\appendix

\section{Typology Details}
\label{sec_appendix_taxonomy}

The 8 categories from which the sentences were taken initially for deriving \taxtypes{} are ``Fashion'', ``Automotive'', ``Books'', ``CDs and Vinyls'', ``Digital Music'', ``Electronics'', ``Movies and TV'', ``Toys and Games''. These were chosen due to the expected differences in aspects associated to the categories, to allow variability and ensure an exhaustive view at types found within reviews.

The \textit{inappropriate} \taxtype{} was not found in the initial data we used, likely because e-commerce websites make an effort to remove such reviews. However this \taxtype{} is a know characteristic in reviews data, and sometimes seeps in anyways.

We used NLTK \citep{bird2006nltk} sentence tokenization to split reviews to sentences, and applied simple normalization techniques to clean the texts. This procedure was used also for the rest of the experiments and analyses in the paper.


\section{Predictor Model Details}
\label{sec_appendix_predictor}

\subsection{Development and Test Evaluation}
\label{sec_appendix_predictor_data_evaluation}

%\paragraph{Development data.}
%The percentage of sentences with each \taxtype{} is as follows... [TODO] In the dev set...

\paragraph{Development evaluation.}
All our initial assessments on the models were done on the development set. We tried several temperatures and engineered some prompts for the models. We only evaluated models that all its responses started with a \say{yes} or \say{no}. 

We extracted \textbf{optimal thresholds} for all \taxtypes{} as follows: For thresholds $[0.1, 1.0]$ incrementing by $0.1$, and for each \taxtype{}, we computed the $F_1$ score against the annotations. The threshold with the maximum $F_1$ score was chosen as the optimal threshold for the corresponding \taxtype.

The macro-$F_1$ score for a model is then the average of $F_1$ scores over all \taxtypes{} with their optimal thresholds. The main results on the development set are presented in \autoref{tab_dev_results}.

%When all responses of a model started with a ``yes'' or ``no'', we calculated the probabilities of each \taxtype{} for each sentence. Then the sentences were marked with a \taxtype{} if its probability was above a $0.5$ threshold. Then, for each sentence individually, we computed the $F_1$ score between the predicted and gold labels. The final score is the average of $F_1$ scores over all sentences. The average of recalls tells us how many of the correct \taxtypes{} were predicted per sentence, and the average of precision tells us how many of the predicted \taxtypes{} are correct. Some of the results on the development set are presented in \autoref{tab_dev_results}.

% \begin{table*}[t]
%     \centering
%     \resizebox{\linewidth}{!}{
%     \begin{tabular}{lcc|c|ccccc|c}
%         \toprule
%         \multirow{2}{*}{\makecell{Model}} &
%         \multirow{2}{*}{\makecell{Temperature}} &
%         \multirow{2}{*}{\makecell{\# Repetitions\\Per Prompt}} &
%         \multirow{2}{*}{\makecell{Macro \\ $F_1$}} &
%         \multicolumn{3}{c}{Sentence-Level} & \multirow{2}{*}{\makecell{\# Pred. \taxtypes{}\\Per Sentence}} &
%         \multirow{2}{*}{\makecell{\# Gold \taxtypes{}\\Per Sentence}} & \multirow{2}{*}{\makecell{Run Time\\Per Sentence}} \\
%         & & & & $F_1$ & Recall & Precision & & & \\
%         \midrule
%         \texttt{flan-t5-xxl} & $0.7$ & 10 & $41.7$ & $17.8$ & $75.3$ & $10.3$ & $4.8$ & $2.2$ & $\sim 10.8$ sec. \\
%         \texttt{flan-t5-xxl} & $0.3$ & 10 & $56.9$ & $27.7$ & $72.6$ & $17.8$ & $4.0$ & $2.2$ & $\sim 10.8$ sec. \\
%         \texttt{flan-t5-xxl} & $0.3$ & 30 & $57.2$ & $23.0$ & $72.6$ & $14.1$ & $3.9$ & $2.2$ & $\sim 10.8$ sec. \\
%         \texttt{flan-ul2}    & $0.3$ & 10 & $51.5$ & $40.1$ & $68.4$ & $31.7$ & $4.7$ & $2.2$ & $\sim 21.0$ sec. \\
%         \texttt{flan-ul2}    & $0.3$ & 30 & $52.2$ & $41.1$ & $66.7$ & $33.5$ & $4.4$ & $2.2$ & $\sim 21.0$ sec. \\
%         \bottomrule
%     \end{tabular}}
%     \caption{The results on the \textit{development} set of the main models in the initial assessment. The threshold for all \taxtypes{} was set to $0.5$ which can make the results noisy. A manual inspection of the results showed good results on both \texttt{flan-t5-xxl} and \texttt{flan-ul2}, and the runtime of \texttt{flan-ul2} is not necesarilly worth the compute resources for our experiments and analyses.}
%     \label{tab_dev_results}
% \end{table*}

%Notice that setting the threshold to $0.5$ strongly affects the results, and hence tend to be noisy or incomparable as-is. A manual inspection of the predictions gave reason to rely confidently enough on the \texttt{flan-t5-xxl} model, and the extra run-time and compute resources incurred by the \texttt{flan-ul2} model was not worth the slight improvement, especially for our uses. Specifically, we even noticed that many of the \taxtypes{} predicted by the models, that were not labeled in the development set, were indeed valid in certain contexts. In any case, our aim is not to produce the best model, but to rely on one enough for our analyses.

\begin{table}[b]
    \centering
    \resizebox{\columnwidth}{!}{
    \begin{tabular}{lcccc}
        \toprule
        \multirow{2}{*}{\makecell{Model}} &
        \multirow{2}{*}{\makecell{Temperature}} &
        \multirow{2}{*}{\makecell{\# Repetitions\\Per Prompt}} &
        \multirow{2}{*}{\makecell{Macro \\ $F_1$}} &
        \multirow{2}{*}{\makecell{Run Time\\Per Sentence}} \\
        & & & & \\
        \midrule
        \texttt{flan-t5-xxl} & $0.7$ & 10 & $41.7$ & $\sim 10.8$ sec. \\
        \texttt{flan-t5-xxl} & $0.3$ & 10 & $56.9$ & $\sim 10.8$ sec. \\
        \texttt{flan-t5-xxl} & $0.3$ & 30 & $57.2$ & $\sim 10.8$ sec. \\
        \texttt{flan-ul2}    & $0.3$ & 10 & $51.5$ & $\sim 21.0$ sec. \\
        \texttt{flan-ul2}    & $0.3$ & 30 & $52.2$ & $\sim 21.0$ sec. \\
        \bottomrule
    \end{tabular}}
    \caption{The results on the \textit{development} set of the main models in the initial assessment. The Macro $F_1$ score is computed based on optimal thresholds for each of the \taxtypes{}.}
    \label{tab_dev_results}
\end{table}

Using 30 repetitions per prompt only slightly improves results, which is not worth the increased compute resources.

A supplementary manual inspection of the predictions gave reason to rely confidently enough on the \texttt{flan-t5-xxl} model. In fact, we noticed that many of the \taxtypes{} predicted by the models, that were not labeled in the development set, were indeed valid in certain contexts. In any case, our aim is not to produce the best model, but to rely on one enough for our analyses.

We note that all other models we tried gave substantially worse results, or irrelevant responses altogether. On the other hand, the models presented in \autoref{tab_dev_results} were very consistent both in output format and final results.


%\paragraph{Test data.}
%The percentage of sentences with each \taxtype{} is as follows... [TODO] In the dev set and in the test set.

\paragraph{Test evaluation.}
We set the thresholds for all \taxtypes{} with those computed on the development set. The $F_1$ score was then computed for each \taxtype{} on the test set, and the average of those produces the final macro-$F_1$ score, as presented in \autoref{tab_test_scores_per_type}. The recall and precision indicate a finer-grained view of the capabilities of the predictor. On manual inspection, many of the incorrectly labeled sentences could be correct in certain contexts. I.e., when the predictor labels incorrectly, it is not nonsensically wrong.


\subsection{Model}
\label{sec_appendix_predictor_model}

\paragraph{Model implementations.}
All FLAN models were run using the Huggingface transformers library \citep{wolf2020huggingface}. The Jurassic model was run using the AI21 Python SDK \citep{ai212023pythonsdk} with a very small monetary budget. In addition to \texttt{flan-t5-\{xl, xxl\}}, \texttt{flan-ul2} and \texttt{j2-jumbo-instruct}, we also tested \texttt{flan-t5-\{base, large\}}, which produced considerably poorer outputs.

\paragraph{Temperature.}
On the models assessed, we mainly tested temperatures of $0.3$ and $0.7$. The FLAN models worked better with $0.3$ and the Jurassic model worked better with $0.7$ (as evaluated on the development set). However, as said, only the \texttt{flan-t5-xxl} and \texttt{flan-ul2} produced reasonable enough results in any case.

\paragraph{Hardware.}
We ran all FLAN predictions on an AWS \texttt{g4dn.12xlarge} EC2 server, which includes 4 NVIDIA T4 GPUs, 64 GB GPU memory and 192 GB RAM. The Jurassic model runs through an API.

\paragraph{Run time.}
The \texttt{flan-t5-xxl} model's run time was about $0.45$ seconds per prompt (including the 10 repetitions for computing probability). Hence, each sentence required an average of $10.8$ seconds for full typology prediction.

\paragraph{Prompt.}
The chosen prompt used in the \texttt{flan-t5-xxl} model is:

\begin{table}[h]
    \vspace{-2mm}
    \centering
    \resizebox{\columnwidth}{!}{
    \begin{tabular}{|l|}
        \hline
        \textsl{
        \begin{tabular}[c]{@{}l@{}}`Given that this sentence is from a product review about\\\texttt{\{product category subprompt\}}, \texttt{\{}\taxtype{} \texttt{question\}}?\\Answer yes or no. The sentence is: \say{\texttt{\{sentence\}}}'\end{tabular}} \\
        \hline
    \end{tabular}}
    \vspace{-4mm}
\end{table}

The \texttt{product category subprompt} is a string prepared for each category. For example the subprompt for the `Electronics' category would be ``\textit{an electronics product}'', and for `Toys and Games' it would be ``\textit{a toy or game}''.

The \texttt{\taxtype{} question} prompt substring is a question prepared for each \taxtype{}, as found in \autoref{tab_taxonomy_types}.

An example of a prompt for a sentence from the ``Books'' category for the \textit{product usage} \taxtype{} would be:

\begin{table}[h]
    \vspace{-2mm}
    \centering
    \resizebox{\columnwidth}{!}{
    \begin{tabular}{|l|}
        \hline
        \textsl{
        \begin{tabular}[c]{@{}l@{}}`Given that this sentence is from a product review about\\a book, does the sentence describe how the product can\\be used? Answer yes or no. The sentence is: ``I love this\\book and just  completed an incredible weekend workshop\\with the author.'''\end{tabular}} \\
        \hline
    \end{tabular}}
    %\vspace{-4mm}
\end{table}

The \taxtype{} predictions on which we rely throughout this paper can be reproduced using these prompts with the \texttt{flan-t5-xxl} Huggingface model. Complementing code will be available as well.


%\section{Experiment Details}
\section{Downstream Task Experiment Details}
\label{sec_appendix_experiments}

This appendix describes the data, training, testing and evaluation details in all the experiments throughout the paper.
%, to facilitate reproducibility.


\subsection{Prediction of Review Helpfulness}
\label{sec_appendix_experiments_review_helpfulness}

\begin{table}[t]
    \centering
    \resizebox{\columnwidth}{!}{
    \begin{tabular}{lccc}
        \toprule
                 & \multirow{2}{*}{\makecell{Helpful\\Reviews}} & \multirow{2}{*}{\makecell{Helpful\\Sentences}} & \multirow{2}{*}{\makecell{Sentiment\\Analysis}} \\
        %\diagbox{Type Set}{Task} & & & \\
        \taxtype{} Set & & & \\
        \midrule
        subjective      & 69.6 & 82.1 & 83.7 \\
        %$\llcorner$~opinions      & 67.0 & 80.6 & 44.5 \\
        %~~~$\llcorner$~\textit{opinion} only    & 35.1 & 72.7 & 44.5 \\
        %~~~$\llcorner$~\textit{op. w/rsn.} only & 67.1 & 80.5 & 44.5 \\
        ~~~opinions      & 66.6 & 80.8 & 80.2 \\
        ~~~~~~(\textit{opinion} only)    & 48.7 & 72.7 & 80.2 \\
        ~~~~~~(\textit{op. w/rsn.} only) & 67.4 & 80.9 & 80.2 \\
        objective       & 67.3 & 83.3 & 80.2 \\
        %$\llcorner$~description & 66.5 & 81.5 & 44.5 \\
        %$\llcorner$~comparisons & 46.0 & 44.3 & 44.5 \\
        ~~~description & 66.6 & 81.9 & 80.2 \\
        ~~~comparisons & 51.4 & 57.9 & 80.1 \\
        personal        & 60.3 & 64.0 & 80.1 \\
        non-product     & 56.8 & 70.4 & 80.1 \\
        stylistic       & 63.1 & 63.7 & 84.0 \\
        \midrule
        All             & 72.6 & 88.3 & 88.1 \\
        \bottomrule
    \end{tabular}}
    \caption{Accuracy scores (\%) on the classification tasks using an SVM with different subsets of \taxtypes{} from the taxonomy (\autoref{tab_taxonomy_groups}). An indented set is contained in the set hierarchically above it. This is a full version extending \autoref{tab_classification_results}.}
    \label{tab_classification_results_full}
\end{table}

\paragraph{Data used.}
\citet{gamzu2021helpfulsentences} released a dataset of helpful sentences, linked to their full review data. The data is based on 123 products from categories ``Toys'', ``Books'', ``Movies'', ``Music'', ``Camera'' and ``Electronics''. We targeted an approximately balanced amount of helpful to unhelpful reviews (regardless of the sentence-level helpfulness which \citeauthor{gamzu2021helpfulsentences} target). Each review has `helpful-count' and `nothelpful-count' fields. By setting helpful-count $>= 9$ and nothelpful-count $== 0$ for helpful reviews, and nothelpful-count $>= 3$ and helpful-count $== 0$ for unhelpful reviews, we produce 458 and 486 reviews respectively. The aim was to collect about 500 reviews for each class.

\paragraph{Training and testing.}
We use 50 iterations of cross validation, with a train set of 70\% and test if 30\%. The procedure is repeated for the different \taxtype{} sets separately.

\paragraph{Evaluation.}
As common in binary classification tasks, we report the accuracy measure \citep{Yadav2020sentAna}.

\paragraph{More results.}
\autoref{tab_classification_results_full} extends \autoref{tab_classification_results}, including all \taxtype{} sets defined in \autoref{tab_taxonomy_groups}.


\subsection{Prediction of Review Sentence Helpfulness}
\label{sec_appendix_experiments_sentence_helpfulness}

\begin{table}[t]
    \centering
    \resizebox{\columnwidth}{!}{
    \begin{tabular}{clccc}
        %\hline
        \toprule
        \multicolumn{2}{l}{Method} & MSE $
        \downarrow$ & PC $\uparrow$ & N@1 $\uparrow$ \\
        \midrule
        \multirow{11}{*}{\rotatebox[origin=c]{90}{{\small Taxonomy \taxtype{} Set w/ Log. Regr.}}}
        & subjective      & 0.096 & 0.66 & 0.86  \\
        %& $\llcorner$~opinions        & 0.101 & 0.64 & 0.75  \\
        %& ~~~$\llcorner$~\textit{opinion} only    & 0.121 & 0.53 & 0.66  \\
        %& ~~~$\llcorner$~\textit{op. w/rsn.} only & 0.116 & 0.56 & 0.74  \\
        & ~~~opinions        & 0.101 & 0.64 & 0.75  \\
        & ~~~~~~(\textit{opinion} only)    & 0.121 & 0.53 & 0.66  \\
        & ~~~~~~(\textit{op. w/rsn.} only) & 0.116 & 0.56 & 0.74  \\
        & objective       & 0.103 & 0.63 & 0.86  \\
        %& $\llcorner$~description     & 0.110 & 0.59 & 0.86  \\
        %& $\llcorner$~comparisons     & 0.165 & 0.08 & 0.69  \\
        & ~~~description     & 0.110 & 0.59 & 0.86  \\
        & ~~~comparisons     & 0.165 & 0.08 & 0.69  \\
        & personal        & 0.152 & 0.31 & 0.62  \\
        & non-product     & 0.139 & 0.42 & 0.64  \\
        & stylistic      & 0.149 & 0.33 & 0.57  \\
        %\cline{2-5}
        \cmidrule{2-5}
        & All             & 0.075 & 0.75 & 0.86  \\
        \midrule
        \multirow{4}{*}{\rotatebox[origin=c]{90}{{\small\citeauthor{gamzu2021helpfulsentences}}}} & Random          & 0.500 & 0.02 & 0.68 \\
        & Baseline 1 {\small(TF-IDF)}        & 0.090 & 0.63 & 0.91 \\
        & Baseline 2 {\small(ST-RIDGE)}        & 0.062 & 0.78 & 0.94 \\
        & Best {\small(BERT)}            & 0.053 & 0.84 & 0.95 \\
        \bottomrule
    \end{tabular}}
    \caption{Mean squared error, Pearson correlation and NDCG@1 scores on the helpful sentence scoring task using Logistic Regression with different subsets of \taxtypes{} from the taxonomy (\autoref{tab_taxonomy_groups}), and compared to results of \citet{gamzu2021helpfulsentences}. An indented set is contained in the set hierarchically above it. This is a full version extending \autoref{tab_helpful_sentences_results}.}
    %\caption{Mean squared error, Pearson correlation and NDCG@1 scores on the helpful sentence scoring task using Logistic Regression with different subsets of \taxtypes{} from the taxonomy (\autoref{tab_taxonomy_groups}), and compared to results of \citet{gamzu2021helpfulsentences}. The combination of \taxtypes{} provides a strong signal for helpfulness.}
    \label{tab_helpful_sentences_results_full}
\end{table}

\paragraph{Data used.}
The full data of \citet{gamzu2021helpfulsentences} consists of 20000 sentences in the train set and 2000 in the test set. Each sentence has a continuous helpfulness score between 0 and 2. In addition, we find the helpfulness scores marking the borders of the top and bottom tertiles in the train set ($1.4$ and $1.0$), and mark the sentences as helpful or unhelpful respectively, or neutral for the mid-section. We use the same border scores to mark the sentences in the test set. Overall there are (7475, 7072, 5453) (unhelpful, helpful, neutral) sentences in the train, and (742, 562, 696) in the test. Notice that classes are not perfectly balanced in the train set since the scores on the borders (1.4 and 1.0) repeat in many sentences.

\paragraph{Training and testing.}
The Linear Regression is conducted on the full original data. The SVM classification is done on the sentences marked with the helpful and unhelpful classes only (neutral ignored).


\paragraph{Evaluation.}
We use the metrics reported in \citep{gamzu2021helpfulsentences} for the regression task, and accuracy for the binary classification task.

\paragraph{More results.}
\autoref{tab_classification_results_full} and \autoref{tab_helpful_sentences_results_full} extend \autoref{tab_classification_results} and \autoref{tab_helpful_sentences_results} respectively, including all \taxtype{} sets defined in \autoref{tab_taxonomy_groups}.



\subsection{Prediction of Sentiment Polarity}
\label{sec_appendix_experiments_sentiment}

\paragraph{Data used.}
As in \S{\ref{sec_appendix_experiments_review_helpfulness}}, we use the subset of products from \citet{gamzu2021helpfulsentences} for convenience. In order to prevent any bias from the helpfulness signal, we randomly sampled 5000 reviews (out of 58205) from the data without any feature pertaining to up-votes and down-votes. The review rating distribution is: $\{\texttt{5}: 3337, \texttt{4}: 675, \texttt{3}: 321, \texttt{2}: 219, \texttt{1}: 448\}$, and the polarity hence distributes to: $\{\texttt{positive}: 4012, \texttt{negative}: 988\}$. There are an average of $4.05$ sentences per review.

\paragraph{Training and testing.}
As in \S\ref{sec_appendix_experiments_review_helpfulness}, we use 50 iterations of cross validation, with a train set of 70\% and test of 30\%, averaging results over the 50 iterations. The procedure is repeated for the different \taxtype{} sets separately.

\paragraph{Evaluation.}
As in \S\ref{sec_appendix_experiments_review_helpfulness}, we report the accuracy measure.

\paragraph{More results.}
\autoref{tab_classification_results_full} extends \autoref{tab_classification_results}, including all \taxtype{} sets defined in \autoref{tab_taxonomy_groups}.



\subsection{Analysis of Reviews and Summaries}
\label{sec_appendix_experiments_analysis}

\paragraph{Data used.}
We iterated over the 31K products in the AmaSum dataset \citep{brazinskas2021amasum}, and filtered out the products without any product category assigned to it. Since products in this dataset are given several hierarchical category options, we heuristically assigned a category by manually clustering related category labels to some general ones. Then for 5 categories (``Books'', ``Electronics'', ``Apparel'', ``Toys and Games'', ``Pet Supplies'') with over 30 products, and chosen manually by their differing aspect-level characteristics, we randomly sampled 20 products. For these products we collected all their reviews and one reference summary (there are rare cases in AmaSum with more than one reference summary for a product). Overall there are 100 products with an average of $77.3$ reviews per product ($7729$ reviews total), $4.2$ sentences per review, and $7.1$ sentences per summary.

\paragraph{Analysis.}
Like in other experiments, a review/summary level vector is the average of its sentence vectors. Review/summary level vectors are then averaged to get the final two vectors to compare (in Figure \ref{fig_review_vs_summaries_vectors}).

For the rhetorical structure analysis, the 6-sentence reviews behave similarly to reviews with other lengths (Figure \ref{fig_review_structure}). Less than 6 sentences does not emphasize the behavior visually as clearly. There are $763$ (out of $7729$) reviews with 6 sentences. For the analysis on summaries, the 7-sentence summaries had the highest number of instances ($24$ out of $100$), and it is visually easier to see the patterns in the data due to the summaries containing 3 subsections (\emph{verdict}, \emph{pros}, \emph{cons}), although other length summaries behave similarly. Here, the sentence vectors at each review/summary position are averaged in order to plot the graphs. In the figures, only the \taxtypes{} with observable changes that have probabilities above 0.2 throughout the review/summary are displayed.



% \begin{table}[t]
    \centering
    % \small {
    \resizebox{\columnwidth}{!}{
    \begin{tabular}{cccc}
        \toprule
        Helpful-S & Unhelpful-S & Helpful-R & Unhelpful-R \\
        % \multirow{2}{*}{\makecell{Helpful\\Sentences}} & \multirow{2}{*}{\makecell{Unhelpful\\Sentences}} & \multirow{2}{*}{\makecell{Helpful\\Reviews}} & \multirow{2}{*}{\makecell{Unhelpful\\Reviews}}  \\
        % & & & \\
        \midrule
        $0.922^{**}$ & $0.359^\dag$ & $0.795^{**}$ & $0.602^{**}$ \\
        \bottomrule
        % \multicolumn{4}{l}{{\small }}
    \end{tabular}}
    \caption{Pearson correlation between \taxtype{} vectors of review sentences (S) or reviews (R), and summaries. 
    % Summaries seem to mainly consist of helpful sentences.
    \\ $**$ $p < 0.001$, $\dag$ $p = 0.085$}
    \label{tab_summary_helpfulness_correlation}
\end{table}




\section{Predictor Evaluation with Specific \taxtypes{}}
\label{sec_appendix_predictor_eval_type_specific}

In addition to the standard evaluation of our \taxtypes{} prediction model in Section \ref{sec_prediction_quality}, we additionally assessed our model's performance on benchmarks of specific \taxtypes{} already identified in previous work, namely \textit{tip}, \textit{opinion}, and \textit{opinion with reason}. The favorable results here only place further emphasis on the reliability of our model, which gives confidence to perform our analyses in Sections \ref{sec_experiments} and \ref{sec_analysis}. \textbf{This assessment is only supplemental to our evaluation in Section \ref{sec_prediction_quality}.}

We obtain existing annotated datasets and establish training sets for the sole purpose of setting a prediction threshold for one of our \taxtypes{}, which is then used as the predicted label for the annotated task.
We experiment with tuning over the entire training set as well as with much smaller subsets of 100 sentences.


\subsection{Tip Classification Evaluation}
\label{sec_appendix_experiments_tips}

Product tips are generally defined as short, concise, practical, self-contained pieces of advice on a product~\citep{hirsch2021producttips}, for example: \say{Read the card that came with it and see all the red flags!}. While previous work~\citep{hirsch2021producttips, farber2022tips} characterized tips into finer-grained sub-types, our definition assumes a generic definition.

\paragraph{Data used.}
We use the data annotated by~\citet{hirsch2021producttips} over Amazon reviews for non-tips or tips. The data used by \citet{hirsch2021producttips} for their experiments (3059 tips and 48,870 non-tip sentences) is slightly different from the data we used (3848 tips and 81,323 non-tips), since we do not apply the initial rule-based filtering that they enforce.

We noticed that some sentences were annotated in the dataset as tips, even though we do not view them as such, e.g.,~\say{The ring that holds the card together is kind of flimsy}. Since this was especially true under the \textit{warning} sub-type, we removed these sentences from the data, to better represent our notion of a \textit{tip}.

Our main goal is to show that our predictor performs decently, and not that we provide a better solution than \citet{hirsch2021producttips}. Hence, exacting the data distribution is not a major concern. Of the available data, we use all 3,848 tip-sentences and sample twice as many non-tip sentences (out of the 81,323).

The dataset includes reviews from 5 categories: ``Musical Instruments'', ``Baby'', ``Toys and Games'', ``Tools and Home Improvement'', and ``Sports and Outdoors''.

\paragraph{Training and testing.}
We follow the cross-validation \citep{xu2001crossVal} procedure of \citet{hirsch2021producttips} to train (find the best tip-\taxtype{} threshold) and to respectively evaluate tip identification using the found threshold. For 50 iterations, all tips are used (or \textit{warning} sub-types are removed) and the same amount of non-tips are sampled (out of the many available). Then the data is split to 80\%/20\% train/test splits. In case a train set size is forced, e.g., only 100 samples, then it is sampled from the training data. The train set is used to find the optimal threshold using Youden's J statistic on the ROC curve \citep{youden1950youdenj}. Then the test set is used to compute the different evaluation metrics. The results are averaged over the 50 iterations and reported, and the bootstrapping method \citep{efron1992bootstrap} is used to compute confidence intervals at alpha=0.025 (95\% confidence percentile). Since the results in \citet{hirsch2021producttips} do not report confidence intervals, we cannot fully compare to their results, however the differences are large enough to assume significant differences, as viewed in our findings.

\paragraph{Evaluation.}
\citet{hirsch2021producttips} report Recall@Precision scores, i.e., setting a specific precision value and producing the corresponding recall value. This approach is used since there is a tradeoff between presenting tips for more products and being confident about the tips presented. We also present the $F_1$ of the tip/no-tip prediction, as common for classification tasks.








\subsection{Opinions Classification Evaluation}
\label{sec_appendix_experiments_opinions}
Product reviews are rich in opinions and are often more convincing when a reason is provided for the opinion. 
A persuasive saying, whether opinionated or objective, is a type of \textit{argument}~\citep{ecklekohler2015arguments}, a class found to be helpful for predicting review helpfulness~\citep{liu2017argsForHelpfullness,passon2018helpfulness,chen2022argumentMiningForHelpfulness}. We thus turn to an argument mining dataset as an assessment benchmark.

\paragraph{Data used.}
The $AM^2$~\citep{chen2022argumentMiningForHelpfulness} dataset is annotated at the clause level for various sub-types of subjective and objective arguments and reasons.
The data contains 878 reviews (pre-split to train and test sets) from 693 products in the ``Headphones'' category, and only sentences that are argumentative are kept in each review. Each review is broken down to its clauses (a sentence or smaller). A clause is annotated as ``Policy'', ``Value'', ``Fact'' or ``Testimony'', where the first two are subjective and the latter two are objective. In addition, a clause can be marked as ``Reason'' or ``Evidence'', where the former provides support for a subjective clause, and the latter for an objective clause. The support clauses are linked to relevant clauses that they support.

We automatically parse and re-label this data and create full-sentence-level instances for \textit{opinion} and \textit{opinion with reason} \taxtypes{}. If a sentence contains a subjective clause, it is marked as an \textit{opinion}, and if it also contains a support clause then it is also marked as an \textit{opinion with reason}. Otherwise a sentence is neither.

We end up with 3132 (2133 train, 999 test) \textit{opinion} sentences, of which 363 (249 train, 114 test) are also \textit{opinion with reason}, and 1359 (972 train, 387 test) non-opinion sentences.

\paragraph{Training and testing.}
We separately classify the \textit{opinion} \taxtype{} and the \textit{opinion with reason} \taxtypes{} against the non-opinionated class. As in the case of \textit{tip}s, we use the train set to find the best threshold for the relevant \taxtype{}, and then evaluate on the test set. We also experimented with sampling just 100 train instances for tuning the threshold.

\paragraph{Evaluation.}
As common in classification tasks, we report the $F_1$ measure.


\begin{table}[t]
    \centering
    \resizebox{\columnwidth}{!}{
    \begin{tabular}{lcc}
        \toprule
        Classification Evaluation & Train Size & Test Size \\
        \midrule
        \textit{Tip} - \citet{hirsch2021producttips}          & 4894 & 1224 \\
        \textit{Tip} - Our experiments                        & 6156 & 1540 \\
        \textit{Tip} - Our experiments, no \textit{warning}s  & 4140 or 100 & 1036 \\
        \textit{Opinion} - Our experiments                    & 3105 or 100 & 1386 \\
        \textit{Opinion with Reason} - Our experiments        & 1221 or 100 & 501 \\
        \midrule
        Standard evaluation (All \taxtypes{} - \S \ref{sec_prediction_quality})                                   & 300 (dev set) & 240 \\
        \bottomrule
    \end{tabular}}
    \caption{The dataset sizes used in the classification evaluations in Appendix \ref{sec_appendix_predictor_eval_type_specific} and Section \ref{sec_prediction_quality}.}
    \label{tab_predictor_dataset_sizes}
\end{table}


\subsection{Results}
\label{sec_appendix_experiments_specific_type_results}
\autoref{tab_predictor_eval} presents the classification results on the three specific \taxtype{} evaluations.
Our zero-shot classifier, which only uses the training sets to find an optimal threshold for the corresponding \taxtype{}, is able to identify \textit{tip}s and \textit{opinion}s effectively, and \textit{opinion with reason} fairly well, on their respective benchmarks.\footnote{\citet{chen2022argumentMiningForHelpfulness} do not provide any intrinsic baseline results on opinion classification to which we can compare. In addition, we manipulated the original data from clause-level to sentence-level.} Moreover, limiting the training sets to only 100 samples appears to hardly have any effect on the quality of the model, showing its robustness to paucity of labeled data.

For the \textit{tip} task, we see in \autoref{tab_tips_results_vs_baseline_full} that the predictor performs on par or much better than existing supervised baselines from \citet{hirsch2021producttips}. Although our definition of \emph{tip} differs slightly from that used in annotating the benchmark, our predictor is still reliable and useful.
When the \textit{warning} subtype of a \textit{tip} is removed from the data (which, as mentioned before, generally do not fit our notion of a tip), the results dramatically improve.

\begin{table}[t]
    \centering
    %\resizebox{0.7\columnwidth}{!}{
    \begin{tabular}{lcc}
        \toprule
         & \multicolumn{2}{c}{$|$Train$|$} \\
        Our Classifier on \taxtype{} & 100 & full \\
        \midrule
        \textit{Tip} (no \textit{warning}s) & 70.0 & 71.0 \\
        \textit{Opinion}                    & 85.8 & 87.6 \\
        \textit{Opinion w/Reason}           & 62.8 & 62.8 \\
        %\midrule
        %\textit{All}                  &  & 56.7  \\
        \bottomrule
    \end{tabular}%}
    \caption{$F_1$ on \taxtype{}-specific datasets when using 100 instances, or all instances in the training set to tune the \taxtype{} threshold.}
    \label{tab_predictor_eval}
\end{table}

\input{Figures/table_tips_results_vs_baseline_full}





\begin{table*}[t]
\centering
    \resizebox{\linewidth}{!}{
    \begin{tabular}{llll}
\hline
\textbf{\taxtype{}}   &	\textbf{Internal Definition}	&	\textbf{Prompt Question for LLM}	&	\textbf{Sentence Example} \\
\hline
\hline
opinion					&	\begin{tabular}[c]{@{}l@{}}a subjective expression regarding\\the product or something else\end{tabular}	&	\begin{tabular}[c]{@{}l@{}}does the sentence express\\an opinion about anything\end{tabular}	&	\begin{tabular}[c]{@{}l@{}}I love this so much that I reordered\\another one online. \end{tabular} \\
\hline
opinion\_with\_reason	&   \begin{tabular}[c]{@{}l@{}}a subjective expression regarding\\the product or something else,\\along with reasoning for the viewpoint\end{tabular}   &	\begin{tabular}[c]{@{}l@{}}does the sentence express an\\opinion about anything and also\\provide reasoning for it \end{tabular} & \begin{tabular}[c]{@{}l@{}}The book is well written in\\an easy to read format.\end{tabular} \\
\hline
improvement\_desire		&	\begin{tabular}[c]{@{}l@{}}something the customer wishes\\that the product could have had\\for improvement\end{tabular}	&	\begin{tabular}[c]{@{}l@{}}does the sentence say how\\the product could be improved\end{tabular}	&	\begin{tabular}[c]{@{}l@{}}More new tiles would have been\\nice for this new release. \end{tabular} \\
\hline
comparative				&	\begin{tabular}[c]{@{}l@{}}compares to another product,\\another version of the product,\\or family of products\end{tabular}	&	\begin{tabular}[c]{@{}l@{}}does the sentence compare\\to another product\end{tabular}	&	\begin{tabular}[c]{@{}l@{}}Hopefully Broan makes better motors. \end{tabular} \\
\hline
comparative\_general	&	\begin{tabular}[c]{@{}l@{}}something that compares the\\product generally to other things\end{tabular}	&	\begin{tabular}[c]{@{}l@{}}does the sentence describe\\something that compares the\\product generally to something\\that is not a product\end{tabular}	&	\begin{tabular}[c]{@{}l@{}}This is not a low carb or Paleo diet. \end{tabular} \\
\hline
buy\_decision			&	\begin{tabular}[c]{@{}l@{}}says something straightforward\\about acquiring or not\\acquiring the product\end{tabular}	&	\begin{tabular}[c]{@{}l@{}}does the sentence explicitly\\talk about buying the product\end{tabular}	&	\begin{tabular}[c]{@{}l@{}}I'll be buying a copy for my\\almost-2-year-old granddaughter. \end{tabular} \\
\hline
speculative				&	\begin{tabular}[c]{@{}l@{}}something the customer thinks\\will be the case with the\\product or that will happen\\because of the product\end{tabular}	&	\begin{tabular}[c]{@{}l@{}}does the sentence speculate\\about something\end{tabular}	&	\begin{tabular}[c]{@{}l@{}}I expect it to last another 2-3 years. \end{tabular} \\
\hline
personal\_usage			&	\begin{tabular}[c]{@{}l@{}}describes something that\\someone did with the product,\\sometimes describes what\\happened to the product after\\some time/use\end{tabular}	&	\begin{tabular}[c]{@{}l@{}}does the sentence describe\\how someone used the product\end{tabular}	&	\begin{tabular}[c]{@{}l@{}}I walked all over Europe in these! \end{tabular} \\
\hline
situation				&	\begin{tabular}[c]{@{}l@{}}explains a situation in which\\the product is used (usually\\relates to "personal/product usage")\end{tabular}	&	\begin{tabular}[c]{@{}l@{}}does the sentence describe\\a condition under which\\the product is used\end{tabular}	&	\begin{tabular}[c]{@{}l@{}}When a car won't jump, we can't help but\\wonder if it is the battery or these cables. \end{tabular} \\
\hline
setup					&	\begin{tabular}[c]{@{}l@{}}something about the\\setup/installation of the product\end{tabular}	&	\begin{tabular}[c]{@{}l@{}}does the sentence describe\\something about the setup\\or installation of the product\end{tabular}	&	\begin{tabular}[c]{@{}l@{}}The instructions were clear and the mounting\\went flawlessly. \end{tabular} \\
\hline
tip						&	\begin{tabular}[c]{@{}l@{}}a suggestion for what to do with\\the product\end{tabular}	&	\begin{tabular}[c]{@{}l@{}}does the sentence provide\\a tip on the product\end{tabular}	&	\begin{tabular}[c]{@{}l@{}}If I were a newby on this, I would just buy the\\older edition for a buck and use it instead. \end{tabular} \\
\hline
product\_usage			&	\begin{tabular}[c]{@{}l@{}}descibes how the product can be\\used\end{tabular}	&	\begin{tabular}[c]{@{}l@{}}does the sentence describe\\how the product can be used\end{tabular}	&	\begin{tabular}[c]{@{}l@{}}they're great for wearing at home or\\gym and doing exercises in one place \end{tabular} \\
\hline
product\_description	&	\begin{tabular}[c]{@{}l@{}}something objective about the\\product like its\\characteristics or story line\end{tabular}	&	\begin{tabular}[c]{@{}l@{}}does the sentence describe\\something objective about\\the product like its characteristics\\or its plot\end{tabular}	&	\begin{tabular}[c]{@{}l@{}}I would estimate that it weighs somewhere\\around 20-25 lbs by itself. \end{tabular} \\
\hline
price					&	\begin{tabular}[c]{@{}l@{}}talks about the price explicitly,\\possibly at a different\\retailers\end{tabular}	&	\begin{tabular}[c]{@{}l@{}}does the sentence explicitly\\talk about the price of the product\end{tabular}	&	\begin{tabular}[c]{@{}l@{}}i bought a pair from the store for 50\$\\and they are very comfortable \end{tabular} \\
\hline
compatibility			&	\begin{tabular}[c]{@{}l@{}}describes usage of the product\\along with another product\end{tabular}	&	\begin{tabular}[c]{@{}l@{}}does the sentence describe\\the compatibility of the product\\with another product\end{tabular}	&	\begin{tabular}[c]{@{}l@{}}Even though it was sold with a particular\\product as an extra battery, it\\isn't, and it doesn't fit into that product. \end{tabular} \\
\hline
personal\_info			&	\begin{tabular}[c]{@{}l@{}}something personal about\\someone, that may or may not\\have to do with the product\end{tabular}	&	\begin{tabular}[c]{@{}l@{}}does the sentence say something\\about someone\end{tabular}	&	\begin{tabular}[c]{@{}l@{}}It's just a beautiful book, one I will keep\\and pass down to my kids' children in 30 years. \end{tabular} \\
\hline
general\_info			&	\begin{tabular}[c]{@{}l@{}}general information that may\\or may not have to do with the\\product\end{tabular}	&	\begin{tabular}[c]{@{}l@{}}does the sentence describe\\general information that is not\\necesarilly in regards to the product\end{tabular}	&	\begin{tabular}[c]{@{}l@{}}There are MANY versions of ``A Christmas \\Carol'' out there. \end{tabular} \\
\hline
comparative\_seller		&	\begin{tabular}[c]{@{}l@{}}comparison between sellers of\\the product\end{tabular}	&	\begin{tabular}[c]{@{}l@{}}does the sentence compare\\between sellers of the product\end{tabular}	&	\begin{tabular}[c]{@{}l@{}}Saved a bundle buying this from Amazon\\and not the cable company. \end{tabular} \\
\hline
seller\_experience		&	\begin{tabular}[c]{@{}l@{}}something about the experience\\with the seller\end{tabular}	&	\begin{tabular}[c]{@{}l@{}}does the sentence describe\\something about the experience\\with the seller\end{tabular}	&	\begin{tabular}[c]{@{}l@{}}I contacted the seller through email\\and they were amazing. \end{tabular} \\
\hline
delivery\_experience	&	\begin{tabular}[c]{@{}l@{}}something about the experience\\of the delivery\end{tabular}	&	\begin{tabular}[c]{@{}l@{}}does the sentence describe\\the shipment of the product\end{tabular}	&	\begin{tabular}[c]{@{}l@{}}The packaging was good too. \end{tabular} \\
\hline
imagery					&	\begin{tabular}[c]{@{}l@{}}a dramatic or figurative\\description of something\end{tabular}	&	\begin{tabular}[c]{@{}l@{}}is the sentence written\\in a figurative style\end{tabular}	&	\begin{tabular}[c]{@{}l@{}}You change them out just about as frequently\\as the oil in your lawn tractor. \end{tabular} \\
\hline
sarcasm					&	\begin{tabular}[c]{@{}l@{}}the sentence has a sarcastic\\expression\end{tabular}	&	\begin{tabular}[c]{@{}l@{}}does the sentence contain\\a sarcastic expression\end{tabular}	&	\begin{tabular}[c]{@{}l@{}}Conveniently, it's not returnable either. \end{tabular} \\
\hline
rhetorical				&	\begin{tabular}[c]{@{}l@{}}something used as a filler or\\for transition, but is not\\really needed\end{tabular}	&	\begin{tabular}[c]{@{}l@{}}is the sentence rhetorical\\or used as a filler or for transition\\without any real value\end{tabular}	&	\begin{tabular}[c]{@{}l@{}}There's not much to say about products like this. \end{tabular} \\
\hline
inappropriate			&	\begin{tabular}[c]{@{}l@{}}contains content that is toxic\\or unnecessarily racy\end{tabular}	&	\begin{tabular}[c]{@{}l@{}}does the sentence contain\\content that is toxic or\\unnecessarily racy\end{tabular}	&	\begin{tabular}[c]{@{}l@{}}Buying this was a f****** waste of money. \end{tabular} \\
\hline
    \end{tabular}}
    \caption{\textbf{The full typology of review sentence \taxtypes{}}. Sentences can be labeled with several \taxtypes{}. The prompt question substring is used for filling in the prompt template (Appendix \ref{sec_appendix_predictor_model}) for tagging the corresponding \taxtype{}.}
    \label{tab_taxonomy_types}
\end{table*}

\begin{table*}[t]
    \centering
    \resizebox{\linewidth}{!}{
    \begin{tabular}{lll}
        \toprule
        \textbf{Review Sentence} & \textbf{Gold \taxtype{} Labels} & \textbf{Predicted \taxtype{} Labels} \\
        \midrule
        \midrule
        \begin{tabular}[c]{@{}l@{}}These weaves are great for\\practicing at home.\end{tabular} & \begin{tabular}[c]{@{}l@{}}opinion, product\_usage, tip\end{tabular} & \begin{tabular}[c]{@{}l@{}}\textbf{opinion}, opinion\_with\_reason,\\personal\_usage, \textbf{product\_usage},\\situation, product\_description, \textbf{tip}\end{tabular} \\
        \midrule
        \begin{tabular}[c]{@{}l@{}}I think you could get 16\\hours out of it if you\\listened on 5\% volume\\while flying in a fighter jet\\over multiple time zones\\and you calculated the\\time change.\end{tabular} & \begin{tabular}[c]{@{}l@{}}opinion, product\_description,\\sarcasm, \underline{imagery}, speculative\end{tabular} & \begin{tabular}[c]{@{}l@{}}\textbf{opinion}, opinion\_with\_reason,\\personal\_usage, product\_usage,\\\textbf{product\_description}, \textbf{speculative},\\comparative\_general, \textbf{sarcasm}\end{tabular} \\
        \midrule
        \begin{tabular}[c]{@{}l@{}}Our dog seems to like them\\just fine but they dont last\\very long because they are\\MAYBE 1/2 of what they\\advertise.\end{tabular} & \begin{tabular}[c]{@{}l@{}}opinion, product\_description,\\\underline{personal\_info}\end{tabular} & \begin{tabular}[c]{@{}l@{}}\textbf{opinion}, opinion\_with\_reason,\\personal\_usage, product\_usage,\\\textbf{product\_description}, speculative,\\price\end{tabular} \\
        \midrule
        \begin{tabular}[c]{@{}l@{}}They are plenty loud for my\\trail riding/work around\\the property!\end{tabular} & \begin{tabular}[c]{@{}l@{}}opinion, personal\_usage,\\\underline{personal\_info}, product\_description,\\situation\end{tabular} & \begin{tabular}[c]{@{}l@{}}\textbf{opinion}, opinion\_with\_reason,\\\textbf{personal\_usage}, product\_usage,\\\textbf{situation}, \textbf{product\_description}\end{tabular} \\
        \midrule
        \begin{tabular}[c]{@{}l@{}}Really, they are pretty basic\\recipes that you could find\\anywhere.\end{tabular} & \begin{tabular}[c]{@{}l@{}}opinion, speculative,\\product\_description,\\comparative\_general\end{tabular} & \begin{tabular}[c]{@{}l@{}}\textbf{opinion}, \textbf{product\_description},\\\textbf{speculative}, \textbf{comparative\_general},\\sarcasm\end{tabular} \\
        \midrule
        \begin{tabular}[c]{@{}l@{}}I bought this Hard Drive for my\\Xbox One because 500 gb\\internal was not enough seeing\\as how an average game is\\almost 50 GB alone.\end{tabular} & \begin{tabular}[c]{@{}l@{}}buy\_decision, compatibility,\\personal\_usage, \underline{general\_info},\\\underline{speculative}\end{tabular} & \begin{tabular}[c]{@{}l@{}}opinion, opinion\_with\_reason,\\\textbf{personal\_usage}, product\_usage,\\product\_description, tip,\\\textbf{compatibility}, \textbf{buy\_decisio}n\end{tabular} \\
        \midrule
        \begin{tabular}[c]{@{}l@{}}Live guide has a learning curve,\\but worth it for sure.\end{tabular} & \begin{tabular}[c]{@{}l@{}}opinion, tip, product\_usage\end{tabular} & \begin{tabular}[c]{@{}l@{}}\textbf{opinion}, opinion\_with\_reason,\\setup, \textbf{product\_usage},\\product\_description, \textbf{tip}\end{tabular} \\
        \midrule
        \begin{tabular}[c]{@{}l@{}}I love the set however it arrived\\with one of the cups that hold\\the poles completely stripped\\and I will have to either find a\\larger screw or super glue it.\end{tabular} & \begin{tabular}[c]{@{}l@{}}opinion, delivery\_experience,\\\underline{situation}, \underline{personal\_usage},\\\underline{personal\_info}, \underline{product\_description}\end{tabular} & \begin{tabular}[c]{@{}l@{}}\textbf{opinion}, opinion\_with\_reason,\\buy\_decision, \textbf{delivery\_experience}\end{tabular} \\
        \midrule
        \begin{tabular}[c]{@{}l@{}}I think I can get a dozen\\washes out of this order,\\so it's a very good value.\end{tabular} & \begin{tabular}[c]{@{}l@{}}opinion, opinion\_with\_reason,\\\underline{product\_description}, \underline{product\_usage},\\speculative\end{tabular} & \begin{tabular}[c]{@{}l@{}}\textbf{opinion}, \textbf{opinion\_with\_reason},\\personal\_info, \textbf{speculative},\\price, buy\_decision\end{tabular} \\
        \bottomrule
    \end{tabular}}
    \caption{Examples of review sentences with their gold \taxtype{} labels and predicted labels (by our predictor outlined in \S{\ref{sec_prediction_model}}). \textbf{Bold} \taxtypes{} are true-positive predictions (correctly predicted) and \underline{underlined} ones are false-negatives (wrongly missed by the predictor). In many cases, false-positive \taxtypes{} (wrongly predicted) are not irrational. The predictor is sufficiently reliable for conducting our experiments and analyses, which is our main objective.}
    \label{tab_example_annotations}
\end{table*}

\begin{table*}[t]
    \centering
    \begin{tabular}{lccc|rr|r}
        \toprule
        & & & & \multicolumn{3}{c}{\# (\%) Sentences w/ \taxtype{}} \\
        \taxtype{} & $F_1$ & Recall & Precision & Gold Test & Pred Test & Gold Val \\
        \midrule
        opinion               & 88.4 & 90.3  & 86.6 & 166 (69.0) & 174 (72.4) & 211 (70.3) \\
        opinion\_with\_reason & 46.7 & 84.8  & 32.2 & 46 (19.2)  & 121 (50.6) & N/A \\
        improvement\_desire   & 50.0 & 77.8  & 36.8 & 9 (3.8)    & 19 (7.9)   & 7 (2.3) \\
        comparative           & 68.9 & 77.8  & 61.8 & 27 (11.3)  & 35 (14.6)  & 16 (5.3) \\
        comparative\_general  & 38.2 & 65.0  & 27.1 & 20 (8.4)   & 49 (20.5)  & 16 (5.3) \\
        buy\_decision         & 69.6 & 94.1  & 55.2 & 34 (14.2)  & 58 (24.3)  & 9 (3.0) \\
        speculative           & 44.2 & 60.7  & 34.7 & 28 (11.7)  & 50 (20.9)  & 18 (6.0) \\
        personal\_usage       & 66.7 & 87.8  & 53.8 & 49 (20.5)  & 80 (33.5)  & 39 (13.0) \\
        situation             & 51.1 & 55.8  & 47.1 & 43 (18.0)  & 51 (21.3)  & 13 (4.3) \\
        setup                 & 64.9 & 85.7  & 52.2 & 14 (5.9)   & 23 (9.6)   & 5 (1.7) \\
        tip                   & 50.0 & 74.1  & 37.7 & 27 (11.3)  & 53 (22.2)  & 19 (6.3) \\
        product\_usage        & 46.2 & 75.0  & 33.3 & 44 (18.4)  & 100 (41.8) & 29 (9.7) \\
        product\_description  & 63.2 & 75.0  & 54.5 & 88 (36.8)  & 122 (51.0) & 97 (32.3) \\
        price                 & 78.4 & 90.9  & 69.0 & 22 (9.2)   & 29 (12.1)  & 7 (2.3) \\
        compatibility         & 55.3 & 86.7  & 40.6 & 15 (6.3)   & 32 (13.4)  & 10 (3.3) \\
        personal\_info        & 65.7 & 60.7  & 71.4 & 108 (45.2) & 91 (38.1)  & 59 (19.7) \\
        general\_info         & 32.3 & 62.5  & 21.7 & 9 (3.8)    & 23 (9.6)   & 29 (9.7) \\
        comparative\_seller   & 54.5 & 85.7  & 40.0 & 7 (2.9)    & 15 (6.3)   & 3 (1.0) \\
        seller\_experience    & 52.9 & 100.0 & 36.0 & 9 (3.8)    & 25 (10.5)  & 7 (2.3) \\
        delivery\_experience  & 72.7 & 80.0  & 66.7 & 15 (6.3)   & 18 (7.5)   & 4 (1.3) \\
        imagery               & 62.5 & 71.4  & 55.6 & 21 (8.8)   & 27 (11.3)  & 28 (9.3) \\
        sarcasm               & 27.6 & 44.4  & 20.0 & 9 (3.8)    & 20 (8.4)   & 7 (2.3) \\
        rhetorical            & 44.4 & 34.8  & 61.5 & 23 (9.6)   & 13 (5.4)   & 17 (5.7) \\
        inappropriate         & 66.7 & 100.0 & 50.0 & 5 (2.1)    & 10 (4.2)   & 0 (0.0) \\
        \midrule
        ALL (Avg.)            & 56.7 & 75.9  & 47.7 & \multicolumn{2}{c|}{240}                  & \multicolumn{1}{c}{300} \\
        \bottomrule
    \end{tabular}
    \caption{$F_1$ scores on the test set per \taxtype{} and overall (macro-$F_1$), as described in the evaluation procedure in \S{\ref{sec_prediction_quality}}, on our \taxtypes{} prediction model. There are 240 annotated sentences in the test set. The threshold of each \taxtype{}, applied on the score from the predictor, is decided by tuning on the development set. The count and percentage of sentences with each one of the \taxtypes{} is shown here as well, to showcase the general distribution of sentences with each \taxtype{}. The rightmost column shows the distribution of sentences in the validation set (300 sentences total). The gold and validation percentages correlate at $\rho=0.91$ (Pearson).}
    \label{tab_test_scores_per_type}
\end{table*}

% \begin{table*}[t]
%     \centering
%     \begin{tabular}{lccc|rr}
%         \toprule
%         & & & & \multicolumn{2}{c}{\% Sentences w/ \taxtype{}} \\
%         \taxtype{} & $F_1$ & Recall & Precision & ~~~~~~~Gold & Pred \\
%         \midrule
%         opinion & 88.4 & 90.3 & 86.6 & 69.0 & 72.4 \\
%         opinion\_with\_reason & 46.7 & 84.8 & 32.2 & 19.2 & 50.6 \\
%         improvement\_desire & 50.0 & 77.8 & 36.8 & 3.8 & 7.9 \\
%         comparative & 68.9 & 77.8 & 61.8 & 11.3 & 14.6 \\
%         comparative\_general & 38.2 & 65.0 & 27.1 & 8.4 & 20.5 \\
%         buy\_decision & 69.6 & 94.1 & 55.2 & 14.2 & 24.3 \\
%         speculative & 44.2 & 60.7 & 34.7 & 11.7 & 20.9 \\
%         personal\_usage & 66.7 & 87.8 & 53.8 & 20.5 & 33.5 \\
%         situation & 51.1 & 55.8 & 47.1 & 18.0 & 21.3 \\
%         setup & 64.9 & 85.7 & 52.2 & 5.9 & 9.6 \\
%         tip & 50.0 & 74.1 & 37.7 & 11.3 & 22.2 \\
%         product\_usage & 46.2 & 75.0 & 33.3 & 18.4 & 41.8 \\
%         product\_description & 63.2 & 75.0 & 54.5 & 36.8 & 51.0 \\
%         price & 78.4 & 90.9 & 69.0 & 9.2 & 12.1 \\
%         compatibility & 55.3 & 86.7 & 40.6 & 6.3 & 13.4 \\
%         personal\_info & 65.7 & 60.7 & 71.4 & 45.2 & 38.1 \\
%         general\_info & 32.3 & 62.5 & 21.7 & 3.8 & 9.6 \\
%         comparative\_seller & 54.5 & 85.7 & 40.0 & 2.9 & 6.3 \\
%         seller\_experience & 52.9 & 100.0 & 36.0 & 3.8 & 10.5 \\
%         delivery\_experience & 72.7 & 80.0 & 66.7 & 6.3 & 7.5 \\
%         imagery & 62.5 & 71.4 & 55.6 & 8.8 & 11.3 \\
%         sarcasm & 27.6 & 44.4 & 20.0 & 3.8 & 8.4 \\
%         rhetorical & 44.4 & 34.8 & 61.5 & 9.6 & 5.4 \\
%         inappropriate & 66.7 & 100.0 & 50.0 & 2.1 & 4.2 \\
%         \midrule
%         ALL (Avg.) & 56.7 & 75.9 & 47.7 & - & - \\
%         \bottomrule
%     \end{tabular}
%     \caption{$F_1$ scores on the test set per \taxtype{} and overall (macro-$F_1$), as described in the direct evaluation procedure in \S{\ref{sec_prediction_direct}}. There are 240 annotated sentences in the test set. The threshold of each \taxtype{}, applied on the score from the predictor, is decided by tuning on the development set. The percentage of sentences with each one of the \taxtypes{} is shown here as well, to showcase the general distribution of sentences with each \taxtype{}.}
%     \label{tab_test_scores_per_type}
% \end{table*}