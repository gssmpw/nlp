As illustrated in \S{\ref{sec_experiments}}, our typology offers an instrument to analyze product-related texts and interpret differences between subsets of texts. In this section we focus on analyzing the structure of reviews and of review summaries. The analysis sheds light on the content patterns that reviewers choose to voice and what summarizers focus on when space is limited.

\begin{figure}
  \centering
  \includegraphics[width=0.9\linewidth]{Figures/img_review_summary_vectors.png}
  \caption{Aggregated \taxtype{} probability vectors in AmaSum~\citep{brazinskas2021amasum} reviews and summaries.}
  \label{fig_review_vs_summaries_vectors}
\end{figure}

\paragraph{Data.}
The AmaSum product reviews summarization dataset~\citep{brazinskas2021amasum} consists of about 31K products, each associated with a list of reviews and a reference summary taken from online platforms for professional product reviews.
Each summary comprises \emph{verdict}, \emph{pros} and \emph{cons} sections. 
We perform some filtering heuristics and then sample 100 products with their reviews and summaries, representing five different product categories (see App.~\ref{sec_appendix_experiments_analysis}), for a total of 7,729 reviews and 100 summaries.

% \subsection{Types of Information}
% \label{sec_analysis_types}

\begin{table}[t]
    \centering
    % \small {
    \resizebox{\columnwidth}{!}{
    \begin{tabular}{cccc}
        \toprule
        Helpful-S & Unhelpful-S & Helpful-R & Unhelpful-R \\
        % \multirow{2}{*}{\makecell{Helpful\\Sentences}} & \multirow{2}{*}{\makecell{Unhelpful\\Sentences}} & \multirow{2}{*}{\makecell{Helpful\\Reviews}} & \multirow{2}{*}{\makecell{Unhelpful\\Reviews}}  \\
        % & & & \\
        \midrule
        $0.922^{**}$ & $0.359^\dag$ & $0.795^{**}$ & $0.602^{**}$ \\
        \bottomrule
        % \multicolumn{4}{l}{{\small }}
    \end{tabular}}
    \caption{Pearson correlation between \taxtype{} vectors of review sentences (S) or reviews (R), and summaries. 
    % Summaries seem to mainly consist of helpful sentences.
    \\ $**$ $p < 0.001$, $\dag$ $p = 0.085$}
    \label{tab_summary_helpfulness_correlation}
\end{table}

\subsection{Reviews vs. Summaries}
We compare \taxtype{} prominence between reviews and summaries by aggregating their
%sentence-level 
probability vectors (averaging review- or summary-level \taxtype{} vectors),
presented in \autoref{fig_review_vs_summaries_vectors}.
% For each text, we predict the vector of probabilities for our \taxtypes{} on the sentence level, and average sentence vectors within their review/summary. 
% The review and summary vectors are averaged separately to yield representing \taxtype{}-vectors for reviews and summaries. 
%We present these macro-averaged vector values in \autoref{fig_review_vs_summaries_vectors}. 
%\rl{The text should explain what these probabilities mean, how they are defined, I don't think its straightforward. I would expect P(opinion) to be num-opinion-sentences/num-sentences within each text type. Maybe also note that they are not expected to sum up to 1 because its multi label}
Comparing the \taxtype{} probabilities in reviews and summaries, it is clear (and unsurprising) that summaries are considerably more efficient in communicating information. 
In particular, summaries predominantly describe the product, provide reasoning for opinions and offer many tips.
%a plethora of tips. \rl{Summaries are almost exclusively comprised of 4 types: description, opinion w/o reason and tips} \rl{Analyze pro/con/verdict separately?}
These findings are consistent with \citet{brazinskas2021amasum}, who describe the \emph{verdict} as emphasizing the most important points about the product, and the \emph{pros} and \emph{cons} as providing details about different aspects of the product.

Reviewers seem to add a non-negligible amount of personal information to the parts eventually excluded from summaries, possibly to make their reviews more relatable, for example, \say{We had used this book when my children were young and were identifying flowers.}
Interestingly, according to our analysis of helpfulness (Figure~\ref{fig_classification_vectors_hs}), readers seem to find this characteristic to be less helpful.
In general, we find (\autoref{tab_summary_helpfulness_correlation}) a much higher correlation between the vector of summaries and that of helpful sentences, than between the vector of summaries and that of unhelpful sentences.
In other words, good summaries tend to contain mainly helpful sentences. 
On the other hand, a helpful \textit{review} has a lower correlation with summaries, meaning that it still contains some potential verbosity, which is natural given the less stringent length constraints.



\begin{figure}
  \centering
  \includegraphics[width=0.75\columnwidth]{Figures/img_review_vectors_per_category.png}
  \caption{Aggregated review-level probabilities of select \taxtypes{}, grouped by product categories.}
  \label{fig_review_vectors_per_category}
\end{figure}

\subsection{Product Categories}
We conduct an examination of the reviews at the product \textit{category} level to learn what kind of information reviewers tend to share for different kinds of products.
\autoref{fig_review_vectors_per_category} reveals, for example, that people share much more \textit{personal information} when reviewing toys and games. 
We find that reviewers purchase these mostly for others, especially children, and therefore provide an explanation of who they bought the product for and their experience with it, e.g.,~\say{Hopefully my 3 year old doesn't toss all of the pieces everywhere.} 
A review is also, expectedly, more likely to discuss \textit{setup} of an electronics device than for other kinds of products. 
A book review is more likely to describe a \textit{situation} or provide a \textit{tip},
% perhaps due to the highly individualistic nature of literary taste. 
two \taxtypes{} we find often co-occurring within sentences suggesting usage for the book, e.g.,
\say{This book should probably be combined with other study methods to help your child perform to the best of their ability.}
%~\say{It was a very helpful book to help go back over the basics of programming at a high level.} \rl{can you find an example that is more clearly a tip?}
% Theres a lot to learn, you need to build on what you learn and keep practicing.
% If you need something to read in the Dr's office this would be a good one.
We suggest that this behaviour may be specific to certain subcategories of books, such as self-help and other nonfiction. 
% I.e., we noticed through this analysis that our subset of books included more guide books than books with story plots. 
Overall, analyzing variability of \taxtypes{} across product categories allows data-driven discovery of insights about products and customers.


\subsection{Rhetorical Structure}
\label{sec_analysis_rhetorical}

Rhetorical Structure Theory studies the organization of a body of text and the relationship between its parts \citep{mann1988rst}, facilitating the assessment of a text's coherence. 
In our setting, we can examine how \taxtypes{} affect each other as a review or summary progresses. 
To demonstrate this, we compute the average \taxtype{} probabilities at each sentence position of reviews with 6 sentences, and summaries with 7 sentences (explained in Appendix~\ref{sec_appendix_experiments_analysis}), and plot them on a graph showing the progression.

\begin{figure}
  \centering
  \includegraphics[width=0.85\linewidth]{Figures/img_review_structure.png}
  \caption{Expected probabilities of \taxtypes{} over the progression of a 6-sentence review in AmaSum.}
  \label{fig_review_structure}
\end{figure}

\begin{figure}
  \centering
  \includegraphics[width=0.85\linewidth]{Figures/img_summary_structure.png}
  \caption{Expected probabilities of \taxtypes{} over the progression of a 7-sentence summary in AmaSum.}
  \label{fig_summary_structure}
\end{figure}

As can be seen in \autoref{fig_review_structure}, reviews tend to open and close with a \textit{buy decision} statement. 
Customers use such statements to introduce the situation, e.g.,~\say{Purchased this SSD in order to put Windows 7 and some games on it to decrease load times.}, or conclude with an emphasizing reiteration of satisfaction (or lack thereof), e.g.,~\say{I only regret not getting it sooner!}
Conversely, reasoned opinions and detailed information tend to occur in the middle of the review. %, as interpreted by the hump in \textit{opinion with reason} and \textit{product description} \taxtypes{}. 
We note that the likelihood of \textit{speculative} statements increases as the review progresses. 
Speculation should be delivered only after it is justified, e.g.,~\say{That being said, they're probably still ok for temporary storage.}
% These behaviors exemplify the capability of assessing the review structure using a rich typology of information types.


AmaSum summaries are formed by concatenating a \textit{verdict} (avg. 1.42 sentences in our data), a \textit{pros} section (avg. 4.04 sentences) and a \textit{cons} section (avg. 1.54 sentences). 
Due to this segmented configuration, the rhetorical structure is only partly relevant since global coherence is not an objective of the summary. 
\autoref{fig_summary_structure} demonstrates the behavior of the \emph{verdict} at the first 1--2 sentences, the next $\sim$4 sentences representing the structure of the \emph{pros}, and the last 1--2 corresponding to the \emph{cons} section. 
The \emph{verdict} embodies features that stand out most in the product with its overall sentiment, e.g., \say{Single camera setup that's reasonably priced because it doesn't require a separate base station.}
Indeed, the \textit{product description} and \textit{opinion with reason} \taxtypes{} are particularly likely. 
The \emph{pros} section generally describes advantages of the product ordered by importance, and accordingly we see reasoning and tip content peaking and then decreasing gradually along these sentences.
The \emph{cons} section provides increased  \textit{speculation} to suggest what the disadvantage might cause, e.g., \say{The 16-hour battery life can become frustrating for some}, as well as vastly increased incidence of \textit{improvement desire}, such as \say{Customers commented that these aren't absorbent enough to handle big messes.}

Future work may further examine review and summary structure through the lens of our typology to improve automatic review summarization and to control content type while maintaining coherence.




%we can see the types written in gold summaries when space is limited. if a customer would like to focus on specific types, the types can be used as-is as a tool for presenting specufuc types of information (transition to next section discussing this).


%\input{Figures/fig_helpful_sentences_per_category_histogram}