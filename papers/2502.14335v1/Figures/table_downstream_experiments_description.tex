\begin{table}[t]
    \centering
    \resizebox{\columnwidth}{!}{
    \begin{tabular}{cccccc}
        \toprule
                 & & \multirow{2}{*}{\makecell{Helpful\\Reviews}} & \multirow{2}{*}{\makecell{Helpful\\Sentences}} & \multirow{2}{*}{\makecell{Sentiment\\Analysis}} \\
        %\diagbox{Type Set}{Task} & & & \\
        \multicolumn{2}{l}{Method} & & \\
        \midrule
        \multirow{6}{*}{\rotatebox[origin=c]{90}{{\small\taxtype{} Set with SVM}}} & subjective      & 69.6 & 82.1 & 83.7 \\
        & ~~~(\textit{opinion} only)    & 48.7 & 72.7 & 80.2 \\
        & ~~~(\textit{op. w/rsn.} only) & 67.4 & 80.9 & 80.2 \\
        & objective       & 67.3 & 83.3 & 80.2 \\
        & stylistic       & 63.1 & 63.7 & 84.0 \\
        \cmidrule{2-5}
        & \textbf{All \taxtypes{}} & 72.6 & 88.3 & 88.1 \\
        \midrule
        \multicolumn{2}{l}{Coarse-grained types w/SVM} & 67.7 & 85.9 & 80.4 \\
        \multicolumn{2}{l}{Random (50-50)}             & 50.0 & 50.0 & 50.0 \\
        \multicolumn{2}{l}{Random (proportion known)}  & 50.0 & 51.0 & 68.3 \\
        \bottomrule
    \end{tabular}}
    \caption{Accuracy scores (\%) on the classification tasks using an SVM with different subsets of \taxtypes{} (\autoref{tab_taxonomy_groups}) from the typology (top). The combination of all \taxtypes{} best predicts helpfulness and sentiment. The bottom part shows prediction results using the coarse-grained types from \autoref{tab_taxonomy_groups} as features to an SVM, and the results of random predictors. Full results are in \autoref{tab_classification_results_full} in the appendix.}
    \label{tab_classification_results}
\end{table}