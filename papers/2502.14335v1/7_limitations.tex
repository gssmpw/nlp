The typology we suggest is partly inspired by previous related studies, and a result of a systematic examination of many review sentences which facilitated the accumulation of the typology types. There are likely other possible types and combinations of types that would establish an alternate typology. However, our research here mainly focuses on the use of a broad typology for revealing insights in product reviews and for assisting downstream applications.

In addition, our experiments and analyses rely on a predictor that does not label sentences at full accuracy. While our evaluation of the predictor gives us good reason to place confidence in its outputs, the conclusions we draw are certainly based on some noisy results. We therefore reported on findings from the analyses that are more distinctly evident from the data. Finally, when comparing or classifying between two subsets of texts (e.g., helpful vs. unhelpful, or reviews vs. summaries), the noise is analogously enforced on both subsets, making them generally comparable.

Lastly, our topology can inspire typologies in other review domains (such as hotels or dining) or other domains entirely. While some types in our typology are likely relevant and can be reused for other domains, developing this further requires a more thorough investigation in future work.