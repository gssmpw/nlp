We aim to define a typology that classifies the vast information available in reviews into types that would, in turn, provide insights to review consumers like potential customers or product vendors~\citep{Rohrdantz2016review_insights}.
Establishing a new typology relies on intuition and theory of the domain, as well as empirical analysis of data \citep{Nickerson2013taxonomydevelop}, and requires an iterative process that combines these two fundamental ingredients. This approach is employed across diverse domains, such as conversational social cues \citep[e.g.,][]{Feine2019conversationTaxonomy} and psychological well-being \citep[e.g.,][]{Desmet2020psychoTypology}.

Following this common practice and based on \citet{Nickerson2013taxonomydevelop}, the starting point for defining our typology is a set of existing text types previously researched individually, such as tips~\citep{hirsch2021producttips}, opinions~\citep{vinodhini2012opinionmining}, toxicity~\citep{Djuric2015hatespeech}, sarcasm~\citep{aditya2017sarcasm}, and the distinction between product, delivery, and seller descriptions~\citep{Bhattacharya2020seller_delivery_classification}.
\citet{Rohrdantz2016review_insights} analyzed reviews and defined a theoretical taxonomy of review types that provides us with further inspiration. The next step requires examining data instances and iteratively deriving the information types in review sentences.

To this end, we sampled reviews from eight categories of the Amazon Product Review Dataset \citep{he2016reviewsDS}, and split them into sentences (technical details in Appendix~\ref{sec_appendix_taxonomy}).
The variability of product categories ensures representation of as many sentence types as possible.
One of the authors manually inspected a random sample of several hundred sentences and accumulated sentence types, always marking a sentence with at least one type (\textit{rhetorical} being a catch-all default tag).
The typology was iteratively fine-tuned during this tagging process, as some types were only noticed after several occurrences.
The process concluded as soon as no new types were encountered for over 100 sentences, yielding a total of roughly 550 reviewed sentences.
The resulting list of types was examined by all authors and slightly refined, resulting in the final typology of 24 \taxtypes{}: \vspace{4pt} \\ \noindent\fbox{%
    \parbox{\columnwidth - 2\fboxsep}{%
        \textit{opinion, opinion with reason, improvement desire, comparative, comparative general, buy decision, speculative, personal usage, situation, setup, tip, product usage, product description, price, compatibility, personal info, general info, comparative seller, seller experience, delivery experience, imagery, sarcasm, rhetorical, inappropriate}
    }%
} \vspace{0.5pt} \\ (see \autoref{tab_taxonomy_types} in the appendix for explanations and example sentences of the \taxtypes{}).
While not hierarchical, the \taxtypes{} can be grouped by mutual semantic characteristics into \textit{coarse-grained} types, such as being objective, subjective or stylistic in nature, with some groups partially overlapping (see \autoref{tab_taxonomy_groups}).
These groupings will come in handy in our analyses in Sections \ref{sec_prediction}, \ref{sec_experiments} and \ref{sec_analysis}.

\begin{table}[t]
\centering
    \resizebox{\columnwidth}{!}{
    \begin{tabular}{ll}
        \toprule
        \textbf{Group} & \textbf{\taxtypes{}} \\
        \midrule
        subjective     & \begin{tabular}[c]{@{}l@{}}opinion, opinion\_with\_reason,\\ improvement\_desire, buy\_decision,\\ speculative, seller\_experience, delivery\_experience\end{tabular} \\
        \cdashline{1-1}
        \cline{2-2}
        opinions       & opinion, opinion\_with\_reason \\
        \hline
        objective      & \begin{tabular}[c]{@{}l@{}}comparative, comparative\_general, personal\_usage,\\ situation, setup, tip, product\_usage, product\_description,\\ price, compatibility, general\_info, comparative\_seller\end{tabular} \\
        \cdashline{1-1}
        \cline{2-2}
        description    & \begin{tabular}[c]{@{}l@{}}setup, tip, product\_usage, product\_description,\\ price, compatibility\end{tabular} \\
        \cdashline{1-1}
        \cline{2-2}
        comparisons    & \begin{tabular}[c]{@{}l@{}}comparative, comparative\_general,\\ comparative\_seller \end{tabular} \\
        \hline
        personal       & personal\_usage, personal\_info \\
        \hline
        non\_product   & \begin{tabular}[c]{@{}l@{}}personal\_info, general\_info, comparative\_seller,\\ seller\_experience, delivery\_experience\end{tabular} \\
        \hline
        stylistic     & imagery, sarcasm, rhetorical, inappropriate \\
        \bottomrule
    \end{tabular}}
    \caption{The \taxtypes{} in our typology grouped by mutual semantic characteristics (into coarse-grained types). Some groups partly overlap or are fully contained in others (dashed sections).}
    \label{tab_taxonomy_groups}
\end{table}

%While taxonomies are often modeled hierarchically, our typology does not seem to have any obvious hierarchic arrangement.
%The \taxtypes{}, however, can be grouped by mutual semantic characteristics, such as being objective, subjective or stylistic, with some groups partially overlapping (see \autoref{tab_typology_groups}).

While the typology was defined systematically and carefully, we note that there are certainly other exhaustive typologies that could be established for categorizing information types in reviews.
Our main goal is to reveal the many advantages of identifying content types in reviews using a highly variable typology, whether for understanding review style or for informing existing downstream classification tasks, as we will show in our analyses. A typology different from ours may yield slightly different outcomes, but the general concept and usability are expected to be persistent.
%Sections \ref{sec_experiments} and \ref{sec_analysis}.
