\section{Introduction}

Observational studies serve as crucial sources of evidence when randomized
controlled trials are infeasible or unethical.
Yet, findings from these studies are often presented as if they establish
causality, despite the inherent limitations of observational data.
This practice raises concerns about overstated conclusions in the scientific
literature, particularly when media coverage and press releases further amplify unsupported claims of causation
\cite{cofield2010use,sumner2014association}.
While some scholars argue that causal language should be restricted to evidence
from randomized controlled trials, as exemplified by medical journals like JAMA, others
contend that rigorous causal inference methods, when applied appropriately,
support valid causal conclusions \cite{pearl2018book,hernan2018c,dahabreh2024causal}.


To help inform this debate, we need a better understanding of the factors that
influence researchers to draw causal conclusions from observational studies.
Ideally, causal language should align with the strength and validity of the
underlying evidence. Yet, since scientists operate within broader social
contexts, this study investigates whether sociocultural influences---such as
authors’ cultural backgrounds and research team dynamics---shape the use of
causal language in observational research.


Specifically, our goal was to examine whether characteristics of research teams and 
individual authors---including their country, gender, authorship position, team size, and
publication history---correlate with the use of causal language when drawing conclusions
in their structured abstracts.
To this end, we developed a
computational linguistic algorithm capable of differentiating between causal and
correlational statements. We applied it to over
80,000 observational study abstracts and analyzed the results using logistic
linear mixed-effects regression.
Our analysis controlled for potential confounders
including journal impact factor, study design, and 
a comprehensive set of over 700 Medical Subject Heading (MeSH) terms that
together capture a wide range of research topics, methodologies,
and key scientific concepts.
Additionally, to account for journal-specific biases related to
editorial policies and temporal trends, we modeled journal ISSN and publication
year as random effects. After adjusting for these factors, we found that causal
claims were more frequently used by less experienced authors, smaller teams,
male last authors, and authors from countries with higher uncertainty avoidance
indices \cite{hofstede2010}.