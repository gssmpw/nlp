\section{Results}


Table~\ref{table__overall_effects} 
summarizes our main findings, revealing 
that author country, gender and authorship position, as well as writing experience and
team size, are associated with the use of causal language in the
conclusions of observational study abstracts.

\noib{Country and uncertainty avoidance culture}.
Geographic location appears to influence authors' tendency to use causal
language. Authors from North America (USA and Canada) and Nordic
countries (Denmark, Norway, and Sweden) were among the least likely to use
causal language. In contrast, authors from Central and Southern European
countries—including Switzerland, Austria, Germany, Italy, Greece, and
Poland—demonstrated a significantly higher likelihood of doing so.
Between these two extremes, authors from East Asia (Japan, South Korea, China, and
Taiwan) ranked in the mid-range of the spectrum.


To further illustrate this pattern, we created a plot
(Figure~\ref{fig__uncertainty_vs_causal_all}) showing the relationship
between causal language use across countries and their respective
uncertainty avoidance index (UAI),
a component of Hofstede's cultural dimensions that measures a culture's tolerance for ambiguity and uncertainty \cite{hofstede2010}.
Among Western cultural countries—represented by blue
markers—we observed a strong positive correlation, where higher UAI scores are
associated with a greater tendency to use causal language. 


\begin{table}[!th]
\centering

%\begin{table}[ht]
%\small
\begin{tabular}{@{}r@{}l@{}}

\begin{tabular}[t]{@{}ll@{ }r@{ }l@{}l@{}}
\multicolumn{5}{l}{\textbf{Author country} (reference: others)} \\
 & United States & -0.298 & (0.028) & $^{***}$ \\
 & Denmark & -0.290 & (0.085) & $^{***}$ \\
 & Canada & -0.270 & (0.067) & $^{***}$ \\
 & Norway & -0.262 & (0.106) & $^{*}$ \\
 & Sweden & -0.180 & (0.072) & $^{*}$ \\
 & Israel & -0.156 & (0.105) & $^{}$ \\
 & United Kingdom & -0.104 & (0.046) & $^{*}$ \\
 & Australia & -0.089 & (0.056) & $^{}$ \\
 & Japan & -0.077 & (0.041) & $^{}$ \\
 & South Korea & -0.060 & (0.056) & $^{}$ \\
 & Turkey & -0.019 & (0.062) & $^{}$ \\
 & Taiwan & -0.006 & (0.081) & $^{}$ \\
 & China & 0.012 & (0.040) & $^{}$ \\
 & Egypt & 0.017 & (0.123) & $^{}$ \\
 & France & 0.050 & (0.047) & $^{}$ \\
 & Brazil & 0.083 & (0.061) & $^{}$ \\
 & Netherlands & 0.117 & (0.053) & $^{*}$ \\
 & Portugal & 0.134 & (0.104) & $^{}$ \\
 & Spain & 0.138 & (0.040) & $^{***}$ \\
 & India & 0.163 & (0.054) & $^{**}$ \\
 & Belgium & 0.185 & (0.112) & $^{}$ \\
 & Greece & 0.221 & (0.122) & $^{}$ \\
 & Switzerland & 0.223 & (0.088) & $^{*}$ \\
 & Germany & 0.253 & (0.048) & $^{***}$ \\
 & Austria & 0.291 & (0.131) & $^{*}$ \\
 & Italy & 0.302 & (0.040) & $^{***}$ \\
 & Poland & 0.308 & (0.100) & $^{**}$ \\ 
[.05in]
\iffalse
\midrule 
\multicolumn{5}{l}{\footnotesize{$^{***}p<0.001$; $^{**}p<0.01$; $^{*}p<0.05$}} \\ [.02in]
\fi
\end{tabular}
&

\begin{tabular}[t]{@{}ll@{ }r@{ }l@{}l@{}}

%GENDER_FIRST_BEGIN%
\multicolumn{5}{l}{\textbf{First author gender} (reference: Women)} \\
&  Men & 0.007 & (0.019) & $^{}$ \\
%&  Low-confidence gender inference & 0.041 & (0.028) & $^{}$ \\
%&  Use of initials as forenames &  gender_firstI \\[.07in]
&  Unknown &  0.041 & (0.028) & $^{}$ \\[.07in]
%GENDER_FIRST_END%

%GENDER_LAST_BEGIN%
\multicolumn{5}{l}{\textbf{Last author gender} (reference: Women)} \\
&  Men & 0.065 & (0.021) & $^{**}$ \\
%&  Low-confidence gender inference & 0.051 & (0.030) & $^{}$\\
%&  Use of initials as forenames & gender_lastI \\[.07in]
&  Unknown &  0.051 & (0.030) & $^{}$ \\[.07in]
%GENDER_LAST_END%

\multicolumn{5}{l}{\textbf{Observational studies published} (log-scaled)} \\
& as first author & -0.127 & (0.029) & $^{***}$ \\
& as last author &   -0.167 & (0.019) & $^{***}$ \\[.07in]

\multicolumn{5}{l}{\textbf{Team size} (log-scaled)} \\
& Num co-authors & -0.091 & (0.018) & $^{***}$ \\ [.07in]



\midrule 
\multicolumn{5}{l}{\textbf{Control Variables}} \\ [.07in]

\iffalse
\multicolumn{5}{l}{\textbf{Observational studies published} (log-scaled)} \\
& as first author & -0.127 & (0.029) & $^{***}$ \\
& as last author &   -0.167 & (0.019) & $^{***}$ \\[.07in]

\multicolumn{5}{l}{\textbf{Team size} (log-scaled)} \\
& Num co-authors & -0.091 & (0.018) & $^{***}$ \\ [.07in]
\fi

\multicolumn{5}{l}{\textbf{Journal} (log-scaled)} \\
%& SciMago journal rank & -0.106 & (0.035) & $^{**}$\\ [.07in]
& SciMago journal rank & -0.106 & (0.035) & $^{**}$\\ 
%\multicolumn{5}{l}{\textbf{Journal exposure} (log-scaled)} \\
& observational studies published & -0.045 & (0.010) & $^{***}$ \\ [.07in]

\multicolumn{5}{l}{\textbf{Study design} (reference: the absence of this)} \\
 & Cross-Sectional Studies & -0.331 & (0.030) & $^{***}$ \\
 & Case-Control Studies & -0.258 & (0.044) & $^{***}$ \\
 & Cohort Studies & -0.171 & (0.029) & $^{***}$ \\
 & Longitudinal Studies & -0.157 & (0.043) & $^{***}$ \\
 & Retrospective Studies & -0.008 & (0.021) & $^{}$ \\
 & Prospective Studies & 0.033 & (0.020) & $^{}$ \\
 & Follow-Up Studies & 0.101 & (0.031) & $^{**}$ \\ 
[.05in]
\multicolumn{5}{l}{\textbf{Topics other than study design} (see SI for details)} \\
& \multicolumn{4}{l}{ 764 MeSH terms listed in at least 150 papers} \\ 
[.03in]

\iffalse
\midrule 
\multicolumn{5}{l}{\textbf{Random Effects}} \\ [.07in]
%\multicolumn{5}{l}{Num sentences in the conclusion, Year published, Journal} \\
& Number of sentences in conclusion & \\ 
& Year published & \\ 
& Journal published & \\ 
[.1in]
\fi
\end{tabular} \\





\midrule 
\multicolumn{2}{l}{\textbf{Random Effects}: Num of sentences in conclusion, Year published, Journal published} \\ [.03in]
%\multicolumn{2}{l}{\footnotesize{$^{***}p<0.001$; $^{**}p<0.01$; $^{*}p<0.05$} }
\end{tabular}

%\caption{?? Effects on causal language use}
%\end{table}
    

\caption{
Estimated effects of author country, gender, authorship position, team size, and writing experience on the likelihood of employing causal language in the conclusion section of observational study abstracts.
\footnotesize{$^{***}p<0.001$; $^{**}p<0.01$; $^{*}p<0.05$}
 }
\label{table__overall_effects}
\end{table}


\begin{figure}[!th]
\centering
\includegraphics[width=.75\textwidth]{figure/uncertainty_vs_causal.eps}
\caption{
   Authors from countries with higher UAI scores—reflecting a
    greater cultural preference for certainty—tend to use causal language more frequently.
    The correlation is moderate across all 27 countries (Pearson's $\rho=0.51, p < 0.01$), 
    and strengthens significantly among Western cultural countries ($\rho = 0.74, p < 0.001$).}
\label{fig__uncertainty_vs_causal_all}
\end{figure}


\noib{Author Gender and Authorship Position.}
Our analysis reveals gender-related patterns in causal language use.
While first authors showed no
significant gender differences, male last authors—typically senior researchers
in biomedical fields—used causal language significantly more often than
their female counterparts. These findings suggest that gender differences in
causal language are mainly associated with senior authorship roles.


\noib{Experience and Team Composition.}
Our analysis shows that greater researcher experience and larger team size are
negatively associated with using causal language. Authors with more publications
were less likely to use causal language, especially experienced last authors,
highlighting the role of senior authors in framing causal conclusions.

Additionally, studies conducted by larger teams were less inclined to use causal language.
The presence of diverse perspectives in larger collaborative groups
may encourage more nuanced discussions and negotiations, which in turn may contribute to a
more restrained interpretation of observational study results.


%------------------------------
%
\noib{Control variables.} 
Our analysis of control variables shows patterns that align with conventional
understandings in scientific communication:

(a) {\em Journal Rank and Publication Volume.}
Higher-ranked journals—as measured by
the SciMago journal ranking—and journals with a larger volume of observational
studies exhibited significantly less frequent use of causal language. This
suggests a filtering effect, whereby leading publications such as JAMA actively
discourage the use of causal language.

(b) {\em Study Design.}
The likelihood of using causal language varied systematically
with study design, ranking from least to most likely as follows:
cross-sectional, case-control, longitudinal, cohort, retrospective, prospective,
and follow-up studies. This order reflects the established hierarchy of
evidence in the Evidence-Based Medicine Pyramid \cite{murad2016new}.

\iffalse
(c) {\em Over 700 MeSH Terms.}
Our regression model further controlled for more than
700 Medical Subject Heading (MeSH) terms, each appearing in at least 150
studies. Detailed effects of these terms on the usage of causal language are
provided in the Supplementary Information (see \ref{meshtermeffect}), offering additional insights into how
specific research topics, methods, and key concepts are tied to authors'
linguistic choices.
\fi

