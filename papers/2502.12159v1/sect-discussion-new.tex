\section{Discussion}


This study provides a new perspective on how human factors—such as authors'
sociocultural backgrounds and team dynamics—might influence the use of causal
language in observational studies. One key finding is the association
between a country's level of uncertainty avoidance \cite{hofstede2010} and the
tendency to use causal language; authors from countries with higher
uncertainty avoidance scores are more inclined to present their findings
as causal. This observation is generally in line with existing literature on
cultural influences in communication and decision-making, suggesting that
individuals in high uncertainty avoidance cultures may lean toward more
definitive conclusions and causal explanations as a way to reduce ambiguity and
convey greater certainty.

Another key finding relates to gender and authorship roles. While no significant
gender differences were observed among first authors, our results suggest that
male last authors—typically senior researchers—tend to use causal language
more frequently than their female counterparts. 
This pattern may reflect the combined influence of academic hierarchies and linguistic preferences.
On the one hand, as senior authors typically shape the overall framing
of a study, their linguistic preferences could influence how conclusions are presented.
On the other hand, male authors, who are more likely to use linguistic boosters 
\cite{tannen1995,nasri2018projecting},
may also be inclined to frame study conclusions more assertively and thus make a
greater use of causal language; in contrast, female authors, who tend to
use more hedging language, may adopt a more cautious approach in their
phrasing. 
These results reveal that gender differences in expressing certainty
extend beyond merely using boosters or hedges—they can fundamentally shape
whether research conclusions are framed as causal or correlational.


Our analysis also indicates that larger teams and more experienced authors may adopt a
more cautious approach when drawing causal conclusions from observational
studies. Larger teams, perhaps due to the benefit of diverse perspectives and
collaborative discussion, tend to moderate their use of causal language.
Similarly, experienced authors—especially those in senior roles—appear to be
more mindful of the limitations inherent in observational research and the
potential consequences of overstated claims, leading to a more careful
presentation of their findings.

One limitation of our study is that our analysis does not assess the intrinsic causal strength
of the evidence presented in observational studies. 
Developing methods to quantitatively estimate and incorporate causal strength could offer a more
complete understanding of how evidence quality influences the use of causal language.
