\section{Materials and Methods}


\noib{Data.}
We queried the PubMed database using the search term ``Observational
Study[Publication Type]'' and retrieved over 160,000 papers, most of which were
published in 2014 or later following the introduction of the MeSH term
``Observational Study'' in 2014. After excluding publications that were also classified as
Randomized Controlled Trials and those lacking a structured abstract, a
conclusion section, an English full text, or explicit causal or correlational
claims in the conclusion, we identified a final sample of over 80,000
observational studies for analysis. (See SI A.1 for details)


\noib{Identifying Causal and Correlational Claims}.
In our regression analysis, we defined the dependent variable as the presence or absence of a causal claim in
an observational study abstract's conclusion—specifically, a conclusion is deemed causal
if it includes at least one sentence asserting a causal relationship. To
automatically classify sentences as causal, correlational, or neither, we
developed a BioBERT-based causal language prediction model \citep{yu2019EMNLPCausalLanguage}
(See SI B.1).


\noib{Author Gender.}
We inferred gender from each author's forename, categorizing names as male,
female, or unknown. The ``unknown'' category was assigned when a forename was
missing or when the algorithm's confidence in the gender prediction was low (a
common challenge with names such as Chinese names, which often lack clear gender
indicators). Using a dataset of six million (name, gender) pairs from WikiData,
we developed a name-to-gender inference model that leverages the
last eight letters of forenames as features. Our algorithm (see SI B.3) achieved
performance comparable to the best of the five name-to-gender inference tools when
evaluated on a benchmark dataset \citep{Santamara2018ComparisonAB}.


\noib{Author Country.}
We designed an algorithm to infer an author's country from the
affiliation metadata associated with each paper. This algorithm was built using
country data from approximately 2.7 million organizations available in the ORCID
public database (see SI B.2).

\noib{Author Name Disambiguation.}
To accurately identify and distinguish individual authors, we used the Semantic
Scholar API to retrieve author data for each paper, representing each researcher
by their unique author ID rather than by name alone (see SI A.3).

\noib{Random Effects.}
In our mixed-effects logistic regression analysis (see SI C), we incorporated three random
effects: (1) Conclusion Length, measured by the number of sentences, to account
for the greater likelihood of causal language in longer conclusions; (2)
Journal, to capture variations in editorial guidelines and practices regarding
causal language across publications; and (3) Publication Year, to control for
temporal shifts in language use and evolving academic norms in response to
changing policies and cultural trends \cite{hyland2019}.
