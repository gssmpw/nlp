\section{Learning Approach} \label{sec:method}
We now turn to the question of how to learn an accurate classifier on the market distribution it induces.
Our general approach will be to 
follow the empirical risk minimization framework
and replace the expected risk in Eq.~\eqref{eq:learning_objective_exp}
with an empirical proxy objective over the sample set $\smplst$, namely:
\begin{equation}
\label{eq:empirical_objective}
\argmin_{h \in H} \frac{1}{m} \sum_{i=1}^m 
% \loss(y_i, h(x_i^h)) 
\loss(x_i^h, y_i; h) 
+ \lambda \reg(g), \,\,\,
x_i^h = \brmrkt_h(x_i;\budget_i)
\end{equation}
Here $\loss$ is a surrogate loss (e.g., hinge),
$\reg$ is an (optional) regularization term with coefficient $\lambda$,
and responses $\brmrkt_h$ are defined w.r.t. empirical market prices,
$\phatvec = \p^*(h;\smplst)$.
% which in itself depends both on $h$ and on the entire dataset $\smplst$.
\squeeze

\extended{\paragraph{Challenges.}}
There are several challenges to optimizing Eq.~\eqref{eq:empirical_objective}.
First, as in standard strategic classification, $\br$ is an argmax operator, which is non-differentiable and even discontinuous.
Second, in our market setting,
the objective no longer decomposes over examples,
since each $x_i^h$ depends on \emph{all} data points in $\smplst$ %(and not just on $x_i$)
% since $\br$ depends on market prices $\phatvec$.
through $\phatvec$.
Third, prices depend not only on $\smplst$,
but are a function of %the learned
$h$, which is 
% optimized.
the target of optimization.
% Finally, strategic settings can require specialized losses
% \citep{levanon2022generalized}.
% \squeeze

\extended{%
it is unclear what are appropriate choices for $\loss$ and $\reg$, since as earlier works have shown,
strategic learning often requires using specialized proxy losses and regularizers \citep{levanon2022generalized}.
}



\paragraph{Approach.}
Our solution will be to replace $\brmrkt_h$ with a differentiable proxy
that permits to take gradients `through the market'. % w.r.t. $h$.
First, notice that \emph{conditioned on prices},
user updates $x_i^h$ become independent---this is precisely the role of prices in efficient markets.
Next, we define the loss function $\loss$.
For standard strategic classification with 2-norm costs,
\citet{levanon2022generalized} propose the \emph{strategic hinge}:
\squeeze
\begin{equation}
\label{eq:s-hinge}
\shinge(x,y;h) = \max\{0, 1 - y(w^\top x + \thresh + 2\|w\|)\}
\end{equation}
which penalizes all points according to the maximal moving distance of 2
(for $y \in \{\pm 1\}$).
Although our costs are linear (and so the s-hinge does not apply),
since market prices are adaptive and of the form $\p = \rho w$,
users move `as if' towards the decision boundary.
The key difference is that in our settings, users have individualized maximal distances:
for a given $\rho$, this is the amount of units that each user $i$ can buy,
namely $\budget_i/\rho$.
This gives our proposed \emph{market hinge loss}: % (or \emph{m-hinge}) is:
\squeeze
\begin{equation}
\label{eq:m-hinge}
\mhinge(x,y;h,\rho) = \max\{0, 1 - y(w^\top x + \thresh + \frac{b}{\rho}\|w\|)\}
\end{equation}
Note that $\mhinge$ does not include $\br$ explicitly,
and requires only the scalar market price $\rho$ (which depends on $h$).
Our final step is to replace $\rho$ with 
a \emph{differentiable market price} $\rhotilde$
as a smooth approximation.
This is achieved by making Algo.~\ref{algo:exact_market_prices} itself differentiable---see full details in Appx.~\ref{appx:learning_algorithm}.

