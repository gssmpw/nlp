
% %========================================================================
% \begin{figure*}[t!]
% \centering
% \includegraphics[width=\textwidth]{synth-naive+semi.v2.tmp.png}
% \caption{%
% \todo{...}
% }
% \label{fig:synth-naive+semi}
% \end{figure*}
% % %========================================================================

%========================================================================
\begin{figure*}[t!]
\centering
\includegraphics[width=0.98\textwidth]{beta_dists.v2-crop.pdf}
\caption{%
\textbf{Demand and price setters.}
Demand distributions $q(u)$ for various $\Betadist$ distributions and $\budget=1$.
Shown are pdfs and revenue curves. Note how revenue-maximizing points
(`price setters') are extreme, suggesting that almost all points cross. 
\squeeze
}
\label{fig:beta}
\end{figure*}
%========================================================================



\section{Learning in Markets: Exploratory Insights} \label{sec:synth}
Our goal in this section is to demonstrate
the basic mechanics underlying how learning creates markets,
and how induced markets affect strategic learning and its outcomes.
As we show, such effects can be quite stark.
We begin with questions regarding fixed $h$, and then consider $h$ that are learned.
\squeeze

% Since the market for a classifier $h=h_{w,\thresh}$ is shaped by demand in the direction of $w$, our analysis will initially focus %w.l.o.g.
% on unidimensional data
% representing distances of points $x$ from the decision boundary of $h$,
% denoted by $z \in \R$.
% This will allow us to explore market behavior for a given direction $w$,
% understand the role of the threshold $\thresh$,
% and gain insight on what makes for effective market-aware $h$.

% \todo{reframe as approaches - naive, semi-strat, strat?}

% \todo{acc is high if negs stay and poses move}

% 1. what happens to naïve h in market setting?
% ○ opt naïve h likely to separate data => consider such data (=distribution, not sample)
% ○ market works  in direction of w => focus on 1D, eg two gaussians
% ○ opt h in the middle => focus on D0=D(x|y=0) - single gaussian "to the left" of h
% ○ in sc, under fixed cost, users move up to some distance
% ○ in market, this depends on b!
%     § remember h is indep of b, takes only x as input
%     § hence, choice of (naive) h agnostic to b
% ○ result: for uniform b=1, under almost any reasonable shape of D0, price setter is extreme

%     § implication: low price => price setter extreme => most points will move => low acc
%         □ market is curse!
% Ø revenue derivative?
% ○ result: can work if b is highly skewed - then price setter will be less extreme

%     § but - this requires b to be very skewed; ratio of max_b/min_b can be huge
%     § take away: naïve h works well under market if separates + v skewed for negs
%     § social: acc will be high only if inequity is also high (within negative points)


%========================================================================
\begin{figure*}[t!]
\centering
\includegraphics[width=\textwidth]{b_pow_u-crop.pdf}
\caption{%
\textbf{Demand under varying budgets.}
When budgets $\budget$ decrease as demand for units $u$ grows,
price setting points become less extreme.
However, this effect is mild, and only very high inequity (Gini$\approx$1)
helps to suppress mass crossing.
\squeeze
}
\label{fig:b_pow_u}
\end{figure*}
%========================================================================
    

\subsection{Typical market behavior} \label{sec:fixed_h}
Given a classifier $h$, how will the market respond?
Let $q$ be the distribution over demand-budget pairs $(u,\budget)$ induced by $h$.
By Sec.~\ref{sec:analysis}, $q$ fully determines prices.
For our analysis we will focus on $q$ that are simple, parametric, and well-behaved.
We first consider uniform budgets,
and then move to heterogeneous budgets that vary across users.
% \squeeze

\paragraph{Uniform budgets.}
When $b=1$ for all users, and so $\ubar=u$, users are naturally `ordered' by their demand.
From Sec.~\ref{sec:analysis}, this means that if $u^*$ is the revenue-maximizing point
(i.e., the price setter),
then all users with $u \le u^*$ move, and all others do not.
The main question is therefore: how \emph{many} users will move?
Fig.~\ref{fig:beta} shows revenue curves for several demand distributions
from the expressive $\mathrm{Beta}$ family and scaled to $[1,10]$.
The figure also shows for each distribution the price setter $u^*$
and the percentage of points that cross (i.e., all $u \le u^*$).
Although the distributions are quite diverse in shape,
market prices are typically low,
and price-setters lie mostly in extreme upper quantiles.
As a result, \emph{almost all points cross},
with over $95\%$ in most cases.
This effect is robust across many distributions---see Appendix~\ref{appx:empirical}.
\squeeze

To gain some intuition as the underlying reason, % for this phenomenon,
the following result provides a simple sufficient condition:
% The following result provides some intuition:
\squeeze
\begin{theorem}
\label{thm:unique_rev_argmax}
Let $q_f(u)$ be a demand distribution with pdf $f(u)$.
Then if the function $f(u)u$ is either
strictly increasing, decreasing, or unimodal, it holds that:
% single-peaked, we have:
\begin{enumerate}[leftmargin=2.2em,topsep=0em,itemsep=0.1em]
\item There is a unique revenue-maximizer $u^*$
\item Let $\umax = \argmax_u f(u) u$, then $u^* \geq \umax$
\end{enumerate}
\end{theorem}
Since $f(u)u$ is unimodal under any log-concave $f$,
Thm.~\ref{thm:unique_rev_argmax} applies to many known distribution classes.%
\footnote{This includes: Normal, uniform, exponential, logistic, Laplace, Gamma, Beta, Weibull, Gumbel, Rayleigh, and Chi$^2$ distributions.}
Appendix~\ref{appx:theory} includes a proof and an in depth analysis of some examples.


% as well as a discussion on why this property holds for most natural distributions.

% phenomenon holds broadly,
% and provide a theoretical motivation for the underlying reason.

% \todo{new theory: conditions for unique max rev price, intuition for why price setters are extreme}



\paragraph{Correlated budgets.}
When users vary in budgets (and so $u \neq \ubar$),
this can be thought of as `distorting' demand by scaling units as
$u \mapsto \frac{1}{\budget} u$.
Note this means that far-away points can now be closer, and close points can move far away, depending on $\budget$.
Potentially, this can lead to less extreme price setters if
the distribution becomes concentrated around smaller values;
this effect occurs mildly for $\mathrm{Beta}(0.5,2)$ in Fig.~\ref{fig:beta},
which is left-skewed.
Because demand is now over $\ubar=u/b$,
then if we think of $b$ as a function of $u$,
demand will be skewed if $b$ is sub-linear in $u$,
since this will ``push'' larger $\ubar$ increasingly further.
If this negative correlation is sufficiently strong,
then market prices should be higher, and we can hope for fewer points to cross.
% \squeeze
Fig.~\ref{fig:b_pow_u} shows revenue, prices, prices-setters, and the percentage of crossers for $b = u^{-\alpha}$ with \blue{$\alpha \in \{0,1,2,\dots,32\}$}.
Here we use Gaussian $u$ scaled to $[1,2]$,
so that $\bmin=1$ for the smallest $u$,
and $\bmax=2^{-\alpha}$ for the largest.
Results show that increasing $\alpha$ does shift the price setter,
and reduces the number of crossers. %, and hence results in better accuracy.
However, this requires $\alpha$ to be large, and even for $\alpha=16$,
\blue{30\%} of points still move.
% there is a 30\% reduction in accuracy.
\squeeze



%========================================================================
\begin{figure}[t!]
\centering
% \includegraphics[width=\columnwidth]{synth-semi.tmp.png}
\includegraphics[width=\columnwidth]{thresh.v2-crop.pdf}
\caption{%
\textbf{Varying threshold.}
For a simple mixture distribution of two class-conditional Gaussians,
$p(x\,|\,y)=\N(y\mu,\sigma)$, varying the threshold $\thresh$ results in surprising outcomes under induced market responses.
For uniform budgets (top left), there is no good solution.
When inequity in budgets $\budget$ is moderate (top right),
accuracy jumps to 1 once a critical point is reached.
When it is low (bottom left), this occurs only at a small interval.
The bottom right plot shows how increased inequity distorts
the demand distribution, causing the price setter to `jump'
from an extreme point of $p(x\,|\,y=0)$
to that of $p(x\,|\,y=1)$, enabling accuracy $\approx 1$.
\squeeze
}
\label{fig:thresh}
\end{figure}
%========================================================================

    



\paragraph{Implications.}
If $h$ is such that most points are able to move, then this can have dire implications on predictive performance.
Because learning generally aims to separate points by their class $y$,
for any moderately accurate classifier
the majority of points that will participate in the market 
(i.e., have $\yhat=0$, and therefore $u>0$)
will be negative ($y=0$).
This means that for classifiers with high \emph{pre}-market accuracy,
we can expect performance to drop to as low as $\sim$50\% \emph{after} the market forms.
% \squeeze
% \blue{This effect is quite robust across many standard distributions;
% for more examples see Appendix~\toref.}
This ill effect can be somewhat mitigated if budgets correlate with distance to $h$,
but \emph{only if inequity is extremely high} within negative points.
Fig.~\ref{fig:b_pow_u} (right) shows the relation between the ratio of crossers and inequity in budgets,
measured in Gini units, as a function of $\alpha$;
in our example, high accuracy is possible only when the gap 
between the lowest and highest budgets is an extreme $2^{32}$-fold.
\squeeze



% %========================================================================
% \begin{figure}[b!]
% \centering
% \includegraphics[width=\columnwidth]{b_corr_y-crop.pdf}
% \caption{
% \todo{...}
% }
% \label{fig:b_corr_y}
% \end{figure}
% %========================================================================


\subsection{Market-aware thresholds} \label{sec:market-aware_thresh}
Strategic movement by negative points is harmful to accuracy;
but positive points that move are actually \emph{helpful}
since they correct the classifier's mistakes.
In standard strategic classification, a useful strategy that exploits this idea
% reduce the number of negative movers and increase the positives
is to `raise the bar' by increasing $\thresh$ for a given $h=h_{w,\thresh}$.
Unfortunately, this idea does not easily transfer to a market setting,
because \emph{prices adapt to changes in $\thresh$}.
\extended{For example, if a change results in scaling demand as
$u \mapsto \alpha u$ for some $\alpha \in \R$,
then prices simply scale inversely,
% as $\p^* \mapsto \frac{1}{\alpha} \p^*$,
and outcomes remain unchanged;
in other words, we simply changed currency.}
Nonetheless, varying $\thresh$ for a given $h$ can still have a positive effect.

Our next construction allows to accommodate for changes in $\thresh$.
Let $h$, and w.l.o.g. assume $h=h_{w,0}$.
Define $p(z)$ as the induced distribution of distances $z$
from points $x$ to the decision boundary of $h$.
For any $\thresh$, we can express the marginal over units as
$q_\thresh(u) = p(z \,|\, z \le \thresh)$.
We can now ask how $\thresh$ shapes demand.
Our next example sets $p(z)$ to be a mixture of two class-conditional Gaussians $p(z \,|\, y)$,
where $p(z \,|\, y=0)$ is scaled to $[-1,0)$ 
and $p(z \,|\, y=1)$ is scaled to $(0,1]$.
Fig.~\ref{fig:thresh} shows prices, accuracy, and the ratio of crossers per class
for the range $\thresh \in [-1,5]$.
When budgets are uniform ($b=1$),
no threshold obtains accuracy above 55\%
\emph{despite the data being separable}.
This is because increasing $\thresh$ causes prices to decrease and remain low enough so that almost all points cross; essentially the same effect of Sec.~\ref{sec:fixed_h}.
However, when $b$ negatively correlates with $z$---even mildly---then it becomes possible to achieve high accuracy:
for $b$ that increases linearly with $z$ from \blue{$\bmin=1$ to $\bmax=5$,}
we see that once \blue{$\thresh \approx 0.75$},
accuracy abruptly jumps from \blue{$\sim$0.5 to $\sim$0.95}.
This is because at this threshold, the price setter shifts from being
an extreme point of $p(z\,|\,y=-1)$ to an extreme point of $p(z\,|\,y=1)$
(see ridgeline plot).
Note that this holds for \emph{any} $\thresh \ge 0.75$;
here the adaptivity of prices plays in favor of learning and provides robustness to the choice of $\thresh$. %
\extended{\footnote{This is in contrast to standard strategic classification which requires a particular $\thresh$ and can be highly sensitive to its choice.
}}%
Interestingly, for a smaller $\bmax=3$,
we get that accuracy is high only for \blue{$\thresh \in [1,2]$};
once $\thresh$ becomes too large, the price returns to be set by an extreme negative point.
\squeeze
% Thus, market classifiers can be robust to the choice of $a$,
% but are prone to quick transitions,
% and in a way that depends on $b$.

\extended{
\todo{implications?}
}


%========================================================================
\begin{figure}[t!]
\centering
% \includegraphics[width=\columnwidth]{synth-strat.v3.tmp.png}
\includegraphics[width=\columnwidth]{2D.v2-crop.pdf}
\caption{%
\textbf{Market classifiers.}
Consider a distribution where $x_1$ enables class separation,
and $x_2$ is uninformative of $y$.
A \naive\ classifier that uses only $x_1$ is unable to prevent negative points from crossing, and attains low accuracy (left).
In contrast, a market-aware classifier that uses $x_2$ is able to capitalize on the variation in budgets to classify well (right).
\squeeze
}
\label{fig:2D}
\end{figure}
%========================================================================

    
    

\subsection{Market-aware classifiers} \label{sec:market-aware_clsfr}
In terms of the market, control over $h$ allows the learner to `choose'
which demand distribution $q$ to work with:
the choice of $w$ determines $p(z)$, and $\thresh$ induces $q(u)$.
This gives learning much power over which users will be in the market,
as well as which of them will cross.
Interestingly, an $h_{w,\thresh}$ that is effective on the induced market
need not be accurate on raw data $(x,y) \sim \dist$.
For example, $w$ can focus weight on features that are entirely uninformative of $y$.
Our next construction demonstrates an extreme version of this idea.

Let $d=2$, and consider $(x_1,x_2) \sim D$
composed of per-feature class-conditional Gaussians $\dist(x_i \,|\, y)$.
The first feature is $x_1 \sim \N(\mu y, \sigma)$,
which allows to separate the classes (we use $\mu=1$ and $\sigma=0.15$).
The second feature is $x_2 \sim \N(0, \sigma)$,
i.e., has the same distribution under both classes,
as so is by definition  inseparable.
We set $\dist(y=0)=0.75$ and $\dist(y=1)=0.25$.
Here we let $\budget$ depend on labels as $\budget = 1+4y$.
Fig.~\ref{fig:2D} shows the behavior of two classifiers on this data:
$\hnaive$ which uses only $x_1$ \blue{(left)},
and $h_{\mathrm{market}}$ which uses only $x_2$ \blue{(right)}.
The idea of $\hnaive$ follows that of Sec.~\ref{sec:market-aware_thresh}:
separate the raw data well, and then tune $\thresh$ on the market.
Here we see that this approach breaks down completely:
the best it can achieve is \blue{$0.29$} accuracy,
since it cannot prevent the bulk of negatives from crossing.
% this by classifying all positives correctly, but 
In contrast, $\hmarket$ exploits the market to \emph{create} separability
over the otherwise ineffective $x_2$.
This is the optimal market-aware classifier.
The correlation between $\budget$ and $y$ results in a demand distribution
$q$ which clusters the positive $\ubar$ close to $h$,
and pushes negative $\ubar$ sufficiently far.
This results in almost only positive points crossing, and accuracy reaches \blue{0.96}.
In Appendix~\ref{appx:additional_experiments} we show this effect can be even more extreme.
\squeeze

\extended{
\todo{implications? learning prefers inequity in budgets; b not in input to h, if b corr y then learning can elicit y by incentivizing users to invest effort and `reveal' their y. if b anti-corr y, then this makes separation impossible (the market is not symmetric - can't just use 1-h, since users want and move towards yhat=1)}
}

% \green{\rule{\columnwidth}{0.5em}}
%==========================================================================

% \subsection{Market response to non-strategic classifiers} \label{sec:synth_non-strat}
% Consider first a classifier $h$ that has been trained {\naive}ly on some data, i.e., without accounting for strategic effects.
% How will the market respond?
% % Let $z \in \R$ be the distance from $x$ to $h$,
% Denote by $\dist(z\,|\,y)$ the induced class-conditional distributions over distances from $h$ for each $y \in \{\pm 1\}$.
% % For simplicity, assume the class-conditional distributions are Gaussian, namely $p(z\,|\,y) = \mathcal{N}(y,\sigma)$ for some $\sigma$.
% If $h$ is reasonably accurate, then by definition
% $\dist(z\,|\,y=-1)$ and $\dist(z\,|\,y=1)$ should be well-separated. 
% But since users are strategic, data points with $z<0$ will seek to cross the decision boundary.
% In standard strategic classification, the set of points that cross are precisely those that are within some fixed, predetermined distance.
% In contrast, in a market setting, which points cross (and which don't)
% depends on \emph{all} distances---and on budgets.
% \squeeze



% % %========================================================================
% % \begin{figure*}[t!]
% % \centering
% % \includegraphics[width=\textwidth]{synth-semi+strat.tmp.png}
% % \caption{%
% % \todo{...}
% % }
% % \label{fig:synth-semi+strat}
% % \end{figure*}
% % %========================================================================

% % %========================================================================
% % \begin{figure*}[t!]
% % \centering
% % \includegraphics[width=\textwidth]{synth-strat.v2.tmp.png}
% % \caption{%
% % \todo{...}
% % }
% % \label{fig:synth-strat}
% % \end{figure*}
% % %========================================================================
    
 
% \paragraph{Uniform budgets.}
% When budgets are uniform, $b=1$,
% and assuming the raw data is separated, then demand for units can be associated with the distribution of negative points,
% $\ubar = u = -z$ for $z \sim \dist(z \,|\, y=-1)$.
% The key question here is how many of these points cross---%
% since this will directly translate to a reduction of accuracy.
% % The questions are: (i) what are optimal prices;
% % (ii) which points become price setters;
% % and, as a result, (iii) what portion of points from each class cross.
% % Since separation means mostly negative points can cross,
% % this translates to the reduction in accuracy due to the market.
% % \squeeze
% Fig.~\toref\ shows revenue curves for several demand distributions
% from the expressive $\mathrm{Beta}$ family and scaled to $[1,10]$.
% The figure also shows for each distribution the revenue-maximizing point $u^*$,
% which is the price-setter, and the percentage of points that cross,
% which in this case are all $u \le u^*$.
% Although the distributions are quite diverse in shape,
% market prices are typically low,
% and price-setters lie mostly in extreme upper quantiles.
% As a result, \emph{almost all points cross},
% with over $95\%$ in most cases.
% The implications for learning are dire:
% a classifier that is perfectly accurate on non-strategic data
% reduces to $\sim$50\% accuracy on its induced market.
% This effect is quite robust across many standard distributions;
% for more examples see Appendix~\toref.


% \paragraph{Correlated budgets.}
% Separating the data can still be a reasonable approach if budgets vary in a way that moderates excessive movement.
% Intuitively, price setters should be less extreme if the distribution is concentrated around smaller values;
% in Fig.~\toref, this occurs (mildly) for $\mathrm{Beta}(0.5,2)$,
% which is left-skewed.
% Because demand is in normalized units $\ubar=u/b$,
% then if we think of $b$ as a function of $u$,
% demand will be skewed if $b$ is sub-linear in $u$,
% since this will ``push'' larger $\ubar$ increasingly further.
% If this negative correlation is sufficiently strong,
% then market prices should be higher, and we can hope fewer points will cross.

% Fig.~\toref\ shows revenue, prices, prices-setters, and the percentage of crossers for $b = u^{-\alpha}$ with \blue{$\alpha \in \{0,1,2,\dots,32\}$}.
% Here we use Gaussian $u$ scaled to $[1,2]$,
% so that $\bmin=1$ for the smallest $u$,
% and $\bmax=2^{-\alpha}$ for the largest.
% Results show that increasing $\alpha$ does shift the price setter and reduces the number of crossers, and hence results in better accuracy.
% However, this requires $\alpha$ to be large, and even for $\alpha=16$
% there is a 30\% reduction in accuracy.
% The implication is that $h$ which are accurate on non-strategic data
% can preserve this accuracy under the market \emph{only if inequity is extremely high} within negative points; in our example, this requires a $2^{32}$-fold gap between the lowest and highest budgets.

\extended{%
\red{%
- "market is a curse" \\
- remember h is indep of b, takes only x as input \\
- hence, choice of (naive) h agnostic to b
% - take away: naïve h works well under market if separates + v skewed for negs
% - social: acc will be high only if inequity is also high (within negative points)
}

% \todo{Gini?}



% \subsection{\Naive\ classifiers with market-aware thresholds}
% \subsection{Market-aware thresholds}

% % 2. strategic threshold
% % ○ standard sc works by separating data and then "raising the bar" (by fixed, known size)
% % ○ will this work for market h?
% % ○ idea: given h that separates, tune thresh strategically (semi-strat)
% % ○ crux: in markets, prices are adaptive
% % ○ ask: how do they adapt? <- negative result
% %     § scaling of w (together with thresh) doesn't matter, just "currency"
% %     § uniform b doesn't help:
% %         □ can't do better than acc=0.5! (see plot below)
% %         □ prices always counter, price setter remains extreme
% % ○ ask: what can we do? <- positive result
% %     1. margins - ?
% %     2. b corrs with x - what matters is ubar!
    
% %         □ can be very mild - here b(u) is linear, with small max_b/min_b (vs. prev result for naïve h)
% %         Ø show plot with acc under optimal market h as function of b strength?
% % • market is blessing!
% % • combined?

% In standard strategic classification, a useful approach is to first find a classifier $h$ with high accuracy on non-strategic data, and then `raise the bar' by increasing $a$.
% If set correctly, then this restores accuracy by
% making it harder for negative points to cross, while incentivizing positives to make the effort.
% Unfortunately, this idea does not easily transfer to a market setting,
% because \emph{prices adapt to changes in $a$}.
% \blue{For example, if a change results in scaling demand as
% $u \mapsto \alpha u$ for some $\alpha \in \R$,
% then prices simply scale inversely,
% % as $\p^* \mapsto \frac{1}{\alpha} \p^*$,
% and outcomes remain unchanged;
% in other words, we simply changed currency.}
% Nonetheless, varying $a$ for a given $h$ can still have a positive effect.

% Fig.~\toref\ shows prices, accuracy, and the percentage of crossers (per class) for a range of thresholds $a \in [-1,5]$.
% Here we use Gaussian $z$ scaled to $[-1,0)$ for negatives and to $(0,1]$ for positives.
% When budgets are uniform ($b=1$),
% no threshold obtains accuracy above 55\%---despite the data being separable.
% This is because increasing $a$ causes prices to decrease and remain low enough to permit
% almost all points cross; essentially the same effect observed in Sec.~\ref{sec:synth_non-strat}.

% However, when $b$ negatively correlates with $z$, even mildly, then it becomes possible to achieve high accuracy:
% for $b$ that increases linearly from \blue{$\bmin=1$ to $\bmax=5$,}
% we see that once \blue{$a \approx 0.75$},
% accuracy abruptly jumps from \blue{$\sim$0.5 to $\sim$0.95}.
% This is because at this threshold, the price setter shifts from being
% an extreme point of $\dist(z\,|\,y=-1)$ to an extreme point of $\dist(z\,|\,y=1)$.
% Note that this holds for \emph{any} $a \ge 0.75$;
% here the adaptivity of prices plays in favor of learning and provides robustness to the choice of $a$.%
% \footnote{This is in contrast to standard strategic classification which requires a particular $a$ and is highly sensitive to its choice \tocite.}
% Interestingly, for a smaller $\bmax=3$,
% we get that accuracy is high only for \blue{$a \in [1,2]$};
% once $a$ becomes too large, the price returns to be set by an extreme negative point.
% % Thus, market classifiers can be robust to the choice of $a$,
% % but are prone to quick transitions,
% % and in a way that depends on $b$.


\red{%
- cluster behavior/hypothesis - appendix? \\
- ridgeline plot? \\
- market is blessing! \\
- social implications / inequity
}

\todo{revenue derivative}



    

% \subsection{Market-aware classifiers}

% % • strategic classifier
% % ○ don't necessarily need h that separates data well
% % ○ instead, gain power from b
% % ○ observation: b (in dir of w) "reorders" u to become ubar
% %     § can potentially make non-sep data become separable
% % ○ one way: b (in dir of w) corr with y
% % ○ (flip side: if b is noisy (in dir of w), then can "ruin" separable u)

% Our analysis so far suggests one way to obtain effective market classifiers:
% find $h_{w,a}$ that separates the data well,
% make sure budgets increase along the positive direction of $w$,
% and tune $a$ to ensure that points that cross are mostly positive.
% But in some cases, budgets
% % if budgets are in themselves informative of labels $y$, they
% can be even more useful for classification than separable features.
% % bypassing the need for separability.



% budgets can greatly influence---for better or worse---the performance of classifiers that are otherwise accurate on non-strategic data.

% Since $b$ is not an input to $h$, 


\red{- overlapping dists - connection to signaling theory? positives pay more to make themselves distinct}

\todo{%
- empirical prices: funky, convergence \\
- recursive construction; min set, max set \\
- revenue derivative
}
}