\section{Discussion}

The use of learned models to inform decisions about humans
has become common practice.
% But while the designers of the system may wish to improve accuracy,
% users of the system may have other interests.
% Our key point is that 
But when those very humans also take interest in prediction outcomes, 
conventional learning tools no longer necessarily apply.
% maximizing accuracy {\naive}ly will not likely be a useful strategy.
This paper advances the idea that when users seek to obtain certain predictions,
learning inevitably becomes a driver of demand.
When this creates an opportunity for profit, 
it is only natural to expect that a market will form.
Learning classifiers that induce markets poses unique challenges as a learning task.
Our paper takes a first step to address these,
and so targets a particular market setting 
and pursues a basic understanding of it.
% to establish basic claims.
But there is of course a plethora of other market settings to explore at this new intersection of machine learning and markets.
Nonetheless, the idea that learning can drive economic outcomes has broader implications to consider.
One example is the question of how learning influences social welfare, as it relates e.g. to market efficiency.
Another example is the question of information asymmetry and the capacity of learning to exploit its informational advantage.
Given the growing influence of learning on our lives,
such questions merit careful thinking and much deliberation.
Our hope is therefore not only to spark interest,
but to also motivate discussion on these important and timely topics.

\extended{%
\red{
    - non-linear prices \\
    - other markets: limited supply (fisher), users also sell (arrow-debreu), indivisible (=discrete) goods \\
    welfare: are induced markets efficient, so maximize social welfare? but this is per given classifier - can we learn $h$ that promotes higher welfare?
}
}

\section*{Impact Statement}
Our paper sets out to study the interplay between learning classifiers and the markets this process can facilitate.
We believe that the impact of prediction on economic outcomes can be significant and widespread when machine learning tools are used in social contexts.
In the market model we propose, the choice of classifier is modeled as affecting both users and sellers: it inadvertently determines who must invest to be classified as positive (i.e., receive the loan or get the job), what this will cost, and which sellers will profit.
These forces arise naturally through how the market coordinates supply and demand.
But whereas the mechanics of conventional markets are well understood both in theory and practice, we believe that the role of learning in markets, and the impact that learning can have, has so far been insufficiently explored.

An understanding of how learning creates and affects markets can be used to advance efficient and fair trade, foster equal opportunity, and promote social welfare.
It can also be used to gain insight as to how learning-driven markets should be regulated and by what means.
But such knowledge and tools should be used with care,
as they can potentially serve to drive markets to undesired outcomes.
One example is economic inequity, which can be exacerbated by learning,
as our results suggest can happen.
Another example is information asymmetry:
Our stylized market model assumes perfect information and efficient prices.
But in a reality where learned models have access to an unparalleled amount of data---certainly more than is accessible to users or sellers---the learning system gains a distinct informational advantage.
It is widely recognized that such settings can lead to the exploitation of consumers and even to market collapse \citep[e.g.,][]{akerlof1978market}.
We hope that our work serves to encourage fruitful discussion on these important topics.

It is also important to note that the market model we study is simple and draws on many assumptions, such as unlimited supply, a fixed number of exclusive sellers, and no externalities.
Results regarding market outcomes, both theoretical and empirical, should therefore be taken under this light.
At the same time, we hope this motivates researchers in both machine learning and economics to deepen our understanding of learning and markets 
in broader and more realistic economic settings.