When learning is used to inform decisions about humans,
such as for loans, hiring, or admissions,
this can incentivize users to strategically modify their features
to obtain positive predictions.
A key assumption is that modifications are costly,
and are governed by a cost function that is exogenous and predetermined.
We challenge this assumption, and assert that
the deployment of a classifier is what \emph{creates costs}.
Our idea is simple:
when users seek positive predictions, 
this creates demand for important features;
and if features are available for purchase,
then a market will form, and competition will give rise to prices.
We extend the strategic classification framework to support this notion,
and study learning in a setting where a classifier can induce a market for features.
We present an analysis of the learning task,
devise an algorithm for computing market prices,
propose a differentiable learning framework,
and conduct experiments to explore our novel setting and approach.
\squeeze