% \section{\blue{Analysis}}
\section{Market Prices: Analysis and Algorithm}
\label{sec:analysis}

Optimizing Eq.~\eqref{eq:learning_objective_exp} requires the ability to anticipate how users respond to the market.
By Eq.~\eqref{eq:br_market}, this can be achieved by computing induced prices $\p^h$.
Our first task is therefore to compute market prices for a given $h$.
We begin with analyzing the market,
and then give an exact pricing algorithm.

% - mix of theory and empirics

% \subsection{On prices, demand, and user behavior}

\paragraph{How users respond to prices.}
% (0) fixed h, fixed price:
% - single point - how it moves
% - linear cost; vs. hardt
Given a classifier $h$ and a general price vector $\p$,
how will users behave?
To gain insight,
notice that computing $x^h$ in Eq.~\eqref{eq:br_market} can be broken down into three steps:
First, compute $\yhat = h(x)$, and proceed only if $\yhat=-1$.
If so, then second, find the least-costly $\delta$ that gives a positive prediction, this by solving the LP:%
\footnote{Since users minimize costs, we can use an equality constraint.}
% \squeeze
\begin{equation}
\label{eq:br_lp}
\delta_* = \argmin_{\delta \ge 0} \delta^\top \p
\quad \text{s.t.} \quad
w^\top (x + \delta) + \thresh = 0
\end{equation}
Third, apply $x^h = x+\delta_*$ iff budget permits,
i.e., $\delta_*^\top \p \le \budget$.

To understand $\delta_*$,
consider a change of variables in Eq.~\eqref{eq:br_lp}
using $z_i = \delta_i w_i$.
The LP can now be rewritten as:
\begin{equation}
\label{eq:br_lp_normalized}
\argmin_{z \ge 0} \sum_{i=1}^d \frac{p_i}{w_i} z_i
\quad \text{s.t.} \quad
\sum_{i=1}^d z_i = \kappa_x
\end{equation}
where the constant $\kappa_x = - w^\top x - \thresh$ is non-negative for
relevant points (i.e., having $\yhat = -1$).
% Solving Eq.~\eqref{eq:br_lp_normalized} amounts to 
This provides an alternative interpretation:
the user must allocate $\kappa_x$ mass across features,
using $z$, 
to minimize total cost-per-value,
where `values' correspond to entries of $w$.
% in the most cost-effective manner.
% given by the ratio $p_i/w_i$.
% where `cost' is due to $p_i$ and `effective' is by $w_i$.
This has a simple solution, which is to set $z_i=\kappa_x$
if $i$ attains the minimal ratio $p_i/w_i$, and $0$ otherwise.
% If we consider $z$ as an optimal allocation of $\kappa$ mass,
% then the solution is obtained by placing all mass on the entry $i$ for which $p_i/w_i$ is minimal, that is, $z_i = \kappa$ and 0 otherwise.
If multiple features attain the minimum,
then these features are substitutable,
and so any allocation of $z$ among them is equivalently optimal.
% and hence any combination of such features (in sufficient quantity) is optimal.
In other words, users will purchase only the most cost-effective features,
% (and so will move only along those dimensions),
but are indifferent within this set of features.
% $i$ (where `cost' is by $p_i$ and `effective' is by $w_i$),
% and so will move only along those dimensions.
% If there is a tie among several features,
% then the user will be equally happy with (a sufficient quantity of) any combination of them.
\squeeze

\extended{%
\todo{give dual as alternative explanation?}
}

\paragraph{How prices adjust to demand.}
Given the above, we next consider how sellers should set prices.
Notice how by Eq.~\eqref{eq:br_lp_normalized},
all user decisions depend on the ratios $p_i/w_i$,
and differ only in the constant $\kappa_x$.
Since all users buy only the most cost-effective features,
any $s_i$ whose ratio is \emph{not} minimal will receive zero market share.
Sellers therefore compete over who attains the minimal $p_i/w_i$.
Since sellers in our setting have no capacity constraints or production costs,
we assume sellers have foresight and so coordinate to prevent price collapse.%
\footnote{This is essentially Bertrand's paradox, for which we invoke the folk theorem to enable the formation of cooperative equilibrium.}
This gives the equilibrium condition:
\squeeze
\begin{equation}
\label{eq:equilibrium_ratio}
\forall \, i \in [d], \qquad
\frac{p_i}{w_i} =  \rho^* >0
\end{equation}
for $\rho^*$ that admits maximal total revenue, $\rev = \sum_i \rev_i$.
This implies a tight connection between the classifier and prices:
\begin{proposition}
Let $h(x)=\sign(w^\top x + \thresh)$ be a linear classifier,
then market prices $\p^*$ are proportional to $w$, namely:
\begin{equation}
\label{eq:scalar_equilibrium_price}
\p^*(h;D) = \rho^* \cdot w % (w_1, \dots, w_d)
% \p^h = \rho^h \cdot w % (w_1, \dots, w_d)
\end{equation}
for some $\rho^* \in \R_+$ which also depends on $h$ and $\dist$.
\end{proposition}
% \blue{For a particular $h$, we will use $\rho^h = \rho^*$. \todo{used?}}


% In principle, different users can have very different demand sets.
% However, notice that by Eq.~\eqref{eq:br_lp_normalized},
% users differ in practice only by the constant $\kappa$, since only this depends on $x$.
% This results from users preferences being homogeneous.
% The implication is that users differ only in the \emph{quantity} they seek,
% and so demand sets differ only in scale.
% Hence, what matters to sellers are only the 
% ratios $p_i/w_i$,
% where each seller $i$ controls its own $p_i$, and depends on all others.
% Since users will buy only the most cost-effective features,
% any $s_i$ whose ratio is \emph{not} minimal will receive zero market share.
% Hence, equilibrium is possible only if $p_i/w_i=\rho$ for some $\rho \in \R$ and for all $i$.

% Consider such $\rho$, and assume all ratio equalities hold.
% Clearly, no seller has incentive to raise her price, since this can only reduce her market share.
% However, each seller has incentive to slightly lower the price,
% since this would undercut all competitors and capture the entire market.
% Since sellers in our setting have no capacity constraints or production costs,
% one possible outcome is that prices spiral down towards zero---essentially Bertrand's paradox. To circumvent this, we will invoke the Folk Theorem, and assume sellers have foresight and so coordinate to prevent price collapse and maintain prices at the cooperative equilibrium:



% \subsection{Computing empirical market prices} \label{sec:exact_algo}
\paragraph{Computing empirical market prices.}
Given a classifier $h$ and sample set $\smplst =\{(x_i,\budget_i,y_i)\}_{i=1}^m$,
we will be interested in computing revenue-maximizing market prices.
Because we only have a sample at hand,
our goal will be to compute optimal \emph{empirical market prices} $\phatvec$.
Applying Eq.~\eqref{eq:scalar_equilibrium_price} to 
the empirical distribution over $\smplst$,
we get that $\phatvec = \rhohat w$ for the scalar $\rhohat$ 
which maximizes total empirical revenue, denoted $\rhat$.
This is highly useful,
since the problem of computing equilibrium for the empirical market 
under a given $h$ reduces to optimizing over scalars $\rho \in \R$.

\extended{%
\todo{describe this `as if' a single seller/item}
}

% This will form the basis of our learning approach in Sec.~\toref.

\extended{%
\blue{The connection between $\phatvec$ and $\p^*$ is discussed in Sec.~\toref.}
}

% Intuitively, the challenge in finding the optimal $\rhohat$ is that if we set $\rho$ too high,
% then some users will not buy at all, and revenue will be lost;
% whereas if $\rho$ is too low, then many users will buy---but at suboptimal profit.
% The trick is therefore to balance \emph{who} buys with \emph{how much} they will pay;
% the first step to which is understanding what users actually want.


%=====================

\begin{algorithm}[tb]
\caption{Exact empirical market prices}
\label{algo:exact_market_prices}
\begin{algorithmic}[1]
    \STATE \textbf{input:} classifier $h_{w,a}$, sample set $S=\{(x_i,\budget_i,y_i)\}_{i=1}^m$
    \STATE {\textbf{initialize:} $\rev = 0$ (revenue), $U = 0$ (total units sold)}
    \FOR{$i=1,\dots,m$}
        \STATE $u_i \gets \distance^+(x_i;h)$
        \STATE $\ubar_i \gets u_i / \budget_i$
    \ENDFOR
    \STATE $(\ubar_{(1)},\dots,\ubar_{(m)}) \gets \sort(\ubar_1,\dots,\ubar_m)$
       \FOR{$i=1,\dots,m$}
        \STATE $p_i \gets 1 / \ubar_{(i)}$
        \STATE $U \gets U + \ubar_{(i)}$
        \IF{$p_i U > \rev$}
            \STATE $\rev \gets p_i U$
            \STATE $\phat \gets p_i$
        \ENDIF
    \ENDFOR
    \STATE \textbf{return:} $\phatvec = \phat \cdot w /\|w\|$
\end{algorithmic}
\end{algorithm}


%=====================



% \paragraph{Interpreting demand.}
% Recall that users seek $\delta$ which allow them to cross the decision boundary (Eq.~\eqref{eq:br_lp}).
By Eq.~\eqref{eq:scalar_equilibrium_price},
market prices align with the direction of $w$.
% it will suffice to consider demand as projected onto $w$.
% Since users are rational (and hence minimize costs),
% this has a simple interpretation:
The demand of a user is then the (directional) distance from $x$ to the decision boundary of $h$, measured in `units' $u \in \R_+$.
\squeeze
\begin{observation}
Let $h$ and $\smplst$, then for a given user $x$, 
and for any $\rho \in \R$,
her demand under prices $\p = \rho w$ is:
\begin{equation}
\label{eq:demand_as_distance}
u = \distance^+(x;h) = \max\left\{0, - \frac{w^\top x + \thresh}{\|w\|} \right\}
% u = \distance^+(x;h) = \max\left\{0,(w^\top x + \thresh) \|w\|^{-1} \right\}
\end{equation}
\end{observation}
% We refer to $u \in \R$ as the amount of \emph{units} that user $x$ demands.
Here the max over 0 ensures that demand is considered only for relevant users, i.e., for which $h(x)=-1$.
This transition to units of demand
lays the ground for our algorithm.
% forms the basis of our main result.
\squeeze

\paragraph{Exact aglorithm.}
Algorithm~\ref{algo:exact_market_prices} provides pseudocode for an algorithm that efficiently computes the optimal (scalar) prices $\rhohat$ for given $h$ and $\smplst$.
Our key observation is that it suffices to work with 
\emph{units-per-budget}, defined as $\ubar_i = u_i/\budget_i$ for each user $x_i$.
The first steps are therefore to project demand onto $w$,
obtain all $u_i$, and normalize by $\budget_i$ to get $\ubar_i$.
Correctness of the algorithm follows from the next result:

% \paragraph{Interpreting prices.}
% Our observation above suggests that to compute the optimal $\rhohat$,
% and in terms of demand,
% it suffices to consider as input the set of pairs
% $(u_i,b_i)$ for all $i \in [m]$.
% We can therefore frame the task as that of a single seller of `units' aiming to maximize its revenue by setting the optimal price $\rhohat$.
% Nonetheless, this is still challenging,
% since revenue remains a complex function even for a single item.


% Our next observation is that from this single seller's perspective,
% what matters is not only the amount of units each user needs,
% but also how much they are willing to pay for them.
% For revenue considerations, this means we can replace each pair $(u,b)$ with
% the ratio $\ubar = u/\budget$,
% which we refer to as \emph{units-per-budget}.%
% \footnote{To see this, consider for example that gains are the same if a user buys 10 units at price 1, or 5 units at price 2.}
% The final step is to notice that a user will purchase at a given price $\rho$
% if $u \rho \le b$, or alternatively, if $\rho \le \ubar^{-1}$.
% This forms the basis of our main result:
% \squeeze
\begin{theorem}
\label{thm:price_is_point}
Given (uni-dimensional) demand $\{(u_i,b_i)\}_{i=1}^m$,
the revenue-maximizing price is
$\rhohat = \ubar_{i^*}^{-1}$
for some $i^* \in [m]$.
\end{theorem}

\begin{proof}
It suffices to show that the set of all local maxima of revenue
(as a function of $\rho = u^{-1}$)
correspond exactly to the set of points $\{\ubar_i^{-1}\}_{i=1}^m$.  
Assume w.l.o.g. that $\ubar_i^{-1}$ are ordered.
Then for any interval $(\ubar_i^{-1},\ubar_{i+1}^{-1})$, revenue is linear in $\rho$;
this is since for all $\rho$ in this interval, the set of users that purchase are precisely $j=1,\dots,i$, each purchasing $u_j$ units at price $\rho$.
Next, notice that at any $\rho=\ubar_i^{-1}$, increasing $\rho$ infinitesimally causes $i$ to \emph{not} purchase, since she can no longer afford her required units $u_i$ at budget $\budget_i$, and revenue exhibits a sharp drop.
Thus, revenue is discontinuous peicewise linear with increasing segments between the $u_i^{-1}$.
\squeeze
% Notice that revenue can be written as $r = \rhohat \sum_{i | \ubar_{i^*}^{-1} >= \rhohat} \ubar_{i^*}^{-1}$.  If $\rhohat$ is not some  $\ubar_i^{-1}$,
% then increasing $\rhohat$ slightly increases the revenue since the sum remains the same and (if $\rhohat$ is higher then all points the revenue is 0).
\end{proof}



% The proof of Thm.~\ref{thm:price_is_point} relies on showing that
% all local minima of revenue (as a function of $\rho$) are characterized exactly by the set of points $\{\ubar_i^{-1}\}_{i=1}^m$.
% Intuitively, this is since for any $i$,
% revenue increases when prices grow beyond $\ubar_i^{-1}$
% (since the same amount of units is sold for a higher price),
% but drops abruptly when it crosses $\ubar_{i+1}^{-1}$
% (since user $i+1$ can no longer afford to buy her units, $u_i$).
% The global optimum is therefore one of these points, 
% and
% This idea is illustrated for a simple example in 

Fig.~\ref{fig:rev_vs_price} illustrates the structure of revenue as a function of price.
% for a small sample $\smplst$.
Thm.~\ref{thm:price_is_point} implies that
it suffices to compute $\rev$ at prices 
$\rho_i=1 / \ubar_i$ for all $i \in [m]$, choose the maximizing $i^*$,
and set $\rhohat = \rho_{i^*}$.
We will henceforth refer to the user $i^*$ as the \emph{price setter}.
Sorting by $\ubar_i$ makes this process efficient:
at price $p_i$, the set of point who purchase are precisely $j$ for which $\ubar_j \le \ubar_i$.
Since revenue at $i$ is  $p_i U = p_i \sum_{\ubar_j \le \ubar_i} \ubar_j$,
we can update $U$ on the fly as a cumulative count of total units bought,
and multiply by price,
giving runtime of $O(m \log m)$.
\squeeze


% Correctness is due to  Thm.~\ref{thm:price_is_point},
% and 
% The first steps are to project demand onto $w$,
% obtain all $u_i$, and normalize by $\budget_i$ to get $\ubar_i$.
% Since Thm.~\ref{thm:price_is_point} states that $\phat$ is obtained from \emph{some} $\ubar_i$,
% it suffices to compute revenue at prices $p_i=1 / \ubar_i$ for all $i \in [m]$, choose the maximizing $i^*$, and set $\phat = p_{i^*}$.
% Sorting by $\ubar_i$ makes this process efficient:
% at price $p_i$, the set of point who purchase are precisely $j$ for which $\ubar_j \le \ubar_i$,
% and since revenue at $i$ is  $\rev = p_i U = p_i \sum_{\ubar_j \le \ubar_i} \ubar_j$,
% we can update $U$ on the fly as a cumulative count of total units bought,
% and multiply by price.
% \squeeze




%========================================================================
\begin{figure}[t!]
\centering
\includegraphics[width=0.9\columnwidth]{rev_vs_price-crop.pdf}
\caption{
\textbf{Empirical revenue as a function of price.}
Revenue increases before each $\ubar_i^{-1}$
and drops immediately after,
implying the argmax is attained at some $i^* \in [m]$
(Thm.~\ref{thm:price_is_point}).
\squeeze
}
\label{fig:rev_vs_price}
\end{figure}
%========================================================================




\extended{%
\paragraph{\red{Market prices in the limit.}}
~\\
\todo{...}


\blue{%
Note $i^*$ need not be an extreme point in the ordering,
and maximal revenue can be at any intermediary point.
}

% \squeeze
% \sum_{j \le i} \ubar_{(j)}$.
% The overall runtime of Algorithm~\ref{algo:exact_market_prices} is $O(m \log m)$.
% \blue{Note that $i^*$ need not be an extreme point in the set;
% in fact, without any assumptions on $\Gamma$, revenue can be maximal at any point (or points).}
% \squeeze


\blue{%
This reveals how prices are agnostic to scale:
if we multiply all $u$ by some factor of $\alpha$,
then by Eqs.~\eqref{eq:scalar_equilibrium_price} and \eqref{eq:demand_as_distance},
market prices would adjust as $\frac{1}{\alpha} \rho^*$,
with no change in outcomes.
}

\red{%
- points with $u_i=0$? \\
- $\phat$ not unique!
}


\todo{better way to say that rev is not a "simple" function of $\ubar$? not increasing, decreasing, convex, concave, etc}


\todo{add proofs! here or in appx?}

\todo{can we generalize to non-linear costs? concave?}



\todo{actually about mapping back from $u$ to $\delta_*$!}
\blue{%
Mapping back from $\rho^*$ to vector prices can be made e.g. by assuming that ties between sellers are broken at random,
which gives $p_i = \rho^* w_i$.
Note this choice affects only how revenue is distributed across sellers,
whereas both users and the classifier remain agnostic to how ties are broken.
% Eq.~\eqref{eq:scalar_equilibrium_price} will prove helpful for computing $\p^*$ %in Sec.~\ref{sec:exact_algo},
% since solving the equilibrium reduces to finding the scalar $\rho^*$.
\squeeze
}
}

