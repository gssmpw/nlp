
%========================================================================
\begin{figure*}[t!]
\centering
\includegraphics[width=\textwidth]{synth-naive+semi.v2.tmp.png}
\caption{%
\todo{...}
}
\label{fig:synth-naive+semi}
\end{figure*}
% %========================================================================

% %========================================================================
% \begin{figure}[t!]
% \centering
% \includegraphics[width=\columnwidth]{synth-naive.tmp.png}
% \caption{%
% \todo{...}
% }
% \label{fig:synth-naive}
% \end{figure}
% %========================================================================



\section{Learning in Markets: Exploratory Insights}
Our goal in this section is to demonstrate
% through simple empirical examples,
the basic mechanics underlying how learning creates markets,
and how induced markets affect strategic learning and its outcomes.
Since the market for a classifier $h=h_{w,\thresh}$ is shaped by demand in the direction of $w$, our analysis will initially focus %w.l.o.g.
on unidimensional data
representing distances of points $x$ from the decision boundary of $h$,
denoted by $z \in \R$.
This will allow us to explore market behavior for a given direction $w$,
understand the role of the threshold $\thresh$,
and gain insight on what makes for effective market-aware $h$.

\todo{reframe as approaches - naive, semi-strat, strat?}

\todo{acc is high if negs stay and poses move}

% 1. what happens to naïve h in market setting?
% ○ opt naïve h likely to separate data => consider such data (=distribution, not sample)
% ○ market works  in direction of w => focus on 1D, eg two gaussians
% ○ opt h in the middle => focus on D0=D(x|y=0) - single gaussian "to the left" of h
% ○ in sc, under fixed cost, users move up to some distance
% ○ in market, this depends on b!
%     § remember h is indep of b, takes only x as input
%     § hence, choice of (naive) h agnostic to b
% ○ result: for uniform b=1, under almost any reasonable shape of D0, price setter is extreme

%     § implication: low price => price setter extreme => most points will move => low acc
%         □ market is curse!
% Ø revenue derivative?
% ○ result: can work if b is highly skewed - then price setter will be less extreme

%     § but - this requires b to be very skewed; ratio of max_b/min_b can be huge
%     § take away: naïve h works well under market if separates + v skewed for negs
%     § social: acc will be high only if inequity is also high (within negative points)



\subsection{Market response to non-strategic classifiers} \label{sec:synth_non-strat}
Consider first a classifier $h$ that has been trained {\naive}ly on some data, i.e., without accounting for strategic effects.
How will the market respond?
% Let $z \in \R$ be the distance from $x$ to $h$,
Denote by $\dist(z\,|\,y)$ the induced class-conditional distributions over distances from $h$ for each $y \in \{\pm 1\}$.
% For simplicity, assume the class-conditional distributions are Gaussian, namely $p(z\,|\,y) = \mathcal{N}(y,\sigma)$ for some $\sigma$.
If $h$ is reasonably accurate, then by definition
$\dist(z\,|\,y=-1)$ and $\dist(z\,|\,y=1)$ should be well-separated. 
But since users are strategic, data points with $z<0$ will seek to cross the decision boundary.
In standard strategic classification, the set of points that cross are precisely those that are within some fixed, predetermined distance.
In contrast, in a market setting, which points cross (and which don't)
depends on \emph{all} distances---and on budgets.
\squeeze



% %========================================================================
% \begin{figure*}[t!]
% \centering
% \includegraphics[width=\textwidth]{synth-semi+strat.tmp.png}
% \caption{%
% \todo{...}
% }
% \label{fig:synth-semi+strat}
% \end{figure*}
% %========================================================================

%========================================================================
\begin{figure}[b!]
\centering
\includegraphics[width=\columnwidth]{synth-strat.v3.tmp.png}
\caption{%
\todo{...}
}
\label{fig:synth-strat}
\end{figure}
%========================================================================
    

% %========================================================================
% \begin{figure*}[t!]
% \centering
% \includegraphics[width=\textwidth]{synth-strat.v2.tmp.png}
% \caption{%
% \todo{...}
% }
% \label{fig:synth-strat}
% \end{figure*}
% %========================================================================
    
 
\paragraph{Uniform budgets.}
When budgets are uniform, $b=1$,
and assuming the raw data is separated, then demand for units can be associated with the distribution of negative points,
$\ubar = u = -z$ for $z \sim \dist(z \,|\, y=-1)$.
The key question here is how many of these points cross---%
since this will directly translate to a reduction of accuracy.
% The questions are: (i) what are optimal prices;
% (ii) which points become price setters;
% and, as a result, (iii) what portion of points from each class cross.
% Since separation means mostly negative points can cross,
% this translates to the reduction in accuracy due to the market.
% \squeeze
Fig.~\toref\ shows revenue curves for several demand distributions
from the expressive $\mathrm{Beta}$ family and scaled to $[1,10]$.
The figure also shows for each distribution the revenue-maximizing point $u^*$,
which is the price-setter, and the percentage of points that cross,
which in this case are all $u \le u^*$.
Although the distributions are quite diverse in shape,
market prices are typically low,
and price-setters lie mostly in extreme upper quantiles.
As a result, \emph{almost all points cross},
with over $95\%$ in most cases.
The implications for learning are dire:
a classifier that is perfectly accurate on non-strategic data
reduces to $\sim$50\% accuracy on its induced market.
This effect is quite robust across many standard distributions;
for more examples see Appendix~\toref.


\paragraph{Correlated budgets.}
Separating the data can still be a reasonable approach if budgets vary in a way that moderates excessive movement.
Intuitively, price setters should be less extreme if the distribution is concentrated around smaller values;
in Fig.~\toref, this occurs (mildly) for $\mathrm{Beta}(0.5,2)$,
which is left-skewed.
Because demand is in normalized units $\ubar=u/b$,
then if we think of $b$ as a function of $u$,
demand will be skewed if $b$ is sub-linear in $u$,
since this will ``push'' larger $\ubar$ increasingly further.
If this negative correlation is sufficiently strong,
then market prices should be higher, and we can hope fewer points will cross.

Fig.~\toref\ shows revenue, prices, prices-setters, and the percentage of crossers for $b = u^{-\alpha}$ with \blue{$\alpha \in \{0,1,2,\dots,32\}$}.
Here we use Gaussian $u$ scaled to $[1,2]$,
so that $\bmin=1$ for the smallest $u$,
and $\bmax=2^{-\alpha}$ for the largest.
Results show that increasing $\alpha$ does shift the price setter and reduces the number of crossers, and hence results in better accuracy.
However, this requires $\alpha$ to be large, and even for $\alpha=16$
there is a 30\% reduction in accuracy.
The implication is that $h$ which are accurate on non-strategic data
can preserve this accuracy under the market \emph{only if inequity is extremely high} within negative points; in our example, this requires a $2^{32}$-fold gap between the lowest and highest budgets.

\red{%
- "market is a curse" \\
- remember h is indep of b, takes only x as input \\
- hence, choice of (naive) h agnostic to b
% - take away: naïve h works well under market if separates + v skewed for negs
% - social: acc will be high only if inequity is also high (within negative points)
}

\todo{Gini?}



% \subsection{\Naive\ classifiers with market-aware thresholds}
\subsection{Market-aware thresholds}

% 2. strategic threshold
% ○ standard sc works by separating data and then "raising the bar" (by fixed, known size)
% ○ will this work for market h?
% ○ idea: given h that separates, tune thresh strategically (semi-strat)
% ○ crux: in markets, prices are adaptive
% ○ ask: how do they adapt? <- negative result
%     § scaling of w (together with thresh) doesn't matter, just "currency"
%     § uniform b doesn't help:
%         □ can't do better than acc=0.5! (see plot below)
%         □ prices always counter, price setter remains extreme
% ○ ask: what can we do? <- positive result
%     1. margins - ?
%     2. b corrs with x - what matters is ubar!
    
%         □ can be very mild - here b(u) is linear, with small max_b/min_b (vs. prev result for naïve h)
%         Ø show plot with acc under optimal market h as function of b strength?
% • market is blessing!
% • combined?

In standard strategic classification, a useful approach is to first find a classifier $h$ with high accuracy on non-strategic data, and then `raise the bar' by increasing $a$.
If set correctly, then this restores accuracy by
making it harder for negative points to cross, while incentivizing positives to make the effort.
Unfortunately, this idea does not easily transfer to a market setting,
because \emph{prices adapt to changes in $a$}.
\blue{For example, if a change results in scaling demand as
$u \mapsto \alpha u$ for some $\alpha \in \R$,
then prices simply scale inversely,
% as $\p^* \mapsto \frac{1}{\alpha} \p^*$,
and outcomes remain unchanged;
in other words, we simply changed currency.}
Nonetheless, varying $a$ for a given $h$ can still have a positive effect.

Fig.~\toref\ shows prices, accuracy, and the percentage of crossers (per class) for a range of thresholds $a \in [-1,5]$.
Here we use Gaussian $z$ scaled to $[-1,0)$ for negatives and to $(0,1]$ for positives.
When budgets are uniform ($b=1$),
no threshold obtains accuracy above 55\%---despite the data being separable.
This is because increasing $a$ causes prices to decrease and remain low enough to permit
almost all points cross; essentially the same effect observed in Sec.~\ref{sec:synth_non-strat}.

However, when $b$ negatively correlates with $z$, even mildly, then it becomes possible to achieve high accuracy:
for $b$ that increases linearly from \blue{$\bmin=1$ to $\bmax=5$,}
we see that once \blue{$a \approx 0.75$},
accuracy abruptly jumps from \blue{$\sim$0.5 to $\sim$0.95}.
This is because at this threshold, the price setter shifts from being
an extreme point of $\dist(z\,|\,y=-1)$ to an extreme point of $\dist(z\,|\,y=1)$.
Note that this holds for \emph{any} $a \ge 0.75$;
here the adaptivity of prices plays in favor of learning and provides robustness to the choice of $a$.%
\footnote{This is in contrast to standard strategic classification which requires a particular $a$ and is highly sensitive to its choice \tocite.}
Interestingly, for a smaller $\bmax=3$,
we get that accuracy is high only for \blue{$a \in [1,2]$};
once $a$ becomes too large, the price returns to be set by an extreme negative point.
% Thus, market classifiers can be robust to the choice of $a$,
% but are prone to quick transitions,
% and in a way that depends on $b$.


\red{%
- cluster behavior/hypothesis - appendix? \\
- ridgeline plot? \\
- market is blessing! \\
- social implications / inequity
}

\todo{revenue derivative}



    

\subsection{Market-aware classifiers}

% • strategic classifier
% ○ don't necessarily need h that separates data well
% ○ instead, gain power from b
% ○ observation: b (in dir of w) "reorders" u to become ubar
%     § can potentially make non-sep data become separable
% ○ one way: b (in dir of w) corr with y
% ○ (flip side: if b is noisy (in dir of w), then can "ruin" separable u)

Our analysis so far suggests one way to obtain effective market classifiers:
find $h_{w,a}$ that separates the data well,
make sure budgets increase along the positive direction of $w$,
and tune $a$ to ensure that points that cross are mostly positive.
But in some cases, budgets
% if budgets are in themselves informative of labels $y$, they
can be even more useful for classification than separable features.
% bypassing the need for separability.



budgets can greatly influence---for better or worse---the performance of classifiers that are otherwise accurate on non-strategic data.

Since $b$ is not an input to $h$, 


- overlapping dists - connection to signaling theory? positives pay more to make themselves distinct

\todo{%
- empirical prices: funky, convergence \\
- recursive construction; min set, max set \\
- revenue derivative
}
