
% %========================================================================
% \begin{figure*}[t!]
% \centering
% \includegraphics[width=0.32\textwidth]{rev_samples.v2-crop.pdf} \,\,
% \includegraphics[width=0.32\textwidth]{sensitivity.v2-crop.pdf}
% \caption{%
% \todo{(left) different samples have different revenue curves}
% }
% \label{fig:synth}
% \end{figure*}
% %========================================================================
    

\section{Learning in Markets: Exploratory Insights}
Our goal in this section is to demonstrate
% through simple empirical examples,
the basic mechanics underlying
how learning shapes markets, and how markets in turn affect learning and its outcomes.
\blue{We focus first on empirical markets, over which the learning objective is defined,
and then transition to consider expected markets, on which performance is evaluated.}

% present basic analysis s understand how markets affect learning.
% - bottom-up: how users behave under, implications on prices, explore learning.



\subsection{Market behavior under fixed $h$}
We begin by exploring market behavior under a given classifier $h$,
with the goal of understanding how variation in inputs maps to changes in outcomes.
Since prices align with $w$, it suffices to consider user behavior along this direction,
and so w.l.o.g. we can abstract away $x$, 
\blue{assume uniform budgets,} % $b=1$,
and work directly with demand sets
$\{u_i\}_{i=1}^m$ where $u_i=\ubar_i$.
% $\{(u,b)\}$. 
% $\smplst = \{(u_i,b_i)\}_{i=1}^m$.
\squeeze

\paragraph{The price-revenue landscape.}
Thm.~\ref{thm:price_is_point} states that the revenue-maximizing price $\rhohat$ is always some $u_i^{-1}$;
hence, we can instead think of revenue as a function of inverse prices $\frac{1}{\rho}=u$,
measured in demand units.
This is useful since we can now consider directly how changes in the demand set affect
revenue, and through it, the optimal price.

Fig.~\ref{fig:rev_samples} plots empirical revenue $\rhat(\smplst)$ for three different samples $\smplst_j$ of size $m=10$ with units $u_i$ scaled to span $[1,10]$:

%========================================================================
\begin{figure}[h!]
\centering
% \vspace{-1em}
\includegraphics[width=\columnwidth]{rev_samples.v3-crop.pdf}
\caption{%
Empirical revenue curves for different samples.
}
\label{fig:rev_samples}
\end{figure}
%========================================================================
Note that revenue always begins at $\frac{1}{m}$ for the smallest $u_i$
since only one unit is sold (to one user) at price $\rho=1$.
From here, however, outcomes can differ considerably across samples,
in terms of the shape of the revenue curve,
the location and index of the price setter $i^*$,
and the optimal price $\rhohat$.
% This shows how prices can be sensitive to the \emph{composition} of demand.
% \squeeze

% \paragraph{Sensitivity.}
% Next, we examine the sensitivity of optimal prices to changes in demand.
This raises the question: how sensitive are market prices to variation in demand?
% We will measure changes when modifying a single point:
For this, we take a sample $u_1,\dots,u_m$,
and measure how prices change due to adding a single new point $u_0$.
Fig.~\ref{fig:sensitivity} shows the outcome of this process for a fixed select demand set
and for increasing $u_0 \in [1,10]$ (x-axis):

%========================================================================
\begin{figure}[h!]
\centering
\includegraphics[width=\columnwidth]{sensitivity.v3-crop.pdf}
\caption{%
The effect on price of adding a single point.
\todo{add legend for $s_i$}
}
\label{fig:sensitivity}
\end{figure}
%========================================================================
As can be seen, the value of $u_0$ has a stark effect on market prices:
even though it is increased gradually,
prices jump at discrete points whenever the price-setter $i^*$ changes.
Generally, prices are down-trending, and $i^*$ appear in increasing order of $u_i$---but this is not necessary, as prices can also jump up,
and some $s_i$ can be price-setters more than once.


\paragraph{Why prices jump.}
One reason for this behavior is that optimal prices
% One challenge with market prices is that they
may not be unique.
The following is an extreme construction in which \emph{all}
points are revenue-maximizing.
% \squeeze
% W.l.o.g. assume budget are uniform, $b_i=1 \, \forall i$.
\begin{proposition}
\label{prop:all_points_max_revenue}
Let $m \ge 3$, and
w.l.o.g. assume uniform budgets.
Define $u_1,\dots,u_m$ recursively as:
\[
u_i = u_2 \cdot \sum\nolimits_{j < i} u_j,
\quad u_2=2,
\quad u_1=1
\]
Then for all $i>1$, prices $\rho_i=u_i^{-1}$ attain the same revenue.
% then revenue $\rhat(\rho)$ is the same at all $\rho_i=u_i^{-1}$ for $i>1$.
\end{proposition}
Together with Thm.~\ref{thm:price_is_point}, this implies that $\rhat$ 
is also maximized under all $\rho_i$.
To see how this lends to price jumps,
consider the minimal case of $m=3$.
If we slightly decrease $u_2$, then it becomes the unique price-setter,
since \todo{...}.
In contrast, if we slightly increase $u_2$,
then $u_3$ becomes the price setter, since \todo{...}.
Thus, small perturbations in $u_2$ can cause prices to jump between
$\rho_2$ and $\rho_3$.
More generally, we can expect that adding another $u$ to the set will
contribute mostly \todo{...}

\blue{Of course another set with all maximizers is the set with $u_i=u$ for all $i>1$.
Interestingly, one attains the minimum revenue achievable with $m$ points, and the other attains the maximum.}

\blue{
Prop.~\ref{prop:all_points_max_revenue} is also suggestive of a more general structure.
For convenience, assume that for each $m$, the sequence $u_1,\dots,u_m$ is 
scaled (after its construction) to be in $[0,1]$.
This new sequence has an interesting limiting behavior:
%  by applying $u_i \gets 2(u_i-1)/(u_m-1)$.
as $m$ grows, we get that $u_{m-k} \to 1/3^k$.
}

\todo{is this also the minimal max-revenue set? can we prove it? is having all $u_i$-s be equal the max max-revenue set?}

\todo{add result on effect of outliers?}

% \paragraph{The role of budgets.}
% By Thm.~\ref{thm:price_is_point}, 

\paragraph{Revenue for large samples.}
Fortunately, prices tend to be more well-behaved
when the number of samples grows
and the demand distribution is reasonable.
For example, for $u \sim \Betadist(0.5,4)$,
% (and scaled to $[1,10]$),
Fig.~\ref{fig:increasing_sample_size} presents
revenue curves (left) as well as maximal revenue (top right),and empirical market prices (bottom right)
for samples of increasing size $m$:
\squeeze

%========================================================================
\begin{figure}[h!]
\centering
\includegraphics[width=\columnwidth]{increasing_sample_size-crop}
\caption{
Revenue and prices for increasing sample size.
}
\label{fig:increasing_sample_size}
\end{figure}
%========================================================================
As can be seen, despite significant variation under small $m$,
both revenue (curve and maximum) and prices tend to stabilize as $m$ grows.
The above shows results for $u \sim \Betadist(0.5,4)$ (and scaled to $[1,10]$).
Fig.~\ref{fig:beta_dists} suggests this behavior holds more broadly for many distributional shapes:
% \squeeze

%========================================================================
\begin{figure}[h!]
\centering
\includegraphics[width=0.95\columnwidth]{beta_dists-crop.pdf}
\caption{
Theoretical revenue curves for different Beta distributions
(all curves and densities are scaled to unit height).
\squeeze
}
\label{fig:beta_dists}
\end{figure}
%========================================================================
As shown, all the above have unique revenue maximizing $u^*$.
The following claim sheds some light on when this occurs.
\squeeze
\begin{lemma}
\label{lem:expected_revenue_derivative}
Let $q(u)$ be a density function,
and denote by $\rev_q(u)$ the expected revenue over $q$ at price $\rho=1/u$.
Then:
\begin{equation}
\label{eq:expected_revenue_derivative}
\frac{\d \rev_q}{\d u} = q(u)-\frac{\expect{q}{u' \mid u' \le u}}{u^2}
\end{equation}
where $\expect{q}{u' \mid u' \le u}$ is the truncated expectation.
\end{lemma}
Intuitively, this imples that a sufficient condition for revenue to have a unique maximum is if
(i) $q(u)$ is \red{...}, and
(ii) \red{...}

\red{
ask when we begin with:
\[
\int_0^u u q(u) \d u' > \int_0^u u' q(u') \d u'
\]
and at what $u$ this becomes an equality.
answer should be - late!
}

\todo{insights! conditions?}




\subsection{Accuracy markets when varying $h$}

- consider two dists $p(x|y=0)$ and $p(x|y=1)$ \\
- role of $b$ in $u/b$


\paragraph{Typical price tendencies.}
Fig.~\ref{fig:beta_dists} reveals an additional property of revenue, which is that
the price-setting point is typically located in the far-right tail of the distribution.
An extreme example for this is the uniform distribution;
plugging in the uniform pdf into Eq.~\eqref{eq:expected_revenue_derivative} gives that the maximum is attained at its upper limit.
% The value of the maximizing $u^*$ also tends to be large, though to a lesser extent.
This phenomenon has concrete implication on learning:
if $h$ induces a well-behaved (normalized) demand distribution,
then prices will be low, and the majority of the population
will cross the decision boundary
(see shaded regions and percentages in Fig.~\ref{fig:beta_dists}).
This, in turn, will likely deteriorate accuracy.
\squeeze



\paragraph{Clustering.}
\todo{...}

\paragraph{Fixed $w$, varying $a$.}


\paragraph{Varying $w$.}

\subsection{Market outcomes under learned $h$}

%----------------

\todo{fix code before plotting! }

\green{%
(1) fixed h: \\
- uniform b: \\
    + small samples
    - price-revenue landscape: \\
      - price as 1/u always; p of smallest u is 1 \\
      - compare revenue of two samples - different shapes, very non-concave \\
      * decide on direction of x-axis

    - multiple argmax rev points - formula, recursive+normalize to $[1,2]$ \\
    - how adding point changes price/argmax - jumpy! \\
    - effect of outliers \\
    - influence of non-moving points (?)

    * from sample to dist: (?)
    - sequentially increase sample, show revenue(m), price(m)
    - limit price result - here?
      - uniform dist
      - MLRP? (what does this even mean here?) 

    + large sample/distribution
    - argmax is extreme on typical dists - unimodal, bimodal \\
      - clusters hypothesis \\    
      - gaussian - limit of variance \\
    - implication: prices will usually be 'low' -> most points will move \\
    - conclusion: for high acc, need argmax to be midpoint; can be achieved if can find direction (w) for which b varies (sufficiently) and correlates with y! \\
      (- also need num pos > num neg? otherwise hard to separate)
    (flip claim: under uniform b, will be very hard to separate classes) \\
    (interpretation: causes $c(y=1) < c(y=-1)$)

- varying b: the power of budgets \\
    - upb as morphing distances \\
    - u-b relation can make sep data unsep, and vice versa \\
      - counterintuitive! \\
      - for this to work, need b cause midpoint to be argmax, by either:
        - morphing upbs to `squeeze' positives (eq, squared)
        - separate data to clusters that match y
      - vary gap between pos and neg dists


(2) learned h, market price: \\
- fix w, vary a: \\
  - how prices adapt/compensate to additive shift \\
  - vs standard sc where `raising bar' via a is key \\
  - loss landscape? 

- vary w: \\
  - 2D example, where x1 is sep and opt for non-strat, and x2 is nonsep but via b is opt for strat \\
  - [interesting/surprising examples]
    - features that *become* informative due to the market through corr with budget - means learning discriminates (learning wise *and* socially) by rich/poor (or - by value, which is good wrt welfare?)


[unsorted:] \\
    - empirical -> limit prices as m increases \\
    - social/welfare/burden implications: \\
        - welfare: if budget=value, then welfare = sum over util-cost.
        do markets improve welfare?
    - market power? \\
    - tell stories

}


% - gain intuition for how prices behave, and what they do
% - uniform vs different budgets: bias, variance, corr with y
% 
% - uniform budget: tendency to prefer further-away points
% - different budgets: corr with y can separate

% - linear vs 2-norm cost
% - revenue at first point always 1; price as 1/(x+1)

% - empirical market price - `reactive' hinge thing
% - vs expected market price

% * how to prevent prices from going too low?

% (2) learned h, market price:


---

% - how budgets `morph' distances via $u_i \mapsto u_i/\budget_i$

% given demand set $\Gamma$, what should we expect $\phat$ to be?
% - empirical vs distribution (continuous)
% - plot revenue as function of $p_i = u_i$ (or $\ubar_i$?)
% - effect of b: (i) b=1 (so $u=\ubar$), (ii) u corr y, (iii) u anticorr y
%   - (or corr with distance to h?)
% - opt price typically near end of "cluster" (`favors' low prices!)
% - social interpretation - markets help users?
% - as rule of thumb, helps learning when c(y=1) < c(y=-1)

% - triple of points that are all rev-maximizing
%   - recursive construction of series; grow vs squeeze to [0,1]
% - how adding new point (or removing? or moving?) changes stuff
%   - does adding far away point affect prices?

% - separable data - `squeeze' positives

% - uniform dist:
%   - p is at extreme
%   - limit price exists, convergence of empirical


%---------------------------



\blue{
Eq.~\eqref{eq:scalar_equilibrium_price} implies that prices align with the direction of $w$.
The role of the offset term $a$, 
which `sets the bar' for how difficult it is to obtain positive predictions (given $w$),
is more nuanced here.
Whereas strategic classification makes extensive use of $a$ to reduce strategic gaming \tocite,
we will see that the market greatly limits its significance by scaling prices to accommodate for its affect.
}
