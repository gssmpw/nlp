\section{Experiments}

% \red{- consider having only a subset of features for sale and all others fixed}

We now turn to demonstrate how our market-aware strategic learning framework performs empirically on real data with simulated market behavior.
We use the common and publicly available \adult\ dataset,%
\footnote{\url{https://archive.ics.uci.edu/dataset/2/adult}} 
and adapt it to our strategic market setting.
For budgets, we use the \texttt{capital\_gain} feature as a proxy for a monetary budget $\budget$.
Further details on data are given in Appendix.~\ref{appx:data}.
Code is available anonymously at
\url{https://github.com/MASC-ICML/MASC}.
\extended{\todo{publicly; plug in public repo}}

\extended{
\paragraph{Method and baselines.}
Our method of market-aware strategic classification (\masc)
works by optimizing the proxy objective proposed in Eq.~\eqref{eq:empirical_objective}
using gradient methods---see Appendix~\ref{appx:optimization} for details.
We compare performance to two baselines:
(i) \naivemthd, which trains a conventional non-strategic classifier,
and (ii) \strat, which trains a strategic classifier under the assumption that users respond to a fixed price vector $\p$.
We set $\p$ according to the market price induced by \naivemthd:
this simulates a setting where an initially \naive\ learner reacts to the market
it has induced, anticipate user behavior in this market---but does not account for \emph{future} price changes.
We also show the accuracy of \naivemthd\ on non-strategic data (for which it is consistent) as a benchmark.
All methods use linear classifiers as the base class.
\squeeze}

\paragraph{Setup.}
Our method of market-aware strategic classification (\masc)
works by optimizing the proxy objective proposed in Eq.~\eqref{eq:empirical_objective}
using gradient methods---see Appendix~\ref{appx:optimization} for details.
We compare to two baselines:
(i) \naivemthd, a conventional non-strategic classifier;
and (ii) \strat, a strategic classifier that anticipates user responses
(to fixed prices) but does not account for how the market adapts.
The latter is done by training \naivemthd, computing optimal prices $\p$,
setting $w=\p$,%
\footnote{This draws on the idea of the main algorithm in \citet{hardt2016strategic}.}
and then optimizing $\thresh$ to maximize accuracy on a held-out set.
We also show the accuracy of \naivemthd\ on non-strategic data (for which it is consistent) as a benchmark.
\squeeze

Our main question regards the effect of budget distribution on learning and its outcomes. 
The distribution of budgets $b$ as found in the data is highly skewed, 
with a ratio of $\bmin$ to $\bmax$ of approximately 1:1000,
which depicts a state of high inequality.
To balance this, we consider the effects of redistributing budgets
to attain lower inequity, achieved by rescaling budgets to
ranges $[1,2^\alpha]$ for \blue{$\alpha=2,\dots,10$}.
For each such $k$,
we measure test accuracy,
welfare (normalized by total budget),
and social burden \citep{milli2019social}
(normalized by total budget of positives). 
% The latter are given by:
We also measure the ratio of positive points that move (high is good)
and of negative points that don't (low is good).
% as measures of successful strategic learning.
All results are averaged over \blue{10} random 
% \blue{80-20} train-test 
splits (mean and standard errors).
% \squeeze

% \paragraph{Setup.}
% Our method of market-aware strategic classification (\masc)
% works by optimizing the proxy objective proposed in Eq.~\eqref{eq:empirical_objective}
% using gradient methods---see Appendix~\ref{appx:optimization} for details.
% We compare performance to a 
% \naivemthd\ baseline which trains a conventional non-strategic classifier.
% We also show the accuracy of \naivemthd\ on non-strategic data (for which it is consistent) as a benchmark.
% %
% Our main question regards the effect of budget distribution on learning and its outcomes. 
% The distribution of budgets $b$ as found in the data is highly skewed, 
% with a ratio of $\bmin$ to $\bmax$ of approximately 1:1000,
% which depicts a state of high inequality.
% To balance this, we consider the effects of redistributing budgets
% to attain lower inequity, achieved by rescaling budgets to
% ranges $[1,2^\alpha]$ for \blue{$\alpha=2,\dots,10$}.
% For each such $k$,
% we measure (i) test accuracy, (ii) welfare, and (iii) social burden
% \citep{milli2019social}. 
% % The latter are given by:
% \extended{%
% Welfare measures the profit (utility minus cost) for all users of the system as:
% \begin{equation}
% \label{eq:welfare}
% \welfare(h) = \frac{1}{B}\sum_i \budget_i \one{h(x_i^h) = 1} - \cost(x_i,x_i^h)
% \end{equation}
% where $B=\sum_i b_i$ normalizes to allow meaningful comparisons across conditions.
% Social burden measures the overall cost required to ensure that all deserving users (i.e., with $y=1$) rightfully obtain positive predictions ($\yhat=1$):
% \begin{equation}
% \label{eq:burden}
% \burden(h) = \frac{1}{P} \sum_{i:y_i=1} \min_{x':h(x')=1} \cost(x_i,x')
% \end{equation}
% with normalizing constant $P=|\{i\,:\,y_i=1\}|$.
% }
% We also measure the ratio of (iv) positive points that move
% and of (v) negative points that don't as measures of successful strategic learning.
% All results are averaged over \blue{10} random 
% % \blue{80-20} train-test 
% splits.
% \squeeze


% \todo{Gini?}


\paragraph{Results.}
Fig.~\ref{fig:adult} shows results across budget redistributions
at different inequity scales \blue{$\frac{\bmax}{\bmin} = 2^2,\dots,2^{10}$}.
In terms of accuracy (left), 
all methods improve as budget gaps increase, but at different rates.
\masc\ clearly outperforms \naivemthd\ by a large margin.
It also outperforms \strat\ in the larger budget gap regime of
% $\frac{\bmax}{\bmin}$ 
$2^4, \dots, 2^{10}$, which includes the data's original budget scale.
For small scales ($\le 2^3$), \strat\ does better;
further analysis reveals that \masc\ attains high train accuracy,
and therefore might be overfitting to the observed market.
This may not be surprising given our results in Sec.~\ref{sec:synth}
which show high sensitivity at small budget scales. %
\extended{\footnote{Another possible reason is that $\thresh$ optimizes over exact prices and the true 0-1 loss, whereas \masc\ resorts to smooth surrogates.}}
This also relates to the abrupt jump in performance for \masc\ 
observed at scale $2^4$.
Such improvement is enabled by an increasing ability to limit the crossing of negative points,
while retaining movement for positive (top right).
In terms of social outcomes,
welfare (normalized) begins reasonably high, but reduces to $\sim 0.5$,
and does not fully recover.
Burden remains flat until scale $2^4$, and then gradually rises,
but remains relatively low throughout.
\squeeze

% \paragraph{Results.}
% Fig.~\ref{fig:adult} shows results across budget redistributions
% at different inequity scales \blue{$\frac{\bmax}{\bmin} = 2^2,\dots,2^{10}$}.
% In terms of accuracy (left), \masc\ clearly outperforms \naivemthd\ by a large margin.
% An increase in budget inequity helps both methods,
% but whereas \naivemthd\ begins to capitalize on this only at around scale $2^5$,
% \masc\ begins earlier, and shows a large jump to attain its maximal value at $2^4$.
% This is enabled by abruptly preventing almost all negatives from crossing (top right).
% As inequity continues to increase, the accuracy of \masc\ remains high but slowly diminishes. This is due to an increasing difficulty to permit positive points to cross.
% Interestingly, welfare changes are quite dramatic, 
% and are mostly (inversely) correlated with positive crossing.
% Burden remains flat until scale $2^4$, and then gradually rises,
% as the distribution of budgets become more concentrated on positives.
% \squeeze



%========================================================================
\begin{figure}[t!]
\centering
% \includegraphics[width=\columnwidth]{real.tmp.png}
% \includegraphics[width=\columnwidth]{adult.v2-crop.pdf}
\includegraphics[width=\columnwidth]{adult.v4-crop.pdf}
\caption{%
\textbf{Results on \adult.}
\textbf{(Left:)}
Accuracy across reduced budget inequity scales,
relative to the data's original highly skewed scale of
$\frac{\bmax}{\bmin} \approx 2^{10}$ (star).
\textbf{(Right:)}
For \masc,
per-class ratio of crossers (top; high is good for $y=1$,
low is good for $y=0$),
welfare, and social burden (bottom).
% for increasingly skewed budget distributions.
}
\label{fig:adult}
\end{figure}
%========================================================================
    
    