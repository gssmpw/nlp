\section{Introduction}

Strategic classification \citep{hardt2016strategic,bruckner2012static}
considers learning in a setting where users 
can strategically manipulate their features %, in response to the classifier,
to obtain positive predictions.
This applies to tasks such as loan approval, job hiring, school admissions, insurance claims, and welfare benefits,
in which the interests of users
(e.g., getting the loan or being hired) may not be aligned with 
the system's learning objective of maximizing accuracy.
The primary goal of strategic learning is to train classifiers that are robust to such responsive user behavior,
% which is made in response to the learned classifier.
an idea that has gained much recent traction
(see Sec.~\ref{sec:related} for a partial list of related work).
%  \tocite.
% \squeeze

\extended{\todo{replace ref to related with actual cites}}

A core assumption of strategic classification is that feature manipulation is \emph{costly},
i.e., that modifying $x$ to some other $x'$ incurs a cost to the user.
These costs are typically modeled via a \emph{cost function} $c(x,x')$ that 
underlies user decisions, and hence governs strategic behavior.
% this is sensible, since features that can be changed `for free' would become entirely uninformative and hence redundant for learning.
The vast majority of the literature considers costs as predetermined and
fixed;
% in other words,
even if unknown to the learner,
costs are still assumed to simply `exist'.
But where do costs come from,
what form do they take,
and how do they come to be?
Challenging the conventional assumption of exogenous costs,
our works sets out to propose and study alternative cost mechanisms.

One such alternative, 
and the focus of our paper,
% and our main thesis in this paper,
is the idea that costs can materialize through \emph{market forces}:
the classifier creates demand,
suppliers set prices,
and users pay for items or services that aid them in securing positive predictions.
As an example, consider university admissions,
which often rely on standardized test scores (e.g., SAT).
Since these affect acceptance decisions,
% The inclusion of these as criteria for acceptance incentivizes students
students are incentivized to improve their scores;
this, in turn,
has created a (billion-dollar) market for preparation courses.
% \blue{Thus, the mere inclusion of a certain feature---namely test scores---created an entire (billion-dollar) industry.}
% Taking university admissions as an example,
% consider how the inclusion of standardized test scores (e.g.,
% SAT) as a criterion has created a (billion-dollar) market for
% test-preparation courses.
We posit that the price of such
courses is determined by the importance of standardized
tests as a feature in the decision rule for admission:
if a policy update
changes the relative weight of test scores, then
prices should adjust accordingly.%
\footnote{For broader discussion on changes in SAT policy as they relate to strategic behavior see \citet{liu2021test}.}
% As an example, consider university admissions,
% which are often based on high school grades (e.g., GPA)
% and standardized test scores (e.g., SAT).
% Since these affect acceptance,
% % The inclusion of these as criteria for acceptance incentivizes students
% students are incentivized to improve in these aspects;
% this, in turn,
% has created (billion-dollar) markets for private tutoring and preparation courses.
% We posit that the price of tutors and courses is determined by the relative importance of each component in the decision rule for admission:
% if a policy update changes these weights,
% then we should expect prices to adjust accordingly.
Note that such changes also affect who will---and even who \emph{can}---take such costly courses.
This, in turn, can affect the eventual composition of admitted students.
If a learned classifier is to be used to inform such decisions,
then learning must be aware of, and accountable for, the market it fosters.
% should be made aware of how the classifier will shape the market it induces.

% \extended{%
% \todo{ref SAT policy changes; nikhil's paper: Test-optional Policies: Overcoming Strategic Behavior and
% Informational Gaps}}


Our paper formalizes this idea 
and applies it to the framework of strategic classification.
% This idea applies to strategic classification as well:
When users seek positive predictions, this creates demand for features that are important for classification;
and if features are available for `purchase' from sellers,
then a competitive market is formed.
The cost of obtaining features is then determined by their market price,
which is reflective of their market value,
and users can purchase any bundle of features whose price is within their budget.
% where demand is due to users' interest in positive prediction, features are available for purchase, and costs 
% reflect prices set by sellers that maximize their revenue.
% are set by prices that reflect competitive equilibrium.
% More precisely, since users seek positive prediction, this creates demand for features that are helpful in improving predictive scores.
% If these features can be `purchased',
% then sellers can set prices (e.g., to maximize their revenue), which in turn determine the cost function $c$.
Crucially, prices are not given nor predetermined;
rather, they depend on the learned classifier through how it shapes demand, as it relates to the entire data distribution.
This means that to obtain a strategically robust classifier,
learning must %be aware of---and 
be able to anticipate %---%
the market it induces.
We refer to this as \emph{market-aware classification}.
% in which prices reflect the joint demand of all users.
To facilitate learning in this setting,
we define `a market for features', as it relates to the learning task;
characterize an appropriate notion of price equilibrium;
and study the task of learning strategic classifiers that induce markets.
% \squeeze
 
% in a setting where users are strategic agents that operate in this market.
% under users that strategically 
% agents that respond based on market prices.


% As such, our goal in this work is to extend beyond the standard setting and consider
% % take a step towards
% broader and better-grounded cost mechanisms.
% We begin by asking: in the real world, where do costs come from?
% One possible answer, which is the focus of our work, is that
% % In this paper we propose that 
% costs %emerge
% materialize through \emph{market forces}.




% Since students wish to get accepted, they are likely to invest effort in improving their scores.
% This, in turn, creates demand for services that can assist students,
% such as test-preparation courses---and a market is formed.

% \todo{mention user budgets here?}

Learning in our market setting admits two key challenges.
First, since market prices rely on the aggregate demand of \emph{all} users,
the behavior of users becomes dependent through the cost function.
This is in sharp contrast to the standard setting in which users respond independently and the objective decomposes over training examples.
As we show, this can have a stark effect on learning, 
since even points that lie far from the decision boundary
%  (and will never move) 
can still have a significant impact on market prices,
and hence on the behavior, of others.
Fortunately, a useful property of equilibrium prices is that if the market is efficient, then prices reflect all relevant information.
In our setting, this implies that \emph{conditioned on prices},
% user behavior becomes independent.
the objective does decompose over users.
This allows us to adapt standard techniques for strategic learning with (independent) user responses to handle (dependent) market-induced cost functions.
%
Our second challenge is therefore to compute market prices effectively
and as part of the training pipeline.
For this, we first give an algorithm for computing prices exactly, and show that it is efficient.
% Based on this,
We then propose a differentiable variant of the algorithm that computes `smooth' prices,
% which we use as a `layer' that permits 
which enables
end-to-end optimization of the entire objective using gradient methods.
\squeeze

Using our approach,
and via both synthetic and semi-synthetic experiments,
% of \emph{market-aware classification},
we proceed to explore the effects of induced markets on learning and its outcomes,
Our main results here are twofold.
First, we show that markets can give rise to complex behavioral patterns that differ significantly from the conventional model of strategic classification, and in some cases, are even counterintuitive.
This is because \emph{prices are adaptive}:
For example,
scaling the data has no effect since this is simply a change of currency;
`raising the bar' on acceptance by increasing the threshold 
has a limited effect since prices will work to counter it;
% be lowered;
and unseparable data can %be disentangled and 
\emph{become} separable 
% if user budgets increase in the direction of the classifier.
if: (i) budgets correlate with labels,
% and prices discriminate users with large budgets.
and (ii) prices discriminate against low-budget users.
\squeeze
% and classifiers can obtain perfect accuracy on completely unseparable data

This drives our second result, which is that budgets play a distinctive role in shaping learning outcomes.
Our framework makes a distinction between what users have (i.e., their features) and their economic stature (via their budget).
Here we show that learning tends to favor users with larger budgets.
The mechanism for this is indirect:
if the classifier separates the data well
but in a way that negative points hold most of the aggregate budget,
then prices will be low, negative points will cross the decision boundary---and accuracy will be reduced.
Thus, classifiers will be accurate under the markets they induce \emph{only if they associate positive predictions with high budgets}.
This raises natural questions regarding fairness and socioeconomic equity.
\extended{which carry important social and policy implications.}

\extended{%
\red{%
- monetary inequity/discrimination  -- important aspect of fairness, received little attention in the learning community \\
- market power held by learner \\
- regulation, subsidizing
}
}

% not only on their endowment (=features) but on their monetary budget / economic stature / capabilities

% questions regarding fairness naturally arise
% social implications of market-inducing learning

% \todo{experiments}

% One implication of learning under market-determined costs is that behavior becomes \emph{dependent} across users---this is since, given a classifier, prices are determined by the aggregate demand exhibited by the collective of all users.
% In standard strategic classification, fixed prices (coupled with the fact that learning optimizes for accuracy) lend to an objective that decouples over users, and in which each user responds independently to the classifier.
% % irrespective of other user's features or decisions.
% In the market setting we consider, dependence across users gives rise to a non-decomposable objective, which is more challenging to optimize.
% However, a useful property of prices is that they channel all information about demand in the market into a single set of numbers;
% as we will show, conditional on prices, user responses decompose---an outcome which enables the tractability of our proposed learning approach.

% \blue{
% % Beyond solving the learning problem, we will be interested in exploring and analyzing the social implication of market-inducing learning.
% Because outcomes in our setting rely on the willingness or capacity of users to expend resources in order to obtain positive prediction,
% this provides us with an opportunity to study inequity and discrimination under a monetary lens---and important aspect of fairness that has received little attention in the learning community.
% Also, because learning affects outcomes through pricing, we will also be interested in understanding the market power held by the learner, i.e., how much power learning sways over price-setting, possibly in ways that promote the learner's interest but at the expense of welfare to users and/or sellers.
% Both notions raise important questions regarding regulation and the ability of a social planner to encourage positive social outcomes.
% }

% \todo{need examples! see notes, mlis, isf}


\extended{%
\blue{
As an abstraction, this formulation is sensible and convenient,
and often permits tractable learning.
However, modeling costs in this way has several significant drawbacks.
First, it implicitly considers costs not only as fixed, but as predetermined, and in a way that is entirely exogenous to the learning task.
% since they depend neither on the user population nor on the learned classifier.
Second, formulating costs as $c(x,x')$ implies that each user responds individually and independently, and is unaffected by the actions (or mere existence) of other users.
Third, learning must assume and commit to a particular functional form
(e.g., $\ell_2$-norm), and often requires full knowledge of its precise specification \cite{rosenfeld2023one}.
% Thus, costs are assumed to
Together, these depict a world in which costs simply `exist', are common knowledge,
and affect users in isolation---%
a view which we believe is both restrictive and unrealistic.
\squeeze}

% \todo{scuc ref is arxiv, update to icml24}

\todo{together with util=pred, responses are independent}

\todo{cost is arbitrary (eg norm), is silent about what costs are - price, effort, time etc}
}