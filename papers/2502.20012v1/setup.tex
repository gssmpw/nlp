
\extended{%
%========================================================================
\begin{figure*}[t!]
\centering
\includegraphics[width=\textwidth]{illust.tmp.png}
\caption{%
\todo{...}
}
\label{fig:illust}
\end{figure*}
%========================================================================   
}


\section{Setup}

\paragraph{Strategic classification.}
In standard strategic classification,
users are described by features $x \in \R^d$, and have binary labels $y \in \{0, 1\}$.
% and can modify their features to obtain positive predictions.
Given a sample set of pairs $(x,y)$ drawn iid from some unknown joint distribution $\dist$,
% Given a sample set $\smplst =\{(x_i,\budget_i,y_i)\}_{i=1}^m$ drawn iid from some unknown joint distribution $(x,\budget,y) \sim \dist$,
the goal in learning is to find a classifier $h$ from some model class $H$ whose predictions $\yhat=h(x)$ obtain high expected accuracy on future samples.
Our focus will be on linear classifiers,
$h_{w,\thresh}(x)=\sign(w^\top x + \thresh)$.
% as is common in most current works on strategic classification \tocite.
%
The challenge in strategic learning is that users
can `game' the system by manipulating their features to obtain positive predictions.
In particular, given the classifier $h$, users are assumed to be rational and therefore modify their features via
% at test time the classifier observes strategically modified inputs $x^h$, 
% where modifications are made in response to the classifier.
% Hence, rather than the true $x$, the system observes the a modified $x^h$
% This is 
% which results from users applying to $x$
the \emph{best-response mapping}:
\squeeze
\begin{equation}
\label{eq:br_general}
% x^h \coloneqq 
% \br_h(x) = \argmax_{x'} \util(h(x')) - \cost(x,x')
\br_h(x) = \argmax_{x'} \, h(x') - \cost(x,x')
\end{equation}
% where $\util$ determines the utility gained from the prediction outcome 
where $h(x') \in \{ \pm 1\}$ is their utility gained from prediction outcomes on the modified input,
% (i.e., either $h(x')=1$ or $-1$),
% $\yhat'=h(x') \in \{\pm 1\}$,
and $\cost(x,x')$ is a cost function that governs the costs of changing $x$ to any other $x'$.
% In the standard setting, $\util$ is simply the identity function.
The goal is then to learn a classifier $h$ that is robust to such strategic responses, and the strategic learning objective is:
\begin{equation}
\label{eq:sc_learning_objective}
\argmin_{h \in H} \expect{\dist}{\one{y \neq h(x^h)}}, \quad
x^h = \br_h(x)
\end{equation}
% Our main distinction will be to allow the cost function $c$ to encode market costs.

\paragraph{Market setting.}
Our setting builds on the above to allow for the formation of a market for features.
To generally enable transactions,
we require two additional structural assumptions.
First, we assume features describe tangibles; this means that each $x_{[i]} \ge 0$, and that a larger value means having `more' of feature $i$.
Second, %to enable purchases, 
we assume each user has an (individualized) monetary budget $b \ge 0$, which limits the amount they are willing to spend
(or, alternatively, the value they attribute to obtaining a positive prediction).
This extends the joint distribution to be over tuples $(x,b,y) \sim \dist$.

% \todo{$x_j$ notation overloads vector entry and j-th example}

% \red{x as endowment?}

Apart from these, the only distinction of our setup is that we use a particular cost function to express market costs.
We will make use of \emph{linear costs} \citep{hardt2016strategic};
given a vector $\p=(p_1,\dots,p_d)\ge 0$
and for $\delta=x'-x$,
define:
% \squeeze
\begin{equation}
\label{eq:linear_costs}
\cost_\p(x,x') = \cost_\p(\delta) = \delta^\top \p
\end{equation}
If we consider $\delta$ as a bundle of features,
then we can interpret $\p$ as a vector of \emph{prices},
where each $p_i$ is the price of purchasing one unit of feature $i$.
%  where $\p$ encodes all prices.
% We will henceforth refer to $\p$ as \emph{prices}.
We assume that users can buy features (but cannot sell),
so that $\delta \ge 0$; together with $\p \ge 0$, this ensures $\cost_\p(\delta) \ge 0$ always.\footnote{%
Note this circumvents an artifact of linear costs,
made apparent in \citet{hardt2016strategic},
which is that points can move `for free' in any direction (and hence to any distance) that is orthogonal to $w$.
}
% \squeeze
Rather than assuming prices are fixed and given,
our main innovation is to let $\p$ be determined by forces of supply and demand.
% which is shaped by the classifier, as we describe next.

\extended{\todo{what can we say about non-linear costs? which of our results generalize?}}

% The crucial aspect of our setting is that, rather than given,
% prices depend on the learned classifier $h$
% through how it shapes demand on the data distribution $\dist$,
% so that $\p=\p(h;D)$.
% We will make this dependence precise in Sec.~\toref.
% \squeeze





\paragraph{Sellers and market prices.}
% Our main innovation is that, rather than given,
% prices are set by \emph{sellers} and in response to the learned classifier $h$.
%  as shaped by the learned classifier. %, where the latter derives from $h$.
% Prices in our settings are set by
Our setting assumes there are 
$d$ distinct sellers,
$s_1,\dots,s_d$,
where each seller $s_i$ sells feature $i$ exclusively and
% that each feature $i$ can be bought only from a distinct seller $s_i$, who
can determine its price $p_i$.
% We consider a setting of unlimited supply (e.g., as in digital goods)
% and in which users can purchase any real quantity of any feature $i \in [d]$.
%% $\delta_i \in \R_+$ (at cost $p_i \delta_i$).
%
% defined as follows.
% Let $\delta(x;p) \in \R^d$ be the bundle of features
% % amount of feature $i$ 
% purchased by user $x$ under prices $p$.
% where $\delta_i$ is the amount of units of feature $i$ that she requires.
The goal of each seller %$s_i$
is to maximize her \emph{expected revenue},
defined as:
% defined for each seller $s_i$ as:
\squeeze
% given a classifier $h$, and for distribution $\dist$, we have:
\begin{equation}
\label{eq:revenue}
\rev_i(\p) = p_i \cdot \expect{\dist}{\delta_i(x;\p)}
% \rev_i(p) = p_i \cdot \expect{\dist}{\delta_i(x) \cdot \one{\delta(x)^\top p \le \budget}}
\end{equation}
where $\delta_i$ is the amount of feature $i$ purchased by user $x$ at prices $\p$.
We consider a setting of unlimited supply (e.g., as in digital goods)
and in which users can purchase any real quantity of any feature,
$\delta_i \in \R_+ \,\forall i \in [d]$.
% where $\delta(x;\p)$ is the bundle purchased by $x$ at prices $\p$.
% i.e., sellers gain $p \delta_i$ profit from any purchase of $\delta_i$ units of their feature $i$.
% user whose budget permits purchase of all required features, namely $\delta^\top p \le \budget$.
% \squeeze

Note revenue to seller $s_i$ depends not only on its $p_i$,
which it controls,
but also on all \emph{other} prices, $p_{-i}$,
which are set by other sellers.
% Although each seller set $p_i$ to maximize their own revenue $\rev_i(\p)$,
% this depends also on prices set be all other sellers, $p_{-i}$.
% which depend on the joint demand for all features $i \in [d]$,
% which are set by the other sellers $s_j \neq s_i$.
As such, % induced by a learned classifier $h$,
we will assume that prices reach 
% \blue{competitive?} 
equilibrium,
denoted $\p^* = (p^*_1, \dots, p^*_d)$,
which is
% These prices are
revenue-maximizing in the sense that
no seller $s_i$ can improve her own revenue by changing $p_i$,
given that all other prices remain fixed. %
% \footnote{
% Because our setting includes substitute goods with no capacity constraints or production costs, 
% it is susceptible to Bertrand's paradox:
% sellers always gain by undercutting competitors,
% which causes all prices to spiral down towards marginal costs.
% To circumvent this, %we assume
% for simplicity we will invoke the Folk Theorem and assume
% sellers have foresight and so coordinate to prevent price collapse and maintain prices at the cooperative equilibrium.
% }
We refer to $\p^*$ as `market prices',
and will define them precisely in Sec.~\ref{sec:analysis}.
A crucial point is that market prices depend on 
the joint demand for all features,
aggregated over all users.
This, in turn, is shaped by the choice of classifier,
as we describe next.
\squeeze

\extended{%
\todo{extends to sellers that sell bundles of features? now it's like they sell $e_i$ vectors, extension is arbitrary vectors; maybe need them to span space together}

\todo{explain why finite supply is more challenging}

\todo{ask moran to go over this!}
}

% \todo{bertrand competition: substitute goods; (equal) marginal costs; no capacity constraints; undercut; cooperation - collude to charge monopoly price (folk theorem = repeated game with infinite horizon; vs single simultaneous move game)}

% We denote such revenue-maximizing prices (when they exist) by .

% for all sellers are then set via:
% \begin{equation}
% \label{eq:revenue_maximizing_prices}
% p^*_i = \argmax_p \rev_i(p)
% \end{equation}

% since features in our setup are substitutable goods (i.e., any user can cross the decision boundary with a sufficient quantity of any single feature), ...


\paragraph{Classifiers that induce markets.}
% The key modeling point in our setup is that market prices are determined by the 
% % demand for features is shaped by the
% learned classifier $h$ through how it shapes demand.
% To see this, note that since
Since by Eq.~\eqref{eq:br_general}
utility to users derives from their prediction $\yhat=h(x)$,
% (see Eq.~\eqref{eq:br}),
any user $x$ who is classified as negative (i.e., has $h(x)=0$) 
% \yonatan{-1?}) 
will be interested in purchasing additional features
$\delta = (\delta_1, \dots, \delta_d)$
if this results in flipping her prediction to $h(x+\delta)=1$.
The demand set of a user therefore includes all $\delta$ for which:
\begin{equation}
\label{eq:demand_set}
% \delta \,:\, 
w^\top (x+\delta) + \thresh \ge 0
\quad \text{and} \quad
\delta^\top p \le b
\end{equation}    
% Since users are rational (and hence minimize costs),
% a user $x$ will be happy with any $\delta$ in her 
% \emph{demand set}, defined as:
% \squeeze
% \begin{equation}
% \label{eq:demand_set}
% % \delta^h 
% % = \delta^h(x;p) 
% % \in 
% \Gamma_x(p,h_{w,\thresh}) = 
% \{ \delta \,:\, w^\top (x+\delta) + a = 0
% \, \wedge \,
% \delta^\top p \le b \}
% \end{equation}
% which satisfies budget constraints.
% Overall demand $\Gamma$ is then obtained by aggregating over all $\Gamma_x$,
Overall demand is then given by aggregating all such $\delta$ over the collection of all users,
and market prices $\p^*$ are set by sellers to maximize revenue under this global demand set.
% Aggregating over all users then gives the overall demand, denoted $\Gamma$.
% In this sense,

Notice how demand, and therefore prices,
depend on the interaction between the data distribution
(i.e., all pairs $(x,b)$) and the classifier $h$ (via $w$ and $a$).
In this sense,
we get that
\emph{each choice of classifier induces a market}.
We will henceforth use $\p^h \coloneqq \p^*(h;\dist)$ to denote
the classifier-dependent equilibrium prices
that govern user responses in the market.
\squeeze

% This implies that over all users, which features will be in demand,
% and in what quantity, depends on $h$.
% Because users are utility maximizers,
% then for a linear classifier $h_{w,\thresh}$,
% they will seek $\delta > 0$ for which $w^\top (x+\delta) + a = 0$,
% as this minimizes costs,
% and will be feasible if $\delta^\top p \le b$.



\paragraph{Strategic learning objective.}
% \todo{sample set; empirical objective}
Given a sample set $\smplst =\{(x_i,\budget_i,y_i)\}_{i=1}^m$ drawn iid from $\dist$, we will be interested in learning a classifier that maximizes expected accuracy under the market it induces.
This requires us to anticipate how users will respond:
% User responses are obtained
for a given $h$,
plugging Eq.~\eqref{eq:linear_costs} into Eq.~\eqref{eq:br_general} 
% and integrating budget constraints 
gives the \emph{market best-response} mapping:
% \todo{make connection between constraints and next eq; talk about $\frac{\cost}{\budget}$ formulation}
\begin{equation}
\label{eq:br_market}
\brmrkt_h(x;\budget) = \argmax_{\delta \ge 0} \,
h(x+\delta) - \frac{1}{\budget} \cost_{\p^h}(\delta)
% h(x+\delta) - \frac{1}{\budget} \cost_{\p^*(h;\dist)}(\delta)
% \,\,\, \text{s.t.} \,\,\,
% c_\p(\delta) \le b
\end{equation}
which satisfies budget constraints implicitly.
Importantly, when $h$ is learned,
the behavior of each individual $x$ via $\br_h$ depends (indirectly)
on the entire distribution $\dist$ through $\p^h$.
% Equivalently, we can think of budgets as scaling costs via $\frac{1}{\budget}\cost$ with utility $h$;
% in either case we get that $x$ can change only if
% $\delta^\top \cost \le \budget$.
\squeeze

Given Eq.~\eqref{eq:br_market}, our strategic learning objective is:
% \squeeze
\begin{equation}
\label{eq:learning_objective_exp}
\argmin_{h \in H} \expect{\dist}{\one{y \neq h(x^h)}}, \quad
x^h = 
% \br_h^\p(x;\budget)
\brmrkt_h(x,\budget)
\end{equation}
which in practice we will replace with an appropriate empirical proxy
(Sec.~\ref{sec:method}).
Note $h$ is a function of features alone---and not of budgets;
% takes as input only (modified) features $x^h$;
thus, we assume budgets are observed at train time,
but at test time are private to users,
and affect their computation of $x^h$.
% \extended{%
This makes $\brmrkt_h$ a special case of the generalized response model proposed in \citet{levanon2022generalized} which supports private information.
% }
% We now turn to discuss market behavior in more depth.
% \squeeze

% We now turn to discussing how classifiers affect prices,
% and how prices influence learning outcomes.
% \blue{%
% In practice we will instead optimize the empirical objective:
% \begin{equation}
% \label{eq:learning_objective_emp}
% \argmin_{h \in H} 
% \sum_{i=1}^m \loss(\brapx_h^\smplst(x_i,\budget_i), y_i; h) + \lambda \reg(h)
% \end{equation}
% where $\loss(x,y;h)$ is a proxy loss (e.g., hinge loss),
% $\reg$ is a possible regularization term with coefficient $\lambda$,
% and $\brapx_h^\smplst$ replaces $\brmrkt_h$ as an approximate  best-response mapping that permits tractable optimization, 
% which depends on $\smplst$,
% and will be defined in Sec.~\toref.
% }


% \blue{%
% One implication of market prices is that to optimize the robust objective in Eq.~\eqref{eq:learning_objective_exp}, learning must appropriately anticipate the market that will form once the classifier is deployed.
% Since learning relies on a sample, we will be interested in the \emph{empirical market prices} $\phatvec$ which maximize the average revenue on the empirical distribution given by $\smplst$.
% We now turn to discuss how $\phatvec$ can be computed efficiently and integrated within an optimization framework for solving Eq.~\eqref{eq:learning_objective}.}
% \squeeze


% \todo{$\budget$ as private info as $z$ in GSC}
% how they respond to $h$.

\extended{%
\todo{make sure this is correct; if so -- is this because of the market? otherwise it would mean we can learn with individualized costs $c_i=\frac{1}{\budget_i}\cost$. what does this mean about learnability?}
}

% We refer to a pair $\mu=(S,h)$ as a \emph{market}.

\extended{%
\blue{
As we will show, the standard (non-market) setting is recovered when $b=1$ for all users. Thus, variation in budgets, which manifests as variation in demand, is naturally required for the market to be meaningful.}
\yonatan{Is this right? even with same budgets, all users don't have fixed and equal $\delta$ like in the standard setting and the movement will still depend on the market}


% \blue{
% % and the collection of all such $\delta$ defines their demand set.%
% % \footnote{Intuitively, this includes all positive vectors originating at $x$ and ending precisely on the decision boundary of $h$.}
% Note however that given prices $\p$, there is a unique $\delta^*$ that solves Eq.~\eqref{eq:br_market};
% in particular, this is the $\delta^*$ which satisfies ${\delta^*}^\top w = 1$,
% i.e., the vector which projects $x$ onto the hyperplane defined by $w$ and $\thresh$.
% This ensures that Eq.~\eqref{eq:revenue} is well-defined under the demand distribution over $\delta_i$ induced by $h_{w,\thresh}$ for all $i \in [d]$.
% % As a result, we get that
% % \emph{each choice of classifier induces a market},
% % in which equilibrium prices $\p^* = \p^*(h;\dist)$
% % govern how users respond.
% % We refer to these as \emph{market prices}.
% % (when clear from context we will suppress the notational dependence of $\p^*$ on $h$ and $\dist$).
% }



\red{
    % - fixed price as special case of hardt's cost (really?); no free moves due to positivity assumptions \\
    - uniform $b$ somewhat degenerate - talk more later \\
    - $\br$ depends on all points, but given prices decomposes - as in standard sc \\
    - extension: exogeneous demand, prices set wrt this and h
}



\red{
    - result: in finite sample, rev max price is always some point \\
    - could be multiple point, so multiple optimal prices \\
}

\red{extensions: \\
 -- exogenous demand? \\
 -- non-linear prices? \\
 -- production cost? (subtract from revenue) \\
 -- min price? (below which profit<0 for sellers) \\
 -- some way to have first point get rev<<1? \\
 -- have only a subset of features for sale and all others fixed
}
}