\appendix
% \section{Appendix}
% \label{sec: appendix}

\section{Effect of Chunk Size and Chunk Overlap on Chunking Quality}
\label{appendix: chunk_hyper}

Elaborate from \S\ref{subsubsec: chunking_result}, Table~\ref{table: chunking_by_size} presents the performance of various line chunking strategies across different chunk sizes. As expected, larger chunk sizes correlate with more text per chunk, resulting in higher Sections/Chunk and Chunks/Section values. Larger chunks also reduce the likelihood of incomplete section coverage within a single chunk.

\begin{table}[H]
\centering

\resizebox{\textwidth}{!}{%
\renewcommand{\arraystretch}{1.5} % This increases the cell height by 1.5 times
% \small % or \scriptsize
\begin{tabular}{@{}lccccc@{}}
\toprule
Chunk Size & \multicolumn{1}{l}{Section/Chunk $\rightarrow$1} & \multicolumn{1}{l}{Chunk/Section $\rightarrow$1} & \multicolumn{1}{l}{Fail Chunk Ratio $\downarrow$} & \multicolumn{1}{l}{Fail Section Ratio $\downarrow$} & \multicolumn{1}{l}{Uncovered Section Ratio $\downarrow$} \\ \midrule

212        & \textbf{1.264}                    & \textbf{1.121}                    & 0.8                                  & 0.551                                  & 0.487                                       \\

250        & \underline{1.373}                       & \underline{1.174}                       & 0.768                                & 0.509                                  & 0.425                                       \\

300        & 1.497                             & 1.227                             & 0.725                                & 0.463                                  & 0.356                                       \\

350        & 1.613                             & 1.264                             & 0.684                                & 0.422                                  & 0.304                                       \\

466        & 1.87                              & 1.284                             & 0.597                                & 0.36                                   & 0.207                                       \\

553        & 2.048                             & 1.285                             & \underline{0.528}                          & \underline{0.315}                            & \underline{0.149}                                 \\

600        & 2.16                              & 1.285                             & \textbf{0.5}                         & \textbf{0.301}                         & \textbf{0.13}                               \\ \bottomrule
\end{tabular}
}
\caption{Line chunking strategy result by chunk size}
\label{table: chunking_by_size}
\end{table}

Table~\ref{table: chunking_by_overlap} presents the evaluation of the line chunking strategy with varying overlaps. As overlap increases, the effective chunk size also increases, leading to higher Sections/Chunk and Chunks/Section values. Furthermore, increasing overlap reduces the Fail Chunk Ratio, Fail Section Ratio, and Uncovered Section Ratio.

\begin{table}[H]
\centering

\resizebox{\textwidth}{!}{%
\renewcommand{\arraystretch}{1.5} % This increases the cell height by 1.5 times
\small % or \scriptsize
\begin{tabular}{@{}lccccc@{}}
\toprule
Chunk Overlap & \multicolumn{1}{l}{Section/Chunk $\rightarrow$1} & \multicolumn{1}{l}{Chunk/Section $\rightarrow$1} & \multicolumn{1}{l}{Fail Chunk Ratio $\downarrow$} & \multicolumn{1}{l}{Fail Section Ratio $\downarrow$} & \multicolumn{1}{l}{Uncovered Section Ratio $\downarrow$} \\ \midrule

50            & \textbf{1.627}                    & \textbf{1.153}                    & \textbf{0.654}                       & 0.423                                  & 0.303                                       \\

100           & \underline{1.666}                       & \underline{1.183}                       & \underline{0.659}                          & 0.422                                  & 0.299                                       \\

150           & 1.717                             & 1.254                             & 0.661                                & \underline{0.417}                            & \underline{0.292}                                 \\

200           & 1.747                             & 1.347                             & 0.657                                & \textbf{0.407}                         & \textbf{0.283}                              \\ \bottomrule
\end{tabular}
}
\caption{Line chunking result by chunk overlap}
\label{table: chunking_by_overlap}
\end{table}

Finally, Table~\ref{table: chunking_map_stats} presents statistics on section coverage within each chunk using the selected line chunking strategy (chunk size: 553 characters, overlap: 50 characters).

\begin{table}[H]
\centering
\begin{tabular}{@{}cccccccc@{}}
\toprule
Count & Mean  & Std.  & Min   & 25th Percentile & 50th Percentile & 75th Percentile & Max \\ \midrule
2610  & 1.331 & 0.577 & 1.000 & 1               & 1               & 2               & 7   \\ \bottomrule
\end{tabular}
\caption{Statistics on section coverage within each chunk}
\label{table: chunking_map_stats}
\end{table}

% \section{Response Prompts}
% \label{appendix: response_prompts}

% We present the prompts used for responses in different systems, including:
% \begin{itemize}
%     \item Retrieval-Augmented Generation (RAG)-based system
%     \item Parametric RAG-based system
%     \item Large-Context Language Model (LCLM)-based system
% \end{itemize}

% Additionally, we provide the few-shot examples used for each system. All examples are in Thai.

% \subsection{RAG-based system}
% \label{subsubsec: rag_based_prompts}

% Below are the prompts we utilized for generating response in a RAG-based system on WCX dataset.

\textbf{System Prompt: }
\begin{Verbatim}[breaklines, breaksymbol=, breakanywheresymbolpre=]

You are an expert paralegal in Thai law domain. You have a job to answer legal questions posed by civillians and rarely make mistakes. The structure of questions you are proficient in consists of the question under the tag <question></question> and your answer and maybe rationale of that answer in the tag <answer></answer> and  may also be provided with the content of the relevant laws to that tax case under the tag <ข้อกฎหมาย></ข้อกฎหมาย> with each law section under the tag <law section="XXX"></law>. The law can also be under the tag <related_law section="XXX" law_name="XXX", parent_section="XXX" parent_law_name="XXX"></related_law> which are nested laws mention inside a given main law.
\end{Verbatim}

\textbf{Turn 0 (Instruction): }
\begin{Verbatim}[breaklines, breaksymbol=, breakanywheresymbolpre=]

<user> Take a deep breath and think carefully. Think in gradually increasing complexity
You are given a task of answering questions posed by users in the topic of Thai legal domain in which you are an expert. Given the question (<question>) and relevant laws (<ข้อกฎหมาย>) with each section under the tag <law section="XXX"></law> or <related_law section="XXX" parent_law="XXX" parent_section="XXX"></related_law>, you need to give a response to the question asked by the users. You do not need to explain your rationale and how you reach the answer. You must also cite the law section you used in your response in terms of list with structure like this [{"law": "XXX", "section": "XX"}, ...]. The cited law must be from the given relevant law only and nothing outside of it. Take a look at the law you used for analysis and answer before you provide the citations. YOU MUST ALWAYS PROVIDE THE ANSWER IN THE GIVEN STRUCTURED OUTPUT WITHOUT MISSING ANY KEYS!! THIS IS REALLY IMPORTANT. DO NOT ANSWER OR PROVIDE ANYTHING I CANNOT PARSE INTO JSON!!! Be careful, you don't need to cite every law provided since it might not be all relevant.
    
Your final output should be a JSON object with the following keys. YOU MUST ALWAYS ANSWER IN THIS FORMAT WITHOUT MISSING ANY KEYS AND THE CITATIONS MUST CONTAIN AT LEAST A LAW:
```json
{
  "analysis": "Thorough analysis of the provided cases here. Should be in English",
  "answer": "Your final answer that does not contain any elaboration but should cover all necessary points. Must be in THAI only.",
  "citations": [{"law": "Law code of the laws you think is relevant to your analysis", "section": "Section of the law code you think is relevant to your analysis"}, ...]
}
```
    
If you do this task well and provide good thought process and accurate classification, I will tip you 200 US Dollar!

<assistant> I will strictly adhere to the guidelines you told me for this task
\end{Verbatim}


\textbf{Turn 1: }
\begin{Verbatim}[breaklines, breaksymbol=, breakanywheresymbolpre=]

<user> <ข้อกฎหมาย>
<law section=31 ประมวลรัษฎากร> ประมวลรัษฎากร มาตรา 31 การอุทธรณ์ไม่เป็นการทุเลาการเสียภาษีอากร ถ้าไม่เสียภาษีอากรภายในเวลาที่กฎหมายกำหนดให้ถือเป็นภาษีอากรค้างตามมาตรา 12 เว้นแต่กรณีที่ผู้อุทธรณ์ได้รับอนุมัติจากอธิบดีให้รอคำวินิจฉัยอุทธรณ์หรือคำพิพากษาได้ ก็ให้มีหน้าที่ชำระภายในสามสิบวันนับแต่วันได้รับแจ้งคำวินิจฉัยอุทธรณ์หรือได้รับทราบคำพิพากษาถึงที่สุด แล้วแต่กรณี
ในกรณีที่มีคำวินิจฉัยอุทธรณ์ให้เสียภาษีอากรเพิ่มขึ้น ผู้อุทธรณ์จะต้องชำระภายในกำหนดเวลาเช่นเดียวกับวรรคก่อน
</law>
</ข้อกฎหมาย>
<question> ถ้าอุทธรณ์จะสามารถทุเลาการเสียภาษีได้หรือไม่ </question>

<assistant> ```json
{"analysis": "From the given question, the user wants to know if an appeal can defer the tax. According to the given relevant law which states that the appeal does not defer tax unless the person who made the appealed is authorized by the director that the payment can be waited until the result of the appeal is finalized. In that case the person who made the appeal must pay the tax in the duration of 30 days after the date in which the result of the appeal is finalized according to the result.", "answer": "ไม่ได้ เนื่องจากการอุทธรณ์ไม่เป็นการทุเล่าการเสียภาษี เว้นแต่กรณีที่ผู้อุทธรณ์ได้รับอนุมัติจากอธิบดีให้รอคำวินิจฉัยอุทธรณ์หรือคำพิพากษาได้ ก็ให้มีหน้าที่ชำระภายใน 30 วันนับแต่วันได้รับแจ้งคำวินิจฉัยอุทธรณ์หรือได้รับทราบคำพิพากษาถึงที่สุด ", "citations": [{'law': 'ประมวลรัษฎากร', 'section': '31'}]}
```
\end{Verbatim}

\textbf{Turn 2: }
\begin{Verbatim}[breaklines, breaksymbol=, breakanywheresymbolpre=]

<user> <ข้อกฎหมาย>
<law section=70 พระราชบัญญัติการจัดซื้อจัดจ้างและการบริหารพัสดุภาครัฐ พ.ศ. 2560> พระราชบัญญัติการจัดซื้อจัดจ้างและการบริหารพัสดุภาครัฐ พ.ศ. 2560 มาตรา 70 งานจ้างที่ปรึกษา ให้หน่วยงานของรัฐเลือกใช้วิธีใดวิธีหนึ่งตามมาตรา 69 ตามหลักเกณฑ์ ดังต่อไปนี้
(1) งานจ้างที่ปรึกษาโดยวิธีประกาศเชิญชวนทั่วไป ให้ใช้กับงานที่ไม่ซับซ้อน งานที่มีลักษณะเป็นงานประจำของหน่วยงานของรัฐ หรืองานที่มีมาตรฐานตามหลักวิชาชีพ และมีที่ปรึกษาซึ่งสามารถทำงานนั้นได้เป็นการทั่วไป
(2) งานจ้างที่ปรึกษาโดยวิธีคัดเลือก ให้กระทำได้ในกรณี ดังต่อไปนี้
(ก) ใช้วิธีประกาศเชิญชวนทั่วไปแล้ว แต่ไม่มีผู้ยื่นข้อเสนอ หรือข้อเสนอนั้นไม่ได้รับการคัดเลือก
(ข) เป็นงานที่ซับซ้อน ซับซ้อนมาก หรือที่มีเทคนิคเฉพาะไม่เหมาะที่จะดำเนินการโดยวิธีประกาศเชิญชวนทั่วไป
(ค) เป็นงานจ้างที่มีที่ปรึกษาในงานที่จะจ้างนั้นจำนวนจำกัด
(ง) กรณีอื่นตามที่กำหนดในกฎกระทรวง
(3) งานจ้างที่ปรึกษาโดยวิธีเฉพาะเจาะจง ให้กระทำได้ในกรณี ดังต่อไปนี้
(ก) ใช้ทั้งวิธีประกาศเชิญชวนทั่วไปและวิธีคัดเลือก หรือใช้วิธีคัดเลือกแล้ว แต่ไม่มีผู้ยื่นข้อเสนอ หรือข้อเสนอนั้นไม่ได้รับการคัดเลือก
(ข) งานจ้างที่มีวงเงินค่าจ้างครั้งหนึ่งไม่เกินวงเงินตามที่กำหนดในกฎกระทรวง
(ค) เป็นงานที่จำเป็นต้องให้ที่ปรึกษารายเดิมทำต่อจากงานที่ได้ทำไว้แล้วเนื่องจากเหตุผลทางเทคนิค
(ง) เป็นงานจ้างที่มีที่ปรึกษาในงานที่จะจ้างนั้นจำนวนจำกัดและมีวงเงินค่าจ้างครั้งหนึ่งไม่เกินวงเงินตามที่กำหนดในกฎกระทรวง
(จ) เป็นงานจ้างที่มีที่ปรึกษาในงานที่จะจ้างนั้นเพียงรายเดียว
(ฉ) เป็นงานที่มีความจำเป็นเร่งด่วนหรือที่เกี่ยวกับความมั่นคงของชาติ หากล่าช้าจะเสียหายแก่หน่วยงานของรัฐหรือความมั่นคงของชาติ
(ช) กรณีอื่นตามที่กำหนดในกฎกระทรวง
ให้คณะกรรมการนโยบายมีอำนาจประกาศในระบบเครือข่ายสารสนเทศของกรมบัญชีกลางกำหนดกรณีตัวอย่างของงานที่ไม่ซับซ้อนตาม (1) หรืองานที่ซับซ้อน ซับซ้อนมาก หรือที่มีเทคนิคเฉพาะตาม (2) (ข)
รัฐมนตรีอาจออกระเบียบเพื่อกำหนดรายละเอียดอื่นของงานจ้างที่ปรึกษาตามวรรคหนึ่งเพิ่มเติมได้ตามความจำเป็นเพื่อประโยชน์ในการดำเนินการ
</law>
<law section=76 พระราชบัญญัติการจัดซื้อจัดจ้างและการบริหารพัสดุภาครัฐ พ.ศ. 2560> พระราชบัญญัติการจัดซื้อจัดจ้างและการบริหารพัสดุภาครัฐ พ.ศ. 2560 มาตรา 76 ในการพิจารณาคัดเลือกข้อเสนอโดยวิธีตามมาตรา 69 (1) หรือ (2) นอกจากให้พิจารณาเกณฑ์ด้านคุณภาพตามมาตรา 75 แล้ว ให้เป็นไปตามเกณฑ์ในการพิจารณาและการให้น้ำหนัก ดังต่อไปนี้ด้วย
(1) กรณีงานจ้างที่ปรึกษาเพื่อดำเนินงานประจำ งานที่มีมาตรฐานเชิงคุณภาพตามหลักวิชาชีพอยู่แล้ว หรืองานที่ไม่ซับซ้อน ให้หน่วยงานของรัฐคัดเลือกผู้ยื่นข้อเสนอที่ผ่านเกณฑ์ด้านคุณภาพแล้วและให้คัดเลือกจากรายที่เสนอราคาต่ำสุด
(2) กรณีงานจ้างที่ปรึกษาที่เป็นไปตามมาตรฐานของหน่วยงานของรัฐหรืองานที่ซับซ้อน ให้หน่วยงานของรัฐคัดเลือกผู้ยื่นข้อเสนอที่ผ่านเกณฑ์ด้านคุณภาพแล้วและให้คัดเลือกจากรายที่ได้คะแนนรวมด้านคุณภาพและด้านราคามากที่สุด
(3) กรณีงานจ้างที่ปรึกษาที่มีความซับซ้อนมาก ให้หน่วยงานของรัฐคัดเลือกผู้ยื่นข้อเสนอที่ผ่านเกณฑ์ด้านคุณภาพแล้วและให้คัดเลือกจากรายที่ได้คะแนนด้านคุณภาพมากที่สุด
รัฐมนตรีอาจออกระเบียบกำหนดหลักเกณฑ์เกี่ยวกับการพิจารณาคัดเลือกข้อเสนอของงานจ้างที่ปรึกษาประเภทใดหรือกำหนดรายละเอียดอื่นของงานจ้างที่ปรึกษาตามวรรคหนึ่งเพิ่มเติมได้ตามความจำเป็นเพื่อประโยชน์ในการดำเนินการ
</law>
</ข้อกฎหมาย>
<question> ช่วยอธิบายความหมายและความสำคัญของ "ความคุ้มค่า" ในการคัดเลือกที่ปรึกษา? </question>


<assistant> ```json{"analysis": "From the question, the user wants to know the meaning of 'value for money' and its importance in the process of procurements (external hirings from government). From the given relevant laws, section 70 states about the 'how' of the process while section 76 states about the 'guideline' of the process and how to get the most value out of the consultants for each type of job. Therefore, according to section 76, one can infer the meaning of 'value for money' as the highest possible return according to the quality of the job and the price. Therefore, The importance of 'value for money' is to ensure that government agencies receive consulting services with quality appropriate to the nature of the work, while also considering efficient use of the budget", "answer": "'ความคุ้มค่า' ในการคัดเลือกที่ปรึกษาตามกฎหมายการจัดซื้อจัดจ้าง หมายถึงการพิจารณาทั้งด้านคุณภาพและราคาเพื่อให้ได้ประโยชน์สูงสุดแก่หน่วยงานของรัฐ และ ความสำคัญของ 'ความคุ้มค่า' คือการทำให้มั่นใจว่าหน่วยงานของรัฐจะได้รับบริการที่ปรึกษาที่มีคุณภาพเหมาะสมกับลักษณะของงาน ในขณะเดียวกันก็คำนึงถึงการใช้งบประมาณอย่างมีประสิทธิภาพ ซึ่งจะช่วยให้การใช้ทรัพยากรของรัฐเป็นไปอย่างมีประสิทธิภาพและเกิดประโยชน์สูงสุด", "citations": [{'law': 'ประมวลรัษฎากร', 'section': '76'}]}
```
\end{Verbatim}

\textbf{Turn 3: }
\begin{Verbatim}[breaklines, breaksymbol=, breakanywheresymbolpre=]

<user> <ข้อกฎหมาย>
<law section=56 พระราชบัญญัติบริษัทมหาชนจำกัด พ.ศ. 2535> พระราชบัญญัติบริษัทมหาชนจำกัด พ.ศ. 2535 มาตรา 56 ใบหุ้นนั้นอย่างน้อยต้องมีรายการดังต่อไปนี้
(1) ชื่อบริษัท
(2) เลขทะเบียนบริษัท และวันที่นายทะเบียนรับจดทะเบียนบริษัท
(3) ชนิด มูลค่า เลขที่ใบหุ้นและ จำนวนหุ้น
(4) ชื่อผู้ถือหุ้น
(5) ลายมือชื่อกรรมการซึ่งลงหรือพิมพ์ไว้อย่างน้อยหนึ่งคน แต่กรรมการจะมอบหมายให้นายทะเบียนหุ้นตามกฎหมายว่าด้วยหลักทรัพย์และตลาดหลักทรัพย์ลงหรือพิมพ์ลายมือชื่อแทนก็ได้
(6) วันเดือนปีที่ออกใบหุ้น
</law>
<law section=6 พระราชบัญญัติหลักทรัพย์และตลาดหลักทรัพย์ พ.ศ. 2535> พระราชบัญญัติหลักทรัพย์และตลาดหลักทรัพย์ พ.ศ. 2535 มาตรา 6 การลงลายมือชื่อของกรรมการหรือนายทะเบียนในใบหลักทรัพย์ตามพระราชบัญญัตินี้ จะใช้เครื่องจักรประทับหรือโดยวิธีอื่นใดแทนตามที่คณะกรรมการกำกับตลาดทุนกำหนดก็ได้ ทั้งนี้ ตามหลักเกณฑ์และวิธีการที่คณะกรรมการกำกับตลาดทุนประกาศกำหนด
</law>
<law section=197 พระราชบัญญัติหลักทรัพย์และตลาดหลักทรัพย์ พ.ศ. 2535> พระราชบัญญัติหลักทรัพย์และตลาดหลักทรัพย์ พ.ศ. 2535 มาตรา 197 ในการออกใบหลักทรัพย์ ให้นายทะเบียนหลักทรัพย์มีอำนาจลงลายมือชื่อในใบหลักทรัพย์แทนการลงลายมือชื่อกรรมการของบริษัทที่ออกหลักทรัพย์จดทะเบียนได้เมื่อได้รับมอบหมายจากบริษัทดังกล่าว
</law>
</ข้อกฎหมาย>
<question> การลงลายมือชื่อของนายทะเบียนหลักทรัพย์มีผลกระทบอย่างไรต่อความถูกต้องของใบหลักทรัพย์ </question>

<assistant> ```json{"analysis": "From the given question, the user wants to know how the signature of the securities registrar affects the validity of the securities certificate. The provided laws, specifically section 56 of พระราชบัญญัติบริษัทมหาชนจำกัด พ.ศ. 2535 and section 197 of พระราชบัญญัติหลักทรัพย์และตลาดหลักทรัพย์ พ.ศ. 2535, states that a share certificate must consist of multiple things including the signature of the company's director or in case that it is authorized, the signature of the securities registrar. Therefore, the signature of the securities registrar is necessary in some cases but does not affect the validity of the certificate more than the signature of the company's director.", "answer":"การลงลายมือชื่อของนายทะเบียนหลักทรัพย์มีผลกระทบอย่างไรต่อความถูกต้องของใบหลักทรัพย์ คือ ไม่มีผลกระทบใด ๆ ต่อความถูกต้องของใบหลักทรัพย์ เพราะ ตามมาตรา 56 ของพระราชบัญญัติบริษัทมหาชนจำกัด พ.ศ. 2535 และมาตรา 197 ของพระราชบัญญัติหลักทรัพย์และตลาดหลักทรัพย์ พ.ศ. 2535 นายทะเบียนหลักทรัพย์มีอำนาจลงลายมือชื่อในใบหลักทรัพย์แทนการลงลายมือชื่อกรรมการของบริษัทที่ออกหลักทรัพย์จดทะเบียนได้เมื่อได้รับมอบหมายจากบริษัทดังกล่าว", "citations": [{'law': 'พระราชบัญญัติบริษัทมหาชนจำกัด พ.ศ. 2535', 'section': '56'}, {'law': 'พระราชบัญญัติหลักทรัพย์และตลาดหลักทรัพย์ พ.ศ. 2535', 'section': '197'}]}
```
\end{Verbatim}
% Below are the prompts we utilized for generating response in a RAG-based system on Tax Case dataset.

\textbf{System Prompt: }
\begin{Verbatim}[breaklines, breaksymbol=, breakanywheresymbolpre=]

You are an expert lawyer in Thai tax legal domain who never make mistakes when reviewing a tax cases. The structure of the Thai Tax case you are proficient in consists of the context and question wrapped under the tag <ข้อหารือ></ข้อหารือ> and may also be provided with the content of the relevant laws to that tax case under the tag <ข้อกฎหมาย></ข้อกฎหมาย> with each law section under the tag <law section="XXX" law_name="XXX"></law>. The law can also be under the tag <related_law section="XXX" law_name="XXX", parent_section="XXX" parent_law_name="XXX"></related_law> which are nested laws mention inside a given main law.
\end{Verbatim}

\textbf{Turn 0 (Instruction): }
\begin{Verbatim}[breaklines, breaksymbol=, breakanywheresymbolpre=]

<user> Take a deep breath and think carefully. Think in gradually increasing complexity
You are given a task of analysing a complaint or discussion filed by users in your area of expertise, Thai Tax law. 
Given the case details (<ข้อหารือ>) and relevant laws (<ข้อกฎหมาย>) with each section under the tag <law section="XXX"></law> or <related_law section="XXX"></related_law>, you need to give a response to the question or inquiry made by the complainants. You do not need to explain your rationale and how you reach the answer. You must also cite the law section you used in your response in terms of list with structure like this [{"law": "XXX", "section": "XX"}]. The cited law must be from the given law only and nothing outside of it. Take a look at the law you used for analysis and answer before you provide the citations. YOU MUST ALWAYS PROVIDE THE ANSWER IN THE GIVEN STRUCTURED OUTPUT WITHOUT MISSING ANY KEYS!! THIS IS REALLY IMPORTANT DO NOT ANSWER OR PROVIDE ANYTHING I CANNOT PARSE INTO JSON!!!. Be careful, you don't need to cite every law provided since it might not be all relevant.
    
Your final output should be a JSON object with the following keys. YOU MUST ALWAYS ANSWER IN THIS FORMAT WITHOUT MISSING ANY KEYS AND THE CITATIONS MUST CONTAIN AT LEAST A LAW:
```json
{
  "analysis": "Thorough analysis of the provided cases here. Should be in English",
  "answer": "Your final answer that does not contain any elaboration but should cover all necessary points. Must be in THAI only.",
  "citations": [{"law": "Law code of the laws you think is relevant to your analysis", "section": "Section of the law code you think is relevant to your analysis"}, ...]
}
```
    
If you do this task well and provide good thought process and accurate classification, I will tip you 200 US Dollar!

<assistant> I will strictly adhere to the guidelines you told me for this task
\end{Verbatim}


\textbf{Turn 1: }
\begin{Verbatim}[breaklines, breaksymbol=, breakanywheresymbolpre=]

<user> <ข้อกฎหมาย>
<law section=81 law_name=ประมวลรัษฎากร> ประมวลรัษฎากร มาตรา 81 ให้ยกเว้นภาษีมูลค่าเพิ่มสำหรับการประกอบกิจการประเภทต่าง ๆ ดังต่อไปนี้
(1) การขายสินค้าที่มิใช่การส่งออก หรือการให้บริการดังต่อไปนี้
(ก) การขายพืชผลทางการเกษตร ไม่ว่าจะเป็นลำต้น กิ่ง ใบ เปลือก หน่อ ราก เหง้า ดอก หัว ฝัก เมล็ด หรือส่วนอื่น ๆ ของพืช และวัตถุพลอยได้จากพืช ทั้งนี้ ที่อยู่ในสภาพสด หรือรักษาสภาพไว้เพื่อมิให้เสียเป็นการชั่วคราวในระหว่างขนส่งด้วยการแช่เย็น แช่เย็นจนแข็ง หรือด้วยการจัดทำหรือปรุงแต่งโดยวิธีการอื่น หรือรักษาสภาพไว้เพื่อมิให้เสียเพื่อการขายปลีกหรือขายส่งด้วยวิธีการแช่เย็น แช่เย็นจนแข็ง ทำให้แห้ง บด ทำให้เป็นชิ้น หรือด้วยวิธีอื่น ข้าวสารหรือผลิตภัณฑ์ที่ได้จากการสีข้าว แต่ไม่รวมถึงไม้ซุง ฟืน หรือผลิตภัณฑ์ที่ได้จากการเลื่อยไม้หรือผลิตภัณฑ์อาหารที่บรรจุกระป๋อง ภาชนะหรือหีบห่อ ที่ทำเป็นอุตสาหกรรม ตามลักษณะ และเงื่อนไขที่อธิบดีกำหนด
(ข) การขายสัตว์ ไม่ว่าจะมีชีวิตหรือไม่มีชีวิต และในกรณีสัตว์ไม่มีชีวิตไม่ว่าจะเป็นเนื้อ ส่วนต่าง ๆ ของสัตว์ ไข่ น้ำนม และวัตถุพลอยได้จากสัตว์ ทั้งนี้ ที่อยู่ในสภาพสดหรือรักษาสภาพไว้เพื่อมิให้เสียเป็นการชั่วคราวในระหว่างขนส่งด้วยการแช่เย็น แช่เย็นจนแข็ง หรือด้วยการจัดทำหรือปรุงแต่งโดยวิธีการอื่น หรือรักษาสภาพไว้เพื่อมิให้เสียเพื่อการขายปลีกหรือขายส่งด้วยวิธีการแช่เย็น แช่เย็นจนแข็ง ทำให้แห้ง บด ทำให้เป็นชิ้น หรือด้วยวิธีอื่น แต่ไม่รวมถึงผลิตภัณฑ์อาหารที่บรรจุกระป๋อง ภาชนะ หรือหีบห่อ ที่ทำเป็นอุตสาหกรรม ตามลักษณะ และ เงื่อนไขที่อธิบดีกำหนด
(ค) การขายปุ๋ย
(ง) การขายปลาป่น อาหารสัตว์
(จ) การขายยาหรือเคมีภัณฑ์ที่ใช้สำหรับพืชหรือสัตว์ เพื่อบำรุงรักษาป้องกัน ทำลายหรือกำจัดศัตรูหรือโรคของพืชและสัตว์
(ฉ) การขายหนังสือพิมพ์ นิตยสาร หรือตำราเรียน
(ช) การให้บริการการศึกษาของสถานศึกษาของทางราชการ สถานศึกษาตามกฎหมายว่าด้วยสถาบันอุดมศึกษาเอกชน หรือโรงเรียนเอกชนตามกฎหมายว่าด้วยโรงเรียนเอกชน
(ซ) การให้บริการที่เป็นงานทางศิลปะและวัฒนธรรมในสาขา และลักษณะการประกอบกิจการที่อธิบดีกำหนดโดยอนุมัติรัฐมนตรี
(ฌ) การให้บริการการประกอบโรคศิลปะ การสอบบัญชี การว่าความหรือการประกอบวิชาชีพอิสระอื่นตามที่อธิบดีกำหนดโดยอนุมัติรัฐมนตรี ทั้งนี้ เฉพาะวิชาชีพอิสระที่มีกฎหมายควบคุมการประกอบวิชาชีพอิสระนั้น
(ญ) การให้บริการรักษาพยาบาลของสถานพยาบาลตามกฎหมายว่าด้วยสถานพยาบาล
(ฎ) การให้บริการวิจัย หรือการให้บริการทางวิชาการ ทั้งนี้ ในสาขา และลักษณะการประกอบกิจการที่อธิบดีกำหนดโดยอนุมัติรัฐมนตรี
(ฎ) การให้บริการห้องสมุด พิพิธภัณฑ์ สวนสัตว์
(ฐ) การให้บริการตามสัญญาจ้างแรงงาน
(ฑ) การให้บริการจัดแข่งขันกีฬาสมัครเล่น
(ฒ) การให้บริการของนักแสดงสาธารณะ ทั้งนี้ เฉพาะบริการในสาขาและลักษณะการประกอบกิจการตามที่อธิบดีกำหนดโดยอนุมัติรัฐมนตรี
(ณ) การให้บริการขนส่งในราชอาณาจักร
(ด) การให้บริการขนส่งระหว่างประเทศ ซึ่งมิใช่เป็นการขนส่งโดยอากาศยานหรือเรือเดินทะเล
(ต) การให้บริการเช่าอสังหาริมทรัพย์
(ถ) การให้บริการของราชการส่วนท้องถิ่น ทั้งนี้ ไม่รวมถึงบริการที่เป็นการพาณิชย์ของราชการส่วนท้องถิ่น หรือเป็นการหารายได้หรือผลประโยชน์ไม่ว่าจะเป็นกิจการสาธารณูปโภคหรือไม่ก็ตาม
(ท) การขายสินค้าหรือการให้บริการของกระทรวง ทบวง กรม ซึ่งส่งรายรับทั้งสิ้นให้แก่รัฐโดยไม่หักรายจ่าย
(ธ) การขายสินค้าหรือการให้บริการเพื่อประโยชน์แก่การศาสนาหรือการสาธารณกุศลภายในประเทศ ซึ่งไม่นำผลกำไรไปจ่ายในทางอื่น
(น) การขายสินค้าหรือการให้บริการตามที่กำหนดโดยพระราชกฤษฎีกา
(2) การนำเข้าสินค้าดังต่อไปนี้
(ก) สินค้าตาม (1) (ก) ถึง (ฉ)
(ข) สินค้าจากต่างประเทศที่นำเข้าไปในเขตปลอดอากร ทั้งนี้ เฉพาะสินค้าที่ได้รับยกเว้นอากรขาเข้าตามกฎหมายว่าด้วยการนั้น
(ค) สินค้าที่จำแนกประเภทไว้ในภาคว่าด้วยของที่ได้รับยกเว้นอากรตามกฎหมายว่าด้วยพิกัดอัตราศุลกากร
(ง) สินค้าซึ่งนำเข้าและอยู่ในอารักขาของศุลกากร แล้วได้ส่งกลับออกไปต่างประเทศ โดยได้คืนอากรขาเข้าตามกฎหมายว่าด้วยศุลกากร
(3) การส่งออกซึ่งสินค้าหรือบริการของผู้ประกอบการจดทะเบียนซึ่งต้องเสียภาษีมูลค่าเพิ่มตามมาตรา 82/16
การยกเว้นภาษีมูลค่าเพิ่มสำหรับการประกอบกิจการตามมาตรานี้ อธิบดีจะเสนอให้คณะกรรมการวินิจฉัยภาษีอากรพิจารณากำหนดลักษณะของกิจการและเงื่อนไขในการประกอบกิจการที่ได้รับการยกเว้นตามมาตรานี้ก็ได้ และเมื่อคณะกรรมการวินิจฉัยภาษีอากรได้วินิจฉัยแล้ว ให้ประกาศคำวินิจฉัยของคณะกรรมการดังกล่าวในราชกิจจานุเบกษาและหากกิจการนั้นมิได้เป็นไปตามลักษณะ และเงื่อนไขที่กำหนด กิจการนั้นจะไม่ได้รับยกเว้นภาษีมูลค่าเพิ่มตามมาตรานี้ </law>
</ข้อกฎหมาย>
<ข้อหารือ> บริษัท A เป็นผู้ประกอบการนำเข้าสินค้า อินทผาลัม มะเดื่อแห้ง อาปิคอต ลูกพลัม จากต่างประเทศนำมาบรรจุ ใส่กล่องกระดาษที่คล้ายกล่องส่งไปรษณีย์ภัณฑ์ เพียงแต่นำมาสอดขัดขึ้นรูปเป็นกล่องบรรจุ เพื่อสะดวกต่อการขนย้าย และจำหน่ายภายในประเทศ แต่มิได้ทำเป็นอุตสาหกรรม และเป็นการขายทั้งปลีกและส่ง การบรรจุภาชนะหรือหีบห่อ มิได้บรรจุกระป๋อง หรือผนึกในลักษณะมั่นคง และสินค้าดังกล่าวไม่จำต้องแปรรูปใดๆ เพียงแต่เก็บไว้ในห้องที่มีอุณหภูมิไม่เกิน 1 องศาเซลเซียส บริษัทฯ จึงขอทราบว่า การขายสินค้าดังกล่าว ได้รับการยกเว้นภาษีมูลค่าเพิ่ม ตามมาตรา 81(1) (ก) แห่งประมวลรัษฎากร หรือไม่ <\ข้อหารือ>

<assistant> ```json
{"analysis": "From the given case details, the company import date palm, dried fig, apricot and plum which is considered agricultural product from foreign country. The products also undergo some process such as freezing, cooling, drying and other to prolong their shelf life. Hence, selling said products falls into the exception of VAT collection according to section 81(1)(ก) of ประมวลรัษฎากร", "answer": "การขายสินค้าดังกล่าวได้รับการยกเว้นภาษีมูลค่าเพิ่มตามมาตรา 81(1)(ก) แห่งประมวลรัษฎากร", "citations": [{'law': 'ประมวลรัษฎากร', 'section': '81'}]}
```

\end{Verbatim}

\textbf{Turn 2: }
\begin{Verbatim}[breaklines, breaksymbol=, breakanywheresymbolpre=]

<user> <ข้อกฎหมาย>
<law section=78 ประมวลรัษฎากร> ประมวลรัษฎากร มาตรา 78 ภายใต้บังคับมาตรา 78/3 ความรับผิดในการเสียภาษีมูลค่าเพิ่มที่เกิดจากการขายสินค้า ให้เป็นไปตามหลักเกณฑ์ดังต่อไปนี้
(1) การขายสินค้านอกจากที่อยู่ในบังคับตาม (2) (3) (4) หรือ(5) ให้ความรับผิดทั้งหมดเกิดขึ้นเมื่อส่งมอบสินค้า เว้นแต่กรณีที่ได้มีการกระทำดังต่อไปนี้เกิดขึ้นก่อนส่งมอบสินค้า ก็ให้ถือว่าความรับผิดเกิดขึ้นเมื่อได้มีการกระทำนั้น ๆ ด้วย
(ก) โอนกรรมสิทธิ์สินค้า
(ข) ได้รับชำระราคาสินค้า หรือ
(ค) ได้ออกใบกำกับภาษี
ทั้งนี้ โดยให้ความรับผิดเกิดขึ้นตามส่วนของการกระทำนั้น ๆ แล้วแต่กรณี
(2) การขายสินค้าตามสัญญาให้เช่าซื้อหรือสัญญาซื้อขายผ่อนชำระที่กรรมสิทธิ์ในสินค้ายังไม่โอนไปยังผู้ซื้อเมื่อได้ส่งมอบ ให้ความรับผิดเกิดขึ้นเมื่อถึงกำหนดชำระราคาตามงวดที่ถึงกำหนดชำระราคาแต่ละงวด เว้นแต่กรณีที่ได้มีการกระทำดังต่อไปนี้เกิดขึ้นก่อนถึงกำหนดชำระราคาแต่ละงวด ก็ให้ถือว่าความรับผิดเกิดขึ้นเมื่อได้มีการกระทำนั้น ๆ ด้วย
(ก) ได้รับชำระราคาสินค้า หรือ
(ข) ได้ออกใบกำกับภาษี
ทั้งนี้ โดยให้ความรับผิดเกิดขึ้นตามส่วนของการกระทำนั้น ๆ แล้วแต่กรณี
(3) การขายสินค้าโดยมีการตั้งตัวแทนเพื่อขายและได้ส่งมอบสินค้าให้ตัวแทนแล้ว ทั้งนี้ เฉพาะสัญญาการตั้งตัวแทนเพื่อขายตามประเภทของสินค้าและเป็นไปตามหลักเกณฑ์ วิธีการ และเงื่อนไขที่อธิบดีกำหนดโดยอนุมัติรัฐมนตรี ให้ความรับผิดทั้งหมดเกิดขึ้นเมื่อตัวแทนได้ส่งมอบสินค้าให้ผู้ซื้อ เว้นแต่กรณีที่ได้มีการกระทำดังต่อไปนี้เกิดขึ้นก่อนการส่งมอบสินค้าให้ผู้ซื้อ ก็ให้ถือว่าความรับผิดเกิดขึ้นเมื่อได้มีการกระทำนั้น ๆ ด้วย
(ก) ตัวแทนได้โอนกรรมสิทธิ์สินค้าให้ผู้ซื้อ
(ข) ตัวแทนได้รับชำระราคาสินค้า
(ค) ตัวแทนได้ออกใบกำกับภาษี หรือ
(ง) ได้มีการนำสินค้าไปใช้ไม่ว่าโดยตัวแทนหรือบุคคลอื่น
ทั้งนี้ โดยให้ความรับผิดเกิดขึ้นตามส่วนของการกระทำนั้น ๆ แล้วแต่กรณี
(4) การขายสินค้าโดยส่งออก ให้ความรับผิดเกิดขึ้นดังต่อไปนี้
(ก) การส่งออกนอกจากที่ระบุใน (ข) หรือ (ค) ให้ความรับผิดเกิดขึ้นเมื่อชำระอากรขาออก วางหลักประกันอากรขาออก หรือจัดให้มีผู้ค้ำประกันอากรขาออก เว้นแต่ในกรณีที่ไม่ต้องเสียอากรขาออกหรือได้รับยกเว้นอากรขาออกแล้วแต่กรณี ก็ให้ถือว่าความรับผิดเกิดขึ้นในวันที่มีการออกใบขนสินค้าขาออกตามกฎหมายว่าด้วยศุลกากร
(ข) การส่งออกในกรณีที่นำสินค้าเข้าไปในเขตปลอดอากรตามมาตรา 77/1 (14) (ก) ให้ความรับผิดเกิดขึ้นในวันที่นำสินค้าในราชอาณาจักรเข้าไปในเขตดังกล่าว
(ค) การส่งออกซึ่งสินค้าที่อยู่ในคลังสินค้าทัณฑ์บนตามกฎหมายว่าด้วยศุลกากร ให้ความรับผิดเกิดขึ้นพร้อมกับความรับผิดตามกฎหมายว่าด้วยศุลกากร
(5) การขายสินค้าที่ได้เสียภาษีมูลค่าเพิ่มในอัตราร้อยละ 0 ตามมาตรา 80/1 (5) และภายหลังได้มีการโอนกรรมสิทธิ์ในสินค้าอันทำให้ผู้รับโอนสินค้ามีหน้าที่ต้องเสียภาษีมูลค่าเพิ่มตามมาตรา 82/1 (2) ให้ความรับผิดทั้งหมดเกิดขึ้นเมื่อโอนกรรมสิทธิ์สินค้า
เพื่อเป็นการบรรเทาภาระในการยื่นแบบแสดงรายการภาษีและการชำระภาษีของผู้ประกอบการจดทะเบียนสำหรับการขายสินค้าแก่กระทรวง ทบวง กรม หรือราชการส่วนท้องถิ่น ทั้งนี้ เฉพาะการขายสินค้าตามสัญญาและมีการชำระราคาที่เป็นไปตามหลักเกณฑ์ วิธีการ และเงื่อนไขที่กำหนดโดยกฎกระทรวง ให้อธิบดีโดยอนุมัติรัฐมนตรีมีอำนาจกำหนดความรับผิดตาม (1) (2) และ (3) เกิดขึ้นเป็นอย่างอื่นได้
</law>
<law section=82/4 ประมวลรัษฎากร> ประมวลรัษฎากร มาตรา 82/4 ภายใต้บังคับมาตรา 83/5 มาตรา 83/6 และมาตรา 83/7 ให้ผู้ประกอบการจดทะเบียนเรียกเก็บภาษีมูลค่าเพิ่มจากผู้ซื้อสินค้าหรือผู้รับบริการเมื่อความรับผิดในการเสียภาษีมูลค่าเพิ่มเกิดขึ้น โดยคำนวณจากฐานภาษีตามส่วน 3 และอัตราภาษีตามส่วน 4
บทบัญญัติมาตรานี้ มิได้เป็นการห้ามผู้ประกอบการจดทะเบียนที่จะเสนอหรือแสดงราคาสินค้าหรือราคาค่าบริการแก่ผู้ซื้อสินค้าหรือผู้รับบริการ ในราคาที่รวมภาษีมูลค่าเพิ่มไว้แล้ว ทั้งนี้ ไม่ว่าผู้ประกอบการจดทะเบียนจะได้แจ้งให้ผู้ซื้อสินค้าหรือผู้รับบริการทราบด้วยหรือไม่ก็ตาม
ภาษีมูลค่าเพิ่มที่ผู้ประกอบการจดทะเบียนเรียกเก็บจากผู้ซื้อสินค้าหรือผู้รับบริการตามมาตรานี้ ย่อมเป็นภาษีขายของผู้ประกอบการจดทะเบียนนั้น
ภาษีมูลค่าเพิ่มที่ผู้ประกอบการจดทะเบียนใดถูกผู้ประกอบการจดทะเบียนอื่นเรียกเก็บตามมาตรานี้ เนื่องจากการซื้อสินค้าหรือรับบริการมาเพื่อใช้ในการประกอบกิจการของตน ย่อมเป็นภาษีซื้อของผู้ประกอบการจดทะเบียนที่เป็นผู้ซื้อสินค้าหรือผู้รับบริการนั้น
</law>
<law section=77/2 ประมวลรัษฎากร> ประมวลรัษฎากร มาตรา 77/2 การกระทำกิจการดังต่อไปนี้ในราชอาณาจักร ให้อยู่ในบังคับต้องเสียภาษีมูลค่าเพิ่ม ตามบทบัญญัติในหมวดนี้
(1) การขายสินค้าหรือการให้บริการโดยผู้ประกอบการ
(2) การนำเข้าสินค้าโดยผู้นำเข้า
การให้บริการในราชอาณาจักรให้หมายถึง บริการที่ทำในราชอาณาจักรโดยไม่คำนึงว่าการใช้บริการนั้นจะอยู่ในต่างประเทศหรือในราชอาณาจักร
การให้บริการที่ทำในต่างประเทศและได้มีการใช้บริการนั้นในราชอาณาจักรให้ถือว่าการให้บริการนั้น เป็นการให้บริการในราชอาณาจักร
</law>
<law section=86 ประมวลรัษฎากร> ประมวลรัษฎากร มาตรา 86 ภายใต้บังคับมาตรา 86/1 มาตรา 86/2 และมาตรา 86/8 ให้ผู้ประกอบการจดทะเบียนจัดทำใบกำกับภาษีและสำเนาใบกำกับภาษีสำหรับการขายสินค้า หรือการให้บริการทุกครั้ง และต้องจัดทำในทันทีที่ความรับผิดในการเสียภาษีมูลค่าเพิ่มเกิดขึ้น พร้อมทั้งให้ส่งมอบใบกำกับภาษีนั้นแก่ผู้ซื้อสินค้าหรือผู้รับบริการส่วนสำเนาใบกำกับภาษีให้เก็บรักษาไว้ตามมาตรา 87/3
ผู้ประกอบการที่ได้รับยกเว้นการจดทะเบียนภาษีมูลค่าเพิ่มและได้จดทะเบียนภาษีมูลค่าเพิ่มชั่วคราวตามมาตรา 85/3 จะออกใบกำกับภาษีได้ต่อเมื่อเป็นไปตามหลักเกณฑ์ วิธีการ และเงื่อนไขที่อธิบดีกำหนด
ใบกำกับภาษีให้ออกเป็นรายสถานประกอบการ ทั้งนี้ เว้นแต่อธิบดีจะกำหนดเป็นอย่างอื่น
การออกใบกำกับภาษีโดยตัวแทนในนามของผู้ประกอบการจดทะเบียนให้เป็นไปตามหลักเกณฑ์ วิธีการ และเงื่อนไขที่อธิบดีกำหนด
</law>
<law section=80 ประมวลรัษฎากร> ประมวลรัษฎากร มาตรา 80 ให้ใช้อัตราภาษีร้อยละ 10.0 ในการคำนวณภาษีมูลค่าเพิ่มสำหรับการประกอบกิจการดังต่อไปนี้ ทั้งนี้ เว้นแต่กรณีที่กำหนดไว้ในมาตรา 80/2
(1) การขายสินค้า
(2) การให้บริการ
(3) การนำเข้า
อัตราภาษีตามวรรคหนึ่ง ให้ลดลงได้โดยตราเป็นพระราชกฤษฎีกาแต่ต้องกำหนดอัตราภาษีให้เป็นอัตราภาษีเดียวกันสำหรับการขายสินค้าการให้บริการและการนำเข้าทุกกรณี
</law>
</ข้อกฎหมาย>
<ข้อหารือ> บริษัทฯ ได้รับคำสั่งซื้อแท้งค์จำนวน 2 ใบ ราคาใบละ 1,364,814 บาท จาก PT ซึ่งเป็นบริษัทที่จัดตั้งขึ้นในประเทศอินโดนีเซีย โดยสรุปข้อเท็จจริงได้ดังนี้
          1. บริษัทฯ ได้ว่าจ้างบริษัท ท. เป็นผู้ผลิตแท้งค์ และบริษัทฯ ได้เก็บแท้งค์ที่ผลิตเสร็จแล้วไว้ที่คลังสินค้าที่จังหวัดชลบุรี ซึ่งบริษัทฯ ได้เช่าจากบริษัท ท.
          2. PT ตกลงจะขนส่งแท้งค์ออกจากคลังสินค้าด้วยตนเอง โดยให้บริษัท ฮ. ซึ่งเป็นบริษัทในเครือของ PT ที่ตั้งอยู่ใน ประเทศไทย เป็นผู้ดำเนินการ
          3. ต่อมาบริษัท ฮ. ได้ว่าจ้าง ขนส่งแท้งค์จากคลังสินค้าของบริษัทฯ ไปยังบริษัท ฮ. สาขาจังหวัดสงขลา เพื่อส่งออก ไปให้ PT ที่ประเทศอินโดนีเซีย
          4. บริษัทฯ ได้ออกใบส่งของให้แก่ PT โดย PT เป็นผู้ชำระเงินให้แก่บริษัทฯ และมีบริษัท ฮ. สาขาจังหวัดสงขลา เป็น ผู้เซ็นรับสินค้าและดำเนินพิธีการศุลกากรเพื่อส่งออกในนามของบริษัท ฮ.
          กรณีที่บริษัทฯ ขายสินค้าให้กับ PT ตามข้อเท็จจริงดังกล่าว ถือเป็นการส่งออก ซึ่งจะได้รับสิทธิเสียภาษีมูลค่าเพิ่ม ในอัตราร้อยละ 0 หรือไม่ <\ข้อหารือ>


<assistant> ```json{"analysis": "From the given case details, the company sells two tanks to PT company which is established under Indonesia's law and PT company agrees to move the tank from the company's inventory by themselves and require the company to sign over the tank to ฮ company which is its subsidary. This is considered selling inside the Kingdom of Thailand and according to section 77/2 of ประมวลรัษฎากร, the company is subjected to 7% VAT which the company must issue the invoice to the buyer from other countries according to section 78, 80, 82/4 and 86 of ประมวลรัษฎากร.", "answer": "บริษัทฯ มีหน้าที่ต้องเสียภาษีมูลค่าเพิ่มในอัตราร้อยละ 7.0 และต้องจัดทำใบกำกับภาษีเรียกเก็บภาษีมูลค่าเพิ่มจากผู้ซื้อในต่างประเทศ", 
"citations": [{'law': 'ประมวลรัษฎากร', 'section': '77/2'}, {'law': 'ประมวลรัษฎากร', 'section': '78'}, {'law': 'ประมวลรัษฎากร', 'section': '80'}, {'law': 'ประมวลรัษฎากร', 'section': '82/4'}, {'law': 'ประมวลรัษฎากร', 'section': '86'}]}
```

\end{Verbatim}

\textbf{Turn 3: }
\begin{Verbatim}[breaklines, breaksymbol=, breakanywheresymbolpre=]

<user> <ข้อกฎหมาย>
<law section=27 ประมวลรัษฎากร> ประมวลรัษฎากร มาตรา 27 บุคคลใดไม่เสียหรือนำส่งภาษีภายในกำหนดเวลาตามที่บัญญัติไว้ในหมวดต่าง ๆ แห่งลักษณะนี้เกี่ยวกับภาษีอากรประเมิน ให้เสียเงินเพิ่มอีกร้อยละ 1.5 ต่อเดือนหรือเศษของเดือนของเงินภาษีที่ต้องเสียหรือนำส่งโดยไม่รวมเบี้ยปรับ
ในกรณีอธิบดีอนุมัติให้ขยายกำหนดเวลาชำระหรือนำส่งภาษีและได้มีการชำระหรือนำส่งภาษีภายในกำหนดเวลาที่ขยายให้นั้น เงินเพิ่มตามวรรคหนึ่งให้ลดลงเหลือร้อยละ 0.75 ต่อเดือนหรือเศษของเดือน
การคำนวณเงินเพิ่มตามวรรคหนึ่งและวรรคสอง ให้เริ่มนับเมื่อพ้นกำหนดเวลาการยื่นรายการหรือนำส่งภาษีจนถึงวันชำระหรือนำส่งภาษี แต่เงินเพิ่มที่คำนวณได้มิให้เกินจำนวนภาษีที่ต้องเสียหรือนำส่ง ไม่ว่าภาษีที่ต้องเสียหรือนำส่งนั้นจะเกิดจากการประเมินหรือคำสั่งของเจ้าพนักงานหรือคำวินิจฉัยอุทธรณ์หรือคำพิพากษาของศาล
</law>
<law section=48 ประมวลรัษฎากร> ประมวลรัษฎากร มาตรา 48 เงินได้พึงประเมินต้องเสียภาษีเงินได้ดังต่อไปนี้
(1) เงินได้พึงประเมินเมื่อได้หักตามมาตรา 42 ทวิ ถึงมาตรา 47 หรือมาตรา 57 เบญจ แล้ว เหลือเท่าใดเป็นเงินได้สุทธิ ต้องเสียภาษีในอัตราที่กำหนดในบัญชีอัตราภาษีเงินได้ท้ายหมวดนี้
(2) สำหรับผู้มีเงินได้พึงประเมินตั้งแต่ 120,000 บาทขึ้นไป การคำนวณภาษีตาม (1) ให้เสียไม่น้อยกว่าร้อยละ 0.5 ของยอดเงินได้พึงประเมิน
การนับจำนวนเงินได้พึงประเมินตาม (2) ไม่รวมถึงเงินได้พึงประเมินตามมาตรา 40 (1)
(3) ผู้มีเงินได้จะเลือกเสียภาษีในอัตราร้อยละ 15.0 ของเงินได้โดยไม่ต้องนำไปรวมคำนวณภาษีตาม (1) และ (2) ก็ได้ สำหรับเงินได้ตามมาตรา 40 (4) (ก) และ (ช) ดังต่อไปนี้
(ก) ดอกเบี้ยพันธบัตร ดอกเบี้ยเงินฝากธนาคารในราชอาณาจักร ดอกเบี้ยเงินฝากสหกรณ์ ดอกเบี้ยหุ้นกู้ ดอกเบี้ยตั๋วเงินที่ได้จากบริษัทหรือห้างหุ้นส่วนนิติบุคคลหรือนิติบุคคลอื่น ดอกเบี้ยเงินกู้ยืมที่ได้จากบริษัทหรือห้างหุ้นส่วนนิติบุคคลหรือนิติบุคคลอื่น ดอกเบี้ยที่ได้จากสถาบันการเงินที่มีกฎหมายโดยเฉพาะของประเทศไทยจัดตั้งขึ้นสำหรับให้กู้ยืมเงินเพื่อส่งเสริมเกษตรกรรม พาณิชยกรรม หรืออุตสาหกรรม
(ข) ผลต่างระหว่างราคาไถ่ถอนกับราคาจำหน่ายตั๋วเงิน หรือตราสารแสดงสิทธิในหนี้ที่บริษัทหรือห้างหุ้นส่วนนิติบุคคลหรือนิติบุคคลอื่นเป็นผู้ออก
(ค) ผลประโยชน์ที่ได้จากการโอนพันธบัตร หุ้นกู้ หรือตั๋วเงิน หรือตราสารแสดงสิทธิในหนี้ที่บริษัทหรือห้างหุ้นส่วนนิติบุคคลหรือนิติบุคคลอื่นเป็นผู้ออก ทั้งนี้ เฉพาะที่ตีราคาเป็นเงินได้เกินกว่าที่ลงทุน
ผู้มีเงินได้ซึ่งเป็นผู้อยู่ในประเทศไทยจะเลือกเสียภาษีในอัตราร้อยละ 10.0 ของเงินได้โดยไม่ต้องนำไปรวมคำนวณภาษีตาม (1) และ (2) ก็ได้สำหรับเงินได้ตามมาตรา 40 (4) (ข) ที่ได้รับจากบริษัทหรือห้างหุ้นส่วนนิติบุคคลที่ตั้งขึ้นตามกฎหมายไทย หรือสถาบันการเงินที่มีกฎหมายโดยเฉพาะของประเทศไทยจัดตั้งขึ้นสำหรับให้กู้ยืมเงินเพื่อส่งเสริมเกษตรกรรม พาณิชยกรรม หรืออุตสาหกรรม
(4) ผู้มีเงินได้จะเลือกเสียภาษีโดยไม่ต้องนำไปรวมคำนวณภาษีตาม(1) และ (2) ก็ได้ เฉพาะเงินได้ตามมาตรา 40 (8) ที่ได้รับจากการขายอสังหาริมทรัพย์อันเป็นมรดกหรืออสังหาริมทรัพย์ที่ได้มาโดยมิได้มุ่งในทางการค้าหรือหากำไรดังต่อไปนี้
(ก) เงินได้จากการขายอสังหาริมทรัพย์อันเป็นมรดกหรืออสังหาริมทรัพย์ที่ได้รับจากการให้โดยเสน่หาให้หักค่าใช้จ่ายร้อยละ 50 ของเงินได้ เหลือเท่าใดถือเป็นเงินได้สุทธิแล้วหารด้วยจำนวนปีที่ถือครองได้ผลลัพธ์เป็นเงินเท่าใดให้คำนวณภาษีตามอัตราภาษีเงินได้ได้เท่าใดให้คูณด้วยจำนวนปีที่ถือครองผลลัพธ์ที่ได้เป็นเงินภาษีที่ต้องเสีย
(ข)เงินได้จากการขายอสังหาริมทรัพย์ที่ได้มาโดยทางอื่นนอกจาก (ก) ให้หักค่าใช้จ่ายได้ตามที่กำหนดโดยพระราชกฤษฎีกาเหลือเท่าใดถือเป็นเงินได้สุทธิ แล้วหารด้วยจำนวนปีที่ถือครองได้ผลลัพธ์เป็นเงินเท่าใด ให้คำนวณภาษีตามอัตราภาษีเงินได้ ได้เท่าใดให้คูณด้วยจำนวนปีที่ถือครองผลลัพธ์ที่ได้เป็นเงินภาษีที่ต้องเสีย
ในกรณีที่เสียภาษีโดยไม่นำไปรวมคำนวณภาษีตาม (1) และ (2) เมื่อคำนวณภาษีแล้วต้องเสียไม่เกินร้อยละ 20 ของราคาขาย
ในกรณีที่เสียภาษีโดยนำไปรวมคำนวณภาษีตาม (1) และ (2) ให้หักค่าใช้จ่ายร้อยละ 50 ของเงินได้ตาม (ก) หรือตามที่กำหนดโดยพระราชกฤษฎีกาตาม (ข) แล้วแต่กรณี เหลือเท่าใดนำไปรวมคำนวณภาษีกับเงินได้อย่างอื่น
คำว่า “จำนวนปีที่ถือครอง” ใน (ก) หรือ (ข) หมายถึงจำนวนปีนับตั้งแต่ปีที่ได้กรรมสิทธิ์หรือสิทธิครอบครองในอสังหาริมทรัพย์ ถึงปีที่โอนกรรมสิทธิ์หรือสิทธิครอบครองในอสังหาริมทรัพย์นั้น ถ้าเกินสิบปีให้นับเพียงสิบปี และเศษของปีให้นับเป็นหนึ่งปี
(4/1) ผู้มีเงินได้จะเลือกเสียภาษีในอัตราร้อยละ 5 ของเงินได้ส่วนที่เกินยี่สิบล้านบาท โดยไม่ต้องนำไปรวมคำนวณภาษีตาม (1) และ (2) ก็ได้ สำหรับเงินได้ตามมาตรา 40 (8) ที่เป็นเงินได้จากการโอนกรรมสิทธิ์หรือสิทธิครอบครองในอสังหาริมทรัพย์โดยไม่มีค่าตอบแทนให้แก่บุตรชอบด้วยกฎหมายซึ่งไม่รวมถึงบุตรบุญธรรม ที่ไม่ได้รับยกเว้นตามมาตรา 42 (26)
(5) ผู้มีเงินได้จะเลือกเสียภาษีโดยไม่ต้องนำไปรวมคำนวณภาษีตาม (1) และ (2) ก็ได้ สำหรับเงินได้พึงประเมินตามมาตรา 40 (1) และ (2) ซึ่งเป็นเงินที่นายจ้างจ่ายให้ครั้งเดียวเพราะเหตุออกจากงาน ซึ่งได้คำนวณจ่ายจากระยะเวลาที่ทำงานและได้จ่ายตามหลักเกณฑ์ วิธีการ และเงื่อนไขที่อธิบดีกำหนด โดยให้นำเงินได้พึงประเมินดังกล่าวหักค่าใช้จ่ายเป็นจำนวนเท่ากับ 7,000 บาท คูณด้วยจำนวนปีที่ทำงานแต่ไม่เกินเงินได้พึงประเมิน เหลือเท่าใดให้หักค่าใช้จ่ายอีกร้อยละ 50 ของเงินที่เหลือนั้นแล้วคำนวณภาษีตามอัตราภาษีเงินได้
ในกรณีเงินได้พึงประเมินดังกล่าวจ่ายในลักษณะเงินบำเหน็จจำนวนหนึ่งและเงินบำนาญอีกจำนวนหนึ่ง ให้ถือว่าเฉพาะเงินที่จ่ายในลักษณะเงินบำเหน็จเป็นเงินซึ่งนายจ้างจ่ายให้ครั้งเดียวเพราะเหตุออกจากงานและให้ลดค่าใช้จ่ายจำนวน 7,000 บาท ลงเหลือ 3,500 บาท
จำนวนปีที่ทำงานตามวรรคหนึ่ง ในกรณีเงินบำเหน็จหรือเงินอื่นใดในลักษณะเดียวกัน ที่ทางราชการจ่าย ให้ถือจำนวนปีที่ใช้เป็นเกณฑ์คำนวณเงินบำเหน็จหรือเงินอื่นในลักษณะเดียวกันนั้นตามกฎหมาย ระเบียบ หรือข้อบังคับของทางราชการ
ในการคำนวณจำนวนปีที่ทำงาน นอกจากกรณีตามวรรคสาม เศษของปี ถ้าถึงหนึ่งร้อยแปดสิบสามวันให้ถือเป็นหนึ่งปี ถ้าไม่ถึงหนึ่งร้อยแปดสิบสามวันให้ปัดทิ้ง
(6) ผู้มีเงินได้จะเลือกเสียภาษีในอัตราร้อยละ 5 ของเงินได้ส่วนที่เกินยี่สิบล้านบาท โดยไม่ต้องนำไปรวมคำนวณภาษีตาม (1) และ (2) ก็ได้ สำหรับเงินได้ตามมาตรา 40 (8) ที่ได้รับจากการอุปการะหรือจากการให้โดยเสน่หาจากบุพการี ผู้สืบสันดาน หรือคู่สมรส ที่ไม่ได้รับยกเว้นตามมาตรา 42 (27)
(7) ผู้มีเงินได้จะเลือกเสียภาษีในอัตราร้อยละ 5 ของเงินได้ส่วนที่เกินสิบล้านบาท โดยไม่ต้องนำไปรวมคำนวณภาษีตาม (1) และ (2) ก็ได้ สำหรับเงินได้ตามมาตรา 40 (8) ที่ได้รับจากการอุปการะโดยหน้าที่ธรรมจรรยาหรือจากการให้โดยเสน่หาเนื่องในพิธีหรือตามโอกาสแห่งขนบธรรมเนียมประเพณี ทั้งนี้ จากบุคคลซึ่งมิใช่บุพการี ผู้สืบสันดาน หรือคู่สมรส ที่ไม่ได้รับยกเว้นตามมาตรา 42 (28)
</law>
<law section=50 ประมวลรัษฎากร> ประมวลรัษฎากร มาตรา 50 ให้บุคคล ห้างหุ้นส่วน บริษัท สมาคม หรือคณะบุคคลผู้จ่ายเงินได้พึงประเมินตามมาตรา 40 หักภาษีเงินได้ไว้ทุกคราวที่จ่ายเงินได้พึงประเมินตามวิธีดังต่อไปนี้
(1)ในกรณีเงินได้พึงประเมินตามมาตรา 40 (1) และ (2) ให้คูณเงินได้พึงประเมินที่จ่ายด้วยจำนวนคราวที่จะต้องจ่ายเพื่อให้ได้จำนวนเงินเสมือนหนึ่งว่าได้จ่ายทั้งปี แล้วคำนวณภาษีตามเกณฑ์ในมาตรา 48เป็นเงินภาษีทั้งสิ้นเท่าใดให้หารด้วยจำนวนคราวที่จะต้องจ่ายได้ผลลัพธ์เป็นเงินเท่าใด ให้หักเป็นเงินภาษีไว้เท่านั้น
ถ้าการหารด้วยจำนวนคราวที่จะต้องจ่ายตามความในวรรคก่อนไม่ลงตัวเหลือเศษเท่าใดให้เพิ่มเงินเท่าจำนวนที่เหลือเศษนั้นรวมเข้ากับเงินภาษีที่จะต้องหักไว้ครั้งสุดท้ายในปีนั้นเพื่อให้ยอดเงินภาษีที่หักรวมทั้งปีเท่ากับจำนวนภาษีที่จะต้องเสียทั้งปี
ในกรณีเงินได้พึงประเมินตามมาตรา 40 (1) และ (2) ซึ่งเป็นเงินที่นายจ้างจ่ายให้ครั้งเดียวเพราะเหตุออกจากงาน ซึ่งได้คำนวณจ่ายจากระยะเวลาที่ทำงานและได้จ่ายตามหลักเกณฑ์ วิธีการ และเงื่อนไขที่อธิบดีกำหนด ให้คำนวณภาษีตามเกณฑ์ในมาตรา 48 (5) เป็นเงินภาษีทั้งสิ้นเท่าใดให้หักเป็นเงินภาษีไว้เท่านั้น
ในกรณีเงินได้พึงประเมินตามมาตรา 40 (2) นอกจากที่ระบุไว้ในวรรคสามที่จ่ายให้แก่ผู้รับซึ่งมิได้เป็นผู้อยู่ในประเทศไทย ให้คำนวณหักในอัตราร้อยละ 15.0 ของเงินได้
(2) ในกรณีเงินได้พึงประเมินตามมาตรา 40 (3) และ (4) ให้คำนวณหักตามอัตราภาษีเงินได้ เว้นแต่
(ก) ในกรณีเงินได้พึงประเมินตามมาตรา 40 (3) และ (4) นอกจากที่ระบุไว้ใน (ข) (ค) (ง) และ (จ) ที่จ่ายให้แก่ผู้รับซึ่งมิได้เป็นผู้อยู่ในประเทศไทยให้คำนวณหักในอัตราร้อยละ 15.0 ของเงินได้
(ข) ในกรณีเงินได้พึงประเมินที่ระบุในมาตรา 48 (3) (ก) และ (ค) ให้คำนวณหักในอัตราร้อยละ 15.0 ของเงินได้
(ค) ในกรณีเงินได้พึงประเมินที่ระบุในมาตรา 48 (3) (ข) ให้ถือว่าผู้ออกตั๋วเงิน ผู้ออกตราสารแสดงสิทธิในหนี้ หรือนิติบุคคลผู้โอนตั๋วเงินหรือตราสารดังกล่าว ให้แก่ผู้มีหน้าที่เสียภาษีเงินได้ตามส่วนนี้ เป็นผู้จ่ายเงินได้พึงประเมิน และให้เรียกเก็บภาษีเงินได้จากผู้มีเงินได้ในอัตราร้อยละ 15.0 ของเงินได้และให้ถือว่าภาษีที่เรียกเก็บนั้นเป็นภาษีที่หักไว้
(ง) ในกรณีเงินได้พึงประเมินตามมาตรา 40 (4) (ก) ที่มิได้ระบุใน (ข) และ (ค) แห่งมาตรานี้ ถ้าผู้จ่ายเงินได้มิใช่เป็นนิติบุคคล และจ่ายให้แก่ผู้รับซึ่งเป็นผู้อยู่ในประเทศไทย ไม่ต้องหักภาษีตามมาตรานี้
(จ) ในกรณีเงินได้พึงประเมินตามมาตรา 40 (4) (ข) ให้คำนวณหักในอัตราร้อยละ 10.0 ของเงินได้
(ฉ) ในกรณีเงินได้พึงประเมินตามมาตรา 40 (4) (ซ) และ (ฌ) ให้คำนวณหักในอัตราร้อยละ 15.0 ของเงินได้
(3)ในกรณีเงินได้พึงประเมินตามมาตรา 40 (5) และ (6) ที่จ่ายให้แก่ผู้รับซึ่งมิได้เป็นผู้อยู่ในประเทศไทยให้คำนวณหักในอัตราร้อยละ 15.0 ของเงินได้
(4) นอกจากกรณีตาม (5) ในกรณีผู้จ่ายเงินตามมาตรานี้เป็นรัฐบาล องค์การของรัฐบาล เทศบาล สุขาภิบาล หรือองค์การบริหารราชการส่วนท้องถิ่นอื่น ซึ่งจ่ายเงินได้พึงประเมินตามมาตรา 40 (5) (6) (7) หรือ (8) แต่ไม่รวมถึงการจ่ายค่าซื้อพืชผลทางการเกษตร ให้กับผู้รับรายหนึ่ง ๆ มีจำนวนรวมทั้งสิ้นตั้งแต่ 10,000 บาทขึ้นไป แม้การจ่ายนั้นจะได้แบ่งจ่ายครั้งหนึ่ง ๆ ไม่ถึง 10,000 บาทก็ดี ให้คำนวณหักในอัตราร้อยละ 1 ของยอดเงินได้พึงประเมิน แต่เฉพาะเงินได้ในการประกวดหรือแข่งขันให้คำนวณหักตามอัตราภาษีเงินได้
(5)ในกรณีเงินได้พึงประเมินตามมาตรา 40 (8) เฉพาะที่จ่ายให้แก่ผู้รับซึ่งขายอสังหาริมทรัพย์ ให้คำนวณหักดังต่อไปนี้
(ก) สำหรับอสังหาริมทรัพย์อันเป็นมรดกหรืออสังหาริมทรัพย์ที่ได้รับจากการให้โดยเสน่หา ให้คำนวณภาษีตามเกณฑ์ในมาตรา 48 (4) (ก) เป็นเงินภาษีทั้งสิ้นเท่าใด ให้หักเป็นเงินภาษีไว้เท่านั้น
(ข)สำหรับอสังหาริมทรัพย์ที่ได้มาโดยทางอื่นนอกจาก (ก) ให้หักค่าใช้จ่ายเป็นการเหมาตามที่กำหนดโดยพระราชกฤษฎีกาแล้วคำนวณภาษีตามเกณฑ์ในมาตรา 48 (4) (ข) เป็นเงินภาษีทั้งสิ้นเท่าใดให้หักเป็นเงินภาษีไว้เท่านั้น
(6) ในกรณีการโอนกรรมสิทธิ์หรือสิทธิครอบครองในอสังหาริมทรัพย์โดยไม่มีค่าตอบแทน ให้ถือว่าผู้โอนเป็นผู้จ่ายเงินได้ โดยให้ผู้โอนหักภาษีตามเกณฑ์ใน (5) เว้นแต่กรณีการโอนให้แก่บุตรชอบด้วยกฎหมายซึ่งไม่รวมถึงบุตรบุญธรรม ให้ผู้โอนหักภาษีไว้ร้อยละ 5 ของเงินได้เฉพาะในส่วนที่เกินยี่สิบล้านบาท
</law>
</ข้อกฎหมาย>
<ข้อหารือ> บริษัท ก จำกัด และบริษัท ข จำกัด เป็นบริษัทในเครือเดียวกันของบริษัท ค ประเทศ
เยอรมัน โดยที่บริษัท ข ประกอบธุรกิจบริหารการขายและการตลาดผลิตภัณฑ์เวชภัณฑ์ซึ่งผลิตโดยบริษัท ก
ต่อมาได้มีการปรับโครงสร้างบริษัททั้งสองใหม่ โดยแยกส่วนบริหารการขายและการตลาดเฉพาะผลิตภัณฑ์
เวชภัณฑ์ของบริษัท ข มารวมกับบริษัท ก และได้โอนย้ายพนักงานผู้ทำงานในส่วนดังกล่าวมาอยู่กับบริษัทฯ
ด้วย โดยได้รับความยินยอมเป็นหนังสือจากพนักงาน ทั้งมีเงื่อนไขว่าบริษัทฯ จะนับอายุการทำงานของ
พนักงานต่อเนื่องไปด้วยและบริษัท ข มิได้บอกกล่าวเลิกจ้างพนักงานและมิได้จ่ายเงินชดเชยใด ๆ บริษัท
ก จึงขอทราบเกี่ยวกับเงินที่นายจ้างจ่ายให้ครั้งเดียวเพราะเหตุออกจากงาน ดังนี้
1. พนักงานสองคนได้ทำงานให้กับบริษัท ข คนที่ 1 ทำงานเป็นเวลา 20 ปี พนักงานคนที่
2 ทำงานเป็นเวลา 3 ปี ต่อมาเมื่อโอนมาทำงานกับบริษัท ก ปรากฏว่าคนที่ 1 ทำงานกับบริษัท ก เป็น
เวลา 2 ปี (รวมทำงาน 22 ปี) และคนที่ 2 ทำงานกับบริษัท ก เป็นเวลา 2 ปี (รวมทำงาน 5 ปี)
พนักงานทั้งสองจึงได้ลาออกจากงานหรือถูกเลิกจ้าง และได้รับเงินชดเชยตามกฎหมายคุ้มครองแรงงาน
จากบริษัท ก กรณีนี้พนักงานทั้งสองจะมีสิทธิเลือกเสียภาษีตามมาตรา 48 (5) แห่งประมวลรัษฎากร
หรือไม่ และบริษัท ก จะต้องคำนวณภาษีเงินได้หัก ณ ที่จ่ายตามเกณฑ์ในมาตรา 48 (5) แห่ง
ประมวลรัษฎากร ใช่หรือไม่
2. กรณีตาม 1. บริษัท ก ได้ออกภาษีให้แก่พนักงาน แต่บริษัท ก ได้คำนวณหักภาษี ณ
ที่จ่าย สำหรับเงินที่นายจ้างจ่ายให้ครั้งเดียวเพราะเหตุออกจากงานตามมาตรา 50 (1) แห่ง
ประมวลรัษฎากรไปแล้ว บริษัท ก จะมีสิทธิขอคืนภาษีเงินได้หัก ณ ที่จ่ายที่นำส่งไว้เกินหรือไม่ และมี
เงื่อนไขหรือข้อจำกัดอย่างไร <\ข้อหารือ>

<assistant> ```json{"analysis": "From the given case details, there are two cases to be considered. The first case is when the company has transfered the provident fund of both employees to the company ก without paying one-time fee to the employees. In this case, the working years of the employees will be continued after their working years in company ข and the employees can choose to pay their tax calculated apart from other income according to section 48(5) of ประมวลรัษฎากร since they have both been working for more than 5 years and the company should deduct their fee accordingly. Therefore, if the company deduct income according to section 50(1) of ประมวลรัษฎากร instead of section 48(5), the company should 1) deliver the additional tax in case it deducts less than actual according to section 27 of ประมวลรัษฎากร 2) inform the employees that they can receive their tax refund in case it deducts more than actual. Lastly, in the second case where the company has not transfered the provident fund of both employees to the company ก, the employees' working years cannot be counted together with their previous company and therefore, there is no special case in tax calculation.", "answer": "พนักงานทั้งสองมีสิทธิเลือกเสียภาษีตามมาตรา 48 (5) แห่งประมวลรัษฎากร และบริษัท ก ต้องคำนวณภาษีเงินได้หัก ณ ที่จ่ายตามเกณฑ์ในมาตรา 48 (5) แห่งประมวลรัษฎากร แต่บริษัท ก ไม่มีสิทธิขอคืนเงินภาษีที่ถูกหักไว้เกินได้",
"citations": [{'law': 'ประมวลรัษฎากร', 'section': '27'}, {'law': 'ประมวลรัษฎากร', 'section': '48'}, {'law': 'ประมวลรัษฎากร', 'section': '50'}]}
```
\end{Verbatim}

% \subsection{Parametric system}
% The prompts used for generating response in a parametric setup as described in Section~\ref{subsubsec: e2e_best_setup} are detailed in this section. The few-shot examples for this system are the same as the one used in the RAG-based system but without the relevant laws provided. The system prompt and instruction prompt are somewhat modified and displayed here.

% \input{prompts/response_param_wcx}
% Below are the system and instruction prompts we utilized for generating response in a Parametric system on Tax Case dataset.

\textbf{System Prompt: }
\begin{Verbatim}[breaklines, breaksymbol=, breakanywheresymbolpre=]

You are an expert lawyer in Thai tax legal domain who never make mistakes when reviewing a tax cases. The structure of the Thai Tax case you are proficient in consists of the context and question wrapped under the tag <ข้อหารือ></ข้อหารือ>. From your knowledge of Thai legal system, you must recall laws that might be relevant to the ข้อหารือ posed and answer them accurately.
\end{Verbatim}

\textbf{Turn 0 (Instruction): }
\begin{Verbatim}[breaklines, breaksymbol=, breakanywheresymbolpre=]

<user> Take a deep breath and think carefully. Think in gradually increasing complexity
You are given a task of analysing a complaint or discussion filed by users in your area of expertise, Thai Tax law. 
Given the case details (<ข้อหารือ>), you need to give a response to the question or inquiry made by the complainants based on legal reasoning with laws that you think are relevant from all possible laws as your reference. You do not need to explain your rationale and how you reach the answer. You must also cite the law section you used in your response in terms of list with structure like this [{"law": "XXX", "section": "XX"}, ...]. The cited law must be from the the realm of all possible laws only and nothing outside of it. You should always cite at least a law. Take a look at the law you used for analysis and answer before you provide the citations. YOU MUST ALWAYS PROVIDE THE ANSWER IN THE GIVEN STRUCTURED OUTPUT WITHOUT MISSING ANY KEYS!! THIS IS REALLY IMPORTANT
    
Your final output should be a JSON object with the following keys:
```json
{
  "analysis": "Thorough analysis of the provided cases here. Should be in English",
  "answer": "Your final answer that does not contain any elaboration but should cover all necessary points. Must be in THAI only.",
  "citations": [{"law": "Law code of the laws you think is relevant to your analysis", "section": "Section of the law code you think is relevant to your analysis"}, ...]
}
```
    
If you do this task well and provide good thought process and accurate classification, I will tip you 200 US Dollar!

<assistant> I will strictly adhere to the guidelines you told me for this task
\end{Verbatim}


% \subsection{LCLM system}
% Similarly, the prompts used for generating response in the LCLM system described in Section~\ref{subsubsec: e2e_best_setup} only differs slightly from the one used in Parametric System in terms of the system prompt and the instruction prompt. They are displayed here.

% Below are the system and instruction prompts we utilized for generating response in a LCLM system on WCX dataset.

\textbf{System Prompt: }
\begin{Verbatim}[breaklines, breaksymbol=, breakanywheresymbolpre=]

You are an expert paralegal in Thai law domain. You have a job to answer legal questions posed by civillians and rarely make mistakes. The structure of questions you are proficient in consists of the question under the tag <question></question> and your answer. You are also given all possible laws under the tag <ข้อกฎหมาย> with each law inside a subtag of format <law section=XX law_name=XX></law> from which you will select relevant laws to the question asked in order to answer accurately
\end{Verbatim}

\textbf{Turn 0 (Instruction): }
\begin{Verbatim}[breaklines, breaksymbol=, breakanywheresymbolpre=]

<user> Take a deep breath and think carefully. Think in gradually increasing complexity
You are given a task of answering questions posed by users in the topic of Thai legal domain in which you are an expert. Given the question (<question>) and all laws under the tag <ข้อกฎหมาย> with each law inside a subtag of format <law section=XX law_name=XX></law>, you need to give a response to the question or inquiry made by the complainants based on legal reasoning with laws that you think are relevant from all possible laws as your reference. You do not need to explain your rationale and how you reach the answer. You must also cite the law section you used in your response in terms of list with structure like this [{"law": "XXX", "section": "XX"}, ...]. The cited law must be from the the realm of all possible laws only and nothing outside of it. You should always cite at least a law. Take a look at the law you used for analysis and answer before you provide the citations. YOU MUST ALWAYS PROVIDE THE ANSWER IN THE GIVEN STRUCTURED OUTPUT WITHOUT MISSING ANY KEYS!! THIS IS REALLY IMPORTANT

Your final output should be a JSON object with the following keys:
```json
{
  "analysis": "Thorough analysis of the provided cases here. Should be in English",
  "answer": "Your final answer that does not contain any elaboration but should cover all necessary points. Must be in THAI only.",
  "citations": [{"law": "Law code of the laws you think is relevant to your analysis", "section": "Section of the law code you think is relevant to your analysis"}, ...]
}
```
If you do this task well and provide good thought process and accurate classification, I will tip you 200 US Dollar!

<assistant> I will strictly adhere to the guidelines you told me for this task
\end{Verbatim}

% \input{prompts/response_lclm_tax}

% As for o1 system discussed in Section~\ref{subsec: reasoning_model}, the prompt used in that experiment is the same as the one in the RAG-based system but without the few-shot examples. Furthermore, since o1 model does not allow modifying system prompt, we combine the system prompt into the instruction prompt.

% Lastly, for LCLM-as-a-retriever task discussed in Section~\ref{subsec: rq3_result}, we opt to use zero-shot approach by providing the LLM with system and instruction prompt only. The full prompt is provided below:

% \input{prompts/retriever_lclm}

% \section{Metric Prompt}
% We provide the LLM-as-a-judge prompt for assessing the response on Coverage and Contradiction as described in Section~\ref{subsec: metric} in this section. The system prompt is the same as the one used in RAG-based system.

% Below are the prompts we utilized for evaluating system response against a reference response on the metric of Coverage and Contradiction on Tax Case dataset. Instruction prompt is the same as the one used in WCX dataset


\textbf{Turn 1: }
\begin{Verbatim}[breaklines, breaksymbol=, breakanywheresymbolpre=]

<user> <ข้อหารือ> บริษัทฯ เป็นผู้ประกอบการจดทะเบียน ประกอบกิจการซื้อมาขายไปกาแฟกระป๋อง หัวน้ำเชื้อ และหัวน้ำเชื้อชาเขียว ช. ได้ยื่นคำร้องของดเบี้ยปรับภาษีมูลค่าเพิ่ม สำหรับเดือนภาษีเมษายน 2551 เดือนภาษีพฤษภาคม 2551 เดือนภาษีกรกฎาคม 2551 และเดือนภาษีตุลาคม 2551 โดยมีข้อเท็จจริงว่า เจ้าพนักงานประเมินได้ตรวจ0สอบการขอคืนภาษี กรณีที่บริษัทฯ ยื่นแบบแสดง รายการภาษีมูลค่าเพิ่ม (ภ.พ.30) ฉบับยื่นปกติรวมภาษีที่ขอคืนเป็นเงินจำนวน 1,510,423.08 บาท\n บริษัทฯ ได้จดทะเบียนเลิกบริษัทต่อกระทรวงพาณิชย์เมื่อวันที่ 30 มกราคม 2551 พร้อมทั้งยื่นแบบคำขอแจ้งการเปลี่ยนแปลง ทะเบียนภาษีมูลค่าเพิ่ม (ภ.พ.09) แจ้งเลิกต่อกรมสรรพากรเมื่อวันที่ 14 กุมภาพันธ์ 2551 ภาษีซื้อที่บริษัทฯ นำมาใช้ในการยื่นแบบ ภ.พ.30 สำหรับเดือนภาษีดังกล่าวเป็นภาษีซื้อที่เกิดจากการส่งเสริมการขายและค่าที่ปรึกษาทางกฎหมาย ซึ่งบริษัทฯ ได้รับบริการ ก่อนวันแจ้งเลิกการประกอบกิจการ แต่ได้รับใบกำกับภาษีภายหลังวันที่ยื่นแบบ ภ.พ.09 โดยใบกำกับภาษีลงวันที่ 4 เมษายน 2551 วันที่ 8 พฤษภาคม 2551 วันที่ 3 กรกฎาคม 2551 และวันที่ 30 ตุลาคม 2551 กรณีดังกล่าวหากบริษัทฯ ยังไม่ถูกขีดชื่อออกจากทะเบียนภาษีมูลค่าเพิ่ม และได้นำภาษีซื้อตามใบกำกับภาษีดังกล่าวมาใช้ในการคำนวณภาษีมูลค่าเพิ่ม โดยมิได้แสดงรายละเอียด การใช้สิทธิขอนำภาษีซื้อตามใบกำกับภาษีดังกล่าวมาหักในการคำนวณภาษีในเอกสารตามที่อธิบดีกำหนดพร้อมกับการยื่นแบบ ภ.พ.09 ตามข้อ 6 ของคำสั่งกรมสรรพากรที่ ป.66/2539 เรื่อง การแจ้งเลิกประกอบกิจการตามมาตรา 85/15 แห่งประมวลรัษฎากร ลงวันที่ 29 ธันวาคม พ.ศ. 2539\n 1. บริษัทฯ มีสิทธินำภาษีซื้อตามใบกำกับภาษีดังกล่าวมาใช้ในการคำนวณภาษีมูลค่าเพิ่มในการยื่นแบบ ภ.พ.30 ภายหลัง การยื่นแบบ ภ.พ.09 ได้หรือไม่\n 2. หากบริษัทฯ ไม่มีสิทธินำภาษีซื้อตามใบกำกับภาษีดังกล่าวมาใช้ตามข้อ 1. บริษัทฯ จะต้องเสียเบี้ยปรับกรณีนำภาษีซื้อ มาใช้เกินไปตามมาตรา 89(4) แห่งประมวลรัษฎากรหรือไม่ และบริษัทฯ จะต้องยื่นแบบ ค.10 เพื่อขอคืนภาษีใช่หรือไม่ </ข้อหารือ>
<reference_answer> 1. บริษัทฯ ไม่มีสิทธินำภาษีซื้อตามใบกำกับภาษีดังกล่าวมาใช้ในการคำนวณภาษีมูลค่าเพิ่มในการยื่นแบบ ภ.พ.30 ภายหลังการยื่นแบบ ภ.พ.09; 2. บริษัทฯ จะต้องเสียเบี้ยปรับกรณีนำภาษีซื้อมาใช้เกินไปตามมาตรา 89(4) และต้องยื่นแบบ ค.10 เพื่อขอคืนภาษี </reference_answer>
<student_answer> 1. บริษัทฯ ไม่มีสิทธินำภาษีซื้อตามใบกำกับภาษีดังกล่าวมาใช้ในการคำนวณภาษีมูลค่าเพิ่มในการยื่นแบบ ภ.พ.30 ภายหลังการยื่นแบบ ภ.พ.09 ได้ 2. บริษัทฯ จะต้องเสียเบี้ยปรับกรณีนำภาษีซื้อมาใช้เกินไปตามมาตรา 89(4) แห่งประมวลรัษฎากร และบริษัทฯ จะต้องยื่นแบบ ค.10 เพื่อขอคืนภาษี </student_answer>

<assistant> ```json
{"point_thought": "From the given reference answer, the student answer and ข้อหารือ, the student answer is making two points as explicitly stated by the numbered list. The reference answer is also making two points stated by the numbered list as well. These two points corresponds with the two questions made by the ข้อหารือ. Therefore, I should put these two points in lists",
 "student_point": ["1. บริษัทฯ ไม่มีสิทธินำภาษีซื้อตามใบกำกับภาษีดังกล่าวมาใช้ในการคำนวณภาษีมูลค่าเพิ่มในการยื่นแบบ ภ.พ.30 ภายหลังการยื่นแบบ ภ.พ.09 ได้", "2. บริษัทฯ จะต้องเสียเบี้ยปรับกรณีนำภาษีซื้อมาใช้เกินไปตามมาตรา 89(4) แห่งประมวลรัษฎากร และบริษัทฯ จะต้องยื่นแบบ ค.10 เพื่อขอคืนภาษี"],
 "reference_point": ["1. บริษัทฯ ไม่มีสิทธินำภาษีซื้อตามใบกำกับภาษีดังกล่าวมาใช้ในการคำนวณภาษีมูลค่าเพิ่มในการยื่นแบบ ภ.พ.30 ภายหลังการยื่นแบบ ภ.พ.09;", "2. บริษัทฯ จะต้องเสียเบี้ยปรับกรณีนำภาษีซื้อมาใช้เกินไปตามมาตรา 89(4) และต้องยื่นแบบ ค.10 เพื่อขอคืนภาษี"],
 "coverage": {"thought": "From the extracted student and reference points, there are two points in the reference answer and in the student answer and the student's answer covers both point of the reference about how the company has no right to use buying tax from the said tax invoice to calculate its VAT when submitting ภ.พ.30 after filing ภ.พ.09. Furthermore, the student answer also covers the point about how the company is subjected to a penalty for using excess input tax under Section 89(4) and must file a return tax form ค.10. Therefore, since the student answer covers all aspect of the reference answer, the student answer has full coverage.", "score": "full-coverage"}, "contradiction": {"thought": "From the points extracted from reference and student answer, point 1 of the student does not directly contradict with the any points from the reference answer same with point 2 of the student. Therefore, the contradiction score would be 0", "score": "no-contradiction"}}
```


\end{Verbatim}

\textbf{Turn 2: }
\begin{Verbatim}[breaklines, breaksymbol=, breakanywheresymbolpre=]

<user> <ข้อหารือ> บริษัท ธ. ประกอบกิจการให้บริการทางกฎหมายแก่บริษัทเงินทุน ธ. และบริษัทในเครือของ
บริษัทเงินทุน ธ. ซึ่งในการให้บริการทางกฎหมายจะแบ่งการให้บริการออกเป็น 2 ประเภท ดังต่อไปนี้
1. การให้บริการงานกฎหมายสำหรับร่างและตรวจสอบสัญญาและเอกสารทางกฎหมาย
รวมทั้งการให้คำปรึกษาทางกฎหมาย ตลอดจนประเมินราคาด้วย
2. การให้บริการงานกฎหมายสำหรับงานว่าความมีขอบเขตให้บริการเริ่มตั้งแต่ติดตาม
ทวงถาม เร่งรัดหนี้ ฟ้องคดี และบังคับคดี
บริษัทฯ จึงหารือว่า การให้บริการทั้ง 2 กรณีดังกล่าว จะได้รับยกเว้นภาษี มูลค่าเพิ่มหรือไม่ </ข้อหารือ>
<reference_answer> การให้บริการงานว่าความได้รับการยกเว้นไม่ต้องเสียภาษีมูลค่าเพิ่ม ส่วนค่าบริการอื่น ๆ ไม่ได้รับการยกเว้นภาษีมูลค่าเพิ่มแต่อย่างใด </reference_answer>
<student_answer> การให้บริการทั้ง 2 กรณีดังกล่าวจะไม่ได้รับยกเว้นภาษีมูลค่าเพิ่ม และจะต้องเสียภาษีมูลค่าเพิ่มตามอัตราปกติ </student_answer>


<assistant> ```json
{"point_thought": "From the given reference answer, the student answer and ข้อหารือ, the reference answer is making two points as clearly stated which are that the legal service is exempted from VAT and that other services are not. As for student answer, as though it may seems it is making one point only, it is actually two point which is that legal service is not exempted from VAT and that other services are not. Therefore, I should put all the points in a list accordingly.",
"student_point": ["การให้บริการงานว่าความจะไม่ได้รับยกเว้นภาษีมูลค่าเพิ่ม และจะต้องเสียภาษีมูลค่าเพิ่มตามอัตราปกติ", "การให้บริการอื่นๆจะไม่ได้รับยกเว้นภาษีมูลค่าเพิ่ม และจะต้องเสียภาษีมูลค่าเพิ่มตามอัตราปกติ"],
"reference_point": ["การให้บริการงานว่าความได้รับการยกเว้นไม่ต้องเสียภาษีมูลค่าเพิ่ม", "ค่าบริการอื่น ๆ ไม่ได้รับการยกเว้นภาษีมูลค่าเพิ่มแต่อย่างใด"],
"coverage": {"thought": "From the extracted points of the student, the student is making two points: 1. Legal service is not exempted from VAT 2. Other services are not exempted from VAT. I should judge it point by point as stated by the caution. The second point aligns well with the reference answer first point which says that the other service is not exempted from VAT while the first point contradicts with the reference answer's first point which says that legal service is exempted from VAT. Therefore, the student answer covers 1 out of 2 points made by reference answer and should get a score of partial-coverage", "score": "partial-coverage"}, "contradiction": {"thought": "Since the student answer has two points, I should evaluate point by point to see if any point contradict with the law or the reference answer. The first point of the student directly contradicts with the first point of the reference answer which says that legal service is exempted from VAT. Therefore, the student answer should get contradiction score of contradiction.", "score": "contradiction"}}
```
\end{Verbatim}

\textbf{Turn 3: }
\begin{Verbatim}[breaklines, breaksymbol=, breakanywheresymbolpre=]
<user> <ข้อหารือ> นาย ช. ได้ขออนุมัติเปลี่ยนแปลงวิธีการยื่นรายการและเสียภาษีสำหรับปีภาษี 2555 มีข้อเท็จจริงสรุปได้ว่า นาย ช. และภริยาจดทะเบียนสมรสกัน เมื่อวันที่ 1 พฤษภาคม 2550 ต่างฝ่ายต่างมีเงินได้พึงประเมินตามมาตรา 40 แห่งประมวลรัษฎากร และได้ยื่นรายการและเสียภาษีด้วยแบบแสดงรายการ ภ.ง.ด. 90 เมื่อวันที่ 21 กุมภาพันธ์ 2556 โดยนำเงินได้พึงประเมินของภริยามารวมคำนวณและเสียภาษีในนามของนาย ช. ตามมาตรา 57 ฉ วรรคสาม แห่งประมวลรัษฎากร และขอคืนเงินภาษีจำนวน 8,590.85 บาท ไว้แล้ว แต่ประสงค์จะขออนุมัติเปลี่ยนแปลงวิธีการยื่นรายการและเสียภาษีจากเดิมเป็นการยื่นรายการและเสียภาษีแยกต่างหากจากกัน </ข้อหารือ>
<reference_answer> อนุมัติให้ท่านทั้งสองเปลี่ยนแปลงวิธีการยื่นรายการและเสียภาษีตามที่ร้องขอได้ </reference_answer>
<student_answer> นาย ช. ต้องขออนุมัติจากอธิบดีเพื่อเปลี่ยนแปลงวิธีการยื่นรายการและเสียภาษีจากการยื่นรวมเป็นการยื่นแยกต่างหาก </student_answer>

<assistant> ```json
{"point_thought": "From the given reference answer and the student answer, it is clear that they are both making one point only and it is their whole answer. Therefore, that's what I would put in the student_point and reference_point",
 "student_point": ["นาย ช. ต้องขออนุมัติจากอธิบดีเพื่อเปลี่ยนแปลงวิธีการยื่นรายการและเสียภาษีจากการยื่นรวมเป็นการยื่นแยกต่างหาก"],
 "reference_point": ["อนุมัติให้ท่านทั้งสองเปลี่ยนแปลงวิธีการยื่นรายการและเสียภาษีตามที่ร้องขอได้"],
 "coverage": {"thought": "From the extracted reference point and student point, I will evaluate point by point. The reference's only point is about allowing the two of the complainants the ability to refile the tax forms. However, the student's only point only states that Mr.ช. must request approval from the director-general for the refiling of the tax forms and does not cover the same point the reference answer is making. Therefore, it would get the coverage score of no-coverage.", "score": "no-coverage"}, "contradiction": {"thought": "From the given reference point and student point, even though the student point does not contain the same message as the one made by the reference. It does not directly contradict the reference point either and therefore would get the score no-contradiction for contradiction", "score": "no-contradiction"}}
```

\end{Verbatim}

\textbf{Turn 4: }
\begin{Verbatim}[breaklines, breaksymbol=, breakanywheresymbolpre=]
<user> <ข้อหารือ> บริษัทฯ มีความประสงค์จะขอขยายเวลาการยื่นคำร้องขอคืนภาษีมูลค่าเพิ่ม กรณีบริษัทฯ ได้
นำเข้าสินค้าและชำระภาษีมูลค่าเพิ่ม ดังนี้
1. นำเข้าสินค้าเมื่อวันที่ 21 มีนาคม 2538 โดยชำระภาษีมูลค่าเพิ่มจากการนำเข้าตาม
ใบเสร็จรับเงินค่าภาษีอากรของกรมศุลกากรเมื่อวันที่ 4 เมษายน 2538 จำนวนภาษีมูลค่าเพิ่ม 91,257
บาท และนำเข้าสินค้าเมื่อวันที่ 29 มีนาคม 2538 โดยชำระภาษีมูลค่าเพิ่มจากการนำเข้าตามใบเสร็จ
ของกรมศุลกากรเมื่อวันที่ 3 เมษายน 2538 ภาษีมูลค่าเพิ่ม 134,675 บาท รวมภาษีมูลค่าเพิ่มจาก
การนำเข้า 225,932 บาท
บริษัทฯ ได้ยื่นแบบ ภ.พ.30 เพิ่มเติมขอคืนภาษีซื้อจากการนำเข้าสินค้าดังกล่าวโดยถือเป็น
ภาษีซื้อของเดือนมีนาคม 2538 โดยยื่นแบบ ภ.พ.30 เพิ่มเติม เพื่อขอคืนภาษีซื้อดังกล่าวรวมกับภาษีซื้อ
อื่น เมื่อเดือนสิงหาคม 2538 ซึ่งเป็นการขอคืนภาษีซื้อผิดเดือนภาษีเพราะภาษีซื้อจากการนำเข้าดังกล่าว
เป็นภาษีซื้อของเดือนเมษายน 2538 ซึ่งบริษัทฯ ก็ได้รับการงดเบี้ยปรับจากการยื่นภาษีซื้อผิดเดือนภาษี
และได้รับแจ้งว่าจะไม่ได้รับการคืนภาษีดังกล่าวเนื่องจากบริษัทฯ ยังมิได้ยื่นคำร้องขอคืนภาษีซื้อดังกล่าว
ซึ่งเป็นของเดือนเมษายน 2538
2. นำเข้าสินค้าเมื่อวันที่ 27 เมษายน 2538 โดยชำระภาษีมูลค่าเพิ่มจากการนำเข้าตาม
ใบเสร็จรับเงินค่าภาษีอากรของกรมศุลกากรเมื่อวันที่ 27 พฤษภาคม 2538 จำนวนภาษีมูลค่าเพิ่ม
46,438 บาท
บริษัทฯ ได้ยื่นแบบ ภ.พ.30 เพิ่มเติมเพื่อขอคืนภาษีซื้อจากการนำเข้าสินค้าดังกล่าว โดย
ถือเป็นภาษีซื้อของเดือนเมษายน 2538 โดยยื่นแบบ ภ.พ.30 เพิ่มเติมเพื่อขอคืนภาษีซื้อดังกล่าวรวมกับ
ภาษีซื้ออื่นเมื่อเดือนสิงหาคม 2538 ซึ่งเป็นการขอคืนภาษีซื้อผิดเดือนภาษี เพราะภาษีซื้อจากการนำเข้า
ดังกล่าวเป็นภาษีซื้อของเดือนพฤษภาคม 2538 ซึ่งบริษัทฯ ก็ได้รับการงดเบี้ยปรับจากการยื่นภาษีซื้อผิด
เดือนภาษีแล้วและได้รับแจ้งว่าจะไม่ได้รับคืนภาษีซื้อดังกล่าว เนื่องจากบริษัทฯ ยังมิได้ยื่นคำร้องขอคืน
ภาษีซื้อดังกล่าวซึ่งเป็นของเดือนพฤษภาคม 2538
บริษัทฯ มีความประสงค์จะขอคืนภาษีซื้อจากการนำเข้าสินค้าดังกล่าว แต่เนื่องจากการ
ขอคืนภาษีตามมาตรา 84/1 แห่งประมวลรัษฎากร ต้องขอคืนภายใน 3 ปี นับแต่วันพ้นกำหนดยื่น
แบบแสดงรายการภาษีสำหรับเดือนภาษี และบริษัทฯ มิได้ยื่นคำร้องขอคืนภายในกำหนดเวลาดังกล่าว
เนื่องจากในช่วงเดือนพฤษภาคม - กรกฎาคม 2539 บริษัทฯ มีการเปลี่ยนแปลงผู้ถือหุ้น ผู้บริหาร และ
ระบบการทำงานภายในบริษัทฯ ประกอบกับพนักงานบัญชีที่มาจากทีมผู้ถือหุ้น และผู้บริหารชุดเดิมซึ่งลาออก
ยกทีมกระทันหัน ทำให้บริษัทฯ ประสบปัญหาในการจัดทำบัญชี จึงไม่มีผู้ที่รู้ด้านระบบบัญชีและภาษีอากร
ทำให้เกิดข้อผิดพลาดในการยื่นแบบขอคืนภาษี ซึ่งบริษัทฯ มิได้มีเจตนา แต่เข้าใจผิดทางปฏิบัติในการยื่น
แบบ บริษัทฯ ไม่เข้าใจข้อกฎหมายและมิได้มีเจตนาหลีกเลี่ยงภาษี บริษัทฯ จึงขอขยายเวลาการยื่น
คำร้องขอคืนภาษีมูลค่าเพิ่มในกรณีดังกล่าวด้วย </ข้อหารือ>
<reference_answer> ไม่อาจพิจารณาคำร้องขอของบริษัทฯ ได้ เนื่องจากไม่มีบทบัญญัติของกฎหมายให้อำนาจอธิบดีกรมสรรพากรขยายเวลาการยื่นคำร้องขอคืนภาษีมูลค่าเพิ่ม </reference_answer>
<student_answer> บริษัทฯ จะไม่สามารถขอคืนภาษีมูลค่าเพิ่มได้ เนื่องจากไม่ได้ยื่นคำร้องขอคืนภาษีภายในกำหนดเวลาตามมาตรา 84/1 แห่งประมวลรัษฎากร และเหตุผลที่นำเสนออาจไม่เพียงพอในการขอขยายเวลา </student_answer>


<assistant> ```json
{"point_thought": "The reference answer is making one point only saying that it cannot process the company's request due to lack of authority. The student answer is making one point as well saying that the company cannot file the request due to the deadline and due to lack of reason for deadline extension. Therefore, I should put these into lists.",
    "student_point": ["บริษัทฯ จะไม่สามารถขอคืนภาษีมูลค่าเพิ่มได้ เนื่องจากไม่ได้ยื่นคำร้องขอคืนภาษีภายในกำหนดเวลาตามมาตรา 84/1 แห่งประมวลรัษฎากร และเหตุผลที่นำเสนออาจไม่เพียงพอในการขอขยายเวลา"],
    "reference_point": ["ไม่อาจพิจารณาคำร้องขอของบริษัทฯ ได้ เนื่องจากไม่มีบทบัญญัติของกฎหมายให้อำนาจอธิบดีกรมสรรพากรขยายเวลาการยื่นคำร้องขอคืนภาษีมูลค่าเพิ่ม"],
    "coverage": {"thought": "From the extracted reference point and student point, the reference answer is making one point saying that it cannot process the company's request due to lack of authority. However, the student point states that the company cannot request for VAT refund due to the request is overdue from the deadline and its reason is not sufficient. At first glance, both of these points may lead to the same outcome which is that the company does not receive a VAT refund but the way that the student answer reaches this outcome is because of a different reasons since the reference points that it cannot process the request while the student says that the company cannot file this request. Therefore, the score would be no-coverage", "score": "no-coverage"}, "contradiction": {"thought": "As stated in the coverage section, the student states that the company cannot file the request which is not mentioned at all in the reference point. In fact, the reference points out that it can file the request just that the receiver cannot process the request due to lack of authority. Therefore, the student point directly contradicts the reference point and the score should be contradiction.", "score": "contradiction"}}
```

\end{Verbatim}
% Below are the prompts we utilized for evaluating system response against a reference response on the metric of Coverage and Contradiction on WCX dataset. 

\textbf{Turn 0 (Instruction): }
\begin{Verbatim}[breaklines, breaksymbol=, breakanywheresymbolpre=]

<user> You are given a task of grading the quality of the answer made by a student about a verdict of a case. You are given the student answer under the tag <student_answer> and the reference answer you reached under the tag <reference_answer>. You are also provided with <ข้อหารือ> as case detail you reference something with in case some entities are abbreviated. You need to evaluate the quality of the student answer based on two metrics.
1. Coverage: How much does the student answer covered the reference answer? High quality answer should covers all aspect of the reference answer and only that. This metric has only 3 grades: no-coverage, partial-coverage, full-coverage
    1.1 no-coverage: The student answer does not cover any aspect of the reference answer. It can be either wrong or irrelevant to the question
    1.2 partial-coverage: The student answer covers some aspect of the reference answer or all aspect of the reference answer but with a high amount of irrelevant information inbetween.
    1.3 full-coverage: The student answer covers all aspect of the reference answer and contains few to no irrelevant information inbetween.
2. Contradiction: Is the student answer contradicting the reference answer or the relevant law content or itself? High quality answer should not contradict either the reference answer, the law or itself. This metric differs from the above metric in some cases. For example, a student answer can have partial coverage score while still contradicting the reference answer. Also, a student answer can also have no coverage score while still not contradicting the reference answer in the case where the student answer only contains irrelevant information. This metric only has 2 grades: no-contradiction, contradiction
    2.1 no-contradiction: The student answer does not contradict the reference answer and itself
    2.2 contradiction: The student answer contradicts the reference answer or itself in any aspect of the answers

DO NOT ATTEMPT TO ASSIGN OTHER SCORE THAN THE ONE LISTED!!!

Here's the full instruction of how to do this task:
1. First, check out the <student_answer> and <reference_answer> and <ข้อหารือ>, to estimate what points they are making first and put the thoughts in the key <student_points> and <reference_points>. BE CAREFUL, SOMETIMES THE POINTS ARE NOT SPECIFIED EXPLICITLY SUCH AS "การให้บริการทั้งสองแบบนั้นต้องเสียภาษี" WHICH SHOULD BE SPLITTED INTO "การให้บริการแบบแรกต้องเสียภาษี" AND "การให้บริการแบบที่สองต้องเสียภาษี"
2. From the points extracted, measure the two metrics. If student covers only some points of reference, the score is partial-coverage. If all, it's full-coverage. If none, it's no-coverage. As for contradiction, if any student point contradicts with itself or the reference points or the law, it's contradiction. Otherwise, it's no-contradiction.
    
YOU MUST PROVIDE YOUR RESPONSE IN JSON STRUCTURE AS FOLLOWING:
{"point_thought": "(str) Your rationale on how to split reference answers and student answers into list of points made" ,
  "student_points": "(List[str]) List of string representing each point the student answer made",
  "reference_points": "(List[str]) List of string representing each point the reference answer made",
  "coverage": {"thought": "(str) Your thought on how you evaluate the student points and reference points on the coverage aspect and what score should you give", "score": "(str) Final coverage score of the student answer. Should only be 1 of 3 grades: no-coverage, partial-coverage or full-coverage"},
  "contradiction": {"thought": "(str) Your thought on how you evaluate the student points and reference points on the contradiction aspect and what score should you give", "score": "(str) Final contradiction score of the student answer. Should only be 1 of 2 grades: no-contradiction or contradiction"}
}


Caution:
    1. MAKE SURE TO PROVIDE A REASONABLE THOUGHT PROCESS AS WELL.
    2. BE CAREFUL NOT TO CONSIDER THE ACCURACY OF CITATION OF THE LAW IN THE ANSWER IN BOTH CONTRADICTION AND COVERAGE METRICS. THERE IS A SEPARATE METRIC FOR THAT. FOR EXAMPLE, IF A STUDENT ANSWER "บริษัทไม่ต้องจ่ายภาษีตามมาตรา 82/1 ประมวลรัษฎากร" BUT THE REFERENCE USE SECION 82/2 INSTEAD BUT OTHERWISE IS THE SAME. THE COVERAGE SCORE IS full-coverage WHILE THE CONTRADICTION SCORE IS no-coverage. PLEASE BE VERY CAREFUL OF THIS!!!. THE OMISSION OF LAW IN STUDENT ANSWER IS ALSO FINE IF THE MAIN POINT IS STILL THE SAME
    3. *** THIS IS REALLY IMPORTANT!! *** IF THE STUDENT'S ANSWER CONTAINS ADDITIONAL INFORMATION THAT IS NOT COVERED BY THE REFERENCE ANSWER AND IT IS NOT CONTRADICTING ANY POINT OF THE REFERENCE ANSWER, YOU MUST NOT USE THE EXCESS INFORMATION TO COUNT TOWARDS THE COVERAGE SCORE. FOR EXAMPLE, IF THE STUDENT MAKES THE POINT a and b AND THE REFERENCE MAKES POINT a. THE STUDENT WOULD STILL GET full-coverage score
    4. *** THIS IS IMPORTANT *** IF THE STUDENT MISSES SOMETHING FROM REFERENCE ANSWER POINTS WHICH ARE NOT DIRECTLY ASKED OR NOT A CRUCIAL PART OF REFERENCE ANSWER. THE COVERAGE SCORE WOULD NOT BE AFFECTED OR DEDUCED BY IT. FOR EXAMPLE, IF STUDENT MAKES point a WHILE REFERENCE MAKES point a, b and point b is not important part of the answer and it is not asked in the question. THE STUDENT WOULD STILL GET full-coverage

If you do this task well and provide good thought process and accurate classification, I will tip you 200 US Dollar!
    
Take a deep breath and think carefully. Think in gradually increasing complexity

<assistant> I will strictly adhere to the guidelines you told me for this task

\end{Verbatim}


\textbf{Turn 1: }
\begin{Verbatim}[breaklines, breaksymbol=, breakanywheresymbolpre=]

<user> <question> คู่สัญญาสามารถเรียกร้องอะไรได้บ้างหากรู้ว่าผู้บังคับหลักประกันไม่เป็นกลาง </question>
<reference_answer> คู่สัญญาสามารถเรียกร้องได้โดยยื่นคำร้องต่อศาลเพื่อมีคำสั่งให้ผู้บังคับหลักประกันพ้นจากตำแหน่ง ตามมาตรา 76 ของพระราชบัญญัติหลักประกันทางธุรกิจ พ.ศ. 2558. </reference_answer>
<student_answer> คู่สัญญาสามารถเรียกร้องให้ศาลมีคำสั่งให้ผู้บังคับหลักประกันพ้นจากตำแหน่งได้ หากพบว่าผู้บังคับหลักประกันไม่เป็นกลางหรือมีพฤติกรรมที่ไม่เหมาะสมตามมาตรา 76 ของพระราชบัญญัติหลักประกันทางธุรกิจ พ.ศ. 2558 </student_answer>

<assistant> ```json
{"point_thought": "From the given reference answer, the student answer and question, both the student answer and the reference answers are making one point. I should put their points in a list",
 "student_point": ["คู่สัญญาสามารถเรียกร้องได้โดยยื่นคำร้องต่อศาลเพื่อมีคำสั่งให้ผู้บังคับหลักประกันพ้นจากตำแหน่ง ตามมาตรา 76 ของพระราชบัญญัติหลักประกันทางธุรกิจ พ.ศ. 2558. "],
 "reference_point": ["คู่สัญญาสามารถเรียกร้องให้ศาลมีคำสั่งให้ผู้บังคับหลักประกันพ้นจากตำแหน่งได้ หากพบว่าผู้บังคับหลักประกันไม่เป็นกลางหรือมีพฤติกรรมที่ไม่เหมาะสมตามมาตรา 76 ของพระราชบัญญัติหลักประกันทางธุรกิจ พ.ศ. 2558 "],
"coverage": {"thought": "From the extracted points, the reference answer is only making one point which is that the contracting party can request the court to issue an order to remove the security enforcer from their position if it is found that the security enforcer is not impartial or has inappropriate behavior. The student answer is also making only one point and it is also that the contracting party can request the court to issue an order to remove the security enforcer from their position. Therefore, the student answer fully cover the reference answer and does not contain any irrelevant information. The score would be 100", "score": "full-coverage"}, "contradiction": {"thought": "From the extracted_points, there is no contradicting point that the student make and the answer does not contradict with itself at all. Therefore, the contradiction score would be 0", "score": "no-contradiction"}}
```



\end{Verbatim}

\textbf{Turn 2: }
\begin{Verbatim}[breaklines, breaksymbol=, breakanywheresymbolpre=]

<user> <question> การจดทะเบียนสมาคมธุรกิจหลักทรัพย์มีเงื่อนไขอะไรบ้าง </question>
<reference_answer> 1. สมาคมต้องมีข้อบังคับที่อย่างน้อยประกอบด้วยข้อความ 7 ข้อ ดังนี้
   - ชื่อ
   - วัตถุที่ประสงค์
   - ที่ตั้งสำนักงาน
   - วิธีรับสมาชิก สิทธิและหน้าที่ของสมาชิก
   - วินัยและการลงโทษสมาชิก
   - การดำเนินกิจการของสมาคมที่เกี่ยวเนื่องกับธุรกิจหลักทรัพย์ การตั้ง การออกจากตำแหน่ง และการประชุมของกรรมการ ตลอดจนการประชุมใหญ่
   - ข้อบังคับอื่นตามที่คณะกรรมการ ก.ล.ต. ประกาศกำหนด

2. ข้อบังคับของสมาคมต้องนำไปจดทะเบียนต่อสำนักงานพร้อมกับการยื่นคำขออนุญาตจัดตั้งสมาคมที่เกี่ยวเนื่องกับธุรกิจหลักทรัพย์
3. บริษัทหลักทรัพย์ผู้เริ่มก่อการจัดตั้งที่มีจำนวนไม่น้อยกว่าสิบห้ารายต้องยื่นคำขอต่อสำนักงานตามหลักเกณฑ์และวิธีการที่คณะกรรมการ ก.ล.ต. ประกาศกำหนด
4. สำนักงานมีอำนาจออกคำสั่งเป็นหนังสือเรียกบุคคลใด ๆ มาสอบถามหรือให้ส่งเอกสารมาเพื่อประกอบการพิจารณาเกี่ยวกับการขออนุญาตจัดตั้งสมาคมที่เกี่ยวเนื่องกับธุรกิจหลักทรัพย์ได้
5. การจัดตั้งสมาคมที่เกี่ยวเนื่องกับธุรกิจหลักทรัพย์ต้องได้รับใบอนุญาตและจดทะเบียนกับสำนักงาน. </reference_answer>
<student_answer> การจดทะเบียนสมาคมธุรกิจหลักทรัพย์มีเงื่อนไขดังนี้: 1. ต้องได้รับใบอนุญาตและจดทะเบียนกับสำนักงาน (มาตรา 231) 2. ต้องมีบริษัทหลักทรัพย์ผู้เริ่มก่อการจัดตั้งไม่น้อยกว่ายี่สิบรายยื่นคำขอต่อสำนักงาน (มาตรา 232) 3. ข้อบังคับของสมาคมต้องมีข้อความตามที่กำหนดในมาตรา 234 </student_answer>

<assistant>```json
{"point_thought": "From the given reference answer, the student answer and question, the reference answer is making 5 points as listed while the student is making 3 points. I should put those points in the list",
 "student_point": ["1. สมาคมต้องมีข้อบังคับที่อย่างน้อยประกอบด้วยข้อความ 7 ข้อ ดังนี้ ชื่อ, วัตถุที่ประสงค์ ที่ตั้งสำนักงาน วิธีรับสมาชิก สิทธิและหน้าที่ของสมาชิก วินัยและการลงโทษสมาชิก การดำเนินกิจการของสมาคมที่เกี่ยวเนื่องกับธุรกิจหลักทรัพย์ การตั้ง การออกจากตำแหน่ง และการประชุมของกรรมการ ตลอดจนการประชุมใหญ่ ข้อบังคับอื่นตามที่คณะกรรมการ ก.ล.ต. ประกาศกำหนด", "2. ข้อบังคับของสมาคมต้องนำไปจดทะเบียนต่อสำนักงานพร้อมกับการยื่นคำขออนุญาตจัดตั้งสมาคมที่เกี่ยวเนื่องกับธุรกิจหลักทรัพย์", "3. บริษัทหลักทรัพย์ผู้เริ่มก่อการจัดตั้งที่มีจำนวนไม่น้อยกว่าสิบห้ารายต้องยื่นคำขอต่อสำนักงานตามหลักเกณฑ์และวิธีการที่คณะกรรมการ ก.ล.ต. ประกาศกำหนด", "4. สำนักงานมีอำนาจออกคำสั่งเป็นหนังสือเรียกบุคคลใด ๆ มาสอบถามหรือให้ส่งเอกสารมาเพื่อประกอบการพิจารณาเกี่ยวกับการขออนุญาตจัดตั้งสมาคมที่เกี่ยวเนื่องกับธุรกิจหลักทรัพย์ได้", "5. การจัดตั้งสมาคมที่เกี่ยวเนื่องกับธุรกิจหลักทรัพย์ต้องได้รับใบอนุญาตและจดทะเบียนกับสำนักงาน."],
 "reference_point": ["การจดทะเบียนสมาคมธุรกิจหลักทรัพย์มีเงื่อนไขดังนี้: 1. ต้องได้รับใบอนุญาตและจดทะเบียนกับสำนักงาน (มาตรา 231)", "2. ต้องมีบริษัทหลักทรัพย์ผู้เริ่มก่อการจัดตั้งไม่น้อยกว่ายี่สิบรายยื่นคำขอต่อสำนักงาน (มาตรา 232)", "3. ข้อบังคับของสมาคมต้องมีข้อความตามที่กำหนดในมาตรา 234"]
 ,"coverage": {"thought": "As seen from the extracted points of student and reference answer, the reference answer is making the total of 5 points of requirements needed for registering securities business association. However, the student is only making three points which covers point 5 and 1 correctly. It also mentions point 3 of the reference answer but the amount of filing which should be 15 is stated as 20. It doesn't mention point 2 and 4 at all. Therefore, the coverage score would be partial-coverage", "score": "partial-coverage"}, "contradiction": {"thought": "As mention, the student answer mentions point 3 of the reference answer but the number of filing should be 15. Hence, the student answer directly contradict the reference answer. Therefore, the contradiction score would be contradiction ", "score": "contradiction"}}
```

\end{Verbatim}

\textbf{Turn 3: }
\begin{Verbatim}[breaklines, breaksymbol=, breakanywheresymbolpre=]
<user> <question> มีความผิดอะไรหากคนต่างชาติยังคงประกอบธุรกิจที่ถูกเพิกถอนใบอนุญาต </question>
<reference_answer> คนต่างด้าวจะต้องระวางโทษจำคุกไม่เกินสามปี หรือปรับจากหนึ่งแสนบาทถึงหนึ่งล้านบาท หรือทั้งจำทั้งปรับ และปรับอีกวันละหนึ่งหมื่นบาทตลอดเวลาที่ยังฝ่าฝืนอยู่. </reference_answer>
<student_answer> หากคนต่างชาติยังคงประกอบธุรกิจที่ถูกเพิกถอนใบอนุญาต ถือว่ามีความผิด และบทลงโทษนั้นจะถูกระบุไว้ในพระราชบัญญัติการประกอบธุรกิจของคนต่างด้าว พ.ศ. 2542 </student_answer>

<assistant>```json
{"point_thought": "From the given reference answer, the student answer and question, the reference answer is making 1 point about the penalty while the student makes two point about whether he is guilty and the penalty.",
 "student_point": ["คนต่างด้าวจะต้องระวางโทษจำคุกไม่เกินสามปี หรือปรับจากหนึ่งแสนบาทถึงหนึ่งล้านบาท หรือทั้งจำทั้งปรับ และปรับอีกวันละหนึ่งหมื่นบาทตลอดเวลาที่ยังฝ่าฝืนอยู่."],
 "reference_point": ["หากคนต่างชาติยังคงประกอบธุรกิจที่ถูกเพิกถอนใบอนุญาต ถือว่ามีความผิด", "บทลงโทษนั้นจะถูกระบุไว้ในพระราชบัญญัติการประกอบธุรกิจของคนต่างด้าว พ.ศ. 2542"],
 "coverage": {"thought": "From the extracted points, the reference answer is only making one point which says that this case can be penalized by jail time of up to 3 years or fine of 100,000 to 1,000,000 baht. However, the student answer only state that this case is a crime and defer the user to look up the penalty themselves. Therefore, it would get the coverage score of no-coverage.", "score": "no-coverage"}, "contradiction": {"thought": "From the given reference and student points, even though the student answer does not cover the same point that the reference answer is making. It does not directly contradict the reference answer either and therefore would get the score no-contradiction for contradiction", "score": "no-contradiction"}}
```


\end{Verbatim}

\textbf{Turn 4: }
\begin{Verbatim}[breaklines, breaksymbol=, breakanywheresymbolpre=]
<user> <question> ถ้ากรรมการลงรายการในเอกสารที่ยื่นนายทะเบียนไม่ตรงกับบัญชีของบริษัท กรรมการจะผิดหรือไม่ </question>
<reference_answer> กรรมการจะต้องรับผิดชอบทางอาญาและแพ่ง ตามพระราชบัญญัติหลักทรัพย์และตลาดหลักทรัพย์ พ.ศ. 2535 มาตรา 312 และอาจต้องระวางโทษจำคุกตั้งแต่ห้าปีถึงสิบปี และปรับตั้งแต่ห้าแสนบาทถึงหนึ่งล้านบาท </reference_answer>
<student_answer> กรรมการไม่มีความผิด </student_answer>

<assistant>```json
{"point_thought": "From the given reference answer, the student answer and question, the reference answer is making 2 points about whether the committee is guilty or not and the penalty which was not asked. The student is only making one point",
 "student_point": ["กรรมการจะต้องรับผิดชอบทางอาญาและแพ่ง ตามพระราชบัญญัติหลักทรัพย์และตลาดหลักทรัพย์ พ.ศ. 2535 มาตรา 312", "อาจต้องระวางโทษจำคุกตั้งแต่ห้าปีถึงสิบปี และปรับตั้งแต่ห้าแสนบาทถึงหนึ่งล้านบาท"],
 "reference_point": ["กรรมการไม่มีความผิด"],
 "coverage": {"thought": "From the reference answer and student points, the reference answer is only making two points about whether the the behavior is a crime or not and the penalty that the law state for this case while the student answer only says that the behavior is not a crime and therefore does not cover a single point of the reference answer. Therefore, the score would be no-coverage", "score": "no-coverage"}, "contradiction": {"thought": "The student points directly contradicts the reference point on the point of whether the behavior is a crime or not. Therefore, the score would be contradiction", "score": "contradiction"}}
```


\end{Verbatim}
