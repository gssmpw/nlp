\section{Problem Statement}

% \subsection{The Challenge of Obtaining Thai Legal Documents}

% Despite the increasing digitization of legal resources, acquiring comprehensive and trustworthy versions of Thai legal documents remains a significant challenge. While Thai legal documents are readily available in physical book format, accessing their digital counterparts presents several obstacles. Multiple sources exist online, but reliability and completeness vary greatly. The \textit{Office of the Council of State of Thailand}\footnote{\url{https://www.ocs.go.th/}}, abbreviated as \textit{OCS}, as the official body responsible for legal drafting and publication, launched a new website in early 2024 to address this need. 


% While this initiative holds promise for the future, as of August 2024, the website is still under development and faces significant challenges. Its instability leads to frequent changes in document paths and locations, making it difficult for researchers and practitioners to reliably access specific legal texts. Additionally, the website suffers from incomplete data, as exemplified by the \textit{Civil and Commercial Code}, where only two out of six books are currently accessible. Although the OCS website is expected to eventually provide a stable and complete digital repository, its current limitations pose a considerable barrier to research and development in areas like legal information retrieval and AI-powered legal applications.
% เหมือนเขียนด่า OCS 5555555
% go to problem settings

\subsection{Lack of tailored components in Legal RAG system}

This section highlights the unique challenges posed by Thai legal documents in the development of RAG systems \textcolor{orange}{for Thai legal domain}. These challenges stem from the inherent characteristics of the documents themselves, particularly their structure and referencing patterns, demanding specialized solutions beyond traditional approaches.

\subsubsection{Limitations of text chunking method on Thai Legal Document}

A key factor determining RAG's effectiveness lies in the chunking strategy employed, i.e., \emph{how a document is broken down into manageable pieces for processing}. Well-defined chunks are essential for ensuring that relevant information is grouped together, facilitating accurate retrieval based on user queries. Furthermore, chunking must maintain the contextual integrity of the legal text, avoiding the fragmentation of crucial information that could hinder comprehension. Finally, efficient chunking must optimize for token limits inherent in LLMs, ensuring chunks remain within these constraints without sacrificing information completeness. \cite{5chunking} have provides overview of chunking strategies into five levels: % Greg Kamradt, 2024 

\begin{enumerate}[label=Level \arabic*]
    \item \textbf{Fixed-Size Chunking}: This method divides text into fixed-length segments based on character count. While simple to implement, it completely disregards textual structure and semantic boundaries.
    \item \textbf{Recursive Chunking}: Building upon fixed-size chunking, this approach utilizes a predefined set of separators (e.g., line breaks, spaces) to recursively split text until desired chunk sizes are achieved. However, reliance on generic separators often results in chunks that misalign with the logical segmentation of legal texts.
    \item \textbf{Document-Based Chunking}: This strategy leverages the inherent structure of documents, utilizing elements like Markdown headings or code blocks as delimiters. While suitable for specific document types, it proves inadequate for capturing the complex hierarchies and semantic relationships within Thai legal documents.
    \item \textbf{Semantic Chunking}: This method attempts to identify semantically related sentences and group them into chunks. It usually involves calculating the semantic similarity between sentences using techniques like embeddings. While a step towards context awareness, it may struggle with long-range dependencies and subtle semantic shifts within legal texts.
    \item \textbf{Agentic Chunking}: Considered the most advanced approach, agentic chunking employs LLMs themselves to determine optimal chunk boundaries. It often involves extracting propositions (standalone statements) and using an LLM-based agent to decide how to group them. However, this method is computationally expensive and requires careful prompt engineering to guide the LLM effectively.
\end{enumerate}

Despite their prevalence, traditional chunking methods suffer from inherent limitations that hinder their effectiveness when applied to legal documents. Primarily, these methods are largely structure agnostic, failing to adequately account for the hierarchical and semantic structures inherent in legal texts. Chunking based solely on character count, generic separators, or even basic semantic similarity can lead to the fragmentation of crucial legal concepts, arguments, and contextual information. This often results in information scattering, where essential legal content is spread across multiple chunks, making retrieval and comprehension more challenging. Furthermore, these methods often struggle with token inefficiency, generating chunks that are either too long – exceeding LLM token limits and leading to information loss – or too short – lacking sufficient context for accurate retrieval and understanding.

% solution -> not sure that should it be here or in 1-experimental-setup
To overcome the limitations of traditional chunking methods, we propose a novel approach specifically designed for Thai legal documents: \textbf{Tier-Based Hierarchical Chunking}. This method leverages the meticulous multi-tiered structure inherent in Thai Acts and Codes, ensuring both structural awareness and token length efficiency. Instead of relying on arbitrary character counts or generic separators, our method directly utilizes the inherent legal hierarchy as the foundation for chunking. We identify and segment documents based on the multi-tiered structure from section \ref{multi-tiered}

In our approach, \textbf{section} level will be served as the smallest unit of chunking. This ensures that each chunk represents a self-contained legal concept or rule. To preserve the hierarchical context, these chunks are stored in a nested JSON format, allowing for easy traversal across different tiers.

This hierarchical representation offers several advantages:
% \begin{itemize}
%     \item \textbf{Structure Awareness}: Each chunk inherently reflects the legal hierarchy, facilitating contextually accurate retrieval and reasoning.
%     \item \textbf{Token Length Efficiency}: By aligning chunks with meaningful legal units, we avoid arbitrarily long or short segments, optimizing token usage within LLM limits.
%     \item \textbf{Flexibility and Traversal}: The nested JSON structure allows for flexible navigation and retrieval at various levels of granularity.
% \end{itemize}


\textbf{Structure Awareness}: Each chunk inherently reflects the legal hierarchy, facilitating contextually accurate retrieval and reasoning.

\textbf{Token Length Efficiency}: By aligning chunks with meaningful legal units, we avoid arbitrarily long or short segments, optimizing token usage within LLM limits.

\textbf{Flexibility and Traversal}: The nested JSON structure allows for flexible navigation and retrieval at various levels of granularity.

% Our chunking process is entirely rule-based, relying on the consistent structure of Thai legal documents. It can be applied to raw text or HTML formats obtained from \textit{Office of the Council of State of Thailand}. This ensures a robust and scalable solution for applying RAG effectively within the Thai legal domain.

Our chunking process is deterministic, leveraging the consistent structure of Thai legal documents. It operates on raw text or HTML formats obtained from the Office of the Council of State of Thailand, providing a robust and scalable solution for applying RAG within the Thai legal domain.

\subsubsection{Referencer}
A significant challenge in applying RAG to Thai legal documents stems from the intricate network of section references that permeate these texts. Legal sections often build upon or modify one another, requiring the reader to navigate a web of related sections to grasp the complete legal context. This challenge is further compounded by two distinct types of references:
\begin{enumerate}
    \item \textbf{Intra-Document Referencing}: This challenge arises when sections within a legal document reference other sections located within the same legal documents.
    \item \textbf{Inter-Document Referencing}: This challenge arises when sections within a legal document reference provisions located in entirely different legal documents.
\end{enumerate}

For instance, consider this excerpt from Thai Criminal Code:

\begin{quote}
    \textit{Section 260: Whoever uses, sells, offers for sale, exchanges or offers to exchange a ticket arising from an act as provided for in Section 258 or Section 259 shall be punished with imprisonment...}
\end{quote}

In this example, understanding Section 260 hinges on also comprehending the content of Sections 258 and 259, which are not provided within the immediate text. This inter-section referencing poses a significant obstacle for traditional RAG systems, which typically rely on isolated chunks of text.

% not sure where should we place it
To tackle these referencing challenges, we propose a system that performs:
\begin{enumerate}
    \item \textbf{Reference Identification}: Accurately identifying all intra and inter-document references within each legal section.
    \item \textbf{Reference Retrieval}: Fetching the full text of all referenced sections, ensuring version accuracy and database consistency.
    \item \textbf{Information Consolidation}: Integrating the retrieved referenced sections into the context of the original section, creating a single, comprehensive text unit for information retrieval.
    \item \textbf{Nested Referencing Handling }: Addressing scenarios where referenced sections themselves contain further references, ensuring all relevant information is captured in a chain-like manner.
\end{enumerate}

By proactively resolving and incorporating referenced information, we aim to provide legal professionals and researchers with complete and contextually rich legal texts, enhancing the accuracy and efficiency of legal information retrieval systems.


\subsection{Complexities in system evaluation}

Apart from difficulty in obtaining Thai legal documents and the lack of specialized components in RAG pipeline, one of the significant hurdles in incorporating LLMs into legal domain is the complexity in evaluating and analysing the performance of such systems. There are main 3 reasons to this: (1) Limited availability of Thai legal corpora, (2) Challenges in retrieval benchmarking, and (3) Multi-faceted nature of LLM evaluation


\subsubsection{Limited Availability of Thai Legal Corpora}
As to our knowledge, there has not been a corpora constructed with the purpose of instruction tuning LLMs toward Thai legal knowledge or retrieving relevant Thai law from the queries. Previous work such as Wiratchawa et al.\cite{thlegalbert} in which a QA dataset is constructed by scraping online legal forums lacks the relevant context which is crucial when evaluating LLM system in knowledge-intensive task. We suspect that the scarcity of Thai legal domain corpora can be attributed to the highly specialized nature of the legal field. Not only is law a complex domain, but it is divided into various subdomains necessitates the involvement of even more specialized experts for dataset construction. Consequently, creating a comprehensive Thai legal corpus of high quality which covers every possible topics in the field demands a collaborative effort among multiple stakeholders. This constitutes a major challenges in developing Thai legal LLM system since without a proper evaluation dataset, the performance of the system remains ungrounded, unreliable and incomparable with other systems. Therefore, there is a high need for a proper initiative project aimed to establish a standard corpora in this domain.

\subsubsection{Challenges in retrieval benchmarking}

Another substantial challenge in developing effective RAG systems for Thai legal documents lies in evaluating retriever performance within a multi-label retrieval context. Traditional information retrieval metrics like Precision, Recall, and F1-score, primarily designed for single-label classification, fall short in capturing the complexity of retrieving multiple relevant sections across different tiers of Thai law and, to the best of our knowledge, there has not been prior work on developing multi-label counterparts of these metrics. This is further compounded by the scarcity of publicly available, annotated Thai legal datasets specifically designed for benchmarking multi-label retrieval systems. This lack of standardized benchmarks makes it difficult to objectively compare different retrieval models and chunking strategies, hindering progress in the field. 

This highlights the critical need for developing novel evaluation metrics tailored to multi-label legal retrieval task
% do we need to spoil our benchmark here?

\subsubsection{Multi-faceted nature of LLM evaluation}

Evaluation of LLM system typically involves letting the system perform a certain task such as Q\&A and computing the metrics based on the generated result and the ground truth. According to \cite{llmeval}, there are various metrics that can be utilized in the evaluation process including statistical metrics such as BLEU\cite{papineni-etal-2002-bleu} and Levenshtein distance, model-based metrics such as G-Eval\cite{liu2023gevalnlgevaluationusing} and Natural Language Inference model, and hybrid metrics such as BERTscore\cite{zhang2020bertscoreevaluatingtextgeneration} and GPTscore\cite{fu2023gptscoreevaluatedesire}. These metrics are mainly used to quantify the similarity or agreement between the generated result and the ground truth. 

However, these metrics do not represent all aspects of the system's performance especially in legal domain in which the faithfulness of the system to the facts or context provided and the transparency of the system's reasoning are desired. In addition, further analyses should also be carried out to identify potential problems in the system which might not be apparent at first sight such as the sycophancy of the system, the positional bias, the multi-step reasoning and ungrounded answers problem. Therefore, a comprehensive evaluation of LLM performance in the legal domain requires one to take in consideration the system’s specific requirements in the domain. This demands the development of tailored metrics and analyses to accurately assess the system’s capabilities in these areas.


% not sure this subsection is necessary, but in lit review there are some mentions about legal LLM
% To future works
% \subsection{The Need for a Comprehensive Thai Legal Language Model}
% The successful integration of LLMs into Thai legal practice faces significant hurdles due to the unique characteristics of Thai legal language and the limitations of current LLMs predominantly trained on English text. The linguistic specificity of Thai legal language, with its distinct syntactic structures, specialized terminology, and subtle nuances, often eludes accurate capture and interpretation by these general-purpose models. This challenge is further exacerbated by the limited availability of large-scale, high-quality Thai legal corpora necessary for training robust LLMs capable of understanding complex Thai legal concepts and reasoning patterns. This scarcity impedes the development of specialized models that can effectively navigate the intricacies of Thai law. Furthermore, the inherent "black-box" nature of many LLMs raises concerns about transparency and trust, creating a significant barrier to their adoption in legal practice, where explainability and accountability in decision-making are paramount.

% To overcome these challenges, research efforts must prioritize the development of a dedicated Thai Legal Language Model trained on a comprehensive corpus of Thai legal documents. This endeavor requires exploring effective data augmentation and collection strategies to address data scarcity while simultaneously investigating techniques for enhancing the explainability and transparency of LLM outputs to foster trust and facilitate their responsible integration into the Thai legal domain.

% เคสถูกข้อยาก ผิดข้อง่าย

% เคส