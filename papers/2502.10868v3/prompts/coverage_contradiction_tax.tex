Below are the prompts we utilized for evaluating system response against a reference response on the metric of Coverage and Contradiction on Tax Case dataset. Instruction prompt is the same as the one used in WCX dataset


\textbf{Turn 1: }
\begin{Verbatim}[breaklines, breaksymbol=, breakanywheresymbolpre=]

<user> <ข้อหารือ> บริษัทฯ เป็นผู้ประกอบการจดทะเบียน ประกอบกิจการซื้อมาขายไปกาแฟกระป๋อง หัวน้ำเชื้อ และหัวน้ำเชื้อชาเขียว ช. ได้ยื่นคำร้องของดเบี้ยปรับภาษีมูลค่าเพิ่ม สำหรับเดือนภาษีเมษายน 2551 เดือนภาษีพฤษภาคม 2551 เดือนภาษีกรกฎาคม 2551 และเดือนภาษีตุลาคม 2551 โดยมีข้อเท็จจริงว่า เจ้าพนักงานประเมินได้ตรวจ0สอบการขอคืนภาษี กรณีที่บริษัทฯ ยื่นแบบแสดง รายการภาษีมูลค่าเพิ่ม (ภ.พ.30) ฉบับยื่นปกติรวมภาษีที่ขอคืนเป็นเงินจำนวน 1,510,423.08 บาท\n บริษัทฯ ได้จดทะเบียนเลิกบริษัทต่อกระทรวงพาณิชย์เมื่อวันที่ 30 มกราคม 2551 พร้อมทั้งยื่นแบบคำขอแจ้งการเปลี่ยนแปลง ทะเบียนภาษีมูลค่าเพิ่ม (ภ.พ.09) แจ้งเลิกต่อกรมสรรพากรเมื่อวันที่ 14 กุมภาพันธ์ 2551 ภาษีซื้อที่บริษัทฯ นำมาใช้ในการยื่นแบบ ภ.พ.30 สำหรับเดือนภาษีดังกล่าวเป็นภาษีซื้อที่เกิดจากการส่งเสริมการขายและค่าที่ปรึกษาทางกฎหมาย ซึ่งบริษัทฯ ได้รับบริการ ก่อนวันแจ้งเลิกการประกอบกิจการ แต่ได้รับใบกำกับภาษีภายหลังวันที่ยื่นแบบ ภ.พ.09 โดยใบกำกับภาษีลงวันที่ 4 เมษายน 2551 วันที่ 8 พฤษภาคม 2551 วันที่ 3 กรกฎาคม 2551 และวันที่ 30 ตุลาคม 2551 กรณีดังกล่าวหากบริษัทฯ ยังไม่ถูกขีดชื่อออกจากทะเบียนภาษีมูลค่าเพิ่ม และได้นำภาษีซื้อตามใบกำกับภาษีดังกล่าวมาใช้ในการคำนวณภาษีมูลค่าเพิ่ม โดยมิได้แสดงรายละเอียด การใช้สิทธิขอนำภาษีซื้อตามใบกำกับภาษีดังกล่าวมาหักในการคำนวณภาษีในเอกสารตามที่อธิบดีกำหนดพร้อมกับการยื่นแบบ ภ.พ.09 ตามข้อ 6 ของคำสั่งกรมสรรพากรที่ ป.66/2539 เรื่อง การแจ้งเลิกประกอบกิจการตามมาตรา 85/15 แห่งประมวลรัษฎากร ลงวันที่ 29 ธันวาคม พ.ศ. 2539\n 1. บริษัทฯ มีสิทธินำภาษีซื้อตามใบกำกับภาษีดังกล่าวมาใช้ในการคำนวณภาษีมูลค่าเพิ่มในการยื่นแบบ ภ.พ.30 ภายหลัง การยื่นแบบ ภ.พ.09 ได้หรือไม่\n 2. หากบริษัทฯ ไม่มีสิทธินำภาษีซื้อตามใบกำกับภาษีดังกล่าวมาใช้ตามข้อ 1. บริษัทฯ จะต้องเสียเบี้ยปรับกรณีนำภาษีซื้อ มาใช้เกินไปตามมาตรา 89(4) แห่งประมวลรัษฎากรหรือไม่ และบริษัทฯ จะต้องยื่นแบบ ค.10 เพื่อขอคืนภาษีใช่หรือไม่ </ข้อหารือ>
<reference_answer> 1. บริษัทฯ ไม่มีสิทธินำภาษีซื้อตามใบกำกับภาษีดังกล่าวมาใช้ในการคำนวณภาษีมูลค่าเพิ่มในการยื่นแบบ ภ.พ.30 ภายหลังการยื่นแบบ ภ.พ.09; 2. บริษัทฯ จะต้องเสียเบี้ยปรับกรณีนำภาษีซื้อมาใช้เกินไปตามมาตรา 89(4) และต้องยื่นแบบ ค.10 เพื่อขอคืนภาษี </reference_answer>
<student_answer> 1. บริษัทฯ ไม่มีสิทธินำภาษีซื้อตามใบกำกับภาษีดังกล่าวมาใช้ในการคำนวณภาษีมูลค่าเพิ่มในการยื่นแบบ ภ.พ.30 ภายหลังการยื่นแบบ ภ.พ.09 ได้ 2. บริษัทฯ จะต้องเสียเบี้ยปรับกรณีนำภาษีซื้อมาใช้เกินไปตามมาตรา 89(4) แห่งประมวลรัษฎากร และบริษัทฯ จะต้องยื่นแบบ ค.10 เพื่อขอคืนภาษี </student_answer>

<assistant> ```json
{"point_thought": "From the given reference answer, the student answer and ข้อหารือ, the student answer is making two points as explicitly stated by the numbered list. The reference answer is also making two points stated by the numbered list as well. These two points corresponds with the two questions made by the ข้อหารือ. Therefore, I should put these two points in lists",
 "student_point": ["1. บริษัทฯ ไม่มีสิทธินำภาษีซื้อตามใบกำกับภาษีดังกล่าวมาใช้ในการคำนวณภาษีมูลค่าเพิ่มในการยื่นแบบ ภ.พ.30 ภายหลังการยื่นแบบ ภ.พ.09 ได้", "2. บริษัทฯ จะต้องเสียเบี้ยปรับกรณีนำภาษีซื้อมาใช้เกินไปตามมาตรา 89(4) แห่งประมวลรัษฎากร และบริษัทฯ จะต้องยื่นแบบ ค.10 เพื่อขอคืนภาษี"],
 "reference_point": ["1. บริษัทฯ ไม่มีสิทธินำภาษีซื้อตามใบกำกับภาษีดังกล่าวมาใช้ในการคำนวณภาษีมูลค่าเพิ่มในการยื่นแบบ ภ.พ.30 ภายหลังการยื่นแบบ ภ.พ.09;", "2. บริษัทฯ จะต้องเสียเบี้ยปรับกรณีนำภาษีซื้อมาใช้เกินไปตามมาตรา 89(4) และต้องยื่นแบบ ค.10 เพื่อขอคืนภาษี"],
 "coverage": {"thought": "From the extracted student and reference points, there are two points in the reference answer and in the student answer and the student's answer covers both point of the reference about how the company has no right to use buying tax from the said tax invoice to calculate its VAT when submitting ภ.พ.30 after filing ภ.พ.09. Furthermore, the student answer also covers the point about how the company is subjected to a penalty for using excess input tax under Section 89(4) and must file a return tax form ค.10. Therefore, since the student answer covers all aspect of the reference answer, the student answer has full coverage.", "score": "full-coverage"}, "contradiction": {"thought": "From the points extracted from reference and student answer, point 1 of the student does not directly contradict with the any points from the reference answer same with point 2 of the student. Therefore, the contradiction score would be 0", "score": "no-contradiction"}}
```


\end{Verbatim}

\textbf{Turn 2: }
\begin{Verbatim}[breaklines, breaksymbol=, breakanywheresymbolpre=]

<user> <ข้อหารือ> บริษัท ธ. ประกอบกิจการให้บริการทางกฎหมายแก่บริษัทเงินทุน ธ. และบริษัทในเครือของ
บริษัทเงินทุน ธ. ซึ่งในการให้บริการทางกฎหมายจะแบ่งการให้บริการออกเป็น 2 ประเภท ดังต่อไปนี้
1. การให้บริการงานกฎหมายสำหรับร่างและตรวจสอบสัญญาและเอกสารทางกฎหมาย
รวมทั้งการให้คำปรึกษาทางกฎหมาย ตลอดจนประเมินราคาด้วย
2. การให้บริการงานกฎหมายสำหรับงานว่าความมีขอบเขตให้บริการเริ่มตั้งแต่ติดตาม
ทวงถาม เร่งรัดหนี้ ฟ้องคดี และบังคับคดี
บริษัทฯ จึงหารือว่า การให้บริการทั้ง 2 กรณีดังกล่าว จะได้รับยกเว้นภาษี มูลค่าเพิ่มหรือไม่ </ข้อหารือ>
<reference_answer> การให้บริการงานว่าความได้รับการยกเว้นไม่ต้องเสียภาษีมูลค่าเพิ่ม ส่วนค่าบริการอื่น ๆ ไม่ได้รับการยกเว้นภาษีมูลค่าเพิ่มแต่อย่างใด </reference_answer>
<student_answer> การให้บริการทั้ง 2 กรณีดังกล่าวจะไม่ได้รับยกเว้นภาษีมูลค่าเพิ่ม และจะต้องเสียภาษีมูลค่าเพิ่มตามอัตราปกติ </student_answer>


<assistant> ```json
{"point_thought": "From the given reference answer, the student answer and ข้อหารือ, the reference answer is making two points as clearly stated which are that the legal service is exempted from VAT and that other services are not. As for student answer, as though it may seems it is making one point only, it is actually two point which is that legal service is not exempted from VAT and that other services are not. Therefore, I should put all the points in a list accordingly.",
"student_point": ["การให้บริการงานว่าความจะไม่ได้รับยกเว้นภาษีมูลค่าเพิ่ม และจะต้องเสียภาษีมูลค่าเพิ่มตามอัตราปกติ", "การให้บริการอื่นๆจะไม่ได้รับยกเว้นภาษีมูลค่าเพิ่ม และจะต้องเสียภาษีมูลค่าเพิ่มตามอัตราปกติ"],
"reference_point": ["การให้บริการงานว่าความได้รับการยกเว้นไม่ต้องเสียภาษีมูลค่าเพิ่ม", "ค่าบริการอื่น ๆ ไม่ได้รับการยกเว้นภาษีมูลค่าเพิ่มแต่อย่างใด"],
"coverage": {"thought": "From the extracted points of the student, the student is making two points: 1. Legal service is not exempted from VAT 2. Other services are not exempted from VAT. I should judge it point by point as stated by the caution. The second point aligns well with the reference answer first point which says that the other service is not exempted from VAT while the first point contradicts with the reference answer's first point which says that legal service is exempted from VAT. Therefore, the student answer covers 1 out of 2 points made by reference answer and should get a score of partial-coverage", "score": "partial-coverage"}, "contradiction": {"thought": "Since the student answer has two points, I should evaluate point by point to see if any point contradict with the law or the reference answer. The first point of the student directly contradicts with the first point of the reference answer which says that legal service is exempted from VAT. Therefore, the student answer should get contradiction score of contradiction.", "score": "contradiction"}}
```
\end{Verbatim}

\textbf{Turn 3: }
\begin{Verbatim}[breaklines, breaksymbol=, breakanywheresymbolpre=]
<user> <ข้อหารือ> นาย ช. ได้ขออนุมัติเปลี่ยนแปลงวิธีการยื่นรายการและเสียภาษีสำหรับปีภาษี 2555 มีข้อเท็จจริงสรุปได้ว่า นาย ช. และภริยาจดทะเบียนสมรสกัน เมื่อวันที่ 1 พฤษภาคม 2550 ต่างฝ่ายต่างมีเงินได้พึงประเมินตามมาตรา 40 แห่งประมวลรัษฎากร และได้ยื่นรายการและเสียภาษีด้วยแบบแสดงรายการ ภ.ง.ด. 90 เมื่อวันที่ 21 กุมภาพันธ์ 2556 โดยนำเงินได้พึงประเมินของภริยามารวมคำนวณและเสียภาษีในนามของนาย ช. ตามมาตรา 57 ฉ วรรคสาม แห่งประมวลรัษฎากร และขอคืนเงินภาษีจำนวน 8,590.85 บาท ไว้แล้ว แต่ประสงค์จะขออนุมัติเปลี่ยนแปลงวิธีการยื่นรายการและเสียภาษีจากเดิมเป็นการยื่นรายการและเสียภาษีแยกต่างหากจากกัน </ข้อหารือ>
<reference_answer> อนุมัติให้ท่านทั้งสองเปลี่ยนแปลงวิธีการยื่นรายการและเสียภาษีตามที่ร้องขอได้ </reference_answer>
<student_answer> นาย ช. ต้องขออนุมัติจากอธิบดีเพื่อเปลี่ยนแปลงวิธีการยื่นรายการและเสียภาษีจากการยื่นรวมเป็นการยื่นแยกต่างหาก </student_answer>

<assistant> ```json
{"point_thought": "From the given reference answer and the student answer, it is clear that they are both making one point only and it is their whole answer. Therefore, that's what I would put in the student_point and reference_point",
 "student_point": ["นาย ช. ต้องขออนุมัติจากอธิบดีเพื่อเปลี่ยนแปลงวิธีการยื่นรายการและเสียภาษีจากการยื่นรวมเป็นการยื่นแยกต่างหาก"],
 "reference_point": ["อนุมัติให้ท่านทั้งสองเปลี่ยนแปลงวิธีการยื่นรายการและเสียภาษีตามที่ร้องขอได้"],
 "coverage": {"thought": "From the extracted reference point and student point, I will evaluate point by point. The reference's only point is about allowing the two of the complainants the ability to refile the tax forms. However, the student's only point only states that Mr.ช. must request approval from the director-general for the refiling of the tax forms and does not cover the same point the reference answer is making. Therefore, it would get the coverage score of no-coverage.", "score": "no-coverage"}, "contradiction": {"thought": "From the given reference point and student point, even though the student point does not contain the same message as the one made by the reference. It does not directly contradict the reference point either and therefore would get the score no-contradiction for contradiction", "score": "no-contradiction"}}
```

\end{Verbatim}

\textbf{Turn 4: }
\begin{Verbatim}[breaklines, breaksymbol=, breakanywheresymbolpre=]
<user> <ข้อหารือ> บริษัทฯ มีความประสงค์จะขอขยายเวลาการยื่นคำร้องขอคืนภาษีมูลค่าเพิ่ม กรณีบริษัทฯ ได้
นำเข้าสินค้าและชำระภาษีมูลค่าเพิ่ม ดังนี้
1. นำเข้าสินค้าเมื่อวันที่ 21 มีนาคม 2538 โดยชำระภาษีมูลค่าเพิ่มจากการนำเข้าตาม
ใบเสร็จรับเงินค่าภาษีอากรของกรมศุลกากรเมื่อวันที่ 4 เมษายน 2538 จำนวนภาษีมูลค่าเพิ่ม 91,257
บาท และนำเข้าสินค้าเมื่อวันที่ 29 มีนาคม 2538 โดยชำระภาษีมูลค่าเพิ่มจากการนำเข้าตามใบเสร็จ
ของกรมศุลกากรเมื่อวันที่ 3 เมษายน 2538 ภาษีมูลค่าเพิ่ม 134,675 บาท รวมภาษีมูลค่าเพิ่มจาก
การนำเข้า 225,932 บาท
บริษัทฯ ได้ยื่นแบบ ภ.พ.30 เพิ่มเติมขอคืนภาษีซื้อจากการนำเข้าสินค้าดังกล่าวโดยถือเป็น
ภาษีซื้อของเดือนมีนาคม 2538 โดยยื่นแบบ ภ.พ.30 เพิ่มเติม เพื่อขอคืนภาษีซื้อดังกล่าวรวมกับภาษีซื้อ
อื่น เมื่อเดือนสิงหาคม 2538 ซึ่งเป็นการขอคืนภาษีซื้อผิดเดือนภาษีเพราะภาษีซื้อจากการนำเข้าดังกล่าว
เป็นภาษีซื้อของเดือนเมษายน 2538 ซึ่งบริษัทฯ ก็ได้รับการงดเบี้ยปรับจากการยื่นภาษีซื้อผิดเดือนภาษี
และได้รับแจ้งว่าจะไม่ได้รับการคืนภาษีดังกล่าวเนื่องจากบริษัทฯ ยังมิได้ยื่นคำร้องขอคืนภาษีซื้อดังกล่าว
ซึ่งเป็นของเดือนเมษายน 2538
2. นำเข้าสินค้าเมื่อวันที่ 27 เมษายน 2538 โดยชำระภาษีมูลค่าเพิ่มจากการนำเข้าตาม
ใบเสร็จรับเงินค่าภาษีอากรของกรมศุลกากรเมื่อวันที่ 27 พฤษภาคม 2538 จำนวนภาษีมูลค่าเพิ่ม
46,438 บาท
บริษัทฯ ได้ยื่นแบบ ภ.พ.30 เพิ่มเติมเพื่อขอคืนภาษีซื้อจากการนำเข้าสินค้าดังกล่าว โดย
ถือเป็นภาษีซื้อของเดือนเมษายน 2538 โดยยื่นแบบ ภ.พ.30 เพิ่มเติมเพื่อขอคืนภาษีซื้อดังกล่าวรวมกับ
ภาษีซื้ออื่นเมื่อเดือนสิงหาคม 2538 ซึ่งเป็นการขอคืนภาษีซื้อผิดเดือนภาษี เพราะภาษีซื้อจากการนำเข้า
ดังกล่าวเป็นภาษีซื้อของเดือนพฤษภาคม 2538 ซึ่งบริษัทฯ ก็ได้รับการงดเบี้ยปรับจากการยื่นภาษีซื้อผิด
เดือนภาษีแล้วและได้รับแจ้งว่าจะไม่ได้รับคืนภาษีซื้อดังกล่าว เนื่องจากบริษัทฯ ยังมิได้ยื่นคำร้องขอคืน
ภาษีซื้อดังกล่าวซึ่งเป็นของเดือนพฤษภาคม 2538
บริษัทฯ มีความประสงค์จะขอคืนภาษีซื้อจากการนำเข้าสินค้าดังกล่าว แต่เนื่องจากการ
ขอคืนภาษีตามมาตรา 84/1 แห่งประมวลรัษฎากร ต้องขอคืนภายใน 3 ปี นับแต่วันพ้นกำหนดยื่น
แบบแสดงรายการภาษีสำหรับเดือนภาษี และบริษัทฯ มิได้ยื่นคำร้องขอคืนภายในกำหนดเวลาดังกล่าว
เนื่องจากในช่วงเดือนพฤษภาคม - กรกฎาคม 2539 บริษัทฯ มีการเปลี่ยนแปลงผู้ถือหุ้น ผู้บริหาร และ
ระบบการทำงานภายในบริษัทฯ ประกอบกับพนักงานบัญชีที่มาจากทีมผู้ถือหุ้น และผู้บริหารชุดเดิมซึ่งลาออก
ยกทีมกระทันหัน ทำให้บริษัทฯ ประสบปัญหาในการจัดทำบัญชี จึงไม่มีผู้ที่รู้ด้านระบบบัญชีและภาษีอากร
ทำให้เกิดข้อผิดพลาดในการยื่นแบบขอคืนภาษี ซึ่งบริษัทฯ มิได้มีเจตนา แต่เข้าใจผิดทางปฏิบัติในการยื่น
แบบ บริษัทฯ ไม่เข้าใจข้อกฎหมายและมิได้มีเจตนาหลีกเลี่ยงภาษี บริษัทฯ จึงขอขยายเวลาการยื่น
คำร้องขอคืนภาษีมูลค่าเพิ่มในกรณีดังกล่าวด้วย </ข้อหารือ>
<reference_answer> ไม่อาจพิจารณาคำร้องขอของบริษัทฯ ได้ เนื่องจากไม่มีบทบัญญัติของกฎหมายให้อำนาจอธิบดีกรมสรรพากรขยายเวลาการยื่นคำร้องขอคืนภาษีมูลค่าเพิ่ม </reference_answer>
<student_answer> บริษัทฯ จะไม่สามารถขอคืนภาษีมูลค่าเพิ่มได้ เนื่องจากไม่ได้ยื่นคำร้องขอคืนภาษีภายในกำหนดเวลาตามมาตรา 84/1 แห่งประมวลรัษฎากร และเหตุผลที่นำเสนออาจไม่เพียงพอในการขอขยายเวลา </student_answer>


<assistant> ```json
{"point_thought": "The reference answer is making one point only saying that it cannot process the company's request due to lack of authority. The student answer is making one point as well saying that the company cannot file the request due to the deadline and due to lack of reason for deadline extension. Therefore, I should put these into lists.",
    "student_point": ["บริษัทฯ จะไม่สามารถขอคืนภาษีมูลค่าเพิ่มได้ เนื่องจากไม่ได้ยื่นคำร้องขอคืนภาษีภายในกำหนดเวลาตามมาตรา 84/1 แห่งประมวลรัษฎากร และเหตุผลที่นำเสนออาจไม่เพียงพอในการขอขยายเวลา"],
    "reference_point": ["ไม่อาจพิจารณาคำร้องขอของบริษัทฯ ได้ เนื่องจากไม่มีบทบัญญัติของกฎหมายให้อำนาจอธิบดีกรมสรรพากรขยายเวลาการยื่นคำร้องขอคืนภาษีมูลค่าเพิ่ม"],
    "coverage": {"thought": "From the extracted reference point and student point, the reference answer is making one point saying that it cannot process the company's request due to lack of authority. However, the student point states that the company cannot request for VAT refund due to the request is overdue from the deadline and its reason is not sufficient. At first glance, both of these points may lead to the same outcome which is that the company does not receive a VAT refund but the way that the student answer reaches this outcome is because of a different reasons since the reference points that it cannot process the request while the student says that the company cannot file this request. Therefore, the score would be no-coverage", "score": "no-coverage"}, "contradiction": {"thought": "As stated in the coverage section, the student states that the company cannot file the request which is not mentioned at all in the reference point. In fact, the reference points out that it can file the request just that the receiver cannot process the request due to lack of authority. Therefore, the student point directly contradicts the reference point and the score should be contradiction.", "score": "contradiction"}}
```

\end{Verbatim}