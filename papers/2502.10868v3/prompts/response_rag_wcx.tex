Below are the prompts we utilized for generating response in a RAG-based system on WCX dataset.

\textbf{System Prompt: }
\begin{Verbatim}[breaklines, breaksymbol=, breakanywheresymbolpre=]

You are an expert paralegal in Thai law domain. You have a job to answer legal questions posed by civillians and rarely make mistakes. The structure of questions you are proficient in consists of the question under the tag <question></question> and your answer and maybe rationale of that answer in the tag <answer></answer> and  may also be provided with the content of the relevant laws to that tax case under the tag <ข้อกฎหมาย></ข้อกฎหมาย> with each law section under the tag <law section="XXX"></law>. The law can also be under the tag <related_law section="XXX" law_name="XXX", parent_section="XXX" parent_law_name="XXX"></related_law> which are nested laws mention inside a given main law.
\end{Verbatim}

\textbf{Turn 0 (Instruction): }
\begin{Verbatim}[breaklines, breaksymbol=, breakanywheresymbolpre=]

<user> Take a deep breath and think carefully. Think in gradually increasing complexity
You are given a task of answering questions posed by users in the topic of Thai legal domain in which you are an expert. Given the question (<question>) and relevant laws (<ข้อกฎหมาย>) with each section under the tag <law section="XXX"></law> or <related_law section="XXX" parent_law="XXX" parent_section="XXX"></related_law>, you need to give a response to the question asked by the users. You do not need to explain your rationale and how you reach the answer. You must also cite the law section you used in your response in terms of list with structure like this [{"law": "XXX", "section": "XX"}, ...]. The cited law must be from the given relevant law only and nothing outside of it. Take a look at the law you used for analysis and answer before you provide the citations. YOU MUST ALWAYS PROVIDE THE ANSWER IN THE GIVEN STRUCTURED OUTPUT WITHOUT MISSING ANY KEYS!! THIS IS REALLY IMPORTANT. DO NOT ANSWER OR PROVIDE ANYTHING I CANNOT PARSE INTO JSON!!! Be careful, you don't need to cite every law provided since it might not be all relevant.
    
Your final output should be a JSON object with the following keys. YOU MUST ALWAYS ANSWER IN THIS FORMAT WITHOUT MISSING ANY KEYS AND THE CITATIONS MUST CONTAIN AT LEAST A LAW:
```json
{
  "analysis": "Thorough analysis of the provided cases here. Should be in English",
  "answer": "Your final answer that does not contain any elaboration but should cover all necessary points. Must be in THAI only.",
  "citations": [{"law": "Law code of the laws you think is relevant to your analysis", "section": "Section of the law code you think is relevant to your analysis"}, ...]
}
```
    
If you do this task well and provide good thought process and accurate classification, I will tip you 200 US Dollar!

<assistant> I will strictly adhere to the guidelines you told me for this task
\end{Verbatim}


\textbf{Turn 1: }
\begin{Verbatim}[breaklines, breaksymbol=, breakanywheresymbolpre=]

<user> <ข้อกฎหมาย>
<law section=31 ประมวลรัษฎากร> ประมวลรัษฎากร มาตรา 31 การอุทธรณ์ไม่เป็นการทุเลาการเสียภาษีอากร ถ้าไม่เสียภาษีอากรภายในเวลาที่กฎหมายกำหนดให้ถือเป็นภาษีอากรค้างตามมาตรา 12 เว้นแต่กรณีที่ผู้อุทธรณ์ได้รับอนุมัติจากอธิบดีให้รอคำวินิจฉัยอุทธรณ์หรือคำพิพากษาได้ ก็ให้มีหน้าที่ชำระภายในสามสิบวันนับแต่วันได้รับแจ้งคำวินิจฉัยอุทธรณ์หรือได้รับทราบคำพิพากษาถึงที่สุด แล้วแต่กรณี
ในกรณีที่มีคำวินิจฉัยอุทธรณ์ให้เสียภาษีอากรเพิ่มขึ้น ผู้อุทธรณ์จะต้องชำระภายในกำหนดเวลาเช่นเดียวกับวรรคก่อน
</law>
</ข้อกฎหมาย>
<question> ถ้าอุทธรณ์จะสามารถทุเลาการเสียภาษีได้หรือไม่ </question>

<assistant> ```json
{"analysis": "From the given question, the user wants to know if an appeal can defer the tax. According to the given relevant law which states that the appeal does not defer tax unless the person who made the appealed is authorized by the director that the payment can be waited until the result of the appeal is finalized. In that case the person who made the appeal must pay the tax in the duration of 30 days after the date in which the result of the appeal is finalized according to the result.", "answer": "ไม่ได้ เนื่องจากการอุทธรณ์ไม่เป็นการทุเล่าการเสียภาษี เว้นแต่กรณีที่ผู้อุทธรณ์ได้รับอนุมัติจากอธิบดีให้รอคำวินิจฉัยอุทธรณ์หรือคำพิพากษาได้ ก็ให้มีหน้าที่ชำระภายใน 30 วันนับแต่วันได้รับแจ้งคำวินิจฉัยอุทธรณ์หรือได้รับทราบคำพิพากษาถึงที่สุด ", "citations": [{'law': 'ประมวลรัษฎากร', 'section': '31'}]}
```
\end{Verbatim}

\textbf{Turn 2: }
\begin{Verbatim}[breaklines, breaksymbol=, breakanywheresymbolpre=]

<user> <ข้อกฎหมาย>
<law section=70 พระราชบัญญัติการจัดซื้อจัดจ้างและการบริหารพัสดุภาครัฐ พ.ศ. 2560> พระราชบัญญัติการจัดซื้อจัดจ้างและการบริหารพัสดุภาครัฐ พ.ศ. 2560 มาตรา 70 งานจ้างที่ปรึกษา ให้หน่วยงานของรัฐเลือกใช้วิธีใดวิธีหนึ่งตามมาตรา 69 ตามหลักเกณฑ์ ดังต่อไปนี้
(1) งานจ้างที่ปรึกษาโดยวิธีประกาศเชิญชวนทั่วไป ให้ใช้กับงานที่ไม่ซับซ้อน งานที่มีลักษณะเป็นงานประจำของหน่วยงานของรัฐ หรืองานที่มีมาตรฐานตามหลักวิชาชีพ และมีที่ปรึกษาซึ่งสามารถทำงานนั้นได้เป็นการทั่วไป
(2) งานจ้างที่ปรึกษาโดยวิธีคัดเลือก ให้กระทำได้ในกรณี ดังต่อไปนี้
(ก) ใช้วิธีประกาศเชิญชวนทั่วไปแล้ว แต่ไม่มีผู้ยื่นข้อเสนอ หรือข้อเสนอนั้นไม่ได้รับการคัดเลือก
(ข) เป็นงานที่ซับซ้อน ซับซ้อนมาก หรือที่มีเทคนิคเฉพาะไม่เหมาะที่จะดำเนินการโดยวิธีประกาศเชิญชวนทั่วไป
(ค) เป็นงานจ้างที่มีที่ปรึกษาในงานที่จะจ้างนั้นจำนวนจำกัด
(ง) กรณีอื่นตามที่กำหนดในกฎกระทรวง
(3) งานจ้างที่ปรึกษาโดยวิธีเฉพาะเจาะจง ให้กระทำได้ในกรณี ดังต่อไปนี้
(ก) ใช้ทั้งวิธีประกาศเชิญชวนทั่วไปและวิธีคัดเลือก หรือใช้วิธีคัดเลือกแล้ว แต่ไม่มีผู้ยื่นข้อเสนอ หรือข้อเสนอนั้นไม่ได้รับการคัดเลือก
(ข) งานจ้างที่มีวงเงินค่าจ้างครั้งหนึ่งไม่เกินวงเงินตามที่กำหนดในกฎกระทรวง
(ค) เป็นงานที่จำเป็นต้องให้ที่ปรึกษารายเดิมทำต่อจากงานที่ได้ทำไว้แล้วเนื่องจากเหตุผลทางเทคนิค
(ง) เป็นงานจ้างที่มีที่ปรึกษาในงานที่จะจ้างนั้นจำนวนจำกัดและมีวงเงินค่าจ้างครั้งหนึ่งไม่เกินวงเงินตามที่กำหนดในกฎกระทรวง
(จ) เป็นงานจ้างที่มีที่ปรึกษาในงานที่จะจ้างนั้นเพียงรายเดียว
(ฉ) เป็นงานที่มีความจำเป็นเร่งด่วนหรือที่เกี่ยวกับความมั่นคงของชาติ หากล่าช้าจะเสียหายแก่หน่วยงานของรัฐหรือความมั่นคงของชาติ
(ช) กรณีอื่นตามที่กำหนดในกฎกระทรวง
ให้คณะกรรมการนโยบายมีอำนาจประกาศในระบบเครือข่ายสารสนเทศของกรมบัญชีกลางกำหนดกรณีตัวอย่างของงานที่ไม่ซับซ้อนตาม (1) หรืองานที่ซับซ้อน ซับซ้อนมาก หรือที่มีเทคนิคเฉพาะตาม (2) (ข)
รัฐมนตรีอาจออกระเบียบเพื่อกำหนดรายละเอียดอื่นของงานจ้างที่ปรึกษาตามวรรคหนึ่งเพิ่มเติมได้ตามความจำเป็นเพื่อประโยชน์ในการดำเนินการ
</law>
<law section=76 พระราชบัญญัติการจัดซื้อจัดจ้างและการบริหารพัสดุภาครัฐ พ.ศ. 2560> พระราชบัญญัติการจัดซื้อจัดจ้างและการบริหารพัสดุภาครัฐ พ.ศ. 2560 มาตรา 76 ในการพิจารณาคัดเลือกข้อเสนอโดยวิธีตามมาตรา 69 (1) หรือ (2) นอกจากให้พิจารณาเกณฑ์ด้านคุณภาพตามมาตรา 75 แล้ว ให้เป็นไปตามเกณฑ์ในการพิจารณาและการให้น้ำหนัก ดังต่อไปนี้ด้วย
(1) กรณีงานจ้างที่ปรึกษาเพื่อดำเนินงานประจำ งานที่มีมาตรฐานเชิงคุณภาพตามหลักวิชาชีพอยู่แล้ว หรืองานที่ไม่ซับซ้อน ให้หน่วยงานของรัฐคัดเลือกผู้ยื่นข้อเสนอที่ผ่านเกณฑ์ด้านคุณภาพแล้วและให้คัดเลือกจากรายที่เสนอราคาต่ำสุด
(2) กรณีงานจ้างที่ปรึกษาที่เป็นไปตามมาตรฐานของหน่วยงานของรัฐหรืองานที่ซับซ้อน ให้หน่วยงานของรัฐคัดเลือกผู้ยื่นข้อเสนอที่ผ่านเกณฑ์ด้านคุณภาพแล้วและให้คัดเลือกจากรายที่ได้คะแนนรวมด้านคุณภาพและด้านราคามากที่สุด
(3) กรณีงานจ้างที่ปรึกษาที่มีความซับซ้อนมาก ให้หน่วยงานของรัฐคัดเลือกผู้ยื่นข้อเสนอที่ผ่านเกณฑ์ด้านคุณภาพแล้วและให้คัดเลือกจากรายที่ได้คะแนนด้านคุณภาพมากที่สุด
รัฐมนตรีอาจออกระเบียบกำหนดหลักเกณฑ์เกี่ยวกับการพิจารณาคัดเลือกข้อเสนอของงานจ้างที่ปรึกษาประเภทใดหรือกำหนดรายละเอียดอื่นของงานจ้างที่ปรึกษาตามวรรคหนึ่งเพิ่มเติมได้ตามความจำเป็นเพื่อประโยชน์ในการดำเนินการ
</law>
</ข้อกฎหมาย>
<question> ช่วยอธิบายความหมายและความสำคัญของ "ความคุ้มค่า" ในการคัดเลือกที่ปรึกษา? </question>


<assistant> ```json{"analysis": "From the question, the user wants to know the meaning of 'value for money' and its importance in the process of procurements (external hirings from government). From the given relevant laws, section 70 states about the 'how' of the process while section 76 states about the 'guideline' of the process and how to get the most value out of the consultants for each type of job. Therefore, according to section 76, one can infer the meaning of 'value for money' as the highest possible return according to the quality of the job and the price. Therefore, The importance of 'value for money' is to ensure that government agencies receive consulting services with quality appropriate to the nature of the work, while also considering efficient use of the budget", "answer": "'ความคุ้มค่า' ในการคัดเลือกที่ปรึกษาตามกฎหมายการจัดซื้อจัดจ้าง หมายถึงการพิจารณาทั้งด้านคุณภาพและราคาเพื่อให้ได้ประโยชน์สูงสุดแก่หน่วยงานของรัฐ และ ความสำคัญของ 'ความคุ้มค่า' คือการทำให้มั่นใจว่าหน่วยงานของรัฐจะได้รับบริการที่ปรึกษาที่มีคุณภาพเหมาะสมกับลักษณะของงาน ในขณะเดียวกันก็คำนึงถึงการใช้งบประมาณอย่างมีประสิทธิภาพ ซึ่งจะช่วยให้การใช้ทรัพยากรของรัฐเป็นไปอย่างมีประสิทธิภาพและเกิดประโยชน์สูงสุด", "citations": [{'law': 'ประมวลรัษฎากร', 'section': '76'}]}
```
\end{Verbatim}

\textbf{Turn 3: }
\begin{Verbatim}[breaklines, breaksymbol=, breakanywheresymbolpre=]

<user> <ข้อกฎหมาย>
<law section=56 พระราชบัญญัติบริษัทมหาชนจำกัด พ.ศ. 2535> พระราชบัญญัติบริษัทมหาชนจำกัด พ.ศ. 2535 มาตรา 56 ใบหุ้นนั้นอย่างน้อยต้องมีรายการดังต่อไปนี้
(1) ชื่อบริษัท
(2) เลขทะเบียนบริษัท และวันที่นายทะเบียนรับจดทะเบียนบริษัท
(3) ชนิด มูลค่า เลขที่ใบหุ้นและ จำนวนหุ้น
(4) ชื่อผู้ถือหุ้น
(5) ลายมือชื่อกรรมการซึ่งลงหรือพิมพ์ไว้อย่างน้อยหนึ่งคน แต่กรรมการจะมอบหมายให้นายทะเบียนหุ้นตามกฎหมายว่าด้วยหลักทรัพย์และตลาดหลักทรัพย์ลงหรือพิมพ์ลายมือชื่อแทนก็ได้
(6) วันเดือนปีที่ออกใบหุ้น
</law>
<law section=6 พระราชบัญญัติหลักทรัพย์และตลาดหลักทรัพย์ พ.ศ. 2535> พระราชบัญญัติหลักทรัพย์และตลาดหลักทรัพย์ พ.ศ. 2535 มาตรา 6 การลงลายมือชื่อของกรรมการหรือนายทะเบียนในใบหลักทรัพย์ตามพระราชบัญญัตินี้ จะใช้เครื่องจักรประทับหรือโดยวิธีอื่นใดแทนตามที่คณะกรรมการกำกับตลาดทุนกำหนดก็ได้ ทั้งนี้ ตามหลักเกณฑ์และวิธีการที่คณะกรรมการกำกับตลาดทุนประกาศกำหนด
</law>
<law section=197 พระราชบัญญัติหลักทรัพย์และตลาดหลักทรัพย์ พ.ศ. 2535> พระราชบัญญัติหลักทรัพย์และตลาดหลักทรัพย์ พ.ศ. 2535 มาตรา 197 ในการออกใบหลักทรัพย์ ให้นายทะเบียนหลักทรัพย์มีอำนาจลงลายมือชื่อในใบหลักทรัพย์แทนการลงลายมือชื่อกรรมการของบริษัทที่ออกหลักทรัพย์จดทะเบียนได้เมื่อได้รับมอบหมายจากบริษัทดังกล่าว
</law>
</ข้อกฎหมาย>
<question> การลงลายมือชื่อของนายทะเบียนหลักทรัพย์มีผลกระทบอย่างไรต่อความถูกต้องของใบหลักทรัพย์ </question>

<assistant> ```json{"analysis": "From the given question, the user wants to know how the signature of the securities registrar affects the validity of the securities certificate. The provided laws, specifically section 56 of พระราชบัญญัติบริษัทมหาชนจำกัด พ.ศ. 2535 and section 197 of พระราชบัญญัติหลักทรัพย์และตลาดหลักทรัพย์ พ.ศ. 2535, states that a share certificate must consist of multiple things including the signature of the company's director or in case that it is authorized, the signature of the securities registrar. Therefore, the signature of the securities registrar is necessary in some cases but does not affect the validity of the certificate more than the signature of the company's director.", "answer":"การลงลายมือชื่อของนายทะเบียนหลักทรัพย์มีผลกระทบอย่างไรต่อความถูกต้องของใบหลักทรัพย์ คือ ไม่มีผลกระทบใด ๆ ต่อความถูกต้องของใบหลักทรัพย์ เพราะ ตามมาตรา 56 ของพระราชบัญญัติบริษัทมหาชนจำกัด พ.ศ. 2535 และมาตรา 197 ของพระราชบัญญัติหลักทรัพย์และตลาดหลักทรัพย์ พ.ศ. 2535 นายทะเบียนหลักทรัพย์มีอำนาจลงลายมือชื่อในใบหลักทรัพย์แทนการลงลายมือชื่อกรรมการของบริษัทที่ออกหลักทรัพย์จดทะเบียนได้เมื่อได้รับมอบหมายจากบริษัทดังกล่าว", "citations": [{'law': 'พระราชบัญญัติบริษัทมหาชนจำกัด พ.ศ. 2535', 'section': '56'}, {'law': 'พระราชบัญญัติหลักทรัพย์และตลาดหลักทรัพย์ พ.ศ. 2535', 'section': '197'}]}
```
\end{Verbatim}