Below are the prompts we utilized for generating response in a RAG-based system on Tax Case dataset.

\textbf{System Prompt: }
\begin{Verbatim}[breaklines, breaksymbol=, breakanywheresymbolpre=]

You are an expert lawyer in Thai tax legal domain who never make mistakes when reviewing a tax cases. The structure of the Thai Tax case you are proficient in consists of the context and question wrapped under the tag <ข้อหารือ></ข้อหารือ> and may also be provided with the content of the relevant laws to that tax case under the tag <ข้อกฎหมาย></ข้อกฎหมาย> with each law section under the tag <law section="XXX" law_name="XXX"></law>. The law can also be under the tag <related_law section="XXX" law_name="XXX", parent_section="XXX" parent_law_name="XXX"></related_law> which are nested laws mention inside a given main law.
\end{Verbatim}

\textbf{Turn 0 (Instruction): }
\begin{Verbatim}[breaklines, breaksymbol=, breakanywheresymbolpre=]

<user> Take a deep breath and think carefully. Think in gradually increasing complexity
You are given a task of analysing a complaint or discussion filed by users in your area of expertise, Thai Tax law. 
Given the case details (<ข้อหารือ>) and relevant laws (<ข้อกฎหมาย>) with each section under the tag <law section="XXX"></law> or <related_law section="XXX"></related_law>, you need to give a response to the question or inquiry made by the complainants. You do not need to explain your rationale and how you reach the answer. You must also cite the law section you used in your response in terms of list with structure like this [{"law": "XXX", "section": "XX"}]. The cited law must be from the given law only and nothing outside of it. Take a look at the law you used for analysis and answer before you provide the citations. YOU MUST ALWAYS PROVIDE THE ANSWER IN THE GIVEN STRUCTURED OUTPUT WITHOUT MISSING ANY KEYS!! THIS IS REALLY IMPORTANT DO NOT ANSWER OR PROVIDE ANYTHING I CANNOT PARSE INTO JSON!!!. Be careful, you don't need to cite every law provided since it might not be all relevant.
    
Your final output should be a JSON object with the following keys. YOU MUST ALWAYS ANSWER IN THIS FORMAT WITHOUT MISSING ANY KEYS AND THE CITATIONS MUST CONTAIN AT LEAST A LAW:
```json
{
  "analysis": "Thorough analysis of the provided cases here. Should be in English",
  "answer": "Your final answer that does not contain any elaboration but should cover all necessary points. Must be in THAI only.",
  "citations": [{"law": "Law code of the laws you think is relevant to your analysis", "section": "Section of the law code you think is relevant to your analysis"}, ...]
}
```
    
If you do this task well and provide good thought process and accurate classification, I will tip you 200 US Dollar!

<assistant> I will strictly adhere to the guidelines you told me for this task
\end{Verbatim}


\textbf{Turn 1: }
\begin{Verbatim}[breaklines, breaksymbol=, breakanywheresymbolpre=]

<user> <ข้อกฎหมาย>
<law section=81 law_name=ประมวลรัษฎากร> ประมวลรัษฎากร มาตรา 81 ให้ยกเว้นภาษีมูลค่าเพิ่มสำหรับการประกอบกิจการประเภทต่าง ๆ ดังต่อไปนี้
(1) การขายสินค้าที่มิใช่การส่งออก หรือการให้บริการดังต่อไปนี้
(ก) การขายพืชผลทางการเกษตร ไม่ว่าจะเป็นลำต้น กิ่ง ใบ เปลือก หน่อ ราก เหง้า ดอก หัว ฝัก เมล็ด หรือส่วนอื่น ๆ ของพืช และวัตถุพลอยได้จากพืช ทั้งนี้ ที่อยู่ในสภาพสด หรือรักษาสภาพไว้เพื่อมิให้เสียเป็นการชั่วคราวในระหว่างขนส่งด้วยการแช่เย็น แช่เย็นจนแข็ง หรือด้วยการจัดทำหรือปรุงแต่งโดยวิธีการอื่น หรือรักษาสภาพไว้เพื่อมิให้เสียเพื่อการขายปลีกหรือขายส่งด้วยวิธีการแช่เย็น แช่เย็นจนแข็ง ทำให้แห้ง บด ทำให้เป็นชิ้น หรือด้วยวิธีอื่น ข้าวสารหรือผลิตภัณฑ์ที่ได้จากการสีข้าว แต่ไม่รวมถึงไม้ซุง ฟืน หรือผลิตภัณฑ์ที่ได้จากการเลื่อยไม้หรือผลิตภัณฑ์อาหารที่บรรจุกระป๋อง ภาชนะหรือหีบห่อ ที่ทำเป็นอุตสาหกรรม ตามลักษณะ และเงื่อนไขที่อธิบดีกำหนด
(ข) การขายสัตว์ ไม่ว่าจะมีชีวิตหรือไม่มีชีวิต และในกรณีสัตว์ไม่มีชีวิตไม่ว่าจะเป็นเนื้อ ส่วนต่าง ๆ ของสัตว์ ไข่ น้ำนม และวัตถุพลอยได้จากสัตว์ ทั้งนี้ ที่อยู่ในสภาพสดหรือรักษาสภาพไว้เพื่อมิให้เสียเป็นการชั่วคราวในระหว่างขนส่งด้วยการแช่เย็น แช่เย็นจนแข็ง หรือด้วยการจัดทำหรือปรุงแต่งโดยวิธีการอื่น หรือรักษาสภาพไว้เพื่อมิให้เสียเพื่อการขายปลีกหรือขายส่งด้วยวิธีการแช่เย็น แช่เย็นจนแข็ง ทำให้แห้ง บด ทำให้เป็นชิ้น หรือด้วยวิธีอื่น แต่ไม่รวมถึงผลิตภัณฑ์อาหารที่บรรจุกระป๋อง ภาชนะ หรือหีบห่อ ที่ทำเป็นอุตสาหกรรม ตามลักษณะ และ เงื่อนไขที่อธิบดีกำหนด
(ค) การขายปุ๋ย
(ง) การขายปลาป่น อาหารสัตว์
(จ) การขายยาหรือเคมีภัณฑ์ที่ใช้สำหรับพืชหรือสัตว์ เพื่อบำรุงรักษาป้องกัน ทำลายหรือกำจัดศัตรูหรือโรคของพืชและสัตว์
(ฉ) การขายหนังสือพิมพ์ นิตยสาร หรือตำราเรียน
(ช) การให้บริการการศึกษาของสถานศึกษาของทางราชการ สถานศึกษาตามกฎหมายว่าด้วยสถาบันอุดมศึกษาเอกชน หรือโรงเรียนเอกชนตามกฎหมายว่าด้วยโรงเรียนเอกชน
(ซ) การให้บริการที่เป็นงานทางศิลปะและวัฒนธรรมในสาขา และลักษณะการประกอบกิจการที่อธิบดีกำหนดโดยอนุมัติรัฐมนตรี
(ฌ) การให้บริการการประกอบโรคศิลปะ การสอบบัญชี การว่าความหรือการประกอบวิชาชีพอิสระอื่นตามที่อธิบดีกำหนดโดยอนุมัติรัฐมนตรี ทั้งนี้ เฉพาะวิชาชีพอิสระที่มีกฎหมายควบคุมการประกอบวิชาชีพอิสระนั้น
(ญ) การให้บริการรักษาพยาบาลของสถานพยาบาลตามกฎหมายว่าด้วยสถานพยาบาล
(ฎ) การให้บริการวิจัย หรือการให้บริการทางวิชาการ ทั้งนี้ ในสาขา และลักษณะการประกอบกิจการที่อธิบดีกำหนดโดยอนุมัติรัฐมนตรี
(ฎ) การให้บริการห้องสมุด พิพิธภัณฑ์ สวนสัตว์
(ฐ) การให้บริการตามสัญญาจ้างแรงงาน
(ฑ) การให้บริการจัดแข่งขันกีฬาสมัครเล่น
(ฒ) การให้บริการของนักแสดงสาธารณะ ทั้งนี้ เฉพาะบริการในสาขาและลักษณะการประกอบกิจการตามที่อธิบดีกำหนดโดยอนุมัติรัฐมนตรี
(ณ) การให้บริการขนส่งในราชอาณาจักร
(ด) การให้บริการขนส่งระหว่างประเทศ ซึ่งมิใช่เป็นการขนส่งโดยอากาศยานหรือเรือเดินทะเล
(ต) การให้บริการเช่าอสังหาริมทรัพย์
(ถ) การให้บริการของราชการส่วนท้องถิ่น ทั้งนี้ ไม่รวมถึงบริการที่เป็นการพาณิชย์ของราชการส่วนท้องถิ่น หรือเป็นการหารายได้หรือผลประโยชน์ไม่ว่าจะเป็นกิจการสาธารณูปโภคหรือไม่ก็ตาม
(ท) การขายสินค้าหรือการให้บริการของกระทรวง ทบวง กรม ซึ่งส่งรายรับทั้งสิ้นให้แก่รัฐโดยไม่หักรายจ่าย
(ธ) การขายสินค้าหรือการให้บริการเพื่อประโยชน์แก่การศาสนาหรือการสาธารณกุศลภายในประเทศ ซึ่งไม่นำผลกำไรไปจ่ายในทางอื่น
(น) การขายสินค้าหรือการให้บริการตามที่กำหนดโดยพระราชกฤษฎีกา
(2) การนำเข้าสินค้าดังต่อไปนี้
(ก) สินค้าตาม (1) (ก) ถึง (ฉ)
(ข) สินค้าจากต่างประเทศที่นำเข้าไปในเขตปลอดอากร ทั้งนี้ เฉพาะสินค้าที่ได้รับยกเว้นอากรขาเข้าตามกฎหมายว่าด้วยการนั้น
(ค) สินค้าที่จำแนกประเภทไว้ในภาคว่าด้วยของที่ได้รับยกเว้นอากรตามกฎหมายว่าด้วยพิกัดอัตราศุลกากร
(ง) สินค้าซึ่งนำเข้าและอยู่ในอารักขาของศุลกากร แล้วได้ส่งกลับออกไปต่างประเทศ โดยได้คืนอากรขาเข้าตามกฎหมายว่าด้วยศุลกากร
(3) การส่งออกซึ่งสินค้าหรือบริการของผู้ประกอบการจดทะเบียนซึ่งต้องเสียภาษีมูลค่าเพิ่มตามมาตรา 82/16
การยกเว้นภาษีมูลค่าเพิ่มสำหรับการประกอบกิจการตามมาตรานี้ อธิบดีจะเสนอให้คณะกรรมการวินิจฉัยภาษีอากรพิจารณากำหนดลักษณะของกิจการและเงื่อนไขในการประกอบกิจการที่ได้รับการยกเว้นตามมาตรานี้ก็ได้ และเมื่อคณะกรรมการวินิจฉัยภาษีอากรได้วินิจฉัยแล้ว ให้ประกาศคำวินิจฉัยของคณะกรรมการดังกล่าวในราชกิจจานุเบกษาและหากกิจการนั้นมิได้เป็นไปตามลักษณะ และเงื่อนไขที่กำหนด กิจการนั้นจะไม่ได้รับยกเว้นภาษีมูลค่าเพิ่มตามมาตรานี้ </law>
</ข้อกฎหมาย>
<ข้อหารือ> บริษัท A เป็นผู้ประกอบการนำเข้าสินค้า อินทผาลัม มะเดื่อแห้ง อาปิคอต ลูกพลัม จากต่างประเทศนำมาบรรจุ ใส่กล่องกระดาษที่คล้ายกล่องส่งไปรษณีย์ภัณฑ์ เพียงแต่นำมาสอดขัดขึ้นรูปเป็นกล่องบรรจุ เพื่อสะดวกต่อการขนย้าย และจำหน่ายภายในประเทศ แต่มิได้ทำเป็นอุตสาหกรรม และเป็นการขายทั้งปลีกและส่ง การบรรจุภาชนะหรือหีบห่อ มิได้บรรจุกระป๋อง หรือผนึกในลักษณะมั่นคง และสินค้าดังกล่าวไม่จำต้องแปรรูปใดๆ เพียงแต่เก็บไว้ในห้องที่มีอุณหภูมิไม่เกิน 1 องศาเซลเซียส บริษัทฯ จึงขอทราบว่า การขายสินค้าดังกล่าว ได้รับการยกเว้นภาษีมูลค่าเพิ่ม ตามมาตรา 81(1) (ก) แห่งประมวลรัษฎากร หรือไม่ <\ข้อหารือ>

<assistant> ```json
{"analysis": "From the given case details, the company import date palm, dried fig, apricot and plum which is considered agricultural product from foreign country. The products also undergo some process such as freezing, cooling, drying and other to prolong their shelf life. Hence, selling said products falls into the exception of VAT collection according to section 81(1)(ก) of ประมวลรัษฎากร", "answer": "การขายสินค้าดังกล่าวได้รับการยกเว้นภาษีมูลค่าเพิ่มตามมาตรา 81(1)(ก) แห่งประมวลรัษฎากร", "citations": [{'law': 'ประมวลรัษฎากร', 'section': '81'}]}
```

\end{Verbatim}

\textbf{Turn 2: }
\begin{Verbatim}[breaklines, breaksymbol=, breakanywheresymbolpre=]

<user> <ข้อกฎหมาย>
<law section=78 ประมวลรัษฎากร> ประมวลรัษฎากร มาตรา 78 ภายใต้บังคับมาตรา 78/3 ความรับผิดในการเสียภาษีมูลค่าเพิ่มที่เกิดจากการขายสินค้า ให้เป็นไปตามหลักเกณฑ์ดังต่อไปนี้
(1) การขายสินค้านอกจากที่อยู่ในบังคับตาม (2) (3) (4) หรือ(5) ให้ความรับผิดทั้งหมดเกิดขึ้นเมื่อส่งมอบสินค้า เว้นแต่กรณีที่ได้มีการกระทำดังต่อไปนี้เกิดขึ้นก่อนส่งมอบสินค้า ก็ให้ถือว่าความรับผิดเกิดขึ้นเมื่อได้มีการกระทำนั้น ๆ ด้วย
(ก) โอนกรรมสิทธิ์สินค้า
(ข) ได้รับชำระราคาสินค้า หรือ
(ค) ได้ออกใบกำกับภาษี
ทั้งนี้ โดยให้ความรับผิดเกิดขึ้นตามส่วนของการกระทำนั้น ๆ แล้วแต่กรณี
(2) การขายสินค้าตามสัญญาให้เช่าซื้อหรือสัญญาซื้อขายผ่อนชำระที่กรรมสิทธิ์ในสินค้ายังไม่โอนไปยังผู้ซื้อเมื่อได้ส่งมอบ ให้ความรับผิดเกิดขึ้นเมื่อถึงกำหนดชำระราคาตามงวดที่ถึงกำหนดชำระราคาแต่ละงวด เว้นแต่กรณีที่ได้มีการกระทำดังต่อไปนี้เกิดขึ้นก่อนถึงกำหนดชำระราคาแต่ละงวด ก็ให้ถือว่าความรับผิดเกิดขึ้นเมื่อได้มีการกระทำนั้น ๆ ด้วย
(ก) ได้รับชำระราคาสินค้า หรือ
(ข) ได้ออกใบกำกับภาษี
ทั้งนี้ โดยให้ความรับผิดเกิดขึ้นตามส่วนของการกระทำนั้น ๆ แล้วแต่กรณี
(3) การขายสินค้าโดยมีการตั้งตัวแทนเพื่อขายและได้ส่งมอบสินค้าให้ตัวแทนแล้ว ทั้งนี้ เฉพาะสัญญาการตั้งตัวแทนเพื่อขายตามประเภทของสินค้าและเป็นไปตามหลักเกณฑ์ วิธีการ และเงื่อนไขที่อธิบดีกำหนดโดยอนุมัติรัฐมนตรี ให้ความรับผิดทั้งหมดเกิดขึ้นเมื่อตัวแทนได้ส่งมอบสินค้าให้ผู้ซื้อ เว้นแต่กรณีที่ได้มีการกระทำดังต่อไปนี้เกิดขึ้นก่อนการส่งมอบสินค้าให้ผู้ซื้อ ก็ให้ถือว่าความรับผิดเกิดขึ้นเมื่อได้มีการกระทำนั้น ๆ ด้วย
(ก) ตัวแทนได้โอนกรรมสิทธิ์สินค้าให้ผู้ซื้อ
(ข) ตัวแทนได้รับชำระราคาสินค้า
(ค) ตัวแทนได้ออกใบกำกับภาษี หรือ
(ง) ได้มีการนำสินค้าไปใช้ไม่ว่าโดยตัวแทนหรือบุคคลอื่น
ทั้งนี้ โดยให้ความรับผิดเกิดขึ้นตามส่วนของการกระทำนั้น ๆ แล้วแต่กรณี
(4) การขายสินค้าโดยส่งออก ให้ความรับผิดเกิดขึ้นดังต่อไปนี้
(ก) การส่งออกนอกจากที่ระบุใน (ข) หรือ (ค) ให้ความรับผิดเกิดขึ้นเมื่อชำระอากรขาออก วางหลักประกันอากรขาออก หรือจัดให้มีผู้ค้ำประกันอากรขาออก เว้นแต่ในกรณีที่ไม่ต้องเสียอากรขาออกหรือได้รับยกเว้นอากรขาออกแล้วแต่กรณี ก็ให้ถือว่าความรับผิดเกิดขึ้นในวันที่มีการออกใบขนสินค้าขาออกตามกฎหมายว่าด้วยศุลกากร
(ข) การส่งออกในกรณีที่นำสินค้าเข้าไปในเขตปลอดอากรตามมาตรา 77/1 (14) (ก) ให้ความรับผิดเกิดขึ้นในวันที่นำสินค้าในราชอาณาจักรเข้าไปในเขตดังกล่าว
(ค) การส่งออกซึ่งสินค้าที่อยู่ในคลังสินค้าทัณฑ์บนตามกฎหมายว่าด้วยศุลกากร ให้ความรับผิดเกิดขึ้นพร้อมกับความรับผิดตามกฎหมายว่าด้วยศุลกากร
(5) การขายสินค้าที่ได้เสียภาษีมูลค่าเพิ่มในอัตราร้อยละ 0 ตามมาตรา 80/1 (5) และภายหลังได้มีการโอนกรรมสิทธิ์ในสินค้าอันทำให้ผู้รับโอนสินค้ามีหน้าที่ต้องเสียภาษีมูลค่าเพิ่มตามมาตรา 82/1 (2) ให้ความรับผิดทั้งหมดเกิดขึ้นเมื่อโอนกรรมสิทธิ์สินค้า
เพื่อเป็นการบรรเทาภาระในการยื่นแบบแสดงรายการภาษีและการชำระภาษีของผู้ประกอบการจดทะเบียนสำหรับการขายสินค้าแก่กระทรวง ทบวง กรม หรือราชการส่วนท้องถิ่น ทั้งนี้ เฉพาะการขายสินค้าตามสัญญาและมีการชำระราคาที่เป็นไปตามหลักเกณฑ์ วิธีการ และเงื่อนไขที่กำหนดโดยกฎกระทรวง ให้อธิบดีโดยอนุมัติรัฐมนตรีมีอำนาจกำหนดความรับผิดตาม (1) (2) และ (3) เกิดขึ้นเป็นอย่างอื่นได้
</law>
<law section=82/4 ประมวลรัษฎากร> ประมวลรัษฎากร มาตรา 82/4 ภายใต้บังคับมาตรา 83/5 มาตรา 83/6 และมาตรา 83/7 ให้ผู้ประกอบการจดทะเบียนเรียกเก็บภาษีมูลค่าเพิ่มจากผู้ซื้อสินค้าหรือผู้รับบริการเมื่อความรับผิดในการเสียภาษีมูลค่าเพิ่มเกิดขึ้น โดยคำนวณจากฐานภาษีตามส่วน 3 และอัตราภาษีตามส่วน 4
บทบัญญัติมาตรานี้ มิได้เป็นการห้ามผู้ประกอบการจดทะเบียนที่จะเสนอหรือแสดงราคาสินค้าหรือราคาค่าบริการแก่ผู้ซื้อสินค้าหรือผู้รับบริการ ในราคาที่รวมภาษีมูลค่าเพิ่มไว้แล้ว ทั้งนี้ ไม่ว่าผู้ประกอบการจดทะเบียนจะได้แจ้งให้ผู้ซื้อสินค้าหรือผู้รับบริการทราบด้วยหรือไม่ก็ตาม
ภาษีมูลค่าเพิ่มที่ผู้ประกอบการจดทะเบียนเรียกเก็บจากผู้ซื้อสินค้าหรือผู้รับบริการตามมาตรานี้ ย่อมเป็นภาษีขายของผู้ประกอบการจดทะเบียนนั้น
ภาษีมูลค่าเพิ่มที่ผู้ประกอบการจดทะเบียนใดถูกผู้ประกอบการจดทะเบียนอื่นเรียกเก็บตามมาตรานี้ เนื่องจากการซื้อสินค้าหรือรับบริการมาเพื่อใช้ในการประกอบกิจการของตน ย่อมเป็นภาษีซื้อของผู้ประกอบการจดทะเบียนที่เป็นผู้ซื้อสินค้าหรือผู้รับบริการนั้น
</law>
<law section=77/2 ประมวลรัษฎากร> ประมวลรัษฎากร มาตรา 77/2 การกระทำกิจการดังต่อไปนี้ในราชอาณาจักร ให้อยู่ในบังคับต้องเสียภาษีมูลค่าเพิ่ม ตามบทบัญญัติในหมวดนี้
(1) การขายสินค้าหรือการให้บริการโดยผู้ประกอบการ
(2) การนำเข้าสินค้าโดยผู้นำเข้า
การให้บริการในราชอาณาจักรให้หมายถึง บริการที่ทำในราชอาณาจักรโดยไม่คำนึงว่าการใช้บริการนั้นจะอยู่ในต่างประเทศหรือในราชอาณาจักร
การให้บริการที่ทำในต่างประเทศและได้มีการใช้บริการนั้นในราชอาณาจักรให้ถือว่าการให้บริการนั้น เป็นการให้บริการในราชอาณาจักร
</law>
<law section=86 ประมวลรัษฎากร> ประมวลรัษฎากร มาตรา 86 ภายใต้บังคับมาตรา 86/1 มาตรา 86/2 และมาตรา 86/8 ให้ผู้ประกอบการจดทะเบียนจัดทำใบกำกับภาษีและสำเนาใบกำกับภาษีสำหรับการขายสินค้า หรือการให้บริการทุกครั้ง และต้องจัดทำในทันทีที่ความรับผิดในการเสียภาษีมูลค่าเพิ่มเกิดขึ้น พร้อมทั้งให้ส่งมอบใบกำกับภาษีนั้นแก่ผู้ซื้อสินค้าหรือผู้รับบริการส่วนสำเนาใบกำกับภาษีให้เก็บรักษาไว้ตามมาตรา 87/3
ผู้ประกอบการที่ได้รับยกเว้นการจดทะเบียนภาษีมูลค่าเพิ่มและได้จดทะเบียนภาษีมูลค่าเพิ่มชั่วคราวตามมาตรา 85/3 จะออกใบกำกับภาษีได้ต่อเมื่อเป็นไปตามหลักเกณฑ์ วิธีการ และเงื่อนไขที่อธิบดีกำหนด
ใบกำกับภาษีให้ออกเป็นรายสถานประกอบการ ทั้งนี้ เว้นแต่อธิบดีจะกำหนดเป็นอย่างอื่น
การออกใบกำกับภาษีโดยตัวแทนในนามของผู้ประกอบการจดทะเบียนให้เป็นไปตามหลักเกณฑ์ วิธีการ และเงื่อนไขที่อธิบดีกำหนด
</law>
<law section=80 ประมวลรัษฎากร> ประมวลรัษฎากร มาตรา 80 ให้ใช้อัตราภาษีร้อยละ 10.0 ในการคำนวณภาษีมูลค่าเพิ่มสำหรับการประกอบกิจการดังต่อไปนี้ ทั้งนี้ เว้นแต่กรณีที่กำหนดไว้ในมาตรา 80/2
(1) การขายสินค้า
(2) การให้บริการ
(3) การนำเข้า
อัตราภาษีตามวรรคหนึ่ง ให้ลดลงได้โดยตราเป็นพระราชกฤษฎีกาแต่ต้องกำหนดอัตราภาษีให้เป็นอัตราภาษีเดียวกันสำหรับการขายสินค้าการให้บริการและการนำเข้าทุกกรณี
</law>
</ข้อกฎหมาย>
<ข้อหารือ> บริษัทฯ ได้รับคำสั่งซื้อแท้งค์จำนวน 2 ใบ ราคาใบละ 1,364,814 บาท จาก PT ซึ่งเป็นบริษัทที่จัดตั้งขึ้นในประเทศอินโดนีเซีย โดยสรุปข้อเท็จจริงได้ดังนี้
          1. บริษัทฯ ได้ว่าจ้างบริษัท ท. เป็นผู้ผลิตแท้งค์ และบริษัทฯ ได้เก็บแท้งค์ที่ผลิตเสร็จแล้วไว้ที่คลังสินค้าที่จังหวัดชลบุรี ซึ่งบริษัทฯ ได้เช่าจากบริษัท ท.
          2. PT ตกลงจะขนส่งแท้งค์ออกจากคลังสินค้าด้วยตนเอง โดยให้บริษัท ฮ. ซึ่งเป็นบริษัทในเครือของ PT ที่ตั้งอยู่ใน ประเทศไทย เป็นผู้ดำเนินการ
          3. ต่อมาบริษัท ฮ. ได้ว่าจ้าง ขนส่งแท้งค์จากคลังสินค้าของบริษัทฯ ไปยังบริษัท ฮ. สาขาจังหวัดสงขลา เพื่อส่งออก ไปให้ PT ที่ประเทศอินโดนีเซีย
          4. บริษัทฯ ได้ออกใบส่งของให้แก่ PT โดย PT เป็นผู้ชำระเงินให้แก่บริษัทฯ และมีบริษัท ฮ. สาขาจังหวัดสงขลา เป็น ผู้เซ็นรับสินค้าและดำเนินพิธีการศุลกากรเพื่อส่งออกในนามของบริษัท ฮ.
          กรณีที่บริษัทฯ ขายสินค้าให้กับ PT ตามข้อเท็จจริงดังกล่าว ถือเป็นการส่งออก ซึ่งจะได้รับสิทธิเสียภาษีมูลค่าเพิ่ม ในอัตราร้อยละ 0 หรือไม่ <\ข้อหารือ>


<assistant> ```json{"analysis": "From the given case details, the company sells two tanks to PT company which is established under Indonesia's law and PT company agrees to move the tank from the company's inventory by themselves and require the company to sign over the tank to ฮ company which is its subsidary. This is considered selling inside the Kingdom of Thailand and according to section 77/2 of ประมวลรัษฎากร, the company is subjected to 7% VAT which the company must issue the invoice to the buyer from other countries according to section 78, 80, 82/4 and 86 of ประมวลรัษฎากร.", "answer": "บริษัทฯ มีหน้าที่ต้องเสียภาษีมูลค่าเพิ่มในอัตราร้อยละ 7.0 และต้องจัดทำใบกำกับภาษีเรียกเก็บภาษีมูลค่าเพิ่มจากผู้ซื้อในต่างประเทศ", 
"citations": [{'law': 'ประมวลรัษฎากร', 'section': '77/2'}, {'law': 'ประมวลรัษฎากร', 'section': '78'}, {'law': 'ประมวลรัษฎากร', 'section': '80'}, {'law': 'ประมวลรัษฎากร', 'section': '82/4'}, {'law': 'ประมวลรัษฎากร', 'section': '86'}]}
```

\end{Verbatim}

\textbf{Turn 3: }
\begin{Verbatim}[breaklines, breaksymbol=, breakanywheresymbolpre=]

<user> <ข้อกฎหมาย>
<law section=27 ประมวลรัษฎากร> ประมวลรัษฎากร มาตรา 27 บุคคลใดไม่เสียหรือนำส่งภาษีภายในกำหนดเวลาตามที่บัญญัติไว้ในหมวดต่าง ๆ แห่งลักษณะนี้เกี่ยวกับภาษีอากรประเมิน ให้เสียเงินเพิ่มอีกร้อยละ 1.5 ต่อเดือนหรือเศษของเดือนของเงินภาษีที่ต้องเสียหรือนำส่งโดยไม่รวมเบี้ยปรับ
ในกรณีอธิบดีอนุมัติให้ขยายกำหนดเวลาชำระหรือนำส่งภาษีและได้มีการชำระหรือนำส่งภาษีภายในกำหนดเวลาที่ขยายให้นั้น เงินเพิ่มตามวรรคหนึ่งให้ลดลงเหลือร้อยละ 0.75 ต่อเดือนหรือเศษของเดือน
การคำนวณเงินเพิ่มตามวรรคหนึ่งและวรรคสอง ให้เริ่มนับเมื่อพ้นกำหนดเวลาการยื่นรายการหรือนำส่งภาษีจนถึงวันชำระหรือนำส่งภาษี แต่เงินเพิ่มที่คำนวณได้มิให้เกินจำนวนภาษีที่ต้องเสียหรือนำส่ง ไม่ว่าภาษีที่ต้องเสียหรือนำส่งนั้นจะเกิดจากการประเมินหรือคำสั่งของเจ้าพนักงานหรือคำวินิจฉัยอุทธรณ์หรือคำพิพากษาของศาล
</law>
<law section=48 ประมวลรัษฎากร> ประมวลรัษฎากร มาตรา 48 เงินได้พึงประเมินต้องเสียภาษีเงินได้ดังต่อไปนี้
(1) เงินได้พึงประเมินเมื่อได้หักตามมาตรา 42 ทวิ ถึงมาตรา 47 หรือมาตรา 57 เบญจ แล้ว เหลือเท่าใดเป็นเงินได้สุทธิ ต้องเสียภาษีในอัตราที่กำหนดในบัญชีอัตราภาษีเงินได้ท้ายหมวดนี้
(2) สำหรับผู้มีเงินได้พึงประเมินตั้งแต่ 120,000 บาทขึ้นไป การคำนวณภาษีตาม (1) ให้เสียไม่น้อยกว่าร้อยละ 0.5 ของยอดเงินได้พึงประเมิน
การนับจำนวนเงินได้พึงประเมินตาม (2) ไม่รวมถึงเงินได้พึงประเมินตามมาตรา 40 (1)
(3) ผู้มีเงินได้จะเลือกเสียภาษีในอัตราร้อยละ 15.0 ของเงินได้โดยไม่ต้องนำไปรวมคำนวณภาษีตาม (1) และ (2) ก็ได้ สำหรับเงินได้ตามมาตรา 40 (4) (ก) และ (ช) ดังต่อไปนี้
(ก) ดอกเบี้ยพันธบัตร ดอกเบี้ยเงินฝากธนาคารในราชอาณาจักร ดอกเบี้ยเงินฝากสหกรณ์ ดอกเบี้ยหุ้นกู้ ดอกเบี้ยตั๋วเงินที่ได้จากบริษัทหรือห้างหุ้นส่วนนิติบุคคลหรือนิติบุคคลอื่น ดอกเบี้ยเงินกู้ยืมที่ได้จากบริษัทหรือห้างหุ้นส่วนนิติบุคคลหรือนิติบุคคลอื่น ดอกเบี้ยที่ได้จากสถาบันการเงินที่มีกฎหมายโดยเฉพาะของประเทศไทยจัดตั้งขึ้นสำหรับให้กู้ยืมเงินเพื่อส่งเสริมเกษตรกรรม พาณิชยกรรม หรืออุตสาหกรรม
(ข) ผลต่างระหว่างราคาไถ่ถอนกับราคาจำหน่ายตั๋วเงิน หรือตราสารแสดงสิทธิในหนี้ที่บริษัทหรือห้างหุ้นส่วนนิติบุคคลหรือนิติบุคคลอื่นเป็นผู้ออก
(ค) ผลประโยชน์ที่ได้จากการโอนพันธบัตร หุ้นกู้ หรือตั๋วเงิน หรือตราสารแสดงสิทธิในหนี้ที่บริษัทหรือห้างหุ้นส่วนนิติบุคคลหรือนิติบุคคลอื่นเป็นผู้ออก ทั้งนี้ เฉพาะที่ตีราคาเป็นเงินได้เกินกว่าที่ลงทุน
ผู้มีเงินได้ซึ่งเป็นผู้อยู่ในประเทศไทยจะเลือกเสียภาษีในอัตราร้อยละ 10.0 ของเงินได้โดยไม่ต้องนำไปรวมคำนวณภาษีตาม (1) และ (2) ก็ได้สำหรับเงินได้ตามมาตรา 40 (4) (ข) ที่ได้รับจากบริษัทหรือห้างหุ้นส่วนนิติบุคคลที่ตั้งขึ้นตามกฎหมายไทย หรือสถาบันการเงินที่มีกฎหมายโดยเฉพาะของประเทศไทยจัดตั้งขึ้นสำหรับให้กู้ยืมเงินเพื่อส่งเสริมเกษตรกรรม พาณิชยกรรม หรืออุตสาหกรรม
(4) ผู้มีเงินได้จะเลือกเสียภาษีโดยไม่ต้องนำไปรวมคำนวณภาษีตาม(1) และ (2) ก็ได้ เฉพาะเงินได้ตามมาตรา 40 (8) ที่ได้รับจากการขายอสังหาริมทรัพย์อันเป็นมรดกหรืออสังหาริมทรัพย์ที่ได้มาโดยมิได้มุ่งในทางการค้าหรือหากำไรดังต่อไปนี้
(ก) เงินได้จากการขายอสังหาริมทรัพย์อันเป็นมรดกหรืออสังหาริมทรัพย์ที่ได้รับจากการให้โดยเสน่หาให้หักค่าใช้จ่ายร้อยละ 50 ของเงินได้ เหลือเท่าใดถือเป็นเงินได้สุทธิแล้วหารด้วยจำนวนปีที่ถือครองได้ผลลัพธ์เป็นเงินเท่าใดให้คำนวณภาษีตามอัตราภาษีเงินได้ได้เท่าใดให้คูณด้วยจำนวนปีที่ถือครองผลลัพธ์ที่ได้เป็นเงินภาษีที่ต้องเสีย
(ข)เงินได้จากการขายอสังหาริมทรัพย์ที่ได้มาโดยทางอื่นนอกจาก (ก) ให้หักค่าใช้จ่ายได้ตามที่กำหนดโดยพระราชกฤษฎีกาเหลือเท่าใดถือเป็นเงินได้สุทธิ แล้วหารด้วยจำนวนปีที่ถือครองได้ผลลัพธ์เป็นเงินเท่าใด ให้คำนวณภาษีตามอัตราภาษีเงินได้ ได้เท่าใดให้คูณด้วยจำนวนปีที่ถือครองผลลัพธ์ที่ได้เป็นเงินภาษีที่ต้องเสีย
ในกรณีที่เสียภาษีโดยไม่นำไปรวมคำนวณภาษีตาม (1) และ (2) เมื่อคำนวณภาษีแล้วต้องเสียไม่เกินร้อยละ 20 ของราคาขาย
ในกรณีที่เสียภาษีโดยนำไปรวมคำนวณภาษีตาม (1) และ (2) ให้หักค่าใช้จ่ายร้อยละ 50 ของเงินได้ตาม (ก) หรือตามที่กำหนดโดยพระราชกฤษฎีกาตาม (ข) แล้วแต่กรณี เหลือเท่าใดนำไปรวมคำนวณภาษีกับเงินได้อย่างอื่น
คำว่า “จำนวนปีที่ถือครอง” ใน (ก) หรือ (ข) หมายถึงจำนวนปีนับตั้งแต่ปีที่ได้กรรมสิทธิ์หรือสิทธิครอบครองในอสังหาริมทรัพย์ ถึงปีที่โอนกรรมสิทธิ์หรือสิทธิครอบครองในอสังหาริมทรัพย์นั้น ถ้าเกินสิบปีให้นับเพียงสิบปี และเศษของปีให้นับเป็นหนึ่งปี
(4/1) ผู้มีเงินได้จะเลือกเสียภาษีในอัตราร้อยละ 5 ของเงินได้ส่วนที่เกินยี่สิบล้านบาท โดยไม่ต้องนำไปรวมคำนวณภาษีตาม (1) และ (2) ก็ได้ สำหรับเงินได้ตามมาตรา 40 (8) ที่เป็นเงินได้จากการโอนกรรมสิทธิ์หรือสิทธิครอบครองในอสังหาริมทรัพย์โดยไม่มีค่าตอบแทนให้แก่บุตรชอบด้วยกฎหมายซึ่งไม่รวมถึงบุตรบุญธรรม ที่ไม่ได้รับยกเว้นตามมาตรา 42 (26)
(5) ผู้มีเงินได้จะเลือกเสียภาษีโดยไม่ต้องนำไปรวมคำนวณภาษีตาม (1) และ (2) ก็ได้ สำหรับเงินได้พึงประเมินตามมาตรา 40 (1) และ (2) ซึ่งเป็นเงินที่นายจ้างจ่ายให้ครั้งเดียวเพราะเหตุออกจากงาน ซึ่งได้คำนวณจ่ายจากระยะเวลาที่ทำงานและได้จ่ายตามหลักเกณฑ์ วิธีการ และเงื่อนไขที่อธิบดีกำหนด โดยให้นำเงินได้พึงประเมินดังกล่าวหักค่าใช้จ่ายเป็นจำนวนเท่ากับ 7,000 บาท คูณด้วยจำนวนปีที่ทำงานแต่ไม่เกินเงินได้พึงประเมิน เหลือเท่าใดให้หักค่าใช้จ่ายอีกร้อยละ 50 ของเงินที่เหลือนั้นแล้วคำนวณภาษีตามอัตราภาษีเงินได้
ในกรณีเงินได้พึงประเมินดังกล่าวจ่ายในลักษณะเงินบำเหน็จจำนวนหนึ่งและเงินบำนาญอีกจำนวนหนึ่ง ให้ถือว่าเฉพาะเงินที่จ่ายในลักษณะเงินบำเหน็จเป็นเงินซึ่งนายจ้างจ่ายให้ครั้งเดียวเพราะเหตุออกจากงานและให้ลดค่าใช้จ่ายจำนวน 7,000 บาท ลงเหลือ 3,500 บาท
จำนวนปีที่ทำงานตามวรรคหนึ่ง ในกรณีเงินบำเหน็จหรือเงินอื่นใดในลักษณะเดียวกัน ที่ทางราชการจ่าย ให้ถือจำนวนปีที่ใช้เป็นเกณฑ์คำนวณเงินบำเหน็จหรือเงินอื่นในลักษณะเดียวกันนั้นตามกฎหมาย ระเบียบ หรือข้อบังคับของทางราชการ
ในการคำนวณจำนวนปีที่ทำงาน นอกจากกรณีตามวรรคสาม เศษของปี ถ้าถึงหนึ่งร้อยแปดสิบสามวันให้ถือเป็นหนึ่งปี ถ้าไม่ถึงหนึ่งร้อยแปดสิบสามวันให้ปัดทิ้ง
(6) ผู้มีเงินได้จะเลือกเสียภาษีในอัตราร้อยละ 5 ของเงินได้ส่วนที่เกินยี่สิบล้านบาท โดยไม่ต้องนำไปรวมคำนวณภาษีตาม (1) และ (2) ก็ได้ สำหรับเงินได้ตามมาตรา 40 (8) ที่ได้รับจากการอุปการะหรือจากการให้โดยเสน่หาจากบุพการี ผู้สืบสันดาน หรือคู่สมรส ที่ไม่ได้รับยกเว้นตามมาตรา 42 (27)
(7) ผู้มีเงินได้จะเลือกเสียภาษีในอัตราร้อยละ 5 ของเงินได้ส่วนที่เกินสิบล้านบาท โดยไม่ต้องนำไปรวมคำนวณภาษีตาม (1) และ (2) ก็ได้ สำหรับเงินได้ตามมาตรา 40 (8) ที่ได้รับจากการอุปการะโดยหน้าที่ธรรมจรรยาหรือจากการให้โดยเสน่หาเนื่องในพิธีหรือตามโอกาสแห่งขนบธรรมเนียมประเพณี ทั้งนี้ จากบุคคลซึ่งมิใช่บุพการี ผู้สืบสันดาน หรือคู่สมรส ที่ไม่ได้รับยกเว้นตามมาตรา 42 (28)
</law>
<law section=50 ประมวลรัษฎากร> ประมวลรัษฎากร มาตรา 50 ให้บุคคล ห้างหุ้นส่วน บริษัท สมาคม หรือคณะบุคคลผู้จ่ายเงินได้พึงประเมินตามมาตรา 40 หักภาษีเงินได้ไว้ทุกคราวที่จ่ายเงินได้พึงประเมินตามวิธีดังต่อไปนี้
(1)ในกรณีเงินได้พึงประเมินตามมาตรา 40 (1) และ (2) ให้คูณเงินได้พึงประเมินที่จ่ายด้วยจำนวนคราวที่จะต้องจ่ายเพื่อให้ได้จำนวนเงินเสมือนหนึ่งว่าได้จ่ายทั้งปี แล้วคำนวณภาษีตามเกณฑ์ในมาตรา 48เป็นเงินภาษีทั้งสิ้นเท่าใดให้หารด้วยจำนวนคราวที่จะต้องจ่ายได้ผลลัพธ์เป็นเงินเท่าใด ให้หักเป็นเงินภาษีไว้เท่านั้น
ถ้าการหารด้วยจำนวนคราวที่จะต้องจ่ายตามความในวรรคก่อนไม่ลงตัวเหลือเศษเท่าใดให้เพิ่มเงินเท่าจำนวนที่เหลือเศษนั้นรวมเข้ากับเงินภาษีที่จะต้องหักไว้ครั้งสุดท้ายในปีนั้นเพื่อให้ยอดเงินภาษีที่หักรวมทั้งปีเท่ากับจำนวนภาษีที่จะต้องเสียทั้งปี
ในกรณีเงินได้พึงประเมินตามมาตรา 40 (1) และ (2) ซึ่งเป็นเงินที่นายจ้างจ่ายให้ครั้งเดียวเพราะเหตุออกจากงาน ซึ่งได้คำนวณจ่ายจากระยะเวลาที่ทำงานและได้จ่ายตามหลักเกณฑ์ วิธีการ และเงื่อนไขที่อธิบดีกำหนด ให้คำนวณภาษีตามเกณฑ์ในมาตรา 48 (5) เป็นเงินภาษีทั้งสิ้นเท่าใดให้หักเป็นเงินภาษีไว้เท่านั้น
ในกรณีเงินได้พึงประเมินตามมาตรา 40 (2) นอกจากที่ระบุไว้ในวรรคสามที่จ่ายให้แก่ผู้รับซึ่งมิได้เป็นผู้อยู่ในประเทศไทย ให้คำนวณหักในอัตราร้อยละ 15.0 ของเงินได้
(2) ในกรณีเงินได้พึงประเมินตามมาตรา 40 (3) และ (4) ให้คำนวณหักตามอัตราภาษีเงินได้ เว้นแต่
(ก) ในกรณีเงินได้พึงประเมินตามมาตรา 40 (3) และ (4) นอกจากที่ระบุไว้ใน (ข) (ค) (ง) และ (จ) ที่จ่ายให้แก่ผู้รับซึ่งมิได้เป็นผู้อยู่ในประเทศไทยให้คำนวณหักในอัตราร้อยละ 15.0 ของเงินได้
(ข) ในกรณีเงินได้พึงประเมินที่ระบุในมาตรา 48 (3) (ก) และ (ค) ให้คำนวณหักในอัตราร้อยละ 15.0 ของเงินได้
(ค) ในกรณีเงินได้พึงประเมินที่ระบุในมาตรา 48 (3) (ข) ให้ถือว่าผู้ออกตั๋วเงิน ผู้ออกตราสารแสดงสิทธิในหนี้ หรือนิติบุคคลผู้โอนตั๋วเงินหรือตราสารดังกล่าว ให้แก่ผู้มีหน้าที่เสียภาษีเงินได้ตามส่วนนี้ เป็นผู้จ่ายเงินได้พึงประเมิน และให้เรียกเก็บภาษีเงินได้จากผู้มีเงินได้ในอัตราร้อยละ 15.0 ของเงินได้และให้ถือว่าภาษีที่เรียกเก็บนั้นเป็นภาษีที่หักไว้
(ง) ในกรณีเงินได้พึงประเมินตามมาตรา 40 (4) (ก) ที่มิได้ระบุใน (ข) และ (ค) แห่งมาตรานี้ ถ้าผู้จ่ายเงินได้มิใช่เป็นนิติบุคคล และจ่ายให้แก่ผู้รับซึ่งเป็นผู้อยู่ในประเทศไทย ไม่ต้องหักภาษีตามมาตรานี้
(จ) ในกรณีเงินได้พึงประเมินตามมาตรา 40 (4) (ข) ให้คำนวณหักในอัตราร้อยละ 10.0 ของเงินได้
(ฉ) ในกรณีเงินได้พึงประเมินตามมาตรา 40 (4) (ซ) และ (ฌ) ให้คำนวณหักในอัตราร้อยละ 15.0 ของเงินได้
(3)ในกรณีเงินได้พึงประเมินตามมาตรา 40 (5) และ (6) ที่จ่ายให้แก่ผู้รับซึ่งมิได้เป็นผู้อยู่ในประเทศไทยให้คำนวณหักในอัตราร้อยละ 15.0 ของเงินได้
(4) นอกจากกรณีตาม (5) ในกรณีผู้จ่ายเงินตามมาตรานี้เป็นรัฐบาล องค์การของรัฐบาล เทศบาล สุขาภิบาล หรือองค์การบริหารราชการส่วนท้องถิ่นอื่น ซึ่งจ่ายเงินได้พึงประเมินตามมาตรา 40 (5) (6) (7) หรือ (8) แต่ไม่รวมถึงการจ่ายค่าซื้อพืชผลทางการเกษตร ให้กับผู้รับรายหนึ่ง ๆ มีจำนวนรวมทั้งสิ้นตั้งแต่ 10,000 บาทขึ้นไป แม้การจ่ายนั้นจะได้แบ่งจ่ายครั้งหนึ่ง ๆ ไม่ถึง 10,000 บาทก็ดี ให้คำนวณหักในอัตราร้อยละ 1 ของยอดเงินได้พึงประเมิน แต่เฉพาะเงินได้ในการประกวดหรือแข่งขันให้คำนวณหักตามอัตราภาษีเงินได้
(5)ในกรณีเงินได้พึงประเมินตามมาตรา 40 (8) เฉพาะที่จ่ายให้แก่ผู้รับซึ่งขายอสังหาริมทรัพย์ ให้คำนวณหักดังต่อไปนี้
(ก) สำหรับอสังหาริมทรัพย์อันเป็นมรดกหรืออสังหาริมทรัพย์ที่ได้รับจากการให้โดยเสน่หา ให้คำนวณภาษีตามเกณฑ์ในมาตรา 48 (4) (ก) เป็นเงินภาษีทั้งสิ้นเท่าใด ให้หักเป็นเงินภาษีไว้เท่านั้น
(ข)สำหรับอสังหาริมทรัพย์ที่ได้มาโดยทางอื่นนอกจาก (ก) ให้หักค่าใช้จ่ายเป็นการเหมาตามที่กำหนดโดยพระราชกฤษฎีกาแล้วคำนวณภาษีตามเกณฑ์ในมาตรา 48 (4) (ข) เป็นเงินภาษีทั้งสิ้นเท่าใดให้หักเป็นเงินภาษีไว้เท่านั้น
(6) ในกรณีการโอนกรรมสิทธิ์หรือสิทธิครอบครองในอสังหาริมทรัพย์โดยไม่มีค่าตอบแทน ให้ถือว่าผู้โอนเป็นผู้จ่ายเงินได้ โดยให้ผู้โอนหักภาษีตามเกณฑ์ใน (5) เว้นแต่กรณีการโอนให้แก่บุตรชอบด้วยกฎหมายซึ่งไม่รวมถึงบุตรบุญธรรม ให้ผู้โอนหักภาษีไว้ร้อยละ 5 ของเงินได้เฉพาะในส่วนที่เกินยี่สิบล้านบาท
</law>
</ข้อกฎหมาย>
<ข้อหารือ> บริษัท ก จำกัด และบริษัท ข จำกัด เป็นบริษัทในเครือเดียวกันของบริษัท ค ประเทศ
เยอรมัน โดยที่บริษัท ข ประกอบธุรกิจบริหารการขายและการตลาดผลิตภัณฑ์เวชภัณฑ์ซึ่งผลิตโดยบริษัท ก
ต่อมาได้มีการปรับโครงสร้างบริษัททั้งสองใหม่ โดยแยกส่วนบริหารการขายและการตลาดเฉพาะผลิตภัณฑ์
เวชภัณฑ์ของบริษัท ข มารวมกับบริษัท ก และได้โอนย้ายพนักงานผู้ทำงานในส่วนดังกล่าวมาอยู่กับบริษัทฯ
ด้วย โดยได้รับความยินยอมเป็นหนังสือจากพนักงาน ทั้งมีเงื่อนไขว่าบริษัทฯ จะนับอายุการทำงานของ
พนักงานต่อเนื่องไปด้วยและบริษัท ข มิได้บอกกล่าวเลิกจ้างพนักงานและมิได้จ่ายเงินชดเชยใด ๆ บริษัท
ก จึงขอทราบเกี่ยวกับเงินที่นายจ้างจ่ายให้ครั้งเดียวเพราะเหตุออกจากงาน ดังนี้
1. พนักงานสองคนได้ทำงานให้กับบริษัท ข คนที่ 1 ทำงานเป็นเวลา 20 ปี พนักงานคนที่
2 ทำงานเป็นเวลา 3 ปี ต่อมาเมื่อโอนมาทำงานกับบริษัท ก ปรากฏว่าคนที่ 1 ทำงานกับบริษัท ก เป็น
เวลา 2 ปี (รวมทำงาน 22 ปี) และคนที่ 2 ทำงานกับบริษัท ก เป็นเวลา 2 ปี (รวมทำงาน 5 ปี)
พนักงานทั้งสองจึงได้ลาออกจากงานหรือถูกเลิกจ้าง และได้รับเงินชดเชยตามกฎหมายคุ้มครองแรงงาน
จากบริษัท ก กรณีนี้พนักงานทั้งสองจะมีสิทธิเลือกเสียภาษีตามมาตรา 48 (5) แห่งประมวลรัษฎากร
หรือไม่ และบริษัท ก จะต้องคำนวณภาษีเงินได้หัก ณ ที่จ่ายตามเกณฑ์ในมาตรา 48 (5) แห่ง
ประมวลรัษฎากร ใช่หรือไม่
2. กรณีตาม 1. บริษัท ก ได้ออกภาษีให้แก่พนักงาน แต่บริษัท ก ได้คำนวณหักภาษี ณ
ที่จ่าย สำหรับเงินที่นายจ้างจ่ายให้ครั้งเดียวเพราะเหตุออกจากงานตามมาตรา 50 (1) แห่ง
ประมวลรัษฎากรไปแล้ว บริษัท ก จะมีสิทธิขอคืนภาษีเงินได้หัก ณ ที่จ่ายที่นำส่งไว้เกินหรือไม่ และมี
เงื่อนไขหรือข้อจำกัดอย่างไร <\ข้อหารือ>

<assistant> ```json{"analysis": "From the given case details, there are two cases to be considered. The first case is when the company has transfered the provident fund of both employees to the company ก without paying one-time fee to the employees. In this case, the working years of the employees will be continued after their working years in company ข and the employees can choose to pay their tax calculated apart from other income according to section 48(5) of ประมวลรัษฎากร since they have both been working for more than 5 years and the company should deduct their fee accordingly. Therefore, if the company deduct income according to section 50(1) of ประมวลรัษฎากร instead of section 48(5), the company should 1) deliver the additional tax in case it deducts less than actual according to section 27 of ประมวลรัษฎากร 2) inform the employees that they can receive their tax refund in case it deducts more than actual. Lastly, in the second case where the company has not transfered the provident fund of both employees to the company ก, the employees' working years cannot be counted together with their previous company and therefore, there is no special case in tax calculation.", "answer": "พนักงานทั้งสองมีสิทธิเลือกเสียภาษีตามมาตรา 48 (5) แห่งประมวลรัษฎากร และบริษัท ก ต้องคำนวณภาษีเงินได้หัก ณ ที่จ่ายตามเกณฑ์ในมาตรา 48 (5) แห่งประมวลรัษฎากร แต่บริษัท ก ไม่มีสิทธิขอคืนเงินภาษีที่ถูกหักไว้เกินได้",
"citations": [{'law': 'ประมวลรัษฎากร', 'section': '27'}, {'law': 'ประมวลรัษฎากร', 'section': '48'}, {'law': 'ประมวลรัษฎากร', 'section': '50'}]}
```
\end{Verbatim}