Below are the prompts we utilized for evaluating system response against a reference response on the metric of Coverage and Contradiction on WCX dataset. 

\textbf{Turn 0 (Instruction): }
\begin{Verbatim}[breaklines, breaksymbol=, breakanywheresymbolpre=]

<user> You are given a task of grading the quality of the answer made by a student about a verdict of a case. You are given the student answer under the tag <student_answer> and the reference answer you reached under the tag <reference_answer>. You are also provided with <ข้อหารือ> as case detail you reference something with in case some entities are abbreviated. You need to evaluate the quality of the student answer based on two metrics.
1. Coverage: How much does the student answer covered the reference answer? High quality answer should covers all aspect of the reference answer and only that. This metric has only 3 grades: no-coverage, partial-coverage, full-coverage
    1.1 no-coverage: The student answer does not cover any aspect of the reference answer. It can be either wrong or irrelevant to the question
    1.2 partial-coverage: The student answer covers some aspect of the reference answer or all aspect of the reference answer but with a high amount of irrelevant information inbetween.
    1.3 full-coverage: The student answer covers all aspect of the reference answer and contains few to no irrelevant information inbetween.
2. Contradiction: Is the student answer contradicting the reference answer or the relevant law content or itself? High quality answer should not contradict either the reference answer, the law or itself. This metric differs from the above metric in some cases. For example, a student answer can have partial coverage score while still contradicting the reference answer. Also, a student answer can also have no coverage score while still not contradicting the reference answer in the case where the student answer only contains irrelevant information. This metric only has 2 grades: no-contradiction, contradiction
    2.1 no-contradiction: The student answer does not contradict the reference answer and itself
    2.2 contradiction: The student answer contradicts the reference answer or itself in any aspect of the answers

DO NOT ATTEMPT TO ASSIGN OTHER SCORE THAN THE ONE LISTED!!!

Here's the full instruction of how to do this task:
1. First, check out the <student_answer> and <reference_answer> and <ข้อหารือ>, to estimate what points they are making first and put the thoughts in the key <student_points> and <reference_points>. BE CAREFUL, SOMETIMES THE POINTS ARE NOT SPECIFIED EXPLICITLY SUCH AS "การให้บริการทั้งสองแบบนั้นต้องเสียภาษี" WHICH SHOULD BE SPLITTED INTO "การให้บริการแบบแรกต้องเสียภาษี" AND "การให้บริการแบบที่สองต้องเสียภาษี"
2. From the points extracted, measure the two metrics. If student covers only some points of reference, the score is partial-coverage. If all, it's full-coverage. If none, it's no-coverage. As for contradiction, if any student point contradicts with itself or the reference points or the law, it's contradiction. Otherwise, it's no-contradiction.
    
YOU MUST PROVIDE YOUR RESPONSE IN JSON STRUCTURE AS FOLLOWING:
{"point_thought": "(str) Your rationale on how to split reference answers and student answers into list of points made" ,
  "student_points": "(List[str]) List of string representing each point the student answer made",
  "reference_points": "(List[str]) List of string representing each point the reference answer made",
  "coverage": {"thought": "(str) Your thought on how you evaluate the student points and reference points on the coverage aspect and what score should you give", "score": "(str) Final coverage score of the student answer. Should only be 1 of 3 grades: no-coverage, partial-coverage or full-coverage"},
  "contradiction": {"thought": "(str) Your thought on how you evaluate the student points and reference points on the contradiction aspect and what score should you give", "score": "(str) Final contradiction score of the student answer. Should only be 1 of 2 grades: no-contradiction or contradiction"}
}


Caution:
    1. MAKE SURE TO PROVIDE A REASONABLE THOUGHT PROCESS AS WELL.
    2. BE CAREFUL NOT TO CONSIDER THE ACCURACY OF CITATION OF THE LAW IN THE ANSWER IN BOTH CONTRADICTION AND COVERAGE METRICS. THERE IS A SEPARATE METRIC FOR THAT. FOR EXAMPLE, IF A STUDENT ANSWER "บริษัทไม่ต้องจ่ายภาษีตามมาตรา 82/1 ประมวลรัษฎากร" BUT THE REFERENCE USE SECION 82/2 INSTEAD BUT OTHERWISE IS THE SAME. THE COVERAGE SCORE IS full-coverage WHILE THE CONTRADICTION SCORE IS no-coverage. PLEASE BE VERY CAREFUL OF THIS!!!. THE OMISSION OF LAW IN STUDENT ANSWER IS ALSO FINE IF THE MAIN POINT IS STILL THE SAME
    3. *** THIS IS REALLY IMPORTANT!! *** IF THE STUDENT'S ANSWER CONTAINS ADDITIONAL INFORMATION THAT IS NOT COVERED BY THE REFERENCE ANSWER AND IT IS NOT CONTRADICTING ANY POINT OF THE REFERENCE ANSWER, YOU MUST NOT USE THE EXCESS INFORMATION TO COUNT TOWARDS THE COVERAGE SCORE. FOR EXAMPLE, IF THE STUDENT MAKES THE POINT a and b AND THE REFERENCE MAKES POINT a. THE STUDENT WOULD STILL GET full-coverage score
    4. *** THIS IS IMPORTANT *** IF THE STUDENT MISSES SOMETHING FROM REFERENCE ANSWER POINTS WHICH ARE NOT DIRECTLY ASKED OR NOT A CRUCIAL PART OF REFERENCE ANSWER. THE COVERAGE SCORE WOULD NOT BE AFFECTED OR DEDUCED BY IT. FOR EXAMPLE, IF STUDENT MAKES point a WHILE REFERENCE MAKES point a, b and point b is not important part of the answer and it is not asked in the question. THE STUDENT WOULD STILL GET full-coverage

If you do this task well and provide good thought process and accurate classification, I will tip you 200 US Dollar!
    
Take a deep breath and think carefully. Think in gradually increasing complexity

<assistant> I will strictly adhere to the guidelines you told me for this task

\end{Verbatim}


\textbf{Turn 1: }
\begin{Verbatim}[breaklines, breaksymbol=, breakanywheresymbolpre=]

<user> <question> คู่สัญญาสามารถเรียกร้องอะไรได้บ้างหากรู้ว่าผู้บังคับหลักประกันไม่เป็นกลาง </question>
<reference_answer> คู่สัญญาสามารถเรียกร้องได้โดยยื่นคำร้องต่อศาลเพื่อมีคำสั่งให้ผู้บังคับหลักประกันพ้นจากตำแหน่ง ตามมาตรา 76 ของพระราชบัญญัติหลักประกันทางธุรกิจ พ.ศ. 2558. </reference_answer>
<student_answer> คู่สัญญาสามารถเรียกร้องให้ศาลมีคำสั่งให้ผู้บังคับหลักประกันพ้นจากตำแหน่งได้ หากพบว่าผู้บังคับหลักประกันไม่เป็นกลางหรือมีพฤติกรรมที่ไม่เหมาะสมตามมาตรา 76 ของพระราชบัญญัติหลักประกันทางธุรกิจ พ.ศ. 2558 </student_answer>

<assistant> ```json
{"point_thought": "From the given reference answer, the student answer and question, both the student answer and the reference answers are making one point. I should put their points in a list",
 "student_point": ["คู่สัญญาสามารถเรียกร้องได้โดยยื่นคำร้องต่อศาลเพื่อมีคำสั่งให้ผู้บังคับหลักประกันพ้นจากตำแหน่ง ตามมาตรา 76 ของพระราชบัญญัติหลักประกันทางธุรกิจ พ.ศ. 2558. "],
 "reference_point": ["คู่สัญญาสามารถเรียกร้องให้ศาลมีคำสั่งให้ผู้บังคับหลักประกันพ้นจากตำแหน่งได้ หากพบว่าผู้บังคับหลักประกันไม่เป็นกลางหรือมีพฤติกรรมที่ไม่เหมาะสมตามมาตรา 76 ของพระราชบัญญัติหลักประกันทางธุรกิจ พ.ศ. 2558 "],
"coverage": {"thought": "From the extracted points, the reference answer is only making one point which is that the contracting party can request the court to issue an order to remove the security enforcer from their position if it is found that the security enforcer is not impartial or has inappropriate behavior. The student answer is also making only one point and it is also that the contracting party can request the court to issue an order to remove the security enforcer from their position. Therefore, the student answer fully cover the reference answer and does not contain any irrelevant information. The score would be 100", "score": "full-coverage"}, "contradiction": {"thought": "From the extracted_points, there is no contradicting point that the student make and the answer does not contradict with itself at all. Therefore, the contradiction score would be 0", "score": "no-contradiction"}}
```



\end{Verbatim}

\textbf{Turn 2: }
\begin{Verbatim}[breaklines, breaksymbol=, breakanywheresymbolpre=]

<user> <question> การจดทะเบียนสมาคมธุรกิจหลักทรัพย์มีเงื่อนไขอะไรบ้าง </question>
<reference_answer> 1. สมาคมต้องมีข้อบังคับที่อย่างน้อยประกอบด้วยข้อความ 7 ข้อ ดังนี้
   - ชื่อ
   - วัตถุที่ประสงค์
   - ที่ตั้งสำนักงาน
   - วิธีรับสมาชิก สิทธิและหน้าที่ของสมาชิก
   - วินัยและการลงโทษสมาชิก
   - การดำเนินกิจการของสมาคมที่เกี่ยวเนื่องกับธุรกิจหลักทรัพย์ การตั้ง การออกจากตำแหน่ง และการประชุมของกรรมการ ตลอดจนการประชุมใหญ่
   - ข้อบังคับอื่นตามที่คณะกรรมการ ก.ล.ต. ประกาศกำหนด

2. ข้อบังคับของสมาคมต้องนำไปจดทะเบียนต่อสำนักงานพร้อมกับการยื่นคำขออนุญาตจัดตั้งสมาคมที่เกี่ยวเนื่องกับธุรกิจหลักทรัพย์
3. บริษัทหลักทรัพย์ผู้เริ่มก่อการจัดตั้งที่มีจำนวนไม่น้อยกว่าสิบห้ารายต้องยื่นคำขอต่อสำนักงานตามหลักเกณฑ์และวิธีการที่คณะกรรมการ ก.ล.ต. ประกาศกำหนด
4. สำนักงานมีอำนาจออกคำสั่งเป็นหนังสือเรียกบุคคลใด ๆ มาสอบถามหรือให้ส่งเอกสารมาเพื่อประกอบการพิจารณาเกี่ยวกับการขออนุญาตจัดตั้งสมาคมที่เกี่ยวเนื่องกับธุรกิจหลักทรัพย์ได้
5. การจัดตั้งสมาคมที่เกี่ยวเนื่องกับธุรกิจหลักทรัพย์ต้องได้รับใบอนุญาตและจดทะเบียนกับสำนักงาน. </reference_answer>
<student_answer> การจดทะเบียนสมาคมธุรกิจหลักทรัพย์มีเงื่อนไขดังนี้: 1. ต้องได้รับใบอนุญาตและจดทะเบียนกับสำนักงาน (มาตรา 231) 2. ต้องมีบริษัทหลักทรัพย์ผู้เริ่มก่อการจัดตั้งไม่น้อยกว่ายี่สิบรายยื่นคำขอต่อสำนักงาน (มาตรา 232) 3. ข้อบังคับของสมาคมต้องมีข้อความตามที่กำหนดในมาตรา 234 </student_answer>

<assistant>```json
{"point_thought": "From the given reference answer, the student answer and question, the reference answer is making 5 points as listed while the student is making 3 points. I should put those points in the list",
 "student_point": ["1. สมาคมต้องมีข้อบังคับที่อย่างน้อยประกอบด้วยข้อความ 7 ข้อ ดังนี้ ชื่อ, วัตถุที่ประสงค์ ที่ตั้งสำนักงาน วิธีรับสมาชิก สิทธิและหน้าที่ของสมาชิก วินัยและการลงโทษสมาชิก การดำเนินกิจการของสมาคมที่เกี่ยวเนื่องกับธุรกิจหลักทรัพย์ การตั้ง การออกจากตำแหน่ง และการประชุมของกรรมการ ตลอดจนการประชุมใหญ่ ข้อบังคับอื่นตามที่คณะกรรมการ ก.ล.ต. ประกาศกำหนด", "2. ข้อบังคับของสมาคมต้องนำไปจดทะเบียนต่อสำนักงานพร้อมกับการยื่นคำขออนุญาตจัดตั้งสมาคมที่เกี่ยวเนื่องกับธุรกิจหลักทรัพย์", "3. บริษัทหลักทรัพย์ผู้เริ่มก่อการจัดตั้งที่มีจำนวนไม่น้อยกว่าสิบห้ารายต้องยื่นคำขอต่อสำนักงานตามหลักเกณฑ์และวิธีการที่คณะกรรมการ ก.ล.ต. ประกาศกำหนด", "4. สำนักงานมีอำนาจออกคำสั่งเป็นหนังสือเรียกบุคคลใด ๆ มาสอบถามหรือให้ส่งเอกสารมาเพื่อประกอบการพิจารณาเกี่ยวกับการขออนุญาตจัดตั้งสมาคมที่เกี่ยวเนื่องกับธุรกิจหลักทรัพย์ได้", "5. การจัดตั้งสมาคมที่เกี่ยวเนื่องกับธุรกิจหลักทรัพย์ต้องได้รับใบอนุญาตและจดทะเบียนกับสำนักงาน."],
 "reference_point": ["การจดทะเบียนสมาคมธุรกิจหลักทรัพย์มีเงื่อนไขดังนี้: 1. ต้องได้รับใบอนุญาตและจดทะเบียนกับสำนักงาน (มาตรา 231)", "2. ต้องมีบริษัทหลักทรัพย์ผู้เริ่มก่อการจัดตั้งไม่น้อยกว่ายี่สิบรายยื่นคำขอต่อสำนักงาน (มาตรา 232)", "3. ข้อบังคับของสมาคมต้องมีข้อความตามที่กำหนดในมาตรา 234"]
 ,"coverage": {"thought": "As seen from the extracted points of student and reference answer, the reference answer is making the total of 5 points of requirements needed for registering securities business association. However, the student is only making three points which covers point 5 and 1 correctly. It also mentions point 3 of the reference answer but the amount of filing which should be 15 is stated as 20. It doesn't mention point 2 and 4 at all. Therefore, the coverage score would be partial-coverage", "score": "partial-coverage"}, "contradiction": {"thought": "As mention, the student answer mentions point 3 of the reference answer but the number of filing should be 15. Hence, the student answer directly contradict the reference answer. Therefore, the contradiction score would be contradiction ", "score": "contradiction"}}
```

\end{Verbatim}

\textbf{Turn 3: }
\begin{Verbatim}[breaklines, breaksymbol=, breakanywheresymbolpre=]
<user> <question> มีความผิดอะไรหากคนต่างชาติยังคงประกอบธุรกิจที่ถูกเพิกถอนใบอนุญาต </question>
<reference_answer> คนต่างด้าวจะต้องระวางโทษจำคุกไม่เกินสามปี หรือปรับจากหนึ่งแสนบาทถึงหนึ่งล้านบาท หรือทั้งจำทั้งปรับ และปรับอีกวันละหนึ่งหมื่นบาทตลอดเวลาที่ยังฝ่าฝืนอยู่. </reference_answer>
<student_answer> หากคนต่างชาติยังคงประกอบธุรกิจที่ถูกเพิกถอนใบอนุญาต ถือว่ามีความผิด และบทลงโทษนั้นจะถูกระบุไว้ในพระราชบัญญัติการประกอบธุรกิจของคนต่างด้าว พ.ศ. 2542 </student_answer>

<assistant>```json
{"point_thought": "From the given reference answer, the student answer and question, the reference answer is making 1 point about the penalty while the student makes two point about whether he is guilty and the penalty.",
 "student_point": ["คนต่างด้าวจะต้องระวางโทษจำคุกไม่เกินสามปี หรือปรับจากหนึ่งแสนบาทถึงหนึ่งล้านบาท หรือทั้งจำทั้งปรับ และปรับอีกวันละหนึ่งหมื่นบาทตลอดเวลาที่ยังฝ่าฝืนอยู่."],
 "reference_point": ["หากคนต่างชาติยังคงประกอบธุรกิจที่ถูกเพิกถอนใบอนุญาต ถือว่ามีความผิด", "บทลงโทษนั้นจะถูกระบุไว้ในพระราชบัญญัติการประกอบธุรกิจของคนต่างด้าว พ.ศ. 2542"],
 "coverage": {"thought": "From the extracted points, the reference answer is only making one point which says that this case can be penalized by jail time of up to 3 years or fine of 100,000 to 1,000,000 baht. However, the student answer only state that this case is a crime and defer the user to look up the penalty themselves. Therefore, it would get the coverage score of no-coverage.", "score": "no-coverage"}, "contradiction": {"thought": "From the given reference and student points, even though the student answer does not cover the same point that the reference answer is making. It does not directly contradict the reference answer either and therefore would get the score no-contradiction for contradiction", "score": "no-contradiction"}}
```


\end{Verbatim}

\textbf{Turn 4: }
\begin{Verbatim}[breaklines, breaksymbol=, breakanywheresymbolpre=]
<user> <question> ถ้ากรรมการลงรายการในเอกสารที่ยื่นนายทะเบียนไม่ตรงกับบัญชีของบริษัท กรรมการจะผิดหรือไม่ </question>
<reference_answer> กรรมการจะต้องรับผิดชอบทางอาญาและแพ่ง ตามพระราชบัญญัติหลักทรัพย์และตลาดหลักทรัพย์ พ.ศ. 2535 มาตรา 312 และอาจต้องระวางโทษจำคุกตั้งแต่ห้าปีถึงสิบปี และปรับตั้งแต่ห้าแสนบาทถึงหนึ่งล้านบาท </reference_answer>
<student_answer> กรรมการไม่มีความผิด </student_answer>

<assistant>```json
{"point_thought": "From the given reference answer, the student answer and question, the reference answer is making 2 points about whether the committee is guilty or not and the penalty which was not asked. The student is only making one point",
 "student_point": ["กรรมการจะต้องรับผิดชอบทางอาญาและแพ่ง ตามพระราชบัญญัติหลักทรัพย์และตลาดหลักทรัพย์ พ.ศ. 2535 มาตรา 312", "อาจต้องระวางโทษจำคุกตั้งแต่ห้าปีถึงสิบปี และปรับตั้งแต่ห้าแสนบาทถึงหนึ่งล้านบาท"],
 "reference_point": ["กรรมการไม่มีความผิด"],
 "coverage": {"thought": "From the reference answer and student points, the reference answer is only making two points about whether the the behavior is a crime or not and the penalty that the law state for this case while the student answer only says that the behavior is not a crime and therefore does not cover a single point of the reference answer. Therefore, the score would be no-coverage", "score": "no-coverage"}, "contradiction": {"thought": "The student points directly contradicts the reference point on the point of whether the behavior is a crime or not. Therefore, the score would be contradiction", "score": "contradiction"}}
```


\end{Verbatim}