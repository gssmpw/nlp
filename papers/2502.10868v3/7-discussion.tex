\section{Discussion}
% I will dump something here from results that I consider should be here first
% Full result of chunking here (Effect of chunk size, overlap, chunk strategy)
% What should be written here -> Start with chunk hyperparameters table -> E2E chunking analysis -> E2E referencer analysis -> E2E retriever analysis (ยาว อาจจะแบ่งได้) -> E2E llm analysis

\subsection{Analysis of Effect of Chunking Strategy on E2E and Retrieval Performance}
\label{subsec: chunk_error_analysis}

Building on the results in Table~\ref{table: chunk_e2e_main}, we further analyze the number of samples where each chunking strategy outperforms the other, as shown in Figure~\ref{fig:chunk_vary_won_lost_wcx}, where \texttt{local\_f1} represents the E2E F1 score.
%
The performance difference between the two strategies remains relatively balanced in terms of coverage score.
%
However, hierarchy-aware chunking achieves better E2E F1 scores, while naive chunking performs slightly better in terms of contradiction scores.

\begin{figure}[h!]
\centering

\includegraphics[width=0.6\textwidth]{images/chunk_vary_won_lost_wcx_reduced.png}  % Replace with your figure file
\caption{Bar chart showing amount of queries performing better under hierarchical-aware and naive chunking on NitiBench-CCL}
\label{fig:chunk_vary_won_lost_wcx}
\end{figure}

For further investigation, we analyze the subset of queries where the naive chunking method achieves a higher coverage score than hierarchy-aware chunking. 
%
Within this subset, we further examine queries where the MRR of naive chunking exceeds that of hierarchy-aware chunking. 
%
This specific focus allows us to identify cases where naive chunking enhances retriever performance and leads to better overall E2E performance despite our assumption that hierarchy-aware chunking inherently provides the most complete and noise-free information for each legislative section.

Upon analyzing this subset of queries, a recurring pattern emerges. 
%
\textbf{Thai legal QA system benefits from the naive chunking strategy when the target section alone does not provide sufficient context and requires context from the section before or after it.}  

For example, a query regarding the endorsement of a bill of exchange without a date requires the application of Section 933 of the Civil and Commercial Code, which discusses the acceptance of a bill of exchange. 
%
However, this section does not explicitly mention the term \enquote{bill of exchange}, as it assumes the reader's familiarity with the term from preceding sections.  

By chunking \enquote{imperfectly}, the naive chunking strategy includes Section 932— which explicitly references a bill of exchange— alongside Section 933.
%
This additional context improves the retriever model's ability to retrieve the correct chunk.

\begin{quote}
\textbf{Question: }If the drawer endorses a bill of exchange but does not include a date, what legal actions can be taken?

\textbf{Relevant Chunk: } 

   \textbf{Section 933 of Civil and Commercial Code:} If an acceptance does not include a date, the last day of the prescribed period for acceptance shall be deemed the date of acceptance.

\textbf{Chunk retrieved under naive chunking setting: }

   \textbf{Section 932 of Civil and Commercial Code:} It is prescribed that if the holder acts in good faith but records an incorrect date due to a material mistake, or if the date is entirely incorrect in any case, the bill of exchange shall not be invalidated solely for this reason, provided that it later comes into the possession of a lawful holder. The bill shall remain valid and enforceable as if the recorded date were correct.

   \textbf{Section 933 of Civil and Commercial Code:} If an acceptance does not include a date, the last day of the prescribed period for acceptance shall be deemed the date of acceptance.

   \textbf{Section 934 of Civil and Commercial Code}
\end{quote}



%% ตัดออกเพราะรู้สึกไม่ค่อยให้ info อะไรเท่าไร
% Figures~\ref{fig:augmenter_vary_recall_diff_tax} and~\ref{fig:augmenter_vary_recall_diff_wcx} present histograms illustrating the difference between retriever recall and referencer recall when the retriever recall is below 1.0. These plots aim to analyze the proportion of queries with potential for improvement that actually show recall gains when the referencer is applied. The results indicate a higher proportion of recall improvements in the Tax dataset compared to the WCX dataset.

% \begin{figure}[H]
% \centering
% \begin{minipage}{0.48\textwidth}
%     \centering
%     \includegraphics[width=\textwidth]{images/augmenter_vary_recall_diff_tax.png}  % Replace with your figure file
%     \caption{Histogram of difference of referencer recall and retriever recall on Tax Case Dataset}
%     \label{fig:augmenter_vary_recall_diff_tax}
% \end{minipage}
% \hfill
% \begin{minipage}{0.48\textwidth}
%     \centering
%     \includegraphics[width=\textwidth]{images/augmenter_vary_recall_diff_wcx.png}  % Replace with your figure file
%     \caption{Histogram of difference of referencer recall and retriever recall on WCX Dataset}
%     \label{fig:augmenter_vary_recall_diff_wcx}
% \end{minipage}
% \end{figure}

% ref depth vs retrieval perf
\subsection{Effect of NitiLink's Reference Depth on Retrieval Performance}
\label{subsec: ref_depth_vs_retrieval_perf}

\textbf{Adding more reference depth improves retrieval performance when the question requires extensive legal reasoning.} 
%
In Table~\ref{table: augmenter_e2e_main}, we observed that increasing the reference depth by one slightly improved retrieval performance for NitiBench-Tax, while NitiBench-CCL showed no notable difference. 
%
This suggests that for NitiBench-CCL, which does not require extensive legal reasoning, resolving law section references provides little benefit. 
%
However, it remains unclear how much retrieval performance—particularly recall—can be improved by further increasing the reference depth.

To investigate this, we examined the relationship between \textit{NitiLink's maximum reference depth depth}, \textit{retrieval performance gains} (\texttt{Mean Diff} on the y-axis), and \textit{the total number of law sections NitiLink resolves} (see Figures~\ref{fig: augmenter_vary_retrieval_diff_tax} and~\ref{fig: augmenter_vary_retrieval_diff_wcx}). 
%

For NitiBench-Tax, retrieval performance improves as reference depth increases, peaking at a depth of 6.
%
However, this comes at the cost of increased context length, reaching approximately 60 sections per query.
%
While the improvement in retrieval performance could be attributed to retrieving more law sections—thereby increasing the hit rate—after extensive recursive reference resolution in NitiBench-Tax, the results for NitiBench-CCL indicate that this is not always the case. 
%
For the NitiBench-CCL, retrieval gains remain minimal and plateau after a depth of 2 despite resolving up to 30 law sections at a depth of 9. 
%
We suspect this is due to NitiBench-CCL requiring only one relevant law per entry, eliminating the need for complex legal reasoning during retrieval.

Based on this analysis, we hypothesize that the performance gains from increasing reference depth stem from \textbf{recursively resolving law section references, which primarily benefits retrieval performance when the query requires extensive legal reasoning.}

% Due to budget constraints, we did not test the effects of increasing reference depth beyond our current limits. We leave this as an open question for future research.

\begin{figure}[!ht]
\centering
\includegraphics[width=0.7\textwidth]{images/augmenter_vary_retrieval_diff_tax.png}
\caption{Plot of the relationship between reference depth of NitiLink and retrieval performance and number of sections per query on NitiBench-Tax. On the left, \texttt{Mean Diff} shows the average retrieval metric difference when increasing law section depth compared to retrieval performance without NitiLink. The right plot shows the number of sections cited when resolving more reference depth.}
\label{fig: augmenter_vary_retrieval_diff_tax}
\end{figure}

\begin{figure}[!ht]
\centering
\includegraphics[width=0.7\textwidth]{images/augmenter_vary_retrieval_diff_wcx.png}
\caption{Plot of the relationship between reference depth of NitiLink and retrieval performance and number of sections per query on NitiBench-CCL.}
\label{fig: augmenter_vary_retrieval_diff_wcx}
\end{figure}

%
\subsection{On Generated Answer Performance When Using NitiLink}
% \label{subsec: referencer_error_analysis}
\label{subsec: generated_ans_perf_with_ref}

\textbf{An increase in retrieval performance from using NitiLink does not always translate to a better generation score.} 
%
Based on the insights from \S\ref{subsec: ref_depth_vs_retrieval_perf}, we hypothesized that NitiLink could boost retrieval performance—especially in NitiBench-Tax, where extensive legal reasoning is required. 
%
To further investigate the effect of adding NitiLink on end-to-end (E2E) metrics (e.g., coverage, contradiction), we analyzed the differences in E2E metrics with and without NitiLink on subsets of queries for which NitiLink improved recall. 
%
This approach isolates our analysis to focus solely on the E2E performance of queries where NitiLink successfully identified the correct law section (unlike the analysis in \S\ref{subsubsec: referencer_result}, which includes all queries). 
%
The results are presented in Figures~\ref{fig:augmenter_vary_e2e_gain}. 
%
The local\_recall, local\_precision, and local\_f1 refer to E2E recall, precision, and f1, respectively.

These plots confirm the trend mentioned in \S\ref{subsubsec: referencer_result}: improved retrieval performance in NitiBench-Tax does not consistently translate to better end-to-end performance due to query complexity and increased context length. 
%
In contrast, retrieval improvements in NitiBench-CCL significantly enhance end-to-end results. 
%
From these findings, we hypothesize that \textbf{improvements in retrieval recall do not always translate to better E2E performance and may even degrade LLM generation performance. 
%
We suspect that this degradation is due to two reasons: 
%
1) NitiLink does not contribute to the LLM reasoning capability necessary for processing queries such as in NitiBench-Tax. 
%
2) The longer context lengths can have adverse effects on the LLM performance.}

\begin{figure}[!ht]
    \centering
    \begin{subfigure}[b]{0.48\textwidth}
        \centering
        \includegraphics[width=\textwidth]{images/augmenter_vary_e2e_gain_tax.png}
        \label{fig:augmenter_vary_e2e_gain_tax}
    \end{subfigure}
    \hfill
    \begin{subfigure}[b]{0.48\textwidth}
        \centering
        \includegraphics[width=\textwidth]{images/augmenter_vary_e2e_gain_wcx.png}
        \label{fig:augmenter_vary_e2e_gain_wcx}
    \end{subfigure}
    \caption{Difference in E2E metrics on NitiBench-Tax (left) and on NitiBench-CCL (right). The Y-axis of the plot denotes the difference in E2E metrics with NitiLink compared to those without using NitiLink. A positive value means that NitiLink improves that specific metric (specified in x-axis) and negative showed performance degradation. \texttt{local} metric represents the citation score obtained from LLM citation in the generated output.}
    \label{fig:augmenter_vary_e2e_gain}
\end{figure}


% \begin{figure}[!ht]
% \centering
% \begin{minipage}{0.48\textwidth}
%     \centering
%     \includegraphics[width=\textwidth]{images/augmenter_vary_e2e_gain_tax.png}  % Replace with your figure file
%     \caption{Difference between End-to-End performance on subset that has improved recall on Tax dataset}
%     \label{fig:augmenter_vary_e2e_gain_tax}
% \end{minipage}
% \hfill
% \begin{minipage}{0.48\textwidth}
%     \centering
%     \includegraphics[width=\textwidth]{images/augmenter_vary_e2e_gain_wcx.png}  % Replace with your figure file
%     \caption{Difference between End-to-End performance on subset that has improved recall on WCX dataset}
%     \label{fig:augmenter_vary_e2e_gain_wcx}
% \end{minipage}
% \end{figure}


%%%%%%

\subsection{Findings from Varying Retrieval Model on Legal Question Answering}
\label{subsec: retriever_e2e_error_analysis}

% 1. Fix LLM/chunk, vary retrieval (tab 20)
% 2. correlation between retrieval vs e2e metrics (fig 12): positive correlation between retriever and e2e
% 3. # of times model best in retrieval score(fig 13,14) -> further analyze to tab21: 
% 4. error analysis on (3)

We measure E2E metrics based on 4 retrievers setups where each retrieval model represents distinct types of embeddings: 
%
three-head embeddings (\textit{Human-Finetuned BGE-M3}), 
%
dense embeddings (\textit{JINA V3}), 
%
sparse embeddings (\textit{BM25}), and 
%
API-based embeddings for retrieval (\textit{Cohere}). 
%
We fix our chunking strategy to use a hierarchy-aware chunking method (see \S~\ref{subsubsec: chunk_setup}) without NitiLink and use \texttt{gpt-4o-2024-08-06} as a LLM for text generation. 
%
The results are summarized in Table~\ref{table: retrieval_e2e_main}.

\begin{table}[!ht]
\centering
\renewcommand{\arraystretch}{1.2} % This increases the cell height by 1.5 times
\resizebox{\textwidth}{!}{%
\begin{tabular}{@{}lcccccccc@{}}
\toprule
\textbf{Setting} & \textbf{Retriever MRR ($\uparrow$)} & \multicolumn{1}{l}{\textbf{Retriever Multi MRR ($\uparrow$)}} & \textbf{Retriever Recall ($\uparrow$)} & \textbf{Coverage ($\uparrow$)} & \textbf{Contradiction ($\downarrow$)} & \textbf{E2E Recall ($\uparrow$)} & \textbf{E2E Precision ($\uparrow$)} & \textbf{E2E F1 ($\uparrow$)} \\ \midrule

\multicolumn{9}{c}{\textbf{NitiBench-CCL}} \\ \midrule
BM25                   & 0.556                & {\color[HTML]{333333} 0.556} & 0.716            & 73.3          & 0.123          & 0.687          & 0.571          & 0.624          \\
JINA V3                & 0.689                & {\color[HTML]{333333} 0.689} & 0.875            & 82.1          & 0.065          & 0.827          & 0.592          & 0.690          \\
Cohere                 & \underline{\textbf{0.760}} & {\color[HTML]{333333} \underline{0.760}}  & \underline{0.912}      & \underline{84.4}    & \underline{0.060}    & \underline{0.857}    & \underline{0.597}    & \underline{0.704}    \\
Human-Finetuned BGE-M3 & \textbf{0.809}       & \textbf{0.809}         & \textbf{0.938}   & \textbf{85.2} & \textbf{0.055} & \textbf{0.880} & \textbf{0.601} & \textbf{0.714} \\

\midrule \multicolumn{9}{c}{\textbf{NitiBench-Tax}} \\ \midrule
BM25                   & 0.327                & {\color[HTML]{333333} 0.183} & 0.320            & 36.0          & 0.600          & 0.271          & 0.582          & 0.370          \\
JINA V3                & \underline{0.524}     & {\color[HTML]{333333} \underline{0.311}}  & \textbf{0.549}   & \underline{47.0}    & \underline{0.520}    & \underline{0.340}    & \textbf{0.690}   & \underline{0.456}    \\
Cohere                 & 0.454                & {\color[HTML]{333333} 0.263} & 0.414            & \underline{47.0}    & \underline{0.520}    & \textbf{0.361} & 0.642          & \textbf{0.462} \\
Human-Finetuned BGE-M3 & \textbf{0.574}       & {\color[HTML]{333333} \textbf{0.333}} & \underline{0.499}      & \textbf{50.0} & \textbf{0.460} & 0.333          & \underline{0.640}    & 0.438          \\  \bottomrule
\end{tabular}%
}
\caption{Effect of retrieval configuration on E2E performance. We use \texttt{gpt-4o-2024-08-06} as an LLM and vary different retrieval models to see the difference in E2E metrics.}
\label{table: retrieval_e2e_main}
\end{table}


The results showed that the \textit{Human-Finetuned BGE-M3} performed best overall, achieving results comparable to JinaAI V3 on NitiBench-Tax and Cohere embeddings on NitiBench-CCL. 
%
This result is consistent with the retrieval result in Table~\ref{table: retrieval_wangchan} and~\ref{table: retrieval_tax}. 
%
Building upon these results, we further analyze several findings based on varying retrieval models.

%
\textbf{Multi-MRR do positively correlates with E2E metrics.} Figure~\ref{fig: retriever_vary_corr} visualizes the relationship between Multi-MRR (a retriever metric) and five end-to-end metrics: coverage, contradiction, E2E recall (local\_recall in the figure), E2E precision (local\_precision in the figure), and E2E F1 (local\_f1 in the figure). 
%
The figure reveals a strong positive correlation between Multi-MRR and all end-to-end metrics except for contradiction, which shows a negative correlation. We also found that the Multi-MRR correlates to the E2E metrics better than the single-label counterpart.
%
This is expected, as better retrieval leads to better overall system performance. 
%
These findings suggested that \textbf{the RAG framework can perform best when the relevant law sections are ranked at the top of the retrieved sections.}

\begin{figure}[!ht]
\centering
\includegraphics[width=0.8\textwidth]{images/retriever_vary_corr.png}
\caption{Scatter plot of retrieval metrics vs E2E metrics}
\label{fig: retriever_vary_corr}
\end{figure}

%
\textbf{There remains a gap in overall retriever performance, indicating room for improvement across all retriever models.} 
%
To identify the current limitation of the retriever models, we analyzed samples where each retriever achieved the highest score for each end-to-end (E2E) metric.
%
Figures~\ref{fig:retriever_vary_win_tax} and~\ref{fig:retriever_vary_win_wcx} show the frequency of how each retrieval model achieves the highest score for each E2E metric. 
%
Note that we omit samples where all retrievers are tied at a maximum score, as the count for the tie column would be too high.
% Figures~\ref{fig:retriever_vary_win_tax} and~\ref{fig:retriever_vary_win_wcx} exclude samples where each model yields the maximum score for every metric, as we aim to focus on the performance gap between the retrievers. 
%
These figures demonstrate that the human-finetuned BGE-M3 model (\textbf{wcx} in the figures) most frequently achieved the highest E2E metric scores. 
%
Notably, a portion of both datasets showed identical E2E metric scores across all four retriever configurations.
%
Apart from the winning retrieval model for each E2E score, we are specifically interested in the case where all retrieval models are tied (while not getting a perfect score).
%
We suspect that the identical E2E metric scores result from the general performance gap in retriever models for certain types of queries, as well as the inherent structure of Thai legislation. 
%
This will be further discussed in more detail in \S~\ref{subsec: retriever_re_error_analysis_wcx} and~\ref{subsec: retriever_re_error_analysis_tax}. 

\begin{figure}[!ht]
\centering
\begin{minipage}{0.48\textwidth}
    \centering
    \includegraphics[width=\textwidth]{images/retriever_vary_win_tax.png}  % Replace with your figure file
    \caption{Bar chart showing amount of times each retriever achieves the highest score on a sample on NitiBench-Tax}
    \label{fig:retriever_vary_win_tax}
\end{minipage}
\hfill
\begin{minipage}{0.48\textwidth}
    \centering
    \includegraphics[width=\textwidth]{images/retriever_vary_win_wcx.png}  % Replace with your figure file
    \caption{Bar chart showing amount of times each retriever achieves the highest score on a sample on NitiBench-CCL}
    \label{fig:retriever_vary_win_wcx}
\end{minipage}
\end{figure}


%%%%%%backup%%%%%%
% Apart from evaluating the effect of retriever models on retrieval performance in Section~\ref{subsubsec: retriever_result}, we also analyze the E2E impact of different retriever configurations: Human-rerank Finetuned BGE-M3, JinaAI Embeddings V3, BM25, and Cohere embeddings. These four configurations were selected to represent four distinct types of embeddings: three-head embeddings, dense embeddings, sparse embeddings, and API-based embeddings for retrieval. We keep other components constant, using section-based chunking, no referencer, and gpt-4o-2024-08-06 for the generation model. The results are summarized in Table~\ref{table: retrieval_e2e_main}.

% \begin{table}[ht]
% \centering
% \caption{Effect of retrieval configuration on E2E performance}
% \renewcommand{\arraystretch}{1.5} % This increases the cell height by 1.5 times
% \label{table: retrieval_e2e_main}
% \resizebox{\textwidth}{!}{%
% \begin{tabular}{@{}c|ccccccccc@{}}
% \toprule
% Dataset               & Setting                & Retriever MRR        & \multicolumn{1}{l}{Retriever Multi MRR} & Retriever Recall & Coverage      & Contradiction  & E2E Recall     & E2E Precision  & E2E F1         \\ \midrule
%                       & BM25                   & 0.327                & {\color[HTML]{333333} 0.183}            & 0.320            & 36.0          & 0.600          & 0.271          & 0.582          & 0.370          \\
%                       & JINA V3                & \underline{0.524}          & {\color[HTML]{333333} \underline{0.311}}      & \textbf{0.549}   & \underline{47.0}    & \underline{0.520}    & \underline{0.340}    & \textbf{0.690} & \underline{0.456}    \\
%                       & Cohere                 & 0.454                & {\color[HTML]{333333} 0.263}            & 0.414            & \underline{47.0}    & \underline{0.520}    & \textbf{0.361} & 0.642          & \textbf{0.462} \\
% \multirow{-4}{*}{Tax} & Human-Finetuned BGE-M3 & \textbf{0.574}       & {\color[HTML]{333333} \textbf{0.333} }           & \underline{0.499}      & \textbf{50.0} & \textbf{0.460} & 0.333          & \underline{0.640}    & 0.438          \\ \midrule
%                       & BM25                   & 0.556                & {\color[HTML]{333333} 0.556}            & 0.716            & 73.3          & 0.123          & 0.687          & 0.571          & 0.624          \\
%                       & JINA V3                & 0.689                & {\color[HTML]{333333} 0.689}            & 0.875            & 82.1          & 0.065          & 0.827          & 0.592          & 0.690          \\
%                       & Cohere                 & \underline{\textbf{0.760}} & {\color[HTML]{333333} \underline{0.760}}      & \underline{0.912}      & \underline{84.4}    & \underline{0.060}    & \underline{0.857}    & \underline{0.597}    & \underline{0.704}    \\
% \multirow{-4}{*}{WCX} & Human-Finetuned BGE-M3 & \textbf{0.809}       & \textbf{0.809}                                   & \textbf{0.938}   & \textbf{85.2} & \textbf{0.055} & \textbf{0.880} & \textbf{0.601} & \textbf{0.714} \\ \bottomrule
% \end{tabular}%
% }
% \end{table}

% The results in Table~\ref{table: retrieval_e2e_main} show that the human-fine-tuned BGE-M3 model performs best overall, achieving comparable results to JinaAI V3 on the Tax Case dataset and Cohere embeddings on the WCX dataset. This result is consistent with the retrieval result in Table~\ref{table: retrieval_wangchan} and~\ref{table: retrieval_tax}. Figure~\ref{fig: retriever_vary_corr} visualizes the relationship between Multi-MRR (a retriever metric) and five end-to-end metrics: coverage, contradiction, E2E recall, E2E precision, and E2E F1. The figure reveals a strong positive correlation between Multi-MRR and all end-to-end metrics except for contradiction, which shows a negative correlation. This is expected, as better retrieval generally leads to better overall system performance.

% \begin{figure}[ht]
% \centering
% \includegraphics[width=\textwidth]{images/retriever_vary_corr.png}
% \caption{Scatter plot of retrieval metrics vs E2E metrics}
% \label{fig: retriever_vary_corr}
% \end{figure}

% We then analyzed the number of samples where each retriever achieved the highest score for each end-to-end (E2E) metric. Figures~\ref{fig:retriever_vary_win_tax} and~\ref{fig:retriever_vary_win_wcx} exclude samples where each model yields the maximum score for every metric, as we aim to focus on the performance gap between the retrievers. These figures demonstrate that the human-fine-tuned BGE-M3 model (\textbf{wcx} in the figures) most frequently achieved the highest E2E metric scores. However, a portion of both datasets showed identical E2E metric scores across all four retriever configurations. To investigate these queries further, we compared their retrieval recall to the full dataset's retrieval recall, as detailed in Table~\ref{table: retrieval_e2e_subset}.

% \begin{figure}[H]
% \centering
% \begin{minipage}{0.48\textwidth}
%     \centering
%     \includegraphics[width=\textwidth]{images/retriever_vary_win_tax.png}  % Replace with your figure file
%     \caption{Bar chart showing amount of times each retriever achieves highest score on a sample on Tax Dataset}
%     \label{fig:retriever_vary_win_tax}
% \end{minipage}
% \hfill
% \begin{minipage}{0.48\textwidth}
%     \centering
%     \includegraphics[width=\textwidth]{images/retriever_vary_win_wcx.png}  % Replace with your figure file
%     \caption{Bar chart showing amount of times each retriever achieves highest score on a sample on WCX Dataset}
%     \label{fig:retriever_vary_win_wcx}
% \end{minipage}
% \end{figure}

% \begin{table}[ht]
% \centering
% \caption{Retriever recall on subset of datasets that have the same coverage score across 4 settings compared to the full dataset}
% \renewcommand{\arraystretch}{1.5} % This increases the cell height by 1.5 times
% \label{table: retrieval_e2e_subset}
% \begin{tabular}{@{}cccc@{}}
% \toprule
% Dataset              & Setting                & Retriever Recall on Tie samples & Retriever Recall on All samples \\ \midrule
% \multirow{4}{*}{Tax} & BM25                   & 0.333                           & 0.320                           \\
%                      & JINA V3                & 0.524                           & 0.549                           \\
%                      & Cohere                 & 0.411                           & 0.414                           \\
%                      & Human-Finetuned BGE-M3 & 0.399                           & 0.499                           \\ \midrule
% \multirow{4}{*}{WCX} & BM25                   & 0.522                           & 0.716                           \\
%                      & JINA V3                & 0.659                           & 0.875                           \\
%                      & Cohere                 & 0.691                           & 0.912                           \\
%                      & Human-Finetuned BGE-M3 & 0.731                           & 0.938                           \\ \bottomrule
% \end{tabular}

% \end{table}

% Table~\ref{table: retrieval_e2e_subset} reveals distinct error patterns in the Tax and WCX datasets. In the Tax dataset, the retrieval recall for the subset of queries with identical E2E scores across all retrievers is similar to the full dataset's recall. This suggests that the bottleneck for these queries lies in the fact that the LLM struggles to use the retrieved documents effectively for legal reasoning and generating accurate answers on the subset. This limitation arises because the dataset's queries are complex and often require reasoning steps not explicitly stated in the retrieved documents.

% For example, consider a query about whether a private fund managed by an asset management company on behalf of an individual investing in debt instruments is subject to a specific business tax on the interest earned. The relevant law (Section 91/2 of the Revenue Code) states that commercial bank-like operations (e.g., lending, guarantees, dealing with promissory notes) are subject to this tax. However, the law doesn't explicitly mention interest from promissory notes. The system needs to reason that such interest also falls under commercial bank-like operations and is therefore taxable. The RAG system, while able to retrieve the correct section, fails to perform this additional reasoning and incorrectly concludes that the interest is not taxable.

% % \begin{quote}
% %     \begin{thai}
% %     \textbf{Question: } นาง ก. ประกอบธุรกิจลงทุนในตราสารหนี้ ได้ทำสัญญาจัดการกองทุนส่วนบุคคลแต่งตั้งให้บริษัทหลักทรัพย์จัดการกองทุน (บลจ.) เป็นผู้รับมอบอำนาจและเป็นผู้บริหารกองทุนส่วนบุคคล “นาง ก. โดย บลจ.” ซึ่ง บลจ. ได้ลงทุนด้วยการซื้อตั๋วแลกเงินจากบริษัทผู้ออกตั๋วและระบุชื่อ “นาง ก. โดย บลจ.” เป็นผู้ทรงตั๋วเงิน … ดอกเบี้ยที่ได้รับจากการลงทุนในตั๋วแลกเงินดังกล่าวต้องเสียภาษีธุรกิจเฉพาะหรือไม่
    
% %     \textbf{Relevant Laws: }
    
% %     ประมวลรัษฎากร มาตรา 91/2 ภายใต้บังคับมาตรา 91/4 การประกอบกิจการดังต่อไปนี้ในราชอาณาจักร ให้อยู่ในบังคับต้องเสียภาษีธุรกิจเฉพาะตามบทบัญญัติในหมวดนี้
    
% % (1) การธนาคาร ตามกฎหมายว่าด้วยการธนาคารพาณิชย์หรือกฎหมายเฉพาะ

% % (2) การประกอบธุรกิจเงินทุน ธุรกิจหลักทรัพย์ ธุรกิจเครดิตฟองซิเอร์ตามกฎหมายว่าด้วยการประกอบธุรกิจเงินทุน ธุรกิจหลักทรัพย์ และธุรกิจเครดิตฟองซิเอร์

% % (3)การรับประกันชีวิตตามกฎหมายว่าด้วยการประกันชีวิต

% % (4) การรับจำนำตามกฎหมายว่าด้วยโรงรับจำนำ

% % \textbf{(5) การประกอบกิจการโดยปกติเยี่ยงธนาคารพาณิชย์ เช่น การให้กู้ยืมเงินค้ำประกัน แลกเปลี่ยนเงินตรา ออก ซื้อ หรือขายตั๋วเงิน หรือรับส่งเงินไปต่างประเทศด้วยวิธีต่าง ๆ}

% % (6) การขายอสังหาริมทรัพย์เป็นทางค้าหรือหากำไร ไม่ว่าอสังหาริมทรัพย์นั้นจะได้มาโดยวิธีใดก็ตาม ทั้งนี้ เฉพาะที่เป็นไปตามหลักเกณฑ์ วิธีการ และเงื่อนไขตามที่กำหนดโดยพระราชกฤษฎีกา…
% %     \end{thai}
% % \end{quote}
% \begin{quote}
%     \textbf{Question: } Ms. K, who runs a business investing in debt securities, has entered into an agreement to manage a personal fund and appointed a fund management company (Fund Co.) as the authorized agent and fund manager for the personal fund "Ms. K by Fund Co.". The Fund Co. invested by purchasing promissory notes from the issuing company and designated "Ms. K by Fund Co." as the holder of the promissory notes. Is the interest received from this investment in promissory notes subject to business tax?
    
%     \textbf{Relevant Laws: }
    
%     Section 91/2 of Revenue Code, under Section 91/4, the following business activities in the Kingdom are subject to business tax under the provisions of this chapter:
    
%     (1) Banking activities under the law governing commercial banking or specific laws.
    
%     (2) Financial business, securities business, credit business under the law governing the operation of financial businesses, securities businesses, and credit businesses.
    
%     (3) Life insurance under the law governing life insurance.
    
%     (4) Pledge activities under the law governing pawnshops.
    
%     \textbf{(5) Business activities typically performed by commercial banks, such as lending, collateralized loans, foreign currency exchange, issuing, purchasing, or selling promissory notes, or sending money abroad by various methods.}
    
%     (6) The sale of real estate for trade or profit, regardless of how the real estate is acquired, but only according to the criteria, methods, and conditions specified by the Royal Decree…
% \end{quote}

% In contrast, WCX dataset shows a significant difference in retrieval recall between the subset and the full dataset, indicating deficient overall retriever performance on this subset. This likely stems from issues similar to the four error categories discussed in Section~\ref{subsec: retriever_re_error_analysis}. Additionally, within this subset, we observe cases where successful retrieval is followed by inaccurate answer generation.

% A pattern occurs when observing these queries. We observe that the generator often struggles to summarize long, multi-conditional sections concisely while addressing all conditions accurately. This often leads to incomplete or incorrectly merged conditions, lowering the coverage score from 100. The prompt's emphasis on conciseness may exacerbate this issue. For example, a query about revoking a declaration of disappearance requires retrieving Section 63 of the Civil and Commercial Code, which outlines two conditions for such revocation. All four RAG configurations retrieve this section correctly. However, the generator, in attempting to summarize, often omits one condition. Prompting the LLM to determine whether direct quotation is appropriate might mitigate this issue.

% \begin{quote}
%     \begin{thai}
%     \textbf{Question: } In what circumstances can a court revoke a declaration of disappearance?

%     \textbf{Relevant Laws: }
%     Section 63 of Civil and Commercial Code

%     When a person who has been declared missing by the court, or a party with a vested interest, or a public prosecutor petitions the court and proves that the person is still alive or that they died at a time different from what was stated in Section 62, the court may revoke the declaration of disappearance. However, this revocation does not affect the validity of actions taken in good faith between the time the court issued the declaration of disappearance and the time of revocation.
%     A person who has acquired property as a result of the court's declaration of disappearance but loses their rights because the court revokes the declaration, is subject to the provisions regarding unjust enrichment in this Code by analogy.

%     \textbf{RAG Answer: }

%     A court can revoke a declaration of disappearance if it is proven that the person is still alive or that they died at a different time than stated in the declaration of disappearance.
%     \end{thai}
% \end{quote}

%%%%%%backup%%%%%%

%%%%%%

\subsection{Error Analysis on Retriever Performances on NitiBench-CCL}
\label{subsec: retriever_re_error_analysis_wcx}

Given the retriever results of the hierarchy-aware chunking strategy (showed in \S\ref{subsubsec: retriever_result}), we aim to further conduct an error analysis to answer the following questions: \textbf{What type of queries caused the retriever to fail, and what is the current limitation of dense retriever?}. 
%
To address these questions, first, we grouped the retrieval model performance of \textit{Human-Finetuned BGE-M3} (from Table~\ref{table: retrieval_e2e_main}) across all legislation on NitiBench-CCL and NitiBench-Tax respectively. 
%
The \enquote{Support} column indicates the number of queries for which a section within the given law code is relevant, while \enquote{Section Count} represents the total number of sections within that law code. 
%
The results are presented in Table~\ref{table: retrieval_wangchan_by_code} and~\ref{table: retrieval_tax_by_code}.

\begin{table}[!ht]
\centering

\resizebox{\textwidth}{!}{%
\renewcommand{\arraystretch}{1.4} % This increases the cell height by 1.5 times

\begin{tabular}{@{}p{0.6\textwidth}cccc@{}}
\toprule
\textbf{Law Code}                                                                                                                              & \multicolumn{1}{c}{\textbf{Multi MRR@10}} & \multicolumn{1}{c}{\textbf{Recall@10}} & \multicolumn{1}{c}{\textbf{Support}} & \multicolumn{1}{c}{\textbf{Section Count}} \\ \midrule
Business Registration Act, B.E. 2499                                                                                                  & \textbf{0.955}                   & \textbf{1.00}                 & 11                          & 22                                \\
Foreign Business Act, B.E. 2542                                                                                                       & \underline{0.900}                      & \textbf{1.00}                 & 36                          & 46                                \\
Trade Association Act, B.E. 2509                                                                                                      & 0.883                            & 0.925                         & 40                          & 56                                \\
Accounting Profession Act, B.E. 2547                                                                                                  & 0.862                            & 0.922                         & 64                          & 78                                \\
Secured Transactions Act, B.E. 2558                                                                                                   & 0.848                            & 0.932                         & 74                          & 91                                \\
Revenue Code                                                                                                                          & 0.836                            & 0.911                         & 484                         & 361                               \\
Provident Fund Act, B.E. 2530                                                                                                         & 0.834                            & \underline{0.972}                   & 36                          & 51                                \\
Emergency Decree on Digital Asset Businesses, B.E. 2561                                                                               & 0.827                            & 0.962                         & 78                          & 100                               \\
Public Limited Companies Act, B.E. 2535                                                                                               & 0.825                            & 0.919                         & 186                         & 231                               \\
Act on Offenses Relating to Registered Partnerships, Limited Partnerships, Companies Limited, Associations and Foundations, B.E. 2499 & 0.819                            & 0.958                         & 48                          & 76                                \\
Financial Institutions Business Act, B.E. 2551                                                                                        & 0.813                            & 0.897                         & 165                         & 168                               \\
Derivatives Act, B.E. 2546                                                                                                            & 0.813                            & 0.927                         & 109                         & 156                               \\
Chamber of Commerce Act, B.E. 2509                                                                                                    & 0.811                            & 0.864                         & 22                          & 62                                \\
Civil and Commercial Code                                                                                                             & 0.807                            & 0.909                         & 1618                        & 1842                              \\
Trust for Transactions in Capital Market Act, B.E. 2550                                                                               & 0.805                            & 0.890                         & 73                          & 94                                \\
Emergency Decree on Special Purpose Juristic Person for Securitization, B.E. 2540                                                     & 0.798                            & 0.879                         & 33                          & 46                                \\
Securities and Exchange Act, B.E. 2535                                                                                                & 0.786                            & 0.898                         & 294                         & 499                               \\
Energy Conservation Promotion Act, B.E. 2535                                                                                          & 0.778                            & 0.948                         & 58                          & 68                                \\
Accounting Act, B.E. 2543                                                                                                             & 0.728                            & 0.895                         & 38                          & 45                                \\
Energy Industry Act, B.E. 2550                                                                                                        & 0.726                            & 0.843                         & 153                         & 155                               \\
Petroleum Income Tax Act, B.E. 2514                                                                                                   & 0.616                            & 0.764                         & 110                         & 115                               \\ \bottomrule

\end{tabular}
}
\caption{Retrieval Evaluation Result by law code on NitiBench-CCL}
\label{table: retrieval_wangchan_by_code}
\end{table}

\begin{table}[!ht]
\centering

\renewcommand{\arraystretch}{1.2} % This increases the cell height by 1.5 times
\small
\begin{tabular}{@{}lcccc@{}}
\toprule
\textbf{Law Code}                               & \textbf{Multi MRR@10} & \textbf{Recall@10} & \textbf{Support} & \textbf{Section Count} \\ \midrule
Revenue Code                           & 0.333        & 0.499     & 50      & 361           \\
Accounting Act, B.E. 2543              & 0.333        & 0.5       & 1       & 45            \\
Civil and Commercial Code              & 0.173        & 0.34      & 8       & 1842          \\
Securities and Exchange Act, B.E. 2535 & 0.065        & 0.333     & 1       & 499           \\ \bottomrule
\end{tabular}
\caption{Retrieval Evaluation Result by law code on NitiBench-Tax}
\label{table: retrieval_tax_by_code}
\end{table}

Focusing on NitiBench-CCL, we analyze a subset of approximately 780 queries (out of 4000) that showed no improvement in MRR@10 after fine-tuning with in-domain data. 
%
Within this subset, 540 queries yielded identical MRR@10 scores for both the base and human-finetuned BGE-M3 models. 
%
A closer examination of these queries reveals that most errors fall into four distinct categories:
% new line for page formatting
% \\
% \\
\begin{itemize}
    \item Hidden Hierarchical Information
    \item Nested Structure
    \item Missable Details
    \item Complex Queries
    \item Incorrect Legislation Retrieval
\end{itemize}

\subsubsection{Hidden Hierarchical Information}
This error type arises from the redundancy and repetition within Thai Financial Law. 
%
Multiple sections often convey identical information, differing only in their location within the legal hierarchy or the specific law code. 
%
For example, Sections 27 and 89/1 of the Revenue Code both address additional tax charges for late payment, but Section 27 resides in Chapter 2: Procedures Related to Tax Assessment, while Section 89/1 belongs to Chapter 4: Value-Added Tax. 
%
Without providing a proper comprehensive definition of all law hierarchies as a context to the prompt, differentiating these sections becomes extremely difficult.

% \begin{quote}
% \begin{thai}
% \enquote{\textbf{ประมวลรัษฎากร มาตรา 27} บุคคลใดไม่เสียหรือนำส่งภาษีภายในกำหนดเวลาตามที่บัญญัติไว้ในหมวดต่าง ๆ แห่งลักษณะนี้เกี่ยวกับภาษีอากรประเมิน ให้เสียเงินเพิ่มอีกร้อยละ 1.5 ต่อเดือนหรือเศษของเดือนของเงินภาษี...}        
% \end{thai}
% \end{quote}

% \begin{quote}
% \begin{thai}
% \enquote{\textbf{ประมวลรัษฎากร มาตรา 89/1} บุคคลใดไม่ชำระภาษีหรือนำส่งภาษีให้ครบถ้วนภายในกำหนดเวลาตามบทบัญญัติในหมวดนี้ ให้เสียเงินเพิ่มอีกร้อยละ 1.5 ต่อเดือน หรือเศษของเดือนของเงินภาษี..."}           
% \end{thai}
% \end{quote}
% CHECK AGAIN!!!
\begin{quote}
\textbf{Section 27 of Revenue Code}  
Any person who fails to pay or remit taxes within the specified timeframes as stipulated in various chapters of this title concerning assessed taxes shall be subject to an additional charge of 1.5\% per month or a fraction thereof on the tax amount...
\end{quote}

\begin{quote}
\textbf{Section 89/1 of Revenue Code}  
Any person who fails to fully pay or remit taxes within the prescribed period under this chapter shall incur an additional charge of 1.5\% per month or a fraction thereof on the tax amount...
\end{quote}


Two main approaches can potentially address this hierarchical information challenge. 
%
First, integrate this information directly into the section content, explicitly stating the chapter, division, and law code (or possibly its comprehensive definition of each level). 
%
This requires retraining the retriever model to incorporate this new information into its embedding generation process. 
%
Alternatively, a heuristic approach or a separate retrieval model could pre-select relevant legislation, titles, and divisions before section-level retrieval, thus reducing redundancy.

\subsubsection{Nested Structure}
This error type stems from the nested structure of Thai financial law, where sections often refer to other sections within the same or different legislation. 
%
These referencing sections may lack complete explanations; instead, they should point to other sections for details. 
%
For instance, section 1409 of the Civil and Commercial Code mandates the analogical application of sections 552–555, 558, 562, and 563 but does not contain their content. 
%
Consequently, retrievers might struggle to match queries with section 1409 directly, instead retrieving the referenced sections containing the relevant details.

% \begin{quote}
%     \begin{thai}
%     \enquote{\textbf{ประมวลกฎหมายแพ่งและพาณิชย์ มาตรา 1409} ท่านให้นำบทบัญญัติแห่งประมวลกฎหมายนี้ว่าด้วยหน้าที่และความรับผิดของผู้เช่าอันกล่าวไว้ในมาตรา 552 ถึง 555 มาตรา 558, 562 และ 563 มาใช้บังคับโดยอนุโลม}        
%     \end{thai}
% \end{quote}
\begin{quote}
\textbf{Section 1409 of Civil and Commercial Code}  
The provisions of this Code regarding the duties and liabilities of a lessee, as stipulated in Sections 552 to 555, Sections 558, 562, and 563, shall apply mutatis mutandis.
\end{quote}

Another common example involves sections specifying penalties. 
%
Determining the penalty for a specific offense often requires locating the section prohibiting the behavior and a separate section defining the corresponding penalty. 
%
This separation can cause retrieval issues. 
%
For example, consider Sections 186 and 291 of the Securities and Exchange Act. 
%
Section 186 specifies the authority of the Securities and Exchange Commission (SEC), while section 291 outlines penalties for non-compliance with SEC orders. 
%
A query about penalties for a specific offense might incorrectly retrieve Section 186 (describing the SEC's authority) instead of Section 291 (detailing the actual penalties) because the penalty section itself lacks details about the specific offense.

% \begin{quote}
% \begin{thai}
% \enquote{\textbf{พระราชบัญญัติหลักทรัพย์และตลาดหลักทรัพย์ พ.ศ. 2535 มาตรา 186 } เพื่อป้องกันความเสียหายอันอาจเกิดแก่ประโยชน์ของประชาชนหรือเศรษฐกิจของประเทศ ให้คณะกรรมการ ก.ล.ต. มีอำนาจ
% (1) ห้ามการซื้อขายหลักทรัพย์จดทะเบียนทั้งหมดในตลาดหลักทรัพย์เป็นการชั่วคราว...}        
% \end{thai}
% \end{quote}

% \begin{quote}
% \begin{thai}
% \enquote{\textbf{พระราชบัญญัติหลักทรัพย์และตลาดหลักทรัพย์ พ.ศ. 2535 มาตรา 291 } ผู้ใดฝ่าฝืนหรือไม่ปฏิบัติตามคำสั่งที่กำหนดตามมาตรา 186 (2) ต้องระวางโทษจำคุกไม่เกินหนึ่งปี หรือปรับไม่เกินสามแสนบาท หรือทั้งจำทั้งปรับ}           
% \end{thai}
% \end{quote}

\begin{quote}
\textbf{Section 186 of Securities and Exchange Act B.E. 2535}  
To prevent potential damage to the public interest or the national economy, the SEC Board has the authority to:  
(1) Temporarily prohibit the trading of all listed securities on the stock exchange...
\end{quote}

\begin{quote}
\textbf{Section 291 of Securities and Exchange Act B.E. 2535}  
Anyone who violates or fails to comply with the order issued under Section 186 (2) shall be subject to imprisonment for a term not exceeding one year, a fine not exceeding 300,000 baht, or both.
\end{quote}

To potentially mitigate these nested structure errors, a knowledge graph representing the relationships between sections could be employed. 
%
This would allow for augmenting retrieved content with referenced or referencing sections, providing the LLM with a more complete context. 
%
We implemented a simple knowledge graph, referred to as \textit{NitiLink}, in the pipeline and observed improved retriever performance, as shown in \S~\ref{subsubsec: referencer_result}, particularly on NitiBench-Tax, which requires multiple sections and legal reasoning. 
%
However, a more advanced implementation of this component should be explored.

\subsubsection{Missable Details}
%
This type of error occurs when subtle but critical details in a query are overlooked by dense embedding models, which often focus on general relevance instead of fine-grained nuances. 
%
For instance, consider a legal query about whether one guarantor can guarantee another guarantor. 
%
The correct answer relies on Section 682 of the Civil and Commercial Code, which explicitly allows such arrangements. 
%
However, the retriever might incorrectly prioritize Section 680, which only provides a general definition of a guarantor, thus missing the specific nuance required to address the query accurately.

% \begin{quote}
% \begin{thai}
% \textbf{Question: } บุคคลสามารถเข้าค้ำประกันให้ผู้ค้ำประกันได้หรือไม่\\
% \textbf{Retrieved Section: } ประมวลกฎหมายแพ่งและพาณิชย์ มาตรา 680
% อันว่าค้ำประกันนั้น คือสัญญาซึ่งบุคคลภายนอกคนหนึ่ง เรียกว่า ผู้ค้ำประกัน ผูกพันตนต่อเจ้าหนี้คนหนึ่ง เพื่อชำระหนี้ในเมื่อลูกหนี้ไม่ชำระหนี้นั้น

% อนึ่ง สัญญาค้ำประกันนั้น ถ้ามิได้มีหลักฐานเป็นหนังสืออย่างใดอย่างหนึ่งลงลายมือชื่อผู้ค้ำประกันเป็นสำคัญ ท่านว่าจะฟ้องร้องให้บังคับคดีหาได้ไม่  \\

% \textbf{Gold Section: } ประมวลกฎหมายแพ่งและพาณิชย์ มาตรา 682 ท่านว่าบุคคลจะยอมเข้าเป็นผู้รับเรือน คือเป็นประกันของผู้ค้ำประกันอีกชั้นหนึ่ง ก็เป็นได้

% ถ้าบุคคลหลายคนยอมตนเข้าเป็นผู้ค้ำประกันในหนี้รายเดียวกันไซร้ ท่านว่าผู้ค้ำประกันเหล่านั้นมีความรับผิดอย่างลูกหนี้ร่วมกัน แม้ถึงว่าจะมิได้เข้ารับค้ำประกันรวมกัน
% \end{thai}
% \end{quote}

\begin{quote}
\textbf{Question:} Can a person act as a guarantor for another guarantor?\\

\textbf{Retrieved Section:}  
Section 680 of Civil and Commercial Code  
Suretyship is a contract in which a third party, called the guarantor, binds themselves to a creditor to fulfill the obligation if the debtor fails to do so.  

Furthermore, a suretyship contract must be evidenced in writing and signed by the guarantor; otherwise, it cannot be enforced in court.\\

\textbf{Gold Section:}  
Section 682 of Civil and Commercial Code 
A person may act as a secondary guarantor, meaning they guarantee the obligations of the primary guarantor.  

If multiple persons act as guarantors for the same debt, they are jointly liable as co-debtors, even if they did not enter into the suretyship agreement together.
\end{quote}

One approach to address these missable detail errors involves adjusting the weighting in late-interaction embedding models to prioritize keyword matching. 
%
This increased focus on keywords could improve the model's ability to capture subtle nuances within queries.

%One approach to potentially address these missable detail errors involves adjusting the weighting in late-interaction embedding models to prioritize keyword matching. This increased focus on keywords can improve the model's ability to capture subtle nuances within queries.

\subsubsection{Complex Queries}
This category encompasses complex queries requiring multi-step reasoning. 
%
These errors are prevalent and difficult to resolve through simple fine-tuning. 
%
For example, a query about returning a purchased ring to someone claiming ownership (who wasn't the seller) requires multiple reasoning steps. 
%
The model must first recognize the ring as movable property, then retrieve Section 1303 of the Civil and Commercial Code, which grants superior ownership rights to a good-faith purchaser for value. 
%
Current retrieval models struggle with this multi-step reasoning, often retrieving more general sections like Section 1332, which addresses returning purchased goods to the seller.

% \begin{quote}
% \begin{thai}
% \textbf{Question: } ซื้อแหวนมาจากคนหนึ่งได้รับของมาแล้ว ต่อมามีอีกคนอ้างว่าตนเป็นเจ้าของ ต้องคืนแหวนให้ผู้ที่มาทวงหรือไม่ \\
% \textbf{Retrieved Section: } ประมวลกฎหมายแพ่งและพาณิชย์ มาตรา 1332
% บุคคลผู้ซื้อทรัพย์สินมาโดยสุจริตในการขายทอดตลาด หรือในท้องตลาด หรือจากพ่อค้าซึ่งขายของชนิดนั้น ไม่จำต้องคืนให้แก่เจ้าของแท้จริง เว้นแต่เจ้าของจะชดใช้ราคาที่ซื้อมา \\
% \textbf{Gold Section: } ประมวลกฎหมายแพ่งและพาณิชย์ มาตรา 1303 ถ้าบุคคลหลายคนเรียกเอาสังหาริมทรัพย์เดียวกัน โดยอาศัยหลักกรรมสิทธิ์ต่างกันไซร้ ท่านว่าทรัพย์สินตกอยู่ในครอบครองของบุคคลใด บุคคลนั้นมีสิทธิยิ่งกว่าบุคคลอื่น ๆ แต่ต้องได้ทรัพย์นั้นมาโดยมีค่าตอบแทนและได้การครอบครองโดยสุจริต

% ท่านมิให้ใช้มาตรานี้บังคับถึงสังหาริมทรัพย์ซึ่งระบุไว้ในมาตราก่อนและในเรื่องทรัพย์สินหาย กับทรัพย์สินที่ได้มาโดยการกระทำผิด
% \end{thai}
% \end{quote}
\begin{quote}
\textbf{Question:} If I buy a ring from someone and later another person claims to be the rightful owner, do I have to return the ring? \\

\textbf{Retrieved Section:}  
Section 1332 of Civil and Commercial Code  
A person who purchases property in good faith at a public auction, in a marketplace, or from a merchant dealing in such goods is not required to return it to the rightful owner unless the owner reimburses the purchase price. \\

\textbf{Gold Section:}  
Section 1303 of Civil and Commercial Code 
If multiple individuals claim ownership of the same movable property based on different legal grounds, the person in possession has superior rights, provided they acquired it for value and in good faith.  

This provision does not apply to items specified in the preceding section, lost property, or assets acquired through wrongful acts.
\end{quote}

Addressing these complex queries requires an additional step to decompose the query into multiple sub-queries, each focusing on a specific aspect. 
%
This allows the model to perform a more nuanced \enquote{thought process}, inferring implicit information and breaking down complex issues into smaller, more manageable parts, ultimately improving retrieval accuracy and comprehensiveness.

\subsection{Error Analysis on Retriever Performances on NitiBench-Tax}
\label{subsec: retriever_re_error_analysis_tax}
Beyond NitiBench-CCL, a detailed error analysis was also performed on the NitiBench-Tax. 

\subsubsection{Generic Section Retrieval Challenge}
This analysis focused on false negatives and false negative rates, particularly within sections of the Revenue Code. 
%
Table~\ref{table: retrieval_tax_section_fn} highlights the five sections with the highest false negative rates and counts.

\begin{table}[!ht]
\centering
\small
\renewcommand{\arraystretch}{1.2} % This increases the cell height by 1.5 times

\begin{tabular}{@{}lcc@{}}
\toprule
\multicolumn{1}{l}{\textbf{Section}} & \multicolumn{1}{l}{\textbf{False Negative}} & \multicolumn{1}{l}{\textbf{False Negative Rate}} \\ \midrule
77/2                        & 6                                  & 1.0                                     \\
77/1                        & 5                                  & 1.0                                     \\
2                           & 5                                  & 1.0                                     \\
3 octo                     & 4                                  & 1.0                                     \\
91/5                        & 2                                  & 1.0                                     \\ \bottomrule
\end{tabular}
\caption{Top 5 false negative and false negative ratio of sections in Revenue Code of NitiBench-Tax}
\label{table: retrieval_tax_section_fn}
\end{table}

These sections largely contain either terminology definitions or broad legal overviews. 
%
Because they lack specific details and conditions, they pose a retrieval challenge: while important, their generality makes them a poor match for the concrete scenarios found in NitiBench-Tax. 
%
The retriever model, therefore, tends to favor sections with greater contextual relevance. We refer to this problem as the \enquote{Generic Section Retrieval Challenge}.

For example, a query about VAT applicability to animal treatment chemicals and medical supplies might retrieve sections discussing specific VAT exemptions or surcharges rather than the overarching section 77/2 of the Revenue Code, which simply states that all sales, imports, and services are subject to VAT. 
%
While some retrieved sections may be relevant, the model fails to surface the foundational Section 77/2, hindering comprehensive legal reasoning.

%     \begin{quote}
%         \begin{thai}
%         \textbf{Question: } บริษัทประกอบกิจการโรงพยาบาลสัตว์ให้บริการเป็นสถานรักษาสัตว์ ได้มีการสั่งซื้อยาเพื่อมาใช้รักษาสัตว์ เมื่อบริษัทฯ คิดค่ารักษาสัตว์กับลูกค้า บริษัทฯ ได้แยกค่ายากับค่าบริการรักษาสัตว์ออกจากกัน บริษัทจึงขอหารือว่า
%     1. บริษัทจะเรียกเก็บภาษีมูลค่าเพิ่มเฉพาะค่ารักษาสัตว์ แต่จะไม่เรียกเก็บภาษีมูลค่าเพิ่มสำหรับค่ายาที่ใช้รักษาสัตว์ ถูกต้องหรือไม่
%     2. ค่าเคมีภัณฑ์ที่ใช้สำหรับการรักษาสัตว์ เช่น น้ำยาล้างแผล แอลกอฮอล์ น้ำเกลือ เป็นต้น บริษัทจะต้องเรียกเก็บภาษีมูลค่าเพิ่มจากลูกค้าหรือไม่
%     3. ค่าเวชภัณฑ์ที่ใช้สำหรับรักษาสัตว์ เช่น สำลี สายน้ำเกลือ เข็มและไหมเย็บ เป็นต้น บริษัทจะต้องเรียกเก็บภาษีมูลค่าเพิ่มหรือไม่       

%         \textbf{Retrieved Section: } ประมวลรัษฎากร มาตรา 83/10 ในการเสียภาษีมูลค่าเพิ่ม
% (1) สำหรับสินค้าที่นำเข้า ให้กรมศุลกากรเรียกเก็บเพื่อกรมสรรพากร และในกรณีของตกค้างตามมาตรา 78/2 (3) ให้กรมศุลกากรหักภาษีมูลค่าเพิ่มพร้อมเบี้ยปรับเงินเพิ่มเพื่อกรมสรรพากรตามหลักเกณฑ์ที่อธิบดีกำหนด
% (2) สำหรับการขายสินค้าหรือการให้บริการที่ต้องเสียภาษีสรรพสามิตด้วย ให้กรมสรรพสามิตเรียกเก็บเพื่อกรมสรรพากร

%         \textbf{Gold Section: } ประมวลรัษฎากร มาตรา 77/2 การกระทำกิจการดังต่อไปนี้ในราชอาณาจักร ให้อยู่ในบังคับต้องเสียภาษีมูลค่าเพิ่ม ตามบทบัญญัติในหมวดนี้

% (1) การขายสินค้าหรือการให้บริการโดยผู้ประกอบการ

% (2) การนำเข้าสินค้าโดยผู้นำเข้า

% การให้บริการในราชอาณาจักรให้หมายถึง บริการที่ทำในราชอาณาจักรโดยไม่คำนึงว่าการใช้บริการนั้นจะอยู่ในต่างประเทศหรือในราชอาณาจักร

% การให้บริการที่ทำในต่างประเทศและได้มีการใช้บริการนั้นในราชอาณาจักรให้ถือว่าการให้บริการนั้น เป็นการให้บริการในราชอาณาจักร
%         \end{thai}
%     \end{quote}

\begin{quote}
    \textbf{Question:}  
    A company operates an animal hospital providing veterinary services. The company purchases medicine for animal treatment and separately charges customers for medicine and treatment services. The company seeks clarification on the following:  
    1. Can the company charge VAT only on treatment services while exempting VAT on medicine used for treatment?  
    2. Should the company charge VAT on chemicals used for treatment, such as disinfectants, alcohol, and saline solution?  
    3. Should the company charge VAT on medical supplies used for treatment, such as cotton, IV tubes, needles, and sutures?  
    
    \textbf{Retrieved Section:}  
    Section 83/10 of Revenue Code – VAT Collection:  
    (1) For imported goods, the Customs Department shall collect VAT on behalf of the Revenue Department. In the case of abandoned goods under Section 78/2 (3), the Customs Department shall deduct VAT, along with penalties and surcharges, according to the criteria set by the Director-General.  
    (2) For the sale of goods or services subject to excise tax, the Excise Department shall collect VAT on behalf of the Revenue Department.  
    
    \textbf{Gold Section:}  
    Section 77/2 of Revenue Code – VAT Liability:  
    The following activities conducted in Thailand are subject to VAT under this section:  
    (1) The sale of goods or the provision of services by an operator.  
    (2) The importation of goods by an importer.  
    
    The provision of services in Thailand refers to services performed in Thailand, regardless of whether they are used domestically or abroad.  
    Services performed abroad but used in Thailand are also considered services rendered in Thailand.
\end{quote}

Two strategies may mitigate this issue. 
%
First, we should consider whether abstract and general sections constitute foundational knowledge that the LLM requires for all queries within a RAG system, rather than retrieving them selectively. 
%
Second, keyword matching retrieval could prove effective for terminology sections, as query terms often directly match defined phrases.

Beyond the Generic Section Retrieval Challenge, our analysis of NitiBench-Tax revealed two additional error categories:

\subsubsection{Incorrect Legislation Retrieval} 
Table~\ref{table: retrieval_tax_code_fp} presents the distribution of false positives at the law code level.

% Please add the following required packages to your document preamble:
% \usepackage{booktabs}
\begin{table}[!h]
\centering

\renewcommand{\arraystretch}{1.2} % This increases the cell height by 1.5 times

\begin{tabular}{@{}lc@{}}
\toprule
\textbf{Law Code}                                                                                                                    & \textbf{False Positive} \\ \midrule
Revenue Code                                                                                                                & 280            \\
Petroleum Income Tax Act, B.E. 2514                                                                                         & 30             \\
Civil and Commercial Code                                                                                                   & 21             \\
Securities and Exchange Act, B.E. 2535                                                                                      & 15             \\
\begin{tabular}[c]{@{}c@{}}Government Procurement and Supplies Management Act, B.E. 2560\end{tabular}                     & 15             \\
Budget Procedure Act, B.E. 2561                                                                                             & 14             \\
Energy Industry Act, B.E. 2550                                                                                              & 12             \\
Business Registration Act, B.E. 2499                                                                                        & 10             \\
Public Limited Companies Act, B.E. 2535                                                                                     & 8              \\
Energy Conservation Promotion Act, B.E. 2535                                                                                & 5              \\
Trust for Transactions in Capital Market Act, B.E. 2550                                                                     & 5              \\
Financial Institutions Business Act, B.E. 2551                                                                              & 4              \\
National Economic and Social Development Act, B.E. 2561                                                                     & 3              \\
Accounting Profession Act, B.E. 2547                                                                                        & 3              \\
\begin{tabular}[c]{@{}c@{}}Act on the Management of Shares and Stocks of Ministers, B.E. 2543\end{tabular}                & 2              \\
State Enterprise Development and Governance Act, B.E. 2562                                                                  & 2              \\
Fiscal Discipline Act, B.E. 2561                                                                                            & 2              \\
Accounting Act, B.E. 2543                                                                                                   & 1              \\
\begin{tabular}[c]{@{}c@{}}Emergency Decree on Special Purpose Juristic Person for Securitization, B.E. 2540\end{tabular} & 1              \\
Provident Fund Act, B.E. 2530                                                                                               & 1              \\
Emergency Decree on Digital Asset Businesses, B.E. 2561                                                                     & 1              \\
Foreign Business Act, B.E. 2542                                                                                             & 1              \\
Derivatives Act, B.E. 2546                                                                                                  & 1              \\ \bottomrule
\end{tabular}
\caption{False positive distribution on NitiBench-Tax on law code level}
\label{table: retrieval_tax_code_fp}
\end{table}

While the NitiBench-Tax's ground truth labels span only 4 legislation, retrieved false positives originate from 21 different legislation. 
%
This mirrors the hidden hierarchical information problem observed in NitiBench-CCL, where similar concepts appear in different legislation. 
%
For instance, both Section 91/15 of the Revenue Code and Section 56 of the Petroleum Act address tax assessor authority, differing only in their conditions of application.

This issue is amplified in NitiBench-Tax because queries directed to the Revenue Department often omit details implicitly covered by the Revenue Code's scope.
%
Besides previously suggested solutions, such as incorporating hidden information into section content and employing level-based retrieval, restricting retrieval to sections within relevant legislation when queries pertain to a specific domain could potentially alleviate this problem. 
%
This targeted approach should reduce false positives from unrelated codes.


%     \begin{quote}
%         \begin{thai}
%         \textbf{ประมวลรัษฎากร มาตรา 91/15 } เจ้าพนักงานประเมินมีอำนาจประเมินภาษี เบี้ยปรับ และเงินเพิ่มตามหมวดนี้ ในเมื่อ
% (1) ปรากฏแก่เจ้าพนักงานประเมินว่าผู้มีหน้าที่เสียภาษีมิได้ยื่นแบบแสดงรายการภายในเวลาที่กฎหมายกำหนด...
        
%         \textbf{พระราชบัญญัติภาษีเงินได้ปิโตรเลียม พ.ศ. 2514 มาตรา 56} เจ้าพนักงานประเมินมีอำนาจประเมินภาษี เบี้ยปรับ และเงินเพิ่มตามพระราชบัญญัตินี้ เมื่อ
% (1) ผู้มีหน้าที่ยื่นแบบแสดงรายการเงินได้มิได้ยื่นแบบแสดงรายการเงินได้ภายในเวลาที่กำหนด...         
%         \end{thai}
%     \end{quote}

% \begin{quote}
% \textbf{Section 91/15 of Revenue Code}  
% An assessment officer has the authority to assess taxes, surcharges, and additional charges under this section when:  
% (1) It is evident to the assessment officer that a taxpayer has failed to file a tax return within the legally prescribed timeframe...  

% \textbf{Section 56 of Petroleum Income Tax Act B.E. 2514}  
% An assessment officer has the authority to assess taxes, surcharges, and additional charges under this Act when:  
% (1) A person liable to file an income declaration fails to submit the declaration within the prescribed period...  
% \end{quote}

\subsubsection{Incorrect Tax Type Retrieval}
Beyond false positives due to similar content, the retrieval model frequently misidentifies the relevant tax type, especially in complex queries. 
%
This resembles the \enquote{hard query} error type observed in NitiBench-CCL.

For example, a query about the tax obligations of an employee in Thailand receiving income from both a subsidiary and its parent company (a personal income tax question) should retrieve Sections 41, 48, 50, and 56 of the Revenue Code, which addresses personal income tax, withholding obligations, and calculating tax on foreign income.

However, the model instead retrieves sections related to corporate and export taxes. 
%
This likely stems from keywords like \enquote{company}, \enquote{corporate}, and \enquote{foreign} influencing the query embedding, shifting its focus away from personal income tax.


% \begin{quote}
%     \begin{thai}
%     \textbf{Question: } บริษัทจัดตั้งขึ้นตามกฎหมายประเทศไทย ประกอบธุรกิจเกี่ยวกับสิ่งทอ และมีการลงทุนในธุรกิจที่เกี่ยวข้องทั้งในประเทศไทย และในต่างประเทศ บริษัทได้แต่งตั้งให้ผู้บริหารหรือพนักงานไปดำรงตำแหน่งกรรมการในบริษัทลูกในต่างประเทศ โดยจะได้รับค่าตอบแทนเงินเดือน สวัสดิการต่าง ๆ จากหน้าที่งานของบริษัทในประเทศไทย โดยผ่านบัญชีเงินฝากของผู้บริหารหรือพนักงานในประเทศไทย และค่าตอบแทนจากการเป็นกรรมการในบริษัทลูกในต่างประเทศ ตามอัตราค่าตอบแทนเปรียบเทียบกับค่าตอบแทนในตำแหน่งเดียวกัน หรือเทียบเคียงของธุรกิจหรืออุตสาหกรรมประเภทเดียวกัน หรือให้เป็นไปตามข้อกำหนดในแต่ละประเทศ โดยบริษัทลูกในต่างประเทศจะจ่ายค่าตอบแทนให้แก่ผู้บริหารโดยตรง โดยผ่านบัญชีเงินฝากในต่างประเทศที่บริษัทลูกในต่างประเทศเปิดไว้ให้ บริษัทขอหารือ ดังนี้
% 1. กรณีบริษัทจ่ายเงินเดือนหรือค่าตอบแทนเฉพาะที่เกิดจากหน้าที่งานของนายจ้างในประเทศไทย บริษัทมีหน้าที่นำเงินได้ที่ผู้บริหารหรือพนักงานได้รับเฉพาะที่จ่ายจากหน้าที่งานที่ทำให้นายจ้างในประเทศไทยมารวมคำนวณเป็นรายได้ เพื่อนำส่งภาษีหัก ณ ที่จ่าย ตามมาตรา 50 (1) แห่งประมวลรัษฎากร ใช่หรือไม่
% 2. กรณีผู้บริหารหรือพนักงานที่ได้รับค่าตอบแทนจากการเป็นกรรมการในบริษัทลูกในต่างประเทศ และได้อยู่ในประเทศไทยในปีภาษีเดียวกันกับที่ได้รับเงินได้ค่าตอบแทนเกิน 180 วัน หากผู้บริหารหรือพนักงานคนดังกล่าว ไม่ได้นำเงินค่าตอบแทนที่ได้รับดังกล่าว เข้ามาในประเทศไทยในปีภาษีเดียวกัน ผู้บริหารคนดังกล่าวไม่มีหน้าที่ต้องเสียภาษีเงินได้บุคคลธรรมดาเนื่องจากหน้าที่งานที่ทำในต่างประเทศตามมาตรา 41 วรรคสอง แห่งประมวลรัษฎากร ใช่หรือไม่        
    
%     \textbf{Retrieved Section: } ประมวลรัษฎากร มาตรา 76 ทวิ บริษัทหรือห้างหุ้นส่วนนิติบุคคลซึ่งตั้งขึ้นตามกฎหมายของต่างประเทศมีลูกจ้าง หรือผู้ทำการแทน หรือผู้ทำการติดต่อ ในการประกอบกิจการในประเทศไทย ซึ่งเป็นเหตุให้ได้รับเงินได้หรือผลกำไรในประเทศไทยให้ถือว่า บริษัทหรือห้างหุ้นส่วนนิติบุคคลนั้นประกอบกิจการในประเทศไทยและให้ถือว่าบุคคลผู้เป็นลูกจ้างหรือผู้ทำการแทน หรือผู้ทำการติดต่อเช่นว่านั้นไม่ว่าจะเป็นบุคคลธรรมดาหรือนิติบุคคลเป็นตัวแทนของบริษัทหรือห้างหุ้นส่วนนิติบุคคลซึ่งตั้งขึ้นตามกฎหมายของต่างประเทศและให้บุคคลนั้นมีหน้าที่และความรับผิดในการยื่นรายการและเสียภาษีตามบทบัญญัติในส่วนนี้เฉพาะที่เกี่ยวกับเงินได้หรือผลกำไรที่กล่าวแล้ว
% ในกรณีที่กล่าวในวรรคแรกถ้าบุคคลผู้มีหน้าที่และความรับผิดในการยื่นรายการและเสียภาษีไม่สามารถจะคำนวณกำไรสุทธิเพื่อเสียภาษีตามบทบัญญัติในส่วนนี้ได้ให้นำบทบัญญัติว่าด้วยการประเมินภาษีตามมาตรา71 (1) มาใช้บังคับโดยอนุโลม
% ในกรณีการประเมินตามความในมาตรานี้จะอุทธรณ์การประเมินก็ได้


%     \end{thai}
% \end{quote}

\begin{quote}
    \textbf{Question: }  
    A company incorporated under Thai law operates in the textile industry and has investments in related businesses both in Thailand and abroad. The company appoints executives or employees as directors in its foreign subsidiaries. These individuals receive salaries and benefits for their duties in Thailand, paid through their Thai bank accounts. Additionally, they receive director’s fees from the foreign subsidiaries based on market rates or country-specific regulations, with payments made directly to their foreign bank accounts. The company seeks clarification on the following:
    
    1. If the company pays salaries or compensation solely for duties performed for the employer in Thailand, is the company required to include only the income derived from such duties in Thailand for withholding tax calculation under Section 50(1) of the Revenue Code?
    2. If an executive or employee receives director’s fees from a foreign subsidiary and stays in Thailand for more than 180 days within the same tax year, but does not remit this income into Thailand within the same tax year, is the individual exempt from personal income tax for foreign-earned income under Section 41, paragraph two of the Revenue Code?
    
    \textbf{Retrieved Section: }  
    Revenue Code Section 76 bis:  
    A company or juristic partnership established under foreign law that has employees, agents, or representatives conducting business in Thailand, resulting in income or profits in Thailand, shall be deemed to be carrying on business in Thailand. Such persons, whether individuals or entities, shall be considered representatives of the foreign company or partnership and shall be responsible for filing tax returns and paying tax under this section for the relevant income or profits.  
    
    If the responsible person cannot calculate net profits for tax purposes, the provisions on tax assessment under Section 71(1) shall apply accordingly.  
    
    In cases of tax assessment under this section, an appeal against the assessment may be filed.  
\end{quote}

%%%%%%
\subsection{Effectiveness of Multi-label Metrics}
\label{subsec: multi_metrics}

To further validate the effectiveness of our proposed multi-label metrics, we compute the correlation between conventional retrieval metrics (Hit Rate and MRR) compared to its multi-label variant.
%
We use eight retriever model performances (see Table~\ref{table: retrieval_e2e_main}) to measure the correlation between retrieval and the E2E metric.
%
The result was presented in Table~\ref{tab:correlation}.

According to the result, we can see that our Multi-MRR and Multi-Hit Rate have a higher correlation compared to conventional MRR and hit rate.
%
These results emphasize the importance of using multi-label metrics in legal QA setups.


\begin{table}[!ht]
\centering
\fontsize{8}{11}\selectfont  % Slightly reduce the font size
\setlength{\tabcolsep}{3.75pt}   % Further reduce horizontal padding
\renewcommand{\arraystretch}{0.85}  % Further reduce vertical spacing
\begin{tabular}{lccc}
\toprule
               & \textbf{Coverage ($\uparrow$)} & \textbf{Contradiction ($\downarrow$)} & \textbf{E2E F1 ($\uparrow$)} \\ 
\midrule
Hit Rate       & 0.741            & -0.672                 & 0.780          \\
\rowcolor[HTML]{EFEFEF} 
Multi Hit Rate & 0.989            & -0.986                 & 0.984          \\
MRR            & 0.906            & -0.859                 & 0.933         \\
\rowcolor[HTML]{EFEFEF} 
Multi MRR      & 0.989            & -0.973                 & 0.991          \\
\bottomrule
\end{tabular}
\caption{Correlation between conventional and multi-retrieval metrics with evaluation measures using data from 8 retrievers (Table~\ref{table: retrieval_e2e_main}})
\label{tab:correlation}
\end{table}




\subsection{Analysis of Effect of LLMs on E2E and Retrieval Performance}
\label{subsec: llm_error_analysis}

We investigate two metrics to assess the gaps between retrieval recall and E2E recall:
\begin{itemize}
    \item \textbf{Recall Difference}: Formally, defined as the difference between retriever recall and E2E recall. This metric reflects the RAG system's ability to utilize retrieved documents for reasoning and answer generation.
    %
    The lower the recall difference, the better LLM can reach upper bound recall performance of the retriever.
    %
    \item \textbf{Zero Recall, Non-Zero Coverage Ratio (Hallucination Rate)}: The ratio of samples with zero E2E recall but non-zero coverage, which indicates the likelihood of generating answers without relying on correctly retrieved documents, potentially due to parametric knowledge or hallucinations.
\end{itemize}
The results are shown in Table~\ref{table: llm_e2e_recall_diff}.
%
From the results, \texttt{claude-3.5-sonnet} minimizes the recall difference across both datasets, demonstrating its strong ability to leverage retrieved documents for reasoning. 
%
This contributes to its high Coverage and low Contradiction scores. 
%
OpenAI achieves the lowest hallucination ratio on NitiBench-Tax, while Claude and Gemini perform best on NitiBench-CCL. 
%
As for Typhoon models, the 70b variant performs well in terms of the hallucination ratio in NitiBench-Tax.
%
However, both Typhoon models exhibit high recall differences on both datasets which brings their abilities to utilize correctly retrieved documents into question. 
%
The recall difference is significantly higher for NitiBench-Tax than NitiBench-CCL, suggesting that NitiBench-Tax poses greater challenges for applying the retrieved information to reasoning and answer generation.

\begin{table}[!ht]
\centering
\renewcommand{\arraystretch}{1.2}
\resizebox{0.8\textwidth}{!}{%
\begin{tabular}{@{}lcc@{}}
\toprule
\textbf{Model} & \textbf{Recall Difference ($\downarrow$)} & \begin{tabular}[c]{@{}c@{}}\textbf{Zero Recall, Non Zero}\\\textbf{Coverage Ratio (Hallucination Rate) ($\downarrow$)}\end{tabular} \\ 
\midrule
\multicolumn{3}{c}{\textbf{NitiBench-CCL}} \\
\midrule
\texttt{gpt-4o-2024-08-06}          & 0.058 & 0.069 \\
\texttt{claude-3.5-sonnet-20240620} & \textbf{0.036} & \underline{0.060} \\
\texttt{gemini-1.5-pro-002}         & \underline{0.045} & \textbf{0.058} \\
\texttt{typhoon-v2-70b-instruct}    & 0.076 & 0.079 \\
\texttt{typhoon-v2-8b-instruct}     & 0.163 & 0.120 \\

\midrule
\multicolumn{3}{c}{\textbf{NitiBench-Tax}} \\
\midrule
\texttt{gpt-4o-2024-08-06}          & \underline{0.100} & \textbf{0.100} \\
\texttt{claude-3.5-sonnet-20240620} & \textbf{0.095} & 0.160 \\
\texttt{gemini-1.5-pro-002}         & 0.102 & 0.140 \\
\texttt{typhoon-v2-70b-instruct}    & 0.148 & \underline{0.120} \\
\texttt{typhoon-v2-8b-instruct}     & 0.246 & 0.200 \\

\bottomrule
\end{tabular}%
}
\caption{The table shows the performance gap between retrieval recall and citation recall (E2E recall), and the hallucination rate defined by a ratio of the sample that has zero recall (LLM fails to cite correct documents) but achieves non-zero coverage (showing that LLM can answer despite not citing a correct document). Both metrics were measured across both NitiBench-CCl and NitiBench-Tax with different LLM setups.}
\label{table: llm_e2e_recall_diff}
\end{table}


Analyzing samples with high retriever recall but low coverage reveals a recurring pattern: 

\textbf{LLM often omits intermediate reasoning steps, directly applying retrieved laws without sufficient explanation.} For example, a query about exempting income sent from a foreign husband to his wife in Thailand (without a registered marriage) requires multi-step reasoning.

First, the relevant law (Section 42 of the Revenue Code) exempts income received from moral obligation or gifts given during ceremonies or customs. 
%
The system should then reason whether the money sent to the wife qualifies as a moral obligation. 
%
Evidence of the unregistered marriage would support this claim and justify the exemption.

However, the LLM often skips this detailed reasoning. 
%
It prematurely concludes that the money isn't from tradition or moral obligation and is, therefore, taxable.
%
This overlooks the possibility of the husband's payment being a moral obligation, failing to provide the necessary intermediate reasoning steps.


% \begin{quote}
%     \begin{thai}
%     \textbf{Question: } 1. นาง ก. ปัจจุบันมีสามีเป็นชาวต่างชาติชื่อ Mr.A แต่มิได้จดทะเบียนสมรส
%   2. กรมสรรพากรมีข้อมูล Exchange of Information จากสรรพากรต่างประเทศ (ประเทศ B) กรณีนาง ก. ผู้มีถิ่นที่อยู่ในประเทศไทยได้รับเงินของขวัญ (Gift) ในฐานะ แฟนสาวจาก Mr.A คิดเป็นเงินบาทไทยตามอัตราซื้อถัวเฉลี่ยเงินโอนของธนาคารแห่งประเทศไทย (ธปท.) ดังนี้
%     2.1 จำนวน 1 ล้านบาท เมื่อวันที่ 20 มีนาคม 2560...
%   4. เงินที่นาง ก. ได้รับในกรณีดังกล่าว เป็นเงินได้ที่ได้รับยกเว้นตามมาตรา 42 (28) แห่ง ประมวลรัษฎากรหรือไม่

    
%     \textbf{Relevant Laws: }
%     ประมวลรัษฎากร มาตรา 42
%     ...
%     (28) เงินได้ที่ได้รับจากการอุปการะโดยหน้าที่ธรรมจรรยาหรือจากการให้โดยเสน่หาเนื่องในพิธีหรือตามโอกาสแห่งขนบธรรมเนียมประเพณี ทั้งนี้ จากบุคคลซึ่งมิใช่บุพการี ผู้สืบสันดาน หรือคู่สมรส เฉพาะเงินได้ในส่วนที่ไม่เกินสิบล้านบาทตลอดปีภาษีนั้น
%     ...

%     \textbf{RAG Analysis: }

%     From the given case details, we need to consider the following points:
    
%     1. นาง ก. and Mr. A are not legally married, but they are in a relationship.
    
%     2. นาง ก. received large sums of money as gifts from Mr. A in 2560, 2561, and 2562.
    
%     3. นาง ก. also receives monthly allowances from Mr. A for household expenses.
    
%     4. The question is whether these gifts are exempt under Section 42(28) of the Revenue Code.
    
%     According to Section 42(28), income received from moral support or gifts given during ceremonies or on occasions according to custom traditions, from persons who are not ancestors, descendants, or spouses, is exempt from income tax, but only for the portion not exceeding 10 million baht throughout the tax year.
    
%     In this case:
%     1. Mr. A is not legally นาง ก.'s spouse, so he falls under the category of \enquote{persons who are not ancestors, descendants, or spouses}.
%     2. The gifts received in each year (1 million, 800,000, and 700,000 baht) are all below the 10 million baht threshold.
%     3. However, these gifts do not seem to be given \enquote{during ceremonies or on occasions according to custom traditions}. They appear to be regular gifts given in a relationship.

%     \end{thai}
% \end{quote}

\begin{quote}
    \textbf{Question: } 
    1. Mrs. K is currently married to a foreigner named Mr. A, but they have not registered their marriage. 
    2. The Revenue Department has received an Exchange of Information from the foreign tax authority (Country B) indicating that Mrs. K, who is a resident of Thailand, received a gift from Mr. A, her boyfriend, in Thai baht, calculated at the average exchange rate of the Bank of Thailand (BOT) as follows:
        2.1 Amount of 1 million baht on March 20, 2017...
    4. Is the money that Mrs. K received in this case considered exempt income under Section 42(28) of the Revenue Code?

    \textbf{Relevant Laws: }
    Revenue Code Section 42
    ...
    (28) Income received from support due to natural duty or from voluntary gifts on ceremonial occasions or in accordance with customs, provided by someone other than parents, descendants, or spouses, but only income not exceeding ten million baht for the entire tax year.
    ...

    \textbf{RAG Analysis: }

    From the given case details, we need to consider the following points:
    
    1. Mrs.K and Mr. A are not legally married, but they are in a relationship.
    
    2. Mrs.K received large sums of money as gifts from Mr. A in 2560, 2561, and 2562.
    
    3. Mrs.K also receives monthly allowances from Mr. A for household expenses.
    
    4. The question is whether these gifts are exempt under Section 42(28) of the Revenue Code.
    
    According to Section 42(28), income received from moral support or gifts given during ceremonies or on occasions according to custom traditions, from persons who are not ancestors, descendants, or spouses, is exempt from income tax, but only for the portion not exceeding 10 million baht throughout the tax year.
    
    In this case:
    1. Mr. A is not legally Mrs.K's spouse, so he falls under the category of \enquote{persons who are not ancestors, descendants, or spouses}.
    2. The gifts received in each year (1 million, 800,000, and 700,000 baht) are all below the 10 million baht threshold.
    3. However, these gifts do not seem to be given \enquote{during ceremonies or on occasions according to custom traditions}. They appear to be regular gifts given in a relationship.

\end{quote}

\textbf{LLM's tendency to incorporate multiple similar, retrieved laws into its answer, retrieved laws into its answer despite being irrelevant.} 
%
This pattern is often exhibited in NitiBench-CCL. 
%
We suspect that this could stem from the query's ambiguity and the lack of context surrounding laws with similar content but different applicability conditions.

For example, a query about the tax payment deadline after receiving an assessment notification should retrieve Section 18 bis of the Revenue Code (specifying 7 days). 
%
However, due to the query's vagueness, the model might also incorporate Section 18 ter of the Revenue Code and Section 44 of the Petroleum Income Tax, both containing similar content but with different conditions.

This issue will likely persist until the system can engage in multi-turn conversations to clarify specific conditions and refine document retrieval.


% \begin{quote}
%     \begin{thai}
%     \textbf{Question: } ผู้ต้องเสียภาษีต้องชำระภาษีภายในกี่วันตั้งแต่วันที่ได้รับแจ้งประเมินภาษี

    
%     \textbf{Relevant Laws: }
    
%     ประมวลรัษฎากร มาตรา 18 ทวิ 
    
%     ในกรณีจำเป็นเพื่อรักษาผลประโยชน์ในการจัดเก็บภาษีอากร เจ้าพนักงานประเมินมีอำนาจที่จะประเมินเรียกเก็บภาษีจากผู้ต้องเสียภาษีก่อนถึงกำหนดเวลายื่นรายการได้ เมื่อได้ประเมินแล้วให้แจ้งจำนวนภาษีที่ต้องเสียไปยังผู้ต้องเสียและให้ผู้ต้องเสียภาษีชำระภาษีภายในเจ็ดวันนับแต่วันได้รับแจ้งการประเมิน ในกรณีนี้จะอุทธรณ์การประเมินก็ได้
%     ภาษีที่ประเมินเรียกเก็บตามความในวรรคก่อน ให้ถือเป็นเครดิตของผู้ต้องเสียภาษีในการคำนวณภาษี
%     ในการใช้อำนาจตามความในมาตรานี้ เจ้าพนักงานประเมินจะสั่งให้ยื่นรายการรตามแบบที่อธิบดีกำหนดด้วยก็ได้ 

%     \textbf{RAG Analysis: }

%     From the given question, we need to determine the timeframe for tax payment after receiving a tax assessment notice. According to the provided laws, there are different timeframes mentioned for various situations:
    
%     1. Section 18 ตรี of the Revenue Code states that the taxpayer must pay the tax within 30 days from the date of receiving the assessment notice.
    
%     2. Section 44 of the Petroleum Income Tax Act B.E. 2514 also mentions a 30-day period for tax payment after receiving the assessment notice.
    
%     3. Section 18 ทวิ of the Revenue Code mentions a special case where the tax officer can assess and collect tax before the due date of filing a tax return. In this case, the taxpayer must pay within 7 days of receiving the assessment notice.
    
%     Since the question doesn't specify a particular type of tax or situation, we should provide the general answer based on the most common timeframe mentioned in the laws. 

%     \end{thai}
% \end{quote}

\begin{quote}
    \textbf{Question: } 
    How many days does a taxpayer have to pay tax after receiving the tax assessment notice?

    \textbf{Relevant Laws: }
    Revenue Code Section 18 Bis

    In cases where necessary to protect the interests of tax collection, the assessing officer has the authority to assess and demand tax from the taxpayer before the due date for filing the return. Once the assessment is made, the amount of tax due must be notified to the taxpayer, and the taxpayer is required to pay the tax within seven days from the date of receiving the assessment notice. In this case, an appeal against the assessment is also allowed. 

    The tax assessed under this provision is considered a credit for the taxpayer in calculating their tax.
    
    When exercising authority under this section, the assessing officer may also order the taxpayer to file a return in the form specified by the Director-General.


    \textbf{RAG Analysis: }

    From the given question, we need to determine the timeframe for tax payment after receiving a tax assessment notice. According to the provided laws, there are different timeframes mentioned for various situations:
    
    1. Section 18 Tri of the Revenue Code states that the taxpayer must pay the tax within 30 days from the date of receiving the assessment notice.
    
    2. Section 44 of the Petroleum Income Tax Act B.E. 2514 also mentions a 30-day period for tax payment after receiving the assessment notice.
    
    3. Section 18 Bis of the Revenue Code mentions a special case where the tax officer can assess and collect tax before the due date of filing a tax return. In this case, the taxpayer must pay within 7 days of receiving the assessment notice.
    
    Since the question doesn't specify a particular type of tax or situation, we should provide the general answer based on the most common timeframe mentioned in the laws. 

\end{quote}

\textbf{Some LLM usually overcited retrieved documents} As shown in Table~\ref{table: llm_e2e_main_tax}, the Gemini model performs significantly worse than others on E2E precision. 
%
Figure~\ref{fig:citation_len}\subref{fig:tax_citation_len} reveals that Gemini cites considerably more laws than other models despite having a shorter average output token length than Claude. 
%
A detailed analysis shows that Gemini often cites laws that contain a few relevant keywords but lack overall relevance. 
%
Conversely, Figure~\ref{fig:citation_len}\subref{fig:wcx_citation_len} shows that Claude cites more laws than other models on average for NitiBench-CCL, likely due to its tendency to incorporate multiple laws for vaguely worded queries, as discussed previously.

\begin{figure}[!ht]
  \centering
  \begin{subfigure}[b]{0.48\textwidth}
    \centering
    \includegraphics[width=\textwidth]{images/tax_citation_len.png}
    \caption{Distribution of the number of citations on NitiBench-Tax}
    \label{fig:tax_citation_len}
  \end{subfigure}
  \hfill
  \begin{subfigure}[b]{0.48\textwidth}
    \centering
    \includegraphics[width=\textwidth]{images/wcx_citation_len.png}
    \caption{Distribution of the number of citations on NitiBench-CCL}
    \label{fig:wcx_citation_len}
  \end{subfigure}
  \caption{Distribution of the number of citations on both NitiBench-CCL and NitiBenth-Tax on different LLM.}
  \label{fig:citation_len}
\end{figure}
% \begin{figure}[!ht]
% \centering
% \includegraphics[width=0.5\textwidth]{images/tax_citation_len.png}
% \caption{Plot of distribution of number of citations each model makes on Tax Case dataset}
% \label{fig: tax_citation_len}
% \end{figure}

% \begin{figure}[!ht]
% \centering
% \includegraphics[width=0.5\textwidth]{images/wcx_citation_len.png}
% \caption{Plot of distribution of number of citations each model makes on WCX dataset}
% \label{fig: wcx_citation_len}
% \end{figure}

\subsection{Analysis of LCLM result}
\label{subsec: lclm_error_analysis}

In this section, the effect of the relevant context position in the overall documents on the performance of the system is analyzed on the sampled NitiBench-CCL with long context Gemini 1.5 Pro ingesting 1.2 million tokens or 2.7 million characters. 
%
The resulting performance is binned every 100,000 characters by the maximum depth of the relevant laws that need to be retrieved, and the coverage, contradiction, and E2E F1 of each bin are averaged and plotted in Figure~\ref{fig: depth_performance}.

\begin{figure}[!ht]
\centering
\includegraphics[width=0.7\textwidth]{images/depth_performance.png}
\caption{Plot of performance grouped by the maximum depth of relevant context in the long context LLM setup.}
\label{fig: depth_performance}
\end{figure}

From the resulting plot, there is only a slight decrease in the coverage score and a slightly greater increase in the contradiction score as the depth increases. 
%
However, there is a significant drop in the E2E F1 score as the depth increases. 
%
Therefore, it can be concluded that \textbf{the depth of the relevant laws only mildly affects the coverage and contradiction score while its ability to cite applicable laws clearly has a negative impact.}

\subsection{Performance of reasoning model in RAG-based system}
\label{subsec: reasoning_model}
% What I need to write -> Importance of reasoning models, How I set up the experiments (0 shot, temperature 1 + same setting as Proposed )
The rise of reasoning LLMs raises a question about their capability to perform legal reasoning in Thai legal QA tasks. 
%
To see the performance of such a reasoning model against a \enquote{normal} LLM, we evaluate the performance of the RAG system with reasoning LLM as a generator and compare its E2E performance to other systems described in \S\ref{subsubsec: e2e_best_setup}. 
%
Specifically, we opt to use the OpenAI's \texttt{o1-preview-2024-09-12}\cite{o1} as a generator with hierarchy-aware chunking, Human-Finetuned BGE-M3 as the retriever and not using NitiLink. 
%
Notably, the o1 model is not provided with any examples since the official reasoning guideline of OpenAI\footnote{\url{https://platform.openai.com/docs/guides/reasoning}} advise that the prompt should be kept as simple as possible. 
%
Furthermore, the API of o1 does not allow temperature changes; thus, it is set to 1.0. The experiment is conducted on the same subset used in \S\ref{subsec: setup_rq3} to limit the cost of the experiments, and the result is shown in Table~\ref{table: main_exp_reasoning}.

\begin{table}[!ht]
\centering
\renewcommand{\arraystretch}{1.2} % This increases the cell height by 1.5 times
\resizebox{\textwidth}{!}{%
\begin{tabular}{@{}lcccccccc@{}}
\toprule
\textbf{Setting} & \textbf{Retriever MRR ($\uparrow$)} & \textbf{Retriever Multi-MRR ($\uparrow$)} & \textbf{Retriever Recall ($\uparrow$)} & \textbf{Coverage ($\uparrow$)} & \textbf{Contradiction ($\downarrow$)} & \textbf{E2E Recall ($\uparrow$)} & \textbf{E2E Precision ($\uparrow$)} & \textbf{E2E F1 ($\uparrow$)} \\ 

\midrule
\multicolumn{9}{c}{\textbf{NitiBench-CCL (20\% subsampled)}} \\
\midrule
Parametric                                                    & -     & -     & -     & 60.6 & 0.198 & 0.197 & 0.147 & 0.169  \\
Naïve RAG                                                     & 0.549 & 0.549 & 0.649 & 77.7 & 0.092 & 0.740 & 0.379 & 0.501  \\
Our RAG                                                       & 0.825 & 0.825 & 0.945 & \textbf{90.1} & \textbf{0.028} & \textbf{0.920} & 0.453 & 0.607  \\
LCLM (Gemini)       & -     & -     & -     & 83.2 & 0.063 & 0.765 & \textbf{0.514} & \textbf{0.615}  \\
o1-preview-2024-09-12 & 0.825 & 0.825 & 0.945 & \underline{83.3} & \underline{0.044} & \underline{0.886} & \underline{0.466} & \underline{0.611} \\ 
\cellcolor{lightgray}Golden Context                                                & \cellcolor{lightgray}1.0   & \cellcolor{lightgray}1.0   & \cellcolor{lightgray}1.0   & \cellcolor{lightgray}94.2 & \cellcolor{lightgray}0.025 & \cellcolor{lightgray}0.999 & \cellcolor{lightgray}1.0   & \cellcolor{lightgray}0.999  \\

\midrule
\multicolumn{9}{c}{\textbf{NitiBench-Tax}} \\
\midrule
Parametric                                                    & -     & -     & -     & 46.0 & 0.480 & \textbf{0.458} & \underline{0.629} & \textbf{0.530}  \\
Naïve RAG                                                     & 0.120 & 0.048 & 0.062 & \underline{50.0} & \underline{0.460} & 0.306 & 0.463 & 0.368  \\
Our RAG                                                       & 0.574 & 0.333 & 0.499 & \textbf{51.0} & \textbf{0.440} & 0.389 & 0.554 & \textbf{0.457}  \\
LCLM (Gemini)       & -     & -     & -     & 36.0 & 0.620 & \underline{0.410} & 0.484 & 0.444  \\
o1-preview-2024-09-12 & 0.574 & 0.333 & 0.499 & 43.0 & \underline{0.460} & 0.333 & \textbf{0.640} & 0.438  \\ 
\cellcolor{lightgray}Golden Context                                                & \cellcolor{lightgray}1.0   & \cellcolor{lightgray}1.0   & \cellcolor{lightgray}1.0   & \cellcolor{lightgray}52.0 & \cellcolor{lightgray}0.460 & \cellcolor{lightgray}0.694 & \cellcolor{lightgray}1.000 & \cellcolor{lightgray}0.820  \\

\bottomrule
\end{tabular}%
}
\caption{Experiment results of reasoning model on sampled NitiBench-CCL and NitiBench-Tax. \textbf{For o1-preview, we did \emph{not} adopt few-shot prompt} to kept prompt simple according to \href{https://platform.openai.com/docs/guides/reasoning}{OpenAI guideline}.}
\label{table: main_exp_reasoning}
\end{table}


The reasoning model performs comparably to LCLM and the parametric setting on NitiBench-Tax while matching the performance of Naive RAG on NitiBench-CCL. 
%
This outcome may stem from the specialized nature of legal reasoning, which could differ from o1's capabilities. 
%
These results suggest that the current reasoning model is not yet well-suited for Thai legal QA tasks.



