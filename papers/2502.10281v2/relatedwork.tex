\section{Related work}
Although Zero Trust Architecture (ZTA) was first introduced over 10 years ago, no comprehensive, real-world implementation has emerged that fully addresses its potential. Most ZTA proposals remain in the design phase due to the complexity of their trust models and the need for substantial changes in infrastructure to accommodate them. Additionally, these solutions are often narrowly focused on specific business sectors, such as cloud computing, IoT etc, rather than providing a universal applicable framework applicable. This section will review these ZTA and security architecture proposals, highlighting the need for a more generalized, adaptable approach.

\subsection{First steps towards Zero-Trust Architecture}
TrustGuard\cite{srivatsa2005trustguard} model serves as an intermediary security layer that enforces strict access controls, monitors communication between entities, and validates interactions in real-time. Designed to reduce the risk of unauthorized access and lateral movement, TrustGuard bridges the gap between traditional network models and the zero-trust paradigm by establishing micro-boundaries of trust that are dynamically managed. It introduces a flow-level reputation-based defense mechanism and it was proposed as early as 2005 as a first step towards reputation and trust management networks. Unlike traditional methods that typically focus on IP addresses or individual packet characteristics, TrustGuard evaluates the reputation of entire network flows. Over time, it evolved in more specific uses cases such as allowing for more precise identification and mitigation of Distributed Denial of Service (DDoS) attack traffic\cite{liu2011trustguard} while reducing the incidence of false positives.

The architecture of TrustGuard encompasses several integral components. The flow collector is responsible for gathering detailed flow-level data, encompassing both traffic characteristics and behavioral patterns. The reputation manager analyzes this data to compute reputation scores for each flow, leveraging historical behavior alongside real-time observations. The decision engine then utilizes these reputation scores to make informed traffic filtering decisions, effectively distinguishing between legitimate and malicious flows. Additionally, a feedback loop continuously refines the reputation scores based on observed behaviors, enabling the system to adapt dynamically.

By incorporating machine learning techniques, TrustGuard enhances its ability to adaptively modify reputation scores in response to shifting traffic patterns and evolving attack characteristics. This combination of advanced analytics and real-time data processing positions TrustGuard as a robust solution for modern network security challenges.

\subsection{SDN and zero trust architecture}
A novel solution is presented in combining Software-defined networks(SDN) with zero-trust principles(\cite{guo2023intelligent}, \cite{zanasi2024flexible}), a security architecture designed to address the complex requirements of Industrial IoT systems, which include real-time operations, reliability, and decentralization. Traditional cybersecurity solutions struggle with the heterogeneity of IIoT devices. The proposed architectures leverages network micro-segmentation and integrates Software-Defined Networking (SDN) for policy enforcement, alongside a centralized security management layer for simplified control. A prototype demonstrates that this system ensures decentralized, resilient, and flexible security management while maintaining central oversight of security policies and network topology. One proposal \cite{zanasi2024flexible} uses Nebula, a software-defined overlay network solution, in an abstraction layer for policy enforcement. This tool relies on a custom Public Key Infrastructure (PKI) system since it uses certificates that are not X.509 compliant. 

Nebula introduces challenges in integrating with standard security frameworks and requires the development of a unique Certificate Authority (CA). Custom PKI solutions increase the complexity of managing certificate requests and key generation, already a demanding task, which may lead to security vulnerabilities. Additionally, isolating the Nebula network, while enhancing security, could introduce maintenance and scalability issues. Moreover, it is acknowledged that some devices might lack native support for Nebula and need to be integrated by introducing an additional device.

\subsection{BeyondCorp}
The "BeyondCorp"\cite{ward2014beyondcorp} model represents a paradigm shift in enterprise security, moving away from the traditional perimeter-centric approach. Developed by Google, it emphasizes user and device authentication regardless of location, allowing secure access to applications without a VPN. BeyondCorp relies on continuous verification through context-aware policies, integrating real-time monitoring and adaptive access controls to enhance security. The BeyondCorp model emphasizes secure device and user identification through a comprehensive management system. It maintains a Device Inventory Database to track managed devices, which are uniquely identified via device certificates stored in secure modules. User access is managed through a User and Group Database, integrated with HR processes, and authenticated via a Single Sign-On (SSO) system, which issues a session token for the access of a specific resource. Additionally, BeyondCorp establishes an unprivileged network that mimics an external network, enhancing security by minimizing trust in the internal network infrastructure.

As one of the few practical examples of ZTA, the model faced challenges during the later stages of Google's BeyondCorp migration, particularly regarding difficult use cases that did not fit the standard HTTPS-based workflow. Issues were signaled with specific applications that required IP-layer connectivity or could not easily integrate with the BeyondCorp access proxy\cite{gonccalves2023beyondcorp}.

\subsection{Zero trust in cloud computing}

Another discussed topic for ZTA is its use in cloud computing where services such as storage, processing power, databases, networking, software, and analytics are delivered over the internet. Thus, safeguarding such critical resources is key and the zero-trust design appears to satisfy the security requirements. For example, strategies with 9 principles of trust have been proposed but formulated just as a "conceptual model"\cite{9104214} for further research.

A novel concept presented for cloud computing is "survivable zero trust". Unlike existing models, the proposed architecture \cite{ferretti2021survivable} acknowledges that trusted components can be compromised. The novel survivable zero trust architecture ensures high security, robustness and can tolerate intrusions and recover from failures, making it suitable for cloud environments under specific conditions. The design is also based on a key pair and signature scheme that assists the communication against different attack scenarios. Even with a strong trust model, the paper recognizes that designing an effective protocol that ensures confidentiality while minimizing performance impacts and disruptions remains an open research challenge.

\subsection{Zero-trust and Blockchain}
Combining zero-trust and blockchain can enhance security in distributed systems addressing challenges such as identity management, secure data sharing, and ensuring compliance in decentralized environments. This movement materialized with projects like ZEBRA\cite{10646352}, a framework that focuses on securing Advanced Metering Infrastructure (AMI) using a Zero Trust Architecture combined with blockchain technology and Ring Oscillator Physical Unclonable Functions (ROPUFs). The design ensures robust device authentication and guarantees data integrity by leveraging the unique properties of ROPUFs, for generating unclonable keys, and blockchain for traceable and tamper-proof communication. This approach enhances the security of smart grid networks, providing resilience against cyber threats like unauthorized access, spoofing and data manipulation. Nevertheless, blockchain technology can introduce latency and require significant computational resources, which may be challenging for the resource-constrained devices used in AMI. Additionally, the reliance on ROPUFs for authentication, while secure, could be affected by environmental factors (such as temperature or voltage variations), impacting the reliability of the cryptographic keys generated. Managing these factors while maintaining system performance could pose challenges for real-world deployment.

Another solution, this time tailored for IoT is Amatista\cite{8473444}, a blockchain-based middleware designed for scalable management of IoT networks. The paper is the first to enumerate cryptography as an option in trust management but it incorporates it in the blockchain consensus algorithm. As IoT expands rapidly, the trustworthiness of millions of connected devices becomes a challenge. Amatista tackles this issue by employing a zero-trust approach, utilizing a novel hierarchical mining process to validate both the infrastructure and transactions at varying levels of trust. By leveraging blockchain features such as a distributed database, consensus mechanisms, smart contracts, and immutability, Amatista ensures reliable transactions without centralized validation nodes. The system is tested on Edison Arduino Boards, demonstrating how it can address trust concerns in IoT through decentralized validation mechanisms. While Amatista shows promise, potential issues include reliance on complex blockchain infrastructure, scalability challenges with numerous devices and, as stated by the authors, not yet tested "in a large scale loT deployment".