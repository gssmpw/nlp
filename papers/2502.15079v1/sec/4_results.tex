\section{Experimental Setup}
\label{sec:exp_setup}

\looseness=-1
\paragraph{Data (detailed in \autoref{app:datasets}).}

We train \method{} using videos and their ground-truth and contrastive descriptions from VideoCon \cite{bansal2024videocon}, generating 115,536 (video, description, correction) triplets for training and 8,312 for validation, which is used for model selection. Synthetic contrast captions are also used to fine-tune the \textit{baseline} entailment task with the same dataset sizes.

We evaluate our trained models on text-to-video retrieval using the temporally-challenging SSv2-Temporal \cite{sevilla2021only} and action-intensive SSv2-Events \cite{bagad2023test}  datasets.
Additionally, we evaluate our models on 
compositional ability over time using the VELOCITI benchmark \cite{saravanan2024velociti}. Each video in the dataset includes a correct caption and an incorrect one.


\paragraph{Baselines.}
(i) \textbf{Pretrained Video-LLMs}:
we employ two pre-trained models with different architectures, \textit{Video-LLaVA}~\cite{lin2023video} 
and
\textit{VideoChat2}~\cite{li2023videochat}. 
More details in \autoref{app:pretrained_vlm}.
(ii) \textbf{Entailment}:
we fine-tune the pretrained Video-LLMs using the entailment task described in \autoref{sec:prelim}. 
More details about the implementation are in \autoref{app:implementation}.


\paragraph{Evaluation metrics.}

We report the accuracy on the VELOCITI benchmark as the proportion of examples in which the positive video-caption pair receives a higher \textpr{Yes} entailment probability than the corresponding negative video-caption pair.
For SSv2, we compute \textpr{Yes} probabilities for each text-video pair, rank their scores, and report mean Average Precision (mAP).


\begin{figure}[t!]
\centering
\includegraphics[width=0.45\textwidth]{images/ssv2.pdf}
\vspace{-0.3cm}
\caption{Mean Average Precision (mAP) scores for pretrained Video-LLaVA and models fine-tuned using various methods on zero-shot text-to-video retrieval tasks. }
\label{fig:ssv2}
\vspace{-0.2cm}
\end{figure}


\begin{table}[t]
% \begin{wraptable}{r}{0.72\linewidth}
\centering
% \tabcolsep=0.10cm
% \vspace{-3mm}
%\label{table:velociti_main}
\begin{adjustbox}{width=\columnwidth}
\begin{tabular}{@{}l@{~~}c@{ }c@{ }c@{~~}c@{ }c@{ }c@{~~}c@{~~}>{\columncolor{gray!10}}c@{}}
%old version: \begin{tabular}{lccccccc>{\columncolor{gray!10}}c}
\toprule
 & \multicolumn{3}{c}{\textbf{Agent Tests}}               & \multicolumn{3}{c}{\textbf{Action Tests}} & & \\ % \textbf{Chrono} & \textbf{Avg} \\
\textbf{Model} & Iden~~          & Bind~          & Coref         & Adv           & Bind~          & Modif         &  \textbf{Chrono}                       &    \textbf{Avg}                  \\
\midrule
Random          & 50.0          & 50.0          & 50.0          & 50.0          & 50.0          & 50.0          & 50.0 & 50.0          \\
Human           & 94.7          & 93.3          & 96.0          & 100.0         & 92.7          & 91.3          & 93.3 & 94.4          \\
% Vera            & 58.4          & 53.3          & 63.6          & 66.4          & 52.8          & 57.0          & 52.0 & 57.7          \\
% \midrule
% Gemini          & 91.8          & 76.4          & 67.8          & 80.0          & 76.4          & 76.9          & 68.3 & 76.8          \\
% \midrule
% \textcolor{gray}{CLIP {\scriptsize B/32} }  & \textcolor{gray}{77.6 } & \textcolor{gray}{56.3} & \textcolor{gray}{52.6} & \textcolor{gray}{64.0} & \textcolor{gray}{57.6} & \textcolor{gray}{52.1} & \textcolor{gray}{49.4} & \textcolor{gray}{58.5} \\
% \textcolor{gray}{CLIP {\scriptsize L/14}} & \textcolor{gray}{82.6} & \textcolor{gray}{55.4} & \textcolor{gray}{56.9} & \textcolor{gray}{66.2} & \textcolor{gray}{58.0} & \textcolor{gray}{56.8} & \textcolor{gray}{50.2} & \textcolor{gray}{60.9} \\
% \textcolor{gray}{EVA-CLIP {\scriptsize L/14}} & \textcolor{gray}{83.3} & \textcolor{gray}{53.4} & \textcolor{gray}{55.0} & \textcolor{gray}{70.2} & \textcolor{gray}{55.3} & \textcolor{gray}{51.2} & \textcolor{gray}{51.1} & \textcolor{gray}{59.9} \\
% \textcolor{gray}{SigLIP {\scriptsize B/16}} & \textcolor{gray}{80.0} & \textcolor{gray}{54.4} & \textcolor{gray}{51.0} & \textcolor{gray}{63.8} & \textcolor{gray}{54.5} & \textcolor{gray}{61.1} & \textcolor{gray}{49.8} & \textcolor{gray}{59.2} \\
% \textcolor{gray}{SigLIP {\scriptsize L/16}} & \textcolor{gray}{78.8} & \textcolor{gray}{53.3} & \textcolor{gray}{52.2} & \textcolor{gray}{61.6} & \textcolor{gray}{52.4} & \textcolor{gray}{57.0} & \textcolor{gray}{49.1} & \textcolor{gray}{57.8} \\
\midrule
% \textcolor{gray}{NegCLIP {\scriptsize B/32}} & \textcolor{gray}{\textbf{83.4}} & \textcolor{gray}{55.6} & \textcolor{gray}{50.5} & \textcolor{gray}{61.8} & \textcolor{gray}{52.3} & \textcolor{gray}{61.1} & \textcolor{gray}{51.2} & \textcolor{gray}{59.4} \\
\textcolor{gray}{CLIP-ViP {\scriptsize B/32}} & \textcolor{gray}{75.3} & \textcolor{gray}{52.4} & \textcolor{gray}{55.7} & \textcolor{gray}{70.2} & \textcolor{gray}{53.5} & \textcolor{gray}{51.2} & \textcolor{gray}{48.5} & \textcolor{gray}{58.1} \\
\textcolor{gray}{ViFi-CLIP {\scriptsize B/16}} & \textcolor{gray}{82.3} & \textcolor{gray}{58.7} & \textcolor{gray}{54.6} & \textcolor{gray}{63.0} & \textcolor{gray}{59.3} & \textcolor{gray}{60.5} & \textcolor{gray}{49.8} & \textcolor{gray}{61.2} \\
\midrule
mPLUG-V & 43.0 & 31.9 & 51.7 & 65.0 & 42.0 & 49.6 & 41.3 & 46.3 \\
PLLaVA & 68.6 & 43.3 & {60.5} & 62.4 & 46.6 & 56.0 & 49.6 & 55.3 \\
VideoCon & 67.4 & 44.6 & 50.0 & 73.0 & 51.1 & 63.2 & 45.6 & 56.4 \\

% Owl-Con & 67.4 & 44.6 & 50.0 & \textbf{73.0} &  51.1 & 63.2 & 45.6 & 56.4 \\
% \multicolumn{9}{l}{Video-LLaVA} \\
\arrayrulecolor{gray!50}
\hdashline
\rowcolor[HTML]{F2F3F4} 
Video-LLaVA & 74.1 & 50.4 & \textbf{60.1} & 63.6 & 47.0 & 47.9 & 56.0 & 57.0  \\
\hdashline
% \midrule
\arrayrulecolor{black}
% \rowcolor[HTML]{F2F3F4} 
$+$ Entail & 73.7  & 59.7  & 55.5  & 68.4  & 57.3  & 64.0  & 57.3  & 62.3 \\
% $+$ Mask & 47.8 & 44.5 & 57.9 & 53.0 & 48.9 & \textbf{69.6} & 53.5 & 53.6 \\
% $+$ Entail+Mask &  78.9 & 60.0 & 55.3 & \textbf{72.8} & 61.2 & 63.2 & \textbf{60.3} & 64.5 \\
% $+$ Entail+Mask &  78.9 & 60.0 & 55.3 & \textbf{72.8} & 61.2 & 63.2 & \textbf{60.3} & 64.5 \\
$+$ \method{} & 80.3 & \textbf{62.6} & 57.9 & \textbf{72.6} & \textbf{60.0} & \textbf{65.8} & 54.5 & 64.8 \\
$+$ \method{}+Mask & \textbf{82.7} & 62.1 & 57.9 & 71.8 & 59.0 & 64.8 & \textbf{57.9} & \textbf{65.2} \\
% $+$ Describe $\to$ entail  & 77.8  & 57.4  & \textbf{60.1}  & 63.8  & 55.4  & 55.1  & 56.1 & 60.8 \\
% \rowcolor[HTML]{F2F3F4} 
% \rowcolor[HTML]{F2F3F4} 
% $+$ Entail+Mask  & \textbf{81.4}  & \textbf{60.3}  & 54.3  & \textbf{70.0}  & \textbf{58.0}  & \textbf{65.4}  & \textbf{59.5}  & \textbf{64.1} \\
\arrayrulecolor{gray!50}
\hdashline
\rowcolor[HTML]{F2F3F4} 
VideoChat2         & 76.8 & 54.4 & 53.1 & 56.0 & 46.2 & 59.3 & \textbf{54.7} & 57.2 \\
\hdashline
\arrayrulecolor{black}
$+$ Entail         & 59.7 & 56.9 & 55.5 & 62.2 & 53.0 & 50.1 & 53.9 & 55.9\\
% $+$ Mask           & 45.7 & 39.9 & 51.7 & 43.6 & 40.5 & \underline{67.9} & 52.8 & 48.9\\
% $+$ Entail+Mask   & 61.1 & 59.5 & \textbf{60.5} & 60.0 & \underline{56.2} & 54.9 & 53.8 & 58.0\\
$+$ \method{}      & 77.2 & \textbf{60.4} & 56.4 & 65.8 & \textbf{55.0} & 61.7 & 53.7 & 61.5\\
$+$ \method{}+Mask & \textbf{79.1} & 59.7 & \textbf{56.9} & \textbf{68.2} & 54.6 & \textbf{66.9} & 51.1 & \textbf{62.4} \\
% $+$ \method{}+Mask$_2$ & 59.2 & 55.3 & 68.6 & 62.6 & 56.2 & 61.7 & 68.2 & 61.7 \\
\bottomrule
\end{tabular}
\end{adjustbox}
% \vspace{-1mm}
\caption{
%Accuracy of models on VELOCITI. We show the results of each model trained with the baseline entailment task and our proposed HACA objective. We include results on contrastive models (CLIP-ViP~\cite{xue2023clipvip}, ViFi-CLIP~\cite{rasheed2023fine}), and generative models (mPLUG-V~\cite{ye2023mplug}, PLLaVA~\cite{Xu2024PLLaVAP}, VideoCon) as reported in VELOCITI.
%Model zero-shot accuracy on VELOCITI: We present results for models trained with the baseline entailment task and our proposed \method
%~objective. This includes contrastive models: CLIP-ViP~\cite{xue2023clipvip}, ViFi-CLIP~\cite{rasheed2023fine}; and generative models: mPLUG-V~\cite{ye2023mplug}, PLLaVA~\cite{Xu2024PLLaVAP}.
Zero-shot accuracy on VELOCITI for models trained with the baseline entailment task, our proposed \method~objective, and other contrastive (CLIP-ViP~\cite{xue2023clipvip}, ViFi-CLIP~\cite{rasheed2023fine}) and generative (mPLUG-V~\cite{ye2023mplug}, PLLaVA~\cite{Xu2024PLLaVAP}) models.
%and VideoCon (as reported in VELOCITI).
% \lj{how to add reference to pllava and mplug-v? i advocate to add contrastive models e.g. videoclip to the table now as it's discussed in related work, as least put in appendix :D}
}
\label{table:velociti}
\vspace{-0.3cm}
\end{table}
% \end{wraptable}


% \begin{table*}[t]
% % \begin{wraptable}{r}{0.72\linewidth}
% \centering
% % \tabcolsep=0.10cm
% % \vspace{-3mm}
% \begin{adjustbox}{max width=\textwidth}
% \label{table:velociti_main}
% \begin{tabular}{l|ccccccc>{\columncolor{gray!10}}c|ccccc>{\columncolor{gray!10}}c|ccccc>{\columncolor{gray!10}}c}
% \toprule
% % \multirow{2}{*}{Model}
% Model & \multicolumn{8}{c}{\textbf{VELOCITI}} & \multicolumn{6}{|c}{\textbf{MVBench}} & \multicolumn{5}{|c}{\textbf{TVBench}} \\
% \midrule 
%  & \multicolumn{3}{c}{\textbf{Agent Tests}} & \multicolumn{3}{c}{\textbf{Action Tests}} & \textbf{Chron} & \textbf{Avg} &  %\multirow{2}{*}{\textbf{Avg}}
% \multirow{2}{*}{\textbf{AS}} &  \multirow{2}{*}{\textbf{AP}} &  \multirow{2}{*}{\textbf{AA}} &  \multirow{2}{*}{\textbf{FA}} &  \multirow{2}{*}{\textbf{UA}} &  \textbf{Avg} &  \multirow{2}{*}{\textbf{AC}} &  \multirow{2}{*}{\textbf{AS}} &  \multirow{2}{*}{\textbf{OS}} &  \multirow{2}{*}{\textbf{ST}} &  \multirow{2}{*}{\textbf{MD}} & \textbf{Avg} \\
% & Iden & Bind & Coref & Adv & Bind & Modif & & &
%  &   &   &   &   &   &   &   &   &   &   &   \\
% \midrule 
% Random & 50.0 & 50.0 & 50.0 & 50.0 & 50.0 & 50.0 & 50.0 & 50.0 &  
% -- &  -- &  -- &  -- &  -- &  -- &  -- &  -- &  -- &  -- &  -- &  -- \\
% Human & 94.7 & 93.3 & 96.0 & 100.0 & 92.7 & 91.3 & 93.3 & 94.4 &  
% -- &  -- &  -- &  -- &  -- &  -- &  -- &  -- &  -- &  -- &  -- &  -- \\
% % Vera            & 58.4          & 53.3          & 63.6          & 66.4          & 52.8          & 57.0          & 52.0 & 57.7          \\
% % \midrule
% % Gemini          & 91.8          & 76.4          & 67.8          & 80.0          & 76.4          & 76.9          & 68.3 & 76.8          \\
% \midrule 
% \textcolor{gray}{CLIP {\scriptsize B/32} }  & \textcolor{gray}{77.6 } & \textcolor{gray}{56.3} & \textcolor{gray}{52.6} & \textcolor{gray}{64.0} & \textcolor{gray}{57.6} & \textcolor{gray}{52.1} & \textcolor{gray}{49.4} & \textcolor{gray}{58.5} 
% &  -- &  -- &  -- &  -- &  -- &  -- &  -- &  -- &  -- &  -- &  -- &  -- \\
% \textcolor{gray}{CLIP {\scriptsize L/14}} & \textcolor{gray}{82.6} & \textcolor{gray}{55.4} & \textcolor{gray}{56.9} & \textcolor{gray}{66.2} & \textcolor{gray}{58.0} & \textcolor{gray}{56.8} & \textcolor{gray}{50.2} & \textcolor{gray}{60.9} 
% &  -- &  -- &  -- &  -- &  -- &  -- &  -- &  -- &  -- &  -- &  -- &  -- \\
% \textcolor{gray}{EVA-CLIP {\scriptsize L/14}} & \textcolor{gray}{83.3} & \textcolor{gray}{53.4} & \textcolor{gray}{55.0} & \textcolor{gray}{70.2} & \textcolor{gray}{55.3} & \textcolor{gray}{51.2} & \textcolor{gray}{51.1} & \textcolor{gray}{59.9} 
% &  -- &  -- &  -- &  -- &  -- &  -- &  -- &  -- &  -- &  -- &  -- &  -- \\
% \textcolor{gray}{SigLIP {\scriptsize B/16}} & \textcolor{gray}{80.0} & \textcolor{gray}{54.4} & \textcolor{gray}{51.0} & \textcolor{gray}{63.8} & \textcolor{gray}{54.5} & \textcolor{gray}{61.1} & \textcolor{gray}{49.8} & \textcolor{gray}{59.2} 
% &  -- &  -- &  -- &  -- &  -- &  -- &  -- &  -- &  -- &  -- &  -- &  -- \\
% \textcolor{gray}{SigLIP {\scriptsize L/16}} & \textcolor{gray}{78.8} & \textcolor{gray}{53.3} & \textcolor{gray}{52.2} & \textcolor{gray}{61.6} & \textcolor{gray}{52.4} & \textcolor{gray}{57.0} & \textcolor{gray}{49.1} & \textcolor{gray}{57.8} 
% &  -- &  -- &  -- &  -- &  -- &  -- &  -- &  -- &  -- &  -- &  -- &  -- \\
% \midrule 
% \textcolor{gray}{NegCLIP {\scriptsize B/32}} & \textcolor{gray}{\textbf{83.4}} & \textcolor{gray}{55.6} & \textcolor{gray}{50.5} & \textcolor{gray}{61.8} & \textcolor{gray}{52.3} & \textcolor{gray}{61.1} & \textcolor{gray}{51.2} & \textcolor{gray}{59.4} 
% &  -- &  -- &  -- &  -- &  -- &  -- &  -- &  -- &  -- &  -- &  -- &  -- \\
% \textcolor{gray}{CLIP-ViP {\scriptsize B/32}} & \textcolor{gray}{75.3} & \textcolor{gray}{52.4} & \textcolor{gray}{55.7} & \textcolor{gray}{70.2} & \textcolor{gray}{53.5} & \textcolor{gray}{51.2} & \textcolor{gray}{48.5} & \textcolor{gray}{58.1} &  
% -- &  -- &  -- &  -- &  -- &  -- &  -- &  -- &  -- &  -- &  -- &  -- \\
% \textcolor{gray}{ViFi-CLIP {\scriptsize B/16}} & \textcolor{gray}{82.3} & \textcolor{gray}{58.7} & \textcolor{gray}{54.6} & \textcolor{gray}{63.0} & \textcolor{gray}{\textbf{59.3}} & \textcolor{gray}{60.5} & \textcolor{gray}{49.8} & \textcolor{gray}{61.2} &  -- &  -- &  -- &  -- &  -- &  -- &  -- &  -- &  -- &  -- &  -- &  -- \\
% \midrule 
% mPLUG-V & 43.0 & 31.9 & 51.7 & 65.0 & 42.0 & 49.6 & 41.3 & 46.3 &  &  &  &  &  &  &  &  &  &  &  & \\
% PLLaVA & 68.6 & 43.3 & 60.5 & 62.4 & 46.6 & 56.0 & 49.6 & 55.3 &  -- &  -- &  -- &  -- &  -- &  -- &  -- &  -- &  -- &  -- &  -- &  -- \\
% Video-LLaVA & 74.1 & 50.4 & 60.1 & 63.6 & 47.0 & 47.9 & 56.0 & 57.0  &  -- &  -- &  -- &  -- &  -- &  -- &  -- &  -- &  -- &  -- &  -- &  -- \\
% Owl-Con & 67.4 & 44.6 & 50.0 & 73.0 &  51.1 & 63.2 & 45.6 & 56.4 &  -- &  -- &  -- &  -- &  -- &  -- &  -- &  -- &  -- &  -- &  -- &  -- \\
% \midrule 
% % \rowcolor[HTML]{F2F3F4} 
% % \multicolumn{21}{l}{Video-LLaVA} \\
% Video-LLaVA & & & & & & & & & & & & & & & & & & & & \\ 
% % \rowcolor[HTML]{F2F3F4} 
% % $+$ Desc $\to$ Ent  & 77.8  & 57.4  & \textbf{60.1}  & 63.8  & 55.4  & 55.1  & 56.1 & 60.8 &  
% % -- &  -- &  -- &  -- &  -- &  -- &  -- &  -- &  -- &  -- &  -- &  -- \\
% % \rowcolor[HTML]{F2F3F4} 
% $+$ Entail  & 74.2  & 59.8  & 59.1  & 66.4  & 55.6  & 65.4  & 55.2  & 62.2 &  
% -- &  -- &  -- &  -- &  -- &  -- &  -- &  -- &  -- &  -- &  -- &  -- \\
% % \rowcolor[HTML]{F2F3F4} 
% $+$ Entail+Mask  & \textbf{81.4}  & \textbf{60.3}  & 54.3  & \textbf{70.0}  & \textbf{58.0}  & \textbf{65.4}  & \textbf{59.5}  & \textbf{64.1} & 
% -- &  -- &  -- &  -- &  -- &  -- &  -- &  -- &  -- &  -- &  -- &  -- \\
% \bottomrule
% \end{tabular}
% \end{adjustbox}
% % \vspace{-3mm}
% \caption{Performance on VELOCITI, MVBench and TVBench.}
% \end{table*}
% % \end{wraptable}


