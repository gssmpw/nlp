%----------------------------------------------------------------------------------------
%
% Financements
%
%----------------------------------------------------------------------------------------
\addchap{Financing}
\label{financements}

The DELEGATION project, "DEep LEarning for Generating humAn moTION," is 
focused on developing a sophisticated deep learning framework for generating 
expressive, skeleton-based human motions. This system addresses complex 
challenges in motion analysis, such as noise and variability in human movement 
data, and aims to create realistic, controllable motion sequences. Its applications 
span areas like physical rehabilitation, where accurate motion representation is essential.

This thesis is funded under the DELEGATION project, which is supported by the 
Agence Nationale de la Recherche (ANR) under grant number ANR-21-CE23-0014. 
The financial support from ANR facilitates in-depth research into cutting-edge 
techniques in human motion generation and analysis, allowing for significant 
contributions to this emerging field.

In addition to its innovative goals, the DELEGATION project involves a strong 
collaboration between several key partners. The project is coordinated by 
Dr. Maxime Devanne from the Institut de Recherche en Informatique, 
Mathématiques, Automatique et Signal (IRIMAS) at Université de Haute-Alsace in Mulhouse, France, 
alongside Prof. Germain Forestier and Dr. Jonathan Weber. Other partners 
include the Media Integration and Communication Center (MICC) at University of Florence in Italy, 
led by Dr. Stefano Berretti as the local PI, the CHRU of Brest, France, with Prof. Olivier 
Remy-Neris as the local PI, and the Centre de Réadaptation de Mulhouse
(CRM) in Mulhouse, France, with Fabienne Ernst Kuteifan as the local PI.

A portion of the DELEGATION grant was dedicated to facilitating an academic 
visit to Dr. Stefano Berretti at the Media Integration and Communication Center (MICC) 
in Florence, Italy. I had the opportunity to spend two weeks there in June 2024, 
where I gained valuable expertise and discussed with other PhD students involved in similar 
projects. The visit was highly beneficial, fostering new ideas and approaches that 
enriched my research, while strengthening international collaboration 
within the project's framework.