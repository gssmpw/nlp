%----------------------------------------------------------------------------------------
%
% Introduction
%
%----------------------------------------------------------------------------------------
\addchap{Introduction}
\label{introduction}

Time series refers to sequential data where data points are equally spaced in time, 
with each point corresponding to a specific timestamp. Time series data is prevalent 
across a wide range of applications, including medical fields such as electrocardiograms 
(ECGs)~\cite{ecg-example-paper} and electroencephalograms (EEGs)~\cite{eeg-example-paper},
human motion~\cite{human-motion-example-paper}, stock
market trends~\cite{stock-exchange-example-paper}, and 
telecommunications signals~\cite{telecom-example-paper} between
base stations and users, among others.
The term ``time'' in time series does not imply that only temporal ordering is relevant; 
rather, any data with a necessary sequence or order can be treated similarly to temporal 
data. For instance, in applications like image contour
extraction~\cite{image-countour-shapelets-paper}, the data has an inherent 
order that, if disrupted, such as by shuffling the image into a jigsaw puzzle, would result 
in a loss of meaningful information. Therefore, this data is treated similarly to
time series data.   
While time series data can, to some extent, be represented in a tabular format, where 
each row corresponds to a time series sample and each column represents a variable at 
a particular timestamp, its analysis cannot be adequately performed using standard tabular 
data analysis methods. Traditional tabular methods fail to consider the sequential nature 
of the data, focusing only on relationships between variables, rather than the critical 
ordering of those variables.

In 2006, time series analysis was recognized as one of the top 10 challenges in the field 
of data mining~\cite{data-mining-challenges-paper}. This area of study involves a 
broad spectrum of tasks that are crucial for understanding temporal data patterns. 
These tasks can be effectively tackled using a variety of approaches, including 
traditional statistical methods, modern machine learning techniques, and advanced 
deep learning models.

\textbf{Forecasting} is a specialized regression task in the domain of time series data, 
aiming at predicting future segments of the input series by utilizing the temporal patterns 
present in the data~\cite{monash-forecasting}.
This task is essential to a wide range of applications, such as weather 
forecasting~\cite{weather-forecasting} and stock market 
prediction~\cite{stock-exchange-example-paper}.
% MDC: integral? what does it mean? HFC: Replaced by essential

\begin{figure}
    \centering
    \caption{Time Series Extrinsic Regression (TSER) is the task of predicting continuous 
    labels of the time series samples.}
    \label{fig:task-tser}
    \includegraphics[width=\textwidth]{Figures/intro/tasks/tser.pdf}
\end{figure}

\textbf{Extrinsic regression}~\cite{tser-archive}, presented in 
Figure~\ref{fig:task-tser} differs significantly from 
time series forecasting. In this task, the goal is to predict a continuous value 
that is not a future point in the input series, but rather a value generated by 
a random variable that depends on the entire time series, including its trends and values.
This type of regression is commonly used in applications such as 
live fuel moisture content estimation~\cite{tser-live-fuel-moisture,deep-live-fuel} 
and human rehabilitation motion 
assessment~\cite{kimore-paper}.

\textbf{Anomaly detection}~\cite{anomaly-detection-review} in time series 
focuses on identifying data points or patterns that deviate from the expected norm. 
This task is essential for spotting unusual events that may indicate issues like system 
failures or fraud. It is widely applied in areas such as network
security monitoring~\cite{network-anomaly-detection} 
and healthcare diagnostics~\cite{healthcare-anomaly-detection}, where early
detection of anomalies can prevent significant problems.

\begin{figure}
    \centering
    \caption{Time Series CLustering (TSCL) is the task of discovering common information 
    between samples of time series in order to group them into clusters.}
    \label{fig:task-tscl}
    \includegraphics[width=\textwidth]{Figures/intro/tasks/tscl.pdf}
\end{figure}

\textbf{Clustering}~\cite{deep-tscl-bakeoff}, presented in Figure~\ref{fig:task-tscl},
% is a well-known data mining challenge, 
% particularly in the context of time series analysis. The task of time series clustering 
involves identifying and extracting patterns within the input series to categorize them 
into distinct groups, which are defined by the nature of the data. This approach is 
applied in various fields, such as detecting daily patterns in stock market 
data~\cite{timeseries-clustering-application} and identifying specific 
behaviors in solar magnetic wind~\cite{wind-clustering}.

\begin{figure}
    \centering
    \caption{Time Series Prototyping (TSP) is the task of finding a representative 
    of a collection of time series of a similar group.}
    \label{fig:task-tsp}
    \includegraphics[width=\textwidth]{Figures/intro/tasks/tsp.pdf}
\end{figure}

\textbf{Prototyping}~\cite{dba-paper}, presented in Figure~\ref{fig:task-tsp}, is 
the process of identifying a representative 
time series from a group of similar series. This task is particularly useful in time 
series clustering, as it helps to summarize and simplify the data by selecting a central 
or typical example from each cluster.
In healthcare, time series prototyping can be valuable for creating a summarized exemplar 
of patients with common health conditions, thereby facilitating comparisons with new
patients~\cite{time-series-snippets}.
% MDC: do you have a ref for the last sentence? HFC: Added an Eamonn paper in 2019

\begin{figure}
    \centering
    \caption{Time Series Classification (TSC) is the task of predicting 
    a discrete label of the time series samples.}
    \label{fig:task-tsc}
    \includegraphics[width=\textwidth]{Figures/intro/tasks/tsc.pdf}
\end{figure}

\textbf{Classification}~\cite{bakeoff-tsc-2}, is a discretized version 
of extrinsic regression as presented in Figure~\ref{fig:task-tsc}, where
the goal is to predict a discrete class label for each time series with prior knowledge of 
the number of possible classes, which distinguishes it from clustering. This task is 
widely applied in various fields, including human activity 
recognition~\cite{human-motion-example-paper} and medical diagnostics, such as 
classifying heart diseases from ECG signals~\cite{ecg-example-paper}.

In this thesis, we mainly focus on four tasks for time series data: classification, clustering,
extrinsic regression and prototyping.
These tasks are particularly relevant to the application of human motion analysis, where the 
input time series consists of sequences of recorded kinematic skeleton data at each time step.
% MDC: last sentence is not very clear. HFC: Fixed
Such data can be used for classification tasks to predict the activity of a subject and 
for extrinsic regression to predict a continuous value associated to rehabilitation motion to assess 
patients' performance.
More details on this application and the contributions in such domain are presented in 
Chapter~\ref{chapitre_6}.

\begin{figure}
    \centering
    \caption{The number of research papers mentioning ``deep learning'' 
    and ``time series classification''
    increased rapidly in the last years.}
    \label{fig:intro-trend-dl4tsc}
    \includegraphics[width=\textwidth]{Figures/intro/dl4tsc-papers-count.pdf}
\end{figure}

In order to address the above mentioned applications, we addressed the approached 
perspective of time series analysis, starting with a study of
traditional methods that have been used for years to solve Time Series Classification (TSC).
However, it was shown in the first 2017 TSC bake-off~\cite{bakeoff-tsc-1} that using 
the traditional techniques does not achieve state-of-the-art performance, instead the authors found
that hybrid approaches work much better.
Moreover, the domain of TSC was then extended by different methods proposed in between, ranging from convolution
methods~\cite{rocket} to bag-of-words methods~\cite{weasel2}.
After the publication of the 2017 bake-off, researchers began to question the role of deep learning 
models in this domain, especially given their significant performance in image 
classification~\cite{lecun2015deep,alexnet-paper}. The number of related papers in deep learning for TSC started to 
increase rapidly, leading to the 2019 deep learning for TSC review~\cite{dl4tsc}. The 2019 
review demonstrated that the best deep learning model achieved performance comparable to 
the state-of-the-art non-deep learning model.
\citet{dl4tsc} paved the way for new research in deep learning for time series classification (TSC).
This trend, illustrated in Figure~\ref{fig:intro-trend-dl4tsc},
also extends to addressing other tasks within the time series domain.
% MDC: the last sentence is not very clear, maybe try to split it into two? HFC: fixed
For instance, deep learning methods for Time Series Clustering (TSCL) was reviewed in~\cite{deep-tscl-bakeoff}
showcasing that deep TSCL methods can outperform traditional elastic methods or shape based methods~\cite{kshape-paper}.
Additionally, the usage of deep learning emerged for the task of Time Series Prototyping (TSP)
with the usage of multi-task Auto-Encoders~\cite{deep-averaging-paper}.
Finally,~\citet{deep-tsc-tser} showed that numerous research papers are addressing 
the task of deep learning for Time Series Extrinsic Regression (TSER).
A significant amount of literature work on these topics are presented in Chapter~\ref{chapitre_1}.

Given the growing interest and proven effectiveness of deep learning in time series analysis, we 
employ
% MDC: should be consistent about the employed tense. Sometimes it is present, sometime futire and sometimes past HFC: fixed thx
this approach to tackle the four tasks of TSC, TSCL, TSP, and TSER. Deep learning's ability 
to capture complex patterns and dependencies in sequential data makes it well-suited for addressing 
these challenges.
However, before presenting any contribution in this thesis in these four tasks
% MDC: domains or tasks? HFC: fixed
, we start 
by addressing the evaluation of discriminative models.
The current evaluation framework, even though having its own advantages,
presents some limitations that are beneficial to any research work that want to 
manipulate the ``view'' of the results to make it seem better than other approaches.
For this reason, we propose, not a replacement, but a complimentary tool
for such an evaluation framework, that we present in Chapter~\ref{chapitre_2}.

After establishing the evaluation framework and recognizing the growing interest in developing  
foundation models for time series data~\cite{foundation-models-example-paper}, we introduce 
in Chapter~\ref{chapitre_3} two key
contributions. These 
contributions converge to form 
a novel approach aiming at defining such a foundation model, specifically tailored 
for the task of TSC. Chapter~\ref{chapitre_3} not only outlines the foundation model but also 
explains how it tackles these unique challenges through the engineering of 
hand-crafted features, paving the way for more robust and generalized models in this domain.

Focusing on the carbon footprint of such complex deep learning models for time series data,
we propose in Chapter~\ref{chapitre_4} to reduce model complexity while keeping the performance 
statistically non significantly worse than the state-of-the-art.
This is done by the proposal of LITE,
the smallest deep learning model for time series data found in the 
literature, that is proven to be very effective in the 
second TSC bake-off~\cite{bakeoff-tsc-2}.

Moreover, acquiring labeled data in time series can be challenging. To address this, 
in Chapter~\ref{chapitre_5}, we propose an unsupervised framework
designed to handle 
situations where only a limited number of labeled samples are available. This 
unsupervised framework, based on representation learning, can be applied to various 
downstream tasks involving time series data.

\begin{table}[h!]
    \hspace*{-1.6cm}
    \centering
    \begin{tabular}{|c|c|c|c|}
    \hline
    \textbf{Contribution}  & \textbf{Task} & \textbf{Chap.} & \textbf{GitHub} \\ \hline
    \cite{mcm-paper}  & Evaluation & \ref{chapitre_2} & \href{https://github.com/MSD-IRIMAS/Multi_Comparison_Matrix}{Multi\_Comparison\_Matrix} \\ \hline
    \cite{hand-crafted-filters-paper}  & Classification & \ref{chapitre_3} & \href{https://github.com/MSD-IRIMAS/CF-4-TSC}{CF-4-TSC} \\ \hline
    \cite{pretext-task-paper}  & Classification & \ref{chapitre_3} & \href{https://github.com/MSD-IRIMAS/DomainFoundationModelsTSC}{DomainFoundationModelsTSC} \\ \hline
    \cite{lite-paper}  & Classification & \ref{chapitre_4} & \href{https://github.com/MSD-IRIMAS/LITE}{LITE} \\ \hline
    \cite{lite-extension-paper}  & Classification & \ref{chapitre_4} & \href{https://github.com/MSD-IRIMAS/LITE}{LITE} \\ \hline
    \cite{trilite-paper}  & \begin{tabular}{c}Self-Supervised/ \\ Classification \end{tabular} & \ref{chapitre_5} & \href{https://github.com/MSD-IRIMAS/TRILITE}{TRILITE} \\ \hline
    \cite{shape-dba-paper}  & Prototyping & \ref{chapitre_6} & \href{https://github.com/MSD-IRIMAS/ShapeDBA}{ShapeDBA} \\ \hline
    \cite{weighted-shape-dba-paper}  & \begin{tabular}{c}Prototyping/\\Regression\end{tabular} & \ref{chapitre_6} & \href{https://github.com/MSD-IRIMAS/Weighted-ShapeDBA-4-Rehab}{Weighted-ShapeDBA-4-Rehab} \\ \hline
    \cite{svae-paper}  & \begin{tabular}{c}Generation/\\Classification \end{tabular} & \ref{chapitre_6} & \href{https://github.com/MSD-IRIMAS/SVAE-4-HMG}{SVAE-4-HMG} \\ \hline
    \cite{metrics-paper}  & \begin{tabular}{c}Generation/\\Evaluation\end{tabular} & \ref{chapitre_7} & \href{https://github.com/MSD-IRIMAS/Evaluating-HMG}{Evaluating-HMG} \\ \hline
    \cite{aeon-paper}  & All & \ref{chapitre_8} & \href{https://github.com/aeon-toolkit/aeon}{aeon} \\ \hline
    \cite{ismail-fawaz2024elastic-vis}  & Visualization & \ref{chapitre_8} & \href{https://github.com/MSD-IRIMAS/Elastic_Warping_Vis}{Elastic\_Warping\_Vis} \\ \hline
    \cite{Ismail-Fawaz2023weighted-ba}  & Prototyping & \ref{chapitre_8} & \href{https://github.com/MSD-IRIMAS/Augmenting-TSC-Elastic-Averaging}{Augmenting-TSC-Elastic-Averaging} \\ \hline
    \cite{Ismail-Fawaz2023kan-c22-4-tsc}  & Classification & \ref{chapitre_8} & \href{https://github.com/MSD-IRIMAS/Simple-KAN-4-Time-Series}{Simple-KAN-4-Time-Series} \\ \hline
    
    \end{tabular}
    \caption{List of contributions including 11 papers and 3 open source published work
    with the companion GitHub repository.}
    \label{tab:intro-contributions}
\end{table}

% To address the application of human motion analysis, Chapter~\ref{chapitre_6}
% focuses on tailoring 
% our contributions specifically for human motion data.
Given that this thesis is conducted within the framework of the ANR JCJ DELEGATION (more details in
Section~\nameref{financements}), 
which targets the analysis of human motion, Chapter~\ref{chapitre_6} addresses the specific challenges 
of human motion data. It highlights the unique characteristics of this data and 
demonstrates how our contributions are particularly effective in this domain, in 
line with the project's objectives.
We demonstrate how using small 
models like LITE, optimized for human motion, can achieve state-of-the-art performance 
in rehabilitation assessment within a classification framework, while also being 
effective in terms of medical explainability. Additionally, we introduce a novel 
TSP approach used as a generative method 
for enhancing extrinsic regression models 
in rehabilitation motion assessment tasks.
Moreover, considering the rise 
generative models for human motion~\cite{action2motion-paper,actor-paper} and 
the strong performance of CNNs in the time series domain~\cite{inceptiontime-paper},
we propose a deep 
generative model with a CNN backbone for human motion data
that nearly matches state-of-the-art results.
Given we focus on the evaluation framework for discriminative models in Chapter~\ref{chapitre_2}, 
we argue in Chapter~\ref{chapitre_7} for the necessity of a unified framework specifically for 
generative models, particularly in the context of human motion data.

In Chapter~\ref{chapitre_8}, we conclude by discussing the importance of 
reproducible research, offering 
a professional perspective on this critical aspect of scientific inquiry. We highlight the 
contributions of this thesis to the open-source Python package \textit{aeon}~\cite{aeon-paper}, 
as well as several other open-source projects developed during the course of this research, 
with or without accompanying publications.

In Table~\ref{tab:intro-contributions} we present all of the contributions in this thesis,
including 11 papers and 3 open source projects.
All of our research work is based on using publicly available datasets, including the UCR
archive~\cite{ucr-archive} for univariate setups, the UEA archive~\cite{uea-archive}
% MDC: add a ref ? (you did it for the UCR) HFC: Added
for multivariate setups and both 
the HumanAct12~\cite{action2motion-paper} and Kimore~\cite{kimore-paper} datasets
% MDC: add a ref ? (you did it for the UCR) HFC: Added
for human motion applications of activity recognition and 
rehabilitation assessment.

\section*{Publications}

%MDC: maybe put the full author list for each publication? You are not limited in space and it will particularly be useful to make you visible in the first journal paper :-)

\subsection*{International Journals}

\begin{itemize}
    \item Middlehurst, Matthew, \textbf{Ali Ismail-Fawaz}, Antoine Guillaume, 
    Christopher Holder, David Guijo Rubio, Guzal Bulatova, Leonidas Tsaprounis, Lukasz Mentel, 
    Martin Walter, Patrick Schäfer, Anthony Bagnall
    (2024) ``aeon: a Python toolkit for learning from time series''.
    In \textcolor{orange}{Journal Machine Learning Research (JMLR), open source software track}.
    doi: \url{https://arxiv.org/abs/2406.14231}
\end{itemize}

\subsection*{International Conferences and Workshops}
\begin{itemize}
    \item \textbf{Ismail-Fawaz, Ali}, Maxime Devanne, 
    Jonathan Weber, and Germain Forestier. (2022). 
    ``Deep learning for time series classification using 
    new hand-crafted convolution filters''. In \textcolor{orange}{IEEE International Conference on Big Data (Big Data)}
    (pp. 972-981). IEEE. doi: \url{https://doi.org/10.1109/BigData55660.2022.10020496}
    
    \item \textbf{Ismail-Fawaz, Ali}, Maxime Devanne, Jonathan Weber, 
    and Germain Forestier. (2023). ``Enhancing time series classification with self-supervised learning''.
    In \textcolor{orange}{International Conference on Agents and Artificial Intelligence (ICAART)}
    (pp. 40-47). SCITEPRESS-Science and Technology Publications. doi: \url{https://doi.org/10.5220/0011611300003393}
   
    \item \textbf{Ismail-Fawaz, Ali}, Hassan Ismail Fawaz, François Petitjean, 
    Maxime Devanne, Jonathan Weber, Stefano Berretti, 
    Geoffrey I. Webb, and Germain Forestier. 
    (2023). ``ShapeDBA: generating effective time series prototypes using 
    shapeDTW barycenter averaging''. In \textcolor{orange}{International Workshop on Advanced Analytics and 
    Learning on Temporal Data; in conjunction with the European Conference on Machine Learning and Principles 
    and Practice of Knowledge Discovery in Databases} (pp. 127-142). Cham: Springer Nature Switzerland.
    doi: \url{https://doi.org/10.1007/978-3-031-49896-1_9}

    \item \textbf{Ismail-Fawaz, Ali}, Maxime Devanne, Stefano Berretti, 
    Jonathan Weber, and Germain Forestier. (2023). 
    ``Lite: Light inception with boosting techniques for time series 
    classification''. In \textcolor{orange}{IEEE 10th International Conference on Data Science and Advanced 
    Analytics (DSAA)} (pp. 1-10). IEEE. doi: \url{https://doi.org/10.1109/DSAA60987.2023.10302569}
    
    \item \textbf{Ismail-Fawaz, Ali}, Maxime Devanne, Stefano Berretti, 
    Jonathan Weber, and Germain Forestier. (2024). 
    ``Finding foundation models for time series classification with a pretext task''.
    In \textcolor{orange}{Pacific-Asia Conference on Knowledge Discovery and Data Mining} 
    (pp. 123-135). Singapore: Springer Nature Singapore. doi: \url{https://doi.org/10.1007/978-981-97-2650-9_10}
    
    \item \textbf{Ismail-Fawaz, Ali}, Maxime Devanne, Stefano Berretti, 
    Jonathan Weber, and Germain Forestier. (2024). 
    ``A Supervised Variational Auto-Encoder for Human Motion 
    Generation using Convolutional Neural Networks''. 
    In \textcolor{orange}{4th International Conference on Pattern 
    Recognition and Artificial Intelligence (ICPRAI)}
    
    \item \textbf{Ismail-Fawaz, Ali}, Maxime Devanne, Stefano Berretti, 
    Jonathan Weber, and Germain Forestier. (2024). 
    ``Weighted Average of Human Motion Sequences for Improving Rehabilitation Assessment''.
    In \textcolor{orange}{International Workshop on Advanced Analytics and 
    Learning on Temporal Data (AALTD); in conjunction with the European Conference on Machine Learning and Principles 
    and Practice of Knowledge Discovery in Databases(ECML/PKDD)}.
    doi: \url{https://ecml-aaltd.github.io/aaltd2024/articles/Fawaz_AALTD24.pdf}
\end{itemize}

\subsection*{National Conferences}
\begin{itemize}
    \item \textbf{Ismail-Fawaz, Ali}, Maxime Devanne, 
    Jonathan Weber, and Germain Forestier. (2023). ``Apprentissage en Profondeur pour la Classification des Séries Temporelles 
    à l'aide de Nouveaux Filtres de Convolution Créés Manuellement''. In \textcolor{orange}{ORASIS}.
    doi: \url{https://hal.science/hal-04219450/document}
    
    \item \textbf{Ismail-Fawaz, Ali}, Maxime Devanne, Stefano Berretti, 
    Jonathan Weber, and Germain Forestier. (2024). ``LITE: Light Inception avec des Techniques de Boosting pour la Classification 
    de Séries Temporelles''. In \textcolor{orange}{Extraction et Gestion des Connaissances (EGC)} (pp. 377-384).
    doi: \url{https://editions-rnti.fr/?inprocid=1002948}
    
    \item \textbf{Ismail-Fawaz, Ali}, Maxime Devanne, Stefano Berretti, 
    Jonathan Weber, and Germain Forestier. (2023). ``Reducing Complexity of Deep Learning for Time Series Classification 
    Using New Hand-Crafted Convolution Filters''. \textcolor{orange}{Upper Rhine Artificial Intelligence Symposium (URAI)}.
    doi: \url{https://maxime-devanne.com/publis/ismail-fawaz_urai2023.pdf}
\end{itemize}

\section*{Under Submission}

\subsection*{International Journals}
\begin{itemize}
    \item \textbf{Ismail-Fawaz, Ali}, Maxime Devanne, Stefano Berretti, 
    Jonathan Weber, and Germain Forestier. ``Look Into the LITE in Deep Learning for Time Series Classification''.
    In \textcolor{orange}{International Journal of Data Science and Analytics}.
    doi: \url{https://arxiv.org/abs/2409.02869}
    \item \textbf{Ismail-Fawaz, Ali}, Maxime Devanne, Stefano Berretti, 
    Jonathan Weber, and Germain Forestier. ``Establishing a Unified Evaluation Framework for Human Motion Generation:
    A Comparative Analysis of Metrics''. In \textcolor{orange}{Computer Vision and Image Understanding}.
    doi: \url{https://arxiv.org/abs/2405.07680}
\end{itemize}

\section*{Arxiv}
\begin{itemize}
    \item \textbf{Ismail-Fawaz, Ali}, Angus Dempster, Chang Wei Tan, 
    Matthieu Herrmann, Lynn Miller, Daniel F. Schmidt, 
    Stefano Berretti, Jonathan Weber, Maxime Devanne,
    Germain Forestier, and Geoffrey I. Webb. (2023). ``An approach to multiple comparison benchmark evaluations that 
    is stable under manipulation of the comparate set''. doi: \url{https://arxiv.org/abs/2305.11921}
\end{itemize}
