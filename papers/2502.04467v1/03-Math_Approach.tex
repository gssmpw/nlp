
% Delete the text and write your Theory/ Background Information here:
%------------------------------------

%In this section, we will study the problemo find a catenary joining two points
%in the 3D space $A$ and $B$ that does not collide with obstacles $O$ in the environment
%$C_{obs}=\cup_{i=1}^NO_i$.

%Deciding whether there exists a collision-free catenary with bounded length in a three-dimensional scenario or not is a crucial task in some problems in Robotics, such as the path planning for a tethered aerial robot \cite{martinez2021optimization}.  

%In this section, we address the problem of the existence of a free-collision catenary ${\cal C}_{AB}$ connecting the ground vehicle position, $A$, and the aerial vehicle position, $B$. Adopting a common terminology, we call $A$ and $B$ the \emph{suspension points}, and the horizontal and vertical distance between $A$ and $B$ are called the \emph{span} and the \emph{sag}, respectively.

In this section, we present a new method for efficient computation of a collision-free catenary curve based on the parabola approximation. As was previously commented, the main idea is that the method enables the planner to calculate trajectories faster and more efficiently, since it avoids the computational complexity associated with the calculation of the catenary model for tether collision detection.

Considering the catenary as a planar curve, our input consists of a set of 2D obstacles $\cal{O}$ defined by polygons in a plane denoted $\pi$. This plane $\pi$ is perpendicular to the ground and passes through points $A$ and $B$. 

%To simplify the analysis, we assume that the obstacles are convex polygons. {\color{blue}
%Note that a non-convex polygon can be decomposed into a collection of convex polygons. Do we need convex polygons?}




We define a curve $C$ as collision-free in $\pi$ if it does not intersect with any obstacle in $\pi$.  Therefore, our focus is directed towards the following two-dimensional problem:


\vspace{.25cm}
\textsc{Catenary Decision Problem (CDP):} \emph{Given a set $\cal O$ of polygons in $\pi$, determine whether there exists a collision-free catenary 
${\cal C}_{AB}$ in $\pi$ linking $A$ and $B$ with length at most $L$.} 
%\vspace{.5cm}

%\subsection{Naive approach: catenary length sampling}
%\label{sec:naive_approach}



%================================================
%\subsection{The Parabola Decision Problem}
%\label{sec:dsproblem}

In \cite{smartinezr2023} a numerical approach was considered to determine the existence of a collision-free catenary suspended between points $A$ and $B$. The approach involves examining catenaries with increasing length, starting from $l = d(A,B)$ to $l=L$, with the addition of a specified increment $\Delta l$ until a collision-free trajectory is identified (success) or the maximum length is reached (failure). Furthermore, in each iteration, the catenary shape is sampled to verify for potential collisions.
Both the calculation of the catenary curve and the collision test are time-consuming. This involves solving transcendental equations, which is computationally demanding \cite{behroozi2014fresh}.
%The problem with this approach is that we have to first determine the catenary that passes through $A$ and $B$ with a given length.
%Furthermore, in order to check for collision on the catenary, we have to sample its shape. Hence, the sampling length $d$ of the catenary is a crucial parameter in this method, as a small $d$ could waste efforts by repeteadly calling to the collision checker, whereas a big step could make the method not able to detect some collisions between the catenary and the obstacles. %In addition, a careful tuning of the step length $\Delta l$ is crucial to make the algorithm accurate and efficient.
In addition, low values on $\Delta l$ could waste efforts by repeatedly calling the collision checker with very similar curves. In contrast, high values $\Delta l$ could cause the method to overlook some catenary lengths that might be collision-free.

\begin{figure}[t!]
  \centering
   \includegraphics[scale=0.35]{Figures/region-T_v2.png}  
%\includegraphics[scale=0.27]{Figures/poligonos3.png}  
%\includegraphics[scale=0.25]{Figures/lemma3-2-2red.png}  
\caption{Polygons inside the $T$ region, which is delimited by the black trapezoid $\overline{ABF_AF_BA}$. $A$ and $B$ are the suspension points; $F_A$ and $F_B$ are their projections on the floor. The blue polygon is \emph{weakly inscribed}  in the parabolic region $R(v^*)$ and $\overline{AB}$, delimited by the blue parabola.}  %\caption{Polygons inside $T$ are \emph{weakly inscribed}  in the parabolic region  $R(v^*)$. }
\label{fig:PDP}
\end{figure}

To overcome these issues, we have devised an alternative approach using a simpler curve.  In certain applications, such as the design of transmission overhead lines \cite{hatibovic2018algorithm}, the catenary has been replaced by a parabolic curve, as the parabola is a good approximation of a catenary if the sag is small \cite{hatibovic2020comparison}. However, in our scenario, the sag is relatively large, and we have to be careful with the parabolic approximation.
Moreover, our approach requires examining the reverse approximation: Given a collision-free parabola, we would like to compute an approximated catenary.

In the remainder of the section, we describe our proposed procedure for obtaining collision-free catenaries between two points. Initially, we address the decision problem using parabolas instead of catenaries. Subsequently, upon discovering a collision-free parabolic curve, we introduce a numerical method to approximate the parabola with a catenary. Finally, we ensure that the obtained catenary %remains
is collision-free by strategically expanding the obstacles.

Let $T$ be the open trapezoidal region in $\pi$ below the segment $\overline{AB}$, bounded by the ground and the two vertical lines passing through $A$ and $B$, as illustrated in Fig. \ref{fig:PDP}.

A parabolic curve between the points $A$ and $B$ divides the plane $\pi$ into two regions, one convex and one non-convex. Given a set of obstacles in the plane $\pi$, a parabolic curve is collision-free when all obstacles are contained in one of the two regions.

Let $P_{AB}(v)$ be the parabola defined by the points $A$, $B$, and $v$, where $v$ is a vertex of an obstacle in $T$. Denote by $\ell(v)$ the length of the parabola $P_{AB}(v)$ and
$R(v)$ the convex region defined by the segment $\overline{AB}$ and the parabola $P_{AB}(v)$.
%\begin{definition}
A polygon is \emph{weakly inscribed} in the region $R(v)$ if it is contained in $R(v)$
and at least one vertex of the polygon lies on the parabola $P_{AB}(v)$. 
%We call such vertex a \textit{tangent vertex}. 
%\end{definition}
%Clearly, if a polygon $O$ is weakly inscribed in the region $R(v)$, then its vertices touching the boundary of $R(v)$ belong to its convex hull $CH(O)$ and $CH(O)$ is also weakly inscribed in $R(v)$. Thus,
%our problem reduces to finding a parabola that does not collide with the convex hull of any polygon within $T$. 
Therefore, we are interested in solving the following auxiliary problem:

\vspace{.25cm}
\textsc{Parabola Decision Problem (PDP):} 
\emph{Given a set $\cal O$ of polygons %completely 
contained in $T$, decide whether there exists a collision-free parabola $P_{AB}(v)$ with a length at most $L$ passing through a vertex $v$ of a polygon in $\cal O$.}
%\vspace{.5cm}

%\ma{We define some notation...}





%================================================
\subsection{An iterative approach for solving PDP}
\label{solvingPDP}
%The proposed method for solving PDP involves computing the length of a parabolic arc or determining a set of polygons intersecting this  arc and, for sort, we use some notation: The length of $P_{AB}$ is $|P_{AB}|$ and $P_{AB} \cap \mathcal{O}$ is the subset of polygons of $\mathcal{O}$ that intersect $P_{AB}$. 
Given a polygon $O=\langle v_1,\cdots,v_n\rangle$ contained in  $T$, we say that a parabolic arc hanging from points $A$ and $B$, $P_{AB}$, intersects with the polygon when it intersects a face of the polygon, i.e., there are points on a face that lie both above and below the curve. We denote  $P_{AB} \cap \mathcal{O}$  the subset of polygons of $\mathcal{O}$ that intersect the parabolic arc $P_{AB}$. We call this set $\mathcal{O'}$ and $CH(\mathcal{O'})$ its convex hull. In what follows, we will denote the length of $P_{AB}$ as $|P_{AB}|$ when we do not expressly know the third point $v$ through which the parabola passes.


\begin{lemma} \label{cor: longest_parabola}  
  Let $O=\langle v_1,\cdots,v_n\rangle$ be a convex polygon contained in the open region $T$ and let $v^*$ be the vertex of $O$ such that $\ell(v^*)=\max_{v_i \in O}\ell(v_i).$ Then the polygon $O$ is weakly inscribed in $R(v^*)$. 
  Furthermore, for any vertex $v_i \in O$ such that $\ell(v_i)<\ell(v^*)$, $O$ is not weakly inscribed in $R(v_i)$.
\end{lemma}	

\begin{proof}
Since two different parabolas intersect at most in two points, for all vertices $v_i\in O$ with $\ell(v_i) \leq \ell(v^*)$, it follows that $R(v_i)\subseteq R(v^*)$ and
%. As a result, 
the polygon $O$ is weakly inscribed in the region $R(v^*)$. On the other hand, for all $v_i$ so that $\ell(v_i) < \ell(v^*)$, $R(v^*) \nsubseteq R(v_i)$. Thus, $v^* \notin R(v_i)$ and $O$ is not weakly inscribed in $R(v_i)$. Fig. \ref{fig:PDP} illustrates the proof.
\end{proof}

As a direct consequence of Lemma \ref{cor: longest_parabola}, we state the following remark:
\begin{remark}
\label{rmk:longest_parabola} 
Let $O$ be a polygon
contained in $T$ that intersects a parabolic arc $P_{AB}$. 
Then the parabolic arc $P'_{AB}$ of minimum length that does not intersect with $O$ such that $|P'_{AB}|>|P_{AB}|$ in the span $(A, B)$ is the parabola of maximum length passing through a vertex of $CH(O)$.
\end{remark}

Algorithm \ref{alg:decision_problem} illustrates an overview of the approach to solving the parabola decision problem. We use the result presented in Remark \ref{rmk:longest_parabola} to progressively increase the length of the parabola until a specified stopping condition is met. To initiate the process, the algorithm computes the line segment $\overline{\rm AB}$ in line 1, representing the parabolic arc of minimum length from $A$ to $B$. The %\inma{quitaría esto: value of the } 
current parabola is stored in $P_{AB}$ and is subsequently updated throughout the algorithm. 
The fundamental invariant upheld at each step ensures that any parabola with a length less than 
%The invariant maintained in every step of the algorithm is that any parabola with length lower than 
$|P_{AB}|$ is not valid.
The algorithm enters a loop from lines [2-11], setting stop conditions for PDP in lines [2-8].

If the current parabola touches the ground or if its length reaches the maximum value, then there is no collision-free parabola hanging from $A$ to $B$, line [2-4]. Then we compute the set of obstacles $\mathcal{O'}$ intersecting with $P_{AB}$, line 5. If this set is empty, $P_{AB}$ returns as a valid parabola. Otherwise, we compute the minimum length parabola $P'_{AB}$ that does not intersect with $\mathcal{O'}$ such that $|P'_{AB}|>|P_{AB}|$. Notice that any parabola of length less than $P'_{AB}$ is not collision-free, so the established invariant is preserved. From Remark \ref{rmk:longest_parabola}, $P'_{AB}$ is the parabola of maximum length that passes through the convex hull of $\mathcal{O'}$; hence, we update $P_{AB}$ with this parabola, lines [9-10]. At this point, a full cycle of the algorithm has been completed, so we return to line 2. The iterative process guarantees termination, as the algorithm systematically increases the length of the current parabola.
%The iterative process ensures termination, given that the algorithm consistently augments the length of the current parabola.


% Algorithm \ref{alg:decision_problem} illustrates an overview of the approach to solve the decision problem PDP, consisting in only one loop, lines 2-10. The loop starts considering the segment $\overline{AB}$ as the first candidate parabola $P_{AB}$, line 1. The loop has two stop conditions: 1) if the parabola $P_{AB}$ touches the ground or the length of ${P}_{AB}$ is greater than $L$, then there is no feasible solution; 2) if %the intersection set ${\cal O'} = P_{AB} \cap {\cal O}$ 
% the set of polygons in $T$ intersecting with the parabola, ${\cal O'}$, is empty, then $P_{AB}$ is a feasible parabola and the algorithm ends. When none of the aforementioned conditions are met, 
% %the set of polygons intersecting the parabola, ${\cal O'} = P_{AB} \cap {\cal O}$, is not empty and 
% the process is repeated considering a new parabola $P_{AB}$. %collision-free with $CH({\cal O'})$. 
% The new parabola \revS{with length $\ell(v_i)$} is calculated using Lemma \ref{cor: longest_parabola}, which guarantees that $P_{AB}$ does not collide with the convex hull of the set ${\cal O'}$, $CH({\cal O'})$. The loop always ends since the length of  $P_{AB}$ increases in every iteration. %If no feasible solution is found, at some point the parabolic arc length will exceed $L_{max}$ and the algorithm will stop. 
% This procedure solves the problem PDP in linear time with respect to the number of polygons in the environment. Note that, in this case, there is no need to sample the parabolic shape when searching for collisions. Similarly, we are not constrained to a predetermined set of catenary lengths, as in the naive method mentioned earlier.

\begin{algorithm}[t!]
\caption{Given the suspension points $A$ and $B$, and a set of polygons $\mathcal{O}$ inside $T$, %$T_{AB}$,
returns, if it exists, a collision-free parabolic arc $P_{AB}$ with maximum length $L$. 
$P_{AB}$ cannot touch the ground.
}
\label{alg:decision_problem}
\begin{algorithmic}[1]
\Require $A, B, \mathcal{O}, L$.
\Ensure Collision-free parabolic arc $P_{AB}$.

 \State $P_{AB} \gets \overline{\rm AB}$
\If{${P}_{AB}$ touches the ground \textbf{or} $|P_{AB}|> L$} 
    \State \Return None %\Comment{Parabolic arc can not touch the ground}
\EndIf

\State $\mathcal{O}' \gets P_{AB} \cap \mathcal{O}$
\If{$\mathcal{O}' = \emptyset$}
    \State \Return $P_{AB}$
\EndIf
% \State $P_{AB} \gets \text{ parabola so that }CH(\mathcal{O}')$ \text{ is weakly inscribed }

%\State $P_{AB} \gets P_{AB}(v)$ \text{with} $CH(\mathcal{O}')$ \text{weakly inscribed in} $R(v)$
\State $v^* \gets \text{argmax}_{v_i \in CH(\mathcal{O}')}\ell(v_i)$
\State $P_{AB} \gets P_{AB}(v^*)$
\State GoTo (line 2)
\end{algorithmic}  
\end{algorithm}

%\subsubsection{Correctness of the approach}

% Lemma \ref{cor: longest_parabola}   guarantees the correctness of the approach outlined in Algorithm \ref{alg:decision_problem}. 


%================================================
\subsection{Computing the collision-free catenary from a parabola}
\label{sec:approxlength}

Algorithm \ref{alg:decision_problem} yields a parabolic curve that solves the PDP. In this section, we introduce three methods for approximating the parabolic curve with a catenary, representing the shape of a hanging tether. Following an experimental comparison, we choose one of these methods.

\subsubsection{Method 1. Approximation by length}
\label{sec:bylength}
In our first method, we just impose the catenary to be of the same length as the obtained parabola. Let $\lambda$ be the length of the parabola that connects the suspension points $A=(0, h_1)$ and $B=(S, h_2)$, and let $(x_{min}, y_{min})$ be the coordinates of the lowest point of the catenary with length $\lambda$ passing through $A$ and $B$. Note that $h_1, h_2, x_{min},  y_{min} > 0$ and $\lambda > d(A,B)=\sqrt{S^2 +(h_2-h_1)^2}$.
Taking the general equation of the catenary \cite{hatibovic2020comparison}: 
\begin{equation}
y_{cat}(x)=2c\sinh\left(\frac{x-x_{min}}{2c}\right)+y_{min}, c>0,
\end{equation}
the solution of the following system of non-linear equations provides the expected catenary: 
\begin{equation}
\left.
    \begin{array}{lll}
    2c \sinh^2 {\left(\frac{-x_{min}}{2c}\right)} + y_{min} & = & h_2 \\
    2c \sinh^2 {\left(\frac{S-x_{min}}{2c}\right)} + y_{min} & = & h_1 \\
    c\left[ \sinh{\left( \frac{S-x_{min}}{c}\right)}-   \sinh{\left( \frac{-x_{min}}{c}\right)} \right] & = & \lambda
    \end{array}
    \right\}
\end{equation}

\medspace

%The independent terms of the system are the coefficients of the general equation: $c$, $x_{min}$, and $y_{min}$. 

%The system has a unique solution since $\ell > \sqrt{S^2 +(h_2-h_1)^2}$.

\subsubsection{Method 2. Fitting the catenary with a parabola}
\label{sec:approxfitting}
An alternative method is to find a catenary that fits the parabola, specifically the catenary that minimizes the maximum vertical distance with the parabola. To design an efficient fitting algorithm, some mathematical properties are used.
%The following result can be proved by elemental Calculus.

From the convexity of the catenary, the following Lemma can be stated:
\begin{lemma}
\label{le:big_area_small_length}
    Let $\cat_1$, $\cat_2$ be two catenaries hanging from the same suspension points $A=(x_1, h_1)$ and $B=(x_2, h_2)$ such that $\forall x\in (x_1, x_2)$ $\cat_1(x)> \cat_2(x)$. Then the area under $\cat_1$ in $[x_1, x_2]$ is greater than the area under $\cat_2$ in the same interval, but its length is lower.
\end{lemma}
% \begin{proof}
%     In a concave region, the curve joining two consecutive vertices $v_1, v_2$ is the curve within the region with the lowest length touching $v_1$ and $v_2$. From this we get than any curve touching $A$ and $B$ in the region defined by $\cat_1$ in the interval $[x_1, x_2]$ has greater length than $\cat_1$. On the other hand, the area under $\cat_1$ can be divided in the area under $\cat_2$ plus the area between both curves; hence the area under $\cat_1$ is bigger than the area under $\cat_2$.
% \end{proof}

We use an additional result extracted from \cite{parker2010property}:
\begin{theorem}
\label{theo:cool_cat_property}
    Given a catenary curve $\cat$ and any horizontal interval $(x_1, x_2)$, the ratio defined by the area under $\cat$ divided by the length of the curve in that interval is independent of the value of $x_1$ and $x_2$.
\end{theorem}

Now we are ready to prove the following statement:
\begin{theorem}\label{teo:only_two_in_common}
Two different catenaries hanging from the same suspension points cannot have more than two points in common.
\end{theorem}
\begin{proof}
    (By contradiction). Let $\cat_1$, $\cat_2$ be two catenaries that hang from $A=(x_1, h_1)$ to $B=(x_2, h_2)$ such that they intersect at at least three points within the interval $[x_1, x_2]$. Let $a=(x_a, h_a)$, $b=(x_b, h_b)$, $c=(x_c, h_c)$ with $x_a<x_b<x_c$ be the intersection points,
    %three consecutive points where $\cat_1$ and $\cat_2$ intersects, 
    and let $\cat_i^a$ and $\cat_i^c$ indicate the arcs of the catenary $\cat_i$ contained in the interval $[x_a, x_b]$ and $[x_b, x_c]$, respectively. Without loss of generality, we assume $\forall x\in (x_a, x_b)$ $\cat_1^a(x)> \cat_2^a(x)$; hence $\forall x\in (x_b, x_c)$ $\cat_1^c(x)< \cat_2^c(x)$. Let $r_1$ ($r_2$) be the area under $\cat_1$ ($\cat_2$) in $[x_a, x_b]$ divided by the length of $\cat_1^a$ ($\cat_2^a$). Let $A_1, A_2, C_1, C_2$ be the area under $\cat_1^a,\cat_2^a, \cat_1^c,\cat_2^c$ respectively, and $L_1, L_2, L_3, L_4$ their lengths. From Theorem \ref{teo:only_two_in_common}, $r_1=\frac{A_1}{L_1}=\frac{C_1}{L_3}$ and $r_2=\frac{A_2}{L_2}=\frac{C_2}{L_4}$. In addition, from Lemma \ref{le:big_area_small_length}, we know that $A_1=A_2+\epsilon_1$ ($\epsilon_1>0$), and $L_1<L_2$; then 
    \begin{equation}
    r_1=\frac{A_2+\epsilon_1}{L_1}>\frac{A_2+\epsilon_1}{L_2}>\frac{A_2}{L_2}=r_2  
    \end{equation}
    \noindent On the other hand, from Lemma \ref{le:big_area_small_length} we know $C_2=C_1+\epsilon_2$ ($\epsilon_2>0$), and $L_3>L_4$; then 
    \begin{equation}
    r_2=\frac{C_1+\epsilon_2}{L_4}>\frac{C_1+\epsilon_2}{L_3}>\frac{C_1}{L_3}=r_1
    \end{equation}
    \noindent From the above we get 
    %$r_1>r_2$ and $r_1<r_2$ which is 
    a contradiction and the result follows. 
\end{proof}

% From \cite{parker2010property} the following theorem can be extracted.
% \begin{theorem}
%     Given a catenary curve $\mathcal{C}$ and any horizontal interval $(t_1, t_2)$, the ratio defined by the area under $\mathcal{C}$ divided by the length of the curve on that interval is independent of the value of $t_1$ and $t_2$.
% \end{theorem}
% This theorem characterize a catenary by its area-length ratio; hence two catenaries with different ratios are different. Using this result, the following theorem can be probed by contradiction.

% \begin{theorem}\label{teo:ivorra2}
% Two different catenaries with suspension point $A$ and $B$ cannot have more than two points in common.
% \end{theorem}

As a direct consequence of Theorem 
% \ref{teo:ivorra2}, 
\ref{teo:only_two_in_common}
%we know that two catenaries with the same suspension points are disjoint and thus, 
we have: 
\begin{corollary}\label{cor:monotony}
%$C_{AB}$ be a catenary of length $l$ connecting the suspension points 
%$A=(0, h_1)$ and $B=(S, h_2)$, $S>0$, be the suspension points and 
Let ${C}_{AB}^{i}(x)$ be the catenary function of length $l_i$ that connects points $A$ and $B$. If $l_i \geq l_j$, $j\neq i$, then ${C}_{AB}^{j}(x) \geq {C}_{AB}^{i}(x)$, for all $x\in [0,S]$.
\end{corollary}

%\begin{proof}
%The proof is straightforward using Theorem \ref{teo:ivorra2}.
%\end{proof}

The monotonicity property of Corollary \ref{cor:monotony} allows us to implement a bisection method within the allowable length interval of the tether, namely $[d(A, B), L]$, to identify an appropriate catenary setting.
%Using this monotonicity property, a bisection method can be implemented on the feasible length interval of the tether, that is, $[d(A,B),L]$, to identify a suitable catenary fit. 
As demonstrated in our experiments, this method proves to be efficient, delivering accurate approximations with minimal iterations. Algorithm \ref{alg:catenaryaproximation} depicts the pseudocode of the solution.

\begin{algorithm}[t!]
\caption{Algorithm to compute a fitting catenary.}
\label{alg:catenaryaproximation}
\begin{algorithmic}[1]
\Require \emph{point} $A$, \emph{point} $B$, \emph{parabola} ${P_{AB}}$, \emph{maximum length} $L$, \emph{approximation error} $\varepsilon$
\Ensure $\mathcal{C}_{AB}:$ a fitting catenary hanging from $A$ and $B$.
\State $L_0 \gets |\overline{AB}|$
\State $L_1 \gets L$
\While{$True$}
\State $L_m \gets (L_0+L_1)/2$
\State $\mathcal{C}_{AB} \gets get\_{cat}atenary(A,B,L_m)$
\If{$L_1-L_0 \leq \varepsilon$}
    \Return $\mathcal{C}_{AB}$
\EndIf
\State $x_{max} \gets get\_max\_distance\_axis({P_{AB}},\mathcal{C}_{AB}, A, B)$
\State $d \gets {P_{AB}}(x_{max}) - \mathcal{C}_{AB}(x_{max})$
\If{$d < 0$}
    \State $L_0 \gets L_m$
\Else{}
    \State $L_1 \gets L_m$
\EndIf

\EndWhile
\end{algorithmic}
\end{algorithm}


In Algorithm \ref{alg:catenaryaproximation}, the suspension points $A=(0, h_1)$ and $B=(S, h_2)$ and the maximum length of the expected catenary, $L$, are taken as input. In addition, an approximation error $\varepsilon$ is received to determine the accuracy of the output. The metric used in the bisection is the difference in terms of lengths between two iterations. First, the minimum and maximum lengths of the expected catenary are defined as $L_0$ and $L_1$, respectively, lines 1 and 2. These values are obtained from the parameters $A$, $B$, and $L$ and define a feasible interval of catenary lengths. Later, the algorithm performs a loop to estimate the optimal catenary length in the range $[L_0, L_1]$, lines 3-15. In each iteration, the mean length between $L_0$ and $L_1$, $L_m$, is calculated, obtaining the catenary of length $L_m$, $\mathcal{C}_{AB}$, lines 4 and 5. In the next step, the value: 
\begin{equation}
 x_{max} = max_{x \in [0, S]}|{P_{AB}}(x) - \mathcal{C}_{AB}(x)|   
\end{equation}
\noindent is calculated using the function $get\_max\_distance\_axis$ on line 8, and the maximum vertical difference $d$ between ${P_{AB}}$ and $\mathcal{C}$ is obtained on line 9. The value $x_{max}$ can be obtained by solving a numerical problem or can be approximated by sampling the interval $[0, S]$. Using Corollary \ref{cor:monotony}, the sign of the difference $d$ is the condition to discard an entire subinterval, $[L_0,L_m]$ or $[L_m,L_1]$.
The stop criterion is outlined at line 6, and the algorithm returns the last calculated catenary $\mathcal{C}_{AB}$. 
%The number $N$ of iterations is: $$ N = \log _2 (L - |\overline{AB}| - \varepsilon)$$
%Another stop criterion could be to define a maximum tolerance $\theta$ and return the first catenary with maximum vertical distance to the parabola smaller than $\theta$. In this case, the value $\theta$ should be carefully defined for the correct convergence of the bisection procedure.


\subsubsection{Method 3. Fitting the catenary by sampling the parabola points}
\label{sec:bypoints}

This method consists of fitting the catenary to the parabola through a nonlinear optimization process to minimize the distance to $n$ points sampled on the catenary and the parabola. For this, the $n$ points of the parabola are considered in a plane X-Y. The nonlinear optimizer creates $n$ constraints that must minimize the Y-distance between both curves, parabola and catenary. This translates into the following optimization problem.

\begin{equation}
     cat(x)^*=\arg \min_{cat} \sum_{i=1}^n ||par(x_i) - cat(x_i)||^2
\end{equation}

%\begin{equation}
%    \label{eq:const_byPoints}
%    R_i = y_c(x_i) - y_p(x_i)  \quad \text{for} \quad i = 1, 2, \dots, n
%\end{equation}

\noindent where $cat(x)=a\ cosh( (x-x_0)/a) + y_0$ represents the catenary equation in which the parameters $a$, $x_0$ and $y_0$ must be estimated, and $par(x)$ is the parabola equation to fit.



\subsubsection{Benchmarking}

Now, the methods for computing an approximation catenary are compared each other. The methods of Sections \ref{sec:bylength}, \ref{sec:approxfitting}, and \ref{sec:bypoints} are denoted by \emph{ByLength}, \emph{ByFitting} and \emph{BySampling}, respectively.
%The first method, denoted as \emph{ByLength}, consists of taking a catenary of the same length as the parabola (see Section \ref{sec:approxlength}). Alternatively, a fitting algorithm is proposed to approximate the optimal catenary that minimizes the maximum vertical distance between the catenary and the parabola, \emph{ByFitting} (see Section \ref{sec:approxfitting}). Fig. \ref{fig:approx_cat} shows the catenaries obtained with both techniques for a given parabola. 
A set of random experiments was performed to compare the methods in terms of the maximum vertical distance between the reference and the approximated curves. In the experiments, the approximation error $\varepsilon$ selected for the \emph{ByFitting}, as appears in Algorithm \ref{alg:catenaryaproximation}, was $10^{-2}$, and the number $n$ of points in \emph{BySampling} was 5.

\begin{comment}
\begin{figure}[t!]
    \centering
    \includegraphics[width=0.45\textwidth]{Figures/approx_curves.png}
    \caption{Approximation catenaries by the three methods.
    }
    \label{fig:approx_cat}
\end{figure}
\end{comment}

A set of 100 experimental scenarios was generated. For each scenario, the suspension point $A$ is set to $(0,1)$, while both the suspension point $B$ $(B_x,B_y)$ and the parabola's longitude are generated randomly. In accordance with the intended application, the parabolic curve is generated in such a way that the point $B$ is positioned higher than the point $A$. The maximum allowed curve length $L$ is set to $30$ meters.

Two metrics within the interval $[0,B_x]$ are taken into account to evaluate the performance of the approximation methods:
\begin{itemize}
%     \item [1.] %Maximum Vertical Distance (
% $\delta_{max}$: The maximum vertical distance between both curves. %between the suspension points, $x\in[0,S]$.  
%     \item [2.] %Total Vertical Distance (
% $\delta_{Total}$: The sum of the vertical distances. 
% %for all $x$ in $[0,S]$. 
%     \item [3.] %Mean Vertical Distance (
% $\delta_{mean}$: The mean vertical distance. %in the interval $[0,S]$.  
    % \item [4.] %Computation Time (
    % $C.T.$: The total computational time needed to compute the approximation curve. 
        \item [1.] $\epsilon_{mean}$: The mean vertical distance in the interval $[0,B_x]$ between both curves, catenary and parabola.  
        \item [2.] $\epsilon_{mean}/L$:  Is the mean vertical distance over the length of the curve in the interval $[0,B_x]$.
\end{itemize} 


% $\delta_{max}$, $\delta_{Total}$ and $\delta_{mean}$ are calculated by sampling the interval. For each scenario, the four metrics are evaluated in the approximation methods. Table \ref{tab:comparison} shows the mean and standard deviation of the results of all experimental scenarios. }%Table \ref{tab:comparison} shows that 
% \emph{ByFitting} presents the best performance. %between all methods in terms of vertical distance. 
% %\emph{ByFitting} presents the best performance between the methods in terms of vertical distance, that means, catenary and parabola curves are similar.
% This method yielded precise results, achieving less than $1.1~m$ of the maximum vertical distance, less than $3~m$ of the total vertical distance, and less than $0.7~m$ of the mean vertical distance. Furthermore, both methods are efficient and can be used on-line.

%$\delta_{mean}$ is calculated by sampling the interval. For each scenario, the metrics are evaluated in the approximation methods. 

\noindent Table \ref{tab:comparison} shows the mean and standard deviation of the results of all experimental scenarios. It can be seen that \emph{BySampling} presents the best performance among the methods in terms of vertical distance, which means that the catenary and parabola curves are similar.

%\revS{This method yielded precise results, achieving less than $0.6~m$ of the mean vertical distance. Furthermore, all the methods are efficient and can be used on-line.}


% \begin{table}[ht]
% \begin{center}
%   \small
%   \scalebox{0.9}{
% \begin{tabular}{|c|c|c|c|c|} 
%  \hline
%  Method & $\delta_{max}~(m)$ & $\delta_{Total}~(m)$ & $\delta_{mean}~(m)$ & $C.T.~(s)$ \\ [0.5ex]
%  \hline
% % \emph{ByPSeries} & $3.69 \pm 6.41$ & 38.39 & $2.47 \pm 4.59$ & $0.00\pm0.00$ \\ 
% % \hline
%  \emph{ByLength} & $0.69\pm1.08$ & 5.12& $0.29\pm0.42$ & $0.00\pm0.00$ \\
%  \hline
%  \emph{ByFitting} & $0.48 \pm 0.58$ & 2.98 & $0.28 \pm 0.37$ & $0.07 \pm 0.01$ 
%  \\
%  \hline
% \end{tabular}
% }
% \end{center}
% \caption{
% Comparison between the approximation methods.
% %using metrics $\delta_T$, $\delta_{max}$, $\delta_{mean}$, and $C.T$ 
% For each metric, the mean and standard deviation are shown.}
% \label{tab:comparison1}
% \end{table}

\begin{table}[t!]
\caption{
Comparison between the approximation methods: by Length, by Fitting and by Points
%using metrics $\delta_T$, $\delta_{max}$, $\delta_{mean}$, and $C.T$ 
For each metric, the mean and standard deviation are shown.}

\begin{center}
  \small
  \scalebox{0.9}{
\begin{tabular}{|c|c|c|} 
 \hline
 \textbf{Method} & $\epsilon_{mean}~[m]$ & $\epsilon_{mean} / L$  \\ [0.8ex]
 \hline
% \emph{ByPSeries} & $3.69 \pm 6.41$ & 38.39 & $2.47 \pm 4.59$ & $0.00\pm0.00$ \\ 
% \hline
 \emph{ByLength} & $0.3187\pm 0.3299$  & $0.0200\pm 0.0214$ \\
 \hline
 \emph{ByFitting} & $0.3127 \pm 0.3108$ & $0.0194 \pm 0.0200$ 
 \\
 \hline
  \emph{BySampling} & $0.2841 \pm 0.3040$ & $0.0173 \pm 0.0178$ 
 \\
 \hline
\end{tabular}
}
\end{center}
\label{tab:comparison}
\end{table}



%shows great performance, taking less than $0.1$ seconds to compute the approximation curve.

%The advantage of the $A.C$ and the $A.L$ methods over $O.C.A$ is the very low computational time needed to find the approximate catenary (taking just a few milliseconds). The only reason to pick another method other than $O.C.A$ is that the required computational speed is extremely fast, in this case, the right method to select is $A.L$. Method $A.L$ takes almost zero time to compute and shows acceptable results in terms of vertical distance, reaching less than $1.8~m$ of the maximum vertical distance, less than $5.2~m$ of the total vertical distance, and less than $0.8~m$ of the mean vertical distance. Finally, the $A.C$ method shows poor results despite its fast performance. 

\subsection{Ensuring a collision-free catenary}

Although the proposed catenary approximations of the parabola are accurate enough, we need to be sure that the computed catenaries are feasible, that is, collision-free and with admissible length.

Ensuring a feasible tether length after the catenary approximation can be easily implemented by constraining the parabola length during optimization. We can set the minimal tether length a $5\%$ longer and the maximum length a $5\%$ shorter, ensuring that the computed catenary will still hold the length constraint even with the distortions applied by the approximation.

However, ensuring a collision-free catenary is more complex. The most usual approach to solve this problem consists in inflating the obstacles, as in Algorithm \ref{alg:decision_problem}. The computed parabola avoids obstacles and passes through a vertex of an obstacle inside the trapezoidal region $T$. To guarantee that the calculated approximate catenary remains collision-free, the obstacles should be initially expanded as follows. For each polygon, we consider the \emph{expanded} polygon, which is defined by lines parallel to the faces of the polygon and located at a distance of $\tau$ (see Fig. \ref{fig:PDP}). The tangent collision-free parabola is built in the Algorithm \ref{alg:decision_problem} taking as input the expanded polygons. In practice, it is crucial to determine the tolerance $\tau$ for both ground and aerial obstacles. Large values of $\tau$ can give negative responses to the problem PDP, while small values of $\tau$ can lead to non-feasible approximation catenaries.  The election of $\tau$ strongly depends on the approximation method used. The more accurate the method, the better the selection of $\tau$. According to Table \ref{tab:comparison}, the most accurate approximation is \emph{BySampling} with a relative error of $0.0173(\sigma:0.0178)$ per meter. We can use such error to estimate the value of $\tau$, setting it to the average approximation error plus its standard deviation is a good trade-off, particularly $\tau=0.035*L$, where $L$ is the longitude of the tether. Note that the mean vertical distance between the parabola and the catenary never exceeded this value in the experiments. This means that both the ground and the obstacles in $\mathcal{O}_{\tau}$ should differ only in $0.6$ meters in the vertical axis from their original shape. Therefore, using \emph{BySampling} and $\tau=0.6~m$, we can ensure that Algorithm \ref{alg:catenaryaproximation} solves the CDP problem and computes a collision-free catenary for the marsupial robotic system.

While the previous method is safe and effective, obstacle inflation suffers from over-constraining the planning problem, eliminating possible feasible solutions under the assumption of the worst-case scenario. Instead, we propose not using obstacle inflation, but reevaluating the solution (now with catenary) to check if it is still collision-free after fitting the parabola. In case the solution is not feasible, we can slightly adjust the length of the tether so that it is collision-free again.


