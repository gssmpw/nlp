Once an initial feasible path is computed, we proceed to optimize the full robot system trajectory as a whole, considering dynamic constraints such as velocities or accelerations, and distance to obstacles. 

In this regard, this paper follows the methodology presented in \cite{smartinezr2023}. Unlike \cite{smartinezr2023}, instead of considering the length of the tether in the state vector, here we consider the set of parameters that actually define the tether curve. For both the catenary and parabola, the number of required parameters is three, so we increase the dimension of the state vector in two.  % with respect to \cite{smartinezr2023}. 
However, this new state vector improves the convergence and decreases the computation to obtain a feasible solution. Including the curve parameters in the state vector (instead of just the length) enables the optimizer to directly sample the tether and properly compute the gradients with respect to the curve itself. Section \ref{sec:experiments} will show that this new state vector reduces the optimization time by at least one order of magnitude.

%Thus, to define the state of the trajectory, we must incorporate the time variable $\Delta t^{i}$ into the path obtained from the path planner. For consistency, the UGV, UAV, and tether trajectories must use the same $\Delta t^{i}$ to coordinate the arrival of both platforms at the planned waypoint. 

Thus, in this paper, the state of the system in each time step includes the position of the UGV $\mathbf{p}_g=(x_g,y_g,z_g)^{T}$, the position of the UAV $\mathbf{p}_a=(x_a,y_a,z_a)^{T}$, the parameters that define the curve of the tether $\mathbf{T}$, and the time variable $t$. For consistency, the UGV, UAV, and tether trajectories must use the same $t$ to coordinate the arrival of both platforms at the planned waypoint. 

We discretize the trajectories so that the states of our problem become the set:
\begin{equation}
\label{eq:traj_params}
%    \Omega = \{\mathbf{p}^i_g,\mathbf{p}^i_a,p^{i}, q^{i}, r^{i},\Delta t^i\}_{i=1,...,n}
\Omega = \{\mathbf{p}^i_g,\mathbf{p}^i_a,\mathbf{T}^i,\Delta t^i\}_{i=1,...,n}
\end{equation}

\noindent where $\Delta t^{i} = t^{i} - t^{i-1}$ is the time increase between states $i$ and $i-1$, allowing temporal aspects. When needed, we also discretize the resultant tether model into a set of $m$ positions $\mathbf{p}_{t}=(x_{t}, y_{t}, z_{t})$:

\begin{equation}
\label{eq:tether}
    P^i = \{\mathbf{p}^j_t\}_{j=0,...,m-1}
\end{equation}

Our problem consists of determining the values of the variables in $\Omega$ (\ref{eq:traj_params}) that optimize a weighted multi-objective function $f(\Omega)$:  

\begin{equation}
\label{eq:cost_function}
     \Omega^* = \arg \min_\Omega f(\Omega) = \arg \min_\Omega  \sum_{i,k} \gamma_k * || \delta_k^i(\Omega) ||^2
\end{equation}

\noindent where $\Omega^*$ denotes the optimized collision-free trajectory for the UAV, the UGV, and the tether from the starting to the goal configurations. $\gamma_k$ is the weight for each component $\delta_k(\Omega)$ (also known as residual) of the objective function.  
%As in \cite{smartinezr2023},
 Each component encodes a different constraint or optimization objective for our problem and should be evaluated in all the time steps $i$.
In addition, the optimization problem is addressed using nonlinear sparse optimization algorithms, with \emph{Ceres-Solver} \cite{ceres-solver} serving as the optimization back-end.


%In this regard, this paper follows the same In this section, we detail the enhancements made to the optimization process of the method for finding feasible trajectories as described in \cite{smartinezr2023}. For the whole process, we consider a solution feasible when the trajectory is collision free for every agent of the system, including the tether. 

%As previously mentioned, the core of this improvement lies in the utilization of parabola parameterization to represent the tether. Unlike the original method, which focused on optimizing the length of the catenary curve, our approach leverages the curve parameters for optimization. 


% %---------------------------------------------------------------------------------------
\subsection{Tether parameterization}
\label{sec:tether_params}

We propose two different curves to model the tether: the catenary and the parabola. Both are parameterized by three variables:
\begin{eqnarray}
    parabola(x) &=& px^2+qx+r \\
    catenary(x) &=& a\ cosh( (x-x_0)/a) + y_0 
\end{eqnarray}

\noindent Thus, the parameters of $\mathbf{T}$ in (\ref{eq:traj_params}) will be defined as \mbox{$\mathbf{T}=(a, x_0, y_0)^T$} for the catenary curve, and $\mathbf{T}=(p,q,r)^T$ for the parabola.

%We will see in Section \ref{sec:experiments} that, depending on the type of curve selected for tether parameterization, both the feasibility and the required computation are improved. 
%Although using the catenary as the dentition of the curve for the tether can be easily derived from the optimization process presented in \cite{smartinezr2023}, 
%Note that the path planner gives us a path consisting of a sequence of UAV and UGV poses with catenary lengths. Therefore, we would need to calculate initial parameters for
%the parabola curves so that we can use the parabola approximation. 
%The next section describes the process of initializing the tether for the parabola.

% %---------------------------------------------------------------------------------------
\subsection{Initialization}
\label{sec:mathappoach}

% described in the previous section is a free collision path for the whole agents. This path presents a 

The variables in the optimization (positions and tether states) are initialized from the feasible path provided by the RRT* planner of Section \ref{sec:path_planning}. This initial path consists of a sequence of UAV and UGV positions with catenary lengths at each position. To initialize each $\mathbf{T}_i$ we need the parabola that best fits such a catenary.

%The output of the decision problem between two suspension points \emph{A} and \emph{B} is a free collision catenary of length \emph{L}. The length is part of the feasible solution path computed by the RRT* planner. We need the parabola that best fits such a catenary.

The traditional approach \cite{9337759} to find the parabola parameters that best fit a catenary involves assigning the length of the catenary $L_C$ to the length of the parabola. Thus, the problem of computing the parameters of the parabola is reduced to computing the parameters of the curve passing through two suspension points, \emph{A} and \emph{B}, with a given length $L_C$. This can be obtained by solving the following nonlinear system of equations:

% \begin{eqnarray}
%   \label{eq:parabola_2}
%   \begin{array}{ccl}
%   y_A = p{x_A^2} + qx_A + r & \\
%   y_B = p{x_B^2} + qx_B + r & \\
%   L =  \int_{A}^{B} \sqrt{1+(\mathbf{P'(x)})^2}  \,dx =  \int_{A}^{B} \sqrt{1+(2px + q)^2}  \,dx  \\
%   \end{array}
% \end{eqnarray}

\begin{equation}
    \label{eq:parabola_2}
    \left.
        \begin{array}{cc}
            y_A = p{x_A^2} + qx_A + r & \\
            y_B = p{x_B^2} + qx_B + r & \\
            %\vphantom{\int_{A}^{B}} L=\int_{A}^{B}\sqrt{1+(\mathbf{P'(x)})^2}  \,dx=\int_{A}^{B}\sqrt{1+(2px+q)^2}\,dx
            \vphantom{\int_{A}^{B}} L_C=\int_{A}^{B}\sqrt{1+(2px+q)^2}\,dx
        \end{array}
        \right\}
\end{equation}

% We use this solution as input for our optimizer, so, we consider the catenary length to calculate the parabola length and optimize its parameters. Thus, the result of the optimization process is a parabola with length \emph{L}*, and the optimal catenary is obtained between \emph{A} and \emph{B} using \emph{L}*. The flow of the algorithm in the optimizer can be illustrated as follows:

% \begin{equation}
% \label{eq:parabola_1}
%     A, B , \mathbf{C}(L) 	\rightarrow  \mathbf{P}(L)  \rightarrow Optimization \rightarrow \mathbf{P}(L^{*}) \rightarrow \mathbf{C}(L^{*}),
% \end{equation}

% where $\mathbf{P}$(\emph{L}) is the parabola of length \emph{L} and $\mathbf{C}$(\emph{L}*) is the catenary of optimal length. 
% The problem of computing the parabola $\mathbf{P(x)}=px^2+qx + r$ passing through two suspension points $A$ and $B$ and length $L$ can be numerically solved using the system:

\noindent Notice that this integral can be solved analytically, but its solution is a complex equation establishing highly nonlinear relations among the parabola parameters. This makes the solution sensitive to parameter initialization.

To avoid nonlinear calculation, we propose an approach consisting of equalizing the area under the parabola between the segment connecting \emph{A} and \emph{B} with the area under the catenary curve, $a_C$. Using the parabola equation $px^2+qx+r$, we define a system of equations to obtain the parabola parameters \emph{p}, \emph{q} and \emph{r} in 2D (as a plane) as follows:

%Now, we assume that the output of the planning are \emph{A}, \emph{B} and the area \emph{a$\mathbf{_C}$} under the catenary (which is easy to calculate and approximate). 

% From \emph{A} and \emph{B} using \emph{L} is easy to calculate the area \emph{a$\mathbf{_C}$} under the catenary. From this, we compute parabola area \emph{a$\mathbf{_P}$}. Considering the parabola hanging from \emph{A} and \emph{B} and area \emph{a}, and using the parabola equation ($px^2+qx+r$) we define a equation system to obtain the parabola parameters \emph{p}, \emph{q} and \emph{r} in 2D as follows:

% \begin{eqnarray}
%   \label{eq:parabola_4}
%   \begin{array}{ll}
%   y_A = p{x^2}_A + qx_A + r & \\
%   y_B = p{x^2}_B + qx_B + r & \\
%   a =  \int_{A}^{B} (px^2 + qx + r)  \,dx  &
%   \end{array}
% \end{eqnarray}

\begin{equation}
    \label{eq:parabola_4}
    \left.
        \begin{array}{lll}
            y_A = p{x^2}_A + qx_A + r & \\
            y_B = p{x^2}_B + qx_B + r & \\
             a_C =  \int_{A}^{B} (px^2 + qx + r)  \,dx       
        \end{array}
        \right\}
\end{equation}

\noindent This is a relatively simple linear system of equations for $p,\ q,\ r$ and $y$, that can be solved efficiently. We will use this method to compute the best-fitting parabola.

The computed parabola parameters are used as part of the initial solution for the optimization process. Through this optimization, we obtain the optimized parabola parameters and then compute the corresponding catenary by means of the Algorithm \ref{alg:catenaryaproximation}. 

%The description of the general procedure to get the \emph{L$^{*}$} free of collision is presented in Algorithm \ref{alg:fromParToCat}.

%\begin{algorithm}[t!]
%\caption{Algorithm to compute optimized Catenary Length \emph{L$^{*}_{\mathbf{C}}$}.}
%\label{alg:fromParToCat}
%\begin{algorithmic}[1]
%\Require \emph{point} $A$, \emph{point} $B$, \emph{catenary area}  \emph{a$\mathbf{_C}$}
%\Ensure \emph{L$^{*}_{\mathbf{C}}$}: a catenary length free of collision hanging from $A$ and $B$ .
%\State $a\mathbf{_P} \gets a\mathbf{_C}$
%\State $p, q, r \gets solve\_equation\_system(A,B, a\mathbf{_P})$
%\State $p^{*}, q^{*}, r^{*} \gets \mathbf{optimization\_process}(A,B, p, q, r)$
%\State ${L^{*}}_{\mathbf{P}} \gets get\_parabola\_length(p^{*}, q^{*}, r^{*})$
%\State ${L^{*}}_{\mathbf{C}} \gets Algorithm\ 2(A,B, {L^{*}}_{\mathbf{P}})$
%\end{algorithmic}
%\end{algorithm}

% The parabola parameters pass to the optimization process, thus, we obtain the optimized parameter, and subsequently the area \emph{a$^{*}\mathbf{_P}$} of the best parabola and the corresponding catenary can be obtained from that. The flow of the new algorithm can be illustrated as follows:

% \begin{equation}
% \label{eq:parabola_3}
%     A, B , a\mathbf{_C}  	\rightarrow  \mathbf{P}(a\mathbf{_C})  \rightarrow Optimization \rightarrow a^{*}\mathbf{_P} \rightarrow \mathbf{P}(a^{*}{\mathbf{_P}})
% \end{equation}


%----------------------------------------------------------
% \subsection{States definition}

% As in our previous work \cite{smartinezr2023}, we define a state in the state space as the combination of the position of the UGV $\mathbf{p}_g=(x_g,y_g,z_g)^{T}$, the UAV $\mathbf{p}_a=(x_a,y_a,z_a)^{T}$ and the tether length $l$. At any instant, the tether length should be longer than the distance from the UGV to the UAV, i. e. $l \geq \|\mathbf{p}_a - \mathbf{p}_g\|$.

% The motion planning problem consists of determining the trajectory for the UGV $\mathbf{p}_g(t)$, the UAV $\mathbf{p}_a(t)$ and the tether length $l(t)$ so that the UAV reaches a given goal position while avoiding obstacles and respecting the constraints of the system. In this new approach, as was explained in \ref{sec:mathappoach}, instead of using the tether length $l(t)$ as a state to optimize, we use the parameters of the parabola $(p^{i}$, $q^{i}$, $r^{i})$.

% We discretize the trajectories so that the states of our problem become the set:
% \begin{equation}
% \label{eq:traj_params}
%     \Omega = \{\mathbf{p}^i_g,\mathbf{p}^i_a,p^{i}, q^{i}, r^{i},\Delta t^i\}_{i=1,...,n}
% \end{equation}

% \noindent where $\Delta t^{i} = t^{i} - t^{i-1}$ is the time increment between states $i$ and $i-1$, allowing us to consider the temporal aspects. This value is the same for UGV and UAV trajectories. For each $\mathbf{p}^i_a$ , $\mathbf{p}^i_{g}$ and $p^{i}$, $q^{i}$, $r^{i}$, there is a tether configuration $T^i$, given by the parable model mentioned above. When needed, we also discretize the resultant tether model into a set of $m$ positions $\mathbf{p}_{t}=(x_{t}, y_{t}, z_{t})$:

% \begin{equation}
% \label{eq:tether}
%     T^i = \{\mathbf{p}^j_t\}_{j=0,...,m-1}
% \end{equation}

% Our problem consists of determining the values of the variables in $\Omega$ \ref{eq:traj_params} that optimize a weighted multi-objective function $f(\Omega)$:  

% \begin{equation}
%      \Omega^* = \arg \min_\Omega f(\Omega) = \arg \min_\Omega  \sum_{i,k} \gamma_k * || \delta_k^i(\Omega) ||^2
% \end{equation}

% \noindent where $\Omega^*$ denotes the optimized collision-free trajectory for UAV, UGV and tether from the starting and goal configurations. $\gamma_k$ is the weight for each component $\delta_k(\Omega)$ (also known as residual) of the objective function. 
% As in \cite{smartinezr2023} each component encodes a different constraint or optimization objective for our problem, and should be evaluated in all the timesteps $i$.
% Also, the optimization problem is solved with non-linear sparse optimization algorithms ( \emph{Ceres-Solver} \cite{ceres-solver} as our optimization back-end). 

%----------------------------------------------------------
\subsection{Optimization process}
\label{sec:implementation}
% The optimizer considers geometric constraints such as obstacle distance, trajectory smoothness, and equi-distance between states. Moreover, it also considers temporal constraints including time, velocity, and acceleration. Temporal considerations apply to both platforms, UGV and UAV. These constraints are described and used in our previous work \cite{smartinezr2023}, where we optimized the mentioned constrained related to UGV, UAV, and tether. We represent the constraints in the problem as penalty costs in the objective function.

% About the tether constraints, we present a new approach compared to our previous work in \cite{smartinezr2023} and \cite{smartinezr2021}, where were consider the constraints related to obstacle distance and length for a catenary curve. Furthermore, these constraints were addressed by the optimizer as an AutoDiff Functor which solves the numerical method of Bisection to compute the transcendental equation for mechanical catenary. The new approach incorporates three constraints centered around a parabola curve. These constraints involve obstacle distances, parabola length, and parabola parameters, which force the parabola to pass through the desired endpoints. Next, we define the constraints included in the objective function to be minimized, which also are addressed as AutoDiff Functor but solving this time an analytic equation.

The optimizer takes into account geometric constraints such as obstacle avoidance, trajectory smoothness, and equi-distance between states. In addition, it considers temporal constraints including time, velocity, and acceleration, applicable to both the UGV and the UAV platforms. %These constraints, which we previously described and utilized in \cite{smartinezr2023}
The main set of constraints is inherited from the optimization method detailed in \cite{smartinezr2023}. In particular, we use the penalty functions related to the UAV, the UGV and the tether feasibility. However, we redefine the maximum tether length constraint and we include a new residual that ensures that the tether passes through the positions of both the UAV and UGV due to the new parametric formulation.

%They involve optimization related to UGVs, UAVs, and tether. We represent the constraints in the problem as penalty costs in the objective function (\ref{eq:tether}).

%Regarding the tether constraints, we introduce a new approach compared to our previous works \cite{smartinezr2023} and \cite{smartinezr2021}, where we considered constraints related to obstacle distance and length for a catenary curve. In these works, the optimizer addressed these constraints as an AutoDiff Functor that solved the numerical method of Bisection to compute the transcendental equation for the mechanical catenary.

%The new approach incorporates three constraints centered around a parabola curve. These constraints involve obstacle distances, parabola length, and parabola parameters, ensuring the parabola passes through the desired endpoints. We define these constraints in the objective function to be minimized, which are also addressed as AutoDiff Functors, but this time solving an analytic equation. 

Next, we present the new constraints required to consider the direct tether parameterization presented in (\ref{eq:traj_params}) for parabola and catenary:

%\subsubsection{Obstacle avoidance constraint}

%The tether state is computed by solving the parabola equation for parameters $p^{i}$, $q^{i}$ and $r^{i}$ between \emph{A} = $\mathbf{p}^i_g$  and \textbf{B} = $\mathbf{p}^i_a$. Using these values a parable is computing and sampling according to (\ref{eq:tether}) into $m$ points.

%This constraint penalizes the proximity of the sampled tether to obstacles. We compute the distance $d^i_{ot,j}$ to the nearest obstacles of every sample of tether $p^{i}$, $q^{i}$, $r^{i}$, and estimate the residual $\delta^i_{op}$ as the sum of the inverse nearest distances. We increase the weight of those samples closer than a safety distance $\rho_{ot}$ to guarantee higher cost in these cases using $\rho_{j}= \beta$, with $\beta >> 1$.

%\begin{eqnarray}
%  \label{eq:eq_teher_obst}
%  \delta^i_{op} &=& \sum_{j=0}^{m-1} \frac{\rho_{j}}{d^i_{op,j}}  , \  \rho_{j} = \left \{
%      \begin{array}{cc}
%      1 & ,if\  \  d^i_{op,j}  >  \rho_{op}\ \\
%      \beta & ,otherwise
%  \end{array}
%  \right .
%\end{eqnarray}

\subsubsection{Length constraint}
This constraint penalizes unfeasible tether lengths for the catenary and parabola. The tether cannot be shorter than the Euclidean distance between the UGV and the UAV $d^i_u=||\mathbf{p}^i_g-\mathbf{p}^i_a||$ and cannot exceed its maximum length $L_{max}$. Given the tether parameters $\mathbf{T}^i$ (parabola or catenary), we can analytically compute the length of the curve between the suspension points $l^i$. With this information, we define the following residual:
\begin{equation}
    \label{eq:eq_length}
    \delta^i_{up} =   e^{d^i_{u} - l^{i}} + e^{l^{i} - L_{max}} 
\end{equation}
%\begin{eqnarray}
%  \label{eq:eq_length}
%  \delta^i_{up} =& \left \{
%  \begin{array}{cc}
%  e^{(d^i_{up} - l^{i})} -1 & ,\textrm{if}\  \  d^i_{up} \ > \ l^{i} \ \\
%  e^{(l^{i} - L_{max})} -1 & ,\textrm{if}\  \  L_{max} \ < \ l^{i} \ 
%  \end{array}
%  \right. 
%\end{eqnarray}
\noindent Notice how the residual starts rising over zero when approaching the minimum and maximum thresholds, growing exponentially as we pass the thresholds.

\subsubsection{Parabola parameter constraint}
This is only applied when $\mathbf{T}^i$ is a parabola; it penalizes unfeasible solutions for $p^{i}$, $q^{i}$, and $r^{i}$. This constraint ensures that the parabola passes through the positions of the UGV and UAV, $\mathbf{p}^i_g$ and $\mathbf{p}^i_a$. The constraint projects such 3D positions to the 2D plane that contains the points and is perpendicular to the floor, obtaining the 2D suspension points A and B. The suspension points must comply with the following equations: 

%\begin{eqnarray}
%  \label{eq:eq_parameters}
%  \delta^i_{pA} = p{x^2_A} + qx_A + r - y_A \\
%  \delta^i_{pB} = p{x^2_B} + qx_B + r - y_B 
%\end{eqnarray}

\begin{equation}
  \label{eq:eq_parameters_parabola}
  \left.
    \begin{array}{lll}
        \delta^i_{pA} = p{x^2_A} + qx_A + r - y_A \\
        \delta^i_{pB} = p{x^2_B} + qx_B + r - y_B 
    \end{array}
    \right\}
\end{equation}


\subsubsection{Catenary parameter constraint}
This is only applied when $\mathbf{T}^i$ is a catenary; it penalizes unfeasible solutions for $a^i$, $x^i_0$, and $y^i_0$. This constraint ensures that the catenary passes through the positions of the UGV and UAV, $\mathbf{p}^i_g$ and $\mathbf{p}^i_a$. The constraint projects such 3D positions to the 2D plane that contains the points and is perpendicular to the floor, obtaining the 2D suspension points A and B. The suspension points must comply with the following equations: 

%\begin{eqnarray}
%  \label{eq:eq_parameters}
%  \delta^i_{cA} = a*cosh(\frac{x_A- x_0}{a}) + y_0 - y_A \\
%  \delta^i_{cB} = a*cosh(\frac{x_B- x_0}{a}) + y_0 - y_B
%\end{eqnarray}

\begin{equation}
  \label{eq:eq_parameters_catenary}
  \left.
  \begin{array}{lll}
  \delta^i_{cA} &=& a\ cosh(\frac{x_A- x_0}{a}) + y_0 - y_A \\
  \delta^i_{cB} &=& a\ %.000000000000000
  cosh(\frac{x_B- x_0}{a}) + y_0 - y_B
  \end{array}
\right\}
\end{equation}


% ME FALTA ACTUALZAR ESTO
%\delta^i_{op},  d^i_{op,j} ,  \rho_{op} , \delta^i_up , L_{max} , d^i_{up} , \delta^i_{pA} , \delta^i_{pB}, \delta^i_{pA}, \delta^i_{pB}

%Debo agregar en algun lado que el tratamamiento de las ecuaciones son en 2D por lo que siempre se hace una conversión de 3D a  2D

