%% Included from original template (https://template-selector.ieee.org/secure/templateSelector/publicationType) 
\documentclass[lettersize,journal]{IEEEtran}
%\usepackage{amsmath,amsfonts}
% \usepackage{algorithmic}
\usepackage{algorithm}
\usepackage{array}
\usepackage[caption=false,font=normalsize,labelfont=sf,textfont=sf]{subfig}
\usepackage{textcomp}
\usepackage{stfloats}
\usepackage{url}
\usepackage{verbatim}
\usepackage{balance}
\usepackage{cite}
\hyphenation{op-tical net-works semi-conduc-tor IEEE-Xplore}

% Added command
\IEEEoverridecommandlockouts   
\usepackage{graphics}
\setlength{\marginparwidth}{2cm}
\usepackage{multirow}
\usepackage{multicol}
\usepackage{graphicx}
\usepackage{adjustbox}
\usepackage{hyperref}
\usepackage{breqn}
\usepackage{siunitx}
\usepackage{hhline}
\usepackage{todonotes}
\usepackage{comment}
\usepackage{csquotes}          % Must be loaded when babel is loaded to avoid error.
\usepackage{amsmath,amssymb,amsthm,amsfonts}
\usepackage{algpseudocode}
\usepackage[left]{lineno}
\newtheorem{theorem}{Theorem}
\newtheorem{remark}[theorem]{Remark}
\newtheorem{lemma}[theorem]{Lemma}
\newtheorem{corollary}[theorem]{Corollary}
\theoremstyle{definition}
\newtheorem{definition}[theorem]{Definition}
\renewcommand{\labelitemii}{$\circ$}
\def\R{{\mathbb R}}
\newcommand{\cat}[0]{\mathcal{C}}

%Review command
\usepackage{todonotes}
\newcommand{\rev}[1]{\textcolor{black}{#1}}
\newcommand{\revS}[1]{\textcolor{black}{#1}}
\newcommand{\inma}[1]{\textcolor{black}{#1}}
\newcommand{\ma}[1]{\textcolor{black}{MA: #1}}
\newcommand{\luis}[1]{\textcolor{black}{#1}}

\newcommand{\reviewComment}[2]{\todo[inline, linecolor=blue,backgroundcolor=blue!25,bordercolor=blue]{\textbf{Reviewer #1: }\emph{#2} } }

\title{\LARGE \bf Efficient variable-length hanging tether parameterization for marsupial robot planning in 3D environments*}

\author{S. Mart\'inez-Rozas$^{1}$, D. Alejo$^{2}$, F. Caballero$^{2}$, L. Merino$^{2}$, M.A. Pérez-Cutiño$^{3}$, \\ F. Rodríguez$^{3}$, V. Sánchez-Canales$^{3}$,
I. Ventura$^{3}$ and J.M. Díaz-Báñez$^{3}$% <-this % stops a space
\thanks{*This work was partially supported by the grants: 1) INSERTION PID2021-127648OB-C31 and NORDIC TED2021-132476B-I00, funded by MCIN/AEI/ 10.13039/501100011033 and the “European Union NextGenerationEU/PRTR”, and 2) PID2020-114154RB-I00 and TED2021-129182B-I00, funded by
MCIN/AEI/10.13039/501100011033 and the “European Union NextGenerationEU/PRTR.}% <-this % stops a space
\thanks{$^{1}$S. Mart\'inez-Rozas is with Universidad de Antofagasta, Antofagasta, Chile. Email: {\tt\small simon.martinez$@$uantof.cl}}
\thanks{$^{2}$D. Alejo, F. Caballero and L. Merino are with Service Robotics Laboratory, Universidad Pablo de Olavide, Seville, Spain. Email: {\tt\small daletei$@$upo.es}, {\tt\small fcaballero$@$us.es}, {\tt\small lmercab$@$upo.es}}
\thanks{$^{3}$ F. Rodríguez, M.A. Pérez-Cutiño, V. Sánchez, I. Ventura and J.M. Díaz-Báñez is with Department of Applied Mathematics II, Universidad de Sevilla, Spain. Email: {\tt\small frodriguex@us.es}, {\tt\small migpercut$@$alum.us.es}, {\tt\small iventura$@$us.es}, {\tt\small vscanales$@$us.es}, {\tt\small dbanez@us.es}} 
}

\begin{document}
\maketitle
\thispagestyle{empty}
\pagestyle{empty}

\begin{abstract}

This paper presents a novel approach to efficiently parameterize and model the state of a hanging tether for path and trajectory planning of a UGV tied to a UAV in a marsupial configuration. Most implementations in the state of the art assume a taut tether or make use of the catenary curve to model the shape of the hanging tether. The catenary model is complex to compute and must be instantiated thousands of times during the planning process, becoming a time-consuming task, while the taut tether assumption simplifies the problem, but might overly restrict the movement of the platforms. In order to accelerate the planning process, this paper proposes defining an analytical model to efficiently compute the hanging tether state, and a method to get a tether state parameterization free of collisions. We exploit the existing similarity between the catenary and parabola curves to derive analytical expressions of the tether state. 
%This paper also presents how to adapt and integrate such parameterization into a motion planning approach for a tethered UAV-UGV robotic system \cite{smartinezr2023}. %In this way, the main contributions of this article are: 1) Present a simplified and efficient approach for modeling collision-free tether paths using a parabola approximation, enhancing computational speed; and 2) Integrating tether parameters directly into the trajectory state, improving optimization accuracy and reducing computation time.
The comparative analysis between the baseline and the proposed new method demonstrates that we can generate obstacle-free trajectories for UAVs, UGVs, and tethers in a fraction of the time while increasing the feasibility of the computed solutions. Finally, the source code of the method is publicly available.

\end{abstract}

%----------------------------------------------------------
\section{Introduction and Related Work}
\label{sec:introduction}
In recent years, there has been a notable increase in the development and research of tethered UAVs, reflecting a growing interest in their diverse applications. One of the main motivations is to carry out long-term missions with aerial vehicles, as these present significant challenges due to the limitations of current battery solutions \cite{robotics12040117}. A UAV tethered to a UGV is an interesting configuration, as the UGV can power the UAV through the tether for longer times given the higher payload of the former.  %According to this, an interesting configuration to allow long-duration flights of a UAV is a tethered robot configuration in which a UGV is tied to the UAV and powering it. 
This introduces a paradigm in robotic collaboration, offering distinct advantages over traditional standalone systems by combining the strengths of each of the robotic agents \cite{MooreIROS2018}. %As we venture UGV tied to UAV into scenarios requiring heightened enhanced situational awareness involving an extended operational endurance, the tethered approach proves invaluable, due to the capability to provide energy to the UAV, thus increasing fly time \cite{6961531}. 
When deploying a UGV tethered to a UAV in scenarios requiring increased situational awareness and extended operational endurance, the tethered configuration can become even more invaluable, not only providing the UAV with power to significantly extend its flight time \cite{6961531},  %In this way, the cable plays an important role in providing 
but also with safe high-bandwidth communications \cite{850822,9202196}. 

However, the tethering mechanism introduces several challenges, particularly in modeling the hanging tether state \cite{XiaoSSRR2018}. Unlike standalone systems, where each vehicle operates independently, the tether requires intricate and permanent coordination between the UGV and the UAV. Understanding and managing the state of the tether becomes a critical aspect, which requires sophisticated algorithms and real-time processing capabilities \cite{9561062}. 

\begin{figure}
  \includegraphics[width=0.2\textwidth]{Figures/setup1.png}
  \hfill
  \includegraphics[width=0.2\textwidth]{Figures/setup2.png}
  \caption{Simplified 2D sketch showing an example for motion planning of a tethered UAV-UGV with a hanging tether. (Left) Initial robots and tether configuration, and UAV goal (red circle). (Right) Sequence of robots positions and tether length to reach the given goal. Notice how the goal cannot be reached by means of a taut tether, a hanging tether must be considered in this case.}
  \label{fig:planning-setup}
\end{figure}

The state of the tether has traditionally been analyzed through parameterization, an approach that employs equations to represent its physical behavior, especially the catenary curve \cite{BOOKOFCURVES}. Numerous methodologies, with the aim of simplifying this process, approximate the tether as a straight line \cite{autonomousvisual}\cite{framworktether}\cite{uavfire}. This straight-line approximation is only suitable in scenarios where there is a direct line of sight between the tether endpoints, and thus it inherently restricts the exploratory range of the UAV.

In general, hanging-tether approaches allow UAVs to access a broader range of areas compared to straight tether setups; see Fig. \ref{fig:planning-setup} for an example. This concept has been explored by incorporating tether parameterization into localization or planning processes. For instance, Lima and Pereira \cite{9476778} use the catenary equation to determine the UAV's position.  % This concept has been explored by incorporating tether parameterization into the localization or planning processes, such as in the work conducted by Lima and Pereira \cite{9476778}, where using the catenary equation is feasible to find the UAV position. 
Similarly, in \cite{9364354}, the focus is on computing the state of a catenary tether to localize two UAVs attached at each end. This setup is specifically designed to suspend an object, providing a novel approach to object manipulation using UAVs while maintaining a constant tether length. Another interesting application of the catenary model is presented in \cite{LARANJEIRA2020107018} for underwater operations, where the catenary is used to monitor the status of a cable connected to an \emph{N}-number of ROVs (Remotely Operated Vehicles) performing exploration tasks, also with a constant tether length.

In \cite{8848946}, the parameterization of the tether is used in the localization and control stages to perform two autonomous motion primitives, reactive feedback-based position control and model-predictive feedforward velocity control, but is not used in the planning stage. An interesting approach is presented in \cite{drones7020073}, where a tied unmanned aerial vehicle (TUAV), named ``Oxpecke'', was designed for the inspection of stone-mine pillars. This system uses a sweeping (lawnmower) pattern path planning method intended to map and inspect an entire rectangular area, such as the surface of a pillar. However, the surface to inspect is simple (a rectangle), and the tether length is not directly included in the path planning.

%A general approach about the consideration of the tether in the planning stage is introduced in \cite{battocletti2024entanglementdefinitionstetheredrobots}, where the authors present the definition of tether entanglement problems. Specifically, it addresses the challenges posed by the presence of a tether, including the geometric constraints on the robot's motion due to the finite tether length. For that, different constraints are considered in the planning stage. However, the method is too general and mainly tested in ground points, so UAV implementation are not considering, and algo this method allow tether contact with the floor while entanglement desnt exit.

A comprehensive approach to incorporate a tether in the planning stage is presented in \cite{battocletti2024entanglementdefinitionstetheredrobots}, where the authors define the challenges associated with tether entanglement. Specifically, this work addresses the constraints imposed by the tether on the motion of the robot, particularly the limitations arising from the finite length of the tether. Various constraints are integrated into the planning stage to account for these challenges. However, the proposed method is limited and mainly focused on ground applications, 
thus limiting its applicability to UAVs. Additionally, the approach allows for tether contact with the ground, as long as it does not result in entanglement.

On the other hand, \cite{capitán2024efficientstrategypathplanning} focuses on the development of a path planning strategy for marsupial robotic systems composed of a UGV tethered to a UAV. The article introduces a sequential planning strategy called MASPA (Marsupial Sequential Path-Planning Approach), which allows calculating collision-free 3D trajectories for the tethered UAV-UGV system in complex scenarios, for which the UGV advances to a point where the UAV executes the take-off and then advances to a desired point. This method considers both the geometric limitations imposed by obstacles and the cable and the properties of the joint motion of both robots. A novel algorithm, the PVA (Polygonal Visibility Algorithm), is also presented to identify feasible take-off points and solve visibility problems for the UAV in a three-dimensional space. Despite the novelty of the approach, it is not able to consider coordinated planning of the UGV and the UAV at the same time.

In \cite{smartinezr2023}, the catenary approximation is used to parameterize the state of the tether and plan a collision-free trajectory, in which the UAV must achieve objectives using a hanging tether. However, using the catenary equation, the planning process becomes a time-consuming task, allowing only offline computations. %which makes the planning process to be carried out offline.

%This paper will focus on reducing the complexity associated with the calculation of the variable length hanging tether. We will propose an approach that efficiently calculates the tether state with a minimum representation error concerning the real state, and integrating it into a trajectory planning algorithm for a UGV-UAV tethered team. To this end, we test our approach in the motion planning method for a mobile UGV-UAV tethered system presented in \cite{smartinezr2023}, which is based on two stages. The first stage computes a free-collision path planning for UAV, UGV, and tether, using the RRT* algorithm. The second stage corresponds to a trajectory planning method based on nonlinear optimization that considers smoothness, speed, acceleration limitations of the UGV and UAV, and optimizes the tether configuration to maximize the distance from obstacles. %Unfortunately, considering the real catenary curve in the planner could make it computationally demanding, as shown in our previous work \cite{martinez2021optimization}. In it, we manage to design, implement and test in experiments a two-step optimized planner which considers the catenary shape. For this reason, we propose to approximate the shape of the tether as a parabola without affecting the safety of the planning system and making use of its simpler description to speed-up the computation of optimal paths.

%Our approach is based on the motion planning mentioned above due to the robustness of computed trajectories. Thus, we include in the first stage, a decision problem to set the initial tether length, to quickly obtain a collision-free state for the whole system. Furthermore, we propose a new planner-state parameterization and replace the use of the catenary equation with a parabola equation for estimating the shape of the tether. Thus, the main contributions of the article are:
%\begin{itemize}
%
%\item In Planner Stage: Solving the decision problem to find a collision-free parabola curve instead of the traditional catenary curve. This change allows the RRT* (Rapidly-exploring Random Tree) planner to calculate trajectories faster and more efficiently, since it avoids the computational complexity associated with the calculation of the catenary. The parabolic curve simplifies the collision decision process and increases the success rate in three-dimensional environments with obstacles.
%
%\item  In Optimizer stage: This stage introduces a direct parameterization of the tether in the trajectory state function, which includes the parameters of the curve (parabola or catenary) in the system state vector. This allows a more accurate evaluation of geometric constraints (such as distance to obstacles) and reduces the optimization time by up to an order of magnitude compared to previous methods, achieving safer and smoother trajectories for the UAV-UGV system.
%\end{itemize}

This paper focuses on reducing the complexity associated with the calculation of the variable length hanging tether. %The paper builds on the previous work of the authors \cite{smartinezr2023}, extending it with a new approach that efficiently calculates the tether state with a minimum representation error related to the actual state and a new parameterization of the tether curve in the trajectory optimizer for faster computation. Thus, 
The main contributions are listed below.

\begin{itemize}
    \item A new method for efficient computation of a collision-free catenary curve based on the parabola approximation. This paper proposes using the parabola curve to model the hanging tether curve, detailing the full pipeline, including the computation of the final catenary model. This method reduces the execution time of the path planner to great extent, since it avoids the computational complexity associated with the calculation of the catenary model for tether collision detection. This model also increases the feasibility of the trajectory planner approach, reaching an averaged 98\% of feasibility in the validation scenarios. 

    \item A direct parameterization of the tether in the trajectory state definition, which includes the parameters of the curve (parabola or catenary) in the system state vector. This allows a more accurate evaluation of geometric constraints (such as distance to obstacles) and reduces the optimization time \rev{by more than an order of magnitude} compared to previous methods, achieving safer and smoother trajectories for the UAV-UGV system. \rev{Such improvement opens the door to apply the proposed method to real-time local re-planning.}
\end{itemize}

%The experimental results will show how this new parameterization boosts the computation, while the parabola model will clearly improve the feasibility of the method over the catenary. 

%Thus, we include in the first stage, a decision problem to set the initial tether length, to quickly obtain a collision-free state for the whole system. Additionally, we replace the traditional catenary equation with a parabolic approximation to estimate the tether shape more efficiently. In the second stage of nonlinear optimization stage, we further simplify the process by parameterizing the tether instead of relying on the catenary model. This approach not only streamlines the representation of the curve but also facilitates more straightforward and efficient gradient calculations during optimization.

The paper is structured as follows. In Section \ref{sec:overview}, we show the general problem to be solved, whereas Section \ref{sec:approach} formalizes the solutions proposed. Section \ref{sec:path_planning} details the implementation of the solution within the planning stage. In Section \ref{sec:optimization_process}, we describe how curve parameterization is utilized to enhance the optimization process for trajectory computation. The experimental results are discussed in Section \ref{sec:experiments}. Finally, the paper is concluded in Section \ref{sec:conclusions}.


%----------------------------------------------------------
\section{Path and trajectory planning of tethered UAV-UGV in 3D overview}
\label{sec:overview}
The trajectory planning of a tethered UAV-UGV system can be summarized as the process of computing the sequence of obstacle-free UAV-tether-UGV positions, velocities, and accelerations that allow the UAV to reach a destination, given the starting configuration of the UAV-tether-UGV system (see Fig. \ref{fig:planning-setup} as an example). For the sake of generality, this paper focuses on a novel planning approach in which the tether has a variable and controllable length and might be hanging depending on the scenario. We can see in the example of Fig. \ref{fig:planning-setup} that the goal cannot be reached if only taut tether configurations are considered.

Using hanging tether configurations increases the chances of finding a suitable sequence of robot actions to the goal, but it also implies significant computational effort to model the tether shape. The catenary curve accurately models the shape of the tether and is the most common representation. Although fitting the catenary ${\cal C}$ can be easily implemented, the planner must solve this problem hundreds or thousands of times to check possible collisions of the tether with the environment in every UAV-UGV configuration, which becomes a time-consuming process. 

Our novel approach addresses the problem of efficiently computing the existence of a free-collision catenary ${\cal C}_{AB}$ that connects the position of the ground vehicle, $A$, and the position of the aerial vehicle, $B$. Adopting a common terminology, we call $A$ and $B$ the \emph{suspension points}, and the horizontal and vertical distance between $A$ and $B$ are called \emph{span} and \emph{sag}, respectively.

This paper proposes an approach that builds on top of the motion planning method for tethered UGV-UAV system presented in \cite{smartinezr2023}, where the computation of the catenary ${\cal C}$ is required as a representation of the tether's shape at two levels. First, in the path planning stage, they focus on a collision-free path for all agents. Second, in an optimization stage, where the objective is to optimize the previous path to obtain a trajectory, considering several constraints related to the UGV-UAV robotic configuration. 

%We will use as based method  the motion planning for tethered UGV-UAV system considered in paper \cite{smartinezr2023}, where the computing of the states and also the catenary ${\cal C}$ is required in two stages. First, into the path planning stage, where the focus is to ascertain a free-collision path for all agents. Second, in an optimization stage. Here, the objective is to optimize the previous path, considering several constraints related to the tethered UGV-UAV robotic configuration. 

%The next section will get into the details of our novel approach for estimating the tether shape efficiently in each stage. 



%\subsection{Technical notes: projection to 2D}

%The methods described so far for solving the Parabola Decision Problem work in a vertical plane $\Pi$. However, we usually describe the scenario by means of a polygonal mesh or a Point Cloud. Let us define $\Pi$ as the plane that holds:

%\begin{align}
%    \Pi & \parallel  \boldsymbol k \\
%    A,B &\in \Pi \\
%\end{align}
%where $A=\left(A_x,A_y,A_z\right)$ and $B=\left(B_x,B_y,B_z\right)$ are the extreme points of the tether. Finally, we process the 3D point cloud describing the obstacles in the environment, adding the points that are closest than a given distance $D_{min}$ to $\Pi$, as shown in Figure \ref{fig:slice}.

%Furthermore, we can define the unit vector of the 2D plane and the points A' and B' as follows:


%\begin{align}
%    \boldsymbol{i'} & \leftarrow  \boldsymbol i cos\alpha  + \boldsymbol j sin\alpha  \\
%    \boldsymbol{j'}& \leftarrow  \boldsymbol k \\
%    A' & =  O' \\ 
%    B' & =  \boldsymbol{i'} \sqrt{\left(B_x-A_x\right)^2+ \left(B_y-A_y\right)^2} + \boldsymbol{j'} \left(B_z - A_z\right) 
%\end{align}





%Thus, in the remainder of the section, we will focus on finding a catenary in
%the plane $\Pi$ that does not collide with any of the obstacles projected in it. For simplicity we will work in the new 2D coordinate system.

%\subsubsection{Preprocessing step}

%As the aforementioned process of obtaining the projection of the 3D obstacles to the plane $\Pi$ can be computationally demanding, in this paper we propose to sample the planes on the environment in a preprocessing step.

%Let us assume that we would like to get all the planes that point in the direction $\alpha_0$. To this end, we will start from one of the corners of the workspace $O$ and obtaining the first plane in this direction. By repeating the process moving in the normal direction to the plane, we will obtain all the planes on the workspace with direction $\alpha_0$ (see Figure \ref{fig:slice}).

%Then, we repeat the process for the different directions of interest, obtaining a planar representation of the workspace from some 3D model, which can be specified in terms of a Point Cloud or a polygonal mesh.

%\begin{figure}
%  \includegraphics[width=0.48\textwidth]{Figures/slice.png}
%  \caption{3D Obstacles of the environment are represented in changing colors
%    from red to green with increasing $z$. Then, the 2D projection in the vertical plane $\Pi$ is represented in blue.}
%  \label{fig:slice}
%\end{figure}


%----------------------------------------------------------

%Each component encodes a different constraint or optimization objective of our problem and will be presented next. Besides, each component should be evaluated in all the timesteps $i$ of the trajectory. These components are local with respect to $i$, as they only depend on a few number of consecutive states in general. Consequently, our optimization problem can be solved with non-linear sparse optimization algorithms. In particular, we use \emph{Ceres-Solver} \cite{ceres-solver} as our optimization back-end. 


%----------------------------------------------------------
\section{Efficient computation of a collision-free catenary curve between two points}
\label{sec:approach}

% Delete the text and write your Theory/ Background Information here:
%------------------------------------

%In this section, we will study the problemo find a catenary joining two points
%in the 3D space $A$ and $B$ that does not collide with obstacles $O$ in the environment
%$C_{obs}=\cup_{i=1}^NO_i$.

%Deciding whether there exists a collision-free catenary with bounded length in a three-dimensional scenario or not is a crucial task in some problems in Robotics, such as the path planning for a tethered aerial robot \cite{martinez2021optimization}.  

%In this section, we address the problem of the existence of a free-collision catenary ${\cal C}_{AB}$ connecting the ground vehicle position, $A$, and the aerial vehicle position, $B$. Adopting a common terminology, we call $A$ and $B$ the \emph{suspension points}, and the horizontal and vertical distance between $A$ and $B$ are called the \emph{span} and the \emph{sag}, respectively.

In this section, we present a new method for efficient computation of a collision-free catenary curve based on the parabola approximation. As was previously commented, the main idea is that the method enables the planner to calculate trajectories faster and more efficiently, since it avoids the computational complexity associated with the calculation of the catenary model for tether collision detection.

Considering the catenary as a planar curve, our input consists of a set of 2D obstacles $\cal{O}$ defined by polygons in a plane denoted $\pi$. This plane $\pi$ is perpendicular to the ground and passes through points $A$ and $B$. 

%To simplify the analysis, we assume that the obstacles are convex polygons. {\color{blue}
%Note that a non-convex polygon can be decomposed into a collection of convex polygons. Do we need convex polygons?}




We define a curve $C$ as collision-free in $\pi$ if it does not intersect with any obstacle in $\pi$.  Therefore, our focus is directed towards the following two-dimensional problem:


\vspace{.25cm}
\textsc{Catenary Decision Problem (CDP):} \emph{Given a set $\cal O$ of polygons in $\pi$, determine whether there exists a collision-free catenary 
${\cal C}_{AB}$ in $\pi$ linking $A$ and $B$ with length at most $L$.} 
%\vspace{.5cm}

%\subsection{Naive approach: catenary length sampling}
%\label{sec:naive_approach}



%================================================
%\subsection{The Parabola Decision Problem}
%\label{sec:dsproblem}

In \cite{smartinezr2023} a numerical approach was considered to determine the existence of a collision-free catenary suspended between points $A$ and $B$. The approach involves examining catenaries with increasing length, starting from $l = d(A,B)$ to $l=L$, with the addition of a specified increment $\Delta l$ until a collision-free trajectory is identified (success) or the maximum length is reached (failure). Furthermore, in each iteration, the catenary shape is sampled to verify for potential collisions.
Both the calculation of the catenary curve and the collision test are time-consuming. This involves solving transcendental equations, which is computationally demanding \cite{behroozi2014fresh}.
%The problem with this approach is that we have to first determine the catenary that passes through $A$ and $B$ with a given length.
%Furthermore, in order to check for collision on the catenary, we have to sample its shape. Hence, the sampling length $d$ of the catenary is a crucial parameter in this method, as a small $d$ could waste efforts by repeteadly calling to the collision checker, whereas a big step could make the method not able to detect some collisions between the catenary and the obstacles. %In addition, a careful tuning of the step length $\Delta l$ is crucial to make the algorithm accurate and efficient.
In addition, low values on $\Delta l$ could waste efforts by repeatedly calling the collision checker with very similar curves. In contrast, high values $\Delta l$ could cause the method to overlook some catenary lengths that might be collision-free.

\begin{figure}[t!]
  \centering
   \includegraphics[scale=0.35]{Figures/region-T_v2.png}  
%\includegraphics[scale=0.27]{Figures/poligonos3.png}  
%\includegraphics[scale=0.25]{Figures/lemma3-2-2red.png}  
\caption{Polygons inside the $T$ region, which is delimited by the black trapezoid $\overline{ABF_AF_BA}$. $A$ and $B$ are the suspension points; $F_A$ and $F_B$ are their projections on the floor. The blue polygon is \emph{weakly inscribed}  in the parabolic region $R(v^*)$ and $\overline{AB}$, delimited by the blue parabola.}  %\caption{Polygons inside $T$ are \emph{weakly inscribed}  in the parabolic region  $R(v^*)$. }
\label{fig:PDP}
\end{figure}

To overcome these issues, we have devised an alternative approach using a simpler curve.  In certain applications, such as the design of transmission overhead lines \cite{hatibovic2018algorithm}, the catenary has been replaced by a parabolic curve, as the parabola is a good approximation of a catenary if the sag is small \cite{hatibovic2020comparison}. However, in our scenario, the sag is relatively large, and we have to be careful with the parabolic approximation.
Moreover, our approach requires examining the reverse approximation: Given a collision-free parabola, we would like to compute an approximated catenary.

In the remainder of the section, we describe our proposed procedure for obtaining collision-free catenaries between two points. Initially, we address the decision problem using parabolas instead of catenaries. Subsequently, upon discovering a collision-free parabolic curve, we introduce a numerical method to approximate the parabola with a catenary. Finally, we ensure that the obtained catenary %remains
is collision-free by strategically expanding the obstacles.

Let $T$ be the open trapezoidal region in $\pi$ below the segment $\overline{AB}$, bounded by the ground and the two vertical lines passing through $A$ and $B$, as illustrated in Fig. \ref{fig:PDP}.

A parabolic curve between the points $A$ and $B$ divides the plane $\pi$ into two regions, one convex and one non-convex. Given a set of obstacles in the plane $\pi$, a parabolic curve is collision-free when all obstacles are contained in one of the two regions.

Let $P_{AB}(v)$ be the parabola defined by the points $A$, $B$, and $v$, where $v$ is a vertex of an obstacle in $T$. Denote by $\ell(v)$ the length of the parabola $P_{AB}(v)$ and
$R(v)$ the convex region defined by the segment $\overline{AB}$ and the parabola $P_{AB}(v)$.
%\begin{definition}
A polygon is \emph{weakly inscribed} in the region $R(v)$ if it is contained in $R(v)$
and at least one vertex of the polygon lies on the parabola $P_{AB}(v)$. 
%We call such vertex a \textit{tangent vertex}. 
%\end{definition}
%Clearly, if a polygon $O$ is weakly inscribed in the region $R(v)$, then its vertices touching the boundary of $R(v)$ belong to its convex hull $CH(O)$ and $CH(O)$ is also weakly inscribed in $R(v)$. Thus,
%our problem reduces to finding a parabola that does not collide with the convex hull of any polygon within $T$. 
Therefore, we are interested in solving the following auxiliary problem:

\vspace{.25cm}
\textsc{Parabola Decision Problem (PDP):} 
\emph{Given a set $\cal O$ of polygons %completely 
contained in $T$, decide whether there exists a collision-free parabola $P_{AB}(v)$ with a length at most $L$ passing through a vertex $v$ of a polygon in $\cal O$.}
%\vspace{.5cm}

%\ma{We define some notation...}





%================================================
\subsection{An iterative approach for solving PDP}
\label{solvingPDP}
%The proposed method for solving PDP involves computing the length of a parabolic arc or determining a set of polygons intersecting this  arc and, for sort, we use some notation: The length of $P_{AB}$ is $|P_{AB}|$ and $P_{AB} \cap \mathcal{O}$ is the subset of polygons of $\mathcal{O}$ that intersect $P_{AB}$. 
Given a polygon $O=\langle v_1,\cdots,v_n\rangle$ contained in  $T$, we say that a parabolic arc hanging from points $A$ and $B$, $P_{AB}$, intersects with the polygon when it intersects a face of the polygon, i.e., there are points on a face that lie both above and below the curve. We denote  $P_{AB} \cap \mathcal{O}$  the subset of polygons of $\mathcal{O}$ that intersect the parabolic arc $P_{AB}$. We call this set $\mathcal{O'}$ and $CH(\mathcal{O'})$ its convex hull. In what follows, we will denote the length of $P_{AB}$ as $|P_{AB}|$ when we do not expressly know the third point $v$ through which the parabola passes.


\begin{lemma} \label{cor: longest_parabola}  
  Let $O=\langle v_1,\cdots,v_n\rangle$ be a convex polygon contained in the open region $T$ and let $v^*$ be the vertex of $O$ such that $\ell(v^*)=\max_{v_i \in O}\ell(v_i).$ Then the polygon $O$ is weakly inscribed in $R(v^*)$. 
  Furthermore, for any vertex $v_i \in O$ such that $\ell(v_i)<\ell(v^*)$, $O$ is not weakly inscribed in $R(v_i)$.
\end{lemma}	

\begin{proof}
Since two different parabolas intersect at most in two points, for all vertices $v_i\in O$ with $\ell(v_i) \leq \ell(v^*)$, it follows that $R(v_i)\subseteq R(v^*)$ and
%. As a result, 
the polygon $O$ is weakly inscribed in the region $R(v^*)$. On the other hand, for all $v_i$ so that $\ell(v_i) < \ell(v^*)$, $R(v^*) \nsubseteq R(v_i)$. Thus, $v^* \notin R(v_i)$ and $O$ is not weakly inscribed in $R(v_i)$. Fig. \ref{fig:PDP} illustrates the proof.
\end{proof}

As a direct consequence of Lemma \ref{cor: longest_parabola}, we state the following remark:
\begin{remark}
\label{rmk:longest_parabola} 
Let $O$ be a polygon
contained in $T$ that intersects a parabolic arc $P_{AB}$. 
Then the parabolic arc $P'_{AB}$ of minimum length that does not intersect with $O$ such that $|P'_{AB}|>|P_{AB}|$ in the span $(A, B)$ is the parabola of maximum length passing through a vertex of $CH(O)$.
\end{remark}

Algorithm \ref{alg:decision_problem} illustrates an overview of the approach to solving the parabola decision problem. We use the result presented in Remark \ref{rmk:longest_parabola} to progressively increase the length of the parabola until a specified stopping condition is met. To initiate the process, the algorithm computes the line segment $\overline{\rm AB}$ in line 1, representing the parabolic arc of minimum length from $A$ to $B$. The %\inma{quitaría esto: value of the } 
current parabola is stored in $P_{AB}$ and is subsequently updated throughout the algorithm. 
The fundamental invariant upheld at each step ensures that any parabola with a length less than 
%The invariant maintained in every step of the algorithm is that any parabola with length lower than 
$|P_{AB}|$ is not valid.
The algorithm enters a loop from lines [2-11], setting stop conditions for PDP in lines [2-8].

If the current parabola touches the ground or if its length reaches the maximum value, then there is no collision-free parabola hanging from $A$ to $B$, line [2-4]. Then we compute the set of obstacles $\mathcal{O'}$ intersecting with $P_{AB}$, line 5. If this set is empty, $P_{AB}$ returns as a valid parabola. Otherwise, we compute the minimum length parabola $P'_{AB}$ that does not intersect with $\mathcal{O'}$ such that $|P'_{AB}|>|P_{AB}|$. Notice that any parabola of length less than $P'_{AB}$ is not collision-free, so the established invariant is preserved. From Remark \ref{rmk:longest_parabola}, $P'_{AB}$ is the parabola of maximum length that passes through the convex hull of $\mathcal{O'}$; hence, we update $P_{AB}$ with this parabola, lines [9-10]. At this point, a full cycle of the algorithm has been completed, so we return to line 2. The iterative process guarantees termination, as the algorithm systematically increases the length of the current parabola.
%The iterative process ensures termination, given that the algorithm consistently augments the length of the current parabola.


% Algorithm \ref{alg:decision_problem} illustrates an overview of the approach to solve the decision problem PDP, consisting in only one loop, lines 2-10. The loop starts considering the segment $\overline{AB}$ as the first candidate parabola $P_{AB}$, line 1. The loop has two stop conditions: 1) if the parabola $P_{AB}$ touches the ground or the length of ${P}_{AB}$ is greater than $L$, then there is no feasible solution; 2) if %the intersection set ${\cal O'} = P_{AB} \cap {\cal O}$ 
% the set of polygons in $T$ intersecting with the parabola, ${\cal O'}$, is empty, then $P_{AB}$ is a feasible parabola and the algorithm ends. When none of the aforementioned conditions are met, 
% %the set of polygons intersecting the parabola, ${\cal O'} = P_{AB} \cap {\cal O}$, is not empty and 
% the process is repeated considering a new parabola $P_{AB}$. %collision-free with $CH({\cal O'})$. 
% The new parabola \revS{with length $\ell(v_i)$} is calculated using Lemma \ref{cor: longest_parabola}, which guarantees that $P_{AB}$ does not collide with the convex hull of the set ${\cal O'}$, $CH({\cal O'})$. The loop always ends since the length of  $P_{AB}$ increases in every iteration. %If no feasible solution is found, at some point the parabolic arc length will exceed $L_{max}$ and the algorithm will stop. 
% This procedure solves the problem PDP in linear time with respect to the number of polygons in the environment. Note that, in this case, there is no need to sample the parabolic shape when searching for collisions. Similarly, we are not constrained to a predetermined set of catenary lengths, as in the naive method mentioned earlier.

\begin{algorithm}[t!]
\caption{Given the suspension points $A$ and $B$, and a set of polygons $\mathcal{O}$ inside $T$, %$T_{AB}$,
returns, if it exists, a collision-free parabolic arc $P_{AB}$ with maximum length $L$. 
$P_{AB}$ cannot touch the ground.
}
\label{alg:decision_problem}
\begin{algorithmic}[1]
\Require $A, B, \mathcal{O}, L$.
\Ensure Collision-free parabolic arc $P_{AB}$.

 \State $P_{AB} \gets \overline{\rm AB}$
\If{${P}_{AB}$ touches the ground \textbf{or} $|P_{AB}|> L$} 
    \State \Return None %\Comment{Parabolic arc can not touch the ground}
\EndIf

\State $\mathcal{O}' \gets P_{AB} \cap \mathcal{O}$
\If{$\mathcal{O}' = \emptyset$}
    \State \Return $P_{AB}$
\EndIf
% \State $P_{AB} \gets \text{ parabola so that }CH(\mathcal{O}')$ \text{ is weakly inscribed }

%\State $P_{AB} \gets P_{AB}(v)$ \text{with} $CH(\mathcal{O}')$ \text{weakly inscribed in} $R(v)$
\State $v^* \gets \text{argmax}_{v_i \in CH(\mathcal{O}')}\ell(v_i)$
\State $P_{AB} \gets P_{AB}(v^*)$
\State GoTo (line 2)
\end{algorithmic}  
\end{algorithm}

%\subsubsection{Correctness of the approach}

% Lemma \ref{cor: longest_parabola}   guarantees the correctness of the approach outlined in Algorithm \ref{alg:decision_problem}. 


%================================================
\subsection{Computing the collision-free catenary from a parabola}
\label{sec:approxlength}

Algorithm \ref{alg:decision_problem} yields a parabolic curve that solves the PDP. In this section, we introduce three methods for approximating the parabolic curve with a catenary, representing the shape of a hanging tether. Following an experimental comparison, we choose one of these methods.

\subsubsection{Method 1. Approximation by length}
\label{sec:bylength}
In our first method, we just impose the catenary to be of the same length as the obtained parabola. Let $\lambda$ be the length of the parabola that connects the suspension points $A=(0, h_1)$ and $B=(S, h_2)$, and let $(x_{min}, y_{min})$ be the coordinates of the lowest point of the catenary with length $\lambda$ passing through $A$ and $B$. Note that $h_1, h_2, x_{min},  y_{min} > 0$ and $\lambda > d(A,B)=\sqrt{S^2 +(h_2-h_1)^2}$.
Taking the general equation of the catenary \cite{hatibovic2020comparison}: 
\begin{equation}
y_{cat}(x)=2c\sinh\left(\frac{x-x_{min}}{2c}\right)+y_{min}, c>0,
\end{equation}
the solution of the following system of non-linear equations provides the expected catenary: 
\begin{equation}
\left.
    \begin{array}{lll}
    2c \sinh^2 {\left(\frac{-x_{min}}{2c}\right)} + y_{min} & = & h_2 \\
    2c \sinh^2 {\left(\frac{S-x_{min}}{2c}\right)} + y_{min} & = & h_1 \\
    c\left[ \sinh{\left( \frac{S-x_{min}}{c}\right)}-   \sinh{\left( \frac{-x_{min}}{c}\right)} \right] & = & \lambda
    \end{array}
    \right\}
\end{equation}

\medspace

%The independent terms of the system are the coefficients of the general equation: $c$, $x_{min}$, and $y_{min}$. 

%The system has a unique solution since $\ell > \sqrt{S^2 +(h_2-h_1)^2}$.

\subsubsection{Method 2. Fitting the catenary with a parabola}
\label{sec:approxfitting}
An alternative method is to find a catenary that fits the parabola, specifically the catenary that minimizes the maximum vertical distance with the parabola. To design an efficient fitting algorithm, some mathematical properties are used.
%The following result can be proved by elemental Calculus.

From the convexity of the catenary, the following Lemma can be stated:
\begin{lemma}
\label{le:big_area_small_length}
    Let $\cat_1$, $\cat_2$ be two catenaries hanging from the same suspension points $A=(x_1, h_1)$ and $B=(x_2, h_2)$ such that $\forall x\in (x_1, x_2)$ $\cat_1(x)> \cat_2(x)$. Then the area under $\cat_1$ in $[x_1, x_2]$ is greater than the area under $\cat_2$ in the same interval, but its length is lower.
\end{lemma}
% \begin{proof}
%     In a concave region, the curve joining two consecutive vertices $v_1, v_2$ is the curve within the region with the lowest length touching $v_1$ and $v_2$. From this we get than any curve touching $A$ and $B$ in the region defined by $\cat_1$ in the interval $[x_1, x_2]$ has greater length than $\cat_1$. On the other hand, the area under $\cat_1$ can be divided in the area under $\cat_2$ plus the area between both curves; hence the area under $\cat_1$ is bigger than the area under $\cat_2$.
% \end{proof}

We use an additional result extracted from \cite{parker2010property}:
\begin{theorem}
\label{theo:cool_cat_property}
    Given a catenary curve $\cat$ and any horizontal interval $(x_1, x_2)$, the ratio defined by the area under $\cat$ divided by the length of the curve in that interval is independent of the value of $x_1$ and $x_2$.
\end{theorem}

Now we are ready to prove the following statement:
\begin{theorem}\label{teo:only_two_in_common}
Two different catenaries hanging from the same suspension points cannot have more than two points in common.
\end{theorem}
\begin{proof}
    (By contradiction). Let $\cat_1$, $\cat_2$ be two catenaries that hang from $A=(x_1, h_1)$ to $B=(x_2, h_2)$ such that they intersect at at least three points within the interval $[x_1, x_2]$. Let $a=(x_a, h_a)$, $b=(x_b, h_b)$, $c=(x_c, h_c)$ with $x_a<x_b<x_c$ be the intersection points,
    %three consecutive points where $\cat_1$ and $\cat_2$ intersects, 
    and let $\cat_i^a$ and $\cat_i^c$ indicate the arcs of the catenary $\cat_i$ contained in the interval $[x_a, x_b]$ and $[x_b, x_c]$, respectively. Without loss of generality, we assume $\forall x\in (x_a, x_b)$ $\cat_1^a(x)> \cat_2^a(x)$; hence $\forall x\in (x_b, x_c)$ $\cat_1^c(x)< \cat_2^c(x)$. Let $r_1$ ($r_2$) be the area under $\cat_1$ ($\cat_2$) in $[x_a, x_b]$ divided by the length of $\cat_1^a$ ($\cat_2^a$). Let $A_1, A_2, C_1, C_2$ be the area under $\cat_1^a,\cat_2^a, \cat_1^c,\cat_2^c$ respectively, and $L_1, L_2, L_3, L_4$ their lengths. From Theorem \ref{teo:only_two_in_common}, $r_1=\frac{A_1}{L_1}=\frac{C_1}{L_3}$ and $r_2=\frac{A_2}{L_2}=\frac{C_2}{L_4}$. In addition, from Lemma \ref{le:big_area_small_length}, we know that $A_1=A_2+\epsilon_1$ ($\epsilon_1>0$), and $L_1<L_2$; then 
    \begin{equation}
    r_1=\frac{A_2+\epsilon_1}{L_1}>\frac{A_2+\epsilon_1}{L_2}>\frac{A_2}{L_2}=r_2  
    \end{equation}
    \noindent On the other hand, from Lemma \ref{le:big_area_small_length} we know $C_2=C_1+\epsilon_2$ ($\epsilon_2>0$), and $L_3>L_4$; then 
    \begin{equation}
    r_2=\frac{C_1+\epsilon_2}{L_4}>\frac{C_1+\epsilon_2}{L_3}>\frac{C_1}{L_3}=r_1
    \end{equation}
    \noindent From the above we get 
    %$r_1>r_2$ and $r_1<r_2$ which is 
    a contradiction and the result follows. 
\end{proof}

% From \cite{parker2010property} the following theorem can be extracted.
% \begin{theorem}
%     Given a catenary curve $\mathcal{C}$ and any horizontal interval $(t_1, t_2)$, the ratio defined by the area under $\mathcal{C}$ divided by the length of the curve on that interval is independent of the value of $t_1$ and $t_2$.
% \end{theorem}
% This theorem characterize a catenary by its area-length ratio; hence two catenaries with different ratios are different. Using this result, the following theorem can be probed by contradiction.

% \begin{theorem}\label{teo:ivorra2}
% Two different catenaries with suspension point $A$ and $B$ cannot have more than two points in common.
% \end{theorem}

As a direct consequence of Theorem 
% \ref{teo:ivorra2}, 
\ref{teo:only_two_in_common}
%we know that two catenaries with the same suspension points are disjoint and thus, 
we have: 
\begin{corollary}\label{cor:monotony}
%$C_{AB}$ be a catenary of length $l$ connecting the suspension points 
%$A=(0, h_1)$ and $B=(S, h_2)$, $S>0$, be the suspension points and 
Let ${C}_{AB}^{i}(x)$ be the catenary function of length $l_i$ that connects points $A$ and $B$. If $l_i \geq l_j$, $j\neq i$, then ${C}_{AB}^{j}(x) \geq {C}_{AB}^{i}(x)$, for all $x\in [0,S]$.
\end{corollary}

%\begin{proof}
%The proof is straightforward using Theorem \ref{teo:ivorra2}.
%\end{proof}

The monotonicity property of Corollary \ref{cor:monotony} allows us to implement a bisection method within the allowable length interval of the tether, namely $[d(A, B), L]$, to identify an appropriate catenary setting.
%Using this monotonicity property, a bisection method can be implemented on the feasible length interval of the tether, that is, $[d(A,B),L]$, to identify a suitable catenary fit. 
As demonstrated in our experiments, this method proves to be efficient, delivering accurate approximations with minimal iterations. Algorithm \ref{alg:catenaryaproximation} depicts the pseudocode of the solution.

\begin{algorithm}[t!]
\caption{Algorithm to compute a fitting catenary.}
\label{alg:catenaryaproximation}
\begin{algorithmic}[1]
\Require \emph{point} $A$, \emph{point} $B$, \emph{parabola} ${P_{AB}}$, \emph{maximum length} $L$, \emph{approximation error} $\varepsilon$
\Ensure $\mathcal{C}_{AB}:$ a fitting catenary hanging from $A$ and $B$.
\State $L_0 \gets |\overline{AB}|$
\State $L_1 \gets L$
\While{$True$}
\State $L_m \gets (L_0+L_1)/2$
\State $\mathcal{C}_{AB} \gets get\_{cat}atenary(A,B,L_m)$
\If{$L_1-L_0 \leq \varepsilon$}
    \Return $\mathcal{C}_{AB}$
\EndIf
\State $x_{max} \gets get\_max\_distance\_axis({P_{AB}},\mathcal{C}_{AB}, A, B)$
\State $d \gets {P_{AB}}(x_{max}) - \mathcal{C}_{AB}(x_{max})$
\If{$d < 0$}
    \State $L_0 \gets L_m$
\Else{}
    \State $L_1 \gets L_m$
\EndIf

\EndWhile
\end{algorithmic}
\end{algorithm}


In Algorithm \ref{alg:catenaryaproximation}, the suspension points $A=(0, h_1)$ and $B=(S, h_2)$ and the maximum length of the expected catenary, $L$, are taken as input. In addition, an approximation error $\varepsilon$ is received to determine the accuracy of the output. The metric used in the bisection is the difference in terms of lengths between two iterations. First, the minimum and maximum lengths of the expected catenary are defined as $L_0$ and $L_1$, respectively, lines 1 and 2. These values are obtained from the parameters $A$, $B$, and $L$ and define a feasible interval of catenary lengths. Later, the algorithm performs a loop to estimate the optimal catenary length in the range $[L_0, L_1]$, lines 3-15. In each iteration, the mean length between $L_0$ and $L_1$, $L_m$, is calculated, obtaining the catenary of length $L_m$, $\mathcal{C}_{AB}$, lines 4 and 5. In the next step, the value: 
\begin{equation}
 x_{max} = max_{x \in [0, S]}|{P_{AB}}(x) - \mathcal{C}_{AB}(x)|   
\end{equation}
\noindent is calculated using the function $get\_max\_distance\_axis$ on line 8, and the maximum vertical difference $d$ between ${P_{AB}}$ and $\mathcal{C}$ is obtained on line 9. The value $x_{max}$ can be obtained by solving a numerical problem or can be approximated by sampling the interval $[0, S]$. Using Corollary \ref{cor:monotony}, the sign of the difference $d$ is the condition to discard an entire subinterval, $[L_0,L_m]$ or $[L_m,L_1]$.
The stop criterion is outlined at line 6, and the algorithm returns the last calculated catenary $\mathcal{C}_{AB}$. 
%The number $N$ of iterations is: $$ N = \log _2 (L - |\overline{AB}| - \varepsilon)$$
%Another stop criterion could be to define a maximum tolerance $\theta$ and return the first catenary with maximum vertical distance to the parabola smaller than $\theta$. In this case, the value $\theta$ should be carefully defined for the correct convergence of the bisection procedure.


\subsubsection{Method 3. Fitting the catenary by sampling the parabola points}
\label{sec:bypoints}

This method consists of fitting the catenary to the parabola through a nonlinear optimization process to minimize the distance to $n$ points sampled on the catenary and the parabola. For this, the $n$ points of the parabola are considered in a plane X-Y. The nonlinear optimizer creates $n$ constraints that must minimize the Y-distance between both curves, parabola and catenary. This translates into the following optimization problem.

\begin{equation}
     cat(x)^*=\arg \min_{cat} \sum_{i=1}^n ||par(x_i) - cat(x_i)||^2
\end{equation}

%\begin{equation}
%    \label{eq:const_byPoints}
%    R_i = y_c(x_i) - y_p(x_i)  \quad \text{for} \quad i = 1, 2, \dots, n
%\end{equation}

\noindent where $cat(x)=a\ cosh( (x-x_0)/a) + y_0$ represents the catenary equation in which the parameters $a$, $x_0$ and $y_0$ must be estimated, and $par(x)$ is the parabola equation to fit.



\subsubsection{Benchmarking}

Now, the methods for computing an approximation catenary are compared each other. The methods of Sections \ref{sec:bylength}, \ref{sec:approxfitting}, and \ref{sec:bypoints} are denoted by \emph{ByLength}, \emph{ByFitting} and \emph{BySampling}, respectively.
%The first method, denoted as \emph{ByLength}, consists of taking a catenary of the same length as the parabola (see Section \ref{sec:approxlength}). Alternatively, a fitting algorithm is proposed to approximate the optimal catenary that minimizes the maximum vertical distance between the catenary and the parabola, \emph{ByFitting} (see Section \ref{sec:approxfitting}). Fig. \ref{fig:approx_cat} shows the catenaries obtained with both techniques for a given parabola. 
A set of random experiments was performed to compare the methods in terms of the maximum vertical distance between the reference and the approximated curves. In the experiments, the approximation error $\varepsilon$ selected for the \emph{ByFitting}, as appears in Algorithm \ref{alg:catenaryaproximation}, was $10^{-2}$, and the number $n$ of points in \emph{BySampling} was 5.

\begin{comment}
\begin{figure}[t!]
    \centering
    \includegraphics[width=0.45\textwidth]{Figures/approx_curves.png}
    \caption{Approximation catenaries by the three methods.
    }
    \label{fig:approx_cat}
\end{figure}
\end{comment}

A set of 100 experimental scenarios was generated. For each scenario, the suspension point $A$ is set to $(0,1)$, while both the suspension point $B$ $(B_x,B_y)$ and the parabola's longitude are generated randomly. In accordance with the intended application, the parabolic curve is generated in such a way that the point $B$ is positioned higher than the point $A$. The maximum allowed curve length $L$ is set to $30$ meters.

Two metrics within the interval $[0,B_x]$ are taken into account to evaluate the performance of the approximation methods:
\begin{itemize}
%     \item [1.] %Maximum Vertical Distance (
% $\delta_{max}$: The maximum vertical distance between both curves. %between the suspension points, $x\in[0,S]$.  
%     \item [2.] %Total Vertical Distance (
% $\delta_{Total}$: The sum of the vertical distances. 
% %for all $x$ in $[0,S]$. 
%     \item [3.] %Mean Vertical Distance (
% $\delta_{mean}$: The mean vertical distance. %in the interval $[0,S]$.  
    % \item [4.] %Computation Time (
    % $C.T.$: The total computational time needed to compute the approximation curve. 
        \item [1.] $\epsilon_{mean}$: The mean vertical distance in the interval $[0,B_x]$ between both curves, catenary and parabola.  
        \item [2.] $\epsilon_{mean}/L$:  Is the mean vertical distance over the length of the curve in the interval $[0,B_x]$.
\end{itemize} 


% $\delta_{max}$, $\delta_{Total}$ and $\delta_{mean}$ are calculated by sampling the interval. For each scenario, the four metrics are evaluated in the approximation methods. Table \ref{tab:comparison} shows the mean and standard deviation of the results of all experimental scenarios. }%Table \ref{tab:comparison} shows that 
% \emph{ByFitting} presents the best performance. %between all methods in terms of vertical distance. 
% %\emph{ByFitting} presents the best performance between the methods in terms of vertical distance, that means, catenary and parabola curves are similar.
% This method yielded precise results, achieving less than $1.1~m$ of the maximum vertical distance, less than $3~m$ of the total vertical distance, and less than $0.7~m$ of the mean vertical distance. Furthermore, both methods are efficient and can be used on-line.

%$\delta_{mean}$ is calculated by sampling the interval. For each scenario, the metrics are evaluated in the approximation methods. 

\noindent Table \ref{tab:comparison} shows the mean and standard deviation of the results of all experimental scenarios. It can be seen that \emph{BySampling} presents the best performance among the methods in terms of vertical distance, which means that the catenary and parabola curves are similar.

%\revS{This method yielded precise results, achieving less than $0.6~m$ of the mean vertical distance. Furthermore, all the methods are efficient and can be used on-line.}


% \begin{table}[ht]
% \begin{center}
%   \small
%   \scalebox{0.9}{
% \begin{tabular}{|c|c|c|c|c|} 
%  \hline
%  Method & $\delta_{max}~(m)$ & $\delta_{Total}~(m)$ & $\delta_{mean}~(m)$ & $C.T.~(s)$ \\ [0.5ex]
%  \hline
% % \emph{ByPSeries} & $3.69 \pm 6.41$ & 38.39 & $2.47 \pm 4.59$ & $0.00\pm0.00$ \\ 
% % \hline
%  \emph{ByLength} & $0.69\pm1.08$ & 5.12& $0.29\pm0.42$ & $0.00\pm0.00$ \\
%  \hline
%  \emph{ByFitting} & $0.48 \pm 0.58$ & 2.98 & $0.28 \pm 0.37$ & $0.07 \pm 0.01$ 
%  \\
%  \hline
% \end{tabular}
% }
% \end{center}
% \caption{
% Comparison between the approximation methods.
% %using metrics $\delta_T$, $\delta_{max}$, $\delta_{mean}$, and $C.T$ 
% For each metric, the mean and standard deviation are shown.}
% \label{tab:comparison1}
% \end{table}

\begin{table}[t!]
\caption{
Comparison between the approximation methods: by Length, by Fitting and by Points
%using metrics $\delta_T$, $\delta_{max}$, $\delta_{mean}$, and $C.T$ 
For each metric, the mean and standard deviation are shown.}

\begin{center}
  \small
  \scalebox{0.9}{
\begin{tabular}{|c|c|c|} 
 \hline
 \textbf{Method} & $\epsilon_{mean}~[m]$ & $\epsilon_{mean} / L$  \\ [0.8ex]
 \hline
% \emph{ByPSeries} & $3.69 \pm 6.41$ & 38.39 & $2.47 \pm 4.59$ & $0.00\pm0.00$ \\ 
% \hline
 \emph{ByLength} & $0.3187\pm 0.3299$  & $0.0200\pm 0.0214$ \\
 \hline
 \emph{ByFitting} & $0.3127 \pm 0.3108$ & $0.0194 \pm 0.0200$ 
 \\
 \hline
  \emph{BySampling} & $0.2841 \pm 0.3040$ & $0.0173 \pm 0.0178$ 
 \\
 \hline
\end{tabular}
}
\end{center}
\label{tab:comparison}
\end{table}



%shows great performance, taking less than $0.1$ seconds to compute the approximation curve.

%The advantage of the $A.C$ and the $A.L$ methods over $O.C.A$ is the very low computational time needed to find the approximate catenary (taking just a few milliseconds). The only reason to pick another method other than $O.C.A$ is that the required computational speed is extremely fast, in this case, the right method to select is $A.L$. Method $A.L$ takes almost zero time to compute and shows acceptable results in terms of vertical distance, reaching less than $1.8~m$ of the maximum vertical distance, less than $5.2~m$ of the total vertical distance, and less than $0.8~m$ of the mean vertical distance. Finally, the $A.C$ method shows poor results despite its fast performance. 

\subsection{Ensuring a collision-free catenary}

Although the proposed catenary approximations of the parabola are accurate enough, we need to be sure that the computed catenaries are feasible, that is, collision-free and with admissible length.

Ensuring a feasible tether length after the catenary approximation can be easily implemented by constraining the parabola length during optimization. We can set the minimal tether length a $5\%$ longer and the maximum length a $5\%$ shorter, ensuring that the computed catenary will still hold the length constraint even with the distortions applied by the approximation.

However, ensuring a collision-free catenary is more complex. The most usual approach to solve this problem consists in inflating the obstacles, as in Algorithm \ref{alg:decision_problem}. The computed parabola avoids obstacles and passes through a vertex of an obstacle inside the trapezoidal region $T$. To guarantee that the calculated approximate catenary remains collision-free, the obstacles should be initially expanded as follows. For each polygon, we consider the \emph{expanded} polygon, which is defined by lines parallel to the faces of the polygon and located at a distance of $\tau$ (see Fig. \ref{fig:PDP}). The tangent collision-free parabola is built in the Algorithm \ref{alg:decision_problem} taking as input the expanded polygons. In practice, it is crucial to determine the tolerance $\tau$ for both ground and aerial obstacles. Large values of $\tau$ can give negative responses to the problem PDP, while small values of $\tau$ can lead to non-feasible approximation catenaries.  The election of $\tau$ strongly depends on the approximation method used. The more accurate the method, the better the selection of $\tau$. According to Table \ref{tab:comparison}, the most accurate approximation is \emph{BySampling} with a relative error of $0.0173(\sigma:0.0178)$ per meter. We can use such error to estimate the value of $\tau$, setting it to the average approximation error plus its standard deviation is a good trade-off, particularly $\tau=0.035*L$, where $L$ is the longitude of the tether. Note that the mean vertical distance between the parabola and the catenary never exceeded this value in the experiments. This means that both the ground and the obstacles in $\mathcal{O}_{\tau}$ should differ only in $0.6$ meters in the vertical axis from their original shape. Therefore, using \emph{BySampling} and $\tau=0.6~m$, we can ensure that Algorithm \ref{alg:catenaryaproximation} solves the CDP problem and computes a collision-free catenary for the marsupial robotic system.

While the previous method is safe and effective, obstacle inflation suffers from over-constraining the planning problem, eliminating possible feasible solutions under the assumption of the worst-case scenario. Instead, we propose not using obstacle inflation, but reevaluating the solution (now with catenary) to check if it is still collision-free after fitting the parabola. In case the solution is not feasible, we can slightly adjust the length of the tether so that it is collision-free again.




%----------------------------------------------------------
\section{Efficient Path Planning Approach for a tethered UGV-UAV system}
\label{sec:path_planning}
% %---------------------------------------------------------------------------------------
%\section{Decision problem applying  to path and trajectory planning of thetered UAV-UGV}

In the previous section, we defined a procedure that allows us to efficiently identify the status of a collision-free catenary using a decision problem. In this section, we use this procedure \rev{to enhance the path planning method for a tethered UAV-UGV robotic configuration proposed in \cite{smartinezr2023}. } We summarize the method here for the sake of completeness.

%------------------------------------
%\subsection{RRT* algorithm using decision problems}

The goal of the path planning algorithm is to devise a safe path for the marsupial system that connects a starting position of the whole system to a goal configuration in which the UAV system has a goal position, while the rest of the system has an arbitrary, but feasible and collision-free, configuration. To this end, we use the RRT* algorithm \cite{karaman_rrt_star}.

To reduce the complexity of the approach, we only consider as decision variables the position of both platforms, omitting the variables related to the state of the tether. %, which are specified in Section \ref{sec:overview}. 
These tether parameters are then obtained by solving the corresponding PDP or CDP associated with the positions of both platforms.

The path planner in this Section does not consider any kinematic or dynamic constraints to obtain the path, leaving them for the optimization stage of the algorithm described in Section \ref{sec:optimization_process}.

% \begin{figure}[t!]
% \centering
% \includegraphics[width=0.4\textwidth]{Figures/rrt_diagram_v2.png}
% %  \includegraphics[width=0.48\textwidth]{Figures/results_rrt_2.png}
%   \caption{Basic flow diagram of our implemented RRT* algorithm. }
%   \label{fig:rrt_diagram}
% \end{figure}

The main steps of the RRT* algorithm are as follows. % in Fig. \ref{fig:rrt_diagram}. 
It creates a tree starting from the initial configuration of the marsupial system which will be expanded in a loop for a given number of iterations. In the loop, it generates a new collision-free sample in the \textit{Sampling} step, generating a random collision-free node ($x_{random}$) containing the positions of the UAV and the UGV. Then, RRT* tries to extend the tree towards the random sample, obtaining a new candidate node to be added to the tree ($x_{new}$). The new node should be \textit{validated} by solving the Decision Problem  of section \ref{sec:approach} connecting the new UAV and UGV poses. If extended, RRT* optimizes the graph by searching for the best parent node and rewiring the tree in the neighborhood of $x_{new}$. These steps lead to an asymptotically optimal solution for the path \cite{karaman_rrt_star}. We explain the main particularities of our implementation in each step below.

\subsection{Sampling}

In this step, we search for valid UGV and UAV positions. To this end, the position of the UGV is sampled at the traversable points of the 3D map with a minimum clearance \cite{driving_pc}. Similarly, the position of the UAV is sampled in the obstacle-free space of the environment, which is composed of points having a minimum clearance. We use Euclidean Distance Fields (EDF) to speed up the sampling process, storing it in a preprocessed grid, which contains the distance from each grid point to its closest obstacle in the environment \cite{edf_survey}. In this way, we can check if a point is collision-free just by checking its EDF.

%Obtaining the projection of the 3D point cloud of obstacles in the environment to a plane $\pi$ can be computationally demanding. Hence, in this paper we propose to sample the planes on the environment in a preprocessing step. This is done by slicing the 3D environment in a given number of headings (see Fig. \ref{fig:slices}). Then, whenever a PDP is to be solved, we use the closest pre-processed plane of the required heading.


%\begin{figure}[!t]
%\includegraphics[width=0.48\textwidth]{Figures/slices_1.png}
%  \includegraphics[width=0.48\textwidth]{Figures/results_rrt_2.png}
%  \caption{Slices of a 3D scenario in a given heading. The obstacles of each plane are represented in small spheres with different colors for different planes. }
%  \label{fig:slices}
%\end{figure}

%The performance of both approaches (PDP and CDP) is tested and compared in Section \ref{sec:experiments}. 


\subsection{Steer}

\label{sec:steering}

%We use the \texttt{Steering} procedure used in \cite{smartinezr2023} 
This step tries to generate a new valid node ($x_{new}$) by extending the tree from the nearest node ($x_{near}$) of the previously generated  configuration ($x_{random}$). We follow the steering function proposed in \cite{smartinezr2023}, which sequentially tries three different steering modes, keeping the new node of the first successful mode. Note that in contrast to the Sampling step, the new node should be validated before adding it to the tree. In each mode, the steering method considers movement in a subset of the agents:

\begin{enumerate}
    \item The first mode steers only the UAV position component, and the UGV position is fixed. 
    \item In the event that the first mode does not succeed, the second mode is executed. The second mode steers both the UGV and the UAV
    \item The last mode just steers the UGV and considers the UAV fixed.  
\end{enumerate}

\subsection{Validating the new node}

Whenever a candidate $x_{new}$ configuration is generated with one of the steering alternatives, we make sure that there exists a collision-free tether connecting the positions of each vehicle. To this end, a decision problem, either CDP or PDP, should be solved. 

%\textcolor{red}{The algorithm in \cite{smartinezr2023}, used the catenary model of the shape of the tether. Therefore, it solved a CDP with the numerical approach described in Section \ref{sec:approach}. }

%In this paper, we check for collisions in the catenary by sampling it and ensuring that the EDF values on the samples exceed a minimum clearance. 

In this paper, we propose to use a parabolic model of the tether to speed up the RRT* algorithm. Therefore, we solve a PDP using our iterative algorithm of Section \ref{solvingPDP}. Note that the proposed iterative algorithm uses a 2D description of the environment and obstacle clustering to accelerate the processing, while the RRT* is working in 3D. In the rest of the section, we detail our proposed procedure that conveniently projects and clusters the obstacles in the environment to a 2D plane, making use of the input EDF.

%As in the catenary formulation
%As stated before, we use the EDF to accelerate the 2D projection and clustering process. 
Once the positions of the UGV and UAV have been defined, we compute the 2D vertical plane that passes through both positions. Then, we uniformly sample the 3D EDF in the T region 
(see Fig. \ref{fig:PDP}). If the EDF at a sample point does not meet a minimum clearance, we assume that the point is an obstacle. Otherwise, it is considered as free space. We use the same resolution in this sampling procedure as the resolution of the provided EDF.

While we are generating the grid containing the obstacles in the 2D plane, we try to group them by connecting the  currently generated point with the previous obstacles in its already computed neighborhood, if any. This method is fast, can be performed just by analyzing the direct neighbors of each sample, but might divide concave obstacles into several clusters depending on their shapes. As an example, Figure \ref{fig:clustering} represents a projection that has been properly clustered into three main obstacles (red, green, and blue). However, the clustering algorithm fails to merge the remaining obstacles in pink, purple, and yellow, to name a few. Therefore, the computation of the PDP will increase slightly due to a larger number of clusters. In spite of it, results of Section \ref{sec:experiments} show that this increase is marginal and that PDP remains the fastest method to solve the problem. 

\rev{Once we have obtained the different clusters as point clouds, we get the convex hull of each cluster, resulting in the input polygons of Algorithm \ref{alg:decision_problem}.}




 %Finally, if no $x_{new}$ node is found, we go back to the sampling procedure. In contrast, $x_{new}$ is first connected to the best possible neighbor, and then the tree is rewired to ensure asymptotical optimality.



\begin{figure}[!t]
  \includegraphics[width=0.48\textwidth]{Figures/2d_projection.png}
  \caption{Example of a 2D projection, in which each cluster is represented with a set of big dots in a different color. The projection is obtained from a 3D Point Cloud, represented in fine-grained dots with colors from red to green depending on its $z$ coordinate. Figure obtained with RViz \cite{rviz}. }
  \label{fig:clustering}
\end{figure}


%----------------------------------------------------------
\section{Trajectory optimization}
\label{sec:optimization_process}
Once an initial feasible path is computed, we proceed to optimize the full robot system trajectory as a whole, considering dynamic constraints such as velocities or accelerations, and distance to obstacles. 

In this regard, this paper follows the methodology presented in \cite{smartinezr2023}. Unlike \cite{smartinezr2023}, instead of considering the length of the tether in the state vector, here we consider the set of parameters that actually define the tether curve. For both the catenary and parabola, the number of required parameters is three, so we increase the dimension of the state vector in two.  % with respect to \cite{smartinezr2023}. 
However, this new state vector improves the convergence and decreases the computation to obtain a feasible solution. Including the curve parameters in the state vector (instead of just the length) enables the optimizer to directly sample the tether and properly compute the gradients with respect to the curve itself. Section \ref{sec:experiments} will show that this new state vector reduces the optimization time by at least one order of magnitude.

%Thus, to define the state of the trajectory, we must incorporate the time variable $\Delta t^{i}$ into the path obtained from the path planner. For consistency, the UGV, UAV, and tether trajectories must use the same $\Delta t^{i}$ to coordinate the arrival of both platforms at the planned waypoint. 

Thus, in this paper, the state of the system in each time step includes the position of the UGV $\mathbf{p}_g=(x_g,y_g,z_g)^{T}$, the position of the UAV $\mathbf{p}_a=(x_a,y_a,z_a)^{T}$, the parameters that define the curve of the tether $\mathbf{T}$, and the time variable $t$. For consistency, the UGV, UAV, and tether trajectories must use the same $t$ to coordinate the arrival of both platforms at the planned waypoint. 

We discretize the trajectories so that the states of our problem become the set:
\begin{equation}
\label{eq:traj_params}
%    \Omega = \{\mathbf{p}^i_g,\mathbf{p}^i_a,p^{i}, q^{i}, r^{i},\Delta t^i\}_{i=1,...,n}
\Omega = \{\mathbf{p}^i_g,\mathbf{p}^i_a,\mathbf{T}^i,\Delta t^i\}_{i=1,...,n}
\end{equation}

\noindent where $\Delta t^{i} = t^{i} - t^{i-1}$ is the time increase between states $i$ and $i-1$, allowing temporal aspects. When needed, we also discretize the resultant tether model into a set of $m$ positions $\mathbf{p}_{t}=(x_{t}, y_{t}, z_{t})$:

\begin{equation}
\label{eq:tether}
    P^i = \{\mathbf{p}^j_t\}_{j=0,...,m-1}
\end{equation}

Our problem consists of determining the values of the variables in $\Omega$ (\ref{eq:traj_params}) that optimize a weighted multi-objective function $f(\Omega)$:  

\begin{equation}
\label{eq:cost_function}
     \Omega^* = \arg \min_\Omega f(\Omega) = \arg \min_\Omega  \sum_{i,k} \gamma_k * || \delta_k^i(\Omega) ||^2
\end{equation}

\noindent where $\Omega^*$ denotes the optimized collision-free trajectory for the UAV, the UGV, and the tether from the starting to the goal configurations. $\gamma_k$ is the weight for each component $\delta_k(\Omega)$ (also known as residual) of the objective function.  
%As in \cite{smartinezr2023},
 Each component encodes a different constraint or optimization objective for our problem and should be evaluated in all the time steps $i$.
In addition, the optimization problem is addressed using nonlinear sparse optimization algorithms, with \emph{Ceres-Solver} \cite{ceres-solver} serving as the optimization back-end.


%In this regard, this paper follows the same In this section, we detail the enhancements made to the optimization process of the method for finding feasible trajectories as described in \cite{smartinezr2023}. For the whole process, we consider a solution feasible when the trajectory is collision free for every agent of the system, including the tether. 

%As previously mentioned, the core of this improvement lies in the utilization of parabola parameterization to represent the tether. Unlike the original method, which focused on optimizing the length of the catenary curve, our approach leverages the curve parameters for optimization. 


% %---------------------------------------------------------------------------------------
\subsection{Tether parameterization}
\label{sec:tether_params}

We propose two different curves to model the tether: the catenary and the parabola. Both are parameterized by three variables:
\begin{eqnarray}
    parabola(x) &=& px^2+qx+r \\
    catenary(x) &=& a\ cosh( (x-x_0)/a) + y_0 
\end{eqnarray}

\noindent Thus, the parameters of $\mathbf{T}$ in (\ref{eq:traj_params}) will be defined as \mbox{$\mathbf{T}=(a, x_0, y_0)^T$} for the catenary curve, and $\mathbf{T}=(p,q,r)^T$ for the parabola.

%We will see in Section \ref{sec:experiments} that, depending on the type of curve selected for tether parameterization, both the feasibility and the required computation are improved. 
%Although using the catenary as the dentition of the curve for the tether can be easily derived from the optimization process presented in \cite{smartinezr2023}, 
%Note that the path planner gives us a path consisting of a sequence of UAV and UGV poses with catenary lengths. Therefore, we would need to calculate initial parameters for
%the parabola curves so that we can use the parabola approximation. 
%The next section describes the process of initializing the tether for the parabola.

% %---------------------------------------------------------------------------------------
\subsection{Initialization}
\label{sec:mathappoach}

% described in the previous section is a free collision path for the whole agents. This path presents a 

The variables in the optimization (positions and tether states) are initialized from the feasible path provided by the RRT* planner of Section \ref{sec:path_planning}. This initial path consists of a sequence of UAV and UGV positions with catenary lengths at each position. To initialize each $\mathbf{T}_i$ we need the parabola that best fits such a catenary.

%The output of the decision problem between two suspension points \emph{A} and \emph{B} is a free collision catenary of length \emph{L}. The length is part of the feasible solution path computed by the RRT* planner. We need the parabola that best fits such a catenary.

The traditional approach \cite{9337759} to find the parabola parameters that best fit a catenary involves assigning the length of the catenary $L_C$ to the length of the parabola. Thus, the problem of computing the parameters of the parabola is reduced to computing the parameters of the curve passing through two suspension points, \emph{A} and \emph{B}, with a given length $L_C$. This can be obtained by solving the following nonlinear system of equations:

% \begin{eqnarray}
%   \label{eq:parabola_2}
%   \begin{array}{ccl}
%   y_A = p{x_A^2} + qx_A + r & \\
%   y_B = p{x_B^2} + qx_B + r & \\
%   L =  \int_{A}^{B} \sqrt{1+(\mathbf{P'(x)})^2}  \,dx =  \int_{A}^{B} \sqrt{1+(2px + q)^2}  \,dx  \\
%   \end{array}
% \end{eqnarray}

\begin{equation}
    \label{eq:parabola_2}
    \left.
        \begin{array}{cc}
            y_A = p{x_A^2} + qx_A + r & \\
            y_B = p{x_B^2} + qx_B + r & \\
            %\vphantom{\int_{A}^{B}} L=\int_{A}^{B}\sqrt{1+(\mathbf{P'(x)})^2}  \,dx=\int_{A}^{B}\sqrt{1+(2px+q)^2}\,dx
            \vphantom{\int_{A}^{B}} L_C=\int_{A}^{B}\sqrt{1+(2px+q)^2}\,dx
        \end{array}
        \right\}
\end{equation}

% We use this solution as input for our optimizer, so, we consider the catenary length to calculate the parabola length and optimize its parameters. Thus, the result of the optimization process is a parabola with length \emph{L}*, and the optimal catenary is obtained between \emph{A} and \emph{B} using \emph{L}*. The flow of the algorithm in the optimizer can be illustrated as follows:

% \begin{equation}
% \label{eq:parabola_1}
%     A, B , \mathbf{C}(L) 	\rightarrow  \mathbf{P}(L)  \rightarrow Optimization \rightarrow \mathbf{P}(L^{*}) \rightarrow \mathbf{C}(L^{*}),
% \end{equation}

% where $\mathbf{P}$(\emph{L}) is the parabola of length \emph{L} and $\mathbf{C}$(\emph{L}*) is the catenary of optimal length. 
% The problem of computing the parabola $\mathbf{P(x)}=px^2+qx + r$ passing through two suspension points $A$ and $B$ and length $L$ can be numerically solved using the system:

\noindent Notice that this integral can be solved analytically, but its solution is a complex equation establishing highly nonlinear relations among the parabola parameters. This makes the solution sensitive to parameter initialization.

To avoid nonlinear calculation, we propose an approach consisting of equalizing the area under the parabola between the segment connecting \emph{A} and \emph{B} with the area under the catenary curve, $a_C$. Using the parabola equation $px^2+qx+r$, we define a system of equations to obtain the parabola parameters \emph{p}, \emph{q} and \emph{r} in 2D (as a plane) as follows:

%Now, we assume that the output of the planning are \emph{A}, \emph{B} and the area \emph{a$\mathbf{_C}$} under the catenary (which is easy to calculate and approximate). 

% From \emph{A} and \emph{B} using \emph{L} is easy to calculate the area \emph{a$\mathbf{_C}$} under the catenary. From this, we compute parabola area \emph{a$\mathbf{_P}$}. Considering the parabola hanging from \emph{A} and \emph{B} and area \emph{a}, and using the parabola equation ($px^2+qx+r$) we define a equation system to obtain the parabola parameters \emph{p}, \emph{q} and \emph{r} in 2D as follows:

% \begin{eqnarray}
%   \label{eq:parabola_4}
%   \begin{array}{ll}
%   y_A = p{x^2}_A + qx_A + r & \\
%   y_B = p{x^2}_B + qx_B + r & \\
%   a =  \int_{A}^{B} (px^2 + qx + r)  \,dx  &
%   \end{array}
% \end{eqnarray}

\begin{equation}
    \label{eq:parabola_4}
    \left.
        \begin{array}{lll}
            y_A = p{x^2}_A + qx_A + r & \\
            y_B = p{x^2}_B + qx_B + r & \\
             a_C =  \int_{A}^{B} (px^2 + qx + r)  \,dx       
        \end{array}
        \right\}
\end{equation}

\noindent This is a relatively simple linear system of equations for $p,\ q,\ r$ and $y$, that can be solved efficiently. We will use this method to compute the best-fitting parabola.

The computed parabola parameters are used as part of the initial solution for the optimization process. Through this optimization, we obtain the optimized parabola parameters and then compute the corresponding catenary by means of the Algorithm \ref{alg:catenaryaproximation}. 

%The description of the general procedure to get the \emph{L$^{*}$} free of collision is presented in Algorithm \ref{alg:fromParToCat}.

%\begin{algorithm}[t!]
%\caption{Algorithm to compute optimized Catenary Length \emph{L$^{*}_{\mathbf{C}}$}.}
%\label{alg:fromParToCat}
%\begin{algorithmic}[1]
%\Require \emph{point} $A$, \emph{point} $B$, \emph{catenary area}  \emph{a$\mathbf{_C}$}
%\Ensure \emph{L$^{*}_{\mathbf{C}}$}: a catenary length free of collision hanging from $A$ and $B$ .
%\State $a\mathbf{_P} \gets a\mathbf{_C}$
%\State $p, q, r \gets solve\_equation\_system(A,B, a\mathbf{_P})$
%\State $p^{*}, q^{*}, r^{*} \gets \mathbf{optimization\_process}(A,B, p, q, r)$
%\State ${L^{*}}_{\mathbf{P}} \gets get\_parabola\_length(p^{*}, q^{*}, r^{*})$
%\State ${L^{*}}_{\mathbf{C}} \gets Algorithm\ 2(A,B, {L^{*}}_{\mathbf{P}})$
%\end{algorithmic}
%\end{algorithm}

% The parabola parameters pass to the optimization process, thus, we obtain the optimized parameter, and subsequently the area \emph{a$^{*}\mathbf{_P}$} of the best parabola and the corresponding catenary can be obtained from that. The flow of the new algorithm can be illustrated as follows:

% \begin{equation}
% \label{eq:parabola_3}
%     A, B , a\mathbf{_C}  	\rightarrow  \mathbf{P}(a\mathbf{_C})  \rightarrow Optimization \rightarrow a^{*}\mathbf{_P} \rightarrow \mathbf{P}(a^{*}{\mathbf{_P}})
% \end{equation}


%----------------------------------------------------------
% \subsection{States definition}

% As in our previous work \cite{smartinezr2023}, we define a state in the state space as the combination of the position of the UGV $\mathbf{p}_g=(x_g,y_g,z_g)^{T}$, the UAV $\mathbf{p}_a=(x_a,y_a,z_a)^{T}$ and the tether length $l$. At any instant, the tether length should be longer than the distance from the UGV to the UAV, i. e. $l \geq \|\mathbf{p}_a - \mathbf{p}_g\|$.

% The motion planning problem consists of determining the trajectory for the UGV $\mathbf{p}_g(t)$, the UAV $\mathbf{p}_a(t)$ and the tether length $l(t)$ so that the UAV reaches a given goal position while avoiding obstacles and respecting the constraints of the system. In this new approach, as was explained in \ref{sec:mathappoach}, instead of using the tether length $l(t)$ as a state to optimize, we use the parameters of the parabola $(p^{i}$, $q^{i}$, $r^{i})$.

% We discretize the trajectories so that the states of our problem become the set:
% \begin{equation}
% \label{eq:traj_params}
%     \Omega = \{\mathbf{p}^i_g,\mathbf{p}^i_a,p^{i}, q^{i}, r^{i},\Delta t^i\}_{i=1,...,n}
% \end{equation}

% \noindent where $\Delta t^{i} = t^{i} - t^{i-1}$ is the time increment between states $i$ and $i-1$, allowing us to consider the temporal aspects. This value is the same for UGV and UAV trajectories. For each $\mathbf{p}^i_a$ , $\mathbf{p}^i_{g}$ and $p^{i}$, $q^{i}$, $r^{i}$, there is a tether configuration $T^i$, given by the parable model mentioned above. When needed, we also discretize the resultant tether model into a set of $m$ positions $\mathbf{p}_{t}=(x_{t}, y_{t}, z_{t})$:

% \begin{equation}
% \label{eq:tether}
%     T^i = \{\mathbf{p}^j_t\}_{j=0,...,m-1}
% \end{equation}

% Our problem consists of determining the values of the variables in $\Omega$ \ref{eq:traj_params} that optimize a weighted multi-objective function $f(\Omega)$:  

% \begin{equation}
%      \Omega^* = \arg \min_\Omega f(\Omega) = \arg \min_\Omega  \sum_{i,k} \gamma_k * || \delta_k^i(\Omega) ||^2
% \end{equation}

% \noindent where $\Omega^*$ denotes the optimized collision-free trajectory for UAV, UGV and tether from the starting and goal configurations. $\gamma_k$ is the weight for each component $\delta_k(\Omega)$ (also known as residual) of the objective function. 
% As in \cite{smartinezr2023} each component encodes a different constraint or optimization objective for our problem, and should be evaluated in all the timesteps $i$.
% Also, the optimization problem is solved with non-linear sparse optimization algorithms ( \emph{Ceres-Solver} \cite{ceres-solver} as our optimization back-end). 

%----------------------------------------------------------
\subsection{Optimization process}
\label{sec:implementation}
% The optimizer considers geometric constraints such as obstacle distance, trajectory smoothness, and equi-distance between states. Moreover, it also considers temporal constraints including time, velocity, and acceleration. Temporal considerations apply to both platforms, UGV and UAV. These constraints are described and used in our previous work \cite{smartinezr2023}, where we optimized the mentioned constrained related to UGV, UAV, and tether. We represent the constraints in the problem as penalty costs in the objective function.

% About the tether constraints, we present a new approach compared to our previous work in \cite{smartinezr2023} and \cite{smartinezr2021}, where were consider the constraints related to obstacle distance and length for a catenary curve. Furthermore, these constraints were addressed by the optimizer as an AutoDiff Functor which solves the numerical method of Bisection to compute the transcendental equation for mechanical catenary. The new approach incorporates three constraints centered around a parabola curve. These constraints involve obstacle distances, parabola length, and parabola parameters, which force the parabola to pass through the desired endpoints. Next, we define the constraints included in the objective function to be minimized, which also are addressed as AutoDiff Functor but solving this time an analytic equation.

The optimizer takes into account geometric constraints such as obstacle avoidance, trajectory smoothness, and equi-distance between states. In addition, it considers temporal constraints including time, velocity, and acceleration, applicable to both the UGV and the UAV platforms. %These constraints, which we previously described and utilized in \cite{smartinezr2023}
The main set of constraints is inherited from the optimization method detailed in \cite{smartinezr2023}. In particular, we use the penalty functions related to the UAV, the UGV and the tether feasibility. However, we redefine the maximum tether length constraint and we include a new residual that ensures that the tether passes through the positions of both the UAV and UGV due to the new parametric formulation.

%They involve optimization related to UGVs, UAVs, and tether. We represent the constraints in the problem as penalty costs in the objective function (\ref{eq:tether}).

%Regarding the tether constraints, we introduce a new approach compared to our previous works \cite{smartinezr2023} and \cite{smartinezr2021}, where we considered constraints related to obstacle distance and length for a catenary curve. In these works, the optimizer addressed these constraints as an AutoDiff Functor that solved the numerical method of Bisection to compute the transcendental equation for the mechanical catenary.

%The new approach incorporates three constraints centered around a parabola curve. These constraints involve obstacle distances, parabola length, and parabola parameters, ensuring the parabola passes through the desired endpoints. We define these constraints in the objective function to be minimized, which are also addressed as AutoDiff Functors, but this time solving an analytic equation. 

Next, we present the new constraints required to consider the direct tether parameterization presented in (\ref{eq:traj_params}) for parabola and catenary:

%\subsubsection{Obstacle avoidance constraint}

%The tether state is computed by solving the parabola equation for parameters $p^{i}$, $q^{i}$ and $r^{i}$ between \emph{A} = $\mathbf{p}^i_g$  and \textbf{B} = $\mathbf{p}^i_a$. Using these values a parable is computing and sampling according to (\ref{eq:tether}) into $m$ points.

%This constraint penalizes the proximity of the sampled tether to obstacles. We compute the distance $d^i_{ot,j}$ to the nearest obstacles of every sample of tether $p^{i}$, $q^{i}$, $r^{i}$, and estimate the residual $\delta^i_{op}$ as the sum of the inverse nearest distances. We increase the weight of those samples closer than a safety distance $\rho_{ot}$ to guarantee higher cost in these cases using $\rho_{j}= \beta$, with $\beta >> 1$.

%\begin{eqnarray}
%  \label{eq:eq_teher_obst}
%  \delta^i_{op} &=& \sum_{j=0}^{m-1} \frac{\rho_{j}}{d^i_{op,j}}  , \  \rho_{j} = \left \{
%      \begin{array}{cc}
%      1 & ,if\  \  d^i_{op,j}  >  \rho_{op}\ \\
%      \beta & ,otherwise
%  \end{array}
%  \right .
%\end{eqnarray}

\subsubsection{Length constraint}
This constraint penalizes unfeasible tether lengths for the catenary and parabola. The tether cannot be shorter than the Euclidean distance between the UGV and the UAV $d^i_u=||\mathbf{p}^i_g-\mathbf{p}^i_a||$ and cannot exceed its maximum length $L_{max}$. Given the tether parameters $\mathbf{T}^i$ (parabola or catenary), we can analytically compute the length of the curve between the suspension points $l^i$. With this information, we define the following residual:
\begin{equation}
    \label{eq:eq_length}
    \delta^i_{up} =   e^{d^i_{u} - l^{i}} + e^{l^{i} - L_{max}} 
\end{equation}
%\begin{eqnarray}
%  \label{eq:eq_length}
%  \delta^i_{up} =& \left \{
%  \begin{array}{cc}
%  e^{(d^i_{up} - l^{i})} -1 & ,\textrm{if}\  \  d^i_{up} \ > \ l^{i} \ \\
%  e^{(l^{i} - L_{max})} -1 & ,\textrm{if}\  \  L_{max} \ < \ l^{i} \ 
%  \end{array}
%  \right. 
%\end{eqnarray}
\noindent Notice how the residual starts rising over zero when approaching the minimum and maximum thresholds, growing exponentially as we pass the thresholds.

\subsubsection{Parabola parameter constraint}
This is only applied when $\mathbf{T}^i$ is a parabola; it penalizes unfeasible solutions for $p^{i}$, $q^{i}$, and $r^{i}$. This constraint ensures that the parabola passes through the positions of the UGV and UAV, $\mathbf{p}^i_g$ and $\mathbf{p}^i_a$. The constraint projects such 3D positions to the 2D plane that contains the points and is perpendicular to the floor, obtaining the 2D suspension points A and B. The suspension points must comply with the following equations: 

%\begin{eqnarray}
%  \label{eq:eq_parameters}
%  \delta^i_{pA} = p{x^2_A} + qx_A + r - y_A \\
%  \delta^i_{pB} = p{x^2_B} + qx_B + r - y_B 
%\end{eqnarray}

\begin{equation}
  \label{eq:eq_parameters_parabola}
  \left.
    \begin{array}{lll}
        \delta^i_{pA} = p{x^2_A} + qx_A + r - y_A \\
        \delta^i_{pB} = p{x^2_B} + qx_B + r - y_B 
    \end{array}
    \right\}
\end{equation}


\subsubsection{Catenary parameter constraint}
This is only applied when $\mathbf{T}^i$ is a catenary; it penalizes unfeasible solutions for $a^i$, $x^i_0$, and $y^i_0$. This constraint ensures that the catenary passes through the positions of the UGV and UAV, $\mathbf{p}^i_g$ and $\mathbf{p}^i_a$. The constraint projects such 3D positions to the 2D plane that contains the points and is perpendicular to the floor, obtaining the 2D suspension points A and B. The suspension points must comply with the following equations: 

%\begin{eqnarray}
%  \label{eq:eq_parameters}
%  \delta^i_{cA} = a*cosh(\frac{x_A- x_0}{a}) + y_0 - y_A \\
%  \delta^i_{cB} = a*cosh(\frac{x_B- x_0}{a}) + y_0 - y_B
%\end{eqnarray}

\begin{equation}
  \label{eq:eq_parameters_catenary}
  \left.
  \begin{array}{lll}
  \delta^i_{cA} &=& a\ cosh(\frac{x_A- x_0}{a}) + y_0 - y_A \\
  \delta^i_{cB} &=& a\ %.000000000000000
  cosh(\frac{x_B- x_0}{a}) + y_0 - y_B
  \end{array}
\right\}
\end{equation}


% ME FALTA ACTUALZAR ESTO
%\delta^i_{op},  d^i_{op,j} ,  \rho_{op} , \delta^i_up , L_{max} , d^i_{up} , \delta^i_{pA} , \delta^i_{pB}, \delta^i_{pA}, \delta^i_{pB}

%Debo agregar en algun lado que el tratamamiento de las ecuaciones son en 2D por lo que siempre se hace una conversión de 3D a  2D



%----------------------------------------------------------
\section{Experimental Results}
\label{sec:experiments}
\section{Exploratory User Study Results}
Building on the formative study and resulting requirements, we present the results of our evaluation of \SystemName following the user study design detailed in \cref{sec:user-study}. We first report the self-reported questionnaire results, followed by four themes derived from a thematic analysis of interviews with \exsitu and \insitu participants. Plots that show an overview of the usage of specific \SystemName features are provided in the Supplementary Materials.

\subsection{Questionnaire Results}
We conducted statistical analyses using linear mixed-effects models, accounting for participant variability as a random effect and controlling for confounding factors such as task location and prior experience. All models converged, and model assumptions were verified (linearity, normality, and homoscedasticity). Where applicable, we report the estimated coefficients ($\beta$), standard errors (SE), p-values ($p$), and 95\% confidence intervals (CI). Unless stated otherwise, each analysis includes 32 observations ($n$=32). In this section, we provide full details for (marginally) statistically significant results, whereas an overview of all statistical results is available in the Supplementary Materials.

Cronbach's alpha values indicated acceptable internal consistency for both engagement and task load measures. Engagement items yielded alpha values of $0.740$ for \insitu users and $0.764$ for \exsitu users. Task load items had alpha values of $0.812$ and $0.772$ for \insitu and \exsitu users, respectively.

\subsubsection{Engagement}
The analysis revealed a significant effect of \textit{collaboration mode} on self-reported engagement for both \exsitu and \insitu participants. \Exsitu participants reported significantly higher engagement in the \sync condition ($\beta = 0.500$, SE = $0.140$, $p < 0.001$, 95\% CI [$0.226$, $0.774$]). A similar effect was observed for \insitu participants, who reported significantly increased engagement in the \sync condition ($\beta = 0.340$, SE = $0.130$, $p = 0.009$, 95\% CI [$0.085$, $0.595$]). A plot of engagement scores is shown in \cref{fig:engagement-task_load-plot}A.

While no significant effect of \textit{location} was found for \exsitu participants ($p = 0.196$), \insitu participants reported significantly higher engagement at \locB ($\beta = 0.299$, SE = $0.130$, $p = 0.022$, 95\% CI [$0.044$, $0.554$]). \textit{Prior experience} with AR or 3D editing tools did not significantly influence self-reported engagement for either group (\exsitu: $p = 0.093$; \insitu: $p = 0.231$).

\subsubsection{Task load}
\begin{figure}
    \centering
    \includegraphics[width=\linewidth]{Figures/Plots/EngagementScoresAndTaskLoadScores_CameraReady.pdf}
    \caption{\textsf{(A)} Engagement scores for each role per condition, with significantly higher engagement in the \sync condition for both \insitu and \exsitu participants. \textsf{(B)} Task load scores for each role (\exsitu and \insitu) per condition. No significant differences were found between conditions.}
    \label{fig:engagement-task_load-plot}
    %
\end{figure}

Self-reported task load showed no significant difference between \sync and \async \textit{collaboration modes} for either \exsitu ($\beta = 0.313$, SE = $0.309$, $p = 0.312$) or \insitu participants ($\beta = 0.213$, SE = $0.345$, $p = 0.538$). A plot of task load scores is shown in \cref{fig:engagement-task_load-plot}B.

\textit{Location} did not significantly impact self-reported task load for \exsitu participants ($p = 0.419$). However, \insitu participants reported a marginally significant decrease in task load at \locB compared to \locA ($\beta = -0.637$, SE = $0.345$, $p = 0.065$, 95\% CI [$-1.314$, $0.039$]). \textit{Prior experience} did not significantly affect self-reported task load for \exsitu participants ($p = 0.162$). However, \insitu participants showed a marginally significant association between lower task load and more AR experience ($\beta = -0.320$, SE = $0.176$, $p = 0.070$, 95\% CI [$-0.666$, $0.026$]).


\begin{figure}
    \centering
    \includegraphics[width=\linewidth]{Figures/Plots/TaskFixing_CameraReady.pdf}
    \caption{Stacked bar plot illustrating response proportions where participants indicated perceived performance and confidence for each task component across \textit{collaboration mode} and \textit{location} combinations.}
    \label{fig:confidence-task-plot}
    %
\end{figure}

\subsubsection{Confidence in authored results}
Confidence in the number of issues fixed and overall self-reported confidence were both significantly influenced by \textit{collaboration mode}. Participants in the \sync condition reported fixing more issues with confidence than those in the \async condition ($\beta = 3.000$, SE = $0.373$, $p < 0.001$, 95\% CI [$2.270$, $3.730$], $n=24$). Additionally, overall confidence scores were significantly higher in the \sync condition ($\beta = 1.125$, SE = $0.387$, $p = 0.004$, 95\% CI [$0.367$, $1.883$]). An overview of the proportion of participant responses regarding perceived performance and confidence for each task component is shown in \cref{fig:confidence-task-plot}.

A marginal effect of \textit{location} was observed for \exsitu participants, with fewer issues confidently fixed at \locB ($\beta = -0.667$, SE = $0.373$, $p = 0.074$, 95\% CI [$-1.397$, $0.064$], $n=24$). However, \textit{location} did not significantly affect overall confidence scores ($p = 0.747$). \textit{Prior experience} did not significantly impact the number of confidently fixed issues ($p = 0.096$) or overall confidence scores ($p = 0.466$).

\subsection{Interview Results}
At the start of the interview, each participant was asked which system they would prefer to use if they were to perform a similar task again. Among \insitu participants, the majority (12 out of 16) expressed a preference for the \sync condition, citing advantages such as real-time feedback, improved communication, and enhanced engagement. Two participants indicated that their preference would depend on the scenario, while two favored the \async condition, highlighting the ability to focus better without the distractions of real-time interaction.

Similarly, \exsitu participants also largely favored the \sync condition (12 out of 16), with many emphasizing the value of real-time collaboration and immediate feedback. Three participants indicated that their preference depended on the context, and one favored the \async condition due to the slower pace, which allowed more time to refine their contributions.

Our thematic analysis of interview transcripts resulted in four main themes comparing the synchronous \SystemName and asynchronous baseline experiences. We denote \insitu participant quotes with an `I' (\eg, I2) and \exsitu participants with an `E' (\eg, E8).

\subsubsection{\textbf{Collaboration with \insitu users facilitated the integration of real-world context into \exsitu users' design process.}}\label{sec:results:interviews:real-world_context}
In the \async condition, all \exsitu participants struggled to integrate real-world context into their design decisions due to missing information.
\Exsitu users were uncertain about object placement and its interaction with the environment, as they could not verify details without live input. One participant noted, ``I'm not 100\% sure about the real way, if it's low enough or is this blocking other ways because I need to see it from the real end'' (E7). Lacking live visual and spatial information, \exsitu participants relied on limited artifacts provided by \insitu participants, restricting their ability to account for dynamic environmental elements. One participant explained, ``I tried to put the map onto that light post but I'm not sure whether that's on the post or not'' (E7).

In contrast, the \sync condition enabled participants to gain a clearer understanding of real-world context, leading to more informed decision-making in the design process (E12, E10, E9, I5, E17). One \exsitu participant emphasized the importance of real-time observation, particularly for user flow, explaining that although an object appeared well-positioned, real-world feedback revealed that it interfered with pedestrian flow: ``We had a situation where people were walking through one of the objects. It looked like an ideal location for the object on the map [\textit{location mesh}], but in real time and in real life, it wasn't going to work because it was in a walkway'' (E10).

The integration of real-world context through synchronous communication also extended to addressing safety and navigation concerns. One \exsitu participant described uncertainty about whether a virtual cart placed in the scene would interfere with foot and cycling traffic, as there were two real obstacles on either side of it. They noted that this uncertainty was mitigated by being able to check the scene in real time with the \insitu user (E12). Another participant highlighted the value of hearing environmental noise in real time, which influenced decisions about setting audio levels in the AR experience: ``I need to know the population density there, like what's the environment like, should I put the speaker very loud or not at all'' (E9).

Participants also reported that changes in the environment were easier to account for in the \sync condition. One \exsitu user noted that objects had shifted between the time the location mesh was captured and their session, something they would not have noticed without real-time feedback. As they explained, ``The barrel, [...] had actually moved... I wouldn't have been able to tell that from the picture or the scan [location mesh]'' (E17). Following this realization, they asked the \insitu participant to place cursor markers indicating the correct position of the objects on the barrel.

\Exsitu participants also emphasized that synchronous collaboration allowed them to indirectly experience contextual elements of the real-world environment. One participant underscored the importance of experiencing real-world audio and visual elements: ``You can actually hear what they're hearing as well, which is quite important'' (E7). This sense of presence contributed to an increased understanding of spatial relationships within the environment. One participant mentioned, ``I got a much better idea of the space when the other person was there,'' as the \insitu user provided real-time feedback and visual information that could not have been fully captured asynchronously (E16). The combination of real-time video, real-time 3D captures of the environment, and verbal feedback gave \exsitu users a more comprehensive view of the environment, enhancing their understanding of how the design fit within real-world context (E7, E9, E10, E12, E16, E17).

\subsubsection{\textbf{Immediate feedback and changes supported confidence in design decisions, mutual understanding, and perceived accuracy of the outcome.}}\label{sec:results:interviews:immedate_feedback}

All \exsitu participants noted that the \sync condition facilitated more informative feedback and iterative adjustments, with many additionally highlighting that this strengthened their confidence in the authored outcome (I2, E3, E5, E7, I10, E12-17).
The ability to communicate in real time facilitated quicker, informed decisions, with several \exsitu participants noting that the immediate feedback loop significantly reduced the guesswork involved in refining the AR prototype. As E16 stated, ``I think that, very strongly, I'd be more confident that the positions that we put those objects, they actually line up with the real world a lot better.'' This confidence was echoed by other \exsitu participants who mentioned that making the changes in real-time felt ``much, much quicker'' (E3), ``was a lot more dynamic'' (E12), and that ``it felt like, once we were done, it was very complete'' (E10).

The ability to communicate synchronously and see changes immediately not only built confidence but also improved mutual understanding among pairs. Participants emphasized that the visual nature of the real-time interaction removed ambiguity, leading to more accurate placement of virtual objects. Participant I14 noted that discrepancies were minimized: ``I don't think we had a lot of discrepancies in what we discussed about [in the \sync condition], which means probably we had a good sync about the scene.''

\begin{figure*}
    \centering
    \includegraphics[width=\linewidth]{Figures/ExampleOverview_Quotes.pdf}
    \caption{Overview of \SystemName feature usage during Phase 1, shown as \textit{ex-situ} perspective screenshots. Participant conversations are shown in color-coded speech bubbles: \insitu (\textcolor[HTML]{6DD268}{green}) and \exsitu (\textcolor[HTML]{67B3E6}{blue}). Speech bubbles with a \textit{glow} indicate utterances made at the moment of the screenshot. \textsf{(A)} Alignment of a boombox based on spatial context captured using the \textit{3D Snapshot} feature; \textsf{(B)} The \textit{ex-situ} participant moving the map to a position on the wall as specified by the \textit{in-situ} participant through \textit{Surface Drawing}; \textsf{(C)} Alignment of a garland to a previously unmapped region of the street using the \textit{Coarse 3D Mesh} feature; \textsf{(D)} Alignment of misplaced food items based on \textit{in-situ} input using the \textit{3D Cursor}.}
    \label{fig:example-overview}
    %
\end{figure*}

Beyond the real-time feed, \textsc{CoCreatAR}'s additional features --- such as the \textit{3D Cursor}, annotation tools, and the ability to capture \textit{Coarse Meshes} and \textit{3D Snapshots} --- enhanced mutual understanding between \exsitu and \insitu users. \Cref{fig:example-overview} presents examples of feature usage throughout the user study. Participant I16 noted that using the annotation tool to mark exact locations allowed them to ``draw exactly where'' changes were needed, acting as ``the bridge between being there in reality and the scene that [\exsitu user] is seeing.'' Similarly, I5 emphasized that ``drawing in real time is much easier than just explaining it,'' particularly when identifying hazardous areas. The 3D cursor helped reduce ambiguity by allowing participants to visually reference key elements and create placeholders for alignment (I2, I5, I6, I8–10, I12, I14, I16–17, E6, E8, E10–12, E14, E16–17). Additionally, E3 highlighted how the ability to capture \textit{Coarse Meshes} and \textit{3D Snapshots} enabled the \insitu user ``to add more detail to the scene,'' helping \exsitu users make more informed design decisions.

In contrast, the \async condition often resulted in lower confidence due to the lack of immediate feedback and reliance on static information (E9, E12). As E12 described, ``With the asynchronous one, I don't have any information in terms of how much should I move... there's no feedback whatsoever.'' The absence of real-time interaction forced participants to work with limited information, leading to hesitant decisions. E5 emphasized this challenge: ``There was some ineffective information for me in the notes,'' indicating that static instructions without live verification did not provide the necessary context for accurate design decisions. This lack of real-time verification made it difficult to gauge whether changes were correctly applied. E4 highlighted this issue, stating, ``You can't see the change you made, like, in the real world.'' 

\Insitu participants also expressed uncertainty about whether they had captured the right content or communicated their observations clearly (I6, I12, I14, I17). One \insitu participant stated, ``I wasn't sure it was clear enough'' (I14), while another noted that the lack of immediate dialogue required them to provide excessive detail, which still might not ensure accurate interpretation (I17).

\subsubsection{\textbf{Synchronous authoring increased engagement and encouraged creative exploration through collaborative interaction.}}\label{sec:results:interviews:engagement}

In the \sync condition, participants frequently reported higher levels of engagement, often attributing this to the sense of real-time collaboration and mutual decision-making (I3, I8, I12, I15, E3, E5, E12, E14, E17). 
For instance, I10 described the synchronous condition as ``a lot more fun and engaging'' due to the opportunity to work together with another person. Similarly, I14 noted that the synchronous session was ``more enjoyable'' and led to ``higher engagement'' because of the collaborative nature of the task. This sentiment was echoed by E17, highlighting the satisfaction of ``working together'' and described the experience as akin to ``live game testing,'' suggesting that seeing immediate reactions and feedback from the \insitu user created a more interactive and stimulating process. The ability to see their ideas come to life in real time enriched the creative aspect of the experience, as noted by multiple participants (E9, I17).

Synchronous collaboration also drove a sense of teamwork and shared ownership of the final result, contributing to a more engaging authoring process (E7, I10, I12, E16). 
One participant explained that ``it was fun to chat to another person while I was doing it,'' which made the task feel more like a collaborative endeavor rather than an individual effort (I6). This increased engagement was reflected in the enthusiasm with which participants approached the synchronous conditions, with one participant stating that they felt ``buzzing'' with excitement after completing the task together with their collaborator (I12).

In contrast, the \async condition was perceived as less engaging by the majority of participants (I3, I4, E4, I6, E7, I8–I10, I12, E12, I14–I17, E14, E16). 
Participant I12 compared the two conditions, explaining that the asynchronous condition felt ``more like a task as opposed to fun,'' a sentiment echoed by other participants who described the asynchronous process as more isolated and procedural (I4, I7, E14, I17). E12, for example, mentioned that the asynchronous condition felt ``quicker but less satisfying,'' as there was no immediate feedback or dynamic interaction. 

\Insitu participants frequently expressed dissatisfaction with working toward a goal without seeing the final result (I3, I4, I5, I6, I7, I12, I16). 
For example, E16 stated, ``It was a weird feeling leaving and then not seeing it change.'' The lack of real-time collaboration in the \async condition also constrained opportunities for creative exploration. One participant explained that without the ability to brainstorm with another person, the task became ``more about checking a box'' rather than experimenting with new ideas (I10).

Collaborative brainstorming emerged as a key driver of engagement in the synchronous sessions. Participants reported that having someone to exchange ideas with in real time not only enhanced the creative process but also led to more diverse and spontaneous solutions (I9, I10, I17, E14). Participant I9 emphasized the ease of ``drafting concepts'' together in the synchronous condition, stating that ``we could brainstorm easily'' and quickly iterate on suggestions. 
These interactions not only facilitated creative exploration but also enhanced the sense of shared authorship, which several participants valued (I17, E10).

\subsubsection{\textbf{Multitasking overwhelmed some participants in synchronous collaboration, while asynchronous workflows were seen as more suitable for certain scenarios.}}\label{sec:results:interviews:mult-tasking}

Although the synchronous condition generally resulted in increased engagement and collaboration, some participants reported that the increased complexity of multitasking within the synchronous workflow was overwhelming (I8, E16, I17). Some participants also noted that the \sync condition increased pressure, as they felt their collaborators had to wait for them to complete tasks (I4, E12, E14). 

Participant E16 pointed out that managing multiple tasks simultaneously --- such as communicating with the \insitu user, observing the scene, and making design adjustments --- led to cognitive overload: ``I felt like I was doing too many things at once. Talking to [the \insitu user] and trying to watch the environment was sometimes just too much'' (E16). Similarly, I10 expressed that for users unfamiliar with the system, synchronous interactions might feel overwhelming, particularly due to the need to navigate while processing real-time feedback: ``For someone who's not experienced, it was just a lot. You're trying to follow directions, but there's so much going on'' (I10). This aligns with the feedback of several other participants less familiar with AR, who wished they had more time to practice ahead of the task (I3, I6, I10, I13, E10, E11, E16).

Several participants indicated that the synchronous condition was ideal for scenarios where immediate feedback was critical, but the asynchronous condition was more suitable when working in overwhelming environments. Participant I10 described how synchronous interaction could become overwhelming in crowded environments with high noise levels, explaining that ``if you're dealing with a busy place and someone is talking in your ear, it gets really overwhelming'' (I10).

Participants also recognized that both synchronous and asynchronous workflows had their strengths depending on the scenario. While some participants felt overwhelmed by multitasking in the synchronous condition, they still acknowledged its value for tasks requiring rapid decision-making or creative brainstorming. E12 noted, ``It's definitely harder when you have to do everything at once, but I still think the synchronous one is better when you're trying to bounce ideas off someone'' (E12). Participant I16 described how they viewed the \sync condition as most applicable for complex experiences, whereas the \async condition might be more suitable ``if it was, like, a smaller experience with one object.'' E9 also noted scenarios where asynchronous workflows could be beneficial, such as when \insitu participants already have ``enough information to already produce what I want'' (E9).

Moreover, participants reflected on how asynchronous and synchronous workflows or features could be complementary and applied at different stages of the development process (I5, E15, E16). For example, E15 highlighted a hybrid approach: ``I kind of see it as you start with asynchronous, then you do synchronous to refine it, to collect feedback on your experience.'' Other participants saw potential in a hybrid system that flexibly integrates asynchronous and synchronous workflows while incorporating all of \textsc{CoCreatAR}'s capturing and annotation features (I5, E15) and enabling lightweight \insitu editing as in addition (E16).


%-----------------------------------------------------------
\section{Conclusions and Future Work}
\label{sec:conclusions}
%\section{Discussion}
% Delete the text and write your Discussion here:
%------------------------------------

The paper proposes a novel approach for the efficient parameterization and estimation of the state of a hanging tether for path and trajectory planning of a tethered marsupial robotic system combining an unmanned ground vehicle (UGV) and an unmanned aerial vehicle (UAV). The paper proposes integrating into the trajectory state to optimize the parameters that define the tether curve, and also demonstrates that the parabola curve approximation is a good and efficient representation of the tether.

Experimental results demonstrate that the approximation using a parameterization curve can generate smooth, collision-free trajectories in a fraction of the time taken by the state-of-the-art methods, thus improving the feasibility and efficiency of the obtained solutions. Furthermore, the implementation of the RRT* planning algorithm with the use of a parabola approach based on PDP improves the time calculation in complex three-dimensional environments, such as confined or obstacle-ridden spaces.

We notice that the parabola approximation is an effective alternative for trajectory planning in UAV-UGV systems, by significantly reducing the computation time without compromising the quality of the solutions. This opens new possibilities for the implementation of marsupial robotic systems in missions where real-time trajectory planning is required and in complex three-dimensional environments.

Future work will consider adapting the optimizer to incorporate local planning capabilities, enabling it to modify existing global planner solutions. This integration would provide greater flexibility in adjusting the trajectory as new information or obstacles arise. In addition, incorporating kinematical models of the UGV, UAV, and tether to allow for more realistic constraints, providing a closer representation of the physical capabilities and limitations of the system. 

%You should try to show insight into what happened and why, and how things could have
%gone differently. If you have presented any background theory, try to tie it together with
%your results. How do they relate? If they differ, try to explain why. Even if things didn’t
%work out as intended, a good discussion shows that you’ve understood what went wrong
%and how you could potentially overcome these obstacles. 


%The conclusion should summarise your main results and main points from the discussion.
%A rule of thumb is to not present any new information (information not found in the results or discussion).

%-----------------------------------------------------------
\bibliographystyle{IEEEtran} 
\bibliography{IEEEabrv}
\end{document}
