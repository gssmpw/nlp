%----------------------------------------------------------------------------------------

Here, we focus on the experimental validation of the proposed method. We will benchmark the solution against \cite{smartinezr2023}, which is, to the best of our knowledge, the only approach for planning the path and trajectory of a UGV and UAV linked by a 3D hanging tether. To this end, the proposed approach will be tested/validated in the same scenarios (S1, S2, S3, S4, S5) and the same initial and final configurations of the robotic system as in \cite{smartinezr2023}, which are publicly available\footnote{\url{https://github.com/robotics-upo/marsupial_optimizer/tree/noetic/experiments_execution_instructions}}. These scenarios are illustrated in Fig. \ref{fig:scenarios}. 

%\rev{The experiments used for comparison with our method, as described in \cite{smartinezr2023}, are available at \footnote{\url{https://github.com/robotics-upo/marsupial_optimizer/tree/noetic/experiments_execution_instructions}}. Our experiments retain the same five scenarios (S1, S2, S3, S4, S5), as illustrated in Fig. \ref{fig:scenarios}. Additionally, we use the same initial positions—two for each scenario—and maintain the same target points as described.}
%The approach is tested in the five scenarios ($S1$, $S2$, $S3$, $S4$, $S5$) 
%The scenarios are shown in Figure \ref{fig:scenarios}, considering open, narrow, confined, and cluttered spaces. Two different starting configurations are used in each scenario: one placed in an open area, which we refer to as the open problem, and another in the clutter, referred to as the cluttered problem. 

All the experiments have been run on the same computer, an eight-core AMD Ryzen 7 6800H CPU, 32 GB of RAM running Ubuntu 20.04 and ROS Noetic. The source code is available on an anonymous GitHub repository \footnote{\url{https://anonymous.4open.science/r/efficient_tether_parameterization-040F/}.}.

\begin{figure*}[t!]
  \centering
  \includegraphics[width=0.9\textwidth]{Figures/all_scenario1.png}
  \caption{Scenarios considered for validation. S1: Open/constrained space with arc as obstacle. S2: Narrow/constrained space with denied access to UGV. S3: Confined space with outlet duct for UAV. S4: Collapsed Fire Station. S5: Open space gas station.}
  \label{fig:scenarios}
  \vspace*{-5mm}
\end{figure*}

%------------------------------------------
\subsection{RRT* results}

In this section, we describe the results obtained by the RRT* path planner (Section \ref{sec:path_planning}) when using the catenary and parabola approaches to model the shape of the tether. 
The RRT* algorithm executes batches of 500 iterations, checking for a valid solution at the end of the batch. If not, another batch starts. The solution is used as an initial guess in the optimization step. We empirically set the maximum number of iterations to 10000, as we found that RRT* usually finds a solution before the 1000th iteration.  

First, we focus on the execution time required to solve a PDP or a CDP. Note that for a configuration to be valid in our proposed RRT* algorithm, a CDP or a PDP has to be solved between the UAV and UGV poses. In these experiments, we made the RRT* solve both problems, and we saved the execution times associated with each method. Figure \ref{fig:rrt_results} shows the violin plots of the execution times distributions for both Decision Problem methods (catenary and parabola). It is easy to see how the parabola method consistently outperforms the catenary method in mean and median time. Depending on the problem to be solved, the improvement in median times can range from a very noticeable 5x factor in S3, to also a significant 2x factor in S5.

Then, we evaluate the impact of each method on the execution time of the RRT* algorithm as a whole. As seen previously, our method for solving the PDP is faster than the method for solving the CDP. Therefore, the RRT* algorithm should also be faster when using the Parabola approach. Due to the random nature of the RRT* algorithm, we have repeated each experiment one hundred times. Table \ref{tab:results_rrt} shows the mean value and the standard deviation of the execution times of each method in each scenario. The table also shows the Decision Problem rate (DP rate), which stands for the percentage of solutions in which the Decision Problem must be computed, that is, there is no direct line of sight between the ground and the aerial robots. A scenario with DP rate of 0\% indicates that the Decision Problem was never computed, so ground and aerial configurations were able to be connected with a taut tether in all cases, while a scenario with DP rate of 100\% indicates that we needed to solve the Decision Problem for all ground and aerial configurations tested. As a rule of thumb, the higher the DP rate, the more complex/cluttered the scenario is.

We can see in Table \ref{tab:results_rrt} how the RRT* using PDP is clearly faster than using CPD in both mean time and standard deviation in all scenarios. The improvement depends on the type of problem to solve. The major improvement, 18x to 25x faster, is found in S1, which is relatively open and with obstacles in the middle of the path. S2, S3 and S4, cluttered closed volumes, are also solved very efficiently, with improvements ranging from 4x to 6x in time. Finally, S5 shows the smallest, yet significant, improvements of 1.9x to 2.8x faster. This is a large and very open scenario, which benefits solutions based on straight lines between robots (taut tether). In these cases, it is not necessary to compute the catenary and, therefore, the improvement of the PDP algorithm becomes less visible. These results match the DP rate, which is highest in S1 (greatest improvement), while S5 provides the lowest rate (smallest improvement).

In summary, the experimental results show that the proposed Parabola Decision Problem is a fast and convenient approximation to model a catenary in the RRT* algorithm.

%Table \ref{tab:results_rrt} also presents

\begin{figure*}[t!]
  \centering
  \includegraphics[width=0.141\linewidth]{Figures/violin_s1.png}
  \hfill
  \includegraphics[width=0.15\linewidth]{Figures/violin_s2.png}
  \hfill
  \includegraphics[width=0.15\linewidth]{Figures/violin_s3.png}
  \hfill
  \includegraphics[width=0.151\linewidth]{Figures/violin_s4.png}
  \hfill
  \includegraphics[width=0.141\linewidth]{Figures/violin_s5.png}
  \caption{Violin plots of the distribution of the execution times of the Decision Problem test bench with Parabola and Catenary methods. In blue, the approximate shape of the distribution is represented. The mean value is represented as a red cross, the median as a white dot and the quartiles are linked with a black line. }
  \label{fig:rrt_results}
\end{figure*}

% \begin{table}[t!]
% 	\caption{Execution time results on RRT* and preprocessing time in each scenario}
%  \label{tab:results_rrt}
% 	\begin{center}
% 		\begin{tabular}{|c|cc|c|}
% 			\hline
% 			\multirow{2}{*}{Scenario} & \multicolumn{2}{c|}{Execution Time (s)} & Preprocessing \\ \cline{2-3}
%             & Catenary & Parabola & Time (s) \\
% 			\hline S1.1 & 4.46$\pm$3.61 & \textbf{1.74}$\pm$0.86 & \multirow{2}{*}{28.1} \\
% 			\cline{1-3} S1.2 & 7.67$\pm$11.88 & \textbf{3.33}$\pm$1.91 & \\
% 			\hline S2.1 & 2.33$\pm$2.53  & \textbf{1.38}$\pm$0.52 &\multirow{2}{*}{152.6} \\
% 			 \cline{1-3} S2.2 & \textbf{1.44}$\pm$0.40 &  1.45$\pm$0.42 & \\
% 			\hline S3.1 & 3.07$\pm$1.55 & \textbf{2.90}$\pm$0.66 & \multirow{2}{*}{400.3} \\
% 			\cline{1-3} S3.2 & 4.01$\pm$1.93 & \textbf{2.64}$\pm$0.66 & \\
% 			 \hline S4.1  & 2.57$\pm$2.22 & \textbf{1.55}$\pm$0.40 & \multirow{2}{*}{3152.2} \\
% 			\cline{1-3} S4.2 & 2.43$\pm$2.20 & \textbf{1.51}$\pm$0.41 & \\
% 			\hline S5.1 & 14.47 \
% 			\cline{1-3} S5.2 & 10.74$\pm$11.21 & \textbf{7.39}$\pm$2.57 & \\
% 			\hline
% 		\end{tabular}
% 	\end{center}
% \vspace{-6mm}	
% \end{table}

\begin{table}[t!]
	\caption{Execution time and decision problem rate results on RRT* }
 \label{tab:results_rrt}
	\begin{center}
		\begin{tabular}{|c|cc|c|}
			\hline
			\multirow{2}{*}{Scenario} & \multicolumn{2}{c|} {Execution Time (s)} & \multirow{2}{*}{DP rate (\%) } \\ \cline{2-3}
            & Catenary & Parabola &  \\
			\hline S1.1 & 5.16$\pm$6.66 & \textbf{0.28}$\pm$0.13 &93 \\
			\cline{1-4} S1.2 & 7.73$\pm$10.31 & \textbf{0.30}$\pm$0.10  & 91 \\
			\hline S2.1 & 0.31$\pm$0.36  & \textbf{0.06}$\pm$0.03  & 81 \\
			 \hline S2.2 & 0.35$\pm$0.39 &  \textbf{0.07}$\pm$0.03  & 80 \\
			\hline S3.1 & 3.08$\pm$1.56 & \textbf{0.47}$\pm$0.10 & 84 \\
			\hline S3.2 & 3.01$\pm$1.93 & \textbf{0.75}$\pm$0.26  & 86 \\
			 \hline S4.1  & 2.57$\pm$1.16 & \textbf{0.48}$\pm$0.19 & 81 \\
			\hline S4.2 & 2.43$\pm$2.20 & \textbf{0.44}$\pm$0.39  & 87 \\
			\hline S5.1 & 14.47$\pm$11.69 & \textbf{5.06}$\pm$2.75 & 69 \\
			\hline S5.2 & 10.74$\pm$5.46 & \textbf{5.56}$\pm$2.08 & 71 \\
			\hline
		\end{tabular}
	\end{center}
\vspace{-6mm}	
\end{table}

%% Preprocessing times not included.

% Final times:

%

%----------------------------------------------------------------------------
\subsection{Optimization results}

Different experiments were carried out to test the parabola approach within the optimization process. The results are presented in Table \ref{tab:experiments_new}. We benchmarked the new approaches with respect to \cite{smartinezr2023}.

We tested the methods in two different initial positions for each scenario. All experiments have been executed 100 times with the same set of parameters detailed below. The weight factors, empirically selected, are in the range $[0, 1]$.

\begin{itemize}
    \item Weighting factors for UGV: $\gamma_{eg}$ = 0.2, $\gamma_{og}$ = 0.08, $\gamma_{trav}$ = 0.5, $\gamma_{sg}$ = 0.12, $\gamma_{vg}$ = 0.05, $\gamma_{ag}$ = 0.005. 
    \item Weighting factors for UAV: $\gamma_{ea}$ = 0.25, $\gamma_{oa}$ = 0.08, $\gamma_{sa}$ = 0.14, $\gamma_{va}$ = 0.05, $\gamma_{aa}$ = 0.005. 
    \item Weighting factors for Tether: $\gamma_{ot}$ = 0.25, $\gamma_{u}$ = 0.1, $\gamma_{p}$ = 0.1
    \item Equidistance threshold: $\rho_{eg}$ and $\rho_{ea}$ are computed based on the initial path.   
    \item Collision threshold: $\rho_{oa}$ = 1.2, $\rho_{ot}$ = 0.1, $\rho_{og}$ = 1.2. 
    \item Traversability threshold: $\rho_{trav}$ = 0.001.
    \item Smoothness threshold: $\rho_{sg} $ = $\frac{\pi}{9}$, $\rho_{sa} $ = $\frac{\pi}{9}$.
    \item Desired velocity (m/s): $\rho_{vg}$ = 1.0, $\rho_{va}$ = 1.0.
    \item Desired velocity (m/s): $\rho_{ag}$ = 0.0, $\rho_{aa}$ = 0.0.
    \item Tether avoidance: $\beta$ = 10.0 .
\end{itemize}

The results of the experiments are summarized in Table \ref{tab:experiments_new}. Next paragraphs evaluate the different metrics:

% \begin{table*}[]
% \caption{Results for parameterization catenary approach related to computing time optimization, distance to obstacles, and trajectory time parameters in simulated environments. SI: Scenario; F: Feasibility [\%]; TCO: Time for optimized solution [$s$];  DPOO: Distance parable obstacles optimized[$m$]; DOO: Distance obstacles optimized[$m$]; VTO: Velocity, optimized trajectory [$m/s$]; ATO: Acceleration, optimized trajectory [$m/s^2$]} 
% \centering
% \label{tab:experiments_new}
% \small
% \begin{adjustbox}{max width=\textwidth}
% \begin{tabular}{|l|ll|ll|llllll|llllll|}
% \hline 
% \multicolumn{3}{|c}{\textbf{ALL}} & 
% \multicolumn{2}{|c}{\textbf{CATENARY}} & 
% \multicolumn{6}{|c|}{\textbf{UGV}}            & 
% \multicolumn{6}{|c|}{\textbf{UAV}}            \\
% \hline 
%   \multicolumn{1}{|c}{\textbf{SI}} &
%   \multicolumn{1}{|c}{\textbf{F}} &
%   \multicolumn{1}{|c}{\textbf{TCO}} &
%   \multicolumn{1}{|c}{\textbf{\begin{tabular}[c]{@{}c@{}}Mean\\ DPOO\end{tabular}}} &
%   \multicolumn{1}{|c}{\textbf{\begin{tabular}[c]{@{}c@{}}Min\\ DPOO\end{tabular}}} &
%   \multicolumn{1}{|c}{\textbf{\begin{tabular}[c]{@{}c@{}}Mean\\ DOO\end{tabular}}} &
%   \multicolumn{1}{|c}{\textbf{\begin{tabular}[c]{@{}c@{}}Min\\ DOO\end{tabular}}} &
%   \multicolumn{1}{|c}{\textbf{\begin{tabular}[c]{@{}c@{}}Mean\\ VTO\end{tabular}}} &
%   \multicolumn{1}{|c}{\textbf{\begin{tabular}[c]{@{}c@{}}Max\\ VTO\end{tabular}}} &
%   \multicolumn{1}{|c}{\textbf{\begin{tabular}[c]{@{}c@{}}Mean\\ ATO\end{tabular}}} &
%   \multicolumn{1}{|c}{\textbf{\begin{tabular}[c]{@{}c@{}}Max\\ ATO\end{tabular}}} &
%   \multicolumn{1}{|c}{\textbf{\begin{tabular}[c]{@{}c@{}}Mean\\ DOO\end{tabular}}} &
%   \multicolumn{1}{|c}{\textbf{\begin{tabular}[c]{@{}c@{}}Min\\ DOO\end{tabular}}} &
%   \multicolumn{1}{|c}{\textbf{\begin{tabular}[c]{@{}c@{}}Mean\\ VTO\end{tabular}}} &
%   \multicolumn{1}{|c}{\textbf{\begin{tabular}[c]{@{}c@{}}Max\\ VTO\end{tabular}}} &
%   \multicolumn{1}{|c}{\textbf{\begin{tabular}[c]{@{}c@{}}Mean\\ ATO\end{tabular}}} &
%   \multicolumn{1}{|c|}{\textbf{\begin{tabular}[c]{@{}c@{}}Max\\ ATO\end{tabular}}} \\
% \hline 
% S1.1 & 85  & 6.666  & 1.404 & 0.480 & 4.592  & 2.022 & 0.528 & 0.880 & 0.262  & 0.173  & 4.593  & 1.829 & 0.880 & 1.086  & 0.313  & 0.046  \\
% S1.2 & 91  & 55.195 & 4.444 & 2.286 & 16.254 & 8.175 & 1.073 & 0.945 & 0.698  & 0.263  & 15.890 & 7.903 & 1.203 & 1.017  & 0.745  & 0.285  \\
% S2.1 & 100 & 9.903  & 0.142 & 0.150 & 0.546  & 0.643 & 0.757 & 0.019 & 0.596  & 0.000  & 0.643  & 0.643 & 1.093 & -0.001 & -0.009 & 0.000  \\
% S2.2 & 99  & 0.371  & 0.725 & 0.125 & 1.186  & 1.060 & 0.087 & 0.467 & 0.014  & 0.308  & 1.192  & 0.686 & 1.021 & 1.087  & -0.002 & -0.026 \\
% S3.1 & 95  & 0.707  & 0.647 & 0.103 & 1.143  & 0.754 & 0.550 & 1.041 & -0.004 & 0.039  & 1.031  & 0.657 & 1.123 & 1.382  & 0.014  & -0.036 \\
% S3.2 & 93  & 0.641  & 0.694 & 0.112 & 1.138  & 0.685 & 0.617 & 1.146 & -0.002 & -0.031 & 1.030  & 0.633 & 1.090 & 1.343  & 0.016  & 0.054  \\
% S4.1 & 55  & 0.851  & 0.668 & 0.066 & 2.024  & 0.634 & 0.808 & 1.288 & 0.025  & 0.074  & 1.576  & 0.678 & 1.130 & 1.467  & 0.030  & -0.044 \\
% S4.2 & 84  & 0.758  & 0.751 & 0.131 & 2.501  & 1.574 & 0.614 & 1.115 & 0.037  & 0.199  & 1.719  & 0.733 & 1.085 & 1.254  & 0.037  & -0.043 \\
% S5.1 & 97  & 2.005  & 1.025 & 0.283 & 3.293  & 0.858 & 0.888 & 1.783 & 0.146  & -0.327 & 2.948  & 1.017 & 1.044 & 1.561  & 0.165  & -0.300 \\
% S5.2 & 80  & 2.050  & 1.039 & 0.223 & 4.184  & 0.794 & 0.875 & 1.613 & 0.115  & -0.065 & 3.455  & 0.966 & 1.049 & 1.507  & 0.103  & -0.025 \\
% \hline 
% \end{tabular}
% \end{adjustbox}
% \vspace*{-2mm}
% \end{table*}


%\begin{table*}[]
%\caption{Results for parameterization parabola approach related to computing time optimization, distance to obstacles, and trajectory time parameters in simulated environments. SI: Scenario; F: Feasibility [\%]; TCO: Time for optimized solution [$s$];  DPOO: Distance parable obstacles optimized[$m$]; DOO: Distance obstacles optimized[$m$]; VTO: Velocity, optimized trajectory [$m/s$]; ATO: Acceleration, optimized trajectory [$m/s^2$]} 
%\centering
%\label{tab:experiments_new}
%\small
%\begin{adjustbox}{max width=\textwidth}
%\begin{tabular}{|l|ll|ll|llllll|llllll|}
%\hline 
%\multicolumn{3}{|c}{\textbf{ALL}} & 
%\multicolumn{2}{|c}{\textbf{PARABOLA}} & 
%\multicolumn{6}{|c|}{\textbf{UGV}}            & 
%\multicolumn{6}{|c|}{\textbf{UAV}}            \\
%\hline 
%  \multicolumn{1}{|c}{\textbf{SI}} &
%  \multicolumn{1}{|c}{\textbf{F}} &
%  \multicolumn{1}{|c}{\textbf{TCO}} &
%  \multicolumn{1}{|c}{\textbf{\begin{tabular}[c]{@{}c@{}}Mean\\ DPOO\end{tabular}}} &
%  \multicolumn{1}{|c}{\textbf{\begin{tabular}[c]{@{}c@{}}Min\\ DPOO\end{tabular}}} &
%  \multicolumn{1}{|c}{\textbf{\begin{tabular}[c]{@{}c@{}}Mean\\ DOO\end{tabular}}} &
%  \multicolumn{1}{|c}{\textbf{\begin{tabular}[c]{@{}c@{}}Min\\ DOO\end{tabular}}} &
%  \multicolumn{1}{|c}{\textbf{\begin{tabular}[c]{@{}c@{}}Mean\\ VTO\end{tabular}}} &
%  \multicolumn{1}{|c}{\textbf{\begin{tabular}[c]{@{}c@{}}Max\\ VTO\end{tabular}}} &
%  \multicolumn{1}{|c}{\textbf{\begin{tabular}[c]{@{}c@{}}Mean\\ ATO\end{tabular}}} &
%  \multicolumn{1}{|c}{\textbf{\begin{tabular}[c]{@{}c@{}}Max\\ ATO\end{tabular}}} &
%  \multicolumn{1}{|c}{\textbf{\begin{tabular}[c]{@{}c@{}}Mean\\ DOO\end{tabular}}} &
%  \multicolumn{1}{|c}{\textbf{\begin{tabular}[c]{@{}c@{}}Min\\ DOO\end{tabular}}} &
%  \multicolumn{1}{|c}{\textbf{\begin{tabular}[c]{@{}c@{}}Mean\\ VTO\end{tabular}}} &
%  \multicolumn{1}{|c}{\textbf{\begin{tabular}[c]{@{}c@{}}Max\\ VTO\end{tabular}}} &
%  \multicolumn{1}{|c}{\textbf{\begin{tabular}[c]{@{}c@{}}Mean\\ ATO\end{tabular}}} &
%  \multicolumn{1}{|c|}{\textbf{\begin{tabular}[c]{@{}c@{}}Max\\ ATO\end{tabular}}} \\
%\hline 
%S1.1 & 98  & 7.049  & 1.278 & 0.959 & 5.378  & 2.629 & 0.518 & 0.799 & 0.230  & 0.292  & 5.299  & 2.498 & 0.928 & 1.085 & 0.340 & 0.159  \\
%S1.2 & 97  & 18.162 & 4.823 & 2.767 & 15.182 & 7.816 & 2.317 & 1.531 & 0.655  & 0.331  & 14.721 & 7.645 & 2.513 & 1.611 & 0.691 & 0.343  \\
%S2.1 & 100 & 1.495  & 0.694 & 0.151 & 0.777  & 0.548 & 0.141 & 0.451 & 0.016  & 0.202  & 1.122  & 0.759 & 1.053 & 1.177 & 0.002 & -0.018 \\
%S2.2 & 100 & 1.188  & 0.735 & 0.152 & 1.156  & 1.003 & 0.082 & 0.576 & 0.019  & 0.554  & 1.225  & 0.782 & 1.026 & 1.119 & 0.001 & 0.013  \\
%S3.1 & 92  & 1.493  & 0.645 & 0.060 & 1.139  & 0.673 & 0.541 & 1.124 & -0.003 & -0.140 & 1.077  & 0.638 & 1.154 & 1.495 & 0.016 & 0.039  \\
%S3.2 & 93  & 1.271  & 0.711 & 0.157 & 1.416  & 0.622 & 0.583 & 1.133 & 0.018  & -0.169 & 1.360  & 0.658 & 1.080 & 1.348 & 0.041 & 0.039  \\
%S4.1 & 94  & 1.495  & 0.666 & 0.078 & 1.786  & 1.055 & 0.713 & 1.122 & 0.011  & 0.129  & 1.227  & 0.777 & 1.110 & 1.242 & 0.000 & 0.029  \\
%S4.2 & 99  & 1.448  & 0.737 & 0.149 & 2.063  & 1.552 & 0.590 & 1.046 & 0.014  & 0.284  & 1.260  & 0.771 & 1.113 & 1.231 & 0.001 & 0.024  \\
%S5.1 & 100 & 3.747  & 1.165 & 0.310 & 2.566  & 1.124 & 0.736 & 0.994 & 0.041  & 0.124  & 2.231  & 1.216 & 1.011 & 1.101 & 0.035 & -0.020 \\
%S5.2 & 100 & 3.366  & 1.160 & 0.309 & 4.196  & 0.935 & 0.745 & 0.988 & 0.084  & 0.284  & 3.520  & 1.049 & 0.996 & 1.130 & 0.075 & -0.010 \\
%\hline 
%\end{tabular}
%\end{adjustbox}
%\vspace*{-2mm}
%\end{table*}


\begin{table*}[t!]
\caption{Results of the optimizer for different parameterizations related to computing time, distance to obstacles, and trajectory time in simulated environments. SI: Scenario; M: Method (P = Parabola, C = Catenary, \cite{smartinezr2023})  F: Feasibility [\%]; T: Time for optimized solution [$s$];  DO: Distance to obstacles[$m$]; V: Velocity of the optimized trajectory [$m/s$]; A: Acceleration of the optimized trajectory [$m/s^2$]} 
\centering
\label{tab:experiments_new}
\footnotesize
\begin{adjustbox}{max width=\textwidth}
\begin{tabular}{|c|ccc|cc|cccccc|cccccc|}
\hline 
\multicolumn{4}{|c}{\textbf{ALL}} & 
\multicolumn{2}{|c}{\textbf{TETHER}} & 
\multicolumn{6}{|c|}{\textbf{UGV}}            & 
\multicolumn{6}{|c|}{\textbf{UAV}}            \\
\hline 
  \multicolumn{1}{|c}{\textbf{SI}} &
  \multicolumn{1}{|c}{\textbf{M}} &
  \multicolumn{1}{|c}{\textbf{F}} &
  \multicolumn{1}{|c}{\textbf{T}} &
  \multicolumn{1}{|c}{\textbf{\begin{tabular}[c]{@{}c@{}}Mean\\ DO\end{tabular}}} &
  \multicolumn{1}{|c}{\textbf{\begin{tabular}[c]{@{}c@{}}Min\\ DO\end{tabular}}} &
  \multicolumn{1}{|c}{\textbf{\begin{tabular}[c]{@{}c@{}}Mean\\ DO\end{tabular}}} &
  \multicolumn{1}{|c}{\textbf{\begin{tabular}[c]{@{}c@{}}Min\\ DO\end{tabular}}} &
  \multicolumn{1}{|c}{\textbf{\begin{tabular}[c]{@{}c@{}}Mean\\ V\end{tabular}}} &
  \multicolumn{1}{|c}{\textbf{\begin{tabular}[c]{@{}c@{}}Max\\ V\end{tabular}}} &
  \multicolumn{1}{|c}{\textbf{\begin{tabular}[c]{@{}c@{}}Mean\\ A\end{tabular}}} &
  \multicolumn{1}{|c}{\textbf{\begin{tabular}[c]{@{}c@{}}Max\\ A\end{tabular}}} &
  \multicolumn{1}{|c}{\textbf{\begin{tabular}[c]{@{}c@{}}Mean\\ DO\end{tabular}}} &
  \multicolumn{1}{|c}{\textbf{\begin{tabular}[c]{@{}c@{}}Min\\ DO\end{tabular}}} &
  \multicolumn{1}{|c}{\textbf{\begin{tabular}[c]{@{}c@{}}Mean\\ V\end{tabular}}} &
  \multicolumn{1}{|c}{\textbf{\begin{tabular}[c]{@{}c@{}}Max\\ V\end{tabular}}} &
  \multicolumn{1}{|c}{\textbf{\begin{tabular}[c]{@{}c@{}}Mean\\ A\end{tabular}}} &
  \multicolumn{1}{|c|}{\textbf{\begin{tabular}[c]{@{}c@{}}Max\\ A\end{tabular}}} \\
\hline 
\multirow{3}{*}{S1.1} & P & \textbf{98}  & 7.049  & 1.278 & \textbf{0.959} & 5.378  & \textbf{2.629} & 0.518 & 0.799 & 0.230  & 0.292  & 5.299  & \textbf{2.498} & 0.928 & 1.085 & 0.340 & 0.159  \\
& C & 85  & \textbf{6.666}  & 1.404 & 0.480 & 4.592  & 2.022 & 0.528 & 0.880 & 0.262  & 0.173  & 4.593  & 1.829 & 0.880 & 1.086  & 0.313  & 0.046  \\
 & \cite{smartinezr2023} & 86 & 465.9 & 1.14 & 0.25 & 2.28 & 1.55 & 0.36 & 1.07 & 0.02 & 0.81  & 1.87 & 0.80 & 0.97 & 1.19 & -0.002 & -0.09 \\ 
\hline 

\multirow{3}{*}{S1.2} & P & \textbf{97}  & \textbf{18.162} & 4.823 & \textbf{2.767} & 15.182 & 7.816 & 2.317 & 1.531 & 0.655  & 0.331  & 14.721 & 7.645 & 2.513 & 1.611 & 0.691 & 0.343  \\
 & C & 91  & 55.195 & 4.444 & 2.286 & 16.254 & \textbf{8.175} & 1.073 & 0.945 & 0.698  & 0.263  & 15.890 & \textbf{7.903} & 1.203 & 1.017  & 0.745  & 0.285  \\
 & \cite{smartinezr2023} & 80 & 529.7 & 1.31 & 0.16 & 3.76 & 1.60 & 0.43 & 1.25 & 0.02 & 0.78  & 2.21 & 0.89 & 0.91 & 1.19 & -0.002 & -0.08 \\
\hline 

\multirow{3}{*}{S2.1} & P & \textbf{100} & \textbf{1.495}  & 0.694 & \textbf{0.151} & 0.777  & 0.548 & 0.141 & 0.451 & 0.016  & 0.202  & 1.122  & \textbf{0.759} & 1.053 & 1.177 & 0.002 & -0.018 \\
 & C & \textbf{100} & 9.903  & 0.142 & 0.150 & 0.546  & \textbf{0.643} & 0.757 & 0.019 & 0.596  & 0.000  & 0.643  & 0.643 & 1.093 & -0.001 & -0.009 & 0.000  \\
 & \cite{smartinezr2023} & 95 & 123.6 & 0.72 & 0.10 & 0.69 & 0.56 & 0.13 & 0.91 & 0.02 & 0.96  & 1.19 & 0.76 & 1.01 & 1.15 & 0.002  & 0.01  \\
\hline 

\multirow{3}{*}{S2.2} & P & \textbf{100} & 1.188  & 0.735 & \textbf{0.152} & 1.156  & 1.003 & 0.082 & 0.576 & 0.019  & 0.554  & 1.225  & \textbf{0.782} & 1.026 & 1.119 & 0.001 & 0.013  \\
 & C & 99  & \textbf{0.371}  & 0.725 & 0.125 & 1.186  & \textbf{1.060} & 0.087 & 0.467 & 0.014  & 0.308  & 1.192  & 0.686 & 1.021 & 1.087  & -0.002 & -0.026 \\
 & \cite{smartinezr2023} &  95 & 109.3 & 0.75 & 0.13 & 0.75 & 0.58 & 0.19 & 1.00 & 0.03 & 1.09  & 1.16 & 0.67 & 0.99 & 1.09 & 0.000  & 0.07  \\
\hline 

\multirow{3}{*}{S3.1} & P & 92  & 1.493  & 0.645 & 0.060 & 1.139  & 0.673 & 0.541 & 1.124 & -0.003 & -0.140 & 1.077  & 0.638 & 1.154 & 1.495 & 0.016 & 0.039  \\
 & C & \textbf{95}  & \textbf{0.707}  & 0.647 & 0.103 & 1.143  & \textbf{0.754} & 0.550 & 1.041 & -0.004 & 0.039  & 1.031  & \textbf{0.657} & 1.123 & 1.382  & 0.014  & -0.036 \\
 & \cite{smartinezr2023} &  85 & 213.6 &  0.67 & \textbf{0.13} & 0.98 & 0.62 & 0.50 & 1.07 & 0.00 & 0.21  & 0.81 & 0.56 & 1.02 & 1.21 & 0.003  & 0.16  \\
\hline 

\multirow{3}{*}{S3.2} & P & \textbf{93}  & 1.271  & 0.711 & \textbf{0.157} & 1.416  & 0.622 & 0.583 & 1.133 & 0.018  & -0.169 & 1.360  & \textbf{0.658} & 1.080 & 1.348 & 0.041 & 0.039  \\
 & C & \textbf{93}  & \textbf{0.641}  & 0.694 & 0.112 & 1.138  & \textbf{0.685} & 0.617 & 1.146 & -0.002 & -0.031 & 1.030  & 0.633 & 1.090 & 1.343  & 0.016  & 0.054  \\
 & \cite{smartinezr2023} & 84 & 214.9  & 0.67 & 0.14 & 0.97 & 0.61 & 0.55 & 1.08 & 0.00 & -0.20 & 0.79 & 0.54 & 0.98 & 1.16 & 0.004  & 0.18  \\
\hline 

\multirow{3}{*}{S4.1} & P & \textbf{94}  & 1.495  & 0.666 & 0.078 & 1.786  & 1.055 & 0.713 & 1.122 & 0.011  & 0.129  & 1.227  & \textbf{0.777} & 1.110 & 1.242 & 0.000 & 0.029  \\
 & C & 55  & \textbf{0.851}  & 0.668 & 0.066 & 2.024  & 0.634 & 0.808 & 1.288 & 0.025  & 0.074  & 1.576  & 0.678 & 1.130 & 1.467  & 0.030  & -0.044 \\
 & \cite{smartinezr2023} & 83 & 539.0  & 0.74 & \textbf{0.14} & 2.12 & \textbf{1.40} & 0.45 & 1.14 & 0.02 & 1.00  & 1.16 & 0.65 & 0.94 & 1.12 & -0.002 & -0.06 \\
\hline 

\multirow{3}{*}{S4.2} & P & \textbf{99}  & 1.448  & 0.737 & \textbf{0.149} & 2.063  & 1.552 & 0.590 & 1.046 & 0.014  & 0.284  & 1.260  & \textbf{0.771} & 1.113 & 1.231 & 0.001 & 0.024  \\
 & C & 84  & \textbf{0.758}  & 0.751 & 0.131 & 2.501  & \textbf{1.574} & 0.614 & 1.115 & 0.037  & 0.199  & 1.719  & 0.733 & 1.085 & 1.254  & 0.037  & -0.043 \\
 & \cite{smartinezr2023} & 76 & 708.6 & 0.76 & 0.10 & 1.70 & 1.07 & 0.47 & 1.32 & 0.01 & 0.86  & 1.15 & 0.65 & 0.92 & 1.22 & -0.002 & -0.32 \\
\hline 

\multirow{3}{*}{S5.1} & P & \textbf{100} & 3.747  & 1.165 & \textbf{0.310} & 2.566  & \textbf{1.124} & 0.736 & 0.994 & 0.041  & 0.124  & 2.231  & \textbf{1.216} & 1.011 & 1.101 & 0.035 & -0.020 \\
 & C & 97  & \textbf{2.005}  & 1.025 & 0.283 & 3.293  & 0.858 & 0.888 & 1.783 & 0.146  & -0.327 & 2.948  & 1.017 & 1.044 & 1.561  & 0.165  & -0.300 \\
 & \cite{smartinezr2023} &  98 & 277.8 & 1.09 & 0.23 & 2.26 & 0.76 & 0.41 & 1.15 & 0.03 & 1.07  & 1.74 & 0.83 & 0.97 & 1.27 & -0.004 & -0.14 \\
\hline 

\multirow{3}{*}{S5.2} & P & \textbf{100} & 3.366  & 1.160 & \textbf{0.309} & 4.196  & \textbf{0.935} & 0.745 & 0.988 & 0.084  & 0.284  & 3.520  & \textbf{1.049} & 0.996 & 1.130 & 0.075 & -0.010 \\
 & C & 80  & \textbf{2.050}  & 1.039 & 0.223 & 4.184  & 0.794 & 0.875 & 1.613 & 0.115  & -0.065 & 3.455  & 0.966 & 1.049 & 1.507  & 0.103  & -0.025 \\
 & \cite{smartinezr2023} &  95 & 447.7  & 1.14 & 0.22 & 2.53 & 0.63 & 0.57 & 1.08 & 0.02 & 1.02  & 1.66 & 0.65 & 0.92 & 1.11 & -0.004 & -0.12\\


\hline 
\end{tabular}
\end{adjustbox}
\vspace*{-2mm}
\end{table*}


% \begin{table*}[]
% \caption{Results for parabola approach related to computing time optimization, distance to obstacles, and trajectory time parameters in simulated environments. SI: Scenario; F: Feasibility [\%]; TCO: Time for optimized solution [$s$];  DPOO: Distance parable obstacles optimized[$m$]; DOO: Distance obstacles optimized[$m$]; VTO: Velocity, optimized trajectory [$m/s$]; ATO: Acceleration, optimized trajectory [$m/s^2$]} 
% \centering
% \label{tab:experiments_new}
% \small
% \begin{adjustbox}{max width=\textwidth}
% \begin{tabular}{|l|ll|ll|llllll|llllll|}
% \hline 
% \multicolumn{3}{|c}{\textbf{ALL}} & 
% \multicolumn{2}{|c}{\textbf{PARABOLA}} & 
% \multicolumn{6}{|c|}{\textbf{UGV}}            & 
% \multicolumn{6}{|c|}{\textbf{UAV}}            \\
% \hline 
%   \multicolumn{1}{|c}{\textbf{SI}} &
%   \multicolumn{1}{|c}{\textbf{F}} &
%   \multicolumn{1}{|c}{\textbf{TCO}} &
%   \multicolumn{1}{|c}{\textbf{\begin{tabular}[c]{@{}c@{}}Mean\\ DPOO\end{tabular}}} &
%   \multicolumn{1}{|c}{\textbf{\begin{tabular}[c]{@{}c@{}}Min\\ DPOO\end{tabular}}} &
%   \multicolumn{1}{|c}{\textbf{\begin{tabular}[c]{@{}c@{}}Mean\\ DOO\end{tabular}}} &
%   \multicolumn{1}{|c}{\textbf{\begin{tabular}[c]{@{}c@{}}Min\\ DOO\end{tabular}}} &
%   \multicolumn{1}{|c}{\textbf{\begin{tabular}[c]{@{}c@{}}Mean\\ VTO\end{tabular}}} &
%   \multicolumn{1}{|c}{\textbf{\begin{tabular}[c]{@{}c@{}}Max\\ VTO\end{tabular}}} &
%   \multicolumn{1}{|c}{\textbf{\begin{tabular}[c]{@{}c@{}}Mean\\ ATO\end{tabular}}} &
%   \multicolumn{1}{|c}{\textbf{\begin{tabular}[c]{@{}c@{}}Max\\ ATO\end{tabular}}} &
%   \multicolumn{1}{|c}{\textbf{\begin{tabular}[c]{@{}c@{}}Mean\\ DOO\end{tabular}}} &
%   \multicolumn{1}{|c}{\textbf{\begin{tabular}[c]{@{}c@{}}Min\\ DOO\end{tabular}}} &
%   \multicolumn{1}{|c}{\textbf{\begin{tabular}[c]{@{}c@{}}Mean\\ VTO\end{tabular}}} &
%   \multicolumn{1}{|c}{\textbf{\begin{tabular}[c]{@{}c@{}}Max\\ VTO\end{tabular}}} &
%   \multicolumn{1}{|c}{\textbf{\begin{tabular}[c]{@{}c@{}}Mean\\ ATO\end{tabular}}} &
%   \multicolumn{1}{|c|}{\textbf{\begin{tabular}[c]{@{}c@{}}Max\\ ATO\end{tabular}}} \\
% \hline 
% S1.1 & 90.0 & 1.793 & 1.163 & 0.301 & 2.402 & 1.926 & 0.124 & 0.220 & 0.0057 & 0.0419 & 2.559 & 1.722 &1.117 & 1.340 & 0.0150 & 0.0129 \\
% S1.2 & 90.0 & 2.118 & 1.494 & 0.481 & 4.994 & 2.021 & 0.076 & 0.306 & 0.0174 & 0.0253 & 5.206 & 2.500 & 0.944 & 1.316 & 0.0149 & 0.0037 \\
% S2.1 & 90.0 & 3.178 & 0.717 & 0.090 & 1.156 & 1.110 & 0.008 & 0.079 & 0.0006 & 0.0125 & 1.087 & 0.993 & 1.158 & 1.330 & 0.0003 & 0.0019 \\
% S2.2 & 100.0 & 0.132 & 0.818 & 0.136 & 1.374 & 1.169 & 0.001 & 0.040 & 0.0003 & 0.0070 & 1.057 & 1.001 & 0.950 & 1.258 & 0.000005 & 0.0026 \\
% S3.1 & 80.0 & 3.667  & 0.823 & 0.155 & 0.989 & 0.756 & 0.587 & 1.071 & -0.015 & -0.497 & 0.893 & 0.803 & 1.094 & 1.304 & 0.0003 & 0.093  \\
% S3.2 & 85.0 & 4.131  & 0.803 & 0.106 & 0.954 & 0.758 & 0.639 & 1.061 & -0.009 & -0.248 & 0.873 & 0.817 & 1.075 & 1.264 & 0.0013 & 0.058  \\
% S4.1 & 90.0 & 3.107 & 0.662 & 0.138 & 1.700 & 1.198 & 0.143 & 0.332 & 0.0013 & 0.0389 & 1.145 & 0.999 & 1.164 & 1.345 & 0.0002 & 0.0225 \\
% S4.2 & 95.0 & 4.790 & 0.728 & 0.260 & 2.194 & 1.727 & 0.116 & 0.338 & 0.0012 & 0.0345 & 1.104 & 1.004 & 1.061 & 1.335 & 0.0001 & -0.0057 \\
% S5.1 & 90.0 & 16.285 & 1.250 & 0.207 & 2.500 & 1.104 & 0.778 & 1.123 & 0.016 & 0.404 & 1.481 & 1.016 & 1.055 & 1.200 & -0.0006 & -0.085 \\
% S5.2 & 90.0 & 11.359 & 1.268 & 0.288 & 1.972 & 1.055 & 0.719 & 1.180 & 0.020 & 0.331 & 1.598 & 1.000 & 1.058 & 1.276 & 0.0001  & 0.021  \\
% \hline 
% \end{tabular}
% \end{adjustbox}
% \vspace*{-2mm}
% \end{table*}

%\begin{table*}[t!]
\caption{Results for catenary approach  taken from \cite{smartinezr2023}, related to computing time optimization, distance to obstacles, and trajectory time parameters in simulated environments. SI: Scenario; F: Feasibility [\%]; TCO: Time for optimized solution [$s$];  DPOO: Distance parable obstacles optimized[$m$]; DOO: Distance obstacles optimized[$m$]; VTO: Velocity, optimized trajectory [$m/s$]; ATO: Acceleration, optimized trajectory [$m/s^2$]} 
\centering
\label{tab:experiments_old}
\small
\begin{adjustbox}{max width=\textwidth}
\begin{tabular}{|l|ll|ll|llllll|llllll|}
\hline
\multicolumn{3}{|c}{\textbf{ALL}} &
\multicolumn{2}{|c}{\textbf{CATENARY}} &
\multicolumn{6}{|c}{\textbf{UGV}} & 
\multicolumn{6}{|c|}{\textbf{UAV}} \\ 
\hline
\multicolumn{1}{|c}{\textbf{SI}} & 
\multicolumn{1}{|c}{\textbf{F}} & 
\multicolumn{1}{|c}{\textbf{TCO}} &
\multicolumn{1}{|c}{\textbf{\begin{tabular}[c]{@{}c@{}}Mean\\ DCOO\end{tabular}}} &
\multicolumn{1}{|c}{\textbf{\begin{tabular}[c]{@{}c@{}}Min\\ DCOO\end{tabular}}} &
\multicolumn{1}{|c}{\textbf{\begin{tabular}[c]{@{}c@{}}Mean\\ DOO\end{tabular}}} &
\multicolumn{1}{|c}{\textbf{\begin{tabular}[c]{@{}c@{}}Min\\ DOO\end{tabular}}} &
\multicolumn{1}{|c}{\textbf{\begin{tabular}[c]{@{}c@{}}Mean\\ VTO\end{tabular}}} & 
\multicolumn{1}{|c}{\textbf{\begin{tabular}[c]{@{}c@{}}Max\\ VTO\end{tabular}}} & 
\multicolumn{1}{|c}{\textbf{\begin{tabular}[c]{@{}c@{}}Mean\\ ATO\end{tabular}}} & 
\multicolumn{1}{|c}{\textbf{\begin{tabular}[c]{@{}c@{}}Max\\ ATO\end{tabular}}} &
\multicolumn{1}{|c}{\textbf{\begin{tabular}[c]{@{}c@{}}Mean\\ DOO\end{tabular}}} &
\multicolumn{1}{|c}{\textbf{\begin{tabular}[c]{@{}c@{}}Min\\ DOO\end{tabular}}} & 
\multicolumn{1}{|c}{\textbf{\begin{tabular}[c]{@{}c@{}}Mean\\ VTO\end{tabular}}} & 
\multicolumn{1}{|c}{\textbf{\begin{tabular}[c]{@{}c@{}}Max\\ VTO\end{tabular}}} & 
\multicolumn{1}{|c}{\textbf{\begin{tabular}[c]{@{}c@{}}Mean\\ ATO\end{tabular}}} & 
\multicolumn{1}{|c|}{\begin{tabular}[c]{@{}c@{}}Max\\ ATO\end{tabular}} \\
\hline
S1.1 &  86.0 & 465.9 & 1.14 & 0.25 & 2.28 & 1.55 & 0.36 & 1.07 & 0.02 & 0.81  & 1.87 & 0.80 & 0.97 & 1.19 & -0.002 & -0.09 \\
S1.2 &  80.0 & 529.7 & 1.31 & 0.16 & 3.76 & 1.60 & 0.43 & 1.25 & 0.02 & 0.78  & 2.21 & 0.89 & 0.91 & 1.19 & -0.002 & -0.08 \\
S2.1 &  94.5 & 123.6 & 0.72 & 0.10 & 0.69 & 0.56 & 0.13 & 0.91 & 0.02 & 0.96  & 1.19 & 0.76 & 1.01 & 1.15 & 0.002  & 0.01  \\
S2.2 &  95.0 & 109.3 & 0.75 & 0.13 & 0.75 & 0.58 & 0.19 & 1.00 & 0.03 & 1.09  & 1.16 & 0.67 & 0.99 & 1.09 & 0.000  & 0.07  \\
S3.1 &  84.8 & 213.6 & 0.67 & 0.13 & 0.98 & 0.62 & 0.50 & 1.07 & 0.00 & 0.21  & 0.81 & 0.56 & 1.02 & 1.21 & 0.003  & 0.16  \\
S3.2 &  84.0 &214.9 & 0.67 & 0.14 & 0.97 & 0.61 & 0.55 & 1.08 & 0.00 & -0.20 & 0.79 & 0.54 & 0.98 & 1.16 & 0.004  & 0.18  \\
S4.1 &  82.5 & 539.0 & 0.74 & 0.14 & 2.12 & 1.40 & 0.45 & 1.14 & 0.02 & 1.00  & 1.16 & 0.65 & 0.94 & 1.12 & -0.002 & -0.06 \\
S4.2 &  76.0 & 708.6 & 0.76 & 0.10 & 1.70 & 1.07 & 0.47 & 1.32 & 0.01 & 0.86  & 1.15 & 0.65 & 0.92 & 1.22 & -0.002 & -0.32 \\
S5.1 &  98.0 & 277.8 & 1.09 & 0.23 & 2.26 & 0.76 & 0.41 & 1.15 & 0.03 & 1.07  & 1.74 & 0.83 & 0.97 & 1.27 & -0.004 & -0.14 \\
S5.2 &  95.0 & 447.7 & 1.14 & 0.22 & 2.53 & 0.63 & 0.57 & 1.08 & 0.02 & 1.02  & 1.66 & 0.65 & 0.92 & 1.11 & -0.004 & -0.12\\
\hline
\end{tabular}
\end{adjustbox}
\vspace*{-2mm}
\end{table*}


subsection{Feasibility}
The result of the planners is considered feasible (F) when the computed trajectory is collision-free for every agent of the system (UAV, UGV, tether). This means that $d^i_{og} > \rho_{og}$ and $d^i_{oa} > \rho_{oa}$ for every UGV and UAV state, respectively. In the case of the tether, $d^i_{ot,j} > \rho_{ot}$ for every sample in each tether configuration. In this context, we can see that the proposed solutions are consistently more feasible than \cite{smartinezr2023}, the average feasibility of the trajectory planner with the parabola approach is 97.3\%. Besides, with the catenary approach, the average feasibility is 87.9\%. In both, the unfeasible solutions are due to tether collisions. %The improvement in the results is mainly due to what was explained above. Using an analytical equation, the optimizer then better finds the gradient in which costs decrease and therefore improves the results according to the defined restriction. Thus the distance of the tether to the obstacles is better optimized than the catenary method.

\subsubsection{Distance to obstacles}
Regarding the tether, the parabola approach provides the larger minimal distance from obstacles (Min DO) in all scenarios except S3.1 and S4.1, followed by the catenary and finally \cite{smartinezr2023}. In general terms, this means that the computed tethers with the parabola and catenary are safer in terms of distance to obstacles. With regard to the obstacle distance of the UGV and UAV, the safety values also improve. On average, the minimum values (Min DO) for UGV are 1.79m and 1.71m, for the parabola and catenary approaches, respectively. In the case of UAV, the average for the minimum value (Min DO) is 1.67m and 1.57m. Therefore, both methods behave similarly in terms of distance to obstacle. In general, the Parabola offers slightly better results than the Catenary, but the difference seems to be negligible. 

\subsubsection{Computation Time}
Regarding the computation time (T), we can see that the parabola and catenary approaches are two orders of magnitude faster on average than \cite{smartinezr2023}. This is mainly produced by the analytical parameterization of the tether in the optimizer (Section \ref{sec:implementation}), unlike the use of numerical methods to solve the transcendental catenary equation \cite{BOOKOFCURVES}. We use the results of \cite{smartinezr2023} as reported in the paper. %Notice that those results were obtained with an older computer compared to the present approach, but this cannot justify the large difference in computational time. 
We can also see that the parabola approach is slightly slower than the catenary approach in most scenarios. This delay comes from a larger number of iterations in the parabola solver, resulting in better solutions and with higher feasibility, as previously pointed out. 

\subsubsection{Velocities and Accelerations}
Table \ref{tab:experiments_new} shows the optimized velocities and accelerations (V, A). In general, both methods keep the velocity and acceleration values close to the desired values ($\rho_{vg}$ = 1.0, $\rho_{va}$ = 1.0, $\rho_{ag}$ = 0.0, $\rho_{aa}$ = 0.0).


In summary, the results provided by the new approaches clearly overcome the results presented in \cite{smartinezr2023}. We can see how the new approaches have similar or higher feasibility (F) in all scenarios, together with a clear reduction in computational time (T) of almost two orders of magnitude. This improvement is directly related to the use of analytical equations for the tether (parabola and catenary), which enables the optimizer to perform faster and more accurate gradients. The rest of the parameters, such as distance to obstacles, velocities, and accelerations, are of the same order in the three solutions, but the parabola and catenary improve consistently \cite{smartinezr2023}. 

Among the proposed approaches, Parabola seems to be the best solution, because it provides a clearly higher feasibility in almost all scenarios, while it offers similar capabilities in the rest of the parameters of the planned trajectory. 
