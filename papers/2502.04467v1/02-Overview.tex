The trajectory planning of a tethered UAV-UGV system can be summarized as the process of computing the sequence of obstacle-free UAV-tether-UGV positions, velocities, and accelerations that allow the UAV to reach a destination, given the starting configuration of the UAV-tether-UGV system (see Fig. \ref{fig:planning-setup} as an example). For the sake of generality, this paper focuses on a novel planning approach in which the tether has a variable and controllable length and might be hanging depending on the scenario. We can see in the example of Fig. \ref{fig:planning-setup} that the goal cannot be reached if only taut tether configurations are considered.

Using hanging tether configurations increases the chances of finding a suitable sequence of robot actions to the goal, but it also implies significant computational effort to model the tether shape. The catenary curve accurately models the shape of the tether and is the most common representation. Although fitting the catenary ${\cal C}$ can be easily implemented, the planner must solve this problem hundreds or thousands of times to check possible collisions of the tether with the environment in every UAV-UGV configuration, which becomes a time-consuming process. 

Our novel approach addresses the problem of efficiently computing the existence of a free-collision catenary ${\cal C}_{AB}$ that connects the position of the ground vehicle, $A$, and the position of the aerial vehicle, $B$. Adopting a common terminology, we call $A$ and $B$ the \emph{suspension points}, and the horizontal and vertical distance between $A$ and $B$ are called \emph{span} and \emph{sag}, respectively.

This paper proposes an approach that builds on top of the motion planning method for tethered UGV-UAV system presented in \cite{smartinezr2023}, where the computation of the catenary ${\cal C}$ is required as a representation of the tether's shape at two levels. First, in the path planning stage, they focus on a collision-free path for all agents. Second, in an optimization stage, where the objective is to optimize the previous path to obtain a trajectory, considering several constraints related to the UGV-UAV robotic configuration. 

%We will use as based method  the motion planning for tethered UGV-UAV system considered in paper \cite{smartinezr2023}, where the computing of the states and also the catenary ${\cal C}$ is required in two stages. First, into the path planning stage, where the focus is to ascertain a free-collision path for all agents. Second, in an optimization stage. Here, the objective is to optimize the previous path, considering several constraints related to the tethered UGV-UAV robotic configuration. 

%The next section will get into the details of our novel approach for estimating the tether shape efficiently in each stage. 



%\subsection{Technical notes: projection to 2D}

%The methods described so far for solving the Parabola Decision Problem work in a vertical plane $\Pi$. However, we usually describe the scenario by means of a polygonal mesh or a Point Cloud. Let us define $\Pi$ as the plane that holds:

%\begin{align}
%    \Pi & \parallel  \boldsymbol k \\
%    A,B &\in \Pi \\
%\end{align}
%where $A=\left(A_x,A_y,A_z\right)$ and $B=\left(B_x,B_y,B_z\right)$ are the extreme points of the tether. Finally, we process the 3D point cloud describing the obstacles in the environment, adding the points that are closest than a given distance $D_{min}$ to $\Pi$, as shown in Figure \ref{fig:slice}.

%Furthermore, we can define the unit vector of the 2D plane and the points A' and B' as follows:


%\begin{align}
%    \boldsymbol{i'} & \leftarrow  \boldsymbol i cos\alpha  + \boldsymbol j sin\alpha  \\
%    \boldsymbol{j'}& \leftarrow  \boldsymbol k \\
%    A' & =  O' \\ 
%    B' & =  \boldsymbol{i'} \sqrt{\left(B_x-A_x\right)^2+ \left(B_y-A_y\right)^2} + \boldsymbol{j'} \left(B_z - A_z\right) 
%\end{align}





%Thus, in the remainder of the section, we will focus on finding a catenary in
%the plane $\Pi$ that does not collide with any of the obstacles projected in it. For simplicity we will work in the new 2D coordinate system.

%\subsubsection{Preprocessing step}

%As the aforementioned process of obtaining the projection of the 3D obstacles to the plane $\Pi$ can be computationally demanding, in this paper we propose to sample the planes on the environment in a preprocessing step.

%Let us assume that we would like to get all the planes that point in the direction $\alpha_0$. To this end, we will start from one of the corners of the workspace $O$ and obtaining the first plane in this direction. By repeating the process moving in the normal direction to the plane, we will obtain all the planes on the workspace with direction $\alpha_0$ (see Figure \ref{fig:slice}).

%Then, we repeat the process for the different directions of interest, obtaining a planar representation of the workspace from some 3D model, which can be specified in terms of a Point Cloud or a polygonal mesh.

%\begin{figure}
%  \includegraphics[width=0.48\textwidth]{Figures/slice.png}
%  \caption{3D Obstacles of the environment are represented in changing colors
%    from red to green with increasing $z$. Then, the 2D projection in the vertical plane $\Pi$ is represented in blue.}
%  \label{fig:slice}
%\end{figure}


%----------------------------------------------------------

%Each component encodes a different constraint or optimization objective of our problem and will be presented next. Besides, each component should be evaluated in all the timesteps $i$ of the trajectory. These components are local with respect to $i$, as they only depend on a few number of consecutive states in general. Consequently, our optimization problem can be solved with non-linear sparse optimization algorithms. In particular, we use \emph{Ceres-Solver} \cite{ceres-solver} as our optimization back-end. 
