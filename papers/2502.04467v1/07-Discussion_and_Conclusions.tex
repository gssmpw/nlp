%\section{Discussion}
% Delete the text and write your Discussion here:
%------------------------------------

The paper proposes a novel approach for the efficient parameterization and estimation of the state of a hanging tether for path and trajectory planning of a tethered marsupial robotic system combining an unmanned ground vehicle (UGV) and an unmanned aerial vehicle (UAV). The paper proposes integrating into the trajectory state to optimize the parameters that define the tether curve, and also demonstrates that the parabola curve approximation is a good and efficient representation of the tether.

Experimental results demonstrate that the approximation using a parameterization curve can generate smooth, collision-free trajectories in a fraction of the time taken by the state-of-the-art methods, thus improving the feasibility and efficiency of the obtained solutions. Furthermore, the implementation of the RRT* planning algorithm with the use of a parabola approach based on PDP improves the time calculation in complex three-dimensional environments, such as confined or obstacle-ridden spaces.

We notice that the parabola approximation is an effective alternative for trajectory planning in UAV-UGV systems, by significantly reducing the computation time without compromising the quality of the solutions. This opens new possibilities for the implementation of marsupial robotic systems in missions where real-time trajectory planning is required and in complex three-dimensional environments.

Future work will consider adapting the optimizer to incorporate local planning capabilities, enabling it to modify existing global planner solutions. This integration would provide greater flexibility in adjusting the trajectory as new information or obstacles arise. In addition, incorporating kinematical models of the UGV, UAV, and tether to allow for more realistic constraints, providing a closer representation of the physical capabilities and limitations of the system. 

%You should try to show insight into what happened and why, and how things could have
%gone differently. If you have presented any background theory, try to tie it together with
%your results. How do they relate? If they differ, try to explain why. Even if things didn’t
%work out as intended, a good discussion shows that you’ve understood what went wrong
%and how you could potentially overcome these obstacles. 


%The conclusion should summarise your main results and main points from the discussion.
%A rule of thumb is to not present any new information (information not found in the results or discussion).