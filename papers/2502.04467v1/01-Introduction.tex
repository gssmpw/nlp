In recent years, there has been a notable increase in the development and research of tethered UAVs, reflecting a growing interest in their diverse applications. One of the main motivations is to carry out long-term missions with aerial vehicles, as these present significant challenges due to the limitations of current battery solutions \cite{robotics12040117}. A UAV tethered to a UGV is an interesting configuration, as the UGV can power the UAV through the tether for longer times given the higher payload of the former.  %According to this, an interesting configuration to allow long-duration flights of a UAV is a tethered robot configuration in which a UGV is tied to the UAV and powering it. 
This introduces a paradigm in robotic collaboration, offering distinct advantages over traditional standalone systems by combining the strengths of each of the robotic agents \cite{MooreIROS2018}. %As we venture UGV tied to UAV into scenarios requiring heightened enhanced situational awareness involving an extended operational endurance, the tethered approach proves invaluable, due to the capability to provide energy to the UAV, thus increasing fly time \cite{6961531}. 
When deploying a UGV tethered to a UAV in scenarios requiring increased situational awareness and extended operational endurance, the tethered configuration can become even more invaluable, not only providing the UAV with power to significantly extend its flight time \cite{6961531},  %In this way, the cable plays an important role in providing 
but also with safe high-bandwidth communications \cite{850822,9202196}. 

However, the tethering mechanism introduces several challenges, particularly in modeling the hanging tether state \cite{XiaoSSRR2018}. Unlike standalone systems, where each vehicle operates independently, the tether requires intricate and permanent coordination between the UGV and the UAV. Understanding and managing the state of the tether becomes a critical aspect, which requires sophisticated algorithms and real-time processing capabilities \cite{9561062}. 

\begin{figure}
  \includegraphics[width=0.2\textwidth]{Figures/setup1.png}
  \hfill
  \includegraphics[width=0.2\textwidth]{Figures/setup2.png}
  \caption{Simplified 2D sketch showing an example for motion planning of a tethered UAV-UGV with a hanging tether. (Left) Initial robots and tether configuration, and UAV goal (red circle). (Right) Sequence of robots positions and tether length to reach the given goal. Notice how the goal cannot be reached by means of a taut tether, a hanging tether must be considered in this case.}
  \label{fig:planning-setup}
\end{figure}

The state of the tether has traditionally been analyzed through parameterization, an approach that employs equations to represent its physical behavior, especially the catenary curve \cite{BOOKOFCURVES}. Numerous methodologies, with the aim of simplifying this process, approximate the tether as a straight line \cite{autonomousvisual}\cite{framworktether}\cite{uavfire}. This straight-line approximation is only suitable in scenarios where there is a direct line of sight between the tether endpoints, and thus it inherently restricts the exploratory range of the UAV.

In general, hanging-tether approaches allow UAVs to access a broader range of areas compared to straight tether setups; see Fig. \ref{fig:planning-setup} for an example. This concept has been explored by incorporating tether parameterization into localization or planning processes. For instance, Lima and Pereira \cite{9476778} use the catenary equation to determine the UAV's position.  % This concept has been explored by incorporating tether parameterization into the localization or planning processes, such as in the work conducted by Lima and Pereira \cite{9476778}, where using the catenary equation is feasible to find the UAV position. 
Similarly, in \cite{9364354}, the focus is on computing the state of a catenary tether to localize two UAVs attached at each end. This setup is specifically designed to suspend an object, providing a novel approach to object manipulation using UAVs while maintaining a constant tether length. Another interesting application of the catenary model is presented in \cite{LARANJEIRA2020107018} for underwater operations, where the catenary is used to monitor the status of a cable connected to an \emph{N}-number of ROVs (Remotely Operated Vehicles) performing exploration tasks, also with a constant tether length.

In \cite{8848946}, the parameterization of the tether is used in the localization and control stages to perform two autonomous motion primitives, reactive feedback-based position control and model-predictive feedforward velocity control, but is not used in the planning stage. An interesting approach is presented in \cite{drones7020073}, where a tied unmanned aerial vehicle (TUAV), named ``Oxpecke'', was designed for the inspection of stone-mine pillars. This system uses a sweeping (lawnmower) pattern path planning method intended to map and inspect an entire rectangular area, such as the surface of a pillar. However, the surface to inspect is simple (a rectangle), and the tether length is not directly included in the path planning.

%A general approach about the consideration of the tether in the planning stage is introduced in \cite{battocletti2024entanglementdefinitionstetheredrobots}, where the authors present the definition of tether entanglement problems. Specifically, it addresses the challenges posed by the presence of a tether, including the geometric constraints on the robot's motion due to the finite tether length. For that, different constraints are considered in the planning stage. However, the method is too general and mainly tested in ground points, so UAV implementation are not considering, and algo this method allow tether contact with the floor while entanglement desnt exit.

A comprehensive approach to incorporate a tether in the planning stage is presented in \cite{battocletti2024entanglementdefinitionstetheredrobots}, where the authors define the challenges associated with tether entanglement. Specifically, this work addresses the constraints imposed by the tether on the motion of the robot, particularly the limitations arising from the finite length of the tether. Various constraints are integrated into the planning stage to account for these challenges. However, the proposed method is limited and mainly focused on ground applications, 
thus limiting its applicability to UAVs. Additionally, the approach allows for tether contact with the ground, as long as it does not result in entanglement.

On the other hand, \cite{capitán2024efficientstrategypathplanning} focuses on the development of a path planning strategy for marsupial robotic systems composed of a UGV tethered to a UAV. The article introduces a sequential planning strategy called MASPA (Marsupial Sequential Path-Planning Approach), which allows calculating collision-free 3D trajectories for the tethered UAV-UGV system in complex scenarios, for which the UGV advances to a point where the UAV executes the take-off and then advances to a desired point. This method considers both the geometric limitations imposed by obstacles and the cable and the properties of the joint motion of both robots. A novel algorithm, the PVA (Polygonal Visibility Algorithm), is also presented to identify feasible take-off points and solve visibility problems for the UAV in a three-dimensional space. Despite the novelty of the approach, it is not able to consider coordinated planning of the UGV and the UAV at the same time.

In \cite{smartinezr2023}, the catenary approximation is used to parameterize the state of the tether and plan a collision-free trajectory, in which the UAV must achieve objectives using a hanging tether. However, using the catenary equation, the planning process becomes a time-consuming task, allowing only offline computations. %which makes the planning process to be carried out offline.

%This paper will focus on reducing the complexity associated with the calculation of the variable length hanging tether. We will propose an approach that efficiently calculates the tether state with a minimum representation error concerning the real state, and integrating it into a trajectory planning algorithm for a UGV-UAV tethered team. To this end, we test our approach in the motion planning method for a mobile UGV-UAV tethered system presented in \cite{smartinezr2023}, which is based on two stages. The first stage computes a free-collision path planning for UAV, UGV, and tether, using the RRT* algorithm. The second stage corresponds to a trajectory planning method based on nonlinear optimization that considers smoothness, speed, acceleration limitations of the UGV and UAV, and optimizes the tether configuration to maximize the distance from obstacles. %Unfortunately, considering the real catenary curve in the planner could make it computationally demanding, as shown in our previous work \cite{martinez2021optimization}. In it, we manage to design, implement and test in experiments a two-step optimized planner which considers the catenary shape. For this reason, we propose to approximate the shape of the tether as a parabola without affecting the safety of the planning system and making use of its simpler description to speed-up the computation of optimal paths.

%Our approach is based on the motion planning mentioned above due to the robustness of computed trajectories. Thus, we include in the first stage, a decision problem to set the initial tether length, to quickly obtain a collision-free state for the whole system. Furthermore, we propose a new planner-state parameterization and replace the use of the catenary equation with a parabola equation for estimating the shape of the tether. Thus, the main contributions of the article are:
%\begin{itemize}
%
%\item In Planner Stage: Solving the decision problem to find a collision-free parabola curve instead of the traditional catenary curve. This change allows the RRT* (Rapidly-exploring Random Tree) planner to calculate trajectories faster and more efficiently, since it avoids the computational complexity associated with the calculation of the catenary. The parabolic curve simplifies the collision decision process and increases the success rate in three-dimensional environments with obstacles.
%
%\item  In Optimizer stage: This stage introduces a direct parameterization of the tether in the trajectory state function, which includes the parameters of the curve (parabola or catenary) in the system state vector. This allows a more accurate evaluation of geometric constraints (such as distance to obstacles) and reduces the optimization time by up to an order of magnitude compared to previous methods, achieving safer and smoother trajectories for the UAV-UGV system.
%\end{itemize}

This paper focuses on reducing the complexity associated with the calculation of the variable length hanging tether. %The paper builds on the previous work of the authors \cite{smartinezr2023}, extending it with a new approach that efficiently calculates the tether state with a minimum representation error related to the actual state and a new parameterization of the tether curve in the trajectory optimizer for faster computation. Thus, 
The main contributions are listed below.

\begin{itemize}
    \item A new method for efficient computation of a collision-free catenary curve based on the parabola approximation. This paper proposes using the parabola curve to model the hanging tether curve, detailing the full pipeline, including the computation of the final catenary model. This method reduces the execution time of the path planner to great extent, since it avoids the computational complexity associated with the calculation of the catenary model for tether collision detection. This model also increases the feasibility of the trajectory planner approach, reaching an averaged 98\% of feasibility in the validation scenarios. 

    \item A direct parameterization of the tether in the trajectory state definition, which includes the parameters of the curve (parabola or catenary) in the system state vector. This allows a more accurate evaluation of geometric constraints (such as distance to obstacles) and reduces the optimization time \rev{by more than an order of magnitude} compared to previous methods, achieving safer and smoother trajectories for the UAV-UGV system. \rev{Such improvement opens the door to apply the proposed method to real-time local re-planning.}
\end{itemize}

%The experimental results will show how this new parameterization boosts the computation, while the parabola model will clearly improve the feasibility of the method over the catenary. 

%Thus, we include in the first stage, a decision problem to set the initial tether length, to quickly obtain a collision-free state for the whole system. Additionally, we replace the traditional catenary equation with a parabolic approximation to estimate the tether shape more efficiently. In the second stage of nonlinear optimization stage, we further simplify the process by parameterizing the tether instead of relying on the catenary model. This approach not only streamlines the representation of the curve but also facilitates more straightforward and efficient gradient calculations during optimization.

The paper is structured as follows. In Section \ref{sec:overview}, we show the general problem to be solved, whereas Section \ref{sec:approach} formalizes the solutions proposed. Section \ref{sec:path_planning} details the implementation of the solution within the planning stage. In Section \ref{sec:optimization_process}, we describe how curve parameterization is utilized to enhance the optimization process for trajectory computation. The experimental results are discussed in Section \ref{sec:experiments}. Finally, the paper is concluded in Section \ref{sec:conclusions}.
