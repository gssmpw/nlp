%%%%%%%% ICML 2025 EXAMPLE LATEX SUBMISSION FILE %%%%%%%%%%%%%%%%%

\documentclass{article}

% Recommended, but optional, packages for figures and better typesetting:
\usepackage{microtype}
\usepackage{graphicx}
\usepackage{subfigure}



\usepackage{comment}
\usepackage{enumitem}
\usepackage[table]{xcolor}

% hyperref makes hyperlinks in the resulting PDF.
% If your build breaks (sometimes temporarily if a hyperlink spans a page)
% please comment out the following usepackage line and replace
% \usepackage{icml2025} with \usepackage[nohyperref]{icml2025} above.
\usepackage{hyperref}
\usepackage[title, toc,page]{appendix}
\usepackage{titlesec}
\usepackage{titletoc}
\usepackage{booktabs} % for professional tables


\usepackage{braket}
\usepackage{multirow}
\usepackage{natbib}
\usepackage{graphicx}
\usepackage{subfigure}
\usepackage{makecell}
\usepackage{booktabs}
\usepackage{array}
\usepackage{url}
\usepackage{algorithm}
\usepackage{algorithmic}
\usepackage{dsfont}



% Attempt to make hyperref and algorithmic work together better:
% \newcommand{\theHalgorithm}{\arabic{algorithm}}

% Use the following line for the initial blind version submitted for review:
\usepackage{icml2025}

% If accepted, instead use the following line for the camera-ready submission:
% \usepackage[accepted]{icml2025}

% For theorems and such
\usepackage{amsmath}
\usepackage{amssymb}
\usepackage{mathtools}
\usepackage{amsthm}
\usepackage{amssymb}

\usepackage{amsfonts}
\usepackage{bm}
\usepackage{bbm}

% if you use cleveref..
\usepackage[capitalize,noabbrev]{cleveref}

%%%%%%%%%%%%%%%%%%%%%%%%%%%%%%%%
% THEOREMS
%%%%%%%%%%%%%%%%%%%%%%%%%%%%%%%%
\theoremstyle{plain}
\newtheorem{theorem}{Theorem}[section]
\newtheorem{proposition}[theorem]{Proposition}
\newtheorem{lemma}[theorem]{Lemma}
\newtheorem{corollary}[theorem]{Corollary}
\theoremstyle{definition}
\newtheorem{definition}[theorem]{Definition}
\newtheorem{assumption}[theorem]{Assumption}
\theoremstyle{remark}
\newtheorem{remark}[theorem]{Remark}
\usepackage{titlesec}
\usepackage{booktabs}

% Todonotes is useful during development; simply uncomment the next line
%    and comment out the line below the next line to turn off comments
%\usepackage[disable,textsize=tiny]{todonotes}
\usepackage[textsize=tiny]{todonotes}
\usepackage{nicefrac}


% math commands 
\newcommand{\bfa}[1]{\boldsymbol{#1}}
\newcommand{\lef}{\left}
\newcommand{\rig}{\right}
\newcommand{\norm}[1]{\left\lvert \left\lvert#1\right\rvert \right\rvert}
\newcommand{\vertiii}[1]{{\left\vert\kern-0.25ex\left\vert\kern-0.25ex\left\vert #1 
    \right\vert\kern-0.25ex\right\vert\kern-0.25ex\right\vert}}


%----- bold fonts -----%

\newcommand{\ab}{\mathbf{a}}
\newcommand{\bbb}{\mathbf{b}}
\newcommand{\cbb}{\mathbf{c}}
\newcommand{\db}{\mathbf{d}}
\newcommand{\eb}{\mathbf{e}}
\newcommand{\fb}{\mathbf{f}}
\newcommand{\gb}{\mathbf{g}}
\newcommand{\hb}{\mathbf{h}}
\newcommand{\ib}{\mathbf{i}}
\newcommand{\jb}{\mathbf{j}}
\newcommand{\kb}{\mathbf{k}}
\newcommand{\lb}{\mathbf{l}}
\newcommand{\mb}{\mathbf{m}}
\newcommand{\nbb}{\mathbf{n}}
\newcommand{\ob}{\mathbf{o}}
\newcommand{\pb}{\mathbf{p}}
\newcommand{\qb}{\mathbf{q}}
\newcommand{\rb}{\mathbf{r}}
\newcommand{\sbb}{\mathbf{s}}
\newcommand{\tb}{\mathbf{t}}
\newcommand{\ub}{\mathbf{u}}
\newcommand{\vb}{\mathbf{v}}
\newcommand{\wb}{\mathbf{w}}
\newcommand{\xb}{\mathbf{x}}
\newcommand{\yb}{\mathbf{y}}
\newcommand{\zb}{\mathbf{z}}

\newcommand{\ba}{\bm{a}}
\newcommand{\bb}{\bm{b}}
\newcommand{\bc}{\bm{c}}
\newcommand{\bd}{\bm{d}}
\newcommand{\be}{\bm{e}}
\newcommand{\bbf}{\bm{f}}
\newcommand{\bg}{\bm{g}}
\newcommand{\bh}{\bm{h}}
\newcommand{\bi}{\bmf{i}}
\newcommand{\bj}{\bm{j}}
\newcommand{\bk}{\bm{k}}
\newcommand{\bl}{\bm{l}}
\newcommand{\bbm}{\bm{m}}
\newcommand{\bn}{\bm{n}}
\newcommand{\bo}{\bm{o}}
\newcommand{\bp}{\bm{p}}
\newcommand{\bq}{\bm{q}}
\newcommand{\br}{\bm{r}}
\newcommand{\bs}{\bm{s}}
\newcommand{\bt}{\bm{t}}
\newcommand{\bu}{\bm{u}}
\newcommand{\bv}{\bm{v}}
\newcommand{\bw}{\bm{w}}
\newcommand{\bx}{\bm{x}}
\newcommand{\by}{\bm{y}}
\newcommand{\bz}{\bm{z}}




\newcommand{\Ab}{\mathbf{A}}
\newcommand{\Bb}{\mathbf{B}}
\newcommand{\Cb}{\mathbf{C}}
\newcommand{\Db}{\mathbf{D}}
\newcommand{\Eb}{\mathbf{E}}
\newcommand{\Fb}{\mathbf{F}}
\newcommand{\Gb}{\mathbf{G}}
\newcommand{\Hb}{\mathbf{H}}
\newcommand{\Ib}{\mathbf{I}}
\newcommand{\Jb}{\mathbf{J}}
\newcommand{\Kb}{\mathbf{K}}
\newcommand{\Lb}{\mathbf{L}}
\newcommand{\Mb}{\mathbf{M}}
\newcommand{\Nb}{\mathbf{N}}
\newcommand{\Ob}{\mathbf{O}}
\newcommand{\Pb}{\mathbf{P}}
\newcommand{\Qb}{\mathbf{Q}}
\newcommand{\Rb}{\mathbf{R}}
\newcommand{\Sbb}{\mathbf{S}}
\newcommand{\Tb}{\mathbf{T}}
\newcommand{\Ub}{\mathbf{U}}
\newcommand{\Vb}{\mathbf{V}}
\newcommand{\Wb}{\mathbf{W}}
\newcommand{\Xb}{\mathbf{X}}
\newcommand{\Yb}{\mathbf{Y}}
\newcommand{\Zb}{\mathbf{Z}}

\newcommand{\bA}{\bm{A}}
\newcommand{\bB}{\bm{B}}
\newcommand{\bC}{\bm{C}}
\newcommand{\bD}{\bm{D}}
\newcommand{\bE}{\bm{E}}
\newcommand{\bF}{\bm{F}}
\newcommand{\bG}{\bm{G}}
\newcommand{\bH}{\bm{H}}
\newcommand{\bI}{\bm{I}}
\newcommand{\bJ}{\bm{J}}
\newcommand{\bK}{\bm{K}}
\newcommand{\bL}{\bm{L}}
\newcommand{\bM}{\bm{M}}
\newcommand{\bN}{\bm{N}}
\newcommand{\bO}{\bm{O}}
\newcommand{\bP}{\bm{P}}
\newcommand{\bQ}{\bm{Q}}
\newcommand{\bR}{\bm{R}}
\newcommand{\bS}{\bm{S}}
\newcommand{\bT}{\bm{T}}
\newcommand{\bU}{\bm{U}}
\newcommand{\bV}{\bm{V}}
\newcommand{\bW}{\bm{W}}
\newcommand{\bX}{\bm{X}}
\newcommand{\bY}{\bm{Y}}
\newcommand{\bZ}{\bm{Z}}


%----- calligraphic fonts -----%

\newcommand{\cA}{\mathcal{A}}
\newcommand{\cB}{\mathcal{B}}
\newcommand{\cC}{\mathcal{C}}
\newcommand{\cD}{\mathcal{D}}
\newcommand{\cE}{\mathcal{E}}
\newcommand{\cF}{\mathcal{F}}
\newcommand{\cG}{\mathcal{G}}
\newcommand{\cH}{\mathcal{H}}
\newcommand{\cI}{\mathcal{I}}
\newcommand{\cJ}{\mathcal{J}}
\newcommand{\cK}{\mathcal{K}}
\newcommand{\cL}{\mathcal{L}}
\newcommand{\cM}{\mathcal{M}}
\newcommand{\cN}{\mathcal{N}}
\newcommand{\cO}{\mathcal{O}}
\newcommand{\cP}{\mathcal{P}}
\newcommand{\cQ}{\mathcal{Q}}
\newcommand{\cR}{\mathcal{R}}
\newcommand{\cS}{{\mathcal{S}}}
\newcommand{\cT}{{\mathcal{T}}}
\newcommand{\cU}{\mathcal{U}}
\newcommand{\cV}{\mathcal{V}}
\newcommand{\cW}{\mathcal{W}}
\newcommand{\cX}{\mathcal{X}}
\newcommand{\cY}{\mathcal{Y}}
\newcommand{\cZ}{\mathcal{Z}}




%----- blackboard bold fonts-----%

\newcommand{\CC}{\mathbb{C}}
\newcommand{\EE}{\mathbb{E}}
\newcommand{\VV}{\mathbb{V}}
\newcommand{\II}{\mathbb{I}}
\newcommand{\KK}{\mathbb{K}}
\newcommand{\LL}{\mathbb{L}}
\newcommand{\MM}{\mathbb{M}}
\newcommand{\NN}{\mathbb{N}}
\newcommand{\PP}{\mathbb{P}}
\newcommand{\QQ}{\mathbb{Q}}
\newcommand{\RR}{\mathbb{R}}
\newcommand{\SSS}{\mathbb{S}}
\newcommand{\ZZ}{\mathbb{Z}}
\newcommand{\XX}{\mathbb{X}}
\newcommand{\YY}{\mathbb{Y}}
\newcommand{\OOmega}{\mathbb{\Omega}}




%----- bold greek fonts -----%

\newcommand{\bmeta}{\bm{\eta}}

\newcommand{\balpha}{\bm{\alpha}}
\newcommand{\bbeta}{\bm{\beta}}
\newcommand{\bgamma}{\bm{\gamma}}
\newcommand{\bepsilon}{\bm{\epsilon}}
\newcommand{\bvarepsilon}{\bm{\varepsilon}}
\newcommand{\bzeta}{\bm{\zeta}}
\newcommand{\btheta}{\bm{\theta}}
\newcommand{\bvartheta}{\bm{\vartheta}}
\newcommand{\bkappa}{\bm{\kappa}}
\newcommand{\blambda}{\bm{\lambda}}
\newcommand{\bmu}{\bm{\mu}}
\newcommand{\bnu}{\bm{\nu}}
\newcommand{\bxi}{\bm{\xi}}
\newcommand{\bpi}{\bm{\pi}}
\newcommand{\bvarpi}{\bm{\varpi}}
\newcommand{\brho}{\bm{\varrho}}
\newcommand{\bsigma}{\bm{\sigma}}
\newcommand{\bvarsigma}{\bm{\varsigma}}
\newcommand{\btau}{\bm{\tau}}
\newcommand{\bupsilon}{\bm{\upsilon}}
\newcommand{\bphi}{\bm{\phi}}
\newcommand{\bvarphi}{\bm{\varphi}}
\newcommand{\bchi}{\bm{\chi}}
\newcommand{\bpsi}{\bm{\psi}}
\newcommand{\bomega}{\bm{\omega}}



\newcommand{\R}{\mathbb{R}}
\newcommand{\N}{\mathbb{N}}
\newcommand{\argmin}{\mathop{\mathrm{argmin}}}
\newcommand{\argmax}{\mathop{\mathrm{argmax}}}


\def\CCC{\textcolor{red}}

% The \icmltitle you define below is probably too long as a header.
% Therefore, a short form for the running title is supplied here:
\icmltitlerunning{Transformers and Their Roles as Time Series Foundation Models}



\begin{document}


% spacing things
% \setlist[itemize]{leftmargin=1em, before=\vspace{-0.5em}, after=\vspace{-0.5em}, itemsep=0.1em}
% \setlist[enumerate]{leftmargin=1.4em, before=\vspace{-0.5em}, after=\vspace{-0.5em}, itemsep=0.1em}



\twocolumn[
\icmltitle{Transformers and Their Roles as Time Series Foundation Models
}

% Understanding Transformers as Time Series Foundation Models
% Transformers and Their Role as Time Series Foundation Models
% provable universal forecasting with MOIRAI
% M

% It is OKAY to include author information, even for blind
% submissions: the style file will automatically remove it for you
% unless you've provided the [accepted] option to the icml2025
% package.

% List of affiliations: The first argument should be a (short)
% identifier you will use later to specify author affiliations
% Academic affiliations should list Department, University, City, Region, Country
% Industry affiliations should list Company, City, Region, Country

% You can specify symbols, otherwise they are numbered in order.
% Ideally, you should not use this facility. Affiliations will be numbered
% in order of appearance and this is the preferred way.
\icmlsetsymbol{equal}{*}

\begin{icmlauthorlist}
\icmlauthor{Firstname1 Lastname1}{equal,yyy}
\icmlauthor{Firstname2 Lastname2}{equal,yyy,comp}
\icmlauthor{Firstname3 Lastname3}{comp}
\icmlauthor{Firstname4 Lastname4}{sch}
\icmlauthor{Firstname5 Lastname5}{yyy}
\icmlauthor{Firstname6 Lastname6}{sch,yyy,comp}
\icmlauthor{Firstname7 Lastname7}{comp}
%\icmlauthor{}{sch}
\icmlauthor{Firstname8 Lastname8}{sch}
\icmlauthor{Firstname8 Lastname8}{yyy,comp}
%\icmlauthor{}{sch}
%\icmlauthor{}{sch}
\end{icmlauthorlist}

\icmlaffiliation{yyy}{Department of XXX, University of YYY, Location, Country}
\icmlaffiliation{comp}{Company Name, Location, Country}
\icmlaffiliation{sch}{School of ZZZ, Institute of WWW, Location, Country}

\icmlcorrespondingauthor{Firstname1 Lastname1}{first1.last1@xxx.edu}
\icmlcorrespondingauthor{Firstname2 Lastname2}{first2.last2@www.uk}

% You may provide any keywords that you
% find helpful for describing your paper; these are used to populate
% the "keywords" metadata in the PDF but will not be shown in the document
\icmlkeywords{Machine Learning, ICML}

\vskip 0.3in
]

% this must go after the closing bracket ] following \twocolumn[ ...

% This command actually creates the footnote in the first column
% listing the affiliations and the copyright notice.
% The command takes one argument, which is text to display at the start of the footnote.
% The \icmlEqualContribution command is standard text for equal contribution.
% Remove it (just {}) if you do not need this facility.

%\printAffiliationsAndNotice{}  % leave blank if no need to mention equal contribution
\printAffiliationsAndNotice{\icmlEqualContribution} % otherwise use the standard text.

% %%%%% NEW MATH DEFINITIONS %%%%%

\usepackage{amsmath,amsfonts,bm}
\usepackage{derivative}
% Mark sections of captions for referring to divisions of figures
\newcommand{\figleft}{{\em (Left)}}
\newcommand{\figcenter}{{\em (Center)}}
\newcommand{\figright}{{\em (Right)}}
\newcommand{\figtop}{{\em (Top)}}
\newcommand{\figbottom}{{\em (Bottom)}}
\newcommand{\captiona}{{\em (a)}}
\newcommand{\captionb}{{\em (b)}}
\newcommand{\captionc}{{\em (c)}}
\newcommand{\captiond}{{\em (d)}}

% Highlight a newly defined term
\newcommand{\newterm}[1]{{\bf #1}}

% Derivative d 
\newcommand{\deriv}{{\mathrm{d}}}

% Figure reference, lower-case.
\def\figref#1{figure~\ref{#1}}
% Figure reference, capital. For start of sentence
\def\Figref#1{Figure~\ref{#1}}
\def\twofigref#1#2{figures \ref{#1} and \ref{#2}}
\def\quadfigref#1#2#3#4{figures \ref{#1}, \ref{#2}, \ref{#3} and \ref{#4}}
% Section reference, lower-case.
\def\secref#1{section~\ref{#1}}
% Section reference, capital.
\def\Secref#1{Section~\ref{#1}}
% Reference to two sections.
\def\twosecrefs#1#2{sections \ref{#1} and \ref{#2}}
% Reference to three sections.
\def\secrefs#1#2#3{sections \ref{#1}, \ref{#2} and \ref{#3}}
% Reference to an equation, lower-case.
\def\eqref#1{equation~\ref{#1}}
% Reference to an equation, upper case
\def\Eqref#1{Equation~\ref{#1}}
% A raw reference to an equation---avoid using if possible
\def\plaineqref#1{\ref{#1}}
% Reference to a chapter, lower-case.
\def\chapref#1{chapter~\ref{#1}}
% Reference to an equation, upper case.
\def\Chapref#1{Chapter~\ref{#1}}
% Reference to a range of chapters
\def\rangechapref#1#2{chapters\ref{#1}--\ref{#2}}
% Reference to an algorithm, lower-case.
\def\algref#1{algorithm~\ref{#1}}
% Reference to an algorithm, upper case.
\def\Algref#1{Algorithm~\ref{#1}}
\def\twoalgref#1#2{algorithms \ref{#1} and \ref{#2}}
\def\Twoalgref#1#2{Algorithms \ref{#1} and \ref{#2}}
% Reference to a part, lower case
\def\partref#1{part~\ref{#1}}
% Reference to a part, upper case
\def\Partref#1{Part~\ref{#1}}
\def\twopartref#1#2{parts \ref{#1} and \ref{#2}}

\def\ceil#1{\lceil #1 \rceil}
\def\floor#1{\lfloor #1 \rfloor}
\def\1{\bm{1}}
\newcommand{\train}{\mathcal{D}}
\newcommand{\valid}{\mathcal{D_{\mathrm{valid}}}}
\newcommand{\test}{\mathcal{D_{\mathrm{test}}}}

\def\eps{{\epsilon}}


% Random variables
\def\reta{{\textnormal{$\eta$}}}
\def\ra{{\textnormal{a}}}
\def\rb{{\textnormal{b}}}
\def\rc{{\textnormal{c}}}
\def\rd{{\textnormal{d}}}
\def\re{{\textnormal{e}}}
\def\rf{{\textnormal{f}}}
\def\rg{{\textnormal{g}}}
\def\rh{{\textnormal{h}}}
\def\ri{{\textnormal{i}}}
\def\rj{{\textnormal{j}}}
\def\rk{{\textnormal{k}}}
\def\rl{{\textnormal{l}}}
% rm is already a command, just don't name any random variables m
\def\rn{{\textnormal{n}}}
\def\ro{{\textnormal{o}}}
\def\rp{{\textnormal{p}}}
\def\rq{{\textnormal{q}}}
\def\rr{{\textnormal{r}}}
\def\rs{{\textnormal{s}}}
\def\rt{{\textnormal{t}}}
\def\ru{{\textnormal{u}}}
\def\rv{{\textnormal{v}}}
\def\rw{{\textnormal{w}}}
\def\rx{{\textnormal{x}}}
\def\ry{{\textnormal{y}}}
\def\rz{{\textnormal{z}}}

% Random vectors
\def\rvepsilon{{\mathbf{\epsilon}}}
\def\rvphi{{\mathbf{\phi}}}
\def\rvtheta{{\mathbf{\theta}}}
\def\rva{{\mathbf{a}}}
\def\rvb{{\mathbf{b}}}
\def\rvc{{\mathbf{c}}}
\def\rvd{{\mathbf{d}}}
\def\rve{{\mathbf{e}}}
\def\rvf{{\mathbf{f}}}
\def\rvg{{\mathbf{g}}}
\def\rvh{{\mathbf{h}}}
\def\rvu{{\mathbf{i}}}
\def\rvj{{\mathbf{j}}}
\def\rvk{{\mathbf{k}}}
\def\rvl{{\mathbf{l}}}
\def\rvm{{\mathbf{m}}}
\def\rvn{{\mathbf{n}}}
\def\rvo{{\mathbf{o}}}
\def\rvp{{\mathbf{p}}}
\def\rvq{{\mathbf{q}}}
\def\rvr{{\mathbf{r}}}
\def\rvs{{\mathbf{s}}}
\def\rvt{{\mathbf{t}}}
\def\rvu{{\mathbf{u}}}
\def\rvv{{\mathbf{v}}}
\def\rvw{{\mathbf{w}}}
\def\rvx{{\mathbf{x}}}
\def\rvy{{\mathbf{y}}}
\def\rvz{{\mathbf{z}}}

% Elements of random vectors
\def\erva{{\textnormal{a}}}
\def\ervb{{\textnormal{b}}}
\def\ervc{{\textnormal{c}}}
\def\ervd{{\textnormal{d}}}
\def\erve{{\textnormal{e}}}
\def\ervf{{\textnormal{f}}}
\def\ervg{{\textnormal{g}}}
\def\ervh{{\textnormal{h}}}
\def\ervi{{\textnormal{i}}}
\def\ervj{{\textnormal{j}}}
\def\ervk{{\textnormal{k}}}
\def\ervl{{\textnormal{l}}}
\def\ervm{{\textnormal{m}}}
\def\ervn{{\textnormal{n}}}
\def\ervo{{\textnormal{o}}}
\def\ervp{{\textnormal{p}}}
\def\ervq{{\textnormal{q}}}
\def\ervr{{\textnormal{r}}}
\def\ervs{{\textnormal{s}}}
\def\ervt{{\textnormal{t}}}
\def\ervu{{\textnormal{u}}}
\def\ervv{{\textnormal{v}}}
\def\ervw{{\textnormal{w}}}
\def\ervx{{\textnormal{x}}}
\def\ervy{{\textnormal{y}}}
\def\ervz{{\textnormal{z}}}

% Random matrices
\def\rmA{{\mathbf{A}}}
\def\rmB{{\mathbf{B}}}
\def\rmC{{\mathbf{C}}}
\def\rmD{{\mathbf{D}}}
\def\rmE{{\mathbf{E}}}
\def\rmF{{\mathbf{F}}}
\def\rmG{{\mathbf{G}}}
\def\rmH{{\mathbf{H}}}
\def\rmI{{\mathbf{I}}}
\def\rmJ{{\mathbf{J}}}
\def\rmK{{\mathbf{K}}}
\def\rmL{{\mathbf{L}}}
\def\rmM{{\mathbf{M}}}
\def\rmN{{\mathbf{N}}}
\def\rmO{{\mathbf{O}}}
\def\rmP{{\mathbf{P}}}
\def\rmQ{{\mathbf{Q}}}
\def\rmR{{\mathbf{R}}}
\def\rmS{{\mathbf{S}}}
\def\rmT{{\mathbf{T}}}
\def\rmU{{\mathbf{U}}}
\def\rmV{{\mathbf{V}}}
\def\rmW{{\mathbf{W}}}
\def\rmX{{\mathbf{X}}}
\def\rmY{{\mathbf{Y}}}
\def\rmZ{{\mathbf{Z}}}

% Elements of random matrices
\def\ermA{{\textnormal{A}}}
\def\ermB{{\textnormal{B}}}
\def\ermC{{\textnormal{C}}}
\def\ermD{{\textnormal{D}}}
\def\ermE{{\textnormal{E}}}
\def\ermF{{\textnormal{F}}}
\def\ermG{{\textnormal{G}}}
\def\ermH{{\textnormal{H}}}
\def\ermI{{\textnormal{I}}}
\def\ermJ{{\textnormal{J}}}
\def\ermK{{\textnormal{K}}}
\def\ermL{{\textnormal{L}}}
\def\ermM{{\textnormal{M}}}
\def\ermN{{\textnormal{N}}}
\def\ermO{{\textnormal{O}}}
\def\ermP{{\textnormal{P}}}
\def\ermQ{{\textnormal{Q}}}
\def\ermR{{\textnormal{R}}}
\def\ermS{{\textnormal{S}}}
\def\ermT{{\textnormal{T}}}
\def\ermU{{\textnormal{U}}}
\def\ermV{{\textnormal{V}}}
\def\ermW{{\textnormal{W}}}
\def\ermX{{\textnormal{X}}}
\def\ermY{{\textnormal{Y}}}
\def\ermZ{{\textnormal{Z}}}

% Vectors
\def\vzero{{\bm{0}}}
\def\vone{{\bm{1}}}
\def\vmu{{\bm{\mu}}}
\def\vtheta{{\bm{\theta}}}
\def\vphi{{\bm{\phi}}}
\def\va{{\bm{a}}}
\def\vb{{\bm{b}}}
\def\vc{{\bm{c}}}
\def\vd{{\bm{d}}}
\def\ve{{\bm{e}}}
\def\vf{{\bm{f}}}
\def\vg{{\bm{g}}}
\def\vh{{\bm{h}}}
\def\vi{{\bm{i}}}
\def\vj{{\bm{j}}}
\def\vk{{\bm{k}}}
\def\vl{{\bm{l}}}
\def\vm{{\bm{m}}}
\def\vn{{\bm{n}}}
\def\vo{{\bm{o}}}
\def\vp{{\bm{p}}}
\def\vq{{\bm{q}}}
\def\vr{{\bm{r}}}
\def\vs{{\bm{s}}}
\def\vt{{\bm{t}}}
\def\vu{{\bm{u}}}
\def\vv{{\bm{v}}}
\def\vw{{\bm{w}}}
\def\vx{{\bm{x}}}
\def\vy{{\bm{y}}}
\def\vz{{\bm{z}}}

% Elements of vectors
\def\evalpha{{\alpha}}
\def\evbeta{{\beta}}
\def\evepsilon{{\epsilon}}
\def\evlambda{{\lambda}}
\def\evomega{{\omega}}
\def\evmu{{\mu}}
\def\evpsi{{\psi}}
\def\evsigma{{\sigma}}
\def\evtheta{{\theta}}
\def\eva{{a}}
\def\evb{{b}}
\def\evc{{c}}
\def\evd{{d}}
\def\eve{{e}}
\def\evf{{f}}
\def\evg{{g}}
\def\evh{{h}}
\def\evi{{i}}
\def\evj{{j}}
\def\evk{{k}}
\def\evl{{l}}
\def\evm{{m}}
\def\evn{{n}}
\def\evo{{o}}
\def\evp{{p}}
\def\evq{{q}}
\def\evr{{r}}
\def\evs{{s}}
\def\evt{{t}}
\def\evu{{u}}
\def\evv{{v}}
\def\evw{{w}}
\def\evx{{x}}
\def\evy{{y}}
\def\evz{{z}}

% Matrix
\def\mA{{\bm{A}}}
\def\mB{{\bm{B}}}
\def\mC{{\bm{C}}}
\def\mD{{\bm{D}}}
\def\mE{{\bm{E}}}
\def\mF{{\bm{F}}}
\def\mG{{\bm{G}}}
\def\mH{{\bm{H}}}
\def\mI{{\bm{I}}}
\def\mJ{{\bm{J}}}
\def\mK{{\bm{K}}}
\def\mL{{\bm{L}}}
\def\mM{{\bm{M}}}
\def\mN{{\bm{N}}}
\def\mO{{\bm{O}}}
\def\mP{{\bm{P}}}
\def\mQ{{\bm{Q}}}
\def\mR{{\bm{R}}}
\def\mS{{\bm{S}}}
\def\mT{{\bm{T}}}
\def\mU{{\bm{U}}}
\def\mV{{\bm{V}}}
\def\mW{{\bm{W}}}
\def\mX{{\bm{X}}}
\def\mY{{\bm{Y}}}
\def\mZ{{\bm{Z}}}
\def\mBeta{{\bm{\beta}}}
\def\mPhi{{\bm{\Phi}}}
\def\mLambda{{\bm{\Lambda}}}
\def\mSigma{{\bm{\Sigma}}}

% Tensor
\DeclareMathAlphabet{\mathsfit}{\encodingdefault}{\sfdefault}{m}{sl}
\SetMathAlphabet{\mathsfit}{bold}{\encodingdefault}{\sfdefault}{bx}{n}
\newcommand{\tens}[1]{\bm{\mathsfit{#1}}}
\def\tA{{\tens{A}}}
\def\tB{{\tens{B}}}
\def\tC{{\tens{C}}}
\def\tD{{\tens{D}}}
\def\tE{{\tens{E}}}
\def\tF{{\tens{F}}}
\def\tG{{\tens{G}}}
\def\tH{{\tens{H}}}
\def\tI{{\tens{I}}}
\def\tJ{{\tens{J}}}
\def\tK{{\tens{K}}}
\def\tL{{\tens{L}}}
\def\tM{{\tens{M}}}
\def\tN{{\tens{N}}}
\def\tO{{\tens{O}}}
\def\tP{{\tens{P}}}
\def\tQ{{\tens{Q}}}
\def\tR{{\tens{R}}}
\def\tS{{\tens{S}}}
\def\tT{{\tens{T}}}
\def\tU{{\tens{U}}}
\def\tV{{\tens{V}}}
\def\tW{{\tens{W}}}
\def\tX{{\tens{X}}}
\def\tY{{\tens{Y}}}
\def\tZ{{\tens{Z}}}


% Graph
\def\gA{{\mathcal{A}}}
\def\gB{{\mathcal{B}}}
\def\gC{{\mathcal{C}}}
\def\gD{{\mathcal{D}}}
\def\gE{{\mathcal{E}}}
\def\gF{{\mathcal{F}}}
\def\gG{{\mathcal{G}}}
\def\gH{{\mathcal{H}}}
\def\gI{{\mathcal{I}}}
\def\gJ{{\mathcal{J}}}
\def\gK{{\mathcal{K}}}
\def\gL{{\mathcal{L}}}
\def\gM{{\mathcal{M}}}
\def\gN{{\mathcal{N}}}
\def\gO{{\mathcal{O}}}
\def\gP{{\mathcal{P}}}
\def\gQ{{\mathcal{Q}}}
\def\gR{{\mathcal{R}}}
\def\gS{{\mathcal{S}}}
\def\gT{{\mathcal{T}}}
\def\gU{{\mathcal{U}}}
\def\gV{{\mathcal{V}}}
\def\gW{{\mathcal{W}}}
\def\gX{{\mathcal{X}}}
\def\gY{{\mathcal{Y}}}
\def\gZ{{\mathcal{Z}}}

% Sets
\def\sA{{\mathbb{A}}}
\def\sB{{\mathbb{B}}}
\def\sC{{\mathbb{C}}}
\def\sD{{\mathbb{D}}}
% Don't use a set called E, because this would be the same as our symbol
% for expectation.
\def\sF{{\mathbb{F}}}
\def\sG{{\mathbb{G}}}
\def\sH{{\mathbb{H}}}
\def\sI{{\mathbb{I}}}
\def\sJ{{\mathbb{J}}}
\def\sK{{\mathbb{K}}}
\def\sL{{\mathbb{L}}}
\def\sM{{\mathbb{M}}}
\def\sN{{\mathbb{N}}}
\def\sO{{\mathbb{O}}}
\def\sP{{\mathbb{P}}}
\def\sQ{{\mathbb{Q}}}
\def\sR{{\mathbb{R}}}
\def\sS{{\mathbb{S}}}
\def\sT{{\mathbb{T}}}
\def\sU{{\mathbb{U}}}
\def\sV{{\mathbb{V}}}
\def\sW{{\mathbb{W}}}
\def\sX{{\mathbb{X}}}
\def\sY{{\mathbb{Y}}}
\def\sZ{{\mathbb{Z}}}

% Entries of a matrix
\def\emLambda{{\Lambda}}
\def\emA{{A}}
\def\emB{{B}}
\def\emC{{C}}
\def\emD{{D}}
\def\emE{{E}}
\def\emF{{F}}
\def\emG{{G}}
\def\emH{{H}}
\def\emI{{I}}
\def\emJ{{J}}
\def\emK{{K}}
\def\emL{{L}}
\def\emM{{M}}
\def\emN{{N}}
\def\emO{{O}}
\def\emP{{P}}
\def\emQ{{Q}}
\def\emR{{R}}
\def\emS{{S}}
\def\emT{{T}}
\def\emU{{U}}
\def\emV{{V}}
\def\emW{{W}}
\def\emX{{X}}
\def\emY{{Y}}
\def\emZ{{Z}}
\def\emSigma{{\Sigma}}

% entries of a tensor
% Same font as tensor, without \bm wrapper
\newcommand{\etens}[1]{\mathsfit{#1}}
\def\etLambda{{\etens{\Lambda}}}
\def\etA{{\etens{A}}}
\def\etB{{\etens{B}}}
\def\etC{{\etens{C}}}
\def\etD{{\etens{D}}}
\def\etE{{\etens{E}}}
\def\etF{{\etens{F}}}
\def\etG{{\etens{G}}}
\def\etH{{\etens{H}}}
\def\etI{{\etens{I}}}
\def\etJ{{\etens{J}}}
\def\etK{{\etens{K}}}
\def\etL{{\etens{L}}}
\def\etM{{\etens{M}}}
\def\etN{{\etens{N}}}
\def\etO{{\etens{O}}}
\def\etP{{\etens{P}}}
\def\etQ{{\etens{Q}}}
\def\etR{{\etens{R}}}
\def\etS{{\etens{S}}}
\def\etT{{\etens{T}}}
\def\etU{{\etens{U}}}
\def\etV{{\etens{V}}}
\def\etW{{\etens{W}}}
\def\etX{{\etens{X}}}
\def\etY{{\etens{Y}}}
\def\etZ{{\etens{Z}}}

% The true underlying data generating distribution
\newcommand{\pdata}{p_{\rm{data}}}
\newcommand{\ptarget}{p_{\rm{target}}}
\newcommand{\pprior}{p_{\rm{prior}}}
\newcommand{\pbase}{p_{\rm{base}}}
\newcommand{\pref}{p_{\rm{ref}}}

% The empirical distribution defined by the training set
\newcommand{\ptrain}{\hat{p}_{\rm{data}}}
\newcommand{\Ptrain}{\hat{P}_{\rm{data}}}
% The model distribution
\newcommand{\pmodel}{p_{\rm{model}}}
\newcommand{\Pmodel}{P_{\rm{model}}}
\newcommand{\ptildemodel}{\tilde{p}_{\rm{model}}}
% Stochastic autoencoder distributions
\newcommand{\pencode}{p_{\rm{encoder}}}
\newcommand{\pdecode}{p_{\rm{decoder}}}
\newcommand{\precons}{p_{\rm{reconstruct}}}

\newcommand{\laplace}{\mathrm{Laplace}} % Laplace distribution

\newcommand{\E}{\mathbb{E}}
\newcommand{\Ls}{\mathcal{L}}
\newcommand{\R}{\mathbb{R}}
\newcommand{\emp}{\tilde{p}}
\newcommand{\lr}{\alpha}
\newcommand{\reg}{\lambda}
\newcommand{\rect}{\mathrm{rectifier}}
\newcommand{\softmax}{\mathrm{softmax}}
\newcommand{\sigmoid}{\sigma}
\newcommand{\softplus}{\zeta}
\newcommand{\KL}{D_{\mathrm{KL}}}
\newcommand{\Var}{\mathrm{Var}}
\newcommand{\standarderror}{\mathrm{SE}}
\newcommand{\Cov}{\mathrm{Cov}}
% Wolfram Mathworld says $L^2$ is for function spaces and $\ell^2$ is for vectors
% But then they seem to use $L^2$ for vectors throughout the site, and so does
% wikipedia.
\newcommand{\normlzero}{L^0}
\newcommand{\normlone}{L^1}
\newcommand{\normltwo}{L^2}
\newcommand{\normlp}{L^p}
\newcommand{\normmax}{L^\infty}

\newcommand{\parents}{Pa} % See usage in notation.tex. Chosen to match Daphne's book.

\DeclareMathOperator*{\argmax}{arg\,max}
\DeclareMathOperator*{\argmin}{arg\,min}

\DeclareMathOperator{\sign}{sign}
\DeclareMathOperator{\Tr}{Tr}
\let\ab\allowbreak


\begin{abstract}
% We study the approximation and generalization of MOIRAI \cite{woo2024unified}, the first time series foundation model that achieves \emph{universal forecasting}.
% MOIRAI is a transformer-based model which stands out for its ability to handle any arbitrary number of covariates and to generalize effectively across unseen domains. 
% For approximation, we show that there exists a MOIRAI transformer that performs least square regression on any multi-variate time series data via in-context learning.
% Notably, the unique encoding method and attention layer in MOIRAI allows it to process multiple covariates efficiently.
% Our approximation results are not only the first that effectively handles any arbitrary number of covariates, but are also highly compatible with existing in-context learning theories.
% For generalization, when the data is sampled from distributions under Dobrushin's condition, we present the generalization bound for both single and multi-source pretraining.
% The empirical results validate our theories, highlighting MOIRAI's efficacy as an universal forecaster.


We give a comprehensive analysis of transformers as time series foundation models, focusing on their approximation and generalization capabilities. 
First, we demonstrate that there exist transformers that fit an autoregressive model on input univariate time series via gradient descent. 
We then analyze MOIRAI \cite{woo2024unified}, a multivariate time series foundation model capable of handling an arbitrary number of covariates. 
We prove that it is capable of automatically fitting autoregressive models with an arbitrary number of covariates, offering insights into its design and empirical success.   
For generalization, we establish bounds for pretraining when the data satisfies Dobrushin’s condition. 
Experiments support our theoretical findings, highlighting the efficacy of transformers as time series foundation models.

% For approximation, we show that there exists a MOIRAI transformer that performs least square regression on any multi-variate time series data via in-context learning.
% we show that there exists a MOIRAI transformer that achieves \emph{one-model-for-all-contexts}, where it learns the pattern of any time series in-context and makes accurate predictions.
% Notably, such an property is achieved by the unique encoding method and attention layer in MOIRAI, allowing it to process multiple covariates efficiently.
% For generalization, when the data is sampled from distributions under Dobrushin's condition, we present the generalization bound for both single and multi-source pretraining.
% The empirical results validate our theories, highlighting MOIRAI's efficacy as an universal forecaster.
\end{abstract}


% \vspace{-1em}
\section{Introduction}
% \CCC{We study the approximation and generalization of transformers as universal forecasters. (Perhaps we do not need a first sentence like this? we can just start with saying how important MOIRAI is.)} 
% A recent study \cite{woo2024unified} propose MOIRAI, the first universal forecaster, i.e., a large multi-variate time series foundation model.
% MOIRAI achieves the \emph{one-model-for-all-dataset} schema by solving two key challenges of building a time series foundation model:
% (a) It must be compatible with any arbitrary number of covariates;
% (b) It must be able to generalize well on unseen domains.
% MOIRAI not only addresses the above mentioned setbacks effectively, but also show impressive zero-shot performance on unseen domains, leading to its universality.
% Despite its empirical success, how does MOIRAI's XXX

The advancement of foundation models is reshaping the field of time series forecasting.
Recent studies demonstrate the empirical success of transformer-based time series foundation models \cite{woo2024unified, ansari2024chronos, liang2024foundation, das2023decoder}.
However, a theoretical understanding of how these models succeed is yet missing.
In this paper, we aim to provide a comprehensive analysis of time series foundation models, with a focus on transformer-based models.
We are interested in how these models achieve the \emph{one-model-for-all-datasets} paradigm in time series forecasting.
Specifically, our results cover both uni-variate \cite{ansari2024chronos, das2023decoder, rasul2023lag}, and multi-variate time series foundation models \cite{woo2024unified}\footnote{The model proposed by \cite{woo2024unified} is compatible with an arbitrary number of covariates}.


Our main discovery is twofold.
First, to address universality, we prove that there exists a transformer that fits an autoregressive model \cite{hamilton2020time, mills1990time} on any given uni-variate time series.
% Further, we show that the special design of MOIRAI allows transformers to further handle any arbitrary number of covariates, making it true universal for all multi-variate time series.
Furthermore, we show that the special design of MOIRAI allows transformers to further handle arbitrary number of covariates, making it process any dimension of time series in a principled way.
Second, to address learnability, we establish a generalization bound for pretraining when the data satisfies Dobrushin's condition \cite{Dobrushin1968TheDO, dobrushin1987completely}.
We refer to these two aspects as approximation and generalization, respectively, throughout the rest of this paper, as they form the theoretical foundation for the success of these models.

% To line up with practical scenarios, our theoretical results take MOIRAI's unique attention mechanism and encoding methods into considerations.
% Specifically, we show that those features of MOIRAI allows it to process any arbitrary number of covariates in a principle way.
% This discovery provides an explanation on MOIRAI's predictive power across different time series, justifying the design of MOIRAI.

Our approximation result is inspired by recent studies on in-context learning \cite{bai2024transformers, von2023transformers, li2023transformers, mahankali2023one, ahn2024transformers, zhang2024trained}. 
In-context learning refers to the ability of large foundation models to make accurate predictions for unseen tasks by observing training examples without parameter updates \cite{brown2020language, garg2022can}.
This capability explains how time series foundation models achieve their universality.
By treating the input time series as in-context examples, we show that transformers are able to implement gradient descent to estimate the parameters of the autoregressive model that best explains the given input.


% For generalization, we assume the model is pretrained on a large number of different time series, where each of them satisfies Dobrushin's condition.
% Next, when the data is generated by some autoregressive $\mathtt{AR}$ processes, we are able to further obtain a sharper rate, which reveals the trade-off between model complexity and training error.
% Finally, we present an application of our theorem on $\mathtt{AR}(1)$ process.
% To the best of the authors knowledge, our generalization bound is the first on time series foundation model pretraining.

% In our work, we delve into the theoretical and practical underpinnings of this universal forecastability by examining the core building blocks of MOIRAI. Specifically, we show the existence of a single model that can perform least-square regression across any time series dataset, and we establish pretraining-based generalization bounds to quantify its performance on unseen domains. This investigation clarifies how MOIRAI can guarantee consistent predictive power despite widely varying sources of data and domain-specific nuances. By presenting both a rigorous foundation and empirical insights, our paper sheds light on MOIRAI’s unique ability to unify diverse time series tasks under a single, versatile architecture, offering a pathway for truly universal time series forecasting

% {\color{red}talk about anyvariate encoding, we aim to understand moirai...}

% Our approximation results are inspired by recent studies on in-context learning \cite{bai2024transformers, von2023transformers, li2023transformers, mahankali2023one, ahn2024transformers, zhang2024trained}. 
% In-context learning refers to the ability of large foundation models to make accurate predictions for unseen tasks by observing training examples without parameter updates \cite{brown2020language, garg2022can}. 
% This capability explains MOIRAI's zero-shot performance on unseen domains. 
% By treating the input time series as in-context examples, MOIRAI can infer underlying patterns and predict the next time step. 
% However, existing in-context learning theories fail to fully explain this due to two key challenges: (a) \CCC{difficulty in handling inputs concatenated into a single dimension, and (b) inability to process an arbitrary number of covariates.} \CCC{(these two challenges are not very clear. Perhaps we can write a short paragraph to explain why existing works cannot explain the success of MOIRAI?)}


% Most in-context learning theories rely on the assumption that the input feature and label and formatted into the same column.
% However, MOIRAI's encodes a multi-variate time series into a single long row vector, preventing XX.
% Next, icl theories only focus on a fixed feature dimension.
% Comparing to moirai, the data dimension can be any arbitrary number.


The contribution of this paper is threefold:
\begin{itemize}
    \item 
    From an algorithmic approximation perspective, we prove the existence of a transformer capable of fitting an autoregressive ($\mathtt{AR}$) model on any given uni-variate time series via gradient descent. 
    Extending this to the multi-variate setting, we show that a MOIRAI transformer can automatically adjust the dimensionality of the $\mathtt{AR}$ model to fit time series with an arbitrary number of covariates. 
    Our approximation results not only explain the strong performance of modern models across diverse datasets but also justify the design of MOIRAI.

    \item
    We present the first pretraining generalization bound for time series foundation models.
    We show that when the pretraining data satisfies Dobrushin's condition, the test error can be effectively bounded even when the data does not satisfy the i.i.d. assumption.
    Specifically, when pretraining MOIRAI on $n$ multi-variate time series, the test error decays by a factor of $\nicefrac{1}{\sqrt{n}}$.

    \item 
    Our experimental results match our theories by showing that the prediction error of transformers reduces as the input time series length increases, corresponding to our approximation result.
    % We also empirically show the trade-off between model complexity and test error, corresponding to our generalization bounds.
    
\end{itemize}



% \CCC{The following two paragraphs can be revised into the contribution bullets? }

% \CCC{Is it possible to move the current two paragraphs to contributions, and add a new paragraph before contributions to discuss the goal of the paper, and informally describe the problem setting we consider, highlighting how the problem setting captures the nature of the important tasks of ``process
% an arbitrary number of covariates'', etc? We also need to introduce the concept of ``approximation'' and ``generalization''}

% We demonstrate that MOIRAI's unique encoding and attention mechanisms effectively address key challenges (a) and (b), offering advantages over standard transformers.
% Specifically, we show that MOIRAI can perform variate-wise operations regardless of the number or length of covariates. 
% This advantage allows MOIRAI to perform least square regression on any multi-variate time series.
% Furthermore, when data is generated by autoregressive processes ($\mathtt{AR}$), we show that MOIRAI is capable of estimating the underlying parameter of any multi-variate $\mathtt{AR}$ processes.
% Notably, compared to existing theoretical results \cite{bai2024transformers} on in-context learning, our approximation rate differs by at best a constant factor in the required number of layers.

% For generalization, we demonstrate the learning bound for MOIRAI pretraining.
% When pretraining MOIRAI on a large volume of multi-variate time series satisfying the Dobrushin's condition \cite{Dobrushin1968TheDO}, we are able to obtain a generalization bound.
% Next, our generalization bounds only degrade by no more than log factors in the length of time series comparing to learning a standard transformer \cite{bai2024transformers}.
% Finally, we provide an example on pretraining a MOIRAI transformer on data generated by $\mathtt{AR}$ process with lag size $1$, where with appropriate setup, the test loss is bounded by factor of {\color{red}XXX}.


% The contribution of this paper is threefold
% \begin{itemize}
%     \item 
%     We are the first to provide a comprehensive analysis on large foundation models as universal forecasting.
%     We show that there exists a MOIRAI that learns the underlying pattern of a time series in-context, providing solid explanation on their empirical performance.

%     \item 
%     Our theoretical results gives profound insights to practice as we study the exact setup of moirai.

%     \item 
%     Our generalization bound for pretraining ...

% \end{itemize}


\paragraph{Organization.}
The organization of this paper is as follows:
Section 2 describes the problem setup; Section 3 provides the approximation result; Section 4 analyzes the generalization bound for pretraining; Section 5 reports the numerical simulations; and we leave discussions to Section 6.
\paragraph{Notations.}
We use the following notation conventions. 
The vector-valued variable is given by boldfaced characters. 
We denote $[n]:=\{1,\ldots,n\}$ and $[i:j]:=\{i,i+1,\ldots, j\}$ for $i<j$. 
The universal constants are given by $C$ and are ad hoc. 
Considering a sequence of vectors $(\bx_1, \cdots, \bx_T)$, we use $\bx$ without index to represent the whole sequence, and $\bx_{i:j}$ represents $(\bx_i, \cdots, \bx_j)$ for $i < j$.
We impose periodic boundary conditions for the negative index, i.e., $\bx_{-1} = \bx_{T}$.
For a vector $\bfa v$ we denote $\Vert\bfa v\Vert_2$ as its $L_2$ norm. 
For a matrix $\bfa A\in  \R^{m\times n}$ we denote its operator norm as $\Vert\bfa A\Vert_2:=\sup_{\bfa v\in \mathbb{S}^{n-1}}\Vert\bfa A\bfa v\Vert_2$. 
Random variables are given by calligraphic characters $\cX$, and elements from a domain set are given by normal font $\text{x}$.
For more details, see Table~\ref{tab:nomenclature}.

% \vspace{-0.5em}
\section{Problem Setup}
This section describes our problem setup.
We introduce the architecture of transformer-based time series foundation models and how we construct our datasets.
% \vspace{-0.5em}
\subsection{Transformers}
We consider a sequence of $N$ input vectors $\{ h_i \}_{i=1}^N \subset \R^D$, where $ \bfa H \coloneqq \left[ h_1, \cdots, h_N \right] \in \R^{D \times N}$.
Given any $ \bfa H \in \R^{D \times N}$, we define the attention layer as follows.
\begin{definition}[Attention layer\normalfont]\label{def:attn}
    {\normalfont
    A self-attention layer with $M$ heads is denoted as $\text{Attn}^{\dagger}_{\bm{\theta}_0}(\cdot)$ with parameters $\bm{\theta}_0 = \{ (\bfa V_m), (\bfa Q_m), (\bfa K_m)  \}_{m \in [M]} \subset \R^{D \times D}$.
    The self-attention layer processes any given input sequence $\bfa H \in \R^{D \times N}$ as
    }
    \begin{align*}
    \text{Attn}_{\bm{\theta}_0}^{\dagger} 
    &\left( \bfa H \right)
    \coloneqq
    \bfa H + 
    \frac{1}{N}
    \sum_{m=1}^M
    (\bfa V_m \bfa H)
    \cdot
   % \CCC{\times} 
    \\
    &\sigma 
    \left( 
    \left( \bfa Q_m \bfa H \right)^\top 
    \left(\bfa K_m \bfa H \right) 
    \right),  
    % \hfill{\text{\R^{D \times N}}}
\end{align*}
where $\sigma \coloneqq t \mapsto \text{ReLU}(t)/N$.
% \CCC{We should not use times for matrix product. times is for outer product.}
\end{definition}
\paragraph{Any-variate Attention.}
Next, we introduce the any-variate attention, where \cite{woo2024unified} uses it to replace the standard attention in transformers.
The any-variate attention introduces two learnable variables: Attention Bias $u_1, u_2 \in \R$,
for disambiguation between variates.
\begin{definition}[Any-variate Attention.]\label{def:any-variate-attn}
    An any-variate attention layer with $M$ heads is denoted as $\text{Attn}_{\bm{\theta}_1}(\cdot)$ with parameters $\bm{\theta}_{1} = \{ (\bfa V_m), (\bfa Q_m), (\bfa K_m), (u_m^1), (u_m^2)  \}_{m \in [M]}$.
    With any input $H \in \R^{D \times N}$, we have
    \begin{align*}
    \text{Attn}_{\bm{\theta}_1} &\left( \bfa H \right)
    \coloneqq
    \bfa H + 
    \frac{1}{N}
    \sum_{m=1}^M
    (\bfa V_m \bfa H)
    \times
    \\
    &\sigma 
    \left( 
    \left( \bfa Q_m \bfa H \right)^\top 
    \left(\bfa K_m \bfa H \right) 
    +
    u_m^1 * \bU
    +
    u_m^2 * \bar{\bU}
    \right),  
    % \hfill{\text{\R^{D \times N}}}
\end{align*}
where $\sigma \coloneqq t \mapsto \text{ReLU}(t)/N$, $\bU \in \R^{N \times N}$ is a block diagonal matrix with block size $T\in \N^+$, such that each block consists of $1$s, $\bar{\bU} = \bI - \bU$, and $*$ denotes a constant multiply to all entries of a matrix.
\end{definition}

\begin{remark}
    In \cite{woo2024unified}, the attention score is calculated with the RoPE embedding \cite{su2024roformer}:
    \[
    \sigma 
    \left( 
    \left( \bfa Q_m \bfa H \right)^\top 
    \bfa R
    \left(\bfa K_m \bfa H \right) 
    +
    u_m^1 * \bU
    +
    u_m^2 * \bar{\bU}
    \right).  
    \]
    We omit the notation of rotary matrix $\bfa R$ as it is not learnable and is invertible and thus merged into $\bfa Q, \bfa K$ in our analysis.
\end{remark}

\begin{definition}[MLP Layer\normalfont]
    {\normalfont
    We denote an MLP layer with hidden state dimension $D^\prime$ as $\text{MLP}_{\bm{\theta}}(\cdot)$ with parameters $\bm{\theta}_2 = ( \bW_1, \bW_2 ) \in \R^{ D^\prime \times D } \times \R^{D \times D^\prime}$.
    The MLP layer processes any given input sequence $\bfa H \in \R^{D \times N}$ as
    }
    \begin{equation*}
        \text{MLP}_{\bm{\theta}_2} (\bfa H)
        \coloneqq
        \bfa H +
        \bW_2 \sigma(\bW_1 \bfa H).
    \end{equation*}
\end{definition}
Finally, we define a transformer with $L \geq 1$ layers, each consisting of any-variate attention and an MLP layer.
\begin{definition}[MOIRAI Transformer\normalfont]\label{def:moirai}
    {
    \normalfont
    We define the $L$-layer MOIRAI transformer \cite{woo2024unified}, $\text{TF}_{\bm{\theta}}(\cdot)$, as
    }
    \begin{equation*}
        \text{TF}_{\bm{\theta}}(\bfa H)
        =
        \text{MLP}_{\bm{\theta}_2^L}
        \left(
        \text{Attn}_{\bm{\theta}_1^L}
        \left(
        \cdot \cdot
        \text{MLP}_{\bm{\theta}_2^1}
        \left(
        \text{Attn}_{\bm{\theta}_1^1}
        (\bfa H)
        \right)
        \right)
        \right)
        .
    \end{equation*}
    Note that this transformer is equipped with any-variate attention instead of the standard attention.
    For transformers with standard attention, we denote it as 
    $\text{TF}_{\bm{\theta}}^{\dagger}(\cdot)$.
\end{definition}

We use $\bm{\theta}$ to denote the vectorization of all parameters in a transformer and super-index $\ell$ to denote the parameter of the $\ell$-th layer.
Thus, the parameter of a transformer is defined by
\begin{equation*}
        \bfa \theta = 
        \left\{
        \left\{
        \left(\{\bfa Q_m^\ell,\bfa K_m^\ell,\bfa V_m^\ell, u_m^{1, \ell}, u_m^{2, \ell}\}_{m\in[M]}, \bfa W_{1}^\ell,\bfa W_{2}^\ell
        \right)
        \right\}_{\ell\in[L]}
        \right\}.
\end{equation*}    

We denote the ``attention-only" transformers with $\bW_1^{(\ell)}, \bW_2^{(\ell)} = 0$, as $\text{TF}_{\bm{\theta}}^0(\cdot)$ for shorthand.
We define the following norm of a MOIRAI transformer as
\begin{align*}
    \Vert\bfa\theta\Vert_{op}
    \coloneqq
    \max_{\ell\in[L]}
    &\Big\{\max_{m\in[M^{\ell}]}
    \Big\{\Vert \bfa Q_m^\ell\Vert_2,\Vert\bfa K_m^\ell\Vert_2,
        \\
        \vert u_m^{1\ell} \vert , \vert u_m^{2\ell} \vert 
    \Big\}
    &+
    \sum_{m=1}^{M^{\ell}}\Vert\bfa V_m^\ell\Vert_2+\Vert\bfa W_1^{\ell}\Vert_2+\Vert\bfa W_2^{\ell}\Vert_2 \Big\}, 
\end{align*}
where $M^{\ell}$ is the number of heads of the $\ell$-th Attention layer.


\subsection{Data Generation}
Here, we first consider the case where we aim to find a multi-layered transformer that performs least squares regression via In-context learning (ICL).
Specifically, we assume our data is generated from an autoregressive process $\mathtt{AR}_d(q)$ as follows, where $q, d$ denotes the steps of lag and number of covariates, respectively. 
Consider a sequence of data 
$\bx \in \R^{d \times T} \coloneqq (\bx_1, \dots, \bx_T)$, where $\bx_t = (x_t^1, \cdots, x_t^d) \in \R^d$.
Assuming our target (variate of interest) is in dimension $1$, we assume the $\mathtt{AR}_d(q)$ process generates $x_t^1$ as follows:
\begin{equation}\label{eqn:AR-data}
    x_{t}^1
    =
    \sum_{i=1}^q
    \sum_{j=1}^d
    a_i^j \cdot x_{t-i}^j
    + \epsilon_t
    =
    \sum_{j=1}^d
    \langle \bw^j , \bx_{t-q: t-1}^j
    \rangle
    +
    \epsilon_t
    ,
\end{equation}
where $\epsilon_t \sim N(0, 1)$, $a_i^j \in \R^1$.
We denote the concatenation of all weights $\bw^\star = (\bw_1, \cdots, \bw^j)\in \R^{qd}$.
We assume bounded features $\norm{\bx_{t-q:t-1}}_2 \leq B_x$ , for all $t = 1,\cdots, T$.
The first equation writes the $\mathtt{AR}$ process in scalar form, and the second writes it in vector form.
In the following chapters, we will start by considering the uni-variate case ($\mathtt{AR}_1(q)$) and then move on to the multi-variate case ($\mathtt{AR}_d(q)$).
\vspace{-1.5em}
\paragraph{Problem Setup.}
Given a function class $\cF: \R^{d\times T}\mapsto \R$, our goal is to find a universal function $f\in\cF$ such that, given any time series generated from any arbitrary $\mathtt{AR}_d(q)$, its prediction error is bounded by some $\varepsilon \geq 0$, i.e.,
\[
\norm{f( \Tilde{\bx}) - x_{T}^1}_2
\leq
\varepsilon,
\]
where $\Tilde{\bx}$ denotes the time series $\bx$ with $x_T^1$ being masked.
\begin{remark}
    In the appendix, we show that even when the autoregressive process follows some non-linear relationship, there still exists a universal $f$ that predicts all non-linear AR process accurately.
\end{remark}
% \paragraph{Multivariate Autoregressive Process.}
% We extend the definition and notation for the \texttt{AR($q$)} process to the multivariate AR process, denoted as \texttt{MAR($q$)}.
% Consider a sequence of data $\bx \in \R^{d \times T} \coloneqq (x_1, \dots, x_T)$, we set the first dimension of $\bx$: $x_t^1$, for $t = 1, \cdots, T$ as target, and the other $d-1$ dimensions of $\bx$ as its covariates.
% We say $\bx$ is generated by a \texttt{MAR($q$)} process if the target of the sequence is generated by the following 
% \begin{equation}\label{eqn:MAR-data}
%     x_{t+1}^1
%     =
%     \sum_{i=1}^q
%     \sum_{j=1}^d
%     \langle
%     \bw_i^\star, x_i^j
%     \rangle
%     +
%     \epsilon_i,
% \end{equation}
% where $\bw_i^\star \in \R^d$ and $\epsilon_i \sim N(0, 1)$, for $i = 1, \cdots , q$,.
% Here we slightly abuse the notation in the \texttt{AR} process of $\bw^\star$, as we will be focusing on \texttt{MAR}($q$) in the rest of the paper.

\section{Connecting Transformers with EM}\label{sect2}
This section discusses the connections between the EM algorithm and the Transformer architecture. 
Our discussion is split into $2$ separate subsections: In \ref{Transformers}, we review the mathematical definitions of the Softmax-based Transformer model; In \ref{emintro}, we review the EM algorithm and connect it with the multiphase Transformer design. In section \ref{pretrain}, we discuss the pretraining procedure of the Transformers.
\subsection{The Transformer Architecture}\label{Transformers}
We consider the Softmax Attention Layer, which is defined as follows:
\begin{definition}[Softmax Attention]\label{def:attention}
  The Softmax Attention layer is defined as a self-attention layer with $M$ heads denoted as $Attn_{\bfa\theta_1}(\cdot)$ with parameters $\bfa\theta_1=\{(\bfa V_m,\bfa Q_m,\bfa K_m)\}_{m\in[M]}\subset\bb R^{D\times D}$. On input sequence $\bfa H\in\bb R^{D\times N}$,
    {\begin{align*}
       Attn_{\bfa\theta_1}&(\bfa H)=
            \bfa H\\
            &+\sum_{m=1}^M(\bfa V_m\bfa H)\softmax\Big((\bfa Q_m\bfa H)^\top(\bfa K_m\bfa H)\Big),
    \end{align*}}
    where $\softmax$ is the activation function defined by
    \begin{align*}
        \softmax(\bfa x) = \begin{bmatrix}
            \frac{\exp(x_1)}{\sum_{i=1}^d\exp(x_i)}&\ldots&\frac{\exp(x_d)}{\sum_{i=1}^d\exp(x_d)}
        \end{bmatrix}^\top,
    \end{align*}
    for all $\bfa x\in\bb R^d$.
\end{definition}
In addition to the Softmax Attention layer, we also consider an un-normalized Attention layer, given by
\begin{definition}[Un-normalized Attention]\label{def:mattention}
  The un-normalized Attention layer is defined as a self-attention layer with $M$ heads and denoted as $nAttn_{\bfa\theta_1}(\cdot)$ with parameters $\bfa\theta_1=\{(\bfa V_m,\bfa Q_m,\bfa K_m)\}_{m\in[M]}\subset\bb R^{D\times D}$. On input sequence $\bfa H\in\bb R^{D\times N}$,
    {\begin{align*}
       nAttn_{\bfa\theta_1}&(\bfa H)=
            \bfa H+\sum_{m=1}^M(\bfa V_m\bfa H)(\bfa Q_m\bfa H)^\top(\bfa K_m\bfa H).
    \end{align*}}
\end{definition}
\begin{remark}
    The un-normalized Attention layer is the Attention layer without the non-linear activation function. This layer is studied mainly for technical reasons. We also provide results not using the un-normalized Attention layer, despite having weaker rates. 
\end{remark}
% \begin{remark}
%     We note that in the literature \citep{bai2024transformers} only considers the approximation guarantees yield by the ReLU Transformers. However, existing LLMs predominantely utilize the Softmax-based Attention layers. Our approximation guarantee for the Softmax Attention layers is new in the literature, relying on our new approximation result for the Softmax function, discussed in section \ref{sect31}. 

%     We also replace concatenation of multihead attentions by their average. Despite adaptations are employed for the technical convinience, in the simulation section we carefully evaluate the effect of these modifications on the performance of the model.
% \end{remark}
The following defines the classical Fully-Connected (FC) layers with residual connections.
\begin{definition}[FC Layer] A FC layer with hidden dimension $D^\prime$ is denoted as $FC_{\bfa\theta}(\cdot)$ with parameter $\bfa\theta_2\in(\bfa W_1,\bfa W_2)\in\bb R^{D^\prime\times D}\times\bb R^{D\times D^\prime}$. On any input sequence $\bfa H\in\bb R^{D\times N}$, we define
$$   FC_{\bfa\theta_2}(\bfa H):=\bfa H+\bfa W_2\sigma(\bfa W_1\bfa H).$$
\end{definition}
Then, we use the above definitions on the FC and the Attn/nAttn layers to define the Transformer model and the Transformer+ model.
\begin{definition}[Transformer]\label{transform}
    We define the Transformer $TF_{\bfa\theta}(\cdot)$ as a composition of the self-attention layers with the FC layers. Consider the output dimension to be $\tilde D$, a $L$-layered Transformer is defined by  \sm{\begin{align*}
        TF_{\bfa\theta}&(\bfa H) :=\\
        &\tda W_0\times FC_{\bfa\theta_{2}^L}(Attn_{\bfa\theta_{1}^L}(\cdots FC_{\bfa\theta_{2}^1}(Attn_{\bfa\theta_{1}^1}(\bfa H)))\times\tda W_1,
    \end{align*}}
    where $\tda W_0\in\bb R^{d_1\times D}$ and $\tda W_1\in\bb R^{N\times d_2}$.
\end{definition}
The two additional matrices $\tda W_0$ and $\tda W_1$ serve for the dimension adjustment purpose such that the output of $TF_{\bfa\theta}(\bfa H)$ or $TF^+_{\bfa\theta}(\bfa H)$ will be of dimension $\bb R^{d_1\times d_2}$.

Then, we introduce a class of models called the Transformer+, which includes the un-normalized Attention layer.
\begin{definition}[Transformer+]
    Under the same notations as definition \ref{transform}. We define the Transformer+ model $TF_{\bfa\theta}^+(\cdot)$ as
    \sm{\begin{align*}
        TF_{\bfa\theta}^+&(\bfa H) :=\tda W_0\times FC_{\bfa\theta_{2}^L}(A_{\bfa\theta_{1}^L}(\cdots FC_{\bfa\theta_{2}^1}(A_{\bfa\theta_{1}^1}(\bfa H)))\times\tda W_1,
    \end{align*}}
    where $A\in\{Attn,nAttn\}$ is either the Attn layer defined in definition 2.1 or the nAttn layer defined in definition 2.2.
\end{definition}

We use $\bfa \theta$ to denote all the parameters in the Transformer and the super-index $\ell$ to denote the parameter matrix corresponding to the $\ell$-th layer.
    Under such definition, the parameter $\bfa\theta$ is given by 
    \sm{\begin{align*}
        \bfa \theta = \{\{(\{\bfa Q_m^{(\ell)},\bfa K_m^{(\ell)},\bfa V_m^{(\ell)}\}_{m\in[M]}, \bfa W_{1}^{(\ell)},\bfa W_{2}^{(\ell)})\}_{\ell\in[L]},\tda W_{0},\tda W_{1}\}.
    \end{align*}}
Following the notations in \citep{bai2024transformers}, we define the operator norm of the parameter $\bfa\theta$ as follows.
    \begin{align*}
&\vertiii{\bfa\theta}:=\max_{\ell\in[L]}\Big\{\max_{m\in[M^{(\ell)}]}\lef\{\Vert \bfa Q_m^{(\ell)}\Vert_2,\Vert\bfa K_m^{(\ell)}\Vert_2\rig\}\\
&+\Vert\tda W_0\Vert_2+\Vert\tda W_1\Vert_2+\sum_{m=1}^{M^{(\ell)}}\Vert\bfa V_m^{(\ell)}\Vert_2+\Vert\bfa W_1^{(\ell)}\Vert_2+\Vert\bfa W_2^{(\ell)}\Vert_2\Big\}, 
    \end{align*}
    where $M^{(\ell)}$ is the number of heads of the $\ell$-th attention layer.
    It is also shown in \citep{bai2024transformers} that such a norm relates to the Lipschitz constant of Transformers, which controls the model complexity and leads to the generalization bound. Hence, in this work, we consider the following space of the model
    \begin{align*}
        \Theta(B_{\bfa\theta},B_{M},B_L) &= \bigg\{(\bfa\theta, \{M^{(\ell)}\}_{\ell\in[L]}, L): \vertiii{\bfa\theta}\leq B_{\bfa\theta},\\
        &\quad\quad  \sup_{\ell\in[L]}M^{(\ell)}\leq B_{M}, L\leq B_L \bigg\}.
    \end{align*}
    And for the subspace of $\bfa\theta$ given $M$ and $L$ as hyperparamaters, we denote by
    $\Theta_{B_M,B_L}(B_{\bfa\theta})$.
    

\subsection{The Learning Problem and EM}\label{emintro}
In this section, we first provide notations for the sub-Gaussian mixture models and the clustering problem. Then, we provide the literature on the EM Algorithm and Lloyd's algorithm.
\subsubsection{Clustering Mixture of Gaussians}
 We take samples $\{\bfa X_i\}_{i\in[N]}$ from a sub-Gaussian mixture model with in total of $k$ centers $\{\bfa\mu_i\}_{i\in[k]}$. In particular, we let
 \begin{align*}
  \bfa X:=\begin{bmatrix}
      \bfa X_1\ldots\bfa X_N
  \end{bmatrix},\quad   \bfa X_i := \bfa\mu_{z_i}+\bfa\omega_i\text{ for all }i\in[N],
 \end{align*}
 where $z_i:\in[k]$ corresponds to the membership of $i$-th index. We assume the following condition to hold for $\bfa\omega_i$.
 \begin{assumption}
     $\{\bfa\omega_i\}_{i\in[N]}$ are i.i.d. zero mean random variables from sub-Gaussian distribution that satisfies 
     $\bb E[\exp(\bfa a^\top\bfa\omega)]\leq\exp\lef(\frac{1}{2}\sigma^2\Vert\bfa a\Vert_2^2\rig)$ for all $\bfa a\in\bb R^d$.
 \end{assumption}
 We consider the mapping from $z$ to a set of one-hot vectors
\sm{\begin{align}\label{defp}
    \bfa P_1(z):=\begin{bmatrix}
        \bfa p_{1,1}&\bfa p_{1,2}&\ldots&\bfa p_{1,N}
    \end{bmatrix}\in\bb R^{k\times N},\bfa p_{1,i,j}:=\mbbm 1_{j=z_i}.
\end{align}}

 
Define $\ca S_k:[k]\to[k]$ as the set of permutations of $[k]$. We consider the following loss function for the Transformer output. 
\begin{align*}
    L(A_{\bfa\theta}(\bfa H),\bfa P_1(\bfa z)):=\inf_{\pi\in\ca S_k}\frac{1}{N}\Vert\bfa P_1(\pi(\bfa z_i))-A_{\bfa\theta}(\bfa H)\Vert_{1,1},
\end{align*}
where $A\in\{TF,TF^+\}$. 
\paragraph{The Parameter Space} This work considers the following space of parameters of the generative model. Here, we denote $F_{\omega}$ as the distribution of the random variable $\omega$.
\begin{align*}
    \Theta_{GM}&=\Big\{(\bfa\mu,\bfa z,F_{\bfa \omega}),\bfa\mu\in\bb R^{d\times k}, \Delta\leq \min_{i\neq j}\Vert\bfa\mu_i-\bfa\mu_j\Vert_2,\\
    &\bfa z:[N]\to[k], |\{i\in[N],\bfa z_i=u\}|\geq\alpha n, \forall u\in[k], \\
    &\omega_i\text{ is i.i.d. }\sigma\text{ sub-Gaussian random variable } \forall i\in[N]\Big\}.
\end{align*}
We further consider the solution space $\Theta_{\bfa A}=\{\bfa A:\sum_{j=1}^N\bfa A_{ij}=1,\forall i\in[k],\bfa A\in[0,1]^{k\times N}\}$.
Then, the fundamental limit of the problem class $\Theta_{GM}$ is given by the following lemma.
\begin{lemma}[Lower Bound \citep{yu2015useful}]\label{minimaxlb} For model class $\Theta_{GM}$, given $\frac{\Delta}{\sigma \log(k/\alpha)}\to \infty$, 
\begin{align*}
    \inf_{\wha A\in\Theta_{\bfa A}}\sup_{(z,\theta,F_{\omega})}\bb E[L(\wha A,\bfa P_1)]\geq\exp\lef(-(1+o(1))\frac{\Delta^2}{8\sigma^2} \rig).
\end{align*}
\end{lemma}
\begin{remark}
    The above result implies that the difficulty of this problem is governed by the Signal-to-Noise ratio $\frac{\Delta}{\sigma}$. In particular, the above results imply that the minimax rate of this problem is largely dependent on the distance between the two closest centroids.
    We also note that the original result is instead on the $0-1$ loss between $\wha z$ and $\bfa z$. However, it is also not difficult to show the same results hold for the solution space $\Theta_{\bfa A}$ and our defined loss $L$.
\end{remark}

\subsubsection{The EM (Lloyd's) Algorithm}
Lloyd's algorithm is a special case of EM algorithm on the Gaussian mixture model, which is formally stated by Algorithm \ref{alg:loyld}. The Lloyd's algorithm iteratively updates: \textbf{(1)} The centroid of each cluster; \textbf{(2)} The membership of each sample. Since Lloyd's algorithm requires an initial input $\{\wha\mu_i^{(0)}\}$, \citet{lu2016statistical} has shown that given a proper initialization algorithm \ref{alg:example}, Lloyd's algorithm provably achieves good performance. An example initialization algorithm is given by algorithm \ref{alg:example} where the spectral algorithm and k-means++ algorithm \citep{kumar2004simple} are first called to obtain approximate solutions.
\begin{algorithm}[h]
   \caption{Lloyd's Algorithm}
   \label{alg:loyld}
\begin{algorithmic}
   \STATE {\bfseries Input:} A sample matrix from Mixture of Gaussians $\bfa X\in\bb R^{d\times N}$, number of iterations $\tau$, and initial centroids $\{\wha\mu^{(0)}_i\}_{i\in[k]}$.
   \STATE Compute the Initial Clusters 
   \begin{align}\label{z0}
       \wha z^{(0)}_i = \argmin_{i\in[k]}\Vert \bfa X_j - \wha\mu_i^{(0)}\Vert_2\quad\text{for all }j\in[N].
   \end{align}
   \FOR{$\ell=1$ {\bfseries to} $\tau$}
   \STATE \textbf{(1) The Expectation Step:} Update the centroid by 
   \sm{\begin{align*}
       \wha \mu_i^{(\ell)}=\frac{\sum_{j=1}^N\mbbm 1_{\wha z_j^{(\ell-1)}=i}\bfa X_j}{\sum_{j=1}^N\mbbm 1_{\wha z_j^{(\ell-1)}=i}}.
   \end{align*}}
   \STATE \textbf{(2) The Maximization Step:} Update the cluster assignment by
   \begin{align*}
       \wha z^{(\ell)}_j=\argmin_{i\in[k]}\Vert \bfa X_j-\wha\mu_i^{(\ell)}\Vert_2\quad\text{for all }j\in[N].
   \end{align*}
   \ENDFOR
\end{algorithmic}
\end{algorithm}


\subsection{Pretraining with Supervised Learning}\label{pretrain}
The clustering problem is unsupervised where no labels are given. Transformers are usually used in the supervised learning setup. To let Transformers learn the algorithms, we perform supervised pre-training. 

In this setup, we are first given in a total of $n$ pretraining instances $\{\bfa X^{(i)}\}_{i\in[n]}$ and $\{\bfa z^{(i)}\}_{i\in[n]}$. We also form the pretraining instances by feeding the Transformer with the initialization given by \ref{alg:example}, encoded in $\{\bfa H^{(i)}\}_{i\in[n]}$. Then, we train the Transformer using the standard supervised learning on this set. Since the optimization of Transformers is non-convex and difficult to analyze, we consider the empirical risk minimizer of the Transformer given by 
\begin{align}\label{ERM}
    \wha\theta :=\argmin_{\bfa\theta\in\Theta_{(B_M,B_L)}(B_{\bfa\theta})}\sum_{i=1}^nL\lef(A_{\bfa\theta}(\bfa H^{(i)}),\bfa P_1(\bfa z^{(i)})\rig),
\end{align}
where $A\in\{TF,TF^+\}$.

In our theoretical analysis, we construct the input of the Transformer as a \emph{context-augmented matrix} given by the following
\sm{\begin{align}\label{aux}
    \bfa H=\begin{bmatrix}
        \bfa X\\
        \bfa P
    \end{bmatrix}, \bfa P=\begin{bmatrix}
        \wha \mu_{1}^{(0)}&\wha \mu_{2}^{(0)}&\ldots&\wha \mu_{k}^{(0)}&\ldots&\bfa 0\\
        \bfa p_{1,1}^{(0)}&\bfa p_{1,2}^{(0)}&\ldots&\bfa p_{1,k}^{(0)}&\ldots&\bfa p_{1,N}^{(0)}\\
        \bfa p_{2,1}&\bfa p_{2,2}&\ldots&\bfa p_{2,k}&\ldots&\bfa p_{2,N}\\
        1&1&\ldots&1&\ldots&1\\
        &&\bfa 0&&&
    \end{bmatrix},
\end{align}}
where $\bfa H\in\bb R^{D\times N}$ and $\bfa P^{(D-d)\times N}$. We let the input dimension $D\leq Ckd$ for some universal constant $C$.
The matrix $\bfa P$ contains contextual information. For the first row, $\{\wha\mu_i^{(0)}\}_{i\in[k]}\subset\bb R^{d}$ are the initial centroid estimates given by the initialization algorithm \ref{alg:example}. Then the next row $\{\bfa p_{1,i}\}_{i\in[N]}\subset[0,1]^{k}$ is given by $\bfa P_1(\wha z^{(0)})$ as in \eqref{defp} where $\wha z^{(0)}$ corresponds to the initialized membership in \eqref{z0}. Then the row $\{\bfa p_{2,i}\}_{i\in[N]}\subset\bb R^d$ satisfies
\begin{align*}
    \bfa p_{2,i,j}=\mbbm 1_{i=j}\text{ if }j\in[d].
\end{align*}
And the last row is set to all $1$ for the technical purpose of introducing constants into the Softmax function. 
\begin{algorithm}[tb]
   \caption{Initialization by Spectral Clustering}
   \label{alg:example}
\begin{algorithmic}
   \STATE {\bfseries Input:} Matrix $\bfa X\in\bb R^{d\times N}$.
   \STATE Perform PCA on $\bfa X\bfa X^\top$ and obtain its top-k eigenvectors $\{\bfa V_i\}_{i\in[k]}$.
   \STATE Project the input matrix by $\tda X=\bfa V^\top \bfa X$.
   \STATE Solve the $k$-means program given by 
   \begin{align*}
       \tda z:=\argmin_{\wh z:[N]\to [k]}\min_{\{\bfa\mu_i\}_{i\in[N]}}\sum_{i=1}^k\Vert\bfa \mu_{\wh z_i}- \tda X_i\Vert_2
   \end{align*}
   by the $k$-means++ algorithm \citep{kumar2004simple}.
   \STATE{\bfseries Return:} Initial Cluster Assignment $\tda z$.
\end{algorithmic}
\end{algorithm}

% \paragraph{The Learning Problem.} Consider a set of samples $\{\bfa X_i\}_{i\in[u]}$ i.i.d. sampled from some distribution $p_{\bfa X}$, we construct their oracle top-$k$ principle components as $\bfa V_i=\begin{bmatrix}
%     \bfa v^{i,\top}_1&\ldots&\bfa v_k^{i,\top}
% \end{bmatrix}^\top$
% and the context-augmented input matrix as $\bfa H_i$ for each $\bfa X_i$. Then, the pretraining procedure is given by minimizing the following objective for some convex loss function $L(\cdot,\cdot):\bb R^{dk}\times \bb R^{dk}\to\bb R$,
% \begin{align}\label{ERM}
%  \wha\theta =\argmin_{\bfa\theta\in\Theta(B_{\bfa\theta}, B_M)}\sum_{i=1}^uL(TF_{\bfa\theta}(\bfa H_i), \bfa V_i).
% \end{align}
% Here we consider $\Theta(B_{\bfa\theta}):=\{\bfa\theta:\Vert\bfa\theta\Vert\leq B_{\bfa\theta}, \max_{\ell}M^{(\ell)}\leq B_M\}$ to be the space of parameters. We also consider guarantees in the $L_2$ norm which states that $L(\bfa x_1,\bfa x_2):=\Vert\bfa x_1-\bfa x_2\Vert_2$ in the theoretical part.
% Since $\wha\theta$ given by minimizing the empirical risk is not obtainable in practice, our theory only gives guarantee on the empirical risk minimizer. We further show that the local minimizers obtained through stochastic gradient optimization achieve good empirical performance in section \ref{sect4}.


\section{Generalization}
In this section, we investigate the generalization bound of pretraining transformer-based time series foundation models.
This section will focus on learning MOIRAI on multi-variate time series, one can easily adapt our proofs into learning uni-variate time series with standard transformers.

Let $\pi$ be a meta distribution, and each distribution drawn from it $ \mathtt{P}^{(T)} \sim \pi$, satisfies Dobrushin's condition \cite{Dobrushin1968TheDO} (which we will introduce shortly).
For pretraining data, we first sample $n$ distributions $\mathtt{P}_j^{(T)}$ i.i.d. from $\pi$, and for each distribution, we sample a time series $(\bx_{1j}, \cdots, \bx_{Tj})$, for $j \in[n]$, and each of them contains no more than $d$ covariates and with lag step no more than $q$.


For each time series, we encode it with any-variate encoding into an input matrix denoted as $\bH \in \R^{D \times N}$, \footnote{Due to any-variate encoding, $N = dT$.}
We define each pretraining sample as $\bz_j \coloneqq \left( \bH_j, y_j \right)$, where $y_j = \bx_{Tj}^1$.
We consider the squared loss between model prediction and the label, i.e.
{\small
\begin{equation*}
    \ell( \bz_t, \bm{\theta} )
    \coloneqq
    \frac{1}{2}
    \Bigg[
    y_t
    -
    \mathtt{Clip}_{B_x}
    \bigg(
    \mathtt{read}_y
    \Big(
    \text{TF}_{\bm{\theta}}^R
    \left(
    \bH
    \right)
    \Big)
    \bigg)
    \Bigg]^2,
\end{equation*}
}
where 
$\mathtt{Clip}_{B_x}(t) \coloneqq \max\{ \min \{ t, B_x \}, -B_x \}$, and $\text{TF}_{\bm{\theta}}^R$ is the MOIRAI transformer defined in \cref{def:moirai} with $\mathtt{Clip}(\cdot)$ applied after each layer.
The pretraining loss and test loss is defined as the following:
\begin{equation}\label{eqn:icl-loss}
    \hat{L}( \bm{\theta} )
    \coloneqq
    \frac{1}{nT}
    \sum_{t=1}^T
    \sum_{j=1}^n
    \ell( \bm{\theta}, \bz_{jt} ),
    \;
    L( \bm{\theta} )
    \coloneqq
    \mathbb{E}_{\bz,\mathtt{P}^{(T)}}
    \left[
    \ell(  \bm{\theta}, \bz )
    \right].
\end{equation}
% where the expectation of test loss is taken over $\mathtt{P}_j^{(T)} \sim \pi, \bz \sim \mathtt{P}^{(T)}_j$.
The goal of our pretraining algorithm is to find an empirical risk minimizer (ERM) over MOIRAI transformers with $L$ layers, $M$ heads, and norm bounded by $B$:
\begin{align}\label{eqn:parameter-regime}
    &\hat{\bm{\theta}}
    \coloneqq
    \underset{ \bm{\theta} \in \Theta_{L,M,D^\prime, B} }{\argmin}
    \hat{L}
    (\bm{\theta}),
    \\
    &\Theta_{L,M,D^\prime,B}
    \coloneqq
    \Bigg\{
    \bm{\theta}
    =
    \left(
    \bm{\theta}_1^{(1:L)},
    \bm{\theta}_2^{(1:L)}
    \right)
    :
    \\
    \max_{\ell\in[L]}
    &M^{(\ell)} \leq M
    , 
    \quad
    \max_{\ell\in[L]}
    % \underset{\ell\in[L]}{\max}
    D^{(\ell)} \leq D^\prime
    ,
    \quad
    \Vert\bfa\theta\Vert_{op}
    \leq 
    B
    \Bigg\}.
\end{align}
% \vspace{-1em}
\subsection{Weakly-Dependent Time Series}
In this scenario, we consider the training data $\bx$ to be drawn from a distribution $\mathtt{P}$ satisfying Dobrushin's condition.
Under this condition, we are able to present several generalization bounds on pretraining.


\begin{definition}[Influence in high dimensional distributions]\label{def:influence}
    Let $\cX = (\cX_1, \cdots , \cX_T)$ be a sequence of random variables over $\cD_{\cX}^{T}$.
    The influence of variable $\cX_j$ on variable $\cX_i$ is defined as
    \begin{align*}
        &\bI_{j \rightarrow i}(\cX)
        \coloneqq
        \max_{ \text{x}_{ -i-j}, \text{x}_j, \text{x}_j^\prime  }
        \\
        % \underset{  z_{-i-j} \in Z^{m-2} }{ z_j, z_j^\prime \in Z}}
        &
        \norm{
        P_{\cX_i | \cX_{-i}} \left( \cdot | \text{x}_{-i-j}, \text{x}_j \right),
        P_{\cX_i | \cX_{-i}} \left( \cdot | \text{x}_{-i-j}, \text{x}_j^\prime \right)
        }_{\texttt{TV}}
        ,
    \end{align*}
    where $\text{x}_{-i-j} \in \cD_{\cX}^{T-2}, \text{x}_j, \text{x}_j^\prime \in \cD_{\cX}$, $\norm{\cdot}_{\texttt{TV}}$ denotes the total variation distance, and $\text{x}_{-i}$ represents the vector \textbf{x} after omitting the $i$-th element.
\end{definition}



\begin{definition}[Dobrushin's Uniqueness Condition]
    Consider a random variable $\cX$ over $\cD_{\cX}^T$.
    The Dobrushin coefficient of $\cX$ is defined as 
    \[
    \alpha(\cX) \coloneqq  \max_{1\leq i \leq T} \sum_{j \neq i} \bI_{j \rightarrow i} (\cX).
    \]
    We say the variable satisfies Dobrushin's uniqueness condition if $\alpha(\cX) < 1$.
    For a distribution $\mathtt{P}$, we denote $\alpha(\mathtt{P}) = \sup_{\cX\sim\mathtt{P}} \alpha(\cX)$.
\end{definition}
\begin{definition}[Log Dobrushin's Coefficients]
        Let $\cX = (\cX_1, \cdots, \cX_T)$ be a random variable over $\cD_{\cX}^T$ and let $\mathtt{P}_z$ denote its density.
        Assume that $\mathtt{P}_{z} > 0$ on all $\Omega^T$.
        For any $i \neq j \in [T]$, the log influence between $j$ and $i$ is defined as:
    {\small
    \begin{equation*}
        I^{\log}_{j, i}(\cX)
        =
        \frac{1}{4}
        \sup
        \log
        \frac{ P
        \left[
        \text{x}_i, \text{x}_j, \text{x}_{-i-j}
        \right] 
        P
        \left[
        \text{x}_i^\prime, \text{x}_j^\prime, \text{x}_{-i-j}
        \right]
        }{
    P
        \left[
        \text{x}_i^\prime, \text{x}_j, \text{x}_{-i-j}
        \right] 
    P
        \left[
        \text{x}_i, \text{x}_j^\prime, \text{x}_{-i-j}
        \right] 
        },
    \end{equation*}
    }
    where the $\sup$ is taken over $\text{x}_{-i-j}, \text{x}_i, \text{x}_i^\prime, \text{x}_j, \text{x}_j^\prime$,
    and the log-coefficient of $\cX$ is defined as $\alpha_{\log}(\cX) = \max_{i \in [T]} \sum_{j \neq i} I^{\log}_{j, i}(\cX)$.
\end{definition}
The coefficient $\alpha(\cdot)$ has a natural bound $0 \leq \alpha(\cdot) \leq T-1$, with $\alpha = 0$, the data reduces to the i.i.d. case.

\begin{remark}
    Dobrushin's condition characterizes a class of distributions whose dependency is mild. 
    However, our empirical evaluation suggests that in certain situations where Dobrushin's condition fails to hold, the Transformers can perform prediction well.
\end{remark}
\subsection{Generalization Bounds of MOIRAI}

\begin{theorem}[Pretraining Generalization Bound]\label{thm:gen-bound-1}
    Let $\Theta_{L,M,D^\prime, B}$ be the parameter space defined in 
    Equation~\ref{eqn:parameter-regime}.
    Assume $\alpha_{\log}( \mathtt{P}^{(T)}) < 1/2$.
    Then with probability at least $1 - \varepsilon$, ERM $\hat{\bm{\theta}}$ satisfies the following:
    \begin{align*}
    L(\hat{\bm{\theta}})
    &\leq 
    \inf_{\bm{\theta} \in \Theta_{L,M,D^\prime, B}} L(\bm{\theta})
    +
    \\
    &
    O
    \left(
    \frac{B_x^2}{1 - \alpha(\mathtt{P}^{(T)}) }
    \sqrt{
    \frac{
    L(MD^2 + D D^\prime) \zeta + \log(\nicefrac{1}{\varepsilon})
    }{n}
    }
    \right),
    \end{align*}
    % where $\iota = \log( 2 + 2(L B_H^{L-1} B_{\Theta}) B \frac{1-\alpha}{B_x})$
    where $C$ is an universal constant, and $\zeta = O(\log(2 + \max \{ B, \mathtt{R}, B_x, T, d \}$. 
\end{theorem}

The proof is in \cref{proof:gen-bound-1}.
Note that when $\alpha(\mathtt{P}) = 0$, the data becomes i.i.d., where the only difference between our generalization and one proposed in \cite{bai2024transformers} is the complexity term. 
The complexity of MOIRAI and standard transformers differs as the complexity of MOIRAI also dependents on the time series length ($T$).
Further, in \cref{thm:gen-bound-1}, we do not assume our data is generated from the $\mathtt{AR}$ process, only its Dobrushin coefficient.
When the data is generated by the $\mathtt{AR}$ process, we are able to give a more explicit bound on the same test loss as described below.

\begin{corollary}[Test Error Bound]\label{thm:test-error-bound-1}
    Following the setup in \cref{thm:gen-bound-1},
    if pretraining samples are generated by some $\mathtt{AR}_d(q)$ process with noise sampled from $N(0, \sigma^2_\epsilon)$\footnote{Here we assume fixed $d, q$ across all samples as one can describe a lower dimension/order $\mathtt{AR}$ process with zero coefficients.}, 
    then with probability $ \Delta(1 - \varepsilon)$, ERM $\hat{\bm{\theta}}$ satisfies the following:
    \begin{align*}
    L(\hat{\bm{\theta}})
    &
    \leq
    O
    \Bigg(
    B_x B_w \exp \left( \frac{-L}{\kappa} \right)
    +
    \\
    &
    \frac{B_x^2}{1 - \alpha(\mathtt{P}^{(T)})}
    \sqrt{ \frac{L(MD^2 + D D^\prime ) \zeta + \log (1 / \varepsilon)}{n} }
    \Bigg).
    \end{align*}
    where $\Delta = O\left(1 - \left(  \nicefrac{\sigma_\epsilon}{B_x  B_w e^{\nicefrac{-L}{2\kappa}}}  \right)^2  \right)$, $C$ is an universal constant, and $\zeta = O(\log(2 + \max \{ B, \mathtt{R}, B_x, T, d \})$. 
\end{corollary}
\begin{remark}
Considering the model parameters ($M,D,D^\prime, d$) are of constant level, one is able to further optimize the bound to
$L(\hat{\bm{\theta}}) \lesssim  n^{-\nicefrac{1}{2}}$, by selecting $L$ appropriately.    
\end{remark}

% \begin{theorem}[Multi-Path Generalization Bound]
%     Let $L_{\mathtt{T}}( \bm{\theta})$, $\hat{L}_{\mathtt{T}})\bm{\theta})$ be the test and empirical loss over samples drawn from a target distribution $\mathtt{T}$.
%     Let $\lambda_k = \inf L_k( \bm{\theta})$, where the $\inf$ is taken over all $\bm{\theta} \in \Theta_{L,M,D^\prime, B}$.
%     Following the constructions above, we have
%     \begin{equation}
%         L_{\mathtt{T}}(\hat{\bm{\theta}})
%         \leq 
%         \frac{1}{K}
%         \sum_{k=1}^K
%         \left(
%         \hat{L}_{\mathtt{T}}( \hat{\bm{\theta}})
%         +
%         C_1 \cdot
%         \hat{\mathfrak{G}}
%         +   
%         \lambda_K
%         \right)
%         +
%         \cdots,
%     \end{equation}
%     where $\lambda_K$ is the average risk of the optimal hypothesis on the combined source and target domain,  $C_1, C_2$ are universal constants whenever $\nicefrac{1}{2} - \alpha_{\log}(D_S^k)$ is bounded away from zero, and $\hat{\mathfrak{G}}$ is the empirical Gaussian complexity.
% \end{theorem}


\subsection{Example: Stationary $\mathtt{AR}(1)$}\label{sec:AR1}
Here we provide an example of the application of \cref{thm:test-error-bound-1} on $\mathtt{AR}(1)$ process with the following form
\begin{equation*}
    \bx_{t+1}
    =
    \langle \bw, \bx_{t} \rangle + \epsilon_t, \quad \epsilon \sim N(0, \sigma_{\epsilon}^2),
\end{equation*}
where $\bx_t \in \R^d$, $\bw \in \R^d, \epsilon \in \R$ and $\by_{t+1} = \bx_{t+1}^1$.

To satisfy the condition of $\alpha( \mathtt{P} ) < \frac{1}{2}$, we assume the following holds
\begin{equation}\label{eqn:condition-weakly-dependent-AR1}
    B_x^2 < \ln \frac{1}{2} + ( \sigma_{\epsilon}^2 ),
    \quad  \norm{\bw}_{\infty} < 1.
\end{equation}
The first condition comes from the fact that we require the pair-wise potential of this time series to be less than $1/2$ (For more details, see \cref{appendix:analysis-ar1}).
The second condition comes from the requirement of it being stationary.
\begin{proposition}[Generalization Bound for Any-Variate Transformer on $\mathtt{AR}(1)$]\label{proposition:ar1}
    Considering an $\mathtt{AR}(1)$ process with Dobrushin's coefficient bounded by $1/2$.
    With probability at least $\delta(1 - \varepsilon)$, ERM $\hat{\bm{\theta}}$ satisfies the following:
    \begin{align*}
    L(\hat{\bm{\theta}})
    &=
    O
    \Bigg(
    \frac{\sigma_\epsilon}{\sqrt{1 - \delta}}
    +
    \frac{\sigma_\epsilon^2}{B_x}
    \exp \left( \frac{-L}{\kappa} \right)
    +
    \\
    &
    \frac{\sigma_\epsilon^2}{1 - \alpha( \mathtt{AR}(1) )}
    \sqrt{ \frac{L(MD^2 + D D^\prime) \zeta + \log (1 / \varepsilon)}{n} }
    \Bigg).
    \end{align*}
    where $\zeta = O(\log(2 + \max \{ B, \mathtt{R}, B_x, d \})$. 
\end{proposition}
If we further optimize the bound by viewing the hyperparameters as constants, the test error obeys $O(e^{-L} + \sqrt{\frac{L}{n}})$ with high probability whenever $\sigma_\epsilon$ is small.

\section{Theoretical Results}\label{sect3}
This section presents our theoretical results and the idea of taking each step in the proof. Our proof constructs a particular instance of the transformers and shows that the forward propagation on our constructed instance approximates the Power Method. We also carefully design the contextual matrix $\bfa P$, explained as follows.

\paragraph{The Design of Auxillary Matrix.} Our design of the matrix $\bfa P$ consists of three parts:
\begin{enumerate}
    \item \emph{Place Holder.} For $\ell\in\{1\}\cup[4:k+3]$ and $i\in[N]$, we let $\tda p_{\ell, i}=\bfa 0\in\bb R^{d\times 1}$. The place holders in $\bfa P$ record the intermediate results in the forward propagation.
    \item \emph{Identity Matrix.} We let $ \begin{bmatrix}
        \tda p_{2,1}&\ldots&\tda p_{2,N}
    \end{bmatrix}=\begin{bmatrix}
        \bfa I_d&\bfa 0_{d\times(N-d)}
    \end{bmatrix}$. The identity matrix in $\bfa P$ helps us screen out all the covariates $\bfa X$ in the forward propagation.
    \item \emph{Random Samples on the Hypersphere.} We let $\tda p_{3,1},\ldots\tda p_{3,k}$ be the i.i.d. samples uniformly distributed on $\bb S^{d-1}$. The random samples on the sphere correspond to the initial vectors $\bfa v_{0,\ell}$ for $\ell\in[k]$ in algorithm \ref{alg:almoexactrecov}.
\end{enumerate}
Given the above construction on the auxiliary matrix $\bfa P$, we are ready to state the existence theorem in this work, given as follows.

\begin{algorithm}[htbp] 
 \caption{Spectral Clustering}
\label{alg:spectralclustering}
\KwData{$\bfa X\in\bb R^{D\times N}$ with $\bfa X_i\in\bb R^D$ for all $i\in[N]$}
 Compute the SVD of the data matrix $\bfa X=UDV^\top$. Let $U_k$ be the first $k$ columns of $U$\;

 Project $\bfa x_1,\ldots,\bfa x_N$ onto $U_k$, i.e. let $\wha x_i\gets U_kU_k^\top \bfa x_i$ \;
 Run an \emph{Transformer k-means++\cite{arthur2006k}} algorithm on the columns of projected matrix $\wha X=\begin{bmatrix}
     \wha x_1,\ldots,\wha x_N
 \end{bmatrix}$\;
\end{algorithm}
\begin{assumption}\label{asumpt1}
    We assume that $\bfa X_1,\ldots,\bfa X_N$ are i.i.d. samples from $\sum_{\ell=1}^k\pi_\ell\bb P(X|\mu)$. Minimal separation is denoted by $\Delta:=\inf_{i\neq j}\Vert\bfa\mu_i-\bfa\mu_j\Vert_2$.
\end{assumption}

And the kmeans++ algorithm is given by the following procedure.
\begin{algorithm}[htbp] 
 \caption{k-means++}
\label{alg:kmeans++}
\KwData{$\bfa X\in\bb R^{k\times N}$ with $\bfa X_i\in\bb R^k$ for all $i\in[N]$}
Choose an initial center $\bfa c_1$ uniformly at random from $\ca X:=\{\bfa X_1,\ldots,\bfa X_N\}$ and let $\ca C=\{\bfa c_1\}$\;
\While{$|\ca C|<k$}{Choose the next center $\bfa c$, selecting $\bfa c=\bfa x\in\ca X$ with probability $\frac{D(\bfa x)}{\sum_{x\in\ca X}D(\bfa x)}$\\
where $D(\bfa x):=\argmin_{\bfa c\in\ca C}\Vert \bfa c-\bfa x\Vert_2$\;
If $\bfa c\notin\ca C$, let $\ca C\gets\ca C\cup\{\bfa c\}$\;
}
Run the Lloyd's algorithm with covariates $\bfa X$ and initial centroids $\ca C$\;
\end{algorithm}
\begin{algorithm}[htbp] 
 \caption{Lloyd's Algorithm}
\label{alg:lloyd}
\KwData{$\bfa X\in\bb R^{k\times N}$ and initial clusters $\ca C=\lef\{\wha\mu_\ell^{(0)}\rig\}_{\ell\in[k]}$}
Let $t\gets 1$\;
Initialize with the starting membership $\wh z_i^{(0)}\gets\argmin_{\ell\in[k]}\Vert \bfa x_i-\wha\mu_\ell^{(0)}\Vert_{\Sigma^{-1}}$\;
\While{$\exists j\in[n]$ such that $\wh z^{(t+1)}_j\neq \wh z^{(t)}_j$}{Update the centroids and memberships through $\wha\mu_{\ell}^{(t)}\gets\frac{\sum_{i=1}^n\bfa x_{i}\mbbm 1_{\wh z_i^{(t-1)}=\ell}}{\sum_{i=1}^n\mbbm 1_{\wh z_i^{(t-1)}=\ell}},\enspace\forall \ell\in[k],\qquad \wh z_i^{(t)}\gets\argmin_{\ell\in[k]}\Vert \bfa x_i-\wha\mu_{\ell}^{(t)}\Vert^2_{\Sigma^{-1}},\enspace\forall i\in[n].$
}
Run the Lloyd's algorithm with $\bfa X$ and $\ca C$\;
\end{algorithm}

In particular, take as input the context given by $\bfa H=\begin{bmatrix}
    \bfa x_1,\ldots,\bfa x_N\\
    \bfa p_1,\ldots,\bfa p_N
\end{bmatrix}\in\bb R^{D\times N}$, we show that the Transformer network can approximate the spectral clustering algorithm in the initialization phase, given by \ref{alg:spectralclustering}.
\subsection{The Power Iteration Method}
In the first step of the power method 
we use the Transformer model to approximate the singular vectors of the matrix $\bfa X^\top\bfa X$. A typical implementation to achieve this is given by the power method.


\begin{theorem}[Transformer Approximation of the Power Iteration]\label{thm3.1} Denote the eigenvalues of $\bfa X\bfa X^\top$ to be $\lambda_1>\lambda_2>\ldots>\lambda_k>\ldots$. Let $\Delta:=\min_{1\leq i<j\leq k}|\lambda_i-\lambda_j|$. Assume that the eigenvalues of $\bfa X$ satisfy $\Vert\bfa X\Vert_2\leq B_X$. Assume that the initialized vectors $\tda p_{3,1},\ldots\tda p_{3,N}$ 
    satisfy  $\tda p_{3,i}^\top\bfa v_{i}\geq\delta$ for all $i\in[k]$ and make the rest of the vectors $\bfa 0$. 
    Then, there exists a transformer model with number of layers $L=2\tau+4k+1$ and number of heads $M\leq \lambda_1^d\frac{C}{\epsilon^2}$ with $\tau\leq\frac{\log(1/\epsilon_0\delta)}{\epsilon_0}$ such that for all $\epsilon_0,\epsilon>0$,
    the final output $\wha v_1,\ldots,\wha v_{k}$ given by the transformer model achieve
    \begin{align*}
        \lef\Vert\wha v_{\eta+1}-\bfa v_{\eta+1}\rig\Vert_2\leq C\tau \epsilon\lambda_1^2+\frac{C\lambda_1\sqrt{\epsilon_0}}{\Delta}\prod_{i=1}^{\eta}\frac{5\lambda_{i+1}}{\Delta}.
    \end{align*} 
    Moreover, consider the accuracy of multiple $\bfa v$s as a whole. There exists $\bfa\theta$ such that 
    \begin{align*}
        L\lef(TF_{\wha\theta}(\bfa H),\bfa V\rig)\leq C\tau\epsilon k\lambda_1^2+C\bl\frac{\epsilon_0\lambda_1^2}{\Delta^2}\sum_{\eta=1}^{k-1}\prod_{i=1}^{\eta}\frac{25\lambda_{i+1}^2}{\Delta^2}\br^{1/2}.
    \end{align*}
\end{theorem}
\begin{remark}
\label{remark3}
    The approximation error consists of two terms. The first term comes from the approximation of the Power Method iterations by transformers. The second term comes from the error caused by finite iteration $\tau$. To acquire a more direct account of the error terms and its order of magnitude, we consider a special case where the eigenvalues $\lambda_1\asymp\lambda_2\asymp\ldots\asymp\lambda_k\asymp\Delta$. Then our results boil down to 
    \begin{align*}
        \lef\Vert TF_{\bfa\theta}(\bfa H)-\begin{bmatrix}
            \bfa v_1^\top,\bfa v_2^\top,\ldots,\bfa v_k^\top
        \end{bmatrix}^\top\rig\Vert_2\leq C\tau\epsilon k \lambda_1^2+ C\frac{\lambda_1}{\Delta}\sqrt{k\epsilon_0}.
    \end{align*}
    These results hide dimension $d$ in the universal constant. Hence the dimension significantly affects the approximation properties of transformers. Our experimental results in section \ref{sect4} also indicate that learning high dimensional principal eigenvectors is challenging.
\end{remark}
We show that the conditions on $\tda p_{3,1},\ldots,\tda p_{3,N}$ can be achieved through sampling from isotropic Gaussians, given by the following lemma.
    \begin{lemma}\label{lm3.1.1}\label{lm3.1}
        Let $\bfa y\in\bb R^d$ be a random vector with isotropic Gaussian as its probability density. Consider $\bfa x=\frac{\bfa y}{\Vert\bfa y\Vert_2}$. Let $\bfa v$ be any unit length vector, then we have for all $\delta<\frac{1}{2}d^{-1}$,
           $ \bb P\lef(|\bfa v^\top\bfa x|\leq\delta\rig)\leq  \frac{1}{\sqrt{\pi}}\sqrt\delta+\exp\lef(-C\delta^{-\frac{1}{2}}\rig)$.
        Therefore, for all $\delta<\frac{1}{2}d^{-1}$, the event in theorem \ref{thm3.1} is achieved with
        \begin{align*}
             \bb P\bl\exists i\in[k]\text{ such that }&\bfa x_i^\top\bfa v_i\leq\frac{\delta}{\sqrt d}\br\leq \frac{k\sqrt{\delta}}{\sqrt{\pi}}+k\exp(-C\delta^{-1}).
        \end{align*}
    \end{lemma}
Given the approximation error provided by theorem \ref{thm3.1}, we further provide the generalization error bound for the ERM defined by \eqref{ERM}. This requires us to consider the following regularity conditions on the underlying distribution of $\bfa X\bfa X^\top$ (which also translates to the distribution of $\bfa X$).
\begin{assumption}\label{assump1}
    The distribution of $\bfa X\bfa X^\top$ supports on $$\bb X:=\lef\{A:A\in\bfa S^d_{++},  B_X\geq\lambda_1(A)>\lambda_2(A)>\ldots>\lambda_k(A),\inf_{1\leq i<j\leq k}\lambda_i(A)-\lambda_j(A)\geq\Delta\rig\}.$$
\end{assumption}
\begin{remark}
    The above assumption can be easily generalized to distribution that supports on $\bb X$ with high probability. Examples of such distribution include the Wishart distribution under the Gaussian design. In this work, we stick to the simplest case where the maximum eigenvalue is bounded from above.
\end{remark}
Given the above assumption, we are ready to state the generalization bound.
\begin{proposition}\label{genbound}
   With probability at least $1-\xi$, the ERM solution $\wha\theta$ satisfies
    \begin{align*}
        \bb E\lef[L\lef(TF_{\wha\theta}(\bfa H),\bfa V\rig)|\wha\theta\rig]&\leq\inf_{\bfa\theta\in\Theta(B_{\bfa\theta},B_M)}\bb E\lef[L\lef(TF_{\bfa\theta}(\bfa H),\bfa V\rig)\rig]\\
        &+ C \sqrt{\frac{k^3LB_Md^2\log(B_{\theta}+B_X+k)+\log(1/\xi)}{n}}.
    \end{align*}
\end{proposition}
   Together with the bound given by theorem \ref{thm3.1} and lemma \ref{lm3.1.1}, which essentially give a high probability upper bound on $\inf_{\bfa\theta\in\Theta(B_{\bfa\theta},B_M)}$ we can derive a general upper bound on the generalization error, given as follows.
    \begin{corollary}
        Under assumption \ref{assump1}, with probability at least $1-\xi-\frac{k\sqrt\delta}{\sqrt\pi}-k\exp\lef(-C\delta^{-1/2}\rig)$ for all $\delta<d^{-1}$ we have for all $\epsilon,\epsilon_0>0$,
        \begin{align*}
            \bb E\lef[L\lef(TF_{\wha\theta}(\bfa H),\bfa V\rig)|\wha\theta\rig]&\leq \bb E\lef[C\tau\epsilon k\lambda_1^2+C\bl\frac{\epsilon_0\lambda_1^2}{\Delta^2}\sum_{\eta=1}^{k-1}\prod_{i=1}^{\eta}\frac{25\lambda_{i+1}^2}{\Delta^2}\br^{1/2}\rig]\\
            &+C\sqrt{\frac{k^3\log(\delta/
            \epsilon_0)\lambda_1^d d^2\log(B_{\theta}+B_X+k)+\log(1/\xi)}{n\epsilon_0\epsilon^2}}.
        \end{align*}
    \end{corollary}
    \begin{remark}
        If we consider optimizing the bound w.r.t. $\epsilon_0$ and $\epsilon$, we obtain that $\bb E\lef[L(TF_{\wha\theta}(\bfa H),\bfa V)|\wha\theta\rig]\lesssim n^{-1/5}$ given that the rest of the parameters are of constant scales. It is not known if the results are improvable or not and the authors believe this question worth future explorations.
    \end{remark}
The next theorem demonstrates that there exists a Transformer that simulates the expectation maximization algorithm on Gaussian mixture model.
\begin{theorem}
    
\end{theorem}
\begin{proof}
We first consider the input matrix to be 
\begin{align*}
    \bfa H_1:=\bfa H=\begin{bmatrix}
        \bfa x_1&\bfa x_2&\ldots&\bfa x_n\\
        \wha \mu_{\wh z^{(0)}(1)}^{(0)}&\wha \mu_{\wh z^{(0)}(2)}^{(0)}&\ldots&\wha \mu_{\wh z^{(0)}(n)}^{(0)}\\
        \bfa p_{1,1}&\bfa p_{1,2}&\ldots&\bfa p_{1,N}\\
        \bfa p_{2,1}&\bfa p_{2,2}&\ldots&\bfa p_{2,N}\\
        1&1&\ldots&1\\
        \bfa p_{3,1}&\bfa p_{3,2}&\ldots&\bfa p_{3,N}
    \end{bmatrix}\in\bb R^{D\times N}, 
\end{align*}
where $\wh z^{(0)}:[n]\to[k]$ is the assignment function, $\wha\mu_i\in\bb R^{d}$ is the initially estimated centroid for the $i$-th cluster. $\bfa p_{1,i}\in\bb R^k$ satisfies $\bfa p_{1,i,j}=\mbbm 1_{\wh z^{(0)}(i)=j}$ for all $j\in[k]$. And for $\bfa p_{2,i}$ we have $\bfa p_{2,i,j} = \mbbm 1_{j=i}$ for $i\leq d$ and $\bfa p_{2,i,j}=0$ for $i\leq N$ and $j\leq d$. We let $\bfa p_{3,1}=\bfa p_{3,2}=\ldots=\bfa p_{3,N} = \bfa 0\in\bb R^k$.
We note that algorithm \ref{alg:lloyd} consists of two iterative steps: (1) The expectation step where we take the averages to get an initial estimate $\wha \mu_{\ell}^{(t)}$. (2) The maximization step where we assign each individual their labels. Our following discussions treat the two steps separately. 
\begin{center}
    \textbf{1. The Expectation Step.}
\end{center}
To achieve the first step, we construct our transformer weights as follows:
\begin{align*}
    \bfa V_1^{(1)}=\begin{bmatrix}
        \tda V_{1,1}^{(1)}&\tda V_{1,2}^{(1)}&\tda V_{1,3}^{(1)}
    \end{bmatrix},\quad\bfa Q_1^{(1)}=\begin{bmatrix}
        \bfa 0_{1\times(3d+k)}&1&\bfa 0\\
        \bfa 0&\bfa 0&\bfa 0
    \end{bmatrix},\quad\bfa K_1^{(1)}=\begin{bmatrix}
        \bfa 0_{1\times(3d+k)}&1&\bfa 0\\
        \bfa 0&\bfa 0&\bfa 0
    \end{bmatrix},
\end{align*}
where $\tda V_{1,1}^{(1)}\in\bb R^{2d\times D}=\bfa 0$, $\tda V_{1,2}^{(1)}=\begin{bmatrix}
        \bfa 0_{3d+k}\\
        I_k\\
        \bfa 0
    \end{bmatrix}\in\bb R^{k\times D}$. Then we can show that
\begin{align*}
   (\bfa K_1^{(1)}\bfa H_1)^\top=(\bfa Q_1^{(1)}\bfa H)^\top = \begin{bmatrix}
        1&\bfa 0\\
        \vdots &\bfa 0\\
        1&\bfa 0
    \end{bmatrix}.
\end{align*}
Then we can show that
\begin{align*}
   (\bfa Q_1^{(1)}\bfa H)^\top(\bfa K_1^{(1)}\bfa H) = \begin{bmatrix}
       \bfa v_1&\ldots&\bfa v_1
   \end{bmatrix},\quad \bfa v_{1,i} = 1\quad\forall i\in[N].
\end{align*}
Hence, we can obtain that
\begin{align*}
    \bfa V_1^{(1)}\bfa H&\times\sigma((\bfa Q_1^{(1)}\bfa H)^\top(\bfa K_1^{(1)}\bfa H))=\bfa V_1\bfa H\times\begin{bmatrix}
        \bfa v_1 &\ldots&\bfa v_1
    \end{bmatrix}=\bfa V_1\times \begin{bmatrix}
        &\bfa A_0&\\
        \bfa v_k &\ldots&\bfa v_k\\
        &\bfa A_1&
    \end{bmatrix}\\
    &=\begin{bmatrix}
        \tda V_{1,1}^{(1)}&\tda V_{1,2}^{(1)}&\tda V_{1,3}^{(1)}
    \end{bmatrix}\begin{bmatrix}
        &\bfa A_0&\\
        \bfa v_k&\ldots&\bfa v_k\\
        &\bfa A_1&
    \end{bmatrix}=\tda V_{1,2}^{(1)}\begin{bmatrix}
        \bfa v_k &\ldots&\bfa v_k
    \end{bmatrix}\\
    &=\begin{bmatrix}
        \bfa 0_{3d+k}\\
        I_k\\
        \bfa 0
    \end{bmatrix}\begin{bmatrix}
        \bfa v_k&\ldots&\bfa v_k
    \end{bmatrix}=\begin{bmatrix}
        &\bfa 0_{3d+k}&\\
        \bfa v_k&\ldots&\bfa v_k\\
        &\bfa 0&
    \end{bmatrix},
\end{align*}
 where $\bfa A_0\in\bb R^{2d\times N}$ and $
    \bfa v_{k,\ell}=\sum_{i=1}^N\mbbm 1_{\wh z_i^{(0)}=\ell}$.
Then it is checked that
\begin{align*}
   \bfa H_2 =  \bfa H_1+ \bfa V_1^{(1)}\bfa H_1\times\sigma\lef((\bfa Q_1^{(1)}\bfa H_1)^\top(\bfa K_1^{(1)}\bfa H_1)\rig)=\begin{bmatrix}
        \bfa x_1&\bfa x_2&\ldots&\bfa x_N\\
        \wha \mu_{\wh z^{(0)}(1)}^{(0)}&\wha \mu_{\wh z^{(0)}(2)}^{(0)}&\ldots&\wha \mu_{\wh z^{(0)}(N)}^{(0)}\\
        \bfa p_{1,1}&\bfa p_{1,2}&\ldots&\bfa p_{1,N}\\
        \bfa p_{2,1}&\bfa p_{2,2}&\ldots&\bfa p_{2,N}\\
        1&1&\ldots&1\\
        \bfa v_k&\bfa v_k&\ldots&\bfa v_k\\
        \bfa p_{4,1}& \bfa p_{4,2}&\ldots &\bfa p_{4,N}
    \end{bmatrix}.
\end{align*}
Therefore, we further construct the following multi-head layer to remove the off-diagonal elements in $\begin{bmatrix}
    \bfa v_k&\bfa v_k&\ldots&\bfa v_k
\end{bmatrix}$, given by 
\begin{align*}
    \bfa V_i^{(2)} = \begin{bmatrix}
        &\bfa 0_{(3d+2k+i)\times D}&\\
        \bfa 0_{1\times(3d+2k+i)}&1&\bfa 0\\
        &\bfa 0&
    \end{bmatrix},\quad \bfa Q_1^{(2)} = \begin{bmatrix}
        b
    \end{bmatrix},\quad\bfa K_1^{(2)} = \begin{bmatrix}
        c
    \end{bmatrix},\qquad\text{ for }i\in[k].
\end{align*}
Given this formulation, we can show that
\begin{align*}
    \sigma((\bfa Q_i^{(2)}\bfa H_2)^\top(\bfa K_i^{(2)}\bfa H_2)) = \begin{bmatrix}
        &\bfa 0_{(i-1)\times N}&\\
        \bfa 0_{1\times(i-1)}&1&\bfa 0\\
        &\bfa 0&
    \end{bmatrix}.
\end{align*}
Hence, we can further show that
\begin{align*}
    \bfa V_{i}^{(2)}\bfa H_2\sigma((\bfa Q_i^{(2)}\bfa H_2)^\top(\bfa K_i^{(2)}\bfa H_2))=\begin{bmatrix}
        &\bfa 0_{(3d+2k+i)\times D}&\\
        \bfa 0_{(i-1)}&\bfa v_{k,i}&\bfa 0\\
        &\bfa 0&
    \end{bmatrix},
\end{align*}
which immediately implies that
\begin{align*}
    \sum_{i=1}^k\bfa V_i^{(2)}\bfa H_2\sigma\lef((\bfa Q_i^{(2)}\bfa H_2)^\top(\bfa K_i^{(2)}\bfa H_2)\rig)=\begin{bmatrix}
        \bfa 0_{(3d+2k)\times N}&\\
        \diag(\bfa v_k)&\bfa 0\\
        \bfa 0
    \end{bmatrix}.
\end{align*}
Given the above design, we can show that
\begin{align*}
    \bfa H_{3,1}=\bfa H_2+\sum_{i=1}^k\bfa V_i^{(2)}\bfa H_2\sigma\lef((\bfa Q_i^{(2)}\bfa H_2)^\top(\bfa K_i^{(2)}\bfa H_2)\rig)=\begin{bmatrix}
        \bfa x_1&\bfa x_2&\ldots&\bfa x_N\\
        \wha \mu_{\wh z^{(0)}(1)}^{(0)}&\wha \mu_{\wh z^{(0)}(2)}^{(0)}&\ldots&\wha \mu_{\wh z^{(0)}(N)}^{(0)}\\
        \bfa p_{1,1}&\bfa p_{1,2}&\ldots&\bfa p_{1,N}\\
        \bfa p_{2,1}&\bfa p_{2,2}&\ldots&\bfa p_{2,N}\\
        1&1&\ldots&1\\
        \bfa v_k&\bfa v_k&\ldots&\bfa v_k\\
        \diag(\bfa v_k)&&\bfa 0&\\
        &\bfa 0&&
    \end{bmatrix}.
\end{align*}
Then, we construct the MLP layer to remove the $\bfa v_k$ part, which is designed by
\begin{align*}
    \bfa W_1^{(2)}=I_D,\quad \bfa W_2^{(2)}= \begin{bmatrix}
        &\bfa 0_{(3d+k)\times D}&&\\
        \bfa 0_{k\times (3d+k)}&-I_k&I_k&\bfa 0_{}\\
       \bfa 0 &-I_k&\bfa 0&\bfa 0\\
       &\bfa 0&&
    \end{bmatrix}.
\end{align*}
Given this formulation, we can show that
\begin{align*}
    \bfa H_3 := \bfa H_{3,1}+\bfa W_1^{(2)}\sigma\lef(\bfa W_2^{(2)}\bfa H_{3,1}\rig)=\begin{bmatrix}
        \bfa x_1&\bfa x_2&\ldots&\bfa x_N\\
        \wha \mu_{\wh z^{(0)}(1)}^{(0)}&\wha \mu_{\wh z^{(0)}(2)}^{(0)}&\ldots&\wha \mu_{\wh z^{(0)}(N)}^{(0)}\\
        \bfa p_{1,1}&\bfa p_{1,2}&\ldots&\bfa p_{1,N}\\
        \bfa p_{2,1}&\bfa p_{2,2}&\ldots&\bfa p_{2,N}\\
        1&1&\ldots&1\\
        \bfa v_k&\bfa v_k&\ldots&\bfa v_k\\
        \diag(\bfa v_k)&&\bfa 0&\\
        &\bfa 0&&
    \end{bmatrix}.
\end{align*}

% Then we consider the following attention head which essentially remove the $\bfa v_k$ part in $\bfa H_2$. We note that the rank of $\sigma((\bfa Q_i\bfa H_2)^\top(\bfa K_i\bfa H_2))$ is at most $D$. To form the 
% \begin{align*}
%     \bfa V_{k+1}^{(2)} = \begin{bmatrix}
%         &\bfa 0_{(3d+k)\times D}&\\
%         \bfa 0_{k\times (3d+k)}&-I_k&\bfa 0_{}\\
%         &\bfa 0&
%     \end{bmatrix},\quad \bfa Q_{k+1}^{(2)}=,\quad \bfa K_{k+1}^{(2)}=
% \end{align*}
% Given the above design, we can show that
% \begin{align*}
%     \bfa V_{k+1}^{(2)}\bfa H_2\sigma\lef((\bfa Q_{k+1}^{(2)}\bfa H_2)^\top(\bfa K_i^{(2)}\bfa H_2)\rig)=\begin{bmatrix}
%         &\bfa 0_{(3d+k)\times N}&\\
%         -\bfa v_k&\ldots&-\bfa v_k\\
%         &\bfa 0&
%     \end{bmatrix}.
% \end{align*}
% Therefore, collecting pieces, we can show that
% \begin{align*}
%    \bfa H_3:= \bfa H_2 + \sum_{i=1}^{k+1}\bfa V_i^{(2)}\bfa H_2\sigma\lef((\bfa Q_i^{(2)}\bfa H_2)^\top(\bfa K_i^{(2)}\bfa H_2)\rig)=\begin{bmatrix}
%         \bfa x_1&\bfa x_2&\ldots&\bfa x_N\\
%         \wha \mu_{\wh z^{(0)}(1)}^{(0)}&\wha \mu_{\wh z^{(0)}(2)}^{(0)}&\ldots&\wha \mu_{\wh z^{(0)}(N)}^{(0)}\\
%         \bfa p_{1,1}&\bfa p_{1,2}&\ldots&\bfa p_{1,N}\\
%         \bfa p_{2,1}&\bfa p_{2,2}&\ldots&\bfa p_{2,N}\\
%         1&1&\ldots&1\\
%         \diag(\bfa v_k)&&\bfa 0&\\
%         \bfa p_{4,1}& \bfa p_{4,2}&\ldots &\bfa p_{4,N}
%     \end{bmatrix}.
% \end{align*}
The following layer converts the term $\diag(\bfa v_k)$ to $\diag(\bfa v_k^\prime)$ where $\bfa v_{k,i}^\prime = 1/\bfa v_{k,i}$. The design is given as follows
\begin{align*}
    \bfa V_{i}^{(3)} &= \begin{bmatrix}
        &\bfa 0_{(3d+3k+1)\times D}&\\
        \bfa 0_{k\times(2d+2k)}&\diag(c_i)_{k\times k}&\bfa 0\\
        &\bfa 0&
    \end{bmatrix},\quad\bfa Q_{i}^{(3)}=\begin{bmatrix}
        &\bfa 0_{(3d+3k+1)\times D}&\\
        \bfa 0_{k\times(2d+2k)} &I_{k}&\bfa 0\\
        &\bfa 0&
    \end{bmatrix},\\
    \bfa K_{i}^{(3)}&=\begin{bmatrix}
        &\bfa 0_{(3d+3k+1)\times D}&\\
        \bfa 0_{k \times (3d+3k+1)}&\diag(a_i)_{k\times k}&\bfa 0\\
        &\bfa 0&
    \end{bmatrix}\begin{bmatrix}
        &\bfa 0_{(3d+3k+1)\times D}&\\
       \bfa 0_{k\times(3d+2k+1)}&I_{k} &\bfa 0\\
       &\bfa 0&
    \end{bmatrix},
\end{align*}
where we show in lemma \ref{reluapprox} that for $M>\frac{1}{\epsilon^2}\log(1+C/\epsilon)$, there exists $\{a_i\}_{i\in[M]}$ such that for $x>1$, we have 
\begin{align*}
    \bigg\Vert\sum_{i=1}^Mc_i\sigma(a_ix) - \frac{1}{x}\bigg\Vert_2\leq\epsilon.
\end{align*}
And when $x=0$, we automatically obtain that 
$   \sum_{i=1}^Mc_i\sigma(0) =0$. We then immediately obtain that
\begin{align*}
    \bfa H_{4,1}:&=\bfa H_3 + \sum_{i=1}^M\bfa V_i^{(3)}\sigma\lef((\bfa Q_i^{(3)}\bfa H_3)^\top(\bfa K_i^{(3)}\bfa H_3)\rig)= \begin{bmatrix}
        \bfa x_1&\bfa x_2&\ldots&\bfa x_N\\
        \wha \mu_{\wh z^{(0)}(1)}^{(0)}&\wha \mu_{\wh z^{(0)}(2)}^{(0)}&\ldots&\wha \mu_{\wh z^{(0)}(N)}^{(0)}\\
        \bfa p_{1,1}&\bfa p_{1,2}&\ldots&\bfa p_{1,N}\\
        \bfa p_{2,1}&\bfa p_{2,2}&\ldots&\bfa p_{2,N}\\
        1&1&\ldots&1\\
        \bfa v_k&\bfa v_k&\ldots&\bfa v_k\\
        \diag(\bfa v_k)&&\bfa 0&\\
        \diag(\bfa v_k^\prime)&&\bfa 0&\\
        &\bfa 0&&
    \end{bmatrix}+O_{2}(\epsilon),
\end{align*}
where $\bfa v_{k,i}^\prime = \bfa v_{k,i}^{-1}$. Then we apply the MLP again with the following design 
\begin{align*}
    \bfa W_1^{(4)}= I_D,\quad\bfa W_2^{(4)}=\begin{bmatrix}
        &\bfa 0_{(3d+k)\times D}&&\\
        \bfa 0_{k\times(3d+k)}&-I_k&I_k&\bfa 0\\
        \bfa 0&-I_k&\bfa 0&\bfa 0\\
        &\bfa 0&&
    \end{bmatrix}.
\end{align*}
The above construction implies that
\begin{align*}
    \bfa H_4 =\bfa W_2^{(3)}\sigma\lef(\bfa W_1^{(3)}\bfa H_3\rig) = \begin{bmatrix}
        \bfa x_1&\bfa x_2&\ldots&\bfa x_N\\
        \wha \mu_{\wh z^{(0)}(1)}^{(0)}&\wha \mu_{\wh z^{(0)}(2)}^{(0)}&\ldots&\wha \mu_{\wh z^{(0)}(N)}^{(0)}\\
        \bfa p_{1,1}&\bfa p_{1,2}&\ldots&\bfa p_{1,N}\\
        \bfa p_{2,1}&\bfa p_{2,2}&\ldots&\bfa p_{2,N}\\
        1&1&\ldots&1\\
        \bfa v_k&\bfa v_k&\ldots&\bfa v_k\\
        \diag(\bfa v_k^\prime)&&\bfa 0&\\
        &\bfa 0&&
    \end{bmatrix}.
\end{align*}
We construct the following layer to perform the normalization, given by 
\begin{align*}
    \bfa V_2^{(4)}&=-\bfa V_{1}^{(4)}= \begin{bmatrix}
    &\bfa 0_{(3d+2k+1)\times D}& \\
    \bfa 0_{k\times(3d+k+1)}&I_{k}&\bfa 0\\
    &\bfa 0&
    \end{bmatrix},\quad \bfa Q_1^{(4)} =-\bfa Q_2^{(4)}= \begin{bmatrix}
        &\bfa 0_{(3d+2k+1)\times D}&\\
        \bfa 0_{k\times 3d}&I_{k}&\bfa 0\\
        &\bfa 0&
    \end{bmatrix},\\
    \bfa K_1^{(4)}&=\bfa K_2^{(4)}=\begin{bmatrix}
        &\bfa 0_{(3d+2k+1)\times D}&\\
        \bfa 0_{k\times (3d+1)}&I_k&\bfa 0\\
        &\bfa 0&
    \end{bmatrix}.
\end{align*}
Then we can show that
\begin{align*}
    \sigma\lef((\bfa Q_1^{(4)}\bfa H_4)^\top(\bfa K_1^{(4)}\bfa H_4)\rig)-\sigma\lef((\bfa Q_2^{(4)}\bfa H_4)^\top(\bfa K_2^{(4)}\bfa H_4)\rig) = \begin{bmatrix}
        &\bfa 0_{(3d+2k+1)\times D}&\\
        \bfa p_{1,1}&\ldots&\bfa p_{1,N}\\
        &\bfa 0
    \end{bmatrix}.
\end{align*}
And we also have
\begin{align*}
    \bfa V_2^{(4)}\bfa H_4 = \begin{bmatrix}
        &\bfa 0_{(3d+3k+1)\times D}&\\
        \bfa 0_{k\times(3d+2k+1)}&\diag(\bfa v_k^\prime) &\bfa 0\\
        &\bfa 0&
\end{bmatrix}+O_2(\epsilon),
\end{align*}
which implies that
\begin{align*}
    \bfa H_{4,1} = \bfa H_3+\sum_{i=1}^2\bfa V_i^{(4)}\bfa H_3\times\sigma\lef((\bfa Q_i^{(4)}\bfa H_3)^\top(\bfa K_i^{(3)}\bfa H_3)\rig)=\begin{bmatrix}
        \bfa x_1&\bfa x_2&\ldots&\bfa x_N\\
        \wha \mu_{\wh z^{(0)}(1)}^{(0)}&\wha \mu_{\wh z^{(0)}(2)}^{(0)}&\ldots&\wha \mu_{\wh z^{(0)}(N)}^{(0)}\\
        \bfa p_{1,1}&\bfa p_{1,2}&\ldots&\bfa p_{1,N}\\
        \bfa p_{2,1}&\bfa p_{2,2}&\ldots&\bfa p_{2,N}\\
        1&1&\ldots&1\\
        \diag(\bfa v_k^\prime)&&\bfa 0&\\
        \bfa p_{1,1}^\prime&\bfa p_{1,2}^\prime&\ldots&\bfa p_{1,N}^\prime
    \end{bmatrix}+O_{2}(\epsilon),
\end{align*}
where $\bfa p_{1,i}^\prime = \diag(\bfa v_k^\prime)\bfa p_{1,i}$ for all $i\in[N]$. We therefore construct an MLP layer to replace the $\bfa p_{1}$ part using the following design
\begin{align*}
    \bfa W_1^{(4)}=I_D,\qquad\bfa W_2^{(4)} = \begin{bmatrix}
        &\bfa 0_{2d\times D}&&\\
        \bfa 0_{k\times 2d}&-I_k&\bfa 0&I_k&\bfa 0\\
        \bfa 0&\bfa 0&\bfa 0&\bfa 0&\bfa 0\\
        \bfa 0_{k\times 2d}&-I_k&\bfa 0&\bfa 0 &\bfa 0\\
    \end{bmatrix},
\end{align*}
which ultimately leads to 
\begin{align*}
    \bfa H_4 = \bfa W_1^{(3)}\sigma\lef(\bfa W_2^{(3)}\bfa H_3\rig) = \begin{bmatrix}
        \bfa x_1&\bfa x_2&\ldots&\bfa x_N\\
        \wha \mu_{\wh z^{(0)}(1)}^{(0)}&\wha \mu_{\wh z^{(0)}(2)}^{(0)}&\ldots&\wha \mu_{\wh z^{(0)}(N)}^{(0)}\\
        \bfa p^\prime_{1,1}&\bfa p^\prime_{1,2}&\ldots&\bfa p^\prime_{1,N}\\
        \bfa p_{2,1}&\bfa p_{2,2}&\ldots&\bfa p_{2,N}\\
        1&1&\ldots&1\\
        \diag(\bfa v_k^\prime)&&\bfa 0&\\
        &\bfa 0&&
    \end{bmatrix}.
\end{align*}
We can further perform the expectation step by the following layer
\begin{align*}
    \bfa V_i^{(4)}=\begin{bmatrix}
        &\bfa 0_{(3d+2k+1)\times D}&\\
        \bfa 0_{k\times2d}& I_k&\bfa 0\\
        &\bfa 0&
    \end{bmatrix},\quad\bfa Q_i^{(4)}=\begin{bmatrix}
        \bfa 0&\bfa 0&\bfa 0\\
        \bfa 0_{k\times2d}& I_k&\bfa 0\\
        \bfa 0&\bfa 0&\bfa 0
    \end{bmatrix},\quad\bfa K_i^{(4)}=\begin{bmatrix}
        \bfa 0&\bfa 0&\bfa 0\\
        \bfa 0&\bfa 0&\bfa 0\\
        \bfa 0&\bfa 0&\bfa 0
    \end{bmatrix}
\end{align*}
Then it is further noted that
\begin{align*}
    \bfa H_5 =\bfa H_4+\sum_{i=1}^2\bfa V_i^{(4)}
\end{align*}
\end{proof}
% The rest of this section presents a proof sketch on theorem \ref{thm3.1}.
% \paragraph{Proof Sketch of Theorem \ref{thm3.1}.}
%     The proof sketch for theorem \ref{thm3.1} goes by two important steps:



%     To obtain $\tda p_{3,i}$ for $i\in[k]$, we generate through the probablistic method, which gives the following lemma.
%     % \begin{lemma}
%     %     Let $\bfa y\in\bb R^d$ be a random vector with isotropic Gaussian as its probability density. Consider $\bfa x=\frac{\bfa y}{\Vert\bfa y\Vert_2}$. Let $\bfa v$ be any unit length vector, then we have
%     %        $$ \bb P\bl|\bfa v^\top\bfa x|\leq\frac{\epsilon}{\sqrt d}\br\leq  \frac{1}{\sqrt{\pi}}\epsilon+\exp\lef(-Cd\rig).$$
%     %     Therefore, the event in theorem \ref{thm3.1} can be estimated as
%     %     \begin{align*}
%     %          \bb P\bl\exists i\text{ such that }&\bfa x_i^\top\bfa v_i\leq\frac{\epsilon}{\sqrt d}\br\leq 1-\bl1-\frac{\epsilon}{\sqrt{\pi}}-\exp(-Cd)\br^k\leq \frac{k\epsilon}{\sqrt{\pi}}+k\exp(-Cd).
%     %     \end{align*}
%     % \end{lemma}

%     % \begin{align*}
%     %    \Bigg\Vert \tda H^{rpe,2}-\begin{bmatrix}
%     %         \bfa X\\
%     %         \tda y\\
%     %         \bfa X\bfa X^\top,\bfa 0\\
%     %         \tda p_{2,1},\ldots,\tda p_{2,N}\\
%     %         \tda p_{3,1},\ldots,\tda p_{3,N}\\
%     %         \tda p_{3,1}^{(\tau)},\bfa 0\\
%     %         \Vert\bfa X\bfa X^\top\tda p_{3,N}^{(\tau)}\Vert_2,\bfa 0
%     %     \end{bmatrix}\Bigg\Vert_2\leq\epsilon \Vert\bfa X\bfa X^\top\tda p_{3,1}\Vert_2.
%     % \end{align*}
%     % Then we can show that
%     % \begin{align*}
%     %     &\Bigg\Vert\bfa K\tda H^{pow,2\tau+2}-\begin{bmatrix}
%     %         \bfa 0_{(5d+1)\times(4d+1)}&\bfa 0_{(5d+1)\times 1}&\bfa 0\\
%     %         \bfa 0_{d\times(4d+1)}&\tda p_{3,1}^{(\tau)}&\bfa 0\\
%     %         \bfa 0&\bfa 0&\bfa 0
%     %     \end{bmatrix}\Bigg\Vert_2\leq \tau\epsilon\Vert\bfa X\bfa X^\top\Vert_2,\\ 
%     %     &\Bigg\Vert\bfa Q\tda H^{pow,2\tau+2}-\begin{bmatrix}
%     %         \bfa 0_{(5d+1)\times 1}&\bfa 0\\
%     %        \Vert\bfa X\bfa X^\top\tda p_{3,1}^{(\tau)}\Vert_2 \tda p_{3,1}^{(\tau)}&\bfa 0\\
%     %         \bfa 0&\bfa 0
%     %     \end{bmatrix}\Bigg\Vert_2\leq\tau\epsilon\Vert\bfa X\bfa X^\top\Vert_2.
%     % \end{align*}
%     % which immediately implies that
%     % \begin{align*}
%     %     \Bigg\Vert(\bfa K\tda H^{pow,2\tau+2})^\top(\bfa Q\tda H^{pow,2\tau+2})-\begin{bmatrix}
%     %         \bfa 0&\bfa 0\\
%     %         \Vert\bfa X\bfa X^\top\tda p_{3,1}\Vert_2\tda p_{3,1}^{(\tau)}\tda p_{3,1}^{(\tau),\top}&\bfa 0\\
%     %         \bfa 0&\bfa 0
%     %     \end{bmatrix}\Bigg\Vert_2
%     % \end{align*}
% \begin{lemma}\label{lm3.2}
% Assume that the correlation matrix $\bfa X\bfa X^\top$ has eigenvalues $\lambda_1>\lambda_2>\ldots>\lambda_{k}$. Assume that the eigenvectors are given by $\bfa v_1,\bfa v_2,\ldots,\bfa v_n$ and the eigenvalues satisfy $\inf_{i\neq j}|\lambda_i-\lambda_j|=\Delta$. Then, given that the estimate for the first $\tau$ eigenvectors satisfy
% $    \bfa v_i^\top\wha v_i\geq 1-\epsilon_i $ and the eigenvalues satisfy $|\lambda_i-\wh\lambda_i|\leq\delta_i$, the principle eigenvector of $\bfa X\bfa X^\top-\sum_{i=1}^{\tau}\wh\lambda_i\wha v_i\wha v_i^\top$ denoted by $\tda v_{\tau+1}$ satisfies
% \begin{align*}
%     \Vert\tda v_{\tau+1}-\bfa v_{\tau+1}\Vert_2 \leq\frac{\max_{i\in[\tau]}\delta_i+\sum_{i=1}^\tau\sqrt8\lambda_i\sqrt{\epsilon_i}}{\Delta}.
% \end{align*}
% Alternatively, we can also show that the eigenvector $\wha v_{\tau+1}$ returned by power method with $k= \frac{\log(1/\epsilon_0\delta)}{2\epsilon_0}$ that is initialized by  satisfies
% \begin{align*}
%     \wha v_{\tau+1}^\top\bfa v_{\tau+1} \geq 1-\epsilon_{\tau+1}:=  1-\frac{1}{2}\Big(\frac{\max_{i\in[\tau]}\delta_i+\sum_{i=1}^{\tau}\sqrt{8}\lambda_i\sqrt{\epsilon_i}}{\Delta}+\sqrt{2\epsilon_0}\Big)^2,
% \end{align*}
% and alternatively we have $\Vert\wha v_{\tau+1}-\bfa v_{\tau+1}\Vert_2\leq\sqrt{2\epsilon_{\tau+1}}$. Furthermore, consider the eigenvalue estimate, we show that
% \begin{align*}
%    \Big|\Big\Vert\Big(\bfa X\bfa X^\top-\sum_{i=1}^{\tau}\wh\lambda_i\wha v_i\wha v_i^\top\Big)\wha v_{\tau+1}\Big\Vert_2-\lambda_{\tau+1}\Big|\leq \frac{2\lambda_{\tau+1}}{\Delta}\Big(\max_{i\in[\tau]}\delta_i+\sum_{i=1}^{\tau}\sqrt{8}\lambda_i\sqrt{\epsilon_i}\Big)+\lambda_{\tau+1}\sqrt{2\epsilon_0}
% \end{align*}
% \end{lemma}

% \begin{lemma}[Approximation of norm by sum of Relu activations by Transformer networks]\label{reluapprox}
%     Assume that there exists a constant $C$ with $\Vert\bfa v\Vert_2\leq C$. There exists a multihead Relu attention layer with number of heads $M<\lef(\frac{\overline R}{\underline R}\rig)^d\frac{C(d)}{\epsilon^2}\log(1+C/\epsilon)$ such that there exsits $\{\bfa a_m\}_{m\in[M]}\subset\bb S^{N-1}$ and $\{c_m\}_{m\in[M]}\subset\bb R$ where for all $\bfa v$ with $\overline R\geq\Vert\bfa v\Vert_2\geq \underline R$, we have 
%     \begin{align*}
%         \bigg|\sum_{m=1}^Mc_m\sigma(\bfa a_m^\top\bfa v)-\frac{1}{\Vert\bfa v\Vert_2}+1\bigg|\leq\epsilon.
%     \end{align*}
%     Similarly, there exists a multihead Relu attention layer with number of heads $M\leq\overline R^{\frac{d}{2}}\frac{C(d)}{\epsilon^2}\log\lef(1+C/\epsilon\rig)$, a set of vectors $\{\bfa b_m\}_{m\in[M]}\subset\bb S^{N-1}$ and $\{d_m\}_{m\in[M]}\subset\bb R$ such that 
%     \begin{align*}
%         \Big|\sum_{m=1}^Md_m\sigma(\bfa b_m^\top\bfa v)-\Vert\bfa v\Vert_2^{1/2}+1\Big|\leq\epsilon.
%     \end{align*}
% \end{lemma}




% \vspace{-1em}
\section{Conclusion}
In this paper, we investigate the theoretical foundations of transformers as time series foundation models.
First, we show that there exists a multi-layer transformer capable of performing least squares regression on any input uni-variate time series.
Next, when considering MOIRAI, we demonstrate the existence of a multi-layer MOIRAI that adapts to the dimensionality $d$ of the input (i.e., the number of covariates) and fits different autoregressive models based on $d$.
When the data is generated by an autoregressive process, such a transformer benefits from its prediction error being exponentially suppressed by the number of layers.
We then establish a generalization bound for pretraining when the data satisfies Dobrushin's condition.
When the pretraining data is sampled from $\mathtt{AR}$ processes, we derive a more explicit bound on the test loss, with a trade-off controlled by the number of layers.
Our analysis not only provides the first theoretical justification for the design and performance of MOIRAI but also represents the first theoretical framework for constructing a time series foundation model.
% \vspace{-0.5em}

\paragraph{Limitations.}
One limitation in our analysis is that we consider ReLU instead of softmax in attention mechanisms.
While the same approach also is in theoretical \cite{bai2024transformers, lin2023transformers, he2025learning} and empirical works \cite{wortsman2023replacing, zhang2021sparse, shen2023study}, one might obtain a different approximation bound comparing to \cref{thm:any-variate-auto}.
However, in our generalization analysis, the difference is small as softmax does not affect the model complexity too much.
Another aspect is that we mainly focus on $\mathtt{AR}$ processes.
While in the appendix, we do show the approximation result for non-linear $\mathtt{AR}$ processes generated by a ReLU network, to achieve universal forecasting, a more general assumption on data is required.

\paragraph{Impact Statement.}
This paper studies the theoretical aspect of transformers as time series foundation models.
No negative societal impacts that the authors feel should be specifically highlighted here.


\bibliography{ref}
\bibliographystyle{icml2025}


%%%%%%%%%%%%%%%%%%%%%%%%%%%%%%%%%%%%%%%%%%%%%%%%%%%%%%%%%%%%%%%%%%%%%%%%%%%%%%%
%%%%%%%%%%%%%%%%%%%%%%%%%%%%%%%%%%%%%%%%%%%%%%%%%%%%%%%%%%%%%%%%%%%%%%%%%%%%%%%
% APPENDIX
%%%%%%%%%%%%%%%%%%%%%%%%%%%%%%%%%%%%%%%%%%%%%%%%%%%%%%%%%%%%%%%%%%%%%%%%%%%%%%%
%%%%%%%%%%%%%%%%%%%%%%%%%%%%%%%%%%%%%%%%%%%%%%%%%%%%%%%%%%%%%%%%%%%%%%%%%%%%%%%
\newpage

\bigskip
\onecolumn

% \begin{appendices}
\begin{center}
{\large\bf SUPPLEMENTARY MATERIAL}
\end{center}

\appendix

{
\setlength{\parskip}{-0em}
\startcontents[sections]
\printcontents[sections]{ }{1}{}
}




% \begin{itemize}
%     \item[\ref{sec:tab_notation}] \hyperref[sec:tab_notation]{Table of Notations}\dotfill \pageref{sec:tab_notation}

%     \item[\ref{sec:related-work}] \hyperref[sec:related-work]{Related Works} \dotfill \pageref{sec:related-work}
    
%     \item[\ref{sec:additional-theory}] \hyperref[sec:additional-theory]{Additional Theoretical Background} \dotfill \pageref{sec:additional-theory}

%     \item[\ref{sec:proofs}] \hyperref[sec:proofs]{Proof of Main Context} \dotfill \pageref{sec:proofs}
    
%     \begin{itemize}
%         \item[$\quad$\ref{proof:lem-input-casual}] \hyperref[proof:lem-input-casual]{Proof of \cref{lem:input-causal}} \dotfill \pageref{proof:lem-input-casual}

%         \item[$\quad$\ref{proof:any-var-enc}] \hyperref[proof:lem-input-casual]{Proof of \cref{lem:mar-group-wise}} \dotfill \pageref{proof:any-var-enc}

%         \item[$\quad$\ref{proof:any-var-enc}] \hyperref[proof:any-var-enc]{Proof of \cref{thm:any-variate-auto}} \dotfill \pageref{proof:any-var-enc}

%         \item[$\quad$\ref{proof:tr-lipschitz}] \hyperref[proof:tr-lipschitz]{Proof of \cref{proposition:lipschitz-moirai}} \dotfill \pageref{proof:tr-lipschitz}

%         \item[$\quad$\ref{proof:gen-bound-1}] \hyperref[proof:gen-bound-1]{Proof of \cref{thm:gen-bound-1}} 
%         \dotfill \pageref{proof:gen-bound-1}

%         \item[$\quad$\ref{appendix:analysis-ar1}] \hyperref[appendix:analysis-ar1]{Proof of \cref{proposition:ar1}}
%         \dotfill \pageref{appendix:analysis-ar1}
        
%     \end{itemize}

%     \item[\ref{sec:exp-details}] \hyperref[sec:exp-details]{Experimental Details} \dotfill \pageref{sec:exp-details}

    
% \end{itemize}




\section{Table of Notations}
\label{sec:tab_notation}

\begin{table}[h]
    \caption{Mathematical Notations and Symbols}
    \centering
    % \resizebox{ \textwidth}{!}{ 
    \begin{tabular}{cl}
    \toprule
        Symbol & Description \\
    \midrule
        $\bx_i$ & The $i$-th component of vector $\bx$ \\
        $\Braket{\ba,\bb}$ & Inner product for vectors $\ba,\bb\in \R^d$ \\
        $[I]$ & Index set $\{1,\cdots,I\}$, where $I\in\mathbb{N}^+$ \\
        $\norm{\cdot}$ & Spectral norm, equivalent to the $l_2$-norm when applied to a vector \\
        $\norm{\cdot}_{2, \infty}$ & The largest L2 norm of column vectors of a matrix \\
        $\bA_{ij}$ & The element on the $i$-th row and $j$-th column of matrix $\bA$ \\
        $\bx_{i:j}$ & The sub-sequence of sequence $\bx$ from coordinate $i$ to $j$ \\
        $\oplus$ & Concatenation between column vectors $\bv \oplus \bu \mapsto ( \bv^\top, \bu^\top)^\top$ \\
        $[\bu ; \bv]$ & Concatenation between two row vectors \\
    \midrule
        $N$ & Length of a transformer input sequence \\
        $T$ & Number of time steps of a time series \\
        $M$ & Number of attention heads. \\
        $q$ & Lag of an $\mathtt{AR}$ process. \\
        $d$ & The number of covariates in an $\mathtt{AR}$ process \\
    \midrule
        $\bv$ & Vector (bold lower) \\
        $\bA$ & Matrix (bold upper) \\
        $\cX$ & random variable (calligraphic) \\
        $\text{x}$ & element from a domain set \\
        $\cD_{\cX}$ & Domain of random variable $\cX$ \\
        $\be_i$ & one-hot vector with its $i$-th entry as 1 \\
    \midrule
        $\mathtt{P}_{\cX}$ & Probability distribution of $\cX$ \\
        $P_{\bz | \bw}(z \mid w )$ & The probability $P \left[ \bz = z \mid \bw = w \right]$ \\
    \bottomrule
    \end{tabular}
    % }
     \label{tab:nomenclature}
\end{table}

\clearpage

\section{Simulations}\label{sect4}

\begin{figure}[!h]
% \vspace{-1em}
\minipage{0.48\textwidth}
    \minipage{0.48\textwidth}
        \includegraphics[width=\linewidth]{charts/multiclass/multiclass_synthetic_dist_metric_softmax.pdf}
    \endminipage\hfill
    \minipage{0.48\textwidth}
        \includegraphics[width=\textwidth]{charts/multiclass/multiclass_synthetic_dist_loss_softmax.pdf}
    \endminipage\hfill
\endminipage\hfill
\minipage{0.48\textwidth}
    \minipage{0.48\textwidth}
        \includegraphics[width=\textwidth]{charts/multiclass/multiclass_synthetic_dim_metric_new_softmax.pdf}
    \endminipage\hfill
    \minipage{0.48\textwidth}
        \includegraphics[width=\textwidth]{charts/multiclass/multiclass_synthetic_dim_loss_new_softmax.pdf}
    \endminipage\hfill
\endminipage\hfill
\minipage{0.48\textwidth}
    \minipage{0.48\textwidth}
        \includegraphics[width=\textwidth]{charts/multiclass/multiclass_synthetic_n_step_metric_softmax.pdf}
    \endminipage\hfill
    \minipage{0.48\textwidth}
        \includegraphics[width=\textwidth]{charts/multiclass/multiclass_synthetic_n_steps_loss_softmax.pdf}
    \endminipage\hfill
\endminipage
\vspace{-1em}
\caption{\textbf{$4$-Class Clustering with Different Minimum Distance, Data Dimension, and Number of Training Data.} 
% \red{Can move detials into appendix.}
We train a small Transformer (layer $=3$, head $=2$, embedding $= 64$) and iterate for $300$ steps for each different setting. Each point in the figure is evaluated on $512$ testing data. 
We report the $10$ runs averaged result with a shaded region representing the standard deviation.
Each training sample is generated according to isotropic Gaussian with covariances $\sigma^2 \bfa I$.
\emph{(1) First Row: Minimum Distance.} We set $\sigma^2 \sim \mathrm{Uniform}[10,40]$.
\emph{(2) Second Row: Data Dimension.} We set $\sigma^2 \sim \mathrm{Uniform}[10,20]$, minimum distance $=5$.
\emph{(3) Three Row: Number of Training Data.} 
We set $\sigma^2 \sim \mathrm{Uniform}[0.5,5]$, minimum distance $=5$.}
\label{fig:dist_dim_step}
% \vspace{-1em}
\end{figure}

\begin{figure}[!h]
% \vspace{-1em}
\minipage{0.48\textwidth}
    \minipage{0.48\textwidth}
        \includegraphics[width=\linewidth]{charts/multiclass/multiclass_synthetic_num_classes_metric_new_softmax.pdf}
    \endminipage\hfill
    \minipage{0.48\textwidth}
        \includegraphics[width=\textwidth]{charts/multiclass/multiclass_synthetic_num_classes_loss_softmax.pdf}
    \endminipage\hfill
\endminipage\hfill
\minipage{0.48\textwidth}
    \minipage{0.48\textwidth}
        \includegraphics[width=\textwidth]{charts/multiclass/multiclass_synthetic_ratio_metric_softmax.pdf}
    \endminipage\hfill
    \minipage{0.48\textwidth}
        \includegraphics[width=\textwidth]{charts/multiclass/multiclass_synthetic_ratio_loss_softmax.pdf}
    \endminipage\hfill
\endminipage
\vspace{-1em}
\caption{\textbf{$4$-Class Clustering with Different Number of Class and Inbalance Ratio.}
% \red{Can move detials into appendix.}
We train a small Transformer (layer $=3$, head $=2$, embedding $= 64$) and train for $300$ steps for each different setting. Each point in the figure is evaluated on $512$ testing data.
We report the $10$ runs averaged result with a shaded region representing the standard deviation.
Each training sample is generated according to isotropic Gaussian with covariances $\sigma^2 \bfa I$.
\emph{(1) First Row: Number of Class.} We set $\sigma^2 \sim \mathrm{Uniform}[10,20]$, minimum distance $=5$.
\emph{(2) Second Row: Inbalance Ratio.} Two clusters each contain 50 data points, while the other two contain $50 \times \mathrm{ratio}$ and $50 \times \mathrm{1-ratio}$ respectively. We set $\sigma^2 \sim \mathrm{Uniform}[10,20]$, minimum distance $=5$.}
\label{fig:inbalance}
% \vspace{-1em}
\end{figure}

% We set $\sigma^2 \sim \mathrm{Uniform}[10,20]$, minimum distance $=5$.}

\begin{figure}[!h]
% \vspace{-1em}
\minipage{0.48\textwidth}
    \minipage{0.48\textwidth}
        \includegraphics[width=\textwidth]{charts/multiclass/multiclass_synthetic_num_classes_metric_lloyd_more_layer.pdf}
    \endminipage\hfill
    \minipage{0.48\textwidth}
        \includegraphics[width=\textwidth]{charts/multiclass/multiclass_synthetic_num_classes_metric_lloyd_iter.pdf}
    \endminipage\hfill
\endminipage
\vspace{-1em}
\caption{\textbf{Comparision between Transformer and Lloyd's Algorithm.}
% \red{Can move detials into appendix.}
We compare the effect of the number of layers in Transformers with the number of iterations $\tau$ in Lloyd's algorithm under the same dataset configuration.  
We use a $6$-class dataset, where each cluster contains $50$ data points in a $d=10$ dimensional space.  
Each training sample is generated according to isotropic Gaussian with covariances $\sigma^2 \bfa I$, where $\sigma^2 \sim \mathrm{Uniform}[20,30]$, and the minimum cluster separation is set to $1$.  
\emph{(1) Left: Transformer.} 
We train Transformers with fixed head $=2$, embedding $= 64$, but vary the number of layers from $3$ to $20$.
Each model is trained for $500$ steps per layer.  
\emph{(2) Right: Lloyd's Algorithm.}
% We use ``sklearn`` to run the Lloyd algorithm, and vary the maximum iteration from $1$ to $6$.
We use \texttt{sklearn}\citep{pedregosa2011scikit} to run the Lloyd's algorithm, varying the maximum iteration count from $1$ to $6$.  
Early convergence is declared when the Frobenius norm of the difference between cluster centers in consecutive iterations falls below $10^{-4}$. 
Each point in the figure represents an evaluation of $512$ test samples.  
Results are averaged over $10$ runs, with the shaded region indicating the standard deviation.}
\label{fig:com_lloyd}
% \vspace{-1em}
\end{figure}

% \red{TODOs: Comparision of Lloyd baseline and Transformer.}
In this section, we verify our theoretical results on the multi-class clustering problem and examine its interplay with five key factors: the minimum distance between centroids $\Delta$, the data dimension $d$, the training sample size $N$, the total number of classes, and an imbalance ratio $\alpha$. These results are presented in  \cref{fig:dist_dim_step} and \cref{fig:inbalance}.
Furthermore, we compare the impact of a number of layers in the Transformer with the number of iterations $\tau$ in \cref{fig:com_lloyd}.

\paragraph{Experimental Setup}
We use a small Transformer with $3$ layers, $2$ heads, and $64$-dimensional embedding size.
All simulations are conducted on NVIDIA A100 80G GPUs. We run each experiment for 300 iterations, initialize the model with $10$ different random seeds, and report the mean and standard deviation of the resulting metrics. The model is trained using the Adam optimizer with a learning rate of $0.0005$ and an exponential decay factor of $0.995$ for each step.
After training, each configuration is evaluated on 512 synthetic and random test samples. Note that our empirical evaluation slightly differs from the theoretical part through removing the auxiliary matrix $\bfa P$ given by \eqref{aux} from the input.

\paragraph{Metrics.}
We compute cross entropy among every permutation of the label and choose the minimum as the loss function since clustering tasks are permutation invariant.
We evaluate the clustering performance using two widely adopted permutation-invariant metrics: Adjusted Rand Index (ARI) and Normalized Mutual Information (NMI) \citep{ma2019learning, huang2020partially, monnier2020deep, sun2024lsenet, li2024image}.


\paragraph{Preparation for the Synthetic Data.}
{We generate our synthetic data as follows: For each input $\bfa X \in \R^{d \times N}$, we sample $50$ data points from every cluster. 
Each sample is generated according to isotropic Gaussian with covariances $\sigma^2 \bfa I$.
The variance $\sigma$ differs from task to task; we specify more details in the caption of figures.}

\paragraph{Results}
Our results suggest that the theoretical threshold given by the minimax rate matches with the trend given in the experiments. Moreover, we also showcase that the pre-trained Transformers can be a strong alternative to Lloyd's algorithm, verifying the strong inference capacities of Transformers on this problem.



% In this section, we verify the theoretical result in section~\ref{sect3} on synthetic and real-world datasets. Our experiments include both prediction of eigenvalues and eigenvectors. 
% % Note that our goal is to quickly understand the influence of different parameters via experiments. 
% % We conduct our experiments on both synthetic and realworld datasets.
% For synthetic datasets, we generate samples according to normal distributions. We focus on evaluating the effects of three major parameters:
%     \textbf{(1)} The Impact of $D$;
%     \textbf{(2)} The Impact of Number of Layers;
%     \textbf{(3)} The Impact of $k_{\text{train}}$\footnote{We denote $k_{\text{train}}$ as the value of $k$ used in training}. For real-world datasets, we perform experiments on MNIST \citep{lecun1998gradient} and Fashion-MNIST \citep{xiao2017fashion}. \emph{All the results presented in this section are errors on the testing set}.


% % the experiments in this section don't aim to demonstrate the optimal performance of our proposed model can achieve, but rather to run various experiments to quickly understand the influence of different parameters.
% % To speed up the training process, we set the number of samples $N \leq 50$ for each input $X$ across all settings.
% % In the first two sections, we use a synthetic dataset to (1) demonstrate transformer's ability to perform PCA to get top-$k$ eigenvalues and eigenvectors, and (2) investigate the influence of input dimension $D$, number of layers and attention heads of transformer, and the number of $k$ principal component to obtain.
% % In the third section, we evaluate transformer's ability on perfoming PCA on 2 real-world datasets: MNIST \citep{lecun1998gradient} and Fashion-MNIST \citep{xiao2017fashion}.  
% % Six figures in the main text?
% % \begin{itemize}
% %     \item A graph with loss v.s. step, predict top-1 eigenvector, varying d.
% %     \item varying k.
% %     \item varying layer
% % \end{itemize}

% \paragraph{Data Preparation.}
% For synthetic data $\bfa X \in \R^{D \times N}$, we generate each column with a randomly initialized multivariate Gaussian distribution $\ca N(\mu, \Sigma)$.
% We then generate the labels as the top-$k$ eigenvalues $\lambda$ and eigenvectors $V$ of the empirical covariance matrix $\bfa X^\top \bfa X/(N-1)$ via \texttt{numpy.linalg.eigh}.
% For real world dataset, we apply SVD to reduce the dimensionality of both datasets and evaluate whether the transformer, previously trained on multivariate Gaussian data, are capable of performing PCA on those real-world datasets.
% If the transformer successfully learns to perform PCA, we expect comparable performance on both synthetic and real-world data. 
% % Note that we set $N < D$ in all our experiment setting to speed up the training process.
% For more details on data generation and configuration, please refer to table~\ref{table:data-param} in appendix~\ref{appendix:data}.
 
% \begin{table}[t]
% \centering
% \caption{\textbf{Cosine Similarity for Different $k$.}
% We dentoe $k_{\text{train}}$ as the number of eigenvectors to predict during training.
% For example, for $k_{\text{train}}=4$, the model is trained to predict 4 eigenvectors.
% }
% \begin{tabular}{lcccc}
% \toprule
% \textbf{k-th eigenvec.} & \textbf{k=1} & \textbf{k=2} & \textbf{k=3} & \textbf{k=4} \\ \hline
% $k_{\text{train}}=4$    & $0.891 (0.006)$ & $0.616 (0.038)$ & $0.282 (0.047)$ & $0.120 (0.022)$ \\
% $k_{\text{train}}=3$    & $0.908 (0.011)$ & $0.706 (0.023)$ & $0.366 (0.018)$ & - \\
% $k_{\text{train}}=2$   & $0.903 (0.006)$ & $0.647 (0.019)$ & - & - \\
% $k_{\text{train}}=1$   & $0.894 (0.009)$ & - & - & - \\
% \bottomrule
% \end{tabular}
% \label{tab:mul_k_vector}
% \end{table}

% \begin{figure}[t]
%     \centering
%     \minipage{0.33\textwidth}
%         \includegraphics[width=\linewidth]{charts/eigenvalue_multiple_k.pdf}
%         % \subcaption{Top-10 Eigenvalues}
%     \endminipage\hfill
%     \minipage{0.33\textwidth}
%         \includegraphics[width=\linewidth]{charts/eigenvalue_vary_d_rmse.pdf}
%         % \subcaption{Cosine Similarity with Varying d}
%     \endminipage\hfill
%     \minipage{0.33\textwidth}
%         \includegraphics[width=\linewidth]{charts/eigenvalue_vary_layer_rmse.pdf}
%         % \subcaption{RMSE with Varying Layers}
%     \endminipage
%     \vspace{-1em}
%     \caption{\textbf{Comparisons of Eigenvalue Prediction on Synthetic Data.} 
%     \emph{(1) Left: Evidence of Transformer's Ability to Predict Multiple Eigenvalues.} 
%     We use a small transformer (layer $=3$, head $=2$, embedding $= 64$) to predict top $10$ eigenvalues with $D=20$ and $N=50$.
%     All of the (Relative MSE) of $10$ eigenvalues are below $2\%$, verifying that the transformer can predict eigenvalue very well.
%     Additionally, the error of prediction grows slightly with $k$.
%     We note that: (i) 
%     % While in theoretical analysis we show transformer can perform power iteration, 
%     Higher-order eigenvalues require additional iterations and models with more layers. 
%      (ii) Smaller eigenvalues are more sensitive to the fluctuations in the predicted values under the relative MSE metric.  
%     \emph{(2) Middle: Predictions of eigenvalues with different input dimension $D$.} We use a small transformer and use $N=10$ in this experiment. We show that prediction errors increase significantly as dimension scales up, corroborating our theoretical remark \ref{remark3}.
%     \emph{(3) Right: Predictions of eigenvalues with different number of layers.} We use the same input as the previous multiple eigenvalues predictions experiment, and use a small transformer to predict top-$3$ eigenvalues. As the number of layers grow, the model performs better on eigenvalue prediction, which aligns with the result in theorem \ref{thm3.1}.
%     }
%     \label{fig:eigenvalues}
% \end{figure}



% % We generate each training sample $\bfa X \in \R^{D \times N}$ from a different multivariate Gaussian distribution $\mathcal{N}(\mu, \Sigma)$, and use the classical method via the NumPy function\footnote{We use \texttt{numpy.linalg.eig} to compute eigenvalues and eigenvectors.} to compute the top-$k$ eigenvalues $\lambda \in \mathbb{R}^k$ and eigenvectors $V \in \mathbb{R}^{D \times k}$ of the empirical covariance matrix $\bfa X^\top \bfa X/(N-1)$ as our ground truth.


% % Specifically, for each $\bfa X_i \in \R^D$, we sample $Z_i \sim N(0, I) \in \R^D$.
% % We then form $Z = [Z_1, \cdots, Z_N]$ and transform it using an invertible matrix $L \sim N(0, I) \in \R ^{D \times D}$, yielding the desired training sample $\bfa X$.



% % \begin{table}[h]
% %         \centering
% %         \caption{Hyperparameters for Multiple Eigenvector Prediction.
% %         }
% %         % \resizebox{ \textwidth}{!}{  
% %         \begin{tabular}{l*{2}{c}}
% %         \toprule
% %             \bf{parameter} & $N=10$  \\ 
% %             \midrule
% %             top-$k$ & $1$ & $5$ & $?$  \\
% %              steps (eigenvector) & 60k & 20k & 60k \\
% %             learning rate  & $1\text{e-}3$ & $5\text{e-}3$ & $5\text{e-}3$ \\
% %             Optimizer  & Adam & Adam & Adam \\
% %             batch size  & $64$ & $64$ & $64$  \\
% %             number of layers & $3$ & $3$ & $3$  \\
% %             hidden dimension & $256$ & $256$& $256$ \\
% %             hidden dimension & $256$ & $256$& $256$ \\
% %             number of heads  & $8$ & $8$& $8$ \\
% %             number of examples (Train) $N$ & $100$ & $100$ & $100$  \\
% %             \bottomrule
% %         \end{tabular}
% %         \label{table:hyperparam}
% %     \end{table} 


% % Use the theorem "Generate $Z \sim N(\mu, I)$, then apply linear transformation we get $X = LZ + \mu$, where $\Sigma = LL^{T}$ and $L$ is a invertible matrix, we get $X \sim N(\mu, Σ)$". 
% % Hence starting from building an random invertible matrix $L$ then we are done.
% % That is, to generate the $k$th training data $X_k$, the outline of procedure is:
% % Generate $L_k$ use `np.random.randn(D,D)`-> get $\Sigma_k = L_k L_k^T$ ->
% % use `np.linalg.eigh(Σ_k)` get the eigenvalue ground truth.
% % use $\Sigma_k$ to generate training data, $X_{k,i} ∈ ℝ^D \sim N(\mu_k, Σ_k)$, where $X_k = [X_{k,1}, X_{k,2},...,X_{k,N}]$.
% \paragraph{Model.}
% We use the GPT2-architecture transformer \citep{radford2019language} as our backbone model.
% We follow most settings in \citet{garg2022can}, but replace the Softmax attention with ReLU attention as constructed in definition~\ref{def:attention}.
% We also provide a empirical comparison between Softmax and ReLU attention in Figure \ref{fig:relu_softmax} in the Appendix.
% We use a slighly differernt architectures to predict eigenvalues and eigenvectors.
% For eigenvalues prediction, we flatten the transformer output $TF_{\bfa\theta}(\bfa H) \in \R^{N \times D}$ and use a linear layer $W_\lambda \in \R^{(N\cdot D) \times k}$ to readout the top $k$ eigenvalues.
% As for eigenvectors, we use one more linear layer $W_v \in \R^{(N\cdot D) \times (k\cdot D)}$ to readout $k$ eigenvectors concatenated in a $1$-dimension vector.
% We use a transformer with layer $=3$, head $=2$, and embedding size $=64$ to speed up the training process for most settings and find that it is sufficient to predict multiple eigenvalues and top-$1$ eigenvector well, see below sections for detailed discussion.
% % We start from a very small transformer with 64 embedding dimensions, 2 heads, and 3 layers only.
% % As the dimension of feature input data increases, we further test on a deeper model with 6, 9, and 12 layers respectively.

% \paragraph{Metrics.}
% For eigenvalues, we use relative mean squared error (RMSE) as loss function $\mathcal{L}_{\text{RMSE}}$ and evaluation metric.
% For the loss of predicting top-$K$ eigenvalue, the loss function is defined as following
% \begin{align*}
%     \mathcal{L}_{\text{RMSE}}( \lambda_i, \hat{\lambda}_i )
%     \coloneqq
%     \frac{1}{K} \sum_{i=1}^{K} \frac{\lambda_{i} - \hat{\lambda}_{i}}{\lambda_{i} + \epsilon}.
% \end{align*}
% For eigenvectors, we use cosine similarity as loss function and evaluation metric.
% For predicting $k$ eigenvectors, the loss function is defined as
% \begin{align*}
%     \mathcal{L}_{\text{cos}}(v_i, \hat{v}_i)
%     \coloneqq
%     \frac{1}{K} \sum_{i=1}^{K} 1 - \frac{v_{i} \cdot \hat{v}_i}{\max (\Vert v_{i}\Vert_2 \cdot \Vert \hat{v}_{i}\Vert_2, \epsilon)},
% \end{align*}
% where $v_{i}$ represent the $i$-th eigenvector.
% The design of these loss functions not only matches the intuition of eigenvectors and eigenvalues, but also stablize training by normalizing the loss values.
% % We use two different metrics for predicting eigenvalues and eigenvectors.
% % For eigenvalues, we use mean relative squared error as the training loss and evaluation metric.
% % To predict top-$K$ eigenvalue, defined as follows:
% % \begin{align*}
% %     \frac{1}{K} \sum_{i=1}^{K} \frac{\lambda_{i} - \hat{\lambda}_{i}}{\lambda_{i} + \epsilon}.
% % \end{align*}
% % For eigenvectors, we use cosine similarity as the training loss and evaluation metric as follows:
% % \begin{align*}
% %     \frac{1}{K} \sum_{i=1}^{K} 1 - \frac{v_{i} \cdot \hat{v}_i}{\max (\Vert v_{i}\Vert_2 \cdot \Vert \hat{v}_{i}\Vert_2, \epsilon)},
% % \end{align*}
% % where $v_{i}$ represent the $i$-th eigenvector.


% \begin{figure}[tbp]
%     \centering
%     \minipage{0.33\textwidth}
%         \includegraphics[width=\linewidth]{charts/eigenvector_vary_d_cos_sim.pdf}
%         % \subcaption{Top-10 Eigenvalues}
%     \endminipage\hfill
%     \minipage{0.33\textwidth}
%         \includegraphics[width=\linewidth]{charts/eigenvector_vary_layer_cos_sim.pdf}
%         % \subcaption{Cosine Similarity with Varying d}
%     \endminipage
%     \minipage{0.33\textwidth}
%         \includegraphics[width=\linewidth]{charts/eigenvector_vary_k_cos_sim.pdf}
%         % \subcaption{RMSE with Varying Layers}
%     \endminipage\hfill
%     \vspace{-1em}
%     \caption{\textbf{Comparison of Eigenvector Prediction on Synthetic Data.}
%     \emph{(1) Left: Prediction of top-$1$ eigenvector with different input dimension $D$.} We use a small transformer and $N=10$. As the dimension $D$ scales up the eigenvector prediction suffers significantly. 
%     \emph{ (2) Middle: Prediction of top=$1$ eigenvector with varying number of layers.} 
%     % We start with a small transformer and predict the top $1$ eigenvector with an increasing number of layers.
%     % We also run on different dimensions $d$ to see the effect number of layers on different dimensions.
%     We start from small transformer and use $N=5$. The result demonstrates an `elbow effect', where we show that the increase of $L$ significantly boost the performance when $L$ is small but halt to progress for larger $L$. We believe this can be explained by the bias-variance tradeoff.
%     % shows that model performance increases with the number of layers, but the rate of improvement slows as the number of layers grows.
%     % Hence we further increase the number of heads from $2$ to $8$, and further increase the performance, suggesting increasing model depth alone is not sufficient to keep enhancing eigenvector prediction.
%     % The result also verifies theorem \ref{thm3.1} that we need at least 2 layers of the transformer to perform one iteration of the power method, as there is a sharper decrease between layer$=1$ and layer$=2$.
%     \emph{(3) 
%     Right: Predictions of eigenvectors with different numbers of $k$} We use $N=10$ and $D=10$ in this experiment.
%     We use a larger transformer with layer$=12$, heads$=8$, and an embedding size$=256$ in this experiment.
%     The result demonstrates a decreasing prediction accuracy and increasing standard deviation as $k$ increases.
%     We list the individual cosine similarities of the predicted k-th eigenvectors in table \ref{tab:mul_k_vector}.
%     All the evaluations in the above three figures are averaged on $10$ runs with different random seed.
%     }
%     \label{fig:eigenvector}
% \end{figure}



% \subsection{Synthetic Data vs Real World Data} 


% \begin{figure}[t]
%     \centering
%     \minipage{0.24\textwidth}
%         \includegraphics[width=\linewidth]{charts/mnist_eigenvalue_multiple_k.pdf}
%         % \subcaption{Top-10 Eigenvalues}
%     \endminipage\hfill
%     \minipage{0.24\textwidth}
%         \includegraphics[width=\linewidth]{charts/mnist_3_eigenvector.pdf}
%         % \subcaption{Cosine Similarity with Varying d}
%     \endminipage\hfill
%     \minipage{0.24\textwidth}
%         \includegraphics[width=\linewidth]{charts/fashion_mnist_eigenvalue_multiple_k.pdf}
%         % \subcaption{RMSE with Varying Layers}
%     \endminipage\hfill
%     \minipage{0.24\textwidth}
%         \includegraphics[width=\linewidth]{charts/fashion_mnist_3_eigenvector.pdf}
%         % \subcaption{Fourth Chart}
%     \endminipage
%     \vspace{-1em}
%     \caption{\textbf{Comparison of Eigenvalues and Eigenvectors Prediction on Real World Data.} 
%     \emph{(1) Left: Predicting Top-$10$ Eigenvalues on MNIST}  
%     \emph{(2) Second Left: Predicting Top-$3$ Eigenvectors on MNIST}
%     \emph{(3) Second Right: Predicting Top-$10$ Eigenvalues on FMNIST} 
%     \emph{(4) Right: Predicting Top-$3$ Eigenvectors on 
%     FMNIST}.
%     % We the transformer's performance on predicting eigenvalues and eigenvectors on real world datasets.
%     % Note that transformers are trained on synthetic samples, whose columns are generated by Gaussians.
%     % This leads to a large gap between training and test data distributions.
%     % However, as we see in the figure, transformers are still capable of prediction eigenvalues well, especially on MNIST.
%     % Further, for eigenvectors, transformers also show similar performance as they do on synthetic datasets, indicating their capability to predict principle eigenvectors accuractly.
%     We show that on real world datasets, Transformers perform similarly to the synthetic datasets. The experimental setup for real world data is analogous to the ones performed for synthetic data.
%     }
%     \label{fig:eigenvalues-realworld}
% \end{figure}


% \paragraph{Synthetic Dataset.}
% The results on synthetic data are in figure~\ref{fig:eigenvalues}.
% We first observe that transformers are capable of predicting top-$10$ eigenvalues with small error ($< 2\%$ error).
% The result also corresponds to theorem \ref{thm3.1}, indicating transformers are able to perform the power iteration method and generate eigenvalues with small error.
% For the impact of $D$, we observe the second subfigure in figure~\ref{fig:eigenvalues}.
% In general, we discover an increasing trend of RMSE when $D$ increases, this coincides with the theoretical findings stated in remark \ref{remark3}.
% We also observe that the error of prediction slightly increases with $k$, which is natural as the prediction dimension grows larger.
% For the impact of layers (right subfigure of figure~\ref{fig:eigenvalues}), we observe that as the number of layer increases, RMSE shows significant reduction.
% This matches our theoretical construction as we show the iteration of power methods correspond to the number of layers, see figure~\ref{fig:diagram} for the visualization of our transformer model.
% % transformer can approximate power iteration methods, and the additional iterations in our transformer can be seen as requiring a deeper model, see figure \ref{fig:diagram} for the visualization of our transformer model.
% One thing to highlight is higher-order eigenvalues (larger $k$) have smaller magnitudes, which are are more sensitive to fluctuations in the predicted values when using the relative MSE metric.
% This explaines the higher variance/error of higher order eigenvalues.
% For eigenvectors, we also observe that transformers are capable of predicting principle eigenvectors.
% In particular, the cosine similarity between predicted eigenvector and ground truth is close to 1 when $D$ is small.

% % We demonstrate that transformer can predict eigenvalue by predicting top-$1$ eigenvalues for $D = 10, 20, 30, 40$.
% % and (ii) Predict top-$10$ eigenvalues for $D=20$.
% % Note that in this experiment we want the scale of eigenvalue across different $D$ to remain the same.
% % Since the total variance of the covariance matrix scales with dimension $D$ and the sum of all eigenvalues equal to the total variance, the eigenvalues also scale with $D$.
% % Therefore we divide the generated $\bfa X^T \bfa X$ by $D$ in this experiment.
% % We set $N=10$ and use a small transformer for this experiment.


% \paragraph{Real World Dataset.}
% The results on real world dataset are in figure~\ref{fig:eigenvalues-realworld}.
% We observe that transformers are also capable of predicting top-$k$ eigenvalues well on both MNIST and FMNIST.
% Despite the difference in data distribution on training and test data, transformers are able to produce small error on predicting eigenvalues.
% Overall, we show that pretrained transformers learn PCA, and is able to generalize to other datasets as well.
% For eigenvectors, we can see that trained transformers show similar behavior on real world datasets when comparing to the synthetic ones.
% Indicating that our model actually learn to perform PCA instead of learn certain inductive bias.

% % To evaluate the impact of $D$, we evaluated transformers' performance with different values of $D \in \{10, 20, 30, 40\}$.


% % \paragraph{Multiple top-$k$ eigenvalues prediction.}
% % \red{Do we need to say that we want the variance of top-$10$ to be small hence using larger $N = 50$ ?}
% % We begin by demonstrating that a small transformer is able to predict top $10$ eigenvalues.

% % In this experiment we set $D = 20$ and $N = 50$ for our input data
% % The result is shown in the left figure in \ref{fig:eigenvalues}, all the relative mean squared errors of top $10$ eigenvalues are below $2\%$.
% % This confirms the transformer's ability to predict eigenvalue very well. 
% % We observe that the error of prediction grows slightly with $k$.
% % We attribute this to two main factors. 
% % First, power iteration methods are optimized for computing the largest eigenvalues, higher-order eigenvalues require additional iteration.
% % Our theoretical construction shows transformer can approximate power iteration methods, and the additional iterations in our transformer can be seen as requiring a deeper model, see figure \ref{fig:diagram} for the visualization of our transformer model.
% % Second, higher-order eigenvalues (larger $k$) have smaller magnitudes. Smaller values are more sensitive to fluctuations in the predicted values when using the relative MSE metric.
% % We include more evidence in section \ref{sec:add_exp}.
% % % Predict top-$10$ eigenvalues for $D=20$.
% % % We want the variance of top-$10$ to be small.
% % % We achieve this by using larger $N = 50$.

% % \paragraph{Vary $D$ to predict eigenvalue.} 
% % Results are Figure~\ref{fig:d_eigenvalues}.
% % We observe that transformers predict top-$1$ eigenvalue well for $D \in \{ 10, 20, 30, 40 \}$.



% % For (ii), we want the variance of top-$10$ to be small.
% % We achieve this by using larger $N = 40$
% % We derive the ground truth eigenvalues from $X^\top X$ where $X$ is generated as $X = LZ$.
% % As $D$ increase, the norm row/column vector in matrix grow with $\sqrt{D}$.
% % When we do matrix multiplication like $X = LZ$, the variance of the  increase when the 
% % And when we 
% % we generate training data by $X = LZ$ and get the ground truth of eigenvalue 


% % \paragraph{Vary layers to predict eigenvalue.}
% % The result is in figure \ref{fig:layer_eigenvalue}.
% % In this experiment, we investigate the relation between the number of layers and the predicted eigenvalue error.
% % In the previous paragraph we see that the transformer predicts the first eigenvalue very well, so we increase the data 
% % We use a larger input data the same as in the multiple top $k$ eigenvalue prediction experiment ($D = 20$ and $N = 50$) and train a small transformer to predict top $3$ eigenvalues.



% \subsection{Prediction with Different Parameter Combination.}\label{sec:eigvec_pred}
% \paragraph{Prediction with different $D$.} 
% The results are given in figure~\ref{fig:eigenvector}.
% In this experiment, we test the influence of increasing feature dimension $D$ affects the ability of a Transformer model to predict the principal eigenvector of a data matrix.
% We use the simplest setting with a small transformer and test on $D = 5,10,20,30,40$, $N=10$ and predict top-$1$ eigenvector.
% As the feature dimension 
% $D$ increases, we observe a clear trend of performance degradation of the Transformer's ability to predict the principal eigenvector accurately.
% This confirms our theoretical results stated in remark~\ref{remark3} that the feature dimension $D$ affects the approximation properties of Transformers significantly.
% % The challenges observed with increasing 
% % $d$ suggests that Transformers require additional capacity or architectural modifications to handle high-dimensional data effectively.






% \paragraph{Prediction with different $L$.}
% The results are given in figure~\ref{fig:eigenvector}.
% In this experiment, we change the number of layers of transformer with head $= 2$ and embedding $=64$, and set $N = 5$ to speed up the experiment.
% We observe that as the number of layers increases, the testing error also decreases, but the decreasing scale is less obvious when the number of layers becomes larger.
% However, the rate of improvement diminishes as the number of layers becomes larger. 
% This suggests that increasing model depth alone is not sufficient for significantly enhancing eigenvector prediction.
% To verify our guess, we increase the number of heads from $2$ to $8$ and find that the cosine similarity increases further and with a slightly steeper incline.
% Note that there is a sharper decrease between layer$=1$ and layer$=2$ across different $d$.
% This finding supports theorem \ref{thm3.1} that we need at least 2 layers of the transformer to perform one iteration of the power method.

% \paragraph{Prediction with different $k$.}
% The results are given in table \ref{tab:mul_k_vector}, and the right subfigure in figure \ref{fig:eigenvector} where the cosine similarity in y-axis is averaged over $k$ eigenvectors. 
% % In this experiments, we verify transformer's ability to predict top $k$ principal components.
% We use $N=10$ and $D=10$ in this experiment.
% We use a larger transformer with layer$=12$, heads$=8$, and an embedding size$=256$ in this experiment.
% As shown in figure~\ref{fig:eigenvector}, the model's ability to predict top $k$ eigenvectors decreases as more eigenvectors are predicted, with increasing standard deviation. 
% Table \ref{tab:mul_k_vector} lists the individual cosine similarities of the predicted $k$-th eigenvectors.
% The results show that most errors come from high-order eigenvectors.
% When trained to predict $k_{\text{train}} = 4$ eigenvectors, the model performs as well at predicting the top $1$ eigenvector as when trained on $k_{\text{train}} = 1$.
% The result shows that most prediction errors come from high-order eigenvectors. 
% This suggests that the pivotal difficulty is in the prediction of higher order eigenvectors.
% % This suggests that the pivotal of the task lies not in predicting more vectors, but in the transformer's performance on high-order eigenvectors. 
% % \red{This aligns with our theoretical result that to find high-order eigenvectors, the model must revisit earlier layers (from block 3 to block 2 in figure \ref{fig:diagram}) during forward propagation, which is equivalent to increasing the depth of the forward pass in the actual experiment.
% % Hence under the fixed number layers, we observe a worse performance for predicting high-order eigenvectors.}





% % \subsubsection{Real-world Datasets}
% % Finally, we test our framwork on two real-world dataset instead of the sythetic data generate from multivariate gaussian distribution.
% % We train the model to predict top-$5$ eigenvalues and top-$1 \sim 5$ eigenvectors, with former using small model and latter using large model.
% % For both dataset we use $64000$ data as training data and $6000$ testing data.
% % We set number of samples $N=10$, this lead to $6400 $ training data and $600$ testing data.


% % \begin{figure}[t]
% %     \centering
% %     \minipage{0.33\textwidth}
% %         \includegraphics[width=\linewidth]{charts/eigenvalue_multiple_k.pdf}
% %         % \subcaption{Top-10 Eigenvalues}
% %     \endminipage\hfill
% %     \minipage{0.33\textwidth}
% %         \includegraphics[width=\linewidth]{charts/eigenvalue_vary_d_rmse.pdf}
% %         % \subcaption{Cosine Similarity with Varying d}
% %     \endminipage\hfill
% %     \minipage{0.33\textwidth}
% %         \includegraphics[width=\linewidth]{charts/eigenvalue_vary_layer_rmse.pdf}
% %         % \subcaption{RMSE with Varying Layers}
% %     \endminipage
% %     \vspace{-1em}
% %     \caption{\textbf{Comparison of eigenvalue metrics.} \textcolor{red}{Make the font bigger for clarity.}
% %     \emph{1. Left:} Evidence of Transformer Ability to Predict Multiple Eigenvalues.
% %     We use a small transformer (layer $=3$, head $=2$, embedding $= 64$) to predict the top $10$ eigenvalues of input data with $D=20$ and $N=50$.
% %     All the Relative Mean Squared Error (Relative MSE) of $10$ eigenvalues are below $2\%$, verifying that the transformer indeed can predict the eigenvalue very well.
% %     Additionally, the error of prediction grows slightly with $k$.
% %     This may be due to: (i) 
% %     % While in theoretical analysis we show transformer can perform power iteration, 
% %     Higher-order eigenvalues require additional iterations in the power iteration method. 
% %     This corresponds to needing a deeper model, which contrasts with our small model. (ii) Larger $k$ eigenvalues have smaller magnitudes, making them more sensitive to fluctuations in the predicted values under the relative MSE metric.  
% %     \emph{Middle:} xxx.
% %     \emph{Right:} xxx.
% %     }
% %     \label{fig:eigenvalues}
% % \end{figure}

% % \begin{table}[t]
% % \centering
% % \caption{\textbf{Cosine Similarity.}
% % We dentoe $k_{\text{train}}$ as the number of eigenvectors to predict during training.
% % For example, for $k_{\text{train}}=4$, the model is trained to predict 4 eigenvectors.
% % }
% % \begin{tabular}{lcccc}
% % \toprule
% % \textbf{k-th eigenvec.} & \textbf{k=1} & \textbf{k=2} & \textbf{k=3} & \textbf{k=4} \\ \hline
% % $k_{\text{train}}=4$    & $0.891 (0.006)$ & $0.616 (0.038)$ & $0.282 (0.047)$ & $0.120 (0.022)$ \\
% % $k_{\text{train}}=3$    & $0.908 (0.011)$ & $0.706 (0.023)$ & $0.366 (0.018)$ & - \\
% % $k_{\text{train}}=2$   & $0.903 (0.006)$ & $0.647 (0.019)$ & - & - \\
% % $k_{\text{train}}=1$   & $0.894 (0.009)$ & - & - & - \\
% % \bottomrule
% % \end{tabular}
% % \label{tab:mul_k_vector}
% % \end{table}

% % \begin{figure}[htbp]
% %     \centering
% %     \minipage{0.33\textwidth}
% %         \includegraphics[width=\linewidth]{charts/eigenvector_vary_d_cos_sim.pdf}
% %         % \subcaption{Top-10 Eigenvalues}
% %     \endminipage\hfill
% %     \minipage{0.33\textwidth}
% %         \includegraphics[width=\linewidth]{charts/eigenvector_vary_layer_cos_sim.pdf}
% %         % \subcaption{Cosine Similarity with Varying d}
% %     \endminipage
% %     \minipage{0.33\textwidth}
% %         \includegraphics[width=\linewidth]{charts/eigenvector_vary_k_cos_sim.pdf}
% %         % \subcaption{RMSE with Varying Layers}
% %     \endminipage\hfill
% %     \vspace{-1em}
% %     \caption{\textbf{Comparison of Eigenvector Metrics.}
% %     \emph{(1) Left: Varing $d$ to predict top $1$ eigenvector.} We run this experiment on a small transformer to verify our theoretical result in remark \ref{remark3}. We clearly find that the model predicts worse when feature dimension  $d$ increases.
% %     \emph{ (2) Middle: Varying number of layer to predict top $1$ eigenvector.} 
% %     % We start with a small transformer and predict the top $1$ eigenvector with an increasing number of layers.
% %     % We also run on different dimensions $d$ to see the effect number of layers on different dimensions.
% %     The result shows that model performance increases with the number of layers, but the rate of improvement slows as the number of layers grows.
% %     Hence we further increase the number of heads from $2$ to $8$, and further increase the performance, suggesting increasing model depth alone is not sufficient to keep enhancing eigenvector prediction.
% %     The result also verifies theorem \ref{thm3.1} that we need at least 2 layers of the transformer to perform one iteration of the power method, as there is a sharper decrease between layer$=1$ and layer$=2$.
% %     \emph{(3) 
% %     Right:} xxx.
% %     All the testing error in above three figure xxx
% %     }
% %     \label{fig:eigenvector}
% % \end{figure}



% % \begin{figure}[htb]
% %     \centering
% %     \minipage{0.33\textwidth}
% %         \includegraphics[width=\linewidth]{charts/mnist_eigenvalue_multiple_k.pdf}
% %         % \subcaption{Top-10 Eigenvalues}
% %     \endminipage\hfill
% %     \minipage{0.33\textwidth}
% %         \includegraphics[width=\linewidth]{charts/fashion_mnist_eigenvalue_multiple_k.pdf}
% %         % \subcaption{Cosine Similarity with Varying d}
% %     \endminipage\hfill
% %     \minipage{0.33\textwidth}
% %         \includegraphics[width=\linewidth]{charts/cifar_eigenvalue_multiple_k.pdf}
% %         % \subcaption{RMSE with Varying Layers}
% %     \endminipage
% %     \vspace{-1em}
% %     \caption{\textbf{Eigenvalue Prediction on 3 Real-World Datasets.}
% %     \emph{Left:} xxx.  
% %     \emph{Middle:} xxx.
% %     \emph{Right:} xxx.
% %     }
% %     \label{fig:real_world_eigenvalue}
% % \end{figure}


% % \begin{figure}[htb]
% %     \centering
% %     \minipage{0.33\textwidth}
% %         \includegraphics[width=\linewidth]{charts/eigenvector_vary_d_cos_sim.pdf}
% %         % \subcaption{Top-10 Eigenvalues}
% %     \endminipage\hfill
% %     \minipage{0.33\textwidth}
% %         \includegraphics[width=\linewidth]{charts/N_LAYER_EIGENVECTOR.png}
% %         % \subcaption{Cosine Similarity with Varying d}
% %     \endminipage\hfill
% %     \minipage{0.33\textwidth}
% %         \includegraphics[width=\linewidth]{charts/eigen.pdf}
% %         % \subcaption{RMSE with Varying Layers}
% %     \endminipage
% %     \vspace{-1em}
% %     \caption{\textbf{Comparison of eigenvalue metrics.} \textcolor{red}{Make the font bigger for clarity.}
% %     \emph{Left:} Evidence of Transformer Ability to Predict Multiple Eigenvalues.
% %     We use a small transformer (layer $=3$, head $=2$, embedding $= 64$) to predict the top $10$ eigenvalues of input data with $D=20$ and $N=50$.
% %     All the Relative Mean Squared Error (Relative MSE) of $10$ eigenvalues are below $2\%$, verifying that the transformer indeed can predict the eigenvalue very well.
% %     Additionally, the error of prediction grows slightly with $k$.
% %     This may be due to: (i) 
% %     % While in theoretical analysis we show transformer can perform power iteration, 
% %     Higher-order eigenvalues require additional iterations in the power iteration method. 
% %     This corresponds to needing a deeper model, which contrasts with our small model. (ii) Larger $k$ eigenvalues have smaller magnitudes, making them more sensitive to fluctuations in the predicted values under the relative MSE metric.  
% %     \emph{Middle:} xxx.
% %     \emph{Right:} xxx.
% %     }
% %     \label{fig:same_layer}
% % \end{figure}




% % \begin{figure}[htb]
% % % \vspace{-1em}
% % \minipage{0.5\textwidth}
% % \minipage{0.5\textwidth}
% % \includegraphics[width=\linewidth]{charts/top-10 eigenvalue.png}
% % \endminipage\hfill
% % \minipage{0.5\textwidth}
% % \includegraphics[width=\linewidth]{charts/eigenvalue_vary_d_cos_sim.pdf}
% % \endminipage
% % \endminipage\hfill
% % \minipage{0.5\textwidth}
% % \minipage{0.5\textwidth}
% % \includegraphics[width=\linewidth]{charts/eigenvalue_vary_layer_rmse.pdf}
% % \endminipage\hfill
% % % \minipage{0.5\textwidth}
% % % \includegraphics[width=\linewidth]{charts/eigenvalue_vary_layer_train_loss.pdf}
% % % \endminipage\hfill
% % \endminipage
% % \vspace{-1em}
% % \caption{\red{Bigger font.}
% % % Maximum infinity norm $\| \mathbf{x} \|_{\infty}$ for different tensor components within layer 10 of BERT. Our work is analysed using two softmax variations: $\mathtt{OutEffHop}$ (represented in red) and vanilla $\Softmax$ (in grey). We find $\mathtt{OutEffHop}$ suppresses the outliers growing in both FFN layers.
% % }
% % \label{fig:same_layer}
% % % \vspace{-1.8em}
% % \end{figure}




% % \begin{figure*}[htb]
% % \begin{minipage}[t]{0.5\textwidth}
% %   \includegraphics[width=0.9\linewidth]{charts/eigenvalue_vary_d_rmse.pdf}
% % \end{minipage}%
% % \hfill
% % \begin{minipage}[t]{0.5\textwidth}
% %   \includegraphics[width=0.9\linewidth]{charts/eigenvalue_vary_d_train_loss.pdf}
% % \end{minipage}%
% % \vspace{-1.5em}
% % \caption{
% % \textbf{Influence of Different Feature Fimension $D$ on Top-$1$ Eigenvalue Prediction.}
% % We use 
% % % \emph{Left}
% % }
% % \label{fig:d_eigenvalues}
% % \end{figure*}

% % \begin{figure*}[htb]
% % \begin{minipage}[t]{0.5\textwidth}
% %   \includegraphics[width=\linewidth]{charts/eigenvalue_vary_layer_rmse.pdf}
% % \end{minipage}%
% % \hfill
% % \begin{minipage}[t]{0.5\textwidth}
% %   \includegraphics[width=\linewidth]{charts/eigenvalue_vary_layer_train_loss.pdf}
% % \end{minipage}%
% % \vspace{-1.5em}
% % \caption{
% % % \textbf{Noise-Robustness.}
% % % Our extensive evaluation of noise robustness across various Hopfield Networks, including Vanilla Modern Hopfield, Sparse Hopfield, 10th Order Hopfield, and our $\mathtt{OutEffHop}$, is conducted on two image datasets: MNIST and CIFAR10. The results show that as the noise level rises, the impact of $\mathtt{OutEffHop}$ on the error rate is minimal.
% % }
% % \label{fig:layer_eigenvalue}
% % \end{figure*}


% % \begin{figure*}[htb]
% % \begin{minipage}[t]{0.5\textwidth}
% %   \includegraphics[width=\linewidth]{charts/eigenvector_vary_d_cos_sim.pdf}
% % \end{minipage}%
% % \hfill
% % \begin{minipage}[t]{0.5\textwidth}
% %   \includegraphics[width=\linewidth]{charts/eigenvector_vary_d_train_loss.pdf}
% % \end{minipage}%
% % \vspace{-1.5em}
% % \caption{
% % % \textbf{Noise-Robustness.}
% % % Our extensive evaluation of noise robustness across various Hopfield Networks, including Vanilla Modern Hopfield, Sparse Hopfield, 10th Order Hopfield, and our $\mathtt{OutEffHop}$, is conducted on two image datasets: MNIST and CIFAR10. The results show that as the noise level rises, the impact of $\mathtt{OutEffHop}$ on the error rate is minimal.
% % }
% % \label{fig:d_eigenvector}
% % \end{figure*}





% % \begin{figure*}[htb]
% % \begin{minipage}[t]{0.5\textwidth}
% %   \includegraphics[width=\linewidth]{charts/eigenvector_vary_layer_cos_sim.pdf}
% % \end{minipage}%
% % \hfill
% % \begin{minipage}[t]{0.5\textwidth}
% %   \includegraphics[width=\linewidth]{charts/training_loss_layers_8x8.pdf}
% % \end{minipage}%
% % \vspace{-1.5em}
% % \caption{
% % % \textbf{Noise-Robustness.}
% % % Our extensive evaluation of noise robustness across various Hopfield Networks, including Vanilla Modern Hopfield, Sparse Hopfield, 10th Order Hopfield, and our $\mathtt{OutEffHop}$, is conducted on two image datasets: MNIST and CIFAR10. The results show that as the noise level rises, the impact of $\mathtt{OutEffHop}$ on the error rate is minimal.
% % }
% % \label{fig:layers_eigenvector}
% % \end{figure*}

% % \begin{figure}[H]
% % \vspace{-1em}\includegraphics[width=\linewidth]{}
% %     \vspace{-2em}
% %     \caption{
% %     }
% %     \label{fig:energy-land}
% %     \vspace{-1em}
% % \end{figure}

% % \begin{figure*}[htb]
% % % \begin{minipage}[t]{0.5\textwidth}
% %   \includegraphics[width=\linewidth]{charts/top-10 eigenvalue.png}
% % % \end{minipage}%
% % % \hfill
% % % \begin{minipage}[t]{0.5\textwidth}
% % %   \includegraphics[width=\linewidth]{charts/top-10 eigenvalue.png}
% % % \end{minipage}%
% % % \vspace{-1.5em}
% % \caption{
% % \textbf{Evidence of Transformer Ability to Predict Multiple Eigenvalues.}
% % We use a small transformer (layer $=3$, head $=2$, embedding $= 64$) to predict the top $10$ eigenvalues of input data with $D=20$ and $N=50$.
% % The result shows that all the Relative Mean Squared Error (Relative MSE) of $10$ eigenvalues are below $2\%$, verifying that the transformer indeed can predict the eigenvalue very well.
% % Additionally, the error of prediction grows slightly with $k$.
% % This may be due to: (i) While in theoretical analysis we show transformer can perform power iteration, higher-order eigenvalues require additional iterations in the power iteration method. This corresponds to needing a deeper model, which contrasts with our small model. (ii) Larger $k$ eigenvalues have smaller magnitudes, making them more sensitive to fluctuations in the predicted values under the relative MSE metric.
% % }
% % \label{fig:d_20_top_10}
% % \end{figure*}

% % \begin{figure*}[htb]
% % \begin{minipage}[t]{0.5\textwidth}
% %   \includegraphics[width=\linewidth]{charts/eigenvector_vary_k_cos_sim.pdf}
% % \end{minipage}%
% % \hfill
% % \begin{minipage}[t]{0.5\textwidth}
% %   \includegraphics[width=\linewidth]{charts/eigenvector_vary_k_train_loss.pdf}
% % \end{minipage}%
% % \vspace{-1.5em}
% % \caption{
% % % \textbf{Noise-Robustness.}
% % % Our extensive evaluation of noise robustness across various Hopfield Networks, including Vanilla Modern Hopfield, Sparse Hopfield, 10th Order Hopfield, and our $\mathtt{OutEffHop}$, is conducted on two image datasets: MNIST and CIFAR10. The results show that as the noise level rises, the impact of $\mathtt{OutEffHop}$ on the error rate is minimal.
% % }
% % \label{fig:k_eigenvector}
% % \end{figure*}







% % \clearpage





\section{Additional Theoretical Background}\label{sec:additional-theory}
Here, we include several technical lemme that are intensively used throughout our paper.
The Lipschitzness of an MLP layer is obtained in \citep[Lemma~J.1]{bai2024transformers}, which we restate it below
\begin{lemma}[\cite{bai2024transformers}]\label{lem:mlp-lipschitz}
    For a single MLP layer, $\bm{\theta}_2 = (\bW_1, \bW_2)$, we introduce its norm
    \begin{equation*}
        \lvert
        \lvert
        \lvert
        \bm{\theta}_2
        \rvert 
        \rvert 
        \rvert
        =
        \norm{\bW_1}_{\text{op}}
        +
        \norm{\bW_2}_{\text{op}}.
    \end{equation*}
    For any fixed hidden dimension $D^\prime$, we consider
    \begin{equation*}
        \Theta_{2, B}
        \coloneqq
        \{ 
        \bm{\theta}_2
        :
        \lvert
        \lVert
        \bm{\theta}_2
        \rVert
        \rvert
        \leq 
        B
        \}.
    \end{equation*}
    Then for $\bH \in \mathcal{H}_R$, $\bm{\theta}_2 \in \Theta_{2, B}$, the function 
    $(\bm{\theta}_2, \bH) \mapsto \text{MLP}_{\bm{\theta}_2}$ is $(BR)$-Lipschitz w.r.t. $\bm{\theta}_2$
    and $(1 + B^2)$-Lipschitz w.r.t. $\bH$.
\end{lemma}

The following lemma shows any-variate attention is capable of performing variate-wise operation on arbitrary number of covariates under any-variate encoding.

\begin{lemma}[Group-Wise Operation via Any-Variate Attention]\label{lem:moirai-group-wise}

    Let $\norm{\bH}_{2,p} \coloneqq ( \sum_{i=1}^N \norm{\bh_i}_2^p )^{1/p}$ denote the column-wise $(2, p)$-norm of $\bH$.
    For any input matrix $\bH = (\bh_1, \cdots, \bh_T)$ such that $\norm{\bH}_{2, \infty} \leq \mathtt{R}$,
    suppose
    $\psi(\cdot): \R^{D \times T} \rightarrow \R^{D \times T}$ is a sequence-to-sequence function implemented by a single layer standard transformer ($\text{TF}_{\bm{\theta}}^\dagger$) such that 
    \[
    \text{TF}_{\bm{\theta}}^\dagger(\bH)
    \coloneqq
        \psi(\bH).
    \]
    Then there exists a single layer MOIRAI transformer $\text{TF}_{\bm{\theta}}(\cdot)$ such that for any input 
    \[
    \bH^\star
    =
    \begin{bmatrix}
        \bH_1 & \bH_2 & \cdots & \bH_K
    \end{bmatrix},
    \]
    where $\bH_k \in \R^{D \times T}$.
    $\text{TF}_{\bm{\theta}}(\cdot)$ performs
    \[
    \text{TF}_{\bm{\theta}}
    (\bH^\star)
    =
    \begin{bmatrix}
     \psi(\bH_1) & \psi(\bH_2) & \cdots & \psi(\bH_K)   
    \end{bmatrix}.
    \]
\end{lemma}

\begin{proof}[Proof of Lemma~\ref{lem:moirai-group-wise}]\label{proof:group-wise}
    We start by showing the case of a single-head, single-layer standard transformer.
    Let 
    \[
    \text{MLP}_{\bm{\theta}_2} \circ \text{Attn}_{\bm{\theta}}^\dagger
    (\bH)
    =
    \text{MLP}_{\bm{\theta}_2}
    \circ
    \bV \bH
    \sigma
    (
    \Braket{
    \bQ \bH, 
    \bK \bH
    }
    )
    =
    \bV \bH
    \bA_{\bH},
    \]
    where $\bA_{\bH} = \sigma(\Braket{
    \bQ \bH, 
    \bK \bH})$.

    Let $\psi_1(\bH) \coloneqq \bV \bH \bA_{\bH}$, to apply group-wise operation of $\psi_1(\cdot)$ on some input such that
    \[
    \psi_1(
    \bH^\star
    )
    =
    \begin{bmatrix}
        \psi_1(\bH_1) & \psi_1(\bH_2) & \cdots & \psi_1(\bH_K)
    \end{bmatrix}.
    \]

    Let $\bm{0} \in \R^{T \times T}$ be a zero matrix, and $\bm{1} \in \R^{T \times T}$ be a $1$s matrix, for for any input $\norm{\bH^\star}_{2, \infty} \leq \mathtt{R}$, one can find some $u^2 < 0$ to decompose $\psi_1(\cdot)$ into the following form.
    \begin{align*}
    \psi_1(\bH^\star)
    &=
    \bV
    \begin{bmatrix}
        \bH_1 \bA_{\bH_1} & 
        \bH_2 \bA_{\bH_2} &
        \cdots &
        \bH_K \bA_{\bH_K}
    \end{bmatrix}
    \\
    &=
    \bV 
    \bH^\star
    \begin{bmatrix}
        \bA_{\bH_1} & \bm{0} & \cdots  & \bm{0} \\
        \bm{0} & \bA_{\bH_2} & \cdots  & \bm{0} \\
        \vdots & \vdots & \ddots & \vdots \\
        \bm{0} & \bm{0} & \cdots & \bA_{\bH_K}
    \end{bmatrix}
    \\
    &=
    \bV
    \bH^\star
    \times
    \\
    &\sigma
    \left(
    \begin{bmatrix}
        \Braket{\bQ \bH_1, \bK \bH_1} & \Braket{\bQ \bH_1, \bK \bH_2} & \cdots  & \Braket{\bQ \bH_1, \bK \bH_K} \\
        \Braket{\bQ \bH_2, \bK \bH_1} & \Braket{\bQ \bH_2, \bK \bH_2} & \cdots  & \Braket{\bQ \bH_2, \bK \bH_K} \\
        \vdots & \vdots & \ddots & \vdots \\
        \Braket{\bQ \bH_K, \bK \bH_1} & \Braket{\bQ \bH_K, \bK \bH_2} & \cdots & \Braket{\bQ \bH_K, \bK \bH_K}
    \end{bmatrix}
    +
    \begin{bmatrix}
        \bm{0} & u^2 \cdot \bm{1} & \cdots  & u^2 \cdot \bm{1} \\
        u^2 \cdot\bm{1} & \bm{0} & \cdots  & u^2 \cdot \bm{1}\\
        \vdots & \vdots & \ddots & \vdots \\
        u^2 \cdot\bm{1} & u^2\cdot \bm{1} & \cdots & \bm{0}
    \end{bmatrix}
    \right).
    \end{align*}
    Further, observe that operations in an MLP layer are either left multiplication or element-wise operations, which implies group-wise as well. 
    We then finish the proof by setting $u^1 = 0$.

\end{proof}


\begin{theorem}[\text{\citep[Section~5.6]{wainwright2019high}}]\label{thm:generalizd-dudley}
    Suppose $\psi: [0,+\infty_ \rightarrow [0, +\infty)$ is a convex, non-decreasing function satisfying $\psi(x+y) \geq \psi(x) \psi(y)$.
    For any random variable $X$, we consider the Orlicz norm induced by $\psi: \norm{X}_{\psi} \coloneqq \inf \{ K > 0: \mathtt{E}_\psi(|X|/K) \} \leq 1$.
    Suppose that $\{ X_{\theta} \}$ is a zero-mean random process indexed by $\theta\in\Theta$ such that $\norm{ X_{\theta} - X_{\theta^\prime} } \leq \rho(\theta, \theta^\prime)$ for some metric $\rho$ on $\Theta$.
    Then the following holds
    \[
    P
    \left(
    \sup_{\theta, \theta^\prime \in \Theta}
    \vert 
    X_{\theta}
    -
    X_{\theta^\prime}
    \vert 
    \leq 
    8(J+t)
    \right)
    \leq
    \frac{1}{\psi(t/D)},
    \quad
    \text{for all} t \geq 0,
    \]
    where $D$ is the diameter of the metric space $(\Theta, \rho)$, and the generalized Dudley entropy integral $J$ is given by 
    \[
    J
    \coloneqq
    \int_0^D
    \psi^{-1}
    (N(\delta; \Theta, \rho))
    d \delta,
    \]
    where $N(\delta; \Theta, \rho)$ is the $\delta$-covering number of $(\Theta, \rho)$.
\end{theorem}

The next technical lemma is in \cite{bai2024transformers}.
Let $\mathtt{B}^k_{\infty}(R) = [-R, R]^k$ denotes the standard $\ell_{\infty}$ ball in $\R^k$ with radius $R > 0$. 
\begin{definition}[Sufficiently smooth $k$-variable function]
We say a function $g : \R^k \mapsto \R$ is $(R, C_{\ell})$-smooth if for $s = \lceil (k-1)/2 \rceil + 2$, $g$ is a $C^s$ fnction on $\mathtt{B}_{\infty}^k(R)$, and
\[
\sup_{\bz \in \mathtt{B}_{\infty}^k(R)}
\norm{\nabla^i g(\bz)}_{\infty}
=
\sup_{\bz \in \mathtt{B}_{\infty}^k(R)}
\sup_{j_1,\cdots,j_i\in[k]}
| \partial_{x_{j1} \cdots x_{ji}}
g(\bx)
\leq 
L_i
\]
for all $i = 0, 1, \cdots ,s$, with $\max_{0\leq i \leq s} L_i R^i \leq C_{\ell}$.
 
\end{definition}


\begin{lemma}[Approximating smooth $k$-variable functions]\label{lem:approx}
    For any $\varepsilon_{\text{approx}} > 0$, $R \geq 1, C_{\ell} > 0$, we have the following:
    Any $(R, C_{\ell})$-smooth function $g : \R^k \mapsto \R$ is $(\varepsilon_{\text{approx}}, R, M, C)$-approximable by sum of relus with $M \leq C(k) C_{\ell}^2 \log( 1 + C_{\ell}/\varepsilon_{\text{approx}}^2 )$ and $C \leq C(k) C_{\ell}$, where $C(k) > 0$ is a constant that depends only on $k$, i.e.,
    \[
    f(\bz) =
    \sum_{m=1}^M c_m \sigma( \ba_m^\top [\bz ; 1])
    \quad\text{with }
    \sum_{m=1}^M |c_m| \leq C,
    \quad
    \max_{m\in[M]}
    \norm{\ba_m}_1 \leq ,    
    \]
    such that $\sup_{\bz \in [-R,R]^k}
    | f(\bz) - g(\bz) |\leq \varepsilon_{\text{approx}}
    $.
\end{lemma}


% \subsection{Kernel Regression with Transformer} 
% Here we first review the Nadaraya-Watson Kernel Estimator for kernel regression.
% %, and then review the leave-one-out cross validation for bandwidth selection.

% \paragraph{The Nadaraya-Watson Kernel Estimator.}
% Let $h > 0$ be the bandwidth, and $K(\cdot)$ be a smoothing kernel.
% The Nadaraya-Watson kernel estimator is formulated as follows:
% \begin{equation*}
%     \hat{r}(\bx)
%     \coloneqq
%     \sum_{i=1}^n
%     \gamma_i(\bx) \by_i,\quad\text{where }
%     \gamma_i(\bx)
%     \coloneqq
%     \frac{ K\left( \frac{\bx- \bx_i}{h} \right)  }{     \sum_{j=1}^n
%  K\left( \frac{\bx- \bx_i}{h} \right)}.
% \end{equation*}
% To approximate the Nadaraya-Watson Kernel Estimator, we consider the following input format
% \begin{equation*}
%     \bH 
%     =
%     \begin{bmatrix}
%         \bx_1 & \bx_2 & \dots & \bx_n & \bx_{\text{test}}
%         \\
%         \by_1 & \by_2 & \dots & \by_n & \by_{\text{test}}
%         \\
%         \bp_1 & \bp_2 & \dots & \bp_n & \bp_{\text{test}},
%     \end{bmatrix}
% \end{equation*}
% where we have $n$ in-context examples, and the goal is to estimate calculate $\hat{r}(\bx_{\text{test}})$ with those $n$ in-context examples.
% Now we show our first result of kernel regression via Transformer.
% \begin{theorem}[\normalfont Nadaraya-Watson kernel regression via Transformer]\label{thm:kernel-regression}
%     Let $K_h(\cdot) : \R^d \mapsto \R$ be a kernel with bandwidth $h > 0$ that is $(\varepsilon, R, M, C)$-approximable by sum of ReLUs.
%     There exists a 2-layered transformer $\text{TF}_\theta(\cdot)$ with first layer having a single head, second layer having $M$ heads,
%     that approximates kernel regression
%     \begin{equation}
%         \norm{ \texttt{read}\left( \text{TF}_\theta(\bH) \right) - \hat{r}( \bx_{\text{test}} ) }_2
%         \leq 
%         \frac{n B_y \cdot \varepsilon^2}{s},
%     \end{equation}
%     where $s \coloneqq  \sum_{j=1}^N K(\nicefrac{\bx - \bx_j}{h})$, and $\norm{\by_i}_2 \leq B_y$ for all $i \in [n]$.
% \end{theorem}




% \textcolor{red}{
% \subsection{Basics Operations}
% \begin{table}[t]
%     \centering
%     \caption{Caption}
%     \begin{tabular}{l|l}
%         \toprule
%             Operations & Method  \\
%         \hline
%         Zero-out a certain row & left matrix multiplication $\bW \bx$
%         \\
%         Zero-out a certain column  & right matrix multiplication $\bx \bW$
%         \\
%         \bottomrule
%     \end{tabular}
%     \label{tab:operation-summary}
% \end{table}
% }

\clearpage

\section{Proofs}\label{sec:proofs}
\subsection{Proof of Lemma~\ref{lem:input-causal}}\label{proof:lem-input-casual}
Here we prove a slightly simpler result with the positional encoding containing only zero vectors and a one-hot vector.
One can easily extend the proof by padding the weight matrices.

\begin{equation}\label{eqn:input-data2}
    \bfa H 
    \coloneqq
    \begin{bmatrix}
        x_1 & x_2 &  \dots & x_T & 0
        \\
        \bp_1 & \bp_2 & \dots & \bp_T &
        \bp_{T+1}
    \end{bmatrix}
    \in \R^{D \times (T+1)}
    ,
    \quad
    \bp_i
    \coloneqq
    \begin{bmatrix}
        \mathbf{0}_{d^\prime}
        \\
        \be_i
    \end{bmatrix}
    \in \R^{d^\prime + T}
    ,
\end{equation}

\begin{lemma}[Lemma~\ref{lem:input-causal} Restate]
    Given a sequence of token $\bfa H$ in the form of Equation~\ref{eqn:input-data2}, there exists a one-layer, $q-1$ head ReLU attention layer, such that the columns of $\text{Attn}_{\bm{\theta}}( \bfa H )$ has the following form:
    \begin{equation}
    \text{Attn}_{\bm{\theta}_1}^{\dagger}( \bfa H )_i
    \coloneqq
        \begin{bmatrix}
            x_i
            \\
            x_{i-1}
            \\
            \vdots
            \\
            x_{i-q}
            \\
            \bp_i^\prime
        \end{bmatrix},
        \quad
        \text{where }
        \bp_i^\prime 
        \coloneqq
        \begin{bmatrix}
        \mathbf{0}_{ d^\prime - q }
        \\
        1
        \\
        1 \{ i < T + 1 \}
        \end{bmatrix}
        \in \R^{ d^\prime - q + 2 }.
    \end{equation}
\end{lemma}




% \begin{proof}
%     We denote $\mathbf{0}_d \in \R^d$ as a $d$-dimensional zero vector. 
%     Considering the input $(x_1, \dots, x_T) \in \R^{d \times (T+1)}$.
%     Let $x_i \in \R^d$, 
%     $$\bW_V^h x_i \coloneqq 
%     \begin{pmatrix}
%          \mathbf{0}_{hd}
%          \\
%          x_i
%          \\
%          \mathbf{0}_{(H-h)d}
%     \end{pmatrix}
%     \in \R^{(H+1) d},$$
%     where $H$ is the number of heads with $h$ as head index.
%     For example, 
%     $$
%     \bW_V^1 x_i = 
%     \begin{pmatrix}
%         \mathbf{0}_d
%         \\
%         x_i
%         \\
%         \mathbf{0}_{(H-1)d}
%     \end{pmatrix}.
%     $$

%     We now view the attention layer, as the following form
%     \begin{align*}
%         \text{Attn}( \bH )
%         &=
%         \bH
%         +
%         \frac{1}{T}
%         \sum_{m=1}^M
%         \Vb_m \bH
%         \cdot
%         \sigma( 
%         \left(
%         \Qb_m \bH
%         \right)^\top
%         \left(
%         \Kb_m \bH
%         \right)
%         )
%         \\
%         &=
%         \bH+
%         \frac{1}{T}
%         \sum_{m=1}^M
%         \Vb_m \bH
%         \cdot
%         \Ab_m,
%     \end{align*}
%     where $\Ab_m = \sigma( 
%         \left(
%         \Qb_m \bH
%         \right)^\top
%         \left(
%         \Kb_m \bH
%         \right)
%         )$.
%     By doing so, the self-attention layer contains two types of operations: column-wise and row-wise moving,
%     with right and left matrix multiplications, respectively.
%     Therefore, with $\Vb_m$ being the row-moving operator, we now only have to design $\Kb_m, \Qb_m$, such that $\Ab_m$ shifts each column to the right by $m$ columns, which is trivial to construct with the positional encoding.
% \end{proof}


\begin{proof}
    Consider an input of the following form
    \begin{equation*}
        \bx
        =
        \begin{bmatrix}
            x_1 & x_2 & \cdots & x_T
            \\
            \bm{0} & \bm{0} & \cdots & \bm{0} 
            \\
            \be_1 & \be_2 & \cdots & \be_T
        \end{bmatrix},
    \end{equation*}
    where $\bx_t \in \R^{d}, \pb_t \in \R^T$, for all $t = 1, \cdots, T$.
    We construct weights of the $m$-th head $\bW_Q^m, \bW_Q^m$ as following,
    \begin{equation*}
        \bW_K^m
        =
        \begin{bmatrix}
            \bm{0}^\top & \bm{0}^\top & \be_1^\top
            \\
            \bm{0}^\top & \bm{0}^\top & \be_2^\top
            \\
            \vdots & \vdots & \vdots 
            \\
            \bm{0}^\top & \bm{0}^\top & \be_T^\top            
        \end{bmatrix},
        \quad
        \bW_Q^m
        =
        \begin{bmatrix}
            \bm{0}^\top & \bm{0}^\top & \be_{1-m}^\top
            \\
            \bm{0}^\top & \bm{0}^\top & \be_{2-m}^\top
            \\
            \vdots & \vdots & \vdots 
            \\
            \bm{0}^\top & \bm{0}^\top & \be_{T-m}^\top            
        \end{bmatrix},
    \end{equation*}
    where we define the negative index as rotational index, i.e., $\be_{-1} = \be_{T}, \be_{-2} = \be_{T-1}$.
    We have
    \begin{align*}
        \left(
        \bW_K^m
        \Xb
        \right)^\top
        \left(
        \bW_Q^m
        \Xb
        \right)
        &=
        \begin{bmatrix}
            \be_1^\top
            \\
            \be_2^\top
            \\
            \vdots 
            \\
            \be_T^\top
        \end{bmatrix}^\top
        \begin{bmatrix}
            \be_{1-m}^\top
            \\
            \be_{2-m}^\top
            \\
            \vdots
            \\
            \be_{T-m}^\top
        \end{bmatrix}
        \\
        &=
        \bI_T
        \begin{bmatrix}
            \be_{1-m}
            \\
            \be_{2-m}
            \\
            \vdots
            \\
            \be_{T-m}
        \end{bmatrix}.
    \end{align*}
    Note that the result of 
        $\sigma\left(\left(
        \bW_K^m
        \Xb
        \right)^\top
        \left(
        \bW_Q^m
        \Xb
        \right)
        \right)$
    is a rotation matrix, where right multiplication on $\Xb$ will rotate the columns of $\Xb$.
    Therefore, we have 
    $\bW_V^m$ that performs row-wise shifting and the attention matrix 
    $\sigma\left(\left(
        \bW_K^m
        \Xb
        \right)^\top
        \left(
        \bW_Q^m
        \Xb
        \right)
        \right)$
        performs column-wise shifting.
\end{proof}


\subsection{Proof of \cref{thm:any-variate-auto}}\label{proof:any-var-enc}

\paragraph{Autoregressive Linear Regression under Any-Variate Encoding.}
The ultimate goal of this setup is to perform the following mechanism.
Let $\bx$ be the target variate we wish to predict, $\bz^j$ be the $j$-th covariate of $\bx$, for $j \in [M]$.
We denote the lookback window size as $q$, and each covariate has length $T$ ($T$-time steps.).
We denote the time encoding as $\bp_i$ for $i \in [T]$, and the variate encoding as $\bq_{j}$ for $j \in [M]$.
Finally, our goal is to predict $\bx_T$.

\begin{equation*}
\begin{bmatrix}
    x_1^1 & \cdots & x_T^1 & x_1^2 & \cdots & x_T^2 & \cdots & x_1^d & \cdots & x_T^d
    \\
    \bp_1 & \cdots & \bp_T & \bp_1 & \cdots & \bp_T & \cdots & \bp_1 & \cdots & \bp_T
    \\
    \be_1 & \cdots & \be_1 & \be_2 & \cdots & \be_2 & \cdots & \be_d & \cdots & \be_d
\end{bmatrix}
\mapsto
\begin{bmatrix}
    \mathtt{A}_1(q) & \cdots
    \\
    \mathtt{A}_2(q) & \cdots
    \\
    \vdots
    \\
    \mathtt{A}_d(q) & \cdots
    \\
    \vdots & \ddots
\end{bmatrix}.
\end{equation*}

Here, different colors represent different covariates.
The motivation for performing such an operation is to apply the in-context learning property of transformers proved in \cite{bai2024transformers}.


\begin{lemma}[Lemma~\ref{lem:mar-group-wise} Restate]
Define the matrix $\mathtt{A}_i(q)$ for the $i$-th covariates $(x_1^i, \cdots, x_T^i)$, with order $q$, such that
\begin{equation*}
    \mathtt{A}_i(q)
    \coloneqq
    \begin{bmatrix}
        x_1^i & x_2^i & \cdots & x_t^i & x_{t+1}^i & x_{t+2}^i & \cdots \\
        x_T^i & x_{T-1}^i & \cdots & x_{t-1}^i & x_t^i & x_{t+1}^i & \cdots \\
        x_{T-1}^i & x_{T-2}^i & \cdots & x_{t-2}^i & x_{t-1}^i & x_{t}^i & \cdots \\
        \vdots & \vdots & \vdots & \vdots & \vdots & \vdots & \cdots \\
        x_{T-q}^i & x_{T-q+1}^i & \cdots & x_{t-q}^i & x_{t-q+1}^i & x_{t-q+2}^i & \cdots \\   
    \end{bmatrix},
\end{equation*}
where in the $j$-th column of $\mathtt{A}_i(q)$, it contains historical values of $x_j^i$ with lag $q$.

Given fixed $D, T \in \N^+$, where $T > q$.
For any input matrix $\bH$ in the form of Any-Variate Encoding in Equation~\ref{eqn:AV-encoding}, such that $\bH \in \R^{ D^\prime \times dT^\prime }$, and $D^\prime \leq D$, $T^\prime < T$.
There exists a 1-layer, $q$ head Any-Variate Attention that performs the following operation.
    \begin{equation*}
    \begin{bmatrix}
        x_1^1 & \cdots & x_T^1 & x_1^2 & \cdots & x_T^2 & \cdots & x_1^d & \cdots & x_T^d
        \\
        \bp_1 & \cdots & \bp_T & \bp_1 & \cdots & \bp_T & \cdots & \bp_1 & \cdots & \bp_T
        \\
        \be_1 & \cdots & \be_1 & \be_2 & \cdots & \be_2 & \cdots & \be_d & \cdots & \be_d
    \end{bmatrix}
    \mapsto
    \begin{bmatrix}
        \mathtt{A}_1(q) & \mathtt{A}_2(q) & \cdots & \mathtt{A}_d(q)
        \\
        \ddots & \ddots & \cdots & \ddots 
    \end{bmatrix}
    \end{equation*}
\end{lemma}

\begin{proof}
    The proof comes as a direct corollary of Lemma~\ref{lem:moirai-group-wise} and \citep[Proposition~A.5]{bai2024transformers}.
    By Lemma~\ref{lem:input-causal}, there exists a single layer standard transformer layer (with $\bW_1, \bW_2$ being $0$s) that generates $\mathtt{A}_i(q)$ for each uni-variate (covariate).
    It then left applying Lemma~\ref{lem:moirai-group-wise} for variate-wise operation and applying \citep[Proposition~A.5]{bai2024transformers} to keep the time indices $\bp_t$ unchanged.
        
\end{proof}


\begin{corollary}\label{cor:second-step-enc}
    There exists a $d_{\max}$ head standard attention layer that performs the following
    \[
        \begin{bmatrix}
            \mathtt{A}_1(q) & \mathtt{A}_2(q) & \cdots & \mathtt{A}_d(q)
            \\
            \ddots & \ddots & \cdots & \ddots 
        \end{bmatrix}
        \mapsto
        \begin{bmatrix}
            \mathtt{A}_1(q) & \cdots 
            \\
            \tilde{\mathtt{A}}_2(q) & \cdots
            \\
            \vdots
            \\
            \tilde{\mathtt{A}}_d(q) & \cdots
            \\
            \ddots & \ddots 
        \end{bmatrix},
        \quad\text{for any }d \leq d_{\max},
    \]
    where $\tilde{\mathtt{A}}_i(q)$ is $\mathtt{A}_i(q)$ without the first row.
\end{corollary}
\begin{proof}
Note that this operation in \cref{cor:second-step-enc} is straightforward with \cref{lem:input-causal} and \citep[Proposition~A.5]{bai2024transformers}.
As for each $i \in [d]$, $i \neq 1$, the attention layer performs two operations to each element of $\mathtt{A}_i(q)$:

\begin{align*}
    \begin{cases}
        iT \text{ columns to the left } & \text{ right multiplication} \\
        q_{\max} \text{ rows below } & \text{ left multiplication} \\
        \text{zero out} & \text{if in first row (left multiplication)}
    \end{cases}.
\end{align*}
Note that one can simply construct weight matrices to perform the above permutations and masking.
In total, we need $d_{\max}$ heads to perform such operations for each $\mathtt{A}_i(q)$, for any $d \leq d_{\max}$.
For $q < q_{\max}$, the remaining entries will be zero padded.
Finally, with at best $2$ layers of $d_{\max}$ head any-variate attention, we then obtain
\[
\tilde{H}^{(2)}
\coloneqq
    \begin{bmatrix}
        \mathtt{A}_1(q) & \cdots 
        \\
        \tilde{\mathtt{A}}_2(q) & \cdots
        \\
        \vdots
        \\
        \tilde{\mathtt{A}}_d(q) & \cdots
        \\
        \bp
        \\
        \be
    \end{bmatrix}
    =
    \begin{bmatrix}
     \cdots & x_{T-1}^1 & \mathcolor{red}{x_T^1} &\cdots 
    \\
    \cdots & x_{T-2}^1 & \mathcolor{blue}{x_{T-1}^1} &\cdots
    \\
    \vdots & \vdots & \mathcolor{blue}\vdots &\cdots
    \\
    \cdots & x_{T-q}^1 & \mathcolor{blue}{x_{T-q}^1} &\cdots
    \\
    \cdots & x_{T-1}^d & \mathcolor{blue}{x_{T-1}^d}
    \\
    \cdots & x_{T-2}^d & \mathcolor{blue}{x_{T-2}^d}
    \\
    \cdots & \vdots & \mathcolor{blue}\vdots \\
    \\
    \cdots & x_{T-q}^d & \mathcolor{blue}{x_{T-q}^d}
    \\
    \cdots & \bp_{T-1} & \bp_{T}
    \\
    \cdots & \be_1 & \be_1
    \end{bmatrix},
\]
where $\bp$ is the matrix of $(\bp_1, \cdots, \bp_T)$, $\be$ is the matrix of $(\be_1, \cdots, \be_1)$.

Note that $x_T^1$ in red is the target we wish to predict (masked as 0 initially), and the entries in blue is considered the input feature of our AR model (a linear regression model in this case),
and we are able to directly apply several theoretical results in \cite{bai2024transformers} with input $\tilde{\bH}^{(2)}$.
Specifically, for \cref{thm:any-variate-auto}, it follows directly from \citep[Theorem~4]{bai2024transformers} by setting $\lambda = 0$.

\end{proof}

Next, we present several approximation results from \cite{bai2024transformers}, which our approximation results follows immediately from.
Considering the general form of autoregressive data:
$\bx \in \R^{d \times T} \coloneqq (\bx_1, \dots, \bx_T)$, where $\bx_t = (x_t^1, \cdots, x_t^d) \in \R^d$.
Assuming our target (variate of interest) is in dimension $1$, we assume the autoregressive process generates $x_t^1$ as follows:
\begin{equation}
    x_{t}^1
    =
    f( \bx_{t-q: t-1}^{1:d})
    +
    \epsilon_t
    ,
\end{equation}
where $\epsilon_t \sim N(0, \sigma^2)$, $a_i^j \in \R^1$, and $f$ is a function of interest.
% Note that $f : \R^{q\times d} \mapsto \R$ is a function of interest.
We then present several results when $f$ varies.
% \paragraph{Generalized Linear Models.}
% Following the above setup, let $f : \R \mapsto \R$ be a link function that is non-decreasing and $C^2$-smooth that operates element-wise on input matrices.
% Considering the following convex ERM problem:
% \begin{align*}
% \hat{\bw}_{\text{GLM}}
% &\coloneqq
% \argmin_{\bw \in \R^{dq}}
% \hat{L}_{GLM} (\bw)
% \coloneqq
% \frac{1}{T-1}
% \sum_{t=1}^{T-1}
% \ell
% \left(
% \langle 
% \bw ,
% [\bx_{t-1:t-q}^1 ; \cdots ; \bx_{t-1:t-q}^d]
% \right),
% \end{align*}
% where $\ell(t, j) = -yt + \int_0^t f(s) ds$ is the convex loss associated with $f$.

% \begin{proposition}[Generalized Linear Models ICL with MOIRAI]
% For any $0 < \alpha < \beta$ with $\kappa= \nicefrac{\beta}{\alpha}$, assume \cref{assumption:effective-regression} holds for the above problem, $B_w > 0$, $B_x > 0$, $\kappa_w$
\paragraph{Non-Linear AR.}
% \end{proposition}
Here we analyze that when the autoregressive process is generated by a 2 layer ReLU network with look back window size $q$.
Suppose the prediction function $\text{pred}(\bx, \bw) \coloneqq \sum_{k=1}^K u_k r ( \bv_k^\top \bx)$ is given by a two-layer neural network, parameterized by $\bw = [\bw_k, u_k]_{k\in[K]} \in \R^{K(d+1)}$.
Consider the ERM problem:
\[
\min_{\bw \in \cW}
\hat{L}_N(\bw)
\coloneq
\frac{1}{2N}
\sum_{i=1}^N
\ell( \text{pred}(\bx_i, \bw), y_i )
=
\frac{1}{2N}
\sum_{i=1}^N
\ell
\left(
\sum_{k=1}^K
u_k r(\bv_k^\top \tilde{\bx}_i), x_T^1 
\right),
\]
where $\cW$ is a bounded domain and $\tilde{\bx}_i \in \R^{qd}$ is a flatten version of $\bx_{t-q:t-q} \in \R^{d\times q}$.
\begin{proposition}
    Fix any $B_w, B_u > 0$, $L \geq 3, \nu > 0$, and $\varepsilon > 0$.
    Suppose that
    \begin{enumerate}
        \item Both the activation function $r$ and the loss function $\ell$ is $C^4$-smooth.
        \item $\cW$ is a closed domain such that $\cW \subset \left\{ \bw = [\bv_k;u_k]_{k\in[K]} \in \R^{K(d+1)} : \norm{\bv_k}_2 \leq B_v, |u_K| \leq B_u \right\}$, and $\text{Proj}_{\cW} = \text{MLP}_{\bm{\theta_2}}$ for some MLP layer with hidden dimension $D_w$ and $\norm{\bm{\theta}_2}_{\text{op}} \leq C_w$.
    \end{enumerate}
    Then there exists a $(L_1 + 2 L_2)$-layer MOIRAI transformer with 
    \begin{align*}
    \max_{\ell\in[L_1+1,2L_2]}
    M^{(\ell)} \leq \tilde{O}(\varepsilon^{-2}), \quad
    \max_{\ell\in[L_1+1,2L_2]}
    D^{(\ell)} \leq \tilde{O}(\varepsilon^{-2}) + D_w, \quad
    \\
    \norm{\bm{\theta}}_{\text{op}} \leq O(1+\eta) + C_w, \quad
    \sum_{\ell=1}^{L_1} M^{(\ell)} = d_{\max} + q_{\max}.
    \end{align*}
    where we hide the constants $K, B_x, B_u, B_v, C^4$,
    satisfies the following
    \[
    \norm{ \hat{\bw} - \bw^{L}_{\text{GD}} }_2
    \leq 
    L_f^{-1} ( 1 + \eta L_f)^L \varepsilon,
    \]
    where $L_f = \sup_{w\in\cW} \norm{ \nabla^2 \hat{L}_N(\bw) }_2$.
    
\end{proposition}

\paragraph{Maximum Likelihood Estimation (Gaussian) via Transformers.}
The next result shows that MOIRAI transformers are also capable of performing maximum likelihood estimation on any input multi-variate time series.
Given a data generated by some $\mathtt{AR}_d(q)$ process with parameter $(\bw_1, \cdots, \bw_q) \subset \R^{d}$: 
$(\bx_1, \cdots, \bx_T) \subset \R^d$, the conditional likelihood $f(\cdot)$ of observing $\bx_t$ is
\[
f(\bx_t \mid \bx_{t-1}, \cdots, \bx_{t-q} )
\coloneq
\frac{1}{\sqrt{2\pi \sigma^2}}
\exp
\left(
\frac{-( \bx_t - \sum_{i=1}^q \langle \bw_i,  \bx_{t-i} \rangle )^2 }{2 \sigma^2}
\right).
\]
The goal is to estimate the mean vector $(\bw_1, \cdots, \bw_q)$ and the variance $\sigma^2$ by minimizing the negative log-likelihood loss.
Note that with $n \geq d$, the loss is strongly convex.
The optimization over the NLL Loss has two steps: estimating the mean vector: $\hat{\bw}$, and then derive the variance $\hat{\sigma}^2$ with the following closed-form solution:
\[
\sigma^2
=
\frac{1}{T}
\sum_{t=1}^T
\left(
\bx_t - \sum_{i=1}^q
\langle \hat{\bw}_i, \bx_{t-i} \rangle
\right)^2.
\]

\begin{theorem}\label{thm:moirai-mle}
    Given a set of input data generated by some $\mathtt{AR}_d(q)$ process: $\bX, \in \R^{n \times d}, \bY \in \R^n$, considering the following negative log-likelihood loss, the goal is to find a set of parameters $\bw \in \R^d, \sigma^2 \in \R^+$ to minimize the following loss
    \[
    L_{\text{NLL}}
    ( \bw, \sigma )
    \coloneq
    \frac{n}{2}
    \log (2 \pi \sigma^2) +
    \frac{1}{2\sigma^2}
    \sum_{t=q+1}^T
    \left(
    \bx_t - \sum_{i=1}^q
    \langle \bw_i, \bx_{t-i} \rangle
    \right)^2
    \]
    We denote $\bw^\star, \sigma^\star$ as the ERM satisfying the NLL Loss.
    There exists a $(L_1 + L_2 + 2)$-layer MOIRAI Transformer such that its first $L_1+L_2$ layers follow the same requirement in \cref{thm:any-variate-auto}, and the last two layers each has two and one heads, it estimates $\bw, \bm{\sigma}$ with bounded error:
    \[
    \norm{ \hat{\bw} - \bw^\star } \leq
    \varepsilon,
    \]
    and the estimated variance is bounded by
    \[
    \left\vert \hat{\sigma}^2 - {\sigma^\star}^2 \right\vert
    \leq
    2 E B_x \varepsilon + B_x^2 \varepsilon
    =
    \tilde{O}(\varepsilon + \varepsilon^2),
    \]
    where $E \leq B(1+B_w)$, and $\tilde{O}$ hides the values dependent on $B_x, B_w$.
\end{theorem}


\begin{proof}[Proof of \cref{thm:moirai-mle}]
    
\[
L_{\text{NLL}}
( \bw, \sigma )
\coloneq
\frac{n}{2}
\log (2 \pi \sigma^2) +
\frac{1}{2\sigma^2}
\sum_{t=1}^T
\left(
\bx_t - \sum_{i=1}^q
\langle \bw_i, \bx_{t-i} \rangle
\right)^2.
\]
Following \cref{thm:any-variate-auto}, the first $L_1 + L_2$ layers of MOIRAI obtains $\hat{\bw}$ such that the $L_1+L_2+1$-th layer takes the following as input
% Finally, by following the proof in \citep[Proposition~C.2]{bai2024transformers}, we get
\begin{align*}
    \tilde{\bh}_i^{(L_1+L_2)}
    =    \left[
    x^1_i ; 
    \bx_{i-1:i-q}^1
    ;
    \bx_{i-1:i-q}^2
    ;
    \cdots 
    ;
    \bx_{i-1:i-q}^d
    ; 
    \bw^\star + \bm{\varepsilon}
    ; \bm{0} ; 1; t_i
    \right],
\end{align*}
where $\bw^\star + \bm{\varepsilon} \in \R^{qd}$ is the flatten mean vectors.
For the simplicity of notations, for the $i$-th column, we denote $x^1_i$ with $\tilde{\by}_i$, and denote $[\bx_{i-1:i-q}^1
    ;
    \bx_{i-1:i-q}^2
    ;
    \cdots 
    ;
    \bx_{i-1:i-q}^d]$ as $\tilde{\bx}_i \in \R^{qd}$, as they correspond to the label and feature of our AR model, respectively.
$\bm{\varepsilon} \in \R^{dq}$ satisfies 
\[
\norm{\bm{\varepsilon}}
\leq
\varepsilon \cdot (\eta B_x).
\]

Now we start to construct the $(L_1+L_2+1)$-th layer.
One can then construct
\begin{align*}
    \bQ_1^{L+1} \bh^L_i &= [ \bm{0} ; \tilde{\bx}_i ; \bm{0} ], \quad \bK^{L+1}_1 \bh^L_j = [ \bm{0} ; \hat{\bw} ; \bm{0} ], \quad \bV^{L+1}_1 \bh_k^L = [   \bm{0} ; 1 ; \bm{0} ]
    \\
    \bQ_2^{L+1} \bh^L_i &= [ \bm{0} ; \tilde{\bx}_i ; \bm{0} ], \quad \bK^{L+1}_2 \bh^L_j = [ \bm{0} ; -\hat{\bw} ; \bm{0} ], \quad \bV^{L+1}_2 \bh_k^L = [ \bm{0} ; -1 ; \bm{0} ].
\end{align*}

The above construction gives us
\begin{align*}
    \bh_i^{L+1}
    &=
    \bh_i^{L}
    +
    \frac{1}{n}
    \sum_{j=1}^{n}
    \sum_{m=1}^2
    \sigma\left(
    \langle 
    \bQ^{L+1}_m \bh_{n+1}^L,
    \bK_m^{L+1}
    \bh_{j}^L
    \rangle
    \right)
    \bV_m^{L+1}
    \bh_j^L
    \\
    &=
    [ \tilde{\by}_i ; \tilde{\bx}_i; \hat{\bw} ; \bm{0} ; 1 ; t_i]
    +
    \left(
    \sigma
    (
    \langle \hat{\bw}, \bx_i \rangle
    )
    -
    \sigma 
    (
    -
    \langle \hat{\bw}, \bx_i \rangle
    )
    \right)
    \cdot [ \bm{0} ; 1 ; \bm{0} ]
    \\
    &=
    [ \tilde{\by}_i ; \tilde{\bx}_i; \hat{\bw} ;  \langle \hat{\bw} , \tilde{\bx_i} \rangle,  \bm{0} ; 1 ; t_i].
\end{align*}

Next, we construct the last layer as
\begin{align*}
    \bQ_1^{L+1} \bh^L_i &= [ ... ; \tilde{\by}_i -  \langle \hat{\bw}, \tilde{\bx}_i \rangle  ; ... ], \quad \bK^{L+1}_1 \bh^L_j = [ ... ; \tilde{\by}_j -  \langle \hat{\bw}, \tilde{\bx}_j \rangle ;  ... ], \quad \bV^{L+1}_1 \bh_k^L = [ \bm{0} ; 1 ; \bm{0} ]
    % \\
    % \bQ_1^{L+1} \bh^L_i &= [ \bm{0} ; \bx_i ; \bm{0} ], \quad \bK^{L+1}_1 \bh^L_j = [ \bm{0} ; -\hat{\bw} ; \bm{0} ], \quad \bV^{L+1}_1 \bh_k^L = [ \bm{0} ; 1 ; \bm{0} ].
\end{align*}

Finally, the result becomes 
\begin{align*}
    \bh_{i}
    =
    \frac{1}{n}
    [ ... ; \sum_{\mu=1}^n ( \by_\mu - \langle \bx_\mu , \hat{\bw} \rangle )^2; ... ]
    =
    [ ... ; \hat{\sigma^2}; ... ].
\end{align*}

% \begin{align*}
%     \langle \hat{\bw}, \tilde{\bx}_t \rangle
%     &=
%     \langle \bw^\star + \bm{\varepsilon}, \tilde{\bx}_t \rangle
%     \\
%     &=
%     \langle \bw^\star, \tilde{\bx}_t \rangle
%     +
%     \langle \bm{\varepsilon}, \tilde{\bx}_t \rangle
%     \\
%     &\leq
%     B_w B_x \cdot d) +
%     \varepsilon \eta B_x^2 (d)
% \end{align*}
Thus, we complete the proof.
\end{proof}

% \subsection{Proof of Kernel Regression}

% \begin{proof}
% We first apply our Lemma~\ref{lem:input-causal} to transform the input sequence into the following using a 1-layer attention layer
% \begin{equation*}
%     \tilde{\bH}^{(1)}
%     =
%     \begin{bmatrix}
%         \bx_1 & \bx_2 & \dots & \bx_{\text{test}}
%         \\
%         \bx_{\text{test}} & \bx_{\text{test}} & \dots & \bx_{\text{test}}
%         \\
%         \by_1 & \by_2 & \dots & \by_{\text{test}}
%         \\
%         \bp_1 & \bp_2 & \dots & \bp_{\text{test}}
%     \end{bmatrix}.
% \end{equation*}
%     We then set the $\bW^{(0)}_1, \bW^{(0)}_2$ to be zero matrices.
%     Note that the $\by$ and $\bp$ rows are remained unchanged due to  \cite[Proposition~A.5]{bai2024transformers}\footnote{ (Joining parallel single-layer transformers}.
    
%     Next we set 
%     \begin{equation*}
%         \tilde{\bW}_Q^{(m)}
%         =
%         \begin{bmatrix}
%             \frac{1}{h} I_{2d}
%             & \mathbf{0}
%             \\ 
%             \mathbf{0} &
%             I
%         \end{bmatrix}
%         +
%         \begin{bmatrix}
%             0 & -1/h & 0 & \dots & 0
%             \\
%             0 & -1/h & 0 & \dots & 0
%             \\
%             \vdots & \vdots & \vdots & \dots & \vdots
%             \\
%             0 & -1 & 0 & \dots & 0
%         \end{bmatrix},
%     \end{equation*}
%     where entries in the first $2d$ rows and columns is multiplied by $1/h$.

%     By setting $\bW_Q$ as following (we omit the layer index here for simplicity)
%     \begin{equation*}
%         \bW^\prime_Q 
%         \tilde{\bW}_Q
%         \tilde{\bH}^{(2)}
%         =
%         \begin{bmatrix}
%             a_m \cdot \frac{\bx_1 - \bx_{\text{test}}}{h}
%             &
%             a_m \cdot\frac{\bx_2 - \bx_{\text{test}}}{h}
%             & 
%             \dots 
%             &
%             \bx_{\text{test}}
%             \\
%          \by_1 & \by_2 & \dots & \by_{\text{test}}
%         \\
%         \bp_1 & \bp_2 & \dots & \bp_{\text{test}}
%         \end{bmatrix},
%     \end{equation*}
%     and set $\bW_K$ as 
%     \begin{equation*}
%         \bW_K \tilde{\bH}^{(2)}
%         =
%         I,
%     \end{equation*}
%     we get
%     \begin{align*}
%         \sum_{m=1}^M
%         \bW_V^{(m)}
%         \tilde{\bH}^{(m)}
%         \sigma \left( \left\langle \bW_Q^{(m)} \tilde{\bH}^{(m)},  \bW_K^{(m)} \tilde{\bH}^{(m)} \right\rangle \right)
%         &=
%         \sum_{m=1}^M
%         c_m
%         \sigma
%         \left(
%         a_m \cdot 
%         \frac{\bx_m - \bx_{\text{test}}}{h}
%         \right)
%         \\
%         &=
%         \begin{bmatrix}
%             f\left(\frac{\bx_1 - \bx_{\text{test}}}{h}\right)
%             &
%             f\left(\frac{\bx_2 - \bx_{\text{test}}}{h}\right)
%             & 
%             \dots 
%             &
%             \bx_{\text{test}}
%             \\
%         \by_1 & \by_2 & \dots & \by_{\text{test}}
%         \\
%         \bp_1 & \bp_2 & \dots & \bp_{\text{test}}
%         \end{bmatrix}
%         \\
%         &=
%         \begin{bmatrix}
%             \epsilon + K\left(\frac{\bx_1 - \bx_{\text{test}}}{h}\right)
%             &
%             \epsilon + K\left(\frac{\bx_2 - \bx_{\text{test}}}{h}\right)
%             & 
%             \dots 
%             &
%             \bx_{\text{test}}
%             \\
%         \by_1 & \by_2 & \dots & \by_{\text{test}}
%         \\
%         \bp_1 & \bp_2 & \dots & \bp_{\text{test}}
%         \end{bmatrix}.
%     \end{align*}

%     The approximation error is then bounded by
%     \begin{align*}
%         \norm{
%         \hat{r}
%         (\bx)
%         -
%         \frac{\sum_{i=1}^N
%         f( \frac{\bx - \bx_i}{h} ) \by_i}{\sum_{i=1}^N
%         f( \frac{\bx - \bx_i}{h} )}
%         }
%         &=
%         \norm{
%         \sum_{i=1}^N
%         \frac{
%         K( \frac{\bx - \bx_i}{h} ) \by_i}{\sum_{j=1}^N
%         K( \frac{\bx - \bx_j}{h} )}
%         -
%         \sum_{i=1}^N
%         \frac{
%         f( \frac{\bx - \bx_i}{h} ) \by_i}{\sum_{j=1}^N
%         f( \frac{\bx - \bx_j}{h} )}
%         }    
%         \\
%         &=
%         \norm{
%         \sum_{i=1}^N
%         \left[
%         \frac{
%         K( \frac{\bx - \bx_i}{h} ) \by_i}{\sum_{j=1}^N
%         K( \frac{\bx - \bx_j}{h} )}
%         -
%         \frac{
%         f( \frac{\bx - \bx_i}{h} ) \by_i}{\sum_{j=1}^N
%         f( \frac{\bx - \bx_j}{h} )}
%         \right]
%         }    
%         \\
%         &=
%         \norm{
%         \sum_{i=1}^N
%         \left[
%         \langle \gamma, \by \rangle
%         -
%         \langle \gamma_f, \by \rangle
%         \right]
%         }    
%         \\
%         &\leq
%         \sum_{i=1}^N
%         \norm{
%         \langle \gamma, \by \rangle
%         -
%         \langle \gamma_f, \by \rangle
%         }    
%         \\
%         &=
%         \sum_{i=1}^N
%         \norm{
%         \langle \gamma - \gamma_f, \by \rangle
%         }    
%         \\
%         &\leq
%         \sum_{i=1}^N
%         \norm{
%         \gamma - \gamma_f
%         }
%         \norm{\by}
%         \\
%         &\leq
%         \frac{n B_y \varepsilon^2}{s},
%     \end{align*}
%     where $s \coloneqq  \sum_{j=1}^N K(\nicefrac{\bx - \bx_j}{h})$, and the last equality comes from Lemma~\ref{lem:prob-perturbation}.
% \end{proof}


\clearpage


\subsection{Proof of the Lipschitzness of Any-Variate Transformers}\label{proof:tr-lipschitz}

We first show the Lipschitzness of each component in an Any-Variate Transformer.
For any $p \in [1, \infty]$, let $\norm{\bH}_{2,p} \coloneqq ( \sum_{i=1}^N \norm{\bh_i}_2^p )^{1/p}$ denote the column-wise $(2, p)$-norm of $\bH$.
For any radius $\mathtt{R} > 0$, we denote 
$\cH_{\mathtt{R}} \coloneqq \{ \bH : \norm{\bH}_{2, \infty} \leq \mathtt{R} \} $ be the ball of radius $\mathtt{R}$ under norm $\norm{\cdot}_{2, \infty}$.


\begin{lemma}\label{lem:attn-lipschitz}
    For a single Any-Variate attention layer, $\bm{\theta}_1 = \{(\bV_m, \bQ_m, \bK_m, u^1_m, u^2_m)\}_{m\in[M]}$, we introduce its norm
    \begin{equation*}
        \Vert
        \bm{\theta}_1
        \Vert
        \coloneqq
        \max_{m\in[M]}
        \{ 
            \norm{\bQ_m}_{\text{op}},
            \norm{\bK_m}_{\text{op}},
            \vert u^1_m \vert,
            \vert u^2_m \vert
        \}
        +
        \sum_{m=1}^M \norm{\bV_m}_{\text{op}}
    \end{equation*}
    For any fixed hidden dimension $D^\prime$, we consider
    \begin{equation*}
        \Theta_{1, B}
        \coloneqq
        \{ 
        \bm{\theta}_{1}
        :
        \lvert \lvert \lvert
        \bm{\theta}_1
        \rvert \rvert \rvert
        \leq 
        B
        \}.
    \end{equation*}
    Then for $\bH \in \cH_{\mathtt{R}}$, $\bm{\theta}_1 \in \Theta_{1, B}$, the function 
    $(\bm{\theta}_1, \bH) \mapsto \text{Attn}_{ \bm{\theta}_1}$ is $(1+\iota)$-Lipschitz w.r.t. $\bm{\theta}_1$, where $\iota = \max \{  B^2\mathtt{R}^2 +T + (T-1)d, B(T-1)d  \}$,
    and $(1 + B^3 \mathtt{R}^2)$-Lipschitz w.r.t. $\bH$.
\end{lemma}

\begin{proof}
    Given some $\epsilon > 0$, some set $X$ and a function class $\mathcal{F}$.
    If $\mathcal{F}$ is $L$-Lipschitzness, i.e., 
    \begin{equation*}
        \norm{f(x_1) - f(x_2)} \leq
        L \norm{x_1 - x_2}, \quad \text{for all }f \in \mathcal{F}.
    \end{equation*}
    Then, the following holds
    \begin{equation*}
        N(\epsilon, \mathcal{F}, \norm{\cdot})
        \leq
        N( \nicefrac{\epsilon}{L}, X, \norm{\cdot} ).
    \end{equation*}

    Define 
    \begin{equation*}
        \Theta_{\text{attn}, B}
        \coloneqq
        \{ \theta_{\text{attn}}:  \lvert \lVert \theta_{\text{attn}} \rVert \rvert  \leq B \}.
    \end{equation*}

    The output of the Any-Variate Attention $[\tilde{h}_i]$ is given by
    \begin{equation*}
        \tilde{h}_i
        =
        h_i
        +
        \sum_{m=1}^M
        \frac{1}{N}
        \sum_{j=1}^N
        \sigma
        \left(
        \Braket{ \Qb_m h_i, \Kb_m h_j }
        \cdot \Vb_m h_j
        +
        u_m^1
        \star 
        \Ub
        +
        u_m^2
        \star
        \bar{\Ub}
        \right).
    \end{equation*}
    We also define $\theta^\prime_{\text{attn}} = \{ \Vb_m^\prime, \Qb_m^\prime, \Kb_m^\prime, u_m^{1\prime}, u_m^{2 \prime} \}_{m \in [M]}$.
    $\tilde{h}^\prime_i$ as
    \begin{equation*}
        \tilde{h}_i^\prime
        =
        h_i
        +
        \sum_{m=1}^M
        \frac{1}{N}
        \sum_{j=1}^N
        \sigma
        \left(
        \Braket{ \Qb_m^\prime h_i, \Kb_m^\prime h_j }
        \cdot \Vb_m^\prime h_j
        +
        u_m^{1\prime}
        \star 
        \Ub
        +
        u_m^{2 \prime}
        \star
        \bar{\Ub}
        \right).
    \end{equation*}

    Now we bound 
    $
    \norm{
    \text{Attn}_{\bm{\theta}_1}( \bH )
    -
    \text{Attn}_{\bm{\theta}_1^\prime}(\bH)
    }_{2, \infty}
    =
    \max_i 
    \norm{ \tilde{\bh}_i - \tilde{\bh}_i^\prime }_2 
    $ as follows
    \begin{align*}
        \norm{ \tilde{\bh}_i - \tilde{\bh}_i^\prime }_2 
        &=
        \Bigg\Vert
        \sum_{m=1}^M
        \frac{1}{N}
        \Bigg[
        \sum_{j=1}^N
        \sigma
        \left(
        \Braket{ \Qb_m h_i, \Kb_m h_j }
        +
        u^1_m
        \star 
        \Ub
        +
        u^2_m
        \star
        \bar{\Ub}
        \right)
        \Vb_m h_j
        -
        \\
        &\quad\quad
        \sum_{j=1}^N
        \sigma
        \left(
        \Braket{ \Qb_m^\prime h_i, \Kb_m^\prime h_j }
        +
        u_m^{1\prime}
        \star 
        \Ub
        +
        u_m^{2 \prime}
        \star
        \bar{\Ub}
        \right)
        \Vb_m^\prime h_j
        \Bigg]
        \Bigg\Vert_2
        \\
        &\leq
        \sum_{m=1}^M
        \frac{1}{N}
        \sum_{j=1}^N
        \Big\Vert
        \sigma
        \left(
        \Braket{ \Qb_m h_i, \Kb_m h_j }
        +
        u^1_m
        \star 
        \Ub
        +
        u^2_m
        \star
        \bar{\Ub}
        \right)
        \Vb_m h_j 
        \\
        &\quad\quad
        -
        \sigma
        \left(
        \Braket{ \Qb_m^\prime h_i, \Kb_m^\prime h_j }
        +
        u_m^{1\prime}
        \star 
        \Ub
        +
        u_m^{2 \prime}
        \star
        \bar{\Ub}
        \right)
        \Vb_m^\prime h_j
        \Big\Vert_2
        \\
        &\leq
        \sum_{m=1}^M
        \frac{1}{N}
        \sum_{j=1}^N
        \norm{h_j}_2
        \Big\Vert
        \sigma
        \left(
        \Braket{ \Qb_m h_i, \Kb_m h_j }
        +
        u_m^1
        \star 
        \Ub
        +
        u_m^2
        \star
        \bar{\Ub}
        \right)
        \Vb_m
        \\
        &\quad\quad
        -
        \sigma
        \left(
        \Braket{ \Qb_m^\prime h_i, \Kb_m^\prime h_j }
        +
        u_m^{1\prime}
        \star 
        \Ub
        +
        u_m^{2 \prime}
        \star
        \bar{\Ub}
        \right)
        \Vb_m^\prime
        \Big\Vert_{\text{op}}.
    \end{align*}


    Let 
    \begin{align*}
        A &= 
        \Braket{ \Qb_m h_i, \Kb_m h_j }
        +
        u^1_m
        \star 
        \Ub
        +
        u_m^2
        \star
        \bar{\Ub}
        \\
        B &= 
        \Braket{ \Qb_m^\prime h_i, \Kb_m^\prime h_j }
        +
        u_m^{1\prime}
        \star 
        \Ub
        +
        u_m^{2 \prime}
        \star
        \bar{\Ub}.
    \end{align*}

    By triangle inequality, we have
    \[
    \norm{
    \sigma(A) V_m - \sigma(B) V_m^\prime
    }
    \leq 
    \norm{\sigma(A)}_{\text{op}}
    \norm{\bV_m - \bV_m^\prime}_{\text{op}}
    +
    \norm{\sigma(A) - \sigma(B)}_{\text{op}}
    \norm{V^\prime_m}_{\text{op}}.
    \]

    Note that $\sigma(\cdot)$ is $1$-Lipschitz, we get
    \begin{align*}
        \norm{\sigma(A) - \sigma(B)}_{\text{op}}
        &\leq
        \norm{A-B}_{\text{op}}
        \\
        &=
        \norm{
        \Braket{ \Qb_m h_i, \Kb_m h_j } 
        -
        \Braket{ \Qb_m^\prime h_i, \Kb_m^\prime h_j }
        +
        ( u^1_m - u_m^{1\prime} ) \bU
        +
        ( u^2_m - u_m^{2 \prime} ) \hat{\bU}
        }_{\text{op}}
        \\
        &\leq
        \norm{
        \big\vert
            \Braket{ \Qb_m h_i, \Kb_m h_j } 
        -
        \Braket{ \Qb_m^\prime h_i, \Kb_m^\prime h_j }
        \big\vert
        +
        \bigg\vert
        \norm{
        ( u^1_m - u_m^{1\prime} ) \star \bU
        }
        \bigg\vert
        +
        \bigg\vert
        \norm{
        ( u^2_m - u_m^{2 \prime} ) \star \bar{\bU}
        }
        \bigg\vert
        }
        .
    \end{align*}

    For the first term in the last inequality, we have
    \begin{align*}
    \Braket{ \Qb_m h_i, \Kb_m h_j } 
        -
    \Braket{ \Qb_m^\prime h_i, \Kb_m^\prime h_j }
    &\leq
    \norm{
    \bQ_m - \bQ_m^\prime
    }
    \norm{h_i}
    \norm{h_j}
    \norm{\bK_m}
    +
    \norm{
    \bK_m - \bK_m^\prime
    }
    \norm{h_i}
    \norm{h_j}
    \norm{\bQ_m}
    \\
    &=
    \mathtt{R}^2 B
    \left(
    \norm{
    \bQ_m - \bQ_m^\prime
    }
    +
    \norm{
    \bK_m - \bK_m^\prime
    }
    \right).
    \end{align*}
    Further, we have
    \[
    \norm{ (u^1_m - u_m^{1\prime}) \star \bU }
    \leq
    \vert 
    u^1_m - u_m^{1\prime}
    \vert 
    \norm{\bU}
    \leq
    T    \vert 
    u^1_m - u_m^{1\prime}
    \vert ,
    \]
    where $T$ is the length of each variate (lookback window size).
    \[
    \norm{ (u^2_m - u_m^{2 \prime}) \star \bar{\bU} }
    \leq
    \vert 
    u^2_m - u_m^{2 \prime}
    \vert 
    \norm{\bar{\bU}}
    \leq
    (T-1)d
    \vert 
    u^2_m - u_m^{2 \prime}
    \vert 
    ,
    \]
    where $d$ is the number of variates.
    
    Thus, we have
    \begin{align*}
        \norm{\sigma(A) - \sigma(B)}_{\text{op}}
        \norm{\bV_m^\prime}_{\text{op}}
        &\leq
        B 
        \left( 
        \mathtt{R}^2 B
        \left(
        \norm{
        \bQ_m - \bQ_m^\prime
        }
        +
        \norm{
        \bK_m - \bK_m^\prime
        }
        \right)
        +
        T \left( \vert u^1_m - u_m^{1\prime} \vert  \right)
        +
        (T-1)d
        \left(
        \vert 
        u^2_m - u_m^{2 \prime}
        \vert
        \right)
        \right)
        \\
        &\leq
        B \cdot \max \{ \mathtt{R}^2 B, (T-1) d \}
        \mathcolor{red} \cdot
        \left(
        \norm{\bQ_m - \bQ^\prime_m}
        +
        \norm{\bK_m - \bK^\prime_m}
        +
        \vert u^1_m - u_m^{1\prime} \vert 
        +
        \vert u^2_m - u_m^{2 \prime} \vert 
        \right)
        .
    \end{align*}
    Next, we bound
    \begin{align*}
        \norm{\sigma(A)}_{\text{op}}
        \leq
        \norm{
        A
        }_{\text{op}}
        \leq
        B^2 \mathtt{R}^2 + ( T + (T-1)d ),
    \end{align*}
    due to the fact that 
    \[
    \norm{A}
    \leq
    \norm{
    \bQ_m h_i
    }
    \norm{
    \bK_m h_j
    }
    \norm{
    u^1_m \bU
    }
    \norm{
    u^2_m \bar{\bU}
    }.
    \]
    Overall, the Any-Variate Attention is 
    $
    \max \{  B^2\mathtt{R}^2 +T + (T-1)d, B(T-1)d  \}
    $-Lipshcitz in $\bm{\theta}_1$.
\end{proof}

\begin{proof}
    We start by considering $\bH^\prime = [\bh_i^\prime]$ and
    \begin{equation*}
        \tilde{\bh}_i^\prime
        =
        \bh^\prime_i
        +
        \sum_{m=1}^M
        \frac{1}{N}
        \sum_{j=1}^N
        \sigma
        \left(
        \langle
        \Qb_m \bh^\prime_i,
        \Kb_m \bh^\prime_j
        \rangle
        +
        u^1_m \cdot \Ub
        +
        u^2_m \cdot \bar{\Ub}
        \right)
        \cdot 
        \Vb_m \bh_j^\prime.
    \end{equation*}
    We then bound 
    \begin{align*}
        &\norm{ (\tilde{\bh}_i^\prime - \bh_i^\prime)
        -
        ( \tilde{\bh}_i - \bh_i )
        }_2
        \\
        &=
        \norm{
        \sum_{m=1}^M
        \frac{1}{N}
        \sum_{j=1}^N
        \left[
        \sigma
        \left(
        \langle
        \Qb_m \bh^\prime_i,
        \Kb_m \bh^\prime_j
        \rangle
        +
        u^1_m \cdot \Ub
        +
        u^2_m \cdot \bar{\Ub}
        \right)
        \Vb_m \bh_j
        -
        \left(
        \langle
        \Qb_m \bh^\prime_i,
        \Kb_m \bh^\prime_j
        \rangle
        +
        u^1_m \cdot \Ub
        +
        u^2_m \cdot \bar{\Ub}
        \right)
        \Vb_m \bh_j^\prime
        \right]
        }_2
        \\
        &\leq
        \sum_{m=1}^M
        \frac{1}{N}
        \sum_{j=1}^N
        \norm{\Vb_m}_{\text{op}}
        \norm{
        \sigma
        \left(
        \langle
        \Qb_m \bh^\prime_i,
        \Kb_m \bh^\prime_j
        \rangle
        +
        u^1_m \cdot \Ub
        +
        u^2_m \cdot \bar{\Ub}
        \right)
         \bh_j
        -
        \left(
        \langle
        \Qb_m \bh^\prime_i,
        \Kb_m \bh^\prime_j
        \rangle
        +
        u^1_m \cdot \Ub
        +
        u^2_m \cdot \bar{\Ub}
        \right)
        \bh_j^\prime
        }_2
        \\
        &\leq
        \sum_{m=1}^M 
        \frac{1}{N}
        \sum_{j=1}^N
        \norm{\Vb_m}_{\text{op}}
        \Big\{
        \vert 
        \sigma
        \left( \langle \Qb_m \bh_i, \Kb_m \bh_j \rangle + u^1_m \Ub + u^2_m \bar{\Ub} \right)
        \vert
        \cdot \norm{\bh_j - \bh_j^\prime}_2
        \\
        &\quad\quad\quad\quad+
        \vert
        \sigma
        \left( \langle \Qb_m \bh_i, \Kb_m \bh_j \rangle + u^1_m \Ub + u^2_m \bar{\Ub} \right)
        -
        \sigma
        \left( \langle \Qb_m \bh_i^\prime, \Kb_m \bh_j \rangle + u^1_m \Ub + u^2_m \bar{\Ub} \right)
        \vert
        \cdot
        \norm{\bh_j^\prime}_2
        \\
        &\quad\quad\quad\quad+
        \vert
        \sigma
        \left( \langle \Qb_m \bh_i, \Kb_m \bh_j \rangle + u^1_m \Ub + u^2_m \bar{\Ub} \right)
        -
        \sigma
        \left( \langle \Qb_m \bh_i^\prime, \Kb_m \bh_j^\prime \rangle + u^1_m \Ub + u^2_m \bar{\Ub} \right)
        \vert
        \norm{\bh_j^\prime}_2
        \Big\}
        \\
        &\leq
        \sum_{m=1}^M
        \frac{1}{N}
        \sum_{j=1}^N
        \norm{\Vb_m}_{\text{op}}
        \cdot
        3
        \norm{\Qb_m}_{\text{op}}
        \norm{\Kb_m}_{\text{op}}
        \mathtt{R}^2
        \norm{\bh_j - \bh_j^\prime}_2
        \\
        &\leq
        B^3 \mathtt{R}^2 \norm{\bH - \bH^\prime}_{2, \infty}.
    \end{align*}
    Where the third inequality comes from the fact that ReLU is $1$-Lipschitzness, and the fourth and fifth inequality comes from the AM-GM inequality.
    For more details, refer \citep[Section~J.2]{bai2024transformers}
\end{proof}

\begin{corollary}[Lipschitz Constant of Single Layer Moirai Transformer]
    For a fixed number of heads $M$ and hidden dimension $D^\prime$, we consider
    \[
    \Theta_{\text{TF}, 1, B}
    =
    \{
    \bm{\theta}
    =
    ( \bm{\theta}_1, \bm{\theta}_2 )
    \}
    :
    M \text{ heads, }
    \text{hidden dimension }
    D^\prime,
    \norm{\bm{\theta}}_{\text{op}}
    \leq 
    B.
    \]
    Then for the function $\text{TF}^{\mathtt{R}}$ given by
    \[
    \text{TF}^{\mathtt{R}}
    :
    ( \bm{\theta}, \bH )
    \mapsto
    \texttt{clip}_{\mathtt{R}}
    (
    \text{MLP}_{\bm{\theta}_2}
    (
    \text{Attn}_{\bm{\theta}_1}
    (
    \bH
    )
    )),
    \quad
    \bm{\theta} \in 
    \Theta_{\text{TF}, 1, B},
    \bH \in \cH_{\mathtt{R}}.
    \]
    $\text{TF}^{\mathtt{R}}$ is $B_{\Theta}$-Lipschitz w.r.t. $\bm{\theta}$ and $B_H$-Lipschitz w.r.t. $\bH$,
    where 
    $B_{\Theta} = (1 + B^2) ( 1 + \iota ) + B\mathtt{R}(1 + B^3\mathtt{R}^2)$
    and 
    $B_H = (1 + B^2)(1 + B^3 \mathtt{R}^2 )$.
\end{corollary}


\begin{proposition}[Lipschitz Constant of Moirai Transformer]\label{proposition:lipschitz-moirai}
    For a fixed number of heads $M$ and hidden dimension $D^\prime$, we consider
    \[
    \Theta_{\text{TF}, L, B}
    =
    \{
    \bm{\theta}
    =
    ( \bm{\theta}_1^{(1:L)}, \bm{\theta}_2^{(1:L)} )
    \}
    :
    M^{(\ell)} = M,
    D^{(\ell)}
    =
    D^\prime,
    \norm{\bm{\theta}}_{\text{op}}
    \leq 
    B.
    \]
    Then for the function $\text{TF}^{\mathtt{R}}$ is $(L B_H^{L-1} B_{\Theta})$-Lipschitz in $\bm{\theta} \in \Theta_{\text{TF}, L, B}$ for any fixed $\bH$.
\end{proposition}


\begin{proof}
    For any $\bm{\theta} = ( \bm{\theta}_1, \bm{\theta}_2 )$, 
    $\bH \in \cH_{\mathtt{R}}$, and $\theta^\prime = (\theta^\prime_1, \theta^\prime_2)$, we have
    \begin{align*}
        \norm{
        \text{TF}_{\bm{\theta}}(\bH)
        -
        \text{TF}_{\theta^\prime}(\bH)
        }_{2, \infty}
        &\leq
        \norm{
        \text{MLP}_{\bm{\theta}_2}
        (
        \text{Attn}_{\bm{\theta}_1}
        (\bH)
        )
        -
        \text{MLP}_{\bm{\theta}_2}
        (
        \text{Attn}_{\theta^\prime_1}
        (\bH)
        )
        }_{2, \infty}
        +
        \\
        &\quad\quad
        \norm{
        \text{MLP}_{\bm{\theta}_2}
        (
        \text{Attn}_{\theta^\prime_1}
        (\bH)
        )
        -
        \text{MLP}_{\theta^\prime_2}
        (
        \text{Attn}_{\theta^\prime_1}
        (\bH)
        )
        }_{2, \infty}
        \\
        &\leq
        (1 + B^2)
        \norm{
        \text{Attn}_{\theta^\prime_1}
        (\bH)
        -
        \text{Attn}_{\theta^\prime_1}
        (\bH)
        }_{2, \infty}
        +
        B \bar{\mathtt{R}}
        \norm{ \bm{\theta}_2 - \theta_2^\prime }_{\text{op}}
        \\
        &\leq
        (1 + B^2) ( 1 + \iota ) \norm{\bm{\theta}_1 - \theta_1^\prime}_{\text{op}}
        +
        B \bar{\mathtt{R}}
        \norm{\bm{\theta}_2 - \theta_2^\prime}_{\text{op}}
        \leq 
        B_{\Theta} \norm{\bm{\theta} - \theta^\prime}_{\text{op}},
    \end{align*}
    where $\bar{\mathtt{R}} = \mathtt{R} + B^3 \mathtt{R}^3$, $\iota = \max \{  B^2\mathtt{R}^2 +T + (T-1)d, B(T-1)d  \}$.
    The second inequality comes from the fact $\norm{\text{Attn}_{\bm{\theta}}(\bH)} \leq \mathtt{R} + B^3 \mathtt{R}^3$.

    Further, for $\bH^\prime \in \cH_{\mathtt{R}}$, we have
    \begin{align*}
    \norm{
        \text{TF}_{\bm{\theta}}(\bH)  
        -
        \text{TF}_{\bm{\theta}}(\bH^\prime)
    }_{2, \infty}
    &\leq 
    (1 + B^2) \norm{
    \text{Attn}_{\bm{\theta}_1}(\bH)
    -
    \text{Attn}_{\bm{\theta}_1}(\bH^\prime)
    }
    \\
    &\leq
    (1 + B^2)(1 + B^3 \mathtt{R}^2 ) \norm{
    \bH - \bH^\prime
    }_{2, \infty}.
    \end{align*}

    For the multi-layer case, one can simply follow \citep[Proposition~J.1]{bai2024transformers} to conclude the proof.
    
\end{proof}


\clearpage


\subsection{Proof of \cref{thm:gen-bound-1}}\label{proof:gen-bound-1}

Let $\pi$ be a meta distribution, and each distribution drawn from $ \mathtt{P}^{(T)} \sim \pi$ satisfies the Dobrushin's condition.
We then define the single-path average loss as
\[
Y_{\theta, \mathtt{P}^{(T)}}
\coloneqq
\frac{1}{T}
\sum_{t=1}^T
\ell( \theta, \bz_t)
-
\mathbb{E}_{\bz \sim \mathtt{P}^{(T)}}
\left[
\ell( \theta, \bz)
\right].
\]

Now, we assume our pretraining data is generated by the following
\begin{enumerate}
    \item Sample $n$ distributions from $\pi$ i.i.d. to get $\mathtt{P}^{(T)}_j$, for $j = 1,\cdots,n$
    \item For each distribution $\mathtt{P}^{(T)}_j$, we sample $(\bz_{j,1}, \cdots, \bz_{j,T})$
\end{enumerate}


% For the pretraining data, we consider two cases when the following assumptions hold or not hold
\begin{assumption}\label{assumption:stationary}
    We assume that for each $j \in [n]$, $(z_{j, t})$ has marginals equal to some distribution $D$ for $t = 1, \cdots ,T$.
\end{assumption}

We first present several lemma and theorems that will be used later.
\begin{lemma}[\text{\citep[Example~5.8]{wainwright2019high}}]\label{lem:covering-number-unit-ball}
    Given any well-defined norm $\norm{\cdot}^\prime$.
    Let $\mathbb{B}$ be the $\R^d$ unit-ball in $\norm{\cdot}^\prime$, i.e. 
    $\mathbb{B} = \{ \theta \in \R^d \mid \norm{\theta}^\prime \leq 1 \}$,
    we have
    \[
    \log N( \delta, \mathbb{B}, \norm{\cdot}^\prime )
    \leq
    d 
    \log
    \left( 1 + \frac{2}{\delta} \right).
    \]
\end{lemma}

\begin{theorem}[\text{\citep[Theorem~5.3]{dagan2019learning}}]\label{thm:gen-bound-dobrushin}
    Given a function class $\cF$, such that $\vert f \vert \leq B$, for all $f \in \cF$.
    Let $\mathtt{P}^{(T)}$ be a distribution over some domain $Z^{(T)}$, assuming \cref{assumption:stationary} holds and $\alpha_{\log}( \mathtt{P}^{(T)} ) < \nicefrac{1}{2}$.
    Then for all $t > 0$,
    \[
    P_{\bz \sim \mathtt{P}^{(T)}}
    \left(
    \sup_{f\in\cF}
    \left\vert 
        \frac{1}{T}
        \sum_{i=1}^T
        f(z_i)
        -
        \mathbb{E}_{z}
        [
        f(z)
        ]
    \right\vert
    >
    C
    \left(
        \mathfrak{G}_{\mathtt{P}^{(T)}}( \cF)
        +
        \frac{B t}{\sqrt{T}}
    \right)
    \right)
    \leq
    e^{-t^2/2},
    \]
    for some universal constant whenever $1/2 - \alpha_{\log}(\mathtt{P}^{(T)})$ is bounded away from zero.
\end{theorem}

The following theorem is from \cite{dagan2019learning, kulske2003concentration}.
\begin{theorem}\label{thm:dobrushin-concentration}
    Let $\mathtt{P}^{(T)}_{\bz}$ be a distribution satisfying the Dobrushin's condition with coefficient $\alpha(\mathtt{P}_{\bz}^{(T)})$.
    Let $(\bz_1,\cdots,\bz_T) \sim \mathtt{P}^{(T)}$, and let $f: \bZ^{(T)} \rightarrow \R$ be a real-valued function with the following bounded difference property, with parameters $\lambda_1, \cdots, \lambda_T \geq 0$:
    \[
    \vert f(\bz) - f(\bz^\prime) \vert 
    \leq
    \sum_{t=1}^T
    \mathbbm{1}_{\bz_t\neq \bz_t^\prime}
    \lambda_t.
    \]
    Then for all $t > 0$,
    \[
    P
    \left(
        \vert 
        f(\bz)
        -
        \mathbb{E}
        [ f(\bz) ]
        \geq
        t
    \right)
    \leq
    2 \exp
    \left(
    -
    \frac{(1 - \alpha)t^2}{2 \sum_t \lambda_t^2}.
    \right)
    \]
\end{theorem}
The following corollary directly follows from the above result
\begin{corollary}\label{cor:subgaussian-loss}
    Following \cref{thm:dobrushin-concentration}, let
    \[
    \ell( \bz)
    \coloneqq
    \frac{1}{T}
    \sum_{t=1}^T
    \ell( \bz_t),
    \]
    where $0 \leq \ell(\bz_t) \leq B$ for all $t=1,\cdots,T$ and all $\bz \sim \mathtt{P}_{\bz}$.
    Then the variance of $\ell(\cdot)$ is bounded by
    \[
    \vert 
        \ell(\bz)
        -
        \ell(\bz^\prime)
    \vert
    \leq
    B.
    \]
    Then, the following holds
    \[
        P
        \left(
        \left\vert
        \ell(\bz)
        -
        \mathtt{E}
        [ \ell(\bz) ]
        \right\vert
        \geq
        t
        \right)
        \leq
        2
        \exp
        \left(
        \frac{- (1 - \alpha) t^2 }{2 \sum_t B^2}
        \right)
        .
    \]
\end{corollary}

\paragraph{Direct Application of \cref{thm:gen-bound-dobrushin}.}
By \cref{thm:gen-bound-dobrushin}, if \cref{assumption:stationary} holds, with probability over $1 - e^{-t^2/2}$, for any $\theta \in \Theta$, $\alpha_{\log}(\mathtt{P}^{(T)}_j) < 1/2$ we have

\[
\sup_{\theta \in \Theta }
\left\vert
Y_{\theta, \mathtt{P}^{(T)}}
\right\vert
\leq
C 
\left[
\mathfrak{G}_{\mathtt{P}^{(T)}_j}(\ell(\Theta))
+
\frac{2tB_x^2}{\sqrt{T}}
\right],
\]
where $\ell(\Theta)$ denotes the function class of $\ell( \theta, )$, for all $\theta\in\Theta$, and $C > 0$ is an universal constant.
Note that the above bound presents the naive learning bound for learning a single time series, which is a direct result from \cite{dagan2019learning}.



\begin{proof}
    

We then define a random process 
$\{ X_\theta \}$ as
\begin{align*}
X_{\theta}
\coloneqq
\frac{1}{n}
\sum^n_{j=1}
Y_{\theta, \mathtt{P}_j^{(T)}}
&=
\frac{1}{n}
\sum^n_{j=1}
\left[
\frac{1}{T}
\sum_{t=1}^T
\ell( \theta, \bz_{j,t})
-
\mathbb{E}_{\bz \sim \mathtt{P}^{(T)}_j}
\left[
\ell( \theta, \bz)
\right]
\right]
\\
&=
\left[
\frac{1}{nT}
\sum^n_{j=1}
\sum_{t=1}^T
\ell( \theta, \bz_{j,t})
\right]
-
\mathbb{E}_{\bz \sim \mathtt{P}^{(T)}_j, \mathtt{P}^{(T)}_j \sim \pi}
\left[
\ell( \theta, \bz)
\right].
\end{align*}

Now, to take supremum over $X_\theta$, we get
\begin{align*}
    \sup_{\theta \in \Theta}
    X_\theta
    &=
    \sup_{\theta \in \Theta}
    \frac{1}{n}
    \sum_{j=1}^n
    Y_{\theta, \mathtt{P}_j^{(T)}}
    \\
    &\leq
    \frac{1}{n}
    \sum_{j=1}^n
    \sup_{\theta \in \Theta}
    Y_{\theta, \mathtt{P}_j^{(T)}}.
\end{align*}

To upper bound $\sup | X_{\theta} |$,   we take a similar approach to \citep[Proposition~A.4]{bai2024transformers}.

Assuming the index set $\Theta$ is equipped with a distance metric $\rho$ and diameter $D$.
We assume that for any ball $\Theta^\prime$ of radius $r$ in $\Theta$, there exists some constant $C_1$ such that the covering number admits upper bound
\[
\log
N( \delta, \Theta^\prime, \rho)
\leq
d \log ( 2 A r / \delta),
\]
for all $0 < \delta \leq 2r$.

Now we select $\Theta_0$ such that it is a $(D_0/2)$-covering of $\Theta$.
The above assumption guarantees us that we can have a $\Theta_0$ such that $\log | \Theta_0 | \leq d \log (2AD/D_0)$.
By \cref{cor:subgaussian-loss}, $X_{\bm{\theta}}$ is a $\nicefrac{2 B_x^2}{(1 - \alpha)}$-subgaussian ($\alpha = \alpha( \mathtt{P}^{(T)})$).
Then, with probability at least $1 - \delta/2$, 
\[
\sup_{\theta \in \Theta_0}
| X_\theta |
\leq
C \frac{2 B_x^2}{(1 - \alpha)}
\sqrt{
d \log (2 A D/ D_0)
+
\log ( 2 / \delta)
}.
\]

Note that the uniform bound for independent subgaussian random variables still applies here as for each $\theta$, we are re-sampling a new chain from a new distribution sampled from $\pi$.

Assume that $\Theta_0 = \{ \theta_1, \cdots, \theta_n\}$.
Now for each $j\in[m]$, we consider $\Theta_j$ is the ball centered at $\theta_j$ of radius $D_0$ in $(\Theta, \rho)$.
With \cref{thm:generalizd-dudley}, for each process $\{ X_{\theta} \}_{\theta \in\Theta_j}$, then
\[
\psi = \psi_2,
\quad
\norm{
X_{\theta}
-
X_{\theta^\prime}
}_{\psi}
\leq
\frac{B^1}{\sqrt{n}}
\rho(\theta, \theta^\prime),
\]
where $\ell(\theta, \bz) - \ell(\theta^\prime, \bz)$ is a $B^1 \rho(\theta, \theta^\prime)$-subgaussian random variable.

We then get
\[
P
\left(
    \sup_{\theta, \theta^\prime \in \Theta_j}
    \vert 
    X_{\theta}
    -
    X_{\theta^\prime}
    \vert 
    \leq
    C^\prime 
    B^1
    D_0
    \left(
    \sqrt{
    \frac{d\log(2A)}{n}
    }
    +t
    \right)
\right)
\leq
2 \exp( -nt^2),
\quad\text{ for all }t \geq 0.
\]

If we further take $t \leq \sqrt{\log(2m/\delta)/n}$, then with probability at least $1 - \delta/2$, it holds that for all $j\in[m]$,
\[
    \sup_{\theta, \theta^\prime \in \Theta_j}
    \vert 
    X_{\theta}
    -
    X_{\theta^\prime}
    \vert 
    \leq
    C^\prime
    B^1
    D_0
    \sqrt{
    \frac{
    2d \log (2AD/D_0) + \log(4/\delta)
    }{n}
    }.
\]

By chaining, we have
\[
|X_{\theta}|
\leq 
|X_{\theta_j}|
+
|X_{\theta}
-
X_{\theta_j}|.
\]

Hence with probability at least $1 - \delta$, it holds that

\[
\sup_{\theta\in\Theta}
|X_{\theta}|
\leq
\sup_{\theta\in\Theta_0} | X_{\theta} |
+
\sup_j
\sup_{\theta \in \Theta_j}
| X_{\theta} - X_{\theta_j} |
\leq 
C^{\prime\prime}
(
\frac{2 B_x^2}{(1 - \alpha)}
+
B^1 D_0
)
\sqrt{
\frac{
d \log (2AD/D_0) + \log(2/\delta)
}{n}
}.
\]

Next by taking $D_0 = D/\kappa, \kappa= 1 + B^1D \frac{(1 - \alpha)}{2 B_x^2}$, we get

\[
\sup_{\theta\in\Theta}
|X_{\theta}|
\leq
C^{\prime\prime}
(
\frac{2 B_x^2}{(1 - \alpha)}
+
B^1 D \kappa
)
\sqrt{
\frac{
d \log (2A \kappa) + \log(2/\delta)
}{n}
}.
\]

Last, we check whether the assumptions we make above hold for our function class $\ell_\Theta$. 
Below, we slightly abuse our notation by using $D$ as the dimension for weight matrices in $\text{TF}_{\bm{\theta}}$.
By \cref{lem:covering-number-unit-ball}, it holds that 
\[
\log N (\delta, B_{\norm{\cdot}_{\text{op}}}(r), \norm{\cdot}_{\text{op}} )
\leq
L(3MD^2 + D D^\prime + 2) \log ( 1 + 2r/\delta),
\]
where $B_{\norm{\cdot}_{\text{op}}}(r)$ is a ball of radius $r$ under norm $\norm{\cdot}_{\text{op}}$.

We check that
\[
\norm{
\ell(\theta, \bz)
-
\ell(\theta^\prime, \bz)
}
\leq
B_x(L B_H^{L-1} B_{\Theta}) \norm{\theta - \theta^\prime}_{\text{op}},
\]
where it is a direct result from \cref{proposition:lipschitz-moirai}.
By plugging all the parameters, we get

\[
\sup_{\theta\in\Theta}
|X_{\theta}|
\leq
C
(
\frac{B_x^2}{(1 - \alpha)}
)
\sqrt{
\frac{
L(3MD^2 + D D^\prime + 2) \iota + \log(2/\delta)
}{n}
},
\]

where $\iota = \log( 2 + 2(L B_H^{L-1} B_{\Theta}) B \frac{1-\alpha}{B_x})$


Finally, by plugging the ERM $\hat{\bm{\theta}}$, we get
\[
L(\hat{\bm{\theta}})
\leq
\inf_{\theta} L(\theta)
+
2 \sup_{\theta} | X_{\theta} |.
\]


\end{proof}

% Our goal is to bound the following,
% \begin{align*}
%     X_{\bm{\theta}}
%     \coloneqq
%     \frac{1}{T}
%     \sum_{t=1}^T
%     \ell(\bm{\theta}, \bz_{t})
%     -
%     \mathtt{E}_{\bz\sim \mathtt{P}_{\bz}}
%     \left[
%     \hat{\ell( \bm{\theta}, \bz )}
%     \right],
% \end{align*}
% where $(\bz_1, \cdots, \bz_T)$ satisfies the Dobrushin's condition such that $\alpha( \mathtt{P}_{\bz} ) < 1$.

% The following Lemma is from \citep[Lemma~5.2]{van2014probability}.
% \begin{lemma}[Maximal Tail Inequality]\label{lemma:max-tail}
%     Suppose that $\log \mathbb{E}[ e^{\lambda X_t} ] \leq \psi(\lambda)$ for all $\lambda \geq 0$ and $t \in [T]$, where $\psi$ is convex and $\psi(0) = \psi^\prime(0) = 0$.
%     Then
%     \[
%         P
%         \left(
%             \sup_{t \in T} X_t \geq \psi^{* -1}( \log |T| + u)
%         \right)
%         \leq 
%         e^{-u}
%         \quad
%         \text{for all }u \geq 0.
%     \]
%     Further, if $X_t$ is $\sigma^2$-subgaussian for every $t \in T$, we have
%     \[
%         P
%         \left(
%             \sup_{t\in T} X_t
%             \geq
%             \sqrt{
%             2 \sigma^2 \log |T|
%             }
%             +x
%         \right)
%         \leq
%         e^{-x^2/2\sigma^2},
%         \quad
%         \text{for all }
%         x \geq 0.
%     \]
% \end{lemma}


% The following theorem is from \cite{dagan2019learning, kulske2003concentration}.
% \begin{theorem}\label{thm:dobrushin-concentration}
%     Let $\mathtt{P}^{(T)}_{\bz}$ be a distribution satisfying the Dobrushin's condition with coefficient $\alpha(\mathtt{P}_{\bz})$.
%     Let $(\bz_1,\cdots,\bz_T) \sim \mathtt{P}^{(T)}$, and let $f: \bZ^{(m)} \rightarrow \R$ be a real valued function with the following bounded difference property, with parameters $\lambda_1, \cdots, \lambda_T \geq 0$:
%     \[
%     \vert f(\bz) - f(\bz^\prime) \vert 
%     \leq
%     \sum_{t=1}^T
%     \mathbbm{1}_{\bz_t\neq \bz_t^\prime}
%     \lambda_t.
%     \]
%     Then for all $t > 0$,
%     \[
%     P
%     \left(
%         \vert 
%         f(\bz)
%         -
%         \mathbb{E}
%         [ f(\bz) ]
%         \geq
%         t
%     \right)
%     \leq
%     2 \exp
%     \left(
%     -
%     \frac{(1 - \alpha)t^2}{2 \sum_t \lambda_t^2}.
%     \right)
%     \]
% \end{theorem}
% The following corollary directly follows from the above result
% \begin{corollary}
%     Following \cref{thm:dobrushin-concentration}, let
%     \[
%     \ell( \bz)
%     \coloneqq
%     \frac{1}{T}
%     \sum_{t=1}^T
%     \ell( \bz_t),
%     \]
%     where $0 \leq \ell(\bz_t) \leq B$ for all $t=1,\cdots,T$ and all $\bz \sim \mathtt{P}_{\bz}$.
%     Then the variance of $\ell(\cdot)$ is bounded by
%     \[
%     \vert 
%         \ell(\bz)
%         -
%         \ell(\bz^\prime)
%     \vert
%     \leq
%     B.
%     \]
%     Then the following holds
%     \[
%         P
%         \left(
%         \left\vert
%         \ell(\bz)
%         -
%         \mathtt{E}
%         [ \ell(\bz) ]
%         \right\vert
%         \geq
%         t
%         \right)
%         \leq
%         2
%         \exp
%         \left(
%         \frac{- (1 - \alpha) t^2 }{2 \sum_t B^2}
%         \right)
%         .
%     \]
% \end{corollary}



% \begin{theorem}[\text{\citep[Theorem~6.7]{dagan2019learning}}] \label{thm:gen-bound-dobrushin}
%     Let $\mathtt{P}^{(m)}$ be a distribution satisfying $\alpha_{\log}(\mathtt{P}^{(m)}) < 1/2$ and let $\cF \coloneqq \bZ \rightarrow \R$ be a function class.
%     For every $f \in \cF$, $\vert f(s) \vert \leq B$.
%     Then for any $t > 0$,
%     \[
%     P_{\bZ \sim \mathtt{P}^{(m)}}
%     \left(
%     \sup_{f \in \cF }
%     \left\vert
%     \sum_{i=1}^m
%     f(\bz_i)
%     -
%     \mathbb{E}_{\bZ}
%     \left[
%     \sum_{i=1}^m
%     f(\bz_i)
%     \right]
%     \right\vert
%     >
%     \frac{C \mathfrak{G}_{\mathtt{P}^{(m)}}(\cF)}{
%     \sqrt{1 - 2 \alpha_{\log}(\mathtt{P}^{(m)})}
%     }
%     +
%     C B \sqrt{m t}
%     \right)
%     \leq 
%     e^{-t^2/2},
%     \]
%     for some universal constant $C > 0$.
% \end{theorem}



% \begin{proof}
%     Considering the quantity $\sup_{\bm{\theta}\in\Theta}| X_{\bm{\theta}}|$.

%     Assuming the index set $\Theta$ is equipped with a distance metric $\rho$ and diameter $D$.
%     We assume that for any ball $\Theta^\prime$ of radius $r$ in $\Theta$, there exists some constant $C_1$ such that the covering number admits upper bound
%     \[
%     \log
%     N( \delta, \Theta^\prime, \rho)
%     \leq
%     d \log ( 2 A r / \delta),
%     \]
%     for all $0 < \delta \leq 2r$.

%     Now we select $\Theta_0$ such that it is a $(D_0/2)$-covering of $\Theta$.
%     The above assumption guarantees us that we can have a $\Theta_0$ such that $\log | \Theta_0 | \leq d \log (2AD/D_0)$.
%     Since $X_{\bm{\theta}}$ is a $\nicefrac{B_x^2}{2}$-subgaussian, by \cref{lemma:max-tail}, we have
%     \[
%         P
%         \left(
%             \sup_{ \bm{\theta} \in \Theta_0} X_{\bm{\theta}}
%             \geq
%             {B_x}
%             \sqrt{
%              \log | \Theta_0 |
%             }
%             +u
%         \right)
%         \leq
%         e^{-u^2/B_x^2},
%         \quad
%         \text{for all }
%         u \geq 0.
%     \]
%     The above result is also equivalent to, with probability at least $1 - \delta$, the following holds
%     \begin{align*}
%         \sup_{ \bm{\theta} \in \Theta_0} X_{\bm{\theta}}
%         &\leq
%         B_x \sqrt{ \log | \Theta_0| }
%         +
%         \sqrt{ B_x^2 \log(\frac{1}{\delta})}
%         \\
%         &\leq
%         B_x\left(
%         \sqrt{ d \log(2AD/D_0)}
%         +
%         \log(\frac{1}{\delta})
%         \right).
%     \end{align*}



% \end{proof}


% \subsection{Proof of XXX}

% We first restate some lemma and theorem.
% % There to bound the following quantity
% % \[
% % \sup_{\bm{\theta} \in \Theta_{L, M, D^\prime, B}}
% % \big\vert
% % \hat{L}(\bm{\theta})
% % -
% % L(\bm{\theta})
% % \big\vert,
% % \]
% % we proceed XXX.


% % From \cite{bai2024transformers}
% % \begin{theorem}
% %     Suppose that $\psi : [0, +\infty) \mapsto [0, +\infty)$ is a convex, non-decreasing function that satisfies 
% %     $\psi(x+y) \geq \psi(x) \psi(y)$.
% %     For any random variable $\cX$, we consider the Orlicz norm induced by $\psi: \norm{\cX}_{\psi}\coloneqq \inf \{ K > 0: \mathbb{E}_{\psi}[ |\cX|/K ]\} \leq 1$.
% %     Suppose that $\{ \cX_{\theta}\}$ is a zero-mean random process indexed by $\theta \in \Theta$ such that 
% %     $\norm{\cX_\theta - \cX_{\theta^\prime}} \leq \rho(\theta, \theta^\prime)$ for some metric $\rho$ on space $\Theta$.
% %     Then it holds that 
% %     \begin{equation*}        
% %     P
% %     \left(
% %     \sup_{\theta, \theta^\prime \in \Theta}
% %     \vert 
% %     \cX_\theta - \cX_{\theta^\prime}
% %     \vert
% %     \leq 
% %     8 (J + t)
% %     \right)
% %     \leq
% %     \frac{1}{\psi(t/D)},
% %     \quad
% %     \forall t \geq 0,
% %     \end{equation*}

% %     where $D$ is the diameter of the metric space $(\Theta, \rho)$, and the generalized Dudley entropy integral $J$ is given by
% %     \[
% %     J \coloneqq
% %     \int_0^D
% %     \psi^{-1}
% %     (N(\delta ; \Theta, 
% %     rho) )
% %     d \delta,
% %     \]
% %     where $N(\delta ; \Theta, \rho)$ is the $\delta$-covering number of $(\Theta, \rho)$.
% % \end{theorem}

% % By , we have 
% % \begin{lemma}[\citep[Example~5.8]{wainwright2019high}]\label{lem:covering-number-unit-ball}
% %     Given any well-defined norm $\norm{\cdot}^\prime$.
% %     Let $\mathbb{B}$ be the $\R^d$ unit-ball in $\norm{\cdot}^\prime$, i.e. 
% %     $\mathbb{B} = \{ \theta \in \R^d \mid \norm{\theta}^\prime \leq 1 \}$,
% %     we have
% %     \[
% %     \log N( \delta, \mathbb{B}, \norm{\cdot}^\prime )
% %     \leq
% %     d 
% %     \log
% %     \left( 1 + \frac{2}{\delta} \right).
% %     \]
% % \end{lemma}


% Now we are interested in bounding the following complexity
% \[
% \mathfrak{G}_{\mathtt{P}_{\bZ}}( \ell(\Theta) )=
% \mathbb{E}_{\bZ \sim \mathtt{P}_{\bZ}}
% \left[
% \mathbb{E}_{\tau}
% \left[
% \sup_{ \bm{\theta} \in \Theta_{L,M,D^\prime, B} }
% \frac{1}{T}
% \sum_{i=1}^T
% \tau_i
% \ell ( \bZ_i, \bm{\theta} )
% \right]
% \right],
% \quad
% \tau_i \sim N(0, 1), \text{ for } i = 1, \cdots , T.
% \]

% Note that by Dudley's metric entropy bound, the above Gaussian complexity is upper bounded by

% \[
% \mathfrak{G}_{\mathtt{P}_{\bZ}}( \ell(\Theta) )
% \leq
% C
% \int_0^{\infty}
% \sqrt{ \log N( \delta, \ell(\Theta), L_2) }
% d \delta.
% \]

% Note that the function class $\ell(\Theta)$ is parameterized by $\bm{\theta}$, and the whole function class is $L_1$-Lipschitz in $\bm{\theta}$.
% This implies to cover $\ell(\Theta)$ at scale $\delta$, it is sufficient to cover $\Theta$ at scale $\delta/L_1$.

% By Lemma~\ref{lem:covering-number-unit-ball}, we have
% \[
% \log N( \delta, \ell(\Theta), L_2)
% \leq
% d
% \log
% \left(
% 1 + 
% \frac{2 L_1}{\delta}
% \right).
% \]

% Now consider Dudley's metric entropy bound, we get
% \begin{align*}
%     \mathfrak{G}
%     &\leq
%     C
%     \int_0^{2B_x^2}
%     \sqrt{
%     \frac{
%     \log N( \delta, \ell(\Theta), L_2)
%     }{n}
%     }
%     d \delta
%     \\
%     &\leq 
%     C
%     \int_0^{2B_x^2}
%     \sqrt{
%     \frac{
%     d
%     \log
%     \left(
%     1 + 
%     \frac{2 L_1}{\delta}
%     \right)   
%     }{n}
%     }
%     d \delta
%     \\
%     &=
%     C \sqrt{\frac{d}{n}}
%     \int_0^{2B_x^2}
%     \sqrt{
%     \log( 1 + \frac{2L_1}{\delta} )
%     }
%     d \delta
%     \\
%     &=
%     O
%     \left(
%     \sqrt{
%     \frac
%     {L( MD^2 + D D^\prime)}
%     {n}
%     }
%     \left(
%     B_x^2 + 
%     L ( B_H^{L-1} B_{\Theta} ) 
%     B_x
%     \right)
%     \right).
% \end{align*}


% Now with Theorem~\ref{thm:gen-bound-dobrushin},
% for any $t > 0$,
% with probability $1 - e^{t^2/2}$,
% the ERM $\hat{\bm{\theta}}$ satisfies
% \begin{align*}
%     L ( \hat{\bm{\theta}}
%     -
%     \hat{L}( \hat{\bm{\theta}})
%     &>
%     \frac{C \mathfrak{G}_{\mathtt{P}^{(m)}}(\cF)}{
%     \sqrt{1 - 2 \alpha_{\log}(\mathtt{P}^{(m)})}
%     }
%     +
%     C B \sqrt{m t}
%     \\
%     &\geq 
%     O
%     \left(
%     \sqrt{
%     \frac
%     {L( MD^2 + D D^\prime)}
%     {n ( 1 - 2 \alpha_{\log})}
%     }
%     \left(
%     B_x^2 + 
%     L ( B_H^{L-1} B_{\Theta} ) 
%     B_x
%     \right)
%     \right)
%     +
%     C B_x^2 \sqrt{n t}
% \end{align*}



% The above theorem serves as a general generalization bound for learning from weakly dependent data satisfying the Dobrushin's condition.

% \begin{proof}
%     By Theorem~\ref{thm:gen-bound-dobrushin}, the ERM $\hat{\bm{\theta}}$ satisfies
%     \[
%     P_{\bZ \sim \mathtt{P}^{(m)}}
%     \left(
%     \hat{
%     L}( \hat{\bm{\theta}} )    
%     -
%     L(
%     \hat{\bm{\theta}}
%     )
%     >
%     \frac{C \mathfrak{G}_{\mathtt{P}^{(m)}}( L(\Theta) )}{
%     \sqrt{1 - 2 \alpha_{\log}(\mathtt{P}^{(m)})}
%     }
%     +
%     C B \sqrt{m t}
%     \right)
%     \leq 
%     e^{-t^2/2},
%     \]
%     where $L(\Theta)$ is a function class of $L \circ \cF$.

%     By general bound for Gaussian Complexity, we have
%     \[
%     O ( \sqrt{L (M D^2 + D D^\prime)} L B B^{L-1}_H B_{\Theta})  ),
%     \]

%     By plugging the bound for Gaussian complexity, we have
%     \[
%     P_{\bZ \sim \mathtt{P}^{(m)}}
%     \left(
%     L(
%     \hat{\bm{\theta}}
%     )
%     >
%     O 
%     \left(
%     \frac{\sqrt{L (M D^2 + D D^\prime)} L B_x B^{L-1}_H B_{\Theta}}{n ( 1 - \alpha)}
%     +
%     B_x \cdot \exp\left( \frac{-L}{\kappa} \right)
%     \right)
%     +
%     C B \sqrt{m t}
%     \right)
%     \leq 
%     e^{-t^2/2},
%     \]
    
% \end{proof}


% \begin{proof}[Proof of Theorem~\ref{thm:gen-bound-1}]
%     Considering the following random process
%     \[
%     \cX_{\theta}
%     \coloneqq
%     \frac{1}{N} \sum_{i=1}^N f(z_i, \theta)
%     -
%     \mathbb{E}_z[ f(z, \theta) ],
%     \]
%     where $z_1, \cdots, z_N$ are sampled from a distribution $\mathtt{P}_z$ satisfies the Dobrushin's condition with $\alpha( \mathtt{P}) < 1$.

%     By \citep[Theorem~2.3]{dagan2019learning} (also \cite{kulske2003concentration}), $\{ \cX_\theta \}$ is a $O(1/(1-\alpha(\mathtt{P}_z  )$ subGaussian process for every $\theta \in \Theta$.

%     By \cite{wainwright2019high}, we are able to show that
%     \[
%     L(\hat{\bm{\theta}}) \leq
%     \inf_{\bm{\theta} \in \Theta_{ L,B,M, }}
%     \mathbb{E}
%     \left[
%     L(\bm{\theta})
%     \right]
%     +
%     2 \sup_{\bm{\theta} \in \Theta_{L,B,M}}
%     \vert \cX_{\bm{\theta}} \vert .
%     \]




%     To further bound the above inequality, the tail bound for empirical process to verify a few conditions on the function $L$ and the set $\Theta$ as follows
%     \begin{itemize}
%         \item[1]
%         The metric entropy of an operator norm ball
%         \[
%         \log N ( \delta, B_{\norm{\cdot}_{\text{op}}} (\delta),
%         \norm{\cdot}_{\text{op}}
%         \leq
%         CL (M D^2 + DD^\prime) \log(1 + 2r  \delta)
%         \]
%         \item[2]
%         $L(\bm{\theta})$ is $2 B_x^2$-subGaussian
%         \item[3]
%         The Moirai transformer is Lipschitz in $\bm{\theta}$.
%     \end{itemize}

%     Note that [1] follows immediately from J.2 in \cite{bai2024transformers} (with $D > 2$), 
%     [2] follows immediately by the $\texttt{clip}$ operation and [3] is shown by Proposition~\ref{proposition:lipschitz-moirai}.

%     $(L B_H^{L-1} B_{\Theta})$

%     Finally, by \citep[Proposition~A.4]{bai2024transformers}, with probability at least $1 - \varepsilon$, we have
%     \[
%     \sup_{\bm{\theta} \in \Theta}
%     \big\vert 
%     \cX_{\bm{\theta}}
%     \big\vert 
%     \leq 
%     C B_x^2
%     \sqrt{ L(M D^2 + D D^\prime) \zeta + \log(1/\varepsilon) },
%     \]
%     where $\zeta = \log ( 2 + B(L B_H^{L-1} B_{\Theta} ) / 2B_x ) = O( 2 + \max\{ B, \mathtt{R}, B_x, T, d \})$, with $T$ is the number of time steps of each covariates, and $d$ is the number of covariates.

%     Finally, we get the final result by plugging XXX.
%     \[
%     L(\hat{\bm{\theta}})
%     \leq 
%     \inf_{\bm{\theta} \in \Theta}
%     \mathbb{E}[ L(\bm{\theta})]
%     +
%     C B_x^2 \sqrt{ \frac{L(MD^2 + D D^\prime) \zeta + \log (1 / \varepsilon)}{n} }.
%     \]


    
% \end{proof}


% % \subsubsection{The i.i.d. Case}
% % Let $\cF$ be a function class such that for all $f \in \cF$, $f$ is $L$-Lipschitz on some domain $\cX$, w.r.t. L2 norm, i.e.
% % \[
% % \norm{ f(x) - f(y)}
% % \leq 
% % L
% % \norm{x-y},
% % \quad
% % \text{for all }x, y \in \cX.
% % \]

% % Given a sample of points $X = x_1, \cdots, x_n \in \cX$, the empirical Gaussian complexity of $\cF$ on these points $\hat{\mathfrak{G}}_X( \cF )$ is defined as
% % \[
% % \hat{\mathfrak{G}}_X( \cF )
% % \coloneqq
% % \mathbb{E}_{\gamma}
% % \left[
% % \frac{1}{n}
% % \sum_{i=1}^n
% % \gamma_i f(x_i)
% % \right], \quad \text{where } \gamma_1, \cdots \gamma_n \sim N(0, 1) \text{ i.i.d.}
% % \]


% % Based on the Dudley's inequality \cite{dudley1967sizes} (the chaining bound), the above empirical Gaussian complexity is bounded by
% % \[
% % \hat{\mathfrak{G}}_X( \cF )
% % \leq 
% % \frac{1}{\sqrt{n}}
% % \int_0^D
% % \sqrt{\log N(\epsilon, \cF, d) }
% % d \epsilon,
% % \]
% % where $D \geq \sup_{f,g \in \cF} d(f, g)$ for metric $d$. $d$ is defined as
% % \[
% % d(f, g) \coloneqq 
% % \sqrt{
% % \frac{1}{n}
% % \sum_{i=1}^n
% % \left(
% % f(x_i) - g(x_i)
% % \right)^2
% % }.
% % \]
% % Note that $\cF$ is $L$-Lipschitz by assumption.
% % Assuming $\norm{x_i} < B_x$ for $i = 1, \cdots, n$, $D$ has a natural upper bound
% % \[
% % D \leq 2 L B_x.
% % \]

% % {\color{red} Below needs to be revised}
% % The metric entropy of a Lipschitz function class is then bounded by
% % \[
% % \log
% % N(\epsilon, \cF, d)
% % \leq
% % d_x \log\left( \frac{LB_x}{\epsilon} \right),
% % \]
% % where $d_x$ is the dimension of $\cX$.
% % Finally, the empirical Gaussian complexity of a $L$-Lipschitz function with $d_x$ dimensional input is bounded by
% % \begin{equation}\label{eqn:gaus-complexity-bound}
% %     \hat{\mathfrak{G}}_X (\cF)
% %     \leq
% %     \frac{1}{n}
% %     \int_0^{L B_x}
% %     \sqrt{d_x \log
% %     \left(
% %     \frac{L B_x}{\epsilon}
% %     \right)
% %     d \epsilon
% %     }
% %     =
% %     \cO\left( \frac{L B_x}{\sqrt{n}} \sqrt{ d_x \log (n L B_x) }  \right).
% % \end{equation}

% % \subsubsection{The Weakly Dependent Case}

% % We will use Dudley's inequality to prove XXX. 
% % First, we prove some lemmas. 
% % The first shows that by partitioning the data sequence into block, then with high probability, these blocks behaves like i.i.d. samples.

% % \begin{lemma}[Block-wise Dependency of Weakly Dependent Data]\label{lem:block-dependence}
% %     Let $\cX = (\cX_1, \cdots, \cX_n)$ be random variables over $\cD_{\cX}^n$, where $\alpha(\cX) < 1$.
% %     Let $B \subset \{ 1, \cdots, n \}$, and $ \vert B \vert = b$.
% %     We the subscript $\cX_B$ to denote the subset of $\cX$ selected with indices $B$,  with $B^c \cap B = [n]$.
% %     We have
% %     \[
% %         d_{\text{TV}}
% %         \left(
% %         \mathbb{P}_{ \cX_B \mid \cX_{B^c} }
% %         ( \cdot
% %         \mid \text{x}_{B^c} )
% %         ,
% %         \mathbb{P}_{ \cX_B \mid \cX_{B^c} }
% %         ( \cdot
% %         \mid \text{x}^\prime_{B^c} )
% %         \right)
% %         =
% %         O
% %         \left(\alpha( \cX )^b\right).
% %     \]
% % \end{lemma}

% % \begin{proof}[Proof of Lemma~\ref{lem:block-dependence}]
% %     Considering the sequence
% %     $\cX = (\cX_1, \cdots, \cX_n)$ over $\cD_{\cX}^n$, we say $\cX$ satisfies the Dobrushin's condition if
% %     \[
% %     \max_{1\leq i \leq T} \sum_{j \neq i} \bI_{j \rightarrow i} (\cX) < 1,
% %     \]
% %     where $\bI$ denotes the influence between \emph{elements} in different coordinates.
% %     To be concrete, we have
% %     \[
% %         \bI_{j \rightarrow i}(\cX)
% %         \coloneqq
% %         \max_{ \text{x}_{-i-j}, \text{x}_j, \text{x}_j^\prime  }
% %         % \underset{  z_{-i-j} \in Z^{m-2} }{ z_j, z_j^\prime \in Z}}
% %         d_{\text{TV}}
% %         \left(
% %         \mathbb{P}_{\cX_i | \cX_{-i}} \left( \cdot | \text{x}_{-i-j}, \text{x}_j \right),
% %         \mathbb{P}_{\cX_i | \cX_{-i}} \left( \cdot | \text{x}_{-i-j}, \text{x}_j^\prime \right)
% %         \right).
% %     \]

% %     % We now proceed the \emph{blocking} technique by grouping subsets of indices into blocks.
% %     Let $B \subset \{ 1, \cdots, n \}, \vert B \vert = b$.
% %     We the subscript $\cX_B$ to denote the subset of $\cX$ selected with indices $B$.
% %     The \emph{between-block} influence is then defined by
% %     \begin{equation}\label{eqn:block-influence}
% %         d_{\text{TV}}
% %         \left(
% %         \mathbb{P}_{ \cX_B \mid \cX_{B^c} }
% %         ( \cdot
% %         \mid \text{x}_{B^c} )
% %         ,
% %         \mathbb{P}_{ \cX_B \mid \cX_{B^c} }
% %         ( \cdot
% %         \mid \text{x}^\prime_{B^c} )
% %         \right),
% %     \end{equation}
% %     where $B^c$ is the complement of $B$. 

% %     Observe
% %     \begin{align*}
% %         &\mathbb{P}
% %         \left(
% %         \cX_{i_1}, \cdots, \cX_{i_b}
% %         \mid 
% %         \cX_{B^c}
% %         =
% %         \text{x}_{B^c}
% %         \right)
% %         \\
% %         &=
% %         \mathbb{P}
% %         \left(
% %         \cX_{i_1}
% %         \mid 
% %         \cX_{B^c}=
% %         \text{x}_{B^c}
% %         \right)
% %         \times
% %         \mathbb{P}
% %         \left(
% %          \cX_{i_2}
% %         \mid 
% %         \cX_{i_1},
% %         \cX_{B^c}=
% %         \text{x}_{B^c}
% %         \right)
% %         \cdots 
% %         \times
% %         \mathbb{P}
% %         \left(
% %          \cX_{i_b}
% %         \mid 
% %         \cX_{i_1}, \cdots, \cX_{i_{b-1}},
% %         \cX_{B^c}=
% %         \text{x}_{B^c}
% %         \right)
% %     \end{align*}

% %     Define 
% %     \[
% %     q_k
% %     \coloneqq
% %     \mathbb{P}
% %     \left( \cX_{i_k} \mid \left( \cX_{i_1}, \cdots, \cX_{k-1} \right) = u, \cX_{B^c}=
% %         \text{x}_{B^c} \right),
% %     \]
% %     for $k=1, \cdots, b$.
% %     Thus,
% %     \[
% %     \mathbb{P}
% %         \left(
% %         \cX_{i_1}, \cdots, \cX_{i_b}
% %         \mid 
% %         \cX_{B^c}
% %         =
% %         \text{x}_{B^c}
% %         \right)
% %         =
% %         \prod_{k=1}^b
% %         q_k,
% %         \quad
% %     \mathbb{P}
% %         \left(
% %         \cX_{i_1}, \cdots, \cX_{i_b}
% %         \mid 
% %         \cX_{B^c}
% %         =
% %         \text{x}^\prime_{B^c}
% %         \right)
% %         =
% %         \prod_{k=1}^b
% %         q_k^\prime.
% %     \]
% %     Let 
% %     \[
% %     P_k 
% %     \coloneqq
% %     \left(
% %     \prod_{j=1}^k
% %     q_j^\prime
% %     \right)
% %     \times
% %     \left(
% %     \prod_{j=k+1}^b
% %     q_j
% %     \right),
% %     \]

% %     we then have
% %     \begin{align*}
% %         \Delta
% %         &=
% %         d_{\text{TV}}
% %         \left(
% %         \mathbb{P}_{ \cX_B \mid \cX_{B^c} }
% %         ( \cdot
% %         \mid \text{x}_{B^c} )
% %         ,
% %         \mathbb{P}_{ \cX_B \mid \cX_{B^c} }
% %         ( \cdot
% %         \mid \text{x}^\prime_{B^c} )
% %         \right)
% %         \\
% %         &=
% %         d_{\text{TV}}
% %         \left(
% %         \prod_{k=1}^b
% %         q_k,
% %         \prod_{k=1}^b
% %         q_k^\prime,
% %         \right)
% %         \\
% %         &=
% %         d_{\text{TV}}
% %         \left(
% %         P_0
% %         ,
% %         P_b
% %         \right)
% %         \\
% %         &\leq
% %         \sum_{k=1}^b
% %         d_{\text{TV}}
% %         \left(
% %         P_{k-1}
% %         ,
% %         P_k
% %         \right),
% %     \end{align*}
% %     where the last inequality comes from the triangle inequality with summation over consecutive differences.

% %     Observe 
% %     \begin{align*}
% %         d_{\text{TV}}
% %         \left(
% %         P_{k-1}
% %         ,
% %         P_k
% %         \right)
% %         &=
% %         d_{\text{TV}}
% %         \left(
% %         \prod_{j=1}^{k-1}
% %         q^\prime_j
% %         \times 
% %         (q_k)
% %         \times 
% %         \prod_{j=k+1}^{b}
% %         q_j
% %         ,
% %         \prod_{j=1}^{k-1}
% %         q^\prime_j
% %         \times 
% %         (q_k^\prime)
% %         \times 
% %         \prod_{j=k+1}^{b}
% %         q_j
% %         \right)
% %         \\
% %         &=
% %         d_{\text{TV}}
% %         \left(
% %         \left(
% %         \prod_{j=1}^{k-1}
% %         q^\prime_j
% %         \prod_{j=k+1}^{b}
% %         q_j
% %         \right)
% %         \cdot
% %         \colorbox{cyan!20}{$\left(
% %         q_k , q^\prime_k
% %         \right)$}
% %         \right)
% %         .
% %     \end{align*}

% %     Note that the factor $\left(
% %         \prod_{j=1}^{k-1}
% %         q^\prime_j
% %         \prod_{j=k+1}^{b}
% %         q_j
% %         \right)$ is a probability measure for other coordinates.
% %         Therefore, the above equation implies that the total variation between $P_{k-1}, P_k$ is controlled by the total variation between $q_k, q_k^\prime$.
        
% %         Thus, we have
% %     \[
% %         d_{\text{TV}}
% %         \left(
% %         P_{k-1}
% %         ,
% %         P_k
% %         \right)
% %         =
% %         d_{\text{TV}}
% %         \left(
% %         \left(
% %         \prod_{j=1}^{k-1}
% %         q^\prime_j
% %         \prod_{j=k+1}^{b}
% %         q_j
% %         \right)
% %         \cdot
% %         \colorbox{cyan!20}{$\left(
% %         q_k , q^\prime_k
% %         \right)$}
% %         \right)
% %         \leq\
% %         \sup
% %         d_{\text{TV}}
% %         \left(
% %         \colorbox{cyan!20}{$
% %         q_k , q^\prime_k
% %         $}
% %         \right),
% %     \]
% %     where the supremum is taken over all possible $({\text{x}_{i_1}, \cdots \text{x}_{i_{k-1}}, \text{x}_{i_{k+1}}, \cdots, \text{x}_{i_b}})$.

% %     Note that by Dobrushin's condition, 
% %     \[
% %     d_{\text{TV}}
% %     (q_k, q_k^\prime)
% %     \leq
% %     a^b.
% %     \]

% %     Therefore, we arrive 
% %     \begin{align*}
% %         \Delta
% %         &=
% %         d_{\text{TV}}
% %         \left(
% %         \mathbb{P}_{ \cX_B \mid \cX_{B^c} }
% %         ( \cdot
% %         \mid \text{x}_{B^c} )
% %         ,
% %         \mathbb{P}_{ \cX_B \mid \cX_{B^c} }
% %         ( \cdot
% %         \mid \text{x}^\prime_{B^c} )
% %         \right)
% %         \\
% %         &=
% %         d_{\text{TV}}
% %         \left(
% %         P_0, P_b
% %         \right)
% %         \\
% %         &\leq 
% %         \sum_{k=1}^b
% %         d_{\text{TV}}(P_{k-1}, P_k)
% %         \\
% %         &\leq
% %         \sum_{k=1}^b
% %         d_{\text{TV}}
% %         (q_k, q^\prime_k)
% %         \\
% %         &\leq
% %         b \cdot a^b = \cO(a^b).
% %     \end{align*}
% % \end{proof}

% % The next lemma bounds the empirical Gaussian complexity of a function class $\cF$ over variables satisfying the Dobrushin's condition.

% % \begin{lemma}[Metric Entropy Bound on Weakly Dependent Data]\label{lem:weak-data-complexity}
% %     Following the above notations,
% %     let $(\text{x}_1, \cdots, \text{x}_n) \sim \cD_{\cX}$ be a sequence sampled from a distribution satisfies the Dobrushin's condition, i.e., $\alpha(\cX) < 1$.
% %     Given a function class $\cF$, with some $b > 1 \in \N^+$ such that $\alpha(\cX)^b \ll 1$, we have
% %     \[
% %     \hat{\mathfrak{G}}_X(\cF)
% %     \leq 
% %     \sqrt{\frac{b}{n}}
% %     \int_0^R \sqrt{\log N(\epsilon, \cF, L_2) } d \epsilon
% %     +
% %     \cO\left( \alpha(\cX)^b \right),
% %     \]
% %     where $R \geq \sup_{f,g \in \cF} L_2(f, g)$, $L_2$ is the L2 distance and $N(\cdot)$ is the covering number of $\cF$.
% % \end{lemma}

% % \begin{corollary}\label{cor:complexity-lipschitz}
% %     Following Lemma~\ref{lem:weak-data-complexity}, with $x_i \in \R^d$, for $i = 1, \cdots, n$, and $\cF$ being a $L$-Lipschitz function, we have
% %     \[
% %     \hat{\mathfrak{G}}_X(\cF)
% %     \leq
% %     \cO\left( \alpha(D_x)^b \right)
% %     +
% %     C_0
% %     \left(
% %     \frac{L R}{\epsilon}
% %     \right)^d,
% %     \]
% %     where $C_0$ is a universal constant.
% % \end{corollary}

% % \begin{proof}[Proof of Lemma~\ref{lem:weak-data-complexity}]

% %     We first partition the sequence $(\text{x}_1, \cdots, \text{x}_n)$ into blocks with size $b$.
% %     The empirical Gaussian complexity can be written as
% %     \begin{align*}
% %     \hat{\mathfrak{G}}_X( \cF )
% %     &\coloneqq
% %     \mathbb{E}_{\gamma}
% %     \left[
% %     \frac{1}{n}
% %     \sum_{i=1}^n
% %     \gamma_i f(x_i)
% %     \right]
% %     \\
% %     &=
% %     \mathbb{E}_{\gamma}
% %     \left[
% %     \frac{b}{n}
% %     \sum_{i=1}^{ \lfloor n/b \rfloor}
% %     \sum_{j=1}^{b}
% %     \gamma_i f( x_{ b(j-1) + j} )
% %     \right]
% %     \end{align*}    

% %     We slightly abuse our notations and define
% %     \[
% %         \sum_{j=1}^{b}
% %         \gamma_i f( x_{ b(j-1) + j} )
% %         \coloneqq
% %         \gamma_i f( B_i ).
% %     \]
% %     Then the empirical Gaussian complexity over sequence $(\text{x}_1, \cdots, \text{x}_n)$ can be written as the empirical Gaussian complexity as \emph{super-samples} $( \text{x}_{B_1}, \cdots, \text{x}_{B_{ \lfloor n/b \rfloor }  })$.
% %     We then have

% %     \begin{align*}
% %         \hat{\mathfrak{G}}_X( \cF )
% %         =
% %         \mathbb{E}_{\gamma}
% %         \left[
% %         \frac{1}{n}
% %         \sum_{i=1}^n
% %         \gamma_i f(\text{x}_{B_i}).
% %         \right]
% %     \end{align*}    

% %     We define the event $A$ as the case where each super-samples behaves like i.i.d. samples and $\bar{A}$ denotes the case whenever $A$ fails, i.e. $\mathbb{P}(A) + \mathbb{P}(\bar{A}) = 1$.

% %     \begin{align*}
% %         \mathbb{E}_{\gamma}
% %         \left[
% %         \frac{1}{n}
% %         \sum_{i=1}^n
% %         \gamma_i f(\text{x}_{B_i}).
% %         \right]
% %         =
% %         \mathbb{P}(A) \cdot
% %         \mathbb{E}_A[ \hat{\mathfrak{G}}_X( \cF ) ]
% %         +
% %         \mathbb{P}(\bar{A})
% %         \mathbb{E}_{\bar{A}}[ \hat{\mathfrak{G}}_X( \cF ) ].
% %     \end{align*}    

% %     By Lemma~\ref{lem:block-dependence}, $\mathbb{P}(A) \leq 1 - b \cdot \alpha(\cX)^b$.
% %     Thus, we have
% %     \begin{align*}
% %         \mathbb{E}_{\gamma}
% %         \left[
% %         \frac{1}{n}
% %         \sum_{i=1}^n
% %         \gamma_i f(\text{x}_{B_i}).
% %         \right]
% %         &=
% %         \mathbb{P}(A) \cdot
% %         \mathbb{E}_A[ \hat{\mathfrak{G}}_X( \cF ) ]
% %         +
% %         \mathbb{P}(\bar{A})
% %         \mathbb{E}_{\bar{A}}[ \hat{\mathfrak{G}}_X( \cF ) ]
% %         \\
% %         &\leq
% %         (1 - b \cdot \alpha(\cX)^b)
% %         \sqrt{\frac{b}{n}}
% %         \int_0^R
% %         \sqrt{ \log N\left( \epsilon, \cF, L_2 \right) } d \epsilon
% %         +
% %         (b \cdot \alpha(\cX)^b)
% %         \mathbb{E}_{\bar{A}}[ \hat{\mathfrak{G}}_X( \cF ) ].
% %     \end{align*}    

% %     Note that we then simply bound the term  
% %     \[
% %     \mathbb{E}_{\bar{A}}[ \hat{\mathfrak{G}}_X( \cF ) ]
% %     \leq
% %     2R,
% %     \]
% %     which is dominated by $(b \cdot \alpha(\cX)^b) = O(\alpha(\cX)^b)$.

% %     Thus, we arrive the final result
% %     \begin{equation*}
% %         \hat{\mathfrak{G}}_X( \cF )
% %         \leq
% %         \sqrt{\frac{b}{n}}
% %         \int_0^R
% %         \sqrt{ \log N\left( \epsilon, \cF, L_2 \right) } d \epsilon
% %         +
% %         O(\alpha(\cX)^b).
% %     \end{equation*}

% %     Additionally, Corollary~\ref{cor:complexity-lipschitz} is a direct corollary of Lemma~\ref{lem:weak-data-complexity}.

% % \end{proof}






% \subsection{Proof of the Multi-path Generalization}

\clearpage

\subsection{Analysis of Section~\ref{sec:AR1}}\label{appendix:analysis-ar1}

\begin{definition}[Markov Random Field (MRF) with pairwise potentials]
The random vector $\mathcal{Z} = (\cZ_1, \cdots, \cZ_d)$ over $Z^d$ is an MRF with pariwise potentials if there exist functions $\psi_i : Z \rightarrow \R$ and $\varphi_{ij}: Z^2 \rightarrow \R$ for $i \neq j \in \{ 1, \cdots, d \}$ such that for all $z \in Z^d$,
\begin{equation*}
    \mathbb{P}_{z \sim \mathtt{P}^d }
    \left[ 
    \cZ = z
    \right]
    =
    \prod_{i=1}^d
    e^{\psi(\cZ_i)}
    \prod_{1 \leq i < j \leq d}
    e^{\varphi_{ij}(\cZ_i, \cZ_j)}
\end{equation*}
The functions $\psi_i$ are called as element-wise potentials and $\phi_{ij}$ are pairwise potentials.  
\end{definition}

\begin{definition}
Given an MRF $\cZ$ with potentials $\{ \varphi_i \}$ and $\{ \psi_{ij}\}$, we define
    \begin{equation*}
        \beta_{i,j}(\cZ)
        \coloneqq
        \sup_{\cZ_i, \cZ_j \in Z}
        \lvert
        \varphi_{ij}( \cZ_i, \cZ_j )
        \rvert;
        \quad
        \beta(\cZ)
        \coloneqq
        \max_{1 \leq i \leq d}
        \sum_{j \neq i}
        \beta_{ij}
        ( \mathtt{P}^d ).
    \end{equation*}
\end{definition}

\begin{lemma}\label{lem:beta-bound}
    Given an MRF $\bz$ with pairwise potentials, for any $i \neq j$, $I_{j \rightarrow i}(\bz) \leq \beta_{j,i}(\bz)$.
    $ \bI_{j\rightarrow i}(\cZ)
    \leq
    \bI^{\log}_{j,i}(\cZ) 
    \leq 
    \beta_{j,i}(\cZ)$
    
    % Hence, $\alpha(\bz) \leq \beta(\bz)$.
    % Further, $\bI_{j,i}^{\log}(\bz) \leq \beta_{j,i}(\bz)$.
\end{lemma}
\cref{lem:beta-bound} implies that to satisfy the condition $\alpha^{\log}(\cdot) < 1/2$, it is sufficient to show that $\beta(\cdot) < 1/2$, leading to the following condition.
\begin{equation}
    \langle \bw \bx_{t}, \bx_{t+1} \rangle < \ln \frac{1}{2} + ( \sigma_{\epsilon}^2 ).
    % \quad \text{where}  \norm{\bw}_{\infty} < 1.
\end{equation}

Observe that
\begin{align*}
        \langle \bw \bx_{t}, \bx_{t+1} \rangle 
        &\leq
        \norm{\bw} \cdot \max_t \norm{\bx_t}
        \\
        &= B_w B_x
        \\&< \ln \frac{1}{2} + ( \sigma_{\epsilon}^2 )
        \sim 0.3
        .
\end{align*}

\subsection{Additional Details}

\paragraph{The History Matrix.}
The matrix form of $\mathtt{A}_i(q)$ is presented below
\begin{equation}\label{eqn:history-matrix}
    \mathtt{A}_i(q)
    \coloneqq
    \begin{bmatrix}
        x_1^i & x_2^i & \cdots & x_t^i & x_{t+1}^i & x_{t+2}^i & \cdots \\
        x_T^i & x_{T-1}^i & \cdots & x_{t-1}^i & x_t^i & x_{t+1}^i & \cdots \\
        x_{T-1}^i & x_{T-2}^i & \cdots & x_{t-2}^i & x_{t-1}^i & x_{t}^i & \cdots \\
        \vdots & \vdots & \vdots & \vdots & \vdots & \vdots & \cdots \\
        x_{T-q}^i & x_{T-q+1}^i & \cdots & x_{t-q}^i & x_{t-q+1}^i & x_{t-q+2}^i & \cdots \\   
\end{bmatrix}
\end{equation}

\clearpage

\section{Experimental Details}\label{sec:exp-details}

\subsection{Environment}
We mostly train our model on NVIDIA-H100 GPUs with 2 cores each with 128GB RAM.
2 GPUs are sufficient for all of our experiments.
We use PyTorch 2.1 and our code is based on the open source published by \cite{woo2024unified}.
Training and evaluate takes roughly $12$ hours for one run.


\subsection{Model Architecture}
For most of our experiments, we use MOIRAI-base model.
The hyperparameters are listed in \cref{table:hyperparameters}.

\begin{table}[h]
        \centering
        \caption{Hyperparameters
        }
        % \resizebox{ \textwidth}{!}{  
        \begin{tabular}{l*{1}{c}}
        \toprule
            \bf{parameter} & \multicolumn{1}{c}{\bf{values}}  \\ 
            \midrule
            batch size & $64$ \\ 
            initial learning rate
            & $1\text{e-}3$ \\
            learning rate decay
            & cosine annealing \\
            hidden dimension
            & $768$ \\
            MLP dimension
            & $3072$ \\
            number of heads
            & $12$\\
            training steps
            & $20k$ \\
            max sequence length & $512$ \\
            \midrule
            optimizer
            & AdamW \\
            beta ($\beta_1, \beta_2$)
            & $(0.9, 0.98)$ \\
            weight decay & $1\text{e-}1$ \\
            warm up steps (linear) & $10k$ \\
            \bottomrule
        \end{tabular}
    \label{table:hyperparameters}
\end{table} 




\subsection{Synthetic Data Generation}
We generate the $\mathtt{AR}$ synthetic data similar to Equation~\eqref{eqn:AR-data} but use normalization to stabilize the values.
The parameters of synthetic data are in \cref{table:data-param}.
Consider a sequence of data 
$\bx \in \R^{d \times T} \coloneqq (\bx_1, \dots, \bx_T)$, where $\bx_t = (x_t^1, \cdots, x_t^d) \in \R^d$.
Assuming our target (variate of interest) is in dimension $1$, we assume the $\mathtt{AR}_d(q)$ process generates $x_t^1$ as follows:
\begin{equation}
    x_{t}^1
    =
    \frac{1}{qd}
    \sum_{i=1}^q
    \sum_{j=1}^d
    a_i^j \cdot x_{t-i}^j
    + \epsilon_t
    ,
\end{equation}
where $\epsilon_t \sim N(0, 1)$, $a_i^j \sim N(0, 1) \in \R$.
After recursively generating the time series, we remove its first 50 time steps as burnout.
Each $\mathtt{AR}$ time series has a number of covariates between $1$ to $5$.
For training data, we sampled $100$ different time series, each with $20k$ time steps.
For test data, we randomly generate one time series with time step $5k$, and evaluate our model on all time steps.
We set $q, d \leq 5$ in our experiments.

\paragraph{Seasonality.}
We also conduct experiments on datasets with seasonality information.
Specifically, we consider monthly information.
After generating a multi-variate time series with $T$ time steps $\bx \in \R^{d \times T}$, we then add the seasonality information.
For each time step $t$, its seasonal information is
\[
a \cdot \sin{ \frac{2 \pi T}{f}} \in \R,
\]
where $a \in \R$ is the amplitude, $f \in \N^+$ is the frequency which is $30$ for monthly information.
The whole seasonal information will be added to the time series.

\begin{table}[h]
        \centering
        \caption{Parameter of Synthetic Data
        }
        % \resizebox{ \textwidth}{!}{  
        \begin{tabular}{l*{1}{c}}
        \toprule
            \bf{parameter} & \multicolumn{1}{c}{\bf{values}}  \\ 
            \midrule
            lag size & $\{ 1, 2, 3, 4, 5 \}$ \\ 
            variance
            &  $\text{unif}(0.1, 1)$ \\
            length ($T$)
            & $20k$ \\
            number of covariates ($d$)
            & $\{1, 2, 3, 4, 5 \}$ \\
            \midrule
            amplitude &  $\text{unif}(0, 1.5)$ \\
            frequency &  $30$ \\
            \bottomrule
        \end{tabular}
    \label{table:data-param}
\end{table} 


\subsection{Baselines}

\paragraph{Least Squares Regression.}\label{appendix:ls-baseline}
Consider MOIRAI taking an input AR sequence $\bx \in \R^{d \times T}$, to match our theoretical results (\cref{thm:any-variate-auto}), we transform $\bx$ into the following input-label pairs
\begin{align*}
    \tilde{\bx}_1 &= 
    \left(
    ( \bx_1, \cdots \bx_q ),
    \bx_{q+1}
    \right)
    \\
    \tilde{\bx}_2 &= 
    \left(
    ( \bx_2, \cdots \bx_{q+1} ),
    \bx_{q+2}
    \right)...
\end{align*}
After fitting least squares on this transformed dataset with $T-q$ samples, it predicts the $T+1$-th time step with the following input
\[
\tilde{\bx}_{\text{test}} = 
\left(
\bx_{T-q+1}, \cdots \bx_{T}
\right).
\]
For least squares, we use learning rate as $0.1$, and perform full gradient descent with $50, 100$ iterations.

% \section{Additional Theoretical Background}


% \begin{lemma}[Error bound on Probability Distribution Perturbation]\label{lem:prob-perturbation}
%     Given $\bz = (\bz_1, \dots, \bz_d) \in \R^d$, $\bz^\prime = \bz + \epsilon$, where $\epsilon \in \R^d$.
%     Assume the following holds,
%     $$
%     \vert 
%     \bz_i - \bz^\prime_i
%     \vert
%     \leq 
%     \varepsilon,
%     $$
%     we have
%     \begin{equation*}
%         \norm{n(\bz) - n(\bz^\prime)}_2
%         \leq 
%         \frac{1}{s}
%         \norm{\epsilon}_2,
%     \end{equation*}
%     where 
%     $n(\bz) = (  \nicefrac{\bz_1}{\sum_{i=1} \bz_i}, \dots, \nicefrac{\bz_d}{\sum_{i=1} \bz_i})$ is a normalization operator.    
% \end{lemma}

% \todo[inline]{Revise this paragraph}

% \begin{proof}

%     The goal is to find the operation norm of $J$, which is the Jacobian of the function $n$, which
%     $\norm{J}$ satisfies 
%     $$
%     \norm{ n(\bz) - n(\bz+\epsilon) } 
%     \simeq
%     \norm{J \epsilon}
%     \leq 
%     \norm{J} \cdot \norm{\epsilon}.
%     $$

%     The normalization function $n(\bz)$ defines the component-wise operation such that
%     \begin{equation*}
%         n(\bz)_i
%         =
%         \frac{\bz_i}{s},
%         \quad\text{where }
%         s = \sum_{j=1}^d \bz_j.
%     \end{equation*}
%     Let $s \geq \delta > 0$.

%     We first calculate the Jacobian of the normalization function such that
%     $$
%     J_{ij}
%     =
%     \frac{\partial n(\bz)_i}{\partial \bz_j}
%     =
%     \frac{\delta_{ij} s - \bz_i}{s^2},
%     $$
%     where
%     \begin{equation*}
%         \delta_{ij}
%         =
%         \begin{cases}
%             1 & \text{if } i = j 
%             \\
%             0 & \text{otherwise}
%         \end{cases}.
%     \end{equation*}
%     The Jacobian can be written as
%     \begin{equation*}
%         J 
%         =
%         \frac{1}{s}
%         (\mathbf{I} 
%         - 
%         \bv e^\top
%         ),
%     \end{equation*}
%     where $e^\top$ is a row vector with all ones, and $\bv = \frac{\bz}{s}$.
%     Note that the Jacobian is symmetric.
%     Therefore, its norm is bounded by the absolute value of its largest eigenvalue.
%     For eigenvectors $\bu$ that are orthogonal to $\bv$, we have
%     \begin{equation}
%         J \bu
%         =
%         \frac{1}{s}
%         (\mathbf{I} 
%         - 
%         \bv e^\top
%         )
%         \bu
%         =
%         \frac{1}{s} \bu.
%     \end{equation}
%     For eigenvectors along $\bv$,
%     \begin{equation*}
%         J \bv
%         =
%         \frac{1}{s}
%         (\mathbf{I} 
%         - 
%         \bv e^\top
%         )
%         \bv
%         =
%         0.
%     \end{equation*}

%     Therefore, given a perturbation vector 
%     \begin{equation*}
%         \norm{J \epsilon}_2
%         \leq 
%         \frac{1}{s} \norm{\epsilon}_2
%     \end{equation*}

    
% \end{proof}



\subsection{Additional Experiments}\label{appendix:additional-exp}



\paragraph{Seasonality Data.}
Here we present the experimental results on training transformers on seasonality data.
The data generation is the same as described above.
We use the same setup for seasonality data, where our training data comes from time series with $d \in \{ 1, 2, 3, 4, 5 \}$, and $q = \{ 1, 2, 3, 4, 5\}$.
The evaluation results on seasonality data is in \cref{fig:seasonal}.
We observe that transformers are capable of inferring data with seasonality.
Note that transformers are capable of achieving nearly optimal performance, while least squares regression fails, indicating that transformers are capable of fitting a more complicated model than $\mathtt{AR}$ on a given time series.

\begin{figure}[h]
    \centering
    \includegraphics[width=0.5\linewidth]{imgs/figure_5season.png}
    \caption{We observe that when least squares regression fails to obtain the optimal error rate for prediction, transformers are capable of having their MSE converge towards $1$ as the lookback size increases.
    This indicates that these models are capable of fitting a more complex model other than linear regression on a given time series.
    }
    \label{fig:seasonal}
\end{figure}

% \paragraph{Strongly Dependent Data.}
% Here we construct a dataset with strongly dependent data and pretrain a MOIRAI transformer on it.
% We generate a dataset with $\mathtt{AR}$ processes plus non-ergodic drift.

% \paragraph{Evaluation on Unseen $d, q$.}
% Here we conduct experiments on transformers to see if they can generalize to time series with unseen values of $d, q$.
% Therefore, we train our models on $d \in \{ 4, 5 \}$, and $q \in \{ 4, 5 \}$.
% We evaluate these models on several scenarios
% \begin{itemize}
%     \item[1] Generalization to higher order data
%     \item[2] Generalization to lower order data
%     \item[3] Generalization to higher dimension data
%     \item[4] Generalization to lower dimension data
% \end{itemize} 


% \end{appendices}

\end{document}


% This document was modified from the file originally made available by
% Pat Langley and Andrea Danyluk for ICML-2K. This version was created
% by Iain Murray in 2018, and modified by Alexandre Bouchard in
% 2019 and 2021 and by Csaba Szepesvari, Gang Niu and Sivan Sabato in 2022.
% Modified again in 2023 and 2024 by Sivan Sabato and Jonathan Scarlett.
% Previous contributors include Dan Roy, Lise Getoor and Tobias
% Scheffer, which was slightly modified from the 2010 version by
% Thorsten Joachims & Johannes Fuernkranz, slightly modified from the
% 2009 version by Kiri Wagstaff and Sam Roweis's 2008 version, which is
% slightly modified from Prasad Tadepalli's 2007 version which is a
% lightly changed version of the previous year's version by Andrew
% Moore, which was in turn edited from those of Kristian Kersting and
% Codrina Lauth. Alex Smola contributed to the algorithmic style files.
