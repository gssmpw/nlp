%%%%%%%% ICML 2025 EXAMPLE LATEX SUBMISSION FILE %%%%%%%%%%%%%%%%%

\documentclass{article}

% Recommended, but optional, packages for figures and better typesetting:
\usepackage{microtype}
\usepackage{graphicx}
\usepackage{subfigure}



\usepackage{comment}
\usepackage{enumitem}
\usepackage[table]{xcolor}

% hyperref makes hyperlinks in the resulting PDF.
% If your build breaks (sometimes temporarily if a hyperlink spans a page)
% please comment out the following usepackage line and replace
% \usepackage{icml2025} with \usepackage[nohyperref]{icml2025} above.
\usepackage{hyperref}
\usepackage[title, toc,page]{appendix}
\usepackage{titlesec}
\usepackage{titletoc}
\usepackage{booktabs} % for professional tables


\usepackage{braket}
\usepackage{multirow}
\usepackage{natbib}
\usepackage{graphicx}
\usepackage{subfigure}
\usepackage{makecell}
\usepackage{booktabs}
\usepackage{array}
\usepackage{url}
\usepackage{algorithm}
\usepackage{algorithmic}
\usepackage{dsfont}



% Attempt to make hyperref and algorithmic work together better:
% \newcommand{\theHalgorithm}{\arabic{algorithm}}

% Use the following line for the initial blind version submitted for review:
\usepackage{icml2025}

% If accepted, instead use the following line for the camera-ready submission:
% \usepackage[accepted]{icml2025}

% For theorems and such
\usepackage{amsmath}
\usepackage{amssymb}
\usepackage{mathtools}
\usepackage{amsthm}
\usepackage{amssymb}

\usepackage{amsfonts}
\usepackage{bm}
\usepackage{bbm}

% if you use cleveref..
\usepackage[capitalize,noabbrev]{cleveref}

%%%%%%%%%%%%%%%%%%%%%%%%%%%%%%%%
% THEOREMS
%%%%%%%%%%%%%%%%%%%%%%%%%%%%%%%%
\theoremstyle{plain}
\newtheorem{theorem}{Theorem}[section]
\newtheorem{proposition}[theorem]{Proposition}
\newtheorem{lemma}[theorem]{Lemma}
\newtheorem{corollary}[theorem]{Corollary}
\theoremstyle{definition}
\newtheorem{definition}[theorem]{Definition}
\newtheorem{assumption}[theorem]{Assumption}
\theoremstyle{remark}
\newtheorem{remark}[theorem]{Remark}
\usepackage{titlesec}
\usepackage{booktabs}

% Todonotes is useful during development; simply uncomment the next line
%    and comment out the line below the next line to turn off comments
%\usepackage[disable,textsize=tiny]{todonotes}
\usepackage[textsize=tiny]{todonotes}
\usepackage{nicefrac}


% math commands 
\newcommand{\bfa}[1]{\boldsymbol{#1}}
\newcommand{\lef}{\left}
\newcommand{\rig}{\right}
\newcommand{\norm}[1]{\left\lvert \left\lvert#1\right\rvert \right\rvert}
\newcommand{\vertiii}[1]{{\left\vert\kern-0.25ex\left\vert\kern-0.25ex\left\vert #1 
    \right\vert\kern-0.25ex\right\vert\kern-0.25ex\right\vert}}


%----- bold fonts -----%

\newcommand{\ab}{\mathbf{a}}
\newcommand{\bbb}{\mathbf{b}}
\newcommand{\cbb}{\mathbf{c}}
\newcommand{\db}{\mathbf{d}}
\newcommand{\eb}{\mathbf{e}}
\newcommand{\fb}{\mathbf{f}}
\newcommand{\gb}{\mathbf{g}}
\newcommand{\hb}{\mathbf{h}}
\newcommand{\ib}{\mathbf{i}}
\newcommand{\jb}{\mathbf{j}}
\newcommand{\kb}{\mathbf{k}}
\newcommand{\lb}{\mathbf{l}}
\newcommand{\mb}{\mathbf{m}}
\newcommand{\nbb}{\mathbf{n}}
\newcommand{\ob}{\mathbf{o}}
\newcommand{\pb}{\mathbf{p}}
\newcommand{\qb}{\mathbf{q}}
\newcommand{\rb}{\mathbf{r}}
\newcommand{\sbb}{\mathbf{s}}
\newcommand{\tb}{\mathbf{t}}
\newcommand{\ub}{\mathbf{u}}
\newcommand{\vb}{\mathbf{v}}
\newcommand{\wb}{\mathbf{w}}
\newcommand{\xb}{\mathbf{x}}
\newcommand{\yb}{\mathbf{y}}
\newcommand{\zb}{\mathbf{z}}

\newcommand{\ba}{\bm{a}}
\newcommand{\bb}{\bm{b}}
\newcommand{\bc}{\bm{c}}
\newcommand{\bd}{\bm{d}}
\newcommand{\be}{\bm{e}}
\newcommand{\bbf}{\bm{f}}
\newcommand{\bg}{\bm{g}}
\newcommand{\bh}{\bm{h}}
\newcommand{\bi}{\bmf{i}}
\newcommand{\bj}{\bm{j}}
\newcommand{\bk}{\bm{k}}
\newcommand{\bl}{\bm{l}}
\newcommand{\bbm}{\bm{m}}
\newcommand{\bn}{\bm{n}}
\newcommand{\bo}{\bm{o}}
\newcommand{\bp}{\bm{p}}
\newcommand{\bq}{\bm{q}}
\newcommand{\br}{\bm{r}}
\newcommand{\bs}{\bm{s}}
\newcommand{\bt}{\bm{t}}
\newcommand{\bu}{\bm{u}}
\newcommand{\bv}{\bm{v}}
\newcommand{\bw}{\bm{w}}
\newcommand{\bx}{\bm{x}}
\newcommand{\by}{\bm{y}}
\newcommand{\bz}{\bm{z}}




\newcommand{\Ab}{\mathbf{A}}
\newcommand{\Bb}{\mathbf{B}}
\newcommand{\Cb}{\mathbf{C}}
\newcommand{\Db}{\mathbf{D}}
\newcommand{\Eb}{\mathbf{E}}
\newcommand{\Fb}{\mathbf{F}}
\newcommand{\Gb}{\mathbf{G}}
\newcommand{\Hb}{\mathbf{H}}
\newcommand{\Ib}{\mathbf{I}}
\newcommand{\Jb}{\mathbf{J}}
\newcommand{\Kb}{\mathbf{K}}
\newcommand{\Lb}{\mathbf{L}}
\newcommand{\Mb}{\mathbf{M}}
\newcommand{\Nb}{\mathbf{N}}
\newcommand{\Ob}{\mathbf{O}}
\newcommand{\Pb}{\mathbf{P}}
\newcommand{\Qb}{\mathbf{Q}}
\newcommand{\Rb}{\mathbf{R}}
\newcommand{\Sbb}{\mathbf{S}}
\newcommand{\Tb}{\mathbf{T}}
\newcommand{\Ub}{\mathbf{U}}
\newcommand{\Vb}{\mathbf{V}}
\newcommand{\Wb}{\mathbf{W}}
\newcommand{\Xb}{\mathbf{X}}
\newcommand{\Yb}{\mathbf{Y}}
\newcommand{\Zb}{\mathbf{Z}}

\newcommand{\bA}{\bm{A}}
\newcommand{\bB}{\bm{B}}
\newcommand{\bC}{\bm{C}}
\newcommand{\bD}{\bm{D}}
\newcommand{\bE}{\bm{E}}
\newcommand{\bF}{\bm{F}}
\newcommand{\bG}{\bm{G}}
\newcommand{\bH}{\bm{H}}
\newcommand{\bI}{\bm{I}}
\newcommand{\bJ}{\bm{J}}
\newcommand{\bK}{\bm{K}}
\newcommand{\bL}{\bm{L}}
\newcommand{\bM}{\bm{M}}
\newcommand{\bN}{\bm{N}}
\newcommand{\bO}{\bm{O}}
\newcommand{\bP}{\bm{P}}
\newcommand{\bQ}{\bm{Q}}
\newcommand{\bR}{\bm{R}}
\newcommand{\bS}{\bm{S}}
\newcommand{\bT}{\bm{T}}
\newcommand{\bU}{\bm{U}}
\newcommand{\bV}{\bm{V}}
\newcommand{\bW}{\bm{W}}
\newcommand{\bX}{\bm{X}}
\newcommand{\bY}{\bm{Y}}
\newcommand{\bZ}{\bm{Z}}


%----- calligraphic fonts -----%

\newcommand{\cA}{\mathcal{A}}
\newcommand{\cB}{\mathcal{B}}
\newcommand{\cC}{\mathcal{C}}
\newcommand{\cD}{\mathcal{D}}
\newcommand{\cE}{\mathcal{E}}
\newcommand{\cF}{\mathcal{F}}
\newcommand{\cG}{\mathcal{G}}
\newcommand{\cH}{\mathcal{H}}
\newcommand{\cI}{\mathcal{I}}
\newcommand{\cJ}{\mathcal{J}}
\newcommand{\cK}{\mathcal{K}}
\newcommand{\cL}{\mathcal{L}}
\newcommand{\cM}{\mathcal{M}}
\newcommand{\cN}{\mathcal{N}}
\newcommand{\cO}{\mathcal{O}}
\newcommand{\cP}{\mathcal{P}}
\newcommand{\cQ}{\mathcal{Q}}
\newcommand{\cR}{\mathcal{R}}
\newcommand{\cS}{{\mathcal{S}}}
\newcommand{\cT}{{\mathcal{T}}}
\newcommand{\cU}{\mathcal{U}}
\newcommand{\cV}{\mathcal{V}}
\newcommand{\cW}{\mathcal{W}}
\newcommand{\cX}{\mathcal{X}}
\newcommand{\cY}{\mathcal{Y}}
\newcommand{\cZ}{\mathcal{Z}}




%----- blackboard bold fonts-----%

\newcommand{\CC}{\mathbb{C}}
\newcommand{\EE}{\mathbb{E}}
\newcommand{\VV}{\mathbb{V}}
\newcommand{\II}{\mathbb{I}}
\newcommand{\KK}{\mathbb{K}}
\newcommand{\LL}{\mathbb{L}}
\newcommand{\MM}{\mathbb{M}}
\newcommand{\NN}{\mathbb{N}}
\newcommand{\PP}{\mathbb{P}}
\newcommand{\QQ}{\mathbb{Q}}
\newcommand{\RR}{\mathbb{R}}
\newcommand{\SSS}{\mathbb{S}}
\newcommand{\ZZ}{\mathbb{Z}}
\newcommand{\XX}{\mathbb{X}}
\newcommand{\YY}{\mathbb{Y}}
\newcommand{\OOmega}{\mathbb{\Omega}}




%----- bold greek fonts -----%

\newcommand{\bmeta}{\bm{\eta}}

\newcommand{\balpha}{\bm{\alpha}}
\newcommand{\bbeta}{\bm{\beta}}
\newcommand{\bgamma}{\bm{\gamma}}
\newcommand{\bepsilon}{\bm{\epsilon}}
\newcommand{\bvarepsilon}{\bm{\varepsilon}}
\newcommand{\bzeta}{\bm{\zeta}}
\newcommand{\btheta}{\bm{\theta}}
\newcommand{\bvartheta}{\bm{\vartheta}}
\newcommand{\bkappa}{\bm{\kappa}}
\newcommand{\blambda}{\bm{\lambda}}
\newcommand{\bmu}{\bm{\mu}}
\newcommand{\bnu}{\bm{\nu}}
\newcommand{\bxi}{\bm{\xi}}
\newcommand{\bpi}{\bm{\pi}}
\newcommand{\bvarpi}{\bm{\varpi}}
\newcommand{\brho}{\bm{\varrho}}
\newcommand{\bsigma}{\bm{\sigma}}
\newcommand{\bvarsigma}{\bm{\varsigma}}
\newcommand{\btau}{\bm{\tau}}
\newcommand{\bupsilon}{\bm{\upsilon}}
\newcommand{\bphi}{\bm{\phi}}
\newcommand{\bvarphi}{\bm{\varphi}}
\newcommand{\bchi}{\bm{\chi}}
\newcommand{\bpsi}{\bm{\psi}}
\newcommand{\bomega}{\bm{\omega}}



\newcommand{\R}{\mathbb{R}}
\newcommand{\N}{\mathbb{N}}
\newcommand{\argmin}{\mathop{\mathrm{argmin}}}
\newcommand{\argmax}{\mathop{\mathrm{argmax}}}


\def\CCC{\textcolor{red}}

% The \icmltitle you define below is probably too long as a header.
% Therefore, a short form for the running title is supplied here:
\icmltitlerunning{Transformers and Their Roles as Time Series Foundation Models}



\begin{document}


% spacing things
% \setlist[itemize]{leftmargin=1em, before=\vspace{-0.5em}, after=\vspace{-0.5em}, itemsep=0.1em}
% \setlist[enumerate]{leftmargin=1.4em, before=\vspace{-0.5em}, after=\vspace{-0.5em}, itemsep=0.1em}



\twocolumn[
\icmltitle{Transformers and Their Roles as Time Series Foundation Models
}

% Understanding Transformers as Time Series Foundation Models
% Transformers and Their Role as Time Series Foundation Models
% provable universal forecasting with MOIRAI
% M

% It is OKAY to include author information, even for blind
% submissions: the style file will automatically remove it for you
% unless you've provided the [accepted] option to the icml2025
% package.

% List of affiliations: The first argument should be a (short)
% identifier you will use later to specify author affiliations
% Academic affiliations should list Department, University, City, Region, Country
% Industry affiliations should list Company, City, Region, Country

% You can specify symbols, otherwise they are numbered in order.
% Ideally, you should not use this facility. Affiliations will be numbered
% in order of appearance and this is the preferred way.
\icmlsetsymbol{equal}{*}

\begin{icmlauthorlist}
\icmlauthor{Firstname1 Lastname1}{equal,yyy}
\icmlauthor{Firstname2 Lastname2}{equal,yyy,comp}
\icmlauthor{Firstname3 Lastname3}{comp}
\icmlauthor{Firstname4 Lastname4}{sch}
\icmlauthor{Firstname5 Lastname5}{yyy}
\icmlauthor{Firstname6 Lastname6}{sch,yyy,comp}
\icmlauthor{Firstname7 Lastname7}{comp}
%\icmlauthor{}{sch}
\icmlauthor{Firstname8 Lastname8}{sch}
\icmlauthor{Firstname8 Lastname8}{yyy,comp}
%\icmlauthor{}{sch}
%\icmlauthor{}{sch}
\end{icmlauthorlist}

\icmlaffiliation{yyy}{Department of XXX, University of YYY, Location, Country}
\icmlaffiliation{comp}{Company Name, Location, Country}
\icmlaffiliation{sch}{School of ZZZ, Institute of WWW, Location, Country}

\icmlcorrespondingauthor{Firstname1 Lastname1}{first1.last1@xxx.edu}
\icmlcorrespondingauthor{Firstname2 Lastname2}{first2.last2@www.uk}

% You may provide any keywords that you
% find helpful for describing your paper; these are used to populate
% the "keywords" metadata in the PDF but will not be shown in the document
\icmlkeywords{Machine Learning, ICML}

\vskip 0.3in
]

% this must go after the closing bracket ] following \twocolumn[ ...

% This command actually creates the footnote in the first column
% listing the affiliations and the copyright notice.
% The command takes one argument, which is text to display at the start of the footnote.
% The \icmlEqualContribution command is standard text for equal contribution.
% Remove it (just {}) if you do not need this facility.

%\printAffiliationsAndNotice{}  % leave blank if no need to mention equal contribution
\printAffiliationsAndNotice{\icmlEqualContribution} % otherwise use the standard text.

% %%%%% NEW MATH DEFINITIONS %%%%%

% \usepackage{amsmath,amsfonts,bm}
\usepackage{amsmath,amsfonts}

\usepackage{pifont}


\newcommand{\R}{\mathbb{R}}


\def\va{{\mathbf{a}}}
\def\vg{{\mathbf{g}}}

% Sets
\def\sR{\mathbb{R}}
\def\sC{\mathbb{C}}
\def\sZ{\mathbb{Z}}
\def\sN{\mathbb{N}}
\def\sQ{\mathbb{Q}}

\def\sS{\mathcal{S}}



% Vectors
\def\vzero{{\mathbf{0}}}
\def\vone{{\mathbf{1}}}
\def\vmu{{\mathbf{\mu}}}
\def\vtheta{{\mathbf{\theta}}}
\def\va{{\mathbf{a}}}
\def\vb{{\mathbf{b}}}
\def\vc{{\mathbf{c}}}
\def\vd{{\mathbf{d}}}
\def\ve{{\mathbf{e}}}
\def\vf{{\mathbf{f}}}
\def\vg{{\mathbf{g}}}
\def\vh{{\mathbf{h}}}
\def\vi{{\mathbf{i}}}
\def\vj{{\mathbf{j}}}
\def\vk{{\mathbf{k}}}
\def\vl{{\mathbf{l}}}
\def\vm{{\mathbf{m}}}
\def\vn{{\mathbf{n}}}
\def\vo{{\mathbf{o}}}
\def\vp{{\mathbf{p}}}
\def\vq{{\mathbf{q}}}
\def\vr{{\mathbf{r}}}
\def\vs{{\mathbf{s}}}
\def\vt{{\mathbf{t}}}
\def\vu{{\mathbf{u}}}
\def\vv{{\mathbf{v}}}
\def\vw{{\mathbf{w}}}
\def\vx{{\mathbf{x}}}
\def\vy{{\mathbf{y}}}
\def\vz{{\mathbf{z}}}
\def\vzeta{{\mathbf{\zeta}}}

% Matrix
\def\mA{{\mathbf{A}}}
\def\mB{{\mathbf{B}}}
\def\mC{{\mathbf{C}}}
\def\mD{{\mathbf{D}}}
\def\mE{{\mathbf{E}}}
\def\mF{{\mathbf{F}}}
\def\mG{{\mathbf{G}}}
\def\mH{{\mathbf{H}}}
\def\mI{{\mathbf{I}}}
\def\mJ{{\mathbf{J}}}
\def\mK{{\mathbf{K}}}
\def\mL{{\mathbf{L}}}
\def\mM{{\mathbf{M}}}
\def\mN{{\mathbf{N}}}
\def\mO{{\mathbf{O}}}
\def\mP{{\mathbf{P}}}
\def\mQ{{\mathbf{Q}}}
\def\mR{{\mathbf{R}}}
\def\mS{{\mathbf{S}}}
\def\mT{{\mathbf{T}}}
\def\mU{{\mathbf{U}}}
\def\mV{{\mathbf{V}}}
\def\mW{{\mathbf{W}}}
\def\mX{{\mathbf{X}}}
\def\mY{{\mathbf{Y}}}
\def\mZ{{\mathbf{Z}}}
\def\mBeta{{\mathbf{\beta}}}
\def\mPhi{{\mathbf{\Phi}}}
\def\mLambda{{\mathbf{\Lambda}}}
\def\mSigma{{\mathbf{\Sigma}}}


% Expectation
% \def\eE{\mathop{\mathbb{E}}\limits}
\def\eE{\mathbb{E}}

% Probability
\def\pP{\mathbb{P}}

% Tilde
\def\tf{\tilde{f}}
\def\tS{\tilde{S}}
\def\wtF{\widetilde{\mathcal{F}}}
\def\whR{\widehat{R}}
\def\tvx{\tilde{\mathbf{x}}}
\def\ty{\tilde{y}}


\def\defeq{\overset{\textup{def}}{=}}
% \def\defeq{\overset{.}{=}}
\def\defone{\overset{\text{\ding{172}}}{=}}
\def\deftwo{\overset{\text{\ding{173}}}{=}}
\def\leqone{\overset{\text{\ding{172}}}{\leq}}
\def\leqtwo{\overset{\text{\ding{173}}}{\leq}}
\def\leqthree{\overset{\text{\ding{174}}}{\leq}}
\def\leqfour{\overset{\text{\ding{175}}}{\leq}}
\def\eqone{\overset{\text{\ding{172}}}{=}}
\def\eqtwo{\overset{\text{\ding{173}}}{=}}
\def\eqthree{\overset{\text{\ding{174}}}{=}}
\def\eqfour{\overset{\text{\ding{175}}}{=}}
\def\geqfive{\overset{\text{\ding{176}}}{\geq}}

\begin{abstract}
% We study the approximation and generalization of MOIRAI \cite{woo2024unified}, the first time series foundation model that achieves \emph{universal forecasting}.
% MOIRAI is a transformer-based model which stands out for its ability to handle any arbitrary number of covariates and to generalize effectively across unseen domains. 
% For approximation, we show that there exists a MOIRAI transformer that performs least square regression on any multi-variate time series data via in-context learning.
% Notably, the unique encoding method and attention layer in MOIRAI allows it to process multiple covariates efficiently.
% Our approximation results are not only the first that effectively handles any arbitrary number of covariates, but are also highly compatible with existing in-context learning theories.
% For generalization, when the data is sampled from distributions under Dobrushin's condition, we present the generalization bound for both single and multi-source pretraining.
% The empirical results validate our theories, highlighting MOIRAI's efficacy as an universal forecaster.


We give a comprehensive analysis of transformers as time series foundation models, focusing on their approximation and generalization capabilities. 
First, we demonstrate that there exist transformers that fit an autoregressive model on input univariate time series via gradient descent. 
We then analyze MOIRAI \cite{woo2024unified}, a multivariate time series foundation model capable of handling an arbitrary number of covariates. 
We prove that it is capable of automatically fitting autoregressive models with an arbitrary number of covariates, offering insights into its design and empirical success.   
For generalization, we establish bounds for pretraining when the data satisfies Dobrushin’s condition. 
Experiments support our theoretical findings, highlighting the efficacy of transformers as time series foundation models.

% For approximation, we show that there exists a MOIRAI transformer that performs least square regression on any multi-variate time series data via in-context learning.
% we show that there exists a MOIRAI transformer that achieves \emph{one-model-for-all-contexts}, where it learns the pattern of any time series in-context and makes accurate predictions.
% Notably, such an property is achieved by the unique encoding method and attention layer in MOIRAI, allowing it to process multiple covariates efficiently.
% For generalization, when the data is sampled from distributions under Dobrushin's condition, we present the generalization bound for both single and multi-source pretraining.
% The empirical results validate our theories, highlighting MOIRAI's efficacy as an universal forecaster.
\end{abstract}


\section{Introduction}
Large Language Models (LLMs) have demonstrated significant success in learning and performing inference on real world high dimensional datasets. Most modern LLMs use the Transformer model \citep{vaswani2017attention} as their backbone. 

Many existing works have considered the theoretical guarantees of the in-context-learning setup of Transformers \citep{bai2024transformers, akyurek2022learning}. However, in practice, LLMs require a significant amount of pretraining data to achieve their empirical advantage. And, little is known about the unsupervised learning guarantees of Transformers, especially after a sufficient number of problem instances are observed by the Transformer model in the pre-training phase. Motivated by the strong empirical performance of Transformers, we provide theoretical analysis of the Transformers on a standard unsupervised learning problem of clustering a mixture of Gaussians in the multi-class setup. Our results suggest that Transformers, like human brains, can benefit from experienced problem instances and learn the way to solve the problem (algorithms). Then, when fed with a new problem instance, Transformers can solve it through the learned algorithms naturally.

The problem of clustering a mixture of multivariate Gaussian is one of the most standard unsupervised learning problems \citep{bishop2006pattern} that can be solved by the EM algorithm or Lloyd's algorithm \citep{lloyd1982least}. The EM algorithm contains both the \emph{Expectation} and the \emph{Maximization} sub-procedures where the \emph{Expectation Step} creates a function for \textbf{the expectation} of the log-likelihood evaluated using the current estimate for the parameters and the \emph{Maximization Step} computes parameters maximizing the expected log-likelihood given by the \emph{Expectation Step}. We draw connections between the Softmax Attention in Transformers and the EM algorithms through the following:
\begin{enumerate}
    \item \emph{The Expectation Step} involves a normalized sum in \textbf{the expectation} whose weight vector is naturally given by the output of the softmax function as 
    $\begin{bmatrix}
        \frac{\exp(z_1)}{\sum_{i=1}^D\exp(z_i)},\ldots,\frac{\exp(z_D)}{\sum_{i=1}^D\exp(z_i)}
    \end{bmatrix}$.
    \item \emph{The Maximization Step} involves finding the index with the maximum value in a vector (the Hardmax Function). This is naturally approximated by the Softmax function as its name suggests.
\end{enumerate}
Given the strong connections of the two steps to the Softmax function, we build an approximation theory for Lloyd's algorithm in a constructive manner. We also note that existing works only build approximation bound for multihead ReLU neural networks \citep{bach2017breaking} while the Softmax approximation of multivariate to multivariate mapping remains a myth. We resolve this obstacle by proving an approximation bound for multi-head Transformers on a class of $\bb R^{d_1}\to\bb R^{d_2}$ mappings that might be of independent interests.



\paragraph{Contributions.} We summarize our major contributions as follows:
\begin{enumerate}
    \item We rigorously show that a pre-trained Transformer can perform multi-class clustering by drawing its connection to Lloyd's algorithm, which is used as a proof machine. We provide constructive proof and error bound for the approximation;
    \item We further consider the setup where the Transformer model is trained with independent instances from a class of clustering problems whose labels are used as supervision. We show that Transformers are able to generalize the mapping on new clustering problem instances. We provide upper bounds on the generalization error for the empirical risk minimizer in the pre-training task. Moreover, we show that given a sufficient number of training instances and proper initialization, pre-trained Transformers reach the fundamental limit of the problem;
    \item We systematically evaluate the performance of Transformers through extensive simulations. These empirical results demonstrate that Transformers perform well in the multi-class clustering task even when the assumptions leading to the theoretical results no longer hold.
\end{enumerate}

\subsection{Related Works}
\paragraph{Transformers are algorithm approximators.} Recently, the capacity of Transformers to automatically performing certain algorithms has drawn great attention from researchers. In particular, a rich line of recent works studied the expressive power of Transformers to perform in-context learning (ICL) \citep{akyurek2022learning,bai2024transformers,abernethy2024mechanism,li2023transformers2,jeon2024informationtheoretic}. Specifically, \citet{akyurek2022learning,bai2024transformers,abernethy2024mechanism} studied how Transformers perform gradient descent based training to perform ordinary or sparse linear regression on the context. \citet{chen2024transformers} further showed how Transformers utilize the multi-head structure to perform in-context sparse linear regression.  \citet{li2023transformers2} studied the generalization and stability of transformers in ICL tasks. \citet{jeon2024informationtheoretic} studies the information-theoretical lower bound of in-context learning. Another closely related line of works studied how Transformers can be pretrained by gradient descent to perform certain tasks. Specifically, \citet{zhang2023trained,huang2023context,chen2024training} studied the pretraining optimization dynamics of Transformers to learn in-context linear prediction rules. \citet{li2024one} showed that one-layer Transformers can be trained to perform one-nearest neighbor classification in context. \citet{ahn2024transformers,giannou2024well} studied the training of Transformers in learning various optimization methods. \citet{li2023transformers}  studied how Transformers can be trained to learn topic models. \citet{jelassi2022vision} proved that Vision Transformers can learn a class of image-like data whose patches follow certain spatial structures. \citet{zhang2025transformer} studied how Transformers can learn to perform variable selection in ``group-sparse'' linear regression.
% \paragraph{In-Context Learning with Transformers.} Recent studies have explored the in-context learning (ICL) capabilities of Transformers \citep{garg2022can, bai2024transformers}. Specifically, \citet{bai2024transformers} examined the approximation and generalization properties of Transformers in ICL tasks, including various linear and logistic regression scenarios. \citet{akyurek2022learning, von2023transformers} investigated how Transformers approximate gradient descent during ICL. In our work, we leverage the Power Method to provide theoretical guarantees for the approximation capabilities of Transformers. Notably, obtaining eigenvectors through gradient descent is challenging due to the lack of an explicit functional form. To our knowledge, this is the first study to offer theoretical guarantees for Transformers' approximation of the Power Method. For further practical insights into ICL, see \citet{dong2022survey} and the references therein.

\paragraph{Other theoretical studies on Transformers.} Various efforts have been made to gain a theoretical understanding of Transformers. \citet{yun2019transformers} analyzed the universal approximation properties of Transformers for sequence-to-sequence functions. \citet{li2023transformers} studied the mean-filed limit of large-scale Transformers and proved global convergence in regression tasks. 
\citet{perez2021attention} showed that Transformers with hard-attention are Turing complete exclusively based on their capacity to compute and access internal dense representations of the data. \citet{bhattamishra2020computational}
 further provided an alternate and simpler proof to show that vanilla Transformers are Turing-complete, and then proved that Transformers with only positional masking and without any positional encoding are also Turing-complete. \citet{liu2022transformers} showed that a low-depth Transformer can represent the computations of any finite-state automaton by hierarchically reparameterizing its recurrent dynamics. \citet{yao2021self} demonstrated that Transformers can efficiently process bounded hierarchical languages, offering better space complexity compared to recurrent neural networks.

 

% The computational power of Transformers has been investigated by \citet{perez2021attention}, \citet{bhattamishra2020computational}, and \citet{liu2022transformers}. \citet{hron2020infinite} studied the behavior of infinitely wide multi-head and single-head attention mechanisms. Additionally, \citet{yao2021self} demonstrated that Transformers can efficiently process bounded hierarchical languages, offering better space complexity compared to recurrent neural networks.




















% This work is related to a few different branches in the literature.
% \paragraph{In Context Learning of Transformers.} Some recent works studied the in-context learning (ICL) capacities of Transformers \citep{garg2022can, bai2024transformers}. In particular, \citep{bai2024transformers} considered the approximation and generalization properties of transformers on the ICL tasks, including many linear regression and logistic regression setups. The problem of PCA is a standard unsupervised learning problem. Hence, it differs from ICL in that there is no individual label that the model needs to learn. \citet{akyurek2022learning, von2023transformers} considered the approximation of transformers on gradient descent when performing ICL. In this work, the proof machine utilizes the Power Method. We also notice that performing gradient descent is difficult to obtain the eigenvectors as no explicit functional form is given. To the best of the authors' knowledge, this is the first work that provides theoretical guarantees for the transformers' approximation of the Power method in the literature. Other related works on the more practical side of ICL can be found in \citet{dong2022survey} and references therein.
% \paragraph{Other Theoretical Works on Transformers.}
% Many other attempts are made to theoretically understand transformers. \citet{yun2019transformers} studied the universal approximation properties of transformers on sequence-to-sequence functions. \citet{perez2021attention,bhattamishra2020computational,liu2022transformers} studied the computational power of transformers. \citet{hron2020infinite} studied the limit of infinite width multi/single head attentions. \citet{yao2021self} showed that transformers can process bounded hierarchical languages and demonstrate better space complexity than recurrent neural networks.

\paragraph{Notations}
In this work we follow the following notation conventions. The vector-valued variable is given by boldfaced characters. We denote $[n]:=\{1,\ldots,n\}$ and $[i:j]:=\{i,i+1,\ldots, j\}$ for $i<j$. The universal constants are given by $C$ and are ad hoc. For a vector $\bfa v$ we denote $\Vert\bfa v\Vert_2$ as its $L_2$ norm. For a matrix $\bfa A\in\bb R^{m\times n}$ we denote its operator norm as $\Vert\bfa A\Vert_2:=\sup_{\bfa v\in\bb S^{n-1}}\Vert\bfa A\bfa v\Vert_2$. Given two sequences $a_n$ and $b_n$, we denote $a_n\lesssim b_n$ or $a_n = O(b_n)$ if $\limsup_{n\to\infty}|\frac{a_n}{b_n}|<\infty$ and $a_n=o(b_n)$ if $\limsup_{n\to\infty}|\frac{a_n}{b_n}|=0$. We denote $\mbbm 1_A$ as the indicator function for event $A$. The universal constants are denoted by $C$ in this work and are ad hoc. We use $\ca B(\Vert\cdot\Vert, r)$ to denote a ball with radius $r$ under the norm $\Vert\cdot\Vert$.  

\paragraph{Organizations} The rest of the paper is organized as follows: Section \ref{sect2} reviews standard contexts and describes the learning problem; Section \ref{sect3} provides rigorous theoretical results and sketches of proof; Section \ref{sect4} provides extensive experimental details and results; Section \ref{sect5} discusses the limitations and potential future works. The detailed proofs and additional figures in experiments are delayed to the appendix. The supplementary materials include the code for the experiments.
% We consider the problem of learning the EM algorithm using Transformers. We consider the unsupervised learning setup, given our input sequence given by
% $
%      X=\begin{bmatrix}
%          X_1,\ldots, X_{N}
%     \end{bmatrix}$. Then we consider the following problems under the class of EM algorithms.
%     \subsection{Clustering Mixture of Gaussians}
%     Moreover, we assume that $\{X_i\}_{i\in[N]}$ are i.i.d. samples generated from the Gaussian mixture model, defined by
%     \begin{align*}
%         p( \bfa x|\bfa\mu,z=\ell)=(2\pi)^{-\frac{m}{2}}|\Sigma|^{-\frac{1}{2}}\exp\lef(-\frac{1}{2}(\bfa x-\bfa\mu_{\ell})^\top\Sigma^{-1}(\bfa x-\bfa \mu_{\ell})\rig),\quad z\in[k],
%     \end{align*}
%     where we use $z$ to denote the index of the group from which the sample is generated from.
    
%     And consider the simple EM algorithm that is standard in the literature of clustering \cite{bishop2006pattern}. We use $\bfa z$. Then it is not hard to check that the Maximum Likelihood estimator is given by 
%     \begin{align*}
%        \lef(\{\wha \mu_{\ell}\}_{\ell\in[k]},\wha z\rig)&=\argmax_{\{\bfa\mu_{\ell}\}_{\ell\in[k]},\bfa z}\log p(\bfa X|\bfa\mu,\bfa z)=\argmin_{\{\bfa\mu_{\ell}\}_{\ell\in[k]},\bfa z}\sum_{i\in[n]}(X_i-\bfa\mu_{z_i})^\top\Sigma^{-1}(X_i-\bfa \mu_{z_{i}}) \\
%        &=\argmin_{\{\bfa\mu_{\ell}\}_{\ell\in[k]},\bfa z}\sum_{i\in[n]}\min_{\ell\in[k]}\Vert\bfa\mu_{z_i}-X_i\Vert_{\Sigma^{-1}}^2
%     \end{align*}
%     Note that the Lloyd's algorithm is used to solve the above $k$-means problem, given by
%     \begin{align*}
%         \wha\mu_{\ell}^{(t)}=\frac{\sum_{i=1}^nX_{\ell}^{(t-1)}\mbbm 1_{\wh z_i^{(t-1)}=\ell}}{\sum_{i=1}^n\mbbm 1_{\wh z_i^{(t-1)}=\ell}},\enspace\forall \ell\in[k],\qquad \wh z_i^{(t)}=\argmin_{\ell\in[k]}\Vert X_i-\wha\mu_{\ell}^{(t)}\Vert^2_{\Sigma^{-1}},\enspace\forall i\in[n].
%     \end{align*}
%     Our goal is to study the algorithmic approximation theory on transformers for problems that can be solved using EM algorithms.
%     \begin{assumption}
%         We assume that the distribution of the samples are given by $$\{\bfa X_i\}_{i\in[d]}\overset{i.i.d.}{\sim}(2\pi)^{-\frac{m}{2}}|\Sigma|^{-\frac{1}{2}}\exp\lef(-\frac{1}{2}(\bfa x-\bfa\mu_{\ell})^\top\Sigma^{-1}(\bfa x-\bfa\mu_{\ell})\rig),\quad z\in[k].$$ And the minimal separation condition is given by $\min_{i\neq j}\Vert\bfa\mu_i-\bfa\mu_j \Vert_2=\Delta$.
%     \end{assumption}
% \subsection{Community Detection in the Stochastic Block Model}
% Then we consider the community detection problem under the sparse network, under the following model specification:
% \begin{align*}
%     X_{ij}\sim\begin{cases}
%         Bern(\frac{p}{n})&\text{ if }z_i=z_j\\
%         Bern(\frac{q}{n})&\text{ otherwise, }
%     \end{cases}\quad z_i\in[k]\text{ for all }i\in[n],\quad\bb P(z_i=\ell)=\pi_{\ell}\text{ for all }\ell\in[k].
% \end{align*}
% Then we study the following algorithm
% \begin{align*}
%     \wh A_{i\ell}^{(t)}=\frac{\sum_{j\neq i}^nX_{ij}\mbbm 1_{\wh z_j^{(t)}=\ell}}{\sum_{j\neq i}^n\mbbm 1_{\wh z_i^{(t)}=\ell}},\quad\forall i\in[n],\ell\in[k],\quad \wh z_i^{(t)}=\argmax_{\ell\in[k]}\wh A_{i,\ell}^{(t-1)}\quad\forall i\in[n],
% \end{align*}
% which approximately minimizes the following objective
% \begin{align*}
%     (\wh A,\wh z)=\argmax_{(A,z)}\sum_{i=1}^n\inf_{\ell\in[k]}\sum_{j\neq i}(A_{i,\ell}-X_{jz(j)})^2
% \end{align*}


\paragraph{Notations.}
We use the following notation conventions. 
The vector-valued variable is given by boldfaced characters. 
We denote $[n]:=\{1,\ldots,n\}$ and $[i:j]:=\{i,i+1,\ldots, j\}$ for $i<j$. 
The universal constants are given by $C$ and are ad hoc. 
Considering a sequence of vectors $(\bx_1, \cdots, \bx_T)$, we use $\bx$ without index to represent the whole sequence, and $\bx_{i:j}$ represents $(\bx_i, \cdots, \bx_j)$ for $i < j$.
We impose periodic boundary conditions for the negative index, i.e., $\bx_{-1} = \bx_{T}$.
For a vector $\bfa v$ we denote $\Vert\bfa v\Vert_2$ as its $L_2$ norm. 
For a matrix $\bfa A\in  \R^{m\times n}$ we denote its operator norm as $\Vert\bfa A\Vert_2:=\sup_{\bfa v\in \mathbb{S}^{n-1}}\Vert\bfa A\bfa v\Vert_2$. 
Random variables are given by calligraphic characters $\cX$, and elements from a domain set are given by normal font $\text{x}$.
For more details, see Table~\ref{tab:nomenclature}.

% \vspace{-0.5em}
\section{Problem Setup}
This section describes our problem setup.
We introduce the architecture of transformer-based time series foundation models and how we construct our datasets.
% \vspace{-0.5em}
\subsection{Transformers}
We consider a sequence of $N$ input vectors $\{ h_i \}_{i=1}^N \subset \R^D$, where $ \bfa H \coloneqq \left[ h_1, \cdots, h_N \right] \in \R^{D \times N}$.
Given any $ \bfa H \in \R^{D \times N}$, we define the attention layer as follows.
\begin{definition}[Attention layer\normalfont]\label{def:attn}
    {\normalfont
    A self-attention layer with $M$ heads is denoted as $\text{Attn}^{\dagger}_{\bm{\theta}_0}(\cdot)$ with parameters $\bm{\theta}_0 = \{ (\bfa V_m), (\bfa Q_m), (\bfa K_m)  \}_{m \in [M]} \subset \R^{D \times D}$.
    The self-attention layer processes any given input sequence $\bfa H \in \R^{D \times N}$ as
    }
    \begin{align*}
    \text{Attn}_{\bm{\theta}_0}^{\dagger} 
    &\left( \bfa H \right)
    \coloneqq
    \bfa H + 
    \frac{1}{N}
    \sum_{m=1}^M
    (\bfa V_m \bfa H)
    \cdot
   % \CCC{\times} 
    \\
    &\sigma 
    \left( 
    \left( \bfa Q_m \bfa H \right)^\top 
    \left(\bfa K_m \bfa H \right) 
    \right),  
    % \hfill{\text{\R^{D \times N}}}
\end{align*}
where $\sigma \coloneqq t \mapsto \text{ReLU}(t)/N$.
% \CCC{We should not use times for matrix product. times is for outer product.}
\end{definition}
\paragraph{Any-variate Attention.}
Next, we introduce the any-variate attention, where \cite{woo2024unified} uses it to replace the standard attention in transformers.
The any-variate attention introduces two learnable variables: Attention Bias $u_1, u_2 \in \R$,
for disambiguation between variates.
\begin{definition}[Any-variate Attention.]\label{def:any-variate-attn}
    An any-variate attention layer with $M$ heads is denoted as $\text{Attn}_{\bm{\theta}_1}(\cdot)$ with parameters $\bm{\theta}_{1} = \{ (\bfa V_m), (\bfa Q_m), (\bfa K_m), (u_m^1), (u_m^2)  \}_{m \in [M]}$.
    With any input $H \in \R^{D \times N}$, we have
    \begin{align*}
    \text{Attn}_{\bm{\theta}_1} &\left( \bfa H \right)
    \coloneqq
    \bfa H + 
    \frac{1}{N}
    \sum_{m=1}^M
    (\bfa V_m \bfa H)
    \times
    \\
    &\sigma 
    \left( 
    \left( \bfa Q_m \bfa H \right)^\top 
    \left(\bfa K_m \bfa H \right) 
    +
    u_m^1 * \bU
    +
    u_m^2 * \bar{\bU}
    \right),  
    % \hfill{\text{\R^{D \times N}}}
\end{align*}
where $\sigma \coloneqq t \mapsto \text{ReLU}(t)/N$, $\bU \in \R^{N \times N}$ is a block diagonal matrix with block size $T\in \N^+$, such that each block consists of $1$s, $\bar{\bU} = \bI - \bU$, and $*$ denotes a constant multiply to all entries of a matrix.
\end{definition}

\begin{remark}
    In \cite{woo2024unified}, the attention score is calculated with the RoPE embedding \cite{su2024roformer}:
    \[
    \sigma 
    \left( 
    \left( \bfa Q_m \bfa H \right)^\top 
    \bfa R
    \left(\bfa K_m \bfa H \right) 
    +
    u_m^1 * \bU
    +
    u_m^2 * \bar{\bU}
    \right).  
    \]
    We omit the notation of rotary matrix $\bfa R$ as it is not learnable and is invertible and thus merged into $\bfa Q, \bfa K$ in our analysis.
\end{remark}

\begin{definition}[MLP Layer\normalfont]
    {\normalfont
    We denote an MLP layer with hidden state dimension $D^\prime$ as $\text{MLP}_{\bm{\theta}}(\cdot)$ with parameters $\bm{\theta}_2 = ( \bW_1, \bW_2 ) \in \R^{ D^\prime \times D } \times \R^{D \times D^\prime}$.
    The MLP layer processes any given input sequence $\bfa H \in \R^{D \times N}$ as
    }
    \begin{equation*}
        \text{MLP}_{\bm{\theta}_2} (\bfa H)
        \coloneqq
        \bfa H +
        \bW_2 \sigma(\bW_1 \bfa H).
    \end{equation*}
\end{definition}
Finally, we define a transformer with $L \geq 1$ layers, each consisting of any-variate attention and an MLP layer.
\begin{definition}[MOIRAI Transformer\normalfont]\label{def:moirai}
    {
    \normalfont
    We define the $L$-layer MOIRAI transformer \cite{woo2024unified}, $\text{TF}_{\bm{\theta}}(\cdot)$, as
    }
    \begin{equation*}
        \text{TF}_{\bm{\theta}}(\bfa H)
        =
        \text{MLP}_{\bm{\theta}_2^L}
        \left(
        \text{Attn}_{\bm{\theta}_1^L}
        \left(
        \cdot \cdot
        \text{MLP}_{\bm{\theta}_2^1}
        \left(
        \text{Attn}_{\bm{\theta}_1^1}
        (\bfa H)
        \right)
        \right)
        \right)
        .
    \end{equation*}
    Note that this transformer is equipped with any-variate attention instead of the standard attention.
    For transformers with standard attention, we denote it as 
    $\text{TF}_{\bm{\theta}}^{\dagger}(\cdot)$.
\end{definition}

We use $\bm{\theta}$ to denote the vectorization of all parameters in a transformer and super-index $\ell$ to denote the parameter of the $\ell$-th layer.
Thus, the parameter of a transformer is defined by
\begin{equation*}
        \bfa \theta = 
        \left\{
        \left\{
        \left(\{\bfa Q_m^\ell,\bfa K_m^\ell,\bfa V_m^\ell, u_m^{1, \ell}, u_m^{2, \ell}\}_{m\in[M]}, \bfa W_{1}^\ell,\bfa W_{2}^\ell
        \right)
        \right\}_{\ell\in[L]}
        \right\}.
\end{equation*}    

We denote the ``attention-only" transformers with $\bW_1^{(\ell)}, \bW_2^{(\ell)} = 0$, as $\text{TF}_{\bm{\theta}}^0(\cdot)$ for shorthand.
We define the following norm of a MOIRAI transformer as
\begin{align*}
    \Vert\bfa\theta\Vert_{op}
    \coloneqq
    \max_{\ell\in[L]}
    &\Big\{\max_{m\in[M^{\ell}]}
    \Big\{\Vert \bfa Q_m^\ell\Vert_2,\Vert\bfa K_m^\ell\Vert_2,
        \\
        \vert u_m^{1\ell} \vert , \vert u_m^{2\ell} \vert 
    \Big\}
    &+
    \sum_{m=1}^{M^{\ell}}\Vert\bfa V_m^\ell\Vert_2+\Vert\bfa W_1^{\ell}\Vert_2+\Vert\bfa W_2^{\ell}\Vert_2 \Big\}, 
\end{align*}
where $M^{\ell}$ is the number of heads of the $\ell$-th Attention layer.


\subsection{Data Generation}
Here, we first consider the case where we aim to find a multi-layered transformer that performs least squares regression via In-context learning (ICL).
Specifically, we assume our data is generated from an autoregressive process $\mathtt{AR}_d(q)$ as follows, where $q, d$ denotes the steps of lag and number of covariates, respectively. 
Consider a sequence of data 
$\bx \in \R^{d \times T} \coloneqq (\bx_1, \dots, \bx_T)$, where $\bx_t = (x_t^1, \cdots, x_t^d) \in \R^d$.
Assuming our target (variate of interest) is in dimension $1$, we assume the $\mathtt{AR}_d(q)$ process generates $x_t^1$ as follows:
\begin{equation}\label{eqn:AR-data}
    x_{t}^1
    =
    \sum_{i=1}^q
    \sum_{j=1}^d
    a_i^j \cdot x_{t-i}^j
    + \epsilon_t
    =
    \sum_{j=1}^d
    \langle \bw^j , \bx_{t-q: t-1}^j
    \rangle
    +
    \epsilon_t
    ,
\end{equation}
where $\epsilon_t \sim N(0, 1)$, $a_i^j \in \R^1$.
We denote the concatenation of all weights $\bw^\star = (\bw_1, \cdots, \bw^j)\in \R^{qd}$.
We assume bounded features $\norm{\bx_{t-q:t-1}}_2 \leq B_x$ , for all $t = 1,\cdots, T$.
The first equation writes the $\mathtt{AR}$ process in scalar form, and the second writes it in vector form.
In the following chapters, we will start by considering the uni-variate case ($\mathtt{AR}_1(q)$) and then move on to the multi-variate case ($\mathtt{AR}_d(q)$).
\vspace{-1.5em}
\paragraph{Problem Setup.}
Given a function class $\cF: \R^{d\times T}\mapsto \R$, our goal is to find a universal function $f\in\cF$ such that, given any time series generated from any arbitrary $\mathtt{AR}_d(q)$, its prediction error is bounded by some $\varepsilon \geq 0$, i.e.,
\[
\norm{f( \Tilde{\bx}) - x_{T}^1}_2
\leq
\varepsilon,
\]
where $\Tilde{\bx}$ denotes the time series $\bx$ with $x_T^1$ being masked.
\begin{remark}
    In the appendix, we show that even when the autoregressive process follows some non-linear relationship, there still exists a universal $f$ that predicts all non-linear AR process accurately.
\end{remark}
% \paragraph{Multivariate Autoregressive Process.}
% We extend the definition and notation for the \texttt{AR($q$)} process to the multivariate AR process, denoted as \texttt{MAR($q$)}.
% Consider a sequence of data $\bx \in \R^{d \times T} \coloneqq (x_1, \dots, x_T)$, we set the first dimension of $\bx$: $x_t^1$, for $t = 1, \cdots, T$ as target, and the other $d-1$ dimensions of $\bx$ as its covariates.
% We say $\bx$ is generated by a \texttt{MAR($q$)} process if the target of the sequence is generated by the following 
% \begin{equation}\label{eqn:MAR-data}
%     x_{t+1}^1
%     =
%     \sum_{i=1}^q
%     \sum_{j=1}^d
%     \langle
%     \bw_i^\star, x_i^j
%     \rangle
%     +
%     \epsilon_i,
% \end{equation}
% where $\bw_i^\star \in \R^d$ and $\epsilon_i \sim N(0, 1)$, for $i = 1, \cdots , q$,.
% Here we slightly abuse the notation in the \texttt{AR} process of $\bw^\star$, as we will be focusing on \texttt{MAR}($q$) in the rest of the paper.

% \vspace{-0.5em}
\section{Approximation}
We study the algorithmic approximation perspective of transformer-based time series foundation models.
We first investigate transformers as uni-variate time series foundation models as a warm-up.
Next, we will move on to MOIRAI \cite{woo2024unified} and analyze how its unique design and pre-processing methods enable its universality.
% construct multi-layer transformers to implement least square regression on data (1) generated by autoregressive process, and (2) structured in the form of any-variate encoding.
% \vspace{-0.5em}
\subsection{Warm Up: Autoregressive Regression}
We start our analysis with a warm-up example on the $\mathtt{AR}_1(q)$ model.
We show that standard transformers are capable of performing gradient descent via in-context learning on autoregressive data.
Here, we consider an input sequence with the following form
\begin{align}\label{eqn:input-data}
    \bfa H 
    &\coloneqq
    \begin{bmatrix}
        x_1 & x_2 &  \dots & x_{T} & 0
        \\
        \bp_1 & \bp_2 & \dots & \bp_{T} &
        \bp_{T+1}
    \end{bmatrix}
    \in \R^{D \times (T+1)}
    ,
    \\
    \bp_i
    &\coloneqq
    \begin{bmatrix}
        \mathbf{0}_{d^\prime}
        \\
        \be_i
        \\
        1
        \\
        1\{ i < {T} \}
    \end{bmatrix}
    \in \R^{d^\prime + T + 3}
    ,
\end{align}
where $\be_i$ is an one-hot vector with 1 at the $i$-th entry, and $d^\prime + T + 3 = D$.
Here, our goal is to predict $x_{T}$.
\begin{remark}\label{remark:format-assumption}
    Most in-context learning studies \cite{akyrek2023what, bai2024transformers, li2023transformers} make an assumption on the input data, where they assume it is formatted with features and labels in the same column, i.e.,
\begin{equation}\label{eqn:format-assumption}
\begin{bmatrix}
    \bx_1 &  \bx_2 & \dots & \bx_{N} 
    \\ 
    \yb_1 & \by_2 & \dots & \by_{N}
    \\
    \pb_1 & \pb_2 & \dots & \pb_{N}
\end{bmatrix}.
\end{equation}
In contrast, we adopt a natural approach that leverages the raw structure of the data, particularly for the $\mathtt{AR}_d(q)$ process. 
In this setting, each time step’s label also serves as a feature for future steps. 
Further, the unknown value of $q$ complicates the task of achieving such a format in Equation~\eqref{eqn:format-assumption}.
\end{remark}

Our next lemma shows that transformers are indeed capable of reformatting $\bfa H$ into the form of Equation~\ref{eqn:format-assumption}.
Notably, the following lemma relaxes the assumption in Remark~\ref{remark:format-assumption} of previous studies as well.

\begin{lemma}\label{lem:input-causal}
    Given a sequence of token $\bfa H$ in the form of Equation~\ref{eqn:input-data}, there exists a one-layer, $q_{\max}$ head attention layer, such that for any $q \leq q_{\max}$, the columns of $\text{Attn}_{\bm{\theta}}^{\dagger}( \bfa H )$ has the following form:
    \begin{equation}
    \text{Attn}_{\bm{\theta}_1}^{\dagger}( \bfa H )_i
    \coloneqq
        \begin{bmatrix}
            x_i
            \\
            x_{i-1}
            \\
            \vdots
            \\
            x_{i-q}
            \\
            \bp_i^\prime
        \end{bmatrix},
        \quad
        \bp_i^\prime 
        \coloneqq
        \begin{bmatrix}
        \mathbf{0}_{ d^\prime - q }
        \\
        \be_i
        \\
        1
        \\
        1 \{ i < T \}
        \end{bmatrix}.
    \end{equation}
\end{lemma}
The proof is in \cref{proof:lem-input-casual}.
\cref{lem:input-causal} is crucial in our analysis as it connects the theoretical results in ICL \cite{bai2024transformers} to uni-variate time series forecasting.
% The implication of \cref{lem:input-causal} is highly related to the results in \cite{bai2024transformers}.
When data formats in the form of Equation~\eqref{eqn:format-assumption}, \cite{bai2024transformers} show that there exists a multi-layer transformer that performs linear regression via gradient descent on the first $N-1$ data points and evaluates the $N$-th one.
Thus, \cref{lem:input-causal} implies transformers are also capable of performing linear regression on time series data, which we present in the following paragraph.

This lemma applies to both any-variate attention and standard attention, as the latter can be viewed as a special case of any-variate attention by setting $u^1, u^2 = 0$.
Additionally, the construction of a single layer with $q$ heads is not a strict requirement; the lemma also holds for $c$ layers of  $\frac{q}{c}$ head attention, for any $c$ satisfies $\frac{q}{c} >= 2$.

With Lemma~\ref{lem:input-causal}, we are able to apply the in-context learning results in \cite{bai2024transformers} on the $\mathtt{AR}_1(q)$ case.
Consider the data generated by the $\mathtt{AR}$ process in Equation~\ref{eqn:AR-data}.
Given an input time series $\bx \in \R^{ d \times T}$, we define the least squares estimator as the empirical risk minimizer over the time series, i.e.,
{\small
\begin{align*}
    \ell_{\text{reg}}(\bw, \bx_{t-1:t-q}  )
    &\coloneqq
    \frac{
    \left[
    \langle
    \bw ,
    [ \bx^1_{t-1:t-q} ; \dots ; \bx^d_{t-1:t-q}]
    \rangle
    -
    x_t^1
    \right]^2
    }{2}
    \\
    L_{\text{reg}}
    (\bw, \bx)
    &\coloneqq
    \frac{1}{T-1}
    \sum_{t=1}^{T-1}
    \ell_{\text{reg}}
    \left(
    \bw, \bx_{t-1:t-q}
    \right)
    \\
    \hat{\bw}_{\text{ERM}} 
    &\coloneqq
    \argmin_{\bw\in\R^{dq}}
    \;
    L_{\text{reg}}
    \left(
    \bw, \bx
    \right),
\end{align*}
}
where $[ \bv ; \bu]$ denotes the concatenation between vectors, as $[\bx^1_{t-1:t-q} ; \bx^2_{t-1:t-q}] = ( \bx^1_{t-1},  \bx^1_{t-2},  \cdots, \bx^2_{t-q+1}, \bx^2_{t-q} ) \in \R^{2q}$, $\tilde{\bx}$ denotes masking out the last time step of the target variate, and $L_{reg}$ is a loss, which is $\alpha$-strongly convex, and $\beta$-smooth over $\R^{dq}$.
We make the following assumption and then present our first result on uni-variate time series ($d=1$).
\begin{assumption}\label{assumption:effective-regression}
    The regression problem above $\hat{\bw}_{\text{ERM}}$ is well-conditioned and has a bounded solution.
\end{assumption}
\begin{proposition}[Uni-variate Autoregressive Regression via Transformers]
    Assume \cref{assumption:effective-regression} holds and fix a $q_{\max} > 0$.
    For any $0 \leq \alpha \leq \beta$ with $\kappa \coloneqq \frac{\beta}{\alpha}$, $B_w > 0$, and $\epsilon < B_x B_w / 2$,
    there exists a $L$-layer transformer
     $\text{TF}_{\bm{\theta}}^{0 \dagger} 
    \left( \cdot \right)$, with
    \begin{align*}
        L = L_1 + L_2, \quad
        L_1& = \lceil 2 \kappa \log( \frac{B_x B_w}{2 \epsilon} ) \rceil,
        \quad
        L_2 = \lceil \frac{q_{\max}}{3} \rceil,
        \\
        \text{max}_{\ell \in [L]} 
        M^{(\ell)} 
        &\leq 3,
        \quad
        \norm{ \bm{\theta} }_{\text{op}}
        \leq
        | 4R + 8\beta^{-1} |
        ,
    \end{align*}
    ($R \coloneqq \text{max}\{ B_x B_w, B_x, 1 \}$), the following holds.
    On any input data $\bx$ generated by any $\mathtt{AR}_1(q)$ process such that
    \begin{equation}
        0 < q \leq q_{\max}
        \quad
       \norm{ \hat{\bw}_{\text{ERM}} }_2
       \leq 
       \frac{B_w}{2},
    \end{equation}
    we have
    \begin{equation}
        \lVert
        \hat{\bx}_{T}
        -
        \left\langle
        \hat{\bw}_{\text{ERM}}, 
        [
        \bx_{t-1:t-q}^1 ; \dots 
        ; \bx_{t-1:t-q}^d
        ]
        \right\rangle
        \rVert
        \leq 
        \epsilon,
    \end{equation}
    where 
    $\hat{\bx}_{T} = \mathtt{read}(\text{TF}_{\bm{\theta}}^{0 \dagger}
    \left( \bfa H \right))$.
    The $\mathtt{read}(\bfa H)$ operation reads out the first entry of $T$-th column of $\bfa H$.
\end{proposition}
This proposition follows immediately from \cref{lem:input-causal} and \citep[Theorem~4]{bai2024transformers}.
The above result applies for MOIRAI with $u^1_m, u^2_m = 0$ in all heads and layers.
Further, one can replace the least squares ERM with lasso or ridge ERM and obtain a similar result by applying Theorem 4, 7, and 13 of \cite{bai2024transformers}.

So far, we show that transformers are capable of solving uni-variate autoregressive regression with, at best, one additional layer compared to the results in \cite{bai2024transformers}.
The result above provides insights on transformer-based uni-variate time series foundation models \cite{ansari2024chronos, rasul2023lag, das2023decoder}.
To study MOIRAI, we then include two ingredients into our analysis: the \textit{any-variate encoding} and the \textit{covariates} in the following chapters.

\subsection{Approximation Error of MOIRAI Transformer}
In this subsection, we extend our results to the multivariate autoregressive process ($d > 1$) and the encoding method of MOIRAI.
Note that in the multi-variate case, we only focus on MOIRAI as it is the only transformer-based model that is compatible with arbitrary number of covariates. 
We start by introducing the any-variate encoding.
\paragraph{Any-Variate Encoding.}
\cite{woo2024unified} propose to flatten a $d$-dimensional time series, $\bx \in \R^{d \times T}$, into a $1$-dimensional sequence, i.e., $\bx^\prime \in \R^{1 \times Td}$.
This operation transforms time series with arbitrary number of covariates ($d$), into a long sequence with fixed dimension, enabling consistent input dimension for transformers.
Following the flattening operation, \cite{woo2024unified} also proposes to add two types of indices into the input sequence: the time and variate ID.
We term the above operations as the any-variate encoding, which transforms a multivariate sequence $\bx \in \R^{d \times T}$, as follows:
{    \small
\begin{equation}\label{eqn:AV-encoding}
    \begin{bmatrix}
        x_1^1 & \cdots & x_T^1 
        \\
         x_1^2 & \cdots & x_T^2
         \\
         \vdots & \vdots & \vdots 
         \\
        x_1^d & \cdots & x_T^d    
    \end{bmatrix}
    \rightarrow
    \begin{bmatrix}
        x_1^1 & \cdots & \cellcolor{cyan!20} x_{T}^1 &  \cdots & x_1^d & \cdots & x_{T}^d
        \\
        \bp_1 & \cdots & \bp_{T} &  \cdots & \bp_1 & \cdots & \bp_T
        \\
        \be_1 & \cdots & \be_1 &  \cdots & \be_d & \cdots & \be_d
    \end{bmatrix},
\end{equation}}
where $\be_i$ is the variate index, a one-hot vector with $i$-th entry being $1$, and $\bp_i$ is the time index, which is defined the same as Equation~\eqref{eqn:input-data}.
This is without loss of generality because the discrete-time and variate ID used in \cite{woo2024unified} can be easily transformed into a high-dimensional vector with the embedding layer.
Note that only the target variate has length $T$, we highlight $x_{T}^1$ as it is our prediction target and will be masked as $0$.
% The variate index is defined as
% \begin{equation*}
%     \bp_i
%     \coloneqq
%     \begin{bmatrix}
%         \mathbf{0}_{d^\prime}
%         \\
%         \be_i
%         \\
%         1
%         \\
%         1\{ i < {T+1} \}
%     \end{bmatrix}.
% \end{equation*}

Now we define the history matrix $\mathtt{A}_i(q) \in \R^{q+1 \times T}$ for the $i$-th covariates $(x_1^i, \cdots, x_T^i)$, with order $q$, such that
\begin{equation*}
    \mathtt{A}_i(q)_{\mu, \nu}
    \coloneqq
        x^i_{\nu - \mu + 1},
        \quad
        \text{ for }\mu\in[d], \nu\in[q].
    % \begin{bmatrix}
    %     x_1^i & x_2^i & \cdots & x_t^i & x_{t+1}^i & x_{t+2}^i & \cdots \\
    %     x_T^i & x_{T-1}^i & \cdots & x_{t-1}^i & x_t^i & x_{t+1}^i & \cdots \\
    %     x_{T-1}^i & x_{T-2}^i & \cdots & x_{t-2}^i & x_{t-1}^i & x_{t}^i & \cdots \\
    %     \vdots & \vdots & \vdots & \vdots & \vdots & \vdots & \cdots \\
    %     x_{T-q}^i & x_{T-q+1}^i & \cdots & x_{t-q}^i & x_{t-q+1}^i & x_{t-q+2}^i & \cdots \\   
    % \end{bmatrix},
\end{equation*}
where in the $j$-th column of $\mathtt{A}_i(q)$, it contains historical values of $x_j^i$ with lag $q > 0$.
% The matrix form of $\mathtt{A}_i(q)$ is in Equation~\eqref{eqn:history-matrix}.

\begin{lemma}\label{lem:mar-group-wise}
Fix $q_{\max}, D \in \N^+$.
Given any $T > 0, d^\prime > q > 0, d > 0$ such that $T > q$, $q_{\max} \geq q$.
For any input matrix $\bH$ in the form of any-variate encoding in Equation~\ref{eqn:AV-encoding}, such that $\bH \in \R^{ D \times dT}$.
There exists a one layer, $q_{\max}$ head \textbf{any-variate attention} that performs the following operation.
    \begin{align*}
    &
    \begin{bmatrix}
        x_1^1 & \cdots & x_{T}^1 & x_1^2 & \cdots & x_T^2 & \cdots & x_1^d & \cdots & x_T^d
        \\
        \bp_1 & \cdots & \bp_{T} & \bp_1 & \cdots & \bp_T & \cdots & \bp_1 & \cdots & \bp_T
        \\
        \be_1 & \cdots & \be_1 & \be_2 & \cdots & \be_2 & \cdots & \be_d & \cdots & \be_d
    \end{bmatrix}
    \\ 
    &\quad\quad\quad\quad\mapsto
    \begin{bmatrix}
        \mathtt{A}_1(q) & \mathtt{A}_2(q) & \cdots & \mathtt{A}_d(q)
        \\
        \bm{0}_{d^{''} \times T} & \bm{0}_{d^{''} \times T} & \cdots & \bm{0}_{d^{''} \times T}   
        \\
        \ddots & \ddots & \cdots & \ddots 
    \end{bmatrix},
    \end{align*}
    where $d^{''}  = d^\prime - q_{\max}$.
\end{lemma}
The proof is in \cref{proof:any-var-enc}.
Intuitively, the above operation performs the same operation in Lemma~\ref{lem:input-causal} but in a variate-wise fashion.
\cref{lem:mar-group-wise} shows that any-variate attention is capable of organizing the history of each variate efficiently.
To again achieve the format in Equation~\eqref{eqn:format-assumption}, one has to stack all $\mathtt{A}_i(q)$ in the same columns, which can be easily done by a single layer of attention via \cref{lem:input-causal} and \citep[Proposition~A.5]{bai2024transformers} (details in \cref{proof:any-var-enc}).
This lemma serves as a foundation for MOIRAI to handle multi-variate time series with in-context learning which we present as the theorem below.
\begin{remark}
        Comparing to Lemma~\ref{lem:input-causal}, \cref{lem:mar-group-wise} is specifically for any-variate attention in our construction, where we demonstrate that several special mechanisms in any-variate attention enables variate-wise operations efficiently.
\end{remark}
\begin{remark}    
        Lemma~\ref{lem:input-causal} and Lemma~\ref{lem:mar-group-wise} can be generalized to Softmax and linear attention by considering perturbations, making them applicable to a wide range of transformers.    
\end{remark}

\begin{theorem}[Any-variate Autoregressive Regression via MOIRAI]\label{thm:any-variate-auto}
    Assume \cref{assumption:effective-regression} holds.
    For any $0 \leq \alpha \leq \beta$ with $\kappa \coloneqq \frac{\beta}{\alpha}$, $B_w > 0$, and $\epsilon < B_x B_w / 2$.
    there exists an $(L_1+L_2)$-layer of MOIRAI transformer equipped with any-variate Attention, satisfies the following
    \begin{align*}
        &
        L_1 = \lceil  \frac{q_{\max}}{3} \rceil  +1, \quad
        L_2 = \lceil 2 \kappa \log \frac{ B_x B_w}{2\epsilon} \rceil, 
        \quad
        \\
        &\quad\quad\quad \quad\quad\max_{\ell \in [L_1+1, L_2]} M^{(\ell)} \leq 3, 
        \\
        &\vertiii{\bm{\theta}} \leq \lvert 4R + 8 \beta^{-1} \rvert,
        \quad \sum_{ \ell=1 }^{L_1} M^{(\ell)} = d_{\max} + q_{\max},
    \end{align*}
    where $d_{\max} > 0$.
    For any input time series $\bx$ with length $T$ generated from an $\mathtt{AR}_d(q)$ process, where
    \begin{align*}
        \bx \in \R^{d \times T},\quad q \leq q_{\max}, \quad d \leq d_{\max}.
    \end{align*}
    Then there exists a MOIRAI transformer with $D \geq (q+1)d_{\max} + T + 2$, satisfies the following
    \begin{equation}
        \lVert
        \hat{\bx}_{T}^1
        -
        \left\langle
        \bw^\star_i, [\bx_{T-1: T-q}^1; \dots ; \bx_{T-1: T-q}^d]
        \right\rangle
        \rVert
        \leq 
        \epsilon,
    \end{equation}
    where $\hat{\bx}_T^1 = \mathtt{read}(\text{TF}^0_{\bm{\theta}}(\bH))$, and $\bH \in \R^{D\times  N}$ is the any-variate encoding of $\bx$.
\end{theorem}
\begin{remark}
    \cref{thm:any-variate-auto} indicates there exists a MOIRAI transformer that fits an autoregressive model on time series as long as the number of covariates no greater than $d_{\max}$ and lags no greater than $q_{\max}$.
    This shows its ability to infer the underlying AR model in a principled way and provides a possible explanation for its zero-shot performance on a wide range of datasets.
\end{remark}

The proof is in \cref{proof:group-wise}.
Observe that there exists two trade-offs in \cref{thm:any-variate-auto}.
First, $q_{\max} d_{\max}$ is upper bounded by the hyperparameter $D$ (up to constant), which is a natural trade-off in our construction.
Second, the approximation error is roughly $O( e^{-L} )$, suppressed exponentially by the number of layers, as in our analysis, each layer of MOIRAI performs a single step of gradient descent on $L_{\text{reg}}$.

Another popular approach of time series prediction is through probabilistic forecasting, where the model estimates the distribution from input data.
In \cref{thm:moirai-mle}, we show that there also exists a MOIRAI that performs Maximum Likelihood Estimation with a small estimation error.

% Observe that with a fixed number of layers and number of heads each layer, there exists a trade-off between the approximation error and the max number of covariates a model handles.
% {\color{red} add more analysis}

% \paragraph{Data Assumption.}
% We assume a $T$ time step time series $(\bx_t, \by_t)_{t=1}^T$, $\bx_t \in \R^d$, $\by_t \in \R$, is generated from the following stochastic process with lookback window size $q \in \N_+$.
% \begin{equation*}
%     \bx_{t+1} = \sum_{i=1}^q \langle \bw_{t-i} , \by_{t-i} \rangle + \epsilon_{t-i},
%     \quad\text{where }
%     \epsilon_t \sim N(0, \sigma_t^2), \quad
%     \bw_t, \in \R^d, \quad \text{ for all }t \in [T].
% \end{equation*}

% \begin{theorem}[Transformers as  Universal Forecaster: Approximation Bound]\label{thm:any-variate-auto}
%     For any $0 \leq \alpha \leq \beta$ with $\kappa \coloneqq \frac{\beta}{\alpha}$, assume the input data is norm bounded by $B_x$.
%     With any input time series $\{(\bx_t, \by_t)\}_{t=1}^T$ following the above assumption,
%     there exists an $(L)$-layer of Transformer satisfies the following
%     \begin{equation*}
%         \lVert f( \bx_{T+1} ) - \by_{T+1} \rVert 
%         =
%         O
%         \left(
%         \frac{\kappa \log B_x}{L-1}
%         \right).
%     \end{equation*}
% \end{theorem}


% \subsection{Generalization Bound}

% \begin{itemize}
%     \item Multi-path + No-sliding window
%     \item Single Path + Sliding Window
%     \item Multi-path + sliding window
% \end{itemize}

% Next we study the generalization bound for XXX.
% Considering learning from X

\section{Generalization}
In this section, we investigate the generalization bound of pretraining transformer-based time series foundation models.
This section will focus on learning MOIRAI on multi-variate time series, one can easily adapt our proofs into learning uni-variate time series with standard transformers.

Let $\pi$ be a meta distribution, and each distribution drawn from it $ \mathtt{P}^{(T)} \sim \pi$, satisfies Dobrushin's condition \cite{Dobrushin1968TheDO} (which we will introduce shortly).
For pretraining data, we first sample $n$ distributions $\mathtt{P}_j^{(T)}$ i.i.d. from $\pi$, and for each distribution, we sample a time series $(\bx_{1j}, \cdots, \bx_{Tj})$, for $j \in[n]$, and each of them contains no more than $d$ covariates and with lag step no more than $q$.


For each time series, we encode it with any-variate encoding into an input matrix denoted as $\bH \in \R^{D \times N}$, \footnote{Due to any-variate encoding, $N = dT$.}
We define each pretraining sample as $\bz_j \coloneqq \left( \bH_j, y_j \right)$, where $y_j = \bx_{Tj}^1$.
We consider the squared loss between model prediction and the label, i.e.
{\small
\begin{equation*}
    \ell( \bz_t, \bm{\theta} )
    \coloneqq
    \frac{1}{2}
    \Bigg[
    y_t
    -
    \mathtt{Clip}_{B_x}
    \bigg(
    \mathtt{read}_y
    \Big(
    \text{TF}_{\bm{\theta}}^R
    \left(
    \bH
    \right)
    \Big)
    \bigg)
    \Bigg]^2,
\end{equation*}
}
where 
$\mathtt{Clip}_{B_x}(t) \coloneqq \max\{ \min \{ t, B_x \}, -B_x \}$, and $\text{TF}_{\bm{\theta}}^R$ is the MOIRAI transformer defined in \cref{def:moirai} with $\mathtt{Clip}(\cdot)$ applied after each layer.
The pretraining loss and test loss is defined as the following:
\begin{equation}\label{eqn:icl-loss}
    \hat{L}( \bm{\theta} )
    \coloneqq
    \frac{1}{nT}
    \sum_{t=1}^T
    \sum_{j=1}^n
    \ell( \bm{\theta}, \bz_{jt} ),
    \;
    L( \bm{\theta} )
    \coloneqq
    \mathbb{E}_{\bz,\mathtt{P}^{(T)}}
    \left[
    \ell(  \bm{\theta}, \bz )
    \right].
\end{equation}
% where the expectation of test loss is taken over $\mathtt{P}_j^{(T)} \sim \pi, \bz \sim \mathtt{P}^{(T)}_j$.
The goal of our pretraining algorithm is to find an empirical risk minimizer (ERM) over MOIRAI transformers with $L$ layers, $M$ heads, and norm bounded by $B$:
\begin{align}\label{eqn:parameter-regime}
    &\hat{\bm{\theta}}
    \coloneqq
    \underset{ \bm{\theta} \in \Theta_{L,M,D^\prime, B} }{\argmin}
    \hat{L}
    (\bm{\theta}),
    \\
    &\Theta_{L,M,D^\prime,B}
    \coloneqq
    \Bigg\{
    \bm{\theta}
    =
    \left(
    \bm{\theta}_1^{(1:L)},
    \bm{\theta}_2^{(1:L)}
    \right)
    :
    \\
    \max_{\ell\in[L]}
    &M^{(\ell)} \leq M
    , 
    \quad
    \max_{\ell\in[L]}
    % \underset{\ell\in[L]}{\max}
    D^{(\ell)} \leq D^\prime
    ,
    \quad
    \Vert\bfa\theta\Vert_{op}
    \leq 
    B
    \Bigg\}.
\end{align}
% \vspace{-1em}
\subsection{Weakly-Dependent Time Series}
In this scenario, we consider the training data $\bx$ to be drawn from a distribution $\mathtt{P}$ satisfying Dobrushin's condition.
Under this condition, we are able to present several generalization bounds on pretraining.


\begin{definition}[Influence in high dimensional distributions]\label{def:influence}
    Let $\cX = (\cX_1, \cdots , \cX_T)$ be a sequence of random variables over $\cD_{\cX}^{T}$.
    The influence of variable $\cX_j$ on variable $\cX_i$ is defined as
    \begin{align*}
        &\bI_{j \rightarrow i}(\cX)
        \coloneqq
        \max_{ \text{x}_{ -i-j}, \text{x}_j, \text{x}_j^\prime  }
        \\
        % \underset{  z_{-i-j} \in Z^{m-2} }{ z_j, z_j^\prime \in Z}}
        &
        \norm{
        P_{\cX_i | \cX_{-i}} \left( \cdot | \text{x}_{-i-j}, \text{x}_j \right),
        P_{\cX_i | \cX_{-i}} \left( \cdot | \text{x}_{-i-j}, \text{x}_j^\prime \right)
        }_{\texttt{TV}}
        ,
    \end{align*}
    where $\text{x}_{-i-j} \in \cD_{\cX}^{T-2}, \text{x}_j, \text{x}_j^\prime \in \cD_{\cX}$, $\norm{\cdot}_{\texttt{TV}}$ denotes the total variation distance, and $\text{x}_{-i}$ represents the vector \textbf{x} after omitting the $i$-th element.
\end{definition}



\begin{definition}[Dobrushin's Uniqueness Condition]
    Consider a random variable $\cX$ over $\cD_{\cX}^T$.
    The Dobrushin coefficient of $\cX$ is defined as 
    \[
    \alpha(\cX) \coloneqq  \max_{1\leq i \leq T} \sum_{j \neq i} \bI_{j \rightarrow i} (\cX).
    \]
    We say the variable satisfies Dobrushin's uniqueness condition if $\alpha(\cX) < 1$.
    For a distribution $\mathtt{P}$, we denote $\alpha(\mathtt{P}) = \sup_{\cX\sim\mathtt{P}} \alpha(\cX)$.
\end{definition}
\begin{definition}[Log Dobrushin's Coefficients]
        Let $\cX = (\cX_1, \cdots, \cX_T)$ be a random variable over $\cD_{\cX}^T$ and let $\mathtt{P}_z$ denote its density.
        Assume that $\mathtt{P}_{z} > 0$ on all $\Omega^T$.
        For any $i \neq j \in [T]$, the log influence between $j$ and $i$ is defined as:
    {\small
    \begin{equation*}
        I^{\log}_{j, i}(\cX)
        =
        \frac{1}{4}
        \sup
        \log
        \frac{ P
        \left[
        \text{x}_i, \text{x}_j, \text{x}_{-i-j}
        \right] 
        P
        \left[
        \text{x}_i^\prime, \text{x}_j^\prime, \text{x}_{-i-j}
        \right]
        }{
    P
        \left[
        \text{x}_i^\prime, \text{x}_j, \text{x}_{-i-j}
        \right] 
    P
        \left[
        \text{x}_i, \text{x}_j^\prime, \text{x}_{-i-j}
        \right] 
        },
    \end{equation*}
    }
    where the $\sup$ is taken over $\text{x}_{-i-j}, \text{x}_i, \text{x}_i^\prime, \text{x}_j, \text{x}_j^\prime$,
    and the log-coefficient of $\cX$ is defined as $\alpha_{\log}(\cX) = \max_{i \in [T]} \sum_{j \neq i} I^{\log}_{j, i}(\cX)$.
\end{definition}
The coefficient $\alpha(\cdot)$ has a natural bound $0 \leq \alpha(\cdot) \leq T-1$, with $\alpha = 0$, the data reduces to the i.i.d. case.

\begin{remark}
    Dobrushin's condition characterizes a class of distributions whose dependency is mild. 
    However, our empirical evaluation suggests that in certain situations where Dobrushin's condition fails to hold, the Transformers can perform prediction well.
\end{remark}
\subsection{Generalization Bounds of MOIRAI}

\begin{theorem}[Pretraining Generalization Bound]\label{thm:gen-bound-1}
    Let $\Theta_{L,M,D^\prime, B}$ be the parameter space defined in 
    Equation~\ref{eqn:parameter-regime}.
    Assume $\alpha_{\log}( \mathtt{P}^{(T)}) < 1/2$.
    Then with probability at least $1 - \varepsilon$, ERM $\hat{\bm{\theta}}$ satisfies the following:
    \begin{align*}
    L(\hat{\bm{\theta}})
    &\leq 
    \inf_{\bm{\theta} \in \Theta_{L,M,D^\prime, B}} L(\bm{\theta})
    +
    \\
    &
    O
    \left(
    \frac{B_x^2}{1 - \alpha(\mathtt{P}^{(T)}) }
    \sqrt{
    \frac{
    L(MD^2 + D D^\prime) \zeta + \log(\nicefrac{1}{\varepsilon})
    }{n}
    }
    \right),
    \end{align*}
    % where $\iota = \log( 2 + 2(L B_H^{L-1} B_{\Theta}) B \frac{1-\alpha}{B_x})$
    where $C$ is an universal constant, and $\zeta = O(\log(2 + \max \{ B, \mathtt{R}, B_x, T, d \}$. 
\end{theorem}

The proof is in \cref{proof:gen-bound-1}.
Note that when $\alpha(\mathtt{P}) = 0$, the data becomes i.i.d., where the only difference between our generalization and one proposed in \cite{bai2024transformers} is the complexity term. 
The complexity of MOIRAI and standard transformers differs as the complexity of MOIRAI also dependents on the time series length ($T$).
Further, in \cref{thm:gen-bound-1}, we do not assume our data is generated from the $\mathtt{AR}$ process, only its Dobrushin coefficient.
When the data is generated by the $\mathtt{AR}$ process, we are able to give a more explicit bound on the same test loss as described below.

\begin{corollary}[Test Error Bound]\label{thm:test-error-bound-1}
    Following the setup in \cref{thm:gen-bound-1},
    if pretraining samples are generated by some $\mathtt{AR}_d(q)$ process with noise sampled from $N(0, \sigma^2_\epsilon)$\footnote{Here we assume fixed $d, q$ across all samples as one can describe a lower dimension/order $\mathtt{AR}$ process with zero coefficients.}, 
    then with probability $ \Delta(1 - \varepsilon)$, ERM $\hat{\bm{\theta}}$ satisfies the following:
    \begin{align*}
    L(\hat{\bm{\theta}})
    &
    \leq
    O
    \Bigg(
    B_x B_w \exp \left( \frac{-L}{\kappa} \right)
    +
    \\
    &
    \frac{B_x^2}{1 - \alpha(\mathtt{P}^{(T)})}
    \sqrt{ \frac{L(MD^2 + D D^\prime ) \zeta + \log (1 / \varepsilon)}{n} }
    \Bigg).
    \end{align*}
    where $\Delta = O\left(1 - \left(  \nicefrac{\sigma_\epsilon}{B_x  B_w e^{\nicefrac{-L}{2\kappa}}}  \right)^2  \right)$, $C$ is an universal constant, and $\zeta = O(\log(2 + \max \{ B, \mathtt{R}, B_x, T, d \})$. 
\end{corollary}
\begin{remark}
Considering the model parameters ($M,D,D^\prime, d$) are of constant level, one is able to further optimize the bound to
$L(\hat{\bm{\theta}}) \lesssim  n^{-\nicefrac{1}{2}}$, by selecting $L$ appropriately.    
\end{remark}

% \begin{theorem}[Multi-Path Generalization Bound]
%     Let $L_{\mathtt{T}}( \bm{\theta})$, $\hat{L}_{\mathtt{T}})\bm{\theta})$ be the test and empirical loss over samples drawn from a target distribution $\mathtt{T}$.
%     Let $\lambda_k = \inf L_k( \bm{\theta})$, where the $\inf$ is taken over all $\bm{\theta} \in \Theta_{L,M,D^\prime, B}$.
%     Following the constructions above, we have
%     \begin{equation}
%         L_{\mathtt{T}}(\hat{\bm{\theta}})
%         \leq 
%         \frac{1}{K}
%         \sum_{k=1}^K
%         \left(
%         \hat{L}_{\mathtt{T}}( \hat{\bm{\theta}})
%         +
%         C_1 \cdot
%         \hat{\mathfrak{G}}
%         +   
%         \lambda_K
%         \right)
%         +
%         \cdots,
%     \end{equation}
%     where $\lambda_K$ is the average risk of the optimal hypothesis on the combined source and target domain,  $C_1, C_2$ are universal constants whenever $\nicefrac{1}{2} - \alpha_{\log}(D_S^k)$ is bounded away from zero, and $\hat{\mathfrak{G}}$ is the empirical Gaussian complexity.
% \end{theorem}


\subsection{Example: Stationary $\mathtt{AR}(1)$}\label{sec:AR1}
Here we provide an example of the application of \cref{thm:test-error-bound-1} on $\mathtt{AR}(1)$ process with the following form
\begin{equation*}
    \bx_{t+1}
    =
    \langle \bw, \bx_{t} \rangle + \epsilon_t, \quad \epsilon \sim N(0, \sigma_{\epsilon}^2),
\end{equation*}
where $\bx_t \in \R^d$, $\bw \in \R^d, \epsilon \in \R$ and $\by_{t+1} = \bx_{t+1}^1$.

To satisfy the condition of $\alpha( \mathtt{P} ) < \frac{1}{2}$, we assume the following holds
\begin{equation}\label{eqn:condition-weakly-dependent-AR1}
    B_x^2 < \ln \frac{1}{2} + ( \sigma_{\epsilon}^2 ),
    \quad  \norm{\bw}_{\infty} < 1.
\end{equation}
The first condition comes from the fact that we require the pair-wise potential of this time series to be less than $1/2$ (For more details, see \cref{appendix:analysis-ar1}).
The second condition comes from the requirement of it being stationary.
\begin{proposition}[Generalization Bound for Any-Variate Transformer on $\mathtt{AR}(1)$]\label{proposition:ar1}
    Considering an $\mathtt{AR}(1)$ process with Dobrushin's coefficient bounded by $1/2$.
    With probability at least $\delta(1 - \varepsilon)$, ERM $\hat{\bm{\theta}}$ satisfies the following:
    \begin{align*}
    L(\hat{\bm{\theta}})
    &=
    O
    \Bigg(
    \frac{\sigma_\epsilon}{\sqrt{1 - \delta}}
    +
    \frac{\sigma_\epsilon^2}{B_x}
    \exp \left( \frac{-L}{\kappa} \right)
    +
    \\
    &
    \frac{\sigma_\epsilon^2}{1 - \alpha( \mathtt{AR}(1) )}
    \sqrt{ \frac{L(MD^2 + D D^\prime) \zeta + \log (1 / \varepsilon)}{n} }
    \Bigg).
    \end{align*}
    where $\zeta = O(\log(2 + \max \{ B, \mathtt{R}, B_x, d \})$. 
\end{proposition}
If we further optimize the bound by viewing the hyperparameters as constants, the test error obeys $O(e^{-L} + \sqrt{\frac{L}{n}})$ with high probability whenever $\sigma_\epsilon$ is small.

\section{Theoretical Results}\label{sect3}
This section presents our theoretical results and the idea of taking each step in the proof. Our proof constructs a particular instance of the transformers and shows that the forward propagation on our constructed instance approximates the Power Method. We also carefully design the contextual matrix $\bfa P$, explained as follows.

\paragraph{The Design of Auxillary Matrix.} Our design of the matrix $\bfa P$ consists of three parts:
\begin{enumerate}
    \item \emph{Place Holder.} For $\ell\in\{1\}\cup[4:k+3]$ and $i\in[N]$, we let $\tda p_{\ell, i}=\bfa 0\in\bb R^{d\times 1}$. The place holders in $\bfa P$ record the intermediate results in the forward propagation.
    \item \emph{Identity Matrix.} We let $ \begin{bmatrix}
        \tda p_{2,1}&\ldots&\tda p_{2,N}
    \end{bmatrix}=\begin{bmatrix}
        \bfa I_d&\bfa 0_{d\times(N-d)}
    \end{bmatrix}$. The identity matrix in $\bfa P$ helps us screen out all the covariates $\bfa X$ in the forward propagation.
    \item \emph{Random Samples on the Hypersphere.} We let $\tda p_{3,1},\ldots\tda p_{3,k}$ be the i.i.d. samples uniformly distributed on $\bb S^{d-1}$. The random samples on the sphere correspond to the initial vectors $\bfa v_{0,\ell}$ for $\ell\in[k]$ in algorithm \ref{alg:almoexactrecov}.
\end{enumerate}
Given the above construction on the auxiliary matrix $\bfa P$, we are ready to state the existence theorem in this work, given as follows.

\begin{algorithm}[htbp] 
 \caption{Spectral Clustering}
\label{alg:spectralclustering}
\KwData{$\bfa X\in\bb R^{D\times N}$ with $\bfa X_i\in\bb R^D$ for all $i\in[N]$}
 Compute the SVD of the data matrix $\bfa X=UDV^\top$. Let $U_k$ be the first $k$ columns of $U$\;

 Project $\bfa x_1,\ldots,\bfa x_N$ onto $U_k$, i.e. let $\wha x_i\gets U_kU_k^\top \bfa x_i$ \;
 Run an \emph{Transformer k-means++\cite{arthur2006k}} algorithm on the columns of projected matrix $\wha X=\begin{bmatrix}
     \wha x_1,\ldots,\wha x_N
 \end{bmatrix}$\;
\end{algorithm}
\begin{assumption}\label{asumpt1}
    We assume that $\bfa X_1,\ldots,\bfa X_N$ are i.i.d. samples from $\sum_{\ell=1}^k\pi_\ell\bb P(X|\mu)$. Minimal separation is denoted by $\Delta:=\inf_{i\neq j}\Vert\bfa\mu_i-\bfa\mu_j\Vert_2$.
\end{assumption}

And the kmeans++ algorithm is given by the following procedure.
\begin{algorithm}[htbp] 
 \caption{k-means++}
\label{alg:kmeans++}
\KwData{$\bfa X\in\bb R^{k\times N}$ with $\bfa X_i\in\bb R^k$ for all $i\in[N]$}
Choose an initial center $\bfa c_1$ uniformly at random from $\ca X:=\{\bfa X_1,\ldots,\bfa X_N\}$ and let $\ca C=\{\bfa c_1\}$\;
\While{$|\ca C|<k$}{Choose the next center $\bfa c$, selecting $\bfa c=\bfa x\in\ca X$ with probability $\frac{D(\bfa x)}{\sum_{x\in\ca X}D(\bfa x)}$\\
where $D(\bfa x):=\argmin_{\bfa c\in\ca C}\Vert \bfa c-\bfa x\Vert_2$\;
If $\bfa c\notin\ca C$, let $\ca C\gets\ca C\cup\{\bfa c\}$\;
}
Run the Lloyd's algorithm with covariates $\bfa X$ and initial centroids $\ca C$\;
\end{algorithm}
\begin{algorithm}[htbp] 
 \caption{Lloyd's Algorithm}
\label{alg:lloyd}
\KwData{$\bfa X\in\bb R^{k\times N}$ and initial clusters $\ca C=\lef\{\wha\mu_\ell^{(0)}\rig\}_{\ell\in[k]}$}
Let $t\gets 1$\;
Initialize with the starting membership $\wh z_i^{(0)}\gets\argmin_{\ell\in[k]}\Vert \bfa x_i-\wha\mu_\ell^{(0)}\Vert_{\Sigma^{-1}}$\;
\While{$\exists j\in[n]$ such that $\wh z^{(t+1)}_j\neq \wh z^{(t)}_j$}{Update the centroids and memberships through $\wha\mu_{\ell}^{(t)}\gets\frac{\sum_{i=1}^n\bfa x_{i}\mbbm 1_{\wh z_i^{(t-1)}=\ell}}{\sum_{i=1}^n\mbbm 1_{\wh z_i^{(t-1)}=\ell}},\enspace\forall \ell\in[k],\qquad \wh z_i^{(t)}\gets\argmin_{\ell\in[k]}\Vert \bfa x_i-\wha\mu_{\ell}^{(t)}\Vert^2_{\Sigma^{-1}},\enspace\forall i\in[n].$
}
Run the Lloyd's algorithm with $\bfa X$ and $\ca C$\;
\end{algorithm}

In particular, take as input the context given by $\bfa H=\begin{bmatrix}
    \bfa x_1,\ldots,\bfa x_N\\
    \bfa p_1,\ldots,\bfa p_N
\end{bmatrix}\in\bb R^{D\times N}$, we show that the Transformer network can approximate the spectral clustering algorithm in the initialization phase, given by \ref{alg:spectralclustering}.
\subsection{The Power Iteration Method}
In the first step of the power method 
we use the Transformer model to approximate the singular vectors of the matrix $\bfa X^\top\bfa X$. A typical implementation to achieve this is given by the power method.


\begin{theorem}[Transformer Approximation of the Power Iteration]\label{thm3.1} Denote the eigenvalues of $\bfa X\bfa X^\top$ to be $\lambda_1>\lambda_2>\ldots>\lambda_k>\ldots$. Let $\Delta:=\min_{1\leq i<j\leq k}|\lambda_i-\lambda_j|$. Assume that the eigenvalues of $\bfa X$ satisfy $\Vert\bfa X\Vert_2\leq B_X$. Assume that the initialized vectors $\tda p_{3,1},\ldots\tda p_{3,N}$ 
    satisfy  $\tda p_{3,i}^\top\bfa v_{i}\geq\delta$ for all $i\in[k]$ and make the rest of the vectors $\bfa 0$. 
    Then, there exists a transformer model with number of layers $L=2\tau+4k+1$ and number of heads $M\leq \lambda_1^d\frac{C}{\epsilon^2}$ with $\tau\leq\frac{\log(1/\epsilon_0\delta)}{\epsilon_0}$ such that for all $\epsilon_0,\epsilon>0$,
    the final output $\wha v_1,\ldots,\wha v_{k}$ given by the transformer model achieve
    \begin{align*}
        \lef\Vert\wha v_{\eta+1}-\bfa v_{\eta+1}\rig\Vert_2\leq C\tau \epsilon\lambda_1^2+\frac{C\lambda_1\sqrt{\epsilon_0}}{\Delta}\prod_{i=1}^{\eta}\frac{5\lambda_{i+1}}{\Delta}.
    \end{align*} 
    Moreover, consider the accuracy of multiple $\bfa v$s as a whole. There exists $\bfa\theta$ such that 
    \begin{align*}
        L\lef(TF_{\wha\theta}(\bfa H),\bfa V\rig)\leq C\tau\epsilon k\lambda_1^2+C\bl\frac{\epsilon_0\lambda_1^2}{\Delta^2}\sum_{\eta=1}^{k-1}\prod_{i=1}^{\eta}\frac{25\lambda_{i+1}^2}{\Delta^2}\br^{1/2}.
    \end{align*}
\end{theorem}
\begin{remark}
\label{remark3}
    The approximation error consists of two terms. The first term comes from the approximation of the Power Method iterations by transformers. The second term comes from the error caused by finite iteration $\tau$. To acquire a more direct account of the error terms and its order of magnitude, we consider a special case where the eigenvalues $\lambda_1\asymp\lambda_2\asymp\ldots\asymp\lambda_k\asymp\Delta$. Then our results boil down to 
    \begin{align*}
        \lef\Vert TF_{\bfa\theta}(\bfa H)-\begin{bmatrix}
            \bfa v_1^\top,\bfa v_2^\top,\ldots,\bfa v_k^\top
        \end{bmatrix}^\top\rig\Vert_2\leq C\tau\epsilon k \lambda_1^2+ C\frac{\lambda_1}{\Delta}\sqrt{k\epsilon_0}.
    \end{align*}
    These results hide dimension $d$ in the universal constant. Hence the dimension significantly affects the approximation properties of transformers. Our experimental results in section \ref{sect4} also indicate that learning high dimensional principal eigenvectors is challenging.
\end{remark}
We show that the conditions on $\tda p_{3,1},\ldots,\tda p_{3,N}$ can be achieved through sampling from isotropic Gaussians, given by the following lemma.
    \begin{lemma}\label{lm3.1.1}\label{lm3.1}
        Let $\bfa y\in\bb R^d$ be a random vector with isotropic Gaussian as its probability density. Consider $\bfa x=\frac{\bfa y}{\Vert\bfa y\Vert_2}$. Let $\bfa v$ be any unit length vector, then we have for all $\delta<\frac{1}{2}d^{-1}$,
           $ \bb P\lef(|\bfa v^\top\bfa x|\leq\delta\rig)\leq  \frac{1}{\sqrt{\pi}}\sqrt\delta+\exp\lef(-C\delta^{-\frac{1}{2}}\rig)$.
        Therefore, for all $\delta<\frac{1}{2}d^{-1}$, the event in theorem \ref{thm3.1} is achieved with
        \begin{align*}
             \bb P\bl\exists i\in[k]\text{ such that }&\bfa x_i^\top\bfa v_i\leq\frac{\delta}{\sqrt d}\br\leq \frac{k\sqrt{\delta}}{\sqrt{\pi}}+k\exp(-C\delta^{-1}).
        \end{align*}
    \end{lemma}
Given the approximation error provided by theorem \ref{thm3.1}, we further provide the generalization error bound for the ERM defined by \eqref{ERM}. This requires us to consider the following regularity conditions on the underlying distribution of $\bfa X\bfa X^\top$ (which also translates to the distribution of $\bfa X$).
\begin{assumption}\label{assump1}
    The distribution of $\bfa X\bfa X^\top$ supports on $$\bb X:=\lef\{A:A\in\bfa S^d_{++},  B_X\geq\lambda_1(A)>\lambda_2(A)>\ldots>\lambda_k(A),\inf_{1\leq i<j\leq k}\lambda_i(A)-\lambda_j(A)\geq\Delta\rig\}.$$
\end{assumption}
\begin{remark}
    The above assumption can be easily generalized to distribution that supports on $\bb X$ with high probability. Examples of such distribution include the Wishart distribution under the Gaussian design. In this work, we stick to the simplest case where the maximum eigenvalue is bounded from above.
\end{remark}
Given the above assumption, we are ready to state the generalization bound.
\begin{proposition}\label{genbound}
   With probability at least $1-\xi$, the ERM solution $\wha\theta$ satisfies
    \begin{align*}
        \bb E\lef[L\lef(TF_{\wha\theta}(\bfa H),\bfa V\rig)|\wha\theta\rig]&\leq\inf_{\bfa\theta\in\Theta(B_{\bfa\theta},B_M)}\bb E\lef[L\lef(TF_{\bfa\theta}(\bfa H),\bfa V\rig)\rig]\\
        &+ C \sqrt{\frac{k^3LB_Md^2\log(B_{\theta}+B_X+k)+\log(1/\xi)}{n}}.
    \end{align*}
\end{proposition}
   Together with the bound given by theorem \ref{thm3.1} and lemma \ref{lm3.1.1}, which essentially give a high probability upper bound on $\inf_{\bfa\theta\in\Theta(B_{\bfa\theta},B_M)}$ we can derive a general upper bound on the generalization error, given as follows.
    \begin{corollary}
        Under assumption \ref{assump1}, with probability at least $1-\xi-\frac{k\sqrt\delta}{\sqrt\pi}-k\exp\lef(-C\delta^{-1/2}\rig)$ for all $\delta<d^{-1}$ we have for all $\epsilon,\epsilon_0>0$,
        \begin{align*}
            \bb E\lef[L\lef(TF_{\wha\theta}(\bfa H),\bfa V\rig)|\wha\theta\rig]&\leq \bb E\lef[C\tau\epsilon k\lambda_1^2+C\bl\frac{\epsilon_0\lambda_1^2}{\Delta^2}\sum_{\eta=1}^{k-1}\prod_{i=1}^{\eta}\frac{25\lambda_{i+1}^2}{\Delta^2}\br^{1/2}\rig]\\
            &+C\sqrt{\frac{k^3\log(\delta/
            \epsilon_0)\lambda_1^d d^2\log(B_{\theta}+B_X+k)+\log(1/\xi)}{n\epsilon_0\epsilon^2}}.
        \end{align*}
    \end{corollary}
    \begin{remark}
        If we consider optimizing the bound w.r.t. $\epsilon_0$ and $\epsilon$, we obtain that $\bb E\lef[L(TF_{\wha\theta}(\bfa H),\bfa V)|\wha\theta\rig]\lesssim n^{-1/5}$ given that the rest of the parameters are of constant scales. It is not known if the results are improvable or not and the authors believe this question worth future explorations.
    \end{remark}
The next theorem demonstrates that there exists a Transformer that simulates the expectation maximization algorithm on Gaussian mixture model.
\begin{theorem}
    
\end{theorem}
\begin{proof}
We first consider the input matrix to be 
\begin{align*}
    \bfa H_1:=\bfa H=\begin{bmatrix}
        \bfa x_1&\bfa x_2&\ldots&\bfa x_n\\
        \wha \mu_{\wh z^{(0)}(1)}^{(0)}&\wha \mu_{\wh z^{(0)}(2)}^{(0)}&\ldots&\wha \mu_{\wh z^{(0)}(n)}^{(0)}\\
        \bfa p_{1,1}&\bfa p_{1,2}&\ldots&\bfa p_{1,N}\\
        \bfa p_{2,1}&\bfa p_{2,2}&\ldots&\bfa p_{2,N}\\
        1&1&\ldots&1\\
        \bfa p_{3,1}&\bfa p_{3,2}&\ldots&\bfa p_{3,N}
    \end{bmatrix}\in\bb R^{D\times N}, 
\end{align*}
where $\wh z^{(0)}:[n]\to[k]$ is the assignment function, $\wha\mu_i\in\bb R^{d}$ is the initially estimated centroid for the $i$-th cluster. $\bfa p_{1,i}\in\bb R^k$ satisfies $\bfa p_{1,i,j}=\mbbm 1_{\wh z^{(0)}(i)=j}$ for all $j\in[k]$. And for $\bfa p_{2,i}$ we have $\bfa p_{2,i,j} = \mbbm 1_{j=i}$ for $i\leq d$ and $\bfa p_{2,i,j}=0$ for $i\leq N$ and $j\leq d$. We let $\bfa p_{3,1}=\bfa p_{3,2}=\ldots=\bfa p_{3,N} = \bfa 0\in\bb R^k$.
We note that algorithm \ref{alg:lloyd} consists of two iterative steps: (1) The expectation step where we take the averages to get an initial estimate $\wha \mu_{\ell}^{(t)}$. (2) The maximization step where we assign each individual their labels. Our following discussions treat the two steps separately. 
\begin{center}
    \textbf{1. The Expectation Step.}
\end{center}
To achieve the first step, we construct our transformer weights as follows:
\begin{align*}
    \bfa V_1^{(1)}=\begin{bmatrix}
        \tda V_{1,1}^{(1)}&\tda V_{1,2}^{(1)}&\tda V_{1,3}^{(1)}
    \end{bmatrix},\quad\bfa Q_1^{(1)}=\begin{bmatrix}
        \bfa 0_{1\times(3d+k)}&1&\bfa 0\\
        \bfa 0&\bfa 0&\bfa 0
    \end{bmatrix},\quad\bfa K_1^{(1)}=\begin{bmatrix}
        \bfa 0_{1\times(3d+k)}&1&\bfa 0\\
        \bfa 0&\bfa 0&\bfa 0
    \end{bmatrix},
\end{align*}
where $\tda V_{1,1}^{(1)}\in\bb R^{2d\times D}=\bfa 0$, $\tda V_{1,2}^{(1)}=\begin{bmatrix}
        \bfa 0_{3d+k}\\
        I_k\\
        \bfa 0
    \end{bmatrix}\in\bb R^{k\times D}$. Then we can show that
\begin{align*}
   (\bfa K_1^{(1)}\bfa H_1)^\top=(\bfa Q_1^{(1)}\bfa H)^\top = \begin{bmatrix}
        1&\bfa 0\\
        \vdots &\bfa 0\\
        1&\bfa 0
    \end{bmatrix}.
\end{align*}
Then we can show that
\begin{align*}
   (\bfa Q_1^{(1)}\bfa H)^\top(\bfa K_1^{(1)}\bfa H) = \begin{bmatrix}
       \bfa v_1&\ldots&\bfa v_1
   \end{bmatrix},\quad \bfa v_{1,i} = 1\quad\forall i\in[N].
\end{align*}
Hence, we can obtain that
\begin{align*}
    \bfa V_1^{(1)}\bfa H&\times\sigma((\bfa Q_1^{(1)}\bfa H)^\top(\bfa K_1^{(1)}\bfa H))=\bfa V_1\bfa H\times\begin{bmatrix}
        \bfa v_1 &\ldots&\bfa v_1
    \end{bmatrix}=\bfa V_1\times \begin{bmatrix}
        &\bfa A_0&\\
        \bfa v_k &\ldots&\bfa v_k\\
        &\bfa A_1&
    \end{bmatrix}\\
    &=\begin{bmatrix}
        \tda V_{1,1}^{(1)}&\tda V_{1,2}^{(1)}&\tda V_{1,3}^{(1)}
    \end{bmatrix}\begin{bmatrix}
        &\bfa A_0&\\
        \bfa v_k&\ldots&\bfa v_k\\
        &\bfa A_1&
    \end{bmatrix}=\tda V_{1,2}^{(1)}\begin{bmatrix}
        \bfa v_k &\ldots&\bfa v_k
    \end{bmatrix}\\
    &=\begin{bmatrix}
        \bfa 0_{3d+k}\\
        I_k\\
        \bfa 0
    \end{bmatrix}\begin{bmatrix}
        \bfa v_k&\ldots&\bfa v_k
    \end{bmatrix}=\begin{bmatrix}
        &\bfa 0_{3d+k}&\\
        \bfa v_k&\ldots&\bfa v_k\\
        &\bfa 0&
    \end{bmatrix},
\end{align*}
 where $\bfa A_0\in\bb R^{2d\times N}$ and $
    \bfa v_{k,\ell}=\sum_{i=1}^N\mbbm 1_{\wh z_i^{(0)}=\ell}$.
Then it is checked that
\begin{align*}
   \bfa H_2 =  \bfa H_1+ \bfa V_1^{(1)}\bfa H_1\times\sigma\lef((\bfa Q_1^{(1)}\bfa H_1)^\top(\bfa K_1^{(1)}\bfa H_1)\rig)=\begin{bmatrix}
        \bfa x_1&\bfa x_2&\ldots&\bfa x_N\\
        \wha \mu_{\wh z^{(0)}(1)}^{(0)}&\wha \mu_{\wh z^{(0)}(2)}^{(0)}&\ldots&\wha \mu_{\wh z^{(0)}(N)}^{(0)}\\
        \bfa p_{1,1}&\bfa p_{1,2}&\ldots&\bfa p_{1,N}\\
        \bfa p_{2,1}&\bfa p_{2,2}&\ldots&\bfa p_{2,N}\\
        1&1&\ldots&1\\
        \bfa v_k&\bfa v_k&\ldots&\bfa v_k\\
        \bfa p_{4,1}& \bfa p_{4,2}&\ldots &\bfa p_{4,N}
    \end{bmatrix}.
\end{align*}
Therefore, we further construct the following multi-head layer to remove the off-diagonal elements in $\begin{bmatrix}
    \bfa v_k&\bfa v_k&\ldots&\bfa v_k
\end{bmatrix}$, given by 
\begin{align*}
    \bfa V_i^{(2)} = \begin{bmatrix}
        &\bfa 0_{(3d+2k+i)\times D}&\\
        \bfa 0_{1\times(3d+2k+i)}&1&\bfa 0\\
        &\bfa 0&
    \end{bmatrix},\quad \bfa Q_1^{(2)} = \begin{bmatrix}
        b
    \end{bmatrix},\quad\bfa K_1^{(2)} = \begin{bmatrix}
        c
    \end{bmatrix},\qquad\text{ for }i\in[k].
\end{align*}
Given this formulation, we can show that
\begin{align*}
    \sigma((\bfa Q_i^{(2)}\bfa H_2)^\top(\bfa K_i^{(2)}\bfa H_2)) = \begin{bmatrix}
        &\bfa 0_{(i-1)\times N}&\\
        \bfa 0_{1\times(i-1)}&1&\bfa 0\\
        &\bfa 0&
    \end{bmatrix}.
\end{align*}
Hence, we can further show that
\begin{align*}
    \bfa V_{i}^{(2)}\bfa H_2\sigma((\bfa Q_i^{(2)}\bfa H_2)^\top(\bfa K_i^{(2)}\bfa H_2))=\begin{bmatrix}
        &\bfa 0_{(3d+2k+i)\times D}&\\
        \bfa 0_{(i-1)}&\bfa v_{k,i}&\bfa 0\\
        &\bfa 0&
    \end{bmatrix},
\end{align*}
which immediately implies that
\begin{align*}
    \sum_{i=1}^k\bfa V_i^{(2)}\bfa H_2\sigma\lef((\bfa Q_i^{(2)}\bfa H_2)^\top(\bfa K_i^{(2)}\bfa H_2)\rig)=\begin{bmatrix}
        \bfa 0_{(3d+2k)\times N}&\\
        \diag(\bfa v_k)&\bfa 0\\
        \bfa 0
    \end{bmatrix}.
\end{align*}
Given the above design, we can show that
\begin{align*}
    \bfa H_{3,1}=\bfa H_2+\sum_{i=1}^k\bfa V_i^{(2)}\bfa H_2\sigma\lef((\bfa Q_i^{(2)}\bfa H_2)^\top(\bfa K_i^{(2)}\bfa H_2)\rig)=\begin{bmatrix}
        \bfa x_1&\bfa x_2&\ldots&\bfa x_N\\
        \wha \mu_{\wh z^{(0)}(1)}^{(0)}&\wha \mu_{\wh z^{(0)}(2)}^{(0)}&\ldots&\wha \mu_{\wh z^{(0)}(N)}^{(0)}\\
        \bfa p_{1,1}&\bfa p_{1,2}&\ldots&\bfa p_{1,N}\\
        \bfa p_{2,1}&\bfa p_{2,2}&\ldots&\bfa p_{2,N}\\
        1&1&\ldots&1\\
        \bfa v_k&\bfa v_k&\ldots&\bfa v_k\\
        \diag(\bfa v_k)&&\bfa 0&\\
        &\bfa 0&&
    \end{bmatrix}.
\end{align*}
Then, we construct the MLP layer to remove the $\bfa v_k$ part, which is designed by
\begin{align*}
    \bfa W_1^{(2)}=I_D,\quad \bfa W_2^{(2)}= \begin{bmatrix}
        &\bfa 0_{(3d+k)\times D}&&\\
        \bfa 0_{k\times (3d+k)}&-I_k&I_k&\bfa 0_{}\\
       \bfa 0 &-I_k&\bfa 0&\bfa 0\\
       &\bfa 0&&
    \end{bmatrix}.
\end{align*}
Given this formulation, we can show that
\begin{align*}
    \bfa H_3 := \bfa H_{3,1}+\bfa W_1^{(2)}\sigma\lef(\bfa W_2^{(2)}\bfa H_{3,1}\rig)=\begin{bmatrix}
        \bfa x_1&\bfa x_2&\ldots&\bfa x_N\\
        \wha \mu_{\wh z^{(0)}(1)}^{(0)}&\wha \mu_{\wh z^{(0)}(2)}^{(0)}&\ldots&\wha \mu_{\wh z^{(0)}(N)}^{(0)}\\
        \bfa p_{1,1}&\bfa p_{1,2}&\ldots&\bfa p_{1,N}\\
        \bfa p_{2,1}&\bfa p_{2,2}&\ldots&\bfa p_{2,N}\\
        1&1&\ldots&1\\
        \bfa v_k&\bfa v_k&\ldots&\bfa v_k\\
        \diag(\bfa v_k)&&\bfa 0&\\
        &\bfa 0&&
    \end{bmatrix}.
\end{align*}

% Then we consider the following attention head which essentially remove the $\bfa v_k$ part in $\bfa H_2$. We note that the rank of $\sigma((\bfa Q_i\bfa H_2)^\top(\bfa K_i\bfa H_2))$ is at most $D$. To form the 
% \begin{align*}
%     \bfa V_{k+1}^{(2)} = \begin{bmatrix}
%         &\bfa 0_{(3d+k)\times D}&\\
%         \bfa 0_{k\times (3d+k)}&-I_k&\bfa 0_{}\\
%         &\bfa 0&
%     \end{bmatrix},\quad \bfa Q_{k+1}^{(2)}=,\quad \bfa K_{k+1}^{(2)}=
% \end{align*}
% Given the above design, we can show that
% \begin{align*}
%     \bfa V_{k+1}^{(2)}\bfa H_2\sigma\lef((\bfa Q_{k+1}^{(2)}\bfa H_2)^\top(\bfa K_i^{(2)}\bfa H_2)\rig)=\begin{bmatrix}
%         &\bfa 0_{(3d+k)\times N}&\\
%         -\bfa v_k&\ldots&-\bfa v_k\\
%         &\bfa 0&
%     \end{bmatrix}.
% \end{align*}
% Therefore, collecting pieces, we can show that
% \begin{align*}
%    \bfa H_3:= \bfa H_2 + \sum_{i=1}^{k+1}\bfa V_i^{(2)}\bfa H_2\sigma\lef((\bfa Q_i^{(2)}\bfa H_2)^\top(\bfa K_i^{(2)}\bfa H_2)\rig)=\begin{bmatrix}
%         \bfa x_1&\bfa x_2&\ldots&\bfa x_N\\
%         \wha \mu_{\wh z^{(0)}(1)}^{(0)}&\wha \mu_{\wh z^{(0)}(2)}^{(0)}&\ldots&\wha \mu_{\wh z^{(0)}(N)}^{(0)}\\
%         \bfa p_{1,1}&\bfa p_{1,2}&\ldots&\bfa p_{1,N}\\
%         \bfa p_{2,1}&\bfa p_{2,2}&\ldots&\bfa p_{2,N}\\
%         1&1&\ldots&1\\
%         \diag(\bfa v_k)&&\bfa 0&\\
%         \bfa p_{4,1}& \bfa p_{4,2}&\ldots &\bfa p_{4,N}
%     \end{bmatrix}.
% \end{align*}
The following layer converts the term $\diag(\bfa v_k)$ to $\diag(\bfa v_k^\prime)$ where $\bfa v_{k,i}^\prime = 1/\bfa v_{k,i}$. The design is given as follows
\begin{align*}
    \bfa V_{i}^{(3)} &= \begin{bmatrix}
        &\bfa 0_{(3d+3k+1)\times D}&\\
        \bfa 0_{k\times(2d+2k)}&\diag(c_i)_{k\times k}&\bfa 0\\
        &\bfa 0&
    \end{bmatrix},\quad\bfa Q_{i}^{(3)}=\begin{bmatrix}
        &\bfa 0_{(3d+3k+1)\times D}&\\
        \bfa 0_{k\times(2d+2k)} &I_{k}&\bfa 0\\
        &\bfa 0&
    \end{bmatrix},\\
    \bfa K_{i}^{(3)}&=\begin{bmatrix}
        &\bfa 0_{(3d+3k+1)\times D}&\\
        \bfa 0_{k \times (3d+3k+1)}&\diag(a_i)_{k\times k}&\bfa 0\\
        &\bfa 0&
    \end{bmatrix}\begin{bmatrix}
        &\bfa 0_{(3d+3k+1)\times D}&\\
       \bfa 0_{k\times(3d+2k+1)}&I_{k} &\bfa 0\\
       &\bfa 0&
    \end{bmatrix},
\end{align*}
where we show in lemma \ref{reluapprox} that for $M>\frac{1}{\epsilon^2}\log(1+C/\epsilon)$, there exists $\{a_i\}_{i\in[M]}$ such that for $x>1$, we have 
\begin{align*}
    \bigg\Vert\sum_{i=1}^Mc_i\sigma(a_ix) - \frac{1}{x}\bigg\Vert_2\leq\epsilon.
\end{align*}
And when $x=0$, we automatically obtain that 
$   \sum_{i=1}^Mc_i\sigma(0) =0$. We then immediately obtain that
\begin{align*}
    \bfa H_{4,1}:&=\bfa H_3 + \sum_{i=1}^M\bfa V_i^{(3)}\sigma\lef((\bfa Q_i^{(3)}\bfa H_3)^\top(\bfa K_i^{(3)}\bfa H_3)\rig)= \begin{bmatrix}
        \bfa x_1&\bfa x_2&\ldots&\bfa x_N\\
        \wha \mu_{\wh z^{(0)}(1)}^{(0)}&\wha \mu_{\wh z^{(0)}(2)}^{(0)}&\ldots&\wha \mu_{\wh z^{(0)}(N)}^{(0)}\\
        \bfa p_{1,1}&\bfa p_{1,2}&\ldots&\bfa p_{1,N}\\
        \bfa p_{2,1}&\bfa p_{2,2}&\ldots&\bfa p_{2,N}\\
        1&1&\ldots&1\\
        \bfa v_k&\bfa v_k&\ldots&\bfa v_k\\
        \diag(\bfa v_k)&&\bfa 0&\\
        \diag(\bfa v_k^\prime)&&\bfa 0&\\
        &\bfa 0&&
    \end{bmatrix}+O_{2}(\epsilon),
\end{align*}
where $\bfa v_{k,i}^\prime = \bfa v_{k,i}^{-1}$. Then we apply the MLP again with the following design 
\begin{align*}
    \bfa W_1^{(4)}= I_D,\quad\bfa W_2^{(4)}=\begin{bmatrix}
        &\bfa 0_{(3d+k)\times D}&&\\
        \bfa 0_{k\times(3d+k)}&-I_k&I_k&\bfa 0\\
        \bfa 0&-I_k&\bfa 0&\bfa 0\\
        &\bfa 0&&
    \end{bmatrix}.
\end{align*}
The above construction implies that
\begin{align*}
    \bfa H_4 =\bfa W_2^{(3)}\sigma\lef(\bfa W_1^{(3)}\bfa H_3\rig) = \begin{bmatrix}
        \bfa x_1&\bfa x_2&\ldots&\bfa x_N\\
        \wha \mu_{\wh z^{(0)}(1)}^{(0)}&\wha \mu_{\wh z^{(0)}(2)}^{(0)}&\ldots&\wha \mu_{\wh z^{(0)}(N)}^{(0)}\\
        \bfa p_{1,1}&\bfa p_{1,2}&\ldots&\bfa p_{1,N}\\
        \bfa p_{2,1}&\bfa p_{2,2}&\ldots&\bfa p_{2,N}\\
        1&1&\ldots&1\\
        \bfa v_k&\bfa v_k&\ldots&\bfa v_k\\
        \diag(\bfa v_k^\prime)&&\bfa 0&\\
        &\bfa 0&&
    \end{bmatrix}.
\end{align*}
We construct the following layer to perform the normalization, given by 
\begin{align*}
    \bfa V_2^{(4)}&=-\bfa V_{1}^{(4)}= \begin{bmatrix}
    &\bfa 0_{(3d+2k+1)\times D}& \\
    \bfa 0_{k\times(3d+k+1)}&I_{k}&\bfa 0\\
    &\bfa 0&
    \end{bmatrix},\quad \bfa Q_1^{(4)} =-\bfa Q_2^{(4)}= \begin{bmatrix}
        &\bfa 0_{(3d+2k+1)\times D}&\\
        \bfa 0_{k\times 3d}&I_{k}&\bfa 0\\
        &\bfa 0&
    \end{bmatrix},\\
    \bfa K_1^{(4)}&=\bfa K_2^{(4)}=\begin{bmatrix}
        &\bfa 0_{(3d+2k+1)\times D}&\\
        \bfa 0_{k\times (3d+1)}&I_k&\bfa 0\\
        &\bfa 0&
    \end{bmatrix}.
\end{align*}
Then we can show that
\begin{align*}
    \sigma\lef((\bfa Q_1^{(4)}\bfa H_4)^\top(\bfa K_1^{(4)}\bfa H_4)\rig)-\sigma\lef((\bfa Q_2^{(4)}\bfa H_4)^\top(\bfa K_2^{(4)}\bfa H_4)\rig) = \begin{bmatrix}
        &\bfa 0_{(3d+2k+1)\times D}&\\
        \bfa p_{1,1}&\ldots&\bfa p_{1,N}\\
        &\bfa 0
    \end{bmatrix}.
\end{align*}
And we also have
\begin{align*}
    \bfa V_2^{(4)}\bfa H_4 = \begin{bmatrix}
        &\bfa 0_{(3d+3k+1)\times D}&\\
        \bfa 0_{k\times(3d+2k+1)}&\diag(\bfa v_k^\prime) &\bfa 0\\
        &\bfa 0&
\end{bmatrix}+O_2(\epsilon),
\end{align*}
which implies that
\begin{align*}
    \bfa H_{4,1} = \bfa H_3+\sum_{i=1}^2\bfa V_i^{(4)}\bfa H_3\times\sigma\lef((\bfa Q_i^{(4)}\bfa H_3)^\top(\bfa K_i^{(3)}\bfa H_3)\rig)=\begin{bmatrix}
        \bfa x_1&\bfa x_2&\ldots&\bfa x_N\\
        \wha \mu_{\wh z^{(0)}(1)}^{(0)}&\wha \mu_{\wh z^{(0)}(2)}^{(0)}&\ldots&\wha \mu_{\wh z^{(0)}(N)}^{(0)}\\
        \bfa p_{1,1}&\bfa p_{1,2}&\ldots&\bfa p_{1,N}\\
        \bfa p_{2,1}&\bfa p_{2,2}&\ldots&\bfa p_{2,N}\\
        1&1&\ldots&1\\
        \diag(\bfa v_k^\prime)&&\bfa 0&\\
        \bfa p_{1,1}^\prime&\bfa p_{1,2}^\prime&\ldots&\bfa p_{1,N}^\prime
    \end{bmatrix}+O_{2}(\epsilon),
\end{align*}
where $\bfa p_{1,i}^\prime = \diag(\bfa v_k^\prime)\bfa p_{1,i}$ for all $i\in[N]$. We therefore construct an MLP layer to replace the $\bfa p_{1}$ part using the following design
\begin{align*}
    \bfa W_1^{(4)}=I_D,\qquad\bfa W_2^{(4)} = \begin{bmatrix}
        &\bfa 0_{2d\times D}&&\\
        \bfa 0_{k\times 2d}&-I_k&\bfa 0&I_k&\bfa 0\\
        \bfa 0&\bfa 0&\bfa 0&\bfa 0&\bfa 0\\
        \bfa 0_{k\times 2d}&-I_k&\bfa 0&\bfa 0 &\bfa 0\\
    \end{bmatrix},
\end{align*}
which ultimately leads to 
\begin{align*}
    \bfa H_4 = \bfa W_1^{(3)}\sigma\lef(\bfa W_2^{(3)}\bfa H_3\rig) = \begin{bmatrix}
        \bfa x_1&\bfa x_2&\ldots&\bfa x_N\\
        \wha \mu_{\wh z^{(0)}(1)}^{(0)}&\wha \mu_{\wh z^{(0)}(2)}^{(0)}&\ldots&\wha \mu_{\wh z^{(0)}(N)}^{(0)}\\
        \bfa p^\prime_{1,1}&\bfa p^\prime_{1,2}&\ldots&\bfa p^\prime_{1,N}\\
        \bfa p_{2,1}&\bfa p_{2,2}&\ldots&\bfa p_{2,N}\\
        1&1&\ldots&1\\
        \diag(\bfa v_k^\prime)&&\bfa 0&\\
        &\bfa 0&&
    \end{bmatrix}.
\end{align*}
We can further perform the expectation step by the following layer
\begin{align*}
    \bfa V_i^{(4)}=\begin{bmatrix}
        &\bfa 0_{(3d+2k+1)\times D}&\\
        \bfa 0_{k\times2d}& I_k&\bfa 0\\
        &\bfa 0&
    \end{bmatrix},\quad\bfa Q_i^{(4)}=\begin{bmatrix}
        \bfa 0&\bfa 0&\bfa 0\\
        \bfa 0_{k\times2d}& I_k&\bfa 0\\
        \bfa 0&\bfa 0&\bfa 0
    \end{bmatrix},\quad\bfa K_i^{(4)}=\begin{bmatrix}
        \bfa 0&\bfa 0&\bfa 0\\
        \bfa 0&\bfa 0&\bfa 0\\
        \bfa 0&\bfa 0&\bfa 0
    \end{bmatrix}
\end{align*}
Then it is further noted that
\begin{align*}
    \bfa H_5 =\bfa H_4+\sum_{i=1}^2\bfa V_i^{(4)}
\end{align*}
\end{proof}
% The rest of this section presents a proof sketch on theorem \ref{thm3.1}.
% \paragraph{Proof Sketch of Theorem \ref{thm3.1}.}
%     The proof sketch for theorem \ref{thm3.1} goes by two important steps:



%     To obtain $\tda p_{3,i}$ for $i\in[k]$, we generate through the probablistic method, which gives the following lemma.
%     % \begin{lemma}
%     %     Let $\bfa y\in\bb R^d$ be a random vector with isotropic Gaussian as its probability density. Consider $\bfa x=\frac{\bfa y}{\Vert\bfa y\Vert_2}$. Let $\bfa v$ be any unit length vector, then we have
%     %        $$ \bb P\bl|\bfa v^\top\bfa x|\leq\frac{\epsilon}{\sqrt d}\br\leq  \frac{1}{\sqrt{\pi}}\epsilon+\exp\lef(-Cd\rig).$$
%     %     Therefore, the event in theorem \ref{thm3.1} can be estimated as
%     %     \begin{align*}
%     %          \bb P\bl\exists i\text{ such that }&\bfa x_i^\top\bfa v_i\leq\frac{\epsilon}{\sqrt d}\br\leq 1-\bl1-\frac{\epsilon}{\sqrt{\pi}}-\exp(-Cd)\br^k\leq \frac{k\epsilon}{\sqrt{\pi}}+k\exp(-Cd).
%     %     \end{align*}
%     % \end{lemma}

%     % \begin{align*}
%     %    \Bigg\Vert \tda H^{rpe,2}-\begin{bmatrix}
%     %         \bfa X\\
%     %         \tda y\\
%     %         \bfa X\bfa X^\top,\bfa 0\\
%     %         \tda p_{2,1},\ldots,\tda p_{2,N}\\
%     %         \tda p_{3,1},\ldots,\tda p_{3,N}\\
%     %         \tda p_{3,1}^{(\tau)},\bfa 0\\
%     %         \Vert\bfa X\bfa X^\top\tda p_{3,N}^{(\tau)}\Vert_2,\bfa 0
%     %     \end{bmatrix}\Bigg\Vert_2\leq\epsilon \Vert\bfa X\bfa X^\top\tda p_{3,1}\Vert_2.
%     % \end{align*}
%     % Then we can show that
%     % \begin{align*}
%     %     &\Bigg\Vert\bfa K\tda H^{pow,2\tau+2}-\begin{bmatrix}
%     %         \bfa 0_{(5d+1)\times(4d+1)}&\bfa 0_{(5d+1)\times 1}&\bfa 0\\
%     %         \bfa 0_{d\times(4d+1)}&\tda p_{3,1}^{(\tau)}&\bfa 0\\
%     %         \bfa 0&\bfa 0&\bfa 0
%     %     \end{bmatrix}\Bigg\Vert_2\leq \tau\epsilon\Vert\bfa X\bfa X^\top\Vert_2,\\ 
%     %     &\Bigg\Vert\bfa Q\tda H^{pow,2\tau+2}-\begin{bmatrix}
%     %         \bfa 0_{(5d+1)\times 1}&\bfa 0\\
%     %        \Vert\bfa X\bfa X^\top\tda p_{3,1}^{(\tau)}\Vert_2 \tda p_{3,1}^{(\tau)}&\bfa 0\\
%     %         \bfa 0&\bfa 0
%     %     \end{bmatrix}\Bigg\Vert_2\leq\tau\epsilon\Vert\bfa X\bfa X^\top\Vert_2.
%     % \end{align*}
%     % which immediately implies that
%     % \begin{align*}
%     %     \Bigg\Vert(\bfa K\tda H^{pow,2\tau+2})^\top(\bfa Q\tda H^{pow,2\tau+2})-\begin{bmatrix}
%     %         \bfa 0&\bfa 0\\
%     %         \Vert\bfa X\bfa X^\top\tda p_{3,1}\Vert_2\tda p_{3,1}^{(\tau)}\tda p_{3,1}^{(\tau),\top}&\bfa 0\\
%     %         \bfa 0&\bfa 0
%     %     \end{bmatrix}\Bigg\Vert_2
%     % \end{align*}
% \begin{lemma}\label{lm3.2}
% Assume that the correlation matrix $\bfa X\bfa X^\top$ has eigenvalues $\lambda_1>\lambda_2>\ldots>\lambda_{k}$. Assume that the eigenvectors are given by $\bfa v_1,\bfa v_2,\ldots,\bfa v_n$ and the eigenvalues satisfy $\inf_{i\neq j}|\lambda_i-\lambda_j|=\Delta$. Then, given that the estimate for the first $\tau$ eigenvectors satisfy
% $    \bfa v_i^\top\wha v_i\geq 1-\epsilon_i $ and the eigenvalues satisfy $|\lambda_i-\wh\lambda_i|\leq\delta_i$, the principle eigenvector of $\bfa X\bfa X^\top-\sum_{i=1}^{\tau}\wh\lambda_i\wha v_i\wha v_i^\top$ denoted by $\tda v_{\tau+1}$ satisfies
% \begin{align*}
%     \Vert\tda v_{\tau+1}-\bfa v_{\tau+1}\Vert_2 \leq\frac{\max_{i\in[\tau]}\delta_i+\sum_{i=1}^\tau\sqrt8\lambda_i\sqrt{\epsilon_i}}{\Delta}.
% \end{align*}
% Alternatively, we can also show that the eigenvector $\wha v_{\tau+1}$ returned by power method with $k= \frac{\log(1/\epsilon_0\delta)}{2\epsilon_0}$ that is initialized by  satisfies
% \begin{align*}
%     \wha v_{\tau+1}^\top\bfa v_{\tau+1} \geq 1-\epsilon_{\tau+1}:=  1-\frac{1}{2}\Big(\frac{\max_{i\in[\tau]}\delta_i+\sum_{i=1}^{\tau}\sqrt{8}\lambda_i\sqrt{\epsilon_i}}{\Delta}+\sqrt{2\epsilon_0}\Big)^2,
% \end{align*}
% and alternatively we have $\Vert\wha v_{\tau+1}-\bfa v_{\tau+1}\Vert_2\leq\sqrt{2\epsilon_{\tau+1}}$. Furthermore, consider the eigenvalue estimate, we show that
% \begin{align*}
%    \Big|\Big\Vert\Big(\bfa X\bfa X^\top-\sum_{i=1}^{\tau}\wh\lambda_i\wha v_i\wha v_i^\top\Big)\wha v_{\tau+1}\Big\Vert_2-\lambda_{\tau+1}\Big|\leq \frac{2\lambda_{\tau+1}}{\Delta}\Big(\max_{i\in[\tau]}\delta_i+\sum_{i=1}^{\tau}\sqrt{8}\lambda_i\sqrt{\epsilon_i}\Big)+\lambda_{\tau+1}\sqrt{2\epsilon_0}
% \end{align*}
% \end{lemma}

% \begin{lemma}[Approximation of norm by sum of Relu activations by Transformer networks]\label{reluapprox}
%     Assume that there exists a constant $C$ with $\Vert\bfa v\Vert_2\leq C$. There exists a multihead Relu attention layer with number of heads $M<\lef(\frac{\overline R}{\underline R}\rig)^d\frac{C(d)}{\epsilon^2}\log(1+C/\epsilon)$ such that there exsits $\{\bfa a_m\}_{m\in[M]}\subset\bb S^{N-1}$ and $\{c_m\}_{m\in[M]}\subset\bb R$ where for all $\bfa v$ with $\overline R\geq\Vert\bfa v\Vert_2\geq \underline R$, we have 
%     \begin{align*}
%         \bigg|\sum_{m=1}^Mc_m\sigma(\bfa a_m^\top\bfa v)-\frac{1}{\Vert\bfa v\Vert_2}+1\bigg|\leq\epsilon.
%     \end{align*}
%     Similarly, there exists a multihead Relu attention layer with number of heads $M\leq\overline R^{\frac{d}{2}}\frac{C(d)}{\epsilon^2}\log\lef(1+C/\epsilon\rig)$, a set of vectors $\{\bfa b_m\}_{m\in[M]}\subset\bb S^{N-1}$ and $\{d_m\}_{m\in[M]}\subset\bb R$ such that 
%     \begin{align*}
%         \Big|\sum_{m=1}^Md_m\sigma(\bfa b_m^\top\bfa v)-\Vert\bfa v\Vert_2^{1/2}+1\Big|\leq\epsilon.
%     \end{align*}
% \end{lemma}




\section{Discussions}\label{sect5}
This section discusses the limitations on the theory part of this work and points to future working directions. 
\paragraph{Limitations.} Our limitations in the theoretical results can be summarized as follows: \textbf{(1)} From the theoretical perspective, our results guarantee the performance of ERM solutions whereas the true estimator is obtained through stochastic gradient descent method; \textbf{(2)} Our theoretical results utilize the context-augmented matrix $\bfa P$, which is verified removable from our empirical results. 
\paragraph{Future Works.} Beyond resolving the limitations in this work, other future working directions from this work include: \textbf{(1)} Taking into consideration of the layer norm in the Transformer architecture. \textbf{(2)} Resolving the universal approximation problem raised in section \ref{approx3}; \textbf{(3)} Removing the initialization procedure in the theory.




\section*{Impact Statement}
This paper theoretically analyzes the capability of Transformers in performing EM algorithm. Due to the theoretical nature of this work, there is no negative sociatal impact.
% Due to the theoretical nature of this work, the sociatal impact is minor. Though it is possible that Transformer models might induce safety issues when used improperly.


\bibliography{ref}
\bibliographystyle{icml2025}


%%%%%%%%%%%%%%%%%%%%%%%%%%%%%%%%%%%%%%%%%%%%%%%%%%%%%%%%%%%%%%%%%%%%%%%%%%%%%%%
%%%%%%%%%%%%%%%%%%%%%%%%%%%%%%%%%%%%%%%%%%%%%%%%%%%%%%%%%%%%%%%%%%%%%%%%%%%%%%%
% APPENDIX
%%%%%%%%%%%%%%%%%%%%%%%%%%%%%%%%%%%%%%%%%%%%%%%%%%%%%%%%%%%%%%%%%%%%%%%%%%%%%%%
%%%%%%%%%%%%%%%%%%%%%%%%%%%%%%%%%%%%%%%%%%%%%%%%%%%%%%%%%%%%%%%%%%%%%%%%%%%%%%%
\newpage

\bigskip
\onecolumn

% \begin{appendices}
\begin{center}
{\large\bf SUPPLEMENTARY MATERIAL}
\end{center}

\appendix



% {
% \setlength{\parskip}{-0em}
% \startcontents[sections]
% \printcontents[sections]{ }{1}{}
% }

% \stopcontents
% \startcontents[sections]
% \printcontents[sections]{ }{1}{}


\section{Additional Theoretical Background}

\begin{definition}[Sufficiently Smooth $d$-variate function]
    Denote $\msf B_{\infty}^d(R):=[-R,R]^d$ as the standard $\ell_\infty$ ball in $\bb R^d$.
    We say a function $g:\bb R^d\to\bb R$ is $(R,C_\ell)$ smooth if for $s=\lceil(d-1)/2\rceil+2$, $g$ is a $C^s$ function on $\msf B^d_\infty(\bb R)$ and
    \begin{align*}
        \sup_{\bfa z\in\msf B_{\infty}^d(R)}\Vert\nabla^dg(\bfa z)\Vert_{\infty}=\sup_{\bfa z\in\msf B^d_\infty(R)}\max_{j_1,\ldots,j_i\in[d]}\lef|\pta_{x_{j_1}\ldots x_{j_{i}}}g(\bfa x)\rig|\leq L_i
    \end{align*}
    for all $i\in\{0,1,\ldots,s\}$, with $\max_{0\leq i\leq s}L_iR^i\leq C_\ell$.
\end{definition}
\begin{definition}[Approximability by sum of Relus \citep{bai2024transformers}]\label{def6} A function $g:\bb R^k\to\bb R$ is $(\epsilon_{approx}, R,M,C)$-approximable by sum of Relus if there exists a function $f_{M,C}$ such that
\begin{align*}
    f_{M,C}(\bfa z)=\sum_{m=1}^Mc_m\sigma(\bfa a_m^\top[\bfa z;1])\text{ with }\sum_{m=1}^M|c_m|\leq C,\;\max_{m\in[M]}\Vert \bfa a_m\Vert_1\leq 1,\quad \bfa a_m\in\bb R^{k+1},\; c_m\in\bb R,
\end{align*}
such that $\sup_{\bfa z\in\msf B_\infty^k(R)}|g(\bfa z)-f_{M,C}(\bfa z)|\leq \epsilon_{approx}$.
\end{definition}



\section{Proofs}
\subsection{Proof of Proposition \ref{genbound}}
\begin{proof}
    The proof follows from \citep{wainwright2019high}, using the fact that for all $\bfa\theta\in \Theta(B_{\bfa\theta},B_M)$, we have 
    \begin{align*}
        \frac{1}{n}\sum_{j=1}^nL\lef(TF_{\wha\theta}(\bfa H_i),\bfa V_i\rig)\leq\frac{1}{n}\sum_{j=1}^n L\lef(TF_{\bfa\theta}(\bfa H_i),\bfa V_i\rig),
    \end{align*} 
    it is not hard to show that
    \begin{align*}
        \bb E\lef[L\lef(TF_{\wha\theta}(\bfa H),\bfa V\rig)\rig]\leq \inf_{\bfa\theta\in \Theta(B_{\bfa\theta},B_M)}\bb E\lef[L(TF_{\wha\theta}(\bfa H),\bfa V)\rig]+2\sup_{\bfa\theta\in \Theta(B_{\bfa\theta},B_M)}| X_{\bfa\theta}|,
    \end{align*}
    where $ X_{\bfa\theta}=\frac{1}{n}\sum_{j=1}^nL(TF_{\bfa\theta}(\bfa H_i),\bfa V_i)-\bb E[L(TF_{\bfa\theta}(\bfa H),\bfa V)]$ is the empirical process indexed by $\bfa\theta$.
    The tail bound for the empirical process requires us to verify a few regularity conditions \citep{gine2016mathematical} on the function $L$ and the set $\Theta$
    \begin{enumerate}
        \item The metric entropy of an operator norm ball $\log N(\delta, B_{\Vert\cdot\Vert_{op}}(r),\Vert\cdot\Vert_{op})\leq CLB_M D^2\log\lef(1+2(B_{\theta}+B_X+k)/\delta\rig)$.
        \item $L(TF_{\bfa\theta}(\bfa H),\bfa V)\leq C\sqrt k$.
        \item The Lipschitz condition of Transformers satisfies that for all $\bfa\theta_1,\bfa\theta_2\in\Theta(B_{\bfa\theta}, B_{M})$, we have $L(TF_{\bfa\theta_1}(\bfa H),\bfa V) - L(TF_{\bfa\theta_2}(\bfa H),\bfa V)\leq C L B_1^{L}\Vert\bfa\theta_1-\bfa\theta_2\Vert_{op}$ where $B_1=B_{\theta}^4B_X^3$.
    \end{enumerate}
    The first and second verifications follow immediately from J.2 in \citep{bai2024transformers}. The third verification is given upon noticing that as $L(\bfa x,\bfa y)=\Vert\bfa x-\bfa y\Vert_2$,
    \begin{align*}
        \sup_{\bfa\theta,\bfa H,\bfa V}L(TF_{\bfa\theta}(\bfa H),\bfa V)\leq C\sqrt k,\qquad \Vert\nabla_{\bfa x} L\Vert\leq C.
    \end{align*}
    Further note that $\Vert\tda W_0\Vert_2\asymp\Vert\tda W_1\Vert_2\asymp 1$.
    Given the above result, and corollary J.1 in \citep{bai2024transformers}, we can show that
    \begin{align*}
        L(TF_{\bfa\theta_1}(\bfa H),\bfa V)-L(TF_{\bfa\theta_2}(\bfa H),\bfa V)\leq C L B_1^{L}\Vert\bfa\theta_1-\bfa\theta_2\Vert_{op},
    \end{align*}
    where $B_1 = B^4_{\theta}B^3_{X}$.
    Therefore, using the uniform concentration bound given by proposition A.4 we can show that with probability at least $1-\xi$, we have
    \begin{align*}
        \sup_{\bfa\theta\in\Theta(B_{\bfa\theta},B_M)} |X_{\theta}|\leq C\sqrt k\sqrt{\frac{LB_MD^2\log(B_{\theta}+B_X+k)+\log(1/\delta)}{n}}.
    \end{align*}
    Therefore, replacing $D$ with $Ckd$ we complete the proof.
\end{proof}

\subsection{Proof of Theorem \ref{thm3.1}}
\begin{proof}
    Our proof can be dissected into the following steps: 1. We construct a Transformer with fixed parameters that performs
    (1) The computation of the symmetrized covariate matrix; (2) The approximation of the power method; (3) The removal of the principal eigenvectors; (4) Adjust the dimension of the output through multiplying the two matrices $\tda W_0$ and $\tda W_1$ on the left and right.
    \begin{center}
        \textbf{1. The Covariate Matrix.}
    \end{center}
    To compute the covariate matrix $\bfa X\bfa X^\top$, we construct $\bfa H=\begin{bmatrix}
        \bfa X_1,\ldots,\bfa X_N\\
        \tda p_{1,1},\ldots,\tda p_{1,N}\\
        \tda p_{2,1},\ldots,\tda p_{2,N}\\
        \vdots\\
        \tda p_{\ell,1},\ldots,\tda p_{\ell,N}
    \end{bmatrix}=\begin{bmatrix}
        \bfa X\\
        \bfa P
    \end{bmatrix}$
    we let $m=2$ and
    \begin{align}
     & \bfa V_1^{cov}=I_D=-\bfa V_2^{cov},\quad \bfa Q_1^{cov,\top}\bfa K_1^{cov}=-\bfa Q_2^\top\bfa K_2=\begin{bmatrix}
         \bfa 0_{N+1\times d}, I_d,\bfa 0\\
         \bfa 0,\bfa 0,\bfa 0
     \end{bmatrix}\in\bb R^{D\times D},\nnb\\
     &\tda p_{1,\ell,j}=\bfa 0,\qquad\tda p_{2,\ell,j}=\begin{cases}
         \mbbm 1_{\ell=j}\quad&\text{ when }\ell\leq d\\
         0&\text{ when }\ell>d
     \end{cases}.
    \end{align}
    Under the above construction, we obtain that
    \begin{align*}
        &\bfa Q_1^\top\bfa K_1\bfa H=\begin{bmatrix}
            I_{d}&\bfa 0\\
            \bfa 0&\bfa 0
        \end{bmatrix}\in\bb R^{D\times N},\quad\bfa Q_2^\top\bfa K_2\bfa H=\begin{bmatrix}
            -I_{d}&\bfa 0\\
            \bfa 0&\bfa 0
        \end{bmatrix}\in\bb R^{D\times N},\\
        &\sigma(\bfa H^\top\bfa Q_1^\top\bfa K_1\bfa H)+\sigma(\bfa H^\top\bfa Q_2^\top\bfa K_2\bfa H)=\begin{bmatrix}
            \bfa X^\top,\bfa 0
        \end{bmatrix}\in\bb R^{N\times N}.
    \end{align*}
   We further obtain that
    \begin{align*}
        \frac{1}{N}\sum_{m=1}^M(\bfa V_m\bfa H)\times \sigma\lef((\bfa Q_m\bfa H)^\top(\bfa K_m\bfa H)\rig)=\begin{bmatrix}
        \bfa 0&\bfa 0\\
            \bfa X\bfa X^\top\in\bb R^{d\times d}&\bfa 0\\
            \bfa 0&\bfa 0
        \end{bmatrix}\in\bb R^{D\times D}.
    \end{align*}
    Therefore, the output is given by  $\tda H^{cov}=\begin{bmatrix}
            \bfa X\\
            \bfa X\bfa X^\top,\bfa 0\\
            \tda p_{2,1},\ldots,\tda p_{2,N}\\
            \tda p_{\ell,1},\ldots,\tda p_{\ell,N}
        \end{bmatrix}$. 
    \begin{center}
        \textbf{2. The Power Iteration.}
    \end{center}
    Then we consider constructing a single attention layer that approximates the power iteration. This step involves two important operations: (1) Obtaining the vector given by $\bfa X\bfa X^\top\bfa v$. (2) Approximation of the value of the inverse norm given by $1/\Vert \bfa X\bfa X^\top\bfa v\Vert_2$. We show that one can use the multi-head Relu Transformer to achieve both goals simultaneously, whose parameters are given by 
    \begin{align*}
        \bfa V_{1}^{pow,1}&=-\bfa V_{2}^{pow,1}=\begin{bmatrix}
            \bfa 0_{(3d+1)\times (2d+1)}&\bfa 0&\bfa 0\\
            \bfa 0_{(d)\times(2d+1)}&I_d&\bfa 0\\
            \bfa 0&\bfa 0&\bfa 0
        \end{bmatrix},\\
        \bfa Q_1^{pow,1}&=-\bfa Q_1^{pow,1}=\begin{bmatrix}
           \bfa 0_{(d+1)\times (d+1)}& \bfa 0&\bfa 0\\
            \bfa 0_{d\times (d+1)}&I_d&\bfa 0\\
            \bfa 0&\bfa 0&\bfa 0
        \end{bmatrix},\\
        \bfa K_1^{pow,1}&=\bfa K_2^{pow,1}=\begin{bmatrix}
            \bfa 0_{(3d+1)\times(3d+1)}&\bfa 0&\bfa 0\\
            \bfa 0_{d\times(3d+1)}&I_d&\bfa 0\\
            \bfa 0&\bfa 0&\bfa 0
        \end{bmatrix},\qquad\tda p_{4,j} = \bfa 0\text{ for all }j\in[N].
    \end{align*}
    Given the above formulation, we can show that 
    \begin{align*}
       &\bfa Q_2^{pow,1}\tda H^{cov}=-\bfa Q_1^{pow,1}\tda H^{cov} = \begin{bmatrix}
        \bfa 0_{(2d+1)\times N}\\
            \bfa X\bfa X^\top,\bfa 0\\
            \bfa 0
        \end{bmatrix},\\
        &\bfa K_2^{pow,1}\tda H^{cov}=\bfa K_1^{pow,1}\tda H^{cov}=\begin{bmatrix}
            \bfa 0_{2d+1}\\
            \tda p_{3,1},\bfa 0\\
            \bfa 0
        \end{bmatrix},
    \end{align*}
    which implies that 
    \begin{align*}
    \sum_{m\in\{1,2\}}\sigma((\bfa Q_m^{pow, 1}\tda H^{cov})^\top\bfa K_m^{pow,1}\tda H^{cov})=\begin{bmatrix}
           \bfa 0& \bfa 0_{d\times(2d+1)}&\\
            \bfa X\bfa X^\top\tda p_{3,1}&\bfa 0_{d\times(N-1)}\\
            \bfa 0&\bfa 0&
        \end{bmatrix}.
    \end{align*}
    Then we can show that 
    \begin{align*}
        \tda H^{pow,1}-\tda H^{cov}&=\sum_{m\in\{1,2\}}\bfa V_m^{pow,1}\tda H^{cov}\times\sigma((\bfa Q_m^{pow,1}\tda H^{cov})^\top\bfa K_m^{pow,1}\tda H^{cov})\\
        &=\begin{bmatrix}
            \bfa 0_{3d+1}\\
            \bfa X\bfa X^\top\tda p_{3,1},\bfa 0_{d\times(N-1)}\\
            \bfa 0
        \end{bmatrix}.
    \end{align*}
    Therefore, we conclude that the output of the first power iteration layer is given by
    \begin{align*}
        \tda H^{pow,1}=\begin{bmatrix}
            \bfa X\\
            \tda y^\top\\
            \bfa X\bfa X^\top,\bfa 0\\
            \tda p_{2,1},\ldots,\tda p_{2,N}\\
            \tda p_{3,1},\ldots,\tda p_{3,N}\\
            \bfa X\bfa X^\top\tda p_{3,1},\bfa 0\\
            \tda p_{5,1},\ldots,\tda p_{5,N}\\
            \vdots\\
            \tda p_{\ell,1},\ldots,\tda p_{\ell,N}
        \end{bmatrix}.
    \end{align*}
    % Then we consider integrating the normalizing constants into $\bfa H$ and remove unnecessary $\tda p_{}$.
    % We further let $\tda p_{3,\ell}$ to be i.i.d. sampled from $\bb S^{d-1}$ and one uses lemma \ref{reluapprox} to obtain that there exists $\{\bfa a_m\}_{m\in[M]}$ with
    % \begin{align*}
    %     \frac{1}{N}\sum_{m=1}^M(\bfa V_m\bfa H)\times\sigma\lef((\bfa Q_m\bfa H)^\top(\bfa K_m\bfa H)\rig)=\begin{bmatrix}
    %         \bfa 0_{(4d+1)\times N}\\
    %         \frac{\bfa X\bfa X^\top\tda p_{1,1}}{\Vert\bfa X\bfa X^\top\tda p_{1,1}\Vert_2},\ldots,\bfa 0\\
    %         \bfa 0
    %     \end{bmatrix}.
    % \end{align*}
    Then, using lemma \ref{reluapprox}, we design an extra attention layer that performs the normalizing procedure, with the following parameters for all $m\in[M]$,
    \begin{align*}
        &\bfa V_{m}^{pow,2}=\begin{bmatrix}
            \bfa 0_{d\times (4d+1)}&c_m\bfa I_d&\bfa 0\\
            \bfa 0&\bfa 0&\bfa 0
        \end{bmatrix},\qquad\bfa Q_{m}^{pow,2}=\begin{bmatrix}
            \bfa 0_{d\times(2d+1)}&\bfa I_{d}&\bfa 0\\
            \bfa 0&\bfa 0&\bfa 0
        \end{bmatrix},\\
        &\bfa K_m^{pow,2}=\begin{bmatrix}
            \bfa 0_{1\times (3d+1)}&\bfa a_m^\top&\bfa 0\\
            \vdots\\
            \bfa 0_{1\times(3d+1)}&\bfa a_m^\top&\bfa 0\\
            \bfa 0_{(D-d)\times(3d+1)}&\bfa 0&\bfa 0
        \end{bmatrix}.
    \end{align*}
    Under the above construction, we obtain that
    \begin{align*}
        (\bfa Q_m^{pow,2}\tda H^{pow,1})^\top =\begin{bmatrix}
            I_{d\times d}&\bfa 0\\
            \bfa 0&\bfa 0
        \end{bmatrix},\qquad \bfa K_m^{pow,2}\tda H^{pow,1} = \begin{bmatrix}
            \bfa a_m^\top\bfa X\bfa X^\top\tda p_{3,1}&\bfa 0\\
            \vdots\\
            \bfa a_m^\top\bfa X\bfa X^\top\tda p_{3,1}&\bfa 0\\
            \bfa 0_{(D-d)\times 1}&\bfa 0
        \end{bmatrix}.
    \end{align*}
    Then, given $\bfa V_m^{pow,2}$ we can show that under the condition given by lemma \ref{reluapprox}, we have
    \begin{align*}
        \bigg\Vert&\sum_{m=1}^M\bfa V_m^{pow,2}\tda H^{pow,1}\sigma\lef((\bfa Q_m^{pow,2}\tda H^{pow,1})^\top(\bfa K_m^{pow,2}\tda H^{pow,1})\rig)-\begin{bmatrix}
        \bfa 0_{4d+1}\\
            \frac{\bfa X\bfa X^\top\tda p_{3,1}}{\Vert\bfa X\bfa X^\top\tda p_{3,1}\Vert_2}-\bfa X\bfa X^\top\tda p_{3,1},\bfa 0\\
            \bfa 0
        \end{bmatrix}\bigg\Vert_{\infty}\\
        &<\epsilon,
    \end{align*}
    Moreover, we can further achieve that
    \sm{\begin{align*}
        \Bigg\Vert\sum_{m=1}^M\bfa V_m^{pow,2j}\tda H^{pow,1}\sigma\lef((\bfa Q_m^{pow,2}\tda H^{pow,1})^\top (\bfa K_m^{pow,2}\tda H^{pow,1})\rig)&-\begin{bmatrix}
            \bfa 0_{4d+1}\\
            \frac{\bfa X\bfa X^\top\tda p_{3,1}}{\Vert\bfa X\bfa X^\top\tda p_{3,1}\Vert_2}-\bfa X\bfa X^\top\tda p_{3,1},\bfa 0\\
            \bfa 0
        \end{bmatrix}\Bigg\Vert_2\\
        &<\epsilon\Vert\bfa X\bfa X^\top\tda p_{3,1}\Vert_2.
    \end{align*}}
    Hence, using the fact that $\tda H^{pow,2}=\tda H^{pow,1}+\sum_{i=1}^m\bfa V_m^{pow,2}\tda H^{pow,1}\sigma\lef((\bfa Q_m^{pow,2}\tda H^{pow,1})^\top(\bfa K_m^{pow,2}\tda H^{pow,1})\rig)$, we obtain that
    \begin{align*}
        \Bigg\Vert\tda H^{pow,2}-\begin{bmatrix}
            \bfa X\\
            \tda y\\
            \bfa X\bfa X^\top,\bfa 0\\
            \tda p_{2,1},\ldots,\tda p_{2,N}\\
            \tda p_{3,1},\ldots,\tda p_{3,N}\\
            \frac{\bfa X\bfa X^\top\tda p_{3,1}}{\Vert\bfa X\bfa X^\top\tda p_{3,1}\Vert_2},\ldots\bfa 0\\
            \vdots
        \end{bmatrix}\Bigg\Vert_{2}<\epsilon\Vert\bfa X\bfa X^\top\tda p_{3,1}\Vert_2.
    \end{align*}
    Then we construct another attention layer, which performs similar calculations as that of $pow,1$ but switch the rows of $\tda p_{3,1}$ with that of $\frac{\bfa X\bfa X^\top\tda p_{3,1}}{\Vert\bfa X\bfa X^\top\tda p_{3,1}\Vert_2}$. Our construction for the third layer is given by
    \begin{align*}
        \bfa V_{1}^{pow,3}&=-\bfa V_{2}^{pow,3}=\begin{bmatrix}
            \bfa 0_{(3d+1)\times (2d+1)}&\bfa 0&\bfa 0\\
            \bfa 0_{d\times(2d+1)}&I_d&\bfa 0\\
            \bfa 0&\bfa 0&\bfa 0
        \end{bmatrix},\\
        \bfa Q_1^{pow,3}&=-\bfa Q_2^{pow,3}=\begin{bmatrix}
           \bfa 0_{(3d+1)\times (d+1)}&\bfa 0& \bfa 0\\
            \bfa 0_{d\times (d+1)}&I_d&\bfa 0\\
            \bfa 0&\bfa 0&\bfa 0
        \end{bmatrix},\\
        \bfa K_1^{pow,3}&=\bfa K_2^{pow,3}=\begin{bmatrix}
            \bfa 0_{(4d+1)\times(4d+1)}&\bfa 0&\bfa 0\\
            \bfa 0_{d\times(4d+1)}&I_d&\bfa 0\\
            \bfa 0&\bfa 0&\bfa 0
        \end{bmatrix},\qquad\tda p_{4,j} = \bfa 0\text{ for all }j\in[N].
    \end{align*}
    Given the above construction, we can show that
    \begin{align*}
       &\bfa Q_2^{pow,3}\tda H^{pow,2}=-\bfa Q_1^{pow,3}\tda H^{pow,2}=\begin{bmatrix}
           &\bfa 0_{(3d+1)\times N}&\\
           \bfa 0_{}&\bfa X\bfa X^\top &\bfa 0\\
           &\bfa 0&
       \end{bmatrix},\quad \bfa K_2^{pow,3}\tda H^{pow,2}=\bfa K_1^{pow,3}\tda H^{pow,2},\\
       &\Bigg\Vert\bfa K_2^{pow,3}\tda H^{pow,2}-\begin{bmatrix}
            \bfa 0_{(3d+1)\times N}\\
            \frac{\bfa X\bfa X^\top\tda p_{3,1}}{\Vert\bfa X\bfa X^\top\tda p_{3,1}\Vert_2},\bfa 0\\
            \bfa 0
        \end{bmatrix}\Bigg\Vert_{2}\leq\epsilon\Vert\bfa X\bfa X^\top\tda p_{3,1}\Vert_2.
    \end{align*}
    Then, using the fact that given $\bfa x_1,\bfa x_2$ with $\Vert\bfa x_1-\bfa x_2\Vert_2\leq\delta_0$, we have
$        \Vert\bfa X\bfa X^\top(\bfa x_1-\bfa x_2)\Vert_2\leq \Vert\bfa X\bfa X^\top\Vert_2\delta_0 $. 
    Hence, by collecting the above pieces, we have
    \begin{align*}
      \Bigg\Vert\sum_{m=1}^2\bfa V_m^{pow,3}\tda H^{pow,2}\sigma\lef((\bfa Q_2^{pow,3}\tda H^{pow,2})^{\top}\bfa K_2^{pow,3}\tda H^{pow,2}\rig)&-\begin{bmatrix}
          \bfa 0_{(3d+1)\times N}\\
          \frac{(\bfa X\bfa X^\top)^2\tda p_{3,1}}{\Vert\bfa X\bfa X^\top\tda p_{3,1}\Vert_2}-\frac{\bfa X\bfa X^\top\tda p_{3,1}}{\Vert\bfa X\bfa X^\top\tda p_{3,1}\Vert_2},\bfa 0\\
          \bfa 0
      \end{bmatrix}\Bigg\Vert_{2}\\
      &\leq\epsilon\Vert\bfa X\bfa X^\top\Vert_2\lef\Vert\bfa X\bfa X^\top\tda p_{3,1}\rig\Vert_2.
    \end{align*}
    Henceforth, one can further show that
$      \Bigg\Vert \tda H^{pow,3}- \begin{bmatrix}
            \bfa X\\
            \tda y\\
            \bfa X\bfa X^\top,\bfa 0\\
            \tda p_{2,1},\ldots,\tda p_{2,N}\\
            \tda p_{3,1},\ldots,\tda p_{3,N}\\
            \frac{(\bfa X\bfa X^\top)^2\tda p_{3,1}}{\Vert\bfa X\bfa X^\top\tda p_{3,1}\Vert_2},\bfa 0\\
            \tda p_{5,1},\ldots,\tda p_{5,N}\\
            \vdots\\
            \tda p_{\ell,1},\ldots,\tda p_{\ell,N}
        \end{bmatrix}\Bigg\Vert_{2}\leq\epsilon \Vert\bfa X\bfa X^\top\Vert_2\Vert\bfa X\bfa X^\top\tda p_{3,1}\Vert_2$.
    Consider we are doing in total of $\tau$ power iterations, we can set for all $\tau\in\bb N^*$,
    \begin{align*}
        &\bfa V_m^{pow,2\tau+1}=\bfa V_m^{pow,3},\quad \bfa Q_m^{pow,2\tau+1}=\bfa Q_m^{pow,3},\quad\bfa K_m^{pow,2\tau+1}=\bfa K_m^{pow,3},\\
        &\bfa V_m^{pow,2\tau+2}=\bfa V_m^{pow,4},\quad\bfa Q_m^{pow,2\tau+2}=\bfa Q_m^{pow,4},\quad\bfa K_m^{pow,2\tau+2}=\bfa K_m^{pow,4}.
    \end{align*}
    Therefore, taking another layer of normalization, we can show that
    \begin{align*}
        \Bigg\Vert \tda H^{pow,3}- \begin{bmatrix}
            \bfa X\\
            \tda y\\
            \bfa X\bfa X^\top,\bfa 0\\
            \tda p_{2,1},\ldots,\tda p_{2,N}\\
            \tda p_{3,1},\ldots,\tda p_{3,N}\\
            \frac{(\bfa X\bfa X^\top)^2\tda p_{3,1}}{\Vert\bfa X\bfa X^\top\tda p_{3,1}\Vert_2^2},\bfa 0\\
            \tda p_{5,1},\ldots,\tda p_{5,N}\\
            \vdots\\
            \tda p_{\ell,1},\ldots,\tda p_{\ell,N}
        \end{bmatrix}\Bigg\Vert_{2}\leq2\epsilon \Vert\bfa X\bfa X^\top\Vert_2.
    \end{align*}
    Then, using the sublinearity of errors, we can show that for $\tau\in\bb N$,
    \begin{align*}
              \Bigg\Vert \tda H^{pow,2\tau+2}- \begin{bmatrix}
            \bfa X\\
            \tda y\\
            \bfa X\bfa X^\top,\bfa 0\\
            \tda p_{2,1},\ldots,\tda p_{2,N}\\
            \tda p_{3,1},\ldots,\tda p_{3,N}\\
            \tda p_{3,1}^{(\tau)},\bfa 0\\
            \tda p_{5,1},\ldots,\tda p_{5,N}\\
            \vdots\\
            \tda p_{\ell,1},\ldots,\tda p_{\ell,N}
        \end{bmatrix}\Bigg\Vert_{\infty}\leq \tau\epsilon\Vert\bfa X\bfa X^\top\Vert_2,\quad\tda p^{(\tau)}_{3,1}=\frac{\bfa X\bfa X^\top\tda p_{3,1}^{(\tau-1)}}{\lef\Vert\bfa X\bfa X^\top\tda p_{3,1}^{(\tau-1)}\rig\Vert_2},\quad\tda p_{3,1}^{(0)}=\tda p_{3,1}.
    \end{align*}
    If we denote $\bfa v_i$ as the eigenvector corresponds to the $i$ th largest eigenvalue of $\bfa X\bfa X^\top$. Let the eigenvalues of $\bfa X\bfa X^\top$ be denoted by $\lambda_1>\lambda_2>\cdots>\lambda_n$. Given $|\tda p_{3,1}^\top\bfa v_1|>\delta$ and $|\sqrt{\lambda_1}-\sqrt{\lambda_2}|=\Omega(1)$. Theorem 3.11 in \citep{blum2020foundations} page 53 shows that given $k=\frac{\log(1/\epsilon_0\delta)}{2\epsilon_0}$ and $\Vert\tda p_{3,1}^{(\tau)}\Vert_2 = \Vert\bfa v_1\Vert_2=1$, one immediately obtains that
    \begin{align*}  
    \tda p_{3,1}^{(\tau),\top}\bfa v_1\geq 1- \epsilon_0,\qquad\Vert\tda p_{3,1}^{(\tau)}-\bfa v_1\Vert_2 = \sqrt{2-2\bfa v_1^\top\tda p_{3,1}^{(\tau)}}=\sqrt{2\epsilon_0}.
    \end{align*}
    We also consider the approximation of the maximum eigenvalue. Note that using $\Vert\bfa v_1\Vert_2=1$, we have 
    \begin{align*}
       \Vert\bfa X\bfa X^\top\Vert_2&=\Vert\bfa X\bfa X^\top\bfa v_1\Vert_2=\lef\Vert\bfa X\bfa X^\top\tda p_{3,1}^{(\tau)}+\bfa X\bfa X^\top(\bfa v_1-\tda p_{3,1}^{(\tau)})\rig\Vert_2\\
       &\leq\Vert\bfa X\bfa X^\top\tda p_{3,1}^{(\tau)}\Vert_2+\Vert\bfa X\bfa X^\top(\bfa v_1-\tda p_{3,1}^{(\tau)})\Vert_2\\
       &\leq\Vert\bfa X\bfa X^\top\tda p_{3,1}^{(\tau)}\Vert_2+\Vert\bfa X\bfa X^\top\Vert_2\Vert\bfa v_1-\tda p_{3,1}^{(\tau)}\Vert_2.
   \end{align*}
   Similarly we can also derive that $\Vert\bfa X\bfa X^\top\Vert_2\geq \Vert\bfa X\bfa X^\top\tda p_{3,1}^{(\tau)}\Vert_2-\Vert\bfa X\bfa X^\top\Vert_2\Vert\bfa v_1-\tda p_{3,1}\Vert_2$. Then we show that
   \begin{align*}
       \Big|\Vert\bfa X\bfa X^\top\Vert_2-\Vert\bfa X\bfa X^\top\tda p_{3,1}^{(\tau)}\Vert_2\Big|\leq\Vert\bfa X\bfa X^\top\Vert_2\Vert\bfa v_1-\tda p_{3,1}^{(\tau)}\Vert_2\leq\sqrt{2\epsilon_0}\Vert\bfa X\bfa X^\top\Vert_2.
   \end{align*}
    \begin{center}
        \textbf{3. The Removal of Principal Eigenvectors.}
    \end{center}
    After $\tau$ iterates on the power method, we need to remove the principal term from the matrix $\bfa X\bfa X^\top$, achieved through two important steps: (1) The computation of the estimated eigenvalue $\Vert\bfa X\bfa X^\top\tda p_{3,1}\Vert_2$. (2) The construction of the low rank update $\tda p_{3,1}\tda p_{3,1}^\top$. For step (1), we consider the following construction:
    \begin{align*}
        &\bfa V_1^{rpe,1}=-\bfa V_2^{rpe,1}=\begin{bmatrix}
            \bfa 0_{(3d+1)\times(2d+1)}&\bfa 0&\bfa 0\\
            \bfa 0_{d\times(2d+1)}&I_d&\bfa 0\\
            \bfa 0&\bfa 0&\bfa 0
        \end{bmatrix},\quad\bfa Q_1^{rpe,1} =-\bfa Q_2^{rpe,1}= \begin{bmatrix}
            \bfa 0_{(d+1)\times(d+1)}&\bfa 0&\bfa 0\\
            \bfa 0_{d\times(d+1)}& I_d&\bfa 0\\
            \bfa 0&\bfa 0&\bfa 0
        \end{bmatrix},\\
        &\bfa K_1^{rpe,1}=\bfa K_2^{rpe,1} = \begin{bmatrix}
            \bfa 0_{(4d+1)\times(4d+1)}&\bfa 0&\bfa 0\\
            \bfa 0_{d\times(4d+1)}& I_d&\bfa 0\\
            \bfa 0&\bfa 0&\bfa 0
        \end{bmatrix}.
    \end{align*}
    Note that the above construction is similar to the first layer of the power method. Under this construction, we can show that
    \begin{align}\label{diffhrpe1}
        &\tda H^{rpe,1}=\tda H^{pow,2\tau+2}+\sum_{m\in\{1,2\}}\bfa V_m^{rpe,1}\sigma((\bfa Q_m^{rpe,1}\tda H^{pow,2\tau+2})^\top(\bfa K_m^{rpe,1}\tda H^{pow,2\tau+2})),\nnb\\
        &\Bigg\Vert\tda H^{rpe,1}-\ub{\begin{bmatrix}
            \bfa X\\
            \tda y\\
            \bfa X\bfa X^\top,\bfa 0\\
            \tda p_{2,1},\ldots,\tda p_{2,N}\\
            \tda p_{3,1},\ldots,\tda p_{3,N}\\
            \tda p_{3,1}^{(\tau)},\bfa 0\\
            \bfa X\bfa X^\top\tda p_{3,N}^{(\tau)},\bfa 0\\
            \tda p_{6,1},\ldots,\tda p_{6,N}\\
            \vdots\\
            \tda p_{\ell,1},\ldots,\tda p_{\ell,N}
        \end{bmatrix}}_{=:\bfa H^{rpe,1}}\Bigg\Vert_2\leq C\tau\epsilon\Vert\bfa X\bfa X^\top\Vert_2^2,\qquad\tda p_{5,i}=\bfa 0,\quad\forall i\in[N].
    \end{align}
    Then, we construct the next layer, using the notations in lemma \ref{reluapprox}, for $M\geq \Vert\bfa X\bfa X^\top\Vert_2^d\frac{C(d)}{\epsilon^2}$ for all $m\in[M]$ we have 
    \begin{align*}
       &\bfa V_m^{rpe,2}=\begin{bmatrix}
            \bfa 0_{d\times(4d+1)}&d_m\bfa I_d&\bfa 0\\
            \bfa 0&\bfa 0&\bfa 0
       \end{bmatrix},\qquad \bfa Q_m^{rpe,2} =\begin{bmatrix}
           \bfa 0_{d\times(2d+1)}&\bfa I_d&\bfa 0\\
           \bfa 0&\bfa 0&\bfa 0
       \end{bmatrix},\\
       &\bfa K_m^{rpe,2} =\begin{bmatrix}
           \bfa 0_{1\times(5d+1)}&\bfa b_m^\top&\bfa 0\\
           \vdots&&\\
           \bfa 0_{1\times(5d+1)}&\bfa b_m^\top&\bfa 0 \\
           \bfa 0_{(D-d)\times(5d+1)}&\bfa 0&\bfa 0
       \end{bmatrix}.
    \end{align*}
    Given the above construction, we subsequently show that
    \begin{align*}
       (\bfa Q_m^{rpe,2}\tda H^{rpe,1})^\top=\begin{bmatrix}
           I_{d\times d}&\bfa 0\\
           \bfa 0&\bfa 0
       \end{bmatrix},\qquad \bfa K_m^{rpe,2}\tda H^{rpe,1} = \begin{bmatrix}
            \bfa b_m^\top\bfa X\bfa X^\top\tda p_{3,1}^{(\tau)}&\bfa 0\\
            \vdots &\vdots\\
            \bfa b_m^\top\bfa X\bfa X^\top\tda p_{3,1}^{(\tau)}&\bfa 0\\
            \bfa 0_{(D-d)\times 1}&\bfa 0
        \end{bmatrix}.
    \end{align*}
    Hence, given the construction of $\bfa V_m^{rpe,2}$, we can show that $
        \tda H^{rpe,2} $ satisfies
        \begin{align*}
           \tda H^{rpe,2} &= \tda H^{rpe,1}+\sum_{m\in[M]}\bfa V_m^{rpe,2}\tda H^{rpe,1}\times\sigma\lef((\bfa K^{rpe,2}_m\tda H^{rpe,1})^\top(\bfa Q_m^{rpe}\tda H^{rpe,1})\rig)\\
           &=\ub{\bfa H^{rpe,1}+\sum_{m\in[M]}\bfa V_m^{rpe,2}\bfa H^{rpe,1}\times\sigma\lef((\bfa K_m^{rpe,2}\bfa H^{rpe,1})^\top(\bfa Q_m^{rpe,2}\bfa H^{rpe,1})\rig)}_{=:\wha H^{rpe,1}}\\
           &+\lef(\tda H^{rpe,1}-\bfa H^{rpe,1}\rig)+\sum_{m\in[M]}\bfa V_m^{rpe,2}\tda H^{rpe,1}\times\sigma\lef((\bfa K_m^{rpe,2}\tda H^{rpe,1})^\top\bfa Q_m^{rpe,2}\tda H^{rpe,1}\rig)\\
           &-\sum_{m\in[M]}\bfa V_m^{rpe,2}\tda H^{rpe,1}\times\sigma\lef((\bfa K_m^{rpe,2}\tda H^{rpe,1})^\top\bfa Q_m^{rpe,2}\tda H^{rpe,1}\rig).
        \end{align*}
    We note that by lemma \ref{reluapprox} we can show that
    \begin{align*}
        \Bigg\Vert\wha H^{rpe,1}-\begin{bmatrix}
            \bfa X\\
            \tda y\\
            \bfa X\bfa X^\top,\bfa 0\\
            \tda p_{2,1}\ldots,\tda p_{2,N}\\
            \tda p_{3,1}\ldots,\tda p_{3,N}\\
            \tda p_{3,1}^{(\tau)},\bfa 0\\
            \Vert\bfa X\bfa X^\top\tda p_{3,N}^{(\tau)}\Vert_2^{\frac{1}{2}}\tda p_{3,1}^{(\tau)},\bfa 0\\
            \vdots\\
            \tda p_{\ell,1},\ldots\tda p_{\ell, N}
        \end{bmatrix}\Bigg\Vert_2\leq C\tau\epsilon\Vert\bfa X\bfa X^\top\Vert_2^2.
    \end{align*}
    Then the rest of the proof focuses on showing that the rest of the terms are small. Note that using \eqref{diffhrpe1}, we show that
    \begin{align*}
        \lef\Vert\tda H^{rpe,1}-\bfa H^{rpe,1}\rig\Vert_2\leq\tau\epsilon\Vert\bfa X\bfa X^\top\Vert_2^2.
    \end{align*}
    And for the last term, we can show that
    \begin{align*}
        \Big\Vert\sum_{m\in[M]}\bfa V_m^{rpe,2}\tda H^{rpe,1}&\times\sigma\lef((\bfa K_m^{rpe,2}\tda H^{rpe,1})^\top(\bfa Q_m^{rpe,2}\tda H^{rpe,1})\rig)\\
        &-\sum_{m\in[M]}\bfa V_m^{rpe,2}\bfa H^{rpe,1}\times\sigma\lef((\bfa K_m^{rpe,2}\bfa H^{rpe,1})^\top(\bfa Q_m^{rpe,2}\bfa H^{rpe,1})\rig)\Big\Vert_2\\
        &\leq C\tau\epsilon\Vert\bfa X\bfa X^\top\Vert_2^2. 
    \end{align*}
    Collecting the above pieces, we finally show that
    \begin{align*}
        \Bigg\Vert\tda H^{rpe,2}-\begin{bmatrix}
            \bfa X\\
            \bfa X\bfa X^\top,\bfa 0\\
            \tda p_{2,1},\ldots,\tda p_{2,N}\\
            \tda p_{3,1},\ldots,\tda p_{3,N}\\
            \tda p_{3,1}^{(\tau)},\bfa 0\\
            \Vert\bfa X\bfa X^\top\tda p_{3,1}^{(\tau)}\Vert_2^{\frac{1}{2}}\tda p_{3,1}^{(\tau)},\bfa 0\\
            \tda p_{6,1},\ldots,\tda p_{6,N}\\
            \vdots\\
            \tda p_{\ell,1},\ldots,\tda p_{\ell,N}
        \end{bmatrix} \Bigg\Vert_2\leq C\tau\epsilon\Vert\bfa X\bfa X^\top\Vert_2^2.
    \end{align*}
    % \begin{align*}
    %     \Bigg\Vert\tda H^{rpe,2}-\begin{bmatrix}
    %         \bfa X\\
    %         \tda y\\
    %         \bfa X\bfa X^\top,\bfa 0\\
    %         \tda p_{2,1}\ldots,\tda p_{2,N}\\
    %         \tda p_{3,1}\ldots,\tda p_{3,N}\\
    %         \Vert\bfa X\bfa X^\top\tda p_{3,1}^{(\tau)}\Vert_2^{\frac{1}{2}}\tda p_{3,1}^{(\tau)},\bfa 0\\
    %         \bfa 0
    %     \end{bmatrix}\Bigg\Vert_2.
    % \end{align*}
    Then we construct another layer to remove the principal components from the matrix $\bfa X\bfa X^\top$, given by
    \begin{align*}
        &-\bfa V_1^{rpe,3}=\bfa V_2^{rpe,3}= \begin{bmatrix}
            \bfa 0_{(d+1)\times(4d+1)}&\bfa 0&\bfa 0\\
            \bfa 0& I_d&\bfa 0\\
            \bfa 0&\bfa 0&\bfa 0
        \end{bmatrix},\qquad\bfa Q_1^{rpe,3}=-\bfa Q_2^{rpe,3}=\begin{bmatrix}
            \bfa 0_{d\times(4d+1)}&I_d&\bfa 0\\
            \bfa 0&\bfa 0&\bfa 0
        \end{bmatrix},\\
        &\bfa K_1^{rpe,3}=\bfa K_2^{rpe,3}=\begin{bmatrix}
            \bfa 0_{d\times(4d+1)}& I_d&\bfa 0\\
            \bfa 0&\bfa 0&\bfa 0
        \end{bmatrix}.
    \end{align*}
    Then we can show that
    \begin{align*}
        (\bfa Q_1^{rpe,3}\tda H^{rpe,2})^\top =\begin{bmatrix}
            \Vert\bfa X\bfa X^\top\tda p_{3,1}^{(\tau)}\Vert_2^{\frac{1}{2}}\tda p_{3,1}^{(\tau),\top}&\bfa 0\\
            \bfa 0&\bfa 0
        \end{bmatrix},\qquad\bfa K_1^{rpe,3}\tda H^{rpe,2}=\begin{bmatrix}
            \bfa 0&\bfa 0\\
            I_d&\bfa 0\\
            \bfa 0&\bfa 0
        \end{bmatrix}.
    \end{align*}
    Then it is further noted that
    $
       -(\bfa Q_2^{rpe,3}\tda H^{rpe,2})^\top\bfa K_2^{rpe,3}\tda H^{rpe,2}= (\bfa Q_1^{rpe,3}\tda H^{rpe,2})^\top\bfa K_1^{rpe,3}\tda H^{rpe,2} 
    $ satisfies
    \begin{align*}
        \Bigg\Vert(\bfa Q_1^{rpe,3}\tda H^{rpe,2})^\top\bfa K_1^{rpe,3}\tda H^{rpe,2}-\begin{bmatrix}
            \Vert\bfa X\bfa X^\top\tda p_{3,1}^{(\tau)}\Vert_2^{\frac{1}{2}}\tda p_{3,1}^{(\tau),\top}&\bfa 0\\
            \bfa 0&\bfa 0
        \end{bmatrix}\Bigg\Vert_2\leq C\tau\epsilon\Vert\bfa X\bfa X^\top\Vert_2^2.
    \end{align*}
    And therefore, combining our construction for $\bfa V_m$, it is noted that
    \begin{align*}
        \Bigg\Vert\sum_{m=1}^2\bfa V_m\tda H^{rpe,2}\times\sigma((\bfa Q_1^{rpe,3}\tda H^{rpe,2})^\top\bfa K_1^{rpe,3}\tda H^{rpe,2})&-\begin{bmatrix}
            \bfa 0_{(d+1)\times N}\\
            -\Vert\bfa X\bfa X^\top\tda p_{3,1}^{(\tau)}\Vert_2\tda p_{3,1}^{(\tau)}\tda p_{3,1}^{(\tau),\top},\bfa 0\\
            \bfa 0
        \end{bmatrix}\Bigg\Vert_2\\
        &\leq C\tau\epsilon\Vert\bfa X\bfa X^\top\Vert_2^2.
    \end{align*}
    Therefore, we can further show that
    \begin{align*}
        \tda H^{rpe,3} =\tda H^{rpe,2}+ \sum_{m=1}^2\bfa V_{m}^{rpe,3}\tda H^{rpe,2}\times\sigma\lef((\bfa Q_m^{rpe,3}\tda H^{rpe,2})^\top\bfa K_m^{rpe,3}\tda H^{rpe,2}\rig)
    \end{align*}
    satisfies
    \begin{align*}
       \Bigg\Vert \tda H^{rpe,3}-\begin{bmatrix}
            \bfa X\\
            \bfa X\bfa X^\top-\Vert\bfa X\bfa X^\top\tda p_{3,1}^{(\tau)}\Vert_2\tda p_{3,1}^{(\tau)}\tda p_{3,1}^{(\tau),\top},\bfa 0\\
            \tda p_{2,1},\ldots,\tda p_{2,N}\\
            \tda p_{3,1},\ldots,\tda p_{3,N}\\
            \tda p_{3,1}^{(\tau)},\bfa 0\\
            \tda p_{5,1},\ldots,\tda p_{5,N}\\
            \vdots\\
            \tda p_{\ell,1},\ldots,\tda p_{\ell,N}
        \end{bmatrix}\Bigg\Vert_2\leq C\tau\epsilon\Vert\bfa X\bfa X^\top\Vert_2^2.
    \end{align*}
    And we can construct another layer to remove the term $\Vert\bfa X\bfa X^\top\tda p_{3,1}^{(\tau)}\Vert_2^{\frac{1}{2}}\tda p_{3,1}^{(\tau)}$, which is achieved by 
    \begin{align*}
        &-\bfa V_1^{rpe,4} = \bfa V_2^{rpe,4} =  \begin{bmatrix}
            \bfa 0_{(4d+1)\times(4d+1)}&\bfa 0&\bfa 0\\
            \bfa 0_{d\times(4d+1)}& I_d&\bfa 0\\
            \bfa 0&\bfa 0&\bfa 0
        \end{bmatrix},\\
        &\bfa Q_1^{rpe,4} = -\bfa Q_2^{rpe,4}=\begin{bmatrix}
            &\bfa 0_{(3d+1)\times D}&\\
            \bfa 0_{d\times (2d+1)}& I_d&\bfa 0\\
            &\bfa 0&
        \end{bmatrix},\\
        &\bfa K_1^{rpe,4}= \bfa K_2^{rpe,4} =\begin{bmatrix}
            &\bfa 0_{(3d+1)\times D}&\\
            \bfa 0_{d\times (2d+1)}& I_d&\bfa 0\\
            &\bfa 0&
        \end{bmatrix}.
    \end{align*}
    Using the above construction, we can further show that
    \begin{align*}
        \Bigg\Vert\tda H^{rpe,4}-\begin{bmatrix}
            \bfa X\\
            \bfa X\bfa X^\top-\Vert\bfa X\bfa X^\top\tda p_{3,1}^{(\tau)}\Vert_2\tda p_{3,1}^{(\tau)}\tda p_{3,1}^{(\tau),\top},\bfa 0\\
            \tda p_{2,1},\ldots,\tda p_{2,N}\\
            \tda p_{3,1},\ldots,\tda p_{3,N}\\
            \tda p_{3,1}^{(\tau)},\bfa 0\\
            \tda p_{5,1},\ldots,\tda p_{5,N}\\
            \vdots\\
            \tda p_{\ell,1},\ldots,\tda p_{\ell,N}
        \end{bmatrix}\Bigg\Vert_2\leq C\tau\epsilon\Vert\bfa X\bfa X^\top\Vert_2^2.
    \end{align*}
    Then we proceed to recover the rest of the $k$ principal eigenvectors using similar model architecture given by the ones used by the Power Iterations. For the computation over the $\tau$-th eigenvector, we denote $\tda H^{pow, \eta, 1}$ till $\tda H^{pow, \eta, \tau}$ to be the intermediate states corresponding to the $\eta$-th power iteration. We denote $\tda H^{rpe,\eta,\tau_0}$ to be the output of $\eta$-th removal of principal eigenvector layers for the $\tau$-th eigenvector. Furthermore, we iteratively define
    \begin{align*}
        \bfa A_1 =\bfa X\bfa X^\top-\Vert\bfa X\bfa X^\top\tda p_{3,1}^{(\tau)}\Vert_2\tda p_{3,1}^{(\tau)}\tda p_{3,1}^{(\tau),\top},\qquad \bfa A_{i+1}=\bfa A_{i}-\Vert\bfa A_i\tda p_{3,i}^{(\tau)}\Vert_2\tda p_{3,i}^{(\tau)}\tda p_{3,i}^{(\tau),\top},\qquad\forall i\in[k].
    \end{align*}
    Then, applying the subadditivity of the $2$-norm, we can show that
    \begin{align*}
        \Bigg\Vert\tda H^{rpe,4,k}-\begin{bmatrix}
            \bfa X\\
            \bfa A_{k+1},\bfa 0\\
            \tda p_{2,1},\ldots,\tda p_{2,N}\\
            \tda p_{3,1},\ldots,\tda p_{3,N}\\
            \tda p_{3,1}^{(\tau)},\bfa 0\\
            \tda p_{3,2}^{(\tau)},\bfa 0\\
            \vdots\\
            \tda p_{3,k}^{(\tau)},\bfa 0
        \end{bmatrix}\Bigg\Vert_2\leq C\tau k\epsilon\Vert\bfa X\bfa X^\top\Vert_2^2.
    \end{align*}
    For simplicity, we denote $\tda A=\begin{bmatrix}
        \bfa X\\
        \bfa A_{k+1},\bfa 0\\
        \tda p_{2,1},\ldots,\tda p_{2,N}\\
        \tda p_{3,1},\ldots,\tda p_{3,N}
    \end{bmatrix}$ and $\tda P=\begin{bmatrix}
        \tda p_{3,1}^{(\tau)}\\
        \tda p_{3,2}^{(\tau)}\\
        \vdots\\
        \tda p_{3,k}^{(\tau)}
    \end{bmatrix}$ from here. 
    \begin{center}
        \textbf{4. Finishing Up.}
    \end{center}
    The finishing-up phase considers constructing $\tda W_0$ and $\tda W_1$ that adjust the final output format. Our construction gives the following 
    \begin{align*}
        \tda W_0 = \begin{bmatrix}
            \bfa 0,I_{kd}
        \end{bmatrix},\qquad\tda W_1=\begin{bmatrix}
            1\\
            \bfa 0_{N-1}
        \end{bmatrix}.
    \end{align*}
    And we can show that
    \begin{align*}
        \Bigg\Vert\tda W_0\tda H^{rpe,4,k}\tda W_1-\begin{bmatrix}
            \tda p_{3,1}^{(\tau)}\\
            \tda p_{3,2}^{(\tau)}\\
            \vdots\\
            \tda p_{3,k}^{(\tau)}
        \end{bmatrix}\Bigg\Vert_2\leq C\tau k\epsilon\Vert\bfa X\bfa X^\top\Vert_2^2.
    \end{align*}
        We further use the result given by lemma \ref{lm3.2}, denote $a_{\eta}:=\lef\Vert\bfa v_{\eta}-\tda p_{3,\eta}^{(\tau)}\rig\Vert_2$, $\wh\lambda_{\eta}=\lef\Vert \bfa A_{\eta}\tda p_{3,\eta}^{(\tau)}\rig\Vert_2$, and $b_{\eta}:=|\lambda_{\eta}-\wh \lambda_{\eta}|$ for $\eta\in[k]$, we obtain that for all $\eta\geq 1$, given the number of iterations $\tau\geq C\frac{\log(1/\epsilon_0\delta)}{2\epsilon_0}$ where the constant value $C$ depends on $d$,
    \begin{align*}
    a_{\eta+1}\leq\frac{\max_{i\in[\eta]}b_{i}+\sum_{i=1}^{\eta}2\lambda_i a_i}{\Delta},\qquad b_{\eta+1}\leq\frac{2\lambda_{\eta+1}}{\Delta}\bl\max_{i\in[\eta]}b_{\eta}+\sum_{i=1}^{\eta} 2\lambda_ia_i\br+\lambda_{\eta+1}\sqrt{2\epsilon_0}.
    \end{align*}
    Further, note that the starting point is given by
    $a_{1}\leq \sqrt{2\epsilon_0}$, $b_1\leq\lambda_{1}\sqrt{2\epsilon_0}$. Introducing $A_{\eta}=\sum_{i=1}^{\eta} 2\lambda_ia_i$, we obtain that
$        A_{\eta+1} =\sum_{i=1}^{\eta+1}2\lambda_ia_i = A_{\eta}+2\lambda_{\eta+1}a_{\eta+1}$ which alternatively implies that
    \begin{align*}
        \frac{1}{2\lambda_{\eta+1}}(A_{\eta+1}-A_{\eta})\leq\frac{\max_{i\in[\eta]}b_i+A_{\eta}}{\Delta},\qquad b_{\eta+1}\leq\frac{2\lambda_{\eta+1}}{\Delta}\bl\max_{i\in[\eta]}b_{\eta}+A_{\eta}\br+\lambda_{\eta+1}\sqrt{2\epsilon_0}.
    \end{align*}
    We use the fact 
        $\frac{\lambda_{\eta}}{\Delta}>1$ for all $\eta\in[k]$
    to show the following
    \begin{align*}
A_{\eta+1}+\max_{i\in[\eta+1]}b_i\leq\frac{5\lambda_{\eta+1}}{\Delta}\lef(A_{\eta}+\max_{i\in[\eta]} b_i\rig)+\lambda_1\sqrt{2\epsilon_0}, \qquad A_1+b_1=2\lambda_1\sqrt{2\epsilon_0},
    \end{align*}
    which implies that
    \begin{align}\label{iterativeeq}
    A_{\eta+1}+\max_{i\in[\eta+1]}b_i+&\frac{\lambda_1\sqrt{2\epsilon_0}}{\frac{5\lambda_{1}}{\Delta}-1}\leq\frac{5\lambda_{1}}{\Delta}\bl A_{\eta}+\max_{i\in[\eta]}b_i +\frac{\lambda_1\sqrt{2\epsilon_0}}{\frac{5\lambda_{1}}{\Delta}-1}\br,\nnb\\
        A_{\eta+1}+\max_{i\in[\eta+1]}b_i+&\frac{\lambda_1\sqrt{2\epsilon_0}}{\frac{5\lambda_{1}}{\Delta}-1}\leq\bl A_1+b_1+\frac{\lambda_1\sqrt{2\epsilon_0}}{\frac{5\lambda_1}{\Delta}-1}\br\prod_{i=1}^{\eta}\bl\frac{5\lambda_{i+1}}{\Delta}\br\nnb\\
        &=\lambda_1\sqrt{2\epsilon_0}\bl 2+\frac{1}{\frac{5\lambda_1}{\Delta}-1} \br\prod_{i=1}^{\eta}\bl\frac{5\lambda_{i+1}}{\Delta}\br.
    \end{align}
    Therefore, applying the inequality given by \eqref{iterativeeq} we can show that, for $\eta\leq k$, we have for all $\eta\in[k-1]$,
    \begin{align*}
        a_{\eta+1}&\leq\frac{1}{\Delta}\bl \lambda_1\sqrt{2\epsilon_0}\bl 2+\frac{1}{\frac{5\lambda_1}{\Delta}-1}\br\prod_{i=1}^{\eta}\bl\frac{5\lambda_{i+1}}{\Delta}\br-\frac{\lambda_1\sqrt{2\epsilon_0}}{\frac{5\lambda_1}{\Delta}-1}\br,\\
        b_{\eta+1}&\leq\frac{2\lambda_{\eta}\lambda_1\sqrt{2\epsilon_0}}{\Delta}\bl 2+\frac{1}{\frac{5\lambda_1}{\Delta}-1}\br\prod_{i=1}^{\eta}\bl\frac{5\lambda_{i+1}}{\Delta}\br+\lambda_{\eta+1}\sqrt{2\epsilon_0}.
    \end{align*}
    Therefore collecting pieces, we conclude that there exists a transformer with a number of layers $2\tau+4k+1$ and a number of heads $M\leq \lambda_1^d\frac{C(d)}{\epsilon^2}$ such that
    the final output $\wha v_1,\ldots,\wha v_{k}$ given by the Transformer model satisfy $\forall \eta\in[k-1]$,
    \begin{align*}
        \lef\Vert\wha v_{\eta+1}-\bfa v_{\eta+1}\rig\Vert_2\leq C\tau \epsilon\lambda_1^2+\frac{1}{\Delta}\bl\lambda_1\sqrt{2\epsilon_0}\bl 2+\frac{1}{\frac{5\lambda_1}{\Delta}-1}\br\prod_{i=1}^{\eta}\bl\frac{5\lambda_{i+1}}{\Delta}\br-\frac{\lambda_1\sqrt{2\epsilon_0}}{\frac{5\lambda_1}{\Delta}-1}\br.
    \end{align*}
    And the rest of the result directly follows.
    \end{proof}

    \subsection{Proof of Lemma \ref{lm3.1}}
        \begin{proof}
        To prove the above result, we consider two events
$            A_1 =\lef\{\Vert \bfa y\Vert_2\geq \sqrt{\frac{1}{\epsilon}}\rig\}$, $A_2=\lef\{|\bfa y^\top\bfa v|\leq \sqrt{\epsilon}\rig\}$, then we can show that
        \begin{align*}
           \lef\{ |\bfa v^\top\bfa x|\leq\frac{1}{\sqrt\epsilon}\rig\}\subset A_1\cup A_2\quad\Rightarrow\quad\bb P\bl|\bfa v^\top\bfa x|\leq\sqrt{\epsilon}\br\leq\bb P(A_1)+\bb P(A_2).
        \end{align*}
        And we use the tail bound for Chi-square given by \citep{laurent2000adaptive} to obtain that as $\epsilon<d^{-1}$,
        \begin{align*}
            \bb P(A_1) = \bb P\lef(\Vert\bfa y\Vert_2^2\geq \epsilon^{-1}\rig)\leq\exp\lef(-C\epsilon^{-1}\rig).
        \end{align*}
        Similarly, consider the event $A_2$, note that $\bfa y^\top\bfa v\sim N(0,1)$, we use the cdf of the folded normal distribution to obtain that
        \begin{align*}
            \bb P\lef(A_2\rig) = \bb P\lef(\lef|\bfa v^\top\bfa y\rig|\leq\sqrt{\epsilon}\rig)=erf\lef(\frac{\sqrt\epsilon}{\sqrt 2}\rig)=\frac{2}{\sqrt{\pi}}\bl \sqrt\epsilon-\frac{(\sqrt\epsilon)^3}{3}+\frac{(\sqrt\epsilon)^5}{10}-\frac{(\sqrt\epsilon)^7}{42}\br\leq\frac{\sqrt\epsilon}{\sqrt{\pi}}.
        \end{align*}
        Then we obtain that 
        \begin{align*}
            \bb P\bl|\bfa v^\top\bfa x|\leq\sqrt\epsilon\br\leq \frac{\sqrt\epsilon}{\sqrt{\pi}}+\exp\lef(-C\epsilon^{-1}\rig).
        \end{align*}
        Consider in total of $k$ independent random vectors $\bfa x_1,\ldots,\bfa x_k$, and arbitrary $k$ vectors $\bfa v_1,\ldots,\bfa v_k$, we can show that
        \begin{align*}
            \bb P\bl\exists i\text{ such that }&\bfa x_i^\top\bfa v_i\leq\epsilon\br\leq k\bb P\bl\bfa x_1^\top\bfa v_1\leq\epsilon\br\leq\frac{k\sqrt{\epsilon}}{\sqrt{\pi}}+k\exp(-C\epsilon^{-1}).
        \end{align*}
    \end{proof}

\subsection{Proof of Lemma \ref{lm3.2}}

\begin{lemma}\label{lm3.2}
Assume that the correlation matrix $\bfa X\bfa X^\top$ has eigenvalues $\lambda_1>\lambda_2>\ldots>\lambda_{k}$. Assume that the eigenvectors are given by $\bfa v_1,\bfa v_2,\ldots,\bfa v_n$ and the eigenvalues satisfy $\inf_{i\neq j}|\lambda_i-\lambda_j|=\Delta$. Then, given that the estimate for the first $\tau$ eigenvectors satisfy
$    \bfa v_i^\top\wha v_i\geq 1-\epsilon_i $ and the eigenvalues satisfy $|\lambda_i-\wh\lambda_i|\leq\delta_i$, the principal eigenvector of $\bfa X\bfa X^\top-\sum_{i=1}^{\tau}\wh\lambda_i\wha v_i\wha v_i^\top$ denoted by $\tda v_{\tau+1}$ satisfies
\begin{align*}
    \Vert\tda v_{\tau+1}-\bfa v_{\tau+1}\Vert_2 \leq\frac{\max_{i\in[\tau]}\delta_i+\sum_{i=1}^\tau\sqrt8\lambda_i\sqrt{\epsilon_i}}{\Delta}.
\end{align*}
Alternatively, we can also show that the eigenvector $\wha v_{\tau+1}$ returned by power method with $k= \frac{\log(1/\epsilon_0\delta)}{2\epsilon_0}$ that is initialized by  satisfies
\begin{align*}
    \wha v_{\tau+1}^\top\bfa v_{\tau+1} \geq 1-\epsilon_{\tau+1}:=  1-\frac{1}{2}\Big(\frac{\max_{i\in[\tau]}\delta_i+\sum_{i=1}^{\tau}\sqrt{8}\lambda_i\sqrt{\epsilon_i}}{\Delta}+\sqrt{2\epsilon_0}\Big)^2,
\end{align*}
\end{lemma}
    \begin{proof}
    Our proof is given by inductive arguments. Consider our obtained estimates $\{\wha v_i\}_{i\in[k]}$ for the eigenvectors $\{\bfa v_i\}_{i\in[k]}$ satisfy
    \begin{align*}
        \bfa v_i^\top\wha v_i\geq 1-\epsilon_i\qquad \forall i\in[\tau],\qquad|\lambda_i-\wh\lambda_i|\leq\delta_i.
    \end{align*}
    We note that for the eigenvectors, we have for a vector $\bfa v_0$,
    \begin{align*}
        \Vert\bfa v_i\bfa v_i^\top-\wha v_i\wha v_i^\top\Vert_2&=\sup_{\bfa v_0\in\bb S^{d-1}}\bfa v_0^\top(\bfa v_i\bfa v_i^\top-\wha v_i\wha v_i^\top)\bfa v_0=\sup_{\bfa v_0\in\bb S^{d-1}}(\bfa v_0^\top\bfa v_i)^2-(\bfa v_0^\top\wha v_i)^2\\
        &=\sup_{\bfa v_0\in\bb S^{d-1}}(\bfa v_0^\top(\bfa v_i-\wha v_i))(\bfa v_0^\top(\bfa v_i+\wha v_i^\top))\\
        &\leq 2\Vert\bfa v_i-\wha v_i\Vert_2=2\sqrt{\Vert\bfa v_i-\wha v_i\Vert_2^2} = 2\sqrt{\Vert\bfa v_i\Vert_2^2+\Vert\wha v_i\Vert_2^2-2\bfa v_i^\top\wha v_i} = 2\sqrt{2\epsilon_i}.
    \end{align*}
    Then, we can show by the subadditivity of the spectral norm,
    \begin{align*}
        \Big\Vert\bfa X\bfa X^\top-\sum_{i=1}^\tau\wh\lambda_i\wha v_i\wha v_i^\top\Big\Vert_2&=\Big\Vert\sum_{i=1}^k\lambda_i\bfa v_i\bfa v_i^\top-\sum_{i=1}^{\tau}\wh\lambda_i\wha v_i\wha v_i^\top\Big\Vert_2\\
        &\leq\Big\Vert\sum_{i=1}^k\lambda_i\bfa v_i\bfa v_i^\top-\sum_{i=1}^\tau\lambda_i\wha v_i\wha v_i^\top\Big\Vert_2+\Big\Vert\sum_{i=1}^{\tau}\delta_i\wha v_i\wha v_i^\top\Big\Vert_2\\
        &\leq\Big\Vert\sum_{i=\tau+1}^k\lambda_i\bfa v_i\bfa v_i^\top\Big\Vert_2+\Big\Vert\sum_{i=1}^\tau\delta_i\wha v_i\wha v_i^\top\Big\Vert_2+\Big\Vert\sum_{i=1}^{\tau}\lambda_i\bfa v_i\bfa v_i^\top-\sum_{i=1}^{\tau}\lambda_i\wha v_i\wha v_i^\top \Big\Vert_2\\
        &\leq\lambda_{\tau+1} + \max_{i\in[\tau]}\delta_i+ \Big\Vert\sum_{i=1}^{\tau}\lambda_i(\bfa v_i\bfa v_i^\top-\wha v_i\wha v_i^\top)\Big\Vert_2 \\
        &\leq\lambda_{\tau+1}+\max_{i\in[\tau]}\delta_i+\sum_{i=1}^{\tau}\lambda_i\Big\Vert\bfa v\bfa v_i^\top-\wha v_i\wha v_i^\top\Big\Vert_2\\
        &\leq\lambda_{\tau+1}+\max_{i\in[\tau]}\delta_i+\sum_{i=1}^{\tau}\sqrt{8}\lambda_i\sqrt{\epsilon_i}.
    \end{align*}
    By a similar argument, we can also show that
    \begin{align*}
        \Big\Vert\bfa X\bfa X^\top-\sum_{i=1}^{\tau}\wh\lambda_i\wha v_i\wha v_i^\top\Big\Vert_2\geq\lambda_{\tau+1}-\max_{i\in[\tau]}\delta_i-\sum_{i=1}^{\tau}\sqrt 8\lambda_i\sqrt{\epsilon_i}.
    \end{align*}
    To study the convergence of the eigenvectors, we notice that by Davis-Kahan Theorem by \citep{yu2015useful} we can show that the principal eigenvector $\tda v_{\tau+1}=\argmax_{\bfa v\in\bb S^{d-1}}$ satisfies
    \begin{align*}
        \Vert\tda v_{\tau+1}-\bfa v_{\tau+1}\Vert_2\leq\frac{\Big|\lef\Vert\bfa X\bfa X^\top-\sum_{i}^{\tau}\wh\lambda_i\wha v_i\wha v_i^\top\rig\Vert-\lambda_{\tau+1}\Big|}{\max\{|\lambda_{\tau+1}-\lambda_{\tau}|,|\lambda_{\tau-1}-\lambda_{\tau}|\}}\leq\frac{\max_{i\in[\tau]}\delta_i+\sum_{i=1}^{\tau}\sqrt 8\lambda_i\sqrt{\epsilon_i}}{\Delta}.
    \end{align*}
    % Alternatively, we can show that
    % \begin{align*}
    %     \tda v_{\tau+1}^\top\bfa v_{\tau+1}=\frac{1}{2}\lef(2-\Vert\tda v_{\tau+1}-\bfa v_{\tau+1}\Vert_2^2\rig)=1-\Big(\frac{\max_{i\in[\tau]}\delta_i+\sum_{i=1}^{\tau}\sqrt 8\lambda_i\sqrt{\epsilon_i}}{2\Delta}\Big)^2.
    % \end{align*}
    Considering the eigenvector returned by the power method, we can show by the subadditivity of $L_2$ norm, we obtain that $\Vert\wha v_{\tau+1}-\bfa v_{\tau+1}\Vert_2\leq \Vert\tda v_{\tau+1}-\wha v_{\tau+1}\Vert_2+\Vert\tda v_{\tau+1}-\bfa v_{\tau+1}\Vert_2\leq \frac{\max_{i\in[\tau]}\delta_i+\sum_{i=1}^{\tau}\sqrt{8}\lambda_i\sqrt{\epsilon_i}}{\Delta}+\sqrt{2\epsilon_0}$
    \begin{align*}
        \wha v_{\tau+1}^\top\bfa v_{\tau+1}&=\frac{1}{2}\lef(2-\Vert\bfa v_{\tau+1}-\wha v_{\tau+1}\Vert_2^2\rig)\geq\frac{1}{2}\lef(2-\lef(\Vert\tda v_{\tau+1}-\wha v_{\tau+1}\Vert_2+\Vert\bfa v_{\tau+1}-\tda v_{\tau+1}\Vert_2\rig)^2\rig)\\
        &=1-\frac{1}{2}\Big(\frac{\max_{i\in[\tau]}\delta_i+\sum_{i=1}^{\tau}\sqrt{8}\lambda_i\sqrt{\epsilon_i}}{\Delta}+\sqrt{2\epsilon_0}\Big)^2.
    \end{align*}
    Moreover, considering the estimate of the eigenvalue, we have 
    \begin{align*}
        &\Big\Vert\Big(\bfa X\bfa X^\top -\sum_{i=1}^{\tau}\wh\lambda_i\wha v_i\wha v_i^\top\Big)\wha v_{\tau+1}\Big\Vert_2\\
        &\leq\Big\Vert\Big(\bfa X\bfa X^\top-\sum_{i=1}^{\tau}\lambda_i\bfa v_i\bfa v_i^\top\Big)\wha v_{\tau+1}\Big\Vert_2+\Big\Vert\sum_{i=1}^\tau\lambda_i\bfa v_i\bfa v_i^\top-\sum_{i=1}^{\tau}\wh\lambda_i\wha v_i\wha v_i^\top \Big\Vert_2\\
        &\leq \Big\Vert\Big(\bfa X\bfa X^\top-\sum_{i=1}^{\tau}\lambda_i\bfa v_i\bfa v_i^\top\Big)\wha v_{\tau+1} \Big\Vert_2+\Big\Vert\sum_{i=1}^{\tau}\lambda_i\bfa v_i\bfa v_i^\top-\sum_{i=1}^{\tau}\lambda_i\wha v_i\wha v_i^\top\Big\Vert_2+\Big\Vert \sum_{i=1}^{\tau}\lef(\lambda_i-\wh\lambda_i\rig)\wha v_i\wha v_i^\top\Big\Vert_2\\
        &\leq\Big\Vert\Big(\bfa X\bfa X^\top-\sum_{i=1}^{\tau}\lambda_i\bfa v_i\bfa v_i^\top\Big)\bfa v_{\tau+1}\Big\Vert_2+\Big\Vert\Big(\bfa X\bfa X^\top-\sum_{i=1}^{\tau}\lambda_i\bfa v_i\bfa v_i^\top\Big)\Big\Vert_2\Vert\wha v_{\tau+1}-\bfa v_{\tau+1}\Vert_2\\
        &+\max_{i\in[\tau]}\delta_i+\sum_{i=1}^{\tau}\sqrt{8}\lambda_i\sqrt{\epsilon_i}\\
        &=\lambda_{\tau+1}+\lambda_{\tau+1}\Big(\frac{\max_{i\in[\tau]}\delta_i+\sum_{i=1}^{\tau}\sqrt 8\lambda_i\sqrt{\epsilon_i}}{\Delta}+\sqrt{2\epsilon_0}\Big)+\max_{i\in[\tau]}\delta_i+\sum_{i=1}^{\tau}\sqrt 8\lambda_i\sqrt{\epsilon_i}.
    \end{align*}
    Therefore, by similar arguments, we can show that
    \begin{align*}
        \Big|\Big\Vert \Big(\bfa X\bfa X^\top-\sum_{i=1}^{\tau}\wh\lambda_i\wha v_i\wha v_i^\top\Big)\wha v_{\tau+1}\Big\Vert_2-\lambda_{\tau+1}\Big|\leq \frac{2\lambda_{\tau+1}}{\Delta}\Big(\max_{i\in[\tau]}\delta_i+\sum_{i=1}^{\tau}\sqrt{8}\lambda_i\sqrt{\epsilon_i}\Big)+\lambda_{\tau+1}\sqrt{2\epsilon_0}.
    \end{align*}
\end{proof}

\subsection{Proof of Lemma \ref{reluapprox}}

    \begin{lemma}[Approximation of norm by sum of Relu activations by Transformer networks]\label{reluapprox}
    Assume that there exists a constant $C$ with $\Vert\bfa v\Vert_2\leq C$. There exists a multi-head Relu attention layer with a number of heads $M>\lef(\frac{\overline R}{\underline R}\rig)^d\frac{C(d)}{\epsilon^2}\log(1+C/\epsilon)$ such that there exists $\{\bfa a_m\}_{m\in[M]}\subset\bb S^{N-1}$ and $\{c_m\}_{m\in[M]}\subset\bb R$ where for all $\bfa v$ with $\overline R\geq\Vert\bfa v\Vert_2\geq \underline R$, we have 
    \begin{align*}
        \bigg|\sum_{m=1}^Mc_m\sigma(\bfa a_m^\top\bfa v)-\frac{1}{\Vert\bfa v\Vert_2}+1\bigg|\leq\epsilon.
    \end{align*}
    Similarly, there exists a multi-head Relu attention layer with number of heads $M\leq\overline R^{\frac{d}{2}}\frac{C(d)}{\epsilon^2}\log\lef(1+C/\epsilon\rig)$, a set of vectors $\{\bfa b_m\}_{m\in[M]}\subset\bb S^{N-1}$ and $\{d_m\}_{m\in[M]}\subset\bb R$ such that 
    \begin{align*}
        \Big|\sum_{m=1}^Md_m\sigma(\bfa b_m^\top\bfa v)-\Vert\bfa v\Vert_2^{1/2}+1\Big|\leq\epsilon.
    \end{align*}
\end{lemma}


\begin{proof}
    Consider a set $\msf C^d(\overline R) :=\msf B^d_{\infty}(\overline R)\setminus\msf B_{2}^d(\underbar R)$, then it is not hard to check that given $\Vert\bfa v\Vert_2>C$ with some $C(d)>0$ depending on $d$ such that we have
    \begin{align*}
        \sup_{\bfa v\in\msf C^d(\overline R)}\pta_{v_{j_1},\ldots,v_{j_i}\in[d]}\bl\frac{1}{\Vert\bfa v\Vert_2}\br\leq\frac{C(d)}{\Vert\bfa v\Vert_2^d}\leq\frac{C(d)}{\underline R^d}.
    \end{align*}
    Therefore, consider the definition \ref{def6}, we have $C_{\ell}=\lef(\frac{\overline R}{\underline R}\rig)^d C(d)$.  Note that by proposition A.1 in \citep{bai2024transformers} shows that for a function that is $(R,C_{\ell})$ smooth with $R\geq 1$ is $(\epsilon_{approx}, R, M,C)$ approximable with $M\leq C(d)C_{\ell}\log(1+C_{\ell}/\epsilon_{approx})/\epsilon_{approx}^2$, we complete the proof. 

    Then we consider the function $\Vert\bfa v\Vert_2^{\frac{1}{2}}$, note that
    \begin{align*}
        \sup_{\bfa v\in\msf C^d(\overline R)}\pta_{v_{j_1},\ldots,v_{j_i}\in[d]}\Vert\bfa v\Vert_2^{\frac{1}{2}}\leq C\Vert\bfa v\Vert_2^{-\frac{1}{2}}\leq C\overline R^{-\frac{1}{2}}.
    \end{align*}
    And the rest of the proof follows similarly to the previous step.
\end{proof}

\clearpage

\section{Experimental Details}\label{sec:exp_details}

\subsection{Setup.}
We run all our experiments on RTX 2080 Ti GPUs.
We use PyTorch to construct our models and training process.
We use sklearn for data generation.
A training process with 2k steps roughly takes $0.5$ hours.

% Finally, we test our framework on two real-world datasets instead of the synthetic data generated from a multivariate Gaussian distribution.
% We train the model to predict top-$5$ eigenvalues and top-$1 \sim 5$ eigenvectors, with the former using a small model and the latter using a large model.


\subsection{Data.}\label{appendix:data}
\paragraph{Synthetic Dataset.}
For each $\bfa X_i \in \R^D$, we sample $Z_i \sim N(0, I) \in \R^D$.
We then form $Z = [Z_1, \cdots, Z_N]$ and transform it using an invertible matrix $L \sim N(0, I) \in \R ^{D \times D}$, yielding the desired training sample $\bfa X$.
To speed up the training process, we set $N < D$ in all our experiment settings.
With this design, the rank of the covariance matrix  $\bfa X^T X$ is at most $N$, meaning there are at least $D-N$ zero eigenvalues.
The eigenvectors corresponding to these zero eigenvalues are less meaningful.
Thus, to ensure predictions focus on meaningful eigenvectors, we increase $N$ to $10$ when predicting multiple eigenvectors.
We also adjust the data generation process to ensure the magnitude of eigenvalue across different $D$ to be at a similar level.

\paragraph{Real-world Dataset.}
For both the MNIST and FMNIST, we first normalize the images to zero mean.
Next, we perform SVD to extract the top-$D$ principal components and project the data onto these components, reducing feature dimension to $D = 10, 20$, and use $N = 10, 50$ for eigenvalue and eigenvector prediction respectively.
Last, we rescale the resulting matrix to ensure its magnitude is roughly the same level as training data (transformers are trained on synthetic data).
The rescaling process is critical to transformers as some images after SVD contain entries as large as $7e3$.
This will largely degrade the transformer's performance as it changes the input domain by a large margin.

% We start by verifying the simpler case with $N=5$ and predicting top-1 eigenvalue and eigenvector, with $D = 5, 10, 30, 50$.
% We then proceed to the harder case with $N=10$ and predict the top-$5$ eigenvalues and eigenvectors.



% and use Adam optimizer with batch size $64$ and learning rate $= 0.001$.
% If not specified, all the experiments is trained for 20k steps, except the multiple eigenvectors prediction experiment in section \ref{sec:eigvec_pred}.


\subsection{Hyperparameters.}
We list the hyperparameters in our experiments as below (table~\ref{table:data-param}).
We separate the hyperparameters used in predicting (1) eigenvalues and single eigenvectors, and (2) multiple eigenvectors.

\begin{table}[h]
        \centering
        \caption{Hyperparameters for Eigenvalue and Eigenvector Prediction.
        }
        % \resizebox{ \textwidth}{!}{  
        \begin{tabular}{l*{4}{c}}
        \toprule
            \bf{parameter} & $N=5$ & $N=10$  & $N=20$ \\ 
            \midrule
             steps (eigenvalue) & 20k & 20k & 20k \\
             steps (eigenvector) & 20k & 20k & 60k \\
            learning rate  & $1\text{e-}3$ & $5\text{e-}3$ & $5\text{e-}3$ \\
            Optimizer  & Adam & Adam & Adam \\
            batch size  & $64$ & $64$ & $64$  \\
            number of layers & $3$ & $3$ & $3$  \\
            hidden dimension & $64$ & $64$& $64$ \\
            number of heads  & $2$ & $2$& $2$ \\
            \bottomrule
        \end{tabular}
        \label{table:data-param}
    \end{table} 


\subsection{Additional Experimental Results}\label{sec:add_exp}

% \paragraph{Evidence of the challenge of predicting high-order eigenvalue.} xxx.

% \paragraph{Convergence Results.}
% We include the training loss curve of all the experiments using synthetic datasets in the figure \ref{fig:loss}.
% All the loss is 

\begin{figure}[!h]
    \centering
    \minipage{0.3\textwidth}
        \includegraphics[width=\linewidth]{charts/eigenvalue_vary_d_train_loss.pdf}
    \endminipage\hfill
    \minipage{0.3\textwidth}
        \includegraphics[width=\linewidth]{charts/eigenvalue_vary_layer_train_loss.pdf}
    \endminipage\hfill
    \minipage{0.3\textwidth}
        \includegraphics[width=\linewidth]{charts/eigenvector_vary_d_train_loss.pdf}
    \endminipage\vspace{1em}
    \minipage{0.45\textwidth}
        \includegraphics[width=\linewidth]{charts/eigenvector_vary_layer_train_loss.pdf}
    \endminipage\hfill
    \minipage{0.45\textwidth}
        \includegraphics[width=\linewidth]{charts/eigenvector_vary_k_train_loss.pdf}
    \endminipage
    \caption{\textbf{Convergence Results on Eigenvalue, Eigenvector Prediction with Different Parameters.}
    \emph{(1) Top left: Loss curve on eigenvalue prediction with different size of $D$}
    \emph{(2) Top middle: Loss curve on eigenvalue prediction with different number of layers}
    \emph{(3) Top right: Loss curve on eigenvector prediction with different size of $D$}
    \emph{(4) Bottom left: Top right: Loss curve on eigenvector prediction with different number of layers}
    \emph{(5) Bottom left: Loss curve on eigenvector prediction with different number of $k_{\text{train}}$}
    For (1), we observe that smaller $D$ is easier for transformers as they present lower losses.
    For (2), we see that with more layers, transformers are also capable of predicting eigenvalues more accurately.
    For (3), transformers also predict eigenvectors better when $D$ is small.
    For (4), similar to (2), transformers with more layers show improved performance.
    For (5), we want to highlight that the loss value is mainly affected by the fact that predicting 3rd or 4th eigenvectors is significantly harder, which contributes to a higher loss value.
    }
    \label{fig:loss}
\end{figure}

% \begin{figure}[!h]
% % \vspace{-1em}
% \minipage{0.5\textwidth}
% \minipage{0.5\textwidth}
% \includegraphics[width=\linewidth]{charts/relu_softmax_loss_d_5_plot.pdf}
% \endminipage\hfill
% \minipage{0.5\textwidth}
% \includegraphics[width=\textwidth]{charts/relu_softmax_loss_d_10_plot.pdf}
% \endminipage
% \endminipage\hfill
% \end{figure}


\begin{figure}[!h]
    \centering
    \begin{minipage}{0.45\textwidth}
        \centering
        \includegraphics[width=\linewidth]{charts/relu_softmax_loss_d_5_plot.pdf}
    \end{minipage}
    \hfill % Adds space between the two minipages
    \begin{minipage}{0.45\textwidth}
        \centering
        \includegraphics[width=\linewidth]{charts/relu_softmax_loss_d_10_plot.pdf}
    \end{minipage}
    \caption{\textbf{Loss Curve Comparison between Softmax and ReLU Transformers (Top-1 Eigenvector Prediction).}
    \emph{Left: $D=5$ }\emph{Right: $D=10$}
    We use a 3-layer, 2-head, 64 hidden dimension transformer to predict the top-1 eigenvector across all experiments in this figure. An explanation for the superior performance of ReLU transformers is that the normalizing behavior of Softmax can potentially hinder the PCA process.
    }
    \label{fig:relu_softmax1}
\end{figure}



\begin{figure}[!h]
    \centering
    \begin{minipage}{0.45\textwidth}
        \centering
        \includegraphics[width=\linewidth]{charts/relu_softmax_loss_d_30.pdf}
    \end{minipage}
    \hfill % Adds space between the two minipages
    \begin{minipage}{0.45\textwidth}
        \centering
        \includegraphics[width=\linewidth]{charts/relu_softmax_loss_d_50_plot.pdf}
    \end{minipage}
    
    \caption{\textbf{Loss Curve Comparison between Softmax and ReLU Transformers (Top-1 Eigenvector Prediction).}
    \emph{Left: $D=30$ }\emph{Right: $D=50$}
    We use a 3-layer, 2-head, 64 hidden dimension transformer to predict the top-1 eigenvector across all experiments in this figure. We also observe that the performance gap enlarges as $D$ increases, likely because the difference between eigenvectors becomes larger with increasing $D$, making the normalizing nature of Softmax unsuitable for PCA.
    }
    \label{fig:relu_softmax2}
\end{figure}



% \subsubsection{}

% \end{appendices}

\end{document}


% This document was modified from the file originally made available by
% Pat Langley and Andrea Danyluk for ICML-2K. This version was created
% by Iain Murray in 2018, and modified by Alexandre Bouchard in
% 2019 and 2021 and by Csaba Szepesvari, Gang Niu and Sivan Sabato in 2022.
% Modified again in 2023 and 2024 by Sivan Sabato and Jonathan Scarlett.
% Previous contributors include Dan Roy, Lise Getoor and Tobias
% Scheffer, which was slightly modified from the 2010 version by
% Thorsten Joachims & Johannes Fuernkranz, slightly modified from the
% 2009 version by Kiri Wagstaff and Sam Roweis's 2008 version, which is
% slightly modified from Prasad Tadepalli's 2007 version which is a
% lightly changed version of the previous year's version by Andrew
% Moore, which was in turn edited from those of Kristian Kersting and
% Codrina Lauth. Alex Smola contributed to the algorithmic style files.
