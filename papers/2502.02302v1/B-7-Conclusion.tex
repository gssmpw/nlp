\section{Conclusions}
\label{sec:Conclude}



% In this paper, we explored a novel problem of online learning 
% in variable feature space with mixed data.
% %
% The challenge of this problem lies in the establishing  
% of feature-wise relationship while the multivariate joint distribution of mixed data 
% (\eg continuous vs discrete) is difficult to be modeled.
% %
% We counter this challenge by proposing the \alg\ approach 
% that exploits Gaussian copula to model the mixed data 
% with a latent space consisting of normal variables.
% %
% A learner trained on this latent space enjoys 1)
% a complete feature information provided by the missing feature reconstruction 
% and 2) a fast convergence rate since optimizing over 
% the continuous latent variables renders gradients in a fine-level of granularity.
% %
% A theoretical study demonstrated the performance advantages 
% of our approach, and extensive empirical results further
% substantiated that.

The message-passing framework leverages the propagation characteristic of graph-structured data to enrich the information of each node, which makes it more suitable for node representation learning than traditional methods. 
%
In this paper, we have developed a novel message-passing based model, namely Edge-empowered Graph Feature Preference Learning (\alg), by integrating the edge representation learning into the messagepassing model, which can help refine the information in finer granularity for nodes. 
%
By doing this, nodes can receive more useful information, and thus promote the node representation learning process. 
Extensive experiments are conducted in both of node classification and node clustering on four datasets. 
%
And the results demonstrate the superior performance of node representations learned by \alg\, and the visualization results further reflect that the learned distribution can be better distributed in latent space. 
%
% In addition, the explainable case study show that the relation representation can refine the sent information for the target node by emphasizing the related features as well as reducing the unrelated features.

% \clearpage
