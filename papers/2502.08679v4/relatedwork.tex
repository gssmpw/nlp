\section{Related Work}
\label{sec:related}

Chen et al. \cite{chen2023overkill} distinguished between feature-space drift and data-space drift in malware detectors, highlighting the predominant influence of data-space drift on model degradation over time. Their findings underscored the necessity for further exploration into the implications of feature-space updates, particularly in the context of Android malware datasets like AndroZoo and EMBER.

Jameel et al. \cite{jameel2020critical}  conducted a critical review of concept drift's adverse effects on machine learning classification models. They proposed the ACNNELM model as optimal for Big Data stream classification but noted the absence of critical parameters for advanced ML models like deep learning. The review also highlighted the lack of a matrix model to measure adaptability factors, suggesting avenues for future research in model evaluation and optimization.

Hashmani et al. \cite{hashmani2020concept} presented a systematic literature review on concept drift evolution in machine learning approaches. Their comprehensive synthesis and categorization of existing research provided valuable insights into the state of the art in handling concept drift. However, the paper itself did not contribute new methodologies, serving primarily as a reference for researchers seeking to understand current trends and challenges in the field.

Lu et al. \cite{lu2018learning} conducted an extensive review focusing on concept drift in machine learning, covering detection, understanding, and adaptation strategies across numerous studies. While offering valuable insights into the breadth of research in this area, the paper did not introduce novel methodologies, functioning primarily as a compilation and analysis of existing approaches.

Farid et al. \cite{farid2013adaptive} proposed an adaptive ensemble classifier for mining concept drifting data streams. Their methodology addressed the challenge of concept drift by leveraging an ensemble approach, demonstrating promise for real-world applications with evolving data streams. However, the effectiveness of their approach was contingent upon the selection of appropriate base classifiers, highlighting a potential area for improvement in future research.