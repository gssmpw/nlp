\section{Conclusion}

In this paper, we address the issues caused by the unclear definition of successful jailbreak attacks and incomplete evaluation elements, which hinder the comparison across methods and lead to an overestimation of jailbreak attack capabilities. We propose \bench, a benchmark for evaluating jailbreak methods that includes a reconstructed harmful question dataset and a scoring system. We examine the shortcomings of existing work and provide detailed scoring rules for all harmful questions by introducing guidelines. We design a guidelines-based LLM scoring system that is fine-grained and stable, significantly reducing the reliance on the inherent values of evaluator LLMs. This reduces the threshold for the jailbreak attack evaluation task to require only basic reading and extraction abilities of contextual information, greatly reducing the cost of jailbreak attack evaluation.\\
\textbf{Limitations}. Currently, \bench~evaluates only 8 jailbreak methods due to cost and time constraints. This limited scope may not fully capture the evolving landscape of jailbreak attacks, potentially leaving out newly emerging or more sophisticated techniques. We promise to continue using \bench~to evaluate more jailbreak methods, providing reliable references for subsequent jailbreak attack research.

In addition, due to the lack of reasonable scoring principles, we do not evaluate the performance of some transferable jailbreak methods. However, we are aware that this is also an important attribution of these methods and should be evaluated in future research.\\
\textbf{Future Works}. To enhance the completeness and utility of \bench~, we plan to expand the benchmark by incorporating additional jailbreak techniques. We also aim to release an open-source toolkit that allows researchers to test their own attack methods against \bench~ and contribute to its continuous improvement.

In addition, we plan to launch an online leaderboard that reports \bench~scores dynamically for various jailbreak methods. This platform enables real-time benchmarking and fosters a more standardized and transparent evaluation of jailbreak attacks across different LLMs.