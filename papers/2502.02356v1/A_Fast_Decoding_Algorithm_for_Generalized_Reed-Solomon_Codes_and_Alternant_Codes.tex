
%% bare_jrnl.tex
%% V1.3
%% 2007/01/11
%% by Michael Shell
%% see http://www.michaelshell.org/
%% for current contact information.
%%
%% This is a skeleton file demonstrating the use of IEEEtran.cls
%% (requires IEEEtran.cls version 1.7 or later) with an IEEE journal paper.
%%
%% Support sites:
%% http://www.michaelshell.org/tex/ieeetran/
%% http://www.ctan.org/tex-archive/macros/latex/contrib/IEEEtran/
%% and
%% http://www.ieee.org/



% *** Authors should verify (and, if needed, correct) their LaTeX system  ***
% *** with the testflow diagnostic prior to trusting their LaTeX platform ***
% *** with production work. IEEE's font choices can trigger bugs that do  ***
% *** not appear when using other class files.                            ***
% The testflow support page is at:
% http://www.michaelshell.org/tex/testflow/


%%*************************************************************************
%% Legal Notice:
%% This code is offered as-is without any warranty either expressed or
%% implied; without even the implied warranty of MERCHANTABILITY or
%% FITNESS FOR A PARTICULAR PURPOSE! 
%% User assumes all risk.
%% In no event shall IEEE or any contributor to this code be liable for
%% any damages or losses, including, but not limited to, incidental,
%% consequential, or any other damages, resulting from the use or misuse
%% of any information contained here.
%%
%% All comments are the opinions of their respective authors and are not
%% necessarily endorsed by the IEEE.
%%
%% This work is distributed under the LaTeX Project Public License (LPPL)
%% ( http://www.latex-project.org/ ) version 1.3, and may be freely used,
%% distributed and modified. A copy of the LPPL, version 1.3, is included
%% in the base LaTeX documentation of all distributions of LaTeX released
%% 2003/12/01 or later.
%% Retain all contribution notices and credits.
%% ** Modified files should be clearly indicated as such, including  **
%% ** renaming them and changing author support contact information. **
%%
%% File list of work: IEEEtran.cls, IEEEtran_HOWTO.pdf, bare_adv.tex,
%%                    bare_conf.tex, bare_jrnl.tex, bare_jrnl_compsoc.tex
%%*************************************************************************

% Note that the a4paper option is mainly intended so that authors in
% countries using A4 can easily print to A4 and see how their papers will
% look in print - the typesetting of the document will not typically be
% affected with changes in paper size (but the bottom and side margins will).
% Use the testflow package mentioned above to verify correct handling of
% both paper sizes by the user's LaTeX system.
%
% Also note that the "draftcls" or "draftclsnofoot", not "draft", option
% should be used if it is desired that the figures are to be displayed in
% draft mode.
%
%\documentclass[journal,draft, one column]{IEEEtranTCOM}
\documentclass[journal,draftcls,onecolumn,12pt,twoside]{IEEEtran}
%
% If IEEEtran.cls has not been installed into the LaTeX system files,
% manually specify the path to it like:
% \documentclass[journal]{../sty/IEEEtran}

\normalsize

\usepackage{lineno}
%\linenumbers
% Some very useful LaTeX packages include:
% (uncomment the ones you want to load)


% *** MISC UTILITY PACKAGES ***
%
%\usepackage{ifpdf}
% Heiko Oberdiek's ifpdf.sty is very useful if you need conditional
% compilation based on whether the output is pdf or dvi.
% usage:
% \ifpdf
%   % pdf code
% \else
%   % dvi code
% \fi
% The latest version of ifpdf.sty can be obtained from:
% http://www.ctan.org/tex-archive/macros/latex/contrib/oberdiek/
% Also, note that IEEEtran.cls V1.7 and later provides a builtin
% \ifCLASSINFOpdf conditional that works the same way.
% When switching from latex to pdflatex and vice-versa, the compiler may
% have to be run twice to clear warning/error messages.






% *** CITATION PACKAGES ***
%
\usepackage{cite}
% cite.sty was written by Donald Arseneau
% V1.6 and later of IEEEtran pre-defines the format of the cite.sty package
%~\cite{} output to follow that of IEEE. Loading the cite package will
% result in citation numbers being automatically sorted and properly
% "compressed/ranged". e.g., [1], [9], [2], [7], [5], [6] without using
% cite.sty will become [1], [2], [5]--[7], [9] using cite.sty. cite.sty's
%~\cite will automatically add leading space, if needed. Use cite.sty's
% noadjust option (cite.sty V3.8 and later) if you want to turn this off.
% cite.sty is already installed on most LaTeX systems. Be sure and use
% version 4.0 (2003-05-27) and later if using hyperref.sty. cite.sty does
% not currently provide for hyperlinked citations.
% The latest version can be obtained at:
% http://www.ctan.org/tex-archive/macros/latex/contrib/cite/
% The documentation is contained in the cite.sty file itself.






% *** GRAPHICS RELATED PACKAGES ***
%
\ifCLASSINFOpdf
  % \usepackage[pdftex]{graphicx}
  % declare the path(s) where your graphic files are
  % \graphicspath{{../pdf/}{../jpeg/}}
  % and their extensions so you won't have to specify these with
  % every instance of \includegraphics
  % \DeclareGraphicsExtensions{.pdf,.jpeg,.png}
\else
  % or other class option (dvipsone, dvipdf, if not using dvips). graphicx
  % will default to the driver specified in the system graphics.cfg if no
  % driver is specified.
  % \usepackage[dvips]{graphicx}
  % declare the path(s) where your graphic files are
  % \graphicspath{{../eps/}}
  % and their extensions so you won't have to specify these with
  % every instance of \includegraphics
  % \DeclareGraphicsExtensions{.eps}
\fi

% graphicx was written by David Carlisle and Sebastian Rahtz. It is
% required if you want graphics, photos, etc. graphicx.sty is already
% installed on most LaTeX systems. The latest version and documentation can
% be obtained at: 
% http://www.ctan.org/tex-archive/macros/latex/required/graphics/
% Another good source of documentation is "Using Imported Graphics in
% LaTeX2e" by Keith Reckdahl which can be found as epslatex.ps or
% epslatex.pdf at: http://www.ctan.org/tex-archive/info/
%
% latex, and pdflatex in dvi mode, support graphics in encapsulated
% postscript (.eps) format. pdflatex in pdf mode supports graphics
% in .pdf, .jpeg, .png and .mps (metapost) formats. Users should ensure
% that all non-photo figures use a vector format (.eps, .pdf, .mps) and
% not a bitmapped formats (.jpeg, .png). IEEE frowns on bitmapped formats
% which can result in "jaggedy"/blurry rendering of lines and letters as
% well as large increases in file sizes.
%
% You can find documentation about the pdfTeX application at:
% http://www.tug.org/applications/pdftex





% *** MATH PACKAGES ***
%
%\usepackage[cmex10]{amsmath}
% A popular package from the American Mathematical Society that provides
% many useful and powerful commands for dealing with mathematics. If using
% it, be sure to load this package with the cmex10 option to ensure that
% only type 1 fonts will utilized at all point sizes. Without this option,
% it is possible that some math symbols, particularly those within
% footnotes, will be rendered in bitmap form which will result in a
% document that can not be IEEE Xplore compliant!
%
% Also, note that the amsmath package sets \interdisplaylinepenalty to 10000
% thus preventing page breaks from occurring within multiline equations. Use:
%\interdisplaylinepenalty=2500
% after loading amsmath to restore such page breaks as IEEEtran.cls normally
% does. amsmath.sty is already installed on most LaTeX systems. The latest
% version and documentation can be obtained at:
% http://www.ctan.org/tex-archive/macros/latex/required/amslatex/math/





% *** SPECIALIZED LIST PACKAGES ***
%
%\usepackage{algorithmic}
% algorithmic.sty was written by Peter Williams and Rogerio Brito.
% This package provides an algorithmic environment fo describing algorithms.
% You can use the algorithmic environment in-text or within a figure
% environment to provide for a floating algorithm. Do NOT use the algorithm
% floating environment provided by algorithm.sty (by the same authors) or
% algorithm2e.sty (by Christophe Fiorio) as IEEE does not use dedicated
% algorithm float types and packages that provide these will not provide
% correct IEEE style captions. The latest version and documentation of
% algorithmic.sty can be obtained at:
% http://www.ctan.org/tex-archive/macros/latex/contrib/algorithms/
% There is also a support site at:
% http://algorithms.berlios.de/index.html
% Also of interest may be the (relatively newer and more customizable)
% algorithmicx.sty package by Szasz Janos:
% http://www.ctan.org/tex-archive/macros/latex/contrib/algorithmicx/



% *** ALIGNMENT PACKAGES ***
%
%\usepackage{array}
% Frank Mittelbach's and David Carlisle's array.sty patches and improves
% the standard LaTeX2e array and tabular environments to provide better
% appearance and additional user controls. As the default LaTeX2e table
% generation code is lacking to the point of almost being broken with
% respect to the quality of the end results, all users are strongly
% advised to use an enhanced (at the very least that provided by array.sty)
% set of table tools. array.sty is already installed on most systems. The
% latest version and documentation can be obtained at:
% http://www.ctan.org/tex-archive/macros/latex/required/tools/


%\usepackage{mdwmath}
%\usepackage{mdwtab}
% Also highly recommended is Mark Wooding's extremely powerful MDW tools,
% especially mdwmath.sty and mdwtab.sty which are used to format equations
% and tables, respectively. The MDWtools set is already installed on most
% LaTeX systems. The lastest version and documentation is available at:
% http://www.ctan.org/tex-archive/macros/latex/contrib/mdwtools/


% IEEEtran contains the IEEEeqnarray family of commands that can be used to
% generate multiline equations as well as matrices, tables, etc., of high
% quality.


%\usepackage{eqparbox}
% Also of notable interest is Scott Pakin's eqparbox package for creating
% (automatically sized) equal width boxes - aka "natural width parboxes".
% Available at:
% http://www.ctan.org/tex-archive/macros/latex/contrib/eqparbox/





% *** SUBFIGURE PACKAGES ***
%\usepackage[tight,footnotesize]{subfigure}
% subfigure.sty was written by Steven Douglas Cochran. This package makes it
% easy to put subfigures in your figures. e.g., "Figure 1a and 1b". For IEEE
% work, it is a good idea to load it with the tight package option to reduce
% the amount of white space around the subfigures. subfigure.sty is already
% installed on most LaTeX systems. The latest version and documentation can
% be obtained at:
% http://www.ctan.org/tex-archive/obsolete/macros/latex/contrib/subfigure/
% subfigure.sty has been superceeded by subfig.sty.



%\usepackage[caption=false]{caption}
%\usepackage[font=footnotesize]{subfig}
% subfig.sty, also written by Steven Douglas Cochran, is the modern
% replacement for subfigure.sty. However, subfig.sty requires and
% automatically loads Axel Sommerfeldt's caption.sty which will override
% IEEEtran.cls handling of captions and this will result in nonIEEE style
% figure/table captions. To prevent this problem, be sure and preload
% caption.sty with its "caption=false" package option. This is will preserve
% IEEEtran.cls handing of captions. Version 1.3 (2005/06/28) and later 
% (recommended due to many improvements over 1.2) of subfig.sty supports
% the caption=false option directly:
%\usepackage[caption=false,font=footnotesize]{subfig}
%
% The latest version and documentation can be obtained at:
% http://www.ctan.org/tex-archive/macros/latex/contrib/subfig/
% The latest version and documentation of caption.sty can be obtained at:
% http://www.ctan.org/tex-archive/macros/latex/contrib/caption/




% *** FLOAT PACKAGES ***
%
%\usepackage{fixltx2e}
% fixltx2e, the successor to the earlier fix2col.sty, was written by
% Frank Mittelbach and David Carlisle. This package corrects a few problems
% in the LaTeX2e kernel, the most notable of which is that in current
% LaTeX2e releases, the ordering of single and double column floats is not
% guaranteed to be preserved. Thus, an unpatched LaTeX2e can allow a
% single column figure to be placed prior to an earlier double column
% figure. The latest version and documentation can be found at:
% http://www.ctan.org/tex-archive/macros/latex/base/



%\usepackage{stfloats}
% stfloats.sty was written by Sigitas Tolusis. This package gives LaTeX2e
% the ability to do double column floats at the bottom of the page as well
% as the top. (e.g., "\begin{figure*}[!b]" is not normally possible in
% LaTeX2e). It also provides a command:
%\fnbelowfloat
% to enable the placement of footnotes below bottom floats (the standard
% LaTeX2e kernel puts them above bottom floats). This is an invasive package
% which rewrites many portions of the LaTeX2e float routines. It may not work
% with other packages that modify the LaTeX2e float routines. The latest
% version and documentation can be obtained at:
% http://www.ctan.org/tex-archive/macros/latex/contrib/sttools/
% Documentation is contained in the stfloats.sty comments as well as in the
% presfull.pdf file. Do not use the stfloats baselinefloat ability as IEEE
% does not allow \baselineskip to stretch. Authors submitting work to the
% IEEE should note that IEEE rarely uses double column equations and
% that authors should try to avoid such use. Do not be tempted to use the
% cuted.sty or midfloat.sty packages (also by Sigitas Tolusis) as IEEE does
% not format its papers in such ways.


%\ifCLASSOPTIONcaptionsoff
%  \usepackage[nomarkers]{endfloat}
% \let\MYoriglatexcaption\caption
% \renewcommand{\caption}[2][\relax]{\MYoriglatexcaption[#2]{#2}}
%\fi
% endfloat.sty was written by James Darrell McCauley and Jeff Goldberg.
% This package may be useful when used in conjunction with IEEEtran.cls'
% captionsoff option. Some IEEE journals/societies require that submissions
% have lists of figures/tables at the end of the paper and that
% figures/tables without any captions are placed on a page by themselves at
% the end of the document. If needed, the draftcls IEEEtran class option or
% \CLASSINPUTbaselinestretch interface can be used to increase the line
% spacing as well. Be sure and use the nomarkers option of endfloat to
% prevent endfloat from "marking" where the figures would have been placed
% in the text. The two hack lines of code above are a slight modification of
% that suggested by in the endfloat docs (section 8.3.1) to ensure that
% the full captions always appear in the list of figures/tables - even if
% the user used the short optional argument of \caption[]{}.
% IEEE papers do not typically make use of \caption[]'s optional argument,
% so this should not be an issue. A similar trick can be used to disable
% captions of packages such as subfig.sty that lack options to turn off
% the subcaptions:
% For subfig.sty:
% \let\MYorigsubfloat\subfloat
% \renewcommand{\subfloat}[2][\relax]{\MYorigsubfloat[]{#2}}
% For subfigure.sty:
% \let\MYorigsubfigure\subfigure
% \renewcommand{\subfigure}[2][\relax]{\MYorigsubfigure[]{#2}}
% However, the above trick will not work if both optional arguments of
% the \subfloat/subfig command are used. Furthermore, there needs to be a
% description of each subfigure *somewhere* and endfloat does not add
% subfigure captions to its list of figures. Thus, the best approach is to
% avoid the use of subfigure captions (many IEEE journals avoid them anyway)
% and instead reference/explain all the subfigures within the main caption.
% The latest version of endfloat.sty and its documentation can obtained at:
% http://www.ctan.org/tex-archive/macros/latex/contrib/endfloat/
%
% The IEEEtran \ifCLASSOPTIONcaptionsoff conditional can also be used
% later in the document, say, to conditionally put the References on a 
% page by themselves.





% *** PDF, URL AND HYPERLINK PACKAGES ***
%
%\usepackage{url}
% url.sty was written by Donald Arseneau. It provides better support for
% handling and breaking URLs. url.sty is already installed on most LaTeX
% systems. The latest version can be obtained at:
% http://www.ctan.org/tex-archive/macros/latex/contrib/misc/
% Read the url.sty source comments for usage information. Basically,
% \url{my_url_here}.





% *** Do not adjust lengths that control margins, column widths, etc. ***
% *** Do not use packages that alter fonts (such as pslatex).         ***
% There should be no need to do such things with IEEEtran.cls V1.6 and later.
% (Unless specifically asked to do so by the journal or conference you plan
% to submit to, of course. )


% correct bad hyphenation here
\hyphenation{op-tical net-works semi-conduc-tor}

\usepackage{amsmath,amsfonts}
\usepackage{algorithmic}
\usepackage{algorithm}
\usepackage{array}
\usepackage[caption=false,font=normalsize,labelfont=sf,textfont=sf]{subfig}
\usepackage{textcomp}
\usepackage{stfloats}
\usepackage{url}
\usepackage{verbatim}
\usepackage{graphicx}
\usepackage{cite}
\usepackage{xcolor}
\usepackage{multirow}

\newcommand{\LCM}{\text{LCM}}
\newcommand{\Ls}{\mathbf{\mathcal{L}}}
\newtheorem{lemma}{Lemma}
\newtheorem{example}{Example}
\newenvironment{proof}{\par{\it{Proof:}}\ }{\hfill $\square$\par}

\renewcommand{\algorithmicrequire}{\textbf{Input:}}
\renewcommand{\algorithmicensure}{\textbf{Output:}}
%{\qed}
%{\hfill $\square$\par}
% A popular package from the American Mathematical Society that provides
% many useful and powerful commands for dealing with mathematics.
%
% Note that the amsmath package sets \interdisplaylinepenalty to 10000
% thus preventing page breaks from occurring within multiline equations. Use:
%\interdisplaylinepenalty=2500
% after loading amsmath to restore such page breaks as IEEEtran.cls normally
% does. amsmath.sty is already installed on most LaTeX systems. The latest
% version and documentation can be obtained at:
% http://www.ctan.org/pkg/amsmath





% *** SPECIALIZED LIST PACKAGES ***
%
\usepackage{algorithm}
\usepackage{algorithmic}
%\usepackage{algorithmicx}
\usepackage{float}
\usepackage{lipsum}
\usepackage{xcolor}
%\newcommand{\FFT}{\text{FFT}}
%\newcommand{\IFFT}{\text{IFFT}}
\DeclareMathOperator{\FFT}{FFT}%The use of \DeclareMathOperator gives a typeface that matches that of the surrounding maths and also gives the correct spacing before and after the defined operator 
\DeclareMathOperator{\IFFT}{IFFT}
\newcommand{\X}{\mathbb{X}}
\newcommand{\F}{\mathbb{F}}
%\newcommand{\rank}{\text{rank}}
\DeclareMathOperator{\rank}{rank}
\renewcommand{\algorithmicrequire}{ \textbf{Input:}} %Use Input in the format of Algorithm
\renewcommand{\algorithmicensure}{ \textbf{Output:}}

\renewcommand{\Im}{\operatorname{Im}}%This gives the correct maths typeface and spacing for ``Im'' denoting ``image'' (the \renewcommand is necessary because \Im would otherwise give the standard LaTeX notation for ``imaginary part'', a Fraktur ``I'')

\makeatletter
\newenvironment{breakablealgorithm}
{% \begin{breakablealgorithm}
	\begin{center}
		\refstepcounter{algorithm}% New algorithm
		\hrule height.8pt depth0pt \kern2pt% \@fs@pre for \@fs@ruled
		\renewcommand{\caption}[2][\relax]{% Make a new \caption
			{\raggedright\textbf{\ALG@name~\thealgorithm} ##2\par}%
			\ifx\relax##1\relax % #1 is \relax
			\addcontentsline{loa}{algorithm}{\protect\numberline{\thealgorithm}##2}%
			\else % #1 is not \relax
			\addcontentsline{loa}{algorithm}{\protect\numberline{\thealgorithm}##1}%
			\fi
			\kern2pt\hrule\kern2pt
		}
	}{% \end{breakablealgorithm}
		\kern2pt\hrule\relax% \@fs@post for \@fs@ruled
	\end{center}
}
\makeatother

%UseOutput in the format of Algorithm
% algorithmic.sty was written by Peter Williams and Rogerio Brito.
% This package provides an algorithmic environment fo describing algorithms.
% You can use the algorithmic environment in-text or within a figure
% environment to provide for a floating algorithm. Do NOT use the algorithm
% floating environment provided by algorithm.sty (by the same authors) or
% algorithm2e.sty (by Christophe Fiorio) as the IEEE does not use dedicated
% algorithm float types and packages that provide these will not provide
% correct IEEE style captions. The latest version and documentation of
% algorithmic.sty can be obtained at:
% http://www.ctan.org/pkg/algorithms
% Also of interest may be the (relatively newer and more customizable)
% algorithmicx.sty package by Szasz Janos:
% http://www.ctan.org/pkg/algorithmicx




% *** ALIGNMENT PACKAGES ***
%
%\usepackage{array}
% Frank Mittelbach's and David Carlisle's array.sty patches and improves
% the standard LaTeX2e array and tabular environments to provide better
% appearance and additional user controls. As the default LaTeX2e table
% generation code is lacking to the point of almost being broken with
% respect to the quality of the end results, all users are strongly
% advised to use an enhanced (at the very least that provided by array.sty)
% set of table tools. array.sty is already installed on most systems. The
% latest version and documentation can be obtained at:
% http://www.ctan.org/pkg/array


% IEEEtran contains the IEEEeqnarray family of commands that can be used to
% generate multiline equations as well as matrices, tables, etc., of high
% quality.




% *** SUBFIGURE PACKAGES ***
%\ifCLASSOPTIONcompsoc
%  \usepackage[caption=false,font=normalsize,labelfont=sf,textfont=sf]{subfig}
%\else
%  \usepackage[caption=false,font=footnotesize]{subfig}
%\fi
% subfig.sty, written by Steven Douglas Cochran, is the modern replacement
% for subfigure.sty, the latter of which is no longer maintained and is
% incompatible with some LaTeX packages including fixltx2e. However,
% subfig.sty requires and automatically loads Axel Sommerfeldt's caption.sty
% which will override IEEEtran.cls' handling of captions and this will result
% in non-IEEE style figure/table captions. To prevent this problem, be sure
% and invoke subfig.sty's "caption=false" package option (available since
% subfig.sty version 1.3, 2005/06/28) as this is will preserve IEEEtran.cls
% handling of captions.
% Note that the Computer Society format requires a larger sans serif font
% than the serif footnote size font used in traditional IEEE formatting
% and thus the need to invoke different subfig.sty package options depending
% on whether compsoc mode has been enabled.
%
% The latest version and documentation of subfig.sty can be obtained at:
% http://www.ctan.org/pkg/subfig




% *** FLOAT PACKAGES ***
%
%\usepackage{fixltx2e}
% fixltx2e, the successor to the earlier fix2col.sty, was written by
% Frank Mittelbach and David Carlisle. This package corrects a few problems
% in the LaTeX2e kernel, the most notable of which is that in current
% LaTeX2e releases, the ordering of single and double column floats is not
% guaranteed to be preserved. Thus, an unpatched LaTeX2e can allow a
% single column figure to be placed prior to an earlier double column
% figure.
% Be aware that LaTeX2e kernels dated 2015 and later have fixltx2e.sty's
% corrections already built into the system in which case a warning will
% be issued if an attempt is made to load fixltx2e.sty as it is no longer
% needed.
% The latest version and documentation can be found at:
% http://www.ctan.org/pkg/fixltx2e


%\usepackage{stfloats}
% stfloats.sty was written by Sigitas Tolusis. This package gives LaTeX2e
% the ability to do double column floats at the bottom of the page as well
% as the top. (e.g., "\begin{figure*}[!b]" is not normally possible in
% LaTeX2e). It also provides a command:
%\fnbelowfloat
% to enable the placement of footnotes below bottom floats (the standard
% LaTeX2e kernel puts them above bottom floats). This is an invasive package
% which rewrites many portions of the LaTeX2e float routines. It may not work
% with other packages that modify the LaTeX2e float routines. The latest
% version and documentation can be obtained at:
% http://www.ctan.org/pkg/stfloats
% Do not use the stfloats baselinefloat ability as the IEEE does not allow
% \baselineskip to stretch. Authors submitting work to the IEEE should note
% that the IEEE rarely uses double column equations and that authors should try
% to avoid such use. Do not be tempted to use the cuted.sty or midfloat.sty
% packages (also by Sigitas Tolusis) as the IEEE does not format its papers in
% such ways.
% Do not attempt to use stfloats with fixltx2e as they are incompatible.
% Instead, use Morten Hogholm'a dblfloatfix which combines the features
% of both fixltx2e and stfloats:
%
% \usepackage{dblfloatfix}
% The latest version can be found at:
% http://www.ctan.org/pkg/dblfloatfix




%\ifCLASSOPTIONcaptionsoff
%  \usepackage[nomarkers]{endfloat}
% \let\MYoriglatexcaption\caption
% \renewcommand{\caption}[2][\relax]{\MYoriglatexcaption[#2]{#2}}
%\fi
% endfloat.sty was written by James Darrell McCauley, Jeff Goldberg and 
% Axel Sommerfeldt. This package may be useful when used in conjunction with 
% IEEEtran.cls'  captionsoff option. Some IEEE journals/societies require that
% submissions have lists of figures/tables at the end of the paper and that
% figures/tables without any captions are placed on a page by themselves at
% the end of the document. If needed, the draftcls IEEEtran class option or
% \CLASSINPUTbaselinestretch interface can be used to increase the line
% spacing as well. Be sure and use the nomarkers option of endfloat to
% prevent endfloat from "marking" where the figures would have been placed
% in the text. The two hack lines of code above are a slight modification of
% that suggested by in the endfloat docs (section 8.4.1) to ensure that
% the full captions always appear in the list of figures/tables - even if
% the user used the short optional argument of \caption[]{}.
% IEEE papers do not typically make use of \caption[]'s optional argument,
% so this should not be an issue. A similar trick can be used to disable
% captions of packages such as subfig.sty that lack options to turn off
% the subcaptions:
% For subfig.sty:
% \let\MYorigsubfloat\subfloat
% \renewcommand{\subfloat}[2][\relax]{\MYorigsubfloat[]{#2}}
% However, the above trick will not work if both optional arguments of
% the \subfloat command are used. Furthermore, there needs to be a
% description of each subfigure *somewhere* and endfloat does not add
% subfigure captions to its list of figures. Thus, the best approach is to
% avoid the use of subfigure captions (many IEEE journals avoid them anyway)
% and instead reference/explain all the subfigures within the main caption.
% The latest version of endfloat.sty and its documentation can obtained at:
% http://www.ctan.org/pkg/endfloat
%
% The IEEEtran \ifCLASSOPTIONcaptionsoff conditional can also be used
% later in the document, say, to conditionally put the References on a 
% page by themselves.




% *** PDF, URL AND HYPERLINK PACKAGES ***
%
%\usepackage{url}
% url.sty was written by Donald Arseneau. It provides better support for
% handling and breaking URLs. url.sty is already installed on most LaTeX
% systems. The latest version and documentation can be obtained at:
% http://www.ctan.org/pkg/url
% Basically, \url{my_url_here}.




% *** Do not adjust lengths that control margins, column widths, etc. ***
% *** Do not use packages that alter fonts (such as pslatex).         ***
% There should be no need to do such things with IEEEtran.cls V1.6 and later.
% (Unless specifically asked to do so by the journal or conference you plan
% to submit to, of course. )


% correct bad hyphenation here
%\hyphenation{op-tical net-works semi-conduc-tor}


%The following is to allow the insertion of queries and comments for the author. 
\newtheorem{aq}{Author Query/Comment}

\newcommand{\baq}{\begin{aq}}
\newcommand{\eaq}{\end{aq}}

\newcommand{\AQ}[1]{{\textcolor{cyan}{#1}}}

\begin{document}
%
% paper title
% Titles are generally capitalized except for words such as a, an, and, as,
% at, but, by, for, in, nor, of, on, or, the, to and up, which are usually
% not capitalized unless they are the first or last word of the title.
% Linebreaks \\ can be used within to get better formatting as desired.
% Do not put math or special symbols in the title.
\title{A Fast Decoding Algorithm for Generalized Reed-Solomon Codes and Alternant Codes}
%
%
% author names and IEEE memberships
% note positions of commas and nonbreaking spaces ( ~ ) LaTeX will not break
% a structure at a ~ so this keeps an author's name from being broken across
% two lines.
% use \thanks{} to gain access to the first footnote area
% a separate \thanks must be used for each paragraph as LaTeX2e's \thanks
% was not built to handle multiple paragraphs
%

\author{Nianqi Tang, Yunghsiang S. Han, \IEEEmembership{Fellow,~IEEE}, Danyang Pei, and Chao Chen 
	\thanks{
		%This work was supported by the National Key Research and Development Program of China under Grant 2022YFA1004902.
		
		Nianqi Tang (724973040@qq.com),  Yunghsiang S. Han (yunghsiangh@gmail.com), and Danyang Pei (202322280103@std.uestc.edu.cn) are with the Shenzhen Institute for Advanced Study, University of Electronic Science and Technology of China, Shenzhen, China. Chao Chen (cchen@xidian.edu.cn) is with the State Key Lab of ISN, Xidian University, Xi'an, China.}% <-this % stops a space
%\thanks{Manuscript received October 18, 2019; revised December 05, 2019.}
%%}
}

% note the % following the last \IEEEmembership and also \thanks - 
% these prevent an unwanted space from occurring between the last author name
% and the end of the author line. i.e., if you had this:
% 
% \author{....lastname \thanks{...} \thanks{...} }
%                     ^------------^------------^----Do not want these spaces!
%
% a space would be appended to the last name and could cause every name on that
% line to be shifted left slightly. This is one of those "LaTeX things". For
% instance, "\textbf{A} \textbf{B}" will typeset as "A B" not "AB". To get
% "AB" then you have to do: "\textbf{A}\mathbf{B}"
% \thanks is no different in this regard, so shield the last } of each \thanks
% that ends a line with a % and do not let a space in before the next \thanks.
% Spaces after \IEEEmembership other than the last one are OK (and needed) as
% you are supposed to have spaces between the names. For what it is worth,
% this is a minor point as most people would not even notice if the said evil
% space somehow managed to creep in.



% The paper headers
%\markboth{IEEE Transactions on Communications}%
%{Submitted paper}
% The only time the second header will appear is for the odd numbered pages
% after the title page when using the twoside option.
% 
% *** Note that you probably will NOT want to include the author's ***
% *** name in the headers of peer review papers.                   ***
% You can use \ifCLASSOPTIONpeerreview for conditional compilation here if
% you desire.




% If you want to put a publisher's ID mark on the page you can do it like
% this:
%\IEEEpubid{0000--0000/00\$00.00~\copyright~2015 IEEE}
% Remember, if you use this you must call \IEEEpubidadjcol in the second
% column for its text to clear the IEEEpubid mark.



% use for special paper notices
%\IEEEspecialpapernotice{(Invited Paper)}




% make the title area
\maketitle
% As a general rule, do not put math, special symbols or citations
% in the abstract or keywords.
\begin{abstract}
In this paper, it is shown that the syndromes of generalized Reed-Solomon (GRS) codes and alternant codes can be characterized in terms of inverse fast Fourier transform, regardless of code definitions. Then a fast decoding algorithm is proposed, which has a computational complexity of $O(n\log(n-k) + (n-k)\log^2(n-k))$  for all $(n,k)$ GRS codes and $(n,k)$ alternant codes. Particularly, this provides a new decoding method for Goppa codes, which is an important subclass of alternant codes. When decoding the binary Goppa code with length $8192$ and correction capability $128$, the new algorithm is nearly 10 times faster than traditional methods. The decoding algorithm is suitable for the McEliece cryptosystem, which is a candidate for post-quantum cryptography techniques.
\end{abstract}
%
\begin{IEEEkeywords}
generalized Reed-Solomon Codes, alternant codes, decoding algorithms, Goppa codes, fast Fourier transform, McEliece cryptosystem 
\end{IEEEkeywords}






% For peer review papers, you can put extra information on the cover
% page as needed:
% \ifCLASSOPTIONpeerreview
% \begin{center} \bfseries EDICS Category: 3-BBND \end{center}
% \fi
%
% For peerreview papers, this IEEEtran command inserts a page break and
% creates the second title. It will be ignored for other modes.
\IEEEpeerreviewmaketitle

\section{Introduction}
Generalized Reed-Solomon (GRS) Codes are an important class of error-correcting codes for both theoretical and practical aspects. On the theoretical aspect, GRS codes are maximum-distance-separable (MDS) and are algebraic geometry codes. Furthermore, some alternant codes, which are a subclass of GRS codes, are able to achieve the Gilbert-Varshamov bound. On the practical aspect, several subclasses of GRS codes have been adopted in many applications. For example, Reed-Solomon (RS) codes and Bose-Ray-Chaudhuri-Hocquenghem (BCH) codes have been adopt as error-correcting codes in wired communication systems and storage systems. Goppa codes, a special class of alternant codes, have been adopted in the McEliece cryptosystem, which is chosen as a candidate for post-quantum cryptography (PQC). Hence, investigating the decoding algorithm for GRS codes is of great importance. In the past years, several decoding techniques have been proposed for various subclasses of GRS codes. For example, based on the conventionally defined syndrome, decoding methods for RS codes and BCH codes, which use the Berlekamp-Massey algorithm and the Euclidean algorithm, can be found in \cite{berlekamp2015algebraic} and \cite{sugiyama1975method}, respectively. Based on the remainder polynomial, decoding techniques for RS codes using the Welch-Berlekamp algorithm and the modular approach appeared in \cite{Error1984} and \cite{Fast1995}. A decoding algorithm for Goppa codes was proposed in \cite{patterson1975algebraic} as well as a simple improvement for irreducible Goppa codes. In addition to the hard-decision decoding mentioned above, many soft-decision decoding algorithms were investigated, e.g., Guruswami-Sudan algorithm for RS codes \cite{Guruswami1999improved}, efficient list decoding for RS or BCH codes \cite{wu2012fast, shany2022a}, and so on.

As GRS codes are defined by the finite field Fourier transform, much effort has been devoted to investigating decoding algorithms based on fast Fourier transform (FFT) over finite fields, see \cite{wu2012reduced, bellini2011structure, gao2010additive} for example. There are several challenges when applying an FFT algorithm to design a decoding algorithm. Firstly, whether the obtained decoding algorithms are both available and efficient for various code parameters. Secondly, whether the decoding algorithm has a regular structure such that it is suitable for hardware implementations. In our previous work \cite{tang2022a}, a decoding technique based on Lin-Chung-Han FFT (LCH-FFT, see Section~\ref{sec:preliminary} for details) was proposed for RS codes,  which has the lowest computational complexity to date and has a regular structure. However, the RS codes, which are taken into account in \cite{tang2022a}, are defined in a different way from conventional GRS codes. More precisely, the $(n,k)$ RS code in \cite{tang2022a} is defined by
\begin{align*}
	\{(f(\omega_0), f(\omega_1), \dots, f(\omega_{n-1}))\mid \deg(f(x)) < 2^m - n + k,\\ f(\omega_j) = 0, j = n, n + 1, \dots, 2^m - 1 \},
\end{align*}
where $f(x)\in GF(2^m)[x]$ and $\omega_j$, whose definition can be found in Section \ref{sec:LCH}, are distinct elements in finite field $GF(2^m)$. This $(n,k)$ RS code can be seen as deleting message symbols for a $(2^m, 2^m - n + k)$ RS code. However, a conventional GRS code is defined as
\begin{equation*}
	\{w_0f(\alpha_0), w_1f(\alpha_1), \dots, w_{n-1}f(\alpha_{n-1})\mid \deg(f(x)) < k\},
\end{equation*}
where $w_i$ are nonzero elements and $\alpha_i$ are distinct elements in $GF(2^m)$. The conventional $(n,k)$ GRS code can be seen as deleting parity symbols from a $(2^m, k)$ GRS code. Hence, it is not straightforward whether a fast decoding algorithm based on FFT exists for arbitrary GRS codes.
On the other hand, it is not clear whether such a fast decoding algorithm exists for alternate codes, such as Goppa codes, since their definitions are not related to the Fourier transform. This paper is devoted to addressing these issues.

In this paper, we first investigate the concept of generalized syndromes. The concept of generalized syndromes was first shown in \cite{araki1992generalized}, which was used for illustrating the relationship between the conventional decoding key equation and Welch-Berlekamp's key equation. We prove that, by specifying the polynomial $T(x)$ (which shall be defined shortly), the generalized syndrome can be characterized in terms of the high degree part of the inverse FFT  (IFFT) of the received vector. Accordingly, an efficient method for computing this type of syndrome is also derived, which is of complexity $O(n\log(n - k))$ for code length $n$ and information dimension $k$. Notably, this type of syndromes is defined with respect to the parity check matrix, regardless of code definitions. As $T(x)$ is fixed, all GRS codes and alternant codes are shown to have the same type of key equation. This leads to a unified decoding algorithm with the computational complexity of $O(n\log(n-k) + (n-k)\log^2(n-k))$, which is the lowest computational complexity to date according to the best of our knowledge. As an example, we investigate the proposed algorithm regarding the Goppa codes. A computational complexity comparison among the proposed algorithm, the famous Patterson method in \cite{patterson1975algebraic}, and a conventional decoding method shown in \cite{macwilliams1977theory} is provided. When decoding the Goppa code of length $8192$, which is defined on $GF(2^{13})$ and has Goppa polynomial $x^{128} + x^7 + x^2 + x + 1$ (this Goppa code has been submitted to National Institute of Standards and Technology (NIST) as a candidate for post-quantum cryptography, see \cite{NIST_PQC}), the proposed algorithm is nearly 10 times faster than other methods. This shows that the proposed algorithm is superior in terms of computational complexity for practical use.

The main contributions of this paper can be summarized as follows:
\begin{enumerate}
	\item We characterize the generalized syndrome in terms of the IFFT for all GRS codes and alternant codes and provide an efficient method for computing the syndrome;
	\item Based on the above syndrome, we show that all GRS codes and alternant codes have the same type of key equation. This leads to a fast decoding algorithm with complexity $O(n\log(n-k) + (n-k)\log^2(n-k))$ for any $(n,k)$ GRS code and alternant code, which is the best complexity to date. This also provides another derivation of the key equation in \cite{Lin20161}, which is more straightforward and comprehensive;
	\item Based on the algorithm, we provide a fast decoding algorithm for Goppa codes and show that the algorithm is efficient for practical use.
\end{enumerate}
%\uppercase\expandafter{\romannumeral2}

The paper is organized as follows. Section \ref{sec:preliminary} first reviews the definition of GRS codes and alternant codes, and a brief introduction to LCH-FFT is then provided. Section \ref{sec:syndrome} discusses the generalized syndrome and characterizes the generalized syndrome in terms of the IFFT of the received vector. Section \ref{sec:dec} proposes a decoding algorithm for any GRS code and alternant code. Section \ref{sec:dec} provides the application of the proposed method to separable Goppa codes and gives a complexity comparison between the proposed algorithm and traditional methods when decoding Goppa codes. Finally, Section \ref{sec:conclusion} concludes the paper and presents several open issues.

\section{Preliminary}\label{sec:preliminary}
This section reviews some basic definitions and concepts, including the GRS codes and alternant codes, and provides a brief introduction to the FFT over a finite field (Lin-Chung-Han FFT), which is crucial for designing a decoding algorithm. Throughout this paper, we mainly focus on codes defined over binary extension field $GF(2^m)$ since they are of great importance in practice. Nevertheless, the technologies presented here can be generalized to any finite field.
\subsection{Generalized Reed-Solomon Codes and Alternant Codes}

Let $n$ and $k$ be integers satisfying $0 < k \leq n\leq 2^m$. Let $\Ls = (\alpha_0, \alpha_1, \dots, \alpha_{n - 1})$ be $n$ distinct elements in $GF(2^m)$ and let $\mathbf{w} = (w_0, w_1, \dots, w_{n - 1})$ whose elements are all nonzero in $GF(2^m)$. The generalized Reed-Solomon (GRS) code $GRS_k(\Ls, \mathbf{w})$ consists of all vectors
\begin{equation}
	(w_0 f(\alpha_0), w_1 f(\alpha_1), \dots, w_{n - 1} f(\alpha_{n - 1})),
\end{equation}
where $f(x)\in GF(2^m)[x]$ satisfying $\deg(f(x)) < k$. $GRS_k(\Ls, \mathbf{w})$ code is an $(n, k, d)$ code over $GF(2^m)$, where $d = n - k + 1$ is the minimum distance.

The dual code of $GRS_k(\Ls, \mathbf{w})$ is also a GRS code. The parity check matrix of $GRS_k(\Ls, \mathbf{w})$ can be written as 
\begin{align*}
	H &= \left(
	\begin{array}{cccc}
		y_0 & y_1 & \dots & y_{n - 1} \\
		y_0\alpha_0 & y_1\alpha_1 & \dots & y_{n - 1}\alpha_{n - 1}\\
		\vdots & \vdots & \dots & \vdots\\
		y_0\alpha_0^{n - k - 1} & y_1\alpha_1^{n - k - 1} & \dots & y_{n - 1}\alpha_{n - 1}^{n - k - 1}
	\end{array}
	\right),
	%	& = 
	%	\left(
	%	\begin{array}{cccc}
		%		1 & 1 & \dots & 1 \\
		%		\alpha_1 & \alpha_2 & \dots & \alpha_n\\
		%		\vdots & \vdots & \dots & \vdots\\
		%		\alpha_1^{r - 1} & \alpha_2^{r - 1} & \dots & \alpha_n^{r - 1}
		%	\end{array}
	%	\right)
\end{align*}
where $\mathbf{y} = (y_0, y_1, \dots, y_{n - 1})\in GF(2^m)$ is a vector whose components are all nonzero. It can be shown that
\begin{equation}\label{eq:inverse}y_i=w_i^{-1}\left(\prod_{0\le j\le n-1,j\neq i}(\alpha_i-\alpha_j)\right)^{-1}.\end{equation}

The alternant code $\mathcal{A}_k(\Ls, \mathbf{y})$ consists of all codewords of $GRS_k(\Ls, \mathbf{w})$ whose components all lie in $GF(2)$. In other words, $\mathcal{A}_k(\Ls, \mathbf{y})$ is the subfield subcode of $GRS_k(\Ls, \mathbf{w})$. $\mathcal{A}_k(\Ls, \mathbf{y})$ is an $(n, \geq n - m(n - k), \geq d)$ code over $GF(2)$, where $d = n - k + 1$. Straightforwardly, $H$ is also the parity check matrix of $\mathcal{A}_k(\Ls, \mathbf{y})$.

In subsequent sections, unless other specifications, we assume that the codes that appear are determined by the parity check matrix $H$. We do not distinguish $GRS_k(\Ls, \mathbf{w})$ and $\mathcal{A}_k(\Ls, \mathbf{y})$ since the decoding algorithm focuses mainly on $H$.

\subsection{Lin-Chung-Han FFT (LCH-FFT)\cite{4319038}}~\label{sec:LCH}

%The Lin-Chung-Han FFT (LCH-FFT) was first proposed in \cite{Lin2014}. It uses a basis that is derived from the subspace polynomial and has the advantage of low complexity. A discussion of the LCH-FFT has been provided.

Let $\{v_0, v_1, \dots, v_{m-1}\}$ be a basis of $GF(2^m)$ over $GF(2)$. The elements of $GF(2^m)$ can be represented by
\begin{equation*}
	\omega_j = j_0v_0 + j_1v_1 + \dots + j_{m-1}v_{m-1}, 0\leq j< 2^m - 1,
\end{equation*}
where $(j_0, j_1, \dots, j_{m-1})$ is the binary representation of the integer $j$. The subspace polynomial $s_{\tau}(x)$ is defined as
\begin{equation*}
	s_{\tau}(x) = \prod_{j = 0}^{2^{\tau} - 1}(x - \omega_j), 0\leq \tau\leq m.
\end{equation*}
Define the polynomial ${X}_j(x)$ as
\begin{equation*}
	{X}_j(x) = s_0(x)^{j_0}s_1(x)^{j_1}\cdots s_{m-1}(x)^{j_{m-1}}.
\end{equation*}
Then the set ${\mathbb{X}} = \{{X}_0(x), {X}_1(x), \dots, {X}_{2^m - 1}(x)\}$ is a basis of $GF({2^m})[x] / (x^{2^m} - x)$ over $GF({2^m})$ since $\deg({X}_j(x)) = j$. Let
\begin{equation*}
	\bar{X}_{j}(x) = {X}_j(x) / p_j, 0 \leq j < 2^m - 1,
\end{equation*}
where $p_j = s_0(v_0)^{j_0}s_1(v_1)^{j_1}\cdots s_{m-1}(v_{m-1})^{j_{m-1}}$. It is easy to check that the set $\bar{\mathbb{X}} = \{\bar{X}_0(x), \bar{X}_1(x), \dots, \bar{X}_{2^m - 1}(x) \}$ is also a basis of $GF({2^m})[x] / (x^{2^m} - x)$ over $GF({2^m})$.

For a polynomial $f(x) \in GF({2^m})[x] / (x^{2^m} - x)$ of degree less than $2^{\tau}$, given its coordinate vector $\bar{\mathbf{f}} = (\bar{f}_0, \bar{f}_1, \dots, \bar{f}_{2^{\tau} - 1})$ with respect to $\bar{\mathbb{X}}$ and $\beta\in GF({2^m})$, the LCH-FFT computes $$\mathbf{F} = (f(\omega_0 + \beta), f(\omega_1 + \beta), \dots, f(\omega_{2^{\tau} - 1} + \beta))$$ within $O(2^{\tau}\log(2^{\tau}))$ field operations, which is denoted by
\begin{equation*}
	\mathbf{F} = \text{FFT}_{\bar{\mathbb{X}}}(\bar{\mathbf{f}}, \tau, \beta).
\end{equation*}
Its inverse transform is written as
\begin{equation*}
	\bar{\mathbf{f}} = \text{IFFT}_{\bar{\mathbb{X}}}(\mathbf{F}, \tau, \beta).
\end{equation*}
Detailed descriptions of $\FFT_{{\bar{\mathbb{X}}}}$ and $\IFFT_{{\bar{\mathbb{X}}}}$ are shown in Algorithms \ref{FFTX} and \ref{IFFTX}, respectively. For more discussions, please refer to \cite{Lin20161}. It should be noted that the LCH-FFT has been generalized to evaluate any number of points. More details can be found in \cite{2207.11079}.

\begin{algorithm}[ht]
	\caption{$\FFT_{{\bar{\mathbb{X}}}}$~\cite{4319038}}
	\label{FFTX}
	\begin{algorithmic}[1]
		\REQUIRE
		$\bar{\mathbf{f}} = (\bar{f}_0,\bar{f}_1,\dots,\bar{f}_{2^\tau-1})$, $\tau$, $\beta$.
		\ENSURE
		$(f(\omega_0 + \beta), f(\omega_1 + \beta), \dots, f(\omega_{2^\tau-1} + \beta))$.
		
		\IF {$\tau = 0$} \RETURN $\bar{f}_0$
		\ENDIF
		\FOR {$l = 0, 1,\dots,2^{\tau-1}-1$}
		\STATE {$a_l^{(0)} = \bar{f}_l + \dfrac{s_{\tau-1}(\beta)}{s_{\tau-1}(v_{\tau-1})}\bar{f}_{l+2^{\tau-1}}$}
		\STATE {$a_l^{(1)} = a_l^{(0)} + \bar{f}_{l+2^{\tau-1}}$}
		\ENDFOR
		\STATE $\mathbf{a}^{(0)}=(a_0^{(0)},\dots,a_{2^{\tau-1}-1}^{(0)}),\mathbf{a}^{(1)}=(a_0^{(1)},\dots,a_{2^{\tau-1}-1}^{(1)})$
		\STATE Calculate $\mathbf{A}_0 = \FFT_{\bar{\mathbb{X}}}(\mathbf{a}^{(0)},\tau-1,\beta)$, $\mathbf{A}_1 = \FFT_{\bar{\mathbb{X}}}(\mathbf{a}^{(1)},\tau-1,v_{\tau-1} + \beta)$
		\RETURN $(\mathbf{A}_0,\mathbf{A}_1)$
	\end{algorithmic}
\end{algorithm}

\begin{algorithm}[ht]
	\caption{$\IFFT_{{\bar{\mathbb{X}}}}$~\cite{4319038}}
	\label{IFFTX}
	\begin{algorithmic}[1]
		\REQUIRE
		$\mathbf{F} = (f(\omega_0 + \beta), f(\omega_1 + \beta), \dots, f(\omega_{2^\tau-1} + \beta)), \tau, \beta$
		\ENSURE
		$\bar{\mathbf{f}}$ such that $ \mathbf{F}= \FFT_{\bar{\mathbb{X}}}(\bar{\mathbf{f}},\tau,\beta)$
		
		\IF {$\tau = 0$}
		\RETURN $f(\omega_0 + \beta)$
		\ENDIF
		
		\STATE  $\mathbf{A}_{0}=(f(\omega_0 + \beta),\dots,f(\omega_{2^{\tau-1}-1} + \beta)),\mathbf{A}_{1}=(f(\omega_{2^{\tau-1}} + \beta)),\dots,f(\omega_{2^{\tau}-1} + \beta))$
		\STATE $\mathbf{a}^{(0)} = \IFFT_{\bar{\mathbb{X}}}(\mathbf{A}_0,\tau-1,\beta)$
		, $\mathbf{a}^{(1)} = \IFFT_{\bar{\mathbb{X}}}(\mathbf{A}_1,\tau-1,v_{\tau-1} + \beta)$
		
		\FOR {$l = 0, 1,\dots,2^{\tau-1}-1$}
		\STATE {$\bar{f}_{l+2^{\tau-1}} = a_l^{(0)} + a_l^{(1)}$}
		\STATE {$\bar{f}_l = a_l^{(0)} + \dfrac{s_{\tau-1}(\beta)}{s_{\tau-1}(v_{\tau-1})}\bar{f}_{l+2^{\tau-1}}$}
		\ENDFOR
		\RETURN $\mathbf{\bar{f}}$
	\end{algorithmic}
\end{algorithm}

\section{Generalized Syndrome}\label{sec:syndrome}

In this section, we first review the definition of generalized syndromes, which was first proposed in \cite{araki1992generalized}. Then, we prove that a specific generalized syndrome can be characterized in terms of the high degree part of the IFFT of the received vector. In the next section, based on this specific generalized syndrome, we shall propose a fast decoding algorithm of complexity $O(n\log(n-k) + (n-k)\log^2(n-k))$ for any GRS and alternant codes.

For a codeword $\mathbf{c} = (c_0, c_1, \dots, c_{n-1})$, the received vector can be written as
\begin{align*}
	\mathbf{r} &= (r_0, r_1, \dots, r_{n - 1})\\
	&= (c_0, c_1, \dots, c_{n-1}) + (e_0, e_1, \dots, e_{n-1})\\
	&= \mathbf{c} + \mathbf{e},
\end{align*}
where $\mathbf{e} = (e_0, e_1, \dots, e_{n-1})$ is the error pattern. If $e_i\neq 0$, an error occurs at position $i$. The error location set $E$ is defined as
$E = \{i|e_i\neq 0, i = 0, 1, \dots, n - 1\}$.

For any polynomial $T(x)\in GF(2^m)[x]$ of degree $n-k$, the generalized syndrome of $GRS_k(\Ls, \mathbf{w})$ and $\mathcal{A}_k(\Ls, \mathbf{y})$ is defined as
\begin{equation}\label{eq:syndrome}
	\mathbf{S}(x) = \sum_{i = 0}^{n - 1}r_iy_i\frac{T(x) - T(\alpha_i)}{x - \alpha_i}.
\end{equation}

\begin{lemma}
	The generalized syndrome satisfies that
	\begin{equation*}
		\mathbf{S}(x) = \sum_{i\in E}e_iy_i\frac{T(x) - T(\alpha_i)}{x - \alpha_i}.
	\end{equation*} 
	\begin{proof}
		This proof can be found in \cite{araki1992generalized}. However, we repeat the proof here to make this paper self-contained.
		
		As $H\mathbf{c}^{T} = 0$, a codeword $\mathbf{c} = (c_0, c_1, \dots, c_{n-1})$ satisfies
		\begin{equation*}
			\sum_{i = 0}^{n - 1}c_iy_i\alpha_i^l = 0,\ \text{for}\ l = 0, 1, \dots, n - k - 1.
		\end{equation*}
		If we write $T(x) = \sum_{j = 0}^{n - k} T_j x^j$, we have
		\begin{align*}
			T(x) - T(\alpha_i) &= \sum_{j = 0}^{n - k}T_j (x^j - \alpha_{i}^j)\\
			&= \sum_{j = 1}^{n - k}T_j (x - \alpha_i)\sum_{l = 0}^{j - 1}x^{j - 1 - l}(\alpha_{i})^l.
		\end{align*}
		This leads to
		\begin{equation*}
			\frac{T(x) - T(\alpha_i)}{x - \alpha_i} = \sum_{j = 1}^{n - k}T_j \sum_{l = 0}^{j - 1}x^{j - 1 - l}\alpha_{i}^l.
		\end{equation*}
		Hence, one has
		\begin{align*}
			\mathbf{S}(x) &=  \sum_{i = 0}^{n - 1}r_iy_i\sum_{j = 1}^{n - k}T_j \sum_{l = 0}^{j - 1}x^{j - 1 - l}\alpha_{i}^l\\
			&= \sum_{j = 1}^{n - k}T_j \sum_{l = 0}^{j - 1}x^{j - 1 - l}\sum_{i = 0}^{n - 1}r_iy_i\alpha_i^l\\
			&= \sum_{j = 1}^{n - k}T_j \sum_{l = 0}^{j - 1}x^{j - 1 - l}\sum_{i = 0}^{n - 1}(c_i + e_i)y_i\alpha_i^l\\
			&= \sum_{j = 1}^{n - k}T_j \sum_{l = 0}^{j - 1}x^{j - 1 - l}\sum_{i\in E}e_iy_i\alpha_i^l\\
			&= \sum_{i\in E}e_iy_i\sum_{j = 1}^{n - k}T_j \sum_{l = 0}^{j - 1} x^{j - 1 - l}\alpha_i^l\\
			&= \sum_{i\in E}e_iy_i\frac{T(x) - T(\alpha_i)}{x - \alpha_i}.
		\end{align*}
	\end{proof}
\end{lemma}

Based on the generalized syndrome defined above, the corresponding decoding procedure is described as follows.
Define the error locator polynomial
\begin{equation*}
	\lambda(x) = \prod_{i\in E}(x - \alpha_i).
\end{equation*}
The key equation is 
\begin{equation}\label{eq:key}
	\mathbf{S}(x)\lambda(x) = q(x)T(x) + z(x),
\end{equation}
where
\begin{align*}
	q(x) &= \sum_{i\in E}e_iy_i\prod_{j\in E \atop j\neq i}(x - \alpha_j),\\
	z(x) &= -\sum_{i\in E}e_iy_iT(\alpha_i)\prod_{j\in E \atop j\neq i}(x - \alpha_j).
\end{align*}
Clearly, $\deg(z(x)) < \deg(\lambda(x))$. Solving the key equation, one can obtain $\lambda(x)$ and $z(x)$. By \eqref{eq:key}, one can obtain $q(x)$. 
The error locations can be computed by finding the roots of $\lambda(x)$, and the error values $e_i$ can be computed by
\begin{equation*}
	e_i = \frac{q(\alpha_i)}{y_i^{-1}\lambda{'}(\alpha_i)},
\end{equation*}
where $\lambda{'}(x)$ denotes the formal derivative of $\lambda(x)$.

The above decoding procedure works for any $T(x)$. In the following, we prove that, by specifying $T(x)$, the generalized syndrome can be related to the IFFT of $\mathbf{r}$. We now specify $T(x) = \prod_{j = 0}^{n - k - 1}(x -\omega_j)$ hereafter, in which the definition of $\omega_j$ appears in Section \ref{sec:LCH}. 

The elements in $GF(2^m)$ can be arranged as 
\begin{equation*}
	(\alpha_0, \alpha_1, \dots, \alpha_{n-1}, \alpha_{n}, \dots, \alpha_{2^m-1}),
\end{equation*}
where $\alpha_0, \alpha_2, \dots, \alpha_{n-1}$ have the same order with $\mathcal{L}$ and the residual elements in $GF(2^m)\setminus\mathcal{L}$ are with arbitrary order. Let $\pi$ denotes a permutation between the indices of $(\omega_0, \omega_1, \dots, \omega_{2^m-1})$ and $(\alpha_0, \alpha_1, \dots, \alpha_{2^m-1})$, in which $\pi(j) = i$ if and only if $\omega_j = \alpha_i$. The inverse mapping of $\pi$ is written by $\pi^{-1}$.

Construct a new vector 
\begin{equation*}
	\mathbf{r}^{'} = (r_0^{'}, r_1^{'}, \dots, r_{2^m-1}^{'}),
\end{equation*}
satisfying $r_j^{'}  = 0$ for all $j$ such that $\pi(j) \geq n$ and satisfying $r^{'}_j = r_{\pi{(j)}}y_{\pi(j)}$ otherwise. 

We are now able to rewrite the generalized syndrome \eqref{eq:syndrome} as
\begin{equation}\label{eq:new-symdrome}
	\mathbf{S}(x) = \sum_{j = 0}^{2^m - 1} r_{j}^{'}\frac{T(x) - T(\omega_j)}{x - \omega_j}.
\end{equation}

For the vector $(r_0^{'}, r_1^{'}, \dots, r_{2^m - 1}^{'})$, there exists an unique polynomial $f(x)\in GF(2^m)[x]$ of degree less than $2^m$ such that 
$$f(\omega_j) = r_j^{'}$$
for all $j$. 
By Lagrange interpolation, $f(x)$ can be represented by
\begin{equation*}
	f(x) = \sum_{j = 0}^{2^m - 1}r_j^{'}\frac{s_m(x)}{x - \omega_j},
\end{equation*}
where $s_m(x) = \prod_{l = 0}^{2^m-1}(x - \omega_l)$. Let $\mu$ be the smallest integer such that $\epsilon = 2^{\mu} \geq n-k$. Define
\begin{equation*}
	\mathbf{S}_1(x) = \sum_{j = 0}^{2^m - 1} r_{j}^{'}\frac{s_{\mu}(x) - s_{\mu}(\omega_j)}{x - \omega_j}.
\end{equation*}

\begin{lemma}\label{lem:s1x}
	$\mathbf{S}_1(x)$ is the quotient of $f(x)$ divided by ${\mathbb{X}}_{2^m - 2^{\mu}}(x)$, i.e.,
	\begin{equation}\label{eq:synd1}
		f(x) = \mathbf{S}_1(x){\mathbb{X}}_{2^m - 2^{\mu}}(x) + \eta_1(x),
	\end{equation}
	where $\deg(\eta_1(x)) < \deg({\mathbb{X}}_{2^m - 2^{\mu}}(x))=2^m - 2^{\mu}$.
	
	\begin{proof}
		The case that $\mu = m$ is trivial since ${\mathbb{X}}_0(x) = 1$ and $s_{m}(\omega_j)=0$ for $0\le j\le 2^m-1$.
		
		If $\mu = 0$, one must have $\mathbf{S}_1(x) = \sum_{j=0}^{2^m-1}r_j'$ and the claim obviously holds.
		
		Now we assume that $0 < \mu < m$. $s_m(x)$ can be written as 
		\begin{equation*}
			s_m(x) = (s_{\mu}(x) + s_{\mu}(v_{\mu})){\mathbb{X}}_{2^m - 2^{\mu}}(x) + \eta_2(x),
		\end{equation*}
		where $\deg(\eta_2(x)) < \deg({\mathbb{X}}_{2^m - 2^{\mu}}(x))$ (This can be referred to \cite[Equation (75)]{FFT2016}). It follows that
		\begin{equation}\label{eq:s_m_divide1}
			\frac{s_m(x)}{x - \omega_j} = \frac{(s_{\mu}(x) + s_{\mu}(v_{\mu})){\mathbb{X}}_{2^m - 2^{\mu}}(x) + \eta_2(x)}{x - \omega_j}.
		\end{equation}
		
		Given the polynomial $\frac{s_{\mu}(x) - s_{\mu}(\omega_j)}{x - \omega_j}{\mathbb{X}}_{2^m - 2^{\mu}}(x)$, we have
		\begin{align}
			\frac{s_m(x)}{x - \omega_j} 
			&= \frac{s_{\mu}(x) - s_{\mu}(\omega_j)}{x - \omega_j}{\mathbb{X}}_{2^m - 2^{\mu}}(x) + \eta_3(x)\label{eq:new_divide}\\
			&= \frac{(s_{\mu}(x) - s_{\mu}(\omega_j)){\mathbb{X}}_{2^m - 2^{\mu}}(x) + \eta_3(x)(x - \omega_j)}{x - \omega_j}.\label{eq:s_m_divide2}
		\end{align}
		Combining  \eqref{eq:s_m_divide1} and \eqref{eq:s_m_divide2}, this leads to
		\begin{equation*}
			(s_{\mu}(v_{\mu}) + s_{\mu}(\omega_j)){\mathbb{X}}_{2^m - 2^{\mu}}(x) + \eta_2(x) - \eta_3(x)(x - \omega_j) = 0.
		\end{equation*}
		If $s_{\mu}(v_{\mu}) + s_{\mu}(\omega_j) = 0$, we must have $\deg(\eta_3(x)) < \deg(\eta_2(x)) < \deg(\mathbb{X}_{2^m - 2^{\mu}}(x))$. On the other hand, if $s_{\mu}(v_{\mu}) + s_{\mu}(\omega_j) \neq 0$, it is obviously that $\deg(\eta_3(x)) < \deg(\mathbb{X}_{2^m - 2^{\mu}}(x))$. Hence, according to \eqref{eq:new_divide}, we can conclude that $\frac{s_{\mu}(x) - s_{\mu}(\omega_j)}{x - \omega_j}$ is the quotient of $\frac{s_m(x)}{x - \omega_j}$ divided by $\mathbb{X}_{2^m - 2^{\mu}}(x)$. Finally, by summation on $j$, it is straightforward to see that $\mathbf{S}_1(x)$ is the quotient of $f(x)$ divided by ${\mathbb{X}}_{2^m - 2^{\mu}}(x)$ and the claim follows.
	\end{proof}
\end{lemma}

\begin{lemma}\label{lem:sx}
	$\mathbf{S}(x)$ is the quotient of $\mathbf{S}_1(x)$ divided by $\prod_{l = n - k}^{\epsilon - 1}(x - \omega_l)$, i.e.,
	\begin{equation}\label{eq:syndrome2}
		\mathbf{S}_1(x) = \mathbf{S}(x)\prod_{l = n - k}^{\epsilon - 1}(x - \omega_l) +\eta_4(x),
	\end{equation}
	where $\deg(\eta_4(x)) < \deg(\prod_{l = n - k}^{\epsilon - 1}(x - \omega_l))$.
	\begin{proof}
		If $n - k = \epsilon$, we have $\mathbf{S}(x) = \mathbf{S}_1(x)$ and $\prod_{l = n - k}^{\epsilon - 1}(x - \omega_l) = 1$. The claim obviously holds.
		
		If $n - k < \epsilon$, we have
		\begin{equation*}
			s_{\mu}(x) = T(x)\prod_{l = n - k}^{\epsilon - 1}(x - \omega_l).
		\end{equation*}
		So
		\begin{equation}\label{eq:s_mu_divide1}
			\frac{s_{\mu}(x) - s_{\mu(\omega_j)}}{x - \omega_j} = \frac{T(x)\prod_{l = n - k}^{\epsilon - 1}(x - \omega_l) - s_{\mu(\omega_j)}}{x - \omega_j}.
		\end{equation}
		Given the polynomial $\frac{T(x) - T(\omega_j)}{x - \omega_j}\prod_{l = n - k}^{\epsilon - 1}(x - \omega_l)$, one can write
		\begin{align}
			&\frac{s_{\mu(x)} - s_{\mu(\omega_j)}}{x - \omega_j} \notag\\=
			&\frac{T(x) - T(\omega_j)}{x - \omega_j}\prod_{l = n - k}^{\epsilon - 1}(x - \omega_l) + \eta_5(x)\label{eq:new_divide1}\\
			=&\frac{T(x)\prod_{l = n - k}^{\epsilon - 1}(x - \omega_l) - T(\omega_j)\prod_{l = n - k}^{\epsilon - 1}(x - \omega_l)}{x - \omega_j}\notag\\
			&+\frac{\eta_5(x)(x - \omega_j)}{x - \omega_j}\label{eq:s_mu_divide2}
		\end{align}
		Combining \eqref{eq:s_mu_divide1} and \eqref{eq:s_mu_divide2}, this leads to
		\begin{equation*}
			T(\omega_j)\prod_{l = n - k}^{\epsilon - 1}(x - \omega_l) - s_{\mu}(\omega_j) - \eta_5(x)(x - \omega_j) = 0.
		\end{equation*}
		Thus one must have $\deg(\eta_5(x)) <  \deg(\prod_{l = n - k}^{\epsilon - 1}(x - \omega_l))$. Hence, according to \eqref{eq:new_divide1}, $\frac{T(x) - T(\omega_j)}{x - \omega_j}$ is the quotient of $\frac{s_{\mu(x)} - s_{\mu(\omega_j)}}{x - \omega_j}$ divided by $\prod_{l = n - k}^{\epsilon - 1}(x - \omega_l)$. By summation on $j$, one can easily verify that the claim holds.
	\end{proof}
\end{lemma}

Lemma \ref{lem:s1x} and \ref{lem:sx} say that $\mathbf{S}(x)$ is the quotient of $\mathbf{S}_1(x)$ divided by $\prod_{l = n - k}^{\epsilon - 1}(x - \omega_l)$, where $\mathbf{S}_1(x)$ is the quotient of $f(x)$ divided by $\bar{\mathbb{X}}_{2^m - 2^{\mu}}(x)$. Therefore, the generalized syndrome $\mathbf{S}(x)$ is determined by the high degree part of $f(x)$. Since $f(x)$ is the IFFT of the vector $\mathbf{r}^{'}$, we have shown that the specific generalized syndrome can be related to the IFFT of $\mathbf{r}^{'}$. Furthermore, as a linear transform relates to any two syndrome definitions, the above comment also proves that any syndrome definition can be characterized in terms of the IFFT regardless of code definition. In the next section, we provide an efficient decoding algorithm base on $\mathbf{S}(x)$.

\section{Proposed Decoding Algorithm}\label{sec:dec}

This section proposes a unified decoding algorithm for $GRS_k(\Ls, \mathbf{w})$ and $\mathcal{A}_k(\Ls, \mathbf{y})$ codes. The term "unified" means that no matter what $\Ls$ and $\mathbf{y}$ are, the algorithm is able to decode the received vector. Furthermore, its computational complexity is $O(n\log(n-k) + (n-k)\log^2(n-k))$ (Note that we assume that $n > 2^{m - 1}$ here. This assumption is reasonable since if $n \leq 2^{m - 1}$, one can construct the code in a smaller field $GF(2^{m-1})$).

Based on the generalized syndrome, the decoding algorithm can be outlined as four steps: 
\begin{enumerate}
	\item computing the generalized syndrome $\mathbf{S}(x)$; 
	\item solving the key equation
	\begin{equation*}
		\mathbf{S}(x)\lambda(x) = q(x)T(x) + z(x);
	\end{equation*}
	\item determining the error locations by finding the roots of $\lambda(x)$; 
	\item computing the error values for the corresponding nonbinary codes.
\end{enumerate}
We provide the detailed algorithm for these four steps in the following. It should be noted that for any positive integer $\upsilon\leq 2^m$, there exist $\upsilon$-points Fourier transform $\text{FFT}_{\bar{\mathbb{X}}}$ and its inverse transform $\text{IFFT}_{\bar{\mathbb{X}}}$, whose complexity are $O(\upsilon\log(\upsilon))$. These transforms are referred to in Appendix of \cite{tang2022a}.

\subsection{Syndrome Computation}
Let $\mathbf{r}^{'}_{l,\epsilon}$ denote a sub-vector of $\mathbf{r}^{'}$:
\begin{equation*}
	\mathbf{r}^{'}_{l,\epsilon} = (r^{'}_{l\cdot\epsilon}, r^{'}_{l\cdot\epsilon + 1}, \dots, r^{'}_{l\cdot\epsilon + \epsilon - 1}).
\end{equation*}
\begin{lemma}
	The coordinate vector of $\mathbf{S}_1(x)$ with respect to $\bar{\mathbb{X}}$ is equal to
	\begin{equation}\label{eq:syndrome_compute}
		\sum_{l = 0}^{2^{m - \mu} - 1}\IFFT_{{\bar{\mathbb{X}}}}(\mathbf{r}^{'}_{l,\epsilon}, \mu, \omega_{\epsilon \cdot l})/p_{2^m - 2^{\mu}}.
	\end{equation}
	\begin{proof}
		Recall that we have $f(\omega_j)  = {r}^{'}_j$ for all $j$. Thus the coordinate vector of $f(x)$ with respect to $\bar{\mathbb{X}}$ can be computed by $\IFFT_{{\bar{\mathbb{X}}}}(\mathbf{r}^{'}, m, \omega_0)$. For any $0\leq \mu \leq m$ and $\epsilon = 2^{\mu}$, the vector
		\begin{equation*}
			(\bar{f}_{2^m - \epsilon}, \bar{f}_{2^m - \epsilon + 1}, \dots, \bar{f}_{2^m - 1}) = \sum_{l = 0}^{2^{m - \mu} - 1}\IFFT_{{\bar{\mathbb{X}}}}(\mathbf{r}^{'}_{l,\epsilon}, \mu, \omega_{\epsilon \cdot l}).
		\end{equation*}
		This property is referred to \cite[Lemma 10]{FFT2016}.
		
		On the other hand, the polynomial $f(x)$ can be represented as
		\begin{align*}
			f(x) = &\sum_{j = 0}^{2^m-1}\bar{f}_j\bar{X}_j(x)\\
			= &\sum_{j = 0}^{2^m-\epsilon-1}\bar{f}_j\bar{X}_j(x) + \sum_{j = 2^m-\epsilon}^{2^m - 1}\bar{f}_j\bar{X}_j(x)\\
			= &\sum_{j = 0}^{2^m-\epsilon-1}\bar{f}_j\bar{X}_j(x) + \bar{X}_{2^m-\epsilon}(x)\sum_{j = 2^m-\epsilon}^{2^m - 1}\bar{f}_j\bar{X}_{j - 2^m-\epsilon}(x)\\
			= &\sum_{j = 0}^{2^m-\epsilon-1}\bar{f}_j\bar{X}_j(x) \\
			&+ {X}_{2^m-\epsilon}(x)\sum_{j = 2^m-\epsilon}^{2^m - 1}\frac{\bar{f}_j}{p_{2^m - 2^{\mu}}}\bar{X}_{j - 2^m-\epsilon}(x)\\
			= &\sum_{j = 0}^{2^m-\epsilon-1}\bar{f}_j\bar{X}_j(x) + {X}_{2^m-\epsilon}(x)\sum_{j = 0}^{2^{\mu} - 1}\frac{\bar{f}_{j + 2^m - 2^{\mu}}}{p_{2^m - 2^{\mu}}}\bar{X}_{j}(x)
		\end{align*}
		According to \eqref{eq:synd1}, we have
		\begin{align*}
			\mathbf{S}_1(x) &= \sum_{j = 0}^{2^{\mu} - 1}\frac{\bar{f}_{j + 2^m - 2^{\mu}}}{p_{2^m - 2^{\mu}}}\bar{X}_{j}(x)
		\end{align*}
		Thus, the coordinate vector of $\mathbf{S}_1(x)$ with respect to $\bar{\mathbb{X}}$ is equal to
		\begin{equation*}
			\sum_{l = 0}^{2^{m - \mu} - 1}\IFFT_{{\bar{\mathbb{X}}}}(\mathbf{r}^{'}_{l,\epsilon}, \mu, \omega_{\epsilon \cdot l})/p_{2^m - 2^{\mu}}.
		\end{equation*}
		This completes the proof.
	\end{proof}
\end{lemma}

Given the coordinate vector of $\mathbf{S}_1(x)$, it remains to calculate $\mathbf{S}(x)$. Recall the equation \eqref{eq:syndrome2}:
\begin{equation*}
	\mathbf{S}_1(x) = \mathbf{S}(x)\prod_{l = n - k}^{\epsilon - 1}(x - \omega_l) +\eta_4(x).
\end{equation*}
One has $\eta_4(\omega_l) = \mathbf{S}_1(\omega_l)$ for $l = n - k, \dots, \epsilon - 1$. As $\deg(\eta_4(x)) < \epsilon - n + k$, $\eta_4(x)$ can be determined by $\epsilon - n + k$ points $\text{IFFT}_{\bar{\mathbb{X}}}$. Furthermore, we have
\begin{equation}\label{eq:syndrome3}
	\mathbf{S}(\omega_j) = (\mathbf{S}_1(\omega_j) - \eta_4(\omega_j))(\prod_{l = n - k}^{\epsilon - 1}(\omega_j - \omega_l))^{-1}.
\end{equation}
Then we can obtain $\mathbf{S}(\omega_j)$ for $j = 0, 1, \dots, n - k - 1$. Finally, the polynomial $\mathbf{S}(x)$ can be computed by $\text{IFFT}_{{\bar{\mathbb{X}}}}$.

We now analyze the computational complexity. Computing the coordinate vector of $\mathbf{S}_1(x)$ costs $O(n\log(n-k))$ operations and evaluating $\mathbf{S}_1(x)$ on $\omega_0, \omega_1, \dots, \omega_{\epsilon - 1}$ costs $O((n-k)\log(n-k))$ operations. Next, determining $\eta_4(x)$ by $\text{IFFT}_{{\bar{\mathbb{X}}}}$ and evaluating it on $\omega_0, \omega_1, \dots, \omega_{\epsilon - 1}$ takes $O((n-k)\log(n-k))$ operations. Finally, the complexity of computing $\mathbf{S}(\omega_j)$ for $j = 0, 1, \dots, n - k - 1$ according to \eqref{eq:syndrome3} is $O(n)$ and the complexity of obtaining $\mathbf{S}(x)$ by $\text{IFFT}_{{\bar{\mathbb{X}}}}$ is $O((n-k)\log(n-k))$. Hence, the complexity of syndrome computation is $O((n-k)\log(n-k))$.

\subsection{The Key Equation}

The Key equation is 
\begin{equation}\label{eq:key_equation}
	\mathbf{S}(x)\lambda(x) = q(x)T(x) + z(x).
\end{equation}
The Euclidean algorithm can solve this key equation. Furthermore, recall that $T(x) = \prod_{j = 0}^{n - k - 1}(x - \omega_j)$. This key equation is an interpolation problem and can be solved by the modular approach within $O((n-k)\log^2(n-k))$ field operations. More details about the modular approach are referred to \cite{tang2022a}.

\subsection{Chien Search}
Once the error locator polynomial $\lambda(x)$ has been found. We can compute the roots of $\lambda(x)$ by
\begin{equation}\label{eq:FFT_search}
	\FFT_{{\bar{\mathbb{X}}}}(\bar{\mathbf{\lambda}}, \mu, \omega_{\epsilon \cdot l}), 0\leq l < 2^{m - \mu},
\end{equation}
where $\bar{\mathbf{\lambda}}$ represents the coordinate vector of $\lambda(x)$ with respect to $\bar{\mathbb{X}}$. The step takes $O(n\log(n-k))$ operations.

\subsection{Forney's formula}

For nonbinary codes, we should compute the error values for each error location. If $\omega_j$ is a root of $\lambda(x)$, the formula for computing the error value at location $\omega_j$ is
\begin{equation}\label{eq:Forney}
	e_j = \frac{q(\omega_j)}{y_i^{-1}\lambda{'}(\omega_j)}.
\end{equation}
Clearly, the computational complexity of this step is at most $O((n-k)\log(n-k))$.

To sum up, the complexity of the proposed unified decoding algorithm is $O(n\log(n-k) + (n-k)\log^2(n-k))$.


\begin{algorithm}[h]
	
	\caption{{Unified Decoding Algorithm}}
	\label{alg:decoding}
	\begin{algorithmic}[1]
		\REQUIRE
		Received vector $\mathbf{r} = \mathbf{c} + \mathbf{e}$.
		\ENSURE
		The codeword $\mathbf{c}$.
		\STATE Compute the syndrome polynomial $\mathbf{S}_1(x)$ according to \eqref{eq:syndrome_compute}.
		\STATE Evaluate $\mathbf{S}_1(x)$ at points $\omega_0, \omega_1, \dots, \omega_{\epsilon-1}$ by Algorithm \ref{FFTX}.
		\STATE Given $\eta_{4}(\omega_{l}) = \mathbf{S}_1(\omega_{l})$ for $l = n - k, \dots, \epsilon - 1$, call $\epsilon - n + k$ points $\text{IFFT}_{\bar{\mathbb{X}}}$ to get $\eta_{4}(x)$.
		\STATE Evaluate $\eta_{4}(x)$ at $\omega_0, \dots, \omega_{\epsilon - 1}$ by $\epsilon$-points FFT and compute $\mathbf{S}(\omega_l)$ for $l = 0, 1, \dots, \epsilon - 1$ according to \eqref{eq:syndrome3};
		\STATE Solve the key equation \eqref{eq:key_equation} by the Euclidean algorithm or the modular approach.
		\STATE Find the error locations by \eqref{eq:FFT_search}.
		\STATE Compute the error pattern $\mathbf{e}$ by \eqref{eq:Forney}.
		\RETURN
		$\mathbf{r} + \mathbf{e}$.
	\end{algorithmic}
	
\end{algorithm}



\section{Separable Goppa Codes}\label{sec:Goppa}

In this section, we discuss the decoding algorithm for the most common separable Goppa codes. Furthermore, it is straightforward to generalize the following discussion for separable Goppa codes to general Goppa codes.

Given $\Ls = (\alpha_0, \alpha_1, \dots, \alpha_{n-1})$, for any vector $\mathbf{a} = (a_0, a_1, \dots, a_{n-1})$ over $GF(2)$, we associate the rational function
\begin{equation*}
	R_{\mathbf{a}}(x) = \sum_{i = 0}^{n - 1}\frac{a_i}{x - \alpha_i}.
\end{equation*}

The Goppa code $\Gamma(\mathcal{L}, G)$ consists of all vectors $\mathbf{a}$ such that
\begin{equation*}
	R_{\mathbf{a}}(x) \equiv 0\bmod{G(x)},
\end{equation*}
in which the polynomial $G(x)\in GF(2^m)$ satisfying $G(\alpha_i)\neq 0$ for $0 \leq i < n$. $G(x)$ is called the Goppa polynomial. If $G(x)$ has no multiple roots, the code is called \textit{separable}. If $G(x)$ is irreducible, the code is called \textit{irreducible}. Evidently, an irreducible Goppa code is separable. 

We assume that $G(x)$ has no multiple roots hereafter, and thus, the corresponding Goppa code is separable.
Given a codeword $\mathbf{a} = (a_0, a_1, \dots, a_{n-1})$ of weight $\kappa$ in $\Gamma(\mathcal{L}, G)$, let $a_{i_0}, a_{i_1}, \dots, a_{i_{\kappa-1}}$ represent the nonzero elements. Define
\begin{equation*}
	\gamma(x) = \prod_{l = 0}^{\kappa - 1}(x - \alpha_{i_l}).
\end{equation*}
The formal derivative of $\gamma(x)$ is
\begin{equation*}
	\gamma^{'}(x) = \sum_{l = 0}^{\kappa - 1}\prod_{j \neq l}(x - \alpha_{i_j}).
\end{equation*}
It follows that
\begin{equation*}
	R_{\mathbf{a}}(x) =  \frac{\gamma^{'}(x)}{\gamma(x)}.
\end{equation*}
As $G(\alpha_i) \neq 0$ for all $i$, $G(x)$ is relatively prime to $\gamma(x)$. Therefore we have $G(x)\mid \gamma^{'}(x)$ since $R_{\mathbf{a}}(x) \equiv 0\bmod{G(x)}$. Note that the coefficients of these polynomials are in a field of characterizing $2$. So there are only even power in $\gamma^{'}(x)$ and $\gamma^{'}(x)$ is a perfect square. Because $G(x)$ has no multiple roots, $\bar{G}(x) = G^2(x)$ is the lowest degree perfect square that is divisible by $G(x)$. It follows that $\bar{G}(x)\mid \gamma^{'}(x)$. Due to $\bar{G}(x)$ must be relatively prime to $\gamma(x)$, we can conclude that
\begin{equation*}
	R_{\mathbf{a}}(x) \equiv 0\bmod{G(x)},
\end{equation*}
if and only if
\begin{equation*}
	R_{\mathbf{a}}(x) \equiv 0\bmod{\bar{G}(x)}.
\end{equation*}
This implies that
\begin{equation*}
	\Gamma(\mathcal{L}, G) = \Gamma(\mathcal{L}, \bar{G}).
\end{equation*}

According to the above discussion, the parity check matrix of $\Gamma(\mathcal{L}, G)$ can be written as
\begin{align}\label{eq:H-Goppa}
	H &= \left(
	\begin{array}{cccc}
		y_0 & y_1 & \dots & y_{n - 1} \\
		y_0\alpha_0 & y_1\alpha_1 & \dots & y_{n - 1}\alpha_{n - 1}\\
		\vdots & \vdots & \dots & \vdots\\
		y_0\alpha_0^{2\rho - 1} & y_1\alpha_1^{2\rho - 1} & \dots & y_{n - 1}\alpha_{n - 1}^{2\rho - 1}
	\end{array}
	\right),
\end{align}
where $y_i = \bar{G}(\alpha_i)^{-1}$ for all $i$ and $\rho = \deg(G(x))$. Hence, the decoding algorithm proposed in Section \ref{sec:dec} can be used for $\Gamma(\mathcal{L}, G)$ and any error with weight less than or equal to $\rho$ can be corrected.

The traditional encoding algorithm for Goppa codes is to extend the $H$ matrix given in ~\eqref{eq:H-Goppa} into a binary matrix $H'$ by replacing each entry in $H$ with its corresponding $m$-tuple over $GF(2)$ arranged in column form. Let $G$ be a binary generator matrix of the Goppa code. Then we have $G(H')^T=\mathbf{0}$, where $T$ is the matrix transpose and $\mathbf{0}$ is the all-zero matrix. Hence, $G$ can be used to encode the information bits.
We next prove that any encoding method resulting in the parity check matrix given in~\eqref{eq:H-Goppa} is suitable for the case when LCH-FFT is being used in the decoding procedure. We prove this fact by giving a systematic encoding procedure to be performed on $\bar{\mathbb{X}}$, which produces the same codeword as the encoding method, resulting in the parity check matrix given in ~\eqref{eq:H-Goppa}. Note that the systematic encoding procedure is an alternative encoding algorithm based on LCH-FFT.

Let $\mathbf{u}=(u_0,u_1,\ldots,u_{k-1})$ be the binary information vector. For ease of presentation, we assume $k$ is a power of $2$. The following is the systematic encoding procedure.
\begin{enumerate}
	\item Calculate 
	$$w_i=\bar G(\alpha_i)\left(\prod_{0\le j\le n-1,j\neq i}(\alpha_i-\alpha_j)\right)^{-1},\ 0\le i\le n-1.$$ This step can be pre-calculated before the decoding procedure.
	\item Calculate $\mathbf{u}'=(u_0',u_1,\ldots, u_{k-1}')$, where
	$$u_i'=u_iw_{\pi(i)}^{-1}.$$
	\item Perform IFFT on $\mathbf{u}'$ to obtain
	$$\bar{\mathbf{u}}=\text{IFFT}_{\bar{\mathbb{X}}}(\mathbf{u}', \log k, \omega_0).$$
	\item Perform FFT on different $\beta$ to obtain
	$$\bar{\mathbf{c}}=(\mathbf{u}',\text{FFT}_{\bar{\mathbb{X}}}(\bar{\mathbf{u}}, \log k, \omega_{k}),\text{FFT}_{\bar{\mathbb{X}}}(\bar{\mathbf{u}}, \log k, \omega_{2k}),\ldots, \text{FFT}_{\bar{\mathbb{X}}}(\bar{\mathbf{u}}, \log k, \omega_{(n/k-1)k}))$$
	when $k|n$; otherwise
	$$\bar{\mathbf{c}}=(\mathbf{u}',\text{FFT}_{\bar{\mathbb{X}}}(\bar{\mathbf{u}}, \log k, \omega_{k}),\text{FFT}_{\bar{\mathbb{X}}}(\bar{\mathbf{u}}, \log k, \omega_{2k}),\ldots, \{\text{FFT}_{\bar{\mathbb{X}}}(\bar{\mathbf{u}}, \log k, \omega_{\lfloor n/k\rfloor k})\}_r),$$
	where the operation $\{\cdot\}_r$ takes the first $r$ element of the vector and $r$ is the reminder when $n$ is divided by $k$.
	\item Multiply each element of $\bar{\mathbf{c}}$ by $w_{\pi(i)}$ to obtain $\mathbf{c}'$, where 
	$$c'_i=\bar{c}_iw_{\pi(i)}.$$
	\item Obtain the codeword $\mathbf{c}$ by permuting elements of $\mathbf{c}'$ by $\pi^{-1}$, where
	$$c_i=c_{\pi^{-1}(i)}.$$
\end{enumerate}
The overall complexity of the encoding procedure is $O(n\log k)$.




Complexity comparisons are provided in Table \ref{tab:Goppa3488}  and Table \ref{tab:Goppa8192}. The code parameters are chosen from the PQC scheme that has been submitted to NIST; see \cite{NIST_PQC} for more details. The code in Table \ref{tab:Goppa3488} is defined over $GF(2^{12}) = GF(2)[x] / (x^{12} + x^3 + 1)$, which is of length $3488$. The Goppa polynomial is $y^{64} + y^3 + y + x$, and its error correction capability is equal to $64$. The code in Table \ref{tab:Goppa8192} is defined over $GF(2^{13}) = GF(2)[x] / (x^{13} + x^4 + x^3 + x + 1)$, which is of length $8192$. The Goppa polynomial is $y^{128} + y^7 + y^2 + y + 1$ and its error correction capability is equal to $128$. The conventional method, the Patterson method, refers to Algorithm 4 in \cite{patterson1975algebraic}. The comparisons are by counting the field operations needed for decoding a codeword, which has been added the most errors that can be corrected. The comparisons given in Table \ref{tab:Goppa3488} and Table \ref{tab:Goppa8192} show that the proposed algorithm is five times and ten times less in terms of computational complexity than the traditional methods, respectively.

\begin{table}[h]
	\caption{Complexity Comparison for decoding Goppa code of length $3488$ and correction ability $64$.}
	\begin{center}
		%		\resizebox{.95\columnwidth}{!}
		{
			\begin{tabular}{|c|c|c|c|}
				\hline
				{Decoding method} & {Additions} & {Multiplications}&
				{Inversions}\\
				\hline
				MacWilliams method & 693,857 & 689,889 & 3,617 \\
				\hline
				Patterson method & 475,079 & 466,458 & 189\\
				\hline
				Proposed method & 103,720 & 63,568 & 128 \\
				\hline
			\end{tabular}
		}
	\end{center}
	\label{tab:Goppa3488}
\end{table}

\begin{table}[h]
	\caption{Complexity Comparison for decoding Goppa code of length $8192$ and correction ability $128$.}
	\begin{center}
		%		\resizebox{.95\columnwidth}{!}
		{
			\begin{tabular}{|c|c|c|c|}
				\hline
				{Decoding method} & {Additions} & {Multiplications}&
				{Inversions}\\
				\hline
				MacWilliams method & 3,235,840 & 3,219,712 & 8,448 \\
				\hline
				Patterson method & 2,206,791 & 2,171,130 & 381\\
				\hline
				Proposed method & 243,176 & 148,976 & 256 \\
				\hline
			\end{tabular}
		}
	\end{center}
	\label{tab:Goppa8192}
\end{table}


\section{Conclusion}\label{sec:conclusion}
This paper characterizes the generalized syndrome in terms of the IFFT of GRS codes and alternant codes and proposes a fast decoding algorithm of complexity $O(n\log(n-k) + (n-k)\log^2(n-k))$, which is the best complexity to date. This algorithm is suitable for all GRS codes and alternant codes. When decoding practical codes, for example, the Goppa codes used in post-quantum cryptography, a significant improvement is obtained in terms of computational complexity. 

Open issues include whether the Guruswami-Sudan algorithm for RS codes can be accelerated by FFT or not and whether the proposed algorithm can be improved further for irreducible Goppa codes.



%% The very first letter is a 2 line initial drop letter followed
%% by the rest of the first word in caps.
%% 
%% form to use if the first word consists of a single letter:
%% \IEEEPARstart{A}{demo} file is ....
%% 
%% form to use if you need the single drop letter followed by
%% normal text (unknown if ever used by the IEEE):
%% \IEEEPARstart{A}{}demo file is ....
%% 
%% Some journals put the first two words in caps:
%% \IEEEPARstart{T}{his demo} file is ....
%% 
%% Here we have the typical use of a "T" for an initial drop letter
%% and "HIS" in caps to complete the first word.


\ifCLASSOPTIONcaptionsoff
  \newpage
\fi
\bibliographystyle{IEEEtran}
\bibliography{IEEEabrv,refs_final}

%\begin{thebibliography}{1}
	
%\bibitem{Polynomial1960}
%Reed I S, Solomon G. Polynomial codes over certain finite fields[J]. \textit{Journal of the society for industrial and applied mathematics}, 1960, 8(2): 300-304.
%
%\bibitem{Error1984}
%Welch L R, Berlekamp E R. Error correction for algebraic block codes: U.S. Patent 4,633,470[P]. 1986-12-30.
%
%%\bibitem{Novel2016}
%%S. J. Lin, T. Y. Al-Naffouri, Y. S. Han and W. H. Chung, ``Novel polynomial basis with fast Fourier transform and its application to Reed–Solomon erasure codes," \textit{IEEE Transactions on Information Theory}, vol.62, no.12, pp. 6284-6299, Nov. 2016.
%
%\bibitem{Error2005}
%Moon T K. Error correction coding: mathematical methods and algorithms[M]. \textit{John Wiley \& Sons}, 2020.
%
%\bibitem{Algebraic2003}
%Blahut R E. Algebraic codes for data transmission[M]. \textit{Cambridge university press}, 2003.
%
%\bibitem{FFT2016}
%S. J. Lin, T. Y. Al-Naffouri and Y. S. Han, ``FFT algorithm for binary extension finite fields and its application to Reed–Solomon codes," \textit{IEEE Transactions on Information Theory}, vol.62, no.10, pp. 5343-5358, Oct. 2016.
%
%\bibitem{Fast2020}
%Tang N, Lin Y. Fast Encoding and Decoding Algorithms for Arbitrary $(n, k) $ Reed-Solomon Codes Over $\mathbb {F} _ {2^ m} $[J]. \textit{IEEE Communications Letters}, 2020, 24(4): 716-719.
%
%\bibitem{Fast1995}
%Dabiri D, Blake I F. Fast parallel algorithms for decoding Reed-Solomon codes based on remainder polynomials[J]. \textit{IEEE Transactions on Information Theory}, 1995, 41(4): 873-885.
%
%\end{thebibliography}

% biography section
% 
% If you have an EPS/PDF photo (graphicx package needed) extra braces are
% needed around the contents of the optional argument to biography to prevent
% the LaTeX parser from getting confused when it sees the complicated
% \includegraphics command within an optional argument. (You could create
% your own custom macro containing the \includegraphics command to make things
% simpler here.)
%\begin{IEEEbiography}[{\includegraphics[width=1in,height=1.25in,clip,keepaspectratio]{mshell}}]{Michael Shell}
% or if you just want to reserve a space for a photo:

%\begin{IEEEbiography}{Michael Shell}
%Biography text here.
%\end{IEEEbiography}

% if you will not have a photo at all:
%\begin{IEEEbiographynophoto}{Nianqi Tang}
%received Ph.D. degree from Xidian University, Xi'an, China, in 2019. Then he joined Huawei Technologies Co., Ltd, where he is now an engineer. His research includes error control coding and information theory.
%\end{IEEEbiographynophoto}

% insert where needed to balance the two columns on the last page with
% biographies
%\newpage

%\begin{IEEEbiographynophoto}{Jane Doe}
%Biography text here.
%\end{IEEEbiographynophoto}

% You can push biographies down or up by placing
% a \vfill before or after them. The appropriate
% use of \vfill depends on what kind of text is
% on the last page and whether or not the columns
% are being equalized.

%\vfill

% Can be used to pull up biographies so that the bottom of the last one
% is flush with the other column.
%\enlargethispage{-5in}



% that's all folks
\end{document}


