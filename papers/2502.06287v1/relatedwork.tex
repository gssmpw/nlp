\section{Related Work}
\subsection{Few-Anchor UWB Localization}
Few-anchor UWB localization has recently attracted increasing attention and has been widely studied due to its low complexity and cost-effective deployment~\cite{tong2022single,9829196,penggang2022novel}. 
These research works focus exclusively
on fusion schemes that integrate UWB measurements with other motion sensors. Mainstream works primarily employ filter-based, optimization-based, and two-stage methods to incorporate additional velocity observations. 
In filter-based solutions, Cao~\textit{et al.}~\cite{cao2020accurate} employed Extended Kalman Filtering (EKF) to combine a single UWB anchor with a nine-DOF IMU for velocity and orientation estimation. Qin~\textit{et al.}~\cite{qin2024single} designed a Set Membership Filter (SMF) method based on constrained zonotopes for single-beacon localization problems. 
Gao~\textit{et al.}~\cite{penggang2022novel} introduced UWB ranging, non-holonomic constraints from the carrier, and trajectory constraints as observations of the EKF system. The proposed method can realize positioning in corridor-like areas. However, it relies on non-holonomic and corridor-specific trajectory constraints, which may not be applicable in unknown environmental structures. 
In optimization-based solutions, Li~\textit{et al.}~\cite{li2021computationally} proposed a Moving Horizon Estimation (MHE) algorithm, which utilizes historical measurement data from a single UWB anchor and IMU during a time horizon. Additionally, a Gradient Aware Levenberg-Marquardt (GALM) algorithm was further proposed to solve the optimization problem with a calculation cost. 
For two-stage methods, Yang~\textit{et al.}~\cite{yang2023novel} proposed an improved positioning method that leverages Bayesian optimization to provide the estimated IMU data based on an error model. A filter then updates the position at the ranging time using the prior position error. 
Zhou~\textit{et al.}~\cite{zhou2024optimization} proposed a two-stage optimization method that could yield accurate solutions with the single-anchor and odometer.
In the first stage, they construct a factor graph that incorporates odometry positions and UWB measurements to optimize state estimation. In the second stage, an adaptive trust region algorithm is performed to refine the location estimate and maintain robustness with inequality constraints.  While these methods approximate the time offsets between state estimation and sensor measurements, the hypothesis of time synchronization may not always hold in practical scenarios. Unlike the existing few-anchor UWB localization methods, our method
formulates the trajectory in continuous-time with B-spline, which allows asynchronous, high-frequency sensor measurements to be aligned at any time instants along the trajectory, rather than relying on linear interpolation of discrete-time poses only at measurement times.

\subsection{Continuous-Time Representation Using B-spline}
Continuous-time representation is a popular and natural choice for formulating trajectories and ensuring smoothness. 
It can estimate the robot's pose as a continuous function of time without the need to introduce additional states at every measurement time. The most common continuous-time-based model is based on splines and Gaussian processes~\cite{cioffi2022continuous}. 
In this article, we focus on B-spline-based trajectory representation,  current literature has paid increasing attention to applying B-splines for continuous-time state estimation in multi-sensor fusion systems, including the LiDAR-Inertial system, visual-inertial system, ultra-wideband-inertial and so on.
Nguyen~\textit{et al.}~\cite{nguyen2024eigen} developed a real-time continuous-time LiDAR-inertial odometry (SLICT2), which achieved efficient optimization with few iterations using a simple solver.
Lu~\textit{et al.}~\cite{lu2023event} proposed an event-based visual-inertial velometer that incrementally incorporates measurements from a stereo event camera and IMU. 
Li~\textit{et al.}~\cite{li2023continuous} proposed a spline-based approach (SFUISE) for continuous-time Ultra-wideband-Inertial sensor fusion, which addressed the limitations of discrete-time sensor fusion schemes in asynchronous multi-sensor fusion and online calibration. 
These methods generally employ uniform knot B-splines for trajectory modeling, relying on assumptions of zero velocity or constant speed. However, these assumptions often fail to capture the dynamic nature of real-world motion. Non-uniform B-splines offer a more flexible distribution of control points, enabling a different density of control points based on the smoothness of the trajectory segment. 
Ovrén~\textit{et al.}~\cite{ovren2018spline} introduced an energy proportionality index in spline fitting to optimize the selection of knot spacing. 
By leveraging the different frequency response characteristics of spline basis functions and specific energy values, appropriate knot spacing can be automatically selected. 
Lang~\textit{et al.}~\cite{lang2023coco} tightly coupled the measurements from LiDAR, IMU, and cameras using non-uniform B-spline curves. 
They adaptively adjusted the number of control points based on IMU observations to detect different motion patterns, thereby improving adaptability to complex environments and motion patterns. However, relying solely on IMU data leads to velocity estimate divergence, which impacts the distribution of control points and ultimately reduces positioning accuracy. 
In contrast, the proposed method integrates an IMU/Odometer fusion model to provide short-term accurate velocity estimates, enabling the dynamic adjustment of control point density for more precise trajectory modeling. 

\begin{figure*}[!t]
\centering
\includegraphics[width=\linewidth,trim=1cm 7cm 2cm 2cm,clip]{1.pdf}
\caption{The framework of CT-UIO: In the preprocessing stage, UWB ranging data is detected, and outliers are removed. Simultaneously,  incoming IMU and odometer data are fused using an adaptive EKF to provide short-term accurate motion priors. In the front-end, the results of the IMU/odometer fusion model are combined with UWB ranging to generate virtual anchors. Additionally,  Based on the motion estimates from the IMU/odometer fusion model, the adaptive knot span adjustment strategy non-uniformly places control points. In the back-end, we conduct a CT-UIO factor graph with an adaptive sliding window for global trajectory estimation.}
\label{fig_0}
\end{figure*}