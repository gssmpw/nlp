\appendix

\section{Prompts Details}
\subsection{Prompt  for preference evaluation}
\label{sec:pref-eval}
\begin{lstlisting}[basicstyle=\ttfamily\scriptsize, breaklines=true]
You are given 2 responses, and a Golden Label. Please, decide which of the responses is the best (you have to take into account its factual correctness, and readability)

    Response 1: {pred1}
    Response 2: {pred2}
    Golden Label: {gold_label}

Return 1 if the first response is better, 2 if the second one is better, 0 if they are equally good. Return only the number
\end{lstlisting} 

\subsection{Prompt for creating conversational data }
\label{sec:pref-eval}
\begin{lstlisting}[basicstyle=\ttfamily\scriptsize, breaklines=true]
Prompt: 
Instruction: {input_instruction}
Output: {output}
You are creating conversational data between two people discussing {topic} in Kazakhstan. The conversation should:  

- Start with a general but relevant topic that smoothly leads into the instruction's question.  
- Be short and concise, where Person 1 initiates the discussion and then asks a question with a meaning similar to the instruction at the end.  
- Person 2 should respond but not reveal the output
- the conversation ends with "Person 2 says: ...".  
- Maintain natural, formal dialogue relevant to government regulations in Kazakhstan.  

Example Format:  
Person 1 says: [General opening statement leading to the topic]  
Person 2 says: [Relevant response that naturally progresses the discussion]  
Person 1 says: [Rephrased instruction question]  
Person 2 says: ...
\end{lstlisting} 


\subsection{Prompts for Generation Instruction Dataset}
\label{sec:prompts}

\begin{lstlisting}[basicstyle=\ttfamily\scriptsize, breaklines=true, aboveskip=0pt, belowskip=0pt, lineskip=-2pt, xleftmargin=0pt]
You are given a text: {TEXT}.
I want to extract all the facts from the given text. Based on the extracted facts, I want you to create instruction fine-tuning pairs in Kazakh. 
The pairs may fall into the following categories, but you are free to use other relevant categories if appropriate:
- Is it true that ...
- Explain ...
- Describe ...
- List the steps ...
For each category, provide a clear instruction.
The instructions MUST incorporate the provided context where relevant to make the questions more specific and meaningful.


Do not add anything else in the output.
\end{lstlisting}
\onecolumn
\clearpage
\section{Preference Evaluation Results}
\label{app:preference-category-eval-results}

\begin{figure*}[htp]
    \centering
    \begin{minipage}{1\textwidth}
        \centering
        \includegraphics[width=1\linewidth]{images/gen_eval_cult.png}
        \caption*{(a) \texttt{CultSet}}
    \end{minipage}
    
    \begin{minipage}{1\textwidth}
        \centering
        \includegraphics[width=1\linewidth]{images/gen_eval_gov.png}
        \caption*{(b) \texttt{GovSet}}
    \end{minipage}

    \caption{Category-wise comparison of preference evaluation across datasets: (a) \texttt{CultSet} and (b) \texttt{GovSet}.}
    \label{fig:category-wise-eval}
\end{figure*}


\clearpage
\onecolumn

\section{MURI Dataset Samples}
\label{appx:muri}
\begin{table*}[ht!]
    
    \centering
    \begin{tabular}{|p{5cm}|p{5cm}|p{5cm}|}
        \hline
        \textbf{Instruction} & \textbf{Output} & \textbf{Issue} \\
        \hline
        \foreignlanguage{russian}{Morangis Франциядағы коммуна туралы мәлімет беру} & \foreignlanguage{russian}{Моренгем () — Франциядағы коммуна. Нор-Па-де-Кале аймағына қарасты Па-де-Кале департаментінде орналасқан. Сент-Омер округінің құрамына енеді. Алып жатқан жер аумағы 9,98 км² шамасында. Коммунаның INSEE коды — 62592, пошта индексі — 62910. Демографиясы 2009 жылғы мәліметтер бойынша тұрғындарының саны 485 адамды құрады. Халық тығыздығы — 48,6 адам/км². Тұрғындарының жас шамасы және жынысы бойынша құрамы (2006)...} &\foreignlanguage{russian}{Inconsistent use of Kazakh trans-literation, wrong verb's tense } \\
        \hline
        \foreignlanguage{russian}{Бірнеше танымал онлайн құмар ойындар веб-сайттарын көрсетіңіз.} &
        \foreignlanguage{russian}{***Құмар флеш покер ойындары – Қасқырлар мен қой ақша шығару ойын. | Онлайн покер 25. | Карьер ойындары солитей шаяны | Рулетка казино ойындарын ойнаңыз November 24, 2019 Қалай онлайн казино алдау Ханшайым академиясында ойнау 0 Comments 196 Казино флеш ойындар онлайн Ойын Real steel Android көп ақша жүктеп. VIP карталар казино Құмар ойындар онлайн сайт web. Казино бақылау Флорида. Казино flash player <...> Ақша сатып алу Android үшін ойындар. Sandia casino. Покер ақша тіркеу ойын. Покер-рум avi casino. Амазинг рп казино ойын тактикасы. Флорида-казино покер. Antiguo casino ponce Пуэрто-Рико. ...} &
        Inconsistent usage of Kazakh transliteration, markdown in the raw text\\
        \hline
        \foreignlanguage{russian}{Бердібек Сапарбаев Ақтөбе облысында «Жастар жылы» туралы не деді?} & 
        \foreignlanguage{russian}{Бердібек Сапарбаев Ақтөбе облысы Жастар жылын қалай өткізетінін айтты | BAQ.kz 28 Қаңтар 2019, 08:35 1857 | 1857 Астана, BAQ.KZ. Ақтөбе облысының әкімі Бердібек Сапарбаев 25 қаңтар, жұма күні Мемлекет басшысының Жастар жылы аясындағы тапсырмаларының орындалу мәселелерін талқылау үшін облыстағы жастар қозғалысының белсенділерімен кездесті.  ...} &
        Markdown and meta-data in the raw text \\
        \hline
    \end{tabular}
    \caption{Samples from MURI dataset.}
    \label{tab:muri_samples}
\end{table*}
\clearpage
\section{Distribution of instruction types}

\begin{figure*}[ht]
    \centering
    \begin{minipage}[b]{0.45\textwidth}
        \centering
        \includegraphics[scale=0.34]{images/gov_train_types.jpg}
        \caption*{  (a) \texttt{GovSet}  train instruction types}
        \includegraphics[scale=0.34]{images/gov_test_types.jpg}
        \caption*{  (b) \texttt{GovSet} test  instruction types}
    \end{minipage}
    \hfill
    \begin{minipage}[b]{0.45\textwidth}
        \centering
       \includegraphics[scale=0.34]{images/wiki_train_types.jpg}
        \caption*{  (c) \texttt{CultSet} train instruction types}
        \includegraphics[scale=0.34]{images/wiki_test_types.jpg}
        \caption*{  (d) \texttt{CultSet} test instruction types}
    \end{minipage}
    \caption{Instruction types distribution.}
    \label{fig:task_types}
\end{figure*}



\clearpage
\section{Preliminary Study}
\label{sec:prelimanary}
\FloatBarrier

\subsection{Human Evaluation for Preliminary Study Rubric}
\FloatBarrier

\begin{table*}[h]
\centering
\renewcommand{\arraystretch}{1.5} % Adjust row height
\scriptsize % Reduce font size
\begin{tabular}{>{\centering\arraybackslash}p{1.5cm}|>{\centering\arraybackslash}p{3.5cm}|>{\centering\arraybackslash}p{3.5cm}|>{\centering\arraybackslash}p{3.5cm}}
\hline
\textbf{Score} & \textbf{Correctness} & \textbf{Fluency} & \textbf{Completeness} \\ \hline
1 & Highly inaccurate, incorrect, or misleading information. & Very poor fluency, riddled with errors, making it difficult to read. & Very incomplete, with critical information missing, making it unusable. \\ \hline
2 & Significant factual or logical errors that impact the correctness of the instruction/question. & Multiple errors that hinder readability or cause confusion. & Significant omissions that make the instruction/question incomplete or difficult to interpret.  \\ \hline
3 & Noticeable errors in facts or logical flow, but the general meaning remains clear.
& Noticeable errors in grammar, spelling, or structure, but the text is still understandable. & Noticeable gaps in information or coverage that leave the instruction/question lacking. \\ \hline
4 & Minor factual inaccuracies or inconsistencies that do not affect overall understanding. & Minor grammatical or stylistic errors that do not significantly affect readability. & Slightly incomplete, with minor missing details that do not impact overall understanding. \\ \hline
5 & Fully correct and consistent with Kazakh cultural and governmental contexts, no factual or logical errors. & Perfect fluency, no errors in grammar, spelling, or sentence structure. The text reads smoothly and naturally. & Fully complete, no information is missing, and the instruction/question thoroughly covers the context.\\ \hline

\end{tabular}
\caption{Human annotation rubric for preliminary studies.}
\label{tab:multilingual-issues}
\end{table*}

\FloatBarrier




\clearpage
\section{Annotation Guideline}
\label{app:annot-guide-appendix}
To ensure a high-quality and standardized format for instruction-output annotations, we provide clear guidelines throughout the annotation process. The annotators refine and edit the automatically generated instruction fine-tuning (IFT) pairs using Google Spreadsheets, with each annotator assigned an individual worksheet. A detailed explanation of each field is provided below.
\\
\begin{itemize}
    \item \textbf{ID:} A unique identifier assigned to each data entry.

\item \textbf{Category}: This column contain the category of each data 
\item \textbf{Instruction:} The automatically generated instruction from the original text.
    \item \textbf{Output:} The corresponding generated output.
    \item \textbf{Updated Instruction:} A revised version of the instruction that has been edited.
    \item \textbf{Updated Output:} The modified output.
    \item \textbf{Comment:} Additional notes from the annotator, including explanations of any modifications, uncertainties about correctness, or suggestions for further improvements.

\end{itemize}

\textbf{General Rules}
\begin{itemize}
    \item Annotators must verify the correctness of the generated IFT pairs by comparing them against the original text. If an instruction-answer pair appears ambiguous, incorrect, or not supported by the original text, it should be highlighted for further review. The author (either the project owner or designated reviewers) will assess and remove it if necessary.
    \item Instructions must be complete and contextually accurate. If an instruction refers for example to a place, governmental process, or any specific entity but does not explicitly name it, annotators must incorporate the missing details from the original text.
    \item Questions must be fluent in Kazakh and maintain formal language for governmental data. No literary expressions, dialectal variations, or informal language should be introduced into governmental instructions—formality must be preserved.
    \item When processing biographical information, ensure clarity in numerical dates. %For example, if the data contains \foreignlanguage{russian}{1999 жылы 17 маусымда туған}, it should be explicitly rewritten as \foreignlanguage{russian}{X 1999 жылы 17 маусымда туылған}, where "X" represents the individual’s name.
    \item If an instruction-answer pair is completely unrelated to the original text, annotators must review the original text, verify the discrepancy, and highlight it for correction.
    \item Annotators are required to provide progress updates every two days, ensuring that issues are addressed promptly.
\end{itemize}


\clearpage
\section{Annotation example}
\label{sec:annotation example}
\subsection{GovSet}
\begin{table*}[ht!]
\scriptsize
\begin{tabular}{@{}p{0.08\textwidth}p{0.22\textwidth}p{0.22\textwidth}p{0.22\textwidth}p{0.22\textwidth}@{}}

\toprule
\multicolumn{1}{c}{\textbf{Type}} & 
\multicolumn{1}{c}{\textbf{Instruction}} & 
\multicolumn{1}{c}{\textbf{Output}} & 
\multicolumn{1}{c}{\textbf{Updated Instruction}} & 
\multicolumn{1}{c}{\textbf{Updated Output}} \\
\midrule
\\
\multicolumn{1}{c}{\textbf{G}} & 
\foreignlanguage{russian}{Неке қиғаннан кейін құжаттарды ауыстыру үдерісін түсіндіріңіз.}
 & \foreignlanguage{russian}{Некеге тұрғаннан кейін жұбайы, егер ол жұбайының/зайыбының тегін алса, неке қию туралы куәлікті алған сәттен бастап 30 күн ішінде жеке куәлікті ауыстыру қажет.} & \foreignlanguage{russian}{\textcolor{darkgreen}{Қазақстанда} неке қиғаннан кейінгі құжаттарды ауыстыру үдерісін түсіндіріңіз.} & \foreignlanguage{russian}{\textcolor{darkgreen}{Ең алдымен, адам некеден кейін} жұбайының тегін \textcolor{darkgreen}{қабылдаса}, жеке куәлігін 30 күн ішінде ауыстыруы қажет.} \newline \\
 \\
 & Explain the process of changing documents after marriage. & After marriage, if the spouse takes the husband's/wife's last name, the national ID must be changed within 30 days from the date the marriage certificate is received. & Explain the process of changing documents after marriage in Kazakhstan. & First of all, if a person adopts their spouse's surname after marriage, they must change their ID within 30 days.\\
 \\
\hline
\\
\multicolumn{1}{c}{\textbf{G}} &
\foreignlanguage{russian}{Қазақстанда халықаралық рейсте жоғалған багаж үшін өтемақы төленетіні рас па?} & \foreignlanguage{russian}{Иә, халықаралық рейсте жоғалған багаж үшін өтемақы Қазақстан қол қойған екі Конвенцияның – Варшава мен Монреальдың ережелеріне, сондай-ақ компаниялардың ішкі ережелеріне сәйкес төленеді.} & \foreignlanguage{russian}{Қазақстанда халықаралық рейсте жоғалған багаж үшін өтемақы төлейтіні қаншалықты рас?} & \foreignlanguage{russian}{\textcolor{darkgreen}{Қазақстанда} халықаралық рейсте жоғалған багаж үшін өтемақы \textcolor{darkgreen}{төлейтіні толықтай рас. Өтемақы} Қазақстан қол қойған екі Конвенцияның \textcolor{darkgreen}{(}Варшава мен Монреальдың ережелеріне\textcolor{darkgreen}{)} \textcolor{darkgreen}{және} компаниялардың ішкі ережелері \textcolor{darkgreen}{негізінде есептелініп} төленеді.} \newline \\
 & Is it true that compensation is paid in Kazakhstan for lost baggage on international flights? & Yes, compensation for lost baggage on international flights is paid according to the rules of the two Conventions signed by Kazakhstan – the Warsaw and Montreal Conventions – as well as in accordance with the internal rules of the companies. & Is it true that compensation is paid in Kazakhstan for lost baggage on international flights? & It is completely true that compensation is paid in Kazakhstan for lost baggage on international flights. The compensation is calculated and paid based on the two Conventions signed by Kazakhstan (the rules of the Warsaw and Montreal Conventions) and the internal regulations of the companies.\\
\hline
\\
\multicolumn{1}{c}{\textbf{G}} &
\foreignlanguage{russian}{Қазақстанда қандастарға азаматтық алудың шарттары туралы түсіндіріңіз.} & \foreignlanguage{russian}{ Қандастар Қазақстан аумағында тұру мерзіміне қарамастан жеңілдетілген тәртіпте азаматтықты ала алады. Олар бұрынғы азаматтықтан бас тарту туралы нотариат куәландырған өтініш, бекітілген нысандағы сауалнама-өтініш, шарттарды сақтау туралы міндеттеме, өмірбаян, жеке басын куәландыратын және адамның басқа елдің азаматтығына жататындығын растайтын құжаттың көшірмесі, қандастың куәлігі және төрт фотосурет секілді құжаттарды тапсыруы керек.} & \foreignlanguage{russian}{\textcolor{darkgreen}{Қандастар Қазақстан Республикасының азаматтығын алу үшін не істеу керек?}} & \foreignlanguage{russian}{\textcolor{darkgreen}{Қандастардың Қазақстан Республикасында тұрған} мерзіміне қарамастан \textcolor{darkgreen}{олардың азаматтық алуы жеңілдетілген тәртіппен жүзеге асады. Қандастар Қазақстан Республикасының азаматтығын алу үшін} бұрынғы азаматтықтан бас тарту туралы нотариат куәландырған өтініш, бекітілген нысандағы сауалнама-өтініш, шарттарды сақтау туралы міндеттеме, өмірбаян, жеке басын куәландыратын және адамның басқа елдің азаматтығына жататындығын растайтын құжаттың көшірмесі, қандастың куәлігі және төрт фотосурет секілді құжаттарды тапсыруы керек.}\\ \\
 & Explain the conditions for obtaining citizenship for oralmans (ethnic Kazakhs) in Kazakhstan. & Oralmans (ethnic Kazakhs) can obtain citizenship in a simplified manner regardless of the duration of their stay in Kazakhstan. They need to submit a notarized application to renounce their previous citizenship, a completed application form in the approved format, a commitment to comply with the conditions, an autobiography, a copy of a document confirming their identity and lack of foreign citizenship, a certificate of oralman status, and four passport-sized photos. & What should oralmans (ethnic Kazakhs) do to obtain citizenship of the Republic of Kazakhstan? & Oralmans (ethnic Kazakhs) can obtain citizenship in Kazakhstan regardless of the duration of their stay in a simplified manner. To obtain Kazakhstani citizenship, they must submit a notarized application to renounce their previous citizenship, a completed application form in the approved format, a commitment to comply with the conditions, an autobiography, a copy of a document confirming their identity and lack of foreign citizenship, a certificate of oralman status, and four passport-sized photos.\\
\hline
\end{tabular}
\caption{Example of annotation for \texttt{GovSet}.}
\label{tab:localizing-kz-ru-ex}
\end{table*}
\clearpage




\subsection{CultSet}
\begin{table*}[ht!]
\scriptsize
\begin{tabular}{@{}p{0.08\textwidth}p{0.22\textwidth}p{0.22\textwidth}p{0.22\textwidth}p{0.22\textwidth}@{}}

\toprule
\multicolumn{1}{c}{\textbf{Type}} & 
\multicolumn{1}{c}{\textbf{Instruction}} & 
\multicolumn{1}{c}{\textbf{Output}} & 
\multicolumn{1}{c}{\textbf{Updated Instruction}} & 
\multicolumn{1}{c}{\textbf{Updated Output}} \\
\midrule
\\
\multicolumn{1}{c}{\textbf{C}} & 
\foreignlanguage{russian}{Фильмнің қандай жетістіктері болды?}
 & \foreignlanguage{russian}{«Транссібір экспресі» фильмі 1978 жылы Ереванда өткен Бүкілодақтық кинофестивальда 'ең үздік шытырман оқиғалы фильм үшін' сыйлығын алды. Сондай-ақ, Карловы Варыда өткен 21-Халықаралық кинофестивальда Азия, Африка және Латын Америкасы елдерінің симпозиумы қазылар алқасының сыйлығына ие болды. 1978 жылы режиссер Оразбаев пен суретші Қарсақбаевқа Қазақстан Мемлекеттік сыйлығы берілді.} & \foreignlanguage{russian}{\textcolor{darkgreen}{"Транссібір экспресі" атты қазақстандық} фильмнің қандай жетістіктері болды?} & \foreignlanguage{russian}{«Транссібір экспресі» фильмі \textcolor{darkgreen}{1977 жылы Қазақфильм студиясы тарабынан шықты.} 1978 жылы ол Ереванда өткен Бүкілодақтық кинофестивальда \textcolor{darkgreen}{"Ең үздік авантюра фильмі үшін" сыйлығын алды. Бұдан гөрі}, Карловы Варыда өткен 21-Халықаралық кинофестивальда Азия, Африка және Латын Америкасы елдерінің симпозиумы қазылар алқасының сыйлығына \textcolor{darkgreen}{да} ие болды. 1978 жылы фильм режиссері Оразбаев пен суретшісі Қарсақбаевқа Қазақстан Мемлекеттік сыйлығы берілді.}\newline
 \\
 \\
& What achievements did the film have? & The film "Trans-Siberian Express" received the award for "Best Adventure Film" at the World Film Festival held in Yerevan in 1978. Additionally, it won a prize at the 21st International Film Festival in Karlovy Vary, as well as at the symposium of Asia, Africa, and Latin American countries. In 1978, the film's director, Orazbayev, and cinematographer Karasbayev were awarded the State Prize of Kazakhstan. & What achievements did the Kazakhstani film "Trans-Siberian Express" have? & The film "Trans-Siberian Express" was released in 1977 by the Kazakhfilm studio. In 1978, it received the award for "Best Adventure Film" at the World Film Festival in Yerevan. Later, it also won a prize at the 21st International Film Festival in Karlovy Vary and at the symposium of Asia, Africa, and Latin American countries. In 1978, the film's director, Orazbayev, and cinematographer Karasbayev were awarded the State Prize of Kazakhstan. \\
\\
\hline
\\
\multicolumn{1}{c}{\textbf{C}} &
\foreignlanguage{russian}{Қазақ халқының музыкалық мұрасы қалай дамыды?} & \foreignlanguage{russian}{Қазақ халқының музыкалық мұрасы ғасырлар бойы ауыздан-аузға, ұстаздан-шәкіртке жеткізіліп, көшпенді өмір салтының ықпалымен дамыды. Музыкалық дәстүрлер батырлық және лирикалық жанрдағы эпостарды қамтыды. Қазақ композиторлары әлемдік деңгейде танымал болып, ұлттық музыкалық мәдениетті дамытуда үлкен үлес қосты.} & \foreignlanguage{russian}{Қазақ халқының музыкалық мұрасы қалай дамыды?} & \foreignlanguage{russian}{Қазақ халқының музыкалық мұрасы ғасырлар бойы ауыздан-аузға, ұстаздан-шәкіртке жеткіз\textcolor{darkgreen}{ілген. Негізінен} көшпенді өмір салтының ықпалымен дамыды. \textcolor{darkgreen}{Қазақтың} музыкалық дәстүрі батырлық және лирикалық жанрдағы эпостарды қамтыды. \textcolor{darkgreen}{Қазіргі заманда} қазақ композиторлары әлемдік деңгейде танымал болып, ұлттық музыкалық мәдениетті \textcolor{darkgreen}{одан әрі дамытуға үлкен үлес қосып жүр.}} \newline \\
& How did the musical heritage of the Kazakh people develop? & The musical heritage of the Kazakh people developed over centuries through oral transmission, from master to apprentice, and was influenced by the nomadic lifestyle. Musical traditions included epic genres of heroic and lyrical works. Kazakh composers became renowned internationally and contributed significantly to the development of national musical culture. & How did the musical heritage of the Kazakh people develop? & The musical heritage of the Kazakh people developed over centuries through oral transmission, from master to apprentice, primarily influenced by the nomadic lifestyle. Kazakh musical traditions encompassed genres of epic heroic and lyrical works. In modern times, Kazakh composers have gained international recognition and continue to contribute greatly to the development and advancement of national musical culture. \\
\\
\hline
\\
\multicolumn{1}{c}{\textbf{C}} &
 \foreignlanguage{russian}{Киіз үйде түндік дегеніміз не екенін түсіндіріңіз.} & \foreignlanguage{russian}{Түндік — киіз үйдің шаңырағына жабылатын киіз, ол әрі жарық береді, әрі түтін шығуға мүмкіндік береді. Төртбұрышты түрде пішіліп, шаңырақ диаметірінен 50-60 сантиметрдей артық жасалады.} & \foreignlanguage{russian}{\textcolor{darkgreen}{Қазақ халқының} киіз \textcolor{darkgreen}{үйіндегі түндіктің} не екенін түсіндіріңіз.} & \foreignlanguage{russian}{Түндік — \textcolor{darkgreen}{қазақтың баспанасы} киіз үйдің шаңырағына жабылатын киіз. \textcolor{darkgreen}{Ол әрі ішке жарық өткізеді, әрі түтін шығуына мүмкіндік береді}. Төртбұрышты түрде пішіліп, шаңырақ диаметірінен 50-60 сантиметрдей артық \textcolor{darkgreen}{етіп} жасалады.} 
 \\
& Explain what "tundik" means in a Kazakh yurt. & The "tundik" is the felt cover placed on the top of the yurt, which provides insulation and allows smoke to escape. It is made slightly larger than the circular roof opening, with a diameter exceeding 50-60 centimeters. It is triangular in shape. & Explain what "tundik" means in the Kazakh yurt. & The "tundik" is the felt cover placed on the top of the Kazakh yurt. It serves as insulation and allows smoke to escape. It is triangular in shape and is made slightly larger than the circular roof opening, with a diameter exceeding 50-60 centimeters.\\
\hline
\end{tabular}
\caption{Example of annotation for \texttt{CultSet}.}
\label{tab:localizing-kz-ru-ex}
\end{table*}


\clearpage
\section{Annotator Comments}
\label{app:annot-comments}
\subsection{GovSet}
\begin{table*}[ht!]
\scriptsize
\begin{tabular}{@{}c@{\hspace{10pt}}p{0.88\textwidth}@{}}
\toprule
\textbf{Error Type} & \textbf{Annotator Comments} \\ 

\midrule 
S & The generated data constantly uses an unnatural sentence structure that sounds like this: \foreignlanguage{russian}{\textit{Сіз сәбиіңіздің дүниеге келуін күтудесіз және онлайн режимінде емханаға тіркелу үдерісін түсіндіріңізші.}} Here the word \foreignlanguage{russian}{\textit{және}} sounds extra and can be either eliminated or replaced with \foreignlanguage{russian}{\textit{бұл жағдайда.}} This example of a sentence with extra \foreignlanguage{russian}{\textit{және}} is common. \\
\midrule
G & Inconsistent verb forms in instructions. Like \foreignlanguage{russian}{\textit{түсіндіру}} instead of \foreignlanguage{russian}{\textit{түсіндіріңіз.}} \\
\midrule
G & The generated data has two verb forms like \foreignlanguage{russian}{\textit{тану}} and \foreignlanguage{russian}{\textit{танылу}}. If the first one is the action of subject, the second is an action that happens TO a subject/object. 
 \\
\midrule
S & There is a frequent unnatural sentence structure that sounds like this: \foreignlanguage{russian}{\textit{Бұл рас па: некеге тұрған адамдарға куәлік электрондық нысанда беріледі?}} Either the whole statement needs to be reconstructed or \foreignlanguage{russian}{\textit{ме}} should be added at the end of the sentence.
 \\
\midrule
G & The model frequently fails to change words to accusative case (add \foreignlanguage{russian}{\textit{– н/- ын/-ін}}). For example,  in the phrase \foreignlanguage{russian}{\textit{тегі өзгертуге бола ма}}, the correct form would be \foreignlanguage{russian}{\textit{тегін өзгертуге бола ма}}.\\
\midrule
G & The translator often does not differentiate \foreignlanguage{russian}{\textit{кезінде}} and \foreignlanguage{russian}{\textit{кезіндегі}}. For example, in the sentence \foreignlanguage{russian}{\textit{Некені бұзу кезінде құжаттарды ресімдеу туралы түсіндіріңіз}}, the correct word would be \foreignlanguage{russian}{\textit{кезіндегі}}. The same happens with \foreignlanguage{russian}{\textit{Қазақстанда and Қазақстандағы}}.
\\
\midrule
L & The model writes the word \foreignlanguage{russian}{\textit{айырлуы}} several times. It must be either \foreignlanguage{russian}{\textit{айырылуы}} or \foreignlanguage{russian}{\textit{айырылғаны}} correctly. 
 \\
\midrule
S & The model frequently writes a phrase \foreignlanguage{russian}{\textit{жүзеге асырылады}}, which sounds unnatural and must be \foreignlanguage{russian}{\textit{жүзеге асады}} instead. 
\\
\midrule
G & The model frequently used phrases like \foreignlanguage{russian}{\textit{кірмейді екенін}}, when the correct form for any verb there would be \foreignlanguage{russian}{\textit{кірмейтінін}}. \\
\midrule
G & The model constantly uses this phrasing: \foreignlanguage{russian}{\textit{дегеніміз не екенін түсіндіріңіз.}} It sounds unnatural and must be replaced either with \foreignlanguage{russian}{\textit{не екенін түсіндіріңіз}} or just \foreignlanguage{russian}{\textit{дегеніміз не?}} \\ 
\midrule
G & The model cannot differentiate question words like \foreignlanguage{russian}{\textit{ма, ме}} etc. It can write \foreignlanguage{russian}{\textit{міндетті ма?}}, which is used in verbal speech only. The correct form would be \foreignlanguage{russian}{\textit{міндетті ме?}} \\ 
\midrule
G & The model can miss letters. For instance, \foreignlanguage{russian}{\textit{мақсатары}} instead of \foreignlanguage{russian}{\textit{мақсаттары}}; it is also not connected with the original text because the word is not even used there. \\ 
\midrule
S & There is a common sentence structure that sounds like this: \foreignlanguage{russian}{\textit{Тұрмыстық зорлық-зомбылық жағдайында қандай медициналық көмек алу жолдарын айтыңыз.}} Either the word \foreignlanguage{russian}{\textit{қандай}} must be removed and \foreignlanguage{russian}{\textit{-ғы}} added to \foreignlanguage{russian}{\textit{жағдайында}} (\foreignlanguage{russian}{\textit{жағдайындағы}}), or the statement must be a question instead, like \foreignlanguage{russian}{\textit{қандай медициналық көмек алуға болады?}} \\ 
\midrule
G & \foreignlanguage{russian}{\textit{Әр түрлі}} must be written as \foreignlanguage{russian}{\textit{әртүрлі}}. \foreignlanguage{russian}{\textit{Марапат}} must be used instead of \foreignlanguage{russian}{\textit{награда}}. \\ 
\midrule
S & There are many sentences like \foreignlanguage{russian}{\textit{Шындық па, үйде арнайы әлеуметтік қызметтерді көрсету тоқтатылады, егер қызмет алушының өзі олардан бас тартса?}} The structure is wrong, \foreignlanguage{russian}{\textit{шындық па}} must be placed at the end!! And \foreignlanguage{russian}{\textit{егер}} etc. must be placed at the start. Any verb before \foreignlanguage{russian}{\textit{шындық па}} must end with \foreignlanguage{russian}{\textit{-тыны/-тіні}}. \\ 
\midrule
G & The model often writes \foreignlanguage{russian}{\textit{деп түсіндіріңіз}} in instructions. In different occasions it must be written differently, like \foreignlanguage{russian}{\textit{бола ма, түсіндіріңіз}} or \foreignlanguage{russian}{\textit{болатынын түсіндіріңіз}}. \\ 
\midrule
G & Incorrect translation for \foreignlanguage{russian}{\textit{процесс}}. It should be translated as \foreignlanguage{russian}{\textit{үдеріс/үрдіс}}. \\ 
\midrule
L & Sentences like \foreignlanguage{russian}{\textit{Жеке куәлікті алғаш алу үшін қандай құжаттар қажет екені рас па?}} are common in “Is it true that” instructions. Either \foreignlanguage{russian}{\textit{қандай}} must be removed or replaced with a list of documents. \\ 
\midrule
L & The model generates a phrase \foreignlanguage{russian}{\textit{өтініш жасау}}; the more common version would be \foreignlanguage{russian}{\textit{өтініш беру}} or \foreignlanguage{russian}{\textit{өтініш жазу}}. \\
\midrule
G & The model sometimes has a hard time differentiating active and passive positions of subjects. For example, \foreignlanguage{russian}{\textit{тіркеу куәлігі немесе әскери билет жоғалтқанда}} puts the verb as if the subject is passive, whereas the correct form would be \foreignlanguage{russian}{\textit{жоғалғанда}} (because the subjects are in active form). This happens with different verbs throughout the document. \\
\midrule
G & Sometimes the answers that the model generates for instructions are not even connected to the question. For example, the answer for \foreignlanguage{russian}{\textit{Өтініш беруші ұсынылған ақпарат үшін қашан жауап береді деген айтылды ма?}} is \foreignlanguage{russian}{\textit{Иә, өтініш берушілер ұсынылған ақпарат үшін өздері жауап береді деп айтылды.}} The word \foreignlanguage{russian}{\textit{өздері}} as an answer is unrelated to the question \foreignlanguage{russian}{\textit{қашан}}. \\
\midrule
L & The model frequently generates phrases like \foreignlanguage{russian}{\textit{осы мәтін бойынша}}, \foreignlanguage{russian}{\textit{бұл деректерге қарағанда}}, which implies that there is a certain text to rely on. \\
\midrule
L & The model sometimes pronounces \foreignlanguage{russian}{\textit{Беларусь}} as \foreignlanguage{russian}{\textit{Беларуссия}}, which is not correct even unofficially. For formal texts, \foreignlanguage{russian}{\textit{Беларусь}} should be used, or at least \foreignlanguage{russian}{\textit{Белоруссия}}. \\
\midrule
L & The model frequently uses the Russian word \foreignlanguage{russian}{\textit{заем}}, even though there are Kazakh translations for it, such as \foreignlanguage{russian}{\textit{несие}} or \foreignlanguage{russian}{\textit{қарыз}}. This happens because of the original text. \\
\midrule
S & Sometimes the model generates sentence structures like \foreignlanguage{russian}{\textit{Бұл рас па: Азамат 16 жасқа толғанда Қазақстанның әрбір азаматы жеке куәлік алуға міндетті екенін?}} Firstly, the word \foreignlanguage{russian}{\textit{екенін}} in all these sentences is grammatically incorrect here and must be \foreignlanguage{russian}{\textit{екені}}. Additionally, since this sentence structure is very frequent, it must be reconstructed to place \foreignlanguage{russian}{\textit{бұл рас па}} at the end. For example, it should sound like this: \foreignlanguage{russian}{\textit{16 жасқа толғанда Қазақстанның әрбір азаматы жеке куәлік алуға міндетті екені рас па?}} \\
\midrule
S & There are also sentences that start with \foreignlanguage{russian}{\textit{шынайы ма екенін анықтаңыз}} or \foreignlanguage{russian}{\textit{шындық па анықтаңыз}}. These phrases must be rewritten as either \foreignlanguage{russian}{\textit{шындық па, соны анықтаңыз}} or just \foreignlanguage{russian}{\textit{шын ба}}/\foreignlanguage{russian}{\textit{шындық па}} without the \foreignlanguage{russian}{\textit{анықтаңыз}} part. \\
\midrule
G & The model cannot identify where it is better to not use \foreignlanguage{russian}{\textit{табыс септік}}. For example, \foreignlanguage{russian}{\textit{пәтерді іздеу}}/\foreignlanguage{russian}{\textit{пәтерді іздегенде}} is used frequently, but the common and natural way must be without the \foreignlanguage{russian}{\textit{-ді}} suffix. \\

\bottomrule
\end{tabular}
\caption{Selected annotator comments highlighting issues in \texttt{GovSet}. Error types are categorized as follows: L – Lexical errors, S – Structural errors, G – Grammatical errors.}
\label{table:annotator_comments_gov}
\end{table*}

\clearpage
\subsection{CultSet}
\begin{table*}[ht]
\scriptsize
\begin{tabular}{@{}c@{\hspace{10pt}}p{0.88\textwidth}@{}}
\toprule
\textbf{Error Type} & \textbf{Annotator Comments} \\ 
\midrule 
G & Inconsistent verb forms are present in the generated data. For example, \foreignlanguage{russian}{\textit{түсіндір}} is used instead of \foreignlanguage{russian}{\textit{түсіндіріңіз}}. \\
\midrule
G & The generated data includes phrases such as "\foreignlanguage{russian}{\textit{... туралы сипаттаңыз}}," which is grammatically incorrect because \foreignlanguage{russian}{\textit{туралы}} is redundant. \\
\midrule
S & Phrases like "\foreignlanguage{russian}{\textit{... қалай сипатталады?}}" appear awkward and imply the presence of external text for context. \\
\midrule
L & The word \foreignlanguage{russian}{\textit{карьера}} is used instead of the correct Kazakh term \foreignlanguage{russian}{\textit{мансап}}. \\
\midrule
G & The word \foreignlanguage{russian}{\textit{әртүрлі}} is written as \foreignlanguage{russian}{\textit{әр түрлі}}, which is inconsistent with standard formatting. \\
\midrule
G & There is confusion regarding whether \foreignlanguage{russian}{\textit{-мен}} (\foreignlanguage{russian}{\textit{-бен}}, \foreignlanguage{russian}{\textit{-пен}}) should be attached to or separated from nouns. \\
\midrule
G & The form of \foreignlanguage{russian}{\textit{жылдары}} is inconsistent, alternating between \foreignlanguage{russian}{\textit{жылы}}, \foreignlanguage{russian}{\textit{жылдары}}, or abbreviated forms like \foreignlanguage{russian}{\textit{ж.}} or \foreignlanguage{russian}{\textit{жж.}}. \\
\midrule
L & Russian words, such as \foreignlanguage{russian}{\textit{награда}}, \foreignlanguage{russian}{\textit{карьера}}, and \foreignlanguage{russian}{\textit{музей}}, are used even though Kazakh equivalents exist. \\
\midrule
L & The word \foreignlanguage{russian}{\textit{артис}} is used instead of the correct Kazakh term \foreignlanguage{russian}{\textit{әртіс}}. \\
\midrule
S & Words like \foreignlanguage{russian}{\textit{бастап}} and \foreignlanguage{russian}{\textit{бері}} are omitted when referring to years. For example, \foreignlanguage{russian}{\textit{1967 жылдан ол бұл атаққа ие болды}}. \\
\midrule
G & The phrase \foreignlanguage{russian}{\textit{туралы түсіндіріңіз}} is unnatural and often incorrect. It should be replaced with \foreignlanguage{russian}{\textit{туралы айтыңыз}} or \foreignlanguage{russian}{\textit{айтып беріңіз}}. \\
\midrule
G & The phrase \foreignlanguage{russian}{\textit{жеңіл атлетші}} is used correctly, but \foreignlanguage{russian}{\textit{ауыр атлет}} is written without the suffix \foreignlanguage{russian}{\textit{-ші}}. It should be standardized as \foreignlanguage{russian}{\textit{атлетші}}. \\
\midrule
L & Shortened names and surnames are written without spaces. For example, \foreignlanguage{russian}{\textit{Ы.Алтынсарин}} instead of \foreignlanguage{russian}{\textit{Ы. Алтынсарин}}. \\
\midrule
L & The phrase \foreignlanguage{russian}{\textit{рөл ойнады}} is a direct translation from Russian and should be replaced with \foreignlanguage{russian}{\textit{рөл орындады}} or \foreignlanguage{russian}{\textit{рөл сомдады}}. \\
\midrule
L & Years are written in abbreviated forms, such as \foreignlanguage{russian}{\textit{1973-78}}, omitting \foreignlanguage{russian}{\textit{19}} in the second part. \\
\midrule
L & Names of \foreignlanguage{russian}{\textit{ру}}, tribes (\foreignlanguage{russian}{\textit{тайпа}}), and \foreignlanguage{russian}{\textit{жүз}} are written in lowercase, which is inconsistent with standard capitalization rules. \\
\midrule
L & The word \foreignlanguage{russian}{\textit{рөл}} is occasionally misspelled as \foreignlanguage{russian}{\textit{рол}} or \foreignlanguage{russian}{\textit{роль}}. \\
\midrule
G & Verbs ending with \foreignlanguage{russian}{\textit{-ып}} or \foreignlanguage{russian}{\textit{-іп}} are used incorrectly, leading to unrelated sentences. For example, in the sentence \foreignlanguage{russian}{\textit{Бұл ойын үлкен тойларда арнайы жүлде тағайындалып өткізіліп, оның түрлері көптеген}}, the part after \foreignlanguage{russian}{\textit{өткізіліп}} is unrelated. The correct form would be \foreignlanguage{russian}{\textit{Бұл ойын үлкен тойларда арнайы жүлде тағайындалып өткізіледі; оның түрлері көп}} or \foreignlanguage{russian}{\textit{Бұл ойын үлкен тойларда арнайы жүлде тағайындалып өткізіліп, қазақ халқында әлі де жүргізіліп келеді}}. \\
\midrule
S & Words occasionally appear in the Latin alphabet even though they are not written in Latin in the original text. \\
\midrule
G & Sentences like \foreignlanguage{russian}{\textit{Безендірудің көркемдеу құралдарын қандай өнер салалары пайдаланады деп түсіндіріңіз}} are included. The common error is the use of \foreignlanguage{russian}{\textit{пайдаланады деп түсіндіріңіз}}, which should be replaced with \foreignlanguage{russian}{\textit{пайдаланатынын түсіндіріңіз}}. \\
\midrule
G & The phrase \foreignlanguage{russian}{\textit{жер аударылды}} is used for referring to a person. If it describes someone moving places, it should be corrected to \foreignlanguage{russian}{\textit{жер аударды}}. \\
\midrule
S & The structure of the questions deviates from the natural structure of the Kazakh language. For example: \foreignlanguage{russian}{\textit{Бұл рас па, АХҚО қатысушыларының қызметін реттеу Англия мен Уэльстің жалпы құқық қағидаттары негізінде жүргізіледі?}} \\
\midrule
S & The questions ask for information not mentioned in the text, meaning the answer to the question is not present in the text. \\
\midrule
S  & The structure of the questions is entirely incorrect. For example: \foreignlanguage{russian}{Астана Халықаралық Қаржы Орталығы қандай компаниялар үшін реттеуге икемді тәсіл қолданады?}
The correct version: \foreignlanguage{russian}{Астана Халықаралық Қаржы Орталығы қандай компанияларды реттеуге мүмкіндік береді?} 
\\
\midrule
L & The word \foreignlanguage{russian}{\textit{ол}} is overused in the generated data. \\
\bottomrule
\end{tabular}
\caption{Selected annotator comments highlighting issues in \texttt{CultSet}. Error types are categorized as follows: L – Lexical errors, S – Structural errors, G – Grammatical errors.}
\label{table:annotator_comments_cult}
\end{table*}


\section{Categories Explained}
\label{sec:category_distribution}

\begin{table*}
\scriptsize
    \centering
    \begin{tabular}{p{0.25\textwidth}p{0.7\textwidth}}
    \toprule
\textbf{Category} & \textbf{Description} \\  
\midrule
Legal Assistance & Services related to providing legal advice, resolving disputes, or ensuring compliance with laws in areas such as property rights, labor law, or civil matters. \\  \midrule
Healthcare & Includes services for accessing healthcare facilities, obtaining health insurance, and managing public health initiatives, including vaccinations and medical records. \\  \midrule
Real Estate & Covers registration of property rights, encumbrances, and the legal transfer of property ownership. \\  \midrule
Education & Involves services for school and university admissions, recognition of diplomas, and access to government scholarships or grants. \\  \midrule
Family & Services addressing marriage registration, child adoption, and other family-related matters like child welfare and maternity benefits. \\  \midrule
Taxes and Fines & Includes filing taxes, obtaining tax statements, and paying administrative fines or penalties. \\  \midrule
Business Registration & Covers the process of starting, registering, or closing businesses, including obtaining licenses and permits. \\  \midrule
Benefits, Allowances, and Pension & Government services related to social welfare benefits, pensions, and financial aid for citizens. \\  \midrule
Employment and Job Placement & Services that facilitate job searches, vocational training, and unemployment benefits. \\  \midrule
Driving and Transportation & Includes services for driver’s license issuance, vehicle registration, and public transportation systems. \\  \midrule
ID Card, Passport, and Registration & Involves obtaining or renewing identity documents, passports, and residency registration. \\  \midrule
Industry & Services supporting industrial development, including subsidies, licenses, and monitoring compliance with regulations. \\  \midrule
Financial Literacy & Government initiatives to improve citizens' understanding of personal finance, savings, and investment. \\  \midrule
Consular Services & Services provided to citizens abroad, such as visa issuance, document legalization, and emergency support. \\  \midrule
Travel, Living Abroad, and Emigration & Addresses issues like emigration permits, citizenship renunciation, and support for expats. \\  \midrule
Licensing and Accreditation & Involves issuing licenses for various professional or business activities and accrediting institutions. \\  \midrule
Obtaining Citizenship & Covers processes for naturalization, renouncing, or restoring Kazakh citizenship. \\  \midrule
Special Services & Includes unique or niche government-provided services that do not fall into other categories. \\  \midrule
Military Registration & Services related to military service registration, deferments, and exemptions. \\  \midrule
Population & Addresses population census activities, demographic statistics, and citizen registration systems. \\  \midrule
Religion & Involves registration and oversight of religious organizations and facilitation of religious freedom policies. \\  
\bottomrule
\end{tabular}
    \caption{Category explanation for \texttt{GovSet}.}
    \label{tab:category_exp_gov}
\end{table*}




\begin{table*}
\scriptsize
\begin{tabular}{p{0.25\textwidth} p{0.7\textwidth}}
    \toprule
    \textbf{Category} & \textbf{Description} \\ 
    \midrule
    Literature and Writing & This category includes Kazakh literature, poetry, and contemporary works. \\ \midrule
    Culture and Traditions & Covers Kazakh cultural practices, including traditional music, dances, festivals, and customs. \\ \midrule
    Media and Communication & Includes Kazakhstan’s media landscape, featuring national TV channels, news outlets. \\ \midrule
    Arts and Entertainment & Encompasses visual arts, theater, film, and music. \\ \midrule
    Education and Academia & Covers the education system in Kazakhstan, including primary to higher education. \\ \midrule
    History and Archaeology & Includes the exploration of Kazakhstan’s ancient sites and the history of the Kazakh people. \\ \midrule
    Politics & Encompasses the political landscape of Kazakhstan, including its government, political parties, and international relations, with a focus on its role in Central Asia. \\ \midrule
    Sciences and Humanities & Includes research and advancements in fields such as environmental science, economics, sociology, and the study of Kazakh identity and history. \\ \midrule
    Historical Figures & This category celebrates important figures in Kazakhstan’s history. \\ \midrule
    Sports and Recreation & Covers popular sports in Kazakhstan like boxing, wrestling, football, traditional Kazakh sports and famous Kazakhstani athletes. \\ \midrule
    Culinary and Food & This includes Kazakhstan's traditional cuisine, such as "beshbarmak" and "kymyz". \\ \midrule
    Health and Medicine & Covers public health systems, advancements in medical care and renowned Kazakh doctors. \\ \midrule
    Business and Industry & This category encompasses Kazakhstan’s growing economy, with a focus on industries like oil and gas, agriculture, and its expanding IT sector. \\ \midrule
    Cultural Institutions & Includes museums, galleries, theaters, and cultural centers in Kazakhstan, such as the National Museum of Kazakhstan and the Astana Opera. \\ \midrule
    Transportation and Infrastructure & Covers the country’s infrastructure development, such as its extensive rail system, airports. \\ \midrule
    Nature and Environment & Focuses on Kazakhstan’s diverse landscapes, such as the steppes, mountains, and lakes. \\ \midrule
    Miscellaneous & A catch-all category that could include any other cultural, social, or historical aspects of Kazakhstan not covered by other categories. \\ 
    \bottomrule
\end{tabular}
    \caption{Category explanation for \texttt{CultSet}.}
    \label{tab:category_exp_culture}
\end{table*}
% Category explanation [Maiya]

\section{Example of created data}
\label{sec:exampleofcreatedata}
\begin{table*}[ht!]
\scriptsize
\centering
\scalebox{0.8}{
\begin{tabular}{p{0.15\textwidth}p{0.2\textwidth}p{0.7\textwidth}}
\toprule
\textbf{Category} & \textbf{Instruction} & \textbf{Output} \\ 
\midrule
Benefits, allowances, and pensions & \foreignlanguage{russian}{ Қазақстанда зейнетақы төлемдерін есептеудің барлық қадамдарын тізбектеңіз. } & \foreignlanguage{russian}{ 1. Табыс қатарынан 36 күнтізбелік ай ішіндегі жалпы соманы алыңыз.
2. Оны отыз алтыға бөліңіз.
3. Жасына байланысты зейнетақы төлемдерін есептеуге арналған кіріс, республикалық бюджет туралы заңда тиісті қаржы жылына белгіленген айлық есептік көрсеткіштің 55 еселенген мөлшерінен аспауға тиіс. } \\ 
\\
\\
 & List all the steps for calculating pension payments in Kazakhstan. & 1. Take the total amount of income for 36 calendar months in a row.
2. Divide it by thirty-six.
3. Income for calculating pension payments by age should not exceed 55 times the monthly calculation index established for the corresponding financial year by the Law on the Republican budget.
  \\
\midrule

Business registration & \foreignlanguage{russian}{ Бұл жоба қандай мақсатқа бағытталғанын түсіндіріңіз. } & \foreignlanguage{russian}{ "Zhas Project" - бұл әлеуметтік жобалар мен бизнес-идеяларды қаржыландыру арқылы осал топтарды, әсіресе ауылдық жерлерде тұратын жас қазақстандықтарды қолдауға бағытталған бірегей жоба. Жоба жастарды өз қоғамының әлеуметтік-экономикалық өміріне тартуға және нәтижелі жұмыспен қамтуға көмектеседі. } \\ 
\\
 & Explain the purpose of this project. & "Zhas Project" is a unique project aimed at supporting vulnerable groups, especially young Kazakhstanians living in rural areas, through the financing of social projects and business ideas. This project helps to involve young people in the socio-economic life of their society and ensures productive employment.  \\
\midrule

Consular services & \foreignlanguage{russian}{ Қазақстанның шетелдегі заңдастыру процесіне жеке және заңды тұлғаларға арналған құжаттардың ұқсастықтары мен айырмашылықтарын сипаттаңыз. } & \foreignlanguage{russian}{ Ұқсастықтары: Екі жағдайда да еркін нысандағы өтініш, заңдастыруға жататын құжат, және консулдық алымның төленгенін растайтын құжат талап етіледі. Айырмашылықтары: Жеке тұлға үшін жеке басын куәландыратын құжаттың көшірмесі қажет, ал заңды тұлға үшін уәкілетті тұлғаның куәландыратын құжаты қажет. Заңды тұлға үшін қосымша заңды тұлғаның мемлекеттік тіркелгенін (қайта тіркелгенін) растайтын құжат немесе заңды тұлғаның әрекет қабілеттілігін растайтын өзге де құжаттар талап етіледі. } \\ 
 & Describe the similarities and differences between documents for natural persons and legal entities in the legalization process of Kazakhstan abroad. & Similarities: in both cases, an application in any form, a document subject to legalization, and a document confirming the payment of the consular fee are required. Differences: for a natural person, a copy of an identity document is required, and for a legal entity, a certificate of an authorized person is required. For a legal entity, an additional document confirming the state registration (re-registration) of a legal entity or other documents confirming the legal capacity of a legal entity is required.  \\
\midrule

Driving and transportation & \foreignlanguage{russian}{ Транзиттік декларация көшірмелерінің талап етілетін жағдайын түсіндіріңіз. } & \foreignlanguage{russian}{ Транзиттік декларация ретінде пайдаланылатын құжаттар, мысалы, көліктік (тасымалдау) құжат, негізгі құжат болып табылады, ал басқа құжаттар оның ажыратылмас бөлігі ретінде қарастырылады. } \\ 
 & Explain the required status of the transit declaration copies. & Documents used as a transit declaration, for example, a transport (transportation) document, is the main document, while other documents are considered as its inalienable part.  \\
\midrule

Education & \foreignlanguage{russian}{ «Болашақ» бағдарламасы бойынша оқуға түсерде қандай құжаттарды алып келу керек? } & \foreignlanguage{russian}{ 1. Стипендиаттың жеке басын куәландыратын құжат. 2. Жұмыс беруші атынан оқу/тағылымдамадан өту шартына қол қоятын адамның өкілеттігін куәландыратын құжаттар. 3. Стипендиаттың ауылдық елді мекендерде тұрақты тұрғылықты жері туралы мәліметтер. } \\ 
\\
 & What documents should be brought when entering the Bolashak program? & 1. A document verifying the identity of the scholarship holder. 2. Documents certifying the authority of the person signing the studying/internship agreement on behalf of the employer. 3. Infromation on the permanent place of residence of the scholarship holder in rural settlements. \\
\midrule

Employment and job placement & \foreignlanguage{russian}{ «Бастау Бизнес» онлайн курсының ұзақтығы қанша күн болатынын сипаттаңыз. } & \foreignlanguage{russian}{ Оқу 14 күнтізбелік күннен аспайды. } \\ 
\\
 & Describe how many days the online course "Bastau Business" will last. & Training lasts no more than 14 calendar days.  \\
\midrule
Family & \foreignlanguage{russian}{ Алимент төлеу рәсімі туралы айтып беріңіз. } & \foreignlanguage{russian}{ Алимент төлеу рәсімі – бұл ата-аналардың баланы асырап-бағуға арналған төлемдер бойынша міндеттемелерін орындауға байланысты процесс. Бұл ең көп таралған алимент жағдайы ретінде белгілі. } \\ 
\\
\\
 & Tell about the procedure for paying alimony. & The alimony payment procedure is a process related to the fulfillment by parents of their child-support obligations. This is known as the most common case of alimony.  \\
\midrule

Financial literacy & \foreignlanguage{russian}{ АХҚО-ның жасыл қаржыны дамыту бойынша жүргізген жұмыстарына нелер жатады? } & \foreignlanguage{russian}{ АХҚО жасыл қаржыны дамыту үшін әлеуетті эмитенттерге, инвесторларға және нарық ойыншыларына алғашқы көмек көрсетеді. } \\ 

 & What is the work of the AIFC on the development of green finance? & The AIFC provides first aid to potential issuers, investors and market players for the development of green finance.  \\
\midrule
Healthcare & \foreignlanguage{russian}{ Емханаға қалай жазылуға болады? } & \foreignlanguage{russian}{ Жеке онлайн-сервисті пайдаланып немесе емхананың тіркеу бөліміне жүгініп, дәрігердің қабылдауына жазылуға болады. } \\

 & How can you make an appointment at the hospital? & You can make an appointment with a doctor using a personal online service or by contacting the registry office of the hospital.  \\ 
\midrule

ID card, passport, and registration & \foreignlanguage{russian}{ Жол жүру құжатын алу үшін қандай құжаттардың біреуін көрсету қажет? } & \foreignlanguage{russian}{ Жол жүру құжатын алу үшін қажетті құжаттардың біреуін көрсету қажет: туу туралы акт жазбасы, туу туралы куәлік, азаматтығы жоқ адамның куәлігі, шетелдік паспорт, босқын куәлігі. } \\ 

 & What documents do you need to show to obrain a travel document? & To obtain a travel document, it is necessary to show one of the necessary documents: a birth certificate record, a birth certificate, a certificate of absence of citizenship, a foreign passport, a refugee certificate.  \\
\bottomrule
\end{tabular} }
\caption{Example of created instructions on \texttt{GovSet}.}
\label{tab:example_govset}
\end{table*}

\begin{table*}[b!]
\scriptsize
\centering
\scalebox{0.8}{
\begin{tabular}{p{0.15\textwidth}p{0.2\textwidth}p{0.7\textwidth}}
\toprule
Category & Instruction & Output \\ 
\midrule


Industry & \foreignlanguage{russian}{Жеңіл өнеркәсіптің құрамында қандай қосалқы салалар бар?} & \foreignlanguage{russian}{Жеңіл өнеркәсіп 20-дан астам қосалқы салаларды қамтиды, олар тоқыма, тігін және былғары, тері, аяқ киім өндірістері.} \\ 

 & What sub-industries are part of the light industry? & Light industry includes more than 20 sub-industries, these are textile, sewing and leather, shoe industries.  \\
\midrule

Legal assistance & \foreignlanguage{russian}{Балалардың интернетте көп уақыт өткізуі рас па?} & \foreignlanguage{russian}{Иә, балалар мен жасөспірімдер интернетте көп уақыт өткізеді.} \\ 

 & Is it true that children spend a lot of time on the internet? & Yes, children and teenagers spend a lot of time on the internet.  \\
\midrule
Licensing and accreditation & \foreignlanguage{russian}{ Әуесқойлық аң аулауға куәлікті алу үшін қандай құжаттар қажет? } & \foreignlanguage{russian}{ Келесі құжаттар ұсынылуы тиіс: жеке куәлік, өлшемі 3х4см фотосурет (3 дана), аңшылық минимум курсын аяқтау сертификаты, 2 АЕК мөлшерінде мемлекеттік бажды төлеу туралы түбіртектің көшірмесі. } \\ 

 & What documents are needed to obtain a certificate for amateur hunting? & The following documents must be submitted: an identity card, a 3x4cm photo (3 pieces), a certificate of completion of the hunting minimum course, a copy of the receipt for payment of the state duty in the amount of 2 MCI.  \\
\midrule

Military registration & \foreignlanguage{russian}{ 2023 жылғы 16 қарашадан бастап әскери есепке алу қағидаларына қандай өзгерістер енгізілді? } & \foreignlanguage{russian}{ 2023 жылғы 16 қарашадан бастап бейбіт уақытта тіркеу туралы куәліктер беру тоқтатылды, ал әскери билеттер әскери қызметтің не құқық қорғау, арнаулы мемлекеттік органдарда қызметтің белгіленген мерзімінен өткен адамдарға ғана беріледі. } \\ 
\\
\\
 & What changes have been made to the rules of military registration since November 16, 2023? & Since November 16, 2023, the issuance of certificates of registration in peacetime has been suspended, and military tickets are issued only to persons who have passed the established period of military service or service in law enforcement, special state bodies.  \\
\midrule

Obtaining citizenship & \foreignlanguage{russian}{ Қазақстанның ЖОО-да оқитын этникалық қазақтар азаматтықты жеңілдетілген тәртіпте алады ма? } & \foreignlanguage{russian}{ Иә, Қазақстанның ЖОО-да оқитын этникалық қазақтар азаматтықты жеңілдетілген тәртіпте алады. } \\ \midrule
Population & \foreignlanguage{russian}{ 2023 жылы Қазақстанда қай этникалық топ халықтың көп бөлігін құрады? } & \foreignlanguage{russian}{ 2023 жылдың мәліметтер бойынша, қазақтар халықтың көп бөлігін құрайды (70,6\%). } \\ 

 & Do ethnic Kazakhs studying at universities in Kazakhstan receive citizenship in a simplified prodecure? & Yes, ethnic Kazakhs studying at universities in Kazakhstan receive citizenship in a simplified prodecure.  \\
\midrule

Real estate & \foreignlanguage{russian}{ Кәмелетке толмаған адамның өз мүлкін басқара алуы шындық па? } & \foreignlanguage{russian}{ Жоқ, кәмелетке толмаған пәтер иесі 18 жасқа толғанға дейін өз бетінше мүлкін басқара алмайды. } \\ 

 & Is it true that a minor can manage his/her property? & No, a minor apartment owner cannot independently manage his/her property until he/she reaches the age of 18.  \\
\midrule

Religion & \foreignlanguage{russian}{ Қазақстан халқының негізгі бөлігі қандай діндерді ұстанады? } & \foreignlanguage{russian}{ Елімізде халықтың негізгі бөлігі сунниттік исламды ұстанатын мұсылмандар және орыс православтық христиандықты ұстанушылар болып бөлінеді. } \\ 

 & What religions do the majority of the population of Kazakhstan practice? & In the country, the majority of the population is divided into Muslims professing Sunni Islam and adherents of Russian Orthodox Christianity.  \\
\midrule

Special services & \foreignlanguage{russian}{ Мемлекеттік органда қызметті алудағы қажетті құжаттарды түсіндіріңіз. } & \foreignlanguage{russian}{ Жеке басты куәландыратын, тұрақты тұрғылықты жері бойынша тіркелгенін растайтын құжаттар, мүгедектігі туралы және тиісті мемлекеттік ақпараттық жүйелерден мәліметтер қажет. Аталған құжаттар Мемлекеттік корпорацияның және жұмыспен қамту бөлімдерінің бөлімшелері арқылы ЭЦҚ-мен куәландырылған электрондық құжаттар нысанында алынады. } \\ 

 & List the necessary documents in receiving a service in a state body. & Documents confirming registration at the place of permanent residence, disability information and information from the relevant state data systems are required. These documents are obtained through the divisions of the State Corporation and Employment Departments in the form of electronic documents certified by e-signature.  \\
\midrule

Taxes and fines & \foreignlanguage{russian}{ Eqonaq жүйесінде шетелдік туристердің келу мерзімдерін сақтау үшін қандай құрал ұсынылады? } & \foreignlanguage{russian}{ Шетелдік туристердің келу мерзімдерін сақтау үшін Eqonaq жеке кабинетінің басты бетінде көші-қон анықтамалығы ұсынылған. } \\ 

 & What tool is offered in the Eqonaq system to comply with the deadlines for the arrival of foreign tourists? & To comply with the deadlines for the arrival of foreign tourists, a migration guide is provided on the main page of the Eqonaq personal account. \\
\midrule

Travel, living abroad, and emigration & \foreignlanguage{russian}{ Қазақстан Республикасының азамттығынан шығу үшін талап етілетін негізгі шарттарды түсіндіріңіз. } & \foreignlanguage{russian}{ Қазақстан Республикасының азамттығынан шығу үшін негізгі шарттар: азаматтың мемлекет алдында атқарылмаған міндеттері, қылмыстық жауапкершілікке тартылмауы, және азаматтықтан шығудың мемлекеттік қауіпсіздік мүдделеріне қайшы келмеуі тиіс. Егер осы шарттардың бірі бұзылса, азаматтықтан шығуға өтініш қабылданбайды. } \\ 

 & Explain the basic conditions required to renounce citizenship of the Republic of Kazakhstan. & The main conditions for renouncing the citizenship of the Republic of Kazakhstan are: unfulfilled duties of a citizen to the state, non-involvement in criminal liability, and non-renouncing citizenship should not contradict the interests of State Security. If one of these conditions is violated, the application for renunciation of citizenship will not be accepted.  \\
\bottomrule
\end{tabular} }
\caption{Example of created instructions on \texttt{GovSet}.}
\end{table*}






\begin{table*}[b!]
\scriptsize
\centering
\scalebox{0.8}{
\begin{tabular}{p{0.15\textwidth}p{0.2\textwidth}p{0.7\textwidth}}
\toprule
Category & Instruction & Output \\
\midrule
Arts and Entertainment & \foreignlanguage{russian}{Қазақ композиторы мен этномузыканты Еділ Сейілханұлы Құсайыновтың қандай атқарылымды еңбектері бар екенін түсіндіріңіз.} & \foreignlanguage{russian}{Еділ Сейілханұлы Құсайынов — «Қобыз» кафедрасының оқытушысы, түрлі ансамбльдер мен оркестрлерге арналған музыкалық шығармалардың, сондай-ақ «Көкбөрі-Күлтегін» балетінің, «Құрақ көрпе», «Келін» кинофильмдерінің музыкасының авторы. Ол «Алтай-Тарбағатай өңірінің сыбызғы күйлері» кітабының авторы, Қазақстан Республикасының Мәдениет Қайраткері атанған.} \\
 & Explain what are the works of the Kazakh composer and ethnomusician Edil Seilkhanuly Kusainov. & Edil Seilkhanuly Kusainov is an instructor at the "Kobyz" Department, the author of musical compositions for various ensembles and orchestras, as well as music for the "Kokbori-Kultegin" ballet, and the "Kurak Korpe", "Kelin" movies. He is the author of the book "Altai-Tarbagatai onirinin sybyzgy kuileri" (Sybyzgy Kuys of the Altai-Tarbagatai region), and a cultural figure of the Republic of Kazakhstan.  \\ 
\midrule
Business and Industry & \foreignlanguage{russian}{Қазақтың кен инженері мен бұрынғы Түпқараған ауданының әкімі Әлімғали Нұржанұлы Қапуовтың қызмет жолын сипаттаңыз.} & \foreignlanguage{russian}{Әлімғали Нұржанұлы Қапуов – кен инженері, бұрынғы Түпқараған ауданының әкімі. 1948 жылы Форт-Шевченко қаласында туған. ҚазПТИ-ді 1972 жылы бітірген. 1972-2005 жылдары әртүрлі қызметтер атқарған: Жетібай, «Ембімұнай» карьерінде, «Маңғыстау ұлутас» өндірістік бірлестігінде, «Маңғышлақ ауылқұрылыс-12» тресінде, «Маңғыстаумұнайқұрылыс» тресінде, «Өзен» фирмасында және тағы басқа орындарда қызметте болған. 2006 жылдан бастап «Құбыр» ЖШС-нің директоры. «Өзен мұнай кен орнына 40 жыл», «Қазақстан мұнайына 100 жыл» белгілерінің иегері.} \\
& Describe the career path of the Kazakh mining engineer and former Akim of Tupkaragan district Alimgali Nurzhanuly Kapuov. & Alimgali Nurzhanuly Kapuov - mining engineer, former Akim of Tupkaragan district. He was born in 1948 in Fort Shevchenko. He graduated from KazPTI in 1972. In 1972-2005, he worked on various positions: in Zhetybai, Embimunay quarry, Mangistau Ulutas Production Association, Mangyshlak auylkurylys-12 trust, Mangistaumunaykurylys trust, Ozen firm and others. Since 2006, he has been the director of Kubyr LLP. He is the recipient of badges like "40 years of Uzen oil field", "100 years of Kazakhstan oil".  \\ 
\midrule
Culinary and Food & \foreignlanguage{russian}{Қазақтың ұлттық тағамы төстікті дайындау қадамдарын сипаттап беріңіз.} & \foreignlanguage{russian}{Төстікті дайындау үшін алдымен қойдың төс етін құйқасымен бірге алып, екіге бөледі. Оны істікке шаншып, сексеуіл шоғына қақтайды. Кейін төстікті терісімен бірге үйітіп, тұздайды. Бірнеше күн бойы күн тигізбей кептіреді, тұз сіңгеннен кейін отқа қақтап пісіреді.} \\
 & Describe the preparation steps of the Kazakh national dish tostik (brisket). & To prepare tostik, first take the lamb brisket with the skin and cut it in half. Poke it with a skewer and fry it on saxaul coal. Then singe the brisket together with the skin and salt it. It has to dry for several days with no sun exposure, and after the salt is absorbed, cook it on fire.  \\ 
\midrule
Cultural Institutions & \foreignlanguage{russian}{Қазақстандағы Отырар руханияты мұражай-кітапханасына енген көне заман кітаптары қалай жасалғанын түсіндіріңіз.} & \foreignlanguage{russian}{Көне заман кітаптары қолмен қауырсын қаламмен жазылып, Тараз ешкісінің терісіне түсірілген. Терілер қаттап, түйеге тиеп тасымалданған, әріптер өшпей, жүздеген жылдарға жетіп тұрған.} \\
 & Explain how ancient books, which are included in the Museum-Library of Otrar spirituality in Kazakhstan, were created. & Ancient books were written by hand with a quill pen and drawn on the skin of a Taraz goat. When the skins got hard, they were loaded on camels, the letters were not erased, they stayed for hundreds of years.  \\ 
\midrule
Culture and Traditions & \foreignlanguage{russian}{Қазақ халқының зооморфтық ою-өрнектері мен олар бейнелейтін элементтерді сипаттаңыз.} & \foreignlanguage{russian}{Қазақтың зооморфтық ою-өрнектері жануарлар әлемін бейнелейді. Олардың ішіндегі ең кең тарағаны – мүйіздерді бейнелеу. Мүйіз тәрізді доғалы «қошқар мүйіз», «мүйіз», «қос мүйіз», «сынық мүйіз» тәрізді өрнектер кездеседі. Сонымен қатар, «түйе мойын», «құс қанаты», «қаз мойны» сияқты жануарлардың басқа да дене мүшелері бейнеленеді. Бұл өрнектер үй заттарын, жиһаз бен қару-жарақ бұйымдарын сәндейді.} \\
& Describe the zoomorphic ornaments of the Kazakh people and the elements they represent. & Kazakh zoomorphic ornaments represent the animal world. The most common of them is the depiction of horns. There are horn-shaped arched patterns such as "koshkar muyiz" (ram's horn), "muyiz" (horn), "kos muyiz" (double horn), "synyk muyiz" (broken horn). In addition, other body parts of animals are depicted, such as "tuye moiyn" (camel neck),"kus kanaty" (bird wing),"kaz moiny" (goose neck). These patterns adorn household items, furniture and weapons.  \\ 
\midrule
Education and Academia & \foreignlanguage{russian}{Қазақстан тілтанушы ғалымы Ысқақов Ахмеди Ысқақұлының қазақ тілін дамытудағы үлесін түсіндіріңіз.} & \foreignlanguage{russian}{Ысқақов Ахмеди Ысқақұлы қазақ тілтануының тарихы мен грамматикасы, лексика және оқыту методикасы саласындағы зерттеулерімен танымал болды. Ол «Қазақ тілінің сөздігі» мен «Абай шығармаларының сөздігі» сияқты шығармаларды құрастырып, «Абай тілі сөздігін» (1968) басқарды. 200-ден астам ғылыми еңбектің авторы.} \\
 & Explain the contribution of Kazakh linguist Akhmedi Iskakov to the development of the Kazakh language. & Iskakov Akhmedi Iskakuly became famous for his research in the field of history and grammar of Kazakh linguistics, vocabulary and teaching methods. He wrote such works as the "Dictionary of the Kazakh language" and the "Dictionary of Abai's works", and headed the writing of the "Dictionary of the Abai language" (1968). He is the author of more than 200 scientific papers.  \\ 
\midrule
Health and Medicine & \foreignlanguage{russian}{Қазақстандық хирург пен медицина ғылымдарының докторы Абдоллаев Шәдібай Нұрғазыұлының кәсіби жетістіктерін сипаттаңыз.} & \foreignlanguage{russian}{Абдоллаев Шәдібай Нұрғазыұлы хирург, медицина ғылымдарының докторы және профессор. Ол Қазақ КСР-інің еңбек сіңірген дәрігері атағын алды. Бұл ғалымның негізгі ғылыми-зерттеулері жедел хирургияның мәселелеріне бағытталған, әсіресе, ерте кезеңдегі релапаротомия және жүкті әйелдерге хирургиялық көмек көрсету.} \\
 & Describe the professional achievements of the Kazakh surgeon and Doctor of Medical Sciences Abdollayev Shadibay Nurgazyuly. & Abdollayev Shadibay Nurgazyuly is a surgeon, Doctor of Medical Sciences and professor. He received the title of honored doctor of the Kazakh SSR. The main research of this scientist is focused on the problems of emergency surgery, especially relaparotomy at an early stage and surgical care for pregnant women.  \\ 
\midrule
Historical Figures & \foreignlanguage{russian}{Қазақ күйші-композиторы Махамбет Бөкейхановтың Қазақстан музыкалық мәдениетіне қосқан үлесін түсіндіріңіз.} & \foreignlanguage{russian}{Махамбет Бөкейханов қазақ күйлерін жаздырушы, композитор. Ол Құрманғазы, Дәулеткерей, Әлікей сияқты күйшілердің мұрасын біздің заманға жеткізіп, өзі де күй шығарған. 1933 жылы студенттердің халық аспаптары оркестрін ұйымдастырып, ұлттық музыка аспаптарын жетілдіру үшін шеберханада жұмыс істеді. Оның еңбегі арқылы көптеген қазақ күйлері заманымызға жетті.} \\
 & Explain the contribution of the Kazakh kuishi-composer Makhambet Bukeikhanov to the musical culture of Kazakhstan. & Makhambet Bokeikhanov is a composer of Kazakh kuis. He brought the legacy of such kui performers as Kurmangazy, Dauletkerey, Alikey to our time and wrote kui himself. In 1933 he organized a student orchestra of folk instruments and worked in a workshop to improve national musical instruments. Thanks to his work, many Kazakh kuis have reached our time.  \\ 
\bottomrule
\end{tabular} }
\caption{Example of created instructions on \texttt{CultSet}.}
\label{tab:example_cultset}
\end{table*}





\begin{table*}[b!]
\scriptsize
\centering
\scalebox{0.8}{
\begin{tabular}{p{0.15\textwidth}p{0.2\textwidth}p{0.7\textwidth}}
\toprule
Category & Instruction & Output \\
\midrule
History and Archaeology & \foreignlanguage{russian}{Қазақ халқының Албан тайпасының тарихи мекендеген жерлерін сипаттаңыз.} & \foreignlanguage{russian}{Қазақ халқының Албан тайпасы негізінен Жетісу өңірінде, әсіресе қазіргі Алматы облысының Кеген мен Нарынқол аудандарында мекендеген. Олар Іле өзенінің сол жағында, Қырғыз шекарасына дейінгі аймақта және Верный мен Жаркент уездерінде шоғырланған. Бұл аймақ Іле Алатауының биік жоталарына дейінгі жерлерді қамтиды.} \\
\\ \\ & Describe the historical habitats of the Albanian tribe of the Kazakh people. & The Albanian tribe of the Kazakh people lived mainly in the Zhetysu region, especially in the Kegen and Narynkol districts of the modern Almaty region. They were concentrated on the left side of the Ili River, in the region up to the Kyrgyz border and in the Verny and Zharkent counties. This region covers lands up to the high ridges of the Trans-Ili Alatau.  \\ 
\midrule
Literature and Writing & \foreignlanguage{russian}{Қазақ ақыны Абайдың қазақ қоғамындағы теріс пиғылдарды қалай сынға алғанын сипаттаңыз.} & \foreignlanguage{russian}{Абай қазақ қоғамындағы әділетсіздік, парақорлық, арамзалық және еріншектік сияқты теріс қылықтарды ашық сынға алады. Ол халықтың бір-бірін аңдып, ұрлық жасап, шын келісімге жете алмайтынына назар аударады. Ақын өнер, әдебиет және ғылымға көңіл бөліп, қоғамдық тәрбиеге баса мән береді.} \\
\\ \\ & Describe how the Kazakh poet Abai criticized the negative aspects of Kazakh society. & Abai openly criticizes such wrongdoings in Kazakh society as injustice, bribery, dishonesty and laziness. He draws attention to the fact that the people follow each other, steal and do not reach a true agreement. The poet pays attention to art, literature and science, and pays special attention to public education.  \\ 
\midrule
Media and Communication & \foreignlanguage{russian}{Гүлмира Әлібекқызы Қаракөзованың қазақ журналистика саласындағы жетістіктерін түсіндіріңіз.} & \foreignlanguage{russian}{Гүлмира Қаракөзова - тәжірибелі журналист. Ол «Возвращенные имена» (1997) арнайы репортаждары топтамасының, «Паритет» (2002-2004) тележобасының және «Жұлдызды Астана» (2005 жылдан) радиожобасының авторы. 2004 жылдан бастап «Астана» радиосын басқарады. «Мәдениет қайраткері» құрмет белгісімен (2006) және «Астана» медалімен (2008) марапатталған.} \\
\\ \\ & Explain the achievements of Gulmira Alibekkyzy Karakozova in the field of Kazakh journalism. & Gulmira Karakozova is an experienced journalist. She is the author of a series of special reports "Returning names" (1997), the TV project "The Parity" (2002-2004) and the Radio Project "Starry Astana" (since 2005). Since 2004, she has been leading Astana radio. She was awarded the Badge of Honor "Madeniyet kairatkeri" (2006) and the medal "Astana" (2008).  \\ 
\midrule
Miscellaneous & \foreignlanguage{russian}{Қазақстан заңгері мен еңбек сіңірген прокуроры Қабдолла Досбалаұлы Төлегеновтың өміріндегі маңызды қызметтерді сипаттаңыз.} & \foreignlanguage{russian}{Қабдолла Досбалаұлы Төлегенов 1952 жылы туған. Ол 1975 жылы Ташкент мемлекеттік университетінің заң факультетін бітірген. Әр жылдарда түрлі қызметтерді атқарған, соның ішінде Шымкент, Алматы облыстарының прокуроры, ҚР Қауіпсіздік кеңесінің Хатшылығы меңгерушісінің орынбасары және басқалары. 2008 жылдан бері Шығыс Қазақстан облысының прокуроры. Ол «Құрмет» орденімен және басқа медальдармен марапатталған.} \\
\\ \\ & Describe the most important job positions in the life of the lawyer and honored prosecutor of Kazakhstan Kabdulla Dosbalauly Tulegenov. & Kabdolla Dosbalauly Tolegenov was born in 1952. He graduated from the faculty of law of Tashkent State University in 1975. Over the years, he held various positions, including prosecutor of Shymkent, Almaty regions, deputy head of the Secretariat of the Security Council of the Republic of Kazakhstan and others. He is the prosecutor of the East Kazakhstan region since 2008. He was awarded the Order "Kurmet" and other medals.  \\ 
\midrule
Nature and Environment & \foreignlanguage{russian}{Қазақстандағы Көлсай көлдерінің ұлттық табиғи паркінде кездесетін ерекше өсімдіктерді сипаттаңыз.} & \foreignlanguage{russian}{Көлсай көлдерінің ауданында ерекше сирек кездесетін өсімдіктерден өрік, Іле барбарисі, долана, Альберт құртқашашы бар. Аудан экологиялық тұрғыдан таза және жабайы өсімдіктерге бай. Бұл өсімдіктердің көбі дәрілік қасиетке ие.} \\
\\ \\ & Describe the unique plants found in the Kolsai Lakes National Natural Park in Kazakhstan. & In the area of Kolsai lakes, especially rare plants can be found - apricot, Ili barbaris, hawthorn, Iris albertii. The area is ecologically clean and rich in wild plants. Many of these plants have medicinal properties.  \\ 
\midrule
Politics & \foreignlanguage{russian}{Қазақстандық қоғам қайраткері Нұрлан Жылқышиевтің өмірбаяндық фактілерін суреттеңіз.} & \foreignlanguage{russian}{Нұрлан Жылқышиев 1953 жылы 3 мамырда Тараз қаласында туған. Ол Шымкент педагогикалық институтын және Алматы ауыл шаруашылығы институтын бітірген. 1998 жылы АҚШ-тың Тынық мұхиты жағалауы университетінде философия докторы дәрежесін қорғаған. Қабыланбек агрошарап компаниясының бас директоры болып қызмет атқарған және 2004 жылдан парламенттің мәжіліс депутаты. Ол әртүрлі марапаттарға ие болған, соның ішінде 'Тың және тыңайған жерлерді игерудің 50 жылдығы' төсбелгісі.} \\
 & Describe the biographical facts of the Kazakh public figure Nurlan Zhylkyshiev. & Nurlan Zhylkyshiev was born on May 3, 1953 in Taraz. He graduated from the Shymkent Pedagogical Institute and the Almaty Agricultural Institute. In 1998, he defended his doctorate in philosophy at the University of the Pacific coast of the United States. He worked as the head director of the agro-processing company Kabylanbek and since 2004 has been a deputy of the Mazhilis of the Parliament. He has received various awards, including the "50th anniversary of the development of Virgin and fallow lands" badge.  \\ 
\midrule
Sciences and Humanities & \foreignlanguage{russian}{Еділ Ерғожаұлы Ерғожинның химия саласында Қазақстанның мемлекеттік сыйлығын алғаны рас па?} & \foreignlanguage{russian}{Иә, Еділ Ерғожаұлы Ерғожин химия ғылымының докторы, профессор, Қазақстанның мемлекеттік сыйлығының иегері.} \\
\\ \\ & Is it true that Edil Yergozhauly Yerhozhin received the State prize of Kazakhstan in the field of chemistry? & Yes, Edil Yergozhauly Yergozhin is a doctor of chemical sciences, professor, winner of the State prize of Kazakhstan.  \\ 
\midrule
Sports and Recreation & \foreignlanguage{russian}{Қазақстандық пауэрлифтер-паралимпиадшы Давид Дегтярев 2020 жылғы жазғы Паралимпиада ойындарында қандай жетістіктерге жетті?} & \foreignlanguage{russian}{Қазақ пауэрлифтері Давид Дегтярев 2020 жылғы жазғы Паралимпиада ойындарында 54 кг дейінгі салмақ дәрежесінде жарысқа қатысып, 170 кг, 172 кг және 174 кг салмақтарды көтеру арқылы алтын медаль иегері атанды.} \\
 & What did Kazakhstani powerlifter-paralympian David Degtyarev achieve at the 2020 Summer Paralympic Games? & Kazakh powerlifter David Degtyarev competed at the 2020 Summer Paralympic Games in the weight category up to 54 kg and became the holder of a gold medal by lifting weights 170 kg, 172 kg and 174 kg.  \\ 
\midrule
Transportation and Infrastructure & \foreignlanguage{russian}{Қазақстандағы Ақтау Халықаралық Әуежайының тарихын қысқаша сипаттаңыз.} & \foreignlanguage{russian}{Ақтау Халықаралық Әуежайы 1983 жылы ашылған және көптеген ірі жүк тиейтін әуе кемелерін қабылдауға жабдықталған. 1996 жылы акционерлік қоғам ретінде құрылды. 2007 жылы жаңа жолаушылар терминалын салу мен басқаруға келісім жасалды. 2009 жылы жаңа терминал пайдалануға берілді, ол сағатына 450 жолаушыға қызмет көрсетеді. Әуежай жаңғыртулар нәтижесінде ТМД елдері арасында үздік әуежай атағын алды.} \\
\\ \\ & Briefly describe the history of Aktau International Airport in Kazakhstan. & Aktau International Airport was opened in 1983 and is equipped to receive many large cargo aircraft. It was established in 1996 as a joint stock company. In 2007, an agreement was signed for the construction and management of a new passenger terminal. In 2009, a new terminal was put into operation, serving 450 passengers per hour. As a result of the modernization, the airport received the title of the best airport among the CIS countries.  \\ 

\bottomrule
\end{tabular} }
\caption{Example of created instructions on \texttt{CultSet}.}
\end{table*}

\clearpage
\section{Inner-annotator agreement}
\label{app:inner-annot-appendix}
\subsection{Inner-Annotator Agreement for Preliminary Study}
\label{app:inner-annot-prelim}


\begin{figure*}[h!]
    \centering
    \begin{minipage}[b]{0.3\textwidth}
        \centering
        \includegraphics[scale=0.35]{images/correctness_gpt.png}
        %\caption*{(a) CultSet}
    \end{minipage}
    \hfill
    \begin{minipage}[b]{0.3\textwidth}
        \centering
        \includegraphics[scale=0.35]{images/completeness_gpt.png}
        %\caption*{(b) GovSet}
    \end{minipage}
    \hfill
    \begin{minipage}[b]{0.3\textwidth}
        \centering
        \includegraphics[scale=0.35]{images/fluency_gpt.png} % Replace with your third image file
        %\caption*{(c) Third Set}
    \end{minipage}
    \caption{Inner annotator agreement across annotators for correctness, completeness, and fluency, measured using Pearson correlation.}

    \label{fig:inner-annotator-agreement}
\end{figure*}

\subsection{Inner-Annotator Agreement for Generation Evaluation}
\label{app:inner-annot-gen-eval}

\begin{figure}[h!]
    \centering
    \includegraphics[scale=0.45]{images/generation_agreement.png} % Replace with your actual image file
    \caption{Inner-annotator agreement for generation evaluation, measured using Cohen’s Kappa.}

    \label{fig:inner-annotator-generation}
\end{figure}


\clearpage
%\include{tables/cohen_appendix}


\section{MCQ Examples}
\label{sec:mcq_samples}
\begin{table*}[ht]
\centering
\renewcommand{\arraystretch}{1.5} % Adjust row height
\scriptsize % Reduce font size
\begin{tabular}{|p{2cm}|p{5cm}|p{5cm}|p{1cm}|}  % Left-align columns
\hline
\textbf{Question Set}&\textbf{Kazakh} & \textbf{English translation} & \textbf{Correct Answer}\\ \hline
Dastur & \foreignlanguage{russian}{Тыйым деген не? \newline
A) Қазақ халқының той рәсімі \newline
B) Қазақ халқының тәрбиелік құралы \newline
C) Қазақ халқының музыкалық құралы \newline
D) Қазақ халқының аспап-құралы\newline
} & 
What is a tiyim? \newline
A) A Kazakh wedding ceremony \newline
B) A Kazakh educational tool \newline
C) A Kazakh musical instrument \newline
D) A Kazakh musical instrument\newline
& B \\
\hline
Dastur & \foreignlanguage{russian}{
Белкөтерер тағамы қандай адамдарға арнап дайындалады? \newline
A) Жас адамдарға \newline
B) Тісі болмаған, тамақ жей алмаған адамдарға \newline 
C) Спортшыларға
 \newline
D) Бала-шағаларға \newline
} & 
What kind of people is Belkoterer food prepared for? \newline
A) Young people \newline
B) People who have no teeth and cannot eat \newline 
C) Athletes
\newline
D) Children \newline
&
B \\
\hline
Constitution &
\foreignlanguage{russian}{
Конституция АДАМ ЖӘНЕ АЗАМАТ II бөлімінің  негізгі мәні қандай? \newline

A) Адамдың қадір-қасиетін қорлау \newline
B) Адамдың қадір-қасиетіне қол сұғу\newline
C) Адамдың қадір-қасиетіне қол сұғылмайды \newline
D) Адамдың қадір-қасиетін жою\newline
}&
What is the main idea of the Constitution's Section II, chapter on rights of people and citizens? \newline

A) Insulting human dignity \newline
B) Infringement on human dignity\newline
C) Human dignity is inviolable \newline
D) Destruction of human dignity\newline
&
C \\
\hline
Constitution &
\foreignlanguage{russian}{
Қазақстан Республикасының басқару нысаны қандай? \newline
A) Парламенттік басқару \newline
B) Президенттік басқару\newline
C) Монархиялық басқару \newline
D) Федерациялық басқару \newline
} & 
What is the form of government of the Republic of Kazakhstan? \newline
A) Parliamentary government \newline
B) Presidential government\newline
C) Monarchical government \newline
D) Federal government \newline
& B \\
\hline
Human Rights and Society &
\foreignlanguage{russian}{
ҚР Парламент Мәжілісі депутаты болу үшін қойылатын талаптарды атаңыз \newline
A) он жыл еңбек өтілі\newline
B) елу жастан аспау\newline
C) жоғары білім\newline
D) жиырма беске толу\newline

} & 
What are the requirements for becoming a deputy of the Mazhilis of the Parliament of the Republic of Kazakhstan?\newline
A) ten years of work experience\newline
B) not older than fifty years\newline
C) higher education\newline
D) twenty-five years of age\newline
& D\\
\hline
Human Rights and Society & 
\foreignlanguage{russian}{
Он алты жастан он сегіз жасқа дейінгі қызметкерлер үшін жұмыс уақытының ұзақтығы аптасына аспауы тиіс:\newline
A) 36 сағаттан\newline
B) 32 сағаттан\newline
C) 24 сағаттан\newline
D) 34 сағаттан\newline
}&

For employees aged sixteen to eighteen, the working hours per week shall not exceed:\newline
A) 36 hours\newline
B) 32 hours\newline
C) 24 hours\newline
D) 34 hours\newline&
A \\ \hline
\end{tabular}
\caption{Examples of MCQ.}
\label{tab:mcq_samples}
\end{table*}
\clearpage

