\section{Background} 
%\subsection{Kazakh Cultural Heritage}
\textbf{Kazakh Cultural Heritage.} Kazakhstan has a rich cultural heritage that reflects a blend of nomadic traditions, Soviet influences, and modern developments. 

The country’s nomadic heritage is evident in many aspects of daily life, from its architecture, with \textit{yurts} (traditional felt tents) still used in rural areas, to its customs of communal gatherings and feasts, such as the celebration of the Turkic New Year, \textit{Nauryz}. Kazakh hospitality is a source of pride for the nation, embodied in proverbs like \textit{"Qonaq keldi — ırysın ala keldi"} ("A guest arrives, bringing fortune with them"). It is deeply important that guests are never left hungry. They are always offered a variety of traditional dishes, such as \textit{beshbarmak} (a hearty dish of boiled meat with pasta) and \textit{baursak} (deep-fried dough). Kazakhstan’s Soviet past has also left a lasting imprint on its culture. Many cities still bear the architectural marks of Soviet planning, while the era also shaped the country’s education and scientific institutions, fostering a strong tradition in mathematics and engineering—most notably reflected in the \textit{Baikonur Cosmodrome}, the world’s first and largest space launch facility. Alongside all of this, modern developments have transformed Kazakhstan. The capital, Astana, is a prime example of this shift, with its futuristic skyline and ambitious urban projects Investments in technology, renewable energy, and digital infrastructure have propelled Kazakhstan onto the global stage, while cultural revitalization efforts have fostered a renewed interest in the Kazakh language, music, and art. \\
At it's core, Kazakhstan’s culture is shaped by beliefs and values, social practices, language, artistic expression, and material culture. Understanding these components is crucial for ensuring accurate and meaningful representation.

\textbf{Kazakhstan’s Institutional Structure and the Public Governance.}
Kazakhstan operates as a presidential republic, and since gaining independence in 1991, the country has focused on modernization, particularly in governance and legal frameworks. The Constitution of Kazakhstan, adopted in 1995, provides the foundation for the country’s laws, establishing the rights and responsibilities of citizens, as well as the structure of the government. A significant shift has been the digitalization of many governmental processes, making services more accessible to the public. Kazakhstan ranks among the top 25 countries in the UN E-Government Development Index (EGDI) \citep{UN_EGDI}. One of the most notable achievements is the development of the \textit{eGov} platform, a centralized system providing citizens with streamlined access to various government services. \textit{eGov} serves as a one-stop portal for activities such as registering businesses, paying taxes, applying for social benefits, and obtaining official documents like passports and ID cards. It is a comprehensive source of governmental services in Kazakhstan, simplifying processes that previously required significant time and paperwork. 

These various aspects of Kazakhstan—its rich cultural heritage, nomadic traditions, modern governance, and the increasing digitalization of public services—reflect the unique characteristics and dynamics of the country. As Kazakhstan continues to modernize and integrate technological advancements into everyday life, it is crucial that these characteristics are properly represented in tools like language models.



% When writing about government we should say that our dataset covers not only how does Egov work, but also nuances of our political structure (what are the main parties, who rules them?), legal aspects, etc

% Why? because it will allow us to justify usage of law-mc
% How governmental processes work in Kazakhstan?
% why it is important? Many processes are digitalized in KZ, people use egov to check all their questions regarding 

% what is culture in kz? traditional food, people, tradition, clothes, museums, plays, musical instruments, history


