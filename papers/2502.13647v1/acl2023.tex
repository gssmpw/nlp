% This must be in the first 5 lines to tell arXiv to use pdfLaTeX, which is strongly recommended.

% In particular, the hyperref package requires pdfLaTeX in order to break URLs across lines.

\documentclass[final]{article}

% Remove the "review" option to generate the final version.
\usepackage[final]{ACL2023}

% Standard package includes
\usepackage{times}
\usepackage{latexsym}
%\usepackage[utf8]{inputenc}
%\usepackage{utf8}
%\setcode{utf8}
\usepackage{tikz}
\usepackage{xcolor}% Adjust margins for more space
\usepackage{xcolor}
\usepackage{lipsum} % For dummy text, remove if unnecessary

\usepackage[utf8]{inputenc}
\usepackage{graphicx}
\usepackage{placeins}

% and will typically save some space.
\usepackage{microtype}
%\usepackage{lmodern} % For modern fonts
\usepackage{listings} 
% This is also not strictly necessary and may be commented out.
% However, it will improve the aesthetics of text in
% the typewriter font.
\usepackage{inconsolata}
\usepackage{import}
\usepackage{microtype}
\usepackage{layout}
\usepackage{tabularx, makecell}
\usepackage{booktabs}
\usepackage{mathrsfs}
\usepackage{amssymb} 
\usepackage{url}
\usepackage{graphicx}
\usepackage{xspace,paralist}
\usepackage{times,latexsym}
\usepackage{amsmath}
\usepackage{appendix}
\usepackage{comment} 
\usepackage{enumitem}
\usepackage{makecell}
\usepackage{multirow}
\usepackage{colortbl}
\usepackage{xcolor}
\usepackage{tablefootnote}
\usepackage{enumitem}  % More control
\usepackage{lipsum}
\usepackage{xspace}
\usepackage{bbding}
\usepackage{pifont}
\usepackage{arydshln}
\usepackage{array}
\usepackage{cleveref}

\definecolor{darkgreen}{rgb}{0.0, 0.5, 0.0}
%\usepackage[T2A]{fontenc}
%\usepackage[T1]{fontenc}
%\usepackage[T2A]{fontenc}   % Set Cyrillic font encoding

\usepackage[russian,english]{babel}

\newcommand{\eqnref}[1]{Eq~\eqref{#1}\xspace}
\newcommand{\tabref}[2][]{Table#1~\ref{#2}\xspace}
\newcommand{\figref}[1]{Figure~\ref{#1}\xspace}
\newcommand{\secref}[1]{Section~\ref{#1}\xspace}
\newcommand{\appref}[1]{Appendix~\ref{#1}\xspace}

\newcommand{\errortype}[1]{\textit{#1}\xspace}
\newcommand{\ftype}{\errortype{Type1}}
\newcommand{\stype}{\errortype{Type2}}
\newcommand{\ttype}{\errortype{Type3}}

\DeclareMathOperator{\inff}{inf}
\newcommand{\metric}[1]{\textsc{#1}\xspace}
\newcommand{\ece}{\metric{ECE}}
\newcommand{\nlpd}{\metric{NLPD}}
\newcommand{\kl}{\metric{KL}}
\newcommand{\NA}{\textsc{n/a}}
\newcommand{\sd}[1]{\smaller{\ensuremath{\pm {#1}}}}

\newcommand{\dataset}[1]{\text{#1}\xspace}
\newcommand{\nrisk}{\text{61}\xspace}

\newcommand{\model}[1]{\text{#1}\xspace}
\newcommand{\mcdropout}{\model{MC-Dropout}}
\newcommand{\chatgpt}{\model{ChatGPT}}
\newcommand{\gptfour}{\model{GPT-4}}
\newcommand{\instructgpt}{\model{InstructGPT}}
\newcommand{\gptthree}{\model{GPT-3}}
\newcommand{\bard}{\model{Bard}}
\newcommand{\llama}{\model{LLaMA}}
\newcommand{\llamatwo}{\model{LLaMA-2}}
\newcommand{\chatglm}{\model{ChatGLM2}}
\newcommand{\claude}{\model{Claude}}
\newcommand{\alpaca}{\model{Alpaca}}
\newcommand{\vicuna}{\model{Vicuna}}
\newcommand{\longformer}{\model{Longformer}}
\newcommand{\bert}{\model{BERT}}
\newcommand{\repo}{\url{https://github.com/Libr-AI/do-not-answer}}
%\newcommand{\cn}[1]{\begin{CJK*}{UTF8}{gbsn}#1\end{CJK*}}
\usepackage{CJKutf8}
% \newcommand{\repo}{\url{https://anonymous.for.review}}


% For proper rendering and hyphenation of words containing Latin characters (including in bib files)
\usepackage{float}
\restylefloat{table}

\usepackage[T1]{fontenc}
% For Vietnamese characters
% \usepackage[T5]{fontenc}
% See https://www.latex-project.org/help/documentation/encguide.pdf for other character sets
\usepackage{arydshln}
% This assumes your files are encoded as UTF8
\usepackage[utf8]{inputenc}

% This is not strictly necessary, and may be commented out.
% However, it will improve the layout of the manuscript,
% and will typically save some space.
\usepackage{microtype}

% This is also not strictly necessary, and may be commented out.
% However, it will improve the aesthetics of text in
% the typewriter font.
\usepackage{inconsolata}
\usepackage{CJK}
\usepackage[T2A,T1]{fontenc}
\AtBeginDocument{%
  \DeclareFontFamilySubstitution{T2A}{\familydefault}{Tempora-TLF}%
}
% If the title and author information does not fit in the area allocated, uncomment the following
%
%\setlength\titlebox{<dim>}
%
% and set <dim> to something 5cm or larger.

% old title
% \title{Instruction Tuning on Public Government and Cultural Data: \\A Case Study in a Low-Resource Language}

% Preslav's title
\title{Instruction Tuning on Public Government and Cultural Data\\ for Low-Resource Language: a Case Study in Kazakh}

%\title{Instruction Tuning for Low-Resource Languages: Enhancing Practical and Cultural Knowledge in Kazakh
%}

\author{
  Nurkhan Laiyk$^1$\thanks{\hspace{0.2cm}These authors contributed equally.}  \quad 
  Daniil Orel$^{1*}$ \quad  
  Rituraj Joshi$^2$ \quad  
  Maiya Goloburda$^1$ \\  
  \textbf{Yuxia Wang}$^1$ \quad  
  \textbf{Preslav Nakov}$^1$ \quad  
  \textbf{Fajri Koto}$^1$ \\  
  \textsuperscript{1}Department of Natural Language Processing, MBZUAI \\  
  \textsuperscript{2}Cerebras Systems \\  
  \texttt{\small \{nurkhan.laiyk,daniil.orel\}@mbzuai.ac.ae} \\
}

% Author information can be set in various styles:
% For several authors from the same institution:
% \author{Author 1 \and ... \and Author n \\
%         Address line \\ ... \\ Address line}
% if the names do not fit well on one line use
%         Author 1 \\ {\bf Author 2} \\ ... \\ {\bf Author n} \\
% For authors from different institutions:
% \author{Author 1 \\ Address line \\  ... \\ Address line
%         \And  ... \And
%         Author n \\ Address line \\ ... \\ Address line}
% To start a seperate ``row'' of authors use \AND, as in
% \author{Author 1 \\ Address line \\  ... \\ Address line
%         \AND
%         Author 2 \\ Address line \\ ... \\ Address line \And
%         Author 3 \\ Address line \\ ... \\ Address line}


\usepackage{graphicx}
\begin{document}
\maketitle

\begin{abstract}

% \todo{Preslav: Why not mention Kazakh in the title? See my proposed title.}
% yes, looks good to me.

Instruction tuning in low-resource languages remains underexplored due to limited text data, particularly in government and cultural domains. To address this, we introduce and open-source a large-scale (10,600 samples) instruction-following (IFT) dataset, covering key institutional and cultural knowledge relevant to Kazakhstan. Our dataset enhances LLMs' understanding of procedural, legal, and structural governance topics.
We employ LLM-assisted data generation, comparing open-weight and closed-weight models for dataset construction, and select GPT-4o as the backbone. Each entity of our dataset undergoes full manual verification to ensure high quality. We also show that fine-tuning Qwen, Falcon, and Gemma on our dataset leads to consistent performance improvements in both multiple-choice and generative tasks, demonstrating the potential of LLM-assisted instruction tuning for low-resource languages.

% Instruction tuning in low-resource languages remains underexplored due to the scarcity of text data, particularly in domains such as government and culture. Public government data, while valuable for real-world applications, is often difficult to access or underrepresented in existing datasets. Similarly, cultural data plays a crucial role in ensuring that models align with local contexts, yet it remains largely untapped for instruction tuning. In this work, we investigate the adaptation of LLMs to these underrepresented domains, using Kazakh as a case study. Our approach leverages high-quality unlabeled texts to facilitate LLM-assisted dataset creation through a single-prompt method that extracts both factual information and corresponding questions. To ensure data quality, we evaluate multiple LLMs for dataset construction and ultimately select GPT-4o as the backbone. Every instance in our dataset undergoes full manual verification to maintain high standards. Our dataset consists of 10,600 instruction-following (IFT) pairs, and we assess the impact of instruction tuning by fine-tuning Qwen, Falcon, and Gemma on our dataset, observing consistent performance improvements in both multiple-choice and generative tasks.


%Instruction tuning has emerged as a prominent method for training Large Language Models (LLMs) to effectively interpret and follow instructions. However, most existing instruction datasets are heavily focused on English or developed using LLMs biased toward English-speaking contexts, often resulting in a Western cultural perspective. This cultural bias can affect the syntactic and semantic accuracy of non-English languages, such as Kazakh, which has unique grammatical structures and cultural nuances. To address this limitation, we present "Instruction Tuning for Low-Resource Languages: Enhancing Practical and Cultural Knowledge in Kazakh"—an instruction-tuning dataset specifically designed for Kazakh and culturally validated by native speakers.

%For dataset collection, we gathered 4,400 texts from Kazakh Wikipedia and 1,300 from government sources using a combination of manual and automated methods. Additionally, we utilized GPT-4o to generate instruction pairs based on this content, expanding the dataset to a total of 9,300 instruction-following question-answer pairs. These datasets were then fine-tuned on Llama-3.1-8B and Qwen-2.5-7B to better capture the linguistic and cultural nuances of the Kazakh language. Our dataset is publicly available at \url{https://github.com/nurkhanl5/KazakhIFT}.

\end{abstract}



\section{Introduction}

Video generation has garnered significant attention owing to its transformative potential across a wide range of applications, such media content creation~\citep{polyak2024movie}, advertising~\citep{zhang2024virbo,bacher2021advert}, video games~\citep{yang2024playable,valevski2024diffusion, oasis2024}, and world model simulators~\citep{ha2018world, videoworldsimulators2024, agarwal2025cosmos}. Benefiting from advanced generative algorithms~\citep{goodfellow2014generative, ho2020denoising, liu2023flow, lipman2023flow}, scalable model architectures~\citep{vaswani2017attention, peebles2023scalable}, vast amounts of internet-sourced data~\citep{chen2024panda, nan2024openvid, ju2024miradata}, and ongoing expansion of computing capabilities~\citep{nvidia2022h100, nvidia2023dgxgh200, nvidia2024h200nvl}, remarkable advancements have been achieved in the field of video generation~\citep{ho2022video, ho2022imagen, singer2023makeavideo, blattmann2023align, videoworldsimulators2024, kuaishou2024klingai, yang2024cogvideox, jin2024pyramidal, polyak2024movie, kong2024hunyuanvideo, ji2024prompt}.


In this work, we present \textbf{\ours}, a family of rectified flow~\citep{lipman2023flow, liu2023flow} transformer models designed for joint image and video generation, establishing a pathway toward industry-grade performance. This report centers on four key components: data curation, model architecture design, flow formulation, and training infrastructure optimization—each rigorously refined to meet the demands of high-quality, large-scale video generation.


\begin{figure}[ht]
    \centering
    \begin{subfigure}[b]{0.82\linewidth}
        \centering
        \includegraphics[width=\linewidth]{figures/t2i_1024.pdf}
        \caption{Text-to-Image Samples}\label{fig:main-demo-t2i}
    \end{subfigure}
    \vfill
    \begin{subfigure}[b]{0.82\linewidth}
        \centering
        \includegraphics[width=\linewidth]{figures/t2v_samples.pdf}
        \caption{Text-to-Video Samples}\label{fig:main-demo-t2v}
    \end{subfigure}
\caption{\textbf{Generated samples from \ours.} Key components are highlighted in \textcolor{red}{\textbf{RED}}.}\label{fig:main-demo}
\end{figure}


First, we present a comprehensive data processing pipeline designed to construct large-scale, high-quality image and video-text datasets. The pipeline integrates multiple advanced techniques, including video and image filtering based on aesthetic scores, OCR-driven content analysis, and subjective evaluations, to ensure exceptional visual and contextual quality. Furthermore, we employ multimodal large language models~(MLLMs)~\citep{yuan2025tarsier2} to generate dense and contextually aligned captions, which are subsequently refined using an additional large language model~(LLM)~\citep{yang2024qwen2} to enhance their accuracy, fluency, and descriptive richness. As a result, we have curated a robust training dataset comprising approximately 36M video-text pairs and 160M image-text pairs, which are proven sufficient for training industry-level generative models.

Secondly, we take a pioneering step by applying rectified flow formulation~\citep{lipman2023flow} for joint image and video generation, implemented through the \ours model family, which comprises Transformer architectures with 2B and 8B parameters. At its core, the \ours framework employs a 3D joint image-video variational autoencoder (VAE) to compress image and video inputs into a shared latent space, facilitating unified representation. This shared latent space is coupled with a full-attention~\citep{vaswani2017attention} mechanism, enabling seamless joint training of image and video. This architecture delivers high-quality, coherent outputs across both images and videos, establishing a unified framework for visual generation tasks.


Furthermore, to support the training of \ours at scale, we have developed a robust infrastructure tailored for large-scale model training. Our approach incorporates advanced parallelism strategies~\citep{jacobs2023deepspeed, pytorch_fsdp} to manage memory efficiently during long-context training. Additionally, we employ ByteCheckpoint~\citep{wan2024bytecheckpoint} for high-performance checkpointing and integrate fault-tolerant mechanisms from MegaScale~\citep{jiang2024megascale} to ensure stability and scalability across large GPU clusters. These optimizations enable \ours to handle the computational and data challenges of generative modeling with exceptional efficiency and reliability.


We evaluate \ours on both text-to-image and text-to-video benchmarks to highlight its competitive advantages. For text-to-image generation, \ours-T2I demonstrates strong performance across multiple benchmarks, including T2I-CompBench~\citep{huang2023t2i-compbench}, GenEval~\citep{ghosh2024geneval}, and DPG-Bench~\citep{hu2024ella_dbgbench}, excelling in both visual quality and text-image alignment. In text-to-video benchmarks, \ours-T2V achieves state-of-the-art performance on the UCF-101~\citep{ucf101} zero-shot generation task. Additionally, \ours-T2V attains an impressive score of \textbf{84.85} on VBench~\citep{huang2024vbench}, securing the top position on the leaderboard (as of 2025-01-25) and surpassing several leading commercial text-to-video models. Qualitative results, illustrated in \Cref{fig:main-demo}, further demonstrate the superior quality of the generated media samples. These findings underscore \ours's effectiveness in multi-modal generation and its potential as a high-performing solution for both research and commercial applications.

\section{Related Work}

\subsection{Large 3D Reconstruction Models}
Recently, generalized feed-forward models for 3D reconstruction from sparse input views have garnered considerable attention due to their applicability in heavily under-constrained scenarios. The Large Reconstruction Model (LRM)~\cite{hong2023lrm} uses a transformer-based encoder-decoder pipeline to infer a NeRF reconstruction from just a single image. Newer iterations have shifted the focus towards generating 3D Gaussian representations from four input images~\cite{tang2025lgm, xu2024grm, zhang2025gslrm, charatan2024pixelsplat, chen2025mvsplat, liu2025mvsgaussian}, showing remarkable novel view synthesis results. The paradigm of transformer-based sparse 3D reconstruction has also successfully been applied to lifting monocular videos to 4D~\cite{ren2024l4gm}. \\
Yet, none of the existing works in the domain have studied the use-case of inferring \textit{animatable} 3D representations from sparse input images, which is the focus of our work. To this end, we build on top of the Large Gaussian Reconstruction Model (GRM)~\cite{xu2024grm}.

\subsection{3D-aware Portrait Animation}
A different line of work focuses on animating portraits in a 3D-aware manner.
MegaPortraits~\cite{drobyshev2022megaportraits} builds a 3D Volume given a source and driving image, and renders the animated source actor via orthographic projection with subsequent 2D neural rendering.
3D morphable models (3DMMs)~\cite{blanz19993dmm} are extensively used to obtain more interpretable control over the portrait animation. For example, StyleRig~\cite{tewari2020stylerig} demonstrates how a 3DMM can be used to control the data generated from a pre-trained StyleGAN~\cite{karras2019stylegan} network. ROME~\cite{khakhulin2022rome} predicts vertex offsets and texture of a FLAME~\cite{li2017flame} mesh from the input image.
A TriPlane representation is inferred and animated via FLAME~\cite{li2017flame} in multiple methods like Portrait4D~\cite{deng2024portrait4d}, Portrait4D-v2~\cite{deng2024portrait4dv2}, and GPAvatar~\cite{chu2024gpavatar}.
Others, such as VOODOO 3D~\cite{tran2024voodoo3d} and VOODOO XP~\cite{tran2024voodooxp}, learn their own expression encoder to drive the source person in a more detailed manner. \\
All of the aforementioned methods require nothing more than a single image of a person to animate it. This allows them to train on large monocular video datasets to infer a very generic motion prior that even translates to paintings or cartoon characters. However, due to their task formulation, these methods mostly focus on image synthesis from a frontal camera, often trading 3D consistency for better image quality by using 2D screen-space neural renderers. In contrast, our work aims to produce a truthful and complete 3D avatar representation from the input images that can be viewed from any angle.  

\subsection{Photo-realistic 3D Face Models}
The increasing availability of large-scale multi-view face datasets~\cite{kirschstein2023nersemble, ava256, pan2024renderme360, yang2020facescape} has enabled building photo-realistic 3D face models that learn a detailed prior over both geometry and appearance of human faces. HeadNeRF~\cite{hong2022headnerf} conditions a Neural Radiance Field (NeRF)~\cite{mildenhall2021nerf} on identity, expression, albedo, and illumination codes. VRMM~\cite{yang2024vrmm} builds a high-quality and relightable 3D face model using volumetric primitives~\cite{lombardi2021mvp}. One2Avatar~\cite{yu2024one2avatar} extends a 3DMM by anchoring a radiance field to its surface. More recently, GPHM~\cite{xu2025gphm} and HeadGAP~\cite{zheng2024headgap} have adopted 3D Gaussians to build a photo-realistic 3D face model. \\
Photo-realistic 3D face models learn a powerful prior over human facial appearance and geometry, which can be fitted to a single or multiple images of a person, effectively inferring a 3D head avatar. However, the fitting procedure itself is non-trivial and often requires expensive test-time optimization, impeding casual use-cases on consumer-grade devices. While this limitation may be circumvented by learning a generalized encoder that maps images into the 3D face model's latent space, another fundamental limitation remains. Even with more multi-view face datasets being published, the number of available training subjects rarely exceeds the thousands, making it hard to truly learn the full distibution of human facial appearance. Instead, our approach avoids generalizing over the identity axis by conditioning on some images of a person, and only generalizes over the expression axis for which plenty of data is available. 

A similar motivation has inspired recent work on codec avatars where a generalized network infers an animatable 3D representation given a registered mesh of a person~\cite{cao2022authentic, li2024uravatar}.
The resulting avatars exhibit excellent quality at the cost of several minutes of video capture per subject and expensive test-time optimization.
For example, URAvatar~\cite{li2024uravatar} finetunes their network on the given video recording for 3 hours on 8 A100 GPUs, making inference on consumer-grade devices impossible. In contrast, our approach directly regresses the final 3D head avatar from just four input images without the need for expensive test-time fine-tuning.


%structure of paper:
%1 intro
%2 backgroud: systolic array: mix size regularity, replace flow: not good at PE array placement, without using regularity info( 1 page)
%3 preliminary and motivation: formulation of mixed size placement problem, (better quality) -> ILP, runtime-> regularity -> reduce searching space( 1 page)
%4 method (1.5 page)
%5 experiment(1 page) 


\begin{figure*}[ht]
    \centering
    \includegraphics[width=0.9\linewidth]{figure/workfolw15.pdf}
    \caption{Overview of the Lorecast methodology.}
    \label{fig:workfolw1}
\end{figure*}

% \thispagestyle{empty}
\section{Background}
\label{sec:background}

\subsection{Large Language Models }
\label{sec:graph_clustering}
Large Language Models (LLMs) are advanced AI systems designed to understand and generate human-like text by processing vast amounts of language data\cite{Radford2019LLM}. These models are typically based on Transformer architectures\cite{Vaswani2017Transformer} and employ billions of parameters to capture complex patterns and relationships within language, enabling them to perform tasks such as text generation, translation, summarization, and question answering\cite{Brown2020LLMLearners}. Prominent examples include OpenAI’s GPT-4 \cite{OpenAI2023GPT-4} and Google’s Gemini 1.5 Pro \cite{Google2024Gemini}, showcasing the immense capabilities of LLMs. With 175 billion parameters and a context capacity of 1 million tokens, these models achieve near-human performance across a variety of Natural Language Processing (NLP) tasks. The primary strength of LLMs lies in their pretraining on extensive text corpora, which gives them a general understanding of language that can be fine-tuned for specific tasks or domains \cite{Bommasani2022finite}\cite{Liu2024RTLCoder}\cite{Thakur2023Benchmarking}. As LLMs continue to evolve, their applications are expanding across various fields, including EDA, where they are increasingly employed for tasks such as code generation\cite{Wang2024ChatCPU}\cite{Wong2024VGV}\cite{Pei2024BetterV}.
\subsection{Verilog and Code Correctness}
Verilog is a hardware description language (HDL) widely used in digital design and circuit development for specifying and modeling electronic systems at various abstraction levels, from high-level functional descriptions to detailed structural representations \cite{Palnitkar2003Verilog}. Ensuring code correctness in Verilog is crucial because errors at the HDL level can lead to significant functional failures, performance inefficiencies, and increased costs when translated to physical hardware. Code correctness in Verilog typically includes both syntax correctness, ensuring code is free from syntax errors, and functional correctness, validating that the code’s behavior aligns with design specifications \cite{Harris2010CMOS}. Recent advances in machine learning and automated code generation, including the use of LLMs, have opened new avenues for improving Verilog code correctness through automated error detection and code synthesis\cite{Wang2024ChatCPU,Pei2024BetterV}.

\iffalse
\subsection{systolic array}
describe the pipeline and parallel computing function 
due to the repeating PE structure and its dataflow.
IO and control logic unmatch to the regular schematic of the PE array
\fi



\section{Data}
\subsection{Document Source}

\paragraph{\texttt{GovSet}} We manually collected 1,376 texts from the official Kazakhstan e-Government portal (gov.kz\footnote{\url{https://www.gov.kz}}), the primary and most comprehensive platform for all public services, governmental processes, and administrative resources in the country. As the central hub for Kazakhstan's digital governance, \texttt{gov.kz} consolidates a wide range of essential information into a single system, covering diverse aspects of public administration, legal frameworks, citizen services, and governmental initiatives. 
% REPETITION The collected texts cover various aspects of Kazakh public life and services, reflecting the country’s ongoing efforts to digitalize and streamline public administration through its e-government framework. 
By incorporating these texts, we ensure that the dataset captures essential institutional aspects of life in Kazakhstan, including its governmental structure and public services. This enrichment enhances instruction-tuning applications, making them more linguistically appropriate and contextually informed.


\paragraph{\texttt{CultSet}} We automatically collected 4,400 texts from Kazakh Wikidata,\footnote{\href{https:_/_/kk.wikipedia.org}{kk.wikipedia.org}} specifically focusing on pages related to Kazakh culture. These pages were identified based on metadata that explicitly indicated their relevance to Kazakh cultural topics. 
The parsed texts include various aspects of Kazakh traditions, heritage, arts, and historical practices, providing a rich source of culturally relevant content.
This ensures that the dataset reflects the depth and diversity of Kazakh culture, making it suitable for instruction-tuning tasks that require a culturally grounded perspective.

%To ensure the uniqueness and quality of the dataset, we performed a MinHash deduplication.

\subsection{LLM-assisted Data Generation}

We benchmark one open-weight LLM: LLaMA 3.1-70B \cite{touvron2023llama}, and three closed-weight LLMs: GPT-4o \cite{openai2024gpt4o}, Gemini-1.5 \cite{google2024gemini15}, and Claude-3.5-Sonnet \cite{anthropic2024claude}, to assess their effectiveness in assisting dataset creation. These models were selected based on their strong performance in multilingual benchmarks. However, their capability in generating instruction datasets specific to Kazakh government and cultural data remains uncertain.

%For this study, we used the following LLMs: LLaMA 3.1-70B \cite{touvron2023llama}, GPT-4o \cite{openai2024gpt4o}, Gemini-1.5 \cite{google2024gemini15}, and Claude-3.5-sonnet \cite{anthropic2024claude}.  We selected these LLMs because of their advanced capabilities and widespread use, making them suitable for evaluating instruction generation across various tasks. 

We design a prompt (see Appendix~\ref{sec:prompts}) that instructs LLMs to first extract factual information from a given Kazakh document and then generate an instruction dataset based on the extracted content. Table~\ref{tab:ds_stats} provides detailed statistics on the source documents and the resulting instruction fine-tuning (IFT) dataset using GPT-4o. Specifically, we use 4,400 Kazakh cultural Wikipedia documents and 1,376 Kazakh government data sources, generating a total of 10,600 IFT instances. Of these, 58\% belong to the government public data category (\texttt{GovSet}), while the remaining samples are derived from Wikipedia (\texttt{CultSet}). The example of generated IFT data can be found in Table~\ref{tab:example_govset} and Table~\ref{tab:example_cultset}.

\paragraph{Human Evaluation Across LLMs}  
For each LLM, we sampled 100 generated IFT instances, drawn from 25 randomly selected \texttt{GovSet} and 25 \texttt{CultSet} documents. Additionally, we randomly sampled 100 instances from MURI~\cite{koksal2024muri}, which also includes Kazakh IFT data, to provide a comparative quality assessment. Two native Kazakh speakers were recruited to manually evaluate the generated data based on the following criteria:
\begin{compactitem}
    \item \textbf{Correctness}: The factual accuracy and alignment with the original text. A high score indicates that the generated pair adheres closely to the source material without introducing errors or inaccuracies.
    \item \textbf{Fluency}: The grammatical and stylistic quality of the generated text. A higher score reflects well-structured, natural, and polished language.
    \item \textbf{Completeness}: The degree to which the instruction-response pair is clear, contextually grounded, and free from ambiguity. High scores indicate that the pair is fully self-contained, with enough context to make it understandable. % and actionable.
\end{compactitem}
All criteria were rated on a Likert scale from 1 to 5, with 5 representing the highest quality. A detailed evaluation rubric is provided in Table~\ref{tab:multilingual-issues}.

%Each response was evaluated on a -pofiveint scale, assessing its alignment with predefined criteria (a detailed rubric is provided in Table \ref{tab:multilingual-issues}).
Table~\ref{tab:llm-instruction} presents the quality assessment of various LLMs in generating IFT data for Kazakh. The inter-annotator agreement, measured using Pearson correlation, is high (ranging from 0.68 to 0.70) across correctness, completeness, and fluency, indicating strong reliability in the evaluation process (see Appendix \ref{app:inner-annot-prelim} for further details).

Among the evaluated models, GPT-4o achieved the highest performance across all three criteria. In contrast, LLaMA-3.1 (70B) lagged significantly, scoring nearly 0.8–1 point lower in all aspects. Notably, MURI's quality was lower than GPT-4o despite both relying on OpenAI models. This discrepancy is likely due to MURI’s reliance on machine translation, where Kazakh text is first translated into English before generating instructions, followed by a final back-translation into Kazakh. This multi-step translation process can introduce errors due to cumulative translation inaccuracies. Additionally, MURI is entirely LLM-generated without human validation, further affecting its quality. 


%We additionally evaluated 100 samples, randomly selected from the publicly available MURI~\cite{koksal2024muri} dataset, as a quality baseline, since as mentioned by its authors, no additional cleaning of this dataset was done. 
%Examples of MURI instructions and their issues are given in Appendix \ref{appx:muri}.

%25 GovSet texts and 25 CultSet. Using these texts, we sampled 100 generated IFT pairs.%, ensuring a diverse set of instructions and responses.
%To assess the quality of the generated IFT pairs, we employed a manual evaluation framework where three annotators evaluated each pair based on the following criteria:


%\begin{table*}[htbp!]
%\centering
%\renewcommand{\arraystretch}{1} % Adjust row height
%%\resizebox{\textwidth}{!}{
%\scalebox{0.75}{
%\begin{tabular}{lcccccccccc}
%\toprule
%\textbf{Model name} & \multicolumn{3}{c}{\textbf{Reverse Instruction Generation}} & \multicolumn{3}{c}{\textbf{Fact Extraction with Reverse Generation}} \\  
%\cmidrule(lr){2-4} \cmidrule(lr){5-7} \cmidrule(lr){8-10}
%
%
%
%& \textbf{Correctness} & \textbf{Completeness} & \textbf{Fluency} 
%& \textbf{Correctness} & \textbf{Completeness} & \textbf{Fluency} \\ 
%\midrule
% Llama 3.1 - 70B  & 3.45 & 3.21 & 2.87 & 3.54 & 3.45 & 3.07\\
%GPT-4o & \textbf{4.05}  & \textbf{3.97} & \textbf{3.84} 
%& \textbf{4.38 $\uparrow$} & \underline{\textbf{4.29 $\uparrow$}} & \textbf{4.04$\uparrow$}  \\ 
%Claude & 3.63 & 3.41 & 2.98 & 3.74 & 3.48 & 3.09\\
%Gemini 1.5  & 3.54 & 3.53 & 3.2 & 3.85 & 3.64 & 3.32
%\\
%\midrule
%\end{tabular}}
%\caption{Comparison of Fine-tuning Conditions Across Models}
%\label{tab:finetuning-comparison}
%\end{table*}




\begin{table}[t]
\centering
%\renewcommand{\arraystretch}{1} % Adjust row height
 \resizebox{\linewidth}{!}{
    \begin{tabular}{lccc}
    \toprule
    \textbf{Model} & \textbf{Correctness} & \textbf{Completeness} & \textbf{Fluency} \\ 
    %\multicolumn{3}{c}{\textbf{LLM Instruction Generation Analysis}} \\  
    %\cmidrule(lr){2-4} 
    \midrule
    Llama 3.1 (70B)  & 3.54 & 3.45 & 3.07 \\
    Claude & 3.74 & 3.48 & 3.09 \\
    Gemini 1.5  & 3.85 & 3.64 & 3.32 \\
    GPT-4o & \textbf{4.38} & {\textbf{4.29}} & \textbf{4.04}  \\ 
    % \midrule
    \hdashline
    MURI & 3.87 & 3.52 & 3.41\\
    \bottomrule
    \end{tabular} 
} % Close scalebox correctly
\caption{Human evaluation on LLM-generated instruction datasets.}


\label{tab:llm-instruction}
\end{table}







%The results of this evaluation are summarized in Table \ref{tab:llm-instruction}. 




% More importantly, the Fact Extraction with Reverse Generation approach yielded the best performance across all criteria, making it the preferred method for subsequent experiments.
%This preliminary study helps us identify effective methods for generating high-quality instruction-tuning data tailored to Kazakh cultural and governmental contexts.
%%\begin{table*}[htbp!]
%\centering
%\renewcommand{\arraystretch}{1} % Adjust row height
%%\resizebox{\textwidth}{!}{
%\scalebox{0.75}{
%\begin{tabular}{lcccccccccc}
%\toprule
%\textbf{Model name} & \multicolumn{3}{c}{\textbf{Reverse Instruction Generation}} & \multicolumn{3}{c}{\textbf{Fact Extraction with Reverse Generation}} \\  
%\cmidrule(lr){2-4} \cmidrule(lr){5-7} \cmidrule(lr){8-10}
%
%
%
%& \textbf{Correctness} & \textbf{Completeness} & \textbf{Fluency} 
%& \textbf{Correctness} & \textbf{Completeness} & \textbf{Fluency} \\ 
%\midrule
% Llama 3.1 - 70B  & 3.45 & 3.21 & 2.87 & 3.54 & 3.45 & 3.07\\
%GPT-4o & \textbf{4.05}  & \textbf{3.97} & \textbf{3.84} 
%& \textbf{4.38 $\uparrow$} & \underline{\textbf{4.29 $\uparrow$}} & \textbf{4.04$\uparrow$}  \\ 
%Claude & 3.63 & 3.41 & 2.98 & 3.74 & 3.48 & 3.09\\
%Gemini 1.5  & 3.54 & 3.53 & 3.2 & 3.85 & 3.64 & 3.32
%\\
%\midrule
%\end{tabular}}
%\caption{Comparison of Fine-tuning Conditions Across Models}
%\label{tab:finetuning-comparison}
%\end{table*}




\begin{table}[t]
\centering
%\renewcommand{\arraystretch}{1} % Adjust row height
 \resizebox{\linewidth}{!}{
    \begin{tabular}{lccc}
    \toprule
    \textbf{Model} & \textbf{Correctness} & \textbf{Completeness} & \textbf{Fluency} \\ 
    %\multicolumn{3}{c}{\textbf{LLM Instruction Generation Analysis}} \\  
    %\cmidrule(lr){2-4} 
    \midrule
    Llama 3.1 (70B)  & 3.54 & 3.45 & 3.07 \\
    Claude & 3.74 & 3.48 & 3.09 \\
    Gemini 1.5  & 3.85 & 3.64 & 3.32 \\
    GPT-4o & \textbf{4.38} & {\textbf{4.29}} & \textbf{4.04}  \\ 
    % \midrule
    \hdashline
    MURI & 3.87 & 3.52 & 3.41\\
    \bottomrule
    \end{tabular} 
} % Close scalebox correctly
\caption{Human evaluation on LLM-generated instruction datasets.}


\label{tab:llm-instruction}
\end{table}



\begin{table}[t]
%\scriptsize
\centering
\renewcommand{\arraystretch}{1.2} % Adjust row height
 \resizebox{\linewidth}{!}{
\begin{tabular}{lrr}
\toprule
 & \textbf{CultSet} & \textbf{GovSet} \\ \midrule
Collected text & 4,400 & 1,376 \\ 
Avg. lengths (\#char) of collected text & 245 & 179\\
\cdashline{1-3} % Dashed line

Generated IFT pairs & 4,400 & 6,200 \\
Avg. lengths (\#char) of instruction & 85 & 76\\
% 10.21 & 8.64 \\
Avg. length (\#) of output & 453 & 215\\
% 56.60 & 26.36 \\
\# of unique tokens & 62,449 & 24,304\\
\bottomrule
\end{tabular}}
\caption{Overall statistics of GPT-4o generated IFT dataset.}
\label{tab:ds_stats}
\end{table}




\begin{table}[t]
\small
\centering
\resizebox{0.7\linewidth}{!}{
\renewcommand{\arraystretch}{1.2}
\begin{tabular}{lcc}
    \toprule
    \multirow{2}{*}{\textbf{Error Type}} & \multicolumn{2}{c}{\textbf{\% of Questions}} \\
    \cmidrule(lr){2-3}
    & \textbf{CultSet} & \textbf{GovSet} \\
    \midrule
    No error & 28.32\% & 19.47\% \\
    \hdashline
    Wrong language & 0.07\% & 0.14\% \\
    \textbf{Structural} & \textbf{28.45\%} & \textbf{33.58\%} \\
    Grammatical & 25.24\% & 28.73\% \\
    Lexical & 17.92\% & 18.08\% \\
    \bottomrule
\end{tabular}}
\caption{Distribution of error types in GPT-4o-generated IFT data from \texttt{CultSet} and \texttt{GovSet}, identified during manual post-editing.}
\label{tab:error_analysis}
\end{table}


\subsection{Manual Post-Editing}

Given GPT-4o’s strong performance, we use it for large-scale IFT data generation while ensuring quality through full human verification. We employ 12 expert annotators, all native Kazakh speakers with advanced degrees in World Languages, Literature, or Political Science from top Kazakhstani universities. Their extensive experience—having lived in Kazakhstan for over 25 years—equips them with the necessary linguistic and cultural expertise.

To maintain consistency, annotators received detailed guidelines outlining task objectives, evaluation criteria, and examples of high-quality IFT pairs (see Appendix \ref{app:annot-guide-appendix}). They were responsible for manually reviewing and correcting errors in the generated data. Before starting the main annotation process, all candidates completed a pilot task to assess their understanding of project requirements and their ability to refine IFT pairs accurately. Only those who met the evaluation criteria were selected. Each annotator's workload was equivalent to five full working days, and they were compensated fairly based on Kazakhstan’s monthly minimum wage. To accommodate flexibility, annotators were given up to one month to complete the task while working part-time.

Table~\ref{tab:error_analysis} summarizes the error types identified during manual post-editing of GPT-4o-generated data across the two document sources. Annotators found that \texttt{CultSet} had a higher proportion of "No error" cases (28.32\%) compared to \texttt{GovSet} (19.47\%), suggesting variations in data quality.

Structural errors were the most common in both datasets, accounting for over 28\% in \texttt{CultSet} and 33\% in \texttt{GovSet}. These errors involve grammatically correct but poorly structured responses, including issues with logical flow, organization, and unnatural phrasing for a Kazakh speaker. Additionally, grammatical and lexical errors were frequently observed, with annotators noting that GPT-4o occasionally replaces Kazakh words with Russian equivalents, even when the correct Kazakh term is explicitly provided in the original text. For a detailed breakdown of annotator observations, see Appendix~\ref{app:annot-comments}.

%Throughout the process, annotators were asked to document their findings as notes, focusing on patterns of errors they observed in the generated pairs. Annotators observe linguistic issues, inconsistent verb usage, unnatural phrasing, and reliance on Russian borrowings instead of existing Kazakh translations. Grammar errors, such as incorrect sentence structure and case application, are prevalent, along with formatting inconsistencies in names, punctuation, and noun linkage. For more details, in the key findings from their observations, please refer to Appendix~\ref{app:annot-comments}.

%Both CultSet and GovSet exhibit recurring linguistic issues, including inconsistent verb usage, unnatural phrasing, and reliance on Russian borrowings instead of existing Kazakh translations. Grammar errors, such as incorrect sentence structure and case application, are prevalent in both datasets. Additionally, formatting inconsistencies, particularly in names, punctuation, and noun linkage, affect the overall quality.
%However, some differences emerge. CultSet is notably affected by capitalization errors and inconsistent suffix application, whereas GovSet struggles more with active vs. passive constructions, improper word order, and inaccuracies in domain-specific terminology. Moreover, GovSet contains frequent issues with missing or redundant question particles, which are less represented in CultSet.
%Despite these variations, both datasets highlight the need for improved consistency in Kazakh text generation, particularly in reducing Russian borrowings and refining grammatical accuracy.

% \textbf{CultSet.}
% The generated IFT pairs exhibit several recurring issues, including inconsistent verb forms, incorrect use of grammar and sentence structure, inconsistent application of suffixes, capitalization errors, and reliance on Russian borrowings instead of existing Kazakh translations. Additionally, formatting inconsistencies in names, years, and noun linkage, as well as unnatural phrasing, were observed. 

% \textbf{GovSet.}
% The GovSet IFT pairs demonstrate several common issues, including inconsistent verb usage, incorrect case application, and improper use of active versus passive constructions. Errors in sentence phrasing, unnatural word order, and missing or redundant question particles were observed. Additionally, reliance on Russian words over their Kazakh equivalents and inaccuracies in translating specific terminologies were prevalent.
% Formatting inconsistencies in punctuation and improper differentiation between similar words further impacted the quality of the generated data. 


\subsection{Final Data Overview}




%which predominantly covers topics related to Kazakh literature, traditions, and media. It contains information on legal assistance in Kazakhstan, the healthcare system, laws related to real estate, and education  Additionally, it covers different aspects of Kazakh culture, including historical figures, academia, handicrafts, fashion, and more. 
As shown in Table~\ref{tab:ds_stats}, the final dataset consists of 4,400 \texttt{CultSet} and 6,200 \texttt{GovSet} IFT instances, totaling 10,600 high-quality samples. We split the dataset into 90\% training and 10\% test, where the training data is used for full fine-tuning of LLMs, and the test set is used for generation evaluation in our experiments.

Since both \texttt{CultSet} and \texttt{GovSet} are topic-based, we include their respective topics as metadata in the final IFT dataset (see Table~\ref{tab:category_exp_gov} and Table~\ref{tab:category_exp_culture} for topic definitions). Figure~\ref{fig:distribution-all} illustrates the topic distribution of the dataset. The most common topics in \texttt{CultSet} include Kazakh literature, traditions, and media, while \texttt{GovSet} primarily covers legal assistance, the healthcare system, real estate laws, and education in Kazakhstan. Examples of GPT-4o-generated IFT data can be found in Table~\ref{tab:example_govset} and Table~\ref{tab:example_cultset}.

Table~\ref{tab:ds_stats} further highlights a notable difference between the two subsets: the average output length in \texttt{CultSet} is significantly longer and contains more unique tokens than \texttt{GovSet}. This difference stems from the nature of \texttt{GovSet} responses, which are strictly factual and concise, whereas \texttt{CultSet} responses tend to be more diverse and expressive.
 


%\textbf{Appendix}
%The generated IFT pairs exhibit several recurring issues, including inconsistent verb forms (e.g., \foreignlanguage{russian}{\textit{түсіндір}} instead of \foreignlanguage{russian}{\textit{түсіндіріңіз}}) and incorrect use of \foreignlanguage{russian}{\textit{туралы}} in instructions. Phrases such as "\foreignlanguage{russian}{\textit{... қалай сипатталады?}}" imply reliance on external text, requiring adjustment. Russian borrowings (e.g., \foreignlanguage{russian}{\textit{награда}}, \foreignlanguage{russian}{\textit{карьера}}, \foreignlanguage{russian}{\textit{музей}}) are sometimes used despite existing Kazakh translations. Words like \foreignlanguage{russian}{\textit{әртүрлі}} and suffixes like \foreignlanguage{russian}{\textit{-мен/-пен}} are inconsistently applied, and shortened names lack spacing (e.g., \foreignlanguage{russian}{\textit{Ы.Алтынсарин}}). Errors in year formatting (e.g., \foreignlanguage{russian}{\textit{1973-78}}), noun capitalization (e.g., \foreignlanguage{russian}{\textit{ру}}), and verb usage (e.g., improper linkage with \foreignlanguage{russian}{\textit{-ып/-іп}}) persist. Additionally, phrases like \foreignlanguage{russian}{\textit{рөл ойнады}} reflect direct translations from Russian and require correction to \textit{\foreignlanguage{russian}{рөл сомдады}}. The translator overuses \foreignlanguage{russian}{\textit{ол}} and occasionally writes Latin-alphabet words. Missteps such as \foreignlanguage{russian}{\textit{пайдаланады деп түсіндіріңіз}} instead of \foreignlanguage{russian}{\textit{пайдаланатынын түсіндіріңіз}} and confusion with \foreignlanguage{russian}{\textit{жылдары}} formatting remain common. While generally faithful to the original text, these issues necessitate ongoing revision for consistency and grammatical accuracy.


\begin{figure*}[h!]
    \centering
    \begin{minipage}[b]{0.45\textwidth}
        %\centering
        \includegraphics[scale=0.35]{images/wikipedia_topics_distribution.png}
        \caption*{(a) CultSet}
    \end{minipage}
    \hfill
    \begin{minipage}[b]{0.45\textwidth}
        \centering
        \includegraphics[scale=0.35]{images/category_gov_barchart.png}
        \caption*{(b) GovSet}
    \end{minipage}
    \caption{Topic distribution of GPT-4 generated IFT dataset in \texttt{CultSet} and \texttt{GovSet}.}
    \label{fig:distribution-all}
\end{figure*}

%\section{Research Methodology}~\label{sec:Methodology}

In this section, we discuss the process of conducting our systematic review, e.g., our search strategy for data extraction of relevant studies, based on the guidelines of Kitchenham et al.~\cite{kitchenham2022segress} to conduct SLRs and Petersen et al.~\cite{PETERSEN20151} to conduct systematic mapping studies (SMSs) in Software Engineering. In this systematic review, we divide our work into a four-stage procedure, including planning, conducting, building a taxonomy, and reporting the review, illustrated in Fig.~\ref{fig:search}. The four stages are as follows: (1) the \emph{planning} stage involved identifying research questions (RQs) and specifying the detailed research plan for the study; (2) the \emph{conducting} stage involved analyzing and synthesizing the existing primary studies to answer the research questions; (3) the \emph{taxonomy} stage was introduced to optimize the data extraction results and consolidate a taxonomy schema for REDAST methodology; (4) the \emph{reporting} stage involved the reviewing, concluding and reporting the final result of our study.

\begin{figure}[!t]
    \centering
    \includegraphics[width=1\linewidth]{fig/methodology/searching-process.drawio.pdf}
    \caption{Systematic Literature Review Process}
    \label{fig:search}
\end{figure}

\subsection{Research Questions}
In this study, we developed five research questions (RQs) to identify the input and output, analyze technologies, evaluate metrics, identify challenges, and identify potential opportunities. 

\textbf{RQ1. What are the input configurations, formats, and notations used in the requirements in requirements-driven
automated software testing?} In requirements-driven testing, the input is some form of requirements specification -- which can vary significantly. RQ1 maps the input for REDAST and reports on the comparison among different formats for requirements specification.

\textbf{RQ2. What are the frameworks, tools, processing methods, and transformation techniques used in requirements-driven automated software testing studies?} RQ2 explores the technical solutions from requirements to generated artifacts, e.g., rule-based transformation applying natural language processing (NLP) pipelines and deep learning (DL) techniques, where we additionally discuss the potential intermediate representation and additional input for the transformation process.

\textbf{RQ3. What are the test formats and coverage criteria used in the requirements-driven automated software
testing process?} RQ3 focuses on identifying the formulation of generated artifacts (i.e., the final output). We map the adopted test formats and analyze their characteristics in the REDAST process.

\textbf{RQ4. How do existing studies evaluate the generated test artifacts in the requirements-driven automated software testing process?} RQ4 identifies the evaluation datasets, metrics, and case study methodologies in the selected papers. This aims to understand how researchers assess the effectiveness, accuracy, and practical applicability of the generated test artifacts.

\textbf{RQ5. What are the limitations and challenges of existing requirements-driven automated software testing methods in the current era?} RQ5 addresses the limitations and challenges of existing studies while exploring future directions in the current era of technology development. %It particularly highlights the potential benefits of advanced LLMs and examines their capacity to meet the high expectations placed on these cutting-edge language modeling technologies. %\textcolor{blue}{CA: Do we really need to focus on LLMs? TBD.} \textcolor{orange}{FW: About LLMs, I removed the direct emphase in RQ5 but kept the discussion in RQ5 and the solution section. I think that would be more appropriate.}

\subsection{Searching Strategy}

The overview of the search process is exhibited in Fig. \ref{fig:papers}, which includes all the details of our search steps.
\begin{table}[!ht]
\caption{List of Search Terms}
\label{table:search_term}
\begin{tabularx}{\textwidth}{lX}
\hline
\textbf{Terms Group} & \textbf{Terms} \\ \hline
Test Group & test* \\
Requirement Group & requirement* OR use case* OR user stor* OR specification* \\
Software Group & software* OR system* \\
Method Group & generat* OR deriv* OR map* OR creat* OR extract* OR design* OR priorit* OR construct* OR transform* \\ \hline
\end{tabularx}
\end{table}

\begin{figure}
    \centering
    \includegraphics[width=1\linewidth]{fig/methodology/search-papers.drawio.pdf}
    \caption{Study Search Process}
    \label{fig:papers}
\end{figure}

\subsubsection{Search String Formulation}
Our research questions (RQs) guided the identification of the main search terms. We designed our search string with generic keywords to avoid missing out on any related papers, where four groups of search terms are included, namely ``test group'', ``requirement group'', ``software group'', and ``method group''. In order to capture all the expressions of the search terms, we use wildcards to match the appendix of the word, e.g., ``test*'' can capture ``testing'', ``tests'' and so on. The search terms are listed in Table~\ref{table:search_term}, decided after iterative discussion and refinement among all the authors. As a result, we finally formed the search string as follows:


\hangindent=1.5em
 \textbf{ON ABSTRACT} ((``test*'') \textbf{AND} (``requirement*'' \textbf{OR} ``use case*'' \textbf{OR} ``user stor*'' \textbf{OR} ``specifications'') \textbf{AND} (``software*'' \textbf{OR} ``system*'') \textbf{AND} (``generat*'' \textbf{OR} ``deriv*'' \textbf{OR} ``map*'' \textbf{OR} ``creat*'' \textbf{OR} ``extract*'' \textbf{OR} ``design*'' \textbf{OR} ``priorit*'' \textbf{OR} ``construct*'' \textbf{OR} ``transform*''))

The search process was conducted in September 2024, and therefore, the search results reflect studies available up to that date. We conducted the search process on six online databases: IEEE Xplore, ACM Digital Library, Wiley, Scopus, Web of Science, and Science Direct. However, some databases were incompatible with our default search string in the following situations: (1) unsupported for searching within abstract, such as Scopus, and (2) limited search terms, such as ScienceDirect. Here, for (1) situation, we searched within the title, keyword, and abstract, and for (2) situation, we separately executed the search and removed the duplicate papers in the merging process. 

\subsubsection{Automated Searching and Duplicate Removal}
We used advanced search to execute our search string within our selected databases, following our designed selection criteria in Table \ref{table:selection}. The first search returned 27,333 papers. Specifically for the duplicate removal, we used a Python script to remove (1) overlapped search results among multiple databases and (2) conference or workshop papers, also found with the same title and authors in the other journals. After duplicate removal, we obtained 21,652 papers for further filtering.

\begin{table*}[]
\caption{Selection Criteria}
\label{table:selection}
\begin{tabularx}{\textwidth}{lX}
\hline
\textbf{Criterion ID} & \textbf{Criterion Description} \\ \hline
S01          & Papers written in English. \\
S02-1        & Papers in the subjects of "Computer Science" or "Software Engineering". \\
S02-2        & Papers published on software testing-related issues. \\
S03          & Papers published from 1991 to the present. \\ 
S04          & Papers with accessible full text. \\ \hline
\end{tabularx}
\end{table*}

\begin{table*}[]
\small
\caption{Inclusion and Exclusion Criteria}
\label{table:criteria}
\begin{tabularx}{\textwidth}{lX}
\hline
\textbf{ID}  & \textbf{Description} \\ \hline
\multicolumn{2}{l}{\textbf{Inclusion Criteria}} \\ \hline
I01 & Papers about requirements-driven automated system testing or acceptance testing generation, or studies that generate system-testing-related artifacts. \\
I02 & Peer-reviewed studies that have been used in academia with references from literature. \\ \hline
\multicolumn{2}{l}{\textbf{Exclusion Criteria}} \\ \hline
E01 & Studies that only support automated code generation, but not test-artifact generation. \\
E02 & Studies that do not use requirements-related information as an input. \\
E03 & Papers with fewer than 5 pages (1-4 pages). \\
E04 & Non-primary studies (secondary or tertiary studies). \\
E05 & Vision papers and grey literature (unpublished work), books (chapters), posters, discussions, opinions, keynotes, magazine articles, experience, and comparison papers. \\ \hline
\end{tabularx}
\end{table*}

\subsubsection{Filtering Process}

In this step, we filtered a total of 21,652 papers using the inclusion and exclusion criteria outlined in Table \ref{table:criteria}. This process was primarily carried out by the first and second authors. Our criteria are structured at different levels, facilitating a multi-step filtering process. This approach involves applying various criteria in three distinct phases. We employed a cross-verification method involving (1) the first and second authors and (2) the other authors. Initially, the filtering was conducted separately by the first and second authors. After cross-verifying their results, the results were then reviewed and discussed further by the other authors for final decision-making. We widely adopted this verification strategy within the filtering stages. During the filtering process, we managed our paper list using a BibTeX file and categorized the papers with color-coding through BibTeX management software\footnote{\url{https://bibdesk.sourceforge.io/}}, i.e., “red” for irrelevant papers, “yellow” for potentially relevant papers, and “blue” for relevant papers. This color-coding system facilitated the organization and review of papers according to their relevance.

The screening process is shown below,
\begin{itemize}
    \item \textbf{1st-round Filtering} was based on the title and abstract, using the criteria I01 and E01. At this stage, the number of papers was reduced from 21,652 to 9,071.
    \item \textbf{2nd-round Filtering}. We attempted to include requirements-related papers based on E02 on the title and abstract level, which resulted from 9,071 to 4,071 papers. We excluded all the papers that did not focus on requirements-related information as an input or only mentioned the term ``requirements'' but did not refer to the requirements specification.
    \item \textbf{3rd-round Filtering}. We selectively reviewed the content of papers identified as potentially relevant to requirements-driven automated test generation. This process resulted in 162 papers for further analysis.
\end{itemize}
Note that, especially for third-round filtering, we aimed to include as many relevant papers as possible, even borderline cases, according to our criteria. The results were then discussed iteratively among all the authors to reach a consensus.

\subsubsection{Snowballing}

Snowballing is necessary for identifying papers that may have been missed during the automated search. Following the guidelines by Wohlin~\cite{wohlin2014guidelines}, we conducted both forward and backward snowballing. As a result, we identified 24 additional papers through this process.

\subsubsection{Data Extraction}

Based on the formulated research questions (RQs), we designed 38 data extraction questions\footnote{\url{https://drive.google.com/file/d/1yjy-59Juu9L3WHaOPu-XQo-j-HHGTbx_/view?usp=sharing}} and created a Google Form to collect the required information from the relevant papers. The questions included 30 short-answer questions, six checkbox questions, and two selection questions. The data extraction was organized into five sections: (1) basic information: fundamental details such as title, author, venue, etc.; (2) open information: insights on motivation, limitations, challenges, etc.; (3) requirements: requirements format, notation, and related aspects; (4) methodology: details, including immediate representation and technique support; (5) test-related information: test format(s), coverage, and related elements. Similar to the filtering process, the first and second authors conducted the data extraction and then forwarded the results to the other authors to initiate the review meeting.

\subsubsection{Quality Assessment}

During the data extraction process, we encountered papers with insufficient information. To address this, we conducted a quality assessment in parallel to ensure the relevance of the papers to our objectives. This approach, also adopted in previous secondary studies~\cite{shamsujjoha2021developing, naveed2024model}, involved designing a set of assessment questions based on guidelines by Kitchenham et al.~\cite{kitchenham2022segress}. The quality assessment questions in our study are shown below:
\begin{itemize}
    \item \textbf{QA1}. Does this study clearly state \emph{how} requirements drive automated test generation?
    \item \textbf{QA2}. Does this study clearly state the \emph{aim} of REDAST?
    \item \textbf{QA3}. Does this study enable \emph{automation} in test generation?
    \item \textbf{QA4}. Does this study demonstrate the usability of the method from the perspective of methodology explanation, discussion, case examples, and experiments?
\end{itemize}
QA4 originates from an open perspective in the review process, where we focused on evaluation, discussion, and explanation. Our review also examined the study’s overall structure, including the methodology description, case studies, experiments, and analyses. The detailed results of the quality assessment are provided in the Appendix. Following this assessment, the final data extraction was based on 156 papers.

% \begin{table}[]
% \begin{tabular}{ll}
% \hline
% QA ID & QA Questions                                             \\ \hline
% Q01   & Does this study clearly state its aims?                  \\
% Q02   & Does this study clearly describe its methodology?        \\
% Q03   & Does this study involve automated test generation?       \\
% Q04   & Does this study include a promising evaluation?          \\
% Q05   & Does this study demonstrate the usability of the method? \\ \hline
% \end{tabular}%
% \caption{Questions for Quality Assessment}
% \label{table:qa}
% \end{table}

% automated quality assessment

% \textcolor{blue}{CA: Our search strategy focused on identifying requirements types first. We covered several sources, e.g., ~\cite{Pohl:11,wagner2019status} to identify different formats and notations of specifying requirements. However, this came out to be a long list, e.g., free-form NL requirements, semi-formal UML models, free-from textual use case models, UML class diagrams, UML activity diagrams, and so on. In this paper, we attempted to primarily focus on requirements-related aspects and not design-level information. Hence, we generalised our search string to include generic keywords, e.g., requirement*, use case*, and user stor*. We did so to avoid missing out on any papers, bringing too restrictive in our search strategy, and not creating a too-generic search string with all the aforementioned formats to avoid getting results beyond our review's scope.}


%% Use \subsection commands to start a subsection.



%\subsection{Study Selection}

% In this step, we further looked into the content of searched papers using our search strategy and applied our inclusion and exclusion criteria. Our filtering strategy aimed to pinpoint studies focused on requirements-driven system-level testing. Recognizing the presence of irrelevant papers in our search results, we established detailed selection criteria for preliminary inclusion and exclusion, as shown in Table \ref{table: criteria}. Specifically, we further developed the taxonomy schema to exclude two types of studies that did not meet the requirements for system-level testing: (1) studies supporting specification-driven test generation, such as UML-driven test generation, rather than requirements-driven testing, and (2) studies focusing on code-based test generation, such as requirement-driven code generation for unit testing.




%\section{Dataset analysis}
%\subsection{Category Distribution}

% \begin{figure}[ht!]
%     \centering
%     \includegraphics[width=0.95\linewidth]{images/category_gov_barchart.png}
%     \caption{Frequency distribution of instruction categories within the governmental data.}
%     \label{fig:category_gov_barchart}
% \end{figure}

%There are 23 categories within the governmental data, distributed based on their frequency of occurrence (Figure~\ref{fig:category_gov_barchart}).. "Legal assistance" is the most frequent category, appearing nearly 700 times, followed by "Healthcare," "Real estate," and "Education." In contrast, categories such as "Military registration," "Population," and "Religion" are the least frequent, with fewer than 100 occurrences each.

% \begin{figure}[ht!]
%     \centering
%     \includegraphics[width=0.95\linewidth]{images/wikipedia_topics_distribution.png}
%     \caption{Frequency distribution of instruction categories within the cultural data.}
%     \label{fig:category_wiki_barchart}
% \end{figure}



% \begin{figure}[ht!]
%     \centering
%     \includegraphics[width=0.95\linewidth]{images/scatter_plot_gov.png}
%     \caption{Instruction-output length for governmental data.}
%     \label{fig:scatter_plot_gov}
% \end{figure}

% \begin{figure}[ht!]
%     \centering
%     \includegraphics[width=0.95\linewidth]{images/scatter_plot_wiki.png}
%     \caption{Instruction-output length for cultural data.}
%     \label{fig:scatter_plot_wiki}
% \end{figure}



% \begin{figure*}[h!] % Start a figure environment that spans across both columns
%     \centering % Center the entire figure
%     \begin{minipage}[b]{0.45\textwidth} % First subfigure occupies 45% of the total text width
%         % \centering % Uncomment this if you want to center the image within this subfigure
%         \includegraphics[scale=0.38]{images/scatter_plot_wiki.png} % Includes the first scatter plot (CultSet)
%         \caption*{(a) CultSet} % Adds an unnumbered caption for the first plot
%     \end{minipage} % End of the first minipage
%     \hfill % Adds horizontal space between the two minipages
%     \begin{minipage}[b]{0.45\textwidth} % Second subfigure occupies 45% of the total text width
%         \centering % Centers the image within this subfigure
%         \includegraphics[scale=0.38]{images/scatter_plot_gov.png} % Includes the second scatter plot (GovSet)
%         \caption*{(b) GovSet} % Adds an unnumbered caption for the second plot
%     \end{minipage} % End of the second minipage
%     \caption{Instruction-output length for (a) CultSet and (b) GovSet} % Main caption for the entire figure
%     \label{fig:distribution-all} % Adds a label for referencing the figure in text
% \end{figure*} % End of the figure* environment

\begin{figure*}[ht!]
    \centering
    \begin{minipage}[b]{0.45\textwidth}
        \centering
        \includegraphics[scale=0.4]{images/new_gen_eval_pref.png}
        \caption*{(a) CultSet}
    \end{minipage}
    \hfill
    \begin{minipage}[b]{0.45\textwidth}
        \centering
        \includegraphics[scale=0.4]{images/new_gen_eval_gov.png}

        \caption*{(b) GovSet}
    \end{minipage}
    \caption{Distribution of preferences for (a) \texttt{CultSet} and (b) \texttt{GovSet} datasets across models. The charts illustrate the percentage of 'Tie', 'Vanilla', and 'After Fine-Tuning' preferences in each dataset.}
    \label{fig:human_fg_1000}
\end{figure*}
\section{Experiments}

We conducted two experiments: multiple-choice questions (MCQ) and text generations evaluation. We will detail each of the evaluation in the following sections.

\paragraph{Model Selection} For both MCQ and generation evaluations, we use three models: Gemma-2-9B (Gemma) \cite{gemma2}, Qwen-2.5-7B (Qwen)\cite{qwen2.5}, and Falcon-3-10B (Falcon)~\cite{Falcon3}. While these LLMs have multilingual capabilities, they were not specifically trained for Kazakh, allowing us to assume that our IFT data is novel to them. 

\paragraph{Fine-tuning} We performed full fine-tuning on Gemma-2-9b, Qwen-2.5-7b, and Falcon-3-10b using the AdamW optimizer with hyperparameters $\beta_1 = 0.9$, $\beta_2 = 0.95$, $\epsilon = 1e{-5}$, and a weight decay of 0.1. We scaled the gradient norms using a maximum norm clipping value of 1.0. The learning rate was kept constant throughout the fine-tuning without any warm-up or decay with a value of $1e{-6}$ for Gemma-2 and Falcon-3, and $1e{-5}$ for Qwen-2.5. The batch size used was 16, and we packed multiple documents until the maximum sequence length was 8,192 tokens. Cross-document attention is disabled by modifying attention masks so the tokens of a document only attend to the tokens from the same document in a causal way. No adjustment were made to the original tokenizer for each model.

\paragraph{Baseline} As a baseline, we include the Kazakh Alpaca dataset,\footnote{\url{https://huggingface.co/datasets/AmanMussa/kazakh-instruction-v2}} which has been translated and localized into Kazakh. For each model, we conduct full fine-tuning with (1) our training dataset, (2) Alpaca, and (3) a combination of Alpaca and our training dataset.


%Additionally, we applied the respective chat templates for all the models and computed the loss only over response/assistant tokens.

%We tried to compare the models in the same settings, so no adjustments were made to the original tokenizers, despite of the impact of language-specific tokenizers on the overall performance~\cite{nikolich-etal-2024-vikhr}.
% S GOES TO LIMITATION OR DELETE?
% There has been significant research on improving tokenization for non-English languages, as seen in the Vikhr study \cite{nikolich-etal-2024-vikhr}, where researchers addressed the inefficiencies of English-oriented tokenizers by reconstructing a language-specific tokenizer tailored to Russian, leveraging a dedicated corpus. This adaptation significantly improved model efficiency and performance during instruction tuning, showcasing how tokenization can be a critical factor in fine-tuning language models for specific linguistic contexts. While such approaches are ideal for enhancing instruction-following capabilities, we chose not to focus on a language-specific tokenizer to ensure a fair comparison with existing models that rely on standard tokenization frameworks. By using pre-existing tokenizers, we aim to isolate the impact of instruction tuning and dataset quality without introducing additional variables related to tokenizer customization.

% \subsubsection{Human evaluation}
% To evaluate the model, we sampled 400 instruction-response pairs and assessed them across three specific criteria: correctness, fluency, and ambiguity. Each criterion was carefully designed to capture distinct aspects of the model's performance and usability. 
% \par
% This setup ensures a holistic understanding of the model's performance. In Appendix Table ~\ref{tab:multilingual-issues}, we provide detailed examples of human evaluation, including the assigned scores and corresponding comments that explain the rationale behind each score. These examples illustrate how correctness, fluency, and ambiguity were assessed, highlighting specific issues such as factual inaccuracies, grammatical errors, and lack of clarity. The provided annotations offer a clear understanding of how the evaluation criteria were applied to real instruction-response pairs.
% \subsubsection{LLM as a Judge}
% To ensure consistent, scalable, and unbiased evaluations, we employed GPT-4o as an automated judge to assess the responses of LLaMA 3.1 - 8B and Qwen-2.5 - 7B models.The evaluation compared the models' outputs under two conditions: without fine-tuning and after fine-tuning on our dataset. GPT-4o was tasked with evaluating the generated responses across three criteria: correctness, ambiguity, and fluency.

% In this setup, the model-generated responses were assessed in a blind manner to ensure unbiased comparisons. For each instruction-response pair, GPT-4o provided scores for each criterion, enabling us to quantitatively evaluate the impact of fine-tuning on the models' performance. This automated approach allowed for consistent and scalable evaluation while providing valuable insights into how fine-tuning enhances the models' ability to produce accurate, coherent, and fluent responses tailored to culturally nuanced instructions. The evaluation process utilized the following prompt to guide GPT-4o in assessing the quality of the generated responses:

% \begin{quote}
% \textit{
% Evaluate the following response based on the given instruction and ground truth using the following criteria:
% \begin{enumerate}
%     \item \textbf{Correctness (0/1):} Does the response accurately answer the instruction based on the ground truth?
%     \item \textbf{Fluency (1-5):} How well is the response written in terms of grammar, clarity, and coherence?
%     \item \textbf{Ambiguity (0/1):} Is the response free from unclear or vague information?
% \end{enumerate} 
% Provide your scores for each criterion and a brief explanation for your evaluation.
% \textbf{Instruction:} [Insert Instruction] \\
% \textbf{Ground Truth:} [Insert Ground Truth] \\
% \textbf{Generated Response:} [Insert Generated Response]}
% \end{quote}

% \\
% \par
% The final results of the two evaluation methods, human evaluation and LLM evaluation, are presented in Table~\ref{tab:finetuning-comparison}. This table highlights the performance of LLaMA-3.1 - 8B and Qwen-2.5 - 7B under two conditions: without fine-tuning and after fine-tuning on our dataset. The evaluation criteria—correctness, ambiguity, and fluency—demonstrate the impact of fine-tuning on improving the models' ability to generate accurate, coherent, and fluent responses. Both human judges and GPT-4o consistently indicate improvements across these metrics, with LLaMA showing stronger overall performance compared to Qwen, particularly in correctness and fluency. We computed the Pearson correlation coefficient to assess the alignment between human evaluation and GPT-4o's automated evaluation. The results indicate a high correlation for fluency (0.85) and correctness (0.78), while ambiguity exhibited moderate alignment (0.72). These findings demonstrate that GPT-4o is largely consistent with human judgments, validating its use as a reliable automated evaluation framework.

% \begin{table*}[ht!]
% \centering
% \renewcommand{\arraystretch}{1} % Adjust row height
% \scalebox{0.7}{
% \begin{tabular}{lcccccccc}
% \toprule
% \textbf{Model name} & \multicolumn{4}{c}{\textbf{Dastur}} & \multicolumn{4}{c}{\textbf{Constitution}} \\  
% \cmidrule(lr){2-5} \cmidrule(lr){6-9}
% & \textbf{No Fine-Tuning} & \textbf{Alpaca} & \textbf{Ours} & \textbf{Alpaca + Ours} 
% & \textbf{No Fine-Tuning} & \textbf{Alpaca} & \textbf{Ours} & \textbf{Alpaca + Ours}  \\
% \midrule
% Gemma 2 - 9b & 0.498 & 0.5133 & 0.543 & \textbf{0.566 $\uparrow$}\\
% Qwen 2.5 - 7b - 9B & 0.403 & 0.421 & 0.443 &\textbf{0.465 $\uparrow$}\\
% Falcon 3 - 10b & 0.45 & 0.458 & 0.464 & \textbf{0.4718 $\uparrow$}
% % & \textbf{0.47 $\uparrow$} & \textbf{0.69 $\downarrow$} & \textbf{3.23 $\uparrow$} 
% % & \textbf{0.53 $\uparrow$} & \textbf{0.62 $\downarrow$}  \\ 
% % Qwen 2.5 - 7B\\
% \\
% \midrule
% \end{tabular}}
% \caption{Comparison of Fine-tuning Conditions Across Models}
% \label{tab:finetuning-comparison}
% \end{table*}


% \begin{table*}[ht!]
% \centering
% \renewcommand{\arraystretch}{1.9} % Adjust row height
% \scalebox{0.6}{
% \begin{tabular}{lcccccccccccc}
% \toprule
% \textbf{Model name} & \multicolumn{4}{c}{\textbf{Dastur}} & \multicolumn{4}{c}{\textbf{Constitution}} & \multicolumn{4}{c}{\textbf{Human Rights}} \\  
% \cmidrule(lr){2-5} \cmidrule(lr){6-9} \cmidrule(lr){10-13}
% & \textbf{No Fine-Tuning} & \textbf{Alpaca} & \textbf{Ours} & \textbf{Alpaca + Ours} 
% & \textbf{No Fine-Tuning} & \textbf{Alpaca} & \textbf{Ours} & \textbf{Alpaca + Ours}  
% & \textbf{No Fine-Tuning} & \textbf{Alpaca} & \textbf{Ours} & \textbf{Alpaca + Ours}  \\
% \midrule
% Gemma 2 - 9b & 0.498 & 0.513 & 0.543 & \textbf{0.566 $\uparrow$} 
%              & 0.600 & 0.627 & 0.640 & \textbf{0.650 $\uparrow$} 
%              & 0.405 & 0.430 & 0.465 & \textbf{0.480 $\uparrow$} \\

% Qwen 2.5 - 7b - 9B & 0.403 & 0.421 & 0.443 &\textbf{0.465 $\uparrow$} 
%                    & 0.520 & 0.609 & 0.670 & \textbf{0.680 $\uparrow$} 
%                    & 0.300 & 0.330 & 0.365 & \textbf{0.375$\uparrow$} \\

% Falcon 3 - 10b & 0.450 & 0.458 & 0.464 & \textbf{0.4718 $\uparrow$}  
%                & 0.430 & 0.450 & 0.490 & \textbf{0.520 $\uparrow$}  
%                & 0.215 & 0.234 & 0.250 & \textbf{0.275 $\uparrow$} \\

% \midrule
% \end{tabular}}
% \caption{Comparison of Fine-tuning Conditions Across Models for Dastur, Constitution, and Rights}
% \label{tab:finetuning-comparison}
% \end{table*}
% \begin{table*}[ht!]
% \centering
% \renewcommand{\arraystretch}{1.9} % Adjust row height
% \scalebox{0.5}{
% \begin{tabular}{lcccccccccccccccc}
% \toprule
% \textbf{Model name} & \multicolumn{5}{c}{\textbf{Dastur}} & \multicolumn{5}{c}{\textbf{Constitution}} & \multicolumn{5}{c}{\textbf{Human Rights}} \\  
% \cmidrule(lr){2-6} \cmidrule(lr){7-11} \cmidrule(lr){12-16}
% & \textbf{No Fine-Tuning} & \textbf{Alpaca} & \textbf{Ours} & \textbf{Alpaca + Ours} & \textbf{RAG}  
% & \textbf{No Fine-Tuning} & \textbf{Alpaca} & \textbf{Ours} & \textbf{Alpaca + Ours} & \textbf{RAG}  
% & \textbf{No Fine-Tuning} & \textbf{Alpaca} & \textbf{Ours} & \textbf{Alpaca + Ours} & \textbf{RAG}  \\
% \midrule
% Gemma 2 - 9b & 0.498 & 0.513 & 0.543 & \textbf{0.566 } & \textit{0.533}
%              & 0.600 & 0.627 & 0.640 & \textbf{0.650 } &\textit{0.655} 
%              & 0.405 & 0.430 & 0.465 & \textbf{0.480 } & \textit{0.450}  \\
%
% Qwen 2.5 - 7b - 9B & 0.403 & 0.421 & 0.443 &\textbf{0.465 } & \textit{0.410}  
%                    & 0.520 & 0.609 & 0.670 & \textbf{0.680 $\uparrow$} & \textit{0.523}  
%                    & 0.300 & 0.330 & 0.365 & \textbf{0.375} & \textit{0.325} \\
%
% Falcon 3 - 10b & 0.450 & 0.458 & 0.464 & \textbf{0.4718 }  &  \textit{0.460} 
%                & 0.430 & 0.450 & 0.490 & \textbf{0.520 $\uparrow$}  & \textit{0.3864}  
%                & 0.215 & 0.234 & 0.250 & \textbf{0.275 }  & \textit{0.220} \\
%
% \midrule
% \end{tabular}}
% \caption{Comparison of Fine-tuning Conditions Across Models for Dastur, Constitution, and Rights}
% \label{tab:finetuning-comparison}
% \end{table*}

\begin{table}[t!]
\centering
\renewcommand{\arraystretch}{1.2} % Adjust row height for compactness
\resizebox{\columnwidth}{!}{ % Automatically scales the table to fit within column width
\begin{tabular}{lccccc}
\toprule
\textbf{Model} & \textbf{Vanilla} & \textbf{RAG} & \textbf{Alpaca} & \textbf{Ours} & \textbf{Alpaca + Ours}  \\  
\midrule
\multicolumn{6}{l}{\cellcolor{blue!7}\textbf{Dastur}} \\
Gemma & 0.498 & {0.533} & 0.513 & 0.543 & \textbf{0.566}  \\

Qwen & 0.403 & {0.410} & 0.421 & 0.443 &\textbf{0.465 }  \\

Falcon & 0.450 & {0.460} & 0.458 & 0.464 & \textbf{0.471 }   \\
\midrule
\multicolumn{6}{l}{\cellcolor{blue!7}\textbf{Constitution}} \\
Gemma & 0.600 & {\textbf{0.655}} & 0.627 & 0.640 & {0.650 }  \\

Qwen & 0.520 & {0.523} & 0.609 & 0.670 & \textbf{0.680 }  \\

Falcon & 0.430 & {0.386} & 0.450 & 0.490 & \textbf{0.520 }  \\
\midrule
\multicolumn{6}{l}{\cellcolor{blue!7}\textbf{Human Rights and Society}} \\
Gemma & 0.405 & 0.450 & 0.430 & 0.465 & \textbf{0.480 }  \\

Qwen & 0.300 & 0.325 & 0.330 & 0.365 & \textbf{0.375}  \\

Falcon & 0.215 & 0.220 & 0.234 & 0.250 & \textbf{0.275 }  \\
\bottomrule
\end{tabular}}
\caption{Zero-shot accuracies of language models in different datasets: (1) Datasur, (2) Constitution, and (3) Human Rights and Society}
\label{tab:result_mcq}
\end{table}


% \begin{table}[t!]
% \centering
% \renewcommand{\arraystretch}{1.2} % Adjust row height for compactness
% \resizebox{\columnwidth}{!}{ % Automatically scales the table to fit within column width
% \begin{tabular}{lccccc}
% \toprule
% \textbf{Model} & \textbf{Vanilla} & \textbf{RAG} & \textbf{Alpaca} & \textbf{Ours} & \textbf{Alpaca + Ours}  \\  
% \midrule
% Gemma & 0.600 & {0.655} & 0.627 & 0.640 & \textbf{0.650 }  \\

% Qwen & 0.520 & {0.523} & 0.609 & 0.670 & \textbf{0.680 }  \\

% Falcon & 0.430 & {0.386} & 0.450 & 0.490 & \textbf{0.520 }  \\

% \midrule
% \end{tabular}}
% \caption{Comparison of fine-tuning conditions across models for Constitution.}
% \label{tab:finetuning-constitution}
% \end{table}


% \begin{table}[t!]
% \centering
% \renewcommand{\arraystretch}{1.2} % Adjust row height for compactness
% \resizebox{\columnwidth}{!}{ % Automatically scales the table to fit within column width
% \begin{tabular}{lccccc}
% \toprule
% \textbf{Model} & \textbf{No Fine-Tuning} & \textbf{RAG} & \textbf{Alpaca} & \textbf{Ours} & \textbf{Alpaca + Ours}  \\  
% \midrule
% Gemma & 0.405 & 0.450 & 0.430 & 0.465 & \textbf{0.480 }  \\

% Qwen & 0.300 & 0.325 & 0.330 & 0.365 & \textbf{0.375}  \\

% Falcon & 0.215 & 0.220 & 0.234 & 0.250 & \textbf{0.275 }  \\

% \midrule
% \end{tabular}}
% \caption{Comparison of fine-tuning conditions across models for Human Rights and the Society.}
% \label{tab:finetuning-humanrights}
% \end{table}


\subsection{Multiple-choice Question Evaluation}
\label{sec:mcq}

\paragraph{Dataset}
A dedicated open-source Kazakh NLP community\footnote{\url{https://huggingface.co/kz-transformers}} has collaboratively developed and crowd-sourced multiple hand-crafted benchmarks to assess the factual knowledge of LLMs in Kazakh.  We use three multiple-choice question (MCQ) datasets: (1) Dastur-MC~\cite{horde_dastur_kk2024}, which evaluates knowledge of Kazakh traditions, (2) Kazakh Constitution-MC~\cite{horde_constitution_kk2024}, which focuses on Kazakhstan’s legal system, and (3) Kazakh Unified National~\cite{horde_unt_kk2024}, which assesses citizen rights, legal protections, and societal knowledge (referred to as the "Human Rights and Society" dataset).\footnote{Examples of test questions are provided in Appendix \ref{sec:mcq_samples}.}

Each dataset consists of multiple-choice questions with four answer options, only one of which is correct. We selected these evaluation benchmarks because they align with the focus of our instruction fine-tuning dataset and are not derived from our document sources (\texttt{CultSet} and \texttt{GovSet}). These datasets cover culturally significant topics, legal frameworks, and citizen-government interactions, reflecting real-world applications that our fine-tuned models aim to support.

Since no documented quality assurance process was available for the three datasets, we conducted a manual verification to ensure the accuracy of the questions. To maintain a fair and valid comparison, only the manually verified samples were used in our evaluation. For the Dastur-MC dataset, we randomly sampled 300 questions and manually verified their correctness. The same process was applied to the Kazakh Constitution-MC and Human Rights and Society datasets, with 200 randomly selected questions from each.

\paragraph{Setup}
In addition to the fine-tuned models, we include retrieval-augmented generation (RAG) without fine-tuning to estimate the upper bound of the original models' performance. For RAG, we use BM25 encoding, as no specialized Kazakh retrieval encoder is available. For each question, we retrieve the top two matching text chunks (each 256 symbols long) from the training texts of our IFT corpus and provide them as additional context.

To assess model capability, we use the \texttt{LM Eval Harness}~\cite{eval-harness} framework in a zero-shot setting. During evaluation, the answer is selected based on the alphabetical option with the highest likelihood.

%We also conduct a series of experiments using the models in a retrieval-augmented generation (RAG) setting without fine-tuning to assess the upper bound of the original models' performance.

\paragraph{Result}
Table~\ref{tab:result_mcq} presents the zero-shot evaluation results across different models and techniques. Overall, our fine-tuned dataset consistently outperforms other approaches across datasets and models. The only exception is the Constitution dataset, where RAG performs better with Gemma. Models fine-tuned on Kazakh Alpaca show some improvement, though it remains lower than that achieved with our instruction fine-tuning (IFT) dataset.

Combining parts of our IFT dataset with the translated Alpaca dataset yields the highest performance gains. This aligns with prior studies~\cite{mixingup,demystifying}, which suggest that incorporating general chat instructions alongside domain-specific ones enhances model performance.

For RAG-enhanced models, performance generally exceeds that of the vanilla models, except for Falcon on the Constitution dataset. However, fine-tuned models consistently achieve higher scores than their RAG-enhanced counterparts. We hypothesize that this is due to the models' limited proficiency in Kazakh, which may hinder their ability to fully understand the retrieved context. As a result, despite the additional information provided by RAG, the models may struggle to extract the necessary details to select the correct answer in MCQs.

%shows that the fine-tuning on Kazakhstan specific data improves the accuracy on the MSQ test within all 3 benchmarks. Interestingly, when fine-tuned on unrelated to Kazakhstan translated Alpaca\footnote{\url{https://huggingface.co/datasets/AmanMussa/kazakh-instruction-v2}} dataset, models still show some improvement, which is likely due to adaptation to questions in Kazakh language. Combining part of the instructions from both our IFT dataset, and the translated Alpaca dataset we achieve the best improvement in scores, which aligns well with the previous studies~\cite{mixingup,demystifying} that have shown, that adding general chat instructions to domain specific ones improves the model performance.



% \subsubsection{Generation Evaluation}
% \label{sec:gen_eval}
% To assess the quality of model outputs, we sampled 1,000 instructions from the dataset and generated corresponding outputs using the models under evaluation. 

% For evaluation, we used the LLM as a judge method. Specifically, GPT-4o was tasked to determine whether the model-generated responses to the instructions were correct by comparing them to the reference outputs (referred to as the "golden generation") provided in the dataset.

% Recognizing the critical importance of factual accuracy in responses to government-related questions—where any inaccuracies could result in significant consequences—we employed an adapted version of FactScores, as introduced in~\cite{min-etal-2023-factscore}. Using GPT-4o, we extracted factual elements and procedural steps (where applicable) from the golden answers. The same process was applied to the model-generated outputs. Precision was calculated by identifying hallucinated facts or steps in the generated outputs that were absent in the golden answers, while recall was measured by evaluating whether the generated outputs fully covered the facts and steps present in the golden answers. These metrics provided a rigorous means to evaluate the factual accuracy and completeness of the model outputs.


% \begin{table}[ht]
% \centering
% \begin{tabular}{|l|c|c|}
% \hline
% \textbf{Model} & \textbf{AC  (binary)} & \textbf{AC (5-scale)} \\ \hline
% Llama-3.1$_{8B}$ & 0.052& 1.236  \\ \hline
% Qwen-2.5$_{7B}$ & 0.02 & 1.488 \\ \hline
% Gemma-2$_{9B}$ &  0.052 & 1.502 \\ \hline
% \end{tabular}
% \caption{Evaluation with LLM-as-a-judge method. AC refers to the average correctness, measured as a binary score or a score on a 5-point scale.}
% \label{tab:model_eval}
% \end{table}
\subsection{Generation Evaluation}
\label{sec:rogue}

We evaluate generation performance using our test set, which consists of 500 questions from both \texttt{CultSet} and \texttt{GovSet} (excluded from fine-tuning). We compare the best models from Section~\ref{sec:mcq} against their vanilla counterparts. In this section, "After Fine-Tuning" refers to models fine-tuned on Alpaca + Our Data, while "Vanilla" refers to the original pre-trained models.


\paragraph{Automatic Evaluation with ROUGE and BERTScore} As shown in Table~\ref{tab:rouge}, fine-tuned models generally outperform their vanilla counterparts, except for Qwen, where fine-tuning results in a lower ROUGE-L score \cite{lin-2004-rouge}. However, a lower ROUGE-L does not necessarily indicate worse performance—it may be due to Qwen generating different phrasings compared to the gold answers.

To further validate the quality of generated responses, we also evaluate BERTScore \cite{zhangbertscore}. We use Kaz-RoBERTa\footnote{\href{https://huggingface.co/kz-transformers/kaz-roberta-conversational}{Huggingface model: kaz-roberta-conversational}} as the encoder model, as it is one of the few open-source Kazakh-language transformers. The BERTScore results in Table~\ref{tab:bertscore} align well with the ROUGE-L scores. However, since Kazakh is a low-resource language, BERTScore should be considered a reference point rather than a definitive metric, as Kaz-RoBERTa embeddings may not perfectly capture synonym relationships.

\begin{table}[t!]
\scriptsize
    \centering
    \begin{tabular}{lcc|cc}
        \toprule
        & \multicolumn{2}{c|}{\textbf{CultSet}} & \multicolumn{2}{c}{\textbf{GovSet}} \\
        \cmidrule(lr){2-3} \cmidrule(lr){4-5}
        & \textbf{After FT} & \textbf{Vanilla} & \textbf{After FT} & \textbf{Vanilla} \\
        \midrule
        Gemma & \textbf{24.87} & 15.76 & \textbf{25.10} & 16.12 \\
        Falcon & \textbf{27.98} & 25.96 & \textbf{28.70} & 26.17 \\
         Qwen & 26.63 & \textbf{27.64} & 28.42 & \textbf{30.27} \\
        \bottomrule
    \end{tabular}
    \caption{Rouge-L comparison on \texttt{CultSet} and \texttt{GovSet}.} % datasets
    \label{tab:rouge}
\end{table}


% \begin{table*}[h]
% \scriptsize
%     \centering
%     \begin{tabular}{lccc ccc}
%         \toprule
%         & \multicolumn{3}{c}{\textbf{CultSet}} & \multicolumn{3}{c}{\textbf{GovSet}} \\
%         \cmidrule(lr){2-4} \cmidrule(lr){5-7}
%         & \textbf{Precision} & \textbf{Recall} & \textbf{F1} & \textbf{Precision} & \textbf{Recall} & \textbf{F1} \\
%         \midrule
%         \multicolumn{7}{c}{\textbf{Ours (Fine-Tuned)}} \\
%         Gemma  & 41.94 & 46.36 & 43.62 & 40.27 & 44.90 & 42.00 \\
%         Falcon & 24.59 & 29.68 & 26.64 & 23.78 & 27.73 & 25.36 \\
%         Qwen   & 39.64 & 45.40 & 41.82 & 36.28 & 40.20 & 37.59 \\
%         \midrule
%         \multicolumn{7}{c}{\textbf{Orig (Before Fine-Tuning)}} \\
%         Gemma  & 29.26 & 33.47 & 30.92 & 27.36 & 34.81 & 30.39 \\
%         Falcon & 23.29 & 28.17 & 25.20 & 20.38 & 24.68 & 22.11 \\
%         Qwen   & 40.58 & 47.46 & 43.40 & 36.57 & 44.14 & 39.50  \\
%         \bottomrule
%     \end{tabular}
%     \caption{BERTScore Precision, Recall, and F1 for CultSet and GovSet (Ours vs. Orig)}
%     \label{tab:bertscore}
% \end{table*}

\begin{table*}[ht]
\scriptsize
    \centering
    \begin{tabular}{llccc ccc}
        \toprule
        & & \multicolumn{3}{c}{\textbf{CultSet}} & \multicolumn{3}{c}{\textbf{GovSet}} \\
        \cmidrule(lr){3-5} \cmidrule(lr){6-8}
        & & \textbf{Precision} & \textbf{Recall} & \textbf{F1} & \textbf{Precision} & \textbf{Recall} & \textbf{F1} \\
        \midrule
        \multirow{3}{*}{\textbf{ After Fine-Tuning}} & Gemma  & 41.94 & 46.36 & 43.62 & 40.27 & 44.90 & 42.00 \\
                                                    & Falcon & 24.59 & 29.68 & 26.64 & 23.78 & 27.73 & 25.36 \\
                                                    & Qwen   & 39.64 & 45.40 & 41.82 & 36.28 & 40.20 & 37.59 \\
        \midrule
        \multirow{3}{*}{\textbf{Vanilla}} & Gemma  & 29.26 & 33.47 & 30.92 & 27.36 & 34.81 & 30.39 \\
                                                            & Falcon & 23.29 & 28.17 & 25.20 & 20.38 & 24.68 & 22.11 \\
                                                            & Qwen   & 40.58 & 47.46 & 43.40 & 36.57 & 44.14 & 39.50  \\
        \bottomrule
    \end{tabular}
    \caption{BERTScore Precision, Recall, and F1 for \texttt{CultSet} and \texttt{GovSet}.}
    \label{tab:bertscore}
\end{table*}


\paragraph{Preference Evaluation with GPT-4o}
We conducted a 1-to-1 preference evaluation using the LLM-as-a-judge approach. Specifically, we prompted GPT-4o to compare responses from different models and determine whether each response wins, loses, or ties. The prompt includes the instruction and the gold response as context for GPT-4o.\footnote{The prompt used for comparison is provided in Appendix~\ref{sec:pref-eval}.} As shown in Figure~\ref{fig:human_fg_1000}, the results align with ROUGE-L and BERTScore, confirming that fine-tuned models generally produce improved outputs. Compared to Falcon, Qwen and Gemma exhibit more significant improvements (63\%–80\% winning rate), likely because their pre-trained versions were less optimized for the task, making fine-tuning more impactful.


Additionally, we analyze the win rate across topics in \texttt{CultSet} and \texttt{GovSet}, as shown in Appendix~\ref{app:preference-category-eval-results}. The results indicate that the impact of fine-tuning varies by topic and is not always consistent. In \texttt{CultSet}, fine-tuning Qwen with our IFT data yields the most improvement in Cultural Institutions and Culture \& Traditions, while the gains are smaller in Science \& Humanities and even lead to a decline in performance for Education \& Academia. In \texttt{GovSet}, fine-tuning Qwen with our dataset significantly enhances performance in Legal Assistance, though the improvement is less noticeable in Employment-related topics.



%after fine-tuning Gemma's and Qwen’s responses are more preferred than the ones of original model in Cultural Institutions category, whereas Falcon's original model performs better in this category. 
%A similar pattern appears in \texttt{GovSet}, where fine-tuned Gemma and Qwen outperform their original versions in Benefits, Allowances, and Pensions, while Falcon shows nearly balanced preferences before and after fine-tuning.
%Overall, the categories where fine-tuning provides the most improvement are largely consistent between Gemma and Qwen (matching in the top three for \texttt{GovSet} and top two for \texttt{CultSet}) but diverge significantly from Falcon. We hypothesize that these differences stem from variations in the models' pretraining data.

While LLM-based evaluations provide scalable comparisons, they may not fully capture human judgment nuances, making human evaluation essential for validating model preferences. Therefore, three human annotators conducted a preference evaluation on a randomly sampled 100 examples for each model (Gemma, Qwen, and Falcon) across both \texttt{CultSet} and \texttt{GovSet}. Their judgments were compared against the GPT-based preference evaluation \textbf{to assess alignment}. We computed Cohen’s Kappa between GPT-4o and the annotators, obtaining 0.63 for \texttt{CultSet} and 0.68 for \texttt{GovSet}, indicating substantial agreement. We have also calculated the agreement rate between annotators (detailed in Appendix \ref{app:inner-annot-gen-eval}). The results show that GPT's alignment with human preferences is moderate, with better agreement on \texttt{GovSet} than \texttt{CultSet}.
%We also conducted a human preference evaluation on a subset of 100 samples from CultSet and 100 from GovSet to compare against the GPT-4o-based evaluation. The results show a moderate agreement between GPT-4o and human preferences, with a Cohen's kappa of 0.59. Additionally, the agreement rate between GPT-4o and human annotators was 61\%, indicating a substantial overlap in judgments while still reflecting differences in evaluation criteria between LLM-based and human assessments.

% Participants were presented with two responses to the same instruction in random order—one generated before tuning and one generated after tuning. They were tasked with selecting the response they found superior, considering both its correctness and overall soundness. 
% This human evaluation method allowed us to assess the perceived improvement in the quality and naturalness of model outputs following the proposed adjustments.
% Then we run 


% \textbf{CultSet} The preference results indicate that fine-tuning on our dataset significantly improved performance across most categories, especially for topics like "Culture and Traditions," "Historical Figures," and "Arts and Entertainment," with Gemma and Qwen models showing a clear preference for fine-tuned outputs. However, categories like "Education and Academia" and "Health and Medicine" favored the original models for Qwen and Falcon, suggesting that these domains may require more specialized fine-tuning data. Falcon displayed a more balanced distribution of preferences, with higher tie rates and narrower gaps between fine-tuned and original outputs, likely due to its robust pretraining.

% \textbf{GovSet} Across the three models (Gemma, Qwen, Falcon), the fine-tuned version ("Ours") consistently outperforms the original model across most categories in GovSet, with Gemma and Qwen showing particularly strong preferences for the fine-tuned outputs in domains like "Legal Assistance," "Healthcare," and "Employment and Job Placement." Falcon, however, demonstrates a higher proportion of ties, indicating a closer alignment between its original and fine-tuned outputs, likely due to its strong baseline performance. Notable exceptions include categories like "Financial Literacy" and "Taxes and Fines," where the original model occasionally outperforms the fine-tuned version, suggesting areas where pretraining data may already provide robust coverage. 

\paragraph{Conversational Evaluation.} As an extension of these experiments, we generated a set of 100 conversations for both \texttt{CultSet} and \texttt{GovSet} combined, covering topics presented in Figure \ref{fig:distribution-all}. These conversations were intentionally left unfinished using a special prompt, as detailed in Appendix \ref{sec:pref-eval}. Both the original and fine-tuned models were tasked with generating the most appropriate continuation for each conversation. Examples of the resulting texts are showin in Table \ref{sec:conversational-data-sample}.
To evaluate the quality of the responses, we employed an LLM-as-a-judge framework. The results, presented in Figure \ref{fig:conversational_preferences}, indicate that models fine-tuned on domain-specific data produced significantly more coherent and contextually appropriate responses compared to their pre-fine-tuning counterparts.
We also see that in the conversational settings there are less ties, compared to simple question answering.

% \begin{figure*}[ht!]
%     \centering
%     \begin{minipage}[b]{0.45\textwidth}
%         \centering
%         \includegraphics[scale=0.4]{images/new_gen_eval_pref.png}
%         \caption*{(a) CultSet}
%     \end{minipage}
%     \hfill
%     \begin{minipage}[b]{0.45\textwidth}
%         \centering
%         \includegraphics[scale=0.4]{images/new_gen_eval_gov.png}
%         \caption*{(b) GovSet}
%     \end{minipage}
%     \caption{Distribution of preferences for CultSet (a) and GovSet (b) datasets across models. The charts illustrate the percentage of 'Tie', 'Before Fine-Tuning', and 'After Fine-Tuning' preferences in each dataset.}
%     \label{fig:human_fg_1000}
% \end{figure*}


%\clearpage
%\section{Gen eval preference}
% \begin{figure*}[ht!]
%     \centering
%     \includegraphics[width=0.7\linewidth]{images/gen_eval_cult.png}
%     \caption{Category-wise comparison of preference evaluation on CultSet.}
%     \label{fig:category-wise-eval-cult}
% \end{figure*}

% \begin{figure*}[ht!]
%     \centering
%     \includegraphics[width=0.7\linewidth]{images/gen_eval_gov.png}
%     \caption{Category-wise comparison of preference evaluation on GovSet.}
%     \label{fig:category-wise-eval-gov}
% \end{figure*}


\begin{figure}[t!]
    \centering
    \includegraphics[width=\linewidth]{images/conversation_chart.png}
    \caption{Conversational data preference evaluation.}
    \label{fig:conversational_preferences}
\end{figure}

% %% New Disucssion 
Our study reveals how heavy users integrate LLMs into their daily tasks through distinct patterns. Rather than simple tool usage, participants demonstrated sophisticated cognitive offloading strategies that transformed their decision-making processes. In our study, we observed participants delegating social and interpersonal reasoning to LLMs, suggesting ways users might leverage AI collaboration to support their social cognition processes.

Participants' mental models of LLMs directly influenced their cognitive strategies---those viewing LLMs as rational entities engaged in cognitive complementarity by leveraging LLM capabilities where they perceived personal limitations, while those viewing LLMs as average decision-makers used cognitive benchmarking, establishing baseline standards while reserving higher-order tasks for themselves.
% While delegating a broad range of decisions raised potential concerns about over-reliance and diminished critical thinking, our findings also highlight a nuanced form of human-AI collaboration where users and LLMs develop complementary relationships. Participants showed diverse usage strategies, treating LLMs as an emerging problem-solving tool and developing sophisticated prompting techniques. Most notably, participants frequently sought LLM guidance on social appropriateness and interpersonal situations. Although some users expressed concerns about potential skill degradation and a sense of unease, LLM consultations often led to a more thorough consideration of social factors and an enhanced understanding of different perspectives.

This raises questions for future research on redefining how we conceptualize and measure over-reliance on LLMs. Current metrics typically assess over-reliance through simplified quantitative measures in controlled settings, primarily focusing on users' acceptance rates of LLM outputs ~\cite{bo2024rely, kim2024rely}. However, our findings reveal more complex patterns of engagement. Participants did not blindly adopt LLM outputs, even in cases where they eventually accepted them. Instead, participants demonstrated thoughtful delegation strategies, using LLMs to validate existing decisions, automate routine tasks, or navigate unfamiliar situations. The critical concern was not users' acceptance of LLM outputs, but rather instances where users adopted LLM reasoning without exploring alternative perspectives. Future research should expand the definition of over-reliance beyond simple acceptance rates to examine how users critically engage with alternative lines of reasoning.

Another key direction for future research involves capturing diverse user contexts. Our participants valued the ability of LLMs to extract necessary contextual information when not initially provided. They appreciated that they could receive meaningful responses without extensively explaining background information, even for context-heavy topics like relationship advice. Future research should explore ways to incorporate multi-modal inputs beyond text-based interactions, allowing users to convey context through various channels. Additionally, LLMs' ability to elicit implicit user intentions without explicit prompting is crucial, as demonstrated by recent advances in reasoning-focused LLM architectures that can proactively identify and address underlying user needs.

The development of active usage patterns with LLMs appeared more prominent among younger users who had less experience managing tasks without these systems. Participants with extensive pre-LLM experience maintained clearer boundaries and showed greater awareness of system limitations. In contrast, users with less experience with LLMs demonstrated fewer reservations, viewing LLM interaction itself as a skill and actively developing their prompting strategies. Conducting design studies focused on younger generations, to better understand and support these emerging interaction patterns represents a crucial direction for future research.
\section{Discussion}\label{sec:discussion}



\subsection{From Interactive Prompting to Interactive Multi-modal Prompting}
The rapid advancements of large pre-trained generative models including large language models and text-to-image generation models, have inspired many HCI researchers to develop interactive tools to support users in crafting appropriate prompts.
% Studies on this topic in last two years' HCI conferences are predominantly focused on helping users refine single-modality textual prompts.
Many previous studies are focused on helping users refine single-modality textual prompts.
However, for many real-world applications concerning data beyond text modality, such as multi-modal AI and embodied intelligence, information from other modalities is essential in constructing sophisticated multi-modal prompts that fully convey users' instruction.
This demand inspires some researchers to develop multimodal prompting interactions to facilitate generation tasks ranging from visual modality image generation~\cite{wang2024promptcharm, promptpaint} to textual modality story generation~\cite{chung2022tale}.
% Some previous studies contributed relevant findings on this topic. 
Specifically, for the image generation task, recent studies have contributed some relevant findings on multi-modal prompting.
For example, PromptCharm~\cite{wang2024promptcharm} discovers the importance of multimodal feedback in refining initial text-based prompting in diffusion models.
However, the multi-modal interactions in PromptCharm are mainly focused on the feedback empowered the inpainting function, instead of supporting initial multimodal sketch-prompt control. 

\begin{figure*}[t]
    \centering
    \includegraphics[width=0.9\textwidth]{src/img/novice_expert.pdf}
    \vspace{-2mm}
    \caption{The comparison between novice and expert participants in painting reveals that experts produce more accurate and fine-grained sketches, resulting in closer alignment with reference images in close-ended tasks. Conversely, in open-ended tasks, expert fine-grained strokes fail to generate precise results due to \tool's lack of control at the thin stroke level.}
    \Description{The comparison between novice and expert participants in painting reveals that experts produce more accurate and fine-grained sketches, resulting in closer alignment with reference images in close-ended tasks. Novice users create rougher sketches with less accuracy in shape. Conversely, in open-ended tasks, expert fine-grained strokes fail to generate precise results due to \tool's lack of control at the thin stroke level, while novice users' broader strokes yield results more aligned with their sketches.}
    \label{fig:novice_expert}
    % \vspace{-3mm}
\end{figure*}


% In particular, in the initial control input, users are unable to explicitly specify multi-modal generation intents.
In another example, PromptPaint~\cite{promptpaint} stresses the importance of paint-medium-like interactions and introduces Prompt stencil functions that allow users to perform fine-grained controls with localized image generation. 
However, insufficient spatial control (\eg, PromptPaint only allows for single-object prompt stencil at a time) and unstable models can still leave some users feeling the uncertainty of AI and a varying degree of ownership of the generated artwork~\cite{promptpaint}.
% As a result, the gap between intuitive multi-modal or paint-medium-like control and the current prompting interface still exists, which requires further research on multi-modal prompting interactions.
From this perspective, our work seeks to further enhance multi-object spatial-semantic prompting control by users' natural sketching.
However, there are still some challenges to be resolved, such as consistent multi-object generation in multiple rounds to increase stability and improved understanding of user sketches.   


% \new{
% From this perspective, our work is a step forward in this direction by allowing multi-object spatial-semantic prompting control by users' natural sketching, which considers the interplay between multiple sketch regions.
% % To further advance the multi-modal prompting experience, there are some aspects we identify to be important.
% % One of the important aspects is enhancing the consistency and stability of multiple rounds of generation to reduce the uncertainty and loss of control on users' part.
% % For this purpose, we need to develop techniques to incorporate consistent generation~\cite{tewel2024training} into multi-modal prompting framework.}
% % Another important aspect is improving generative models' understanding of the implicit user intents \new{implied by the paint-medium-like or sketch-based input (\eg, sketch of two people with their hands slightly overlapping indicates holding hand without needing explicit prompt).
% % This can facilitate more natural control and alleviate users' effort in tuning the textual prompt.
% % In addition, it can increase users' sense of ownership as the generated results can be more aligned with their sketching intents.
% }
% For example, when users draw sketches of two people with their hands slightly overlapping, current region-based models cannot automatically infer users' implicit intention that the two people are holding hands.
% Instead, they still require users to explicitly specify in the prompt such relationship.
% \tool addresses this through sketch-aware prompt recommendation to fill in the necessary semantic information, alleviating users' workload.
% However, some users want the generative AI in the future to be able to directly infer this natural implicit intentions from the sketches without additional prompting since prompt recommendation can still be unstable sometimes.


% \new{
% Besides visual generation, 
% }
% For example, one of the important aspect is referring~\cite{he2024multi}, linking specific text semantics with specific spatial object, which is partly what we do in our sketch-aware prompt recommendation.
% Analogously, in natural communication between humans, text or audio alone often cannot suffice in expressing the speakers' intentions, and speakers often need to refer to an existing spatial object or draw out an illustration of her ideas for better explanation.
% Philosophically, we HCI researchers are mostly concerned about the human-end experience in human-AI communications.
% However, studies on prompting is unique in that we should not just care about the human-end interaction, but also make sure that AI can really get what the human means and produce intention-aligned output.
% Such consideration can drastically impact the design of prompting interactions in human-AI collaboration applications.
% On this note, although studies on multi-modal interactions is a well-established topic in HCI community, it remains a challenging problem what kind of multi-modal information is really effective in helping humans convey their ideas to current and next generation large AI models.




\subsection{Novice Performance vs. Expert Performance}\label{sec:nVe}
In this section we discuss the performance difference between novice and expert regarding experience in painting and prompting.
First, regarding painting skills, some participants with experience (4/12) preferred to draw accurate and fine-grained shapes at the beginning. 
All novice users (5/12) draw rough and less accurate shapes, while some participants with basic painting skills (3/12) also favored sketching rough areas of objects, as exemplified in Figure~\ref{fig:novice_expert}.
The experienced participants using fine-grained strokes (4/12, none of whom were experienced in prompting) achieved higher IoU scores (0.557) in the close-ended task (0.535) when using \tool. 
This is because their sketches were closer in shape and location to the reference, making the single object decomposition result more accurate.
Also, experienced participants are better at arranging spatial location and size of objects than novice participants.
However, some experienced participants (3/12) have mentioned that the fine-grained stroke sometimes makes them frustrated.
As P1's comment for his result in open-ended task: "\emph{It seems it cannot understand thin strokes; even if the shape is accurate, it can only generate content roughly around the area, especially when there is overlapping.}" 
This suggests that while \tool\ provides rough control to produce reasonably fine results from less accurate sketches for novice users, it may disappoint experienced users seeking more precise control through finer strokes. 
As shown in the last column in Figure~\ref{fig:novice_expert}, the dragon hovering in the sky was wrongly turned into a standing large dragon by \tool.

Second, regarding prompting skills, 3 out of 12 participants had one or more years of experience in T2I prompting. These participants used more modifiers than others during both T2I and R2I tasks.
Their performance in the T2I (0.335) and R2I (0.469) tasks showed higher scores than the average T2I (0.314) and R2I (0.418), but there was no performance improvement with \tool\ between their results (0.508) and the overall average score (0.528). 
This indicates that \tool\ can assist novice users in prompting, enabling them to produce satisfactory images similar to those created by users with prompting expertise.



\subsection{Applicability of \tool}
The feedback from user study highlighted several potential applications for our system. 
Three participants (P2, P6, P8) mentioned its possible use in commercial advertising design, emphasizing the importance of controllability for such work. 
They noted that the system's flexibility allows designers to quickly experiment with different settings.
Some participants (N = 3) also mentioned its potential for digital asset creation, particularly for game asset design. 
P7, a game mod developer, found the system highly useful for mod development. 
He explained: "\emph{Mods often require a series of images with a consistent theme and specific spatial requirements. 
For example, in a sacrifice scene, how the objects are arranged is closely tied to the mod's background. It would be difficult for a developer without professional skills, but with this system, it is possible to quickly construct such images}."
A few participants expressed similar thoughts regarding its use in scene construction, such as in film production. 
An interesting suggestion came from participant P4, who proposed its application in crime scene description. 
She pointed out that witnesses are often not skilled artists, and typically describe crime scenes verbally while someone else illustrates their account. 
With this system, witnesses could more easily express what they saw themselves, potentially producing depictions closer to the real events. "\emph{Details like object locations and distances from buildings can be easily conveyed using the system}," she added.

% \subsection{Model Understanding of Users' Implicit Intents}
% In region-sketch-based control of generative models, a significant gap between interaction design and actual implementation is the model's failure in understanding users' naturally expressed intentions.
% For example, when users draw sketches of two people with their hands slightly overlapping, current region-based models cannot automatically infer users' implicit intention that the two people are holding hands.
% Instead, they still require users to explicitly specify in the prompt such relationship.
% \tool addresses this through sketch-aware prompt recommendation to fill in the necessary semantic information, alleviating users' workload.
% However, some users want the generative AI in the future to be able to directly infer this natural implicit intentions from the sketches without additional prompting since prompt recommendation can still be unstable sometimes.
% This problem reflects a more general dilemma, which ubiquitously exists in all forms of conditioned control for generative models such as canny or scribble control.
% This is because all the control models are trained on pairs of explicit control signal and target image, which is lacking further interpretation or customization of the user intentions behind the seemingly straightforward input.
% For another example, the generative models cannot understand what abstraction level the user has in mind for her personal scribbles.
% Such problems leave more challenges to be addressed by future human-AI co-creation research.
% One possible direction is fine-tuning the conditioned models on individual user's conditioned control data to provide more customized interpretation. 

% \subsection{Balance between recommendation and autonomy}
% AIGC tools are a typical example of 
\subsection{Progressive Sketching}
Currently \tool is mainly aimed at novice users who are only capable of creating very rough sketches by themselves.
However, more accomplished painters or even professional artists typically have a coarse-to-fine creative process. 
Such a process is most evident in painting styles like traditional oil painting or digital impasto painting, where artists first quickly lay down large color patches to outline the most primitive proportion and structure of visual elements.
After that, the artists will progressively add layers of finer color strokes to the canvas to gradually refine the painting to an exquisite piece of artwork.
One participant in our user study (P1) , as a professional painter, has mentioned a similar point "\emph{
I think it is useful for laying out the big picture, give some inspirations for the initial drawing stage}."
Therefore, rough sketch also plays a part in the professional artists' creation process, yet it is more challenging to integrate AI into this more complex coarse-to-fine procedure.
Particularly, artists would like to preserve some of their finer strokes in later progression, not just the shape of the initial sketch.
In addition, instead of requiring the tool to generate a finished piece of artwork, some artists may prefer a model that can generate another more accurate sketch based on the initial one, and leave the final coloring and refining to the artists themselves.
To accommodate these diverse progressive sketching requirements, a more advanced sketch-based AI-assisted creation tool should be developed that can seamlessly enable artist intervention at any stage of the sketch and maximally preserve their creative intents to the finest level. 

\subsection{Ethical Issues}
Intellectual property and unethical misuse are two potential ethical concerns of AI-assisted creative tools, particularly those targeting novice users.
In terms of intellectual property, \tool hands over to novice users more control, giving them a higher sense of ownership of the creation.
However, the question still remains: how much contribution from the user's part constitutes full authorship of the artwork?
As \tool still relies on backbone generative models which may be trained on uncopyrighted data largely responsible for turning the sketch into finished artwork, we should design some mechanisms to circumvent this risk.
For example, we can allow artists to upload backbone models trained on their own artworks to integrate with our sketch control.
Regarding unethical misuse, \tool makes fine-grained spatial control more accessible to novice users, who may maliciously generate inappropriate content such as more realistic deepfake with specific postures they want or other explicit content.
To address this issue, we plan to incorporate a more sophisticated filtering mechanism that can detect and screen unethical content with more complex spatial-semantic conditions. 
% In the future, we plan to enable artists to upload their own style model

% \subsection{From interactive prompting to interactive spatial prompting}


\subsection{Limitations and Future work}

    \textbf{User Study Design}. Our open-ended task assesses the usability of \tool's system features in general use cases. To further examine aspects such as creativity and controllability across different methods, the open-ended task could be improved by incorporating baselines to provide more insightful comparative analysis. 
    Besides, in close-ended tasks, while the fixing order of tool usage prevents prior knowledge leakage, it might introduce learning effects. In our study, we include practice sessions for the three systems before the formal task to mitigate these effects. In the future, utilizing parallel tests (\textit{e.g.} different content with the same difficulty) or adding a control group could further reduce the learning effects.

    \textbf{Failure Cases}. There are certain failure cases with \tool that can limit its usability. 
    Firstly, when there are three or more objects with similar semantics, objects may still be missing despite prompt recommendations. 
    Secondly, if an object's stroke is thin, \tool may incorrectly interpret it as a full area, as demonstrated in the expert results of the open-ended task in Figure~\ref{fig:novice_expert}. 
    Finally, sometimes inclusion relationships (\textit{e.g.} inside) between objects cannot be generated correctly, partially due to biases in the base model that lack training samples with such relationship. 

    \textbf{More support for single object adjustment}.
    Participants (N=4) suggested that additional control features should be introduced, beyond just adjusting size and location. They noted that when objects overlap, they cannot freely control which object appears on top or which should be covered, and overlapping areas are currently not allowed.
    They proposed adding features such as layer control and depth control within the single-object mask manipulation. Currently, the system assigns layers based on color order, but future versions should allow users to adjust the layer of each object freely, while considering weighted prompts for overlapping areas.

    \textbf{More customized generation ability}.
    Our current system is built around a single model $ColorfulXL-Lightning$, which limits its ability to fully support the diverse creative needs of users. Feedback from participants has indicated a strong desire for more flexibility in style and personalization, such as integrating fine-tuned models that cater to specific artistic styles or individual preferences. 
    This limitation restricts the ability to adapt to varied creative intents across different users and contexts.
    In future iterations, we plan to address this by embedding a model selection feature, allowing users to choose from a variety of pre-trained or custom fine-tuned models that better align with their stylistic preferences. 
    
    \textbf{Integrate other model functions}.
    Our current system is compatible with many existing tools, such as Promptist~\cite{hao2024optimizing} and Magic Prompt, allowing users to iteratively generate prompts for single objects. However, the integration of these functions is somewhat limited in scope, and users may benefit from a broader range of interactive options, especially for more complex generation tasks. Additionally, for multimodal large models, users can currently explore using affordable or open-source models like Qwen2-VL~\cite{qwen} and InternVL2-Llama3~\cite{llama}, which have demonstrated solid inference performance in our tests. While GPT-4o remains a leading choice, alternative models also offer competitive results.
    Moving forward, we aim to integrate more multimodal large models into the system, giving users the flexibility to choose the models that best fit their needs. 
    


\section{Conclusion}\label{sec:conclusion}
In this paper, we present \tool, an interactive system designed to help novice users create high-quality, fine-grained images that align with their intentions based on rough sketches. 
The system first refines the user's initial prompt into a complete and coherent one that matches the rough sketch, ensuring the generated results are both stable, coherent and high quality.
To further support users in achieving fine-grained alignment between the generated image and their creative intent without requiring professional skills, we introduce a decompose-and-recompose strategy. 
This allows users to select desired, refined object shapes for individual decomposed objects and then recombine them, providing flexible mask manipulation for precise spatial control.
The framework operates through a coarse-to-fine process, enabling iterative and fine-grained control that is not possible with traditional end-to-end generation methods. 
Our user study demonstrates that \tool offers novice users enhanced flexibility in control and fine-grained alignment between their intentions and the generated images.


\section{Limitations}

MuJoCo Playground inherits the \href{https://mujoco.readthedocs.io/en/stable/mjx.html#mjx-the-sharp-bits}{limitations of MJX} due to constraints imposed by JAX. First, just-in-time (JIT) compilation can be slow (1-3 minutes on Playground's tasks). Second, computation time related to contacts does not scale like the number of \emph{active} contacts in the scene, but like the number of \emph{possible} contacts in the scene. This is due to JAX's requirement of static shapes at compile time. This limitation can be overcome by using more flexible frameworks like Warp~\cite{macklin2022warp} and Taichi~\cite{Genesis}. This upgrade is an active area of development. Finally we should note that the vision-based training using Madrona is still at an early stage.


\bibliography{acl2023}
\bibliographystyle{acl_natbib}


\newpage
\appendix
\section{Applicability of SparseTransX for dense graphs} 
\label{A:density}
Even for fully dense graphs, our KGE computations remain highly sparse. This is because our SpMM leverages the incidence matrix for triplets, rather than the graph's adjacency matrix. In the paper, the sparse matrix $A \in \{-1,0,1\}^{M \times (N+R)}$ represents the triplets, where $N$ is the number of entities, $R$ is the number of relations, and $M$ is the number of triplets. This representation remains extremely sparse, as each row contains exactly three non-zero values (or two in the case of the "ht" representation). Hence, the sparsity of this formulation is independent of the graph's structure, ensuring computational efficiency even for dense graphs.

\section{Computational Complexity}
\label{A:complexity}
 For a sparse matrix $A$ with $m \times k$ having $nnz(A)=$ number of non zeros and dense matrix $X$ with $k \times n$ dimension, the computational complexity of the SpMM is $O(nnz(A) \cdot n)$ since there are a total of $nnz(A)$ number of dot products each involving $n$ components. Since our sparse matrix contains exactly three non-zeros in each row, $nnz(A) = 3m$. Therefore, the complexity of SpMM is $O(3m \cdot n)$ or $O(m \cdot n)$, meaning the complexity increases when triplet counts or embedding dimension is increased. Memory access pattern will change when the number of entities is increased and it will affect the runtime, but the algorithmic complexity will not be affected by the number of entities/relations.

\section{Applicability to Non-translational Models}
\label{A:non_trans}
Our paper focused on translational models using sparse operations, but the concept extends broadly to various other knowledge graph embedding (KGE) methods. Neural network-based models, which are inherently matrix-multiplication-based, can be seamlessly integrated into this framework. Additionally, models such as DistMult, ComplEx, and RotatE can be implemented with simple modifications to the SpMM operations. Implementing these KGE models requires modifying the addition and multiplication operators in SpMM, effectively changing the semiring that governs the multiplication.   

In the paper, the sparse matrix $A \in \{-1,0,1\}^{M \times (N+R)}$ represents the triplets, and the dense matrix $E \in \mathbb{R}^{(N+R) \times d}$ represents the embedding matrix, where $N$ is the number of entities, $R$ is the number of relations, and $M$ is the number of triplets. TransE’s score function, defined as $h + r - t$, is computed by multiplying $A$ and $E$ using an SpMM followed by the L2 norm. This operation can be generalized using a semiring-based SpMM model: $Z_{ij} = \bigoplus_{k=1}^{n} (A_{ik} \otimes E_{kj})$

Here, $\oplus$ represents the semiring addition operator, and $\otimes$ represents the semiring multiplication operator. For TransE, these operators correspond to standard arithmetic addition and multiplication, respectively.

\subsection*{DistMult} 
DistMult’s score function has the expression $h \odot r \odot t$. To adapt SpMM for this model, two key adjustments are required: The sparse matrix $A$ stores $+1$ at the positions corresponding to $h_{\text{idx}}$, $t_{\text{idx}}$, and $r_{\text{idx}}$. Both the semiring addition and multiplication operators are set to arithmetic multiplication. These changes enable the use of SpMM for the DistMult score function.

\subsection*{ComplEx} 
ComplEx’s score function has $h \odot r \odot \bar{t}$, where embeddings are stored as complex numbers (e.g., using PyTorch). In this case, the semiring operations are similar to DistMult, but with complex number multiplication replacing real number multiplication.

\subsection*{RotatE} 
RotatE’s score function has $h \odot r - t$. For this model, the semiring requires both arithmetic multiplication and subtraction for $\oplus$. With minor modifications to our SpMM implementation, the semiring addition operator can be adapted to compute $h \odot r - t$.

\subsection*{Support from other libraries}
Many existing libraries, such as GraphBLAS (Kimmerer, Raye, et al., 2024), Ginkgo (Anzt, Hartwig, et al., 2022), and Gunrock (Wang, Yangzihao, et al., 2017), already support custom semirings in SpMM. We can leverage C++ templates to extend support for KGE models with minimal effort.


\begin{figure*}[t]
\centering     %%% not \center
\includegraphics[width=\textwidth]{figures/all-eval.pdf}
\caption{Loss curve for sparse and non-sparse approach. Sparse approach eventually reaches the same loss value with similar Hits@10 test accuracy.}
\label{fig:loss_curve}
\end{figure*}

\section{Model Performance Evaluation and Convergence}
\label{A:eval}
SpTransX follows a slightly different loss curve (see Figure \ref{fig:loss_curve}) and eventually converges with the same loss as other non-sparse implementations such as TorchKGE. We test SpTransX with the WN18 dataset having embedding size 512 (128 for TransR and TransH due to memory limitation) and run 200-1000 epochs. We compute average Hits@10 of 9 runs with different initial seeds and a learning rate scheduler. The results are shown below. We find that Hits@10 is generally comparable to or better than the Hits@10 achieved by TorchKGE.

\begin{table}[h]
\centering
\caption{Average of 9 Hits@10 Accuracy for WN18 dataset}
\begin{tabular}{|c|c|c|}
\hline
\textbf{Model} & \textbf{TorchKGE} & \textbf{SpTransX} \\ \hline
TransE         & 0.79 ± 0.001700   & 0.79 ± 0.002667   \\ \hline
TransR         & 0.29 ± 0.005735   & 0.33 ± 0.006154   \\ \hline
TransH         & 0.76 ± 0.012285   & 0.79 ± 0.001832   \\ \hline
TorusE         & 0.73 ± 0.003258   & 0.73 ± 0.002780   \\ \hline
\end{tabular}
\label{table:perf_eval}
\end{table}

% We also plot the loss curve for different models in Figure \ref{fig:loss_curve}. We observe that the sparse approach follows a similar loss curve and eventually converges to the same final loss.

\section{Distributed SpTransX and Its Applicability to Large KGs}
\label{A:dist}
SpTransX framework includes several features to support distributed KGE training across multi-CPU, multi-GPU, and multi-node setups. Additionally, it incorporates modules for model and dataset streaming to handle massive datasets efficiently. 

Distributed SpTransX relies on PyTorch Distributed Data Parallel (DDP) and Fully Sharded Data Parallel (FSDP) support to distribute sparse computations across multiple GPUs. 

\begin{table}[h]
\centering
\caption{Average Time of 15 Epochs (seconds). Training time of TransE model with Freebase dataset (250M triplets, 77M entities. 74K relations, batch size 393K)  on 32 NVIDIA A100 GPUs. FSDP enables model training with larger embedding when DDP fails.}
\begin{tabular}{|p{2cm}|p{2.5cm}|p{2.5cm}|}
\hline
\textbf{Embedding Size} & \textbf{DDP (Distributed Data Parallel)} & \textbf{FSDP (Fully Sharded Data Parallel)} \\ \hline
16                      & 65.07 ± 1.641                            & 63.35 ± 1.258                               \\ \hline
20                      & Out of Memory                            & 96.44 ± 1.490                               \\ \hline
\end{tabular}
\end{table}

We run an experiment with a large-scale KG to showcase the performance of distributed SpTransX. Freebase (250M triplets, 77M entities. 74K relations, batch size 393K) dataset is trained using the TransE model on 32 NVIDIA A100 GPUs of NERSC using various distributed settings. SpTransX’s Streaming dataset module allows fetching only the necessary batch from the dataset and enables memory-efficient training. FSDP enables model training with larger embedding when DDP fails.

\section{Scaling and Communication Bottlenecks for Large KG Training}
\label{A:scaling}
Communication can be a significant bottleneck in distributed KGE training when using SpMM. However, by leveraging Distributed Data-Parallel (DDP) in PyTorch, we successfully scale distributed SpTransX to 64 NVIDIA A100 GPUs with reasonable efficiency. The training time for the COVID-19 dataset with 60,820 entities, 62 relations, and 1,032,939 triplets is in Table \ref{table:scaling}. 
% \vspace{-.3cm}
\begin{table}[h]
\centering
\caption{Scaling TransE model on COVID-19 dataset}
\begin{tabular}{|c|c|}
\hline
\textbf{Number of GPUs} & \textbf{500 epoch time (seconds)} \\ \hline
4                       & 706.38                            \\ \hline
8                       & 586.03                            \\ \hline
16                      & 340.00                               \\ \hline
32                      & 246.02                            \\ \hline
64                      & 179.95                            \\ \hline
\end{tabular}
\label{table:scaling}
\end{table}
% \vspace{-.2cm}
It indicates that communication is not a bottleneck up to 64 GPUs. If communication becomes a performance bottleneck at larger scales, we plan to explore alternative communication-reducing algorithms, including 2D and 3D matrix distribution techniques, which are known to minimize communication overhead at extreme scales. Additionally, we will incorporate model parallelism alongside data parallelism for large-scale knowledge graphs.

\section{Backpropagation of SpMM}
\label{A:backprop}
 Our main computational kernel is the sparse-dense matrix multiplication (SpMM). The computation of backpropagation of an SpMM w.r.t. the dense matrix is also another SpMM. To see how, let's consider the sparse-dense matrix multiplication $AX = C$ which is part of the training process. As long as the computational graph reduces to a single scaler loss $\mathfrak{L}$, it can be shown that $\frac{\partial C}{\partial X} = A^T$. Here, $X$ is the learnable parameter (embeddings), and $A$ is the sparse matrix. Since $A^T$ is also a sparse matrix and $\frac{\partial \mathfrak{L}}{\partial C}$ is a dense matrix, the computation $\frac{\partial \mathfrak{L}}{\partial X} = \frac{\partial C}{\partial X} \times \frac{\partial \mathfrak{L}}{\partial C} = A^T \times \frac{\partial \mathfrak{L}}{\partial C} $ is an SpMM. This means that both forward and backward propagation of our approach benefit from the efficiency of a high-performance SpMM.

\subsection*{Proof that $\frac{\partial C}{\partial X} = A^T$}
 To see why $\frac{\partial C}{\partial X} = A^T$ is used in the gradient calculation, we can consider the following small matrix multiplication without loss of generality.
\begin{align*}
A &= \begin{bmatrix}
a_1 & a_2 \\
a_3 & a_4
\end{bmatrix} \\ 
 X &= \begin{bmatrix}
x_1 & x_2 \\
x_3 & x_4
\end{bmatrix} \\
 C &=  \begin{bmatrix}
c_1 & c_2 \\
c_3 & c_4
\end{bmatrix}
\end{align*}
Where $C=AX$, thus-
\begin{align*}
c_1&=f(x_1, x_3) \\
c_2&=f(x_2, x_4) \\
c_3&=f(x_1, x_3) \\
c_4&=f(x_2, x_4) \\
\end{align*}
Therefore-
\begin{align*}
\frac{\partial \mathfrak{L}}{\partial x_1} &= \frac{\partial \mathfrak{L}}{\partial c_1} \times \frac{\partial c_1}{\partial x_1} + \frac{\partial \mathfrak{L}}{\partial c_2} \times \frac{\partial c_2}{\partial x_1} + \frac{\partial \mathfrak{L}}{\partial c_3} \times \frac{\partial c_3}{\partial x_1} + \frac{\partial \mathfrak{L}}{\partial c_4} \times \frac{\partial c_4}{\partial x_1}\\
&= \frac{\partial \mathfrak{L}}{\partial c_1} \times \frac{\partial \mathfrak{c_1}}{\partial x_1} + 0 + \frac{\partial \mathfrak{L}}{\partial c_3} \times \frac{\partial \mathfrak{c_3}}{\partial x_1} + 0\\
&= a_1 \times \frac{\partial \mathfrak{L}}{\partial c_1} + a_3 \times \frac{\partial \mathfrak{L}}{\partial c_3}\\
\end{align*}

Similarly-
\begin{align*}
\frac{\partial \mathfrak{L}}{\partial x_2}
&= a_1 \times \frac{\partial \mathfrak{L}}{\partial c_2} + a_3 \times \frac{\partial \mathfrak{L}}{\partial c_4}\\
\frac{\partial \mathfrak{L}}{\partial x_3}
&= a_2 \times \frac{\partial \mathfrak{L}}{\partial c_1} + a_4 \times \frac{\partial \mathfrak{L}}{\partial c_3}\\
\frac{\partial \mathfrak{L}}{\partial x_4}
&= a_2 \times \frac{\partial \mathfrak{L}}{\partial c_2} + a_4 \times \frac{\partial \mathfrak{L}}{\partial c_4}\\
\end{align*}
This can be expressed as a matrix equation in the following manner-
\begin{align*}
\frac{\partial \mathfrak{L}}{\partial X} &= \frac{\partial C}{\partial X} \times \frac{\partial \mathfrak{L}}{\partial C}\\
\implies \begin{bmatrix}
\frac{\partial \mathfrak{L}}{\partial x_1} & \frac{\partial \mathfrak{L}}{\partial x_2} \\
\frac{\partial \mathfrak{L}}{\partial x_3} & \frac{\partial \mathfrak{L}}{\partial x_4}
\end{bmatrix} &= \frac{\partial C}{\partial X} \times \begin{bmatrix}
\frac{\partial \mathfrak{L}}{\partial c_1} & \frac{\partial \mathfrak{L}}{\partial c_2} \\
\frac{\partial \mathfrak{L}}{\partial c_3} & \frac{\partial \mathfrak{L}}{\partial c_4}
\end{bmatrix}
\end{align*}
By comparing the individual partial derivatives computed earlier, we can say-

\begin{align*}
\begin{bmatrix}
\frac{\partial \mathfrak{L}}{\partial x_1} & \frac{\partial \mathfrak{L}}{\partial x_2} \\
\frac{\partial \mathfrak{L}}{\partial x_3} & \frac{\partial \mathfrak{L}}{\partial x_4}
\end{bmatrix} &= \begin{bmatrix}
a_1 & a_3 \\
a_2 & a_4
\end{bmatrix} \times \begin{bmatrix}
\frac{\partial \mathfrak{L}}{\partial c_1} & \frac{\partial \mathfrak{L}}{\partial c_2} \\
\frac{\partial \mathfrak{L}}{\partial c_3} & \frac{\partial \mathfrak{L}}{\partial c_4}
\end{bmatrix}\\
\implies \begin{bmatrix}
\frac{\partial \mathfrak{L}}{\partial x_1} & \frac{\partial \mathfrak{L}}{\partial x_2} \\
\frac{\partial \mathfrak{L}}{\partial x_3} & \frac{\partial \mathfrak{L}}{\partial x_4}
\end{bmatrix} &= A^T \times \begin{bmatrix}
\frac{\partial \mathfrak{L}}{\partial c_1} & \frac{\partial \mathfrak{L}}{\partial c_2} \\
\frac{\partial \mathfrak{L}}{\partial c_3} & \frac{\partial \mathfrak{L}}{\partial c_4}
\end{bmatrix}\\
\implies \frac{\partial \mathfrak{L}}{\partial X} &= A^T \times \frac{\partial \mathfrak{L}}{\partial C}\\
\therefore \frac{\partial C}{\partial X} &= A^T \qed
\end{align*}

%\input{tables/prompts_appendix}
%% \begin{table*}[ht!]
% \centering
% \renewcommand{\arraystretch}{1} % Adjust row height
% \scalebox{0.7}{
% \begin{tabular}{lcccccccc}
% \toprule
% \textbf{Model name} & \multicolumn{4}{c}{\textbf{Dastur}} & \multicolumn{4}{c}{\textbf{Constitution}} \\  
% \cmidrule(lr){2-5} \cmidrule(lr){6-9}
% & \textbf{No Fine-Tuning} & \textbf{Alpaca} & \textbf{Ours} & \textbf{Alpaca + Ours} 
% & \textbf{No Fine-Tuning} & \textbf{Alpaca} & \textbf{Ours} & \textbf{Alpaca + Ours}  \\
% \midrule
% Gemma 2 - 9b & 0.498 & 0.5133 & 0.543 & \textbf{0.566 $\uparrow$}\\
% Qwen 2.5 - 7b - 9B & 0.403 & 0.421 & 0.443 &\textbf{0.465 $\uparrow$}\\
% Falcon 3 - 10b & 0.45 & 0.458 & 0.464 & \textbf{0.4718 $\uparrow$}
% % & \textbf{0.47 $\uparrow$} & \textbf{0.69 $\downarrow$} & \textbf{3.23 $\uparrow$} 
% % & \textbf{0.53 $\uparrow$} & \textbf{0.62 $\downarrow$}  \\ 
% % Qwen 2.5 - 7B\\
% \\
% \midrule
% \end{tabular}}
% \caption{Comparison of Fine-tuning Conditions Across Models}
% \label{tab:finetuning-comparison}
% \end{table*}


% \begin{table*}[ht!]
% \centering
% \renewcommand{\arraystretch}{1.9} % Adjust row height
% \scalebox{0.6}{
% \begin{tabular}{lcccccccccccc}
% \toprule
% \textbf{Model name} & \multicolumn{4}{c}{\textbf{Dastur}} & \multicolumn{4}{c}{\textbf{Constitution}} & \multicolumn{4}{c}{\textbf{Human Rights}} \\  
% \cmidrule(lr){2-5} \cmidrule(lr){6-9} \cmidrule(lr){10-13}
% & \textbf{No Fine-Tuning} & \textbf{Alpaca} & \textbf{Ours} & \textbf{Alpaca + Ours} 
% & \textbf{No Fine-Tuning} & \textbf{Alpaca} & \textbf{Ours} & \textbf{Alpaca + Ours}  
% & \textbf{No Fine-Tuning} & \textbf{Alpaca} & \textbf{Ours} & \textbf{Alpaca + Ours}  \\
% \midrule
% Gemma 2 - 9b & 0.498 & 0.513 & 0.543 & \textbf{0.566 $\uparrow$} 
%              & 0.600 & 0.627 & 0.640 & \textbf{0.650 $\uparrow$} 
%              & 0.405 & 0.430 & 0.465 & \textbf{0.480 $\uparrow$} \\

% Qwen 2.5 - 7b - 9B & 0.403 & 0.421 & 0.443 &\textbf{0.465 $\uparrow$} 
%                    & 0.520 & 0.609 & 0.670 & \textbf{0.680 $\uparrow$} 
%                    & 0.300 & 0.330 & 0.365 & \textbf{0.375$\uparrow$} \\

% Falcon 3 - 10b & 0.450 & 0.458 & 0.464 & \textbf{0.4718 $\uparrow$}  
%                & 0.430 & 0.450 & 0.490 & \textbf{0.520 $\uparrow$}  
%                & 0.215 & 0.234 & 0.250 & \textbf{0.275 $\uparrow$} \\

% \midrule
% \end{tabular}}
% \caption{Comparison of Fine-tuning Conditions Across Models for Dastur, Constitution, and Rights}
% \label{tab:finetuning-comparison}
% \end{table*}
% \begin{table*}[ht!]
% \centering
% \renewcommand{\arraystretch}{1.9} % Adjust row height
% \scalebox{0.5}{
% \begin{tabular}{lcccccccccccccccc}
% \toprule
% \textbf{Model name} & \multicolumn{5}{c}{\textbf{Dastur}} & \multicolumn{5}{c}{\textbf{Constitution}} & \multicolumn{5}{c}{\textbf{Human Rights}} \\  
% \cmidrule(lr){2-6} \cmidrule(lr){7-11} \cmidrule(lr){12-16}
% & \textbf{No Fine-Tuning} & \textbf{Alpaca} & \textbf{Ours} & \textbf{Alpaca + Ours} & \textbf{RAG}  
% & \textbf{No Fine-Tuning} & \textbf{Alpaca} & \textbf{Ours} & \textbf{Alpaca + Ours} & \textbf{RAG}  
% & \textbf{No Fine-Tuning} & \textbf{Alpaca} & \textbf{Ours} & \textbf{Alpaca + Ours} & \textbf{RAG}  \\
% \midrule
% Gemma 2 - 9b & 0.498 & 0.513 & 0.543 & \textbf{0.566 } & \textit{0.533}
%              & 0.600 & 0.627 & 0.640 & \textbf{0.650 } &\textit{0.655} 
%              & 0.405 & 0.430 & 0.465 & \textbf{0.480 } & \textit{0.450}  \\
%
% Qwen 2.5 - 7b - 9B & 0.403 & 0.421 & 0.443 &\textbf{0.465 } & \textit{0.410}  
%                    & 0.520 & 0.609 & 0.670 & \textbf{0.680 $\uparrow$} & \textit{0.523}  
%                    & 0.300 & 0.330 & 0.365 & \textbf{0.375} & \textit{0.325} \\
%
% Falcon 3 - 10b & 0.450 & 0.458 & 0.464 & \textbf{0.4718 }  &  \textit{0.460} 
%                & 0.430 & 0.450 & 0.490 & \textbf{0.520 $\uparrow$}  & \textit{0.3864}  
%                & 0.215 & 0.234 & 0.250 & \textbf{0.275 }  & \textit{0.220} \\
%
% \midrule
% \end{tabular}}
% \caption{Comparison of Fine-tuning Conditions Across Models for Dastur, Constitution, and Rights}
% \label{tab:finetuning-comparison}
% \end{table*}

\begin{table}[t!]
\centering
\renewcommand{\arraystretch}{1.2} % Adjust row height for compactness
\resizebox{\columnwidth}{!}{ % Automatically scales the table to fit within column width
\begin{tabular}{lccccc}
\toprule
\textbf{Model} & \textbf{Vanilla} & \textbf{RAG} & \textbf{Alpaca} & \textbf{Ours} & \textbf{Alpaca + Ours}  \\  
\midrule
\multicolumn{6}{l}{\cellcolor{blue!7}\textbf{Dastur}} \\
Gemma & 0.498 & {0.533} & 0.513 & 0.543 & \textbf{0.566}  \\

Qwen & 0.403 & {0.410} & 0.421 & 0.443 &\textbf{0.465 }  \\

Falcon & 0.450 & {0.460} & 0.458 & 0.464 & \textbf{0.471 }   \\
\midrule
\multicolumn{6}{l}{\cellcolor{blue!7}\textbf{Constitution}} \\
Gemma & 0.600 & {\textbf{0.655}} & 0.627 & 0.640 & {0.650 }  \\

Qwen & 0.520 & {0.523} & 0.609 & 0.670 & \textbf{0.680 }  \\

Falcon & 0.430 & {0.386} & 0.450 & 0.490 & \textbf{0.520 }  \\
\midrule
\multicolumn{6}{l}{\cellcolor{blue!7}\textbf{Human Rights and Society}} \\
Gemma & 0.405 & 0.450 & 0.430 & 0.465 & \textbf{0.480 }  \\

Qwen & 0.300 & 0.325 & 0.330 & 0.365 & \textbf{0.375}  \\

Falcon & 0.215 & 0.220 & 0.234 & 0.250 & \textbf{0.275 }  \\
\bottomrule
\end{tabular}}
\caption{Zero-shot accuracies of language models in different datasets: (1) Datasur, (2) Constitution, and (3) Human Rights and Society}
\label{tab:result_mcq}
\end{table}


% \begin{table}[t!]
% \centering
% \renewcommand{\arraystretch}{1.2} % Adjust row height for compactness
% \resizebox{\columnwidth}{!}{ % Automatically scales the table to fit within column width
% \begin{tabular}{lccccc}
% \toprule
% \textbf{Model} & \textbf{Vanilla} & \textbf{RAG} & \textbf{Alpaca} & \textbf{Ours} & \textbf{Alpaca + Ours}  \\  
% \midrule
% Gemma & 0.600 & {0.655} & 0.627 & 0.640 & \textbf{0.650 }  \\

% Qwen & 0.520 & {0.523} & 0.609 & 0.670 & \textbf{0.680 }  \\

% Falcon & 0.430 & {0.386} & 0.450 & 0.490 & \textbf{0.520 }  \\

% \midrule
% \end{tabular}}
% \caption{Comparison of fine-tuning conditions across models for Constitution.}
% \label{tab:finetuning-constitution}
% \end{table}


% \begin{table}[t!]
% \centering
% \renewcommand{\arraystretch}{1.2} % Adjust row height for compactness
% \resizebox{\columnwidth}{!}{ % Automatically scales the table to fit within column width
% \begin{tabular}{lccccc}
% \toprule
% \textbf{Model} & \textbf{No Fine-Tuning} & \textbf{RAG} & \textbf{Alpaca} & \textbf{Ours} & \textbf{Alpaca + Ours}  \\  
% \midrule
% Gemma & 0.405 & 0.450 & 0.430 & 0.465 & \textbf{0.480 }  \\

% Qwen & 0.300 & 0.325 & 0.330 & 0.365 & \textbf{0.375}  \\

% Falcon & 0.215 & 0.220 & 0.234 & 0.250 & \textbf{0.275 }  \\

% \midrule
% \end{tabular}}
% \caption{Comparison of fine-tuning conditions across models for Human Rights and the Society.}
% \label{tab:finetuning-humanrights}
% \end{table}

%
\subsection{Human Evaluation for Preliminary Study Rubric}
\FloatBarrier

\begin{table*}[h]
\centering
\renewcommand{\arraystretch}{1.5} % Adjust row height
\scriptsize % Reduce font size
\begin{tabular}{>{\centering\arraybackslash}p{1.5cm}|>{\centering\arraybackslash}p{3.5cm}|>{\centering\arraybackslash}p{3.5cm}|>{\centering\arraybackslash}p{3.5cm}}
\hline
\textbf{Score} & \textbf{Correctness} & \textbf{Fluency} & \textbf{Completeness} \\ \hline
1 & Highly inaccurate, incorrect, or misleading information. & Very poor fluency, riddled with errors, making it difficult to read. & Very incomplete, with critical information missing, making it unusable. \\ \hline
2 & Significant factual or logical errors that impact the correctness of the instruction/question. & Multiple errors that hinder readability or cause confusion. & Significant omissions that make the instruction/question incomplete or difficult to interpret.  \\ \hline
3 & Noticeable errors in facts or logical flow, but the general meaning remains clear.
& Noticeable errors in grammar, spelling, or structure, but the text is still understandable. & Noticeable gaps in information or coverage that leave the instruction/question lacking. \\ \hline
4 & Minor factual inaccuracies or inconsistencies that do not affect overall understanding. & Minor grammatical or stylistic errors that do not significantly affect readability. & Slightly incomplete, with minor missing details that do not impact overall understanding. \\ \hline
5 & Fully correct and consistent with Kazakh cultural and governmental contexts, no factual or logical errors. & Perfect fluency, no errors in grammar, spelling, or sentence structure. The text reads smoothly and naturally. & Fully complete, no information is missing, and the instruction/question thoroughly covers the context.\\ \hline

\end{tabular}
\caption{Human annotation rubric for preliminary studies.}
\label{tab:multilingual-issues}
\end{table*}

\FloatBarrier





%\subsection{GovSet}
\begin{table*}[ht!]
\scriptsize
\begin{tabular}{@{}p{0.08\textwidth}p{0.22\textwidth}p{0.22\textwidth}p{0.22\textwidth}p{0.22\textwidth}@{}}

\toprule
\multicolumn{1}{c}{\textbf{Type}} & 
\multicolumn{1}{c}{\textbf{Instruction}} & 
\multicolumn{1}{c}{\textbf{Output}} & 
\multicolumn{1}{c}{\textbf{Updated Instruction}} & 
\multicolumn{1}{c}{\textbf{Updated Output}} \\
\midrule
\\
\multicolumn{1}{c}{\textbf{G}} & 
\foreignlanguage{russian}{Неке қиғаннан кейін құжаттарды ауыстыру үдерісін түсіндіріңіз.}
 & \foreignlanguage{russian}{Некеге тұрғаннан кейін жұбайы, егер ол жұбайының/зайыбының тегін алса, неке қию туралы куәлікті алған сәттен бастап 30 күн ішінде жеке куәлікті ауыстыру қажет.} & \foreignlanguage{russian}{\textcolor{darkgreen}{Қазақстанда} неке қиғаннан кейінгі құжаттарды ауыстыру үдерісін түсіндіріңіз.} & \foreignlanguage{russian}{\textcolor{darkgreen}{Ең алдымен, адам некеден кейін} жұбайының тегін \textcolor{darkgreen}{қабылдаса}, жеке куәлігін 30 күн ішінде ауыстыруы қажет.} \newline \\
 \\
 & Explain the process of changing documents after marriage. & After marriage, if the spouse takes the husband's/wife's last name, the national ID must be changed within 30 days from the date the marriage certificate is received. & Explain the process of changing documents after marriage in Kazakhstan. & First of all, if a person adopts their spouse's surname after marriage, they must change their ID within 30 days.\\
 \\
\hline
\\
\multicolumn{1}{c}{\textbf{G}} &
\foreignlanguage{russian}{Қазақстанда халықаралық рейсте жоғалған багаж үшін өтемақы төленетіні рас па?} & \foreignlanguage{russian}{Иә, халықаралық рейсте жоғалған багаж үшін өтемақы Қазақстан қол қойған екі Конвенцияның – Варшава мен Монреальдың ережелеріне, сондай-ақ компаниялардың ішкі ережелеріне сәйкес төленеді.} & \foreignlanguage{russian}{Қазақстанда халықаралық рейсте жоғалған багаж үшін өтемақы төлейтіні қаншалықты рас?} & \foreignlanguage{russian}{\textcolor{darkgreen}{Қазақстанда} халықаралық рейсте жоғалған багаж үшін өтемақы \textcolor{darkgreen}{төлейтіні толықтай рас. Өтемақы} Қазақстан қол қойған екі Конвенцияның \textcolor{darkgreen}{(}Варшава мен Монреальдың ережелеріне\textcolor{darkgreen}{)} \textcolor{darkgreen}{және} компаниялардың ішкі ережелері \textcolor{darkgreen}{негізінде есептелініп} төленеді.} \newline \\
 & Is it true that compensation is paid in Kazakhstan for lost baggage on international flights? & Yes, compensation for lost baggage on international flights is paid according to the rules of the two Conventions signed by Kazakhstan – the Warsaw and Montreal Conventions – as well as in accordance with the internal rules of the companies. & Is it true that compensation is paid in Kazakhstan for lost baggage on international flights? & It is completely true that compensation is paid in Kazakhstan for lost baggage on international flights. The compensation is calculated and paid based on the two Conventions signed by Kazakhstan (the rules of the Warsaw and Montreal Conventions) and the internal regulations of the companies.\\
\hline
\\
\multicolumn{1}{c}{\textbf{G}} &
\foreignlanguage{russian}{Қазақстанда қандастарға азаматтық алудың шарттары туралы түсіндіріңіз.} & \foreignlanguage{russian}{ Қандастар Қазақстан аумағында тұру мерзіміне қарамастан жеңілдетілген тәртіпте азаматтықты ала алады. Олар бұрынғы азаматтықтан бас тарту туралы нотариат куәландырған өтініш, бекітілген нысандағы сауалнама-өтініш, шарттарды сақтау туралы міндеттеме, өмірбаян, жеке басын куәландыратын және адамның басқа елдің азаматтығына жататындығын растайтын құжаттың көшірмесі, қандастың куәлігі және төрт фотосурет секілді құжаттарды тапсыруы керек.} & \foreignlanguage{russian}{\textcolor{darkgreen}{Қандастар Қазақстан Республикасының азаматтығын алу үшін не істеу керек?}} & \foreignlanguage{russian}{\textcolor{darkgreen}{Қандастардың Қазақстан Республикасында тұрған} мерзіміне қарамастан \textcolor{darkgreen}{олардың азаматтық алуы жеңілдетілген тәртіппен жүзеге асады. Қандастар Қазақстан Республикасының азаматтығын алу үшін} бұрынғы азаматтықтан бас тарту туралы нотариат куәландырған өтініш, бекітілген нысандағы сауалнама-өтініш, шарттарды сақтау туралы міндеттеме, өмірбаян, жеке басын куәландыратын және адамның басқа елдің азаматтығына жататындығын растайтын құжаттың көшірмесі, қандастың куәлігі және төрт фотосурет секілді құжаттарды тапсыруы керек.}\\ \\
 & Explain the conditions for obtaining citizenship for oralmans (ethnic Kazakhs) in Kazakhstan. & Oralmans (ethnic Kazakhs) can obtain citizenship in a simplified manner regardless of the duration of their stay in Kazakhstan. They need to submit a notarized application to renounce their previous citizenship, a completed application form in the approved format, a commitment to comply with the conditions, an autobiography, a copy of a document confirming their identity and lack of foreign citizenship, a certificate of oralman status, and four passport-sized photos. & What should oralmans (ethnic Kazakhs) do to obtain citizenship of the Republic of Kazakhstan? & Oralmans (ethnic Kazakhs) can obtain citizenship in Kazakhstan regardless of the duration of their stay in a simplified manner. To obtain Kazakhstani citizenship, they must submit a notarized application to renounce their previous citizenship, a completed application form in the approved format, a commitment to comply with the conditions, an autobiography, a copy of a document confirming their identity and lack of foreign citizenship, a certificate of oralman status, and four passport-sized photos.\\
\hline
\end{tabular}
\caption{Example of annotation for \texttt{GovSet}.}
\label{tab:localizing-kz-ru-ex}
\end{table*}
\clearpage




\subsection{CultSet}
\begin{table*}[ht!]
\scriptsize
\begin{tabular}{@{}p{0.08\textwidth}p{0.22\textwidth}p{0.22\textwidth}p{0.22\textwidth}p{0.22\textwidth}@{}}

\toprule
\multicolumn{1}{c}{\textbf{Type}} & 
\multicolumn{1}{c}{\textbf{Instruction}} & 
\multicolumn{1}{c}{\textbf{Output}} & 
\multicolumn{1}{c}{\textbf{Updated Instruction}} & 
\multicolumn{1}{c}{\textbf{Updated Output}} \\
\midrule
\\
\multicolumn{1}{c}{\textbf{C}} & 
\foreignlanguage{russian}{Фильмнің қандай жетістіктері болды?}
 & \foreignlanguage{russian}{«Транссібір экспресі» фильмі 1978 жылы Ереванда өткен Бүкілодақтық кинофестивальда 'ең үздік шытырман оқиғалы фильм үшін' сыйлығын алды. Сондай-ақ, Карловы Варыда өткен 21-Халықаралық кинофестивальда Азия, Африка және Латын Америкасы елдерінің симпозиумы қазылар алқасының сыйлығына ие болды. 1978 жылы режиссер Оразбаев пен суретші Қарсақбаевқа Қазақстан Мемлекеттік сыйлығы берілді.} & \foreignlanguage{russian}{\textcolor{darkgreen}{"Транссібір экспресі" атты қазақстандық} фильмнің қандай жетістіктері болды?} & \foreignlanguage{russian}{«Транссібір экспресі» фильмі \textcolor{darkgreen}{1977 жылы Қазақфильм студиясы тарабынан шықты.} 1978 жылы ол Ереванда өткен Бүкілодақтық кинофестивальда \textcolor{darkgreen}{"Ең үздік авантюра фильмі үшін" сыйлығын алды. Бұдан гөрі}, Карловы Варыда өткен 21-Халықаралық кинофестивальда Азия, Африка және Латын Америкасы елдерінің симпозиумы қазылар алқасының сыйлығына \textcolor{darkgreen}{да} ие болды. 1978 жылы фильм режиссері Оразбаев пен суретшісі Қарсақбаевқа Қазақстан Мемлекеттік сыйлығы берілді.}\newline
 \\
 \\
& What achievements did the film have? & The film "Trans-Siberian Express" received the award for "Best Adventure Film" at the World Film Festival held in Yerevan in 1978. Additionally, it won a prize at the 21st International Film Festival in Karlovy Vary, as well as at the symposium of Asia, Africa, and Latin American countries. In 1978, the film's director, Orazbayev, and cinematographer Karasbayev were awarded the State Prize of Kazakhstan. & What achievements did the Kazakhstani film "Trans-Siberian Express" have? & The film "Trans-Siberian Express" was released in 1977 by the Kazakhfilm studio. In 1978, it received the award for "Best Adventure Film" at the World Film Festival in Yerevan. Later, it also won a prize at the 21st International Film Festival in Karlovy Vary and at the symposium of Asia, Africa, and Latin American countries. In 1978, the film's director, Orazbayev, and cinematographer Karasbayev were awarded the State Prize of Kazakhstan. \\
\\
\hline
\\
\multicolumn{1}{c}{\textbf{C}} &
\foreignlanguage{russian}{Қазақ халқының музыкалық мұрасы қалай дамыды?} & \foreignlanguage{russian}{Қазақ халқының музыкалық мұрасы ғасырлар бойы ауыздан-аузға, ұстаздан-шәкіртке жеткізіліп, көшпенді өмір салтының ықпалымен дамыды. Музыкалық дәстүрлер батырлық және лирикалық жанрдағы эпостарды қамтыды. Қазақ композиторлары әлемдік деңгейде танымал болып, ұлттық музыкалық мәдениетті дамытуда үлкен үлес қосты.} & \foreignlanguage{russian}{Қазақ халқының музыкалық мұрасы қалай дамыды?} & \foreignlanguage{russian}{Қазақ халқының музыкалық мұрасы ғасырлар бойы ауыздан-аузға, ұстаздан-шәкіртке жеткіз\textcolor{darkgreen}{ілген. Негізінен} көшпенді өмір салтының ықпалымен дамыды. \textcolor{darkgreen}{Қазақтың} музыкалық дәстүрі батырлық және лирикалық жанрдағы эпостарды қамтыды. \textcolor{darkgreen}{Қазіргі заманда} қазақ композиторлары әлемдік деңгейде танымал болып, ұлттық музыкалық мәдениетті \textcolor{darkgreen}{одан әрі дамытуға үлкен үлес қосып жүр.}} \newline \\
& How did the musical heritage of the Kazakh people develop? & The musical heritage of the Kazakh people developed over centuries through oral transmission, from master to apprentice, and was influenced by the nomadic lifestyle. Musical traditions included epic genres of heroic and lyrical works. Kazakh composers became renowned internationally and contributed significantly to the development of national musical culture. & How did the musical heritage of the Kazakh people develop? & The musical heritage of the Kazakh people developed over centuries through oral transmission, from master to apprentice, primarily influenced by the nomadic lifestyle. Kazakh musical traditions encompassed genres of epic heroic and lyrical works. In modern times, Kazakh composers have gained international recognition and continue to contribute greatly to the development and advancement of national musical culture. \\
\\
\hline
\\
\multicolumn{1}{c}{\textbf{C}} &
 \foreignlanguage{russian}{Киіз үйде түндік дегеніміз не екенін түсіндіріңіз.} & \foreignlanguage{russian}{Түндік — киіз үйдің шаңырағына жабылатын киіз, ол әрі жарық береді, әрі түтін шығуға мүмкіндік береді. Төртбұрышты түрде пішіліп, шаңырақ диаметірінен 50-60 сантиметрдей артық жасалады.} & \foreignlanguage{russian}{\textcolor{darkgreen}{Қазақ халқының} киіз \textcolor{darkgreen}{үйіндегі түндіктің} не екенін түсіндіріңіз.} & \foreignlanguage{russian}{Түндік — \textcolor{darkgreen}{қазақтың баспанасы} киіз үйдің шаңырағына жабылатын киіз. \textcolor{darkgreen}{Ол әрі ішке жарық өткізеді, әрі түтін шығуына мүмкіндік береді}. Төртбұрышты түрде пішіліп, шаңырақ диаметірінен 50-60 сантиметрдей артық \textcolor{darkgreen}{етіп} жасалады.} 
 \\
& Explain what "tundik" means in a Kazakh yurt. & The "tundik" is the felt cover placed on the top of the yurt, which provides insulation and allows smoke to escape. It is made slightly larger than the circular roof opening, with a diameter exceeding 50-60 centimeters. It is triangular in shape. & Explain what "tundik" means in the Kazakh yurt. & The "tundik" is the felt cover placed on the top of the Kazakh yurt. It serves as insulation and allows smoke to escape. It is triangular in shape and is made slightly larger than the circular roof opening, with a diameter exceeding 50-60 centimeters.\\
\hline
\end{tabular}
\caption{Example of annotation for \texttt{CultSet}.}
\label{tab:localizing-kz-ru-ex}
\end{table*}

%\begin{table*}[ht]
\scriptsize
\begin{tabular}{@{}c@{\hspace{10pt}}p{0.88\textwidth}@{}}
\toprule
\textbf{Error Type} & \textbf{Annotator Comments} \\ 
\midrule 
G & Inconsistent verb forms are present in the generated data. For example, \foreignlanguage{russian}{\textit{түсіндір}} is used instead of \foreignlanguage{russian}{\textit{түсіндіріңіз}}. \\
\midrule
G & The generated data includes phrases such as "\foreignlanguage{russian}{\textit{... туралы сипаттаңыз}}," which is grammatically incorrect because \foreignlanguage{russian}{\textit{туралы}} is redundant. \\
\midrule
S & Phrases like "\foreignlanguage{russian}{\textit{... қалай сипатталады?}}" appear awkward and imply the presence of external text for context. \\
\midrule
L & The word \foreignlanguage{russian}{\textit{карьера}} is used instead of the correct Kazakh term \foreignlanguage{russian}{\textit{мансап}}. \\
\midrule
G & The word \foreignlanguage{russian}{\textit{әртүрлі}} is written as \foreignlanguage{russian}{\textit{әр түрлі}}, which is inconsistent with standard formatting. \\
\midrule
G & There is confusion regarding whether \foreignlanguage{russian}{\textit{-мен}} (\foreignlanguage{russian}{\textit{-бен}}, \foreignlanguage{russian}{\textit{-пен}}) should be attached to or separated from nouns. \\
\midrule
G & The form of \foreignlanguage{russian}{\textit{жылдары}} is inconsistent, alternating between \foreignlanguage{russian}{\textit{жылы}}, \foreignlanguage{russian}{\textit{жылдары}}, or abbreviated forms like \foreignlanguage{russian}{\textit{ж.}} or \foreignlanguage{russian}{\textit{жж.}}. \\
\midrule
L & Russian words, such as \foreignlanguage{russian}{\textit{награда}}, \foreignlanguage{russian}{\textit{карьера}}, and \foreignlanguage{russian}{\textit{музей}}, are used even though Kazakh equivalents exist. \\
\midrule
L & The word \foreignlanguage{russian}{\textit{артис}} is used instead of the correct Kazakh term \foreignlanguage{russian}{\textit{әртіс}}. \\
\midrule
S & Words like \foreignlanguage{russian}{\textit{бастап}} and \foreignlanguage{russian}{\textit{бері}} are omitted when referring to years. For example, \foreignlanguage{russian}{\textit{1967 жылдан ол бұл атаққа ие болды}}. \\
\midrule
G & The phrase \foreignlanguage{russian}{\textit{туралы түсіндіріңіз}} is unnatural and often incorrect. It should be replaced with \foreignlanguage{russian}{\textit{туралы айтыңыз}} or \foreignlanguage{russian}{\textit{айтып беріңіз}}. \\
\midrule
G & The phrase \foreignlanguage{russian}{\textit{жеңіл атлетші}} is used correctly, but \foreignlanguage{russian}{\textit{ауыр атлет}} is written without the suffix \foreignlanguage{russian}{\textit{-ші}}. It should be standardized as \foreignlanguage{russian}{\textit{атлетші}}. \\
\midrule
L & Shortened names and surnames are written without spaces. For example, \foreignlanguage{russian}{\textit{Ы.Алтынсарин}} instead of \foreignlanguage{russian}{\textit{Ы. Алтынсарин}}. \\
\midrule
L & The phrase \foreignlanguage{russian}{\textit{рөл ойнады}} is a direct translation from Russian and should be replaced with \foreignlanguage{russian}{\textit{рөл орындады}} or \foreignlanguage{russian}{\textit{рөл сомдады}}. \\
\midrule
L & Years are written in abbreviated forms, such as \foreignlanguage{russian}{\textit{1973-78}}, omitting \foreignlanguage{russian}{\textit{19}} in the second part. \\
\midrule
L & Names of \foreignlanguage{russian}{\textit{ру}}, tribes (\foreignlanguage{russian}{\textit{тайпа}}), and \foreignlanguage{russian}{\textit{жүз}} are written in lowercase, which is inconsistent with standard capitalization rules. \\
\midrule
L & The word \foreignlanguage{russian}{\textit{рөл}} is occasionally misspelled as \foreignlanguage{russian}{\textit{рол}} or \foreignlanguage{russian}{\textit{роль}}. \\
\midrule
G & Verbs ending with \foreignlanguage{russian}{\textit{-ып}} or \foreignlanguage{russian}{\textit{-іп}} are used incorrectly, leading to unrelated sentences. For example, in the sentence \foreignlanguage{russian}{\textit{Бұл ойын үлкен тойларда арнайы жүлде тағайындалып өткізіліп, оның түрлері көптеген}}, the part after \foreignlanguage{russian}{\textit{өткізіліп}} is unrelated. The correct form would be \foreignlanguage{russian}{\textit{Бұл ойын үлкен тойларда арнайы жүлде тағайындалып өткізіледі; оның түрлері көп}} or \foreignlanguage{russian}{\textit{Бұл ойын үлкен тойларда арнайы жүлде тағайындалып өткізіліп, қазақ халқында әлі де жүргізіліп келеді}}. \\
\midrule
S & Words occasionally appear in the Latin alphabet even though they are not written in Latin in the original text. \\
\midrule
G & Sentences like \foreignlanguage{russian}{\textit{Безендірудің көркемдеу құралдарын қандай өнер салалары пайдаланады деп түсіндіріңіз}} are included. The common error is the use of \foreignlanguage{russian}{\textit{пайдаланады деп түсіндіріңіз}}, which should be replaced with \foreignlanguage{russian}{\textit{пайдаланатынын түсіндіріңіз}}. \\
\midrule
G & The phrase \foreignlanguage{russian}{\textit{жер аударылды}} is used for referring to a person. If it describes someone moving places, it should be corrected to \foreignlanguage{russian}{\textit{жер аударды}}. \\
\midrule
S & The structure of the questions deviates from the natural structure of the Kazakh language. For example: \foreignlanguage{russian}{\textit{Бұл рас па, АХҚО қатысушыларының қызметін реттеу Англия мен Уэльстің жалпы құқық қағидаттары негізінде жүргізіледі?}} \\
\midrule
S & The questions ask for information not mentioned in the text, meaning the answer to the question is not present in the text. \\
\midrule
S  & The structure of the questions is entirely incorrect. For example: \foreignlanguage{russian}{Астана Халықаралық Қаржы Орталығы қандай компаниялар үшін реттеуге икемді тәсіл қолданады?}
The correct version: \foreignlanguage{russian}{Астана Халықаралық Қаржы Орталығы қандай компанияларды реттеуге мүмкіндік береді?} 
\\
\midrule
L & The word \foreignlanguage{russian}{\textit{ол}} is overused in the generated data. \\
\bottomrule
\end{tabular}
\caption{Selected annotator comments highlighting issues in \texttt{CultSet}. Error types are categorized as follows: L – Lexical errors, S – Structural errors, G – Grammatical errors.}
\label{table:annotator_comments_cult}
\end{table*}

%\begin{table*}[ht!]
\scriptsize
\begin{tabular}{@{}c@{\hspace{10pt}}p{0.88\textwidth}@{}}
\toprule
\textbf{Error Type} & \textbf{Annotator Comments} \\ 

\midrule 
S & The generated data constantly uses an unnatural sentence structure that sounds like this: \foreignlanguage{russian}{\textit{Сіз сәбиіңіздің дүниеге келуін күтудесіз және онлайн режимінде емханаға тіркелу үдерісін түсіндіріңізші.}} Here the word \foreignlanguage{russian}{\textit{және}} sounds extra and can be either eliminated or replaced with \foreignlanguage{russian}{\textit{бұл жағдайда.}} This example of a sentence with extra \foreignlanguage{russian}{\textit{және}} is common. \\
\midrule
G & Inconsistent verb forms in instructions. Like \foreignlanguage{russian}{\textit{түсіндіру}} instead of \foreignlanguage{russian}{\textit{түсіндіріңіз.}} \\
\midrule
G & The generated data has two verb forms like \foreignlanguage{russian}{\textit{тану}} and \foreignlanguage{russian}{\textit{танылу}}. If the first one is the action of subject, the second is an action that happens TO a subject/object. 
 \\
\midrule
S & There is a frequent unnatural sentence structure that sounds like this: \foreignlanguage{russian}{\textit{Бұл рас па: некеге тұрған адамдарға куәлік электрондық нысанда беріледі?}} Either the whole statement needs to be reconstructed or \foreignlanguage{russian}{\textit{ме}} should be added at the end of the sentence.
 \\
\midrule
G & The model frequently fails to change words to accusative case (add \foreignlanguage{russian}{\textit{– н/- ын/-ін}}). For example,  in the phrase \foreignlanguage{russian}{\textit{тегі өзгертуге бола ма}}, the correct form would be \foreignlanguage{russian}{\textit{тегін өзгертуге бола ма}}.\\
\midrule
G & The translator often does not differentiate \foreignlanguage{russian}{\textit{кезінде}} and \foreignlanguage{russian}{\textit{кезіндегі}}. For example, in the sentence \foreignlanguage{russian}{\textit{Некені бұзу кезінде құжаттарды ресімдеу туралы түсіндіріңіз}}, the correct word would be \foreignlanguage{russian}{\textit{кезіндегі}}. The same happens with \foreignlanguage{russian}{\textit{Қазақстанда and Қазақстандағы}}.
\\
\midrule
L & The model writes the word \foreignlanguage{russian}{\textit{айырлуы}} several times. It must be either \foreignlanguage{russian}{\textit{айырылуы}} or \foreignlanguage{russian}{\textit{айырылғаны}} correctly. 
 \\
\midrule
S & The model frequently writes a phrase \foreignlanguage{russian}{\textit{жүзеге асырылады}}, which sounds unnatural and must be \foreignlanguage{russian}{\textit{жүзеге асады}} instead. 
\\
\midrule
G & The model frequently used phrases like \foreignlanguage{russian}{\textit{кірмейді екенін}}, when the correct form for any verb there would be \foreignlanguage{russian}{\textit{кірмейтінін}}. \\
\midrule
G & The model constantly uses this phrasing: \foreignlanguage{russian}{\textit{дегеніміз не екенін түсіндіріңіз.}} It sounds unnatural and must be replaced either with \foreignlanguage{russian}{\textit{не екенін түсіндіріңіз}} or just \foreignlanguage{russian}{\textit{дегеніміз не?}} \\ 
\midrule
G & The model cannot differentiate question words like \foreignlanguage{russian}{\textit{ма, ме}} etc. It can write \foreignlanguage{russian}{\textit{міндетті ма?}}, which is used in verbal speech only. The correct form would be \foreignlanguage{russian}{\textit{міндетті ме?}} \\ 
\midrule
G & The model can miss letters. For instance, \foreignlanguage{russian}{\textit{мақсатары}} instead of \foreignlanguage{russian}{\textit{мақсаттары}}; it is also not connected with the original text because the word is not even used there. \\ 
\midrule
S & There is a common sentence structure that sounds like this: \foreignlanguage{russian}{\textit{Тұрмыстық зорлық-зомбылық жағдайында қандай медициналық көмек алу жолдарын айтыңыз.}} Either the word \foreignlanguage{russian}{\textit{қандай}} must be removed and \foreignlanguage{russian}{\textit{-ғы}} added to \foreignlanguage{russian}{\textit{жағдайында}} (\foreignlanguage{russian}{\textit{жағдайындағы}}), or the statement must be a question instead, like \foreignlanguage{russian}{\textit{қандай медициналық көмек алуға болады?}} \\ 
\midrule
G & \foreignlanguage{russian}{\textit{Әр түрлі}} must be written as \foreignlanguage{russian}{\textit{әртүрлі}}. \foreignlanguage{russian}{\textit{Марапат}} must be used instead of \foreignlanguage{russian}{\textit{награда}}. \\ 
\midrule
S & There are many sentences like \foreignlanguage{russian}{\textit{Шындық па, үйде арнайы әлеуметтік қызметтерді көрсету тоқтатылады, егер қызмет алушының өзі олардан бас тартса?}} The structure is wrong, \foreignlanguage{russian}{\textit{шындық па}} must be placed at the end!! And \foreignlanguage{russian}{\textit{егер}} etc. must be placed at the start. Any verb before \foreignlanguage{russian}{\textit{шындық па}} must end with \foreignlanguage{russian}{\textit{-тыны/-тіні}}. \\ 
\midrule
G & The model often writes \foreignlanguage{russian}{\textit{деп түсіндіріңіз}} in instructions. In different occasions it must be written differently, like \foreignlanguage{russian}{\textit{бола ма, түсіндіріңіз}} or \foreignlanguage{russian}{\textit{болатынын түсіндіріңіз}}. \\ 
\midrule
G & Incorrect translation for \foreignlanguage{russian}{\textit{процесс}}. It should be translated as \foreignlanguage{russian}{\textit{үдеріс/үрдіс}}. \\ 
\midrule
L & Sentences like \foreignlanguage{russian}{\textit{Жеке куәлікті алғаш алу үшін қандай құжаттар қажет екені рас па?}} are common in “Is it true that” instructions. Either \foreignlanguage{russian}{\textit{қандай}} must be removed or replaced with a list of documents. \\ 
\midrule
L & The model generates a phrase \foreignlanguage{russian}{\textit{өтініш жасау}}; the more common version would be \foreignlanguage{russian}{\textit{өтініш беру}} or \foreignlanguage{russian}{\textit{өтініш жазу}}. \\
\midrule
G & The model sometimes has a hard time differentiating active and passive positions of subjects. For example, \foreignlanguage{russian}{\textit{тіркеу куәлігі немесе әскери билет жоғалтқанда}} puts the verb as if the subject is passive, whereas the correct form would be \foreignlanguage{russian}{\textit{жоғалғанда}} (because the subjects are in active form). This happens with different verbs throughout the document. \\
\midrule
G & Sometimes the answers that the model generates for instructions are not even connected to the question. For example, the answer for \foreignlanguage{russian}{\textit{Өтініш беруші ұсынылған ақпарат үшін қашан жауап береді деген айтылды ма?}} is \foreignlanguage{russian}{\textit{Иә, өтініш берушілер ұсынылған ақпарат үшін өздері жауап береді деп айтылды.}} The word \foreignlanguage{russian}{\textit{өздері}} as an answer is unrelated to the question \foreignlanguage{russian}{\textit{қашан}}. \\
\midrule
L & The model frequently generates phrases like \foreignlanguage{russian}{\textit{осы мәтін бойынша}}, \foreignlanguage{russian}{\textit{бұл деректерге қарағанда}}, which implies that there is a certain text to rely on. \\
\midrule
L & The model sometimes pronounces \foreignlanguage{russian}{\textit{Беларусь}} as \foreignlanguage{russian}{\textit{Беларуссия}}, which is not correct even unofficially. For formal texts, \foreignlanguage{russian}{\textit{Беларусь}} should be used, or at least \foreignlanguage{russian}{\textit{Белоруссия}}. \\
\midrule
L & The model frequently uses the Russian word \foreignlanguage{russian}{\textit{заем}}, even though there are Kazakh translations for it, such as \foreignlanguage{russian}{\textit{несие}} or \foreignlanguage{russian}{\textit{қарыз}}. This happens because of the original text. \\
\midrule
S & Sometimes the model generates sentence structures like \foreignlanguage{russian}{\textit{Бұл рас па: Азамат 16 жасқа толғанда Қазақстанның әрбір азаматы жеке куәлік алуға міндетті екенін?}} Firstly, the word \foreignlanguage{russian}{\textit{екенін}} in all these sentences is grammatically incorrect here and must be \foreignlanguage{russian}{\textit{екені}}. Additionally, since this sentence structure is very frequent, it must be reconstructed to place \foreignlanguage{russian}{\textit{бұл рас па}} at the end. For example, it should sound like this: \foreignlanguage{russian}{\textit{16 жасқа толғанда Қазақстанның әрбір азаматы жеке куәлік алуға міндетті екені рас па?}} \\
\midrule
S & There are also sentences that start with \foreignlanguage{russian}{\textit{шынайы ма екенін анықтаңыз}} or \foreignlanguage{russian}{\textit{шындық па анықтаңыз}}. These phrases must be rewritten as either \foreignlanguage{russian}{\textit{шындық па, соны анықтаңыз}} or just \foreignlanguage{russian}{\textit{шын ба}}/\foreignlanguage{russian}{\textit{шындық па}} without the \foreignlanguage{russian}{\textit{анықтаңыз}} part. \\
\midrule
G & The model cannot identify where it is better to not use \foreignlanguage{russian}{\textit{табыс септік}}. For example, \foreignlanguage{russian}{\textit{пәтерді іздеу}}/\foreignlanguage{russian}{\textit{пәтерді іздегенде}} is used frequently, but the common and natural way must be without the \foreignlanguage{russian}{\textit{-ді}} suffix. \\

\bottomrule
\end{tabular}
\caption{Selected annotator comments highlighting issues in \texttt{GovSet}. Error types are categorized as follows: L – Lexical errors, S – Structural errors, G – Grammatical errors.}
\label{table:annotator_comments_gov}
\end{table*}



\section{Example of conversational data}
\label{sec:conversational-data-sample}
\begin{table*}[ht]
\centering
\renewcommand{\arraystretch}{1.5} % Adjust row height
\scriptsize % Reduce font size
\begin{tabular}{p{6cm}|p{6cm}}  % Left-align columns
\hline
\textbf{Kazakh} & \textbf{English translation} \\ \hline
\textcolor{darkgreen}{\textbf{\textit{Person 1 says:}}} \foreignlanguage{russian}{Қазақстанда жылжымайтын мүлік алу және сату кезінде заңдық талаптар маңызды рөл атқарады. Бұл процесс кейде ұзаққа созылуы мүмкін, әсіресе, меншік иелері бірнешеу болған жағдайда.} 
& \textcolor{darkgreen}{\textbf{\textit{Person 1 says:}}} In Kazakhstan, legal requirements play an important role when buying and selling real estate. This process can sometimes be prolonged, especially when there are multiple property owners. \\[0.3cm]

\textcolor{orange}{\textbf{\textit{Person 2 says:}}} \foreignlanguage{russian}{Иә, әсіресе, мүлікті мұрагерлік жолмен алған жағдайда немесе кәмелетке толмаған бала иелік еткенде қосымша құжаттар талап етіледі.} 
& \textcolor{orange}{\textbf{\textit{Person 2 says:}}} Yes, especially when the property is inherited or owned by a minor, additional documents are required. \\[0.3cm]

\textcolor{darkgreen}{\textbf{\textit{Person 1 says:}}} \foreignlanguage{russian}{Қазақстандағы жылжымайтын мүліктің сатып алу-сату үрдісі өте маңызды болып табылады. Сіз білесіз бе, кәмелетке толмаған баланың мүлкін сату үшін қандай құжаттар қажет?} 
& \textcolor{darkgreen}{\textbf{\textit{Person 1 says:}}} The process of buying and selling real estate in Kazakhstan is very important. Do you know what documents are required to sell the property of a minor? \\[0.3cm]

\textcolor{orange}{\textbf{\textit{Person 2 says:}}} \foreignlanguage{russian}{...} 
& \textcolor{orange}{\textbf{\textit{Person 2 says:}}} ... \\[0.3cm]

\textcolor{red}{\textbf{\textit{Gemma's response:}}} \foreignlanguage{russian}{"...Кәмелетке толмаған баланың мүлкін сату үшін ата-анасының немесе олардың орынбасарының қол қоюымен жасалған өтініш, баланың туу туралы куәлігі, мүліктің сатып алу-сату туралы құжаттар, мүліктің құқықтық мәртебесін анықтайтын құжаттар қажет."}  
& \textcolor{red}{\textbf{\textit{Gemma's response:}}} "...To sell the property of a minor, an application signed by the parents or their legal representatives, the child's birth certificate, documents regarding the purchase and sale of the property, and documents confirming the legal status of the property are required." \\ 

\hline
\end{tabular}
\caption{Example of conversational data on \texttt{GovSet}.}
\label{tab:conversational-data-sample}
\end{table*}

%
\begin{table*}
\scriptsize
    \centering
    \begin{tabular}{p{0.25\textwidth}p{0.7\textwidth}}
    \toprule
\textbf{Category} & \textbf{Description} \\  
\midrule
Legal Assistance & Services related to providing legal advice, resolving disputes, or ensuring compliance with laws in areas such as property rights, labor law, or civil matters. \\  \midrule
Healthcare & Includes services for accessing healthcare facilities, obtaining health insurance, and managing public health initiatives, including vaccinations and medical records. \\  \midrule
Real Estate & Covers registration of property rights, encumbrances, and the legal transfer of property ownership. \\  \midrule
Education & Involves services for school and university admissions, recognition of diplomas, and access to government scholarships or grants. \\  \midrule
Family & Services addressing marriage registration, child adoption, and other family-related matters like child welfare and maternity benefits. \\  \midrule
Taxes and Fines & Includes filing taxes, obtaining tax statements, and paying administrative fines or penalties. \\  \midrule
Business Registration & Covers the process of starting, registering, or closing businesses, including obtaining licenses and permits. \\  \midrule
Benefits, Allowances, and Pension & Government services related to social welfare benefits, pensions, and financial aid for citizens. \\  \midrule
Employment and Job Placement & Services that facilitate job searches, vocational training, and unemployment benefits. \\  \midrule
Driving and Transportation & Includes services for driver’s license issuance, vehicle registration, and public transportation systems. \\  \midrule
ID Card, Passport, and Registration & Involves obtaining or renewing identity documents, passports, and residency registration. \\  \midrule
Industry & Services supporting industrial development, including subsidies, licenses, and monitoring compliance with regulations. \\  \midrule
Financial Literacy & Government initiatives to improve citizens' understanding of personal finance, savings, and investment. \\  \midrule
Consular Services & Services provided to citizens abroad, such as visa issuance, document legalization, and emergency support. \\  \midrule
Travel, Living Abroad, and Emigration & Addresses issues like emigration permits, citizenship renunciation, and support for expats. \\  \midrule
Licensing and Accreditation & Involves issuing licenses for various professional or business activities and accrediting institutions. \\  \midrule
Obtaining Citizenship & Covers processes for naturalization, renouncing, or restoring Kazakh citizenship. \\  \midrule
Special Services & Includes unique or niche government-provided services that do not fall into other categories. \\  \midrule
Military Registration & Services related to military service registration, deferments, and exemptions. \\  \midrule
Population & Addresses population census activities, demographic statistics, and citizen registration systems. \\  \midrule
Religion & Involves registration and oversight of religious organizations and facilitation of religious freedom policies. \\  
\bottomrule
\end{tabular}
    \caption{Category explanation for \texttt{GovSet}.}
    \label{tab:category_exp_gov}
\end{table*}




\begin{table*}
\scriptsize
\begin{tabular}{p{0.25\textwidth} p{0.7\textwidth}}
    \toprule
    \textbf{Category} & \textbf{Description} \\ 
    \midrule
    Literature and Writing & This category includes Kazakh literature, poetry, and contemporary works. \\ \midrule
    Culture and Traditions & Covers Kazakh cultural practices, including traditional music, dances, festivals, and customs. \\ \midrule
    Media and Communication & Includes Kazakhstan’s media landscape, featuring national TV channels, news outlets. \\ \midrule
    Arts and Entertainment & Encompasses visual arts, theater, film, and music. \\ \midrule
    Education and Academia & Covers the education system in Kazakhstan, including primary to higher education. \\ \midrule
    History and Archaeology & Includes the exploration of Kazakhstan’s ancient sites and the history of the Kazakh people. \\ \midrule
    Politics & Encompasses the political landscape of Kazakhstan, including its government, political parties, and international relations, with a focus on its role in Central Asia. \\ \midrule
    Sciences and Humanities & Includes research and advancements in fields such as environmental science, economics, sociology, and the study of Kazakh identity and history. \\ \midrule
    Historical Figures & This category celebrates important figures in Kazakhstan’s history. \\ \midrule
    Sports and Recreation & Covers popular sports in Kazakhstan like boxing, wrestling, football, traditional Kazakh sports and famous Kazakhstani athletes. \\ \midrule
    Culinary and Food & This includes Kazakhstan's traditional cuisine, such as "beshbarmak" and "kymyz". \\ \midrule
    Health and Medicine & Covers public health systems, advancements in medical care and renowned Kazakh doctors. \\ \midrule
    Business and Industry & This category encompasses Kazakhstan’s growing economy, with a focus on industries like oil and gas, agriculture, and its expanding IT sector. \\ \midrule
    Cultural Institutions & Includes museums, galleries, theaters, and cultural centers in Kazakhstan, such as the National Museum of Kazakhstan and the Astana Opera. \\ \midrule
    Transportation and Infrastructure & Covers the country’s infrastructure development, such as its extensive rail system, airports. \\ \midrule
    Nature and Environment & Focuses on Kazakhstan’s diverse landscapes, such as the steppes, mountains, and lakes. \\ \midrule
    Miscellaneous & A catch-all category that could include any other cultural, social, or historical aspects of Kazakhstan not covered by other categories. \\ 
    \bottomrule
\end{tabular}
    \caption{Category explanation for \texttt{CultSet}.}
    \label{tab:category_exp_culture}
\end{table*}
% Category explanation [Maiya]
% \input{tables/category_cultural}
%\begin{table*}[b!]
\scriptsize
\centering
\scalebox{0.8}{
\begin{tabular}{p{0.15\textwidth}p{0.2\textwidth}p{0.7\textwidth}}
\toprule
Category & Instruction & Output \\ 
\midrule


Industry & \foreignlanguage{russian}{Жеңіл өнеркәсіптің құрамында қандай қосалқы салалар бар?} & \foreignlanguage{russian}{Жеңіл өнеркәсіп 20-дан астам қосалқы салаларды қамтиды, олар тоқыма, тігін және былғары, тері, аяқ киім өндірістері.} \\ 

 & What sub-industries are part of the light industry? & Light industry includes more than 20 sub-industries, these are textile, sewing and leather, shoe industries.  \\
\midrule

Legal assistance & \foreignlanguage{russian}{Балалардың интернетте көп уақыт өткізуі рас па?} & \foreignlanguage{russian}{Иә, балалар мен жасөспірімдер интернетте көп уақыт өткізеді.} \\ 

 & Is it true that children spend a lot of time on the internet? & Yes, children and teenagers spend a lot of time on the internet.  \\
\midrule
Licensing and accreditation & \foreignlanguage{russian}{ Әуесқойлық аң аулауға куәлікті алу үшін қандай құжаттар қажет? } & \foreignlanguage{russian}{ Келесі құжаттар ұсынылуы тиіс: жеке куәлік, өлшемі 3х4см фотосурет (3 дана), аңшылық минимум курсын аяқтау сертификаты, 2 АЕК мөлшерінде мемлекеттік бажды төлеу туралы түбіртектің көшірмесі. } \\ 

 & What documents are needed to obtain a certificate for amateur hunting? & The following documents must be submitted: an identity card, a 3x4cm photo (3 pieces), a certificate of completion of the hunting minimum course, a copy of the receipt for payment of the state duty in the amount of 2 MCI.  \\
\midrule

Military registration & \foreignlanguage{russian}{ 2023 жылғы 16 қарашадан бастап әскери есепке алу қағидаларына қандай өзгерістер енгізілді? } & \foreignlanguage{russian}{ 2023 жылғы 16 қарашадан бастап бейбіт уақытта тіркеу туралы куәліктер беру тоқтатылды, ал әскери билеттер әскери қызметтің не құқық қорғау, арнаулы мемлекеттік органдарда қызметтің белгіленген мерзімінен өткен адамдарға ғана беріледі. } \\ 
\\
\\
 & What changes have been made to the rules of military registration since November 16, 2023? & Since November 16, 2023, the issuance of certificates of registration in peacetime has been suspended, and military tickets are issued only to persons who have passed the established period of military service or service in law enforcement, special state bodies.  \\
\midrule

Obtaining citizenship & \foreignlanguage{russian}{ Қазақстанның ЖОО-да оқитын этникалық қазақтар азаматтықты жеңілдетілген тәртіпте алады ма? } & \foreignlanguage{russian}{ Иә, Қазақстанның ЖОО-да оқитын этникалық қазақтар азаматтықты жеңілдетілген тәртіпте алады. } \\ \midrule
Population & \foreignlanguage{russian}{ 2023 жылы Қазақстанда қай этникалық топ халықтың көп бөлігін құрады? } & \foreignlanguage{russian}{ 2023 жылдың мәліметтер бойынша, қазақтар халықтың көп бөлігін құрайды (70,6\%). } \\ 

 & Do ethnic Kazakhs studying at universities in Kazakhstan receive citizenship in a simplified prodecure? & Yes, ethnic Kazakhs studying at universities in Kazakhstan receive citizenship in a simplified prodecure.  \\
\midrule

Real estate & \foreignlanguage{russian}{ Кәмелетке толмаған адамның өз мүлкін басқара алуы шындық па? } & \foreignlanguage{russian}{ Жоқ, кәмелетке толмаған пәтер иесі 18 жасқа толғанға дейін өз бетінше мүлкін басқара алмайды. } \\ 

 & Is it true that a minor can manage his/her property? & No, a minor apartment owner cannot independently manage his/her property until he/she reaches the age of 18.  \\
\midrule

Religion & \foreignlanguage{russian}{ Қазақстан халқының негізгі бөлігі қандай діндерді ұстанады? } & \foreignlanguage{russian}{ Елімізде халықтың негізгі бөлігі сунниттік исламды ұстанатын мұсылмандар және орыс православтық христиандықты ұстанушылар болып бөлінеді. } \\ 

 & What religions do the majority of the population of Kazakhstan practice? & In the country, the majority of the population is divided into Muslims professing Sunni Islam and adherents of Russian Orthodox Christianity.  \\
\midrule

Special services & \foreignlanguage{russian}{ Мемлекеттік органда қызметті алудағы қажетті құжаттарды түсіндіріңіз. } & \foreignlanguage{russian}{ Жеке басты куәландыратын, тұрақты тұрғылықты жері бойынша тіркелгенін растайтын құжаттар, мүгедектігі туралы және тиісті мемлекеттік ақпараттық жүйелерден мәліметтер қажет. Аталған құжаттар Мемлекеттік корпорацияның және жұмыспен қамту бөлімдерінің бөлімшелері арқылы ЭЦҚ-мен куәландырылған электрондық құжаттар нысанында алынады. } \\ 

 & List the necessary documents in receiving a service in a state body. & Documents confirming registration at the place of permanent residence, disability information and information from the relevant state data systems are required. These documents are obtained through the divisions of the State Corporation and Employment Departments in the form of electronic documents certified by e-signature.  \\
\midrule

Taxes and fines & \foreignlanguage{russian}{ Eqonaq жүйесінде шетелдік туристердің келу мерзімдерін сақтау үшін қандай құрал ұсынылады? } & \foreignlanguage{russian}{ Шетелдік туристердің келу мерзімдерін сақтау үшін Eqonaq жеке кабинетінің басты бетінде көші-қон анықтамалығы ұсынылған. } \\ 

 & What tool is offered in the Eqonaq system to comply with the deadlines for the arrival of foreign tourists? & To comply with the deadlines for the arrival of foreign tourists, a migration guide is provided on the main page of the Eqonaq personal account. \\
\midrule

Travel, living abroad, and emigration & \foreignlanguage{russian}{ Қазақстан Республикасының азамттығынан шығу үшін талап етілетін негізгі шарттарды түсіндіріңіз. } & \foreignlanguage{russian}{ Қазақстан Республикасының азамттығынан шығу үшін негізгі шарттар: азаматтың мемлекет алдында атқарылмаған міндеттері, қылмыстық жауапкершілікке тартылмауы, және азаматтықтан шығудың мемлекеттік қауіпсіздік мүдделеріне қайшы келмеуі тиіс. Егер осы шарттардың бірі бұзылса, азаматтықтан шығуға өтініш қабылданбайды. } \\ 

 & Explain the basic conditions required to renounce citizenship of the Republic of Kazakhstan. & The main conditions for renouncing the citizenship of the Republic of Kazakhstan are: unfulfilled duties of a citizen to the state, non-involvement in criminal liability, and non-renouncing citizenship should not contradict the interests of State Security. If one of these conditions is violated, the application for renunciation of citizenship will not be accepted.  \\
\bottomrule
\end{tabular} }
\caption{Example of created instructions on \texttt{GovSet}.}
\end{table*}






\begin{table*}[b!]
\scriptsize
\centering
\scalebox{0.8}{
\begin{tabular}{p{0.15\textwidth}p{0.2\textwidth}p{0.7\textwidth}}
\toprule
Category & Instruction & Output \\
\midrule
Arts and Entertainment & \foreignlanguage{russian}{Қазақ композиторы мен этномузыканты Еділ Сейілханұлы Құсайыновтың қандай атқарылымды еңбектері бар екенін түсіндіріңіз.} & \foreignlanguage{russian}{Еділ Сейілханұлы Құсайынов — «Қобыз» кафедрасының оқытушысы, түрлі ансамбльдер мен оркестрлерге арналған музыкалық шығармалардың, сондай-ақ «Көкбөрі-Күлтегін» балетінің, «Құрақ көрпе», «Келін» кинофильмдерінің музыкасының авторы. Ол «Алтай-Тарбағатай өңірінің сыбызғы күйлері» кітабының авторы, Қазақстан Республикасының Мәдениет Қайраткері атанған.} \\
 & Explain what are the works of the Kazakh composer and ethnomusician Edil Seilkhanuly Kusainov. & Edil Seilkhanuly Kusainov is an instructor at the "Kobyz" Department, the author of musical compositions for various ensembles and orchestras, as well as music for the "Kokbori-Kultegin" ballet, and the "Kurak Korpe", "Kelin" movies. He is the author of the book "Altai-Tarbagatai onirinin sybyzgy kuileri" (Sybyzgy Kuys of the Altai-Tarbagatai region), and a cultural figure of the Republic of Kazakhstan.  \\ 
\midrule
Business and Industry & \foreignlanguage{russian}{Қазақтың кен инженері мен бұрынғы Түпқараған ауданының әкімі Әлімғали Нұржанұлы Қапуовтың қызмет жолын сипаттаңыз.} & \foreignlanguage{russian}{Әлімғали Нұржанұлы Қапуов – кен инженері, бұрынғы Түпқараған ауданының әкімі. 1948 жылы Форт-Шевченко қаласында туған. ҚазПТИ-ді 1972 жылы бітірген. 1972-2005 жылдары әртүрлі қызметтер атқарған: Жетібай, «Ембімұнай» карьерінде, «Маңғыстау ұлутас» өндірістік бірлестігінде, «Маңғышлақ ауылқұрылыс-12» тресінде, «Маңғыстаумұнайқұрылыс» тресінде, «Өзен» фирмасында және тағы басқа орындарда қызметте болған. 2006 жылдан бастап «Құбыр» ЖШС-нің директоры. «Өзен мұнай кен орнына 40 жыл», «Қазақстан мұнайына 100 жыл» белгілерінің иегері.} \\
& Describe the career path of the Kazakh mining engineer and former Akim of Tupkaragan district Alimgali Nurzhanuly Kapuov. & Alimgali Nurzhanuly Kapuov - mining engineer, former Akim of Tupkaragan district. He was born in 1948 in Fort Shevchenko. He graduated from KazPTI in 1972. In 1972-2005, he worked on various positions: in Zhetybai, Embimunay quarry, Mangistau Ulutas Production Association, Mangyshlak auylkurylys-12 trust, Mangistaumunaykurylys trust, Ozen firm and others. Since 2006, he has been the director of Kubyr LLP. He is the recipient of badges like "40 years of Uzen oil field", "100 years of Kazakhstan oil".  \\ 
\midrule
Culinary and Food & \foreignlanguage{russian}{Қазақтың ұлттық тағамы төстікті дайындау қадамдарын сипаттап беріңіз.} & \foreignlanguage{russian}{Төстікті дайындау үшін алдымен қойдың төс етін құйқасымен бірге алып, екіге бөледі. Оны істікке шаншып, сексеуіл шоғына қақтайды. Кейін төстікті терісімен бірге үйітіп, тұздайды. Бірнеше күн бойы күн тигізбей кептіреді, тұз сіңгеннен кейін отқа қақтап пісіреді.} \\
 & Describe the preparation steps of the Kazakh national dish tostik (brisket). & To prepare tostik, first take the lamb brisket with the skin and cut it in half. Poke it with a skewer and fry it on saxaul coal. Then singe the brisket together with the skin and salt it. It has to dry for several days with no sun exposure, and after the salt is absorbed, cook it on fire.  \\ 
\midrule
Cultural Institutions & \foreignlanguage{russian}{Қазақстандағы Отырар руханияты мұражай-кітапханасына енген көне заман кітаптары қалай жасалғанын түсіндіріңіз.} & \foreignlanguage{russian}{Көне заман кітаптары қолмен қауырсын қаламмен жазылып, Тараз ешкісінің терісіне түсірілген. Терілер қаттап, түйеге тиеп тасымалданған, әріптер өшпей, жүздеген жылдарға жетіп тұрған.} \\
 & Explain how ancient books, which are included in the Museum-Library of Otrar spirituality in Kazakhstan, were created. & Ancient books were written by hand with a quill pen and drawn on the skin of a Taraz goat. When the skins got hard, they were loaded on camels, the letters were not erased, they stayed for hundreds of years.  \\ 
\midrule
Culture and Traditions & \foreignlanguage{russian}{Қазақ халқының зооморфтық ою-өрнектері мен олар бейнелейтін элементтерді сипаттаңыз.} & \foreignlanguage{russian}{Қазақтың зооморфтық ою-өрнектері жануарлар әлемін бейнелейді. Олардың ішіндегі ең кең тарағаны – мүйіздерді бейнелеу. Мүйіз тәрізді доғалы «қошқар мүйіз», «мүйіз», «қос мүйіз», «сынық мүйіз» тәрізді өрнектер кездеседі. Сонымен қатар, «түйе мойын», «құс қанаты», «қаз мойны» сияқты жануарлардың басқа да дене мүшелері бейнеленеді. Бұл өрнектер үй заттарын, жиһаз бен қару-жарақ бұйымдарын сәндейді.} \\
& Describe the zoomorphic ornaments of the Kazakh people and the elements they represent. & Kazakh zoomorphic ornaments represent the animal world. The most common of them is the depiction of horns. There are horn-shaped arched patterns such as "koshkar muyiz" (ram's horn), "muyiz" (horn), "kos muyiz" (double horn), "synyk muyiz" (broken horn). In addition, other body parts of animals are depicted, such as "tuye moiyn" (camel neck),"kus kanaty" (bird wing),"kaz moiny" (goose neck). These patterns adorn household items, furniture and weapons.  \\ 
\midrule
Education and Academia & \foreignlanguage{russian}{Қазақстан тілтанушы ғалымы Ысқақов Ахмеди Ысқақұлының қазақ тілін дамытудағы үлесін түсіндіріңіз.} & \foreignlanguage{russian}{Ысқақов Ахмеди Ысқақұлы қазақ тілтануының тарихы мен грамматикасы, лексика және оқыту методикасы саласындағы зерттеулерімен танымал болды. Ол «Қазақ тілінің сөздігі» мен «Абай шығармаларының сөздігі» сияқты шығармаларды құрастырып, «Абай тілі сөздігін» (1968) басқарды. 200-ден астам ғылыми еңбектің авторы.} \\
 & Explain the contribution of Kazakh linguist Akhmedi Iskakov to the development of the Kazakh language. & Iskakov Akhmedi Iskakuly became famous for his research in the field of history and grammar of Kazakh linguistics, vocabulary and teaching methods. He wrote such works as the "Dictionary of the Kazakh language" and the "Dictionary of Abai's works", and headed the writing of the "Dictionary of the Abai language" (1968). He is the author of more than 200 scientific papers.  \\ 
\midrule
Health and Medicine & \foreignlanguage{russian}{Қазақстандық хирург пен медицина ғылымдарының докторы Абдоллаев Шәдібай Нұрғазыұлының кәсіби жетістіктерін сипаттаңыз.} & \foreignlanguage{russian}{Абдоллаев Шәдібай Нұрғазыұлы хирург, медицина ғылымдарының докторы және профессор. Ол Қазақ КСР-інің еңбек сіңірген дәрігері атағын алды. Бұл ғалымның негізгі ғылыми-зерттеулері жедел хирургияның мәселелеріне бағытталған, әсіресе, ерте кезеңдегі релапаротомия және жүкті әйелдерге хирургиялық көмек көрсету.} \\
 & Describe the professional achievements of the Kazakh surgeon and Doctor of Medical Sciences Abdollayev Shadibay Nurgazyuly. & Abdollayev Shadibay Nurgazyuly is a surgeon, Doctor of Medical Sciences and professor. He received the title of honored doctor of the Kazakh SSR. The main research of this scientist is focused on the problems of emergency surgery, especially relaparotomy at an early stage and surgical care for pregnant women.  \\ 
\midrule
Historical Figures & \foreignlanguage{russian}{Қазақ күйші-композиторы Махамбет Бөкейхановтың Қазақстан музыкалық мәдениетіне қосқан үлесін түсіндіріңіз.} & \foreignlanguage{russian}{Махамбет Бөкейханов қазақ күйлерін жаздырушы, композитор. Ол Құрманғазы, Дәулеткерей, Әлікей сияқты күйшілердің мұрасын біздің заманға жеткізіп, өзі де күй шығарған. 1933 жылы студенттердің халық аспаптары оркестрін ұйымдастырып, ұлттық музыка аспаптарын жетілдіру үшін шеберханада жұмыс істеді. Оның еңбегі арқылы көптеген қазақ күйлері заманымызға жетті.} \\
 & Explain the contribution of the Kazakh kuishi-composer Makhambet Bukeikhanov to the musical culture of Kazakhstan. & Makhambet Bokeikhanov is a composer of Kazakh kuis. He brought the legacy of such kui performers as Kurmangazy, Dauletkerey, Alikey to our time and wrote kui himself. In 1933 he organized a student orchestra of folk instruments and worked in a workshop to improve national musical instruments. Thanks to his work, many Kazakh kuis have reached our time.  \\ 
\bottomrule
\end{tabular} }
\caption{Example of created instructions on \texttt{CultSet}.}
\label{tab:example_cultset}
\end{table*}





\begin{table*}[b!]
\scriptsize
\centering
\scalebox{0.8}{
\begin{tabular}{p{0.15\textwidth}p{0.2\textwidth}p{0.7\textwidth}}
\toprule
Category & Instruction & Output \\
\midrule
History and Archaeology & \foreignlanguage{russian}{Қазақ халқының Албан тайпасының тарихи мекендеген жерлерін сипаттаңыз.} & \foreignlanguage{russian}{Қазақ халқының Албан тайпасы негізінен Жетісу өңірінде, әсіресе қазіргі Алматы облысының Кеген мен Нарынқол аудандарында мекендеген. Олар Іле өзенінің сол жағында, Қырғыз шекарасына дейінгі аймақта және Верный мен Жаркент уездерінде шоғырланған. Бұл аймақ Іле Алатауының биік жоталарына дейінгі жерлерді қамтиды.} \\
\\ \\ & Describe the historical habitats of the Albanian tribe of the Kazakh people. & The Albanian tribe of the Kazakh people lived mainly in the Zhetysu region, especially in the Kegen and Narynkol districts of the modern Almaty region. They were concentrated on the left side of the Ili River, in the region up to the Kyrgyz border and in the Verny and Zharkent counties. This region covers lands up to the high ridges of the Trans-Ili Alatau.  \\ 
\midrule
Literature and Writing & \foreignlanguage{russian}{Қазақ ақыны Абайдың қазақ қоғамындағы теріс пиғылдарды қалай сынға алғанын сипаттаңыз.} & \foreignlanguage{russian}{Абай қазақ қоғамындағы әділетсіздік, парақорлық, арамзалық және еріншектік сияқты теріс қылықтарды ашық сынға алады. Ол халықтың бір-бірін аңдып, ұрлық жасап, шын келісімге жете алмайтынына назар аударады. Ақын өнер, әдебиет және ғылымға көңіл бөліп, қоғамдық тәрбиеге баса мән береді.} \\
\\ \\ & Describe how the Kazakh poet Abai criticized the negative aspects of Kazakh society. & Abai openly criticizes such wrongdoings in Kazakh society as injustice, bribery, dishonesty and laziness. He draws attention to the fact that the people follow each other, steal and do not reach a true agreement. The poet pays attention to art, literature and science, and pays special attention to public education.  \\ 
\midrule
Media and Communication & \foreignlanguage{russian}{Гүлмира Әлібекқызы Қаракөзованың қазақ журналистика саласындағы жетістіктерін түсіндіріңіз.} & \foreignlanguage{russian}{Гүлмира Қаракөзова - тәжірибелі журналист. Ол «Возвращенные имена» (1997) арнайы репортаждары топтамасының, «Паритет» (2002-2004) тележобасының және «Жұлдызды Астана» (2005 жылдан) радиожобасының авторы. 2004 жылдан бастап «Астана» радиосын басқарады. «Мәдениет қайраткері» құрмет белгісімен (2006) және «Астана» медалімен (2008) марапатталған.} \\
\\ \\ & Explain the achievements of Gulmira Alibekkyzy Karakozova in the field of Kazakh journalism. & Gulmira Karakozova is an experienced journalist. She is the author of a series of special reports "Returning names" (1997), the TV project "The Parity" (2002-2004) and the Radio Project "Starry Astana" (since 2005). Since 2004, she has been leading Astana radio. She was awarded the Badge of Honor "Madeniyet kairatkeri" (2006) and the medal "Astana" (2008).  \\ 
\midrule
Miscellaneous & \foreignlanguage{russian}{Қазақстан заңгері мен еңбек сіңірген прокуроры Қабдолла Досбалаұлы Төлегеновтың өміріндегі маңызды қызметтерді сипаттаңыз.} & \foreignlanguage{russian}{Қабдолла Досбалаұлы Төлегенов 1952 жылы туған. Ол 1975 жылы Ташкент мемлекеттік университетінің заң факультетін бітірген. Әр жылдарда түрлі қызметтерді атқарған, соның ішінде Шымкент, Алматы облыстарының прокуроры, ҚР Қауіпсіздік кеңесінің Хатшылығы меңгерушісінің орынбасары және басқалары. 2008 жылдан бері Шығыс Қазақстан облысының прокуроры. Ол «Құрмет» орденімен және басқа медальдармен марапатталған.} \\
\\ \\ & Describe the most important job positions in the life of the lawyer and honored prosecutor of Kazakhstan Kabdulla Dosbalauly Tulegenov. & Kabdolla Dosbalauly Tolegenov was born in 1952. He graduated from the faculty of law of Tashkent State University in 1975. Over the years, he held various positions, including prosecutor of Shymkent, Almaty regions, deputy head of the Secretariat of the Security Council of the Republic of Kazakhstan and others. He is the prosecutor of the East Kazakhstan region since 2008. He was awarded the Order "Kurmet" and other medals.  \\ 
\midrule
Nature and Environment & \foreignlanguage{russian}{Қазақстандағы Көлсай көлдерінің ұлттық табиғи паркінде кездесетін ерекше өсімдіктерді сипаттаңыз.} & \foreignlanguage{russian}{Көлсай көлдерінің ауданында ерекше сирек кездесетін өсімдіктерден өрік, Іле барбарисі, долана, Альберт құртқашашы бар. Аудан экологиялық тұрғыдан таза және жабайы өсімдіктерге бай. Бұл өсімдіктердің көбі дәрілік қасиетке ие.} \\
\\ \\ & Describe the unique plants found in the Kolsai Lakes National Natural Park in Kazakhstan. & In the area of Kolsai lakes, especially rare plants can be found - apricot, Ili barbaris, hawthorn, Iris albertii. The area is ecologically clean and rich in wild plants. Many of these plants have medicinal properties.  \\ 
\midrule
Politics & \foreignlanguage{russian}{Қазақстандық қоғам қайраткері Нұрлан Жылқышиевтің өмірбаяндық фактілерін суреттеңіз.} & \foreignlanguage{russian}{Нұрлан Жылқышиев 1953 жылы 3 мамырда Тараз қаласында туған. Ол Шымкент педагогикалық институтын және Алматы ауыл шаруашылығы институтын бітірген. 1998 жылы АҚШ-тың Тынық мұхиты жағалауы университетінде философия докторы дәрежесін қорғаған. Қабыланбек агрошарап компаниясының бас директоры болып қызмет атқарған және 2004 жылдан парламенттің мәжіліс депутаты. Ол әртүрлі марапаттарға ие болған, соның ішінде 'Тың және тыңайған жерлерді игерудің 50 жылдығы' төсбелгісі.} \\
 & Describe the biographical facts of the Kazakh public figure Nurlan Zhylkyshiev. & Nurlan Zhylkyshiev was born on May 3, 1953 in Taraz. He graduated from the Shymkent Pedagogical Institute and the Almaty Agricultural Institute. In 1998, he defended his doctorate in philosophy at the University of the Pacific coast of the United States. He worked as the head director of the agro-processing company Kabylanbek and since 2004 has been a deputy of the Mazhilis of the Parliament. He has received various awards, including the "50th anniversary of the development of Virgin and fallow lands" badge.  \\ 
\midrule
Sciences and Humanities & \foreignlanguage{russian}{Еділ Ерғожаұлы Ерғожинның химия саласында Қазақстанның мемлекеттік сыйлығын алғаны рас па?} & \foreignlanguage{russian}{Иә, Еділ Ерғожаұлы Ерғожин химия ғылымының докторы, профессор, Қазақстанның мемлекеттік сыйлығының иегері.} \\
\\ \\ & Is it true that Edil Yergozhauly Yerhozhin received the State prize of Kazakhstan in the field of chemistry? & Yes, Edil Yergozhauly Yergozhin is a doctor of chemical sciences, professor, winner of the State prize of Kazakhstan.  \\ 
\midrule
Sports and Recreation & \foreignlanguage{russian}{Қазақстандық пауэрлифтер-паралимпиадшы Давид Дегтярев 2020 жылғы жазғы Паралимпиада ойындарында қандай жетістіктерге жетті?} & \foreignlanguage{russian}{Қазақ пауэрлифтері Давид Дегтярев 2020 жылғы жазғы Паралимпиада ойындарында 54 кг дейінгі салмақ дәрежесінде жарысқа қатысып, 170 кг, 172 кг және 174 кг салмақтарды көтеру арқылы алтын медаль иегері атанды.} \\
 & What did Kazakhstani powerlifter-paralympian David Degtyarev achieve at the 2020 Summer Paralympic Games? & Kazakh powerlifter David Degtyarev competed at the 2020 Summer Paralympic Games in the weight category up to 54 kg and became the holder of a gold medal by lifting weights 170 kg, 172 kg and 174 kg.  \\ 
\midrule
Transportation and Infrastructure & \foreignlanguage{russian}{Қазақстандағы Ақтау Халықаралық Әуежайының тарихын қысқаша сипаттаңыз.} & \foreignlanguage{russian}{Ақтау Халықаралық Әуежайы 1983 жылы ашылған және көптеген ірі жүк тиейтін әуе кемелерін қабылдауға жабдықталған. 1996 жылы акционерлік қоғам ретінде құрылды. 2007 жылы жаңа жолаушылар терминалын салу мен басқаруға келісім жасалды. 2009 жылы жаңа терминал пайдалануға берілді, ол сағатына 450 жолаушыға қызмет көрсетеді. Әуежай жаңғыртулар нәтижесінде ТМД елдері арасында үздік әуежай атағын алды.} \\
\\ \\ & Briefly describe the history of Aktau International Airport in Kazakhstan. & Aktau International Airport was opened in 1983 and is equipped to receive many large cargo aircraft. It was established in 1996 as a joint stock company. In 2007, an agreement was signed for the construction and management of a new passenger terminal. In 2009, a new terminal was put into operation, serving 450 passengers per hour. As a result of the modernization, the airport received the title of the best airport among the CIS countries.  \\ 

\bottomrule
\end{tabular} }
\caption{Example of created instructions on \texttt{CultSet}.}
\end{table*}

%\include{tables/cohen_appendix.tex}
%\subsection{Lloyd-Max Algorithm}
\label{subsec:Lloyd-Max}
For a given quantization bitwidth $B$ and an operand $\bm{X}$, the Lloyd-Max algorithm finds $2^B$ quantization levels $\{\hat{x}_i\}_{i=1}^{2^B}$ such that quantizing $\bm{X}$ by rounding each scalar in $\bm{X}$ to the nearest quantization level minimizes the quantization MSE. 

The algorithm starts with an initial guess of quantization levels and then iteratively computes quantization thresholds $\{\tau_i\}_{i=1}^{2^B-1}$ and updates quantization levels $\{\hat{x}_i\}_{i=1}^{2^B}$. Specifically, at iteration $n$, thresholds are set to the midpoints of the previous iteration's levels:
\begin{align*}
    \tau_i^{(n)}=\frac{\hat{x}_i^{(n-1)}+\hat{x}_{i+1}^{(n-1)}}2 \text{ for } i=1\ldots 2^B-1
\end{align*}
Subsequently, the quantization levels are re-computed as conditional means of the data regions defined by the new thresholds:
\begin{align*}
    \hat{x}_i^{(n)}=\mathbb{E}\left[ \bm{X} \big| \bm{X}\in [\tau_{i-1}^{(n)},\tau_i^{(n)}] \right] \text{ for } i=1\ldots 2^B
\end{align*}
where to satisfy boundary conditions we have $\tau_0=-\infty$ and $\tau_{2^B}=\infty$. The algorithm iterates the above steps until convergence.

Figure \ref{fig:lm_quant} compares the quantization levels of a $7$-bit floating point (E3M3) quantizer (left) to a $7$-bit Lloyd-Max quantizer (right) when quantizing a layer of weights from the GPT3-126M model at a per-tensor granularity. As shown, the Lloyd-Max quantizer achieves substantially lower quantization MSE. Further, Table \ref{tab:FP7_vs_LM7} shows the superior perplexity achieved by Lloyd-Max quantizers for bitwidths of $7$, $6$ and $5$. The difference between the quantizers is clear at 5 bits, where per-tensor FP quantization incurs a drastic and unacceptable increase in perplexity, while Lloyd-Max quantization incurs a much smaller increase. Nevertheless, we note that even the optimal Lloyd-Max quantizer incurs a notable ($\sim 1.5$) increase in perplexity due to the coarse granularity of quantization. 

\begin{figure}[h]
  \centering
  \includegraphics[width=0.7\linewidth]{sections/figures/LM7_FP7.pdf}
  \caption{\small Quantization levels and the corresponding quantization MSE of Floating Point (left) vs Lloyd-Max (right) Quantizers for a layer of weights in the GPT3-126M model.}
  \label{fig:lm_quant}
\end{figure}

\begin{table}[h]\scriptsize
\begin{center}
\caption{\label{tab:FP7_vs_LM7} \small Comparing perplexity (lower is better) achieved by floating point quantizers and Lloyd-Max quantizers on a GPT3-126M model for the Wikitext-103 dataset.}
\begin{tabular}{c|cc|c}
\hline
 \multirow{2}{*}{\textbf{Bitwidth}} & \multicolumn{2}{|c|}{\textbf{Floating-Point Quantizer}} & \textbf{Lloyd-Max Quantizer} \\
 & Best Format & Wikitext-103 Perplexity & Wikitext-103 Perplexity \\
\hline
7 & E3M3 & 18.32 & 18.27 \\
6 & E3M2 & 19.07 & 18.51 \\
5 & E4M0 & 43.89 & 19.71 \\
\hline
\end{tabular}
\end{center}
\end{table}

\subsection{Proof of Local Optimality of LO-BCQ}
\label{subsec:lobcq_opt_proof}
For a given block $\bm{b}_j$, the quantization MSE during LO-BCQ can be empirically evaluated as $\frac{1}{L_b}\lVert \bm{b}_j- \bm{\hat{b}}_j\rVert^2_2$ where $\bm{\hat{b}}_j$ is computed from equation (\ref{eq:clustered_quantization_definition}) as $C_{f(\bm{b}_j)}(\bm{b}_j)$. Further, for a given block cluster $\mathcal{B}_i$, we compute the quantization MSE as $\frac{1}{|\mathcal{B}_{i}|}\sum_{\bm{b} \in \mathcal{B}_{i}} \frac{1}{L_b}\lVert \bm{b}- C_i^{(n)}(\bm{b})\rVert^2_2$. Therefore, at the end of iteration $n$, we evaluate the overall quantization MSE $J^{(n)}$ for a given operand $\bm{X}$ composed of $N_c$ block clusters as:
\begin{align*}
    \label{eq:mse_iter_n}
    J^{(n)} = \frac{1}{N_c} \sum_{i=1}^{N_c} \frac{1}{|\mathcal{B}_{i}^{(n)}|}\sum_{\bm{v} \in \mathcal{B}_{i}^{(n)}} \frac{1}{L_b}\lVert \bm{b}- B_i^{(n)}(\bm{b})\rVert^2_2
\end{align*}

At the end of iteration $n$, the codebooks are updated from $\mathcal{C}^{(n-1)}$ to $\mathcal{C}^{(n)}$. However, the mapping of a given vector $\bm{b}_j$ to quantizers $\mathcal{C}^{(n)}$ remains as  $f^{(n)}(\bm{b}_j)$. At the next iteration, during the vector clustering step, $f^{(n+1)}(\bm{b}_j)$ finds new mapping of $\bm{b}_j$ to updated codebooks $\mathcal{C}^{(n)}$ such that the quantization MSE over the candidate codebooks is minimized. Therefore, we obtain the following result for $\bm{b}_j$:
\begin{align*}
\frac{1}{L_b}\lVert \bm{b}_j - C_{f^{(n+1)}(\bm{b}_j)}^{(n)}(\bm{b}_j)\rVert^2_2 \le \frac{1}{L_b}\lVert \bm{b}_j - C_{f^{(n)}(\bm{b}_j)}^{(n)}(\bm{b}_j)\rVert^2_2
\end{align*}

That is, quantizing $\bm{b}_j$ at the end of the block clustering step of iteration $n+1$ results in lower quantization MSE compared to quantizing at the end of iteration $n$. Since this is true for all $\bm{b} \in \bm{X}$, we assert the following:
\begin{equation}
\begin{split}
\label{eq:mse_ineq_1}
    \tilde{J}^{(n+1)} &= \frac{1}{N_c} \sum_{i=1}^{N_c} \frac{1}{|\mathcal{B}_{i}^{(n+1)}|}\sum_{\bm{b} \in \mathcal{B}_{i}^{(n+1)}} \frac{1}{L_b}\lVert \bm{b} - C_i^{(n)}(b)\rVert^2_2 \le J^{(n)}
\end{split}
\end{equation}
where $\tilde{J}^{(n+1)}$ is the the quantization MSE after the vector clustering step at iteration $n+1$.

Next, during the codebook update step (\ref{eq:quantizers_update}) at iteration $n+1$, the per-cluster codebooks $\mathcal{C}^{(n)}$ are updated to $\mathcal{C}^{(n+1)}$ by invoking the Lloyd-Max algorithm \citep{Lloyd}. We know that for any given value distribution, the Lloyd-Max algorithm minimizes the quantization MSE. Therefore, for a given vector cluster $\mathcal{B}_i$ we obtain the following result:

\begin{equation}
    \frac{1}{|\mathcal{B}_{i}^{(n+1)}|}\sum_{\bm{b} \in \mathcal{B}_{i}^{(n+1)}} \frac{1}{L_b}\lVert \bm{b}- C_i^{(n+1)}(\bm{b})\rVert^2_2 \le \frac{1}{|\mathcal{B}_{i}^{(n+1)}|}\sum_{\bm{b} \in \mathcal{B}_{i}^{(n+1)}} \frac{1}{L_b}\lVert \bm{b}- C_i^{(n)}(\bm{b})\rVert^2_2
\end{equation}

The above equation states that quantizing the given block cluster $\mathcal{B}_i$ after updating the associated codebook from $C_i^{(n)}$ to $C_i^{(n+1)}$ results in lower quantization MSE. Since this is true for all the block clusters, we derive the following result: 
\begin{equation}
\begin{split}
\label{eq:mse_ineq_2}
     J^{(n+1)} &= \frac{1}{N_c} \sum_{i=1}^{N_c} \frac{1}{|\mathcal{B}_{i}^{(n+1)}|}\sum_{\bm{b} \in \mathcal{B}_{i}^{(n+1)}} \frac{1}{L_b}\lVert \bm{b}- C_i^{(n+1)}(\bm{b})\rVert^2_2  \le \tilde{J}^{(n+1)}   
\end{split}
\end{equation}

Following (\ref{eq:mse_ineq_1}) and (\ref{eq:mse_ineq_2}), we find that the quantization MSE is non-increasing for each iteration, that is, $J^{(1)} \ge J^{(2)} \ge J^{(3)} \ge \ldots \ge J^{(M)}$ where $M$ is the maximum number of iterations. 
%Therefore, we can say that if the algorithm converges, then it must be that it has converged to a local minimum. 
\hfill $\blacksquare$


\begin{figure}
    \begin{center}
    \includegraphics[width=0.5\textwidth]{sections//figures/mse_vs_iter.pdf}
    \end{center}
    \caption{\small NMSE vs iterations during LO-BCQ compared to other block quantization proposals}
    \label{fig:nmse_vs_iter}
\end{figure}

Figure \ref{fig:nmse_vs_iter} shows the empirical convergence of LO-BCQ across several block lengths and number of codebooks. Also, the MSE achieved by LO-BCQ is compared to baselines such as MXFP and VSQ. As shown, LO-BCQ converges to a lower MSE than the baselines. Further, we achieve better convergence for larger number of codebooks ($N_c$) and for a smaller block length ($L_b$), both of which increase the bitwidth of BCQ (see Eq \ref{eq:bitwidth_bcq}).


\subsection{Additional Accuracy Results}
%Table \ref{tab:lobcq_config} lists the various LOBCQ configurations and their corresponding bitwidths.
\begin{table}
\setlength{\tabcolsep}{4.75pt}
\begin{center}
\caption{\label{tab:lobcq_config} Various LO-BCQ configurations and their bitwidths.}
\begin{tabular}{|c||c|c|c|c||c|c||c|} 
\hline
 & \multicolumn{4}{|c||}{$L_b=8$} & \multicolumn{2}{|c||}{$L_b=4$} & $L_b=2$ \\
 \hline
 \backslashbox{$L_A$\kern-1em}{\kern-1em$N_c$} & 2 & 4 & 8 & 16 & 2 & 4 & 2 \\
 \hline
 64 & 4.25 & 4.375 & 4.5 & 4.625 & 4.375 & 4.625 & 4.625\\
 \hline
 32 & 4.375 & 4.5 & 4.625& 4.75 & 4.5 & 4.75 & 4.75 \\
 \hline
 16 & 4.625 & 4.75& 4.875 & 5 & 4.75 & 5 & 5 \\
 \hline
\end{tabular}
\end{center}
\end{table}

%\subsection{Perplexity achieved by various LO-BCQ configurations on Wikitext-103 dataset}

\begin{table} \centering
\begin{tabular}{|c||c|c|c|c||c|c||c|} 
\hline
 $L_b \rightarrow$& \multicolumn{4}{c||}{8} & \multicolumn{2}{c||}{4} & 2\\
 \hline
 \backslashbox{$L_A$\kern-1em}{\kern-1em$N_c$} & 2 & 4 & 8 & 16 & 2 & 4 & 2  \\
 %$N_c \rightarrow$ & 2 & 4 & 8 & 16 & 2 & 4 & 2 \\
 \hline
 \hline
 \multicolumn{8}{c}{GPT3-1.3B (FP32 PPL = 9.98)} \\ 
 \hline
 \hline
 64 & 10.40 & 10.23 & 10.17 & 10.15 &  10.28 & 10.18 & 10.19 \\
 \hline
 32 & 10.25 & 10.20 & 10.15 & 10.12 &  10.23 & 10.17 & 10.17 \\
 \hline
 16 & 10.22 & 10.16 & 10.10 & 10.09 &  10.21 & 10.14 & 10.16 \\
 \hline
  \hline
 \multicolumn{8}{c}{GPT3-8B (FP32 PPL = 7.38)} \\ 
 \hline
 \hline
 64 & 7.61 & 7.52 & 7.48 &  7.47 &  7.55 &  7.49 & 7.50 \\
 \hline
 32 & 7.52 & 7.50 & 7.46 &  7.45 &  7.52 &  7.48 & 7.48  \\
 \hline
 16 & 7.51 & 7.48 & 7.44 &  7.44 &  7.51 &  7.49 & 7.47  \\
 \hline
\end{tabular}
\caption{\label{tab:ppl_gpt3_abalation} Wikitext-103 perplexity across GPT3-1.3B and 8B models.}
\end{table}

\begin{table} \centering
\begin{tabular}{|c||c|c|c|c||} 
\hline
 $L_b \rightarrow$& \multicolumn{4}{c||}{8}\\
 \hline
 \backslashbox{$L_A$\kern-1em}{\kern-1em$N_c$} & 2 & 4 & 8 & 16 \\
 %$N_c \rightarrow$ & 2 & 4 & 8 & 16 & 2 & 4 & 2 \\
 \hline
 \hline
 \multicolumn{5}{|c|}{Llama2-7B (FP32 PPL = 5.06)} \\ 
 \hline
 \hline
 64 & 5.31 & 5.26 & 5.19 & 5.18  \\
 \hline
 32 & 5.23 & 5.25 & 5.18 & 5.15  \\
 \hline
 16 & 5.23 & 5.19 & 5.16 & 5.14  \\
 \hline
 \multicolumn{5}{|c|}{Nemotron4-15B (FP32 PPL = 5.87)} \\ 
 \hline
 \hline
 64  & 6.3 & 6.20 & 6.13 & 6.08  \\
 \hline
 32  & 6.24 & 6.12 & 6.07 & 6.03  \\
 \hline
 16  & 6.12 & 6.14 & 6.04 & 6.02  \\
 \hline
 \multicolumn{5}{|c|}{Nemotron4-340B (FP32 PPL = 3.48)} \\ 
 \hline
 \hline
 64 & 3.67 & 3.62 & 3.60 & 3.59 \\
 \hline
 32 & 3.63 & 3.61 & 3.59 & 3.56 \\
 \hline
 16 & 3.61 & 3.58 & 3.57 & 3.55 \\
 \hline
\end{tabular}
\caption{\label{tab:ppl_llama7B_nemo15B} Wikitext-103 perplexity compared to FP32 baseline in Llama2-7B and Nemotron4-15B, 340B models}
\end{table}

%\subsection{Perplexity achieved by various LO-BCQ configurations on MMLU dataset}


\begin{table} \centering
\begin{tabular}{|c||c|c|c|c||c|c|c|c|} 
\hline
 $L_b \rightarrow$& \multicolumn{4}{c||}{8} & \multicolumn{4}{c||}{8}\\
 \hline
 \backslashbox{$L_A$\kern-1em}{\kern-1em$N_c$} & 2 & 4 & 8 & 16 & 2 & 4 & 8 & 16  \\
 %$N_c \rightarrow$ & 2 & 4 & 8 & 16 & 2 & 4 & 2 \\
 \hline
 \hline
 \multicolumn{5}{|c|}{Llama2-7B (FP32 Accuracy = 45.8\%)} & \multicolumn{4}{|c|}{Llama2-70B (FP32 Accuracy = 69.12\%)} \\ 
 \hline
 \hline
 64 & 43.9 & 43.4 & 43.9 & 44.9 & 68.07 & 68.27 & 68.17 & 68.75 \\
 \hline
 32 & 44.5 & 43.8 & 44.9 & 44.5 & 68.37 & 68.51 & 68.35 & 68.27  \\
 \hline
 16 & 43.9 & 42.7 & 44.9 & 45 & 68.12 & 68.77 & 68.31 & 68.59  \\
 \hline
 \hline
 \multicolumn{5}{|c|}{GPT3-22B (FP32 Accuracy = 38.75\%)} & \multicolumn{4}{|c|}{Nemotron4-15B (FP32 Accuracy = 64.3\%)} \\ 
 \hline
 \hline
 64 & 36.71 & 38.85 & 38.13 & 38.92 & 63.17 & 62.36 & 63.72 & 64.09 \\
 \hline
 32 & 37.95 & 38.69 & 39.45 & 38.34 & 64.05 & 62.30 & 63.8 & 64.33  \\
 \hline
 16 & 38.88 & 38.80 & 38.31 & 38.92 & 63.22 & 63.51 & 63.93 & 64.43  \\
 \hline
\end{tabular}
\caption{\label{tab:mmlu_abalation} Accuracy on MMLU dataset across GPT3-22B, Llama2-7B, 70B and Nemotron4-15B models.}
\end{table}


%\subsection{Perplexity achieved by various LO-BCQ configurations on LM evaluation harness}

\begin{table} \centering
\begin{tabular}{|c||c|c|c|c||c|c|c|c|} 
\hline
 $L_b \rightarrow$& \multicolumn{4}{c||}{8} & \multicolumn{4}{c||}{8}\\
 \hline
 \backslashbox{$L_A$\kern-1em}{\kern-1em$N_c$} & 2 & 4 & 8 & 16 & 2 & 4 & 8 & 16  \\
 %$N_c \rightarrow$ & 2 & 4 & 8 & 16 & 2 & 4 & 2 \\
 \hline
 \hline
 \multicolumn{5}{|c|}{Race (FP32 Accuracy = 37.51\%)} & \multicolumn{4}{|c|}{Boolq (FP32 Accuracy = 64.62\%)} \\ 
 \hline
 \hline
 64 & 36.94 & 37.13 & 36.27 & 37.13 & 63.73 & 62.26 & 63.49 & 63.36 \\
 \hline
 32 & 37.03 & 36.36 & 36.08 & 37.03 & 62.54 & 63.51 & 63.49 & 63.55  \\
 \hline
 16 & 37.03 & 37.03 & 36.46 & 37.03 & 61.1 & 63.79 & 63.58 & 63.33  \\
 \hline
 \hline
 \multicolumn{5}{|c|}{Winogrande (FP32 Accuracy = 58.01\%)} & \multicolumn{4}{|c|}{Piqa (FP32 Accuracy = 74.21\%)} \\ 
 \hline
 \hline
 64 & 58.17 & 57.22 & 57.85 & 58.33 & 73.01 & 73.07 & 73.07 & 72.80 \\
 \hline
 32 & 59.12 & 58.09 & 57.85 & 58.41 & 73.01 & 73.94 & 72.74 & 73.18  \\
 \hline
 16 & 57.93 & 58.88 & 57.93 & 58.56 & 73.94 & 72.80 & 73.01 & 73.94  \\
 \hline
\end{tabular}
\caption{\label{tab:mmlu_abalation} Accuracy on LM evaluation harness tasks on GPT3-1.3B model.}
\end{table}

\begin{table} \centering
\begin{tabular}{|c||c|c|c|c||c|c|c|c|} 
\hline
 $L_b \rightarrow$& \multicolumn{4}{c||}{8} & \multicolumn{4}{c||}{8}\\
 \hline
 \backslashbox{$L_A$\kern-1em}{\kern-1em$N_c$} & 2 & 4 & 8 & 16 & 2 & 4 & 8 & 16  \\
 %$N_c \rightarrow$ & 2 & 4 & 8 & 16 & 2 & 4 & 2 \\
 \hline
 \hline
 \multicolumn{5}{|c|}{Race (FP32 Accuracy = 41.34\%)} & \multicolumn{4}{|c|}{Boolq (FP32 Accuracy = 68.32\%)} \\ 
 \hline
 \hline
 64 & 40.48 & 40.10 & 39.43 & 39.90 & 69.20 & 68.41 & 69.45 & 68.56 \\
 \hline
 32 & 39.52 & 39.52 & 40.77 & 39.62 & 68.32 & 67.43 & 68.17 & 69.30  \\
 \hline
 16 & 39.81 & 39.71 & 39.90 & 40.38 & 68.10 & 66.33 & 69.51 & 69.42  \\
 \hline
 \hline
 \multicolumn{5}{|c|}{Winogrande (FP32 Accuracy = 67.88\%)} & \multicolumn{4}{|c|}{Piqa (FP32 Accuracy = 78.78\%)} \\ 
 \hline
 \hline
 64 & 66.85 & 66.61 & 67.72 & 67.88 & 77.31 & 77.42 & 77.75 & 77.64 \\
 \hline
 32 & 67.25 & 67.72 & 67.72 & 67.00 & 77.31 & 77.04 & 77.80 & 77.37  \\
 \hline
 16 & 68.11 & 68.90 & 67.88 & 67.48 & 77.37 & 78.13 & 78.13 & 77.69  \\
 \hline
\end{tabular}
\caption{\label{tab:mmlu_abalation} Accuracy on LM evaluation harness tasks on GPT3-8B model.}
\end{table}

\begin{table} \centering
\begin{tabular}{|c||c|c|c|c||c|c|c|c|} 
\hline
 $L_b \rightarrow$& \multicolumn{4}{c||}{8} & \multicolumn{4}{c||}{8}\\
 \hline
 \backslashbox{$L_A$\kern-1em}{\kern-1em$N_c$} & 2 & 4 & 8 & 16 & 2 & 4 & 8 & 16  \\
 %$N_c \rightarrow$ & 2 & 4 & 8 & 16 & 2 & 4 & 2 \\
 \hline
 \hline
 \multicolumn{5}{|c|}{Race (FP32 Accuracy = 40.67\%)} & \multicolumn{4}{|c|}{Boolq (FP32 Accuracy = 76.54\%)} \\ 
 \hline
 \hline
 64 & 40.48 & 40.10 & 39.43 & 39.90 & 75.41 & 75.11 & 77.09 & 75.66 \\
 \hline
 32 & 39.52 & 39.52 & 40.77 & 39.62 & 76.02 & 76.02 & 75.96 & 75.35  \\
 \hline
 16 & 39.81 & 39.71 & 39.90 & 40.38 & 75.05 & 73.82 & 75.72 & 76.09  \\
 \hline
 \hline
 \multicolumn{5}{|c|}{Winogrande (FP32 Accuracy = 70.64\%)} & \multicolumn{4}{|c|}{Piqa (FP32 Accuracy = 79.16\%)} \\ 
 \hline
 \hline
 64 & 69.14 & 70.17 & 70.17 & 70.56 & 78.24 & 79.00 & 78.62 & 78.73 \\
 \hline
 32 & 70.96 & 69.69 & 71.27 & 69.30 & 78.56 & 79.49 & 79.16 & 78.89  \\
 \hline
 16 & 71.03 & 69.53 & 69.69 & 70.40 & 78.13 & 79.16 & 79.00 & 79.00  \\
 \hline
\end{tabular}
\caption{\label{tab:mmlu_abalation} Accuracy on LM evaluation harness tasks on GPT3-22B model.}
\end{table}

\begin{table} \centering
\begin{tabular}{|c||c|c|c|c||c|c|c|c|} 
\hline
 $L_b \rightarrow$& \multicolumn{4}{c||}{8} & \multicolumn{4}{c||}{8}\\
 \hline
 \backslashbox{$L_A$\kern-1em}{\kern-1em$N_c$} & 2 & 4 & 8 & 16 & 2 & 4 & 8 & 16  \\
 %$N_c \rightarrow$ & 2 & 4 & 8 & 16 & 2 & 4 & 2 \\
 \hline
 \hline
 \multicolumn{5}{|c|}{Race (FP32 Accuracy = 44.4\%)} & \multicolumn{4}{|c|}{Boolq (FP32 Accuracy = 79.29\%)} \\ 
 \hline
 \hline
 64 & 42.49 & 42.51 & 42.58 & 43.45 & 77.58 & 77.37 & 77.43 & 78.1 \\
 \hline
 32 & 43.35 & 42.49 & 43.64 & 43.73 & 77.86 & 75.32 & 77.28 & 77.86  \\
 \hline
 16 & 44.21 & 44.21 & 43.64 & 42.97 & 78.65 & 77 & 76.94 & 77.98  \\
 \hline
 \hline
 \multicolumn{5}{|c|}{Winogrande (FP32 Accuracy = 69.38\%)} & \multicolumn{4}{|c|}{Piqa (FP32 Accuracy = 78.07\%)} \\ 
 \hline
 \hline
 64 & 68.9 & 68.43 & 69.77 & 68.19 & 77.09 & 76.82 & 77.09 & 77.86 \\
 \hline
 32 & 69.38 & 68.51 & 68.82 & 68.90 & 78.07 & 76.71 & 78.07 & 77.86  \\
 \hline
 16 & 69.53 & 67.09 & 69.38 & 68.90 & 77.37 & 77.8 & 77.91 & 77.69  \\
 \hline
\end{tabular}
\caption{\label{tab:mmlu_abalation} Accuracy on LM evaluation harness tasks on Llama2-7B model.}
\end{table}

\begin{table} \centering
\begin{tabular}{|c||c|c|c|c||c|c|c|c|} 
\hline
 $L_b \rightarrow$& \multicolumn{4}{c||}{8} & \multicolumn{4}{c||}{8}\\
 \hline
 \backslashbox{$L_A$\kern-1em}{\kern-1em$N_c$} & 2 & 4 & 8 & 16 & 2 & 4 & 8 & 16  \\
 %$N_c \rightarrow$ & 2 & 4 & 8 & 16 & 2 & 4 & 2 \\
 \hline
 \hline
 \multicolumn{5}{|c|}{Race (FP32 Accuracy = 48.8\%)} & \multicolumn{4}{|c|}{Boolq (FP32 Accuracy = 85.23\%)} \\ 
 \hline
 \hline
 64 & 49.00 & 49.00 & 49.28 & 48.71 & 82.82 & 84.28 & 84.03 & 84.25 \\
 \hline
 32 & 49.57 & 48.52 & 48.33 & 49.28 & 83.85 & 84.46 & 84.31 & 84.93  \\
 \hline
 16 & 49.85 & 49.09 & 49.28 & 48.99 & 85.11 & 84.46 & 84.61 & 83.94  \\
 \hline
 \hline
 \multicolumn{5}{|c|}{Winogrande (FP32 Accuracy = 79.95\%)} & \multicolumn{4}{|c|}{Piqa (FP32 Accuracy = 81.56\%)} \\ 
 \hline
 \hline
 64 & 78.77 & 78.45 & 78.37 & 79.16 & 81.45 & 80.69 & 81.45 & 81.5 \\
 \hline
 32 & 78.45 & 79.01 & 78.69 & 80.66 & 81.56 & 80.58 & 81.18 & 81.34  \\
 \hline
 16 & 79.95 & 79.56 & 79.79 & 79.72 & 81.28 & 81.66 & 81.28 & 80.96  \\
 \hline
\end{tabular}
\caption{\label{tab:mmlu_abalation} Accuracy on LM evaluation harness tasks on Llama2-70B model.}
\end{table}

%\section{MSE Studies}
%\textcolor{red}{TODO}


\subsection{Number Formats and Quantization Method}
\label{subsec:numFormats_quantMethod}
\subsubsection{Integer Format}
An $n$-bit signed integer (INT) is typically represented with a 2s-complement format \citep{yao2022zeroquant,xiao2023smoothquant,dai2021vsq}, where the most significant bit denotes the sign.

\subsubsection{Floating Point Format}
An $n$-bit signed floating point (FP) number $x$ comprises of a 1-bit sign ($x_{\mathrm{sign}}$), $B_m$-bit mantissa ($x_{\mathrm{mant}}$) and $B_e$-bit exponent ($x_{\mathrm{exp}}$) such that $B_m+B_e=n-1$. The associated constant exponent bias ($E_{\mathrm{bias}}$) is computed as $(2^{{B_e}-1}-1)$. We denote this format as $E_{B_e}M_{B_m}$.  

\subsubsection{Quantization Scheme}
\label{subsec:quant_method}
A quantization scheme dictates how a given unquantized tensor is converted to its quantized representation. We consider FP formats for the purpose of illustration. Given an unquantized tensor $\bm{X}$ and an FP format $E_{B_e}M_{B_m}$, we first, we compute the quantization scale factor $s_X$ that maps the maximum absolute value of $\bm{X}$ to the maximum quantization level of the $E_{B_e}M_{B_m}$ format as follows:
\begin{align}
\label{eq:sf}
    s_X = \frac{\mathrm{max}(|\bm{X}|)}{\mathrm{max}(E_{B_e}M_{B_m})}
\end{align}
In the above equation, $|\cdot|$ denotes the absolute value function.

Next, we scale $\bm{X}$ by $s_X$ and quantize it to $\hat{\bm{X}}$ by rounding it to the nearest quantization level of $E_{B_e}M_{B_m}$ as:

\begin{align}
\label{eq:tensor_quant}
    \hat{\bm{X}} = \text{round-to-nearest}\left(\frac{\bm{X}}{s_X}, E_{B_e}M_{B_m}\right)
\end{align}

We perform dynamic max-scaled quantization \citep{wu2020integer}, where the scale factor $s$ for activations is dynamically computed during runtime.

\subsection{Vector Scaled Quantization}
\begin{wrapfigure}{r}{0.35\linewidth}
  \centering
  \includegraphics[width=\linewidth]{sections/figures/vsquant.jpg}
  \caption{\small Vectorwise decomposition for per-vector scaled quantization (VSQ \citep{dai2021vsq}).}
  \label{fig:vsquant}
\end{wrapfigure}
During VSQ \citep{dai2021vsq}, the operand tensors are decomposed into 1D vectors in a hardware friendly manner as shown in Figure \ref{fig:vsquant}. Since the decomposed tensors are used as operands in matrix multiplications during inference, it is beneficial to perform this decomposition along the reduction dimension of the multiplication. The vectorwise quantization is performed similar to tensorwise quantization described in Equations \ref{eq:sf} and \ref{eq:tensor_quant}, where a scale factor $s_v$ is required for each vector $\bm{v}$ that maps the maximum absolute value of that vector to the maximum quantization level. While smaller vector lengths can lead to larger accuracy gains, the associated memory and computational overheads due to the per-vector scale factors increases. To alleviate these overheads, VSQ \citep{dai2021vsq} proposed a second level quantization of the per-vector scale factors to unsigned integers, while MX \citep{rouhani2023shared} quantizes them to integer powers of 2 (denoted as $2^{INT}$).

\subsubsection{MX Format}
The MX format proposed in \citep{rouhani2023microscaling} introduces the concept of sub-block shifting. For every two scalar elements of $b$-bits each, there is a shared exponent bit. The value of this exponent bit is determined through an empirical analysis that targets minimizing quantization MSE. We note that the FP format $E_{1}M_{b}$ is strictly better than MX from an accuracy perspective since it allocates a dedicated exponent bit to each scalar as opposed to sharing it across two scalars. Therefore, we conservatively bound the accuracy of a $b+2$-bit signed MX format with that of a $E_{1}M_{b}$ format in our comparisons. For instance, we use E1M2 format as a proxy for MX4.

\begin{figure}
    \centering
    \includegraphics[width=1\linewidth]{sections//figures/BlockFormats.pdf}
    \caption{\small Comparing LO-BCQ to MX format.}
    \label{fig:block_formats}
\end{figure}

Figure \ref{fig:block_formats} compares our $4$-bit LO-BCQ block format to MX \citep{rouhani2023microscaling}. As shown, both LO-BCQ and MX decompose a given operand tensor into block arrays and each block array into blocks. Similar to MX, we find that per-block quantization ($L_b < L_A$) leads to better accuracy due to increased flexibility. While MX achieves this through per-block $1$-bit micro-scales, we associate a dedicated codebook to each block through a per-block codebook selector. Further, MX quantizes the per-block array scale-factor to E8M0 format without per-tensor scaling. In contrast during LO-BCQ, we find that per-tensor scaling combined with quantization of per-block array scale-factor to E4M3 format results in superior inference accuracy across models. 






\end{document}
