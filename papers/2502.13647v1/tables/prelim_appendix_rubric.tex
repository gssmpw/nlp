
\subsection{Human Evaluation for Preliminary Study Rubric}
\FloatBarrier

\begin{table*}[h]
\centering
\renewcommand{\arraystretch}{1.5} % Adjust row height
\scriptsize % Reduce font size
\begin{tabular}{>{\centering\arraybackslash}p{1.5cm}|>{\centering\arraybackslash}p{3.5cm}|>{\centering\arraybackslash}p{3.5cm}|>{\centering\arraybackslash}p{3.5cm}}
\hline
\textbf{Score} & \textbf{Correctness} & \textbf{Fluency} & \textbf{Completeness} \\ \hline
1 & Highly inaccurate, incorrect, or misleading information. & Very poor fluency, riddled with errors, making it difficult to read. & Very incomplete, with critical information missing, making it unusable. \\ \hline
2 & Significant factual or logical errors that impact the correctness of the instruction/question. & Multiple errors that hinder readability or cause confusion. & Significant omissions that make the instruction/question incomplete or difficult to interpret.  \\ \hline
3 & Noticeable errors in facts or logical flow, but the general meaning remains clear.
& Noticeable errors in grammar, spelling, or structure, but the text is still understandable. & Noticeable gaps in information or coverage that leave the instruction/question lacking. \\ \hline
4 & Minor factual inaccuracies or inconsistencies that do not affect overall understanding. & Minor grammatical or stylistic errors that do not significantly affect readability. & Slightly incomplete, with minor missing details that do not impact overall understanding. \\ \hline
5 & Fully correct and consistent with Kazakh cultural and governmental contexts, no factual or logical errors. & Perfect fluency, no errors in grammar, spelling, or sentence structure. The text reads smoothly and naturally. & Fully complete, no information is missing, and the instruction/question thoroughly covers the context.\\ \hline

\end{tabular}
\caption{Human annotation rubric for preliminary studies.}
\label{tab:multilingual-issues}
\end{table*}

\FloatBarrier



