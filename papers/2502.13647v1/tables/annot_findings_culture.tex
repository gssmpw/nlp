\begin{table*}[ht]
\scriptsize
\begin{tabular}{@{}c@{\hspace{10pt}}p{0.88\textwidth}@{}}
\toprule
\textbf{Error Type} & \textbf{Annotator Comments} \\ 
\midrule 
G & Inconsistent verb forms are present in the generated data. For example, \foreignlanguage{russian}{\textit{түсіндір}} is used instead of \foreignlanguage{russian}{\textit{түсіндіріңіз}}. \\
\midrule
G & The generated data includes phrases such as "\foreignlanguage{russian}{\textit{... туралы сипаттаңыз}}," which is grammatically incorrect because \foreignlanguage{russian}{\textit{туралы}} is redundant. \\
\midrule
S & Phrases like "\foreignlanguage{russian}{\textit{... қалай сипатталады?}}" appear awkward and imply the presence of external text for context. \\
\midrule
L & The word \foreignlanguage{russian}{\textit{карьера}} is used instead of the correct Kazakh term \foreignlanguage{russian}{\textit{мансап}}. \\
\midrule
G & The word \foreignlanguage{russian}{\textit{әртүрлі}} is written as \foreignlanguage{russian}{\textit{әр түрлі}}, which is inconsistent with standard formatting. \\
\midrule
G & There is confusion regarding whether \foreignlanguage{russian}{\textit{-мен}} (\foreignlanguage{russian}{\textit{-бен}}, \foreignlanguage{russian}{\textit{-пен}}) should be attached to or separated from nouns. \\
\midrule
G & The form of \foreignlanguage{russian}{\textit{жылдары}} is inconsistent, alternating between \foreignlanguage{russian}{\textit{жылы}}, \foreignlanguage{russian}{\textit{жылдары}}, or abbreviated forms like \foreignlanguage{russian}{\textit{ж.}} or \foreignlanguage{russian}{\textit{жж.}}. \\
\midrule
L & Russian words, such as \foreignlanguage{russian}{\textit{награда}}, \foreignlanguage{russian}{\textit{карьера}}, and \foreignlanguage{russian}{\textit{музей}}, are used even though Kazakh equivalents exist. \\
\midrule
L & The word \foreignlanguage{russian}{\textit{артис}} is used instead of the correct Kazakh term \foreignlanguage{russian}{\textit{әртіс}}. \\
\midrule
S & Words like \foreignlanguage{russian}{\textit{бастап}} and \foreignlanguage{russian}{\textit{бері}} are omitted when referring to years. For example, \foreignlanguage{russian}{\textit{1967 жылдан ол бұл атаққа ие болды}}. \\
\midrule
G & The phrase \foreignlanguage{russian}{\textit{туралы түсіндіріңіз}} is unnatural and often incorrect. It should be replaced with \foreignlanguage{russian}{\textit{туралы айтыңыз}} or \foreignlanguage{russian}{\textit{айтып беріңіз}}. \\
\midrule
G & The phrase \foreignlanguage{russian}{\textit{жеңіл атлетші}} is used correctly, but \foreignlanguage{russian}{\textit{ауыр атлет}} is written without the suffix \foreignlanguage{russian}{\textit{-ші}}. It should be standardized as \foreignlanguage{russian}{\textit{атлетші}}. \\
\midrule
L & Shortened names and surnames are written without spaces. For example, \foreignlanguage{russian}{\textit{Ы.Алтынсарин}} instead of \foreignlanguage{russian}{\textit{Ы. Алтынсарин}}. \\
\midrule
L & The phrase \foreignlanguage{russian}{\textit{рөл ойнады}} is a direct translation from Russian and should be replaced with \foreignlanguage{russian}{\textit{рөл орындады}} or \foreignlanguage{russian}{\textit{рөл сомдады}}. \\
\midrule
L & Years are written in abbreviated forms, such as \foreignlanguage{russian}{\textit{1973-78}}, omitting \foreignlanguage{russian}{\textit{19}} in the second part. \\
\midrule
L & Names of \foreignlanguage{russian}{\textit{ру}}, tribes (\foreignlanguage{russian}{\textit{тайпа}}), and \foreignlanguage{russian}{\textit{жүз}} are written in lowercase, which is inconsistent with standard capitalization rules. \\
\midrule
L & The word \foreignlanguage{russian}{\textit{рөл}} is occasionally misspelled as \foreignlanguage{russian}{\textit{рол}} or \foreignlanguage{russian}{\textit{роль}}. \\
\midrule
G & Verbs ending with \foreignlanguage{russian}{\textit{-ып}} or \foreignlanguage{russian}{\textit{-іп}} are used incorrectly, leading to unrelated sentences. For example, in the sentence \foreignlanguage{russian}{\textit{Бұл ойын үлкен тойларда арнайы жүлде тағайындалып өткізіліп, оның түрлері көптеген}}, the part after \foreignlanguage{russian}{\textit{өткізіліп}} is unrelated. The correct form would be \foreignlanguage{russian}{\textit{Бұл ойын үлкен тойларда арнайы жүлде тағайындалып өткізіледі; оның түрлері көп}} or \foreignlanguage{russian}{\textit{Бұл ойын үлкен тойларда арнайы жүлде тағайындалып өткізіліп, қазақ халқында әлі де жүргізіліп келеді}}. \\
\midrule
S & Words occasionally appear in the Latin alphabet even though they are not written in Latin in the original text. \\
\midrule
G & Sentences like \foreignlanguage{russian}{\textit{Безендірудің көркемдеу құралдарын қандай өнер салалары пайдаланады деп түсіндіріңіз}} are included. The common error is the use of \foreignlanguage{russian}{\textit{пайдаланады деп түсіндіріңіз}}, which should be replaced with \foreignlanguage{russian}{\textit{пайдаланатынын түсіндіріңіз}}. \\
\midrule
G & The phrase \foreignlanguage{russian}{\textit{жер аударылды}} is used for referring to a person. If it describes someone moving places, it should be corrected to \foreignlanguage{russian}{\textit{жер аударды}}. \\
\midrule
S & The structure of the questions deviates from the natural structure of the Kazakh language. For example: \foreignlanguage{russian}{\textit{Бұл рас па, АХҚО қатысушыларының қызметін реттеу Англия мен Уэльстің жалпы құқық қағидаттары негізінде жүргізіледі?}} \\
\midrule
S & The questions ask for information not mentioned in the text, meaning the answer to the question is not present in the text. \\
\midrule
S  & The structure of the questions is entirely incorrect. For example: \foreignlanguage{russian}{Астана Халықаралық Қаржы Орталығы қандай компаниялар үшін реттеуге икемді тәсіл қолданады?}
The correct version: \foreignlanguage{russian}{Астана Халықаралық Қаржы Орталығы қандай компанияларды реттеуге мүмкіндік береді?} 
\\
\midrule
L & The word \foreignlanguage{russian}{\textit{ол}} is overused in the generated data. \\
\bottomrule
\end{tabular}
\caption{Selected annotator comments highlighting issues in \texttt{CultSet}. Error types are categorized as follows: L – Lexical errors, S – Structural errors, G – Grammatical errors.}
\label{table:annotator_comments_cult}
\end{table*}
