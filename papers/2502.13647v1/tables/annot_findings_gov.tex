\begin{table*}[ht!]
\scriptsize
\begin{tabular}{@{}c@{\hspace{10pt}}p{0.88\textwidth}@{}}
\toprule
\textbf{Error Type} & \textbf{Annotator Comments} \\ 

\midrule 
S & The generated data constantly uses an unnatural sentence structure that sounds like this: \foreignlanguage{russian}{\textit{Сіз сәбиіңіздің дүниеге келуін күтудесіз және онлайн режимінде емханаға тіркелу үдерісін түсіндіріңізші.}} Here the word \foreignlanguage{russian}{\textit{және}} sounds extra and can be either eliminated or replaced with \foreignlanguage{russian}{\textit{бұл жағдайда.}} This example of a sentence with extra \foreignlanguage{russian}{\textit{және}} is common. \\
\midrule
G & Inconsistent verb forms in instructions. Like \foreignlanguage{russian}{\textit{түсіндіру}} instead of \foreignlanguage{russian}{\textit{түсіндіріңіз.}} \\
\midrule
G & The generated data has two verb forms like \foreignlanguage{russian}{\textit{тану}} and \foreignlanguage{russian}{\textit{танылу}}. If the first one is the action of subject, the second is an action that happens TO a subject/object. 
 \\
\midrule
S & There is a frequent unnatural sentence structure that sounds like this: \foreignlanguage{russian}{\textit{Бұл рас па: некеге тұрған адамдарға куәлік электрондық нысанда беріледі?}} Either the whole statement needs to be reconstructed or \foreignlanguage{russian}{\textit{ме}} should be added at the end of the sentence.
 \\
\midrule
G & The model frequently fails to change words to accusative case (add \foreignlanguage{russian}{\textit{– н/- ын/-ін}}). For example,  in the phrase \foreignlanguage{russian}{\textit{тегі өзгертуге бола ма}}, the correct form would be \foreignlanguage{russian}{\textit{тегін өзгертуге бола ма}}.\\
\midrule
G & The translator often does not differentiate \foreignlanguage{russian}{\textit{кезінде}} and \foreignlanguage{russian}{\textit{кезіндегі}}. For example, in the sentence \foreignlanguage{russian}{\textit{Некені бұзу кезінде құжаттарды ресімдеу туралы түсіндіріңіз}}, the correct word would be \foreignlanguage{russian}{\textit{кезіндегі}}. The same happens with \foreignlanguage{russian}{\textit{Қазақстанда and Қазақстандағы}}.
\\
\midrule
L & The model writes the word \foreignlanguage{russian}{\textit{айырлуы}} several times. It must be either \foreignlanguage{russian}{\textit{айырылуы}} or \foreignlanguage{russian}{\textit{айырылғаны}} correctly. 
 \\
\midrule
S & The model frequently writes a phrase \foreignlanguage{russian}{\textit{жүзеге асырылады}}, which sounds unnatural and must be \foreignlanguage{russian}{\textit{жүзеге асады}} instead. 
\\
\midrule
G & The model frequently used phrases like \foreignlanguage{russian}{\textit{кірмейді екенін}}, when the correct form for any verb there would be \foreignlanguage{russian}{\textit{кірмейтінін}}. \\
\midrule
G & The model constantly uses this phrasing: \foreignlanguage{russian}{\textit{дегеніміз не екенін түсіндіріңіз.}} It sounds unnatural and must be replaced either with \foreignlanguage{russian}{\textit{не екенін түсіндіріңіз}} or just \foreignlanguage{russian}{\textit{дегеніміз не?}} \\ 
\midrule
G & The model cannot differentiate question words like \foreignlanguage{russian}{\textit{ма, ме}} etc. It can write \foreignlanguage{russian}{\textit{міндетті ма?}}, which is used in verbal speech only. The correct form would be \foreignlanguage{russian}{\textit{міндетті ме?}} \\ 
\midrule
G & The model can miss letters. For instance, \foreignlanguage{russian}{\textit{мақсатары}} instead of \foreignlanguage{russian}{\textit{мақсаттары}}; it is also not connected with the original text because the word is not even used there. \\ 
\midrule
S & There is a common sentence structure that sounds like this: \foreignlanguage{russian}{\textit{Тұрмыстық зорлық-зомбылық жағдайында қандай медициналық көмек алу жолдарын айтыңыз.}} Either the word \foreignlanguage{russian}{\textit{қандай}} must be removed and \foreignlanguage{russian}{\textit{-ғы}} added to \foreignlanguage{russian}{\textit{жағдайында}} (\foreignlanguage{russian}{\textit{жағдайындағы}}), or the statement must be a question instead, like \foreignlanguage{russian}{\textit{қандай медициналық көмек алуға болады?}} \\ 
\midrule
G & \foreignlanguage{russian}{\textit{Әр түрлі}} must be written as \foreignlanguage{russian}{\textit{әртүрлі}}. \foreignlanguage{russian}{\textit{Марапат}} must be used instead of \foreignlanguage{russian}{\textit{награда}}. \\ 
\midrule
S & There are many sentences like \foreignlanguage{russian}{\textit{Шындық па, үйде арнайы әлеуметтік қызметтерді көрсету тоқтатылады, егер қызмет алушының өзі олардан бас тартса?}} The structure is wrong, \foreignlanguage{russian}{\textit{шындық па}} must be placed at the end!! And \foreignlanguage{russian}{\textit{егер}} etc. must be placed at the start. Any verb before \foreignlanguage{russian}{\textit{шындық па}} must end with \foreignlanguage{russian}{\textit{-тыны/-тіні}}. \\ 
\midrule
G & The model often writes \foreignlanguage{russian}{\textit{деп түсіндіріңіз}} in instructions. In different occasions it must be written differently, like \foreignlanguage{russian}{\textit{бола ма, түсіндіріңіз}} or \foreignlanguage{russian}{\textit{болатынын түсіндіріңіз}}. \\ 
\midrule
G & Incorrect translation for \foreignlanguage{russian}{\textit{процесс}}. It should be translated as \foreignlanguage{russian}{\textit{үдеріс/үрдіс}}. \\ 
\midrule
L & Sentences like \foreignlanguage{russian}{\textit{Жеке куәлікті алғаш алу үшін қандай құжаттар қажет екені рас па?}} are common in “Is it true that” instructions. Either \foreignlanguage{russian}{\textit{қандай}} must be removed or replaced with a list of documents. \\ 
\midrule
L & The model generates a phrase \foreignlanguage{russian}{\textit{өтініш жасау}}; the more common version would be \foreignlanguage{russian}{\textit{өтініш беру}} or \foreignlanguage{russian}{\textit{өтініш жазу}}. \\
\midrule
G & The model sometimes has a hard time differentiating active and passive positions of subjects. For example, \foreignlanguage{russian}{\textit{тіркеу куәлігі немесе әскери билет жоғалтқанда}} puts the verb as if the subject is passive, whereas the correct form would be \foreignlanguage{russian}{\textit{жоғалғанда}} (because the subjects are in active form). This happens with different verbs throughout the document. \\
\midrule
G & Sometimes the answers that the model generates for instructions are not even connected to the question. For example, the answer for \foreignlanguage{russian}{\textit{Өтініш беруші ұсынылған ақпарат үшін қашан жауап береді деген айтылды ма?}} is \foreignlanguage{russian}{\textit{Иә, өтініш берушілер ұсынылған ақпарат үшін өздері жауап береді деп айтылды.}} The word \foreignlanguage{russian}{\textit{өздері}} as an answer is unrelated to the question \foreignlanguage{russian}{\textit{қашан}}. \\
\midrule
L & The model frequently generates phrases like \foreignlanguage{russian}{\textit{осы мәтін бойынша}}, \foreignlanguage{russian}{\textit{бұл деректерге қарағанда}}, which implies that there is a certain text to rely on. \\
\midrule
L & The model sometimes pronounces \foreignlanguage{russian}{\textit{Беларусь}} as \foreignlanguage{russian}{\textit{Беларуссия}}, which is not correct even unofficially. For formal texts, \foreignlanguage{russian}{\textit{Беларусь}} should be used, or at least \foreignlanguage{russian}{\textit{Белоруссия}}. \\
\midrule
L & The model frequently uses the Russian word \foreignlanguage{russian}{\textit{заем}}, even though there are Kazakh translations for it, such as \foreignlanguage{russian}{\textit{несие}} or \foreignlanguage{russian}{\textit{қарыз}}. This happens because of the original text. \\
\midrule
S & Sometimes the model generates sentence structures like \foreignlanguage{russian}{\textit{Бұл рас па: Азамат 16 жасқа толғанда Қазақстанның әрбір азаматы жеке куәлік алуға міндетті екенін?}} Firstly, the word \foreignlanguage{russian}{\textit{екенін}} in all these sentences is grammatically incorrect here and must be \foreignlanguage{russian}{\textit{екені}}. Additionally, since this sentence structure is very frequent, it must be reconstructed to place \foreignlanguage{russian}{\textit{бұл рас па}} at the end. For example, it should sound like this: \foreignlanguage{russian}{\textit{16 жасқа толғанда Қазақстанның әрбір азаматы жеке куәлік алуға міндетті екені рас па?}} \\
\midrule
S & There are also sentences that start with \foreignlanguage{russian}{\textit{шынайы ма екенін анықтаңыз}} or \foreignlanguage{russian}{\textit{шындық па анықтаңыз}}. These phrases must be rewritten as either \foreignlanguage{russian}{\textit{шындық па, соны анықтаңыз}} or just \foreignlanguage{russian}{\textit{шын ба}}/\foreignlanguage{russian}{\textit{шындық па}} without the \foreignlanguage{russian}{\textit{анықтаңыз}} part. \\
\midrule
G & The model cannot identify where it is better to not use \foreignlanguage{russian}{\textit{табыс септік}}. For example, \foreignlanguage{russian}{\textit{пәтерді іздеу}}/\foreignlanguage{russian}{\textit{пәтерді іздегенде}} is used frequently, but the common and natural way must be without the \foreignlanguage{russian}{\textit{-ді}} suffix. \\

\bottomrule
\end{tabular}
\caption{Selected annotator comments highlighting issues in \texttt{GovSet}. Error types are categorized as follows: L – Lexical errors, S – Structural errors, G – Grammatical errors.}
\label{table:annotator_comments_gov}
\end{table*}
