\subsection{GovSet}
\begin{table*}[ht!]
\scriptsize
\begin{tabular}{@{}p{0.08\textwidth}p{0.22\textwidth}p{0.22\textwidth}p{0.22\textwidth}p{0.22\textwidth}@{}}

\toprule
\multicolumn{1}{c}{\textbf{Type}} & 
\multicolumn{1}{c}{\textbf{Instruction}} & 
\multicolumn{1}{c}{\textbf{Output}} & 
\multicolumn{1}{c}{\textbf{Updated Instruction}} & 
\multicolumn{1}{c}{\textbf{Updated Output}} \\
\midrule
\\
\multicolumn{1}{c}{\textbf{G}} & 
\foreignlanguage{russian}{Неке қиғаннан кейін құжаттарды ауыстыру үдерісін түсіндіріңіз.}
 & \foreignlanguage{russian}{Некеге тұрғаннан кейін жұбайы, егер ол жұбайының/зайыбының тегін алса, неке қию туралы куәлікті алған сәттен бастап 30 күн ішінде жеке куәлікті ауыстыру қажет.} & \foreignlanguage{russian}{\textcolor{darkgreen}{Қазақстанда} неке қиғаннан кейінгі құжаттарды ауыстыру үдерісін түсіндіріңіз.} & \foreignlanguage{russian}{\textcolor{darkgreen}{Ең алдымен, адам некеден кейін} жұбайының тегін \textcolor{darkgreen}{қабылдаса}, жеке куәлігін 30 күн ішінде ауыстыруы қажет.} \newline \\
 \\
 & Explain the process of changing documents after marriage. & After marriage, if the spouse takes the husband's/wife's last name, the national ID must be changed within 30 days from the date the marriage certificate is received. & Explain the process of changing documents after marriage in Kazakhstan. & First of all, if a person adopts their spouse's surname after marriage, they must change their ID within 30 days.\\
 \\
\hline
\\
\multicolumn{1}{c}{\textbf{G}} &
\foreignlanguage{russian}{Қазақстанда халықаралық рейсте жоғалған багаж үшін өтемақы төленетіні рас па?} & \foreignlanguage{russian}{Иә, халықаралық рейсте жоғалған багаж үшін өтемақы Қазақстан қол қойған екі Конвенцияның – Варшава мен Монреальдың ережелеріне, сондай-ақ компаниялардың ішкі ережелеріне сәйкес төленеді.} & \foreignlanguage{russian}{Қазақстанда халықаралық рейсте жоғалған багаж үшін өтемақы төлейтіні қаншалықты рас?} & \foreignlanguage{russian}{\textcolor{darkgreen}{Қазақстанда} халықаралық рейсте жоғалған багаж үшін өтемақы \textcolor{darkgreen}{төлейтіні толықтай рас. Өтемақы} Қазақстан қол қойған екі Конвенцияның \textcolor{darkgreen}{(}Варшава мен Монреальдың ережелеріне\textcolor{darkgreen}{)} \textcolor{darkgreen}{және} компаниялардың ішкі ережелері \textcolor{darkgreen}{негізінде есептелініп} төленеді.} \newline \\
 & Is it true that compensation is paid in Kazakhstan for lost baggage on international flights? & Yes, compensation for lost baggage on international flights is paid according to the rules of the two Conventions signed by Kazakhstan – the Warsaw and Montreal Conventions – as well as in accordance with the internal rules of the companies. & Is it true that compensation is paid in Kazakhstan for lost baggage on international flights? & It is completely true that compensation is paid in Kazakhstan for lost baggage on international flights. The compensation is calculated and paid based on the two Conventions signed by Kazakhstan (the rules of the Warsaw and Montreal Conventions) and the internal regulations of the companies.\\
\hline
\\
\multicolumn{1}{c}{\textbf{G}} &
\foreignlanguage{russian}{Қазақстанда қандастарға азаматтық алудың шарттары туралы түсіндіріңіз.} & \foreignlanguage{russian}{ Қандастар Қазақстан аумағында тұру мерзіміне қарамастан жеңілдетілген тәртіпте азаматтықты ала алады. Олар бұрынғы азаматтықтан бас тарту туралы нотариат куәландырған өтініш, бекітілген нысандағы сауалнама-өтініш, шарттарды сақтау туралы міндеттеме, өмірбаян, жеке басын куәландыратын және адамның басқа елдің азаматтығына жататындығын растайтын құжаттың көшірмесі, қандастың куәлігі және төрт фотосурет секілді құжаттарды тапсыруы керек.} & \foreignlanguage{russian}{\textcolor{darkgreen}{Қандастар Қазақстан Республикасының азаматтығын алу үшін не істеу керек?}} & \foreignlanguage{russian}{\textcolor{darkgreen}{Қандастардың Қазақстан Республикасында тұрған} мерзіміне қарамастан \textcolor{darkgreen}{олардың азаматтық алуы жеңілдетілген тәртіппен жүзеге асады. Қандастар Қазақстан Республикасының азаматтығын алу үшін} бұрынғы азаматтықтан бас тарту туралы нотариат куәландырған өтініш, бекітілген нысандағы сауалнама-өтініш, шарттарды сақтау туралы міндеттеме, өмірбаян, жеке басын куәландыратын және адамның басқа елдің азаматтығына жататындығын растайтын құжаттың көшірмесі, қандастың куәлігі және төрт фотосурет секілді құжаттарды тапсыруы керек.}\\ \\
 & Explain the conditions for obtaining citizenship for oralmans (ethnic Kazakhs) in Kazakhstan. & Oralmans (ethnic Kazakhs) can obtain citizenship in a simplified manner regardless of the duration of their stay in Kazakhstan. They need to submit a notarized application to renounce their previous citizenship, a completed application form in the approved format, a commitment to comply with the conditions, an autobiography, a copy of a document confirming their identity and lack of foreign citizenship, a certificate of oralman status, and four passport-sized photos. & What should oralmans (ethnic Kazakhs) do to obtain citizenship of the Republic of Kazakhstan? & Oralmans (ethnic Kazakhs) can obtain citizenship in Kazakhstan regardless of the duration of their stay in a simplified manner. To obtain Kazakhstani citizenship, they must submit a notarized application to renounce their previous citizenship, a completed application form in the approved format, a commitment to comply with the conditions, an autobiography, a copy of a document confirming their identity and lack of foreign citizenship, a certificate of oralman status, and four passport-sized photos.\\
\hline
\end{tabular}
\caption{Example of annotation for \texttt{GovSet}.}
\label{tab:localizing-kz-ru-ex}
\end{table*}
\clearpage




\subsection{CultSet}
\begin{table*}[ht!]
\scriptsize
\begin{tabular}{@{}p{0.08\textwidth}p{0.22\textwidth}p{0.22\textwidth}p{0.22\textwidth}p{0.22\textwidth}@{}}

\toprule
\multicolumn{1}{c}{\textbf{Type}} & 
\multicolumn{1}{c}{\textbf{Instruction}} & 
\multicolumn{1}{c}{\textbf{Output}} & 
\multicolumn{1}{c}{\textbf{Updated Instruction}} & 
\multicolumn{1}{c}{\textbf{Updated Output}} \\
\midrule
\\
\multicolumn{1}{c}{\textbf{C}} & 
\foreignlanguage{russian}{Фильмнің қандай жетістіктері болды?}
 & \foreignlanguage{russian}{«Транссібір экспресі» фильмі 1978 жылы Ереванда өткен Бүкілодақтық кинофестивальда 'ең үздік шытырман оқиғалы фильм үшін' сыйлығын алды. Сондай-ақ, Карловы Варыда өткен 21-Халықаралық кинофестивальда Азия, Африка және Латын Америкасы елдерінің симпозиумы қазылар алқасының сыйлығына ие болды. 1978 жылы режиссер Оразбаев пен суретші Қарсақбаевқа Қазақстан Мемлекеттік сыйлығы берілді.} & \foreignlanguage{russian}{\textcolor{darkgreen}{"Транссібір экспресі" атты қазақстандық} фильмнің қандай жетістіктері болды?} & \foreignlanguage{russian}{«Транссібір экспресі» фильмі \textcolor{darkgreen}{1977 жылы Қазақфильм студиясы тарабынан шықты.} 1978 жылы ол Ереванда өткен Бүкілодақтық кинофестивальда \textcolor{darkgreen}{"Ең үздік авантюра фильмі үшін" сыйлығын алды. Бұдан гөрі}, Карловы Варыда өткен 21-Халықаралық кинофестивальда Азия, Африка және Латын Америкасы елдерінің симпозиумы қазылар алқасының сыйлығына \textcolor{darkgreen}{да} ие болды. 1978 жылы фильм режиссері Оразбаев пен суретшісі Қарсақбаевқа Қазақстан Мемлекеттік сыйлығы берілді.}\newline
 \\
 \\
& What achievements did the film have? & The film "Trans-Siberian Express" received the award for "Best Adventure Film" at the World Film Festival held in Yerevan in 1978. Additionally, it won a prize at the 21st International Film Festival in Karlovy Vary, as well as at the symposium of Asia, Africa, and Latin American countries. In 1978, the film's director, Orazbayev, and cinematographer Karasbayev were awarded the State Prize of Kazakhstan. & What achievements did the Kazakhstani film "Trans-Siberian Express" have? & The film "Trans-Siberian Express" was released in 1977 by the Kazakhfilm studio. In 1978, it received the award for "Best Adventure Film" at the World Film Festival in Yerevan. Later, it also won a prize at the 21st International Film Festival in Karlovy Vary and at the symposium of Asia, Africa, and Latin American countries. In 1978, the film's director, Orazbayev, and cinematographer Karasbayev were awarded the State Prize of Kazakhstan. \\
\\
\hline
\\
\multicolumn{1}{c}{\textbf{C}} &
\foreignlanguage{russian}{Қазақ халқының музыкалық мұрасы қалай дамыды?} & \foreignlanguage{russian}{Қазақ халқының музыкалық мұрасы ғасырлар бойы ауыздан-аузға, ұстаздан-шәкіртке жеткізіліп, көшпенді өмір салтының ықпалымен дамыды. Музыкалық дәстүрлер батырлық және лирикалық жанрдағы эпостарды қамтыды. Қазақ композиторлары әлемдік деңгейде танымал болып, ұлттық музыкалық мәдениетті дамытуда үлкен үлес қосты.} & \foreignlanguage{russian}{Қазақ халқының музыкалық мұрасы қалай дамыды?} & \foreignlanguage{russian}{Қазақ халқының музыкалық мұрасы ғасырлар бойы ауыздан-аузға, ұстаздан-шәкіртке жеткіз\textcolor{darkgreen}{ілген. Негізінен} көшпенді өмір салтының ықпалымен дамыды. \textcolor{darkgreen}{Қазақтың} музыкалық дәстүрі батырлық және лирикалық жанрдағы эпостарды қамтыды. \textcolor{darkgreen}{Қазіргі заманда} қазақ композиторлары әлемдік деңгейде танымал болып, ұлттық музыкалық мәдениетті \textcolor{darkgreen}{одан әрі дамытуға үлкен үлес қосып жүр.}} \newline \\
& How did the musical heritage of the Kazakh people develop? & The musical heritage of the Kazakh people developed over centuries through oral transmission, from master to apprentice, and was influenced by the nomadic lifestyle. Musical traditions included epic genres of heroic and lyrical works. Kazakh composers became renowned internationally and contributed significantly to the development of national musical culture. & How did the musical heritage of the Kazakh people develop? & The musical heritage of the Kazakh people developed over centuries through oral transmission, from master to apprentice, primarily influenced by the nomadic lifestyle. Kazakh musical traditions encompassed genres of epic heroic and lyrical works. In modern times, Kazakh composers have gained international recognition and continue to contribute greatly to the development and advancement of national musical culture. \\
\\
\hline
\\
\multicolumn{1}{c}{\textbf{C}} &
 \foreignlanguage{russian}{Киіз үйде түндік дегеніміз не екенін түсіндіріңіз.} & \foreignlanguage{russian}{Түндік — киіз үйдің шаңырағына жабылатын киіз, ол әрі жарық береді, әрі түтін шығуға мүмкіндік береді. Төртбұрышты түрде пішіліп, шаңырақ диаметірінен 50-60 сантиметрдей артық жасалады.} & \foreignlanguage{russian}{\textcolor{darkgreen}{Қазақ халқының} киіз \textcolor{darkgreen}{үйіндегі түндіктің} не екенін түсіндіріңіз.} & \foreignlanguage{russian}{Түндік — \textcolor{darkgreen}{қазақтың баспанасы} киіз үйдің шаңырағына жабылатын киіз. \textcolor{darkgreen}{Ол әрі ішке жарық өткізеді, әрі түтін шығуына мүмкіндік береді}. Төртбұрышты түрде пішіліп, шаңырақ диаметірінен 50-60 сантиметрдей артық \textcolor{darkgreen}{етіп} жасалады.} 
 \\
& Explain what "tundik" means in a Kazakh yurt. & The "tundik" is the felt cover placed on the top of the yurt, which provides insulation and allows smoke to escape. It is made slightly larger than the circular roof opening, with a diameter exceeding 50-60 centimeters. It is triangular in shape. & Explain what "tundik" means in the Kazakh yurt. & The "tundik" is the felt cover placed on the top of the Kazakh yurt. It serves as insulation and allows smoke to escape. It is triangular in shape and is made slightly larger than the circular roof opening, with a diameter exceeding 50-60 centimeters.\\
\hline
\end{tabular}
\caption{Example of annotation for \texttt{CultSet}.}
\label{tab:localizing-kz-ru-ex}
\end{table*}
