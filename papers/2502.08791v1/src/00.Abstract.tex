\begin{abstract}
Vision-language navigation (VLN) has emerged as a promising paradigm, enabling mobile robots to perform zero-shot inference and execute tasks without specific pre-programming. However, current systems often separate map exploration and path planning, with exploration relying on inefficient algorithms due to limited (partially observed) environmental information. In this paper, we present a novel navigation pipeline named ``ClipRover" for simultaneous exploration and target discovery in unknown environments, leveraging the capabilities of a vision-language model named CLIP. Our approach requires only monocular vision and operates without any prior map or knowledge about the target. For comprehensive evaluations, we design the functional prototype of a UGV (unmanned ground vehicle) system named ``Rover Master", a customized platform for general-purpose VLN tasks. We integrate and deploy the ClipRover pipeline on Rover Master to evaluate its throughput, obstacle avoidance capability, and trajectory performance across various real-world scenarios. Experimental results demonstrate that ClipRover consistently outperforms traditional map traversal algorithms and achieves performance comparable to path-planning methods that depend on prior map and target knowledge. Notably, ClipRover offers real-time active navigation without requiring pre-captured candidate images or pre-built node graphs, addressing key limitations of existing VLN pipelines.
\end{abstract}

% \begin{IEEEkeywords}
\textbf{Keywords.}
Vision-Language Navigation;
Zero-Shot Visual Servoing;
Path Planning;
GPS-denied Navigation.
% \end{IEEEkeywords}
