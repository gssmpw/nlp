% %\section{UGV Platform: Mechanical Design}
% \section{Rover Master: System Design}
% We develop a novel robotic platform ``\textit{Rover Master}'' (Fig.\ref{fig:system_design}). This platform is designed to be a lower cost yet higher performance alternative to the \textit{TurtleBot} \TODO{Ref.}. It is completely designed from scratch, with most of its parts designed for 3D-printing.

% Both the \textit{TurtleBot} and our \textit{Rover Master} platforms are targeted to fast prototyping of research or engineering projects. Such robotic platforms have been widely used in various research fields and provided significant convenience towards the users.

%\section{Rover Master: System Design}
%Current UGV platforms such as the \textit{TurtleBots}~\cite{turtlebot} are widely used in various research fields and provide significant convenience to users. However, they often come with a high price tag while lacking features and performances demanded by modern AI applications.

%\draft{ Struggling to find a suitable platform with a reasonable price tag, we set off to develop a new platform that not only meets the needs of this project, but might also come handy for other researchers with similar requirements.}


% We designed and developed a novel UGV named ``{Rover Master}'' as low-cost yet higher performance alternative to the \textit{TurtleBot} \TODO{Ref.}.

\section{Rover Master: System Design}

We develop a novel robotic platform ``\textit{Rover Master}'' as a low-power and low-cost UGV platform that is portable and scalable for 2D navigation tasks. The major sensors and actuation components are shown in Fig.~\ref{fig:system_design}; it includes a monocular RGB camera and a 2D LiDAR for exteroceptive perception. Four independent wheel assemblies are responsible for actuation; each wheel assembly is modular and self-contained, \ie, it consists of a gearbox, a brushless DC motor, and a suspension system; see Fig.~\ref{fig:suspension:cad}b. It also includes a pseudo odometer that uses telemetry data from an electronic speed controller (ESC) which drives the motors. Additionally, the driver stack consists of a flight-controller module originally designed for quad-copters that report onboard sensory data to the host computer for planning and navigation. Unlike existing platforms like the \textit{TurtleBots}~\cite{turtlebot}, \textit{RoverMaster} is designed with an open and spacious chassis to facilitate easy adaptation to any single board computers (SBCs). In our setup, we configured the platform to: (\textbf{i}) deliver sufficient computational power to handle a general-purpose vision-language model and other computationally intensive tasks; (\textbf{ii}) have enough mechanical stability to hold the camera in an elevated position without excess vibrations even on uneven surfaces; and (\textbf{iii}) support 3-DOF motions (forward/backward, sideways and rotation), whereas most existing UGV systems have only 2-DOF (forward surge and twist rotation).

%which is needed for a comprehensive evaluation of the proposed VLN pipeline.

%We use an {Nvidia Jetson Orin} for low-power on-board computing. The portable design allows the use of other single-board computers (SBCs) such as Nvidia Jetson Nano/TX2 as well as Raspberry PIs as well. As shown in Fig.~\ref{fig:connection}c, a dedicated \JI{Mention which} microcontroller and a quad ESC take care of motion and intrinsic sensing, which is connected to the host computer via USB 2. The chassis also comes with an optional pair of WiFi-compatible antennas to help extend the range and stability of wireless communication. \JI{talk about what sensors (camera/imu etc) are onboard by default and what can be integrated.}
% \JI{Mention the SBC and other single board devices / micro controllers onboard. Then talk about the networking / routing components.}

\subsection{Task Specific Design for ClipRover}
We customize the Rover Master platform with a configuration that supports real-time VLN capability for 2D environments. An omnidirectional drive system is chosen to account for the moderate surface irregularities. The four wheels are connected to separate brushless DC motors via a planetary gearbox with a reduction ratio of $16$:$1$. This allows the robot to travel at higher speeds and easily overcome moderate obstacles. A throttle limit of $20$\% is imposed to ensure operational safety, capping the robot's maximum speed at approximately $2$\,m/s. As shown in Fig.~\ref{fig:suspension:cad}b, the gearbox and brushless DC motor are integrated into the wheel hub, optimizing the spatial efficiency and smoothness of the drive system. Moreover, the computing pipeline is powered by an Nvidia {\tt Jetson Orin} module. A {\tt FLIR} global shutter RGB camera, lifted approximately $30$\,cm above ground, is used for visual perception. This design ensures that the optical center of the lens aligns with the geometric center of the robot. Besides, the \textit{heading} value from the flight controller's IMU facilitates $360^{\circ}$ scanning capabilities. A 2D LiDAR is mounted at the front of the upper deck as is shown in Fig.~\ref{fig:system_design}b. The LiDAR is only used for safety and visualization purposes and does not influence VLN's decision-making. Further details regarding the LiDAR's utility are explained in Sec.~\ref{sec:LiDAR}.

\begin{figure}
    \centering
    \includegraphics[width=\columnwidth]{res/SuspensionAndConnection.pdf}%
    \vspace{-1mm}
    \caption{
        The wheel hub and suspension design are shown on the left: (\textbf{a}) suspension system coupled with the omnidirectional wheel; and (\textbf{b}) cross-section view of the `discovery wheel', in-wheel drive system, and suspension links, assembled and rendered by CAD software. The connection diagram is shown on the right: (\textbf{c}) the dashed line represents physical connection, while solid lines represent logic functions.
    % \JI{make it more compact, remove the empty spaces} [DONE]
    }%
    \label{fig:suspension:cad}
    \label{fig:connection}
    \vspace{-2mm}
\end{figure}

%The purpose and functionality of this LiDAR sensor is explained in Sec.~\ref{sec:LiDAR}.
% \subsection{Features and Capabilities}
% \subsubsection{\textbf{Enhanced Performance}}
% Our proposed platform features four separately driven wheels. Each wheel is connected to a brushless DC motor via a planetary gearbox with a reduction ratio of $16:1$. This allows the robot to travel at higher speeds and easily overcome obstacles on its path. Due to safety concerns, a hard limit of $20\%$ throttle is currently enforced. At this limit, the robot can travel at approx. $2m/s$. The absolute maximum speed has not been tested. One unique advantage of our design is the compactness of wheel and motor assembly. As shown in Fig.~\ref{fig:suspension:cad}b, the gearbox and brushless DC motor are packed into the center of the wheel (a.k.a. wheel hub). This design significantly reduces the space occupied by the driving system and also improves motion smoothness for the platform.
% \\ [-2mm]
% % \item \textbf{Wider Application Areas.}

% \subsubsection{\textbf{Wider Application Areas}}
% We introduced four-wheel independent suspension onto our platform. The suspension extends the robot's application to uneven surfaces. When using the \textit{Discovery Wheel} (TPU), the platform is proven to perform well on grasslands, rocky pavements, soil or sand\footnote{Reduced maneuverability, spinning at a same spot will trap the robot.} surfaces. In addition, the suspension system can be configured to filter out surface bumps and provide better stability for the onboard sensors.
% \\ [-2mm]
% % \item \textbf{Great Flexibility and Adaptability.}

\subsection{Features and Capabilities}
One unique advantage of our Rover Master system design is the compactness of its wheels and motor assembly. The four-wheel independent suspension extends its application to uneven surfaces; when using the \textit{Discovery Wheel} (TPU) shown in Fig.~\ref{fig:system_design}c and Fig.~\ref{fig:suspension:cad}b, it performs well on grasslands, rocky pavements, and even sand (with reduced maneuverability). The suspension system can also be configured to filter out surface bumps and provide better stability for onboard sensors.
% Overall, it has the following two useful features.

%\subsubsection{Flexibility and Adaptability}

The Rover Master platform can be configured to switch between different \textit{drive types} (omnidirectional or differential), \textit{chassis sizes} (with parameterized CAD design), and task-specific \textit{actuation and sensory add-ons} (cameras, LiDARs, manipulators for both indoor and field robotics applications). The chassis plates are designed to house various additional sensors and actuators according to tasks.
% Additionally, it is configurable with multiple types of onboard computers, such as {Raspberry Pis}, Nvidia Jetson Modules, as well as mini X86 computers.
A comparison of Rover Master with two widely used \textit{TurtleBot} UGV variants is presented in Table~\ref{tab:cost}.

%A metric $M3$ hole array of $20$\,mm spacing was placed on both plates, making mounting many types of components easy. Due to the extremely compact design of the in-wheel drive system, the chassis is very spacious compared to existing platforms. The extra space allows for easier adaptation to various custom add-on modules and sensors.

% % \item \textbf{Lower Cost and Better Scalability.}
% %\subsubsection{Scalability and Low-cost Design}
% We compare the Rover Master with In Table. \ref{tab:cost} we compare the Rover Master system with some popular options currently available in the market. Since the majority of the parts are designed for 3D printing, we achieved a significant advantage on the overall cost. \\
% Scalability is another potential advantage to our platform. In addition to the lower cost which makes it easier to batch produce, the interchangeable battery pack design could make it easier to work with a multi-unit charging station. We have plans to support metal shell Lithium-ion batteries (e.g. 18650 or 21700 industry standard batteries) with an onboard battery management system (BMS).
% %It will further improve the field deployment experience.
% % \end{itemize}

\newcommand{\dRow}[1]{\multirow{2}{*}{#1}}
\newcommand{\acronym}[1]{\textbf{\small{[\,#1\,]}}}

\begin{table}
    \centering
    \renewcommand{\arraystretch}{1.1}
    \caption{Comparison of \textit{Rover Master} with other standard UGV platforms is shown; the acronyms: \acronym{DF} Differential, \acronym{OD} Omni-directional, \acronym{CFG} Configurable.}
    \vspace{-1mm}
    \resizebox{\columnwidth}{!}{
    \begin{tabular}{cccccc}
    \Xhline{2\arrayrulewidth}
    \dRow{Platform} &  Drive  & \dRow{SBC} & Onboard & \dRow{Add-ons} & Est. Cost  \\
                    &  Type   &            & Sensors &                & \footnotesize{(USD)} \\
    \Xhline{2\arrayrulewidth}
    TurtleBot3  & \dRow{\acronym{DF}} & \dRow{RPi4} &  PiCamera, IMU, & \dRow{Limited} & \dRow{$1500$} \\
    (Waffle Pi) & & & LDS laser & & \\ \hline TurtleBot4 & \dRow{\acronym{DF}} & \dRow{RPi4} & Stereo camera, & \dRow{Limited} & \dRow{$2100$} \\
    (Standard)  & & &  IMU, RPLiDAR & & \\\hline
    \dRow{\Large{$^\textit{~Rover}_\textit{Master}$}}
                & \acronym{DF} & \dRow{\acronym{CFG}} & RGB camera, & \dRow{\acronym{CFG}} & $\mathbf{\z650}$ \\\cline{2-2}\cline{6-6}
                & \acronym{OD} & & IMU, LiDAR & & $\z850$ \\ \Xhline{2\arrayrulewidth}
    %\multicolumn{6}{l}{\footnotesize{
    %    $^\dagger$ Configurable, can be equipped with various other SBCs such as Jetson Orin, Rock Pi, offering more computational power.
    %}} \\[-2mm]
    %\multicolumn{6}{l}{\footnotesize{
    %    $^\ddagger$ Including LiDAR modlue, monocular or stereo camera, robotic arm, actuated craws, \etc
    %}}
    % \multicolumn{6}{l}{\small{
    %     \textbf{\textit{Acronyms}}: \acronym{DF} Differential, \acronym{OD} Omni-directional, \acronym{CFG} Configurable.
    % }}
    \end{tabular}
    }
    \label{tab:cost}
    \vspace{-3mm}
\end{table}

