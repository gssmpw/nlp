
%% bare_jrnl_compsoc.tex
%% V1.4b
%% 2015/08/26
%% by Michael Shell
%% See:
%% http://www.michaelshell.org/
%% for current contact information.
%%
%% This is a skeleton file demonstrating the use of IEEEtran.cls
%% (requires IEEEtran.cls version 1.8b or later) with an IEEE
%% Computer Society journal paper.
%%
%% Support sites:
%% http://www.michaelshell.org/tex/ieeetran/
%% http://www.ctan.org/pkg/ieeetran
%% and
%% http://www.ieee.org/

%%*************************************************************************
%% Legal Notice:
%% This code is offered as-is without any warranty either expressed or
%% implied; without even the implied warranty of MERCHANTABILITY or
%% FITNESS FOR A PARTICULAR PURPOSE! 
%% User assumes all risk.
%% In no event shall the IEEE or any contributor to this code be liable for
%% any damages or losses, including, but not limited to, incidental,
%% consequential, or any other damages, resulting from the use or misuse
%% of any information contained here.
%%
%% All comments are the opinions of their respective authors and are not
%% necessarily endorsed by the IEEE.
%%
%% This work is distributed under the LaTeX Project Public License (LPPL)
%% ( http://www.latex-project.org/ ) version 1.3, and may be freely used,
%% distributed and modified. A copy of the LPPL, version 1.3, is included
%% in the base LaTeX documentation of all distributions of LaTeX released
%% 2003/12/01 or later.
%% Retain all contribution notices and credits.
%% ** Modified files should be clearly indicated as such, including  **
%% ** renaming them and changing author support contact information. **
%%*************************************************************************


% *** Authors should verify (and, if needed, correct) their LaTeX system  ***
% *** with the testflow diagnostic prior to trusting their LaTeX platform ***
% *** with production work. The IEEE's font choices and paper sizes can   ***
% *** trigger bugs that do not appear when using other class files.       ***                          ***
% The testflow support page is at:
% http://www.michaelshell.org/tex/testflow/


\documentclass[10pt,journal,compsoc]{IEEEtran}
%
% If IEEEtran.cls has not been installed into the LaTeX system files,
% manually specify the path to it like:
% \documentclass[10pt,journal,compsoc]{../sty/IEEEtran}





% Some very useful LaTeX packages include:
% (uncomment the ones you want to load)


% *** MISC UTILITY PACKAGES ***
%
%\usepackage{ifpdf}
% Heiko Oberdiek's ifpdf.sty is very useful if you need conditional
% compilation based on whether the output is pdf or dvi.
% usage:
% \ifpdf
%   % pdf code
% \else
%   % dvi code
% \fi
% The latest version of ifpdf.sty can be obtained from:
% http://www.ctan.org/pkg/ifpdf
% Also, note that IEEEtran.cls V1.7 and later provides a builtin
% \ifCLASSINFOpdf conditional that works the same way.
% When switching from latex to pdflatex and vice-versa, the compiler may
% have to be run twice to clear warning/error messages.






% *** CITATION PACKAGES ***
%
\ifCLASSOPTIONcompsoc
  % IEEE Computer Society needs nocompress option
  % requires cite.sty v4.0 or later (November 2003)
  \usepackage[nocompress]{cite}
\else
  % normal IEEE
  \usepackage{cite}
\fi
% cite.sty was written by Donald Arseneau
% V1.6 and later of IEEEtran pre-defines the format of the cite.sty package
% \cite{} output to follow that of the IEEE. Loading the cite package will
% result in citation numbers being automatically sorted and properly
% "compressed/ranged". e.g., [1], [9], [2], [7], [5], [6] without using
% cite.sty will become [1], [2], [5]--[7], [9] using cite.sty. cite.sty's
% \cite will automatically add leading space, if needed. Use cite.sty's
% noadjust option (cite.sty V3.8 and later) if you want to turn this off
% such as if a citation ever needs to be enclosed in parenthesis.
% cite.sty is already installed on most LaTeX systems. Be sure and use
% version 5.0 (2009-03-20) and later if using hyperref.sty.
% The latest version can be obtained at:
% http://www.ctan.org/pkg/cite
% The documentation is contained in the cite.sty file itself.
%
% Note that some packages require special options to format as the Computer
% Society requires. In particular, Computer Society  papers do not use
% compressed citation ranges as is done in typical IEEE papers
% (e.g., [1]-[4]). Instead, they list every citation separately in order
% (e.g., [1], [2], [3], [4]). To get the latter we need to load the cite
% package with the nocompress option which is supported by cite.sty v4.0
% and later. Note also the use of a CLASSOPTION conditional provided by
% IEEEtran.cls V1.7 and later.





% *** GRAPHICS RELATED PACKAGES ***
%
\ifCLASSINFOpdf
  % \usepackage[pdftex]{graphicx}
  % declare the path(s) where your graphic files are
  % \graphicspath{{../pdf/}{../jpeg/}}
  % and their extensions so you won't have to specify these with
  % every instance of \includegraphics
  % \DeclareGraphicsExtensions{.pdf,.jpeg,.png}
\else
  % or other class option (dvipsone, dvipdf, if not using dvips). graphicx
  % will default to the driver specified in the system graphics.cfg if no
  % driver is specified.
  % \usepackage[dvips]{graphicx}
  % declare the path(s) where your graphic files are
  % \graphicspath{{../eps/}}
  % and their extensions so you won't have to specify these with
  % every instance of \includegraphics
  % \DeclareGraphicsExtensions{.eps}
\fi
% graphicx was written by David Carlisle and Sebastian Rahtz. It is
% required if you want graphics, photos, etc. graphicx.sty is already
% installed on most LaTeX systems. The latest version and documentation
% can be obtained at: 
% http://www.ctan.org/pkg/graphicx
% Another good source of documentation is "Using Imported Graphics in
% LaTeX2e" by Keith Reckdahl which can be found at:
% http://www.ctan.org/pkg/epslatex
%
% latex, and pdflatex in dvi mode, support graphics in encapsulated
% postscript (.eps) format. pdflatex in pdf mode supports graphics
% in .pdf, .jpeg, .png and .mps (metapost) formats. Users should ensure
% that all non-photo figures use a vector format (.eps, .pdf, .mps) and
% not a bitmapped formats (.jpeg, .png). The IEEE frowns on bitmapped formats
% which can result in "jaggedy"/blurry rendering of lines and letters as
% well as large increases in file sizes.
%
% You can find documentation about the pdfTeX application at:
% http://www.tug.org/applications/pdftex


% *** MATH PACKAGES ***
%
%\usepackage{amsmath}
% A popular package from the American Mathematical Society that provides
% many useful and powerful commands for dealing with mathematics.
%
% Note that the amsmath package sets \interdisplaylinepenalty to 10000
% thus preventing page breaks from occurring within multiline equations. Use:
%\interdisplaylinepenalty=2500
% after loading amsmath to restore such page breaks as IEEEtran.cls normally
% does. amsmath.sty is already installed on most LaTeX systems. The latest
% version and documentation can be obtained at:
% http://www.ctan.org/pkg/amsmath




%\ifCLASSOPTIONcaptionsoff
%  \usepackage[nomarkers]{endfloat}
% \let\MYoriglatexcaption\caption
% \renewcommand{\caption}[2][\relax]{\MYoriglatexcaption[#2]{#2}}
%\fi
% endfloat.sty was written by James Darrell McCauley, Jeff Goldberg and 
% Axel Sommerfeldt. This package may be useful when used in conjunction with 
% IEEEtran.cls'  captionsoff option. Some IEEE journals/societies require that
% submissions have lists of figures/tables at the end of the paper and that
% figures/tables without any captions are placed on a page by themselves at
% the end of the document. If needed, the draftcls IEEEtran class option or
% \CLASSINPUTbaselinestretch interface can be used to increase the line
% spacing as well. Be sure and use the nomarkers option of endfloat to
% prevent endfloat from "marking" where the figures would have been placed
% in the text. The two hack lines of code above are a slight modification of
% that suggested by in the endfloat docs (section 8.4.1) to ensure that
% the full captions always appear in the list of figures/tables - even if
% the user used the short optional argument of \caption[]{}.
% IEEE papers do not typically make use of \caption[]'s optional argument,
% so this should not be an issue. A similar trick can be used to disable
% captions of packages such as subfig.sty that lack options to turn off
% the subcaptions:
% For subfig.sty:
% \let\MYorigsubfloat\subfloat
% \renewcommand{\subfloat}[2][\relax]{\MYorigsubfloat[]{#2}}
% However, the above trick will not work if both optional arguments of
% the \subfloat command are used. Furthermore, there needs to be a
% description of each subfigure *somewhere* and endfloat does not add
% subfigure captions to its list of figures. Thus, the best approach is to
% avoid the use of subfigure captions (many IEEE journals avoid them anyway)
% and instead reference/explain all the subfigures within the main caption.
% The latest version of endfloat.sty and its documentation can obtained at:
% http://www.ctan.org/pkg/endfloat
%
% The IEEEtran \ifCLASSOPTIONcaptionsoff conditional can also be used
% later in the document, say, to conditionally put the References on a 
% page by themselves.




% *** PDF, URL AND HYPERLINK PACKAGES ***
%
%\usepackage{url}
% url.sty was written by Donald Arseneau. It provides better support for
% handling and breaking URLs. url.sty is already installed on most LaTeX
% systems. The latest version and documentation can be obtained at:
% http://www.ctan.org/pkg/url
% Basically, \url{my_url_here}.

%========== my usepackage========
 \usepackage{threeparttable}
 \usepackage{xspace}
 \usepackage{amsmath}
 \usepackage{makecell}
 \usepackage{colortbl}
 \usepackage[dvipsnames]{xcolor}
\usepackage{xcolor}
\usepackage{tabularx} 
\usepackage{arydshln}
\usepackage{booktabs}       % professional-quality tables
\usepackage{amsfonts}       % blackboard math symbols
\usepackage{nicefrac}       % compact symbols for 1/2, etc.
\usepackage{microtype}      % microtypography
\usepackage{diagbox}
\usepackage{slashbox}
\usepackage{array}
% \usepackage{hyperref}
\usepackage{tabularx}
\usepackage{array}
\usepackage{multicol}
\usepackage{longtable}
\usepackage{makecell}


\usepackage[pagebackref=false,breaklinks=true,letterpaper=true,colorlinks,bookmarks=false,urlcolor=red,citecolor=cyan]{hyperref}

\newcolumntype{g}{>{\columncolor{Gray}}r}
\newcommand{\transfer}[2]{#1\% [#2]}
\newcommand{\knowledgetransfer}[1]{\mathbb{T}_{\textit{UK}}(#1)}
\newcommand{\algorithm}{\mathcal{A}}
\newcommand{\vilt}{ViLT}
\newcommand{\forgetting}[2]{\mathbb{T}_{F}(#1 \leftarrow #2)}

\newlength\savewidth\newcommand\shline{\noalign{\global\savewidth\arrayrulewidth\global\arrayrulewidth 1pt}\hline\noalign{\global\arrayrulewidth\savewidth}}

\belowrulesep=0pt
\aboverulesep=0pt

% 定义颜色 NA
\definecolor{shallowblue}{RGB}{227, 240, 249}
\definecolor{deepblue}{RGB}{192, 216, 240}
\definecolor{NA}{rgb}{1, 1, 1}
\definecolor{lightblue}{rgb}{0.93, 0.95, 1.0}
\definecolor{mygray}{gray}{.9}





% *** Do not adjust lengths that control margins, column widths, etc. ***
% *** Do not use packages that alter fonts (such as pslatex).         ***
% There should be no need to do such things with IEEEtran.cls V1.6 and later.
% (Unless specifically asked to do so by the journal or conference you plan
% to submit to, of course. )

\usepackage{epsfig}
\usepackage{graphicx}
\usepackage{amsmath}
\usepackage{amssymb}
\usepackage{enumitem}

% Include other packages here, before hyperref.

% If you comment hyperref and then uncomment it, you should delete
% egpaper.aux before re-running latex.  (Or just hit 'q' on the first latex
% run, let it finish, and you should be clear).
% \usepackage[pagebackref=true,breaklinks=true,letterpaper=true,colorlinks,bookmarks=false,urlcolor=red,citecolor=cyan]{hyperref}
\usepackage{gensymb,graphics,subfigure,inputenc}
\usepackage[linesnumbered,algo2e,boxed]{algorithm2e}
\graphicspath{{Figure/}}
\usepackage[figuresright]{rotating}
\usepackage{amsmath,amssymb,textcomp,subfigure,multirow,upgreek}
\usepackage{booktabs}
\usepackage{color,mdwlist}
% \newcommand{\thr}[1]{\rotatebox[origin=c]{90}{#1}}
% \newcommand{\tht}[2]{\begin{tabular}{@{}#1@{}}#2\end{tabular}}
% \newcommand{\hao}[1]{\textcolor{black}{#1}}

\usepackage{ragged2e}

\usepackage{review}

\usepackage{graphicx,fontawesome}
\graphicspath{{Figures/}}

\def\ci#1{\textcircled{\resizebox{.4em}{!}{#1}}}

\usepackage{caption}
\captionsetup{skip=0pt}
\usepackage{wrapfig}

\usepackage{enumitem}
\setitemize{noitemsep,topsep=0pt,parsep=0pt,partopsep=0pt}

\newcommand*\samethanks[1][\value{footnote}]{\footnotemark[#1]}


% correct bad hyphenation here
\hyphenation{op-tical net-works semi-conduc-tor}

\begin{document}
%
% paper title
% Titles are generally capitalized except for words such as a, an, and, as,
% at, but, by, for, in, nor, of, on, or, the, to and up, which are usually
% not capitalized unless they are the first or last word of the title.
% Linebreaks \\ can be used within to get better formatting as desired.
% Do not put math or special symbols in the title.
\title{When Continue Learning Meets Multimodal Large Language Model: A Survey}
%
%
% author names and IEEE memberships
% note positions of commas and nonbreaking spaces ( ~ ) LaTeX will not break
% a structure at a ~ so this keeps an author's name from being broken across
% two lines.
% use \thanks{} to gain access to the first footnote area
% a separate \thanks must be used for each paragraph as LaTeX2e's \thanks
% was not built to handle multiple paragraphs
%
%
%\IEEEcompsocitemizethanks is a special \thanks that produces the bulleted
% lists the Computer Society journals use for "first footnote" author
% affiliations. Use \IEEEcompsocthanksitem which works much like \item
% for each affiliation group. When not in compsoc mode,
% \IEEEcompsocitemizethanks becomes like \thanks and
% \IEEEcompsocthanksitem becomes a line break with idention. This
% facilitates dual compilation, although admittedly the differences in the
% desired content of \author between the different types of papers makes a
% one-size-fits-all approach a daunting prospect. For instance, compsoc 
% journal papers have the author affiliations above the "Manuscript
% received ..."  text while in non-compsoc journals this is reversed. Sigh.

%\author{Michael~Shell,~\IEEEmembership{Member,~IEEE,}
%        John~Doe,~\IEEEmembership{Fellow,~OSA,}
%        and~Jane~Doe,~\IEEEmembership{Life~Fellow,~IEEE}% <-this % stops a space
%\IEEEcompsocitemizethanks{\IEEEcompsocthanksitem M. Shell was with the Department
%of Electrical and Computer Engineering, Georgia Institute of Technology, Atlanta,
%GA, 30332.\protect\\
%% note need leading \protect in front of \\ to get a newline within \thanks as
%% \\ is fragile and will error, could use \hfil\break instead.
%E-mail: see http://www.michaelshell.org/contact.html
%\IEEEcompsocthanksitem J. Doe and J. Doe are with Anonymous University.}% <-this % stops an unwanted space
%\thanks{Manuscript received April 19, 2005; revised August 26, 2015.}}
\author{Yukang Huo \quad Hao Tang$^*$
	\IEEEcompsocitemizethanks{
	   \IEEEcompsocthanksitem Yukang Huo is with the School of College of Information and Electrical Engineering, China Agricultural University, Beijing 100193, China. E-mail: yukanghuo.ai@gmail.com\protect
        \IEEEcompsocthanksitem Hao Tang is with the School of Computer Science, Peking University, Beijing 100871, China. E-mail: haotang@pku.edu.cn \protect
        }% <-this % stops an unwanted space
	\thanks{$^*$Corresponding author: Hao Tang.}
 }
% note the % following the last \IEEEmembership and also \thanks - 
% these prevent an unwanted space from occurring between the last author name
% and the end of the author line. i.e., if you had this:
% 
% \author{....lastname \thanks{...} \thanks{...} }
%                     ^------------^------------^----Do not want these spaces!
%
% a space would be appended to the last name and could cause every name on that
% line to be shifted left slightly. This is one of those "LaTeX things". For
% instance, "\textbf{A} \textbf{B}" will typeset as "A B" not "AB". To get
% "AB" then you have to do: "\textbf{A}\textbf{B}"
% \thanks is no different in this regard, so shield the last } of each \thanks
% that ends a line with a % and do not let a space in before the next \thanks.
% Spaces after \IEEEmembership other than the last one are OK (and needed) as
% you are supposed to have spaces between the names. For what it is worth,
% this is a minor point as most people would not even notice if the said evil
% space somehow managed to creep in.

% The paper headers
\markboth{IEEE Transactions on Pattern Analysis and Machine Intelligence}%
{Shell \MakeLowercase{\textit{et al.}}: Bare Demo of IEEEtran.cls for Computer Society Journals}
% The only time the second header will appear is for the odd numbered pages
% after the title page when using the twoside option.
% 
% *** Note that you probably will NOT want to include the author's ***
% *** name in the headers of peer review papers.                   ***
% You can use \ifCLASSOPTIONpeerreview for conditional compilation here if
% you desire.



% The publisher's ID mark at the bottom of the page is less important with
% Computer Society journal papers as those publications place the marks
% outside of the main text columns and, therefore, unlike regular IEEE
% journals, the available text space is not reduced by their presence.
% If you want to put a publisher's ID mark on the page you can do it like
% this:
%\IEEEpubid{0000--0000/00\$00.00~\copyright~2015 IEEE}
% or like this to get the Computer Society new two part style.
%\IEEEpubid{\makebox[\columnwidth]{\hfill 0000--0000/00/\$00.00~\copyright~2015 IEEE}%
%\hspace{\columnsep}\makebox[\columnwidth]{Published by the IEEE Computer Society\hfill}}
% Remember, if you use this you must call \IEEEpubidadjcol in the second
% column for its text to clear the IEEEpubid mark (Computer Society jorunal
% papers don't need this extra clearance.)



% use for special paper notices
%\IEEEspecialpapernotice{(Invited Paper)}



% for Computer Society papers, we must declare the abstract and index terms
% PRIOR to the title within the \IEEEtitleabstractindextext IEEEtran
% command as these need to go into the title area created by \maketitle.
% As a general rule, do not put math, special symbols or citations
% in the abstract or keywords.
\IEEEtitleabstractindextext{%
%\begin{abstract}
%The abstract goes here.
%\end{abstract}
\begin{abstract}
We present an image blending pipeline, \textit{IBURD}, that creates realistic synthetic images to assist in the training of deep detectors for use on underwater autonomous vehicles (AUVs) for marine debris detection tasks. 
Specifically, IBURD generates both images of underwater debris and their pixel-level annotations, using source images of debris objects, their annotations, and target background images of marine environments. 
With Poisson editing and style transfer techniques, IBURD is even able to robustly blend transparent objects into arbitrary backgrounds and automatically adjust the style of blended images using the blurriness metric of target background images. 
These generated images of marine debris in actual underwater backgrounds address the data scarcity and data variety problems faced by deep-learned vision algorithms in challenging underwater conditions, and can enable the use of AUVs for environmental cleanup missions. 
Both quantitative and robotic evaluations of IBURD demonstrate the efficacy of the proposed approach for robotic detection of marine debris. 
\end{abstract}




% Note that keywords are not normally used for peerreview papers.
\begin{IEEEkeywords}
Multimodal Large Language Model, Continual Learning, Benchmark Evaluations, Model Innovation, Catastrophic Forgetting
\end{IEEEkeywords}}


% make the title area
\maketitle


% To allow for easy dual compilation without having to reenter the
% abstract/keywords data, the \IEEEtitleabstractindextext text will
% not be used in maketitle, but will appear (i.e., to be "transported")
% here as \IEEEdisplaynontitleabstractindextext when the compsoc 
% or transmag modes are not selected <OR> if conference mode is selected 
% - because all conference papers position the abstract like regular
% papers do.
\IEEEdisplaynontitleabstractindextext
% \IEEEdisplaynontitleabstractindextext has no effect when using
% compsoc or transmag under a non-conference mode.



% For peer review papers, you can put extra information on the cover
% page as needed:
% \ifCLASSOPTIONpeerreview
% \begin{center} \bfseries EDICS Category: 3-BBND \end{center}
% \fi
%
% For peerreview papers, this IEEEtran command inserts a page break and
% creates the second title. It will be ignored for other modes.
\IEEEpeerreviewmaketitle




% Computer Society journal (but not conference!) papers do something unusual
% with the very first section heading (almost always called "Introduction").
% They place it ABOVE the main text! IEEEtran.cls does not automatically do
% this for you, but you can achieve this effect with the provided
% \IEEEraisesectionheading{} command. Note the need to keep any \label that
% is to refer to the section immediately after \section in the above as
% \IEEEraisesectionheading puts \section within a raised box.


% The very first letter is a 2 line initial drop letter followed
% by the rest of the first word in caps (small caps for compsoc).
% 
% form to use if the first word consists of a single letter:
% \IEEEPARstart{A}{demo} file is ....
% 
% form to use if you need the single drop letter followed by
% normal text (unknown if ever used by the IEEE):
% \IEEEPARstart{A}{}demo file is ....
% 
% Some journals put the first two words in caps:
% \IEEEPARstart{T}{his demo} file is ....
% 
% Here we have the typical use of a "T" for an initial drop letter
% and "HIS" in caps to complete the first word.

\section{Introduction}

In today’s rapidly evolving digital landscape, the transformative power of web technologies has redefined not only how services are delivered but also how complex tasks are approached. Web-based systems have become increasingly prevalent in risk control across various domains. This widespread adoption is due their accessibility, scalability, and ability to remotely connect various types of users. For example, these systems are used for process safety management in industry~\cite{kannan2016web}, safety risk early warning in urban construction~\cite{ding2013development}, and safe monitoring of infrastructural systems~\cite{repetto2018web}. Within these web-based risk management systems, the source search problem presents a huge challenge. Source search refers to the task of identifying the origin of a risky event, such as a gas leak and the emission point of toxic substances. This source search capability is crucial for effective risk management and decision-making.

Traditional approaches to implementing source search capabilities into the web systems often rely on solely algorithmic solutions~\cite{ristic2016study}. These methods, while relatively straightforward to implement, often struggle to achieve acceptable performances due to algorithmic local optima and complex unknown environments~\cite{zhao2020searching}. More recently, web crowdsourcing has emerged as a promising alternative for tackling the source search problem by incorporating human efforts in these web systems on-the-fly~\cite{zhao2024user}. This approach outsources the task of addressing issues encountered during the source search process to human workers, leveraging their capabilities to enhance system performance.

These solutions often employ a human-AI collaborative way~\cite{zhao2023leveraging} where algorithms handle exploration-exploitation and report the encountered problems while human workers resolve complex decision-making bottlenecks to help the algorithms getting rid of local deadlocks~\cite{zhao2022crowd}. Although effective, this paradigm suffers from two inherent limitations: increased operational costs from continuous human intervention, and slow response times of human workers due to sequential decision-making. These challenges motivate our investigation into developing autonomous systems that preserve human-like reasoning capabilities while reducing dependency on massive crowdsourced labor.

Furthermore, recent advancements in large language models (LLMs)~\cite{chang2024survey} and multi-modal LLMs (MLLMs)~\cite{huang2023chatgpt} have unveiled promising avenues for addressing these challenges. One clear opportunity involves the seamless integration of visual understanding and linguistic reasoning for robust decision-making in search tasks. However, whether large models-assisted source search is really effective and efficient for improving the current source search algorithms~\cite{ji2022source} remains unknown. \textit{To address the research gap, we are particularly interested in answering the following two research questions in this work:}

\textbf{\textit{RQ1: }}How can source search capabilities be integrated into web-based systems to support decision-making in time-sensitive risk management scenarios? 
% \sq{I mention ``time-sensitive'' here because I feel like we shall say something about the response time -- LLM has to be faster than humans}

\textbf{\textit{RQ2: }}How can MLLMs and LLMs enhance the effectiveness and efficiency of existing source search algorithms? 

% \textit{\textbf{RQ2:}} To what extent does the performance of large models-assisted search align with or approach the effectiveness of human-AI collaborative search? 

To answer the research questions, we propose a novel framework called Auto-\
S$^2$earch (\textbf{Auto}nomous \textbf{S}ource \textbf{Search}) and implement a prototype system that leverages advanced web technologies to simulate real-world conditions for zero-shot source search. Unlike traditional methods that rely on pre-defined heuristics or extensive human intervention, AutoS$^2$earch employs a carefully designed prompt that encapsulates human rationales, thereby guiding the MLLM to generate coherent and accurate scene descriptions from visual inputs about four directional choices. Based on these language-based descriptions, the LLM is enabled to determine the optimal directional choice through chain-of-thought (CoT) reasoning. Comprehensive empirical validation demonstrates that AutoS$^2$-\ 
earch achieves a success rate of 95–98\%, closely approaching the performance of human-AI collaborative search across 20 benchmark scenarios~\cite{zhao2023leveraging}. 

Our work indicates that the role of humans in future web crowdsourcing tasks may evolve from executors to validators or supervisors. Furthermore, incorporating explanations of LLM decisions into web-based system interfaces has the potential to help humans enhance task performance in risk control.








\begin{table*}[htbp]
\small
\renewcommand\arraystretch{1.2}
  \centering
  \caption{Innovations in MLLM Frameworks.}
    \begin{tabularx}{\textwidth}{>{\centering\arraybackslash}m{2cm}|p{0.4\textwidth}|p{0.4\textwidth}}
    \hline
   \multicolumn{1}{c|}{MLLMs} & \multicolumn{1}{c|}{Starting point of the problem} & \multicolumn{1}{c}{How to solve} \\
   
   \hline
   \multirow{2}{*}{\textbf{MaVEn}~\cite{jiang2024maven}} & 
Enhancing the image visual understanding of MLLMs.& MaVEn proposes an effective multi-granularity hybrid visual encoding framework.\\
   
   \hline
   \multirow{2}{*}{\textbf{MoVA}~\cite{zong2024mova}} & 
No single visual encoder can dominate the understanding of various image contents.& MoVA incorporates coarse-grained context-aware expert routing and fine-grained expert fusion.\\

   \hline
   \multirow{2}{*}{\textbf{MoME}~\cite{shen2024mome}} & 
The performance of general-purpose MLLMs is typically inferior to that of expert MLLMs. & MoME combines the MoVE and the MoLE to reduce task interference.\\

   \hline
   \multirow{2}{*}{\textbf{Meteor}~\cite{lee2024meteor}} & 
The performance gap of MLLMs in understanding and answering complex questions.& Meteor introduced the new concept of "traversal of rationales."\\

    \hline
   \multirow{2}{*}{\textbf{CORY}~\cite{ma2024coevolving}} & 
The stability and performance issues MLLMs encounter in RL fine-tuning.& CORY leverages the inherent cooperative evolution and emergence capabilities of multi-agent systems.\\

    \hline
   \multirow{2}{*}{\textbf{Lumen}~\cite{jiao2024lumen}} & 
MLMs overlook the intrinsic characteristics of different visual tasks.& Lumen enhances multimodal understanding by separating task-agnostic and task-specific learning.\\

    \hline
   \multirow{2}{*}{\textbf{Octopus}~\cite{zhaooctopus}} & 
MLLMs combine visual recognition and understanding sequentially at the LLM, which is suboptimal.& Octopus proposed the "Parallel Recognition → Sequential Understanding" MLLM framework.\\

    \hline
   \multirow{2}{*}{\textbf{Wings}~\cite{zhang2024wings}} & 
MLLMs tend to forget knowledge acquired from text-only instructions during training.& Wings introduces additional modules and mechanisms to compensate for attention shifts.\\

    \hline
   \multirow{2}{*}{\textbf{Cantor}~\cite{gao2024cantor}} & 
The "hallucination" problem in decision-making is caused by insufficient visual information.& Cantor inspires a multimodal chain-of-thought of MLLM.\\

    \hline
   \multirow{2}{*}{\textbf{AutoM3L}~\cite{luo2024autom3l}} & 
The limitations of automation in multimodal machine learning.& AutoM3L proposes an automated multimodal machine learning framework with MLLMs.\\

    \hline
   \multirow{2}{*}{\textbf{DI-MML}~\cite{fan2024detached}} & 
The modality competition issue in multimodal learning.& DI-MML proposes detached and interactive multimodal learning.\\

    \hline
   \multirow{2}{*}{\textbf{MEM}~\cite{liu2024multimodal}} & 
Data scraped from networks may leak personal privacy.& MEM optimizes by combining image noise and text triggers to mislead the model into learning shortcuts.\\

    \hline
   \multirow{2}{*}{\textbf{CREAM}~\cite{zhang2024cream}} & 
The lack of cross-page interaction support in document visual question answering.& CREAM proposes Coarse-to-Fine retrieval and multi-modal efficient tuning for document VQA.\\

    \hline
   \multirow{3}{*}{\textbf{SLUDA}~\cite{zheng2024self}} & 
Insufficient labeled data and the underutilization of unlabeled data.& SLUDA generates fine-grained data, optimizes unlabeled data usage, and employs adaptive selection and dynamic threshold strategies. \\

    \hline
   \multirow{2}{*}{\textbf{SAM}~\cite{wu2024semantic}} & 
The semantic alignment issue in MLLMs when processing multi-image instructions.& SAM enhances image-semantic associations through a bidirectional semantic guidance mechanism. \\

    \hline
   \multirow{3}{*}{\textbf{CTVLMs}~\cite{lu2024collaborative}} & 
Improving performance and reducing computational resource demands in MLLMs for multimodal tasks.& CTVLMs use knowledge distillation and multimodal alignment to transfer knowledge from large models to smaller ones.\\

    \hline
   \multirow{2}{*}{\textbf{Bloom}~\cite{kim2024efficient}} & 
Reducing the high computational cost of large-scale multilingual visual data modeling.& Bloom proposes pre-training with discretized visual speech representation.\\

    \hline
   \multirow{4}{*}{\shortstack{\textbf{MA-AGIQA} \\ \cite{wang2024large}}}& 
The quality evaluation issue of AI-generated images (AGIs).& MA-AGIQA combines multimodal models and traditional DNNs, utilizing semantic information extraction and the mixture of experts (MoE) structure to dynamically integrate quality-aware features.\\

    \hline
   \multirow{2}{*}{\makecell{\textbf{WorldGPT}\\~\cite{ge2024worldgpt}}} & Enhancing the applicability and generalization ability of MLLMs.& WorldGPT includes memory offloading, knowledge retrieval, and a Context Reflector.\\

    \hline
   \multirow{2}{*}{\makecell{\textbf{Q-ALIGN}\\~\cite{wu2023q}}} & 
Enhancing the applicability and generalization ability of MLLMs.& Q-ALIGN unifies IQA, IAA, and VQA tasks to enhance the model's cross-task generalization ability.\\

    \hline
   \multirow{2}{*}{\textbf{Flextron}~\cite{cai2024flextron}} & 
The deployment challenges of MLLMs in resource-constrained environments.& Flextron selects different sub-models or sub-networks by using routers.\\

    \hline
   \multirow{2}{*}{\makecell{\textbf{NExT-GPT}\\~\cite{wu2023next}}} & 
Existing MLLMs can only understand the input modality.& NExT-GPT proposes lightweight alignment techniques and modality-switching instruction tuning.\\


    \hline
    \end{tabularx}
  \label{MLLM_frame}%
  \vspace{-5mm} 
\end{table*}%



\section{Multimodal Large Language Model}

\subsection{Preliminary}
In this section, we provide an overview of the latest research on MLLMs, including various model innovation strategies, a range of benchmarks, and the application of MLLMs in diverse domains.

\begin{table*}[htbp]
\small
\renewcommand\arraystretch{1.2}
  \centering
  \caption{Innovations in MLLM Methods.}
    \begin{tabularx}{\textwidth}{>{\centering\arraybackslash}m{2cm}|p{0.4\textwidth}|p{0.4\textwidth}}
    \hline
   \multicolumn{1}{c|}{Method} & \multicolumn{1}{c|}{Starting point of the problem} & \multicolumn{1}{c}{How to solve} \\
   
   \hline
   \multirow{2}{*}{\makecell{\textbf{DenseFusion}\\~\cite{li2024densefusion}}} & 
Enhancing the visual perception ability of MLLMs.& DenseFusion proposes a multimodal perception fusion method that integrates visual experts.\\
      
   \hline
   \multirow{2}{*}{\textbf{E2E-MFD}~\cite{zhang2024e2e}} & 
The complex training process hinders the broader application of MLLMs.& E2E-MFD proposes a novel end-to-end algorithm for multimodal fusion detection.\\
      
   \hline
   \multirow{2}{*}{\textbf{NAM}~\cite{fangtowards}} & 
Neuron attribution in MLLMs has not been fully explored yet.& NAM proposes a neuron attribution method tailored for MLLMs.\\
      
   \hline
   \multirow{2}{*}{\textbf{CODE}~\cite{kim2024code}} & 
Addressing the hallucination problem in MLLMs when generating visual content.& CODE utilizes self-generated descriptions as contrastive references to adjust the information flow.\\
      
   \hline
   \multirow{2}{*}{\makecell{\textbf{MULTEDIT}\\~\cite{basu2024understanding}}} & 
To correct errors and insert new information. & MULTEDIT introduces a multimodal causal tracking method.\\
      
   \hline
   \multirow{2}{*}{\textbf{QSLAW}~\cite{xie2024advancing}} & 
Tackling the resource consumption issue faced by MLLMs in visual-language instruction tuning. & QSLAW learns group scale factors of quantized weights and adopts multimodal pretraining method.\\
      
   \hline
   \multirow{2}{*}{\textbf{LECCR}~\cite{wang2024multimodal}} & 
To improve the quality of cross-modal alignment.& LECCR proposes the MLLM-enhanced cross-lingual, cross-modal retrieval method.\\
      
   \hline
   \multirow{2}{*}{\textbf{ERL-MR}~\cite{han2024erl}} & 
To address the modality imbalance problem in MLLMs.& ERL-MR uses Euler transformations and multimodal constraint loss.\\
      
   \hline
   \multirow{2}{*}{\textbf{AMMPL}~\cite{wu2024adaptive}} & 
Enhancing the model's performance and reasoning ability. & AMMPL proposes an adaptive multimodal prompt learning method.\\
      
   \hline
   \multirow{2}{*}{\textbf{PaRe}~\cite{cai2024enhancing}} & 
Enhancing the model's performance and reasoning ability. & PaRe progressively generates intermediate modalities and replaces modality-agnostic fragments.\\
      
   \hline
   \multirow{2}{*}{\textbf{MCL}~\cite{liimproving}} & 
Addressing the insufficient interaction problem when handling complex multimodal scenarios. & MCL proposes the multimodal combination learning (MCL) method.\\
      
   \hline
   \multirow{2}{*}{\textbf{FARE}~\cite{schlarmann2024robust}} & 
MLLMs are vulnerable to adversarial attacks in the visual modality. & FARE proposes the unsupervised adversarial fine-tuning scheme.\\
      
   \hline
   \multirow{2}{*}{\textbf{DICL}~\cite{huang2023machine}} & 
Reducing the reliance on manual annotations. & DICL leverages MLLMs knowledge to enhance the robustness of visual models.\\
      
   \hline
   \multirow{2}{*}{\textbf{API}~\cite{yu2025attention}} & 
Addressing the limitations of traditional visual prompting techniques. & API enhances model perception through attention heatmaps guided by text queries.\\
      
   \hline
   \multirow{2}{*}{\textbf{IVTP}~\cite{huang2025ivtp}} & 
Addressing the high computational cost problem in MLLMs. & IVTP proposeS the instruction-guided visual token pruning method.\\


   \hline
   \multirow{2}{*}{\makecell{\textbf{ChatTracker}\\~\cite{sun2024chattracker}}} & 
Enhancing the tracking performance of MLLM trackers.& ChatTracker proposes a novel reflection-based prompt optimization module.\\
   
   \hline
   \multirow{2}{*}{\makecell{\textbf{Optimus-1}\\~\cite{li2024optimus}}} & 
Current general agents lack the necessary world knowledge and multimodal experience.& Optimus-1 proposes a hybrid multimodal memory module.\\
   
   \hline
   \multirow{2}{*}{\textbf{CuMo}~\cite{li2024cumo}} & 
Improving the performance of MLLMs on multimodal tasks.& CuMo integrates sparse gated Top-K MoE blocks in the visual encoder and MLP connectors.\\
   
   \hline
   \multirow{2}{*}{\makecell{\textbf{AcFormer}\\~\cite{liu2024visual}}} & 
The connection between visual encoders and LLMs has limitations.& AcFormer identified visual anchors and proposed a novel vision-language connector\\
   
   \hline
   \multirow{2}{*}{\makecell{\textbf{Chain-of-Sight}\\~\cite{huang2024accelerating}}} & 
Accelerating the pretraining process and improving model performance.& Chain-of-Sight captures visual details at different spatial scales through a multi-scale visual resampler.\\
   
   \hline
   \multirow{2}{*}{\makecell{\textbf{Dense Con-}\\ \textbf{nector}~\cite{yao2024dense}}} & 
Existing MLLMs underutilise the visual encoder while overly emphasising the language modality.& Dense Connector enhances the visual perception ability by integrating multi-layer visual features.\\
   
   \hline
   \multirow{2}{*}{\textbf{GCG}~\cite{wang2024weakly}} & 
In video question answering, MLLMs overlook visually relevant cues related to the question.& GCG learns to represent the temporal structure of videos and selects key frames.\\
   
   \hline
   \multirow{2}{*}{\textbf{Q-MoE}~\cite{wang2024q}} & 
Connection structure struggles with filtering visual information according to task requirements.& Q-MoE proposes a query-based hybrid expert connector.\\



    \hline
    \end{tabularx}
  \label{MLLM_method}%
  \vspace{-5mm} 
\end{table*}%

\subsection{Model Innovation}
With the continuous development of MLLMs, researchers have made various innovations in their structure, methods, and functional modules to enhance model performance, generalization ability, and adaptability. This section reviews the main innovations, which focus on three core directions: framework design, method optimization, and functional module improvements. These innovations collectively drive the performance of MLLMs in complex multimodal tasks. This section will explore the latest research advancements in these areas.

\subsubsection{Framework Innovation}
Framework innovation is the foundation of MLLM development, aiming to achieve efficient fusion and processing of cross-modal information by improving the overall architectural design. In recent years, researchers have proposed many efficient framework designs. As shown in Table \ref{MLLM_frame}, researchers have proposed several efficient framework designs, such as MaVEn, MoVA, AutoM3L, DI-MML and et. These framework innovations provide more efficient tools and methods for MLLMs to handle multimodal tasks involving language, vision, and hearing. They enable MLLMs to achieve more precise reasoning and decision-making in the interaction of multimodal data, thereby offering strong support for solving complex problems in practical applications.
More details of the innovation of MLLMs frameworks are provided in Section \ref{appendix_MLLM_MI} of the Appendix.

\subsubsection{Method Innovation}

Method innovation is the core driving force behind the performance improvement of MLLMs. By designing more efficient training methods and optimization objectives, it helps models better adapt to dynamic task environments. As shown in Table~\ref{MLLM_method}, in recent years, researchers have proposed numerous novel and efficient methods to enhance the accuracy and robustness of MLLMs. These method research has explored cutting-edge techniques such as multimodal contrastive learning, self-supervised learning objectives, and multimodal alignment mechanisms. These methods not only enhance the model's generalization ability but also significantly improve the accuracy and robustness of cross-modal tasks.
More details of the innovation of MLLMs methods are provided in Section \ref{appendix_MLLM_MI} of the Appendix.



% \subsubsection{Module Innovation}

% In terms of module innovation, researchers have focused on refining the internal design of models by improving specific modules to enhance cross-modal interaction and representational capabilities. These modules have increased the flexibility and efficiency of the models. As shown in Table~\ref{MLLM_module}, in recent years, many efficient and streamlined modules have been proposed, further boosting the overall performance of the models.These modules, through independent optimization and joint modeling of different modalities, enable the model to perform exceptionally well in tasks such as multimodal reasoning, generation, and understanding. They allow for more accurate cross-modal knowledge fusion and semantic understanding, providing crucial support for the functional expansion and practical application of MLLMs.

% More details of the innovation of MLLMs Modules are provided in Section \ref{appendix_MLLM_MI} of the Appendix.





% \begin{table*}[htbp]
% \small
% \renewcommand\arraystretch{1.2}
%   \centering
%   \caption{Innovations in Multimodal Large Model Modules.}
%     \begin{tabularx}{\textwidth}{>{\centering\arraybackslash}m{2cm}|p{0.4\textwidth}|p{0.4\textwidth}}
%     \hline
%    \multicolumn{1}{c|}{Module} & \multicolumn{1}{c|}{Starting point of the problem} & \multicolumn{1}{c}{How to solve} \\
   




%     \hline
%     \end{tabularx}
%   \label{MLLM_module}%
%   \vspace{-5mm} 
% \end{table*}%


\begin{table*}[htbp]
\small
\renewcommand\arraystretch{1.2}
  \centering
  \caption{Innovations in Non-LLM Unimodal CL Frameworks.}
    \begin{tabularx}{\textwidth}{>{\centering\arraybackslash}m{2cm}|p{0.4\textwidth}|p{0.4\textwidth}}
    \hline
   \multicolumn{1}{c|}{Framework} & \multicolumn{1}{c|}{Starting point of the problem} & \multicolumn{1}{c}{How to solve} \\
   
   \hline
   \multirow{2}{*}{\textbf{NTE}~\cite{benjamin2024continual}} & 
Addressing the catastrophic forgetting problem in graph neural networks.& NTE views a neural network as an ensemble of fixed experts.\\
   
   \hline
   \multirow{2}{*}{\textbf{IsCiL}~\cite{lee2024incremental}} & 
To address the issue of new data lacking labels due to annotation delays in continual learning.&IsCiL improves sample efficiency and task adaptability by incrementally learning shared skills.\\
   
   \hline
   \multirow{3}{*}{\textbf{CKP}~\cite{xu2024mitigate}} & 
To address the performance degradation caused by incorrect labels in the Lifelong Person Re-Identification task.& CKP purifies data through the CDP and ILR modules, and filters out erroneous knowledge using the EKF algorithm.\\
   
   \hline
   \multirow{2}{*}{\textbf{PBR}~\cite{liu2024prior}} & 
To reduce forgetting and enhances long-tail continual learning performance.& PBR proposes an uncertainty-guided sampling strategy and two prior-free constraints.\\
   
   \hline
   \multirow{2}{*}{\textbf{OSN}~\cite{hutask}} & 
Reducing the interference of new tasks on old tasks.& OSN explores shared knowledge between old and new tasks through parameter sharing.\\
   
   \hline
   \multirow{2}{*}{\textbf{MoDE}~\cite{lee2024becotta}} & 
Improving adaptation to new domains while preserving old knowledge.& MoDE includes domain-adaptive routing and domain-expert collaborative loss.\\
   
   \hline
   \multirow{2}{*}{\textbf{SB-MCL}~\cite{lee2024learning}} & 
To address the catastrophic forgetting problem in continual learning.& SB-MCL achieves continual learning through sequential Bayesian updates.\\
   
   \hline
   \multirow{2}{*}{\textbf{PNR}~\cite{charegularizing}} & 
Addressing the knowledge transfer and catastrophic forgetting issues.& PNR Generates pseudo-negative samples and optimizing knowledge transfer.\\
   
   \hline
   \multirow{2}{*}{\makecell{\textbf{CompoNet}\\~\cite{malagonself}}} & 
Addressing the issue of old task forgetting caused in continual reinforcement learning.& CompoNet proposes a modular neural network with linearly growing parameters.\\
   
   \hline
   \multirow{2}{*}{\makecell{\textbf{Vector-HaSH}\\~\cite{wangrapid}}} & 
To enable fast learning and continual memory.& Vector-HaSH combines hetero-associative memory and spatially invariant CNNs.\\
   
   \hline
   \multirow{2}{*}{\textbf{DDDR}~\cite{liang2025diffusion}} & 
Addressing the issue of catastrophic forgetting in federated continual learning.& DDDR uses diffusion models to generate historical data and employs contrastive learning.\\
   
   \hline
   \multirow{2}{*}{\makecell{\textbf{PromptCCD}\\~\cite{cendra2025promptccd}}} & 
Mitigating catastrophic forgetting.& PromptCCD introduces the GMP, which dynamically generates prompts to adapt to new classes.\\
   
   \hline
   \multirow{2}{*}{\textbf{Mecoin}~\cite{li2024efficient}} & 
To reduce parameter fine-tuning, lower the forgetting rate.& Mecoin employs SMU and a MeCo for efficient storage and updating of class prototypes.\\
   
   \hline
   \multirow{2}{*}{\textbf{RP2F}~\cite{sun2024incremental}} & 
Enabling effective knowledge sharing and backward knowledge transfer.& RP2F uses perturbation methods to approximate the Hessian matrix and introduces a prior.\\
   
   \hline
   \multirow{2}{*}{\makecell{\textbf{HAMMER}\\~\cite{liu2024hierarchical}}} & 
To address the catastrophic forgetting issue in multilingual text recognition.& HAMMER proposes online knowledge analysis and a hierarchical language evaluation mechanism.\\
   
   \hline
   \multirow{2}{*}{\textbf{FedCBC}~\cite{yu2024overcoming}} & 
Mitigating catastrophic forgetting.& FedCBC proposes category-specific binary classifiers and selective knowledge fusion.\\
   
   \hline
   \multirow{2}{*}{\textbf{TS-ILM}~\cite{xiaochen2024ts}} & 
Reducing information redundancy and enhancing memory retention.& TS-ILM proposes a task-level temporal pattern extractor and a time-sensitive example selector.\\
   
   \hline
   \multirow{2}{*}{\makecell{\textbf{AutoActivator}\\~\cite{li2024harnessing}}} & 
To address the issue of model forgetting old classes when continuously learning new classes.& AutoActivator dynamically adapts neural units to new tasks, enabling on-demand network expansion.\\
   
   \hline
   \multirow{2}{*}{\textbf{iNeMo}~\cite{fischer2024inemo}} & 
To achieve efficient class-incremental learning.& iNeMo proposes latent space initialization and position regularization.\\
   
   \hline
   \multirow{3}{*}{\textbf{TACO}~\cite{han2024topology}} & 
Offering a novel perspective for understanding and mitigating catastrophic forgetting.& TACO combines graph coarsening and continual learning to dynamically store information from previous tasks.\\



    \hline
    \end{tabularx}
  \label{CL_NonL_Framework}%
  \vspace{-5mm} 
\end{table*}%

\subsection{Benchmarks}
As MLLMs continue to achieve breakthroughs in multimodal tasks such as vision, language, and speech, comprehensive benchmarks have become crucial for systematically evaluating and comparing model performance. These benchmarks not only provide standardized datasets and tasks, but also define metrics for assessing models' abilities in cross-modal reasoning, generation, classification, and other areas. They play a key role in guiding research directions, identifying model limitations, and advancing technological progress. More details of the overview of MLLM benchmarks are provided in Section~\ref{appendix_benchmarks} of the Appendix. Section~\ref{appendix_benchmarks} in the Appendix introduces some of the recent representative benchmarks, covering a wide range of scenarios from academic research to practical applications, reflecting the diverse needs and challenges in the multimodal field.


\begin{table*}[htbp]
\small
\renewcommand\arraystretch{1.2}
  \centering
  \caption{Innovations in Non-LLM Unimodal CL Methods.}
    \begin{tabularx}{\textwidth}{>{\centering\arraybackslash}m{2cm}|p{0.4\textwidth}|p{0.4\textwidth}}
    \hline
   \multicolumn{1}{c|}{Method} & \multicolumn{1}{c|}{Starting point of the problem} & \multicolumn{1}{c}{How to solve} \\
   
   \hline
   \multirow{2}{*}{\textbf{GACL}~\cite{zhuang2024gacl}} & 
Addressing the catastrophic forgetting problem of models in class-incremental learning.& GACL establishes the equivalence between incremental learning and joint training.\\
   
   \hline
   \multirow{2}{*}{\textbf{C-Flat}~\cite{zhuang2024gacl}} & 
Addressing the balance between new task training sensitivity and memory retention.& C-Flat optimizes the flatness of the loss landscape.\\
   
   \hline
   \multirow{2}{*}{\textbf{DSGD}~\cite{fan2024dynamic}} & 
Addressing the practical deployment challenge.& DSGD uses structural and semantic information for stable knowledge distillation.\\
   
   \hline
   \multirow{2}{*}{\makecell{\textbf{VQ-Prompt}\\~\cite{jiao2024vector}}} & 
To improve continual learning performance.& VQ-Prompt utilizes vector quantization to achieve end-to-end optimization of discrete prompt selection.\\
   
   \hline
   \multirow{2}{*}{\makecell{\textbf{RanDumb}\\~\cite{prabhurandom}}} & 
Exploring whether the representations generated by continual learning algorithms are truly effective.& RanDumb uses random transformations and linear classifiers to address.\\
   
   \hline
   \multirow{2}{*}{\textbf{IWMS}~\cite{csabalabel}} & 
The label delay issue in online continual learning.& IWMS prioritizes the memory of samples similar to new data.\\
   
   \hline
   \multirow{2}{*}{\textbf{PPE}~\cite{li2024progressive}} & 
To address the catastrophic forgetting problem in non-sample online continual learning.& PPE learns class prototypes during the online learning phase.\\
   
   \hline
   \multirow{2}{*}{\textbf{GPCNS}~\cite{yang2024introducing}} & 
Improving the performance of continual learning.& GPCNS enhances plasticity by utilizing gradient information from old tasks.\\
   
%    \hline
%    \multirow{2}{*}{\makecell{\textbf{Bayesian}\\ \textbf{Adaptation}~\cite{thapabayesian}}} & 
% Improving the performance of continual learning.& Non-parametric Bayesian adapts the width through a conjugate Bernoulli process.\\
   
   \hline
   \multirow{2}{*}{\textbf{CILA}~\cite{wen2024provable}} & 
Improving the performance of continual learning.& CILA proposes an adaptive distillation coefficient and theoretical performance guarantees.\\
   
   \hline
   \multirow{2}{*}{\textbf{POCL}~\cite{wumitigating}} & 
Existing methods fail to fully leverage the inter-task dependencies.& POCL models task relationships through Pareto optimization and dynamically adjusts weights.\\
   
   \hline
   \multirow{2}{*}{\textbf{Powder}~\cite{piaofederated}} & 
Addressing the cross-task and cross-client knowledge transfer in federated continual learning.& Powder enables prompt-based dual knowledge transfer.\\
   
   \hline
   \multirow{2}{*}{\makecell{\textbf{AdaPromptCL}\\~\cite{kim2023one}}} & 
Addressing the challenge of task-specific semantic variations.& AdaPromptCL proposes dynamic semantic grouping and prompt adjustment.\\
   
   \hline
   \multirow{2}{*}{\textbf{LPR}~\cite{kim2023one}} & 
To reduce catastrophic forgetting and underfitting.& LPR adjusts the optimization geometry to balance the learning of new and old data.\\
   
   \hline
   \multirow{2}{*}{\textbf{InfLoRA}~\cite{liang2024inflora}} & 
To address the issue of forgetting old tasks when adapting to new tasks.& InfLoRA injects parameter reparameterization into pre-trained weights.\\
   
   \hline
   \multirow{2}{*}{\textbf{F-OAL}~\cite{zhuangf}} & 
To alleviate the issue of catastrophic forgetting in online class-incremental learning.& F-OAL proposes a forward online analytical learning method.\\
   
   \hline
   \multirow{2}{*}{\textbf{PRL}~\cite{shiprospective}} & 
Improving performance in non-sample class-incremental learning.& PRL aligns reserved space and latent space to adapt new class features to the reserved space.\\
   
   \hline
   \multirow{2}{*}{\textbf{CIL}~\cite{hao2024addressing}} & 
To address the issue of catastrophic forgetting.& CIL proposes the CIL-balanced classification loss and distribution margin loss.\\
   
   \hline
   \multirow{2}{*}{\textbf{DSSP}~\cite{yang2024domain}} & 
To eliminate the need for sample replay.& DSSP leverages domain sharing and task-specific prompt learning.\\
   
   \hline
   \multirow{2}{*}{\textbf{MRFA}~\cite{zhengmulti}} & 
To reduce catastrophic forgetting.& MRFA optimizes the entire layer margin by enhancing the features of review samples.\\
   
   \hline
   \multirow{2}{*}{\textbf{DARE}~\cite{jeeveswaran2024gradual}} & 
Improving the model's performance on old tasks.& DARE reduces representation drift through a three-stage training process.\\
   
   \hline
   \multirow{2}{*}{\textbf{EASE}~\cite{zhou2024expandable}} & 
To reduce catastrophic forgetting.& EASE constructs task-specific subspaces using lightweight adapters.\\



    \hline
    \end{tabularx}
  \label{CL_NonL_Method}%
  \vspace{-5mm} 
\end{table*}%



\subsection{Applications of MLLMs}

Multimodal large models (MLLMs) have emerged as a significant direction in artificial intelligence research in recent years~\cite{zhan2024anygpt,chiang2023vicuna,jiang2023motiongpt,zhang2024motiongpt,huang2023visual,li2023large,liu2024improved,mu2024embodiedgpt}. With the rapid development of technologies such as natural language processing, computer vision, and speech recognition, single-modal intelligent systems can no longer meet the increasingly complex requirements of real-world applications~\cite{park2023generative,radford2021learning,rocamonde2023vision,sun2023aligning}. Multimodal learning, by integrating different types of data inputs, simulates the diversity and complexity of human information processing, offering more comprehensive and flexible intelligent services. At the same time, with the deepening of interdisciplinary research, MLLMs will not only play a role in traditional AI tasks but will also expand into more edge domains, driving artificial intelligence from closed systems to a more open and intelligent ecosystem.
More details of the applications of MLLMs are provided in Section \ref{applications_MLLM} of the Appendix.

In summary, the application prospects of multimodal large models are vast. However, to fully unleash their potential, this requires the combined advancement of technological innovation and theoretical breakthroughs. In the future, with ongoing progress in algorithms, hardware, and cross-domain collaboration, it is expected that MLLMs will achieve more efficient and intelligent performance in a wider range of practical applications, further advancing the development of artificial intelligence.




\begin{table*}[htbp]
\small
\renewcommand\arraystretch{1.2}
  \centering
  \caption{Innovations in Non-LLM Multimodal CL Methods.}
    \begin{tabularx}{\textwidth}{>{\centering\arraybackslash}m{2cm}|p{0.4\textwidth}|p{0.4\textwidth}}
        \hline
   \multicolumn{1}{c|}{Method} & \multicolumn{1}{c|}{Starting point of the problem} & \multicolumn{1}{c}{How to solve} \\
   
   \hline
   \multirow{2}{*}{\textbf{CPP}~\cite{yuan2024continual}} & 
Improving the performance of continual learning.& CPP incorporates the CCE, TKD, and TPL mechanisms to achieve multimodal vision perception.\\
      
   \hline
   \multirow{2}{*}{\makecell{\textbf{CP-Prompt}\\~\cite{feng2024cp}}} & 
To reduce catastrophic forgetting.& CP-Prompt utilizes a dual-prompt strategy and parameter-efficient adjustments.\\
      
   \hline
   \multirow{2}{*}{\textbf{MMAL}~\cite{yue2024mmal}} & 
Reducing forgetting and enhancing incremental learning performance.& MMAL proposes the modality fusion module and MSKC module.\\
      
   \hline
   \multirow{2}{*}{\textbf{MSPT}~\cite{chen2023continual}} & 
To reduce catastrophic forgetting.& MSPT optimizes multimodal learning through gradient modulation and attention distillation.\\
      
   \hline
   \multirow{2}{*}{\makecell{\textbf{MedCoSS}\\~\cite{ye2024continual}}} & 
To reduce catastrophic forgetting.& MSPT propose a staged multimodal self-supervised learning framework that avoids modality conflicts.\\

   
   \hline
   \multirow{2}{*}{\textbf{ZiRa}~\cite{deng2024zero}} & 
Retaining zero-shot generalization ability.& ZiRa proposes zero-interference loss and a reparameterized dual-branch structure.\\
   
   \hline
   \multirow{2}{*}{\textbf{STELLA}~\cite{leestella}} & 
To reduce forgetting of previously learned knowledge.& STELLA proposes a localized patch importance scoring method.\\
   
   \hline
   \multirow{2}{*}{\makecell{\textbf{RCS-Prompt}\\~\cite{yang2024rcs}}} & 
To address the issue of overlap between old and new category spaces.& RCS-Prompt proposes bidirectional prompt optimization and prompt magnitude normalization.\\
   
   \hline
   \multirow{2}{*}{\textbf{ZSCL}~\cite{zheng2023preventing}} & 
To reduce catastrophic forgetting.& ZSCL proposes feature space distillation and parameter space weight integration.\\
   
   \hline
   \multirow{3}{*}{\textbf{CoCoOp}~\cite{zhou2022conditional}} & 
To address the issue of pretrained models lacking generalization ability to unseen classes when adapting to new tasks.& CoCoOp generates dynamic prompts using a lightweight neural network.\\
   
   \hline
   \multirow{2}{*}{\textbf{RAIL}~\cite{xu2024advancing}} & 
Improving cross-domain classification capabilities during continual learning.& RAIL uses recursive ridge regression and a no-training fusion module.\\

    \hline
    \end{tabularx}
  \label{CL_NonLM_Method}%
  \vspace{-5mm} 
\end{table*}%



\section{Continue Learning}

\subsection{Preliminary}

Continual Learning (CL) has become a central focus in AI research due to the rapid growth of deep learning and LLMs~\cite{li2017learning,loo2020generalized,pellegrini2021continual,sarfraz2023sparse,abbasi2022sparsity,huang2024etag,huang2024kfc,ke2020continual,yu2020semantic}. The challenge is to enable models to retain and enhance learning capabilities when faced with continuously changing data and tasks. Traditional methods assume that models can learn all tasks at once and maintain a fixed knowledge base, but in reality, data and tasks evolve, often leading to ``Catastrophic Forgetting''~\cite{chaudhry2018efficient,chen2020mitigating,de2021continual,miao2021continual,pham2021dualnet,konishi2023parameter,li2023memory,li2023variational}. Therefore, CL, as a learning paradigm that better aligns with real-world application needs, aims to enable models to effectively accumulate and update knowledge across multiple stages, thereby better adapting to dynamic and evolving environments.


This section will provide a detailed classification and overview of the latest innovative research in continual learning. The specific content is divided into three parts: 1) Exploring non-LLMs unimodal continual learning and focusing on traditional models' continual learning research in unimodal data; 2) Analyzing non-LLMs multimodal continual learning and discussing the challenges and research in continual learning across multi-modal data; 3) Analyzing and summarizing the latest advancements in continual learning for LLMs and examining the unique challenges and solutions they face when handling large-scale textual data.

% \subsection{Non-LLM Unimodal Continual Learning}
\subsection{Non-LLM Unimodal CL}
In traditional unimodal learning, research on continual learning primarily focuses on how to prevent models from forgetting previously learned knowledge when learning new tasks. Many researchers have proposed solutions to this problem, including strategies based on knowledge retention, incremental learning methods, and improvements to neural network architectures~\cite{shin2017continual,tao2020topology,wang2021training,sun2023decoupling,sun2021ilcoc,rypesc2024divide,sarfraz2023sparse,shi2023prototype}. For non-large models, the challenges of continual learning are particularly pronounced due to limitations in computational resources. Furthermore, the unimodal continual learning for non-large models primarily focuses on individual modalities such as vision, speech, and text. As show in Tables~\ref{CL_NonL_Framework} and~\ref{CL_NonL_Method}, to address the specific characteristics of these tasks, researchers have proposed a variety of innovative frameworks and methods. Overall, unimodal continual learning with non-large models has made significant progress in scenarios with limited computational resources. Many innovative frameworks and methods have been developed to effectively mitigate catastrophic forgetting. However, how to scale these approaches to multimodal and large-scale data remains an important direction for future research.
More details of the non-LLM unimodal continual learning are provided in Section \ref{appendix_cl_nlu} of the Appendix.


% \subsection{Non-LLM Multimodal Continual Learning}
\subsection{Non-LLM Multimodal CL}
Compared to unimodal continual learning, multimodal continual learning presents more complex challenges. Data from different modalities often exhibit heterogeneity, and the key difficulty in multimodal continual learning for non-large models lies in how to effectively fuse information across modalities while retaining previously acquired knowledge during the process of learning new modalities. In recent years, researchers have proposed various methods to address these challenges, including inter-modal collaborative learning, shared and independent representations for each modality, and others~\cite{kirkpatrick2017overcoming,rebuffi2017icarl,ahn2019uncertainty,zenke2017continual,yoon2017lifelong,lee2020neural,madaan2021representational,cossu2024continual,fini2022self,yoon2023continual,yan2022generative,pian2023audio,mo2023class}. As shown in Table~\ref{CL_NonLM_Method}, these innovative methods enable non-large models to perform continual learning in multimodal environments, while minimizing knowledge conflicts between different modalities.
More details of the non-LLM multimodal continual learning are provided in Section \ref{appendix_cl_nlm} of the Appendix.

\begin{table*}[htbp]
\small
\renewcommand\arraystretch{1.2}
  \centering
  \caption{Innovations in LLM Instruction Fine-tuning Methods.}
    \begin{tabularx}{\textwidth}{>{\centering\arraybackslash}m{2cm}|p{0.4\textwidth}|p{0.4\textwidth}}
    \hline
   \multicolumn{1}{c|}{Method} & \multicolumn{1}{c|}{Starting point of the problem} & \multicolumn{1}{c}{How to solve} \\
   
   \hline
   \multirow{2}{*}{\makecell{\textbf{ConTinTin}\\~\cite{yin2022contintin}}} & 
To reduce catastrophic forgetting.& InstructionSpeak learns from negative outputs and revisites the instructions of previous tasks.\\
      
   \hline
   \multirow{2}{*}{\textbf{OLoRA}~\cite{wang2023orthogonal}} & 
Improving the performance of continual learning.& OLoRA introduces orthogonal low-rank adaptation for CIT.\\
      
   \hline
   \multirow{2}{*}{\textbf{DAPT}~\cite{zhao2024dapt}} & 
To reduce catastrophic forgetting.&DAPT proposes a dual-attention learning and selection module.\\
      
   \hline
   \multirow{2}{*}{\textbf{ELM}~\cite{jang2023exploring}} & 
To reduce catastrophic forgetting.& ELM trains a small expert adapter for each task on top of the LLM.\\
      
   \hline
   \multirow{2}{*}{\makecell{\textbf{LLaMA PRO}\\~\cite{wu2024llama}}} & 
Retaining the initial functionality through post-training. & LLaMA PRO introduces an innovative block expansion technique.\\
      
   \hline
   \multirow{3}{*}{\makecell{\textbf{AdaptLLM}\\~\cite{cheng2023adapting}}} & 
To help the model leverage domain-specific knowledge while enhancing prompt performance. & AdaptLLM adapts the LLM to different domains by enriching the original training corpus with a series of content-related reading comprehension tasks.\\
      
   \hline
   \multirow{2}{*}{\textbf{DynaInst}~\cite{mok2023large}} & 
To enhance the generalization of the LLM.& DynaInst combines dynamic instruction replay with a local minima-inducing regularizer.\\

   \hline
   \multirow{2}{*}{\textbf{TAALM}~\cite{seo2024train}} & 
Enabling targeted knowledge updates and reducing forgetting.& TAALM uses meta-learning to dynamically predict token importance.\\
   
   \hline
   \multirow{2}{*}{\makecell{\textbf{D-CPT Law}\\~\cite{que2024d}}} & 
To reduce GPU resource consumption and improve domain adaptability.& D-CPT Law predicts the optimal training ratio.\\
   
   \hline
   \multirow{2}{*}{\textbf{COPAL}~\cite{malla2024copal}} & 
High computational demands and model adaptability limitations. & COPAL enables continual pruning without the need for retraining.\\
   
   \hline
   \multirow{2}{*}{\textbf{MagMax}~\cite{marczak2025magmax}} & 
To reduce catastrophic forgetting. & MagMax proposes sequential fine-tuning and maximum magnitude weight selection.\\
   
   \hline
   \multirow{2}{*}{\textbf{SAPT}~\cite{zhao2024sapt}} & 
Enabling effective knowledge retention and transfer.& SAPT aligns the learning and selection of PET blocks through a shared attention mechanism.\\
   
   \hline
   \multirow{2}{*}{\textbf{SSR}~\cite{huang2024mitigating}} & 
To reduce catastrophic forgetting.& SSR utilizes LLM-generated synthetic instances for rehearsal.\\
   
   \hline
   \multirow{2}{*}{\makecell{\textbf{LoRAMoE}\\~\cite{dou2024loramoe}}} & 
Enhancing multi-task handling capabilities.& LoRAMoE integrates LoRA and router networks, and introduces local balance constraints.\\
   
   \hline
   \multirow{2}{*}{\makecell{\textbf{F-Learning}\\ \textbf{paradigm}~\cite{dou2024loramoe}}} & 
Improving the performance of continual learning.& F-Learning paradigm first forgets old knowledge before learning new knowledge.\\




    \hline
    \end{tabularx}
  \label{CL_LLM_instr}%
  \vspace{-5mm} 
\end{table*}%




% \subsection{Continual Learning in LLM}
\subsection{CL in LLM}

LLMs such as GPT and BERT, with their powerful language understanding and generation capabilities, have achieved remarkable results on various natural language processing tasks~\cite{devlin2018bert,du2024chinese,eloundou2023gpts,kukreja2024literature,kasneci2023chatgpt,zhao2023survey,naveed2023comprehensive,chang2024survey,chen2021evaluating,unlu2023interpretutor,wu2024survey,zhang2023instruction}. However, LLMs still face unique challenges in continual learning. Particularly in the context of increasing data volume and task diversity, how to effectively update models, avoid catastrophic forgetting, and maintain efficient computational capabilities are key focuses in the research of LLMs for continual learning. As shown in Table~\ref{CL_LLM_instr}, researchers have proposed a variety of instruction fine-tuning methods. Through model improvements and methods such as instruction fine-tuning, LLMs are able to expand their knowledge while effectively addressing the issue of catastrophic forgetting. However, as model sizes continue to grow, core challenges in the field of continual learning for LLMs remain, such as how to handle updates and learning with large-scale data, and how to maintain good adaptability in multi-task and cross-modal environments. These remain critical issues that need to be addressed.
More details of the LLM continual learning are provided in Section \ref{appendix_cl_llm} of the Appendix.


Continual learning is a multidimensional and complex research field, characterized by both challenges and opportunities. From unimodal to multimodal, and then to continual learning in LLMs, each category of methods and strategies presents its own unique challenges and innovations. Future research will not only need to deepen the understanding of existing methods, but also explore how to achieve more efficient and robust continual learning in environments with large-scale, multimodal data and tasks. As computational power and data scale continue to expand, research in continual learning will provide a more solid theoretical and technological foundation for the adaptability, robustness, and sustainability of intelligent systems.














\begin{table*}[htbp]
\small
\renewcommand\arraystretch{1.2}
  \centering
  \caption{Innovations in MLModel CL Frameworks.}
    \begin{tabularx}{\textwidth}{>{\centering\arraybackslash}m{2cm}|p{0.4\textwidth}|p{0.4\textwidth}}
    \hline
   \multicolumn{1}{c|}{Framework} & \multicolumn{1}{c|}{Starting point of the problem} & \multicolumn{1}{c}{How to solve} \\
         
   \hline
   \multirow{2}{*}{\makecell{\textbf{PathWeave}\\~\cite{yu2024llms}}} & 
To reduce the dependency on large-scale joint pre-training.& PathWeave enhances modality alignment and collaboration.\\
   
   \hline
   \multirow{2}{*}{\textbf{CLAP}~\cite{jha2024clap4clip}} & 
To enhance the model's uncertainty estimation capabilities.& CLAP is compatible with various prompt methods.\\
   
   \hline
   \multirow{2}{*}{\textbf{DIKI}~\cite{tang2025mind}} & 
To reduce catastrophic forgetting.& DIKI proposes a residual mechanism and distribution-aware calibration.\\
         
   \hline
   \multirow{2}{*}{\textbf{GMM}~\cite{cao2024generative}} & 
To reduce catastrophic forgetting.& GMM implements incremental learning through generated label text and feature matching.\\
         
   \hline
   \multirow{2}{*}{\makecell{\textbf{PriViLege}\\~\cite{park2024pre}}} & 
To address catastrophic forgetting and overfitting in MLLMs.& PriViLege proposes prompt functionality and knowledge distillation.\\
         
   \hline
   \multirow{2}{*}{\makecell{\textbf{ModalPrompt}\\~\cite{zeng2024modalprompt}}} & 
To address catastrophic forgetting and overfitting in MLLMs.& ModalPrompt proposes bi-modal guided prototype prompts and knowledge transfer.\\
         
   \hline
   \multirow{2}{*}{\textbf{CGIL}~\cite{frascaroli2024clip}} & 
To reduce catastrophic forgetting.& CGIL uses VAEs to learn class-conditioned distributions and generate synthetic samples.\\
         
   \hline
   \multirow{2}{*}{\makecell{\textbf{CoLeCLIP}\\~\cite{li2024coleclip}}} & 
To  reduce interference between tasks.& CoLeCLIP proposes joint learning of task prompts and cross-domain vocabularies.\\
         
   \hline
   \multirow{2}{*}{\textbf{ICL}~\cite{qi2024interactive}} & 
To enhance the efficiency of continual learning in MLLMs.& ICL enables interaction between a fast intuition model and a slow deep thinking model.\\
         
   \hline
   \multirow{2}{*}{\textbf{EMT}~\cite{zhai2023investigating}} & 
To evaluate catastrophic forgetting in MLLMs.& EMT offers a new perspective for improving fine-tuning strategies in MLLMs.\\
         
   \hline
   \multirow{2}{*}{\makecell{\textbf{Freeze-Omni}\\~\cite{wang2024freeze}}} & 
To reduce catastrophic forgetting.& Freeze-Omni implements a three-stage training strategy.\\
         
   \hline
   \multirow{2}{*}{\textbf{Adapt-$\infty$}~\cite{maharana2024adapt}} & 
To reduce catastrophic forgetting.& Adapt-$\infty$ proposes dynamic data selection and a clustering-based permanent pruning strategy.\\
         
   \hline
   \multirow{3}{*}{\makecell{\textbf{Mono-}\\ \textbf{InternVL}~\cite{luo2024mono}}} & 
To address the performance degradation and catastrophic forgetting issues that arise when expanding the visual and language capabilities of MLLMs.& Mono-InternVL integrates visual experts using a MOE structure and introduces endogenous visual pretraining.\\
         
   \hline
   \multirow{2}{*}{\makecell{\textbf{MoExtend}\\~\cite{zhong2024moextend}}} & 
To  address the issues of catastrophic forgetting and high training costs.& MoExtend designes a three-stage training process, including alignment, extension, and fine-tuning.\\


    \hline
    \end{tabularx}
  \label{CL_MLLM_Framework}%
  \vspace{-5mm} 
\end{table*}%



\section{Continual Learning in MLLMs}

\subsection{Preliminary}

Recent advancements in MLLMs have shown remarkable capabilities across various domains. However, as their scale grows, maintaining long-term effectiveness in dynamic environments is a critical challenge~\cite{young2014image,achiam2023gpt,anil2023palm,bai2023qwen,chen2015microsoft,chen2024internvl,panagopoulou2023x,dong2024internlm,fu2024video,goyal2017making,gurari2018vizwiz,liu2024llava}. CL addresses this by enabling models to learn new tasks without forgetting previously acquired knowledge in evolving data and task contexts. For MLLMs, continual learning is more complex due to the vast data and complex computations involved, requiring significant computational resources and storage. Although existing research provides valuable theoretical and experimental insights~\cite{liu2025mmbench,luo2024cheap,yang2023dawn,team2023gemini,team2023internlm,touvron2023llama,wang2023cogvlm,wu2024parameter,yue2024mmmu}, applying MLLMs to continual learning still faces many challenges. This section explores innovations in multimodal large model continual learning and the related evaluation benchmarks.


\subsection{Model Innovation}

As shown in Tables~\ref{CL_MLLM_Framework} and~\ref{CL_MLLM_Method}, to achieve multi-task CL in multimodal large models and avoid catastrophic forgetting, researchers have proposed numerous innovative frameworks and methods~\cite{li2024coleclip,qi2024interactive,maharana2024adapt,luo2024mono,lester2021power,yan2022generative,villa2023pivot,he2024towards}. These innovations not only facilitate knowledge sharing and transfer between multiple tasks but also effectively address challenges such as catastrophic forgetting, modality conflicts, and computational resource constraints. These efforts collectively advance the continual learning capabilities of multimodal large models in dynamic environments.
More details of the model innovation in the continual learning of MLLMs are provided in Section \ref{Appendix_MLLMCL} of the Appendix.

\begin{table*}[htbp]
\small
\renewcommand\arraystretch{1.2}
  \centering
  \caption{Innovations in MLLModel CL Methods.}
    \begin{tabularx}{\textwidth}{>{\centering\arraybackslash}m{2cm}|p{0.4\textwidth}|p{0.4\textwidth}}
    \hline
   \multicolumn{1}{c|}{Method} & \multicolumn{1}{c|}{Starting point of the problem} & \multicolumn{1}{c}{How to solve} \\
         
   \hline
   \multirow{2}{*}{\textbf{NoRGa}~\cite{yu2024llms}} & 
To enhance the continual learning performance of multimodal large language models. & NoRGa proposes the non-linear residual gate.\\
         
   \hline
   \multirow{2}{*}{\textbf{ZAF}~\cite{gaostabilizing}} & 
To reduce catastrophic forgetting.& ZAF preserves knowledge through zero-shot stability regularization. \\
         
   \hline
   \multirow{3}{*}{\makecell{\textbf{DualLoRA}\\~\cite{chen2024dual}}} & 
Improving the efficiency and effectiveness of continual learning in multimodal large language models.& DualLoRA utilizes orthogonal and residual low-rank adapters along with a dynamic memory mechanism to balance model stability and plasticity. \\
         
   \hline
   \multirow{3}{*}{\textbf{LPI}~\cite{yan2024low}} & 
To address the insufficient interaction between modalities and tasks.& LPI enhances inter-modal and inter-task interactions through low-rank decomposition and contrastive learning. \\
         
   \hline
   \multirow{2}{*}{\makecell{\textbf{Model Tailor}\\~\cite{zhu2024model}}} & 
To reduce catastrophic forgetting.& Retaining most of the pre-trained parameters and  replacing a small number of fine-tuned parameters. \\
         
   \hline
   \multirow{2}{*}{\textbf{HVCLIP}~\cite{vesdapunt2025hvclip}} & 
Enhancing the model's ability to retain critical information while adapting to new tasks or domains.& HVCLIP uses strategies such as forgetting reduction, discrepancy reduction, and feature enhancement. \\
         
   \hline
   \multirow{2}{*}{\makecell{\textbf{Continual}\\ \textbf{LLaVA}~\cite{cao2024continual}}} & 
Enhancing the ability to preserve knowledge from previous tasks while accommodating new ones..& Continual LLaVA proposes a parameter-efficient tuning method that does not require rehearsal. \\
         
   \hline
   \multirow{2}{*}{\textbf{LLaCA}~\cite{qiao2024llaca}} & 
To reduce forgetting and lower computational costs.& LLaCA dynamically adjusts the EMA weights and introduces an approximation mechanism. \\
         
   \hline
   \multirow{2}{*}{\textbf{CVM}~\cite{rebillard2024continually}} & 
To reduce forgetting and improve generalization.&CVM maps the representations of small visual models to the knowledge space of a fixed LLM. \\
         
   \hline
   \multirow{2}{*}{\textbf{RE-tune}~\cite{mistretta2024re}} & 
Addressing challenges related to computational resources, data privacy, and catastrophic forgetting.& RE-tune freezes the backbone of the model and trains adapters, using text prompts to guide training. \\
         
   \hline
   \multirow{2}{*}{\textbf{CluMo}~\cite{cai2024clumo}} & 
Enhancing the performance of MLLMs in CL and improving their ability to retain old knowledge.& CluMo employs a two-stage training and modality fusion prompt strategy. \\
         
   \hline
   \multirow{2}{*}{\makecell{\textbf{Fwd-Prompt}\\~\cite{zheng2024beyond}}} & 
To achieve anti-forgetting and positive transfer.& Fwd-Prompt utilizes gradient projection techniques and proposes a multimodal prompt pool. \\
         
   \hline
   \multirow{2}{*}{\makecell{\textbf{CPE-CLIP}\\~\cite{d2023multimodal}}} & 
Enhancing the performance of few-shot class incremental learning in MLLMs.& CPE-CLIP using learnable prompts and regularization strategies. \\
         
   \hline
   \multirow{2}{*}{\textbf{TG}~\cite{zhang2024preserving}} & 
To reduce catastrophic forgetting.& TG proposes the model-agnostic self-uncompression method. \\
         
   \hline
   \multirow{2}{*}{\textbf{LiNeS}~\cite{wang2024lines}} & 
Preserving the generalization ability of pretraining while improving fine-tuning task performance. & LiNeS proposes parameter updates with differentiated layer depth. \\
         
   \hline
   \multirow{2}{*}{\textbf{AttriCLIP}~\cite{wang2023attriclip}} & 
Enhancing the generalization and continual learning capabilities of MLLMs in multimodal tasks. & AttriCLIP adapts to new tasks using an attribute lexicon and textual prompts. \\
         
   \hline
   \multirow{2}{*}{\textbf{AttriCLIP}~\cite{wang2023attriclip}} & 
Enhancing the generalization and continual learning capabilities of MLLMs in multimodal tasks. & AttriCLIP adapts to new tasks using an attribute lexicon and textual prompts. \\
         
   \hline
   \multirow{2}{*}{\textbf{C-LoRA}~\cite{smith2023continual}} & 
To reduce catastrophic forgetting.& C-LoRA performs continual adaptive low-rank adjustments in the cross-attention layers of MLLMs. \\


    \hline
    \end{tabularx}
  \label{CL_MLLM_Method}%
  \vspace{-5mm} 
\end{table*}%

\subsection{Benchmarks}

As the application of multimodal large models in continual learning increases, evaluating their CL capability has become a key issue. To comprehensively assess the continual learning performance of multimodal large models, benchmarks and evaluation frameworks have emerged. However, benchmarks specifically designed for continual learning in multimodal large models are still relatively scarce, and the relevant evaluation standards are still in the process of development. Section~\ref{MLLMCLBenchmark} in the Appendix analyzes and lists the few existing benchmarks to evaluate the continual learning capability of multimodal large models, exploring their design concepts, evaluation metrics, and applicability in different application scenarios.

Existing benchmarks for multimodal large model continual learning provide some reference value for assessing a model's learning ability. However, due to the scarcity of such benchmarks, with only a few available for use, many issues and limitations remain to be addressed. In the future, there is a need to design more comprehensive, flexible, and scalable evaluation benchmarks to meet the evolving demands of multimodal large model continual learning technologies. 




% todo
% 1. 图标风格统一(格式统一)
% 2. 表格重新绘画一下
% 3. pipline

\section{Challenges and Future Trends in Multimodal Large Model Continual Learning}

\subsection{Catastrophic Forgetting}

\subsubsection{Challenges Encountered}

Catastrophic forgetting has long been a classic problem in continual learning tasks, and its presence significantly limits the adaptability and generalization ability of models in real-world dynamic environments. For multimodal large models, this issue becomes even more complex due to the need for training on large-scale data, as well as the immense computational resources and storage space required.

\subsubsection{Future Trends}

Balancing forgetting management with learning efficiency, especially as tasks increase, is a complex optimization challenge. The goal is to prevent catastrophic forgetting while maintaining learning efficiency. Future research should focus on strategies to mitigate forgetting, such as frameworks or algorithms that preserve old knowledge while learning new information, or mechanisms for periodic knowledge consolidation. In addition, techniques such as self-supervised learning and transfer learning can be utilized. By sharing latent features or representations across different modalities, these methods can reduce interference between tasks, thereby alleviating the impact of catastrophic forgetting.

\subsection{Improvement and Standardization of Evaluation Benchmarks}

\subsubsection{Challenges Encountered}

Evaluation benchmarks should not only consider a model's performance in learning new tasks but also assess its ability to retain knowledge across different modalities, the effectiveness of cross-task transfer, and its stability over long-term learning. Currently, benchmarks for evaluating continual learning in multimodal large models are still relatively scarce. As multimodal large models become increasingly complex in real-world applications, developing comprehensive and systematic evaluation benchmarks for their continual learning capabilities is an urgent problem that needs to be addressed.

\subsubsection{Future Trends}

Future research should focus on designing more comprehensive and flexible evaluation benchmarks that support the assessment of continual learning in multimodal large models within multi-task environments. Researchers need to develop evaluation metrics capable of measuring a model's performance in multi-task learning, knowledge transfer, catastrophic forgetting, and cross-modal consistency. Furthermore, the standardization of evaluation benchmarks will be a key direction for future development. By establishing unified evaluation frameworks, it will be possible to more effectively compare the strengths and weaknesses of different models, thereby advancing research in this field.

\subsection{Improving the Interpretability and Transparency of Continual Learning in Multimodal Large Models}

\subsubsection{Challenges Encountered}

In multimodal learning tasks, models need to integrate information from different modalities (such as images, text, audio, etc.), which makes their decision-making process more complex and harder to trace. In particular in continual learning environments, the model must continuously learn new tasks while retaining knowledge from previous tasks. The integration and transfer of information across different modalities during this learning process make the model's decision mechanism even more challenging to interpret. Enhancing the interpretability of multimodal large models in continual learning not only helps increase the model's trustworthiness but also provides effective debugging and error diagnosis mechanisms during the learning process.

\subsubsection{Future Trends}

In future research on continual learning for multimodal large models, to enhance model interpretability, researchers can design more transparent and traceable architectures that allow for clear tracking and analysis of the model's decision-making rationale when handling different tasks. At the model design level, researchers can integrate the latest advances in explainable AI (XAI) to incorporate highly interpretable model structures, thus improving transparency in the decision-making process. Furthermore, by combining techniques such as cross-modal learning and transfer learning, researchers can effectively facilitate the transfer and retention of cross-task knowledge during continual learning, while also enhancing the understanding and explainability of the knowledge transfer mechanisms.

\section{Conclusion}\label{sec:conclusion}

%%% Summarize main points and give brief overview of results %%%
Model-based approaches for time series classification can be effectively utilized when a model of the underlying dynamical process is available~\cite{shen2017classification}. 
Using structural identifiability (SI) analysis, structurally identifiable parameter combinations of the dynamical model can be obtained. 
Individual time series observations may then be represented as point estimates in the original parameter space or in the space of structurally identifiable parameter combinations. 
We introduced a novel method \eco{dubbed} \textbf{S}tructural-\textbf{I}dentifiability \textbf{M}apping (\myMethod{}) and demonstrated that \myMethod{} improves classification performance for the classification of time series data when taking a model-based approach and the underlying dynamical model is structurally unidentifiable.

% improved performance on classification task
Furthermore, it has been shown on a set of relevant example systems that classification performance is significantly improved when learning with data represented in the space of structurally identifiable parameter combinations. 
The increase in performance also persists when time series data of varying quality \mjc{are} \eco{produced}: for all types of time grids (dense, sparse and irregular) as well as for \eco{varying levels of} the observation\eco{al} noise introduced, learning in the space of structurally identifiable parameter combinations outperforms learning in the space of the original model parameters.

This work presents a first success in incorporating SI analysis directly into the learning process for classification. 
The \myMethod{} approach is straightforward and can be applied whenever a SI analysis can be carried out. 
An explicit reparametrisation of a given dynamical model in terms of fewer, structurally identifiable parameters is not needed in order to benefit from SI analysis. 
\pt{This is especially important in situations where explicit expressions for structurally identifiable parameter combinations are available following a SI analysis, but suitable model reparametrizations are not possible.}

Finally, outcomes of the learning process stay interpretable: while interpretation in the space of structurally identifiable parameter combinations is not straightforward, any insight in this space may be translated back to the space of the original model parameters $g^{-1}(\boldsymbol{\Phi})$, which, in turn, are meaningful in the domain-specific context.




\small
\bibliographystyle{IEEEtran}
\bibliography{ref}


\clearpage

\section{Theoretical Analysis}
\subsection{Notations}
We dedicate Table~\ref{tab:Notation} to index the notations used in this paper. Note that every notation is also defined when it is introduced.
\begin{table*}[h!]
\caption{Notations.}\label{tab:Notation}%\\
\centering  
% \resizebox{\textwidth}{!}{
\begin{tabular}{l l l}
\toprule
 $\gG$ &$\triangleq$ & Input graph with a vertex set $\gV$, an edge set $\gE$, and features $\mX$\\
 $\boldsymbol{A}$ &$\triangleq$ & Adjacency matrix of $\gG$\\
 $\gE$ & $\triangleq$ & Edges of $\gG$\\
 $\gV$ & $\triangleq$ & Nodes of $\gG$\\
 $\mX$ & $\triangleq$ & Matrix containing node features of $\gG$\\
 $\vy$ & $\triangleq$ & Vector of node labels of $\gG$\\
 $C$ & $\triangleq$ & An ordered set containing all possible node labels of $\gG$\\
 $F$ & $\triangleq$ & Dimension of node features in $\gG$\\
 $L$ & $\triangleq$ & Number of GNN layers\\
 $H$ & $\triangleq$ & Node embedding dimension\\ 
 ${\mH}$ & $\triangleq$ & Node embedding matrix\\
 $\vh_u$ & $\triangleq$ & Embedding of node u\\
 $\vw$ & $\triangleq$ & Vector of edge weights in  $\gG$\\
 $q$ & $\triangleq$ & Ratio of \# edges in sparse graph and \# edges in input graph in \%\\
 $k$ & $\triangleq$ & \# edges in the sparse graph, $k\triangleq\floor{\frac{q|\gE|}{100}}$\\ 
 $\tilde{p}$ & $\triangleq$ & Learned probability distribution by \sgs \\
 $\tilde{\gE}$ & $\triangleq$ & Set of edges sampled from $\gE$ by \sgs following $\tilde{p}$\\
$\tilde{\gG}$ &$\triangleq$ & Sparse subgraph $(\gV,\tilde{\gE},\mX)$ constructed by \sgs \\  
 $\mA_{\tilde{\gG}}$ or $\tmA$ & $\triangleq$ & Adjacency matrix of $\tilde{\gG}$\\
 $\tilde{\vw}$ & $\triangleq$ & Edge weight of sparse graph learned by \sgs \\
 $p_\mathrm{prior}$ & $\triangleq$ & Probability distribution of a fixed prior on $\gG$ \\
  $\tilde{p}_a$ & $\triangleq$ & Augmented learned probability distribution  \\
 $p^*$ & $\triangleq$ & True probability distribution known by the idealized learning ORACLE\\
 ${\gE^*}$ & $\triangleq$ & Set of edges sampled from $\gE$ by the learning ORACLE following distribution $p^*$\\   
 $\gG^*$ &$\triangleq$ & True sparse subgraph $(\gV,\gE^*,\mX)$ constructed by the learning ORACLE \\
 $\mA_{\gG^*}$ or $\mA^*$ & $\triangleq$ & Adjacency matrix of $\gG^*$\\
 % $\gG^*$ &$\triangleq$ & True sparse subgraph constructed by the idealized learning ORACLE \\
 % $\mA_{\gG^*}$ or $\mA^*$ & $\triangleq$ & Adjacency matrix of $\gG^*$\\
 $\gL_\mathrm{CE}$ & $\triangleq$ & Cross entropy loss\\
 $\gL_\mathrm{assor}$ & $\triangleq$ & Assortative loss\\
 $\gL_\mathrm{cons}$ & $\triangleq$ & Consistency loss\\
$\gL$ & $\triangleq$ & Total loss\\ 
 
 
 
 
 
 
 \bottomrule
\end{tabular}
% }
\end{table*}
\subsection{Bounding \#common edges wrt. true subgraph}
\label{theo:commonedges}
Let $\mathcal{E}^*$ and $\mathcal{\tilde{E}}$ denote the ordered collection of edges sampled by the idealized learning ORACLE according to true distribution $p^*$ and by \sgs according to learned probability $\tilde{p}$ respectively. For analytical convenience, let us assume that both learning algorithms sample $k = \floor{q|\mathcal{E}|/100}$ edges with replacement independently.
 
% We will first show that $\mathbf{Pr}(\mathcal{E}_i^* = \mathcal{\tilde{E}}_i) \geq \min_j (p^*_i, \tilde{p}_i)$ and $\min(p^*_i,\tilde{p}_i) = 1- \frac{1}{2} \|\tilde{p} - p^*\|_1$. These two results will lead us to prove that $\mathbf{Pr}(\mathcal{E}_i^* = \mathcal{\tilde{E}}_i) \geq 1 - \frac{\epsilon}{2}$. 
First, we will prove lemma~\ref{lem:singleedge}, which show that the probability of an edge chosen by \sgs coincides with that chosen by the ORACLE has a lower bound. Finally, we will prove one of the main results (Theorem~\ref{theo:commonedges}), which shows that given $q \in [0,100]$, we can lower-bound the expected number of common edges between \sgs and the learning ORACLE. 

\begin{lemma} 
\label{lem:singleedge}
For any arbitrarily chosen $i \in \{1,2,\ldots, k\}$
\[
\mathbf{Pr}(\mathcal{E}_i^* = \mathcal{\tilde{E}}_i) \geq \sum_{j=1}^{|\mathcal{E}|} \frac{(p^*_j + \tilde{p}_j - \epsilon)^2}{4},
\]
where $k = \floor{q|\mathcal{E}|/100}$ and $0 \leq q \leq 100$ is a user-specified parameter and $\epsilon\in [0,1]$ is the error.
\end{lemma}

\begin{proof} We prove the above lemma in two parts.


\paragraph{Part 1: Universal approximation of probability distribution over edges.}
%\label{tho:uap}
The Universal Approximation Theorem~\cite{cybenko1989approximation,augustine2024survey} states that a feed-forward neural network with at least one hidden layer and a finite number of neurons can approximate any continuous function $f: \mathbb{R}^n \rightarrow \mathbb{R}$ on a compact subset of $\mathbb{R}^n$, given a suitable choice of weights and activation functions. 

In our case, $p^* = f$ is the true edge probability distribution for the downstream task, $\tilde{p} = f_{\text{MLP},\phi}$ is the learned approximate distribution and $\vx_e$ is a vector of edge features, for instance, $\vx_e =  ((\vh_u - \vh_v) \oplus (\vh_u \odot \vh_v))$ as used in equation~\ref{eq:w_uv}. The following universal approximation property holds for the module I component of \sgs,
\begin{equation}
\label{eq:uapp}
\sup_{e \in \mathcal{E}} \|\tilde{p}(\vx_e) - p^*(\vx_e)\|_1 \leq \epsilon.
\end{equation}
 Here, we have two underlying assumptions: (i) the optimal distribution $p^*$ is a function of node features $\mX$ and (ii) $\mX$ is a compact subset (bounded and closed) of Euclidean space $\mathbb{R}^n$. The first assumption is made to simplify the problem. The second assumption is quite practical since the node features are typically normalized. Hence, we can show that the embeddings $\vh_u,\vh_v$, which are continuous images of $\mX$, are also compact due to the extreme value theorem. As a result, the edge features $\vx_e$ which, in a sense, \emph{lifts} the end-point node features into higher-dimensional Euclidean space are also compact. The approximation error $\epsilon$ can be made arbitrarily small by increasing the capacity of the MLP, e.g., adding more neurons or layers. 

\paragraph{Part 2: Common edges wrt. optimal subgraph.}

The event $\mathcal{E}_i^* = \mathcal{\tilde{E}}_i$ means that both $\mathcal{E}_i^*$ and $\mathcal{\tilde{E}}_i$ contain the same edge. But there are $|\mathcal{E}|$ such candidates. Hence, the probability of this event is given by,

\begin{align*}
    \mathbf{Pr}(\mathcal{E}_i^* = \mathcal{\tilde{E}}_i) &= \sum_{j=1}^{|\mathcal{E}|} \mathbf{Pr}(\mathcal{E}_i^* = \mathcal{E}_j \land \mathcal{\tilde{E}}_i = \mathcal{E}_j), \\
    &= \sum_{j=1}^{|\mathcal{E}|} \mathbf{Pr}(\mathcal{E}_i^* = \mathcal{E}_j) \cdot \mathbf{Pr}(\mathcal{\tilde{E}}_i = \mathcal{E}_j), \\
    & = \sum_{j=1}^{|\mathcal{E}|} p^*_j \cdot \tilde{p}_j, \\
    &\geq \sum_{j=1}^{|\mathcal{E}|} \frac{(p^*_j + \tilde{p}_j - |p^*_j - \tilde{p}_j|)^2}{4}, \\
    & \geq \sum_{j=1}^{|\mathcal{E}|} \frac{(p^*_j + \tilde{p}_j - \epsilon)^2}{4}.
\end{align*}
The second line follows since the optimal sampler is a different algorithm independent from the sampler used in \sgs. The last line follows because $\|p^*_j - \tilde{p}_j\|_1 \leq \epsilon \implies |p^*_j - \tilde{p}_j| \leq \epsilon$ (from eq.~\ref{eq:uapp}). 
\end{proof}

% \begin{lemma}
% \[
%     \min(p^*_i,\tilde{p}_i) = 1- \frac{1}{2} \|\tilde{p} - p^*\|_1
% \]
% \end{lemma}
% \begin{proof}
%     It is known (for instance, see~\cite{xie2024distributionally}) that the total variation distance $d_{TV}(\tilde{p},p^*)$ satisfies
%     \[
%     d_{TV}(\tilde{p},p^*) = \frac{1}{2}\|\tilde{p} - p^*\|_1
%     = 1 - \min({\tilde{p},p^*})
%     \]
% \end{proof}

We have the following theorem that lower-bounds the number of common edges with respect to the optimal sampler $|\mathcal{E} ^* \cap \mathcal{\tilde{E}}|$: 
\begin{theorem}[Lower-bound]
\begin{equation}
\mathbb{E}[|\mathcal{E}^* \cap \mathcal{\tilde{E}}|] \geq k \sum_{j=1}^{|\mathcal{E}|} \frac{(p^*_j + \tilde{p}_j - \epsilon)^2}{4},
\end{equation}
where $k = \floor{q|\mathcal{E}|/100}$ and $0 \leq q \leq 100$ is a user-specified parameter.
\end{theorem}
\begin{proof}
    Since we are drawing $k$ edges independently at random, the theorem follows by applying the linearity of expectation on the following:
\begin{align*}
\mathbb{E}[|\mathcal{E}^* \cap \mathcal{\tilde{E}}|] = \mathbb{E}[\sum_{i=1}^k \mathbb{I}(\mathcal{E}_i^* = \mathcal{\tilde{E}}_i)] &= \sum_{i=1}^k \mathbf{Pr}(\mathcal{E}_i^* = \mathcal{\tilde{E}}_i) \\
& = k\cdot \mathbf{Pr}(\mathcal{E}_i^* = \mathcal{\tilde{E}}_i)\\
& \geq k \sum_{j=1}^{|\mathcal{E}|} \frac{(p^*_j + \tilde{p}_j - \epsilon)^2}{4}
\end{align*}
\end{proof}
This theorem shows that the expected number of common edges between the sample subgraph obtained by \sgs $\mathcal{\tilde{G}}$ and the true optimal sample subgraph $\mathcal{G}^*$ is non-trivial. 

% \paragraph{The implication of the lower-bound.} 
% (1) Suppose, the true distribution is uniform. In the best case scenario $\epsilon \rightarrow 0$ and $\tilde{p} = p^* = \frac{1}{|\mathcal{E}|}$. Thus there are at least $\frac{k}{|\mathcal{E}|}$ common edges between $\tilde{\gG}$ and $\gG^*$. However, since $k < \abs{\mathcal{E}}$, the lower-bound of $\mathbb{E}[|\mathcal{E}^* \cap \mathcal{\tilde{E}}|] \geq \frac{k}{|\mathcal{E}|}$ is not very useful even though the learned distribution is accurate. This suggests that \emph{learning the optimum uniform distribution is less likely to produce the optimum sparse subgraph}. 

% (2) Suppose, the true distribution is the Dirac distribution (often called the $\delta$ distribution) where all probability mass is concentrated on a single edge. In other words, suppose $\tilde{p} = p^* = \delta_{ij}$ where $\delta_{ij}$ is the Kronecker-delta. In such a skewed distribution, as $\epsilon \rightarrow 0$, the lower bound reduces to 
% \[
% \mathbb{E}[|\mathcal{E}^* \cap \mathcal{\tilde{E}}|] \geq k \sum_{j=1}^{|\mathcal{E}|} (\tilde{p}_j)^2 = k.
% \]
% This identity suggests that the sampled edges are expected to 100\% overlap with the true, optimal sparse subgraph.

\begin{theorem}[Upper-bound]
\begin{equation}
\mathbb{E}[|\mathcal{E}^* \cap \mathcal{\tilde{E}}|] \leq k (1 - \frac{\|p^* - \tilde{p}\|_1}{2}), 
\end{equation}
where $k = \floor{q|\mathcal{E}|/100}$ and $0 \leq q \leq 100$ is a user-specified parameter.
\end{theorem}
\begin{proof}
\begin{align*}
    \mathbf{Pr}(\mathcal{E}_i^* = \mathcal{\tilde{E}}_i) &= \sum_{j=1}^{|\mathcal{E}|} p^*_j \cdot \tilde{p}_j \\
    & \leq \sum_{j=1}^{|\mathcal{E}|} \min(p^*_j,\tilde{p}_j) \\
    &= 1 - d_{TV}(p^*,\tilde{p}) \\
    &= 1 - \frac{1}{2} \|p^* - \tilde{p}\|_1    
\end{align*}
\end{proof}
Here $d_{TV}$ is the total variation distance. The result used in the last line regarding $d_{TV}$ can be found in~\citet{xie2024distributionally}. 

\paragraph{The implication of the upper-bound.} 
When $\tilde{p} \rightarrow p^*$, the norm $\|p^* - \tilde{p}\|_1 \rightarrow 0$; therefore, the number of common edges could be close to $k$.

\subsection{Upper-bounding the error in the learned Adjacency matrix} 
With the bound proven earlier on the \#common edges by the sparse subgraph of \sgs with that by a learning ORACLE, in this section, we want to obtain an upper-bound on the error in terms of the norm of the Adjacency matrices. As adjacency matrices are used by GNNs for computing node embeddings, such result is important for obtaining error bound on the embeddings later on.

Let $\mA_{\tilde{\gG}}$ and $\mA_{\gG^*}$ be the corresponding adjacency matrices of the learned sparse graph $\tilde{\gG}$ and true optimal sparse graph $\gG^*$. The dimension of these matrices is the same as the input adjacency matrix $\mA_{\mathcal{G}}$ except that $\mA_{\mathcal{G}}$ is denser. Let us also denote the Frobenius norm of a matrix $\mA$ as $\|\mA\|_F$ and the spectral norm of $\mA$ as $\|\mA\|_2$. The Frobenius norm of $\mA$ is defined as $\sqrt{\sum_{ij} \mA^2_{ij}}$, whereas the spectral norm of $\mA$ is the largest singular value $\sigma_{max}(\mA)$ of $\mA$.


Since \sgs do not know the true probability distribution $p^*$, error is introduced in the learned adjacency matrix $\mA_{\tilde{\gG}}$ of the downstream sparse subgraph. We are interested in analyzing the expected error introduced in $\mA_{\tilde{\gG}}$ in terms of the spectral norm, to be precise, $\mathbb{E}[\|\mA_{\tilde{\gG}} - \mA_{\gG^*}\|_2]$. To this end, we will exploit the lower bound derived in Theorem 1 and the fact that $\|\mA\|_2 \leq \|\mA\|_F$. 

\begin{lemma}[Error in Adjacency matrix approximation] Let $\mA_{\tilde{\gG}}$ and $\mA_{\gG^*}$ be the corresponding adjacency matrices of the learned sparse graph $\tilde{\gG}$ and true optimal sparse graph $\gG^*$. If the downstream sampler sampled $k$ edges independently at random (with replacement) to construct those matrices following their respective distributions $\tilde{p}$ and $p^*$, then 
    \[
    \mathbb{E}[\|\mA_{\tilde{\gG}} - \mA_{\gG^*}\|_2] \leq \sqrt{2k(1-\sum_{j=1}^{\abs{\mathcal{E}}} \frac{(p^*_j + \tilde{p}_j - \epsilon)^2}{4})},
    \]
    where $k = \floor{q|\mathcal{E}|/100}$ and $0 \leq q \leq 100$ is a user-specified parameter.
\end{lemma}
\begin{proof}
Since the entries in adjacency matrices are either $0$ or $1$, the difference $\mA_{\tilde{\gG}}(i,j) - \mA_{\gG^*}(i,j)$ are in $\{-1,0,1\}$ for all $i,j$. The following holds by definition of Frobenus norm,

\[
\|\mA_{\tilde{G}} - \mA_{G^*}\|^2_F = \sum_{ij}(\mA_{\tilde{\gG}}(i,j) - \mA_{\gG^*}(i,j))^2.
\] 
As a result, only the non-zero entries in $\mA_{\tilde{\gG}} - \mA_{\gG^*}$ contribute to the square of Frobenius norm $\|\mA_{\tilde{G}} - \mA_{G^*}\|^2_F$.
The expected number of non-zero entries in $\|\mA_{\tilde{\gG}} - \mA_{\gG^*}\|^2_F$ corresponds to the expected cardinality $\abs{(\mathcal{\tilde{E}} \setminus \mathcal{E}^*) \cup (\mathcal{E}^* \setminus \mathcal{\tilde{E}})}$. Thus

\begin{align*}
    \mathbb{E}[\|\mA_{\tilde{\gG}} - \mA_{\gG^*}\|^2_F] &= \mathbb{E}[\abs{(\mathcal{\tilde{E}} \setminus \mathcal{E}^*) \cup (\mathcal{\tilde{E}} \setminus \mathcal{E}^*)}] \\ 
    &= \mathbb{E}[\abs{\mathcal{\tilde{E}}} + \abs{\mathcal{E}^*} - 2 \abs{\mathcal{\tilde{E}} \cap \mathcal{E}^*}] \\
    &= 2k - 2\mathbb{E}[\abs{\mathcal{\tilde{E}} \cap \mathcal{E}^*}] \\
    &\leq 2k - 2k \sum_{j=1}^{|\mathcal{E}|} \frac{(p^*_j + \tilde{p}_j - \epsilon)^2}{4} \\
    &= 2k (1 - \sum_{j=1}^{|\mathcal{E}|} \frac{(p^*_j + \tilde{p}_j - \epsilon)^2}{4}).
\end{align*}
Applying Jensen's inequality for convex functions, in particular, applying $(\mathbb{E}[\rX])^2 \leq \mathbb{E}[\rX^2])$ yields,
\begin{align*}
     (\mathbb{E}[\|\mA_{\tilde{\gG}} - \mA_{\gG^*}\|_F])^2 &\leq  \mathbb{E}[\|\mA_{\tilde{\gG}} - \mA_{\gG^*}\|^2_F] \\
     &\leq 2k (1 - \sum_{j=1}^{|\mathcal{E}|} \frac{(p^*_j + \tilde{p}_j - \epsilon)^2}{4}).
\end{align*}
Taking square-root on both sides yields,
\[
 \mathbb{E}[\|\mA_{\tilde{\gG}} - \mA_{\gG^*}\|_F] \leq \sqrt{2k (1 - \sum_{j=1}^{|\mathcal{E}|} \frac{(p^*_j + \tilde{p}_j - \epsilon)^2}{4})}.
\]
We obtain the theorem using the following relation between the Frobenius and spectral norms.
\begin{align*}
    \|\mA_{\tilde{\gG}} - \mA_{\gG^*}\|_2 &\leq \|\mA_{\tilde{\gG}} - \mA_{\gG^*}\|_F \\
    \implies \mathbb{E}[\|\mA_{\tilde{\gG}} - \mA_{\gG^*}\|_2] &\leq \mathbb{E}[\|\mA_{\tilde{\gG}} - \mA_{\gG^*}\|_F] \\
    &= \sqrt{2k (1 - \sum_{j=1}^{|\mathcal{E}|} \frac{(p^*_j + \tilde{p}_j - \epsilon)^2}{4})}
\end{align*}
\end{proof}
% \begin{corollary} Let us assume $\epsilon \rightarrow 0$ and the model learned the true pmf. Then the spectral norm approximation error is
%     \[
%     \mathbb{E}[\|\mA_{\tilde{\gG}} - \mA_{\gG^*}\|_2] \leq \sqrt{2k(1-\frac{1}{4\abs{\mathcal{E}}})}
%     \]
%     where $k = \floor{q|\mathcal{E}|/100}$ and $0 \leq q \leq 100$ is a user-specified parameter.
% \end{corollary}
% \begin{proof}
% Since we assumed $\epsilon \rightarrow 0$ and $p^*_j = \tilde{p}_j$, Theorem 3 reduces to
% \[
% \mathbb{E}[\|\mA_{\tilde{\gG}} - \mA_{\gG^*}\|_2] \leq \sqrt{2k(1- \frac{\sum_{j=1}^{\abs{\mathcal{E}}} (p^*_j)^2}{4})}
% \]
% For any probability mass function the following inequality holds
% $\sum_{i=1}^n p^2_i \geq 1/n$. This holds with equality when the distribution is uniform. Thus,
% \[
% \sum_{j=1}^{\abs{\mathcal{E}}} (p^*_j)^2 \geq \frac{1}{\abs{\mathcal{E}}}\\
% \implies \mathbb{E}[\|\mA_{\tilde{\gG}} - \mA_{\gG^*}\|_2] \leq \sqrt{2k(1-\frac{1}{4\abs{\mathcal{E}}})}
% \]
% \end{proof}
\subsection{Upper-bounding the error in the predicted node embeddings}
\label{theo:gcnembed}
We consider vanilla GCN as proof of concept to understand how the changes in the sparse subgraph affect the node embeddings produced by a trained GCN. Our goal is to analyze the respective encodings produced by an $L$-layer GCN when the input subgraphs are $\gG^*$ (corresponding to $\mA_{\gG^*}$) and $\tilde{\gG}$ (corresponding to $\mA_{\tilde{\gG}}$) respectively. For simplicity, we will shorten the matrices $\mA_{\gG^*}$ as $\mA^*$ and $\mA_{\tilde{\gG}}$ as $\tmA$. 

A single GCN layer is defined as,

\[
\mH^{(l+1)} = \sigma(\hat{\mA}\mH^{(l)}\mW^{(l)}),
\]
where $\hat{\mA} = \mD^{-1/2}\mA\mD^{-1/2}$ is the normalized adjacency matrix, $\mH^{(l)}$ is the input to the $l$-th layer with $\mH^{(0)} = \mX$, $\mW^{(l)}$ is the learnable weight matrix for $l$-th layer and $\sigma$ is non-linear activation function. Let us suppose an $L$-layer GCN produces embeddings $\tmH^{(L)}$ and $\mH^{*(L)}$ when it takes sparse matrices $\tmA$ and $\mA^*$ as input. We want to upper-bound,
\[
\mathbb{E}[\normLtwo{\tmH^{(L)} - \mH^{*(L)}}],
\]
in other words, the loss in the downstream node encodings is due to using our learned subgraph. 

\paragraph{Assumptions.} We assume that for all $l$, $\normLtwo{\mW} \leq \alpha < 1$ where $\alpha$ is a constant no more than 1. This is reasonable since each $\mW^{(l)}$ is typically controlled during training using regularization techniques, e.g., weight decay. Assuming that the input features in $\mX$ are bounded, we can also assume that there exists a constant $\beta$ such that $\forall l$, $\normLtwo{H}^{(l)} \leq \beta$. We also assume that $\sigma$ is \emph{Lipschitz continuous} with \emph{Lipschitz constant} $L_\sigma$; for instance,  activation functions such as \relu, sigmoid, or tanH are Lipschitz continuous. In particular, we assume \relu activation for our theoretical analysis because \relu has \emph{Lipschitz constant} $L_\sigma = 1$, which simplifies our analysis.

Under these assumptions, we have the following theorem,
\begin{theorem}[Error in GCN encodings]
For sufficiently deep L-layer GCN (large L), the error 
{
\[
\mathbb{E}[\lim_{L \to \infty} \normLtwo{\tmH^{(L)} - \mH^{*(L)}}] < \frac{\beta}{1-\alpha}\sqrt{2k (1 - \sum_{j=1}^{|\mathcal{E}|} \frac{(p^*_j + \tilde{p}_j - \epsilon)^2}{4})}.
\]
}
\end{theorem}
\begin{proof}
{
\[
\tmH^{(L)} - \mH^{*(L)} = \sigma(\hat{\tmA}\tmH^{(L-1)}\mW^{(L-1)}) - \sigma(\hat{\mA}^*\mH^{*(L-1)}\mW^{(L-1)})
\]
}
Since $\sigma$ is a Lipschitz continuous function, we have
{
\begin{align*}
\normLtwo{\tmH^{(L)} - \mH^{*(L)}} \leq L_\sigma\normLtwo{\hat{\tmA}\tmH^{(L-1)}\mW^{(L-1)} - \hat{\mA}^*\mH^{*(L-1)}\mW^{(L-1)}} \\
= \normLtwo{\hat{\tmA}\tmH^{(L-1)}\mW^{(L-1)} - \hat{\mA}^*\mH^{*(L-1)}\mW^{(L-1)}}\\
= \normLtwo{(\hat{\tmA} -\hat{\mA}^*) \tmH^{(L-1)}\mW^{(L-1)} + \hat{\mA}^*(\tmH^{(L-1)}- \mH^{*(L-1)})\mW^{(L-1)}}
\end{align*}
}
For notational convenience, let us suppose $D^{(L)} = \normLtwo{\tmH^{(L)} - \mH^{*(L)}}$. Applying the sub-multiplicative property of the spectral norm and triangle inequality, we obtain the following recurrence relation
{
\begin{align*}
    D^{(L)} &\leq \normLtwo{(\hat{\tmA} -\hat{\mA}^*)}\normLtwo{\tmH^{(L-1)}}\normLtwo{\mW^{(L-1)}} +   \normLtwo{\hat{\mA}^*}D^{(L-1)}\normLtwo{\mW^{(L-1)}} \\
    &\leq \normLtwo{(\hat{\tmA} -\hat{\mA}^*)} \beta\alpha + \normLtwo{\hat{\mA}^*}D^{(L-1)}\alpha \\
    &\leq \normLtwo{(\hat{\tmA} -\hat{\mA}^*)} \beta\alpha + D^{(L-1)}\alpha 
\end{align*}
}
The last inequality holds because normalized adjacency matrix satisfies $\normLtwo{\hat{\mA}^*} \leq 1$. This is because $\hat{\mA}^*$ is symmetric, row-stochastic matrix. Thus the singular values of $\hat{\mA}^*$ is the absolute values of eigenvalues of $\hat{\mA}^*$ and the largest singular value of $\hat{\mA}^*$ is the largest eigenvalue of $\hat{\mA}^*$. But $\hat{\mA}^*$ being row-stochastic, its largest eigenvalue is at most 1 hence $\normLtwo{\hat{\mA}^*} = \sigma_{max}(\hat{\mA}^*) \leq 1$.

By unrolling the recursion from earlier inequality:
\begin{align*}
     D^{(L)} &\leq \normLtwo{(\hat{\tmA} -\hat{\mA}^*)} \beta \alpha\sum_{l=0}^{L-1} \alpha^{l} + D^{(0)}\alpha^L
\end{align*}
$D^{(0)} = \normLtwo{\tmH^{(0)} - \mH^{*(0)}} = \normLtwo{\mX - \mX} = 0$. Since $\alpha < 1$, The geometric series simplifies to:
\begin{align*}
\sum_{l=0}^{L-1} \alpha^{l} = \frac{1-\alpha^L}{1-\alpha} \\
\lim_{L \to \infty} \sum_{l=0}^{L-1} \alpha^{l} = \frac{1}{1-\alpha}
\end{align*}
Thus our earlier inequality becomes:
\[
\lim_{L \to \infty} D^{(L)} \leq \frac{\beta\alpha}{1-\alpha}\normLtwo{(\hat{\tmA} -\hat{\mA}^*)} < \frac{\beta}{1-\alpha}\normLtwo{(\hat{\tmA} -\hat{\mA}^*)}
\]
Taking expectation on both sides gives us our desired result:
\small{
\begin{align*}
    \mathbb{E}[\lim_{L \to \infty} \normLtwo{\tmH^{(L)} - \mH^{*(L)}}] = \mathbb{E}[D^{(L}] < \frac{\beta}{1-\alpha}\mathbb{E}[\normLtwo{(\hat{\tmA} -\hat{\mA}^*)}] \\
    < \frac{\beta}{1-\alpha}\mathbb{E}[\normLtwo{(\hat{\mA} - \mA^*)}] \\
    = \frac{\beta}{1-\alpha}\sqrt{2k (1 - \sum_{j=1}^{|\mathcal{E}|} \frac{(p^*_j + \tilde{p}_j - \epsilon)^2}{4})}
\end{align*}
}
% Expanding the difference in the RHS:
% {
% \[
% \normLtwo{\hat{\tmA}\tmH^{(L-1)}\mW^{(L-1)} - \hat{\mA}^*\mH^{*(L-1)}\mW^{(L-1)}} = 
% \normLtwo{(\hat{\tmA} -\hat{\mA}^*) \tmH^{(L-1)}\mW^{(L-1)} - \hat{\mA}^*\mH^{*(L-1)}\mW^{(L-1)}}
% \]
% }
\end{proof}
% \begin{corollary}[Condition for convergence in embedding]
%     Assuming infinite depth of GCN layer $L$, the learned node embeddings converge to the `true embeddings' if the number of sampled edges satisfies
%     \[
%     k < \frac{1}{2}(\frac{1-\alpha}{\beta})^2
%     \]
% Here, by `true embedding', we mean the embedding generated by applying GCN on a subgraph sampled following the optimal probability distribution.
% \end{corollary}
% \begin{proof}
    
% \end{proof}
% We know that the total variation distance between two probability distributions is 1/2 of the $\text{L}_1$-distance between them. Hence, the following holds by applying universal approximation theorem (equation~\ref{eq:uapp})

% \begin{align}
% \text{TVD}(\tilde{p} , p^*) = \frac{1}{2}\|\tilde{p} - p^*\|_1 \leq \frac{\epsilon}{2}
% \end{align}

% \paragraph{2. The induced sparse subgraphs spectrum is a good approximation to the true graphs spectrum}
% $\forall \vx, \exists \epsilon$ such that
% \[
% (1-\epsilon) \leq \frac{\vx^{\top}\tilde{L}\vx}{\vx^{\top}L\vx} \leq (1+\epsilon)
%  \]
%  where $\tilde{L}$ is the Laplacian of the sparse graph sampled following the learned distribution $\tilde{p}$ and L is the original graph Laplacian.

% How to bound the ratio of the quadratic forms. Some results here: https://arxiv.org/pdf/2403.13268

\FloatBarrier



\clearpage
\section{Analyzing the Effectiveness of \sgs with a Synthetic Graph}
\label{app:toymoon}
In this section, we demonstrate and analyze the effectiveness of \sgs with a synthetically generated heterophilic graph.

\paragraph{Synthetic Graph: Moon.}
The moon dataset has the following properties: number of nodes $|\gV|=150$, number of edges $|\gE|=870$, average degree $d=5.8$, node homophily $\gH_n=0.2$, edge homophily $\gH_e = 0.32$, training/test split = $30\%/70\%$, and 2D coordinates of the points representing the nodes are the node features $\mX$.
The dataset comprises two half-moons representing two communities with $68\%$ edges connecting them as bridge edges.

%n_samples=150, degree=4, train=0.3, h = 0.2


\paragraph{Explaining the Effectiveness of \sgs on Heterophilic graph.} Fig.~\ref{fig:moongraph} juxtaposes the input moon graph (Fig.~\ref{fig:moongraph}, left) and the sparsified moon graph by \sgs (Fig.~\ref{fig:moongraph}, right). \sgs removes a significant portion of bridge edges, causing an increase in edge homophily from $0.32$ to $1.0$. As a result, the accuracy of vanilla GCN increased from $80\%$ on the full graph to $100\%$ on the sparsified graph. Since heterophilous edges significantly hinder the node representation learning, \sgs identifies them during training and learns to put less probability mass on such edges for downstream node classification. 
% \sgs identifies such edges that are detrimental to a downstream task by learning to put less probability mass on them. 
Due to this learning dynamics, \sgs is more effective on heterophilic graphs such as the Moon graph.

%%%%%%%%%%%%%%%%%%%%%%%%%%%%%%%%%%%%%%%%%%%%
\begin{figure}[!htbp]
\centering
\includegraphics[width=\linewidth]{Figures/SGS-moon.png}
\caption{Toy example with two half moon demonstrates the effectiveness of \sgs. The original graph has $68\%$ edges with different node labels; in contrast, the learned sparse subgraph from \sgs contains no such bridge edges.}
\label{fig:moongraph}
\end{figure}

%%%%%%%%%%%%%%%%%%%%%%%%%%%%%%%%%%%%%%%%%%%%

\clearpage
\section{Additional algorithmic details of \sgs }
\label{app:algorithm}

\paragraph{Conditional update of \edgemlp.}
Backward propagation is often the most computationally intensive part of training, so we employ a conditional mechanism to update \edgemlp selectively. 
We evaluate the learned sparse subgraph (line 9, Alg.~\ref{alg:sgstrainingpriorfull}) against a subgraph from the prior probability distribution $p_\mathrm{prior}$ (line 11, Alg.~\ref{alg:sgstrainingpriorfull}). If the training F1-score from the learned sparse subgraph is better than the baseline, parameters of \edgemlp are updated (line 19, Alg.~\ref{alg:sgstrainingpriorfull}). Otherwise, the update to \edgemlp is skipped (line 22, Alg.~\ref{alg:sgstrainingpriorfull}). 

The detailed algorithm for \sgs with conditional updates is in Alg.~\ref{alg:sgstrainingpriorfull}.


% If the size of $\gG$ is large, we compute a sparse subgraph from a prior probability distribution $p_\mathrm{prior}$ for \edgemlp. Later, we sample another sparse subgraph from the learned distribution to use with downstream GNN. The degree-proportionate prior is,

%  \begin{equation}
%  p_\mathrm{prior}(u,v) = \frac{1/d_u + 1/d_v}{\sum_{i,j\in \gE} (1/d_i + 1/d_j)}.
% \end{equation}

% % %%%%%%%%%%%%%%%%%%
% % \begin{wrapfigure}{c}{0.8\textwidth}
% \begin{center}
% \begin{algorithm}[!htbp]
% \caption{\sgs Training}
% \label{alg:sgstraining}
% \begin{algorithmic}[1] % The [1] here is for line numbering
% \STATE \textbf{Input:} $\gG (\gV, \gE, \mX)$, sample percent $q$, $num\_hops$
% \STATE \textbf{Output:} \texttt{EdgeMLP}, \texttt{GNN}

% %\STATE Sample size, $Q =\frac{q\gE}{100}$
% %\STATE Degree Norm, $p(u,v) = \frac{1/d_u + 1/d_v}{\sum_{i,j\in \gE} (1/d_i + 1/d_j)}$

% \FOR{$\mathrm{epochs}$ in $\mathrm{max\_epochs}$}

%     \STATE $\tilde{p},\vw =\edgemlp(\gE,p_\mathrm{prior})$
    
%     \STATE $\gE',\vw' = \mathrm{Sample}(\tilde{p}, \vw, \floor{\frac{q|\gE|}{100})}$ \COMMENT{Sparse graph for downstream GNN}
%     \STATE $\hat{\mY}, \mH' = \mathrm{GNN}_\theta(\gE',\vw',\mX)$

%     \STATE $\gL_\mathrm{CE} = \mathrm{CrossEntropy} (\mY\mathrm{[train]}, \hat{\mY}\mathrm{[train]})$
%     \STATE $\mathrm{mask[u,v]}=True:\forall_{(u,v)\in \gE} u \in \gV_L \land v \in \gV_L$
%     \STATE $\gL_\mathrm{assor} = \mathrm{CrossEntropy}(\gE \mathrm{[mask]},\vw \mathrm{[mask]})$

%     \STATE $\gL_\mathrm{cons} = \mathrm{Similarity} (\vw, \mathrm{Cosine}(\mH'[\gE[s]],\mH'[\gE[t]]))$

%     \STATE $\gL = \alpha_1\cdot \gL_\mathrm{CE}+ \alpha_2\cdot \gL_\mathrm{assor}+ \alpha_3\cdot \gL_\mathrm{cons}$
%     \STATE Backward propagate through $\gL$ and optimize $\edgemlp_\phi, \mathrm{GNN}_\theta$.
% \ENDFOR

% \STATE \textbf{Return} \texttt{EdgeMLP}, \texttt{GNN} 
% \end{algorithmic}
% \end{algorithm}
% \end{center}
% % \end{wrapfigure}
% % %%%%%%%%%%%%%%%%%%

% \sgs also supports conditional updates of \edgemlp, and the pseudocode is outlined in Algorithm~\ref{alg:sgstrainingpriorfull}. 


% %%%%%%%%%%%%%%%%%%
\begin{algorithm}[!htbp]
\caption{\sgs Training with conditional updates}
\begin{algorithmic}[1] % The [1] here is for line numbering
\STATE \textbf{Input:} $\gG (\gV, \gE, \mX)$, sample percent $q$, $\mathrm{hops}$, METIS Parts, $n$
\STATE \textbf{Output:} \texttt{EdgeMLP}, \texttt{GNN}
\STATE Compute $p_\mathrm{prior}(u,v) \gets \frac{1/d_u + 1/d_v}{\sum_{i,j\in \gE} (1/d_i + 1/d_j)}$

\STATE $\gG_\mathrm{parts}=\{\gG_1,\gG_2,\cdots,\gG_n\}\gets \mathrm{METIS} (\gG(\gV,\gE, p), n)$

\FOR{$\mathrm{epochs}$ in $\mathrm{max\_epochs}$}

    \FOR {$\gG_i(\gV_i,\gE_i,\mX_i,p^i_\mathrm{prior}) \in \gG_\mathrm{parts}$}
        \STATE $\tilde{p},\vw \gets \edgemlp(\gE_i,p^i_\mathrm{prior}, \mX_i,\mathrm{hops})$        
        \STATE $\tilde{p}_a \gets \lambda \tilde{p}+(1-\lambda)p^i_\mathrm{prior}$
        \STATE $\tilde{\gE},\tilde{\vw} \gets \mathrm{Sample}(\tilde{p}_a,\vw,\floor{\frac{q|\gE_i|}{100}})$ \COMMENT{\textbf{Learned sparse subgraph}}
        
        \STATE $\hat{\mY}, \tilde{\mH} \gets \mathrm{GNN}_\theta(\tilde{\gE},\tilde{\vw},\mX)$
        
        \STATE $\gE_\mathrm{prior} \gets \mathrm{Sample}({p_i},\floor{\frac{q|\gE_i|}{100}})$ \COMMENT{\textbf{Sparse subgraph from prior}}

        \STATE $\hat{\mY}_\mathrm{prior}  \gets \mathrm{GNN}_\theta(\gE_\mathrm{prior},\mX)$

        \IF {Evaluate$(\hat{\mY}) \ge $ Evaluate$(\hat{\mY}_\mathrm{prior})$}
            \STATE $\gL_\mathrm{CE} \gets \mathrm{CrossEntropy} (\mY_{\gV_L}, \hat{\mY}_{\gV_L})$

            \STATE $\forall_{(u,v)\in \gE_i} u \in \gV_L \land v \in \gV_L : \mathrm{mask[u,v]} \gets \text{True}$
        
            \STATE $\gL_\mathrm{assor} \gets \mathrm{CrossEntropy}(\gE \mathrm{[mask]},\vw \mathrm{[mask]})$
        
            \STATE $\gL_\mathrm{cons} \gets \mathrm{Sim} (\vw, \mathrm{Cosine}(\vh_u,\vh_v): \forall_{(u,v)\in \gE})$ 
        
            \STATE $\gL \gets \alpha_1\cdot \gL_\mathrm{CE}+ \alpha_2\cdot \gL_\mathrm{assor}+ \alpha_3\cdot \gL_\mathrm{cons}$
            \STATE Backward Propagate through $\gL$ and optimize EdgeMLP$_\phi, \mathrm{GNN}_\theta$.
        
        \ELSE
            \STATE $\gL_\mathrm{CE} \gets \mathrm{CrossEntropy} (\mY_{\gV_L}, \hat{\mY}_{\gV_L})$
            \STATE Backward Propagate through $\gL_\mathrm{CE}$ and optimize $\mathrm{GNN}_\theta$.            
        \ENDIF
    
    \ENDFOR
    
\ENDFOR
\STATE \textbf{Return} \texttt{EdgeMLP}, \texttt{GNN} 
\end{algorithmic}
\label{alg:sgstrainingpriorfull}
\end{algorithm}
% %%%%%%%%%%%%%%%%%%

\clearpage
\paragraph{Inference.} 
During inference, we use the learned probability distribution from \edgemlp. We keep track of the best temperature $T$ that gave the best validation accuracy and use that to sample an ensemble of sparse subgraphs. Then, we mean-aggregate their representations to produce the final
prediction on a test node. 

The reason we consider ensemble of subgraphs is because there are variability in the edges of the sample subgraphs even if they are all sampled from the same distribution. Thus mean-aggregation of node embeddings is an effective way to improve the robustness of the learned node embeddings. 

The inference pseudocode is provided in Algorithm~\ref{alg:sgsinference}.

% %%%%%%%%%%%%%%%%%%
\begin{algorithm}[!htbp]
\caption{\sgs Inference}
\begin{algorithmic}[1] % The [1] here is for line numbering
\STATE \textbf{Input:} Graph $\gG (\gV, \gE, \mX)$, sample \% $q$, Ensemble size, $R$.
\STATE \textbf{Output:} Prediction, $\hat{\mY}$

%\STATE Sample size, $Q =\frac{q\gE}{100}$
%\STATE Degree Norm, $p(u,v) = \frac{1/d_u + 1/d_v}{\sum_{i,j\in \gE} (1/d_i + 1/d_j)}$
    \STATE $\vw, \tilde{p} = \edgemlp(\gE, \mX, T_\mathrm{best})$ \COMMENT{\textbf{Use $T$ that gave best validation accuracy}.}   
    \STATE $S_y \gets \emptyset$ \COMMENT{Predictions}
    
    \FOR {$i$ in $R$}
        \STATE $\tilde{\gE}, \tilde{\vw} \gets \mathrm{Sample}(\tilde{p},\floor{\frac{q|\gE|}{100}})$        
        \STATE $\hat{\mY}_i \gets \mathrm{GNN}_\theta(\tilde{\gE},\tilde{\vw},\mX)$
        \STATE $S_y \gets S_y \cup \hat{\mY}_i$
    \ENDFOR

    \STATE Predict, $\hat{\mY} \gets \mathrm{Mean} (S_y)$
    
\STATE \textbf{Return} $\hat{\mY}$
\end{algorithmic}
\label{alg:sgsinference}
\end{algorithm}
% %%%%%%%%%%%%%%%%%%
\clearpage


% \FloatBarrier
\section{Dataset Description}
\label{app:dataset}
\begin{table}[!htbp]
\caption{Additional details of the dataset are provided. $\gH_\mathrm{adj}$ refers to adjusted homophily. $d$ corresponds to the average degree, $C$ number of classes, and $F$ is the feature dimension. \textit{Tr.} is the training label rate.}
\label{tab:datasetdescription}
% \begin{wrapfigure}{c}{1.0\textwidth}
\centering
\begin{sc}
\resizebox{1.0\linewidth}{!}
{
\def\arraystretch{1.0}
\begin{tabular}{@{}crrrccrcccl@{}}
\toprule
\textbf{Dataset} &
  $|\gV|$ &
  $|\gE|$ &
  \textbf{$d$} &
  \textbf{$\gH_\mathrm{adj}$} &
  \textbf{$C$} &
  \textbf{$F$} &
  \textbf{Tr.} &
  \textbf{Self-Loop} &
  \textbf{Isolated} &
  \textbf{Context} \\ \midrule
Cornell        & 183       & 557         & 3.04   & -0.42 & 5  & 1703 & 0.48 & TRUE  & FALSE & Web Pages           \\
Texas          & 183       & 574         & 3.14   & -0.26 & 5  & 1703 & 0.48 & TRUE  & FALSE & Web Pages           \\
Wisconsin      & 251       & 916         & 3.65   & -0.20 & 5  & 1703 & 0.48 & TRUE  & FALSE & Web Pages           \\
reed98         & 962       & 37,624      & 39.11  & -0.10 & 3  & 1001 & 0.6  & FALSE & FALSE & Social Network      \\
amherst41      & 2,235     & 181,908     & 81.39  & -0.07 & 3  & 1193 & 0.6  & FALSE & FALSE & Social Network      \\
penn94         & 41,554    & 2,724,458   & 65.56  & -0.06 & 2  & 4814 & 0.47 & FALSE & FALSE & Social Network      \\
Roman-empire   & 22,662    & 65,854      & 2.91   & -0.05 & 18 & 300  & 0.5  & FALSE & FALSE & Wikipedia           \\
cornell5       & 18,660    & 1,581,554   & 84.76  & -0.04 & 3  & 4735 & 0.6  & FALSE & FALSE & Web pages           \\
Squirrel       & 5,201     & 396,846     & 76.30  & -0.01 & 5  & 2345 & 0.48 & TRUE  & FALSE & Wikipedia           \\
johnshopkins55 & 5,180     & 373,172     & 72.04  & 0.00  & 3  & 2406 & 0.6  & FALSE & FALSE & Web Pages           \\
Actor          & 7,600     & 53,411      & 7.03   & 0.01  & 5  & 932  & 0.48 & TRUE  & FALSE & Actors in Movies    \\
Minesweeper    & 10,000    & 78,804      & 7.88   & 0.01  & 2  & 7    & 0.5  & FALSE & FALSE & Synthetic           \\
Questions      & 48,921    & 307,080     & 6.28   & 0.02  & 2  & 301  & 0.5  & FALSE & FALSE & Yandex Q            \\
Chameleon      & 2,277     & 62,792      & 27.58  & 0.03  & 5  & 2581 & 0.48 & TRUE  & FALSE & Wiki Pages          \\
Tolokers       & 11,758    & 1,038,000   & 88.28  & 0.09  & 2  & 10   & 0.5  & FALSE & FALSE & Toloka Platform     \\
Amazon-ratings & 24,492    & 186,100     & 7.60   & 0.14  & 5  & 556  & 0.5  & FALSE & FALSE & Co-purchase network \\
genius         & 421,961   & 1,845,736   & 4.37   & 0.17  & 2  & 12   & 0.6  & FALSE & TRUE  & Social Network      \\
pokec          & 1,632,803 & 44,603,928  & 27.32  & 0.42  & 3  & 65   & 0.6  & FALSE & FALSE & Social Network      \\
arxiv-year     & 169,343   & 2,315,598   & 13.67  & 0.26  & 5  & 128  & 0.6  & FALSE & FALSE & Citation            \\
snap-patents   & 2,923,922 & 27,945,092  & 9.56   & 0.21  & 5  & 269  & 0.6  & TRUE  & TRUE  & Citation            \\
ogbn-proteins  & 132,534   & 79,122,504  & 597.00 & 0.05  & 94 & 8    & 0.2  & FALSE & FALSE & Protein Network     \\\midrule \midrule
Cora           & 19,793    & 126,842     & 6.41   & 0.56  & 70 & 8710 & 0.2  & FALSE & FALSE & Citation Network    \\
DBLP           & 17,716    & 105,734     & 5.97   & 0.68  & 4  & 1639 & 0.2  & FALSE & FALSE & Citation Network    \\
Computers      & 13,752    & 491,722     & 35.76  & 0.68  & 10 & 767  & 0.6  & FALSE & TRUE  & Co-purchase Network \\
PubMed         & 19,717    & 88,648      & 4.50   & 0.69  & 3  & 500  & 0.2  & FALSE & FALSE & Social Network      \\
Cora\_ML        & 2,995     & 16,316      & 5.45   & 0.75  & 7  & 2879 & 0.2  & FALSE & FALSE & Citation Network    \\
SmallCora      & 2,708     & 10,556      & 3.90   & 0.77  & 7  & 1433 & 0.05 & FALSE & FALSE & Citation Network    \\
CS             & 18,333    & 163,788     & 8.93   & 0.78  & 15 & 6805 & 0.2  & FALSE & FALSE & Co-author Network   \\
Photo          & 7,650     & 238,162     & 31.13  & 0.79  & 8  & 745  & 0.2  & FALSE & TRUE  & Co-purchase Network \\
Physics        & 34,493    & 495,924     & 14.38  & 0.87  & 5  & 8415 & 0.2  & FALSE & FALSE & Co-author Network   \\
CiteSeer       & 4,230     & 10,674      & 2.52   & 0.94  & 6  & 602  & 0.2  & FALSE & FALSE & Citation Network    \\
wiki           & 11,701    & 431,726     & 36.90  & 0.58  & 10 & 300  & 0.99 & TRUE  & TRUE  & Wikipedia           \\
Reddit         & 232,965   & 114,615,892 & 491.99 & 0.74  & 41 & 602  & 0.66 & FALSE & FALSE & Social Network      \\ \bottomrule
\end{tabular}
}
\end{sc}
\end{table}
% \end{wrapfigure}
Table~\ref{tab:datasetdescription} shows the details of the characteristics of the graph datasets, including the splits used throughout the experimentation.

Along with synthetic dataset, for heterophily, we used, 
\textit{Cornell, Texas}, \textit{Wisconsin} from the \textit{WebKB}~\cite{pei2020geom}; \textit{Chameleon}, \textit{Squirrel} ~\cite{rozemberczki2021multi}; \textit{Actor} ~\cite{pei2020geom}; \textit{Wiki, ArXiv-year, Snap-Patents, Penn94, Pokec, Genius, reed98, amherst41, cornell5}, and \textit{Yelp}~\cite{lim2021large}. 
We also experiment on some recent benchmark datasets, \textit{Roman-empire, Amazon-ratings, Minesweeper, Tolokers}, and \textit{Questions} from~\cite{platonov2023critical}.

For homophily, we used
\textit{Cora}~\cite{sen2008collective}; \textit{Citeseer}~\cite{giles1998citeseer}; \textit{pubmed} \cite{namata2012query}; \textit{Coauthor-cs}, \textit{Coauthor-physics}~\cite{shchur2018pitfalls}; \textit{Amazon-computers},  \textit{Amazon-photo} ~\cite{shchur2018pitfalls}; \textit{Reddit}~\cite{hamilton2017inductive}; and, \textit{DBLP}~\cite{fu2020magnn}. 

\noindent\textbf{Heterophily Characterization.} The term \emph{homophily} in a graph describes the likelihood that nodes with the same labels are neighbors. Although there are several ways to measure homophily, three commonly used measures are {\em homophily of the nodes} ($\gH_{n}$), {\em homophily of the edges} ($\gH_{e}$), and {\em adjusted homophily} ($\gH_\mathrm{adj}$).
The {\em node homophily}~\cite{pei2020geom} is defined as,  
\begin{align}
\gH_{n} & = \frac{1}{|\gV|} \sum_{u\in \gV}\frac{| \{v\in \gN(u) : y_v = y_u\}|}{|\gN(u)|}.
\end{align}
The {\em edge homophily}~\cite{zhu2020beyond} of a graph is,

\begin{equation}
    \gH_{e} = \frac{|\{(u,v)\in \gE : y_u = y_v\}|}{|\gE|}. 
\end{equation}

The 
{\em adjusted homophily}~\cite{platonov2022characterizing} is defined as,
\begin{equation}
    \gH_\mathrm{adj} = \frac{\gH_{e}-\sum_{k=1}^{c} D_k^2/(2|\gE|^2)}{1-\sum_{k=1}^c D_k^2/2|\gE|^2}. 
\end{equation}

Here, $D_k = \sum_{v:y_v=k}d_v$ denote the sum of degrees of the nodes belonging to class $k$. 

The values of the node homophily and the edge homophily range from $0$ to $1$, and the adjusted homophily ranges from $-\frac{1}{3}$ to $+1$ (Proposition 1 in~\cite{platonov2022characterizing}). 
Among these measures, adjusted homophily considers the class imbalance. Thus, this work classifies graphs with adjusted homophily, $\gH_\mathrm {adj} \le 0.50$ as heterophilic.

% \paragraph{Dataset used in experiments.} 

\clearpage


% \FloatBarrier
\section{Runtime Comparison}
\label{app:runtime}

\subsection{Impact of Conditional Updates on Runtime}
Table.~\ref{tab:largescaleruntime} compares the runtime of \sgs with and without conditional updates for large-scale graphs (with $|\gE| \ge 1M$). The results indicate that conditional updates are similar to our standard training algorithm in terms of computational efficiency while providing improvements in F1-score under identical conditions. The additional computational costs of evaluation with prior get compensated by fewer updates of \edgemlp.

% Please add the following required packages to your document preamble:
% \usepackage{booktabs}
\begin{table}[!htbp]
\caption{Comparison of runtime of \sgs with and without conditional updates on large-scale graphs (with $|\gE| \ge 1M$). Here, \textit{Runtime (s)} refers to the mean training time per epoch. The terms \edgemlp/\gnn represent the proportion of time the \edgemlp module is updated relative to the \gnn. The results indicate that conditional updates are not significantly slower than our standard training algorithm, yet provide performance improvements to \sgs under similar conditions.}
\label{tab:largescaleruntime}
% \begin{wrapfigure}{c}{1.0\textwidth}
\centering
\begin{sc}
\resizebox{1.0\columnwidth}{!}
{
\def\arraystretch{1.0}
\begin{tabular}{@{}crrr|cc|cc|c@{}}
\toprule
\multirow{2}{*}{\textbf{Dataset}} & \multirow{2}{*}{\textbf{Node}} & \multirow{2}{*}{\textbf{Edges}} & \multirow{2}{*}{\textbf{Degree}} & \multicolumn{2}{c|}{\textbf{\sgs Runtime (s)}} & \multicolumn{2}{c|}{\textbf{\sgs F1-Score}} & \multirow{2}{*}{\textbf{\#EdgeMLP/\#GNN}} \\
 &  &  &  & \textbf{w/o. cond} & \textbf{w. cond} & \textbf{w/o. cond} & \textbf{w. cond} &  \\\midrule
cornell5 & 18,660 & 1,581,554 & 84.76 & \textbf{0.3625} & 0.3795 & 69.02 $\pm$ 0.09 & \textbf{69.12 $\pm$ 0.20} & 0.94 \\
Tolokers & 11,758 & 1,038,000 & 88.28 & 0.1743 & \textbf{0.1630} & 78.12 $\pm$ 0.13 & \textbf{78.13 $\pm$ 0.17} & 0.42 \\
genius & 421,961 & 1,845,736 & 4.37 & \textbf{0.3884} & 0.4799 & 79.92 $\pm$ 0.08 & \textbf{80.07 $\pm$ 0.11} & 0.43 \\
pokec & 1,632,803 & 44,603,928 & 27.32 & 6.7984 & \textbf{6.4885} & 62.05 $\pm$ 0.33 & \textbf{62.20 $\pm$ 0.10} & 0.75 \\
arxiv-year & 169,343 & 2,315,598 & 13.67 & \textbf{0.4571} & 0.4580 & \textbf{36.99 $\pm$ 0.11} & 36.98 $\pm$ 0.13 & 0.23 \\
snap-patents & 2,923,922 & 27,945,092 & 9.56 & \textbf{6.3470} & 7.1236 & 34.86 $\pm$ 0.15 & \textbf{34.95 $\pm$ 0.16} & 0.84 \\
Reddit & 232,965 & 114,615,892 & 491.99 & \textbf{8.0892} & 8.2960 & \textbf{91.45 $\pm$ 0.07} & 91.43 $\pm$ 0.02 & 0.44\\\bottomrule
\end{tabular}
}
\end{sc}
\end{table}
% \end{wrapfigure}

\subsection{Comparison with Baseline GNN based Sparsifiers}
\label{app:runtimerelated}
Table~\ref{tab:runtimerelated} shows related algorithms' mean training time (s). Although \sgs is slower than the unsupervised sparsification-based GNNs, it is significantly faster than supervised sparsifiers.
% Please add the following required packages to your document preamble:
% \usepackage{booktabs}
\begin{table}[!htbp]
% \begin{wrapfigure}{c}{1.0\textwidth}
\caption{Mean training time (s) per epoch of related methods. OOM refers to out-of-memory.}
\label{tab:runtimerelated}
\centering
\begin{sc}
\resizebox{1.0\columnwidth}{!}
{
\def\arraystretch{1.0}
\begin{tabular}{@{}c|ccccccc@{}}
\toprule
\textbf{Method} & \textbf{ClusterGCN} & \textbf{GraphSAINT} & \textbf{DropEdge} & \textbf{MOG} & \textbf{SparseGAT} & \textbf{Neural Sparse} & \textbf{SGS-GNN} \\ \midrule
CS & 0.0095 & 0.0089 & 0.0146 & OOM & 0.1009 & 0.1515 & 0.0221 \\
Questions & 0.0082 & 0.0072 & 0.0290 & 0.1263 & 0.0236 & 0.1221 & 0.0261 \\
Amazon-ratings & 0.0068 & 0.0062 & 0.0169 & 0.1054 & 0.0152 & 0.0499 & 0.0178 \\
johnshopkins55 & 0.0071 & 0.0061 & 0.0207 & OOM & 0.0102 & 0.1234 & 0.0244 \\
amherst41 & 0.0062 & 0.0058 & 0.0101 & OOM & 0.0053 & 0.0368 & 0.0162 \\ \bottomrule
\end{tabular}
}
\end{sc}
\end{table}
% \end{wrapfigure}
\clearpage


\section{Ablation Studies}
\label{app:ablationstudy}

This section investigates how different components of \sgs behave and contribute to overall performance. We organize this section as follows,

\begin{enumerate}
    \item Section~\ref{subsec:ab_edgemlpgnn} investigates $\gL_\mathrm{assor}, \gL_{cons}$, \edgemlp, \gnn, and Conditional Updates mechanism. We also compare its runtime against standard \sgs training vs \sgs with conditional updates. We also show \sgs can be used with other GNNs in Sec~\ref{app:othergnn}.

    \item Section~\ref{app:parameters} explores parameter settings with/without prior, different normalization and sampling methods, and inference with/without an ensemble of subgraphs.

    \item Section~\ref{app:gridsearch} shows ideal settings for regularizer coefficients $\alpha_1, \alpha_2, \alpha_3$. We also show the impact of $\lambda$ for augmenting the learned probability distribution $p$ using $p_\mathrm{prior}$.
\end{enumerate}


\subsection{$\gL_\mathrm{assor}, \gL_{cons}$, \edgemlp, \gnn, and Conditional Updates}
\label{subsec:ab_edgemlpgnn}

Table~\ref{tab:ablationgnn} illustrates the performance of \sgs with various combinations of regularizers, embedding layers in \edgemlp, and convolutional layers in \gnn. 

\begin{enumerate}
    \item $\gL_\mathrm{assor}$: Case 1, 2 shows improvement in results when $\gL_\mathrm{assor}$ is used.

    \item $\gL_\mathrm{cons}$: From cases 4, 6, 8 shows $L_\mathrm{cons}$ improves results when $\texttt{GCN}$ module is used in the GNN.

    \item \edgemlp: In general, we found that the \texttt{GCN} layers for \edgemlp encodings performs best (cases 5-6, 11-12). 
    
    \item \gnn: Both \texttt{GCN} and  \texttt{GAT} modules yielded overall the best results (case 6, 11).
    
    \item Conditional updates: Case 3 shows that conditional updates can benefit some graphs.     
    
    We also investigated the runtime and quality of \sgs with and without conditional updates for large-scale graphs. We found both have similar runtime as the condition check expense gets compensated by fewer updates of \edgemlp. Detailed comparisons of conditional updates in large graphs ($|\gE|\ge 1M$) are included in the Table~\ref{tab:largescaleruntime}.   
\end{enumerate}







\begin{table}[!htbp]
\caption{Combination of \edgemlp, \gnn, Conditional update and $L_\mathrm{cons}.$}
\label{tab:ablationgnn}
% \begin{wrapfigure}{c}{1.0\linewidth}
\centering
\begin{sc}
\resizebox{0.9\linewidth}{!}
{
\def\arraystretch{1.0}
\begin{tabular}{cccccc|ccc}
\toprule
% \rowcolor[HTML]{B7B7B7} 
\textbf{} & {$\mathbf{L_\mathrm{assor}}$} & {$\mathbf{L_\mathrm{cons}}$} & \textbf{\edgemlp} & \textbf{\gnn} & \textbf{Cond.} & {\textbf{SmallCora}} & {\textbf{CoraFull}} & {\textbf{johnshopkin}} \\ \midrule
1 & N & N & \cellcolor[HTML]{F4CCCC}MLP & \cellcolor[HTML]{FFF2CC}GCN & N & 73.80 $\pm$ 0.67 & 61.78 $\pm$ 0.20 & 66.12 $\pm$ 1.38 \\
2 & Y & N & \cellcolor[HTML]{F4CCCC}MLP & \cellcolor[HTML]{FFF2CC}GCN & N & 74.88 $\pm$ 0.15 & 63.99 $\pm$ 0.24 & 66.18 $\pm$ 1.05 \\
3 & Y & N & \cellcolor[HTML]{F4CCCC}MLP & \cellcolor[HTML]{FFF2CC}GCN & Y & 75.82 $\pm$ 0.46 & 64.07 $\pm$ 0.31 & 66.87 $\pm$ 0.93 \\
4 & Y & Y & \cellcolor[HTML]{F4CCCC}MLP & \cellcolor[HTML]{FFF2CC}GCN & Y & 76.58 $\pm$ 0.47 & 65.33 $\pm$ 0.28 & 69.25 $\pm$ 0.76 \\
5 & Y & N & \cellcolor[HTML]{D0E0E3}GCN & \cellcolor[HTML]{FFF2CC}GCN & Y & 75.80 $\pm$ 0.77 & 65.66 $\pm$ 0.14 & 71.06 $\pm$ 0.32 \\
\rowcolor[HTML]{D9D9D9} 
6 & Y & Y & \cellcolor[HTML]{D0E0E3}GCN & \cellcolor[HTML]{FFF2CC}GCN & Y & 77.50 $\pm$ 0.62 & \textbf{66.56 $\pm$ 0.22} & 70.79 $\pm$ 0.18 \\
7 & Y & N & \cellcolor[HTML]{CFE2F3}GSAGE & \cellcolor[HTML]{FFF2CC}GCN & Y & 75.82 $\pm$ 0.44 & 63.70 $\pm$ 0.09 & 67.53 $\pm$ 0.80 \\
8 & Y & Y & \cellcolor[HTML]{CFE2F3}GSAGE & \cellcolor[HTML]{FFF2CC}GCN & Y & 77.48 $\pm$ 0.61 & 65.12 $\pm$ 0.11 & 68.63 $\pm$ 0.66 \\
9 & Y & N & \cellcolor[HTML]{F4CCCC}MLP & \cellcolor[HTML]{D9EAD3}GAT & Y & 77.72 $\pm$ 1.63 & 66.40 $\pm$ 0.08 & 67.92 $\pm$ 0.73 \\
10 & Y & Y & \cellcolor[HTML]{F4CCCC}MLP & \cellcolor[HTML]{D9EAD3}GAT & Y & 75.78 $\pm$ 3.22 & 66.46 $\pm$ 0.16 & 68.17 $\pm$ 0.33 \\
\rowcolor[HTML]{D9D9D9} 
11 & Y & N & \cellcolor[HTML]{D0E0E3}GCN & \cellcolor[HTML]{D9EAD3}GAT & Y & \textbf{78.18 $\pm$ 0.74} & 66.33 $\pm$ 0.20 & \textbf{71.97 $\pm$ 0.59} \\
12 & Y & Y & \cellcolor[HTML]{D0E0E3}GCN & \cellcolor[HTML]{D9EAD3}GAT & Y & 76.94 $\pm$ 2.76 & 66.39 $\pm$ 0.18 & 71.00 $\pm$ 0.96 \\
13 & Y & N & \cellcolor[HTML]{CFE2F3}GSAGE & \cellcolor[HTML]{D9EAD3}GAT & Y & 77.98 $\pm$ 0.79 & 66.38 $\pm$ 0.23 & 69.29 $\pm$ 1.56 \\
14 & Y & Y & \cellcolor[HTML]{CFE2F3}GSAGE & \cellcolor[HTML]{D9EAD3}GAT & Y & 75.74 $\pm$ 2.02 & 66.41 $\pm$ 0.25 & 68.82 $\pm$ 0.24\\\bottomrule
\end{tabular}
}
\end{sc}
\end{table}
% \end{wrapfigure}




\subsubsection{\sgs with other GNN modules}
\label{app:othergnn}
The sampled sparse subgraphs from \edgemlp can be fed into any downstream GNNs and demonstrate a couple of variants of \sgs. Chebnet from Chebyshev~\cite{he2022convolutional}, Graph Attention Network (GAT)~\cite{velivckovic2017graph}, Graph Isomorphic Network (GIN)~\cite{xu2018powerful}, Graph Convolutional Network (GCN)~\cite{kipf2016semi} are some of the GNNs used for demonstration. 

Fig.~\ref{fig:sparsityvsgnn} shows the performance of these GNNs on homophilic and heterophilic datasets. \texttt{SGS-GCN} and \texttt{SGS-GAT} are two best performing models.


%%%%%%%%%%%%%%%%%%%%%%%%%%%%%%%%%%%%%%%%%%%%
\begin{figure}[!htbp]
\centering
% \includegraphics[width=0.5\linewidth]{Figures/SparsityvsGNN.png}
\includegraphics[width=0.6\linewidth]{Figures/SGS-differentgnn.pdf}

\caption{Performance of \sgs with different GNN modules using $20\%$ edges.}
\label{fig:sparsityvsgnn}
\end{figure}
%%%%%%%%%%%%%%%%%%%%%%%%%%%%%%%%%%%%%%%%%%%%


% \subsection{Additional Ablation studies}
% \label{app:moreablation}

\subsection{Impact of  $p_\mathrm{prior}$, Normalization \& Sampling schemes, and Ensembling on \sgs}
\label{app:parameters}

Table~\ref{tab:ablation} highlights the impact of the following components: 

% Please add the following required packages to your document preamble:
% \usepackage{booktabs}
\begin{table}[t]
\caption{Ablation Studies different components of \sgs.}
\label{tab:ablation}
% \begin{wrapfigure}{c}{1.0\textwidth}
\centering
\begin{sc}
\resizebox{1.0\linewidth}{!}
{
\def\arraystretch{1.0}
\begin{tabular}{@{}cccccc|cccccc@{}}
\toprule
\textbf{Case} & \textbf{Prior} & \textbf{Norm.} & \textbf{Sampl.}  & \multicolumn{1}{c|}{\textbf{Ensem.}} & \textbf{SmallCora} & \textbf{Cora\_ML} & \textbf{CiteSeer} & \textbf{Squirrel} & \textbf{johnshopkins55} & \textbf{Roman-empire} \\ \midrule
1 & N  & Sum & Mult  & \multicolumn{1}{c|}{N} & 69.30 $\pm$ 1.20 & 81.05 $\pm$ 0.74 & 82.84 $\pm$ 0.47 & 48.90 $\pm$ 1.06 & 63.86 $\pm$ 0.58 & 63.27 $\pm$ 0.31 \\
2 & N  & Sum & Mult  & \multicolumn{1}{c|}{Y} & 72.84 $\pm$ 0.91 & 82.92 $\pm$ 0.73 & \textbf{87.42 $\pm$ 0.42} & 46.30 $\pm$ 1.18 & 65.14 $\pm$ 1.14 & \textbf{64.31 $\pm$ 0.13} \\
% 3 & N  & Sum & Mult & $L_\mathrm{a*}$ & \multicolumn{1}{c|}{Y} & 73.48 $\pm$ 1.11 & \textbf{85.01 $\pm$ 0.33} & 86.43 $\pm$ 0.23 & 49.68 $\pm$ 0.73 & \textbf{74.73 $\pm$ 0.51} & 63.20 $\pm$ 0.21 \\
%4 & N  & Sum & Mult & $L_\mathrm{a*}$, $L_\mathrm{c*}$ & \multicolumn{1}{c|}{Y} & 75.86 $\pm$ 0.74 & 84.82 $\pm$ 0.24 & 86.55 $\pm$ 0.33 & 48.03 $\pm$ 0.79 & 72.08 $\pm$ 1.16 & 63.16 $\pm$ 0.30 \\
% 5 & Y & 0 & Sum & Mult & $L_\mathrm{a*}$, $L_\mathrm{c*}$ & \multicolumn{1}{c|}{Y} & 74.82 $\pm$ 0.64 & 84.21 $\pm$ 0.60 & 86.55 $\pm$ 0.25 & 47.53 $\pm$ 0.32 & 68.44 $\pm$ 0.46 & 63.06 $\pm$ 0.11 \\
% 6 & Y & 1 & Sum & Mult & $L_\mathrm{a*}$, $L_\mathrm{c*}$ & \multicolumn{1}{c|}{Y} & 75.54 $\pm$ 0.23 & 84.01 $\pm$ 0.74 & 86.34 $\pm$ 0.21 & 48.63 $\pm$ 0.44 & 70.77 $\pm$ 0.40 & 63.27 $\pm$ 0.12 \\
3 & Y  & Sum & Mult & \multicolumn{1}{c|}{Y} & 75.54 $\pm$ 0.41 & \textbf{83.87 $\pm$ 0.69} & 86.31 $\pm$ 0.26 & 47.97 $\pm$ 0.60 & 72.68 $\pm$ 0.51 & 62.88 $\pm$ 0.19 \\
4 & Y & Softmax & Mult & \multicolumn{1}{c|}{Y} & 75.44 $\pm$ 0.51 & 83.81 $\pm$ 0.72 & 86.31 $\pm$ 0.26 & 47.90 $\pm$ 0.42 & \textbf{72.97 $\pm$ 0.20} & 62.98 $\pm$ 0.16 \\
5 & Y & Gumbel & TopK & \multicolumn{1}{c|}{Y} & \textbf{76.24 $\pm$ 0.43} & 83.36 $\pm$ 0.34 & 86.44 $\pm$ 0.16 & \textbf{51.49 $\pm$ 0.72} & 71.83 $\pm$ 1.00 & 63.00 $\pm$ 0.11 \\ \midrule
\multicolumn{11}{l}{\textbf{Prior:} Use of prior, \textbf{Sum:} Sum-Normalization, \textbf{Softmax:} \textit{Softmax} with temperature annealing}\\
\multicolumn{11}{l}{\textbf{Mult:} \textit{Multinonmial} Sampling, \textbf{Gumbel:} \textit{Gumbel-Softmax} with \textit{TopK}}\\
\end{tabular}
}
\end{sc}
\end{table}
% \end{wrapfigure}

\begin{enumerate}
    \item Prior $p_\mathrm{prior}$: Cases 2-3 show that augmenting the learned probability distribution $\tilde{p}$ with prior $p_\mathrm{prior}$ can benefit some datasets. We have also conducted an in-depth comparison between the distributions $\tilde{p}$ and augmented distribution $\tilde{p}_a$. Figure~\ref{fig:augment_p} shows that there $\tilde{p}_a$ is left skewed whereas $\tilde{p}$ is not. Since rare edges still get some negligible mass, it is possible for $\tilde{\gG}$ constructed from $\tilde{p}_a$ to retain some bridge edges from these tails, if there are any. 
    
\begin{figure}[!htbp]
    \centering
    %\subfigure{\includegraphics[width=0.48\linewidth]{Figures/SparsityvsHomophily.png}
    \subfigure{\includegraphics[width=0.4\linewidth]{Figures/SGS-learnedp.png}
    \label{subfig:learnedp}} 
    %\hfill
    % \subfigure{\includegraphics[width=0.48\linewidth]{Figures/SparsityvsAccuracy2.png}
    %\hfill     
     \subfigure{\includegraphics[width=0.4\linewidth]{Figures/SGS-learnedp_a.png}
     \label{subfig:priorpa}} 
     \subfigure{\includegraphics[width=0.4\linewidth]{Figures/SGS-prior.png}
     \label{subfig:priorp}}    
    \caption{The learned probability distribution $\tilde{p}$ (top-left), augmented distribution $\tilde{p}_a$(top-right) and fixed prior $p_\mathrm{prior}$ (bottom). Augmentation puts negligible mass on some rare yet critical edges in the left tail of $\tilde{p}_a$.}
    \label{fig:augment_p}
\end{figure}
    \item Normalization and Sampling: We considered three normalization and sampling techniques. i) sum-normalization with multinomial sampling, ii) softmax-normalization with temperature with multinomial sampling, and iii) Gumbel softmax normalization with Topk selection. Cases 3-5 show that each of these techniques can improve results in certain datasets, and thus, it is difficult to nominate a single one as best. However, in our experiments, we opted for multinomial sampling with softmax temperature annealing for training to encourage exploration in early iterations.

    \item Ensemble subgraphs during inference: Case 2 demonstrates that using multiple subgraphs for ensemble prediction yields better results than a single subgraph (Case 1).    
\end{enumerate}

 % The addition of assortative loss $L_\mathrm{assor}$ in Case 3 enhances performance on heterophilic graphs by promoting homophily in sampled subgraphs. %Case 4 shows that incorporating consistency loss $L_\mathrm{cons}$ benefits certain graphs. 
%While our base training method updates \edgemlp at each iteration, conditional update with a prior distribution speeds up results by reducing the search space. Cases 5-7 indicate that hop size at \edgemlp influences performance. 

% % Please add the following required packages to your document preamble:
% \usepackage{booktabs}
\begin{table}[t]
\caption{Ablation Studies different components of \sgs.}
\label{tab:ablation}
% \begin{wrapfigure}{c}{1.0\textwidth}
\centering
\begin{sc}
\resizebox{1.0\linewidth}{!}
{
\def\arraystretch{1.0}
\begin{tabular}{@{}cccccc|cccccc@{}}
\toprule
\textbf{Case} & \textbf{Prior} & \textbf{Norm.} & \textbf{Sampl.}  & \multicolumn{1}{c|}{\textbf{Ensem.}} & \textbf{SmallCora} & \textbf{Cora\_ML} & \textbf{CiteSeer} & \textbf{Squirrel} & \textbf{johnshopkins55} & \textbf{Roman-empire} \\ \midrule
1 & N  & Sum & Mult  & \multicolumn{1}{c|}{N} & 69.30 $\pm$ 1.20 & 81.05 $\pm$ 0.74 & 82.84 $\pm$ 0.47 & 48.90 $\pm$ 1.06 & 63.86 $\pm$ 0.58 & 63.27 $\pm$ 0.31 \\
2 & N  & Sum & Mult  & \multicolumn{1}{c|}{Y} & 72.84 $\pm$ 0.91 & 82.92 $\pm$ 0.73 & \textbf{87.42 $\pm$ 0.42} & 46.30 $\pm$ 1.18 & 65.14 $\pm$ 1.14 & \textbf{64.31 $\pm$ 0.13} \\
% 3 & N  & Sum & Mult & $L_\mathrm{a*}$ & \multicolumn{1}{c|}{Y} & 73.48 $\pm$ 1.11 & \textbf{85.01 $\pm$ 0.33} & 86.43 $\pm$ 0.23 & 49.68 $\pm$ 0.73 & \textbf{74.73 $\pm$ 0.51} & 63.20 $\pm$ 0.21 \\
%4 & N  & Sum & Mult & $L_\mathrm{a*}$, $L_\mathrm{c*}$ & \multicolumn{1}{c|}{Y} & 75.86 $\pm$ 0.74 & 84.82 $\pm$ 0.24 & 86.55 $\pm$ 0.33 & 48.03 $\pm$ 0.79 & 72.08 $\pm$ 1.16 & 63.16 $\pm$ 0.30 \\
% 5 & Y & 0 & Sum & Mult & $L_\mathrm{a*}$, $L_\mathrm{c*}$ & \multicolumn{1}{c|}{Y} & 74.82 $\pm$ 0.64 & 84.21 $\pm$ 0.60 & 86.55 $\pm$ 0.25 & 47.53 $\pm$ 0.32 & 68.44 $\pm$ 0.46 & 63.06 $\pm$ 0.11 \\
% 6 & Y & 1 & Sum & Mult & $L_\mathrm{a*}$, $L_\mathrm{c*}$ & \multicolumn{1}{c|}{Y} & 75.54 $\pm$ 0.23 & 84.01 $\pm$ 0.74 & 86.34 $\pm$ 0.21 & 48.63 $\pm$ 0.44 & 70.77 $\pm$ 0.40 & 63.27 $\pm$ 0.12 \\
3 & Y  & Sum & Mult & \multicolumn{1}{c|}{Y} & 75.54 $\pm$ 0.41 & \textbf{83.87 $\pm$ 0.69} & 86.31 $\pm$ 0.26 & 47.97 $\pm$ 0.60 & 72.68 $\pm$ 0.51 & 62.88 $\pm$ 0.19 \\
4 & Y & Softmax & Mult & \multicolumn{1}{c|}{Y} & 75.44 $\pm$ 0.51 & 83.81 $\pm$ 0.72 & 86.31 $\pm$ 0.26 & 47.90 $\pm$ 0.42 & \textbf{72.97 $\pm$ 0.20} & 62.98 $\pm$ 0.16 \\
5 & Y & Gumbel & TopK & \multicolumn{1}{c|}{Y} & \textbf{76.24 $\pm$ 0.43} & 83.36 $\pm$ 0.34 & 86.44 $\pm$ 0.16 & \textbf{51.49 $\pm$ 0.72} & 71.83 $\pm$ 1.00 & 63.00 $\pm$ 0.11 \\ \midrule
\multicolumn{11}{l}{\textbf{Prior:} Use of prior, \textbf{Sum:} Sum-Normalization, \textbf{Softmax:} \textit{Softmax} with temperature annealing}\\
\multicolumn{11}{l}{\textbf{Mult:} \textit{Multinonmial} Sampling, \textbf{Gumbel:} \textit{Gumbel-Softmax} with \textit{TopK}}\\
\end{tabular}
}
\end{sc}
\end{table}
% \end{wrapfigure}




\clearpage
\subsection{Choosing Values for Regularizer coefficient \(\alpha_3\) and Parameter \(\lambda\)}
\label{app:gridsearch}
Recall that \sgs computes the total loss at each epoch as
\[
\gL = \alpha_1\gL_{CE}+\alpha_2\gL_\mathrm{assor}+\alpha_3\gL_\mathrm{cons},
\]
where $0 \leq \alpha_1,\alpha_2,\alpha_3 \leq 1$ are regularizer coefficients corresponding to the cross-entropy loss $\gL_{CE}$, assortativity loss $L_\mathrm{assor}$ and  consistency loss $\gL_\mathrm{cons}$ respectively. 

Also recall that, when we use a prior probability distribution, the learned distribution values of $\tilde{p}$ are weighted through $\lambda$ in
$\Tilde{p} = \lambda\Tilde{p}+(1-\lambda) p_\mathrm{prior}$

To avoid numerous combinations of values of three coefficients + the parameter $\lambda$, we have fixed $\alpha_1 = 1$, and $\alpha_2 = 1$. In the following, we investigate the performance of \sgs with different values for $\alpha_3$ and $\lambda$.

Fig.~\ref{fig:consbias} shows a grid search for different combinations of $\lambda$ and $\alpha_3$. As per our observation, the recommended values are $\lambda \in [0.3, 0.7], \alpha_3=0.5$.
%%%%%%%%%%%%%%%%%%%%%%%%%%%%%%%%%%%%%%%%%%%%
\begin{figure}[h]
\centering
\includegraphics[width=0.4\linewidth]{Figures/SGS-biasgrid.pdf}
\caption{Grid search for the parameter $\lambda$ for prior, and consistency loss, $\alpha_3$ (Cora dataset).}
\label{fig:consbias}
\end{figure}
%%%%%%%%%%%%%%%%%%%%%%%%%%%%%%%%%%%%%%%%%%%%






%\IEEEPARstart{T}{his} demo file is intended to serve as a ``starter file''
%for IEEE Computer Society journal papers produced under \LaTeX\ using
%IEEEtran.cls version 1.8b and later.
%% You must have at least 2 lines in the paragraph with the drop letter
%% (should never be an issue)
%I wish you the best of success.
%
%\hfill mds
% 
%\hfill August 26, 2015
%
%\subsection{Subsection Heading Here}
%Subsection text here.
%
%% needed in second column of first page if using \IEEEpubid
%%\IEEEpubidadjcol
%
%\subsubsection{Subsubsection Heading Here}
%Subsubsection text here.
%
%
%% An example of a floating figure using the graphicx package.
%% Note that \label must occur AFTER (or within) \caption.
%% For figures, \caption should occur after the \includegraphics.
%% Note that IEEEtran v1.7 and later has special internal code that
%% is designed to preserve the operation of \label within \caption
%% even when the captionsoff option is in effect. However, because
%% of issues like this, it may be the safest practice to put all your
%% \label just after \caption rather than within \caption{}.
%%
%% Reminder: the "draftcls" or "draftclsnofoot", not "draft", class
%% option should be used if it is desired that the figures are to be
%% displayed while in draft mode.
%%
%%\begin{figure}[!t]
%%\centering
%%\includegraphics[width=2.5in]{myfigure}
%% where an .eps filename suffix will be assumed under latex, 
%% and a .pdf suffix will be assumed for pdflatex; or what has been declared
%% via \DeclareGraphicsExtensions.
%%\caption{Simulation results for the network.}
%%\label{fig_sim}
%%\end{figure}
%
%% Note that the IEEE typically puts floats only at the top, even when this
%% results in a large percentage of a column being occupied by floats.
%% However, the Computer Society has been known to put floats at the bottom.
%
%
%% An example of a double column floating figure using two subfigures.
%% (The subfig.sty package must be loaded for this to work.)
%% The subfigure \label commands are set within each subfloat command,
%% and the \label for the overall figure must come after \caption.
%% \hfil is used as a separator to get equal spacing.
%% Watch out that the combined width of all the subfigures on a 
%% line do not exceed the text width or a line break will occur.
%%
%%\begin{figure*}[!t]
%%\centering
%%\subfloat[Case I]{\includegraphics[width=2.5in]{box}%
%%\label{fig_first_case}}
%%\hfil
%%\subfloat[Case II]{\includegraphics[width=2.5in]{box}%
%%\label{fig_second_case}}
%%\caption{Simulation results for the network.}
%%\label{fig_sim}
%%\end{figure*}
%%
%% Note that often IEEE papers with subfigures do not employ subfigure
%% captions (using the optional argument to \subfloat[]), but instead will
%% reference/describe all of them (a), (b), etc., within the main caption.
%% Be aware that for subfig.sty to generate the (a), (b), etc., subfigure
%% labels, the optional argument to \subfloat must be present. If a
%% subcaption is not desired, just leave its contents blank,
%% e.g., \subfloat[].
%
%
%% An example of a floating table. Note that, for IEEE style tables, the
%% \caption command should come BEFORE the table and, given that table
%% captions serve much like titles, are usually capitalized except for words
%% such as a, an, and, as, at, but, by, for, in, nor, of, on, or, the, to
%% and up, which are usually not capitalized unless they are the first or
%% last word of the caption. Table text will default to \footnotesize as
%% the IEEE normally uses this smaller font for tables.
%% The \label must come after \caption as always.
%%
%%\begin{table}[!t]
%%% increase table row spacing, adjust to taste
%%\renewcommand{\arraystretch}{1.3}
%% if using array.sty, it might be a good idea to tweak the value of
%% \extrarowheight as needed to properly center the text within the cells
%%\caption{An Example of a Table}
%%\label{table_example}
%%\centering
%%% Some packages, such as MDW tools, offer better commands for making tables
%%% than the plain LaTeX2e tabular which is used here.
%%\begin{tabular}{|c||c|}
%%\hline
%%One & Two\\
%%\hline
%%Three & Four\\
%%\hline
%%\end{tabular}
%%\end{table}
%
%
%% Note that the IEEE does not put floats in the very first column
%% - or typically anywhere on the first page for that matter. Also,
%% in-text middle ("here") positioning is typically not used, but it
%% is allowed and encouraged for Computer Society conferences (but
%% not Computer Society journals). Most IEEE journals/conferences use
%% top floats exclusively. 
%% Note that, LaTeX2e, unlike IEEE journals/conferences, places
%% footnotes above bottom floats. This can be corrected via the
%% \fnbelowfloat command of the stfloats package.
%
%
%
%
%\section{Conclusion}
%The conclusion goes here.
%
%
%
%
%
%% if have a single appendix:
%%\appendix[Proof of the Zonklar Equations]
%% or
%%\appendix  % for no appendix heading
%% do not use \section anymore after \appendix, only \section*
%% is possibly needed
%
%% use appendices with more than one appendix
%% then use \section to start each appendix
%% you must declare a \section before using any
%% \subsection or using \label (\appendices by itself
%% starts a section numbered zero.)
%%
%
%
%\appendices
%\section{Proof of the First Zonklar Equation}
%Appendix one text goes here.
%
%% you can choose not to have a title for an appendix
%% if you want by leaving the argument blank
%\section{}
%Appendix two text goes here.
%
%
%% use section* for acknowledgment
%\ifCLASSOPTIONcompsoc
%  % The Computer Society usually uses the plural form
%  \section*{Acknowledgments}
%\else
%  % regular IEEE prefers the singular form
%  \section*{Acknowledgment}
%\fi
%
%
%The authors would like to thank...
%
%
%% Can use something like this to put references on a page
%% by themselves when using endfloat and the captionsoff option.
%\ifCLASSOPTIONcaptionsoff
%  \newpage
%\fi
%
%
%
%% trigger a \newpage just before the given reference
%% number - used to balance the columns on the last page
%% adjust value as needed - may need to be readjusted if
%% the document is modified later
%%\IEEEtriggeratref{8}
%% The "triggered" command can be changed if desired:
%%\IEEEtriggercmd{\enlargethispage{-5in}}
%
%% references section
%
%% can use a bibliography generated by BibTeX as a .bbl file
%% BibTeX documentation can be easily obtained at:
%% http://mirror.ctan.org/biblio/bibtex/contrib/doc/
%% The IEEEtran BibTeX style support page is at:
%% http://www.michaelshell.org/tex/ieeetran/bibtex/
%%\bibliographystyle{IEEEtran}
%% argument is your BibTeX string definitions and bibliography database(s)
%%\bibliography{IEEEabrv,../bib/paper}
%%
%% <OR> manually copy in the resultant .bbl file
%% set second argument of \begin to the number of references
%% (used to reserve space for the reference number labels box)
%\begin{thebibliography}{1}
%
%\bibitem{IEEEhowto:kopka}
%H.~Kopka and P.~W. Daly, \emph{A Guide to \LaTeX}, 3rd~ed.\hskip 1em plus
%  0.5em minus 0.4em\relax Harlow, England: Addison-Wesley, 1999.
%
%\end{thebibliography}
%
%% biography section
%% 
%% If you have an EPS/PDF photo (graphicx package needed) extra braces are
%% needed around the contents of the optional argument to biography to prevent
%% the LaTeX parser from getting confused when it sees the complicated
%% \includegraphics command within an optional argument. (You could create
%% your own custom macro containing the \includegraphics command to make things
%% simpler here.)
%%\begin{IEEEbiography}[{\includegraphics[width=1in,height=1.25in,clip,keepaspectratio]{mshell}}]{Michael Shell}
%% or if you just want to reserve a space for a photo:
%
%\begin{IEEEbiography}{Michael Shell}
%Biography text here.
%\end{IEEEbiography}
%
%% if you will not have a photo at all:
%\begin{IEEEbiographynophoto}{John Doe}
%Biography text here.
%\end{IEEEbiographynophoto}
%
%% insert where needed to balance the two columns on the last page with
%% biographies
%%\newpage
%
%\begin{IEEEbiographynophoto}{Jane Doe}
%Biography text here.
%\end{IEEEbiographynophoto}

% biography section

%\begin{IEEEbiography}[{\includegraphics[width=1in,height=1.25in,clip,keepaspectratio]{Figures/Hao_Tang.png}}]{Hao Tang} is currently a Postdoctoral with Computer Vision Lab, ETH Zurich, Switzerland.
%	He received the master’s degree from the School of Electronics and Computer Engineering, Peking University, China and the Ph.D. degree from Multimedia and Human Understanding Group, University of Trento, Italy.
%	He was a visiting scholar in the Department of Engineering Science at the University of Oxford. His research interests are deep learning, machine learning, and their applications to computer vision.
%\end{IEEEbiography}

% You can push biographies down or up by placing
% a \vfill before or after them. The appropriate
% use of \vfill depends on what kind of text is
% on the last page and whether or not the columns
% are being equalized.

%\vfill

% Can be used to pull up biographies so that the bottom of the last one
% is flush with the other column.
%\enlargethispage{-5in}



% that's all folks
\end{document}


\grid


% todo: 1. 讲解framework、method和module的划分界限
% todo: 2. 表格字体调小一些,能够整合最好
% todo: 3. 参考文献的引用,有没有中论文,中了就把arxiv的格式改掉,会议全改成缩写
% todo: 4. 第一次出现用缩写即可