\section{Characterization of injective polynomials over unrolls}\label{section:poly_unroll}
	
	The search for a solution to polynomial equations over FDDS can be seen as the search for a compatible solution between solutions of polynomial equations of transient states and solutions of polynomial equations of periodic states.
	This is why we can start by looking for solutions for transient states.
	In order to do this, we will use the notion of unroll and determine the number of solutions to polynomial equations over unrolls.
	
	More precisely, our final goal is to prove the following theorem:
	\begin{theorem}\label{th:injPolyUnrolls}
		All univariate polynomial over unrolls are injective. 
	\end{theorem}
	
	For the proof of this theorem, we begin by remarking that for all polynomial~$P$ over unrolls, and for all FDDS $X,Y$ the equality $P(\unroll{X}) = P(\unroll{Y})$ implies $\cut{P(\unroll{X})}{n} = \cut{P(\unroll{Y})}{n}$ for all $n$. In order to study polynomial equations over forests of finite trees, we start by proving one technical lemma on the behavior of the order on powers of trees with the same depth.
	
		
		\begin{lemma}\label{lemma:behaviorOrderWithProduct}
			Let $\tree{x}, \tree{y}$ be two finite trees with the same depth and $k$ be a  positive integer. Then $\tree{y}^k \ge \tree{x}^k$ if and only if $\tree{y} \ge \tree{x}$.
		\end{lemma}
	
		\begin{proof}
			If $\tree{y} \ge \tree{x}$ and $\depth{\tree{x}} = \depth{\tree{y}}$, by \cite[Corollary 20]{article_arbre}, we have $\tree{y}^2 \ge \tree{x}^2$ and $\depth{\tree{y}^2} = \depth{\tree{x}^2}$. 
			Then, by induction, we obtain $\tree{y}^k \ge \tree{x}^k$.
			
			For the other direction, assume that $\tree{y}^k \ge \tree{x}^k$ and, by contradiction, that $\tree{x} > \tree{y}$. 
			Since $\depth{\tree{x}} = \depth{\tree{y}}$, by \cite[Corollary 20]{article_arbre}, we have $\tree{y}^k \tree{x} \ge \tree{y}^k \tree{y}$. 
			Thus, by \cite[Lemma 21]{article_arbre}, we have $\tree{y}^{k-1} \ge \tree{x}^{k-1}$. 
			By induction, we obtain $\tree{y} \ge \tree{x}$, a contradiction.
	\end{proof}
		
		Let $\forest{A}$ be a forest and $i\ge0$ be an integer. 
		We denote by % $\gamma(\forest{A},i)$ the sub-multiset of trees of $\forest{A}$ having depth \emph{exactly} $i$; we also
	        % denote by
                $\Gamma(\forest{A},i)$ the sub-multiset of trees of $\forest{A}$ where each tree has depth \emph{at least} $i$.
		We focus our attention on the trees in $P(\forest{X})$ with a
		certain depth and form.
		
		\begin{lemma}\label{lemme:ordreProdFin}
	
			Let $P = \polyF{1}{m}{a}{x}$ be a polynomial without constant term over forests and $\forest{X}= \tree{x}_1 + \ldots + \tree{x}_n$ be a forest sorted in nondecreasing order.
			Let $\forest{A}_i = \sum_{j = 1}^{m_i} \tree{a}_{i,j}$ be in nondecreasing order, hence $\tree{a}_{i,1} = \min(\forest{a}_i)$, for all~$i$.
			Then, there exists $\alpha \in \{1, \ldots, m\}$ such that $\min_{i}(\forest{A}_i \forest{X}^{i-1} \tree{x}_k)$ is isomorphic to $\tree{a}_{\alpha,1} \tree{x}_1^{\alpha-1}\tree{x}_k$.
		\end{lemma}
		
		\begin{proof}
			Let $\tree{t}_{min}$ be the smallest tree in $P(\forest{X})$. 
			Hence, there exists $\alpha \in \{1, \ldots, m\}$ such that $\tree{t}_{min} \in \forest{A}_\alpha \forest{X}^\alpha$. 
			In addition, by~\cite[Lemma 24]{article_arbre}, we have $\tree{x}_1^2 \le \tree{x}_1 \tree{x}_j$ for all $j$. 
			By induction, we deduce that $\tree{t}_{min} \in \forest{A}_\alpha \tree{x}_1^\alpha$. 
			Analogously, we have $\tree{a}_{\alpha,1} \tree{x}_1^\alpha \le \tree{a}_{\alpha,j} \tree{x}_1^\alpha$ for all $\tree{a}_{\alpha,j} \in \forest{A}_\alpha$, and we conclude that $\tree{t}_{min} = \tree{a}_{\alpha,1} \tree{x}_1^\alpha$.
			
			We recall that $\tree{t}_1 \tree{t}_3 < \tree{t}_2 \tree{t}_3$ if and only if $\cut{\tree{t}_1}{\depth{\tree{t}_3}} < \cut{\tree{t}_2}{\depth{\tree{t}_3}}$ (Lemma~\ref{lemma:ordre_comp_produit}). 
			Therefore, for all $j$ and $\tree{a} \in \forest{A}_j$, we have:
			\[\tree{a}_{\alpha,1} \tree{x}_1^\alpha \le \tree{a} \tree{x}_1^j \Leftrightarrow \cut{\tree{a}_{\alpha,1} \tree{x}_1^{\alpha-1}}{\depth{\tree{x}_1}} \le \cut{\tree{a} \tree{x}_1^{j-1}}{\depth{\tree{x}_1}}\]
			Three cases are possible.
			First, if  $\alpha = j$ then $\tree{a}_{\alpha,1} \le  \tree{a}$ by minimality of $\tree{a}_{\alpha,1}$.
			This implies that $\tree{a}_{\alpha,1} \tree{x}_1^{\alpha-1} \tree{x}_k \le \tree{a} \tree{x}_1^{\alpha-1} \tree{x}_k$.
	
			Second, if~$\alpha \ne j$ and $\alpha \neq 1$, then $\cut{\tree{a}_{\alpha,1} \tree{x}_1^{\alpha-1}}{\depth{\tree{x}_1}} = \tree{a}_{\alpha,1} \tree{x}_1^{\alpha-1}$. 
			Since $\cut{\tree{a} \tree{x}_1^{j-1}}{\depth{\tree{x}_1}} \le\tree{a} \tree{x}_1^{j-1}$, we have $\tree{a}_{\alpha,1} \tree{x}_1^{\alpha-1} \tree{x}_k \le \tree{a} \tree{x}_1^{j-1} \tree{x}_k$. 
			
			Thirdly, if $\alpha \ne j$ but $\alpha = 1$, then either
                        \[
                        \cut{\tree{a}_{\alpha,1}\tree{x}_1^{\alpha-1}}{\depth{\tree{x}_1}} = \cut{\tree{a} \tree{x}_1^{j-1}}{\depth{\tree{x}_1}} = \tree{a} \tree{x}_1^{j-1}
                        \]
                        which implies $\tree{a}_{\alpha,1}\tree{x}_1^{\alpha} = \tree{a}_{j,1}\tree{x}_1^{j}$, and we can just choose $\alpha=j$, which is already covered by the second case. 
			Otherwise $\cut{\tree{a}_{\alpha,1}\tree{x}_1^{\alpha-1}}{\depth{\tree{x}_1}} < \tree{a} \tree{x}_1^{j-1}$. 
			In this case, we need to recall the definition of the order from~\cite{article_arbre}~and the code upon which it is defined. 
			Indeed, by~\cite[Lemma 17]{article_arbre}, we know that the leftmost difference between the codes of $\cut{\tree{a}_{\alpha,1}\tree{x}_1^{\alpha-1}}{\depth{\tree{x}_1}}$ and $\cut{\tree{a} \tree{x}_1^{j-1}}{\depth{\tree{x}_1}}$ corresponds to a node of depth at most $\depth{\tree{x}_1} - 1$. 
			Therefore, this difference is propagated into $\cut{\tree{a}_{\alpha,1}\tree{x}_1^{\alpha-1}}{\depth{\tree{x}_1}} \tree{x}_k$ and $\tree{a} \tree{x}_1^{j-1} \tree{x}_k$ if $\depth{\tree{x}_k} \ge \depth{\tree{x}_1}-1$, and $\tree{a}_{\alpha,1} \tree{x}_1^{\alpha-1} \tree{x}_k < \tree{a} \tree{x}_1^{j-1} \tree{x}_k$; if $\depth{\tree{x}_k} < \depth{\tree{x}_1}-1$ then $\tree{a}_{\alpha,1} \tree{x}_1^{\alpha-1} \tree{x}_k = \tree{a} \tree{x}_1^{j-1} \tree{x}_k$.
			
			Hence, in all cases we have $\tree{a}_{\alpha,1} \tree{x}_1^{\alpha-1} \tree{x}_k \le \tree{a} \tree{x}_1^{j-1} \tree{x}_k$.
			Since for all $p$ we have $\tree{x}_1^{p-1} \le \tree{t}$ with $\tree{t} \in \forest{X}^{p-1}$, the lemma follows.
		\end{proof}
		
		Thanks to Lemma~\ref{lemme:ordreProdFin} we can prove the first important result of this section. Remark that, due to closure under sums and products, the set of forests of finite trees of depth at most~$d_{max}$ is a semiring for every natural number~$d_{max}$.
		
		\begin{proposition}\label{prop:injDesFinis}
	        Let~$P = \sum_{i=0}^{m} \forest{A}_i \forest{X}^i$ be a polynomial with~$d_{max} = \max_{i > 0}(\depth{\forest{A}_i)}$ and~$\depth{\forest{A}_0} \le d_{max}$. Then~$P$ is injective over the semiring of forests of finite trees of depth at most~$d_{max}$. 
	        
	        % Let~$\forest{A}_0$ and~$\forest{A}_1, \ldots, \forest{A}_m$ be forests of finite trees with~$d_{max} = \max_{i > 0}(\depth{\forest{A}_i)}$ and~$\depth{\forest{A}_0} \le d_{max}$. Consider the polynomial~$P = \sum_{i=0}^{m} \forest{A}_i \forest{X}^i$ over the semiring of forests of finite trees of depth at most~$d_{max}$. Then~$P$ is injective.
			% Let $d_{max}$ be a natural number and $P = \sum_{i=1}^{m} \forest{A}_i \forest{X}^i$ a polynomial over forests without constant terms with depth at most $d_{max}$\mr{formulation?} such that $d_{max} = \max(\depth{\forest{A}_i})$.
			% Then $P$ is injective.
	%		Then, for all $0\le d \le d_{max}$ then $\sum_{i=1}^{m} \Gamma(\forest{A}_i,d) \Gamma(\forest{X}^i, d)$ is injective.
		\end{proposition}
		
		
		\begin{proof}
	        Let us first consider polynomials~$P$ without a constant term.
			Let $\forest{X} = \tree{x}_1 + \ldots + \tree{x}_{m_x}$ and $\forest{Y} = \tree{y}_1 + \ldots + \tree{y}_{m_y}$ be two forest (sorted in nondecreasing order), and assume $P(\forest{X}) = P(\forest{Y})$. 
			Thus, by Lemma~\ref{lemme:ordreProdFin}, there exist $i, j \in \{1, \ldots, m\}$ such that, for each $k$, the smallest tree $\tree{t}_s$ with factor $\tree{x}_k$ (resp., $\tree{y}_k$) in $\polyF{1}{m}{a}{x}$ is isomorphic to $\tree{a}_{i,1} \tree{x}_1^{i-1}\tree{x}_k$ (resp., $\tree{a}_{j,1} \tree{y}_1^{j-1}\tree{y}_k$), with $\tree{a}_{i,1} = \min(\forest{A}_i)$ and $\tree{a}_{j,1} = \min(\forest{A}_j)$. 
			From the hypothesis, we deduce that $\tree{a}_{i,1} \tree{x}_1 = \min(P(\forest{x})) =  \min(P(\forest{y})) = \tree{a}_{j,1} \tree{y}_1$. 
			
			First, we show that $\tree{x}_1 = \tree{y}_1$.
			By contradiction, and without loss of generality, assume $\tree{x}_1 < \tree{y}_1$.
			Then, by Lemma~\ref{lemma:behaviorOrderWithProduct}, it follows that $\tree{x}_1^{j} < \tree{y}_1^{j}$.
			By \cite[Lemma 24]{article_arbre}, we deduce that $\tree{a}_{j,1} \tree{x}_1^{j} < \tree{a}_{j,1} \tree{y}_1^j = \tree{a}_{i,1} \tree{x}_1^i$.
			Nevertheless, $\tree{a}_{j,1} \tree{x}_1^{j} \in P(\forest{X})$,
			contradicting the minimality of $\tree{a}_{i,1} \tree{x}_1^i$. 
			
			Finally, since $\tree{x}_1 = \tree{y}_1$, we deduce that $P(\forest{X}) - P(\tree{x}_1) = P(\forest{Y}) - P(\tree{y}_1)$.  
			In addition, the smallest tree in $P(\forest{X}) - P(\tree{x}_1)$ is $\tree{a}_{i,1} \tree{x}_1^{i-1} \tree{x}_2$ and the smallest tree in $P(\forest{y}) - P(\tree{y}_1)$ is $\tree{a}_{j,1} \tree{y}_1^{j-1} \tree{y}_2$.  
			However, since $\tree{a}_{i,1} \tree{x}_1^i = \tree{a}_{j,1} \tree{y}_1^j$ and $\tree{x}_1 = \tree{y}_1$, we have $\tree{a}_{i,1} \tree{x}_1^{i-1} = \tree{a}_{j,1} \tree{y}_1^{j-1}$. 
			Thus, $\tree{x}_2 = \tree{y}_2$. 
			By induction, we conclude that $\forest{X} = \forest{Y}$.
	
	        This proof extends to polynomials with a constant term~$\forest{A}_0$. Indeed, let~$Q = P + \forest{A}_0$ and suppose~$Q(\forest{X}) = Q(\forest{Y})$. Then $Q(\forest{X}) - \forest{A}_0 = Q(\forest{Y}) - \forest{A}_0$, that is~$P(\forest{X}) = P(\forest{Y})$, which implies~$\forest{X} = \forest{Y}$. Thus, $Q$ is also injective.
		\end{proof}
	
	        Our main theorem follows directly from Proposition~\ref{prop:injDesFinis}.
	
	   \begin{proof}[Proof of Theorem~\ref{th:injPolyUnrolls}]
	        Suppose~$P(\unroll{X}) = P(\unroll{Y})$. Then~$\cut{P(\unroll{X})}{n} = \cut{P(\unroll{Y})}{n}$ for all~$n$. If~$\unroll{X}$ and~$\unroll{Y}$ were different, there would exist an~$n$ such that~$\cut{\unroll{X}}{n} \ne \cut{\unroll{Y}}{n}$. However, by Proposition~\ref{prop:injDesFinis} and since~$\cut{\cdot}{n}$ is a homomorphism, we have~$\cut{\unroll{X}}{n} = \cut{\unroll{Y}}{n}$ for all~$n$.
	    \end{proof}
	        
		% From this theorem, we can draw two remarks. Indeed, by Proposition~\ref{prop:polyUnroll2PolyForestFini}, we have that the polynomial $\sum_{i=1}^{m} \unroll{A_i} \unroll{X}^i$ is injective if and only if $\sum_{i=1}^{m} \cut{\unroll{A_i}}{n} \cut{\forest{y}}{n}^i$ is injective for a certain positive integer $n$.
		% And, for this kind of polynomial, each forest has depth at most $n$.  
		% Then, by Proposition~\ref{prop:injDesFinis}, the resulting function is injective.
	        % and the proof of Theorem \ref{th:injPolyUnrolls} follows.
		
		The proof of Proposition~\ref{prop:injDesFinis} also suggests an algorithm (inspired by \cite[Algorithm~1]{kroot}) for solving polynomial equations over forests of finite trees. 
		Indeed, starting with the maximum depth $d_{max}$ (Fig.~\ref{fig:algo} \textcircled{1}), we can solve the equation $\Gamma(P(\forest{x}), d_{max}) = \Gamma(\forest{b}, d_{max})$ \textcircled{2}.
		In fact, we construct the solution $\Gamma(\forest{X},d_{max})$ by first solving the equation $\min(\Gamma(\forest{A}_i,d_{max})) \tree{x}_1^i = \min(\Gamma(\forest{b}, d_{max}))$ \textcircled{3}\textcircled{4} for an~$i \in \{1, \ldots, m\}$. 
		The tree~$\tree{x}_1$ is minimal among the trees of maximum depth appearing in the solution~$\forest{X}$. 
		Thus, by Lemma~\ref{lemme:ordreProdFin}, either we can construct the solution $Sol$ inductively \textcircled{4} or $P(Sol) \nsubseteq \forest{B}$ \textcircled{3}, and we can deduce that this~$i$ is not the one appearing in the statement of Lemma~\ref{lemme:ordreProdFin}, and we try the next value of~$i$, if any.
		Indeed, the $j$-th tree~$\tree{x}_i$ of the solution is equal to
	        \[
	        \cfrac{\min(\Gamma(\forest{b}, d_{max}) - \Gamma(P(\tree{x}_1 + \ldots + \tree{x}_{j-1}), d_{max}))}{\min(\Gamma(\forest{A}_i,d_{max})) \tree{x}_1^{i-1}}.
	        \]
     	Then, we can inductively construct $\tree{x} = \Gamma(\forest{X}, d)$ from $\Gamma(\forest{x}, d')$ where $d < d_{max}$ is the depth of a tree in $\forest{B}$ and $d'$ the smallest depth of a tree in $\forest{B}$ greater than~$d$~\textcircled{5}. In order to do so, we need to find~$\min(\Gamma(\forest{X}, d))$. Let $\tree{a} = \min(\Gamma(\forest{A}_j,d))$ for a~$j \in \{1, \ldots, m\}$ and~$\tree{b} = \min(\Gamma(\forest{B},d))$.
     	\begin{itemize}
	        \item If $\tree{b}$ has depth at least~$d'$ \textcircled{6}, or $\tree{a}$ has depth~$d$ and~$\tree{a} \times \min(\Gamma(\forest{X}, d'))^j \le \tree{b}$, then $\tree{x}$ has depth at least~$d'$ and thus is already in $\Gamma(\forest{x}, d')$.
	        \item Otherwise~$\tree{b}$ necessarily has depth~$d$. In that case, either~$\tree{a}$ has depth at least~$d'$, or~$\tree{a} \times \min(\Gamma(\forest{X}, d'))^j > \tree{b}$, and in both cases we have~$\tree{x} = \sqrt[j]{\tree{b} / \tree{a}}$.
     	\end{itemize}
		Then, similarly to the step for depth~$d_{max}$, we can construct a portion of the solution~$\forest{X}$ \textcircled{8}, or find a mistake \textcircled{7} and try another value of~$j$, if any. If during the iteration we find mistakes for all coefficients, this implies that the equation does not have a solution.
		
		Moreover, the number of iterations is bounded by the number of coefficients~$m$ times the depth~$d_{max}$, which are both polynomial with respect to the size of the inputs, and the operations carried out in each iteration can be performed in polynomial time \cite{article_arbre,kroot}.
	
	        We cannot directly apply this algorithm to polynomial equations over unrolls, since they contain infinite trees, but we can show that, in fact, we only need to consider a finite cut at depth polynomial with respect to the sizes of the FDDS in order to decide if a solution exists:
	
	% \begin{proposition}\label{prop:polyUnroll2PolyForestFini}
	% 	Let $P = \sum_{i=0}^{m} A_i X^i$ be polynomial over FDDS, $B$ an FDDS and $\alpha$ the number of unroll trees of $\unroll{B}$.
	% 	Let $n \ge 2 \times \alpha^2 + \depth{\unroll{B}}$ be an integer.
	% 	If there exists a forest $\forest{y}$ with $\depth{\forest{y}} \le n$ such that $\sum_{i=0}^{m}\cut{\unroll{A_i}}{n} \forest{y}^i = \cut{\unroll{B}}{n}$, then there exists a FDDS $X$ such that $\unroll{P(X)} = \unroll{B}$ and $\cut{\unroll{X}}{n} = \forest{y}$.
	% \end{proposition}

        \begin{figure}[p]
        \centering
        \includegraphics[page=3,width=\textwidth]{pictures}
        % \vspace{-1em}
        \caption{A run of the algorithm for polynomial equations over forests of finite trees.}
        \label{fig:algo}
        \end{figure}


	\begin{proposition}\label{prop:polyUnroll2PolyForestFini}
		Let $P = \sum_{i=0}^{m} A_i X^i$ be polynomial over FDDS, $B$ an FDDS and $\alpha$ the number of unroll trees of $\unroll{B}$.
		Let $n \ge 2 \times \alpha^2 + \depth{\unroll{B}}$ be an integer.
		Then, there exists a forest $\forest{y}$ with $\depth{\forest{y}} \le n$ such that $\sum_{i=0}^{m}\cut{\unroll{A_i}}{n}  \forest{y}^i = \cut{\unroll{B}}{n}$ if and only if there exists a FDDS $X$ such that $\unroll{P(X)} = \unroll{B}$. Furthermore, if such an $X$ exists, then necessarily $\cut{\unroll{X}}{n} = \forest{y}$.
	\end{proposition} 
	
	\begin{proof}
		Assume that $\sum_{i=0}^{m}\cut{\unroll{A_i}}{n} \forest{y}^i = \cut{\unroll{B}}{n}$.
		First, remark that we can apply a similar reasoning to \cite[Lemma 11]{kroot} in order prove that if there exists $\forest{x}$ such that $\sum_{i=0}^{m} \unroll{A_i} \forest{X} = \unroll{B}$ then there exists an FDDS $Y$ such that $\unroll{Y} = \forest{X}$. 
		Since $\alpha$ bounds the greatest cycle length in $A_0,\ldots,A_m,B$ and since, from the hypothesis, $\depth{\unroll{B}}$ bounds $\depth{\unroll{A_i}}$ for each $i$, we have all the tools needed in order to prove (similarly to \cite[Theorem 5]{kroot}) that there exists a FDDS $X$ such that $\unroll{P(X)} = \unroll{B}$. The inverse implication is trivial.
        \end{proof}
                
        This proposition has two consequences. 
        First of all, the solution (if it exists) to an equation $\sum_{i=0}^{m} \unroll{A_i} \forest{X}^i = \unroll{B}$ is always the unroll of an FDDS. 
        Furthermore, by employing the same method for rebuilding $\unroll{X}$ from the solution~$\forest{Y}$ with~$\depth{\forest{Y}} \le n$ of $\sum_{i=0}^{m}\cut{\unroll{A_i}}{n} \forest{y}^i = \cut{\unroll{B}}{n}$ exploited in the proof of \cite[Theorem 6]{kroot}, we deduce an algorithm for solving equations over unrolls. In order to encode unrolls (which are infinite graphs), we represent them finitely by FDDS having those unrolls.
        
		
		\begin{theorem}\label{theorem:equation_forest_P}
			Let $P$ be a polynomial over unrolls (encoded as a polynomial over FDDS) and~$B$ an FDDS.
			Then, we can solve the equation~$P(\unroll{X}) = \unroll{B}$ in polynomial time.
		\end{theorem} 	
	
%%% Local Variables:
%%% mode: LaTeX
%%% TeX-master: "main"
%%% End:
