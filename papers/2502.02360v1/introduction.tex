\section{Introduction}
\label{sec:introduction}

Finite discrete-time dynamical systems (FDDS) are pairs $(X,f)$ where $X$ is a finite set of states and $f: X \to X$ is a transition function (when~$f$ is implied, we denote $(X,f)$ simply as $X$).
We find these systems in the analysis of concrete models such as Boolean networks~\cite{gershenson2004random_bn,automata_book}
and we can apply them to biology~\cite{thomas1990biological_feedback,thomas1973genetic_control_circuits,bernot2013computational_biology} to represent, for example, genetic regulatory networks or epidemic models. 
They also appear in chemistry~\cite{reaction_systems}, or information theory~\cite{gadouleau2011graph_entropy}.

In the following, we focus on deterministic FDDS, and often identify them with their transition digraphs, which have uniform out-degree one.
These graphs are also known as functional digraphs. 
Their general shape is a collection of cycles, where each node of a cycle is the root of a finite in-tree (a directed tree with the edges directed toward the root).
The nodes of the cycles are periodic states, while the others are transient states.

The set $(\mathbb{D}, +, \times)$ of FDDS up to isomorphism, with the alternative execution of two systems as \emph{addition} and the synchronous parallel 
execution of two FDDS as \emph{multiplication} is a commutative semiring~\cite{article_fondateur}. 
As all semirings, we can define a semiring of polynomials over~$\mathbb{D}$ and thus polynomial equations.

Although it has already been proven that general polynomial equations over~$\mathbb{D}$ (with variables on both sides of the equation) are undecidable, it is easily proved that, if one member of the equation is constant, then the problem is decidable (there is just a finite number of possible solutions)~\cite{article_fondateur}.
This variant of the problem is actually in $\NP$ because, since sums and products can be computed in polynomial time, we can just guess the values of the variables (their size is bounded by the constant side of the equation), evaluate the polynomial, and compare it with the constant side (for out-degree one graphs, isomorphism can be checked in linear time~\cite{testIsoLinear}).

However, more restricted equations are not yet classified in terms of complexity.
For example, we do not know if monomial equations of the form $AX = B$ are $\NP$-hard, in $\P$, or possible candidates for an intermediate difficulty. 
However, it has been shown that when~$A$ and~$B$ are connected and with a cycle of length one (i.e., they are trees with a loop, or fixed point, on the root), the equation $AX = B$ can be solved in polynomial time~\cite{article_arbre}.
It has also been proved that, if $X$ is connected, we can solve $AX^k = B$ in polynomial time with respect to the size of the inputs ($A$ and $B$) for all positive integer~$k$~\cite{kroot}.

These results share a common feature: all these equations have \emph{at most one solution}. 
It is natural, then, to investigate whether \emph{all} polynomial equations with at most one solution can be solved in polynomial time. 
However, we still lack a characterization for this type of equation; indeed, for the same polynomial on one side of the equation, if we change the constant side, the number of solutions can change.
For this reason, considering the case where the polynomial is injective seems to be a relevant starting point, requiring a characterization of this class of polynomials, at least for the univariate case.

In this paper, we focus on this class of polynomials. 
After giving the necessary preliminaries (Section~\ref{sec:preliminaries}), we first focus on the semiring of unrolls (equivalence classes of FDDS sharing the same transient behavior, introduced in~\cite{article_arbre}) and prove that all univariate polynomials over unrolls are injective and that every equation $\sum_{i=0}^{m} \unroll{A_i} \unroll{X}^i = \unroll{B}$ can be solved in polynomial time (Section~\ref{section:poly_unroll}).
Then, we prove that all univariate polynomial $P = \sum_{i=0}^{m} A_i X^i$ with at least one cancelable $A_i$ (for some $i > 0$) is injective, by using a proof technique implying the existence of a polynomial-time algorithm for solving the associated equations (Section~\ref{section:cond_suf_poly_FDDS_inj}).
Finally, we show that this is also a necessary condition for injectivity (Section~\ref{section:cond_nec_poly_FDDS_inj}).

%%% Local Variables:
%%% mode: LaTeX
%%% TeX-master: "main"
%%% End:
