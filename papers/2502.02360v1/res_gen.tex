\section{Preliminaries}
\label{sec:preliminaries}

In this paper we compose and decompose FDDS in terms of two algebraic operations.
Thee \emph{sum} of two FDDS 
consists in the mutually
exclusive alternative between their behaviors, while the \emph{product} is their synchronous parallel execution.
The set of FDDS taken up to isomorphism, with these two operations, forms a semiring~\cite{article_fondateur}.
This semiring is isomorphic to the semiring of functional digraphs with the operations of disjoint union and direct product.
We recall~\cite{livreGraphe} that the direct product of two graphs $A$ and $B$ is the graph $C$
having vertices $V(C) = V(A) \times V(B)$ and edges
\[
E(C) = \{((u,u'), (v,v')) \mid (u,v) \in E(A), (u',v') \in E(B)\}.
\]
Within the class of \emph{connected} FDDS, we further distinguish those having a (necessarily  unique) cycle of size $1$ (\ie, a fixed point in terms of dynamics) and refer to them as \emph{dendrons}. 

In this semiring, the structure of the product is particularly rich. 
For example, the semiring is not factorial, \ie, there exist irreducible FDDS $A,B,C,D$ such that $A \neq C$ and $A \neq D$ but $AB = CD$.
This richness might originate from the interactions between the periodic and the transient behaviors of the two FDDS being multiplied.

% For brevity, we use the term \emph{trees} in reference to in-trees constituting the transient behaviors of FDDS.

For the periodic behavior of FDDS, the product of a connected $A$, with a cycle of size $p$,
and a connected $B$, with a cycle of size $q$,
generates $\gcd(p,q)$ connected components, each with a cycle of size $\lcm(p,q)$~\cite{livreGraphe}.
For the analysis of transient behaviors, we use the notion of unroll introduced in~\cite{article_arbre}.

\begin{definition}[Unroll]
	Let $A = (X,f)$ be an FDDS.
	For each state $u\in X$ and $k \in \mathbb{N}$, we denote by $f^{-k}(u) = \{ v \in X \mid f^k(v) = u \}$ the set of $k$-th preimages of $u$. 
	For each $u$ in a cycle of $A$, we call the \emph{unroll tree of $A$ in $u$} the infinite tree $\tree{t}_u = (V,E)$ having vertices $V = \{(s,k) \mid s \in f^{-k}(u), k \in \mathbb{N}\}$ and edges
	$E = \big\{ \big((v,k),(f(v),k-1) \big) \big\} \subseteq V^2$.
	We call \emph{unroll of $A$}, denoted $\unroll{A}$, the set of its unroll trees (see Fig.~\ref{fig:unroll}).
\end{definition}

\begin{figure}[t]
\centering
\includegraphics[page=1]{pictures}
\caption{An FDDS~$A$ with two connected components (on the left), a finite portion of its unroll~$\unroll{A}$ (on the right) and the cut~$\cut{\unroll{A}}{4}$ (below the dashed line). A few vertex names are shown in order to highlight their contribution to the unroll.}
\label{fig:unroll}
\end{figure}

In the rest of this paper, \emph{unrolls will always be taken up to isomorphism}, \ie, as multisets or sums (disjoint unions) of unlabeled trees; the equality sign will thus represent the isomorphism relation.

An unroll tree contains exactly one infinite branch, onto which are periodically rooted the trees representing the transient behaviors of the corresponding connected component. 
% In order to employ unrolls to study the transient behaviors with respect to the product,
We also exploit a notion of ``levelwise'' product on trees. 
For readability, we will denote trees and forests using bold letters (in lower and upper case
respectively) to distinguish them from FDDS.

\begin{definition}[Product of trees]\label{prodintrees} 
	Let $\tree{t}_1=(V_1,E_1)$ and $\tree{t}_2=(V_2,E_2)$ be two trees with roots $r_1$ and $r_2$, respectively.
        Their \emph{product} is the tree $\tree{t}_1 \times \tree{t}_2=(V,E)$ with vertices
        $V=\set{(u,v)\in V_1\times V_2 \mid \depth{u}=\depth{v}}$, where $\depth{u}$ is the length of the shortest path between $u$ and the root of its tree, and edges $E=\set{((u,u'),(v,v')) \mid (u,v)\in E_1, (u',v')\in E_2}\subseteq V^2$ (see Fig.~\ref{fig:tree-product}).
\end{definition}

\begin{figure}[t]
\centering
\includegraphics[page=2]{pictures}
\caption{The levelwise product of two finite trees. Remark how the depth of the result is given by the minimum depth of the two factors.}
\label{fig:tree-product}
\end{figure}

We deduce from~\cite{article_arbre} that the set of unrolls up to isomorphism, with disjoint union for addition and product of trees for multiplication, is a semiring with the infinite path as the multiplicative identity.
In addition, the unroll operation is a homomorphism between the semiring~$\mathbb{D}$ of FDDS and the semiring of unrolls.
% Here and in the following, the equality sign will denote graph isomorphism.

Unrolls have one major algorithmic issue: they are infinite. 
To overcome this obstacle, we will consider only a finite portion of the unrolls. 
For this purpose, % we first define the \emph{depth} of a finite tree as the maximum length of a shortest path between a leaf and its root.
we extend the notion of \emph{depth} to forests of finite trees by taking the maximum depth of its trees. 
We also define the \emph{depth of an FDDS} as the maximum depth among the trees rooted in one of its periodic states.

We remark that the set of forests of bounded depth~$d$ up to isomorphism is a also semiring (with the restrictions of the same tree sum and product operations) for all depth~$d$, where the multiplicative identity is the path of depth~$d$.

We can now define the \emph{cut} of $\tree{t}$ at depth $k \ge 0$, denoted $\cut{\tree{t}}{k}$,
as the induced sub-tree of $\tree{t}$ restricted to the vertices of depth lesser or equal to $k$ (see Fig.~\ref{fig:unroll}). 
This operation generalizes (homomorphically) to forests $\forest{f} = \tree{t}_{1} + \ldots + \tree{t}_{n}$ as  $\cut{\forest{F}}{k} = \cut{\tree{t}_{1}}{k} + \ldots + \cut{\tree{t}_{n}}{k}$. 
The cut of an unroll at a certain depth has already been used to exhibit properties of unrolls~\cite{article_arbre,kroot}.
This approach notably gave a characterization of \emph{cancelable} FDDS, \ie FDDS $A$ such that $AB = AC$ implies $B = C$ for all FDDS $B,C$.
Indeed, a FDDS $A$ is cancelable if and only if at least one of its connected components is a dendron~\cite[Theorem 30]{article_arbre}. 

One of the tools for the analysis of trees and forests is the total order $\le$ on finite and infinite trees introduced in~\cite{article_arbre}.
Indeed, this order is compatible with the product for infinite trees\footnote{This compatibility with the product of~$\le$ is sufficient for most applications in this paper; we refer the reader to the original paper for the actual definition.}, \ie, if $\tree{t}_1, \tree{t}_2$ are two infinite trees then $\tree{t}_1 \le \tree{t}_2$ if and only if $\tree{t}_1 \tree{t} \le \tree{t}_2 \tree{t}$ for all tree $\tree{t}$~\cite[Lemma 24]{article_arbre}. 
A similar property has been proved for the case of finite trees:

\begin{lemma}[\cite{article_arbre}]\label{lemma:ordre_comp_produit}
	Let $\tree{t}_1, \tree{t}_2, \tree{t}$ then $\cut{\tree{t}_1}{\depth{\tree{t}}} \le \cut{\tree{t}_2}{\depth{\tree{t}}}$ if and only if $\tree{t}_1 \tree{t} \le \tree{t}_2 \tree{t}$.
\end{lemma}

In the rest of the paper all polynomials will be implicitly univariate and, by convention, the symbol~$\forest{a}^0$ (i.e., the $0$-th power of a forest of finite trees~$\forest{a}$) will denote the path of length~$\depth{\forest{a}}$, which behaves as an identity for the product with forests of equal or less depth.


%%% Local Variables:
%%% mode: LaTeX
%%% TeX-master: "main"
%%% End:
