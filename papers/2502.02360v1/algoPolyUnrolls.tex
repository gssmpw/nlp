Dans cette partie, nous allons des algorithmes polynomial pour calculer le $X$ tel $P(X) = B$ avec $P$ un polynôme à une variable et $B$ s'il existe.
Commençons par le cas des déroulements.
Pour ce faire, on pose le lemme suivant.

\begin{lemma}\label{lemme:baseAlgoRaisoEquaPoly}
	Soit $P(\tree{X}) = \sum_{i=1}^{n} \tree{A}_i\tree{X}^{k_i}$ avec $k_i < k_{i+1}$ un polynôme à une variable.
	Soit $d$ la plus grande des profondeurs des $\tree{A}_i$, $\tree{A}_\alpha$ le premier coefficient de $P$ qui a profondeur $d$, $\tree{a}_1$ le plus petit arbre de profondeur $d$ de $\tree{A}_\alpha$ et soit $\tree{B}$ une forêt.
	S'il existe $\tree{X}$ une forêt de profondeur au plus $d$ tel que $P(\tree{X}) = \tree{B}$ alors le plus petit arbre de $\tree{B}$ qui est divisible par $\tree{a}_1$ et dont le quotient admet une racine $k_\alpha$e est $\tree{a}_1 \tree{x}_1^{k_\alpha}$ avec $\tree{x}_1$ le plus petit arbre de $\tree{X}$.
\end{lemma}

\begin{proof}
	Supposons qu'il existe $\tree{X}$ une forêt de profondeur au plus $d$ tel que $P(\tree{X}) = \tree{B}$.
	On pose $\tree{b}$ le plus petit arbre de $\tree{B}$ qui est divisible par $\tree{a}_1$. 
	Supposons, par l'absurde, qu'il existe $\tree{y} \neq \tree{x}_1$ tel que $\tree{a}_1 \tree{y}^{k_\alpha} = \tree{b}$ et $\depth{y} \le d$.
	Donc comme $depth(\tree{x}_1) \le d$ par les lemmes 21 et 43 de \cite{1}, si $\tree{a}_1 \tree{x}_1^{k_\alpha} = \tree{b}$ alors $\tree{x}_1 = \tree{y}$, et puisque $\tree{b}$ est minimal, on déduit donc que $\tree{y} < \tree{x}_1$.
	Ainsi, comme $P(\tree{X}) = \tree{B}$, il existe $\ell$ et $i$ tel que $\tree{a}_\ell \tree{x}_i^{k_\ell} = \tree{b}$.
	Delà, deux cas sont possible.
	\begin{itemize}
		\item $\tree{a}_1 < \tree{a}_\ell$. 
		Donc $\tree{y}^{k_\alpha} > \tree{x}_i^{k_\ell}$.
		Mais comme $k_\ell \ge k_\alpha$, on déduit que $\tree{y} > \tree{x}_i \ge \tree{x}_1$. Contradiction.
		
		\item $\tree{a}_\ell \ge \tree{a}_1$. 
		Donc $\tree{a}_1 \tree{x}_1^{k_\ell} \le \tree{a}_\ell \tree{x}_1^{k_\ell} \le \tree{a}_\ell \tree{x}_i^{k_\ell} = \tree{a}_1 \tree{y}^{k_\alpha}$.
		Or comme $k_\ell \ge k_\alpha$, on a que $\tree{a}_1 \tree{x}_1^{k_\alpha} \le \tree{a}_1 \tree{x}_1^{k_\ell}$.
		Donc $\tree{x}_1^{k_\alpha} \le \tree{y}^{k_\alpha}$. 
		Donc $\tree{x}_1 \le \tree{y}$.
		Contradiction.
	\end{itemize}
\end{proof}