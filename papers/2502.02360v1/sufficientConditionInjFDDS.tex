\section{Sufficient condition for the injectivity of polynomials}\label{section:cond_suf_poly_FDDS_inj}

	Even if all polynomials over unrolls are injective, we can easily remark that is not the case for the FDDS.
	For example, let $A$ be the cycle of length $2$; then, the polynomial $AX$ is not injective since $A^2=2A$.
	Hence, since the existence of injective polynomials is trivial (just consider the polynomial $X$), we want to obtain a characterization of this class of polynomials. 
	% In order to do this, we begin by considering polynomials $P = \sum_{i=1}^{m} A_i X^i$ without constant terms.
	
	Proposition 40 of \cite{article_arbre} gives a sufficient condition for the injectivity of polynomials without constant term, namely if the coefficient~$A_1$ is cancelable, \ie, at least one of its connected component is a dendron.
	However, since~$X^k$ is injective for all~$k>0$ (by uniqueness of $k$-th roots~\cite{article_arbre}), this condition is not necessary. 
	Hence, we start by giving a sufficient condition which covers these cases.
	
	\begin{proposition}\label{prop:condSufInj}
		Let $P = \sum_{i=0}^{m} A_i X^{i}$ be a polynomial.
		If at least one coefficient of $P$ is cancelable then $P$ is injective.
	\end{proposition}
	
	% For the proof of this proposition, we start by showing the case of polynomials $P = \sum_{i=1}^{m} A_i X^i$ without a constant terms. 
	% For that, we need to analyse the structure of the results of $P$. 
	% We first define two preorders over a subset of FDDS, which are actually orders on connected FDDS:
	
	% \begin{definition}
	% 	Let $a,b$ be two positive integer, $A,B$ be two FDDS such all of connected component of $A$ (resp., $B$) have size $a$ (resp., $b$), $\tree{t},\tree{t'}$ be two unroll trees such that $\tree{t}= \min(\unroll{A})$ and $\tree{t'} = \min(\unroll{B})$.
	% 	We define:
	% 	\begin{itemize}
	% 		\item $A \oct B$ if and only if $a > b$ or $a = b$ and $\tree{t} \ge \tree{t'}$.
	% 		\item $A \otc B$ if and only if $\tree{t} > \tree{t'}$ or $\tree{t} = \tree{t'}$ and $a \ge b$.
	% 	\end{itemize}
	% \end{definition}
	
	
	% We can easily prove that these two preorders are total because they are based on two total orders. Our proof technique for Proposition~\ref{prop:condSufInj} consists in studying the structure of the results of the polynomial equation according to the preorders~$\oct$. Remark that $\oct$ and~$\otc$ are not compatible with the product: if~$C_n$ is the cycle of length~$n$, then~$C_4 \oct C_3$ and~$C_4 \otc C_3$, but~$C_4 C_2 \ioct C_3 C_2$ and~$C_4 C_2 \iotc C_3 C_2$. Thus, we can only prove an approximation of the compatibility with the product.
	
	% \begin{lemma}\label{lemme:compatibleOct}
	% 	Let $p$ be an integer and $A,B,X$ be FDDSs such that all connected component of $A$ and $B$ have a cycle of length $p$.
	% 	Let $A'$ (resp., $B'$) be the multiset of components of $AX$ (resp., $BX$) having minimum cycle length.
	% 	If $A \oct B$ then $A' \oct B'$ and $A'$ and $B'$ have the same cycle lengths.
	% \end{lemma}  
	
	% \begin{proof}
	% 	Assume that $A \oct B$.
	% 	Let $\{x_1, \ldots, x_n\}$ be the different lengths of cycles in $X$ and let $i \in \{1,\ldots, n\}$ be such that $\lcm(p, x_i) \le \lcm(p, x_j)$ for all $j \in \{1, \ldots, n\}$.
		
	% 	From the hypotheses on $A$ and $B$ it follows that $\min(\unroll{A}) \ge \min(\unroll{B})$ and $A'$ and $B'$ have cycles of size $\lcm(p, x_i)$.
	%         Let~$X_j$ is minimal with respect to~$\otc$ among the connected components~$X_k$ of~$X$ such that~$\lcm(p, \cycle(X_k)) = \lcm(p, x_i)$.
	% 	% So, $b' = \lcm(p, j) = a'$ with $b'$ (resp., $j, a'$) the size of the cycle of $B'$ (resp., $X_j, A'$).
	%         Then~$\min(\unroll{\min_{\oct}(A')}) = \min(\unroll{A})\min(\unroll{X_j})$; similarly,~$\min(\unroll{\min_{\oct}(B')}) = \min(\unroll{B})\min(\unroll{X_j})$. Since~$A \oct B$ we have~$\min(\unroll{A}) \ge \min(\unroll{B})$ and from the compatibility with the product~\cite{article_arbre} we obtain~$\min(\unroll{A})\min(\unroll{X_j}) \ge \min(\unroll{B})\min(\unroll{X_j})$ and thus~$A' \oct B'$.
	% \end{proof}
	
	% This property is weaker than compatibility with the product, but it turns out to be sufficient in order to find necessary conditions for the injectivity of some classes of polynomials. We begin with one the easier cases: the polynomial $X^k$ for $k > 0$.
	
	% Accordingly, let $k$ be a positive integer and $X = X_1 + \ldots + X_m $ be an FDDS such that each $X_i$ are connected and $X_{i+1} \oct X_i$.  
	% We know that $X^k = \binom{k}{k_1,\ldots,k_n} \prod_{i=1}^{m} X_i^{k_i}$.
	% Furthermore, since the sum of FDDSs is their disjoint union, each FDDS can be written as a sum of trees in a unique way (up to reordering of the terms).
	% Thus, $X^k$ has only this form.
	
	% For studies more precisely the structure of $X^k$, we need to define two multi-set of connected component in an FDDS.
	For the proof of this proposition, we start by showing the case of polynomials $P = \sum_{i=1}^{m} A_i X^i$ without a constant term. 
	For that, we need to analyze the structure of the possible evaluations of $P$, and more precisely their multisets of cycles.
	We denote by $\cycle(A)$ the length of the cycle of a connected FDDS $A$.
	% For this analyze, we need two tools. 
	Let us introduce two total orders over connected FDDS:
	
	\begin{definition}
		% Let $a,b$ be two positive integer, $\tree{t},\tree{t'}$ be two unroll trees and
	        Let $A,B$ be two connected FDDS, let~$a = \cycle(A)$, $b = \cycle(B)$, $\tree{t}= \min(\unroll{A})$, and $\tree{t'} = \min(\unroll{B})$.
		We define:	
		\begin{itemize}
			\item $A \oct B$ if and only if $a > b$ or $a = b$ and $\tree{t} \ge \tree{t'}$.
			\item $A \otc B$ if and only if $\tree{t} > \tree{t'}$ or $\tree{t} = \tree{t'}$ and $a \ge b$.
		\end{itemize}
	\end{definition}
	
	% The second tool for our analysis is two functions of connected component.
	For an FDDS $X$ and a positive integer~$p$, let us consider the function~$\setSize{X}{p}$, which computes the multiset of connected components of $X$ with cycles of length $p$, and the function~$\setDive{X}{p}$, computing the multiset of connected components of $X$ with cycles of length \emph{dividing} $p$.
	
	\begin{lemma}\label{lemme:closureProduct}
		Let $p > 0$ be an integer; then
		$\setDive{\cdot}{p}$ is an endomorphism over FDDS. 
	\end{lemma}
	
	\begin{proof}
		First, if we denote by $\mathbf{1}$ the fixed point (the multiplicative identity in~$\mathbb{D}$), we obviously have $\setDive{\mathbf{1}}{p} = \mathbf{1}$. 
		Now, let $A,B$ be two FDDS. 
		It is clear that $\setDive{A + B}{p} = \setDive{A}{p} + \setDive{B}{p}$ since the sum is the disjoint union.
		All that remains to prove is that $\setDive{AB}{p} = \setDive{A}{p} \setDive{B}{p}$.
		By the distributivity of the sum over the product, it is sufficient to show the property in the case of connected $A$ and $B$. 
		Recall that, by the definition of product of FDDS, the product of two connected FDDS $U,V$ has $\gcd(\cycle(U),\cycle(V))$ connected component with cycle length $\lcm(\cycle(U), \cycle(V))$.  
		We deduce that all elements in $\setDive{AB}{p}$ are generated by the product of elements in $\setDive{A}{p}$ and $\setDive{B}{p}$.
		Hence $\setDive{AB}{p} \subseteq \setDive{A}{p}\setDive{B}{p}$.
		Furthermore, by the definition of $\setDive{A}{p}$, each element of $\setDive{A}{p}$ has a cycle length dividing $p$.
		Thus, the cycle length of the product of elements of $\setDive{A}{p}$ and $\setDive{B}{p}$ is bounded by $p$.
		Since the $\lcm$ of integers dividing $p$ is also a divisor of $p$, we have $\setDive{A}{p} \setDive{B}{p} \subseteq \setDive{AB}{p}$.
	\end{proof}
	
	%A directed consequences of this lemma which can be seen like the cloture in terms of the cycle size of $\setSize{X}{p}$ over the power \ie $\setDive{X^k}{p} = (\setDive{X}{p})^k$ for als positive integer $k$. 
	%Thus, in $X^k$, we can identify for each $X_i$ in $X$, the smallest connected component, by the order $\oct$, which comes from a product with $X_i$.
	
	
	
	%\begin{lemma}\label{lemme:structurePower}
	%	Let $k,n,m$ be an positive integers with $n \le m$ and  $X = X_1 + \ldots + X_m$ be a FDDS such that $X_i$ is a connected component for all $i$ and $X_{i+1} \oct X_i$. 
	%	Let $B$ be the smallest connected component over $\oct$ in $X^k - (\sum_{i=1}^{n-1} X_i)^k$ and $p = \cycle(B)$.
	%	Let $X_l$ be the smallest element of $\setDive{X}{p}$ over $\otc$.
	%	Then $B$ come from $X_l^{k-1} X_n$.
	%\end{lemma}
	%
	%\begin{proof}
	%	Let $q$ be the size of the cycle of $X_n$.
	%	First, we have that $B$ come from the product of $k$ element of $\setDive{X}{p}$.
	%	In addition, since $B$ is in $X^k - (\sum_{i=1}^{n-1} X_i)^k$, we deduce that at least one terms of this product is at least $X_n$.
	%	So $q \le p$.
	%	Besides, since $X_n^k$ is in $X^k - (\sum_{i=1}^{n-1} X_i)^k$ and that the size of each cycle of $X_n$ is $q$, by the minimality of $B$, we deduce that $p \le q$.
	%	Thus, we conclude that $q = p$ and also each connected component of $X_l^{k-1} X_n$ has a cycle of size $p$.
	%	
	%	Second, by the minimality of $B$, we have that $\min(\unroll{B}) = \min(\unroll{\setDive{X}{p}^k - (\setDive{\sum_{i=1}^{n-1} X_i}{p})^k})$.
	%	However, since $\unroll{\setDive{X}{p}^k - (\setDive{\sum_{i=1}^{n-1} X_i}{p})^k}$ is equals to $\unroll{\setDive{X}{p}^k} - \unroll{ (\setDive{\sum_{i=1}^{n-1} X_i}{p})^k}$, by a adaptation for infinite trees of \cite[Lemma 4]{kroot} and the minimality of $X_n$, we have that $\min(
	%	\unroll{\setDive{X}{p}^k} - \unroll{ (\setDive{\sum_{i=1}^{n-1} X_i}{p})^k})$ is equals of $\min(\unroll{X_l})^{k-1} \min(\unroll{X_n})$.
	%	Thus at least one connected component of $X_l^{k-1} X_n$ has this minimal unroll tree which equals to $\min(\unroll{B})$. 
	%	
	%	For resume, all connected components of $X_l^{k-1} X_n$ have their cycle of $p$ and at least one connected component of $X_l^{k-1} X_n$ has its minimal unroll tree equals to $\min(\unroll{B})$.
	%	This implies that $B$ is a connected component in $X_l^{k-1} X_n$.  
	%\end{proof}
	%
	%To this proof, we can show that $X_n \oct X_l$.
	%Indeed, since $\unroll{X_l}$ have the minimal unroll tree of $\unroll{\setDive{X}{p}}$, we have that $\min(\unroll{X_n}) \ge \min(\unroll{X_l})$.
	%And, since $X_l$ is an element of $\setDive{X}{p}$, the size of the cycle of $X_l$ is inferior or equals to $X_n$.
	%So, the property follows.
	%
	%In addition, we deduce an algorithm for solve the $k$-th root of a FDDS. 
	%Indeed, from the Lemma~\ref{lemme:structurePower}, the length of the cycle of $\min(X^k)$ over $\oct$ is the length of the cycle of $\min(X)$ over $\oct$. 
	%We set $min$ this length.
	%And $\unroll{\min(X)}$ over $\oct$ is equals to $\sqrt[k]{\min(\unroll{X}^k)}$. 
	%Thus, we can roll $\min(\unroll{\min(X)})$ to period $min$.
	%
	%After that, we can inductively construct the $i+1$-th smallest component of $X$ over $\oct$. 
	%In fact, from the Lemma~\ref{lemme:structurePower}, the length of the cycle of $\min(X^k - (X_1 + \ldots + X_i)^k)$ over $\oct$ is the length of the cycle of $\min(X - (X_1 + \ldots + X_i))$ over $\oct$.
	%We set $p$ this length.   
	%Two cases are possible. 
	%Either $\min(X^k - (X_1 + \ldots + X_i)^k)$ over $\oct$ is equals to $\min(X^{k}\{p\})$ over $\otc$ and $\unroll{X_{i+1}} = \sqrt[k]{\min(X^k - (X_1 + \ldots + X_i)^k)}$, or not and $\min(\unroll{X_{i+1}})$ is the quotient of the division of the minimal unroll tree of $\min(\unroll{\min(X^k - (X_1 + \ldots + X_i)^k)})$ over $\oct$ by the minimal unroll tree in $\unroll{\min(X^{k}\{p\})}^{k-1}$ over $\otc$. 
	%In this two cases, we can identify the minimal unroll tree of $\unroll{X_{i+1}}$. 
	%Thus, we can roll this tree to period $p$. 
	%
	%Moreover, by \cite{article_arbre, kroot}, all of these operation are in polynomial times. Thus, this prove the following proposition.
	%
	%\begin{proposition}
	%	Given $A$ an FDDS and $k$ an integer, we can find in polynomial times the FDDS $X$ such that $X^k = A$ if it exists. 
	%\end{proposition}
	%
	%At this point, we can moving to the studies of general polynomials whose at least one cancelable coefficient. 
	%As previously, our goal is to identify some connected component of $P(X)$ and the product of $P$ whose they are generated. 
	%However, this case is more complicated than $X^k$ because we have to find the coefficient and the connected component in this coefficient in addition to the part in $X$.  
	
	Our goal is to prove, by a simple induction, the core of the proof of Proposition~\ref{prop:condSufInj}. % , namely Lemma~\ref{lemme:condSufInjSpe}. 
	Lemma~\ref{lemme:closureProduct} is a first step to show the induction case.
	Indeed, it allows us to treat sequentially the different cycle lengths. We use the order~$\ioct$ to process sequentially connected components having the same cycle length, as shown by the following lemma.
	
	\begin{lemma}\label{lemme:structurePolyFDDS}
		Let $X = X_1 + \cdots + X_k$ be an FDDS with $k$ connected components sorted by $\ioct$ and $P = \sum_{i=1}^{m} A_i X^i$ a polynomial over FDDS without constant term and with at least one cancelable coefficient. 
		Then $B = \min_{\oct}(P(X) - P(\sum_{i=1}^{n-1} X_i))$ implies $\cycle(X_n) = \cycle(B)$ for all~$1 \le n \le k$.
	\end{lemma}
	
	\begin{proof}
	        We have $\cycle(X_n) \le \lcm(a, \cycle(X_{i}))$ for all integers $i$ with $n \le i \le k$ and $a > 0$. 
		It follows that $\cycle(X_n) \le \cycle(B)$. 
		In addition, since at least one coefficient of $P$, say $A_{c}$, is cancelable, it contains one connected component whose cycle length is one. 
		This implies that $A_{c} X_{n}^c$ contains a connected component $C$ such that $\cycle(C) = \cycle(X_n)$. 
		If $B$ is minimal, then $\cycle(X_n) \ge \cycle(B)$.
	\end{proof}
	
	We can now identify recursively and unambiguously each connected component in $X$. 
	Therefore, we can use Lemma~\ref{lemme:structurePolyFDDS}, as part of the induction step for the following lemma.
	
	\begin{lemma}\label{lemme:condSufInjSpe}
		Let $P = \sum_{i=1}^{m} A_i X^{i}$ be a polynomial without constant term.
		If at least one coefficient of $P$ is cancelable then $P$ is injective.
	\end{lemma}
	
	\begin{proof}
	        Let $X = X_1 + \cdots + X_{n_1}$, resp., $Y = Y_1 + \cdots + Y_{n_2}$ be FDDS consisting of~$n_1$ (resp., $n_2$) connected components sorted by~$\ioct$, and let~$B$ be an FDDS. Suppose~$P(X) = B = P(Y)$.
		We prove by induction on the number~$n_1$ of connected components in $X$ that $X = Y$.
		
		If $X$ has $0$ connected components, since $P$ does not have constant term, we have $P(X) = 0$.
		In addition, since $P$ has at least a cancelable coefficient, it trivially contains at least one nonzero coefficient.
		Thus, since $P(Y) = P(X) = 0$, we deduce that $Y = 0$, hence $X = Y$.
		The property is true in the base case.
		
		Let $n$ be a positive integer.
		Suppose that the properties is true for $n$, \ie $X_1 + \cdots + X_n = Y_1 + \cdots + Y_n$.
		Then, $P(X_1 + \cdots + X_n) = P( Y_1 + \cdots + Y_n)$.
		This implies that $P(X) - P(X_1 + \cdots + X_n) = B - P(X_1 + \cdots + X_n) = P(Y) -  P(Y_1 + \cdots + Y_n)$.
		Let $p$ be the length of the cycle of the smallest connected component of $B - P(X_1 + \cdots + X_n)$ according to $\oct$.
		By Lemma \ref{lemme:structurePolyFDDS}, 
		we have $\cycle(X_{n+1}) = \cycle(Y_{n+1}) = p$.
		
		Besides, thanks to Lemma \ref{lemme:closureProduct}, we deduce that $\setDive{P(X)}{p} = \sum_{i=1}^{m} \setDive{A_i}{p} \setDive{X^i}{p}$ and $\setDive{P(Y)}{p} = \sum_{i=1}^{m} \setDive{A_i}{p} \setDive{Y^i}{p}$.
		Since $\setDive{P(X)}{p} = \setDive{B}{p} = \setDive{P(Y)}{p}$, we deduce that $ \sum_{i=1}^{m} \setDive{A_i}{p} \setDive{X^i}{p} =  \sum_{i=1}^{m} \setDive{A_i}{p} \setDive{Y^i}{p}$.
		By taking unrolls, we obtain $\sum_{i=1}^{m} \unroll{\setDive{A_i}{p}} \unroll{\setDive{X^i}{p}} =  \sum_{i=1}^{m} \setDive{A_i}{p} \setDive{Y^i}{p}$.
		However, by injectivity of polynomials over unrolls (Theorem \ref{th:injPolyUnrolls}), we have $\unroll{\setDive{X}{p}} = \unroll{\setDive{Y}{p}}$.
		Thus, the smallest tree of $\unroll{\setDive{X}{p}} - \unroll{\setDive{X_1 + \cdots + X_n}{p}}$ is equal to the smallest of $\unroll{\setDive{Y}{p}} - \unroll{\setDive{Y_1 + \cdots + Y_n}{p}}$.
		However, the smallest tree of $\unroll{\setDive{X}{p}} - \unroll{\setDive{X_1 + \cdots + X_n}{p}}$ is the smallest tree of $\unroll{X_{n+1}}$ and of $\unroll{Y_{n+1}}$.
		This implies that $X_{n+1}$ and $Y_{n+1}$ are two connected components whose cycle length is $p$ and have the same minimal unroll tree, so $X_{n+1} = Y_{n+1}$.
		The induction step and the statement follow.
	\end{proof}
	
	What remains in order to conclude the proof of Proposition \ref{prop:condSufInj} is to extend Lemma \ref{lemme:condSufInjSpe} to polynomials with a constant term.
	
	\begin{proof}[Proof of Proposition \ref{prop:condSufInj}]
		Let $P = \sum_{i=0}^{m} A_i X^{i}$ be a polynomial with at least one non-constant cancelable coefficient.
		Let $P' = \sum_{i=1}^{m} A_i X^{i}$ be the polynomial obtained from $P$ by removing the constant term $A_0$.
		Hence $P = P' + A_0$.
		If there exist two FDDS $X,Y$ such that $P(X) = P(Y)$, then $P'(X) + A_0 = P'(Y) + A_0$.
		This implies that $P'(X) = P'(Y)$.
		Thus, by Lemma \ref{lemme:condSufInjSpe}, we conclude that $X = Y$. 
	\end{proof}
	
	From the proof of Lemma~\ref{lemme:structurePolyFDDS} and Theorem~\ref{theorem:equation_forest_P}, we deduce that we can solve in polynomial time all equations of the form~$P(X) = B$ with~$P$ having a cancelable non-constant coefficient. 
	Indeed, we can just select the connected components with the correct cycle lengths, and cut their unrolls to the correct depth. 
	Then, we solve the equation over the forest thus obtained, select the smallest tree of the result, and re-roll it to the correct period. 
	Finally, we remove the corresponding multiset of connected components and reiterate until either all connected components have been removed, or we find a connected component that we cannot remove (which implies that the equation has no solution). 
	
	%%% Local Variables:
	%%% mode: LaTeX
	%%% TeX-master: "main"
	%%% End:
