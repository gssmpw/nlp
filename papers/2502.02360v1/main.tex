\documentclass[runningheads,a4paper]{llncs}
% \usepackage[a4paper,showframe,textwidth=12.2cm,textheight=19.3cm]{geometry}
\usepackage[utf8]{inputenc}
\usepackage{microtype}
\usepackage{amsmath,amssymb,amsfonts}
\let\proof\relax
\let\endproof\relax
\usepackage{amsthm}
\usepackage{macro}

\usepackage[final]{pdfpages} 

\DeclareMathOperator{\lcm}{lcm}
\newcommand{\unroll}[1]{\mathcal{U}(#1)}
\newcommand{\tree}[1]{\mathbf{\lowercase{#1}}}
\newcommand{\forest}[1]{\mathbf{\uppercase{#1}}}
\newcommand{\depth}[2][]{\mathrm{depth}_{#1}(#2)}
\newcommand{\cycle}{\ell}
\newcommand{\cut}[2]{\mathcal{C}(#1,#2)}
\newcommand{\compCo}[2]{C_{\mathbf{#1},#2}}
\newcommand{\dt}[1]{\mathcal{D}(#1)}
\newcommand{\nd}{\mathcal{D}}
\newcommand{\rf}[1]{\mathcal{R}(#1)}
\newcommand{\nr}{\mathcal{R}}
\newcommand{\Ls}[1]{L_{#1}}
\newcommand{\bigo}[1]{\mathcal{O}(#1)}

\newcommand{\ie}{\emph{i.e.}\@\xspace}

\newcommand{\polyU}[4]{\sum_{i=#1}^{#2} \unroll{#3_i} \unroll{#4}^{i}}

\newcommand{\polyUC}[5]{\sum_{i=#1}^{#2} \cut{\unroll{#3_i}}{#4} \cut{\unroll{#5}}{#4}^{i}}

\newcommand{\polyUCR}[5]{\sum_{i=#1}^{#2} \rf{\cut{\unroll{#3_i}}{#4}} \rf{\cut{\unroll{#5}}{#4}}^{i}}

\newcommand{\oct}{\ge_{c}}
\newcommand{\ioct}{\le_{c}}
\newcommand{\octs}{<_{c}}
\newcommand{\otc}{\ge_{t}}
\newcommand{\iotc}{\le_{t}}
\newcommand{\setSize}[2]{(#1)(#2)}
\newcommand{\setDive}[2]{(#1)\{#2\}}

\usepackage[color=red!60,textsize=small]{todonotes}
\newcommand{\mr}[2][]{\todo[#1,color=violet!50]{MR: #2}}
\newcommand{\ep}[2][]{\todo[#1,color=vertb!65]{EP: #2}}
\newcommand{\kp}[2][]{\todo[#1,color=brique!70]{KP: #2}}

\title{Injectivity of polynomials over finite\\discrete dynamical systems}
\author{
	Antonio E. Porreca
	\and
	Marius Rolland
}

\institute{
  Aix-Marseille Université, CNRS, LIS, Marseille, France\\
  \email{marius.rolland@lis-lab.fr}
}
\authorrunning{A.E. Porreca \and M. Rolland}

\date{\today}



\begin{document}
	\maketitle
	\begin{abstract}
        The analysis of observable phenomena (for instance, in biology or physics) allows the detection of dynamical behaviors and, conversely, starting from a desired behavior allows the design of objects exhibiting that behavior in engineering. The decomposition of dynamics into simpler subsystems allows us to simplify this analysis (or design). Here we focus on an algebraic approach to decomposition, based on alternative and synchronous execution as the sum and product operations; this gives rise to polynomial equations (with a constant side). In this article we focus on univariate, injective polynomials, giving a characterization in terms of the form of their coefficients and a polynomial-time algorithm for solving the associated equations.
	\end{abstract}
	
	\section{Introduction}
\label{sec:introduction}
The business processes of organizations are experiencing ever-increasing complexity due to the large amount of data, high number of users, and high-tech devices involved \cite{martin2021pmopportunitieschallenges, beerepoot2023biggestbpmproblems}. This complexity may cause business processes to deviate from normal control flow due to unforeseen and disruptive anomalies \cite{adams2023proceddsriftdetection}. These control-flow anomalies manifest as unknown, skipped, and wrongly-ordered activities in the traces of event logs monitored from the execution of business processes \cite{ko2023adsystematicreview}. For the sake of clarity, let us consider an illustrative example of such anomalies. Figure \ref{FP_ANOMALIES} shows a so-called event log footprint, which captures the control flow relations of four activities of a hypothetical event log. In particular, this footprint captures the control-flow relations between activities \texttt{a}, \texttt{b}, \texttt{c} and \texttt{d}. These are the causal ($\rightarrow$) relation, concurrent ($\parallel$) relation, and other ($\#$) relations such as exclusivity or non-local dependency \cite{aalst2022pmhandbook}. In addition, on the right are six traces, of which five exhibit skipped, wrongly-ordered and unknown control-flow anomalies. For example, $\langle$\texttt{a b d}$\rangle$ has a skipped activity, which is \texttt{c}. Because of this skipped activity, the control-flow relation \texttt{b}$\,\#\,$\texttt{d} is violated, since \texttt{d} directly follows \texttt{b} in the anomalous trace.
\begin{figure}[!t]
\centering
\includegraphics[width=0.9\columnwidth]{images/FP_ANOMALIES.png}
\caption{An example event log footprint with six traces, of which five exhibit control-flow anomalies.}
\label{FP_ANOMALIES}
\end{figure}

\subsection{Control-flow anomaly detection}
Control-flow anomaly detection techniques aim to characterize the normal control flow from event logs and verify whether these deviations occur in new event logs \cite{ko2023adsystematicreview}. To develop control-flow anomaly detection techniques, \revision{process mining} has seen widespread adoption owing to process discovery and \revision{conformance checking}. On the one hand, process discovery is a set of algorithms that encode control-flow relations as a set of model elements and constraints according to a given modeling formalism \cite{aalst2022pmhandbook}; hereafter, we refer to the Petri net, a widespread modeling formalism. On the other hand, \revision{conformance checking} is an explainable set of algorithms that allows linking any deviations with the reference Petri net and providing the fitness measure, namely a measure of how much the Petri net fits the new event log \cite{aalst2022pmhandbook}. Many control-flow anomaly detection techniques based on \revision{conformance checking} (hereafter, \revision{conformance checking}-based techniques) use the fitness measure to determine whether an event log is anomalous \cite{bezerra2009pmad, bezerra2013adlogspais, myers2018icsadpm, pecchia2020applicationfailuresanalysispm}. 

The scientific literature also includes many \revision{conformance checking}-independent techniques for control-flow anomaly detection that combine specific types of trace encodings with machine/deep learning \cite{ko2023adsystematicreview, tavares2023pmtraceencoding}. Whereas these techniques are very effective, their explainability is challenging due to both the type of trace encoding employed and the machine/deep learning model used \cite{rawal2022trustworthyaiadvances,li2023explainablead}. Hence, in the following, we focus on the shortcomings of \revision{conformance checking}-based techniques to investigate whether it is possible to support the development of competitive control-flow anomaly detection techniques while maintaining the explainable nature of \revision{conformance checking}.
\begin{figure}[!t]
\centering
\includegraphics[width=\columnwidth]{images/HIGH_LEVEL_VIEW.png}
\caption{A high-level view of the proposed framework for combining \revision{process mining}-based feature extraction with dimensionality reduction for control-flow anomaly detection.}
\label{HIGH_LEVEL_VIEW}
\end{figure}

\subsection{Shortcomings of \revision{conformance checking}-based techniques}
Unfortunately, the detection effectiveness of \revision{conformance checking}-based techniques is affected by noisy data and low-quality Petri nets, which may be due to human errors in the modeling process or representational bias of process discovery algorithms \cite{bezerra2013adlogspais, pecchia2020applicationfailuresanalysispm, aalst2016pm}. Specifically, on the one hand, noisy data may introduce infrequent and deceptive control-flow relations that may result in inconsistent fitness measures, whereas, on the other hand, checking event logs against a low-quality Petri net could lead to an unreliable distribution of fitness measures. Nonetheless, such Petri nets can still be used as references to obtain insightful information for \revision{process mining}-based feature extraction, supporting the development of competitive and explainable \revision{conformance checking}-based techniques for control-flow anomaly detection despite the problems above. For example, a few works outline that token-based \revision{conformance checking} can be used for \revision{process mining}-based feature extraction to build tabular data and develop effective \revision{conformance checking}-based techniques for control-flow anomaly detection \cite{singh2022lapmsh, debenedictis2023dtadiiot}. However, to the best of our knowledge, the scientific literature lacks a structured proposal for \revision{process mining}-based feature extraction using the state-of-the-art \revision{conformance checking} variant, namely alignment-based \revision{conformance checking}.

\subsection{Contributions}
We propose a novel \revision{process mining}-based feature extraction approach with alignment-based \revision{conformance checking}. This variant aligns the deviating control flow with a reference Petri net; the resulting alignment can be inspected to extract additional statistics such as the number of times a given activity caused mismatches \cite{aalst2022pmhandbook}. We integrate this approach into a flexible and explainable framework for developing techniques for control-flow anomaly detection. The framework combines \revision{process mining}-based feature extraction and dimensionality reduction to handle high-dimensional feature sets, achieve detection effectiveness, and support explainability. Notably, in addition to our proposed \revision{process mining}-based feature extraction approach, the framework allows employing other approaches, enabling a fair comparison of multiple \revision{conformance checking}-based and \revision{conformance checking}-independent techniques for control-flow anomaly detection. Figure \ref{HIGH_LEVEL_VIEW} shows a high-level view of the framework. Business processes are monitored, and event logs obtained from the database of information systems. Subsequently, \revision{process mining}-based feature extraction is applied to these event logs and tabular data input to dimensionality reduction to identify control-flow anomalies. We apply several \revision{conformance checking}-based and \revision{conformance checking}-independent framework techniques to publicly available datasets, simulated data of a case study from railways, and real-world data of a case study from healthcare. We show that the framework techniques implementing our approach outperform the baseline \revision{conformance checking}-based techniques while maintaining the explainable nature of \revision{conformance checking}.

In summary, the contributions of this paper are as follows.
\begin{itemize}
    \item{
        A novel \revision{process mining}-based feature extraction approach to support the development of competitive and explainable \revision{conformance checking}-based techniques for control-flow anomaly detection.
    }
    \item{
        A flexible and explainable framework for developing techniques for control-flow anomaly detection using \revision{process mining}-based feature extraction and dimensionality reduction.
    }
    \item{
        Application to synthetic and real-world datasets of several \revision{conformance checking}-based and \revision{conformance checking}-independent framework techniques, evaluating their detection effectiveness and explainability.
    }
\end{itemize}

The rest of the paper is organized as follows.
\begin{itemize}
    \item Section \ref{sec:related_work} reviews the existing techniques for control-flow anomaly detection, categorizing them into \revision{conformance checking}-based and \revision{conformance checking}-independent techniques.
    \item Section \ref{sec:abccfe} provides the preliminaries of \revision{process mining} to establish the notation used throughout the paper, and delves into the details of the proposed \revision{process mining}-based feature extraction approach with alignment-based \revision{conformance checking}.
    \item Section \ref{sec:framework} describes the framework for developing \revision{conformance checking}-based and \revision{conformance checking}-independent techniques for control-flow anomaly detection that combine \revision{process mining}-based feature extraction and dimensionality reduction.
    \item Section \ref{sec:evaluation} presents the experiments conducted with multiple framework and baseline techniques using data from publicly available datasets and case studies.
    \item Section \ref{sec:conclusions} draws the conclusions and presents future work.
\end{itemize}

	\section{Preliminaries}
\label{sec:preliminaries}

In this paper we compose and decompose FDDS in terms of two algebraic operations.
Thee \emph{sum} of two FDDS 
consists in the mutually
exclusive alternative between their behaviors, while the \emph{product} is their synchronous parallel execution.
The set of FDDS taken up to isomorphism, with these two operations, forms a semiring~\cite{article_fondateur}.
This semiring is isomorphic to the semiring of functional digraphs with the operations of disjoint union and direct product.
We recall~\cite{livreGraphe} that the direct product of two graphs $A$ and $B$ is the graph $C$
having vertices $V(C) = V(A) \times V(B)$ and edges
\[
E(C) = \{((u,u'), (v,v')) \mid (u,v) \in E(A), (u',v') \in E(B)\}.
\]
Within the class of \emph{connected} FDDS, we further distinguish those having a (necessarily  unique) cycle of size $1$ (\ie, a fixed point in terms of dynamics) and refer to them as \emph{dendrons}. 

In this semiring, the structure of the product is particularly rich. 
For example, the semiring is not factorial, \ie, there exist irreducible FDDS $A,B,C,D$ such that $A \neq C$ and $A \neq D$ but $AB = CD$.
This richness might originate from the interactions between the periodic and the transient behaviors of the two FDDS being multiplied.

% For brevity, we use the term \emph{trees} in reference to in-trees constituting the transient behaviors of FDDS.

For the periodic behavior of FDDS, the product of a connected $A$, with a cycle of size $p$,
and a connected $B$, with a cycle of size $q$,
generates $\gcd(p,q)$ connected components, each with a cycle of size $\lcm(p,q)$~\cite{livreGraphe}.
For the analysis of transient behaviors, we use the notion of unroll introduced in~\cite{article_arbre}.

\begin{definition}[Unroll]
	Let $A = (X,f)$ be an FDDS.
	For each state $u\in X$ and $k \in \mathbb{N}$, we denote by $f^{-k}(u) = \{ v \in X \mid f^k(v) = u \}$ the set of $k$-th preimages of $u$. 
	For each $u$ in a cycle of $A$, we call the \emph{unroll tree of $A$ in $u$} the infinite tree $\tree{t}_u = (V,E)$ having vertices $V = \{(s,k) \mid s \in f^{-k}(u), k \in \mathbb{N}\}$ and edges
	$E = \big\{ \big((v,k),(f(v),k-1) \big) \big\} \subseteq V^2$.
	We call \emph{unroll of $A$}, denoted $\unroll{A}$, the set of its unroll trees (see Fig.~\ref{fig:unroll}).
\end{definition}

\begin{figure}[t]
\centering
\includegraphics[page=1]{pictures}
\caption{An FDDS~$A$ with two connected components (on the left), a finite portion of its unroll~$\unroll{A}$ (on the right) and the cut~$\cut{\unroll{A}}{4}$ (below the dashed line). A few vertex names are shown in order to highlight their contribution to the unroll.}
\label{fig:unroll}
\end{figure}

In the rest of this paper, \emph{unrolls will always be taken up to isomorphism}, \ie, as multisets or sums (disjoint unions) of unlabeled trees; the equality sign will thus represent the isomorphism relation.

An unroll tree contains exactly one infinite branch, onto which are periodically rooted the trees representing the transient behaviors of the corresponding connected component. 
% In order to employ unrolls to study the transient behaviors with respect to the product,
We also exploit a notion of ``levelwise'' product on trees. 
For readability, we will denote trees and forests using bold letters (in lower and upper case
respectively) to distinguish them from FDDS.

\begin{definition}[Product of trees]\label{prodintrees} 
	Let $\tree{t}_1=(V_1,E_1)$ and $\tree{t}_2=(V_2,E_2)$ be two trees with roots $r_1$ and $r_2$, respectively.
        Their \emph{product} is the tree $\tree{t}_1 \times \tree{t}_2=(V,E)$ with vertices
        $V=\set{(u,v)\in V_1\times V_2 \mid \depth{u}=\depth{v}}$, where $\depth{u}$ is the length of the shortest path between $u$ and the root of its tree, and edges $E=\set{((u,u'),(v,v')) \mid (u,v)\in E_1, (u',v')\in E_2}\subseteq V^2$ (see Fig.~\ref{fig:tree-product}).
\end{definition}

\begin{figure}[t]
\centering
\includegraphics[page=2]{pictures}
\caption{The levelwise product of two finite trees. Remark how the depth of the result is given by the minimum depth of the two factors.}
\label{fig:tree-product}
\end{figure}

We deduce from~\cite{article_arbre} that the set of unrolls up to isomorphism, with disjoint union for addition and product of trees for multiplication, is a semiring with the infinite path as the multiplicative identity.
In addition, the unroll operation is a homomorphism between the semiring~$\mathbb{D}$ of FDDS and the semiring of unrolls.
% Here and in the following, the equality sign will denote graph isomorphism.

Unrolls have one major algorithmic issue: they are infinite. 
To overcome this obstacle, we will consider only a finite portion of the unrolls. 
For this purpose, % we first define the \emph{depth} of a finite tree as the maximum length of a shortest path between a leaf and its root.
we extend the notion of \emph{depth} to forests of finite trees by taking the maximum depth of its trees. 
We also define the \emph{depth of an FDDS} as the maximum depth among the trees rooted in one of its periodic states.

We remark that the set of forests of bounded depth~$d$ up to isomorphism is a also semiring (with the restrictions of the same tree sum and product operations) for all depth~$d$, where the multiplicative identity is the path of depth~$d$.

We can now define the \emph{cut} of $\tree{t}$ at depth $k \ge 0$, denoted $\cut{\tree{t}}{k}$,
as the induced sub-tree of $\tree{t}$ restricted to the vertices of depth lesser or equal to $k$ (see Fig.~\ref{fig:unroll}). 
This operation generalizes (homomorphically) to forests $\forest{f} = \tree{t}_{1} + \ldots + \tree{t}_{n}$ as  $\cut{\forest{F}}{k} = \cut{\tree{t}_{1}}{k} + \ldots + \cut{\tree{t}_{n}}{k}$. 
The cut of an unroll at a certain depth has already been used to exhibit properties of unrolls~\cite{article_arbre,kroot}.
This approach notably gave a characterization of \emph{cancelable} FDDS, \ie FDDS $A$ such that $AB = AC$ implies $B = C$ for all FDDS $B,C$.
Indeed, a FDDS $A$ is cancelable if and only if at least one of its connected components is a dendron~\cite[Theorem 30]{article_arbre}. 

One of the tools for the analysis of trees and forests is the total order $\le$ on finite and infinite trees introduced in~\cite{article_arbre}.
Indeed, this order is compatible with the product for infinite trees\footnote{This compatibility with the product of~$\le$ is sufficient for most applications in this paper; we refer the reader to the original paper for the actual definition.}, \ie, if $\tree{t}_1, \tree{t}_2$ are two infinite trees then $\tree{t}_1 \le \tree{t}_2$ if and only if $\tree{t}_1 \tree{t} \le \tree{t}_2 \tree{t}$ for all tree $\tree{t}$~\cite[Lemma 24]{article_arbre}. 
A similar property has been proved for the case of finite trees:

\begin{lemma}[\cite{article_arbre}]\label{lemma:ordre_comp_produit}
	Let $\tree{t}_1, \tree{t}_2, \tree{t}$ then $\cut{\tree{t}_1}{\depth{\tree{t}}} \le \cut{\tree{t}_2}{\depth{\tree{t}}}$ if and only if $\tree{t}_1 \tree{t} \le \tree{t}_2 \tree{t}$.
\end{lemma}

In the rest of the paper all polynomials will be implicitly univariate and, by convention, the symbol~$\forest{a}^0$ (i.e., the $0$-th power of a forest of finite trees~$\forest{a}$) will denote the path of length~$\depth{\forest{a}}$, which behaves as an identity for the product with forests of equal or less depth.


%%% Local Variables:
%%% mode: LaTeX
%%% TeX-master: "main"
%%% End:

	
	\section{Characterization of injective polynomials over unrolls}\label{section:poly_unroll}
	
	The search for a solution to polynomial equations over FDDS can be seen as the search for a compatible solution between solutions of polynomial equations of transient states and solutions of polynomial equations of periodic states.
	This is why we can start by looking for solutions for transient states.
	In order to do this, we will use the notion of unroll and determine the number of solutions to polynomial equations over unrolls.
	
	More precisely, our final goal is to prove the following theorem:
	\begin{theorem}\label{th:injPolyUnrolls}
		All univariate polynomial over unrolls are injective. 
	\end{theorem}
	
	For the proof of this theorem, we begin by remarking that for all polynomial~$P$ over unrolls, and for all FDDS $X,Y$ the equality $P(\unroll{X}) = P(\unroll{Y})$ implies $\cut{P(\unroll{X})}{n} = \cut{P(\unroll{Y})}{n}$ for all $n$. In order to study polynomial equations over forests of finite trees, we start by proving one technical lemma on the behavior of the order on powers of trees with the same depth.
	
		
		\begin{lemma}\label{lemma:behaviorOrderWithProduct}
			Let $\tree{x}, \tree{y}$ be two finite trees with the same depth and $k$ be a  positive integer. Then $\tree{y}^k \ge \tree{x}^k$ if and only if $\tree{y} \ge \tree{x}$.
		\end{lemma}
	
		\begin{proof}
			If $\tree{y} \ge \tree{x}$ and $\depth{\tree{x}} = \depth{\tree{y}}$, by \cite[Corollary 20]{article_arbre}, we have $\tree{y}^2 \ge \tree{x}^2$ and $\depth{\tree{y}^2} = \depth{\tree{x}^2}$. 
			Then, by induction, we obtain $\tree{y}^k \ge \tree{x}^k$.
			
			For the other direction, assume that $\tree{y}^k \ge \tree{x}^k$ and, by contradiction, that $\tree{x} > \tree{y}$. 
			Since $\depth{\tree{x}} = \depth{\tree{y}}$, by \cite[Corollary 20]{article_arbre}, we have $\tree{y}^k \tree{x} \ge \tree{y}^k \tree{y}$. 
			Thus, by \cite[Lemma 21]{article_arbre}, we have $\tree{y}^{k-1} \ge \tree{x}^{k-1}$. 
			By induction, we obtain $\tree{y} \ge \tree{x}$, a contradiction.
	\end{proof}
		
		Let $\forest{A}$ be a forest and $i\ge0$ be an integer. 
		We denote by % $\gamma(\forest{A},i)$ the sub-multiset of trees of $\forest{A}$ having depth \emph{exactly} $i$; we also
	        % denote by
                $\Gamma(\forest{A},i)$ the sub-multiset of trees of $\forest{A}$ where each tree has depth \emph{at least} $i$.
		We focus our attention on the trees in $P(\forest{X})$ with a
		certain depth and form.
		
		\begin{lemma}\label{lemme:ordreProdFin}
	
			Let $P = \polyF{1}{m}{a}{x}$ be a polynomial without constant term over forests and $\forest{X}= \tree{x}_1 + \ldots + \tree{x}_n$ be a forest sorted in nondecreasing order.
			Let $\forest{A}_i = \sum_{j = 1}^{m_i} \tree{a}_{i,j}$ be in nondecreasing order, hence $\tree{a}_{i,1} = \min(\forest{a}_i)$, for all~$i$.
			Then, there exists $\alpha \in \{1, \ldots, m\}$ such that $\min_{i}(\forest{A}_i \forest{X}^{i-1} \tree{x}_k)$ is isomorphic to $\tree{a}_{\alpha,1} \tree{x}_1^{\alpha-1}\tree{x}_k$.
		\end{lemma}
		
		\begin{proof}
			Let $\tree{t}_{min}$ be the smallest tree in $P(\forest{X})$. 
			Hence, there exists $\alpha \in \{1, \ldots, m\}$ such that $\tree{t}_{min} \in \forest{A}_\alpha \forest{X}^\alpha$. 
			In addition, by~\cite[Lemma 24]{article_arbre}, we have $\tree{x}_1^2 \le \tree{x}_1 \tree{x}_j$ for all $j$. 
			By induction, we deduce that $\tree{t}_{min} \in \forest{A}_\alpha \tree{x}_1^\alpha$. 
			Analogously, we have $\tree{a}_{\alpha,1} \tree{x}_1^\alpha \le \tree{a}_{\alpha,j} \tree{x}_1^\alpha$ for all $\tree{a}_{\alpha,j} \in \forest{A}_\alpha$, and we conclude that $\tree{t}_{min} = \tree{a}_{\alpha,1} \tree{x}_1^\alpha$.
			
			We recall that $\tree{t}_1 \tree{t}_3 < \tree{t}_2 \tree{t}_3$ if and only if $\cut{\tree{t}_1}{\depth{\tree{t}_3}} < \cut{\tree{t}_2}{\depth{\tree{t}_3}}$ (Lemma~\ref{lemma:ordre_comp_produit}). 
			Therefore, for all $j$ and $\tree{a} \in \forest{A}_j$, we have:
			\[\tree{a}_{\alpha,1} \tree{x}_1^\alpha \le \tree{a} \tree{x}_1^j \Leftrightarrow \cut{\tree{a}_{\alpha,1} \tree{x}_1^{\alpha-1}}{\depth{\tree{x}_1}} \le \cut{\tree{a} \tree{x}_1^{j-1}}{\depth{\tree{x}_1}}\]
			Three cases are possible.
			First, if  $\alpha = j$ then $\tree{a}_{\alpha,1} \le  \tree{a}$ by minimality of $\tree{a}_{\alpha,1}$.
			This implies that $\tree{a}_{\alpha,1} \tree{x}_1^{\alpha-1} \tree{x}_k \le \tree{a} \tree{x}_1^{\alpha-1} \tree{x}_k$.
	
			Second, if~$\alpha \ne j$ and $\alpha \neq 1$, then $\cut{\tree{a}_{\alpha,1} \tree{x}_1^{\alpha-1}}{\depth{\tree{x}_1}} = \tree{a}_{\alpha,1} \tree{x}_1^{\alpha-1}$. 
			Since $\cut{\tree{a} \tree{x}_1^{j-1}}{\depth{\tree{x}_1}} \le\tree{a} \tree{x}_1^{j-1}$, we have $\tree{a}_{\alpha,1} \tree{x}_1^{\alpha-1} \tree{x}_k \le \tree{a} \tree{x}_1^{j-1} \tree{x}_k$. 
			
			Thirdly, if $\alpha \ne j$ but $\alpha = 1$, then either
                        \[
                        \cut{\tree{a}_{\alpha,1}\tree{x}_1^{\alpha-1}}{\depth{\tree{x}_1}} = \cut{\tree{a} \tree{x}_1^{j-1}}{\depth{\tree{x}_1}} = \tree{a} \tree{x}_1^{j-1}
                        \]
                        which implies $\tree{a}_{\alpha,1}\tree{x}_1^{\alpha} = \tree{a}_{j,1}\tree{x}_1^{j}$, and we can just choose $\alpha=j$, which is already covered by the second case. 
			Otherwise $\cut{\tree{a}_{\alpha,1}\tree{x}_1^{\alpha-1}}{\depth{\tree{x}_1}} < \tree{a} \tree{x}_1^{j-1}$. 
			In this case, we need to recall the definition of the order from~\cite{article_arbre}~and the code upon which it is defined. 
			Indeed, by~\cite[Lemma 17]{article_arbre}, we know that the leftmost difference between the codes of $\cut{\tree{a}_{\alpha,1}\tree{x}_1^{\alpha-1}}{\depth{\tree{x}_1}}$ and $\cut{\tree{a} \tree{x}_1^{j-1}}{\depth{\tree{x}_1}}$ corresponds to a node of depth at most $\depth{\tree{x}_1} - 1$. 
			Therefore, this difference is propagated into $\cut{\tree{a}_{\alpha,1}\tree{x}_1^{\alpha-1}}{\depth{\tree{x}_1}} \tree{x}_k$ and $\tree{a} \tree{x}_1^{j-1} \tree{x}_k$ if $\depth{\tree{x}_k} \ge \depth{\tree{x}_1}-1$, and $\tree{a}_{\alpha,1} \tree{x}_1^{\alpha-1} \tree{x}_k < \tree{a} \tree{x}_1^{j-1} \tree{x}_k$; if $\depth{\tree{x}_k} < \depth{\tree{x}_1}-1$ then $\tree{a}_{\alpha,1} \tree{x}_1^{\alpha-1} \tree{x}_k = \tree{a} \tree{x}_1^{j-1} \tree{x}_k$.
			
			Hence, in all cases we have $\tree{a}_{\alpha,1} \tree{x}_1^{\alpha-1} \tree{x}_k \le \tree{a} \tree{x}_1^{j-1} \tree{x}_k$.
			Since for all $p$ we have $\tree{x}_1^{p-1} \le \tree{t}$ with $\tree{t} \in \forest{X}^{p-1}$, the lemma follows.
		\end{proof}
		
		Thanks to Lemma~\ref{lemme:ordreProdFin} we can prove the first important result of this section. Remark that, due to closure under sums and products, the set of forests of finite trees of depth at most~$d_{max}$ is a semiring for every natural number~$d_{max}$.
		
		\begin{proposition}\label{prop:injDesFinis}
	        Let~$P = \sum_{i=0}^{m} \forest{A}_i \forest{X}^i$ be a polynomial with~$d_{max} = \max_{i > 0}(\depth{\forest{A}_i)}$ and~$\depth{\forest{A}_0} \le d_{max}$. Then~$P$ is injective over the semiring of forests of finite trees of depth at most~$d_{max}$. 
	        
	        % Let~$\forest{A}_0$ and~$\forest{A}_1, \ldots, \forest{A}_m$ be forests of finite trees with~$d_{max} = \max_{i > 0}(\depth{\forest{A}_i)}$ and~$\depth{\forest{A}_0} \le d_{max}$. Consider the polynomial~$P = \sum_{i=0}^{m} \forest{A}_i \forest{X}^i$ over the semiring of forests of finite trees of depth at most~$d_{max}$. Then~$P$ is injective.
			% Let $d_{max}$ be a natural number and $P = \sum_{i=1}^{m} \forest{A}_i \forest{X}^i$ a polynomial over forests without constant terms with depth at most $d_{max}$\mr{formulation?} such that $d_{max} = \max(\depth{\forest{A}_i})$.
			% Then $P$ is injective.
	%		Then, for all $0\le d \le d_{max}$ then $\sum_{i=1}^{m} \Gamma(\forest{A}_i,d) \Gamma(\forest{X}^i, d)$ is injective.
		\end{proposition}
		
		
		\begin{proof}
	        Let us first consider polynomials~$P$ without a constant term.
			Let $\forest{X} = \tree{x}_1 + \ldots + \tree{x}_{m_x}$ and $\forest{Y} = \tree{y}_1 + \ldots + \tree{y}_{m_y}$ be two forest (sorted in nondecreasing order), and assume $P(\forest{X}) = P(\forest{Y})$. 
			Thus, by Lemma~\ref{lemme:ordreProdFin}, there exist $i, j \in \{1, \ldots, m\}$ such that, for each $k$, the smallest tree $\tree{t}_s$ with factor $\tree{x}_k$ (resp., $\tree{y}_k$) in $\polyF{1}{m}{a}{x}$ is isomorphic to $\tree{a}_{i,1} \tree{x}_1^{i-1}\tree{x}_k$ (resp., $\tree{a}_{j,1} \tree{y}_1^{j-1}\tree{y}_k$), with $\tree{a}_{i,1} = \min(\forest{A}_i)$ and $\tree{a}_{j,1} = \min(\forest{A}_j)$. 
			From the hypothesis, we deduce that $\tree{a}_{i,1} \tree{x}_1 = \min(P(\forest{x})) =  \min(P(\forest{y})) = \tree{a}_{j,1} \tree{y}_1$. 
			
			First, we show that $\tree{x}_1 = \tree{y}_1$.
			By contradiction, and without loss of generality, assume $\tree{x}_1 < \tree{y}_1$.
			Then, by Lemma~\ref{lemma:behaviorOrderWithProduct}, it follows that $\tree{x}_1^{j} < \tree{y}_1^{j}$.
			By \cite[Lemma 24]{article_arbre}, we deduce that $\tree{a}_{j,1} \tree{x}_1^{j} < \tree{a}_{j,1} \tree{y}_1^j = \tree{a}_{i,1} \tree{x}_1^i$.
			Nevertheless, $\tree{a}_{j,1} \tree{x}_1^{j} \in P(\forest{X})$,
			contradicting the minimality of $\tree{a}_{i,1} \tree{x}_1^i$. 
			
			Finally, since $\tree{x}_1 = \tree{y}_1$, we deduce that $P(\forest{X}) - P(\tree{x}_1) = P(\forest{Y}) - P(\tree{y}_1)$.  
			In addition, the smallest tree in $P(\forest{X}) - P(\tree{x}_1)$ is $\tree{a}_{i,1} \tree{x}_1^{i-1} \tree{x}_2$ and the smallest tree in $P(\forest{y}) - P(\tree{y}_1)$ is $\tree{a}_{j,1} \tree{y}_1^{j-1} \tree{y}_2$.  
			However, since $\tree{a}_{i,1} \tree{x}_1^i = \tree{a}_{j,1} \tree{y}_1^j$ and $\tree{x}_1 = \tree{y}_1$, we have $\tree{a}_{i,1} \tree{x}_1^{i-1} = \tree{a}_{j,1} \tree{y}_1^{j-1}$. 
			Thus, $\tree{x}_2 = \tree{y}_2$. 
			By induction, we conclude that $\forest{X} = \forest{Y}$.
	
	        This proof extends to polynomials with a constant term~$\forest{A}_0$. Indeed, let~$Q = P + \forest{A}_0$ and suppose~$Q(\forest{X}) = Q(\forest{Y})$. Then $Q(\forest{X}) - \forest{A}_0 = Q(\forest{Y}) - \forest{A}_0$, that is~$P(\forest{X}) = P(\forest{Y})$, which implies~$\forest{X} = \forest{Y}$. Thus, $Q$ is also injective.
		\end{proof}
	
	        Our main theorem follows directly from Proposition~\ref{prop:injDesFinis}.
	
	   \begin{proof}[Proof of Theorem~\ref{th:injPolyUnrolls}]
	        Suppose~$P(\unroll{X}) = P(\unroll{Y})$. Then~$\cut{P(\unroll{X})}{n} = \cut{P(\unroll{Y})}{n}$ for all~$n$. If~$\unroll{X}$ and~$\unroll{Y}$ were different, there would exist an~$n$ such that~$\cut{\unroll{X}}{n} \ne \cut{\unroll{Y}}{n}$. However, by Proposition~\ref{prop:injDesFinis} and since~$\cut{\cdot}{n}$ is a homomorphism, we have~$\cut{\unroll{X}}{n} = \cut{\unroll{Y}}{n}$ for all~$n$.
	    \end{proof}
	        
		% From this theorem, we can draw two remarks. Indeed, by Proposition~\ref{prop:polyUnroll2PolyForestFini}, we have that the polynomial $\sum_{i=1}^{m} \unroll{A_i} \unroll{X}^i$ is injective if and only if $\sum_{i=1}^{m} \cut{\unroll{A_i}}{n} \cut{\forest{y}}{n}^i$ is injective for a certain positive integer $n$.
		% And, for this kind of polynomial, each forest has depth at most $n$.  
		% Then, by Proposition~\ref{prop:injDesFinis}, the resulting function is injective.
	        % and the proof of Theorem \ref{th:injPolyUnrolls} follows.
		
		The proof of Proposition~\ref{prop:injDesFinis} also suggests an algorithm (inspired by \cite[Algorithm~1]{kroot}) for solving polynomial equations over forests of finite trees. 
		Indeed, starting with the maximum depth $d_{max}$ (Fig.~\ref{fig:algo} \textcircled{1}), we can solve the equation $\Gamma(P(\forest{x}), d_{max}) = \Gamma(\forest{b}, d_{max})$ \textcircled{2}.
		In fact, we construct the solution $\Gamma(\forest{X},d_{max})$ by first solving the equation $\min(\Gamma(\forest{A}_i,d_{max})) \tree{x}_1^i = \min(\Gamma(\forest{b}, d_{max}))$ \textcircled{3}\textcircled{4} for an~$i \in \{1, \ldots, m\}$. 
		The tree~$\tree{x}_1$ is minimal among the trees of maximum depth appearing in the solution~$\forest{X}$. 
		Thus, by Lemma~\ref{lemme:ordreProdFin}, either we can construct the solution $Sol$ inductively \textcircled{4} or $P(Sol) \nsubseteq \forest{B}$ \textcircled{3}, and we can deduce that this~$i$ is not the one appearing in the statement of Lemma~\ref{lemme:ordreProdFin}, and we try the next value of~$i$, if any.
		Indeed, the $j$-th tree~$\tree{x}_i$ of the solution is equal to
	        \[
	        \cfrac{\min(\Gamma(\forest{b}, d_{max}) - \Gamma(P(\tree{x}_1 + \ldots + \tree{x}_{j-1}), d_{max}))}{\min(\Gamma(\forest{A}_i,d_{max})) \tree{x}_1^{i-1}}.
	        \]
     	Then, we can inductively construct $\tree{x} = \Gamma(\forest{X}, d)$ from $\Gamma(\forest{x}, d')$ where $d < d_{max}$ is the depth of a tree in $\forest{B}$ and $d'$ the smallest depth of a tree in $\forest{B}$ greater than~$d$~\textcircled{5}. In order to do so, we need to find~$\min(\Gamma(\forest{X}, d))$. Let $\tree{a} = \min(\Gamma(\forest{A}_j,d))$ for a~$j \in \{1, \ldots, m\}$ and~$\tree{b} = \min(\Gamma(\forest{B},d))$.
     	\begin{itemize}
	        \item If $\tree{b}$ has depth at least~$d'$ \textcircled{6}, or $\tree{a}$ has depth~$d$ and~$\tree{a} \times \min(\Gamma(\forest{X}, d'))^j \le \tree{b}$, then $\tree{x}$ has depth at least~$d'$ and thus is already in $\Gamma(\forest{x}, d')$.
	        \item Otherwise~$\tree{b}$ necessarily has depth~$d$. In that case, either~$\tree{a}$ has depth at least~$d'$, or~$\tree{a} \times \min(\Gamma(\forest{X}, d'))^j > \tree{b}$, and in both cases we have~$\tree{x} = \sqrt[j]{\tree{b} / \tree{a}}$.
     	\end{itemize}
		Then, similarly to the step for depth~$d_{max}$, we can construct a portion of the solution~$\forest{X}$ \textcircled{8}, or find a mistake \textcircled{7} and try another value of~$j$, if any. If during the iteration we find mistakes for all coefficients, this implies that the equation does not have a solution.
		
		Moreover, the number of iterations is bounded by the number of coefficients~$m$ times the depth~$d_{max}$, which are both polynomial with respect to the size of the inputs, and the operations carried out in each iteration can be performed in polynomial time \cite{article_arbre,kroot}.
	
	        We cannot directly apply this algorithm to polynomial equations over unrolls, since they contain infinite trees, but we can show that, in fact, we only need to consider a finite cut at depth polynomial with respect to the sizes of the FDDS in order to decide if a solution exists:
	
	% \begin{proposition}\label{prop:polyUnroll2PolyForestFini}
	% 	Let $P = \sum_{i=0}^{m} A_i X^i$ be polynomial over FDDS, $B$ an FDDS and $\alpha$ the number of unroll trees of $\unroll{B}$.
	% 	Let $n \ge 2 \times \alpha^2 + \depth{\unroll{B}}$ be an integer.
	% 	If there exists a forest $\forest{y}$ with $\depth{\forest{y}} \le n$ such that $\sum_{i=0}^{m}\cut{\unroll{A_i}}{n} \forest{y}^i = \cut{\unroll{B}}{n}$, then there exists a FDDS $X$ such that $\unroll{P(X)} = \unroll{B}$ and $\cut{\unroll{X}}{n} = \forest{y}$.
	% \end{proposition}

        \begin{figure}[p]
        \centering
        \includegraphics[page=3,width=\textwidth]{pictures}
        % \vspace{-1em}
        \caption{A run of the algorithm for polynomial equations over forests of finite trees.}
        \label{fig:algo}
        \end{figure}


	\begin{proposition}\label{prop:polyUnroll2PolyForestFini}
		Let $P = \sum_{i=0}^{m} A_i X^i$ be polynomial over FDDS, $B$ an FDDS and $\alpha$ the number of unroll trees of $\unroll{B}$.
		Let $n \ge 2 \times \alpha^2 + \depth{\unroll{B}}$ be an integer.
		Then, there exists a forest $\forest{y}$ with $\depth{\forest{y}} \le n$ such that $\sum_{i=0}^{m}\cut{\unroll{A_i}}{n}  \forest{y}^i = \cut{\unroll{B}}{n}$ if and only if there exists a FDDS $X$ such that $\unroll{P(X)} = \unroll{B}$. Furthermore, if such an $X$ exists, then necessarily $\cut{\unroll{X}}{n} = \forest{y}$.
	\end{proposition} 
	
	\begin{proof}
		Assume that $\sum_{i=0}^{m}\cut{\unroll{A_i}}{n} \forest{y}^i = \cut{\unroll{B}}{n}$.
		First, remark that we can apply a similar reasoning to \cite[Lemma 11]{kroot} in order prove that if there exists $\forest{x}$ such that $\sum_{i=0}^{m} \unroll{A_i} \forest{X} = \unroll{B}$ then there exists an FDDS $Y$ such that $\unroll{Y} = \forest{X}$. 
		Since $\alpha$ bounds the greatest cycle length in $A_0,\ldots,A_m,B$ and since, from the hypothesis, $\depth{\unroll{B}}$ bounds $\depth{\unroll{A_i}}$ for each $i$, we have all the tools needed in order to prove (similarly to \cite[Theorem 5]{kroot}) that there exists a FDDS $X$ such that $\unroll{P(X)} = \unroll{B}$. The inverse implication is trivial.
        \end{proof}
                
        This proposition has two consequences. 
        First of all, the solution (if it exists) to an equation $\sum_{i=0}^{m} \unroll{A_i} \forest{X}^i = \unroll{B}$ is always the unroll of an FDDS. 
        Furthermore, by employing the same method for rebuilding $\unroll{X}$ from the solution~$\forest{Y}$ with~$\depth{\forest{Y}} \le n$ of $\sum_{i=0}^{m}\cut{\unroll{A_i}}{n} \forest{y}^i = \cut{\unroll{B}}{n}$ exploited in the proof of \cite[Theorem 6]{kroot}, we deduce an algorithm for solving equations over unrolls. In order to encode unrolls (which are infinite graphs), we represent them finitely by FDDS having those unrolls.
        
		
		\begin{theorem}\label{theorem:equation_forest_P}
			Let $P$ be a polynomial over unrolls (encoded as a polynomial over FDDS) and~$B$ an FDDS.
			Then, we can solve the equation~$P(\unroll{X}) = \unroll{B}$ in polynomial time.
		\end{theorem} 	
	
%%% Local Variables:
%%% mode: LaTeX
%%% TeX-master: "main"
%%% End:

	
	\section{Sufficient condition for the injectivity of polynomials}\label{section:cond_suf_poly_FDDS_inj}

	Even if all polynomials over unrolls are injective, we can easily remark that is not the case for the FDDS.
	For example, let $A$ be the cycle of length $2$; then, the polynomial $AX$ is not injective since $A^2=2A$.
	Hence, since the existence of injective polynomials is trivial (just consider the polynomial $X$), we want to obtain a characterization of this class of polynomials. 
	% In order to do this, we begin by considering polynomials $P = \sum_{i=1}^{m} A_i X^i$ without constant terms.
	
	Proposition 40 of \cite{article_arbre} gives a sufficient condition for the injectivity of polynomials without constant term, namely if the coefficient~$A_1$ is cancelable, \ie, at least one of its connected component is a dendron.
	However, since~$X^k$ is injective for all~$k>0$ (by uniqueness of $k$-th roots~\cite{article_arbre}), this condition is not necessary. 
	Hence, we start by giving a sufficient condition which covers these cases.
	
	\begin{proposition}\label{prop:condSufInj}
		Let $P = \sum_{i=0}^{m} A_i X^{i}$ be a polynomial.
		If at least one coefficient of $P$ is cancelable then $P$ is injective.
	\end{proposition}
	
	% For the proof of this proposition, we start by showing the case of polynomials $P = \sum_{i=1}^{m} A_i X^i$ without a constant terms. 
	% For that, we need to analyse the structure of the results of $P$. 
	% We first define two preorders over a subset of FDDS, which are actually orders on connected FDDS:
	
	% \begin{definition}
	% 	Let $a,b$ be two positive integer, $A,B$ be two FDDS such all of connected component of $A$ (resp., $B$) have size $a$ (resp., $b$), $\tree{t},\tree{t'}$ be two unroll trees such that $\tree{t}= \min(\unroll{A})$ and $\tree{t'} = \min(\unroll{B})$.
	% 	We define:
	% 	\begin{itemize}
	% 		\item $A \oct B$ if and only if $a > b$ or $a = b$ and $\tree{t} \ge \tree{t'}$.
	% 		\item $A \otc B$ if and only if $\tree{t} > \tree{t'}$ or $\tree{t} = \tree{t'}$ and $a \ge b$.
	% 	\end{itemize}
	% \end{definition}
	
	
	% We can easily prove that these two preorders are total because they are based on two total orders. Our proof technique for Proposition~\ref{prop:condSufInj} consists in studying the structure of the results of the polynomial equation according to the preorders~$\oct$. Remark that $\oct$ and~$\otc$ are not compatible with the product: if~$C_n$ is the cycle of length~$n$, then~$C_4 \oct C_3$ and~$C_4 \otc C_3$, but~$C_4 C_2 \ioct C_3 C_2$ and~$C_4 C_2 \iotc C_3 C_2$. Thus, we can only prove an approximation of the compatibility with the product.
	
	% \begin{lemma}\label{lemme:compatibleOct}
	% 	Let $p$ be an integer and $A,B,X$ be FDDSs such that all connected component of $A$ and $B$ have a cycle of length $p$.
	% 	Let $A'$ (resp., $B'$) be the multiset of components of $AX$ (resp., $BX$) having minimum cycle length.
	% 	If $A \oct B$ then $A' \oct B'$ and $A'$ and $B'$ have the same cycle lengths.
	% \end{lemma}  
	
	% \begin{proof}
	% 	Assume that $A \oct B$.
	% 	Let $\{x_1, \ldots, x_n\}$ be the different lengths of cycles in $X$ and let $i \in \{1,\ldots, n\}$ be such that $\lcm(p, x_i) \le \lcm(p, x_j)$ for all $j \in \{1, \ldots, n\}$.
		
	% 	From the hypotheses on $A$ and $B$ it follows that $\min(\unroll{A}) \ge \min(\unroll{B})$ and $A'$ and $B'$ have cycles of size $\lcm(p, x_i)$.
	%         Let~$X_j$ is minimal with respect to~$\otc$ among the connected components~$X_k$ of~$X$ such that~$\lcm(p, \cycle(X_k)) = \lcm(p, x_i)$.
	% 	% So, $b' = \lcm(p, j) = a'$ with $b'$ (resp., $j, a'$) the size of the cycle of $B'$ (resp., $X_j, A'$).
	%         Then~$\min(\unroll{\min_{\oct}(A')}) = \min(\unroll{A})\min(\unroll{X_j})$; similarly,~$\min(\unroll{\min_{\oct}(B')}) = \min(\unroll{B})\min(\unroll{X_j})$. Since~$A \oct B$ we have~$\min(\unroll{A}) \ge \min(\unroll{B})$ and from the compatibility with the product~\cite{article_arbre} we obtain~$\min(\unroll{A})\min(\unroll{X_j}) \ge \min(\unroll{B})\min(\unroll{X_j})$ and thus~$A' \oct B'$.
	% \end{proof}
	
	% This property is weaker than compatibility with the product, but it turns out to be sufficient in order to find necessary conditions for the injectivity of some classes of polynomials. We begin with one the easier cases: the polynomial $X^k$ for $k > 0$.
	
	% Accordingly, let $k$ be a positive integer and $X = X_1 + \ldots + X_m $ be an FDDS such that each $X_i$ are connected and $X_{i+1} \oct X_i$.  
	% We know that $X^k = \binom{k}{k_1,\ldots,k_n} \prod_{i=1}^{m} X_i^{k_i}$.
	% Furthermore, since the sum of FDDSs is their disjoint union, each FDDS can be written as a sum of trees in a unique way (up to reordering of the terms).
	% Thus, $X^k$ has only this form.
	
	% For studies more precisely the structure of $X^k$, we need to define two multi-set of connected component in an FDDS.
	For the proof of this proposition, we start by showing the case of polynomials $P = \sum_{i=1}^{m} A_i X^i$ without a constant term. 
	For that, we need to analyze the structure of the possible evaluations of $P$, and more precisely their multisets of cycles.
	We denote by $\cycle(A)$ the length of the cycle of a connected FDDS $A$.
	% For this analyze, we need two tools. 
	Let us introduce two total orders over connected FDDS:
	
	\begin{definition}
		% Let $a,b$ be two positive integer, $\tree{t},\tree{t'}$ be two unroll trees and
	        Let $A,B$ be two connected FDDS, let~$a = \cycle(A)$, $b = \cycle(B)$, $\tree{t}= \min(\unroll{A})$, and $\tree{t'} = \min(\unroll{B})$.
		We define:	
		\begin{itemize}
			\item $A \oct B$ if and only if $a > b$ or $a = b$ and $\tree{t} \ge \tree{t'}$.
			\item $A \otc B$ if and only if $\tree{t} > \tree{t'}$ or $\tree{t} = \tree{t'}$ and $a \ge b$.
		\end{itemize}
	\end{definition}
	
	% The second tool for our analysis is two functions of connected component.
	For an FDDS $X$ and a positive integer~$p$, let us consider the function~$\setSize{X}{p}$, which computes the multiset of connected components of $X$ with cycles of length $p$, and the function~$\setDive{X}{p}$, computing the multiset of connected components of $X$ with cycles of length \emph{dividing} $p$.
	
	\begin{lemma}\label{lemme:closureProduct}
		Let $p > 0$ be an integer; then
		$\setDive{\cdot}{p}$ is an endomorphism over FDDS. 
	\end{lemma}
	
	\begin{proof}
		First, if we denote by $\mathbf{1}$ the fixed point (the multiplicative identity in~$\mathbb{D}$), we obviously have $\setDive{\mathbf{1}}{p} = \mathbf{1}$. 
		Now, let $A,B$ be two FDDS. 
		It is clear that $\setDive{A + B}{p} = \setDive{A}{p} + \setDive{B}{p}$ since the sum is the disjoint union.
		All that remains to prove is that $\setDive{AB}{p} = \setDive{A}{p} \setDive{B}{p}$.
		By the distributivity of the sum over the product, it is sufficient to show the property in the case of connected $A$ and $B$. 
		Recall that, by the definition of product of FDDS, the product of two connected FDDS $U,V$ has $\gcd(\cycle(U),\cycle(V))$ connected component with cycle length $\lcm(\cycle(U), \cycle(V))$.  
		We deduce that all elements in $\setDive{AB}{p}$ are generated by the product of elements in $\setDive{A}{p}$ and $\setDive{B}{p}$.
		Hence $\setDive{AB}{p} \subseteq \setDive{A}{p}\setDive{B}{p}$.
		Furthermore, by the definition of $\setDive{A}{p}$, each element of $\setDive{A}{p}$ has a cycle length dividing $p$.
		Thus, the cycle length of the product of elements of $\setDive{A}{p}$ and $\setDive{B}{p}$ is bounded by $p$.
		Since the $\lcm$ of integers dividing $p$ is also a divisor of $p$, we have $\setDive{A}{p} \setDive{B}{p} \subseteq \setDive{AB}{p}$.
	\end{proof}
	
	%A directed consequences of this lemma which can be seen like the cloture in terms of the cycle size of $\setSize{X}{p}$ over the power \ie $\setDive{X^k}{p} = (\setDive{X}{p})^k$ for als positive integer $k$. 
	%Thus, in $X^k$, we can identify for each $X_i$ in $X$, the smallest connected component, by the order $\oct$, which comes from a product with $X_i$.
	
	
	
	%\begin{lemma}\label{lemme:structurePower}
	%	Let $k,n,m$ be an positive integers with $n \le m$ and  $X = X_1 + \ldots + X_m$ be a FDDS such that $X_i$ is a connected component for all $i$ and $X_{i+1} \oct X_i$. 
	%	Let $B$ be the smallest connected component over $\oct$ in $X^k - (\sum_{i=1}^{n-1} X_i)^k$ and $p = \cycle(B)$.
	%	Let $X_l$ be the smallest element of $\setDive{X}{p}$ over $\otc$.
	%	Then $B$ come from $X_l^{k-1} X_n$.
	%\end{lemma}
	%
	%\begin{proof}
	%	Let $q$ be the size of the cycle of $X_n$.
	%	First, we have that $B$ come from the product of $k$ element of $\setDive{X}{p}$.
	%	In addition, since $B$ is in $X^k - (\sum_{i=1}^{n-1} X_i)^k$, we deduce that at least one terms of this product is at least $X_n$.
	%	So $q \le p$.
	%	Besides, since $X_n^k$ is in $X^k - (\sum_{i=1}^{n-1} X_i)^k$ and that the size of each cycle of $X_n$ is $q$, by the minimality of $B$, we deduce that $p \le q$.
	%	Thus, we conclude that $q = p$ and also each connected component of $X_l^{k-1} X_n$ has a cycle of size $p$.
	%	
	%	Second, by the minimality of $B$, we have that $\min(\unroll{B}) = \min(\unroll{\setDive{X}{p}^k - (\setDive{\sum_{i=1}^{n-1} X_i}{p})^k})$.
	%	However, since $\unroll{\setDive{X}{p}^k - (\setDive{\sum_{i=1}^{n-1} X_i}{p})^k}$ is equals to $\unroll{\setDive{X}{p}^k} - \unroll{ (\setDive{\sum_{i=1}^{n-1} X_i}{p})^k}$, by a adaptation for infinite trees of \cite[Lemma 4]{kroot} and the minimality of $X_n$, we have that $\min(
	%	\unroll{\setDive{X}{p}^k} - \unroll{ (\setDive{\sum_{i=1}^{n-1} X_i}{p})^k})$ is equals of $\min(\unroll{X_l})^{k-1} \min(\unroll{X_n})$.
	%	Thus at least one connected component of $X_l^{k-1} X_n$ has this minimal unroll tree which equals to $\min(\unroll{B})$. 
	%	
	%	For resume, all connected components of $X_l^{k-1} X_n$ have their cycle of $p$ and at least one connected component of $X_l^{k-1} X_n$ has its minimal unroll tree equals to $\min(\unroll{B})$.
	%	This implies that $B$ is a connected component in $X_l^{k-1} X_n$.  
	%\end{proof}
	%
	%To this proof, we can show that $X_n \oct X_l$.
	%Indeed, since $\unroll{X_l}$ have the minimal unroll tree of $\unroll{\setDive{X}{p}}$, we have that $\min(\unroll{X_n}) \ge \min(\unroll{X_l})$.
	%And, since $X_l$ is an element of $\setDive{X}{p}$, the size of the cycle of $X_l$ is inferior or equals to $X_n$.
	%So, the property follows.
	%
	%In addition, we deduce an algorithm for solve the $k$-th root of a FDDS. 
	%Indeed, from the Lemma~\ref{lemme:structurePower}, the length of the cycle of $\min(X^k)$ over $\oct$ is the length of the cycle of $\min(X)$ over $\oct$. 
	%We set $min$ this length.
	%And $\unroll{\min(X)}$ over $\oct$ is equals to $\sqrt[k]{\min(\unroll{X}^k)}$. 
	%Thus, we can roll $\min(\unroll{\min(X)})$ to period $min$.
	%
	%After that, we can inductively construct the $i+1$-th smallest component of $X$ over $\oct$. 
	%In fact, from the Lemma~\ref{lemme:structurePower}, the length of the cycle of $\min(X^k - (X_1 + \ldots + X_i)^k)$ over $\oct$ is the length of the cycle of $\min(X - (X_1 + \ldots + X_i))$ over $\oct$.
	%We set $p$ this length.   
	%Two cases are possible. 
	%Either $\min(X^k - (X_1 + \ldots + X_i)^k)$ over $\oct$ is equals to $\min(X^{k}\{p\})$ over $\otc$ and $\unroll{X_{i+1}} = \sqrt[k]{\min(X^k - (X_1 + \ldots + X_i)^k)}$, or not and $\min(\unroll{X_{i+1}})$ is the quotient of the division of the minimal unroll tree of $\min(\unroll{\min(X^k - (X_1 + \ldots + X_i)^k)})$ over $\oct$ by the minimal unroll tree in $\unroll{\min(X^{k}\{p\})}^{k-1}$ over $\otc$. 
	%In this two cases, we can identify the minimal unroll tree of $\unroll{X_{i+1}}$. 
	%Thus, we can roll this tree to period $p$. 
	%
	%Moreover, by \cite{article_arbre, kroot}, all of these operation are in polynomial times. Thus, this prove the following proposition.
	%
	%\begin{proposition}
	%	Given $A$ an FDDS and $k$ an integer, we can find in polynomial times the FDDS $X$ such that $X^k = A$ if it exists. 
	%\end{proposition}
	%
	%At this point, we can moving to the studies of general polynomials whose at least one cancelable coefficient. 
	%As previously, our goal is to identify some connected component of $P(X)$ and the product of $P$ whose they are generated. 
	%However, this case is more complicated than $X^k$ because we have to find the coefficient and the connected component in this coefficient in addition to the part in $X$.  
	
	Our goal is to prove, by a simple induction, the core of the proof of Proposition~\ref{prop:condSufInj}. % , namely Lemma~\ref{lemme:condSufInjSpe}. 
	Lemma~\ref{lemme:closureProduct} is a first step to show the induction case.
	Indeed, it allows us to treat sequentially the different cycle lengths. We use the order~$\ioct$ to process sequentially connected components having the same cycle length, as shown by the following lemma.
	
	\begin{lemma}\label{lemme:structurePolyFDDS}
		Let $X = X_1 + \cdots + X_k$ be an FDDS with $k$ connected components sorted by $\ioct$ and $P = \sum_{i=1}^{m} A_i X^i$ a polynomial over FDDS without constant term and with at least one cancelable coefficient. 
		Then $B = \min_{\oct}(P(X) - P(\sum_{i=1}^{n-1} X_i))$ implies $\cycle(X_n) = \cycle(B)$ for all~$1 \le n \le k$.
	\end{lemma}
	
	\begin{proof}
	        We have $\cycle(X_n) \le \lcm(a, \cycle(X_{i}))$ for all integers $i$ with $n \le i \le k$ and $a > 0$. 
		It follows that $\cycle(X_n) \le \cycle(B)$. 
		In addition, since at least one coefficient of $P$, say $A_{c}$, is cancelable, it contains one connected component whose cycle length is one. 
		This implies that $A_{c} X_{n}^c$ contains a connected component $C$ such that $\cycle(C) = \cycle(X_n)$. 
		If $B$ is minimal, then $\cycle(X_n) \ge \cycle(B)$.
	\end{proof}
	
	We can now identify recursively and unambiguously each connected component in $X$. 
	Therefore, we can use Lemma~\ref{lemme:structurePolyFDDS}, as part of the induction step for the following lemma.
	
	\begin{lemma}\label{lemme:condSufInjSpe}
		Let $P = \sum_{i=1}^{m} A_i X^{i}$ be a polynomial without constant term.
		If at least one coefficient of $P$ is cancelable then $P$ is injective.
	\end{lemma}
	
	\begin{proof}
	        Let $X = X_1 + \cdots + X_{n_1}$, resp., $Y = Y_1 + \cdots + Y_{n_2}$ be FDDS consisting of~$n_1$ (resp., $n_2$) connected components sorted by~$\ioct$, and let~$B$ be an FDDS. Suppose~$P(X) = B = P(Y)$.
		We prove by induction on the number~$n_1$ of connected components in $X$ that $X = Y$.
		
		If $X$ has $0$ connected components, since $P$ does not have constant term, we have $P(X) = 0$.
		In addition, since $P$ has at least a cancelable coefficient, it trivially contains at least one nonzero coefficient.
		Thus, since $P(Y) = P(X) = 0$, we deduce that $Y = 0$, hence $X = Y$.
		The property is true in the base case.
		
		Let $n$ be a positive integer.
		Suppose that the properties is true for $n$, \ie $X_1 + \cdots + X_n = Y_1 + \cdots + Y_n$.
		Then, $P(X_1 + \cdots + X_n) = P( Y_1 + \cdots + Y_n)$.
		This implies that $P(X) - P(X_1 + \cdots + X_n) = B - P(X_1 + \cdots + X_n) = P(Y) -  P(Y_1 + \cdots + Y_n)$.
		Let $p$ be the length of the cycle of the smallest connected component of $B - P(X_1 + \cdots + X_n)$ according to $\oct$.
		By Lemma \ref{lemme:structurePolyFDDS}, 
		we have $\cycle(X_{n+1}) = \cycle(Y_{n+1}) = p$.
		
		Besides, thanks to Lemma \ref{lemme:closureProduct}, we deduce that $\setDive{P(X)}{p} = \sum_{i=1}^{m} \setDive{A_i}{p} \setDive{X^i}{p}$ and $\setDive{P(Y)}{p} = \sum_{i=1}^{m} \setDive{A_i}{p} \setDive{Y^i}{p}$.
		Since $\setDive{P(X)}{p} = \setDive{B}{p} = \setDive{P(Y)}{p}$, we deduce that $ \sum_{i=1}^{m} \setDive{A_i}{p} \setDive{X^i}{p} =  \sum_{i=1}^{m} \setDive{A_i}{p} \setDive{Y^i}{p}$.
		By taking unrolls, we obtain $\sum_{i=1}^{m} \unroll{\setDive{A_i}{p}} \unroll{\setDive{X^i}{p}} =  \sum_{i=1}^{m} \setDive{A_i}{p} \setDive{Y^i}{p}$.
		However, by injectivity of polynomials over unrolls (Theorem \ref{th:injPolyUnrolls}), we have $\unroll{\setDive{X}{p}} = \unroll{\setDive{Y}{p}}$.
		Thus, the smallest tree of $\unroll{\setDive{X}{p}} - \unroll{\setDive{X_1 + \cdots + X_n}{p}}$ is equal to the smallest of $\unroll{\setDive{Y}{p}} - \unroll{\setDive{Y_1 + \cdots + Y_n}{p}}$.
		However, the smallest tree of $\unroll{\setDive{X}{p}} - \unroll{\setDive{X_1 + \cdots + X_n}{p}}$ is the smallest tree of $\unroll{X_{n+1}}$ and of $\unroll{Y_{n+1}}$.
		This implies that $X_{n+1}$ and $Y_{n+1}$ are two connected components whose cycle length is $p$ and have the same minimal unroll tree, so $X_{n+1} = Y_{n+1}$.
		The induction step and the statement follow.
	\end{proof}
	
	What remains in order to conclude the proof of Proposition \ref{prop:condSufInj} is to extend Lemma \ref{lemme:condSufInjSpe} to polynomials with a constant term.
	
	\begin{proof}[Proof of Proposition \ref{prop:condSufInj}]
		Let $P = \sum_{i=0}^{m} A_i X^{i}$ be a polynomial with at least one non-constant cancelable coefficient.
		Let $P' = \sum_{i=1}^{m} A_i X^{i}$ be the polynomial obtained from $P$ by removing the constant term $A_0$.
		Hence $P = P' + A_0$.
		If there exist two FDDS $X,Y$ such that $P(X) = P(Y)$, then $P'(X) + A_0 = P'(Y) + A_0$.
		This implies that $P'(X) = P'(Y)$.
		Thus, by Lemma \ref{lemme:condSufInjSpe}, we conclude that $X = Y$. 
	\end{proof}
	
	From the proof of Lemma~\ref{lemme:structurePolyFDDS} and Theorem~\ref{theorem:equation_forest_P}, we deduce that we can solve in polynomial time all equations of the form~$P(X) = B$ with~$P$ having a cancelable non-constant coefficient. 
	Indeed, we can just select the connected components with the correct cycle lengths, and cut their unrolls to the correct depth. 
	Then, we solve the equation over the forest thus obtained, select the smallest tree of the result, and re-roll it to the correct period. 
	Finally, we remove the corresponding multiset of connected components and reiterate until either all connected components have been removed, or we find a connected component that we cannot remove (which implies that the equation has no solution). 
	
	%%% Local Variables:
	%%% mode: LaTeX
	%%% TeX-master: "main"
	%%% End:

	
	\section{Necessary condition for the injectivity of polynomials}\label{section:cond_nec_poly_FDDS_inj}
	
	We will now show that the sufficient condition for injectivity of Proposition \ref{prop:condSufInj} is actually necessary.
	As previously, we start by considering polynomials without constant terms.
	Our starting point for this proof is given by  \cite[Theorem 34]{article_arbre}, which characterizes the injectivity of linear monomials.
	Furthermore \cite[Lemma 33]{article_arbre} proves that this condition is necessary for linear monomials.
	We begin by extending the proof of \cite[Lemma 33]{article_arbre} to show that this is necessary for \emph{all} monomials. 
	
	For this, let $\delta_J$ be the sequence recursively defined by $\delta_\emptyset = 1$ and $\delta_{J \cup \{a\}} = \gcd(a, \lcm(J)) \delta_J$ for all $a \in \N$. 
	Now let~$\mathcal{A}$ be a set of integers strictly larger than $1$ and for $I \subseteq \mathcal{A}$ we define $\alpha_I = \delta_\mathcal{A}\prod_{a \in \mathcal{A}} a$ and $\beta_I = \alpha_I + (-1)^{|I|}\delta_I \prod_{a \in \mathcal{A} - I} a$.
	
	We will use $\alpha_I$ and $\beta_I$ to construct two different FDDS $X$ and $Y$ such that $AX = AY$ if $A$ is not cancelable.
	We will start with the case where $A$ is a cycle, denoted by $C_b$ where $b = \cycle(C_b)$.
	% We First set a technical lemma.
	
	\begin{lemma}\label{lemme:cycleNonInjTech}
		% Let $\mathcal{A}$ be a set of integers strictly larger than $1$ and
                Let $k \ge 1$ be an integer, $b \in \mathcal{A}$, $I \subseteq \mathcal{A} - \{b\}$ and $J = I \cup \{b\}$. Then
                \begin{equation}
                \label{eq:kToKMinus1}
                C_b (\beta_I C_{\lcm(I)} + \beta_J C_{\lcm(J)})^k = C_b (\alpha_I C_{\lcm(I)} + \alpha_J C_{\lcm(J)})^k.
                \end{equation}
	\end{lemma}
	
	\begin{proof}
		% We have
		% \begin{equation}
		% 	C_b (\beta_I C_{\lcm(I)} + \beta_J C_{\lcm(J)})^k = C_b (\beta_I C_{\lcm(I)} + \beta_J C_{\lcm(J)}) (\beta_I C_{\lcm(I)} + \beta_J C_{\lcm(J)})^{k-1}.
		% \end{equation}
		The proof of \cite[Lemma 33]{article_arbre} shows that $C_b (\beta_I C_{\lcm(I)} + \beta_J C_{\lcm(J)}) = C_b (\alpha_I C_{\lcm(I)} + \alpha_J C_{\lcm(J)})$.
		Thus, we can replace one instance $C_b (\beta_I C_{\lcm(I)} + \beta_J C_{\lcm(J)})$ in \eqref{eq:kToKMinus1} % by $C_b (\alpha_I C_{\lcm(I)} + \alpha_J C_{\lcm(J)})$
                and obtain:
		\[C_b (\beta_I C_{\lcm(I)} + \beta_J C_{\lcm(J)})^k =(\alpha_I C_{\lcm(I)} + \alpha_J C_{\lcm(J)}) C_b (\beta_I C_{\lcm(I)} + \beta_J C_{\lcm(J)})^{k-1}.\]
		By repeating the same substitution, the thesis follows.
	\end{proof}
	
	Now we can prove that monomials of the form $A X^k$, with $k \ge 1$ and where $A$ is a sum of cycles (\ie, a
        \emph{permutation}) with no cycle of length~$1$, are never injective.
	
	\begin{lemma}\label{lemme:noInjMonomeP1}
		% Let $\mathcal{A}$ be a set of integer strictly superior to $1$ and
                Let~$A = A_1 + \cdots + A_m$ be a sum of cycles of length greater than~$1$ and $k \ge 1$ be an integer.
		Then, there exist two different $FDDS$ $X,Y$ such that $A_j X^k = A_j Y^k$ for all~$1 \le j \le m$ and, by consequence, $A X^k = A Y^k$.
	\end{lemma}

	\begin{proof}
                Let~$\mathcal{A}$ be the set of cycle lengths of~$A$.      
                Let $X = \sum_{I \subseteq \mathcal{A}} \alpha_I C_{\lcm(I)}$ and $Y = \sum_{I \subseteq \mathcal{A}} \beta_I C_{\lcm(I)}$. Remark that~$\alpha_I \ne \beta_I$ for all~$I \subseteq \mathcal{A}$, since~$\beta_I$ is $\alpha_I$ plus a nonzero term; in particular, $\alpha_\emptyset \ne \beta_\emptyset$ and, since these two integers are the number of fixed points of~$X$ and~$Y$ respectively, we have~$X \ne Y$.

                Let us consider a generic term $A_j$, which is a cycle $C_b$ for some $b>1$ for all~$j$. Let~$I_1, \ldots, I_n$ be an enumeration of the subsets of~$\mathcal{A}-\{b\}$ and let~$J_i = I_i \cup \{b\}$ for all~$1 \le i \le n$. Then~$Y = \sum_{i=1}^n (\beta_{I_i} C_{\lcm(I_i)} + \beta_{J_i} C_{\lcm(J_i)})$
		
		% We have that $C_b Y^k = C_b (\sum_{I \subseteq \mathcal{A} - \{b\}} \beta_I C_{\lcm(I)} + \beta_J C_{\lcm(J)})^k$.
		In order to compute~$A_j Y^k = C_b Y^k$ we can distribute the power over the sum, then the product over the sum, and we obtain:
		\begin{equation}\label{eq:distProdSum}
			C_b Y^k = \sum_{k_1+\cdots+k_n=k}\binom{k}{k_1,\ldots,k_n} C_b (\beta_{I_1} C_{\lcm(I_1)} + \beta_{J_1} C_{\lcm(J_1)})^{k_1}\prod_{i=2}^{n} (\beta_{I_i} C_{\lcm(I_i)} + \beta_{J_i} C_{\lcm(J_i)})^{k_i}.
		\end{equation}
		By Lemma \ref{lemme:cycleNonInjTech}, we have $C_b (\beta_{I_1} C_{\lcm(I_1)} + \beta_{J_1} C_{\lcm(J_1)})^{k_1} = C_b (\alpha_{I_1} C_{\lcm(I_1)} + \alpha_{J_1} C_{\lcm(J_1)})^{k_1}$.
		Thus, by replacing this in Equation \eqref{eq:distProdSum}, we have
		\[C_b Y^k = \sum_{k_1+\cdots+k_n=k}\binom{k}{k_1,\ldots,k_n} C_b (\alpha_{I_1} C_{\lcm(I_1)} + \alpha_{J_1} C_{\lcm(J_1)})^{k_1} \prod_{i=2}^{n} (\beta_{I_i} C_{\lcm(I_i)} + \beta_{J_i} C_{\lcm(J_i)})^{k_i}.\]
		If we repeat this substitution for all $2 \le i \le n$ and factor $C_b$ from each term of the sum we obtain
		\[C_b Y^k = C_b \sum_{k_1+\cdots+k_n=k}\binom{k}{k_1,\ldots,k_n}  \prod_{i=1}^{n} (\alpha_{I_i} C_{\lcm(I_i)} + \alpha_{J_i} C_{\lcm(J_i)})^{k_i}.\] 
		Since $\sum_{k_1+\cdots+k_n=k}\binom{k}{k_1,\ldots,k_n}  \prod_{i=1}^{n} (\alpha_{I_i} C_{\lcm(I_i)} + \alpha_{J_i} C_{\lcm(J_i)})^{k_i}$ is just $X^k$, we conclude that $C_b Y^k = C_b X^k$, \ie, $A_j Y^k = A_j X^k$.

                This holds separately for each connected component~$A_j$ of~$A$; by adding all terms and factoring $X^k$ and $Y^k$ we obtain $AX^k = AY^k$.
	\end{proof}
	
	We remark that the $X$ and $Y$ which we have built are just permutations. 
	An interesting property of permutations is that they are closed under sums and products (and thus nonnegative integer powers).
	Thus $X^k$ and $Y^k$ are also permutations.
	Another useful property of permutations is, intuitively, that if we multiply a sum of connected components $A = A_1 + \cdots + A_n$ with a sum of cycles $X = X_1 + \cdots + X_m$, the trees rooted in each connected component of $A_i X_j$ are just a periodic repetition of the trees rooted in the cycle of $A_i$ for all $i,j$ \cite[Corollary 14]{article_arbre}.
	Thanks to these properties, we can characterize the injectivity of monomials.
	
	\begin{proposition}\label{prop:charaInjMonome}
		A monomial over FDDS is injective if and only if its coefficient is cancelable.
	\end{proposition}
	
	\begin{proof}
		Assume that the monomial $AX^k$, where $A = A_1 + \cdots + A_n$ as a sum of connected components and~$k>0$, is not cancelable. Hence, by \cite[Theorem 34]{article_arbre}, $A$ does not have a cycle of length 1 (a fixed point).
		Let $B = B_1 + \cdots + B_n$ the permutation such that~$\cycle(B_i) = \cycle(A_i)$ for all~$i$.
		By Lemma \ref{lemme:noInjMonomeP1}, there exist two distinct permutations $X,Y$ such that $B_i X^k = B_i Y^k$ for all $i$. 
		Thus, by~\cite[Corollary 14]{article_arbre}, we have $A_i X^k = A_i Y^k$ and, by summing, $A X^k = A Y^k$.

		The inverse implication was already proved in Proposition \ref{prop:condSufInj}.
        \end{proof}
	
	In the rest of this section, we will prove that a polynomial is injective if and only if a certain monomial is injective.
	Remark that the sum of any FDDS with a cancelable FDDS is also cancelable.
	Hence, the sum of the coefficients of a polynomial is cancelable if and only if at least one of them is cancelable.
	
	From the characterization of injective monomials (Proposition~\ref{prop:charaInjMonome}) we can deduce that there exists an positive integer $k$ such that $A X^k$ is injective if and only if $A X^n$ is injective for \emph{all} positive integer $n$.
	
	\begin{proposition}\label{prop:charaInj}
		Let $P = \sum_{i=1}^{m} A_i X^{i}$ be a polynomial without constant term and consider the monomial~$M = (\sum_{i=1}^{m} A_i)X$. Then, $P$ is injective if and only if $M$ is injective.
	\end{proposition}
	
	\begin{proof}
		Assume that $M$ is injective.
		Thus, by the Proposition \ref{prop:charaInjMonome}, its coefficient is cancelable.
		Hence, at least one coefficient of $P$ is cancelable.
		By Proposition \ref{prop:condSufInj}, we conclude that $P$ is injective. 
		
		For the other direction, assume that $M$ is not injective.
		Let $X$ and $Y$ the FDDS constructed in the proof of Proposition \ref{prop:charaInjMonome}.
		As previously, $X \neq Y$ but $A_i X^i = A_i Y^i$ for all $i \in \{1,\ldots, m\}$. 
		So, by sum, we have $P(X) = P(Y)$.
	\end{proof}
	
	Thanks to this characterization, we conclude that the condition given in Proposition \ref{prop:condSufInj} is also necessary for polynomials without a constant term.
	All it remains to prove is that this is also necessary in the presence of a constant term.
	
	\begin{theorem}\label{th:charaInjPoly}
		Let $P = \sum_{i=0}^{m} A_i X^{i}$ a polynomial. Then $P$ is injective if and only if at least one of $A_i$ is cancelable for some~$i \ge 1$.
	\end{theorem}
	
	\begin{proof}
                Assume that no non-constant coefficient of $P$ is cancelable.
		Let $P' = \sum_{i=1}^{m} A_i X^{i}$ the polynomial $P$ without the term $A_0$.
		By the Proposition \ref{prop:charaInj}, there exist two different FDDS $X,Y$ such that $P'(X) = P'(Y)$.
		Then $P'(X) + A_0 = P'(Y) + A_0$, which implies $P(X) = P(Y)$. 
		Thus, $P$ is not injective.

		The inverse implication is due to Proposition \ref{prop:condSufInj}.
	\end{proof}
	%%% Local Variables:
	%%% mode: LaTeX
	%%% TeX-master: "main"
	%%% End:


	\section{Conclusion}
In this work, we propose a simple yet effective approach, called SMILE, for graph few-shot learning with fewer tasks. Specifically, we introduce a novel dual-level mixup strategy, including within-task and across-task mixup, for enriching the diversity of nodes within each task and the diversity of tasks. Also, we incorporate the degree-based prior information to learn expressive node embeddings. Theoretically, we prove that SMILE effectively enhances the model's generalization performance. Empirically, we conduct extensive experiments on multiple benchmarks and the results suggest that SMILE significantly outperforms other baselines, including both in-domain and cross-domain few-shot settings.
	
	\begin{credits}
        \subsubsection{\ackname}
        This work has been partially supported by the HORIZON-MSCA-2022-SE-01 project 101131549 ``Application-driven Challenges for Automata Networks and Complex Systems (ACANCOS)'' and by the project ANR-24-CE48-7504 ``ALARICE''.
	\end{credits}

	\bibliography{biblio}
	\bibliographystyle{unsrt}
	
\end{document}
