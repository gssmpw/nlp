
% \definecolor{c1}{RGB}{224,175,107}  % orangle
% \definecolor{c2}{RGB}{138,170,214}  % light blue
% \definecolor{c3}{RGB}{222,117,123} %  % red
% \definecolor{c4}{RGB}{216,174,174} % light pink
% \definecolor{c5}{RGB}{163,137,214} % purple
% \definecolor{c6}{RGB}{248,199,1} %  % yellow
% \definecolor{c7}{RGB}{205,205,205} % grey
% \definecolor{C8}{RGB}{255,0,127} % purple

\definecolor{c1}{RGB}{216,174,174} % light pink
\definecolor{c2}{RGB}{224,175,107}  % orangle
\definecolor{c3}{RGB}{222,117,123} %  % red
% \definecolor{c4}{RGB}{163,137,214} % purple
\definecolor{c4}{RGB}{250, 221, 104} % vanilla

\definecolor{c5}{RGB}{174,223,172} % green
\definecolor{c6}{RGB}{138,170,214}  % light blue
\definecolor{c7}{RGB}{248,199,1} %  % yellow
\definecolor{c8}{RGB}{255,0,127} % pink red
% \definecolor{c8}{RGB}{205,68,50} % red

\definecolor{c11}{RGB}{172, 196, 226} % High
\definecolor{c12}{RGB}{208, 221, 238} % Hybrid

% \definecolor{c21}{RGB}{250, 221, 104} % Global
\definecolor{c21}{RGB}{163,137,214} % Global purple


% \definecolor{c1}{RGB}{235,230,68} % zero-shot
% \definecolor{c2}{RGB}{239,171,110} % cot
% \definecolor{c3}{RGB}{128,125,180} % bm25
% \definecolor{c4}{RGB}{167,206,159} % vanilla
% \definecolor{c5}{RGB}{162,200,221} % hipporag
% \definecolor{c6}{RGB}{227,118,111} % light
% \definecolor{c7}{RGB}{102,185,143} % graphrag
% \definecolor{c8}{RGB}{87,154,196} %  ArchRAG


\definecolor{c9}{RGB}{150,195,125} % 
\definecolor{c10}{RGB}{230,189,69} % 

% \definecolor{c11}{RGB}{33,26,62} %
% \definecolor{c12}{RGB}{208,108,157} %
\definecolor{c13}{RGB}{115,107,157} % 
\definecolor{c14}{RGB}{208,108,157} % 

\pgfplotstableread[row sep=\\,col sep=&]{
datasets & CSH & CSSH & SSCS & SSCSMP & Ecore & Vcore & SCANS \\ 
2 & 3.132 & 3.5 & 3.284 & 2.267 & 2.45 & 2.075 & 7.317 \\
4 & 8.327 & 9.288 & 3.143 & 2.363 & 3.381 & 3.122 & 4.066 \\
6 & 12.593 & 6.744 & 1.581 & 1.619 & 10.857 & 10.833 & 5.621 \\
}\DIACom
\pgfplotstableread[row sep=\\,col sep=&]{
datasets & CSH & CSSH & SSCS & SSCSMP & Ecore & Vcore & SCANS \\ 
2 & 0.57 & 0.629 & 0.889 & 0.898 & 0.85 & 0.64 & 0.266 \\
4 & 0.137 & 0.153 & 0.778 & 0.857 & 0.335 & 0.36 & 0.211 \\
6 & 0.536 & 0.708 & 0.822 & 0.825 & 0.425 & 0.594 & 0.568 \\
}\CCcom

\pgfplotstableread[row sep=\\,col sep=&]{
datasets & zero & cot & BM25 & vanilla & hippo & lightl & lighth & lighthy & graphl & graphg & ArchRAG \\ 
2 & 436.93 & 988.13 & 2590.56 & 3042.43 & 2359.64 & 5193.792 & 7152.56 & 7120.468 & 22118.98 & 246373.92 & 8890.07 \\ 
4 & 211.26 & 488.74 & 351.49 & 332.37 & 1392.55 & 1673.48 & 1852.6 & 2843.442 & 11457.59 & 123030.84 & 2024.94 \\ 
}\Qtime
\pgfplotstableread[row sep=\\,col sep=&]{
datasets & zero & cot & BM25 & vanilla & hippo & lightl & lighth & lighthy & graphl & graphg & ArchRAG \\ 
2 & 426105 & 795941 & 5594336 & 5556774 & 6121633 & 14617559 & 15467710 & 18515561 & 1536156 & 605937663 & 12575315 \\ 
4 & 89649 & 172217 & 675569 & 203654 & 886793 & 3088800 & 3223300 & 4817210 & 601000 & 1394436159 & 5191529 \\ 
}\Qtoken


\pgfplotstableread[row sep=\\,col sep=&]{
datasets & Hippo & light & GraphRAG & ArchRAG \\ 
2 & 3370 & 6689.42 & 7591 & 9320.95 \\
4 & 4521 & 18963.89 & 16118 & 14383.94 \\
}\Indextime
\pgfplotstableread[row sep=\\,col sep=&]{
datasets & Hippo & light & GraphRAG & ArchRAG \\ 
2 & 3044748 & 12537458 & 12212572 & 13328928 \\
4 & 8023838 & 52436836 & 29140517 & 35248693 \\
}\Indextoken



\begin{figure*}[h]
\centering
\setlength{\abovecaptionskip}{-0.1cm}
\setlength{\belowcaptionskip}{-0.3cm}
\small
\renewcommand{\arraystretch}{1.5} % 调整行高,默认值是1
\Large
\subfigure[Comprehensiveness]{
    \resizebox{0.23\textwidth}{!}{ % 调整宽度为页面的24%
    \begin{tabular}{c*{5}{c}}
  & VR & LR & C1 & C2 & AR \\ 
{VR} & \colorcell{50} & \colorcell{46} & \colorcell{18} & \colorcell{18} & \colorcell{1} \\ 
{LR} & \colorcell{54} & \colorcell{50} & \colorcell{21} & \colorcell{29} & \colorcell{16} \\ 
{C1} & \colorcell{82} & \colorcell{79} & \colorcell{50} & \colorcell{86} & \colorcell{18} \\ 
{C2} & \colorcell{82} & \colorcell{71} & \colorcell{14} & \colorcell{50} & \colorcell{16} \\ 
{AR} & \colorcell{99} & \colorcell{84} & \colorcell{82} & \colorcell{84} & \colorcell{50} \\ 
    \end{tabular}
    }
}
% 
\subfigure[Diversity]{
    \resizebox{0.23\textwidth}{!}{ % 调整宽度为页面的24%
    \begin{tabular}{c*{5}{c}}
  & VR & LR & C1 & C2 & AR \\ 
{VR} & \colorcell{50} & \colorcell{64} & \colorcell{3} & \colorcell{12} & \colorcell{8} \\ 
{LR} & \colorcell{36} & \colorcell{50} & \colorcell{52} & \colorcell{63} & \colorcell{33} \\ 
{C1} & \colorcell{97} & \colorcell{48} & \colorcell{50} & \colorcell{52} & \colorcell{46} \\ 
{C2} & \colorcell{88} & \colorcell{37} & \colorcell{48} & \colorcell{50} & \colorcell{42} \\ 
{AR} & \colorcell{92} & \colorcell{67} & \colorcell{54} & \colorcell{58} & \colorcell{50} \\ 
    \end{tabular}
    }
}
% 
\subfigure[Empowerment]{
    \resizebox{0.23\textwidth}{!}{ % 调整宽度为页面的24%
    \begin{tabular}{c*{5}{c}}
  & VR & LR & C1 & C2 & AR \\ 
{VR} & \colorcell{50} & \colorcell{39} & \colorcell{15} & \colorcell{21} & \colorcell{8} \\ 
{LR} & \colorcell{61} & \colorcell{50} & \colorcell{59} & \colorcell{30} & \colorcell{35} \\ 
{C1} & \colorcell{85} & \colorcell{41} & \colorcell{50} & \colorcell{60} & \colorcell{42} \\ 
{C2} & \colorcell{79} & \colorcell{70} & \colorcell{40} & \colorcell{50} & \colorcell{38} \\ 
{AR} & \colorcell{92} & \colorcell{65} & \colorcell{58} & \colorcell{62} & \colorcell{50} \\ 
    \end{tabular}
    }
}
% 
\subfigure[Overall]{
    \resizebox{0.23\textwidth}{!}{ % 调整宽度为页面的24%
    \begin{tabular}{c*{5}{c}}
 & VR & LR & C1 & C2 & AR \\ 
{VR} & \colorcell{50} & \colorcell{46} & \colorcell{14} & \colorcell{18} & \colorcell{4} \\ 
{LR} & \colorcell{54} & \colorcell{50} & \colorcell{48} & \colorcell{31} & \colorcell{22} \\ 
{C1} & \colorcell{86} & \colorcell{52} & \colorcell{50} & \colorcell{70} & \colorcell{31} \\ 
{C2} & \colorcell{82} & \colorcell{69} & \colorcell{30} & \colorcell{50} & \colorcell{30} \\ 
{AR} & \colorcell{96} & \colorcell{78} & \colorcell{69} & \colorcell{70} & \colorcell{50} \\ 
    \end{tabular}
    }
}
\caption{The abstract QA task results are presented as head-to-head win rate percentages, comparing the performance of the row method over the column method. VR, LR, C1, C2, and AR denote {Vanilla RAG}, {LightRAG-Hybrid}, {GraphRAG-Global} with high-level communities, {GraphRAG-Global} with intermediate-level communities, and {ArchRAG}, respectively.}
\label{fig:winmap}
\end{figure*}

\subsection{Overall results}
\label{sec:overallExp}
To answer the question \textit{RQ1}, we compare our method with baseline methods in solving both specific and abstract QA tasks.

$\bullet$ \textbf{Results of specific QA tasks.}
Table \ref{tab:docqa} reports the performance of each method on three datasets.
% 
Clearly, {ArchRAG} demonstrates a substantial performance advantage over other baseline methods on these datasets.
% 
On the Multihop-RAG dataset, {HippoRAG} performs worse than retrieval-only methods while outperforming retrieval-based methods on the HotpotQA dataset.
% 
This is mainly because, on the HotpotQA dataset, passages are segmented by the expert annotators; that is, passages can provide more concise information, whereas on  Multihop-RAG, passages are segmented based on chunk size, which may cause the LLM to lose context and produce incorrect answers.
% 
The experimental results suggest that not all communities are suitable for specific QA tasks, as the {GraphRAG-Global} performs poorly. 
% 
Furthermore, GraphRAG does not consider node attributes during clustering, which causes the community's summary to become dispersed, making it difficult for the LLM to extract relevant information.
% 
Thus, we gain an interesting insight: {\it LLM may not be a good retriever, but is a good analyzer.}
% 
% Therefore, using C-HNSW to search the most relevant communities proves to be highly effective.
% 




$\bullet$ \textbf{Results of abstract QA tasks.}
We compare {ArchRAG} against baselines across four dimensions on the Multihop-RAG dataset.
% 
Here, we only compare the {LightRAG-Hybrid} method, as it represents the best version of {LightRAG} methods \cite{guo2024lightrag}.
% 
As shown in Figure \ref{fig:winmap}, our method achieves comparable performance with {GraphRAG-Global} on the diversity and empowerment dimensions while significantly surpassing it on the comprehensive dimension.
% 
Overall, {ArchRAG} demonstrates outstanding performance in solving abstract QA tasks.


$\bullet$ \textbf{Efficiency of {ArchRAG}.}
To answer the question \textit{RQ2}, we compare the time cost and token usage of {ArchRAG} with those of other baseline methods and the index consumption against that of graph-based RAG methods.
%
As shown in Figure \ref{fig:query-efficiency}, {ArchRAG} demonstrates significant time and cost efficiency for online queries.
% 
For example, on the HotpotQA dataset, {ArchRAG} reduces token usage by a factor of 250 compared to {GraphRAG-Global}, utilizing 5.1M tokens and 1,394M tokens, respectively.

\begin{figure}[h]
    \centering
    \setlength{\abovecaptionskip}{-0.1cm}
    \setlength{\belowcaptionskip}{-0.2cm}
    \quad \ref{eff_leg}\\
    \subfigure[Time cost]{
    \begin{tikzpicture}[scale=0.45]
            \begin{axis}[
                ybar=0.5pt,
                bar width=0.6cm,
                width=\textwidth,
                height=0.26\textwidth,
                % xlabel={\Huge \bf datasets}, 
                xtick=data,
                xticklabels={\huge Multihop-RAG,\huge HotpotQA},
                legend style={
                % at={(0.5,0.78)},
                 at={(0.0,1.05)}, % 将图例放置在图表上方左对齐
    anchor=south west, % 确保图例的左下角对齐图表的左上角
                    align=left, % 确保文字左对齐
                % anchor=north,
                legend columns=3,
                draw=none},
                legend image code/.code={
                    \draw [#1, line width=0.5pt] (0cm,-0.1cm) rectangle (0.3cm,0.2cm); },
                legend to name=eff_leg,
                xmin=1,xmax=5,
                ymin=100,ymax=1000000,
                ytick = {100, 1000, 10000, 100000},
                ymode = log,
                tick align=inside,
                ticklabel style={font=\Huge},
                every axis plot/.append style={line width = 2.5pt},
                every axis/.append style={line width = 2.5pt},
                ylabel={\textbf{\Huge time (s)}}
                ]
% \addplot[pattern=horizontal lines, pattern color=c1] table[x=datasets,y=zero]{\Qtime};
% \addplot[pattern=dots, pattern color=c2] table[x=datasets,y=cot]{\Qtime};
% \addplot[pattern=grid, pattern color=c3] table[x=datasets,y=BM25]{\Qtime};
% \addplot[pattern=crosshatch, pattern color=c4] table[x=datasets,y=vanilla]{\Qtime};
% \addplot[pattern=north west lines, pattern color=c5] table[x=datasets,y=hippo]{\Qtime};
% \addplot[pattern=north west lines, pattern color=c6] table[x=datasets,y=lightl]{\Qtime};
% \addplot[pattern=north west lines, pattern color=c7] table[x=datasets,y=lighth]{\Qtime};
% \addplot[pattern=north west lines, pattern color=c8] table[x=datasets,y=lighthy]{\Qtime};
% \addplot[pattern=vertical lines, pattern color=c9] table[x=datasets,y=graphl]{\Qtime};
% \addplot[pattern=north east lines, pattern color=c10] table[x=datasets,y=graphg]{\Qtime};
% \addplot[pattern=vertical lines, pattern color=c11] table[x=datasets,y=ArchRAG]{\Qtime};

\addplot[fill=c1] table[x=datasets,y=zero]{\Qtime};
\addplot[fill=c2] table[x=datasets,y=cot]{\Qtime};
\addplot[fill=c3] table[x=datasets,y=BM25]{\Qtime};
\addplot[fill=c4] table[x=datasets,y=vanilla]{\Qtime};
\addplot[fill=c5] table[x=datasets,y=hippo]{\Qtime};
\addplot[fill=c6] table[x=datasets,y=lightl]{\Qtime};
% \addplot[pattern=vertical lines, pattern color=c6] table[x=datasets,y=lighth]{\Qtime};
% \addplot[pattern=crosshatch, pattern color=c6] table[x=datasets,y=lighthy]{\Qtime};
\addplot[fill=c11] table[x=datasets,y=lighth]{\Qtime};
\addplot[fill=c12] table[x=datasets,y=lighthy]{\Qtime};
\addplot[fill=c7] table[x=datasets,y=graphl]{\Qtime};
% \addplot[pattern=vertical lines, pattern color=c7] table[x=datasets,y=graphg]{\Qtime};
\addplot[fill=c21] table[x=datasets,y=graphg]{\Qtime};
\addplot[fill=c8] table[x=datasets,y=ArchRAG]{\Qtime};

\legend{\small {Zero-Shot},\small {CoT},\small {BM25}, \small {Vanilla RAG}, \small {HippoRAG}, \small {LightRAG-Low}, \small {LightRAG-High}, \small {LightRAG-Hybrid}, \small {GraphRAG-Local}, \small {GraphRAG-Global}, \small {ArchRAG}}
            \end{axis}
        \end{tikzpicture}
        \label{fig:query-time}
    }
    % 
    \subfigure[Token cost]{
		\begin{tikzpicture}[scale=0.45]
            \begin{axis}[
                ybar=0.5pt,
                bar width=0.6cm,
                width=\textwidth,
                height=0.26\textwidth,
                % xlabel={\Huge \bf datasets}, 
                xtick=data,
                xticklabels={\huge Multihop-RAG,\huge HotpotQA},
                xmin=1,xmax=5,
                ymin=50000,ymax=5000000000,
                % ytick = {0.4, 0.8, 1},
                ytick = {100000, 1000000, 10000000, 100000000, 1000000000},
                % ytick = {100000, 10000000, 1000000000},
                yticklabels = {$10^{-1}$, $10^{0}$, $10^{1}$, $10^{2}$, $10^{3}$},
                ymode = log,
                tick align=inside,
                ticklabel style={font=\Huge},
                every axis plot/.append style={line width = 2.5pt},
                every axis/.append style={line width = 2.5pt},
                ylabel={\textbf{\huge token (M)}}
                ]
% \addplot[pattern=horizontal lines, pattern color=c1] table[x=datasets,y=zero]{\Qtoken};
% \addplot[pattern=dots, pattern color=c2] table[x=datasets,y=cot]{\Qtoken};
% \addplot[pattern=grid, pattern color=c3] table[x=datasets,y=BM25]{\Qtoken};
% \addplot[pattern=crosshatch, pattern color=c4] table[x=datasets,y=vanilla]{\Qtoken};
% \addplot[pattern=north west lines, pattern color=c5] table[x=datasets,y=hippo]{\Qtoken};
% \addplot[pattern=north west lines, pattern color=c6] table[x=datasets,y=lightl]{\Qtoken};
% \addplot[pattern=north west lines, pattern color=c7] table[x=datasets,y=lighth]{\Qtoken};
% \addplot[pattern=north west lines, pattern color=c8] table[x=datasets,y=lighthy]{\Qtoken};
% \addplot[pattern=vertical lines, pattern color=c9] table[x=datasets,y=graphl]{\Qtoken};
% \addplot[pattern=north east lines, pattern color=c10] table[x=datasets,y=graphg]{\Qtoken};
% \addplot[pattern=vertical lines, pattern color=c11] table[x=datasets,y=ArchRAG]{\Qtoken};

\addplot[fill=c1] table[x=datasets,y=zero]{\Qtoken};
\addplot[fill=c2] table[x=datasets,y=cot]{\Qtoken};
\addplot[fill=c3] table[x=datasets,y=BM25]{\Qtoken};
\addplot[fill=c4] table[x=datasets,y=vanilla]{\Qtoken};
\addplot[fill=c5] table[x=datasets,y=hippo]{\Qtoken};
\addplot[fill=c6] table[x=datasets,y=lightl]{\Qtoken};
% \addplot[pattern=vertical lines, pattern color=c6] table[x=datasets,y=lighth]{\Qtoken};
% \addplot[pattern=crosshatch, pattern color=c6] table[x=datasets,y=lighthy]{\Qtoken};
\addplot[fill=c11] table[x=datasets,y=lighth]{\Qtoken};
\addplot[fill=c12] table[x=datasets,y=lighthy]{\Qtoken};
\addplot[fill=c7] table[x=datasets,y=graphl]{\Qtoken};
% \addplot[pattern=vertical lines, pattern color=c7] table[x=datasets,y=graphg]{\Qtoken};
\addplot[fill=c21] table[x=datasets,y=graphg]{\Qtoken};
\addplot[fill=c8] table[x=datasets,y=ArchRAG]{\Qtoken};
            \end{axis}
        \end{tikzpicture}
        \label{fig:query-token}
	}
    \caption{Comparison of query efficiency.}
    \label{fig:query-efficiency}
\end{figure}

Besides, Figure \ref{fig:index-efficiency} shows the index construction time and token usage.
% 
The cost of building index for {ArchRAG} is similar to that of {GraphRAG}, but due to the need for community summarization, both take higher time cost and token usage than {HippoRAG}.

\begin{figure}[h]
    \centering
    \setlength{\abovecaptionskip}{-0.1cm}
    \setlength{\belowcaptionskip}{-0.2cm}
    \quad \ref{eff_index}\\
    \subfigure[Time cost]{
    \begin{tikzpicture}[scale=0.45]
            \begin{axis}[
                ybar=0.5pt,
                bar width=0.6cm,
                width=0.5\textwidth,
                height=0.25\textwidth,
                % xlabel={\Huge \bf datasets}, 
                xtick=data,	
                xticklabels={\huge Multihop-RAG,\huge HotpotQA},
                legend style={
                % at={(0.5,0.78)},
                  at={(0.0,1.05)}, % 将图例放置在图表上方左对齐
    anchor=south west, % 确保图例的左下角对齐图表的左上角
    legend columns=-1, % 使图例按行排列
                align=left, % 确保文字左对齐
                % anchor=north,
                % legend columns=4,
                draw=none},
                legend image code/.code={
                    \draw [#1, line width=0.5pt] (0cm,-0.1cm) rectangle (0.3cm,0.2cm); },
                legend to name=eff_index,
                xmin=1,xmax=5,
                ymin=1000,ymax=100000,
                ytick = {1000,10000,100000},
                ymode = log,
                tick align=inside,
                ticklabel style={font=\Huge},
                every axis plot/.append style={line width = 2.5pt},
                every axis/.append style={line width = 2.5pt},
                ylabel={\textbf{\Huge time (s)}}
                ]
% \addplot[pattern=north west lines, pattern color=c5] table[x=datasets,y=Hippo]{\Indextime};
% \addplot[pattern=north east lines, pattern color=c2] table[x=datasets,y=GraphRAG]{\Indextime};
% \addplot[pattern=vertical lines, pattern color=c11] table[x=datasets,y=ArchRAG]{\Indextime};
\addplot[fill=c5] table[x=datasets,y=Hippo]{\Indextime};
\addplot[fill=c6] table[x=datasets,y=light]{\Indextime};
\addplot[fill=c7] table[x=datasets,y=GraphRAG]{\Indextime};
\addplot[fill=c8] table[x=datasets,y=ArchRAG]{\Indextime};
\legend{\small {HippoRAG},\small {LightRAG},\small {GraphRAG},\small {ArchRAG}}
            \end{axis}
        \end{tikzpicture}
        \label{fig:index-time}
    }
    % 
    \subfigure[Token cost]{
		\begin{tikzpicture}[scale=0.45]
            \begin{axis}[
                ybar=0.5pt,
                bar width=0.6cm,
                width=0.5\textwidth,
                height=0.25\textwidth,
                % xlabel={\Huge \bf datasets}, 
                xtick=data,
                xticklabels={\huge Multihop-RAG,\huge HotpotQA},
                xmin=1,xmax=5,
                ymin=1000000,ymax=100000000,
                ytick = {1000000, 10000000, 100000000},
                yticklabels = {1,10,100},
                ymode = log,
                tick align=inside,
                ticklabel style={font=\Huge},
                every axis plot/.append style={line width = 2.5pt},
                every axis/.append style={line width = 2.5pt},
                ylabel={\textbf{\Huge token (M)}}
                ]
% \addplot[pattern=north west lines, pattern color=c5] table[x=datasets,y=Hippo]{\Indextoken};
% \addplot[pattern=north east lines, pattern color=c10] table[x=datasets,y=GraphRAG]{\Indextoken};
% \addplot[pattern=vertical lines, pattern color=c11] table[x=datasets,y=ArchRAG]{\Indextoken};
\addplot[fill=c5] table[x=datasets,y=Hippo]{\Indextoken};
\addplot[fill=c6] table[x=datasets,y=light]{\Indextoken};
\addplot[fill=c7] table[x=datasets,y=GraphRAG]{\Indextoken};
\addplot[fill=c8] table[x=datasets,y=ArchRAG]{\Indextoken};
            \end{axis}
        \end{tikzpicture}
        \label{fig:index-token}
	}
    \caption{Comparison of indexing efficiency.}
    \label{fig:index-efficiency}
\end{figure}
