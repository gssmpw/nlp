\pgfplotstableread[row sep=\\,col sep=&]{
topk & Accuracy & Recall \\
1 & 63.2 & 37 \\
3 & 66.1 & 36.5 \\
5 & 68.8 & 37.2 \\
7 & 63.4 & 37.1 \\
9 & 67.4 & 37 \\
}\topkmultihop
\pgfplotstableread[row sep=\\,col sep=&]{
topk & Accuracy & Recall \\
1 & 61.5 & 65.6 \\
3 & 61.2 & 65 \\
5 & 65.4 & 69.2 \\
7 & 63.6 & 67.3 \\
9 & 47.7 & 52.5 \\
}\topkhotpot


\subsection{Detailed Analysis}

To answer the question {\it RQ3}, we first evaluate the community qualities under our proposed LLM-based attributed clustering framework and then conduct an ablation study.

% analyze the performance of the proposed community framework within RAG.


$\bullet$ \textbf{Community quality of different clustering methods.} We evaluate 4 different clustering methods with 2 graph augmentation methods. The clustering methods we choose are weighted Leiden, Spectral Clustering~\cite{von2007tutorial}, SCAN~\cite{xu2007scan}, and node2vec~\cite{grover2016node2vec} with KMeans. 
%
The graph augmentation methods we choose are the KNN algorithm, which computes the similarity between each node, and CODICIL~\cite{CODICIL2013efficient}, which selects and adds the top similar edges to generate better clustering results. 
%
We utilize the Calinski-Harabasz (CH) Index \cite{calinski1974CHIndex} and Cosine Similarity \cite{charikar2002similarity}, abbreviated as CHI and Sim respectively, to evaluate the quality of generated attributed communities. 
%
The higher CH Index and Cosine Similarity indicate a higher quality of the community. The details of evaluation metrics can be found in Appendix~\ref{sec:metrics}.
%
As shown in Figures \ref{fig:ch} and \ref{fig:sim}, both KNN and CODICIL help generate high-quality communities well.

\pgfplotstableread[row sep=\\,col sep=&]{
methods & cos & topk  \\ 
2 & 0.8836 & 0.8772 \\
4 & 0.8549 & 0.7791 \\
6 & 0.6963 & 0.7813 \\
8 & 0.7689 & 0.7662 \\
}\simhotpot
\pgfplotstableread[row sep=\\,col sep=&]{
methods & cos & topk  \\ 
2 & 4.8161 & 4.9554 \\
4 & 4.6785 & 1.3950 \\
6 & 1.1961 & 1.5548 \\
8 & 1.0184 & 1.0041 \\
}\chhotpot

\pgfplotstableread[row sep=\\,col sep=&]{
methods & cos & topk  \\ 
2 & 0.8869 & 0.8737 \\
4 & 0.8441 & 0.7856 \\
6 & 0.6941 & 0.7748 \\
8 & 0.7653 & 0.4793 \\
}\simmultihop
\pgfplotstableread[row sep=\\,col sep=&]{
methods & cos & topk  \\ 
2 & 4.6783 & 4.5866 \\
4 & 4.7229 & 1.5135 \\
6 & 1.4620 & 1.4930 \\
8 & 1.0127 & 1.0021 \\
}\chmultihop

\begin{figure}[h]
    \centering
    \setlength{\abovecaptionskip}{-0.1cm}
    \setlength{\belowcaptionskip}{-0.1cm}
    \quad \ref{quality_ch}\\
    \subfigure[Multihop-RAG]{
    \begin{tikzpicture}[scale=0.45]
            \begin{axis}[
                ybar=0.5pt,
                bar width=0.5cm,
                width=0.45\textwidth,
                height=0.25\textwidth,
                % xlabel={\Huge \bf datasets}, 
                xtick=data,	
                xticklabels={{\fontsize{12pt}{14pt}\selectfont Leiden}, {\fontsize{12pt}{14pt}\selectfont Spectral}, {\fontsize{12pt}{14pt}\selectfont SCAN}, {\fontsize{12pt}{14pt}\selectfont Node2Vec}},
                legend style={
                % at={(0.5,0.78)},
                anchor=north,legend columns=4,
                draw=none},
                legend image code/.code={
                    \draw [#1, line width=0.5pt] (0cm,-0.1cm) rectangle (0.3cm,0.2cm); },
                legend to name=quality_ch,
                xmin=1,xmax=9,
                ymin=0,ymax=5.5,
                % ytick = {0,1,2,3,4,5},
                % ymode = log,
                tick align=inside,
                ticklabel style={font=\Huge},
                every axis plot/.append style={line width = 2.5pt},
                every axis/.append style={line width = 2.5pt},
                ylabel={\textbf{{\Huge CHI}}}
                ]
\addplot[fill=c6] table[x=methods,y=cos]{\chmultihop};
\addplot[fill=c3] table[x=methods,y=topk]{\chmultihop};
\legend{\small {KNN},\small {CODICIL}}
            \end{axis}
        \end{tikzpicture}
        \label{fig:ch-multihop}
    }
    % 
    \subfigure[HotpotQA]{
		\begin{tikzpicture}[scale=0.45]
            \begin{axis}[
                ybar=0.5pt,
                bar width=0.5cm,
                width=0.45\textwidth,
                height=0.25\textwidth,
                % xlabel={\Huge \bf datasets}, 
                xtick=data,	
                xticklabels={{\fontsize{12pt}{14pt}\selectfont Leiden}, {\fontsize{12pt}{14pt}\selectfont Spectral}, {\fontsize{12pt}{14pt}\selectfont SCAN}, {\fontsize{12pt}{14pt}\selectfont Node2Vec}},
                legend style={
                % at={(0.5,0.78)},
                anchor=north,legend columns=4,
                draw=none},
                legend image code/.code={
                    \draw [#1, line width=0.5pt] (0cm,-0.1cm) rectangle (0.3cm,0.2cm); },
                xmin=1,xmax=9,
                ymin=0,ymax=5.5,
                % ytick = {0,1,2,3,4,5},
                % ymode = log,
                tick align=inside,
                ticklabel style={font=\Huge},
                every axis plot/.append style={line width = 2.5pt},
                every axis/.append style={line width = 2.5pt},
                ylabel={\textbf{{\Huge CHI}}}
                ]
\addplot[fill=c6] table[x=methods,y=cos]{\chhotpot};
\addplot[fill=c3] table[x=methods,y=topk]{\chhotpot};
            \end{axis}
        \end{tikzpicture}
        \label{fig:ch-hotpotqa}
	}
    \caption{Community quality evaluated by CH Index.}
    \label{fig:ch}
\end{figure}



\begin{figure}[h]
    \centering
    \setlength{\abovecaptionskip}{-0.1cm}
    \setlength{\belowcaptionskip}{-0.2cm}
    \quad \ref{quality_sim}\\
    \subfigure[Multihop-RAG]{
    \begin{tikzpicture}[scale=0.45]
            \begin{axis}[
                ybar=0.5pt,
                bar width=0.5cm,
                width=0.45\textwidth,
                height=0.25\textwidth,
                % xlabel={\Huge \bf datasets}, 
                xtick=data,	
                xticklabels={{\fontsize{12pt}{14pt}\selectfont Leiden}, {\fontsize{12pt}{14pt}\selectfont Spectral}, {\fontsize{12pt}{14pt}\selectfont SCAN}, {\fontsize{12pt}{14pt}\selectfont Node2Vec}},
                legend style={
                % at={(0.5,0.78)},
                anchor=north,legend columns=4,
                draw=none},
                legend image code/.code={
                    \draw [#1, line width=0.5pt] (0cm,-0.1cm) rectangle (0.3cm,0.2cm); },
                legend to name=quality_sim,
                xmin=1,xmax=9,
                ymin=0.4,ymax=1,
                % ytick = {0,1,2,3,4,5},
                % ymode = log,
                tick align=inside,
                ticklabel style={font=\Huge},
                every axis plot/.append style={line width = 2.5pt},
                every axis/.append style={line width = 2.5pt},
                ylabel={\textbf{{\Huge Sim}}}
                ]
\addplot[fill=c6] table[x=methods,y=cos]{\simmultihop};
\addplot[fill=c3] table[x=methods,y=topk]{\simmultihop};
\legend{\small {KNN},\small {CODICIL}}
            \end{axis}
        \end{tikzpicture}
        \label{fig:sim-multihop}
    }
    % 
    \subfigure[HotpotQA]{
		\begin{tikzpicture}[scale=0.45]
            \begin{axis}[
                 ybar=0.5pt,
                bar width=0.5cm,
                width=0.45\textwidth,
                height=0.25\textwidth,
                % xlabel={\Huge \bf datasets}, 
                xtick=data,	
                xticklabels={{\fontsize{12pt}{14pt}\selectfont Leiden}, {\fontsize{12pt}{14pt}\selectfont Spectral}, {\fontsize{12pt}{14pt}\selectfont SCAN}, {\fontsize{12pt}{14pt}\selectfont Node2Vec}},
                legend style={
                % at={(0.5,0.78)},
                anchor=north,legend columns=4,
                draw=none},
                legend image code/.code={
                    \draw [#1, line width=0.5pt] (0cm,-0.1cm) rectangle (0.3cm,0.2cm); },
                xmin=1,xmax=9,
                ymin=0.6,ymax=1,
                ytick={0.6,0.8,1},
                % ymode = log,
                tick align=inside,
                ticklabel style={font=\Huge},
                every axis plot/.append style={line width = 2.5pt},
                every axis/.append style={line width = 2.5pt},
                ylabel={\textbf{{\Huge Sim}}}
                ]
\addplot[fill=c6] table[x=methods,y=cos]{\simhotpot};
\addplot[fill=c3] table[x=methods,y=topk]{\simhotpot};

            \end{axis}
        \end{tikzpicture}
        \label{fig:sim-hotpotqa}
	}
    \caption{Community quality evaluated by Cosine Similarity.}
    \label{fig:sim}
\end{figure}



% $\bullet$ \textbf{Evaluation of community's RAG framework performance.}
$\bullet$ \textbf{Ablation study.}
To evaluate the effect of the attributed community in ArchRAG, we design some variants of ArchRAG and conduct the ablation experiments. 
\begin{itemize}
    \item {Spec.}: Replaces the Weighted Leiden community detection algorithm with spectral clustering.
    \item {Spec. (No Aug)}: Removes the graph augmentation and uses spectral clustering.
    \item {Leiden}: Replaces our clustering framework with the Leiden algorithm.
    \item {Single-Layer}: Replaces our hierarchical index and search with a single-layer community.
    \item {Entity-Only}: Generate the response using entities only.
    \item {Direct Prompt}: Direct prompts the LLM to generate the response without the adaptive filtering-based generation.
\end{itemize}

The first two variants replace the algorithm in the LLM-based hierarchical clustering framework, while the last three variants modify key components of {ArchRAG}: attributes, hierarchy, and communities. 
% 
As shown in Table \ref{tab:variant}, the performance of {ArchRAG} on specific QA tasks decreases when each feature is removed, with the removal of the community component resulting in the most significant drop.
% 
Additionally, the Direct variant demonstrates that the adaptive filtering-based generation process can effectively extract relevant information from retrieved elements.


\begin{table}[h]
    \centering
\caption{Comparing the performance of different variants of ArchRAG on the specific QA tasks.}
    \small
    \begin{tabular}{lcccc}
        \toprule
\multirow{2}{*}{Method variants} &  \multicolumn{2}{c}{Multihop-RAG} & \multicolumn{2}{c}{HotpotQA} \\
\cmidrule(lr){2-3} \cmidrule(lr){4-5}
        & (Accuracy) & (Recall) & (Accuracy) & (Recall)\\
\midrule
{ ArchRAG}  & 68.8 & 37.2 & 65.4 & 69.2 \\
\footnotesize{ - Spec.}  & 67.1 & 36.7 & 60.5 & 63.7 \\
\footnotesize{ - Spec. (No Aug)}  & 62.8 & 34.2 & 64.8 & 63.2 \\
\footnotesize{ - Leiden}  & 63.2 & 34.1 & 61.7 & 64.8 \\
\footnotesize{ - Single-Layer}  & 63.8 & 36.4 & 60.1 & 63.6 \\
\footnotesize{ - Entity-Only}  & 61.2 & 34.7 & 59.9 & 63.1 \\
\footnotesize{ - Direct Prompt}  & 59.9 & 29.6 & 40.7 & 45.4 \\
        \bottomrule
    \end{tabular}
    \label{tab:variant}
\end{table}

In addition, we evaluate the performance of ArchRAG under different LLM backbones and various top-k retrieval content to address question {\it RQ4}.

$\bullet$ \textbf{Effect of LLM backbones.}
As strong LLMs with hundreds of billions of parameters (e.g., GPT-3.5-turbo) possess enhanced capabilities, our proposed ArchRAG may also benefit from performance improvement.
% 
As shown in Table \ref{tab:backbone}, the results of Llama 3.1-8B are similar to those of GPT-3.5-turbo, as Llama3.1's capabilities are comparable to those of GPT-3.5-turbo \cite{dubey2024llama}.
%
Besides, GPT-4o-mini performs better than other LLM backbones because of its exceptional reasoning capabilities.

\begin{table}[h]
    \centering
    \caption{Comparing ArchRAG with other RAG methods on the specific QA tasks under different LLM backbone models.}
    %\caption{The specific QA performance comparison of ArchRAG with different LLM backbone models.}
    \small
    \begin{tabular}{lcccc}
        \toprule
\multirow{2}{*}{LLM backbone} &  \multicolumn{2}{c}{Multihop-RAG} & \multicolumn{2}{c}{HotpotQA} \\
\cmidrule(lr){2-3} \cmidrule(lr){4-5}
        & (Accuracy) & (Recall) & (Accuracy) & (Recall)\\
\midrule
Llama3.1-8B &  68.8 & 37.2 & 65.4 & 69.2 \\
GPT-3.5-turbo & 67.2 & 31.5 & 62.8 & 65.0 \\
GPT-4o-mini & 77.3 & 33.8 & 69.9 & 73.8 \\
        \bottomrule
    \end{tabular}
    \label{tab:backbone}
\end{table}




$\bullet$ \textbf{Effect of $k$ values.}
We compare the performance of ArchRAG under different retrieved elements.
% 
As shown in Figure \ref{fig:topk}, the performance of ArchRAG shows little variation when selecting different retrieval elements (i.e., communities and entities in each layer). 
% 
This suggests that the adaptive filtering process can reliably extract the most relevant information from the retrieval elements and integrate it to generate the answer.

\begin{figure}[h]
    \centering
    \setlength{\abovecaptionskip}{-0.05cm}
    \setlength{\belowcaptionskip}{-0.1cm}

    \quad \ref{topk_leg} \\
    \vspace{-5pt}
    \subfigure[Multihop-RAG]{
		\begin{tikzpicture}[scale=0.45]
			\begin{axis}[
                legend style = {
                    legend columns=-1,
                    inner sep=0pt,
                    draw=none,
                },
legend image post style={scale=1.1, line width=1pt},
                legend to name=topk_leg,
                xmin=-1, xmax=11,
				ymin=20, ymax=80,
				xtick = {1,3,5,7,9},
				xticklabels = {2,3,5,7,9},
				mark size=6.0pt, 
				width=0.5\textwidth,
                height=0.25\textwidth,
ylabel={\Huge \bf Metrics},
				% xlabel={\Huge \bf percentage}, 
ticklabel style={font=\Huge},
every axis plot/.append style={line width = 2.5pt},
every axis/.append style={line width = 2.5pt},
				]
\addplot [mark=triangle,color=c6] table[x=topk,y=Accuracy]{\topkmultihop};
\addplot [mark=o,color=c3] table[x=topk,y=Recall]{\topkmultihop};
\legend{\small Accuracy,\small Recall}
			\end{axis}
		\end{tikzpicture}
	}
	%
    \subfigure[HotpotQA]{
		\begin{tikzpicture}[scale=0.45]
			\begin{axis}[
                xmin=-1, xmax=11,
				ymin=20, ymax=80,
				xtick = {1,3,5,7,9},
				xticklabels = {2,3,5,7,9},
				mark size=6.0pt, 
				width=0.5\textwidth,
                height=0.25\textwidth,
ylabel={\Huge \bf Metrics},
ticklabel style={font=\Huge},
every axis plot/.append style={line width = 2.5pt},
every axis/.append style={line width = 2.5pt},
				]
\addplot [mark=triangle,color=c6] table[x=topk,y=Accuracy]{\topkhotpot};
\addplot [mark=o,color=c3] table[x=topk,y=Recall]{\topkhotpot};
			\end{axis}
		\end{tikzpicture}
	}
	%
    \caption{Comparative analysis of the different numbers of retrieval elements in ArchRAG.}
    \label{fig:topk}
\end{figure}












% $\bullet$ \textbf{Evaluation of HCHNSW.}
% % efficieny + recall
% \textbf{draft} 
% To evaluate the performance of the HCHNSW, we use the top searching results of level 0 and report the searching time and recall of both HCHNSW and HNSW. As shown in 
