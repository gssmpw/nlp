\section{Analysis of SOTA Graph-based RAG methods}
\label{sec:motivation}


% 1. 现有方法的总结
% HippoRAG 和GraphRAG
% 分析具体做法,透露出

% 2. Limitation 检索框架的问题:
% 指定模式的搜索策略(子图、连通算法等)只能适用于特殊范式的问题,如多跳、集合融合等。
% 只能关注与局部信息,无法适用于应用全局信息的任务。
% GraphRAG采用community-based RAG,但指定某一层的社区,推理过程中资源消耗极大,如实验所示。



In this section, we extensively review the SOTA graph-based RAG methods, HippoRAG \cite{gutierrez2024hipporag} and GraphRAG \cite{edge2024local}.
% 
Both methods adopt an offline index and online retrieval framework.
% , with HippoRAG designed for basic QA tasks and GraphRAG tailored for global sensemaking tasks.


HippoRAG \cite{gutierrez2024hipporag} adopts a Personalized PageRank algorithm (PPR) to rank the retrieved candidate information for filtering relevant chunks.
% 
Specifically, in the offline index, HippoRAG first constructs a schemaless KG by extracting entities and relations from original documents.
% 
Then they add synonymy relations for nodes whose text's cosine similarity exceeds a certain threshold.
% 
This process stimulates the relevance-based connectivity in hippocampal memory indexing theory \cite{teyler1986hippocampal}.
% 
During online retrieval, HippoRAG begins by extracting entities from the question and uses these entities as seeds to run the PPR algorithm, filtering the most relevant entities.
% 
It then maps filtered entities to the most relevant original chunks based on the frequency and integrates them to prompt the LLM to generate a response.

GraphRAG \cite{edge2024local} utilizes LLMs to generate report-like summaries for each detected graph community.
% 
Communities, which represent dense clusters of nodes within a graph network, have been widely applied across various domains \cite{dudleyExploitingDrugdiseaseRelationships2011,pesantez-cabreraEfficientDetectionCommunities2019,fortunatoCommunityDetectionGraphs2010,fangSurveyCommunitySearch2019,fortunatoCommunityDetectionNetworks2016,newmanFindingEvaluatingCommunity2004}. 
% 
A summarized community combines specific details, providing a global perspective and making it suitable for the abstract QA task.
% 
In the offline phase, GraphRAG uses the Leiden \cite{traag2019louvain} algorithm to construct hierarchical communities and generate a summary for each community.
% 
During the query phase, GraphRAG selects specific layers of communities, uses the LLM to analyze the relevance of each community to the query, and assigns a score. 
% 
It then integrates the content with the highest score to generate the final response.


{\bf Limitations. } Existing methods either focus on local information or are constrained to tasks that require global information summarization.
% 
These methods either retrieve text chunks related to entities only or use all communities, limiting them to a specific task.
% 
Additionally, HippoRAG does not leverage LLMs to analyze the retrieved content, not only restricting its ability to integrate more comprehensive information but also providing redundant data to the LLM.
% 
GraphRAG adopts a community-based RAG approach and employs LLM analysis, but specifying specific layers of all communities for generation leads to significant resource consumption during the online inference process, as demonstrated in our experiments.
% 
Furthermore, GraphRAG directly constructs communities using the Leiden algorithm, which only considers the graph structure while neglecting the attribute information associated with the entities.

