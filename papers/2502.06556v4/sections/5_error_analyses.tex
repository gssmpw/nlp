\subsection{Error Analyses}
\label{sec: error analyses}
% \tikzstyle{my-box}=[
    rectangle,
    draw=hidden-draw,
    rounded corners,
    text opacity=1,
    minimum height=1.5em,
    minimum width=5em,
    inner sep=2pt,
    align=center,
    fill opacity=.5,
    line width=0.8pt,
]
\tikzstyle{leaf}=[my-box, minimum height=1.5em,
    fill=hidden-pink!80, text=black, align=left,font=\normalsize,
    inner xsep=2pt,
    inner ysep=4pt,
    line width=0.8pt,
]
\begin{figure}[]
    \centering
    \resizebox{\linewidth}{!}{
        \begin{forest}
            forked edges,
            for tree={
                grow'=0,
                draw,
                reversed=true,
                anchor=base west,
                parent anchor=east,
                child anchor=west,
                base=left,
                font=\large,
                rectangle,
                rounded corners,
                align=left,
                minimum width=4em,
                edge+={darkgray, line width=1pt},
                s sep=3pt,
                inner xsep=2pt,
                inner ysep=3pt,
                line width=0.8pt,
                ver/.style={rotate=90, child anchor=north, parent anchor=south, anchor=center},
            },
            where level=1{text width=4.4em,font=\normalsize,}{},
            where level=2{text width=12em,font=\normalsize,}{},
            where level=3{text width=25em,font=\normalsize,}{},
            % where level=4{text width=5em,font=\normalsize,}{},
			[
			    Compilation Error Analysis, ver
			    [
		              Python, 
                        fill=lgreen
    			            [
                                Confuse between non-package and package-based projects
                                , leaf, text width=28em, fill=lgreen
    			            ]
                                [
                                Hallucinate the imported functions/classes:\\
                                1. Paths of the imported functions/classes are wrong\\
                                2. Names of the imported functions/classes are wrong
                                , leaf, text width=28em, fill=lgreen
    			            ]
                                [
                                Syntax Error: Mismatched parentheses
                                , leaf, text width=28em, fill=lgreen
                                ]
			        ]
			    [
    			      Java, fill=lblue
    			            [
                                Hallucinate methods/constructors/functions/classes:\\
                                1. Paths of the imported functions/classes are wrong \\
                                2. Names of the imported functions/classes are wrong \\
                                3. Non-existed methods/constructors
                                , leaf, text width=28em, fill=lblue
    			            ]
                                [
                                Missing information: \\
                                1. Required functions/classes/packages are missing \\
                                2. Required package information is missing \\
                                3. Unreported exception \\
                                , leaf, text width=28em, fill=lblue
                                ]
                                [
                                Illegal access to private/protected functions/classes
                                , leaf, text width=28em, fill=lblue
                                ]
                                [
                                Invalid generation: \\
                                1. Generate textual instructions instead of codes
                                2. Block by model \\
                                , leaf, text width=28em, fill=lblue
                                ]
                                [
                                Incorrect use of mocking: \\
                                1. Wrong objects provided to Mockito \\
                                2. Missing MockMvc injection 
                                3. Inappropriate mockmvc \\
                                4. Argument mismatch
                                , leaf, text width=28em, fill=lblue
                                ]
                                [
                                Incorrect use of other functions/classes/packages: \\
                                1. Arguments type error 2. Ambiguous reference \\
                                3. Incompatible types
                                , leaf, text width=28em, fill=lblue
                                ]
			    ]
                    [
                        JavaScript, fill=lyellow[
                                Hallucinate the imported functions/classes: \\
                                1. Paths of the imported functions/classes are wrong
                                , leaf, text width=28em, fill=lyellow
                                ]
                                [
                                Invalid generation: \\
                                1. Cannot understand the prompt
                                2. Require more/specific codes \\
                                3. Assume the codes are part of a larger project and \\ decline to generate unit tests
                                , leaf, text width=28em, fill=lyellow
                                ]
                                [
                                Test suits have empty unit tests \\
                                , leaf, text width=28em, fill=lyellow
                                ]
                                [
                                Syntax Error: \\
                                1. Incomplete generation 
                                2. Mismatched parentheses
                                , leaf, text width=28em, fill=lyellow
                                ]
                    ]
			]
            \end{forest}
    }
    \caption{Frequent Compilation Errors in Main Results.}
    \label{fig: errors1}
\end{figure}


\tikzstyle{my-box}=[
    rectangle,
    draw=hidden-draw,
    rounded corners,
    text opacity=1,
    minimum height=1.5em,
    minimum width=5em,
    inner sep=2pt,
    align=center,
    fill opacity=.5,
    line width=0.8pt,
]
\tikzstyle{leaf}=[my-box, minimum height=1.5em,
    fill=hidden-pink!80, text=black, align=left,font=\normalsize,
    inner xsep=2pt,
    inner ysep=4pt,
    line width=0.8pt,
]
\begin{figure}[t]
    \centering
    \resizebox{\linewidth}{!}{
        \begin{forest}
            forked edges,
            for tree={
                grow'=0,
                draw,
                reversed=true,
                anchor=base west,
                parent anchor=east,
                child anchor=west,
                base=left,
                font=\large,
                rectangle,
                rounded corners,
                align=left,
                minimum width=4em,
                edge+={darkgray, line width=1pt},
                s sep=3pt,
                inner xsep=2pt,
                inner ysep=3pt,
                line width=0.8pt,
                ver/.style={rotate=90, child anchor=north, parent anchor=south, anchor=center},
            },
            where level=1{text width=4.4em,font=\normalsize,}{},
            where level=2{text width=12em,font=\normalsize,}{},
            where level=3{text width=20em,font=\normalsize,}{},
            % where level=4{text width=5em,font=\normalsize,}{},
			[
			    Cascade Error Analysis, ver
			    [
		              Python, 
                        fill=lgreen
    			            [
                                Required functions/classes/libraries are missing:\\
                                1. Import numpy or unittest.mock\\
                                2. Import functions/classes of the tested project
                                , leaf, text width=25em, fill=lgreen
    			            ]
    			            [
                                FileNotFoundError
                                , leaf, text width=25em, fill=lgreen
    			            ]
			        ]
			    [
    			      Java, fill=lblue
    			            [
                                Missing/Invalid mock of user interactions
                                , leaf, text width=25em, fill=lblue
    			            ]
			    ]
                    [
                        JavaScript, fill=lyellow
                                [
                                Required functions/classes/libraries are missing:\\
                                1. Import chai or three\\
                                2. Import functions/classes of the tested project
                                , leaf, text width=25em, fill=lyellow
                                ]
                                [
                                Confuse between name import and default import
                                , leaf, text width=25em, fill=lyellow
                                ]
                                [
                                Do not follow the Jest framework
                                , leaf, text width=25em, fill=lyellow
                                ]
                    ]
			]
            \end{forest}
    }
    \caption{Frequent Cascade Errors.}
    \label{fig: errors2}
\end{figure}


\tikzstyle{my-box}=[
    rectangle,
    draw=hidden-draw,
    rounded corners,
    text opacity=1,
    minimum height=1.5em,
    minimum width=5em,
    inner sep=2pt,
    align=center,
    fill opacity=.5,
    line width=0.8pt,
]
\tikzstyle{leaf}=[my-box, minimum height=1.5em,
    fill=hidden-pink!80, text=black, align=left,font=\normalsize,
    inner xsep=2pt,
    inner ysep=4pt,
    line width=0.8pt,
]
\begin{figure}[]
    \centering
    \resizebox{\linewidth}{!}{
        \begin{forest}
            forked edges,
            for tree={
                grow'=0,
                draw,
                reversed=true,
                anchor=base west,
                parent anchor=east,
                child anchor=west,
                base=left,
                font=\large,
                rectangle,
                rounded corners,
                align=left,
                minimum width=4em,
                edge+={darkgray, line width=1pt},
                s sep=3pt,
                inner xsep=2pt,
                inner ysep=3pt,
                line width=0.8pt,
                ver/.style={rotate=90, child anchor=north, parent anchor=south, anchor=center},
            },
            where level=1{text width=4.4em,font=\normalsize,}{},
            where level=2{text width=12em,font=\normalsize,}{},
            where level=3{text width=25em,font=\normalsize,}{},
            % where level=4{text width=5em,font=\normalsize,}{},
			[
			    Post-fix Error Analysis, ver
			    [
		              Python, 
                        fill=lgreen
                                [
                                1. AttributeError
                                2. AssertionError
                                3. TypeError 
                                4. ValueError \\
                                5. IndexError 
                                6. \_csv.Error 
                                7. NameError
                                8. KeyError 
                                9. Others
                                , leaf, text width=30em, fill=lgreen
    			            ]
			        ]
			    [
    			      Java, fill=lblue
    			            [
                                1. Mismatch between expected and received 
                                2. NullPointer Error \\
                                3. Zero interactions with mock 
                                4. Failed to release mocks \\
                                5. MissingMethodInvocation 
                                6. Misplaced or misused argument matcher \\
                                7. Spring framework error 
                                8. NoSuchElement 
                                9. Others
                                , leaf, text width=30em, fill=lblue
    			            ]
			    ]
                    [
                        JavaScript, fill=lyellow
                                [
                                1. Mismatch between expected and received 
                                2. TypeError 
                                3. RangeError \\
                                4. RuntimeError
                                5. ReferenceError 
                                6. SyntaxError 
                                7. Others
                                % 7. Invalid component type 
                                % 8. Image given has not completed loading 
                                % 9. Invalid Chai property
                                , leaf, text width=30em, fill=lyellow
                                ]
                    ]
			]
            \end{forest}
    }
    \caption{Frequent Post-Fix Errors.}
    \label{fig: errors3}
\end{figure}

We conduct complex analyses of compilation, cascade, and post-fix errors per programming language, highlighting the common errors and potential reasons behind the errors. The full analyses are presented in Appendix~\ref{sec: full_error_analyses}.


\noindent\textbf{Compilation Error Analyses. }
% \input{sections/5_figure_main_results}
% Figure~\ref{fig: errors1} highlights the detailed compilation errors that occurred.
% One of the most common compilation errors in \textbf{\textit{Python}} arises from the LLM's inability to 
% % determine whether the project being tested is a package. Specifically, LLMs struggle to recognize the presence or absence of \textit{\_\_init\_\_.py} files, which define a package, leading to confusion between package-based and non-package projects. This inability leads LLM to fail to 
% correctly import functions or classes from the tested project, often using incorrect import paths. Other compilation errors include hallucinating the paths or names of imported functions/classes and mismatched parentheses.
In \textbf{\textit{Python}}, common compilation errors arise from incorrect import paths for project functions or classes, hallucinated import names or paths, and mismatched parentheses.
\textbf{\textit{Java}}, a syntax-heavy programming language compared to Python and JavaScript, encounters various compilation errors, 
% resulting in a significantly lower compilation rate than other languages. 
% Java compilation errors often arise from issues like hallucinated methods, constructors, or classes, such as incorrect or non-existent imports and references. Missing essential information, such as required functions, classes, or packages, and package declarations, is also a common problem. Errors frequently occur due to illegal access to private or protected elements, invalid code generation (e.g., generating text instead of code), and improper use of mocking frameworks like Mockito, including incorrect objects, missing or misused MockMvc injections, and argument mismatches. Other errors include incorrect usage of other functions, classes, or packages—such as argument type errors, ambiguous references, or incompatible types.
% Java compilation errors commonly arise from hallucinated methods, constructors, or classes, as well as missing essential elements like package declarations, illegal access to private or protected elements, and invalid code generation. Errors also arise from improper use of mocking frameworks like Mockito, along with argument type mismatches, ambiguous references, and incompatible types.
like hallucinated methods, constructors, or classes, missing essential elements like package declarations, illegal access to private or protected elements, invalid code generation, and improper use of mocking frameworks like Mockito, along with argument type mismatches, ambiguous references, and incompatible types.
% One of the most common compilation errors in \textbf{\textit{JavaScript}} is the hallucination of imported functions or classes, where the issue often lies in incorrect paths for the imported functions or classes. 
% % CodeQwen1.5 has a particularly common compilation error involving invalid generation. This typically occurs due to difficulty understanding the prompt, the need for more specific or detailed code requirements, or the assumption that the code is part of a larger project, leading it to decline generating unit tests. 
% Other compilation errors include test suites containing empty unit tests and syntax errors caused by incomplete code generation or mismatched parentheses.
In \textbf{\textit{JavaScript}}, common errors include hallucinated imports with incorrect paths, empty test suites, and syntax errors from incomplete code generation or mismatched parentheses.

\noindent\textbf{Cascade Error Analyses. }
% Figure~\ref{fig: errors2} highlights the detailed cascade errors that occurred.
For \textbf{\textit{Python}}, 
% the cascade errors include missing imports of commonly used packages such as numpy and unittest, missing imports of functions or classes from the tested project, and FileNotFoundError. 
cascade errors include missing imports (e.g., numpy, unittest, project functions/classes) and FileNotFoundError due to unmocked external files.
% The FileNotFoundError indicates that the generated unit tests fail to mock the external files.
For \textbf{\textit{Java}}, the most common cascade error 
% is missing or invalid mocking of user interactions. A proper unit test should simulate user interactions through mocking rather than relying on real user inputs. This issue also results in unusable coverage reports for some tested projects, as the error forces an abrupt termination, preventing the generation of coverage data.
is improper or missing mocking of user interactions, leading to unusable coverage reports when tests terminate abruptly.
For \textbf{\textit{JavaScript}}, the cascade errors include 
% missing imports of commonly used packages such as chai and three, and missing imports of functions or classes from the tested project. Two other common errors specific to JavaScript are that LLMs may confuse named imports with default imports and fail to comply with the Jest framework. 
missing imports (e.g., chai, three, project functions/classes), confusion between named and default imports, and Jest framework compliance issues.


\noindent\textbf{Post-Fix Error Analyses. }
% Figure~\ref{fig: errors3} highlights the incorrectness reasons after all manual fixes.
For all programming languages, the mismatch between expected and actual values is the most common error.
% Another frequent error in \textbf{\textit{Python}} is AttributeError, typically caused by LLMs hallucinating non-existent attributes.
In \textbf{\textit{Python}}, AttributeError often occurs due to LLMs hallucinating non-existent attributes.
% Other frequent problems in \textbf{\textit{Java}} include NullPointer Errors, zero interactions with mocks, and failures to release mocks, often due to improper mock usage. 
In \textbf{\textit{Java}}, frequent errors include NullPointerException, zero interactions with mocks, and failures to release mocks due to improper usage.
% For projects tested with the Spring framework, errors specific to Spring are also common.
Another frequent error in \textbf{\textit{JavaScript}} is TypeError, typically caused by LLMs hallucinating non-existent functions and constructors or LLMs invalidly mocking some variables.

\noindent\textbf{Overall. }
Common errors across different programming languages include hallucinations of functions or classes, and missing required functions or classes. 
%\textit{possibly caused by training data bias and a lack of proper references}.
% Missing required functions/classes is another common error (Compilation error for Java and Cascade error for Python and JavaScript).
Missing required functions or classes often occurs because LLMs \textit{prioritize logical structure over boilerplate code} and \textit{fail to understand the codebase structure and the dependencies between functions, classes, or modules}. Failure to understand the codebase structure and dependencies can also cause other mistakes, such as confusing non-package and package-based projects (Python) or incorrectly using functions, classes, or packages (Java).
% For after-fix errors, the mismatch between expected and received values is the most common error. 
The most common post-fix error is the mismatch between expected and received values, often caused by incorrect expected values due to the \textit{weak reasoning abilities} of LLMs.

