\section{Dataset}
\label{appendix: dataset}
We provide the detailed information of our datasets in Table~\ref{tab: individual_dataset_py}, Table~\ref{tab: individual_dataset_java}, and Table~\ref{tab: individual_dataset_js}. We provide programming language, project name, license, link, number of stars, and number of forks for each individual project.
\begin{table}[t]
\caption{Dataset Details (Python).}
\small
\resizebox{\linewidth}{!}{% <-
\begin{tabular}{llllllll}
\hline
 Project Name & License & Link & \#Stars & \#Forks \\
\hline
blackjack &MIT license &\href{https://github.com/datamllab/rlcard/tree/master/rlcard/games/blackjack}{blackjack} &2937	&641 \\
bridge &MIT license &\href{https://github.com/datamllab/rlcard/tree/master/rlcard/games/bridge}{bridge} &2937 &641 \\
doudizhu &MIT license &\href{https://github.com/datamllab/rlcard/tree/master/rlcard/games/doudizhu}{doudizhu} &2937 &641 \\
fuzzywuzzy &MIT license &\href{https://github.com/seatgeek/fuzzywuzzy/tree/master/fuzzywuzzy}{fuzzywuzzy} &9200 &876 \\
gin\_rummy &GPL-2.0 license &\href{https://github.com/datamllab/rlcard/tree/master/rlcard/games/gin_rummy}{gin\_rummy} &2937 &641 \\
keras\_preprocessing &MIT license &\href{https://github.com/keras-team/keras-preprocessing/tree/master/keras_preprocessing}{keras\_preprocessing} &1024 &443 \\
leducholde &MIT license &\href{https://github.com/datamllab/rlcard/tree/master/rlcard/games/leducholde}{leducholde} &2937 &641 \\
limitholdem &MIT license &\href{https://github.com/datamllab/rlcard/tree/master/rlcard/games/limitholdem}{limitholdem} &2937 &641 \\
mahjong &MIT license &\href{https://github.com/datamllab/rlcard/tree/master/rlcard/games/mahjong}{mahjong} &2937 &641 \\
nolimitholdem &MIT license &\href{https://github.com/datamllab/rlcard/tree/master/rlcard/games/nolimitholdem}{nolimitholdem} &2937 &641 \\
slugify &MIT license &\href{https://github.com/un33k/python-slugify/tree/master/slugify}{slugify} &1500 &109 \\
stock &CC-BY-SA-4.0 license &\href{https://github.com/dabeaz-course/python-mastery/tree/main/Solutions/7_3}{stock} &10700 &1800 \\
stock2 &CC-BY-SA-4.0 license &\href{https://github.com/dabeaz-course/python-mastery/tree/main/Solutions/7_6}{stock2} &10700 &1800 \\
stock3 &CC-BY-SA-4.0 license &\href{https://github.com/dabeaz-course/python-mastery/tree/main/Solutions/8_1}{stock3} &10700 &1800 \\
stock4 &CC-BY-SA-4.0 license &\href{https://github.com/dabeaz-course/python-mastery/tree/main/Solutions/8_2}{stock4} &10700 &1800 \\
structly &CC-BY-SA-4.0 license &\href{https://github.com/dabeaz-course/python-mastery/tree/main/Solutions/9_2}{structly} &10700 &1800 \\
svm &MIT license &\href{https://github.com/rushter/MLAlgorithms/tree/master/mla/svm}{svm} &10800 &1800 \\
the fuzz &CC-BY-SA-4.0 license &\href{https://github.com/seatgeek/thefuzz/tree/master/thefuzz}{the fuzz} &2949 &141 \\
tree &CC-BY-SA-4.0 license &\href{https://github.com/rushter/MLAlgorithms/blob/master/mla/ensemble/tree.py}{tree} &10800 &1800 \\
uno &MIT license &\href{https://github.com/datamllab/rlcard/tree/master/rlcard/games/uno}{uno} &2937 &641 \\
\hline
\end{tabular}
}
\label{tab: individual_dataset_py}
\end{table}

\begin{table}[t]
\caption{Dataset Details (Java).}
\small
\resizebox{\linewidth}{!}{% <-
\begin{tabular}{llllllll}
\hline
Project Name & License & Link & \#Stars & \#Forks \\
\hline
Actor\_relationship\_game & Apache-2.0 license &\href{https://github.com/open-compass/DevEval/tree/main/benchmark_data/java/Actor_relationship_game/src/main/java/Actor_relationship_game}{Actor\_relationship\_game} &85	&5 \\
banking application &MIT license &\href{https://github.com/kishanrajput23/Java-Projects-Collections/tree/main/banking\%20application}{banking application} &341 &366 \\
Calculator\-OOPS &MIT license &\href{https://github.com/kishanrajput23/Java-Projects-Collections/tree/main/Calculator-OOPS}{Calculator\-OOPS} &525 &513 \\
% Email\-Administration\-Application & -&\href{https://github.com/KrishGaur1354/Java-Projects-for-Beginners/tree/main/Email-Administration-Application}{Email\-Administration\-Application} &33 &17 \\
emailgenerator &MIT license &\href{https://github.com/kishanrajput23/Java-Projects-Collections/tree/main/Email_Generator/src/emailgenerator}{emailgenerator} &525 &513 \\
heap &MIT license &\href{https://github.com/TheAlgorithms/Java/tree/5ab6356090c17cddd953c801eac4abb6ef48c9f1/src/main/java/com/thealgorithms/datastructures/heaps}{heap} &60500 &19600 \\
idcenter &Apache-2.0 license &\href{https://github.com/adyliu/idcenter}{idcenter} &146 &136 \\
libraryApp &MIT license &\href{https://github.com/kishanrajput23/Java-Projects-Collections/tree/main/LibraryApp/libraryApp}{libraryApp} &341 &366 \\
libraryManagement &MIT license &\href{https://github.com/kishanrajput23/Java-Projects-Collections/tree/main/LibraryMangement/src}{libraryManagement} &341 &366 \\
logrequestresponseundertow &Author Permission  &\href{https://github.com/frandorado/spring-projects/tree/master/log-request-response-undertow}{logrequestresponseundertow} &152 &131 \\
Password\_Generator &MIT license &\href{https://github.com/kishanrajput23/Java-Projects-Collections/tree/main/Password_Generator/Password\%20Generator/src}{Password\_Generator} &341 &366 \\
Pong Game &MIT license &\href{https://github.com/kishanrajput23/Java-Projects-Collections/tree/main/Pong\%20Game}{Pong Game} &341 &366 \\
redis &Apache-2.0 license &\href{https://github.com/mybatis/redis-cache}{redis} &413 &218 \\
servlet &MIT license &\href{https://github.com/kishanrajput23/Java-Projects-Collections/tree/main/Online\%20Voting\%20System/Online_Voting_System/src/main/java/vote/com/servlet}{servlet} &341 &366 \\
simpleChat &MIT license &\href{https://github.com/abhpd/hacktoberfest2021/tree/main/Java/Projects/SimpleChat}{simpleChat} &543 &1500 \\
springdatamongowithcluster &Author Permission &\href{https://github.com/frandorado/spring-projects/tree/master/spring-data-mongo-with-cluster}{springdatamongowithcluster} &152 &131 \\
springmicrometerundertow &Author Permission &\href{https://github.com/frandorado/spring-projects/tree/master/spring-micrometer-undertow}{springmicrometerundertow} &152 &131 \\
springreactivenonreactive &Author Permission &\href{https://github.com/frandorado/spring-projects/tree/master/spring-reactive-nonreactive}{springreactivenonreactive} &152 &131 \\
springuploads3 &Author Permission &\href{https://github.com/frandorado/spring-projects/tree/master/spring-upload-s3-localstack}{springuploads3} &152 &131 \\
Train &MIT license &\href{https://github.com/abhpd/hacktoberfest2021/tree/main/Java/Projects/Train}{Train} &545 &1600 \\
\hline
\end{tabular}
}
\label{tab: individual_dataset_java}
\end{table}

The license of "Author Permission" in Table \ref{tab: individual_dataset_java} means that we obtain the usage permission from the author of the corresponding repository\footnote{https://github.com/frandorado/spring-projects/tree/master}.


\begin{table}[t]
\caption{Dataset Details (JavaScript).}
\small
\resizebox{\linewidth}{!}{% <-
\begin{tabular}{llllllll}
\hline
 Project Name & License & Link & \#Stars & \#Forks \\
\hline
aggregate &MIT license &\href{https://github.com/ehmicky/modern-errors/blob/main/src/merge/aggregate.js}{aggregate} &1500	&18 \\
animation &MIT license &\href{https://github.com/mrdoob/three.js/blob/dev/src/animation/AnimationAction.js}{animation} &103000 &35400 \\
check &MIT license &\href{https://github.com/ehmicky/modern-errors/blob/main/src/subclass/check.js}{check} &1500 &18 \\
circle &MIT license &\href{https://github.com/schteppe/p2.js/blob/master/src/shapes/Circle.js}{circle} &2700 &330 \\
ckmeans &ISC license &\href{https://github.com/simple-statistics/simple-statistics/blob/main/src/ckmeans.js}{ckmeans} &3400 &226 \\
controls &MIT license &\href{https://github.com/mrdoob/three.js/blob/dev/src/extras/Controls.js}{controls} &103000 &35400 \\
convex &MIT license &\href{https://github.com/schteppe/p2.js/blob/master/src/shapes/Convex.js}{convex} &2700 &330 \\
easing &MIT license &\href{https://github.com/alienkitty/space.js/blob/main/src/tween/Easing.js}{easing} &418 &9 \\
magnetic &MIT license &\href{https://github.com/alienkitty/space.js/blob/main/src/extras/Magnetic.js}{magnetic} &418 &9 \\
overlapkeeper &MIT license &\href{https://github.com/schteppe/p2.js/blob/master/src/utils/OverlapKeeper.js}{overlapkeeper} &2700 &330 \\
particle &MIT license &\href{https://github.com/schteppe/p2.js/blob/master/src/shapes/Particle.js}{particle} &2700 &330 \\
pixelrender &MIT license &\href{https://github.com/drawcall/Proton/blob/master/src/render/PixelRenderer.js}{pixelrender} &2400 &274 \\
plane &MIT license &\href{https://github.com/schteppe/p2.js/blob/master/src/shapes/Plane.js}{plane} &2700 &330 \\
solver &MIT license &\href{https://github.com/schteppe/p2.js/blob/master/src/solver/Solver.js}{solver} &2700 &330 \\
span &MIT license &\href{https://github.com/drawcall/Proton/blob/master/src/math/Span.js}{span} &2400 &274 \\
spherical &MIT license &\href{https://github.com/mrdoob/three.js/blob/dev/src/math/Spherical.js}{spherical} &103000 &35400 \\
synergy &MIT license &\href{https://github.com/defx/synergy/tree/master/src}{synergy} &310 &3 \\
t\_test &ISC license &\href{https://github.com/simple-statistics/simple-statistics/blob/main/src/t_test.js}{t\_test} &3400 &226 \\
validate &MIT license &\href{https://github.com/ehmicky/modern-errors/blob/main/src/subclass/validate.js}{validate} &1500 &18 \\
zone &MIT license &\href{https://github.com/drawcall/Proton/blob/master/src/zone/Zone.js}{zone} &2400 &274 \\
\hline
\end{tabular}
}
\label{tab: individual_dataset_js}
\end{table}

\section{More Implementation Details}
\subsection{Prompts}
\label{appendix: prompts}
%four prompts

%Python:
%Please generate enough unit test cases for each python file in the {method_signature} project. Ensure that the import path is correct, depending on whether the project is structured as a package. Make sure the tests can successfully compile. Make sure the tests have correct results. Try to achieve the highest coverage rate.

%Java:
%Please generate enough unit test cases for each java file in the {method_signature} project. Ensure to use mock properly for unit tests.
%Make sure the tests can successfully compile. Make sure the tests have correct results. Try to achieve the highest coverage rate.


%JS:
%Please generate enough unit test cases for every javascript file in {method_signature} project. Make sure the tests can successfully compile. Make sure the tests have correct results. Try to achieve the highest coverage rate.


%Python with comment signs:
%# {classname}\_test.py\n 
%# Test class of {classname}.\n 
%# Please generate enough unit test cases for each python file in the {method_signature} project. Ensure that the import path is correct, depending on whether the project is structured as a package. Make sure the tests can successfully compile. Make sure the tests have correct results. Try to achieve the highest coverage rate. \n 
%# class {classname}\_test\n         




% \begin{lstlisting}[breaklines=true, breakatwhitespace=true,backgroundcolor=\color{codebg}, basicstyle=\small,
% title=Prompt for generating unit test cases for Python project
% ]
% Please generate enough unit test cases for each python file in the {method_signature} project. Ensure that the import path is correct, depending on whether the project is structured as a package. Make sure the tests can successfully compile. Make sure the tests have correct results. Try to achieve the highest coverage rate.
% \end{lstlisting}


% \begin{lstlisting}[breaklines=true, breakatwhitespace=true,backgroundcolor=\color{codebg}, basicstyle=\small,
% title=Prompt for generating unit test cases for Java project
% ]
% Please generate enough unit test cases for each java file in the {method_signature} project. Ensure to use mock properly for unit tests.
% Make sure the tests can successfully compile. Make sure the tests have correct results. Try to achieve the highest coverage rate.
% \end{lstlisting}

% \begin{lstlisting}[breaklines=true, breakatwhitespace=true,backgroundcolor=\color{codebg}, basicstyle=\small,
% title=Prompt for generating unit test cases for JavaScript project
% ]
% Please generate enough unit test cases for every javascript file in {method_signature} project. Make sure the tests can successfully compile. Make sure the tests have correct results. Try to achieve the highest coverage rate.
% \end{lstlisting}

% \begin{lstlisting}[breaklines=true, breakatwhitespace=true,backgroundcolor=\color{codebg}, basicstyle=\small,
% title=Prompt for generating unit test cases for Python project with comment signs
% ]
% # {classname}\_test.py\n 
% # Test class of {classname}.\n 
% # Please generate enough unit test cases for each python file in the {method_signature} project. Ensure that the import path is correct, depending on whether the project is structured as a package. Make sure the tests can successfully compile. Make sure the tests have correct results. Try to achieve the highest coverage rate. \n 
% # class {classname}\_test\n 
% \end{lstlisting}

\begin{prompt_java}
\scriptsize{
\textbf{System Prompt}: You are a coding assistant. You generate only source code. \\
\textbf{User Prompt}: \textit{\{Original Codes\}} Please generate enough unit test cases for each java file in the \{method\_signature\} project. Ensure to use mock properly for unit tests. Make sure the tests can successfully compile. Make sure the tests have correct results. Try to achieve the highest coverage rate. }
\end{prompt_java}

\begin{prompt_js}
\scriptsize{
\textbf{System Prompt}: You are a coding assistant. You generate only source code. \\
\textbf{User Prompt}: \textit{\{Original Codes\}} Please generate enough unit test cases for every javascript file in \{method\_signature\} project. Make sure the tests can successfully compile. Make sure the tests have correct results. Try to achieve the highest coverage rate.}
\end{prompt_js}

\begin{prompt_comment}
\scriptsize{
\textbf{System Prompt}: You are a coding assistant. You generate only source code. \\
\textbf{User Prompt}: \textit{\{Original Codes\}} 
\# {classname}\_test.py\textbackslash n 
\# Test class of \{classname\}.\textbackslash n 
\# Please generate enough unit test cases for each python file in the \{method\_signature\} project. Ensure that the import path is correct, depending on whether the project is structured as a package. Make sure the tests can successfully compile. Make sure the tests have correct results. Try to achieve the highest coverage rate. \textbackslash n 
\# class \{classname}\_test\}\textbackslash n 
\end{prompt_comment}

The prompts are displayed in Figure~\ref{fig: prompt_java}, ~\ref{fig: prompt_js}, and ~\ref{fig: prompt_comment}.

\subsection{Models}
\label{appendix: models}
%models
%name, license, link
%2 + 4
\begin{table*}[t]
\caption{Model Details.}
\small
\resizebox{\linewidth}{!}{% <-
\begin{tabular}{llllll}
\hline
Model Type & Model Name & License & Link \\
\hline
Close-sourced &GPT-4-Turbo &- &https://platform.openai.com/docs/models/gpt-4\#gpt-4-turbo-and-gpt-4 \\
Close-sourced &GPT-3.5-Turbo &- &https://platform.openai.com/docs/models/gpt-4\#gpt-3-5-turbo \\
Close-sourced & GPT-o1 & - & https://platform.openai.com/docs/models\#o1\\
Close-sourced & Gemini-2.0-Flash & - & https://ai.google.dev/gemini-api/docs/models/gemini\#gemini-2.0-Flash\\
Close-sourced & Claude-3.5-Sonnet & - & https://www.anthropic.com/claude/sonnet \\
\hline
Open-sourced &CodeQwen1.5-7B-Chat &Tongyi Qianwen LICENSE AGREEMENT &https://huggingface.co/Qwen/CodeQwen1.5-7B-Chat \\
Open-sourced &DeepSeek-Coder-6.7b-Instruct &DEEPSEEK LICENSE AGREEMENT &https://huggingface.co/deepseek-ai/deepseek-coder-6.7b-instruct \\
Open-sourced &CodeLlama-7b-Instruct-hf &LLAMA 2 COMMUNITY LICENSE AGREEMENT	 &https://huggingface.co/codellama/CodeLlama-7b-Instruct-hf \\
Open-sourced &CodeGemma-7b-it &Gemma Terms of Use &https://huggingface.co/google/codegemma-7b-it \\
\hline
\end{tabular}
}
\label{tab: models}
\end{table*}
The detailed information of models, including license and link, is provided in Table~\ref{tab: models}.

\section{More Statistics}
\label{appendix: assert_statistics}
\begin{table*}[t]
\caption{Percentages of the vanilla unit tests containing expected and actual value comparisons.}
\resizebox{\linewidth}{!}{
\begin{tabular}{cccccccccc}
\hline
\textbf{Model} & \textbf{GPT-4-Turbo} & \textbf{GPT-3.5-Turbo} & \textbf{GPT-o1} & \textbf{Gemini} & \textbf{Claude} & \textbf{CodeQwen} & \textbf{DeepSeek-Coder} & \textbf{CodeLlama} & \textbf{CodeGemma} \\
\hline
\textbf{Python} & 98\% & 99\% & 98\% & 89\% & 99\% & 97\% & 96\% & 99\% & 88\% \\
\textbf{Java} & 97\% & 90\% & 98\% & 98\% & 97\% & 89\% & 94\% & 85\% & 93\% \\
\textbf{JavaScript} & 100\% & 89\% & 96\% & 100\% & 100\% & 100\% & 96\% & 86\% & 100\% \\
\hline
\end{tabular}
}
\label{tab: asser_statistics}
\end{table*}
Table~\ref{tab: asser_statistics} presents the percentages of the vanilla-generated unit tests containing comparisons between expected and actual values per language and per model.

\section{Ablation Study}
\label{appendix: ablation}

\subsection{Ablation Study on Prompts}
\label{sec: ablation}
% Please add the following required packages to your document preamble:
% \usepackage{multirow}
% \begin{table}[t]
% \centering
% \caption{Ablation Study. The Performance of Unit Test Generation by GPT-4-Turbo Using Different Prompts.}
% \resizebox{\linewidth}{!}{
% \begin{tabular}{cccccccc}
% \hline
% \textbf{Phase} & \textbf{Settings} & \textbf{\#Tests} & \textbf{\#Correct Tests} & \textbf{CR} & \textbf{ComR} & \textbf{LC} & \textbf{BC} \\
% \hline
% \multirow{6}{*}{Phase 1} & Full Prompt & 12.60 & 6.15 & 47\% & 65\% & 40\% & 36\% \\
%  & w/o CR & 12.75 & 4.75 & 33\% $\downarrow$ & 65\% & 42\% & 38\% \\
%  & w/o ComR & 11.20 & 3.95 & 35\% & 63\% $\downarrow$ & 41\% & 38\% \\
%  & w/o Coverage & 9.80 & 4.20 & 43\% & 75\% & 46\% $\uparrow$ & 42\% $\uparrow$ \\
%  & w/o PL & 9.95 & 4.35 & 47\% & 75\% & 53\% & 49\% \\
%  & w/ Comments & 10.65 & 4.15 & 41\% & 65\% & 45\% & 41\% \\
%  \hline
% \multirow{6}{*}{Phase 2} & Full Prompt & 12.60 & 9.10 & 73\% & 100\% & 65\% & 59\% \\
%  & w/o CR & 12.75 & 7.65 & 60\% $\downarrow$ & 100\% & 69\% & 64\% \\
%  & w/o ComR & 11.20 & 6.90 & 62\% & 100\% & 69\% & 64\% \\
%  & w/o Coverage & 9.80 & 5.95 & 61\% & 100\% & 64\% $\downarrow$ & 59\% $\rightarrow$ \\
%  & w/o PL & 9.95 & 6.40 & 65\% & 100\% & 70\% & 66\% \\
%  & w/ Comments & 10.65 & 6.70 & 63\% & 100\% & 67\% & 61\% \\
%  \hline
% \multirow{6}{*}{Phase 3} & Full Prompt & 12.60 & 9.30 & 74\% & 100\% & 65\% & 59\% \\
%  & w/o CR & 12.75 & 9.9 & 76\% $\uparrow$ & 100\% & 69\% & 64\% \\
%  & w/o ComR & 11.20 & 8.35 & 75\% & 100\% & 70\% & 65\% \\
%  & w/o Coverage & 9.80 & 6.75 & 68\% & 100\% & 66\% $\uparrow$ & 61\% $\uparrow$ \\
%  & w/o PL & 9.95 & 6.90 & 70\% & 100\% & 70\% & 66\% \\
%  & w/ Comments & 10.65 & 7.00 & 66\% & 100\% & 68\% & 62\% \\
%  \hline
% \end{tabular}
% }
% \label{tab: ablation}
% \end{table}

% \begin{table}[t]
% \centering
% \caption{Ablation Study. The Performance of Unit Test Generation by GPT-4-Turbo Using Different Prompts.}
% \resizebox{\linewidth}{!}{
% \begin{tabular}{cccccccc}
% \hline
% \textbf{Phase} & \textbf{Settings} & \textbf{\#Tests} & \textbf{\#Correct Tests} & \textbf{CR} & \textbf{ComR} & \textbf{LC} & \textbf{BC} \\
% \hline
% \multirow{6}{*}{Vanilla} & Full Prompt & 12.60 & 6.15 & 47\% & 65\% & 40\% & 36\% \\
%  & w/o CR & 12.75 & 4.75 & 33\% $\downarrow$ & 65\% & 42\% & 38\% \\
%  & w/o ComR & 11.20 & 3.95 & 35\% & 63\% $\downarrow$ & 41\% & 38\% \\
%  & w/o Coverage & 9.80 & 4.20 & 43\% & 75\% & 46\% $\uparrow$ & 42\% $\uparrow$ \\
%  & w/o PL & 9.95 & 4.35 & 47\% & 75\% & 53\% & 49\% \\
%  & w/ Comments & 10.65 & 4.15 & 41\% & 65\% & 45\% & 41\% \\
%  \hline
% \multirow{6}{*}{Phase 2} & Full Prompt & 12.60 & 9.10 & 73\% & 100\% & 65\% & 59\% \\
%  & w/o CR & 12.75 & 7.65 & 60\% $\downarrow$ & 100\% & 69\% & 64\% \\
%  & w/o ComR & 11.20 & 6.90 & 62\% & 100\% & 69\% & 64\% \\
%  & w/o Coverage & 9.80 & 5.95 & 61\% & 100\% & 64\% $\downarrow$ & 59\% $\rightarrow$ \\
%  & w/o PL & 9.95 & 6.40 & 65\% & 100\% & 70\% & 66\% \\
%  & w/ Comments & 10.65 & 6.70 & 63\% & 100\% & 67\% & 61\% \\
%  \hline
% \multirow{6}{*}{Manual Fixing} & Full Prompt & 12.60 & 9.30 & 74\% & 100\% & 65\% & 59\% \\
%  & w/o CR & 12.75 & 9.9 & 76\% $\uparrow$ & 100\% & 69\% & 64\% \\
%  & w/o ComR & 11.20 & 8.35 & 75\% & 100\% & 70\% & 65\% \\
%  & w/o Coverage & 9.80 & 6.75 & 68\% & 100\% & 66\% $\uparrow$ & 61\% $\uparrow$ \\
%  & w/o PL & 9.95 & 6.90 & 70\% & 100\% & 70\% & 66\% \\
%  & w/ Comments & 10.65 & 7.00 & 66\% & 100\% & 68\% & 62\% \\
%  \hline
% \end{tabular}
% }
% \label{tab: ablation}
% \end{table}

\begin{table}[t]
\centering
\caption{Ablation Study. The Performance of Unit Test Generation by GPT-4-Turbo Using Different Prompts.}
\resizebox{\linewidth}{!}{
\begin{tabular}{cccccccc}
\hline
\textbf{Phase} & \textbf{Settings} & \textbf{CR} & \textbf{ComR} & \textbf{LC} & \textbf{BC} & \textbf{\#Tests} & \textbf{\#Correct Tests} \\
\hline
\multirow{6}{*}{\textbf{Vanilla}} & Full Prompt & 47\% & 65\% & 40\% & 36\% & 12.60 & 6.15 \\
 & w/o CR & 33\% $\downarrow$ & 65\% & 42\% & 38\% & 12.75 & 4.75 \\
 & w/o ComR & 35\% & 63\% $\downarrow$ & 41\% & 38\% & 11.20 & 3.95 \\
 & w/o Coverage & 43\% & 75\% & 46\% $\uparrow$ & 42\% $\uparrow$ & 9.80 & 4.20 \\
 & w/o PL & 47\% & 75\% & 53\% & 49\% & 9.95 & 4.35 \\
 & w/ Comments & 41\% & 65\% & 45\% & 41\% & 10.65 & 4.15 \\
 \hline
% \multirow{6}{*}{Phase 2} & Full Prompt & 73\% & 100\% & 65\% & 59\% & 12.60 & 9.10 \\
%  & w/o CR & 60\% $\downarrow$ & 100\% & 69\% & 64\% & 12.75 & 7.65 \\
%  & w/o ComR & 62\% & 100\% & 69\% & 64\% & 11.20 & 6.90 \\
%  & w/o Coverage & 61\% & 100\% & 64\% $\downarrow$ & 59\% $\rightarrow$ & 9.80 & 5.95 \\
%  & w/o PL & 65\% & 100\% & 70\% & 66\% & 9.95 & 6.40 \\
%  & w/ Comments & 63\% & 100\% & 67\% & 61\% & 10.65 & 6.70 \\
%  \hline
\multirow{6}{*}{\textbf{Manual Fixing}} & Full Prompt & 74\% & 100\% & 65\% & 59\% & 12.60 & 9.30 \\
 & w/o CR & 76\% $\uparrow$ & 100\% & 69\% & 64\% & 12.75 & 9.90 \\
 & w/o ComR & 75\% & 100\% & 70\% & 65\% & 11.20 & 8.35 \\
 & w/o Coverage & 68\% & 100\% & 66\% $\uparrow$ & 61\% $\uparrow$ & 9.80 & 6.75 \\
 & w/o PL & 70\% & 100\% & 70\% & 66\% & 9.95 & 6.90 \\
 & w/ Comments & 66\% & 100\% & 68\% & 62\% & 10.65 & 7.00 \\
 \hline
\end{tabular}
}
\label{tab: ablation}
\end{table}

We perform a detailed ablation study to analyze the impact of prompts on the performance of unit test generation by LLMs.
As mentioned in \S~\ref{unit_test_generation}, the prompt is composed of programming language-specific requirements (PL), as well as requirements related to the correctness rate (CR), the compilation rate (ComR), and the coverage rate metrics (Coverage). We ablate each component and analyze the performance of unit test generation of GPT-4-Turbo using different prompts as shown in Table~\ref{tab: ablation}. 
Requirements related to CR and ComR can help improve performance in vanilla unit tests. 
Coverage-related requirements are not always beneficial, possibly because a high coverage rate is too abstract for LLMs to interpret effectively.
Programming language-specific requirements improve performance in CR but have the opposite effect on ComR, LC, and BC.

Besides, we follow the prompt template from previous work like~\citet{siddiq2024using} to move the prompts into comments (e.g., /*...*/). We compare the performance with and without comment signs in Table~\ref{tab: ablation}. Experimental results show that our prompt demonstrates a significant advantage in CR, while the prompt with comment signs exhibits marginal advantages in ComR, LC, and BC.


\subsection{Effect of Compilation Errors and Cascade Errors}
\label{appendix: alabtion_compilation}
% \begin{table}[t]
% \centering
% \caption{Evaluation Results When Only Manually Fixing Compilation Errors}
% \resizebox{\linewidth}{!}{
% \begin{tabular}{cccccccc}
% \hline
% \textbf{Language} & \textbf{Model} & \textbf{\#Tests} & \textbf{\#Correct Tests} & \textbf{CR} & \textbf{ComR} & \textbf{LC} & \textbf{BC} \\
% \hline
% \multirow{8}{*}{Python} & GPT-4-Turbo & 12.60 & 9.10 & 73\% & 100\% & 65\% & 59\% \\
%  & GPT-3.5-Turbo & 16.90 & 10.40 & 63\% & 100\% & 62\% & 56\% \\
%  & GPT-o1 & 36.35 & 32.25 & \underline{89\%} & 100\% & \textbf{88\%} & 85\%\\
%  & Gemini-2.0-Flash & 34.95 & 22.10 & 61\% & 100\% & 71\% & 68\% \\
%  & Claude-3.5-Sonnet & 18.05 & 16.40 & \textbf{92\%} & 100\% & \underline{74\%} & \underline{70\%} \\
%  & CodeQwen1.5  & 25.40 & 9.60 & 40\% & 100\% & 65\% & 59\% \\
%  & DeepSeek-Coder & 7.20 & 4.10 & 53\% & 100\% & 60\% & 54\% \\
%  & CodeLlama & 19.30 & 6.15 & 26\% & 100\% & 56\% & 50\% \\
%  & CodeGemma & 15.00 & 6.15 & 30\% & 100\% & 52\% & 47\% \\
%  \hline
% \multirow{8}{*}{Java} & GPT-4-Turbo & 7.05 & 5.05 & 59\% & 100\% & 42\% & 34\% \\
%  & GPT-3.5-Turbo & 7.50 & 4.20 & 48\% & 100\% & 37\% & 29\% \\
%  & GPT-o1 & 15.70 & 10.50 & \underline{62\%} & 100\% & \textbf{67\%} & \underline{56\%} \\
%  & Gemini-2.0-Flash & 23.30 & 15.00 & 55\% & 100\% & 54\% & 53\% \\
%  & Claude-3.5-Sonnet & 12.35 & 9.60 & \textbf{73\%} & 100\% & \underline{63\%} & \textbf{57\%} \\
%  & CodeQwen1.5  & 12.95 & 7.50 & 49\% & 100\% & 49\% & 39\% \\
%  & DeepSeek-Coder& 7.00 & 2.85 & 40\% & 100\% & 36\% & 19\% \\
%  & CodeLlama & 7.85 & 4.25 & 30\% & 100\% & 26\% & 21\% \\
%  & CodeGemma & 10.50 & 5.55 & 46\% & 100\% & 44\% & 26\% \\
%  \hline
% \multirow{8}{*}{JavaScript} & GPT-4-Turbo & 16.30 & 14.15 & \underline{89\%} & 100\% & 75\% & 59\% \\
%  & GPT-3.5-Turbo & 13.25 & 10.65 & 71\% & 100\% & 56\% & 44\% \\
%  & GPT-o1 & 39.40 & 35.15 & \textbf{91\%} & 100\% & \textbf{92\%} & \underline{79\%}\\
%  & Gemini-2.0-Flash & 45.85 & 33.30 & 76\% & 100\% & \underline{88\%} & \textbf{80\%} \\
%  & Claude-3.5-Sonnet & 20.25 & 16.75 & 83\% & 100\% & 75\% & 66\% \\
%  & CodeQwen1.5  & 8.45 & 5.65 & 28\% & 100\% & 29\% & 22\% \\
%  & DeepSeek-Coder& 11.85 & 8.05 & 66\% & 100\% & 58\% & 43\% \\
%  & CodeLlama & 48.75 & 21.40 & 28\% & 100\% & 20\% & 15\% \\
%  & CodeGemma & 9.00 & 5.75 & 45\% & 100\% & 43\% & 30\% \\
%  \hline
% \end{tabular}
% }
% \label{tab: results_2}
% \end{table}

\begin{table}[t]
\centering
\caption{Evaluation Results When Only Manually Fixing Compilation Errors}
\resizebox{\linewidth}{!}{
\begin{tabular}{cccccccc}
\hline
\textbf{Language} & \textbf{Model} & \textbf{CR} & \textbf{ComR} & \textbf{LC} & \textbf{BC} & \textbf{\#Tests} & \textbf{\#Correct Tests} \\
\hline
\multirow{8}{*}{Python} 
 & GPT-4-Turbo & 73\% & 100\% & 65\% & 59\% & 12.60 & 9.10 \\
 & GPT-3.5-Turbo & 63\% & 100\% & 62\% & 56\% & 16.90 & 10.40 \\
 & GPT-o1 & \underline{89\%} & 100\% & \textbf{88\%} & \textbf{85\%} & 36.35 & 32.25 \\
 & Gemini-2.0-Flash & 61\% & 100\% & 71\% & 68\% & 34.95 & 22.10 \\
 & Claude-3.5-Sonnet & \textbf{92\%} & 100\% & \underline{74\%} & \underline{70\%} & 18.05 & 16.40 \\
 & CodeQwen1.5  & 40\% & 100\% & 65\% & 59\% & 25.40 & 9.60 \\
 & DeepSeek-Coder & 53\% & 100\% & 60\% & 54\% & 7.20 & 4.10 \\
 & CodeLlama & 26\% & 100\% & 56\% & 50\% & 19.30 & 6.15 \\
 & CodeGemma & 30\% & 100\% & 52\% & 47\% & 15.00 & 6.15 \\
 \hline
\multirow{8}{*}{Java} 
 & GPT-4-Turbo & 59\% & 100\% & 42\% & 34\% & 7.05 & 5.05 \\
 & GPT-3.5-Turbo & 48\% & 100\% & 37\% & 29\% & 7.50 & 4.20 \\
 & GPT-o1 & \underline{62\%} & 100\% & \textbf{67\%} & \underline{56\%} & 15.70 & 10.50 \\
 & Gemini-2.0-Flash & 55\% & 100\% & 54\% & 53\% & 23.30 & 15.00 \\
 & Claude-3.5-Sonnet & \textbf{73\%} & 100\% & \underline{63\%} & \textbf{57\%} & 12.35 & 9.60 \\
 & CodeQwen1.5  & 49\% & 100\% & 49\% & 39\% & 12.95 & 7.50 \\
 & DeepSeek-Coder & 40\% & 100\% & 36\% & 19\% & 7.00 & 2.85 \\
 & CodeLlama & 30\% & 100\% & 26\% & 21\% & 7.85 & 4.25 \\
 & CodeGemma & 46\% & 100\% & 44\% & 26\% & 10.50 & 5.55 \\
 \hline
\multirow{8}{*}{JavaScript} 
 & GPT-4-Turbo & \underline{89\%} & 100\% & 75\% & 59\% & 16.30 & 14.15 \\
 & GPT-3.5-Turbo & 71\% & 100\% & 56\% & 44\% & 13.25 & 10.65 \\
 & GPT-o1 & \textbf{91\%} & 100\% & \textbf{92\%} & \underline{79\%} & 39.40 & 35.15 \\
 & Gemini-2.0-Flash & 76\% & 100\% & \underline{88\%} & \textbf{80\%} & 45.85 & 33.30 \\
 & Claude-3.5-Sonnet & 83\% & 100\% & 75\% & 66\% & 20.25 & 16.75 \\
 & CodeQwen1.5  & 28\% & 100\% & 29\% & 22\% & 8.45 & 5.65 \\
 & DeepSeek-Coder & 66\% & 100\% & 58\% & 43\% & 11.85 & 8.05 \\
 & CodeLlama & 28\% & 100\% & 20\% & 15\% & 48.75 & 21.40 \\
 & CodeGemma & 45\% & 100\% & 43\% & 30\% & 9.00 & 5.75 \\
 \hline
\end{tabular}
}
\label{tab: ablation2}
\end{table}

We manually fix only compilation errors and evaluate the corrected unit tests in Table~\ref{tab: ablation2}.
% Comparing Table~\ref{tab: ablation2} with Table~\ref{tab: main_results} and Table~\ref{tab: manual_results_improvements}, we can analyze the impact of compilation errors and cascade errors on performance.

By fixing compilation errors, Table~\ref{tab: ablation2} shows significant improvements across all programming languages and LLMs compared to Table~\ref{tab: main_results}, indicating that all the programming languages and LLMs are highly sensitive to compilation errors.
Comparing Table~\ref{tab: ablation2} with Table~\ref{tab: manual_results_improvements}, we can observe that CodeQwen1.5, CodeGemma, and CodeLlama are more sensitive to cascade errors.
% For \textbf{\textit{Python}} and \textbf{\textit{JavaScript}}, CodeQwen1.5, CodeGemma, and CodeLlama are more sensitive to cascade errors. The results of the three models increase apparently compared to \textsc{Phase 2}.
For Java, the changes in Table~\ref{tab: manual_results_improvements} compared to Table~\ref{tab: ablation2} are primarily due to missing or invalid mocks of user interactions\footnote{We consider coverage rates as not applicable when requiring user interactions.} which occur more frequently in unit tests generated by CodeQwen1.5 and CodeGemma. 
% This frequency is evident in the changes observed in the results between Table~\ref{tab: ablation2} and Table~\ref{tab: manual_results_improvements}. 
% For \textbf{\textit{JavaScript}},
% CodeQwen1.5, CodeGemma, and CodeLlama are more sensitive to \textcolor{blue}{default or global errors}. The results of the three models increase apparently compared to \textsc{Phase 2}.


\section{Detailed Error Analyses}
\label{sec: full_error_analyses}
We conduct complex analyses of compilation, cascade, and post-fix errors, highlighting the common errors and potential reasons behind the errors.

\paragraph{Compilation Error Analyses}
% \input{sections/5_figure_main_results}
Figure~\ref{fig: errors1} highlights the detailed compilation errors that occurred.
One of the most common compilation errors in \textbf{\textit{Python}} arises from the LLM's inability to determine whether the project being tested is a package. Specifically, LLMs struggle to recognize the presence or absence of \textit{\_\_init\_\_.py} files, which define a package, leading to confusion between package-based and non-package projects. This inability leads LLM to fail to correctly import functions or classes from the tested project.
Other compilation errors include hallucinating the paths or names of imported functions/classes and mismatched parentheses.
\textbf{\textit{Java}}, a syntax-heavy programming language compared to Python and JavaScript, encounters various compilation errors, resulting in a significantly lower compilation rate than other languages. Java compilation errors often arise from issues like hallucinated methods, constructors, or classes, such as incorrect or non-existent imports and references. Missing essential information, such as required functions, classes, or packages, and package declarations, is also a common problem. Errors frequently occur due to illegal access to private or protected elements, invalid code generation (e.g., generating text instead of code), and improper use of mocking frameworks like Mockito, including incorrect objects, missing or misused MockMvc injections, and argument mismatches. Other errors include incorrect usage of other functions, classes, or packages—such as argument type errors, ambiguous references, or incompatible types.
One of the most common compilation errors in \textbf{\textit{JavaScript}} is the hallucination of imported functions or classes, where the issue often lies in incorrect paths for the imported functions or classes. CodeQwen1.5 has a particularly common compilation error involving invalid generation. This typically occurs due to difficulty understanding the prompt, the need for more specific or detailed code requirements, or the assumption that the code is part of a larger project, leading it to decline generating unit tests. Other compilation errors include test suites containing empty unit tests and syntax errors caused by incomplete code generation or mismatched parentheses.


\paragraph{Cascade Error Analyses}
Figure~\ref{fig: errors2} highlights the detailed cascade errors that occurred.
For \textbf{\textit{Python}}, the cascade errors include missing imports of commonly used packages such as numpy and unittest, missing imports of functions or classes from the tested project, and FileNotFoundError. 
% The FileNotFoundError indicates that the generated unit tests fail to mock the external files.
For \textbf{\textit{Java}}, the most common cascade error is missing or invalid mocking of user interactions. A proper unit test should simulate user interactions through mocking rather than relying on real user inputs. This issue also results in unusable coverage reports for some tested projects, as the error forces an abrupt termination, preventing the generation of coverage data.
For \textbf{\textit{JavaScript}}, the cascade errors include missing imports of commonly used packages such as chai and three, and missing imports of functions or classes from the tested project. Two other common errors specific to JavaScript are that LLMs may confuse named imports with default imports and fail to comply with the Jest framework. 


\paragraph{Post-Fix Error Analyses}
Figure~\ref{fig: errors3} highlights the incorrectness reasons after all manual fixes.
For all programming languages, the mismatch between expected and actual values (AssertionError) is the most common error.
Another frequent error in \textbf{\textit{Python}} is AttributeError, typically caused by LLMs hallucinating non-existent attributes.
Other frequent problems in \textbf{\textit{Java}} include NullPointer Errors, zero interactions with mocks, and failures to release mocks, often due to improper mock usage. For projects tested with the Spring framework, errors specific to Spring are also common.
Another frequent error in \textbf{\textit{JavaScript}} is TypeError, mostly caused by LLMs hallucinating non-existent functions and constructors or LLMs invalidly mocking some variables.

\tikzstyle{my-box}=[
    rectangle,
    draw=hidden-draw,
    rounded corners,
    text opacity=1,
    minimum height=1.5em,
    minimum width=5em,
    inner sep=2pt,
    align=center,
    fill opacity=.5,
    line width=0.8pt,
]
\tikzstyle{leaf}=[my-box, minimum height=1.5em,
    fill=hidden-pink!80, text=black, align=left,font=\normalsize,
    inner xsep=2pt,
    inner ysep=4pt,
    line width=0.8pt,
]
\begin{figure}[]
    \centering
    \resizebox{\linewidth}{!}{
        \begin{forest}
            forked edges,
            for tree={
                grow'=0,
                draw,
                reversed=true,
                anchor=base west,
                parent anchor=east,
                child anchor=west,
                base=left,
                font=\large,
                rectangle,
                rounded corners,
                align=left,
                minimum width=4em,
                edge+={darkgray, line width=1pt},
                s sep=3pt,
                inner xsep=2pt,
                inner ysep=3pt,
                line width=0.8pt,
                ver/.style={rotate=90, child anchor=north, parent anchor=south, anchor=center},
            },
            where level=1{text width=4.4em,font=\normalsize,}{},
            where level=2{text width=12em,font=\normalsize,}{},
            where level=3{text width=25em,font=\normalsize,}{},
            % where level=4{text width=5em,font=\normalsize,}{},
			[
			    Compilation Error Analysis, ver
			    [
		              Python, 
                        fill=lgreen
    			            [
                                Confuse between non-package and package-based projects
                                , leaf, text width=28em, fill=lgreen
    			            ]
                                [
                                Hallucinate the imported functions/classes:\\
                                1. Paths of the imported functions/classes are wrong\\
                                2. Names of the imported functions/classes are wrong
                                , leaf, text width=28em, fill=lgreen
    			            ]
                                [
                                Syntax Error: Mismatched parentheses
                                , leaf, text width=28em, fill=lgreen
                                ]
			        ]
			    [
    			      Java, fill=lblue
    			            [
                                Hallucinate methods/constructors/functions/classes:\\
                                1. Paths of the imported functions/classes are wrong \\
                                2. Names of the imported functions/classes are wrong \\
                                3. Non-existed methods/constructors
                                , leaf, text width=28em, fill=lblue
    			            ]
                                [
                                Missing information: \\
                                1. Required functions/classes/packages are missing \\
                                2. Required package information is missing \\
                                3. Unreported exception \\
                                , leaf, text width=28em, fill=lblue
                                ]
                                [
                                Illegal access to private/protected functions/classes
                                , leaf, text width=28em, fill=lblue
                                ]
                                [
                                Invalid generation: \\
                                1. Generate textual instructions instead of codes
                                2. Block by model \\
                                , leaf, text width=28em, fill=lblue
                                ]
                                [
                                Incorrect use of mocking: \\
                                1. Wrong objects provided to Mockito \\
                                2. Missing MockMvc injection 
                                3. Inappropriate mockmvc \\
                                4. Argument mismatch
                                , leaf, text width=28em, fill=lblue
                                ]
                                [
                                Incorrect use of other functions/classes/packages: \\
                                1. Arguments type error 2. Ambiguous reference \\
                                3. Incompatible types
                                , leaf, text width=28em, fill=lblue
                                ]
			    ]
                    [
                        JavaScript, fill=lyellow[
                                Hallucinate the imported functions/classes: \\
                                1. Paths of the imported functions/classes are wrong
                                , leaf, text width=28em, fill=lyellow
                                ]
                                [
                                Invalid generation: \\
                                1. Cannot understand the prompt
                                2. Require more/specific codes \\
                                3. Assume the codes are part of a larger project and \\ decline to generate unit tests
                                , leaf, text width=28em, fill=lyellow
                                ]
                                [
                                Test suits have empty unit tests \\
                                , leaf, text width=28em, fill=lyellow
                                ]
                                [
                                Syntax Error: \\
                                1. Incomplete generation 
                                2. Mismatched parentheses
                                , leaf, text width=28em, fill=lyellow
                                ]
                    ]
			]
            \end{forest}
    }
    \caption{Frequent Compilation Errors in Main Results.}
    \label{fig: errors1}
\end{figure}


\tikzstyle{my-box}=[
    rectangle,
    draw=hidden-draw,
    rounded corners,
    text opacity=1,
    minimum height=1.5em,
    minimum width=5em,
    inner sep=2pt,
    align=center,
    fill opacity=.5,
    line width=0.8pt,
]
\tikzstyle{leaf}=[my-box, minimum height=1.5em,
    fill=hidden-pink!80, text=black, align=left,font=\normalsize,
    inner xsep=2pt,
    inner ysep=4pt,
    line width=0.8pt,
]
\begin{figure}[t]
    \centering
    \resizebox{\linewidth}{!}{
        \begin{forest}
            forked edges,
            for tree={
                grow'=0,
                draw,
                reversed=true,
                anchor=base west,
                parent anchor=east,
                child anchor=west,
                base=left,
                font=\large,
                rectangle,
                rounded corners,
                align=left,
                minimum width=4em,
                edge+={darkgray, line width=1pt},
                s sep=3pt,
                inner xsep=2pt,
                inner ysep=3pt,
                line width=0.8pt,
                ver/.style={rotate=90, child anchor=north, parent anchor=south, anchor=center},
            },
            where level=1{text width=4.4em,font=\normalsize,}{},
            where level=2{text width=12em,font=\normalsize,}{},
            where level=3{text width=20em,font=\normalsize,}{},
            % where level=4{text width=5em,font=\normalsize,}{},
			[
			    Cascade Error Analysis, ver
			    [
		              Python, 
                        fill=lgreen
    			            [
                                Required functions/classes/libraries are missing:\\
                                1. Import numpy or unittest.mock\\
                                2. Import functions/classes of the tested project
                                , leaf, text width=25em, fill=lgreen
    			            ]
    			            [
                                FileNotFoundError
                                , leaf, text width=25em, fill=lgreen
    			            ]
			        ]
			    [
    			      Java, fill=lblue
    			            [
                                Missing/Invalid mock of user interactions
                                , leaf, text width=25em, fill=lblue
    			            ]
			    ]
                    [
                        JavaScript, fill=lyellow
                                [
                                Required functions/classes/libraries are missing:\\
                                1. Import chai or three\\
                                2. Import functions/classes of the tested project
                                , leaf, text width=25em, fill=lyellow
                                ]
                                [
                                Confuse between name import and default import
                                , leaf, text width=25em, fill=lyellow
                                ]
                                [
                                Do not follow the Jest framework
                                , leaf, text width=25em, fill=lyellow
                                ]
                    ]
			]
            \end{forest}
    }
    \caption{Frequent Cascade Errors.}
    \label{fig: errors2}
\end{figure}


\tikzstyle{my-box}=[
    rectangle,
    draw=hidden-draw,
    rounded corners,
    text opacity=1,
    minimum height=1.5em,
    minimum width=5em,
    inner sep=2pt,
    align=center,
    fill opacity=.5,
    line width=0.8pt,
]
\tikzstyle{leaf}=[my-box, minimum height=1.5em,
    fill=hidden-pink!80, text=black, align=left,font=\normalsize,
    inner xsep=2pt,
    inner ysep=4pt,
    line width=0.8pt,
]
\begin{figure}[]
    \centering
    \resizebox{\linewidth}{!}{
        \begin{forest}
            forked edges,
            for tree={
                grow'=0,
                draw,
                reversed=true,
                anchor=base west,
                parent anchor=east,
                child anchor=west,
                base=left,
                font=\large,
                rectangle,
                rounded corners,
                align=left,
                minimum width=4em,
                edge+={darkgray, line width=1pt},
                s sep=3pt,
                inner xsep=2pt,
                inner ysep=3pt,
                line width=0.8pt,
                ver/.style={rotate=90, child anchor=north, parent anchor=south, anchor=center},
            },
            where level=1{text width=4.4em,font=\normalsize,}{},
            where level=2{text width=12em,font=\normalsize,}{},
            where level=3{text width=25em,font=\normalsize,}{},
            % where level=4{text width=5em,font=\normalsize,}{},
			[
			    Post-fix Error Analysis, ver
			    [
		              Python, 
                        fill=lgreen
                                [
                                1. AttributeError
                                2. AssertionError
                                3. TypeError 
                                4. ValueError \\
                                5. IndexError 
                                6. \_csv.Error 
                                7. NameError
                                8. KeyError 
                                9. Others
                                , leaf, text width=30em, fill=lgreen
    			            ]
			        ]
			    [
    			      Java, fill=lblue
    			            [
                                1. Mismatch between expected and received 
                                2. NullPointer Error \\
                                3. Zero interactions with mock 
                                4. Failed to release mocks \\
                                5. MissingMethodInvocation 
                                6. Misplaced or misused argument matcher \\
                                7. Spring framework error 
                                8. NoSuchElement 
                                9. Others
                                , leaf, text width=30em, fill=lblue
    			            ]
			    ]
                    [
                        JavaScript, fill=lyellow
                                [
                                1. Mismatch between expected and received 
                                2. TypeError 
                                3. RangeError \\
                                4. RuntimeError
                                5. ReferenceError 
                                6. SyntaxError 
                                7. Others
                                % 7. Invalid component type 
                                % 8. Image given has not completed loading 
                                % 9. Invalid Chai property
                                , leaf, text width=30em, fill=lyellow
                                ]
                    ]
			]
            \end{forest}
    }
    \caption{Frequent Post-Fix Errors.}
    \label{fig: errors3}
\end{figure}


% \paragraph{Overall}
% Common errors across different programming languages include hallucinations of functions or classes, \textit{possibly caused by training data bias and a lack of proper references}.
% % Missing required functions/classes is another common error (Compilation error for Java and Cascade error for Python and JavaScript).
% Another common error is missing required functions or classes, which often occurs because LLMs \textit{prioritize logical structure over boilerplate code} and \textit{fail to understand the codebase structure and the dependencies between functions, classes, or modules}. Failure to understand the codebase structure and dependencies can also cause other mistakes, such as confusing non-package and package-based projects (Python) or incorrectly using functions, classes, or packages (Java).
% % For after-fix errors, the mismatch between expected and received values is the most common error. 
% The most common post-fix error is the mismatch between expected and received values, often caused by incorrect expected values due to the \textit{weak reasoning abilities} of LLMs.
