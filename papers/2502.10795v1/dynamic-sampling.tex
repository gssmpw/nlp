\section{Application in dynamic sampling}\label{sec:dynamic}

In this section, we will show the application of our local sampler to a dynamic sampling problem, and prove \Cref{theorem:dynamic-sampler}.


We maintain the spin system $\+S=(G=(V,E),\bm{\lambda},\bm{A})$, a query set $\Lambda\subseteq V$, and the sample $X\in [q]^{\Lambda}$ where $X\sim \mu^{\+S}(\Lambda)$. We consider the types of updates already described in \Cref{definition:updates}. The dynamic sampling problem is, when an update is given, which either changes the query set $\Lambda$ or modifies the parameters of the spin system $\+S$ we need to simultaneously update $X$ and ensure that $X \sim \mu^{\+S}(\Lambda)$ always holds.
We use a data structure to implement the functionality described above. The data structure maintains a variable set $\Lambda\subseteq V$ and the local sampling result $\DDS \in [q]^{\Lambda}$. 
%Initially, let $T=\emptyset$.
The data structure supports all updates in \Cref{definition:updates}. After each update, the data structure ensures that the maintained $\DDS$ is distributed as $\mu^{\+S}(\Lambda)$. 

\begin{comment}
\begin{center}
  \begin{tcolorbox}[=sharpish corners, colback=white, width=1\linewidth]
  	\begin{center}
	\textbf{\emph{A dynamic sampling data structure for $q$-spin systems}}
  	\end{center}
  	\vspace{6pt}
%    \vspace{4pt}
    \textbf{Input:} a $q$-spin system $\+S=(G=(V,E),(\bm{\lambda},\bm{A})))$.
    \\
    \textbf{Global Variables:} $\Lambda\subseteq V, \DDS \in [q]^{\Lambda}$.
    \\
    \textbf{Initialization:} $\Lambda=\emptyset$.
    \\
    \textbf{Maintain:} $\DDS \sim \mu^{\+S}(\Lambda)$.
    \\
    \textbf{Types of updates:}
    \begin{enumerate}
    \item modifying the query set, in one of the following two terms:
     \begin{enumerate}
        \item \textbf{adding query vertices}: choose some $\Lambda'\subseteq V$ such that $\Lambda'\cap \Lambda=\emptyset$, and update $\Lambda\gets \Lambda\cup \Lambda'$; \label{item:add-vertices}
        \item \textbf{removing query vertices}: choose some $\Lambda'\subseteq \Lambda$ and update $\Lambda\gets \Lambda\setminus \Lambda'$ ; \label{item:remove-vertices}
    \end{enumerate}
    \item modifying the parameters of the spin system $\+S$, in one of the following two terms:
    \begin{enumerate}
         \item \textbf{updates for the external fields}: modifying $\lambda_v$ for some $v\in V$; \label{item:update-external-field}
        \item \textbf{updates for the interaction matrices}: modifying $A_e$ for some $e\in E$. \label{item:update-interaction-matrix}
    \end{enumerate}

    \end{enumerate}
  \end{tcolorbox} 
\end{center}
\end{comment}

We assume in the process of $\lsample(\Lambda)$, the randomness required in $\eval(v_{i(t)})$ called within $\resolve(t)$ is global for each $t\leq 0$.

\begin{remark}[global randomness]\label{remark:global-randomness}
For each $t\leq 0$, a random sequence $\+R(t) = \{(r_i,(r_{i,u})_{u\in N(v)})\}_{1\leq i<\infty}$ is predetermined at the beginning of the entire dynamic sampling algorithm, where $v=v_{i(t)}$ and each $r_i$ and each $r_{i,u}$ is independent and uniformly from $[0,1]$.
The randomness is maintained in a global map and sampled upon access using the principle of deferred decision, as already mentioned in \Cref{remark:lazy-samples}.
During the entire process of the dynamics sampling algorithm, every time $\eval^{\+O}(v_{i(t)})$ is called by $\resolve(t)$, the random sequence at \Cref{Line:eval-sample} is directly determined by $\+R(t)$, where for each $c_i$, it follows the distribution of $\lambda_v$ and its randomness is determined by $r_i$.
%During multiple calls to $\eval(t)$, the sequence $\{(c_i,(r_{i,u})_{u\in N(v)})\}_{1\leq i<\infty}$ remains consistent. 
\end{remark}

Here, we maintain the needed information of $\lsample(\Lambda)$ with $\eval(t)$ implemented as in \Cref{Alg:eval}. We also need to maintain and update the following two global data structures:
\begin{itemize}
    \item The map $M$, as already explicitly stated in \Cref{Alg:resolve}, that maintains the outcome of resolved updates
    \item A map $R$ that lazily stores all generated random bits of sequence $\+R(t) = \{(r_i,(r_{i,u})_{u\in N(v)})\}_{1\leq i<\infty}$ for each timestamp $t\leq 0$, as mentioned in \Cref{remark:global-randomness}.
    %\item A map $R$ that lazily stores all generated random bits $r_{u,i}$ and $r_{i}$ in \Cref{Line:eval-sample} of \Cref{Alg:eval} for each timestamp $t\leq 0$, as already mentioned in \Cref{remark:global-randomness}.
\end{itemize}

Initially, the maps $M$ and $R$ are set as empty. Here, the update of removing query vertices can simply be implemented by updating $X\to X_{\Lambda\setminus \Lambda'}$. Also, the update of adding query vertices can be simply implemented by calling $\lsample(\Lambda)$ in \Cref{Alg:lsample}, with the map $R$ updated by sampling $r_{i,u}$ and $r_{i}$  visited by the algorithm lazily, according to the correct distribution.

\begin{algorithm}[H]
\caption{dynamic sampler: adding query vertices} \label{Alg:update-variables}
\SetKwInput{KwData}{Global variables}
\SetKwInput{KwUpdate}{Update}
\SetKwInput{KwMaintain}{Maintain}
\KwIn{a $q$-spin system $\+S=(G=(V,E),(\bm{\lambda},\bm{A})))$, a subset of variables $\Lambda'\subseteq V$ to be added}
\KwData{two maps 
$M: \mathbb{Z}\to [q]\cup\{\perp\}, R:\mathbb{Z}\to \{(r_i,(r_{i,u})_{u\in N(v)})\}_{1\leq i<\infty}$;
the query set $\Lambda\subseteq V$ and configuration $\DDS \in [q]^\Lambda$}
\KwMaintain{$\DDS \sim \mu^{\+S}(\Lambda)$}
%\KwOut{A random value $X \in [q]^S$ distributed as $\mu_S$}
\ForAll{$v\in \Lambda'$}{
    $\DDS(v)\gets \resolve(\pred_0(v))$\;\label{Line:dynamic-sampler-resolve}
}
%$\DDS \gets \DDS \cup \lsample(S)$\;
$\Lambda\gets \Lambda\cup \Lambda'$\;
\end{algorithm}

%\begin{lemma}[correctness of \cref{Alg:update-variables}]\label{lemma:dynamic-sampler-upd1-correctness}
%\end{lemma}

For the update on either the external field or the interaction matrices, we unify the two cases and assume that the update is given in the form of $(D,\Phi_D)$ such that either $(D,\Phi_D)=(\{v\},\lambda'_v)$ for some $v\in V$ where $\lambda'_v(e)$ is the external field of $v$ to be updated to; or $(D,\Phi_D)=(\{e\},A'_e)$ for some $e\in V$ where $A'_e$ is the interaction matrix of $e$ to be updated to. 
%When the $q$-spin system is updated from $\+S$ to $\+S'$, we assume that we use the same infinite long sequence of i.i.d tuples $\{(c_i,(r_{i,u})_{u\in N(v)})\}_{1\leq i<\infty}$ for each time stamp $t\leq 0$ to execute the $\lsample$ algorithm.
Our dynamic sampling algorithm will consider all resolved updates where the results may change and call $\resolve$ to resolve these updates again. To determine which resolved update results may change, we first establish the following definitions.

\begin{definition}[variable sets]\label{def:vbl-set}
For a subset $D\subseteq V\cup E$, we use $\vbl(D)$ to denote the set of variables involved in $D$:
\begin{align}
\vbl(D) \triangleq \left(D\cap V\right) \cup \left( \cup_{e\in {D\cap E}} e\right).
\end{align}
\end{definition}

During the recursive invocation process of $\resolve$, we need to maintain certain information related to the procedure of our local sampling algorithm, which is formally defined as follows:

\begin{definition}[calling sets]\label{def:call}
For each $v\in V$, let $\call(v)$ denote the set of resolved updates at the vertex $v$, i.e., the set of $t$s such that $t\leq 0$ and $\resolve(t)$ is recursively called (directly or indirectly) within $\lsample(\Lambda)$:
\begin{align}
\call(v) \triangleq \{t\leq 0 \mid v_{i(t)} = v \text{ and $\resolve(t)$ is recursively called within $\lsample(\Lambda)$} \}. 
\end{align}
\end{definition}
\begin{definition}[successor updates]\label{def:seccessor}
For each time $t$ $(t\leq 0)$, if $\resolve(t)$ is recursively called (directly or indirectly) within the execution of $\lsample(\Lambda)$, then we let $\suc(t)$ denote the set of all $t'$ such that $\resolve(t)$ is recursively called directly (through $\eval^{\+O}(v_{i(t')})$) within $\resolve(t')$:
\begin{align}
\suc(t) \triangleq \{t < t'\leq 0 \mid \resolve(t) \text{ is recursively called directly within } \resolve(t') \}. 
\end{align}
Otherwise, if $\resolve(t)$ is not recursively called (directly or indirectly) within the execution of $\lsample(\Lambda)$, we let $\suc(t)=\emptyset$.
\end{definition}

As the $q$-spin system changes from $\+S$ to $\+S'$ by an update,
the result of update at $t$ $(t\leq 0)$ which is obtained by $\eval^{\+O}(v_{i(t)})$ in \Cref{Alg:eval}
can change only when one of the following cases occurs:
\begin{itemize}
    \item $v_{i(t)} \in \vbl(D)$.
    \item The result of an update at $t'$ changes where $t\in \suc(t')$.
\end{itemize}
where the first case implies that $b_{v,u,c_i}$ in \Cref{Line:eval-cond} and $A_e(x,c_i)$ in \Cref{Line:eval-if} might be updated and the second case implies the result of $\resolve(\pred_t(u))$ for some $u$ in \Cref{Line:eval-if} might change.

Our dynamic sampling algorithm for updating $q$-spin systems is presented as follows.

\begin{algorithm}[H]
\caption{dynamic sampler: modifying the parameters of the spin system} \label{Alg:update-constraints}
\SetKwInput{KwData}{Global variables}
\SetKwInput{KwUpdate}{Update}
\SetKwInput{KwMaintain}{Maintain}
\KwIn{a $q$-spin system $\+S=(G=(V,E),(\bm{\lambda},\bm{A})))$, an update $(D,\Phi_D)$ which modifies $\+S$ to $\+S'$}
\KwData{two maps $M: \mathbb{Z}\to [q]\cup\{\perp\}, R:\mathbb{Z}\to \{(r_i,(r_{i,u})_{u\in N(v)})\}_{1\leq i<\infty}$; the query set $\Lambda\subseteq V$ and configuration $\DDS \in [q]^\Lambda$}
\KwMaintain{$\DDS \sim \mu^{\+S}(\Lambda)$}
\For{$\forall v\in \vbl(D)$}{
\lFor{$\forall t\in \call(v)$}{$\update(t)$}\label{Line:call-upd}
}
%\Return $X$\;
\end{algorithm}

\begin{algorithm}[H]
\caption{$\update(t)$} \label{Alg:update}
\SetKwInput{KwData}{Global variables}
\KwIn{the updated $q$-spin system $\+S'=(G=(V,E),(\bm{\lambda},\bm{A})))$, a timestamp $t\leq 0$, a set of variables $\Lambda'\subseteq V$}
\KwOut{the updated random value $X \in [q]$ updated at time $t$ in the process $\+P(T)$}
\KwData{two maps $M: \mathbb{Z}\to [q]\cup\{\perp\}, R:\mathbb{Z}\to \{(r_i,(r_{i,u})_{u\in N(v)})\}_{1\leq i<\infty}$; the query set $\Lambda\subseteq V$ and configuration $\DDS \in [q]^{\Lambda}$}
$M(t) \gets \perp$\;
$c \gets \resolve(t)$\;\label{Line:call-resolve}
\lIf{$t=\pred_0(v_{i(t)})$}{$\DDS(v_{i(t)}) \gets c$}
\lFor{$\forall t'\in \suc(t)$}{$\update(t')$}
\Return $c$\;
\end{algorithm}

In \Cref{Alg:update-constraints}, for all resolved updates at some time $t$ where $v_{i(t)}\in \vbl(D)$, we call $\update(t)$ to update the result we maintain. Whenever the result at a time $t$ is updated, we also recursively broadcast to all its successors $t'$ and call $\update(t')$.

%\begin{lemma}[correctness of \cref{Alg:update-constraints}]\label{lemma:dynamic-sampler-upd2-correctness}
%\end{lemma}
%\begin{lemma}[efficiency of \cref{Alg:update-constraints}]\label{lemma:dynamic-sampler-upd2-efficiency}
%\end{lemma}

%\begin{lemma}[Correctness of the global map]\label{lem:global-map-correctness}
%Let $R$ denote the global randomness, after the execution of \Cref{Alg:update-variables} or \Cref{Alg:update-constraints}, for all resolved updates $t(t\leq 0)$ where $M(t) \neq \perp$, let map $M_t$ be obtained by clearing all keys in $M$ with $t'\geq t$,
%it ensures 
%$$
%M(t) = \resolve^{R}(t;M_t,\+S).
%$$
%\end{lemma}

We present the following lemma that states the correctness of our algorithm upon updates for the spin system.
\begin{lemma}[correctness of the dynamic sampling algorithm]\label{lem:dynamic-sampling-correctness}
Assume the conditions of \Cref{theorem:dynamic-sampler}. After each the execution of \Cref{Alg:update-variables} or \Cref{Alg:update-constraints}, let $\DDS$ be the configuration maintained by \Cref{Alg:update-variables} and \Cref{Alg:update-constraints}, the random sequence at \Cref{Line:eval-sample} is
then $X\in [q]^{\Lambda}$ is distributed as $\mu^{\+S}_{\Lambda}$.
\end{lemma}

Note that the result $\DDS$ obtained after executing \Cref{Alg:update-variables} or \Cref{Alg:update-constraints} is equivalent to performing $\lsample(\Lambda)$, where for each $t\leq 0$, the random sequence at \Cref{Line:eval-sample} is predetermined and maintained in a global map. \Cref{remark:global-randomness} guarantees that the infinite long sequence $\{(c_i,(r_{i,u})_{u\in N(v)})\}_{1\leq i<\infty}$ is generated i.i.d. Hence, \Cref{lem:dynamic-sampling-correctness} is ensured by \Cref{lemma:lsample-correctness}.
%for any $T\subseteq V$, let $X\in [q]^T$ be constructed where for each $v\in T$, $X(v) \gets \resolve^{R}(\pred_0(v);M,\+S)$, then $\DDS$ is distributed as $\mu^{\+S}_{T}$.

We are now ready to prove \Cref{theorem:dynamic-sampler}.

\begin{proof}[Proof of \Cref{theorem:dynamic-sampler}]

Note that \Cref{item:thm-update-removing-query-vertices} is direct. We prove the remaining items.

We use \Cref{Alg:update-variables} to implement the update of adding query variables and use \Cref{Alg:update-constraints} to implement updates of the parameters of the spin system. for the dynamic sampling data structure. \Cref{lem:dynamic-sampling-correctness} ensures the correctness of the dynamic sampling data structure; it remains to ensure that the algorithm maintains the efficiency claimed in \Cref{theorem:dynamic-sampler}.

Comparing \Cref{Alg:update-variables} with \Cref{Alg:lsample}, the only difference is that the results at some time $t$ might be memorized by the map $M$, which only reduce the number of recursive calls, hence the number of recursive calls in \Cref{Alg:update-variables} does not exceed that in \Cref{Alg:lsample}. According to Lemmas \ref{lemma:eval-efficiency} and \ref{lemma:lsample-efficiency}, the expected number of $\resolve(t)$ calls by \Cref{Alg:update-variables} is $O(|\Lambda|)$, it proves \Cref{item:thm-update-adding-query-vertices} of \Cref{theorem:dynamic-sampler}.

To prove the efficiency of \Cref{Alg:update-constraints}, we first present the following technical lemmas, whose proofs are similar to those established in \Cref{sec:local}. %and are deferred to \Cref{sec:appendix-dynamic}.

\begin{lemma}\label{lem:dependency-probability}
Assume the $q$-spin system $\+S=(G=(V,E),(\bm{\lambda},\bm{A})))$ satisfies \Cref{cond:main} for some constant $0<\delta<1$ before and after the update, then for any resolved update at $t\leq 0$, let $v=v_{i(t)}$,
%and $R$ be the global randomness,
\[
\forall e=(u,v)\in E, \quad\Pr{t\in \suc(\pred_t(u))} \leq \frac{1-\delta}{\Delta}.
\]
where the randomness originates from $\{(c_i,(r_{i,u})_{u\in N(v)})\}_{1\leq i<\infty}$ which is generated in \Cref{Line:eval-sample} of $\eval^{\+O}(v)$.
\end{lemma}

\begin{proof}%[Proof of \Cref{lem:dependency-probability}]
%Suppose the $q$-spin system $\+S=(G=(V,E),(\bm{\lambda},\bm{A})))$ satisfies \Cref{cond:main} for some constant $0<\delta<1$ throughout the entire update process,

For any $t\leq 0$, let $v=v_{i(t)}$, for each $e=(u,v)\in E$, the expression $t\in \suc(\pred_t(u))$ means that $\resolve(\pred_t(u))$ in \Cref{Line:eval-if} of \Cref{Alg:eval} is invoked by $\eval^{\+O}(v)$ at least once throughout the entire update process. 
Suppose $\{(c_i,(r_{i,u})_{u\in N(v)})\}_{1\leq i<\infty}$ is the infinite random sequence generated in \Cref{Line:eval-sample} during the execution of $\eval^{\+O}(v)$ within $\resolve(t)$.
Given constant $0<\delta<1$, define $\+I_t^*$ as follow:
\[
\+I_t^* = \min \left\{i\mid \forall e=(u,v)\in E, r_{i,u}<1-\frac{1-\delta}{2\Delta}\right\}.
\]
For each $i\geq 1$, $\Pr{\forall e=(u,v)\in E, r_{i,u}<1-\frac{1-\delta}{2\Delta}}$ is at least $1- \frac{1-\delta}{2} = \frac{1+\delta}{2}$ by taking the union bound, thus
\begin{align}
    \E{\+I_t^*} \leq \frac{2}{1+\delta} < 2.
\end{align}

For each $e=(u,v)\in E$, the event $\+B_t^u$ is defined as follows:
\[
\+B_t^u: \quad \forall 1\leq i\leq \+I_t^*, r_{i,u} < 1-\frac{1-\delta}{2\Delta}.
\]
By the union bound, the probability $\neg \+B_t^u$ can be bounded as:
\begin{align}
\Pr{\neg \+B_t^u} \leq \E{\+I_t^*} \cdot \frac{1-\delta}{2\Delta}
\leq \frac{1-\delta}{\Delta} \label{eq:probability-event-B}
\end{align}

According to \Cref{remark:global-randomness}, the sequence $\{(r_{i,u})_{u\in N(v)}\}_{1\leq i<\infty}$ remains consistent during multiple calls to $\eval(t)$. 
Assume that the $q$-spin system $\+S=(G=(V,E),(\bm{\lambda},\bm{A})))$ satisfies \Cref{cond:main} for constant $0<\delta<1$ throughout the entire update process, for any calls to $\eval^{\+O}(v)$ within $\resolve(t)$, the loop in \Cref{Line:reject} is guaranteed to terminate within $\+I_t^*$ iterations because if it reaches the $\+I_t^*$th iteration, the expression $r_{i,u}\geq \lb$ in \Cref{Line:eval-cond} is always false, ensuring termination in that round. 

If $\+B_t^u$ occurs, the expression $r_{i,u}\geq \lb$ is always false in each iteration, then $\eval^{\+O}(v)$ will never invoke $\resolve(\pred_t(u))$ during the entire update process. 
Hence, $\+B_t^u$ implies $t\notin \suc(\pred_t(u))$, 
\[
\Pr{t\in \suc(\pred_t(u))}\leq \Pr{\neg \+B_t^*}.
\]
Due to \eqref{eq:probability-event-B}, \Cref{lem:dependency-probability} is established.
\end{proof}
%Thus, \Cref{theorem:dynamic-sampling-main} can be directly derived from \Cref{lemma:dynamic-sampler-upd1-correctness}, \Cref{lemma:lsample-efficiency}, \Cref{lemma:dynamic-sampler-upd2-correctness} and \Cref{lemma:dynamic-sampler-upd2-efficiency}

\begin{lemma}\label{lem:call-set-size}
Assume the $q$-spin system $\+S=(G=(V,E),(\bm{\lambda},\bm{A})))$ satisfies \Cref{cond:main} for some constant $0<\delta<1$ throughout the entire update process, 
%let $R$ be the global randomness,
\[
\E{ |\call(v)| } \leq 1/\delta .
\]
\end{lemma}

\begin{proof}%[Proof of \Cref{lem:call-set-size}]
For each recursive call of $\resolve(t)$ during the entire updating process, let $k$ represent the depth of this recursive call, record its recursive path using sequence $(t_0,t_1,...,t_k)$ such that $t_k=t$ and for each $1\leq i\leq k$, $\resolve(t_i)$ is invoked within $\resolve(t_{i-1})$ (through $\eval^{\+O}(v_{t_{i-1}})$). Let the sequence $(u_0,u_1,...,u_k)$ be the corresponding variable sequence where $u_j=v_{i(t_j)}$ for each $0\leq j\leq k$.

For any recursive call of $\resolve(t)$ with depth $i$, its recursive path $(t_0,t_1,...,t_k)$ and its corresponding variable sequence $(u_0,u_1,...,u_k)$ must satisfy the following conditions:
\begin{itemize}
    \item $\forall 1\leq i\leq k$, $u_i\in N(u_{i-1})$.
    \item $t_0=\pred_0(u_0)$ and $\forall 1\leq i\leq k, t_i = \pred_{t_{i-1}}(u_i)$.
\end{itemize}

If the above recursive path occurs throughout the entire updating process, it indicates:
\begin{equation}
\label{eq:dependency-chain}
\forall 1\leq i\leq k, t_{i-1} \in \suc(t_i).
\end{equation}

Recall that $\call(v)$ denotes the set of $t\leq 0$ such that $\resolve(t)$ is invoked at least once during the entire updating process, then $|\call(v)|$ is dominated by the number of recursive path which satisfies \Cref{eq:dependency-chain} and ends at $u_k=v$.

Fix $k$ be the depth of recursion, there are at most $\Delta^k$ variable sequence $(u_0,u_1,...,u_k)$ such that $u_k=v$ and $\forall 1\leq i\leq k,u_{i}\in N(u_{i-1})$. 
Note that once the variable sequence $(u_k,u_{k-1},...,u_1,u_0)$ is fixed, sequence $(t_0,t_1,...,t_k)$ is uniquely determined. According to \Cref{lem:dependency-probability}, \
\[
\Pr{\forall 1\leq i\leq k, t_{i-1} \in \suc(t_i)} \leq \left(\frac{1-\delta}{\Delta}\right)^k
\]

Hence,
\begin{equation}
    \E{|\call(v)|} \leq \sum\limits_{k\geq 0} \Delta^k \left(\frac{1-\delta}{\Delta}\right)^k = \sum\limits_{k\geq 0} (1-\delta)^k = \frac{1}{\delta}.
\end{equation}
\end{proof}

Similar to the proof of the running time of $\resolve(\pred_0(v))$ in \Cref{lemma:lsample-efficiency}, the behavior of $\update(t)$ can be stochastically dominated by the following multitype Galton-Watson branching process:
\begin{itemize}
    \item Start with a root node labelled with $t$ with depth $0$,
    \item for $i=0,1,\ldots$: for all current leaves labelled with some timestamp $t$ of depth $i$:
    \begin{itemize}
        \item Make an independent run of $\update(t)$, and for each timestamp $t' \in \suc(t)$, add an new node labeled with $t'$ as a child of $t$.
    \end{itemize}
\end{itemize}
For any timestamp $t\leq 0$, let $v=v_{i(t)}$, then for each $u\in N(v)$, there exists at most one $t'$ such that $v_{i(t')}=u$ and $\pred_{t'}(u) = t$. 
By \Cref{lem:dependency-probability}, the expected number of offsprings of a node with label $t$ equals $\E{|\suc(t)|} \leq |N(v)|\cdot \frac{1-\delta}{\Delta} \leq 1-\delta$. 
Then by leveraging the theory of branching processes, the expected number of nodes generated by this process is $\delta^{-1}=O(1)$. 
Combining with \Cref{lemma:lsample-efficiency}, the expected running time of $\update(t)$ is $O(1)$ for each $t\leq 0$.

Under the assumption of \Cref{lem:call-set-size}, the expected number of times \Cref{Alg:update-constraints} executes $\update(t)$ in \Cref{Line:call-upd} is $O(|D|)$. 
Hence, the expected running time of \Cref{Alg:update-constraints} for an update $(D,\Phi_D)$ is $O(|D|)=O(1)$, proving Items \ref{item:thm-update-external-field} and \ref{item:thm-update-interaction-matrix} of \Cref{theorem:dynamic-sampler}.
%Hence, the expected number of $\upd(t)$ calls within $\lsample(S)$ is $O(|S|)$, proving the lemma.
\end{proof}



%\subsection{Efficiency of the dynamic sampler}\label{sec:appendix-dynamic}

%In this section, we finish the omitted proofs of the technical lemmas regarding the efficiency of the dynamic sampler (\Cref{Alg:update-constraints}). Specifically, we will prove Lemmas \ref{lem:dependency-probability} and \ref{lem:call-set-size}.



