%%%%%%%% ICML 2024 EXAMPLE LATEX SUBMISSION FILE %%%%%%%%%%%%%%%%%

\documentclass{article}

% Recommended, but optional, packages for figures and better typesetting:
\usepackage{microtype}
\usepackage{graphicx}
\usepackage{booktabs} % for professional tables

% hyperref makes hyperlinks in the resulting PDF.
% If your build breaks (sometimes temporarily if a hyperlink spans a page)
% please comment out the following usepackage line and replace
% \usepackage{icml2024} with \usepackage[nohyperref]{icml2024} above.
\usepackage{hyperref}


% Attempt to make hyperref and algorithmic work together better:
\newcommand{\theHalgorithm}{\arabic{algorithm}}

% Use the following line for the initial blind version submitted for review:
% \usepackage{icml2025}

% If accepted, instead use the following line for the camera-ready submission:
\usepackage[preprint]{icml2025}

% For theorems and such
\usepackage{amsmath}
\usepackage{amssymb}
\usepackage{mathtools}
\usepackage{amsthm}
\usepackage{wrapfig}
\usepackage[textsize=tiny]{todonotes}
\usepackage{bbm}
\usepackage{placeins}
\usepackage{multirow}
\usepackage{subcaption}
% if you use cleveref..
\usepackage[capitalize,noabbrev]{cleveref}

%%%%%%%%%%%%%%%%%%%%%%%%%%%%%%%%
% THEOREMS
%%%%%%%%%%%%%%%%%%%%%%%%%%%%%%%%
\theoremstyle{plain}
\newtheorem{theorem}{Theorem}[section]
\newtheorem{proposition}[theorem]{Proposition}
\newtheorem{lemma}[theorem]{Lemma}
\newtheorem{corollary}[theorem]{Corollary}
\theoremstyle{definition}
\newtheorem{definition}[theorem]{Definition}
\newtheorem{assumption}[theorem]{Assumption}
\theoremstyle{remark}
\newtheorem{remark}[theorem]{Remark}


% Todonotes is useful during development; simply uncomment the next line
%    and comment out the line below the next line to turn off comments
%\usepackage[disable,textsize=tiny]{todonotes}
\usepackage{mathtools}
%%%%% NEW MATH DEFINITIONS %%%%%

\usepackage{amsmath,amsfonts,bm}
\usepackage{derivative}
% Mark sections of captions for referring to divisions of figures
\newcommand{\figleft}{{\em (Left)}}
\newcommand{\figcenter}{{\em (Center)}}
\newcommand{\figright}{{\em (Right)}}
\newcommand{\figtop}{{\em (Top)}}
\newcommand{\figbottom}{{\em (Bottom)}}
\newcommand{\captiona}{{\em (a)}}
\newcommand{\captionb}{{\em (b)}}
\newcommand{\captionc}{{\em (c)}}
\newcommand{\captiond}{{\em (d)}}

% Highlight a newly defined term
\newcommand{\newterm}[1]{{\bf #1}}

% Derivative d 
\newcommand{\deriv}{{\mathrm{d}}}

% Figure reference, lower-case.
\def\figref#1{figure~\ref{#1}}
% Figure reference, capital. For start of sentence
\def\Figref#1{Figure~\ref{#1}}
\def\twofigref#1#2{figures \ref{#1} and \ref{#2}}
\def\quadfigref#1#2#3#4{figures \ref{#1}, \ref{#2}, \ref{#3} and \ref{#4}}
% Section reference, lower-case.
\def\secref#1{section~\ref{#1}}
% Section reference, capital.
\def\Secref#1{Section~\ref{#1}}
% Reference to two sections.
\def\twosecrefs#1#2{sections \ref{#1} and \ref{#2}}
% Reference to three sections.
\def\secrefs#1#2#3{sections \ref{#1}, \ref{#2} and \ref{#3}}
% Reference to an equation, lower-case.
\def\eqref#1{equation~\ref{#1}}
% Reference to an equation, upper case
\def\Eqref#1{Equation~\ref{#1}}
% A raw reference to an equation---avoid using if possible
\def\plaineqref#1{\ref{#1}}
% Reference to a chapter, lower-case.
\def\chapref#1{chapter~\ref{#1}}
% Reference to an equation, upper case.
\def\Chapref#1{Chapter~\ref{#1}}
% Reference to a range of chapters
\def\rangechapref#1#2{chapters\ref{#1}--\ref{#2}}
% Reference to an algorithm, lower-case.
\def\algref#1{algorithm~\ref{#1}}
% Reference to an algorithm, upper case.
\def\Algref#1{Algorithm~\ref{#1}}
\def\twoalgref#1#2{algorithms \ref{#1} and \ref{#2}}
\def\Twoalgref#1#2{Algorithms \ref{#1} and \ref{#2}}
% Reference to a part, lower case
\def\partref#1{part~\ref{#1}}
% Reference to a part, upper case
\def\Partref#1{Part~\ref{#1}}
\def\twopartref#1#2{parts \ref{#1} and \ref{#2}}

\def\ceil#1{\lceil #1 \rceil}
\def\floor#1{\lfloor #1 \rfloor}
\def\1{\bm{1}}
\newcommand{\train}{\mathcal{D}}
\newcommand{\valid}{\mathcal{D_{\mathrm{valid}}}}
\newcommand{\test}{\mathcal{D_{\mathrm{test}}}}

\def\eps{{\epsilon}}


% Random variables
\def\reta{{\textnormal{$\eta$}}}
\def\ra{{\textnormal{a}}}
\def\rb{{\textnormal{b}}}
\def\rc{{\textnormal{c}}}
\def\rd{{\textnormal{d}}}
\def\re{{\textnormal{e}}}
\def\rf{{\textnormal{f}}}
\def\rg{{\textnormal{g}}}
\def\rh{{\textnormal{h}}}
\def\ri{{\textnormal{i}}}
\def\rj{{\textnormal{j}}}
\def\rk{{\textnormal{k}}}
\def\rl{{\textnormal{l}}}
% rm is already a command, just don't name any random variables m
\def\rn{{\textnormal{n}}}
\def\ro{{\textnormal{o}}}
\def\rp{{\textnormal{p}}}
\def\rq{{\textnormal{q}}}
\def\rr{{\textnormal{r}}}
\def\rs{{\textnormal{s}}}
\def\rt{{\textnormal{t}}}
\def\ru{{\textnormal{u}}}
\def\rv{{\textnormal{v}}}
\def\rw{{\textnormal{w}}}
\def\rx{{\textnormal{x}}}
\def\ry{{\textnormal{y}}}
\def\rz{{\textnormal{z}}}

% Random vectors
\def\rvepsilon{{\mathbf{\epsilon}}}
\def\rvphi{{\mathbf{\phi}}}
\def\rvtheta{{\mathbf{\theta}}}
\def\rva{{\mathbf{a}}}
\def\rvb{{\mathbf{b}}}
\def\rvc{{\mathbf{c}}}
\def\rvd{{\mathbf{d}}}
\def\rve{{\mathbf{e}}}
\def\rvf{{\mathbf{f}}}
\def\rvg{{\mathbf{g}}}
\def\rvh{{\mathbf{h}}}
\def\rvu{{\mathbf{i}}}
\def\rvj{{\mathbf{j}}}
\def\rvk{{\mathbf{k}}}
\def\rvl{{\mathbf{l}}}
\def\rvm{{\mathbf{m}}}
\def\rvn{{\mathbf{n}}}
\def\rvo{{\mathbf{o}}}
\def\rvp{{\mathbf{p}}}
\def\rvq{{\mathbf{q}}}
\def\rvr{{\mathbf{r}}}
\def\rvs{{\mathbf{s}}}
\def\rvt{{\mathbf{t}}}
\def\rvu{{\mathbf{u}}}
\def\rvv{{\mathbf{v}}}
\def\rvw{{\mathbf{w}}}
\def\rvx{{\mathbf{x}}}
\def\rvy{{\mathbf{y}}}
\def\rvz{{\mathbf{z}}}

% Elements of random vectors
\def\erva{{\textnormal{a}}}
\def\ervb{{\textnormal{b}}}
\def\ervc{{\textnormal{c}}}
\def\ervd{{\textnormal{d}}}
\def\erve{{\textnormal{e}}}
\def\ervf{{\textnormal{f}}}
\def\ervg{{\textnormal{g}}}
\def\ervh{{\textnormal{h}}}
\def\ervi{{\textnormal{i}}}
\def\ervj{{\textnormal{j}}}
\def\ervk{{\textnormal{k}}}
\def\ervl{{\textnormal{l}}}
\def\ervm{{\textnormal{m}}}
\def\ervn{{\textnormal{n}}}
\def\ervo{{\textnormal{o}}}
\def\ervp{{\textnormal{p}}}
\def\ervq{{\textnormal{q}}}
\def\ervr{{\textnormal{r}}}
\def\ervs{{\textnormal{s}}}
\def\ervt{{\textnormal{t}}}
\def\ervu{{\textnormal{u}}}
\def\ervv{{\textnormal{v}}}
\def\ervw{{\textnormal{w}}}
\def\ervx{{\textnormal{x}}}
\def\ervy{{\textnormal{y}}}
\def\ervz{{\textnormal{z}}}

% Random matrices
\def\rmA{{\mathbf{A}}}
\def\rmB{{\mathbf{B}}}
\def\rmC{{\mathbf{C}}}
\def\rmD{{\mathbf{D}}}
\def\rmE{{\mathbf{E}}}
\def\rmF{{\mathbf{F}}}
\def\rmG{{\mathbf{G}}}
\def\rmH{{\mathbf{H}}}
\def\rmI{{\mathbf{I}}}
\def\rmJ{{\mathbf{J}}}
\def\rmK{{\mathbf{K}}}
\def\rmL{{\mathbf{L}}}
\def\rmM{{\mathbf{M}}}
\def\rmN{{\mathbf{N}}}
\def\rmO{{\mathbf{O}}}
\def\rmP{{\mathbf{P}}}
\def\rmQ{{\mathbf{Q}}}
\def\rmR{{\mathbf{R}}}
\def\rmS{{\mathbf{S}}}
\def\rmT{{\mathbf{T}}}
\def\rmU{{\mathbf{U}}}
\def\rmV{{\mathbf{V}}}
\def\rmW{{\mathbf{W}}}
\def\rmX{{\mathbf{X}}}
\def\rmY{{\mathbf{Y}}}
\def\rmZ{{\mathbf{Z}}}

% Elements of random matrices
\def\ermA{{\textnormal{A}}}
\def\ermB{{\textnormal{B}}}
\def\ermC{{\textnormal{C}}}
\def\ermD{{\textnormal{D}}}
\def\ermE{{\textnormal{E}}}
\def\ermF{{\textnormal{F}}}
\def\ermG{{\textnormal{G}}}
\def\ermH{{\textnormal{H}}}
\def\ermI{{\textnormal{I}}}
\def\ermJ{{\textnormal{J}}}
\def\ermK{{\textnormal{K}}}
\def\ermL{{\textnormal{L}}}
\def\ermM{{\textnormal{M}}}
\def\ermN{{\textnormal{N}}}
\def\ermO{{\textnormal{O}}}
\def\ermP{{\textnormal{P}}}
\def\ermQ{{\textnormal{Q}}}
\def\ermR{{\textnormal{R}}}
\def\ermS{{\textnormal{S}}}
\def\ermT{{\textnormal{T}}}
\def\ermU{{\textnormal{U}}}
\def\ermV{{\textnormal{V}}}
\def\ermW{{\textnormal{W}}}
\def\ermX{{\textnormal{X}}}
\def\ermY{{\textnormal{Y}}}
\def\ermZ{{\textnormal{Z}}}

% Vectors
\def\vzero{{\bm{0}}}
\def\vone{{\bm{1}}}
\def\vmu{{\bm{\mu}}}
\def\vtheta{{\bm{\theta}}}
\def\vphi{{\bm{\phi}}}
\def\va{{\bm{a}}}
\def\vb{{\bm{b}}}
\def\vc{{\bm{c}}}
\def\vd{{\bm{d}}}
\def\ve{{\bm{e}}}
\def\vf{{\bm{f}}}
\def\vg{{\bm{g}}}
\def\vh{{\bm{h}}}
\def\vi{{\bm{i}}}
\def\vj{{\bm{j}}}
\def\vk{{\bm{k}}}
\def\vl{{\bm{l}}}
\def\vm{{\bm{m}}}
\def\vn{{\bm{n}}}
\def\vo{{\bm{o}}}
\def\vp{{\bm{p}}}
\def\vq{{\bm{q}}}
\def\vr{{\bm{r}}}
\def\vs{{\bm{s}}}
\def\vt{{\bm{t}}}
\def\vu{{\bm{u}}}
\def\vv{{\bm{v}}}
\def\vw{{\bm{w}}}
\def\vx{{\bm{x}}}
\def\vy{{\bm{y}}}
\def\vz{{\bm{z}}}

% Elements of vectors
\def\evalpha{{\alpha}}
\def\evbeta{{\beta}}
\def\evepsilon{{\epsilon}}
\def\evlambda{{\lambda}}
\def\evomega{{\omega}}
\def\evmu{{\mu}}
\def\evpsi{{\psi}}
\def\evsigma{{\sigma}}
\def\evtheta{{\theta}}
\def\eva{{a}}
\def\evb{{b}}
\def\evc{{c}}
\def\evd{{d}}
\def\eve{{e}}
\def\evf{{f}}
\def\evg{{g}}
\def\evh{{h}}
\def\evi{{i}}
\def\evj{{j}}
\def\evk{{k}}
\def\evl{{l}}
\def\evm{{m}}
\def\evn{{n}}
\def\evo{{o}}
\def\evp{{p}}
\def\evq{{q}}
\def\evr{{r}}
\def\evs{{s}}
\def\evt{{t}}
\def\evu{{u}}
\def\evv{{v}}
\def\evw{{w}}
\def\evx{{x}}
\def\evy{{y}}
\def\evz{{z}}

% Matrix
\def\mA{{\bm{A}}}
\def\mB{{\bm{B}}}
\def\mC{{\bm{C}}}
\def\mD{{\bm{D}}}
\def\mE{{\bm{E}}}
\def\mF{{\bm{F}}}
\def\mG{{\bm{G}}}
\def\mH{{\bm{H}}}
\def\mI{{\bm{I}}}
\def\mJ{{\bm{J}}}
\def\mK{{\bm{K}}}
\def\mL{{\bm{L}}}
\def\mM{{\bm{M}}}
\def\mN{{\bm{N}}}
\def\mO{{\bm{O}}}
\def\mP{{\bm{P}}}
\def\mQ{{\bm{Q}}}
\def\mR{{\bm{R}}}
\def\mS{{\bm{S}}}
\def\mT{{\bm{T}}}
\def\mU{{\bm{U}}}
\def\mV{{\bm{V}}}
\def\mW{{\bm{W}}}
\def\mX{{\bm{X}}}
\def\mY{{\bm{Y}}}
\def\mZ{{\bm{Z}}}
\def\mBeta{{\bm{\beta}}}
\def\mPhi{{\bm{\Phi}}}
\def\mLambda{{\bm{\Lambda}}}
\def\mSigma{{\bm{\Sigma}}}

% Tensor
\DeclareMathAlphabet{\mathsfit}{\encodingdefault}{\sfdefault}{m}{sl}
\SetMathAlphabet{\mathsfit}{bold}{\encodingdefault}{\sfdefault}{bx}{n}
\newcommand{\tens}[1]{\bm{\mathsfit{#1}}}
\def\tA{{\tens{A}}}
\def\tB{{\tens{B}}}
\def\tC{{\tens{C}}}
\def\tD{{\tens{D}}}
\def\tE{{\tens{E}}}
\def\tF{{\tens{F}}}
\def\tG{{\tens{G}}}
\def\tH{{\tens{H}}}
\def\tI{{\tens{I}}}
\def\tJ{{\tens{J}}}
\def\tK{{\tens{K}}}
\def\tL{{\tens{L}}}
\def\tM{{\tens{M}}}
\def\tN{{\tens{N}}}
\def\tO{{\tens{O}}}
\def\tP{{\tens{P}}}
\def\tQ{{\tens{Q}}}
\def\tR{{\tens{R}}}
\def\tS{{\tens{S}}}
\def\tT{{\tens{T}}}
\def\tU{{\tens{U}}}
\def\tV{{\tens{V}}}
\def\tW{{\tens{W}}}
\def\tX{{\tens{X}}}
\def\tY{{\tens{Y}}}
\def\tZ{{\tens{Z}}}


% Graph
\def\gA{{\mathcal{A}}}
\def\gB{{\mathcal{B}}}
\def\gC{{\mathcal{C}}}
\def\gD{{\mathcal{D}}}
\def\gE{{\mathcal{E}}}
\def\gF{{\mathcal{F}}}
\def\gG{{\mathcal{G}}}
\def\gH{{\mathcal{H}}}
\def\gI{{\mathcal{I}}}
\def\gJ{{\mathcal{J}}}
\def\gK{{\mathcal{K}}}
\def\gL{{\mathcal{L}}}
\def\gM{{\mathcal{M}}}
\def\gN{{\mathcal{N}}}
\def\gO{{\mathcal{O}}}
\def\gP{{\mathcal{P}}}
\def\gQ{{\mathcal{Q}}}
\def\gR{{\mathcal{R}}}
\def\gS{{\mathcal{S}}}
\def\gT{{\mathcal{T}}}
\def\gU{{\mathcal{U}}}
\def\gV{{\mathcal{V}}}
\def\gW{{\mathcal{W}}}
\def\gX{{\mathcal{X}}}
\def\gY{{\mathcal{Y}}}
\def\gZ{{\mathcal{Z}}}

% Sets
\def\sA{{\mathbb{A}}}
\def\sB{{\mathbb{B}}}
\def\sC{{\mathbb{C}}}
\def\sD{{\mathbb{D}}}
% Don't use a set called E, because this would be the same as our symbol
% for expectation.
\def\sF{{\mathbb{F}}}
\def\sG{{\mathbb{G}}}
\def\sH{{\mathbb{H}}}
\def\sI{{\mathbb{I}}}
\def\sJ{{\mathbb{J}}}
\def\sK{{\mathbb{K}}}
\def\sL{{\mathbb{L}}}
\def\sM{{\mathbb{M}}}
\def\sN{{\mathbb{N}}}
\def\sO{{\mathbb{O}}}
\def\sP{{\mathbb{P}}}
\def\sQ{{\mathbb{Q}}}
\def\sR{{\mathbb{R}}}
\def\sS{{\mathbb{S}}}
\def\sT{{\mathbb{T}}}
\def\sU{{\mathbb{U}}}
\def\sV{{\mathbb{V}}}
\def\sW{{\mathbb{W}}}
\def\sX{{\mathbb{X}}}
\def\sY{{\mathbb{Y}}}
\def\sZ{{\mathbb{Z}}}

% Entries of a matrix
\def\emLambda{{\Lambda}}
\def\emA{{A}}
\def\emB{{B}}
\def\emC{{C}}
\def\emD{{D}}
\def\emE{{E}}
\def\emF{{F}}
\def\emG{{G}}
\def\emH{{H}}
\def\emI{{I}}
\def\emJ{{J}}
\def\emK{{K}}
\def\emL{{L}}
\def\emM{{M}}
\def\emN{{N}}
\def\emO{{O}}
\def\emP{{P}}
\def\emQ{{Q}}
\def\emR{{R}}
\def\emS{{S}}
\def\emT{{T}}
\def\emU{{U}}
\def\emV{{V}}
\def\emW{{W}}
\def\emX{{X}}
\def\emY{{Y}}
\def\emZ{{Z}}
\def\emSigma{{\Sigma}}

% entries of a tensor
% Same font as tensor, without \bm wrapper
\newcommand{\etens}[1]{\mathsfit{#1}}
\def\etLambda{{\etens{\Lambda}}}
\def\etA{{\etens{A}}}
\def\etB{{\etens{B}}}
\def\etC{{\etens{C}}}
\def\etD{{\etens{D}}}
\def\etE{{\etens{E}}}
\def\etF{{\etens{F}}}
\def\etG{{\etens{G}}}
\def\etH{{\etens{H}}}
\def\etI{{\etens{I}}}
\def\etJ{{\etens{J}}}
\def\etK{{\etens{K}}}
\def\etL{{\etens{L}}}
\def\etM{{\etens{M}}}
\def\etN{{\etens{N}}}
\def\etO{{\etens{O}}}
\def\etP{{\etens{P}}}
\def\etQ{{\etens{Q}}}
\def\etR{{\etens{R}}}
\def\etS{{\etens{S}}}
\def\etT{{\etens{T}}}
\def\etU{{\etens{U}}}
\def\etV{{\etens{V}}}
\def\etW{{\etens{W}}}
\def\etX{{\etens{X}}}
\def\etY{{\etens{Y}}}
\def\etZ{{\etens{Z}}}

% The true underlying data generating distribution
\newcommand{\pdata}{p_{\rm{data}}}
\newcommand{\ptarget}{p_{\rm{target}}}
\newcommand{\pprior}{p_{\rm{prior}}}
\newcommand{\pbase}{p_{\rm{base}}}
\newcommand{\pref}{p_{\rm{ref}}}

% The empirical distribution defined by the training set
\newcommand{\ptrain}{\hat{p}_{\rm{data}}}
\newcommand{\Ptrain}{\hat{P}_{\rm{data}}}
% The model distribution
\newcommand{\pmodel}{p_{\rm{model}}}
\newcommand{\Pmodel}{P_{\rm{model}}}
\newcommand{\ptildemodel}{\tilde{p}_{\rm{model}}}
% Stochastic autoencoder distributions
\newcommand{\pencode}{p_{\rm{encoder}}}
\newcommand{\pdecode}{p_{\rm{decoder}}}
\newcommand{\precons}{p_{\rm{reconstruct}}}

\newcommand{\laplace}{\mathrm{Laplace}} % Laplace distribution

\newcommand{\E}{\mathbb{E}}
\newcommand{\Ls}{\mathcal{L}}
\newcommand{\R}{\mathbb{R}}
\newcommand{\emp}{\tilde{p}}
\newcommand{\lr}{\alpha}
\newcommand{\reg}{\lambda}
\newcommand{\rect}{\mathrm{rectifier}}
\newcommand{\softmax}{\mathrm{softmax}}
\newcommand{\sigmoid}{\sigma}
\newcommand{\softplus}{\zeta}
\newcommand{\KL}{D_{\mathrm{KL}}}
\newcommand{\Var}{\mathrm{Var}}
\newcommand{\standarderror}{\mathrm{SE}}
\newcommand{\Cov}{\mathrm{Cov}}
% Wolfram Mathworld says $L^2$ is for function spaces and $\ell^2$ is for vectors
% But then they seem to use $L^2$ for vectors throughout the site, and so does
% wikipedia.
\newcommand{\normlzero}{L^0}
\newcommand{\normlone}{L^1}
\newcommand{\normltwo}{L^2}
\newcommand{\normlp}{L^p}
\newcommand{\normmax}{L^\infty}

\newcommand{\parents}{Pa} % See usage in notation.tex. Chosen to match Daphne's book.

\DeclareMathOperator*{\argmax}{arg\,max}
\DeclareMathOperator*{\argmin}{arg\,min}

\DeclareMathOperator{\sign}{sign}
\DeclareMathOperator{\Tr}{Tr}
\let\ab\allowbreak



% The \icmltitle you define below is probably too long as a header.
% Therefore, a short form for the running title is supplied here:
\icmltitlerunning{Membership Inference Risks in Quantized Models: A Theoretical and Empirical Study}

\begin{document}
\twocolumn[
\icmltitle{Membership Inference Risks in Quantized Models: \\ A Theoretical and Empirical Study}

% It is OKAY to include author information, even for blind
% submissions: the style file will automatically remove it for you
% unless you've provided the [accepted] option to the icml2024
% package.

% List of affiliations: The first argument should be a (short)
% identifier you will use later to specify author affiliations
% Academic affiliations should list Department, University, City, Region, Country
% Industry affiliations should list Company, City, Region, Country

% You can specify symbols, otherwise they are numbered in order.
% Ideally, you should not use this facility. Affiliations will be numbered
% in order of appearance and this is the preferred way.
\icmlsetsymbol{equal}{*}

\begin{icmlauthorlist}
\icmlauthor{Eric Aubinais}{equal,ups,lmo}
\icmlauthor{Philippe Formont}{equal,ups,ets,mila}
\icmlauthor{Pablo Piantanida}{ups,ets,mila,cnrs}
\icmlauthor{Elisabeth Gassiat}{ups,lmo}
\end{icmlauthorlist}

\icmlaffiliation{ups}{Université Paris-Saclay}
\icmlaffiliation{lmo}{Laboratoire de mathématiques d’Orsay, France}
\icmlaffiliation{ets}{Ecole de technologie superieur, ILLS - International Laboratory on Learning Systems, Montreal, Canada}
\icmlaffiliation{mila}{MILA - Quebec AI Institute, Montreal, Canada}
\icmlaffiliation{cnrs}{CNRS, CentraleSupélec}

\icmlcorrespondingauthor{Philippe Formont}{philippe.formont@mila.quebec}
\icmlcorrespondingauthor{Eric Aubinais}{eric.aubinais@universite-paris-saclay.fr}

% You may provide any keywords that you
% find helpful for describing your paper; these are used to populate
% the "keywords" metadata in the PDF but will not be shown in the document
\icmlkeywords{Machine Learning, ICML}

\vskip 0.3in
]

% this must go after the closing bracket ] following \twocolumn[ ...

% This command actually creates the footnote in the first column
% listing the affiliations and the copyright notice.
% The command takes one argument, which is text to display at the start of the footnote.
% The \icmlEqualContribution command is standard text for equal contribution.
% Remove it (just {}) if you do not need this facility.

%\printAffiliationsAndNotice{}  % leave blank if no need to mention equal contribution
\printAffiliationsAndNotice{\icmlEqualContribution} % otherwise use the standard text.



%
\setlength\unitlength{1mm}
\newcommand{\twodots}{\mathinner {\ldotp \ldotp}}
% bb font symbols
\newcommand{\Rho}{\mathrm{P}}
\newcommand{\Tau}{\mathrm{T}}

\newfont{\bbb}{msbm10 scaled 700}
\newcommand{\CCC}{\mbox{\bbb C}}

\newfont{\bb}{msbm10 scaled 1100}
\newcommand{\CC}{\mbox{\bb C}}
\newcommand{\PP}{\mbox{\bb P}}
\newcommand{\RR}{\mbox{\bb R}}
\newcommand{\QQ}{\mbox{\bb Q}}
\newcommand{\ZZ}{\mbox{\bb Z}}
\newcommand{\FF}{\mbox{\bb F}}
\newcommand{\GG}{\mbox{\bb G}}
\newcommand{\EE}{\mbox{\bb E}}
\newcommand{\NN}{\mbox{\bb N}}
\newcommand{\KK}{\mbox{\bb K}}
\newcommand{\HH}{\mbox{\bb H}}
\newcommand{\SSS}{\mbox{\bb S}}
\newcommand{\UU}{\mbox{\bb U}}
\newcommand{\VV}{\mbox{\bb V}}


\newcommand{\yy}{\mathbbm{y}}
\newcommand{\xx}{\mathbbm{x}}
\newcommand{\zz}{\mathbbm{z}}
\newcommand{\sss}{\mathbbm{s}}
\newcommand{\rr}{\mathbbm{r}}
\newcommand{\pp}{\mathbbm{p}}
\newcommand{\qq}{\mathbbm{q}}
\newcommand{\ww}{\mathbbm{w}}
\newcommand{\hh}{\mathbbm{h}}
\newcommand{\vvv}{\mathbbm{v}}

% Vectors

\newcommand{\av}{{\bf a}}
\newcommand{\bv}{{\bf b}}
\newcommand{\cv}{{\bf c}}
\newcommand{\dv}{{\bf d}}
\newcommand{\ev}{{\bf e}}
\newcommand{\fv}{{\bf f}}
\newcommand{\gv}{{\bf g}}
\newcommand{\hv}{{\bf h}}
\newcommand{\iv}{{\bf i}}
\newcommand{\jv}{{\bf j}}
\newcommand{\kv}{{\bf k}}
\newcommand{\lv}{{\bf l}}
\newcommand{\mv}{{\bf m}}
\newcommand{\nv}{{\bf n}}
\newcommand{\ov}{{\bf o}}
\newcommand{\pv}{{\bf p}}
\newcommand{\qv}{{\bf q}}
\newcommand{\rv}{{\bf r}}
\newcommand{\sv}{{\bf s}}
\newcommand{\tv}{{\bf t}}
\newcommand{\uv}{{\bf u}}
\newcommand{\wv}{{\bf w}}
\newcommand{\vv}{{\bf v}}
\newcommand{\xv}{{\bf x}}
\newcommand{\yv}{{\bf y}}
\newcommand{\zv}{{\bf z}}
\newcommand{\zerov}{{\bf 0}}
\newcommand{\onev}{{\bf 1}}

% Matrices

\newcommand{\Am}{{\bf A}}
\newcommand{\Bm}{{\bf B}}
\newcommand{\Cm}{{\bf C}}
\newcommand{\Dm}{{\bf D}}
\newcommand{\Em}{{\bf E}}
\newcommand{\Fm}{{\bf F}}
\newcommand{\Gm}{{\bf G}}
\newcommand{\Hm}{{\bf H}}
\newcommand{\Id}{{\bf I}}
\newcommand{\Jm}{{\bf J}}
\newcommand{\Km}{{\bf K}}
\newcommand{\Lm}{{\bf L}}
\newcommand{\Mm}{{\bf M}}
\newcommand{\Nm}{{\bf N}}
\newcommand{\Om}{{\bf O}}
\newcommand{\Pm}{{\bf P}}
\newcommand{\Qm}{{\bf Q}}
\newcommand{\Rm}{{\bf R}}
\newcommand{\Sm}{{\bf S}}
\newcommand{\Tm}{{\bf T}}
\newcommand{\Um}{{\bf U}}
\newcommand{\Wm}{{\bf W}}
\newcommand{\Vm}{{\bf V}}
\newcommand{\Xm}{{\bf X}}
\newcommand{\Ym}{{\bf Y}}
\newcommand{\Zm}{{\bf Z}}

% Calligraphic

\newcommand{\Ac}{{\cal A}}
\newcommand{\Bc}{{\cal B}}
\newcommand{\Cc}{{\cal C}}
\newcommand{\Dc}{{\cal D}}
\newcommand{\Ec}{{\cal E}}
\newcommand{\Fc}{{\cal F}}
\newcommand{\Gc}{{\cal G}}
\newcommand{\Hc}{{\cal H}}
\newcommand{\Ic}{{\cal I}}
\newcommand{\Jc}{{\cal J}}
\newcommand{\Kc}{{\cal K}}
\newcommand{\Lc}{{\cal L}}
\newcommand{\Mc}{{\cal M}}
\newcommand{\Nc}{{\cal N}}
\newcommand{\nc}{{\cal n}}
\newcommand{\Oc}{{\cal O}}
\newcommand{\Pc}{{\cal P}}
\newcommand{\Qc}{{\cal Q}}
\newcommand{\Rc}{{\cal R}}
\newcommand{\Sc}{{\cal S}}
\newcommand{\Tc}{{\cal T}}
\newcommand{\Uc}{{\cal U}}
\newcommand{\Wc}{{\cal W}}
\newcommand{\Vc}{{\cal V}}
\newcommand{\Xc}{{\cal X}}
\newcommand{\Yc}{{\cal Y}}
\newcommand{\Zc}{{\cal Z}}

% Bold greek letters

\newcommand{\alphav}{\hbox{\boldmath$\alpha$}}
\newcommand{\betav}{\hbox{\boldmath$\beta$}}
\newcommand{\gammav}{\hbox{\boldmath$\gamma$}}
\newcommand{\deltav}{\hbox{\boldmath$\delta$}}
\newcommand{\etav}{\hbox{\boldmath$\eta$}}
\newcommand{\lambdav}{\hbox{\boldmath$\lambda$}}
\newcommand{\epsilonv}{\hbox{\boldmath$\epsilon$}}
\newcommand{\nuv}{\hbox{\boldmath$\nu$}}
\newcommand{\muv}{\hbox{\boldmath$\mu$}}
\newcommand{\zetav}{\hbox{\boldmath$\zeta$}}
\newcommand{\phiv}{\hbox{\boldmath$\phi$}}
\newcommand{\psiv}{\hbox{\boldmath$\psi$}}
\newcommand{\thetav}{\hbox{\boldmath$\theta$}}
\newcommand{\tauv}{\hbox{\boldmath$\tau$}}
\newcommand{\omegav}{\hbox{\boldmath$\omega$}}
\newcommand{\xiv}{\hbox{\boldmath$\xi$}}
\newcommand{\sigmav}{\hbox{\boldmath$\sigma$}}
\newcommand{\piv}{\hbox{\boldmath$\pi$}}
\newcommand{\rhov}{\hbox{\boldmath$\rho$}}
\newcommand{\upsilonv}{\hbox{\boldmath$\upsilon$}}

\newcommand{\Gammam}{\hbox{\boldmath$\Gamma$}}
\newcommand{\Lambdam}{\hbox{\boldmath$\Lambda$}}
\newcommand{\Deltam}{\hbox{\boldmath$\Delta$}}
\newcommand{\Sigmam}{\hbox{\boldmath$\Sigma$}}
\newcommand{\Phim}{\hbox{\boldmath$\Phi$}}
\newcommand{\Pim}{\hbox{\boldmath$\Pi$}}
\newcommand{\Psim}{\hbox{\boldmath$\Psi$}}
\newcommand{\Thetam}{\hbox{\boldmath$\Theta$}}
\newcommand{\Omegam}{\hbox{\boldmath$\Omega$}}
\newcommand{\Xim}{\hbox{\boldmath$\Xi$}}


% Sans Serif small case

\newcommand{\Gsf}{{\sf G}}

\newcommand{\asf}{{\sf a}}
\newcommand{\bsf}{{\sf b}}
\newcommand{\csf}{{\sf c}}
\newcommand{\dsf}{{\sf d}}
\newcommand{\esf}{{\sf e}}
\newcommand{\fsf}{{\sf f}}
\newcommand{\gsf}{{\sf g}}
\newcommand{\hsf}{{\sf h}}
\newcommand{\isf}{{\sf i}}
\newcommand{\jsf}{{\sf j}}
\newcommand{\ksf}{{\sf k}}
\newcommand{\lsf}{{\sf l}}
\newcommand{\msf}{{\sf m}}
\newcommand{\nsf}{{\sf n}}
\newcommand{\osf}{{\sf o}}
\newcommand{\psf}{{\sf p}}
\newcommand{\qsf}{{\sf q}}
\newcommand{\rsf}{{\sf r}}
\newcommand{\ssf}{{\sf s}}
\newcommand{\tsf}{{\sf t}}
\newcommand{\usf}{{\sf u}}
\newcommand{\wsf}{{\sf w}}
\newcommand{\vsf}{{\sf v}}
\newcommand{\xsf}{{\sf x}}
\newcommand{\ysf}{{\sf y}}
\newcommand{\zsf}{{\sf z}}


% mixed symbols

\newcommand{\sinc}{{\hbox{sinc}}}
\newcommand{\diag}{{\hbox{diag}}}
\renewcommand{\det}{{\hbox{det}}}
\newcommand{\trace}{{\hbox{tr}}}
\newcommand{\sign}{{\hbox{sign}}}
\renewcommand{\arg}{{\hbox{arg}}}
\newcommand{\var}{{\hbox{var}}}
\newcommand{\cov}{{\hbox{cov}}}
\newcommand{\Ei}{{\rm E}_{\rm i}}
\renewcommand{\Re}{{\rm Re}}
\renewcommand{\Im}{{\rm Im}}
\newcommand{\eqdef}{\stackrel{\Delta}{=}}
\newcommand{\defines}{{\,\,\stackrel{\scriptscriptstyle \bigtriangleup}{=}\,\,}}
\newcommand{\<}{\left\langle}
\renewcommand{\>}{\right\rangle}
\newcommand{\herm}{{\sf H}}
\newcommand{\trasp}{{\sf T}}
\newcommand{\transp}{{\sf T}}
\renewcommand{\vec}{{\rm vec}}
\newcommand{\Psf}{{\sf P}}
\newcommand{\SINR}{{\sf SINR}}
\newcommand{\SNR}{{\sf SNR}}
\newcommand{\MMSE}{{\sf MMSE}}
\newcommand{\REF}{{\RED [REF]}}

% Markov chain
\usepackage{stmaryrd} % for \mkv 
\newcommand{\mkv}{-\!\!\!\!\minuso\!\!\!\!-}

% Colors

\newcommand{\RED}{\color[rgb]{1.00,0.10,0.10}}
\newcommand{\BLUE}{\color[rgb]{0,0,0.90}}
\newcommand{\GREEN}{\color[rgb]{0,0.80,0.20}}

%%%%%%%%%%%%%%%%%%%%%%%%%%%%%%%%%%%%%%%%%%
\usepackage{hyperref}
\hypersetup{
    bookmarks=true,         % show bookmarks bar?
    unicode=false,          % non-Latin characters in AcrobatÕs bookmarks
    pdftoolbar=true,        % show AcrobatÕs toolbar?
    pdfmenubar=true,        % show AcrobatÕs menu?
    pdffitwindow=false,     % window fit to page when opened
    pdfstartview={FitH},    % fits the width of the page to the window
%    pdftitle={My title},    % title
%    pdfauthor={Author},     % author
%    pdfsubject={Subject},   % subject of the document
%    pdfcreator={Creator},   % creator of the document
%    pdfproducer={Producer}, % producer of the document
%    pdfkeywords={keyword1} {key2} {key3}, % list of keywords
    pdfnewwindow=true,      % links in new window
    colorlinks=true,       % false: boxed links; true: colored links
    linkcolor=red,          % color of internal links (change box color with linkbordercolor)
    citecolor=green,        % color of links to bibliography
    filecolor=blue,      % color of file links
    urlcolor=blue           % color of external links
}
%%%%%%%%%%%%%%%%%%%%%%%%%%%%%%%%%%%%%%%%%%%



\begin{abstract}
    Since 2020, GitGuardian has been detecting checked-in hard-coded secrets in GitHub repositories. During 2020-2023, GitGuardian has observed an upward annual trend and a four-fold increase in hard-coded secrets, with 12.8 million exposed in 2023. However, removing all the secrets from software artifacts is not feasible due to time constraints and technical challenges. Additionally, the security risks of the secrets are not equal, protecting assets ranging from obsolete databases to sensitive medical data. Thus, secret removal should be prioritized by security risk reduction, which existing secret detection tools do not support. \textit{The goal of this research is to aid software practitioners in prioritizing secrets removal efforts through our security risk-based tool}. We present RiskHarvester, a risk-based tool to compute a security risk score based on the value of the asset and ease of attack on a database. We calculated the value of asset by identifying the sensitive data categories present in a database from the database keywords in the source code. We utilized data flow analysis, SQL, and Object Relational Mapper (ORM) parsing to identify the database keywords. To calculate the ease of attack, we utilized passive network analysis to retrieve the database host information. To evaluate RiskHarvester, we curated RiskBench, a benchmark of 1,791 database secret-asset pairs with sensitive data categories and host information manually retrieved from 188 GitHub repositories. RiskHarvester demonstrates precision of (95\%) and recall (90\%) in detecting database keywords for the value of asset and precision of (96\%) and recall (94\%) in detecting valid hosts for ease of attack. Finally, we conducted a survey (52 respondents) to understand whether developers prioritize secret removal based on security risk score. We found that 86\% of the developers prioritized the secrets for removal with descending security risk scores.
\end{abstract}

\section{Introduction}
\label{sec:intro}
% A probably terrible starting point, but a starting point still \o/


Reducing the computational and memory costs of machine learning models is a critical aspect of their deployment, particularly on edge devices and resource-constrained environments. Quantization stands out among the various methods available to enhance inference efficiency in neural networks, such as knowledge distillation and pruning, due to its distinct advantages and proven practical success~\cite{gholami2022survey}. One key benefit is that the storage and latency improvements achieved through quantization are deterministically defined by the chosen quantization level (e.g., using 8-bit integers instead of 32-bit floating-point numbers). Moreover, uniform quantization is inherently hardware-friendly, facilitating the practical realization of theoretical efficiency gains.

%A widely adopted strategy to achieve this is model quantization, where the precision of data types storing the model's weights and activations is reduced, for instance, using 8-bit integers instead of 32-bit floating-point numbers. This approach not only decreases memory usage, but also accelerates computation, enabling the deployment of large-scale models with minimal loss in performance.

While quantization effectively improves efficiency, its impact on the privacy of machine learning models remains largely under-explored. A particularly intriguing question is whether quantization can also strengthen a model's resilience against adversarial threats, such as the extraction of sensitive information. By reducing the precision of a model’s parameters, quantization naturally discards some information~\cite{6451278}, which leads to the hypothesis that this process could potentially reduce the risk of recovering the model’s training data or other private information. However, to our knowledge, the security of quantized models against such privacy attacks has not yet been theoretically investigated.

%Although quantization addresses efficiency, its implications for the privacy of machine learning models remain unexplored. Specifically, an intriguing question arises: can quantization also enhance the model's resilience against adversarial threats, such as the extraction of sensitive information? Intuitively, by reducing the precision of a model’s parameters, quantization inherently discards some information. This raises the hypothesis that such a process could mitigate the risk of recovering the model’s training set or other private data. However, to the best of our knowledge, the security of quantized models against such attacks has yet to be studied.


In this paper, we explore the effect of quantization on the Membership Inference vulnerability of machine learning models. We propose a novel privacy metric, based upon the Membership Inference Security (MIS) \cite{aubinais2023fundamental}  and derive asymptotic bounds—relative to the training sample size—to quantify the privacy implications of model quantization. Using these theoretical insights, we present a systematic framework for evaluating and comparing fundamental quantization techniques, with a focus on the observed loss values. This methodology provides a thorough analysis of how different quantization methods strike a balance between privacy (against the most powerful attacks) and performance. To validate our approach, we compare it against an established baseline technique, showing consistent rankings of quantization methods based on their ability to preserve privacy.

%In this paper, we study the effect of quantization on the Memebership Inference level of a machine learning model. Specifically, we provide asymptotic bounds on the privacy level of a machine learning model that has been quantized, based on an interpretable metric called \textbf{Membership Inference Security} (MIS). We exploit the theoretical results to establish a systemic methodology to compare several quantization methods, wholly based on the different observed loss values, leading to a comprehensive overview on the capability of different quantization methods to provide security while preserving efficacy. We validate our novel methodology by comparing its results to those of some \textit{baseline methodology} on which we show consistency on the order displayed between the different quantizations. 

\subsection{Our contributions}
Our contributions can be summarized as follows: 
\begin{itemize}
    \item We show that, for a fixed model and quantization procedure, as the size of the training dataset increases towards infinity, the MIS of the learning algorithm is fully determined by the distribution of the loss per sample for the quantized models (\autoref{thm:fixed_pos}). Building upon this result, we further show that when the model architecture and/or quantization procedure adapts to the training set size, a similar dependency persists (\autoref{thm:seqQ}), although with a more precise dependence on the covariance structure of the loss per sample.
    
    %We show that for a fixed model and quantization procedure, as the size of the training dataset approaches infinity, the Membership Inference Security (MIS) of the learning algorithm is entirely determined by the distribution of the quantized models' loss per sample (\autoref{thm:fixed_pos}). Extending this result, we prove that when the model architecture and/or quantization procedure adapts to the training set size, a similar dependency holds (\autoref{thm:seqQ}).
    \item Building on the result of \autoref{thm:seqQ}, whose direct estimation is computationally prohibitive, we propose a methodology (\autoref{ssec:algo}) enabling the comparison of quantizers in terms of privacy.
    \item We apply our methodology to several Post-Training Quantization techniques on both synthetic data (see~\autoref{subsec:synthetic_expe}) and real-world data (see~\autoref{subsec:molecular_expe}).
    We show that the rankings provided by our method consistently correlates with the ones obtained with a baseline estimation of the MIS (see~\autoref{sssec:val}), and study the privacy-performance trade-off of quantization on molecular property prediction tasks (see~\autoref{ssec:tradeoff}).
\end{itemize}


%\begin{itemize}
%    \item 
%
%    \item Building on the result of Theorem \ref{thm:seqQ}, we propose a methodology (see Section \ref{ssec:algo}) to systematically construct a hierarchy between quantization procedures, providing a comprehensive perspective on the trade-off between efficacy and privacy. We apply our methodology to several Post-Training Quantization techniques and evaluate its performance on both synthetic data (see Section \ref{subsec:synthetic_expe}) and real-world data (see Section \ref{subsec:molecular_expe}).We apply our methodology to several Post-Training Quantization techniques and evaluate its performance on both synthetic data (see Section \ref{subsec:synthetic_expe}) and real-world data (see Section \ref{subsec:molecular_expe}).
%
%    \item We empirically validate our novel methodology by comparing the ranking we obtain to a baseline estimation of the MIS (see Section \ref{ssec:baseline_estimation}), showing that our methodology consistently correlates with the MIS (see Section \ref{sssec:val}).
%    We highlight the advantages of our methodology, including its computational efficiency, reduced memory requirements, and adaptability to molecular datasets.We highlight the advantages of our methodology, including its computational efficiency, reduced memory requirements, and adaptability to molecular datasets.
%\end{itemize}


Our work is dedicated to providing a theoretically grounded methodology to compare different quantization procedures in the context of MIS evaluation.

\subsection{Related work}

\textbf{Quantization of neural networks.} With the deployment of neural networks on edge devices \cite{yuan2024vit, lin2024awq}, where inference should be time and memory efficient, several quantization procedures have been studied and employed. Quantization usually answers this task by reducing the (bit-)precision of the parameters of the neural networks, demonstrating effectiveness in Large Language Models \cite{gong2024survey, zhu2024survey} even when the quantization is as strong as 1-bit precision quantization \cite{wang2023bitnet, ma2024fbi}, 1.58-bits precision quantization \cite{ma2024era1bitllmslarge}, arbitrary bits precision \cite{zeng2024abq}.
The most adopted framework of quantization is Post-Training Quantization (PTQ) \cite{jacob2018quantization, nagel2019data, gholami2022survey} which provides simple training-free implementation. PTQ is usually adopted over Quantization-Aware Training (QAT) \cite{bengio2013estimating, banner2018scalable, nagel2021white, nagel2022overcoming, pang2024push} due to their limitations to scale up to larger models \cite{gholami2022survey, lin2024awq}. Additionally, some lines of work study "hardware-aware" quantization procedures \cite{wang2024ladder, balaskas2024hardware} where optimization is made directly on the hardware. During our experiments, we will focus on PTQ.



\textbf{Membership Inference Attacks.}
Membership Inference Attacks (MIAs) can reveal sensible information \cite{shokri2017membership, song2017machine,carlini2022membership, carlini2023extracting} about one's data by leveraging the information stored in the parameters of the ML model \cite{hartley2022measuring, del2023bounding}. An extensive line of work has developed in the past decade to construct ever so powerful MIAs in embedding models \cite{song2020information}, regression models \cite{gupta2021membership} or generative models \cite{hayes202588705membership}, systematically summarized in \cite{hu2022membership}. Recent works have leveraged the predictive power of LLMs to construct new MIAs \cite{staab2023beyond, wang2025survey}. While few works have delved into the theoretical intricacies of MIAs \cite{sablayrolles2019white, del2023bounding, aubinais2023fundamental}, several Privacy benchmarks have been developed to audit the privacy risks of ML models \cite{murakonda2020ml, liu2022ml} by evaluating state-of-the-art MIAs on the target model. Although these benchmarks offer valuable insights into the privacy leakage of an ML model, a single MIA alone cannot provide a comprehensive assessment of an algorithm's overall privacy resilience against various attacks. We briefly discuss it in \autoref{ssec:privacy_ass}. 

\textbf{Quantization and Privacy.} Various strategies to protect models from attacks like MIAs have been proposed. In federated learning, the effects of input and gradient quantization have been analyzed through the lens of differential privacy~\cite{youn2023randomizedquantizationneeddifferential, yan_killing_2024, pmlr-v180-chaudhuri22a}. However, the impact of model quantization on security has been primarily assessed through empirical evaluations of MIAs, with no existing theoretical analysis~\cite{kowalski_towards_2022, s23187722}. Our work aims to fill this gap by providing a rigorous theoretical evaluation of the security implications of model quantization.

\section{Background and Notations}
\label{sec:context}
%We give in this section the notations and the fundamental concepts relevant to understanding MIS in the context of quantization.

\subsection{Predictive tasks}
Throughout the article, we consider a dataset $\gD_n\coloneqq\{\rz_1,\cdots,\rz_n\}$ of $n$ independent and identically distributed (i.i.d.) data drawn from a common distribution $P$ over a space $\gZ$. We assume that our goal is to infer a predictive function $\hat{\Psi}$ from a set of predictors $\gF\coloneqq\{\Psi_\theta : \theta\in\Theta\}$ indexed by some space $\Theta\subseteq\R^d$. We define a (learning) \textbf{algorithm} as a function $\gA:\bigcup_{n\geq1}\gZ^n\to\gP(\Theta)$, where $\gP(\Theta)$ is the space of all probability measures on $\Theta$. By denoting $\hat{\theta}_n\sim\gA(\rz_1,\cdots,\rz_n)\in\Theta$, we systematically set $\hat{\Psi} = \Psi_{\hat{\theta}_n}$.
This definition of a learning algorithm includes all (stochastic) algorithms. We will assume in the following that the algorithm can be written as a function of the empirical distribution of the data. More specifically, this means that there exists a function $G$ and a random variable $\xi$ such that $\gA\pp{\rz_1,\cdots,\rz_n} = G(\hat{P}_n,\xi)$, where $\hat{P}_n$ is the empirical distribution of the training dataset. In this case, the random variable $\xi$ encompasses the stochasticity of the algorithm. This assumption is especially satisfied for algorithms minimizing an empirical loss $\theta\mapsto\frac{1}{n}\sum_{j=1}^{n}\ell (\theta,\rz_j)$ where $\ell:\Theta\times\gZ\to\R^+$ is the loss function.
%\vspace{1mm}
\begin{example}{Classification.} For a classification task, we may note $\gZ\coloneqq\gX\times\mathcal{Y}$ where 
$\mathcal{Y}=\{1,\cdots,|\mathcal{Y}|\}$ is the number of classes and $\gX$ is the input space. $\gF$ can be the set of all neural networks, where $\theta\in\Theta$ then represents the parameters of such predictors. The algorithm $\gA$ can be any (stochastic) optimizing procedure (e.g., Adam optimizer) on any adequate loss function (e.g., the cross-entropy loss).
\end{example}
\subsection{Quantization}
\label{subsec:quantization}
We define a \textbf{quantizer}~\cite{citeulike:12927267} as any measurable function $\gQ:\Theta\to\Bar{\Theta}\subseteq\Theta$ for some discrete space $\Bar{\Theta}\coloneqq\{\thetaq_1,\cdots,\thetaq_K\}$. 
A quantizer canonically induces a \textbf{quantized algorithm} $\gA_\gQ$, and a loss function $\ell_{\gQ}$, which we simply write $\ell$, as long as there is no ambiguity. For any $\theta\in\Theta$, we denote by $m_\theta=\E\cc{\ell(\theta,\rz)}$ the expected loss evaluated on $\theta$, where $\rz$ is a random variable with distribution $P$. Additionally, for a given quantizer $\gQ$, we will assume without loss of generality that $m_{\thetaq_1}\leq\cdots\leq m_{\thetaq_K}$. 
In the following, we introduce two specific examples to illustrate particular quantization methods.
% \vspace{1mm}

% \textcolor{red}{They don't do that exactly in the paper. TBC and changed.}
\begin{example}[Binarized Neural Networks \cite{wang2023bitnet}.]
\label{ex:BNN}
Let $\gF$ be a set of neural networks with fixed architecture. The set $\Theta$ then represents the parameters of the neural network. A scalar quantizer $\gQ$ maps coordinate-wise the parameters to its sign, namely $\gQ(\theta) = \pp{\theta_j/|\theta_j|}_j$. Here, for $1-$layer unbiased neural networks with width $d$ (number of parameters), the set $\bar{\Theta}$ would consist of all $d-$dimensional vectors in $\{-1,+1\}^d$ where $K = 2^{d}$. Usually, computers store parameters in a 32-bits (or 64-bits) format. Low-precision quantization procedures, e.g. $2-$bits (or $q-$bits in general) quantization, reduce the number of bits required from 32 (or 64) to 2 (or $q$ in general).
\end{example}
%\vspace{.5mm}

\begin{example}[Vector Quantization]
Another quantization procedure, albeit under-used in practice, is vector quantization. A \textit{codebook} $\Thetaq$ is usually pre-computed, which the vector quantizer $\gQ$ maps $\theta$ onto, usually performed by a nearest neighbor algorithm, which makes it efficient and memory-friendly. The constant $K$ here corresponds to the number of values stored in the codebook.
    


\end{example}

\subsection{Privacy assessment}
\label{ssec:privacy_ass}

In the present work, we evaluate the privacy of an algorithm $\gA$ trained on a task $P$ through Membership Inference Attacks (MIAs). Particularly, in a scenario where $\gD_n$ consists of sensible data and the model $\hat\Psi$ has been shared (such as a sold product), MIAs are known to pose a notable threat to the privacy of the dataset. MIAs aim at inferring membership of a test sample $\Tilde{\rz}$ to the dataset $\gD_n$ by observing $\hat{\theta}_n$. MIAs can be defined as follows.

\begin{definition}[Membership Inference Attack - MIA] Any measurable map $\phi : \Theta\times\gZ\to\{\text{member},\text{non-member}\}$ is considered to be a \textit{Membership Inference Attack}.
\end{definition}

The existence of successful MIAs constitute a major threat against the privacy of personal data by revealing sensible information. However, for most algorithms, there may always exist pathological datasets for which models trained on would be highly attackable by MIAs. Specifically, although individual MIAs can reveal information leakage, it is alone insufficient to disclose a complete overview on the privacy level of a machine learning model. To adequately tackle down the question of privacy of an algorithm, it is compulsory to address all possibilities of attacks. We then will say that an algorithm is private if it \textit{usually} produces parameters $\hat{\theta}_n$ that are private against \textit{most} MIAs. We use the notion of accuracy of an MIA, defined as the probability of successfully guessing the membership of the test point. 
Letting $T\in\{\text{member},\text{non-member}\}$ encode the membership of a test point $\Tilde{\rz}$, we define the accuracy as follows. 


% Defining a test point as $\Tilde{\rz}\coloneqq (1-T)\rz_0 + TU$ where $T$ is a Bernoulli random variable with parameter $1/2$, ${\rz_0}\sim \gP$ be a sample independent from the training dataset and $U$ a random variable whose distribution is $\hat{P}_n$ conditionally to the training dataset, we define the accuracy as follows.


\begin{definition}[Accuracy of a given MIA] The \textit{accuracy of an MIA} $\phi$ is defined as 
\begin{equation}
\label{def:perf}
{\text{Acc}}_n(\phi; P,\gA ) \coloneqq P\left (\phi(\hat{\theta}_n,\Tilde{z})= T\right ),
\end{equation}
where the probability is considered over all sources of randomness inherent in the underlying training model and the data used for both training and evaluation.
\end{definition}

% \noindent For most algorithms, there may always exist pathological datasets for which models trained on would be highly attackable by MIAs. %\textcolor{red}{This indicates that data specific metrics are insufficient to gauge the privacy level of an algorithm, justifying the use of the accuracy. \# I do not understand this sentence... \#}

%We relate the privacy of an algorithm to the highest achievable accuracy.


\begin{definition}[MIS] 
The \textit{Membership Inference Security} (MIS) of an algorithm $\gA$ is defined as 
\begin{equation}
\label{def:sec}
{\MIS}_n(P, \gA)\coloneqq 2\left(1-\underset{\phi}{\sup}\;{\text{Acc}}_n(\phi; P,\gA )\right), 
\end{equation}
where the sup is taken over all MIAs and thus, ${\MIS}_n(P, \gA)\leq 2(1-{\text{Acc}}_n(\phi; P,\gA ))$ for all MIAs $\phi$.
\end{definition}
We notice that MIAs with an accuracy of at least \( \frac{1}{2} \) always exist (e.g., constant MIAs). As a result, the MIS metric ranges from \( 0 \) (completely non-private) to \( 1 \) (fully private).

\begin{remark}
The presence of the supremum makes the MIS a metric encompassing all possible MIAs, including all state-of-the-art MIAs, and most importantly, all unknown MIAs. Indeed, even though state-of-the-art MIAs provide a strong indicator on the security of ML models, more powerful MIAs are likely to emerge in the future, beating the state-of-the-art MIAs. Consequently, it is of paramount importance to consider all attacks when designing privacy metrics. 
\end{remark}

\section{Theoretical Results}
\label{sec:theo_res}
We provide in this section the main theoretical results on the MIS of quantized algorithms. For the results of \autoref{sec:theo_res} and \autoref{sec:algo} to hold, it is mandatory to give a proper mathematical setting, although not required for the reader to pursue. We therefore refer to \autoref{ssec:math-sett} for a formal mathematical setting.

\subsection{Fixed Quantizer}
We start off by giving a simple result. Let $\gQ$ be a fixed quantizer, then under mild additional assumptions presented in \autoref{ssec:fq-add-ass}, the following result holds.
\begin{theorem} 
\label{thm:fixed_pos}
There exists a constant $C_P^1 > 0$ satisfying 
\begin{equation}
    \liminf{n}-\frac{1}{n}\log\pp{1-{\MIS}_n(P, \gA_\gQ)} \geq C_P^1.
\end{equation}
\end{theorem}
Importantly, the constant \( C_P^1 \) depends solely on the distribution of \( \ell(\theta, \rz) \) for all \( \theta \in \Bar{\Theta} \), where $\rz$ has for distribution $P$. The detailed formulation of the theorem can be found in \autoref{thm:main_fixed_quant}. This theorem establishes that the MIS is of order \( 1 - e^{-n C_P^1 (1 + o(1))} \) for some constant \( C_P^1 > 0 \) for a given algorithm. The result suggests that the approximation holds as the dataset size approaches infinity, assuming the architecture size remains fixed. However, in practice, when developing machine learning models, it is common to adjust the architecture based on the dataset, particularly its size.


%Importantly, the constant $C_P^1$ only depends on the distribution of $l(\theta,\rz)$ for all $\theta\in\Bar{\Theta}$. The details of the theorem are given in Theorem \ref{thm:main_fixed_quant} in Appendix ??. Theorem \ref{thm:main_fixed_quant} states that the MIS is of order $1 - e^{-nC_P^1(1+o(1))}$ for some constant $C_P^1>0$ for a given algorithm. The limit suggests that this result is accurate when the dataset size grow to infinity, without the architecture size to adapt to it. However, when building machine learning models, it is natural to adapt the architecture to the dataset, specifically to its size. 


\subsection{Size-Adaptive Quantizers}
\label{subsec:Qn}

As an example, it is common to over-parameterize models relative to the dataset size, as seen in the case of LLMs. 
Therefore, \autoref{thm:fixed_pos} may not provide an accurate approximation for very large datasets. 
We now let our quantizer \( \gQ_n  \) (and therefore the number of quantized values $K_n$ and $\Thetaq_n\coloneqq\{\thetaq_1,\cdots,\thetaq_{K_n}\}$) be \textit{Size-Adaptive}, i.e. depends on the sample size. Let $\delta_k^n\coloneqq m_{\thetaq^n_k}-m_{\thetaq^n_1}$ be the \textbf{loss gaps} and  $\pp{\sigma_k^n}^2 = \textrm{Var}\big(\ell\pp{\thetaq^n_k,\rz} - \ell\pp{\thetaq^n_1,\rz} \big)$ be the \textbf{loss variabilities}, which corresponds to the variance of the difference of the losses between $\thetaq^n_k$ and $\thetaq^n_1$. 

The dependence of \( \gQ_n \) on the dataset size \( n \) formalizes at least two scenarios: either the quantization procedure remains the same, but the architecture size adapts to the training dataset, or the architecture size is fixed while the quantization procedure changes. Specifically, the first interpretation can be seen as the common practice in machine learning to scale models to the dataset size.
\begin{example}[Scaling Architecture]
Let the number of parameters of our original models follow a scaling law~\cite{hoffmann2022trainingcomputeoptimallargelanguage, kaplan2020scalinglawsneurallanguage} $f$ such that its number of parameters is: $f(n)$. Let the quantization method be fixed to a 1-bit quantization (mapping each parameter to its sign for instance).
Then, following Example~\ref{ex:BNN}, for a dataset size $n$, the size of $\Thetaq_n$ is $K_n=2^{f(n)}$.
\end{example}
Under some mild assumptions outlined in Appendix \ref{ssec:saq-add-ass}, we obtain the following result.
\begin{theorem}\label{thm:seqQ} Assume that ${\sqrt{n}\mkn{2}\toinf{n}\infty}$ and ${\mkn{2}\toinf{n}0}$. Then, we have
\begin{equation}
    \liminf{n}-\frac{1}{n\pp{\mkn{2}}^2}\log\big(1-\MIS_n(P, \gA_{\gQ_n})\big)\geq \frac{1}{2\sigma^2},
\end{equation}
where $\sigma^2 = \underset{k}{\max} \liminf{n} \pp{\mkn{2}/\mkn{k}}^2\pp{\sigma_k^n}^2$.
\end{theorem}



For a size-adaptive Quantizer $\gQ\coloneqq (\gQ_n)$, let $r_{\gQ}^n\coloneqq \pp{\delta_2^n}^2/(2\sigma^2)$ be the constant of \autoref{thm:seqQ} multiplied by the square of the minimal loss gap. \autoref{thm:seqQ} then stipulates that the MIS of a quantized algorithm, whose quantization $\gQ$ is Size-Adaptive, is of order $1 - e^{-nr_\gQ^n(1+o(1))}$ which by hypothesis converges to $1$ as $n$ grows to infinity, ensuring asymptotic security.
Furthermore, \autoref{thm:seqQ} suggests that for two size-adaptive Quantizers $\gQ$ and $\gR$,  $r_\gQ^n \geq r_\gR^n$ implies that $\gA_{\gQ}$ produces more secure parameters than $\gA_{\gR}$ (asymptotically). \autoref{thm:seqQ} then proposes to use $r_\gQ^n$ as a measure to compare different quantizers. Most importantly, this quantity wholly relies upon the asymptotic expectations and variances of the random variables $\ell(\thetaq_k^n,\rz) - \ell(\thetaq_1^n,\rz)$. 




% \textcolor{blue}{
% In the following sections, we will call $\qcertif = \pp{\delta_{\gQ,2}^n}^2r_{\gQ}$, the metric we  aim to estimate to compare the privacy level of different embedders.
% }


% \subsection{Discussion}
\begin{remark}
The theoretical setting of \autoref{subsec:Qn} encompasses modern architectures and habits of machine learning designs. This setting enables us to explicit the quantity $r_\gQ^n$ controlling the asymptotic MIS of a quantized algorithm. The most vital point of \autoref{thm:seqQ} is that this result holds for all attacks, including currently unknown (and possibly more powerful) attacks. 
% This comes at the drawback of requiring few mild assumptions, and being an asymptotic result.
% The method we propose in \autoref{ssec:algo} to approximate $r_\gQ^n$ demonstrates a high correlation with the baseline method as shown in \autoref{fig:scatterplot_synth}. This ensures, although being an asymptotic result, that $r_\gQ^n$ is a sufficiently accurate privacy measure.
The following sections focus on empirically demonstrating that the estimate of $r_\gQ^n$ is adequate to rank quantizers by their privacy level.   
\end{remark}



\section{Estimating Quantized Algorithm Privacy}
\label{sec:algo}
\subsection{Measuring the privacy of quantized models}
\label{ssec:algo}

To estimate the security of an algorithm $\gA_{\gQ}$ using~\autoref{thm:seqQ}, we must compute the loss gaps $\mkn{k}$ between the best quantized model and all possible quantizers in $\Bar{\Theta}$.
However, this is computationally infeasible, as even a simple quantizer like 1-bit quantization leads to an exponentially large $\Bar{\Theta}$ with respect to the number of parameters.
We address this intractability by observing that only a few quantizers dominate the estimation of $\qcertif$.
Specifically, $\qcertif$ depends on:
\begin{itemize}
\item The two lowest average quantized losses $m_{\thetaq_1}$ and $m_{\thetaq_2}$.
\item The values $\liminf{n}\pp{\mkn{2}/\mkn{k}}^2(\sigma_k^n)^2$ which measure the trade-off between the mean loss gap and the per-sample variance of quantized models.
\end{itemize}
Crucially, $\max_k\liminf{n}\pp{\mkn{2}/\mkn{k}}^2(\sigma_k^n)^2$ is empirically dominated by low-loss quantizers (see~\autoref{subsec:eval_privacy}), suggesting that exploring the entire set $\Bar{\Theta}$ is unnecessary, as focusing on low-loss quantizers is sufficient to estimate $\qcertif$.
% Crucially, $\max_k{\Lambda_{k,k}}$ is empirically dominated by low-loss quantizers (see~\autoref{subsec:eval_privacy}), suggesting that exploring the entire set $\Bar{\Theta}$ is unnecessary, as focusing on low-loss quantizers is sufficient to estimate $\qcertif$.

We thus propose estimating using quantized models derived from the training trajectory of $\hat{\theta}_n$.
Since $\hat{\theta}_n$ is optimized to minimize $\ell$, the quantizers with the lowest loss likely reside near this trajectory.
This justifies our focus on the training trajectory, which efficiently captures critical quantizers.
The complete procedure is outlined in Algorithm~\ref{algo:main}.
We propose an implementation of Algorithm~\ref{algo:main} as a wrapper of Pytorch Lightning's ``LightningModule" class~\cite{Falcon_PyTorch_Lightning_2019}, as well as the code to replicate all experiments.\footnote{\url{https://anonymous.4open.science/r/Mol_Downstream-B3DB/README.md}}

%To estimate the value of $\qcertif$ using a subset of $\Bar{\Theta}$, we propose to focus on the quantized models obtained by quantizing $\theta_n$ at each epoch of its training.
%As $\theta_n$ is trained to optimize $l$, the quantized model achieving the minimal values $m_{\thetaq_1}$ and $m_{\thetaq_2}$ most likely lies close to its training trajectory.
%On the other hand, $\max_{k}{\Lambda_{k,k}}$ is controled by a trade-off between how close the loss of the quantized models are to $\thetaq_1$ looking at their mean, and yet how different they are on every samples, as measured by the variances $\sigma_k^n = Var\pp{l\pp{\thetaq_k,\rz} - l\pp{\thetaq_1,\rz}}$.
%If the maximum was to be obtained on a quantized model with a high average loss, following the training trajectory of $\theta_n$ might not provide the best estimate of $\qcertif$.
%However, we show in~\autoref{subsec:eval_privacy} that ${\max_k{\Lambda_{k,k}}}$ is consistently achieved for quantized models with among the lowest average loss.
%We propose in~\autoref{algo:main} the complete estimation procedure of $\qcertif$.

\begin{algorithm}
    \caption{Estimation of $\qcertif$}
    \label{algo:main}
    \begin{algorithmic}[1]
        \STATE \textbf{Input:} A training dataset $\gD_n$, a validation dataset $\gD_{\textrm{val}}$, a learning algorithm $\gA$, a quantizer $\gQ$, an initialized model $\theta$, a number of epochs $K$.
        \STATE \textbf{Output:} An estimate of $\qcertif$.
        \STATE \textbf{Initialization:} Set the list of all quantized loss $\Ls_{\textrm{val}} = \{\}$, of all average quantized losses $m_{\textrm{val}} = \{\}$, and of all variances $\sigma_{\textrm{val}}^2 = \{\}$.
        \FOR{$k=1$ to $K$}
            \STATE $\theta \leftarrow \gA(\theta, \gD_n)$.
            \STATE $\thetaq \leftarrow \gQ(\theta)$.
            \STATE $\Ls_k \leftarrow \{\ell(\thetaq, \rz) : \rz\in\gD_{\textrm{val}}\}$.
            \STATE $m_k \leftarrow \frac{1}{|\gD_{\textrm{val}}|}\sum_{\rz\in\gD_{\textrm{val}}}\ell(\thetaq, \rz)$.
            \STATE $\Ls_{\text{val}}[k] \leftarrow \Ls_k$.
        \ENDFOR
        \STATE $\text{idx} \leftarrow \text{argsort}(m_{\text{val}})$.
        \STATE $m_{\textrm{val}} \leftarrow m_{\textrm{val}}[\text{idx}]$.
        \STATE $\Ls_{\textrm{val}} \leftarrow \Ls_{\textrm{val}}[\text{idx}]$.
        \FOR{$k=2$ to K}
            \STATE $\sigma_{\textrm{val}}^2[k] \leftarrow \textrm{Var}\pp{\Ls_{\text{val}}[k] - \Ls_{\text{val}}[1]}$.
        \ENDFOR
        \STATE $r_{\gQ} \leftarrow \frac{1}{2}\cc{\underset{2\leq k\leq K}\max{\left(\sigma_{\text{val}}^2[k] \times \pp{\frac{m_{\text{val}}[2] - m_{\text{val}}[1]}{m_{\text{val}}[k] - m_{\text{val}}[1]}}^2\right)}}^{-1}$.
        \STATE \textbf{return} $\qcertif = r_{\gQ} \pp{m_{\text{val}}[2] - m_{\text{val}}[1]}^2$.

    \end{algorithmic}
\end{algorithm}


\subsection{Baseline estimation of the MIS}
\label{ssec:baseline_estimation}
Inferring membership of $\Tilde{\rz}$ by an MIA can naturally be considered as a statistical test,
\begin{equation*}
\left\{
    \begin{array}{ll}
  H_0: &``\Tilde{\rz}{\text{ belongs to the training dataset}}".  \\
  H_1: & ``\Tilde{\rz}{\text{ does not belong to the training dataset}}".
\end{array}\right.
\end{equation*}
Under the mathematical setting presented in Appendix \ref{ssec:math-sett}, these hypotheses can be apprehended as deciding whether $\hat{\theta}_n$ is independent or not to $\Tilde{\rz}$. Specifically, testing $H_0$ against $H_1$ is equivalent to testing $H_0'$ against $H_1'$, where
\begin{equation}
\left\{
    \begin{array}{ll}
    H_0': & (\hat{\theta}_n, \Tilde{\rz})\sim P_{\pp{\hat{\theta}_n,\rz_1}}. \\
    H_1': & (\hat{\theta}_n,\Tilde{\rz})\sim P_{\hat{\theta}_n}\otimes P.
\end{array}\right. 
\end{equation}
For any dominating measure $\zeta$ on $P_{\pp{\hat{\theta}_n,\rz_1}}$ (and $P_{\hat{\theta}_n}\otimes P$), denoting by $f$ (resp. $g$) the density of  $P_{\pp{\hat{\theta}_n,\rz_1}}$ (resp. $P_{\hat{\theta}_n}\otimes P$) with respect to $\zeta$, we have that $\phi^*(\theta, z) = \mathbbm{1}\left\{\frac{f(\theta,z)}{g(\theta,z)} \geq1\right\}$ satisfies:
\begin{equation}
    \underset{\phi}{\sup}\;{\text{Acc}}_n(\phi; P,\gA ) = {\text{Acc}}_n(\phi^*; P,\gA ).
\end{equation}
\noindent The function $\phi^*$ is the Neyman-Pearson test for $H_0'$ against $H_1'$. This suggests that evaluating empirically the MIS of an algorithm amounts down to training a discriminator, and evaluate it.
% Specifically, Lemma \ref{lem:np} states that to accurately quantify the security level of an algorithm $\gA$, one has to estimate the optimal discriminator between $H_0'$ and $H_1'$. Additionally, denoting by $\|\cdot\|_{TV}$ the total variation distance, we may rewrite the MIS as ${\text{MIS}}_n\left(P,\gA\right) = 1-\Delta_n(P,\gA)$ \cite{aubinais2023fundamental} where 
% \begin{equation}
% \label{eq:delta_def}
%     \Delta_n(P,\gA)\coloneqq \|P_{\left(\hat{\theta}_n,\rz_1\right)} - P_{\hat{\theta}_n}\otimes P\|_{TV}.
% \end{equation}
% Under the mathematical setting presented in \autoref{ssec:math-sett}, and based on the work of \cite{aubinais2023fundamental}, 
% Under the mathematical setting presented in \autoref{ssec:math-sett}, \autoref{eq:mis_delta} states that,
% \begin{equation}
% \label{eq:delta_tv}
%     {\MIS}_n(P, \gA) = 1 - \|P_{(\thetan,\rz_1)} - P_{\thetan}\otimes P\|_{TV},
% \end{equation}
% for any distribution $P$ and any algorithm $\gA$, where $\|\cdot\|_{TV}$ is the total variation distance. In particular, it holds for any (Size-Adaptive) quantized algorithm. For any dominating measure $\zeta$ on $P_{(\thetan,\rz_1)}$ (and $P_{\thetan}\otimes P$), denoting by $f$ (resp. $g$) the densities with respect to $\zeta$ of $P_{(\thetan,\rz_1)}$ (resp. $P_{\thetan}\otimes P$) we have that
% \begin{equation}
%     \|P_{(\thetan,\rz_1)} - P_{\thetan}\otimes P\|_{TV} = \mathbb{P}\pp{\frac{f(\thetan,\rz)}{g(\thetan,\rz)} \geq 1},
% \end{equation}
% from the definition of the total variation distance. Here, the probability $\mathbb{P}$ is taken with respect to $(\hat{\theta}_n,\rz)\sim P_{\thetan}\otimes P$.
The baseline approach therefore consists in training a binary classifier $\bsl: \Theta \times \mathbb{R}^d \times \mathbb{R} \rightarrow [0,1]$ to distinguish between samples $z$ sampled from the training set of a given $\thetan$ and sampled from the product distribution $P_{\thetan}\otimes P$, and evaluate its accuracy.
For a sample $z=(x,y)\sim P$, where $x\in \mathbb{R}^d$ is the input and $y$ its corresponding label, and a model $\thetan$, the discriminator minimizes the binary cross-entropy loss:
\[
    \ell_{\textrm{DISC}}(\thetan, z) = \text{BinaryCE}\left(\bsl(\thetan, x, y), \mathbbm{1}_{z\in \mathcal{D}_{\textrm{train}}(\thetan)}\right),
\]
where $\mathcal{D}_{\textrm{train}}(\thetan)$ is the training set of $\thetan$.

The discriminator is implemented as a feed-forward neural network that takes as input: $x$ the input data, the flattened parameters of $\thetan$ and the loss of the model $\thetan$ on $x,y$.
For instance, if the model $\thetan$ is a binary classifier, $\bsl$ is a neural network with input $x\in \mathbb{R}^d$, $\thetan$ and the binary cross-entropy loss of $\thetan$ on $(x,y)$.

We train the discriminator using a set of models, $\{\thetan{_ i}\}_{i\in\{1,\ldots,k_{\text{run}}\}}$ where each $\thetan{_ i}$ is trained on an independent dataset $\{z_{i,1},\ldots, z_{i,n}\} \sim P$ of $n$ i.i.d. samples.
To generate negative samples, we independently sample additional sets, $z^{\textrm{neg}}_{i,1},\ldots,z^{\textrm{neg}}_{i,n} \sim P$, ensuring no overlap with the training sets of any models, or between the negative samples of different models.
This independence between datasets is critical for preventing information leakage but may be restrictive in practice due to high data requirements.


A key drawback of this approach arises when $\thetan$ contains multiple layers.
Different permutations of the weights of the hidden units can represent identical models, but the discriminator $\bsl$, which relies on flattened parameters, is not invariant to such permutations.


\section{Numerical Experiments}
\label{sec:num_exp}



\begin{figure*}
    \centering
    \includegraphics[width=\linewidth]{Figures/synth_figs/synth_corr_ranking}
    \caption{
    \textbf{Stability of Privacy rankings.}
    To obtain reliable estimates of $\qcertif$, we average its value over multiple runs $k_{\textrm{run}}$ (number of classifiers trained).
    The central plot illustrates how the rankings of quantizers, based on $\qcertif$, evolve with the number of runs.
    Each column of pie charts represents the proportion of quantizers predicted at each rank (across 100 different subsets of $k_{\textrm{run}}$ runs) with connecting lines showing shifts in predicted rankings.
    As the number of runs increases, the rankings stabilize, and when averaged over 50 runs, each quantizer is ranked at its final position 90\% of the time (except for the \texttt{2 bits} and \texttt{1.58b 33\%} quantizers).
    The top figure shows the evolution of the average Spearman correlation between $\qcertif$ (resp. the baseline's estimation of the MIS) when evaluated over $k_{\textrm{run}}\leq 100$ and $k_{\textrm{run}} =300$.
    The confusion matrices on the right compare rankings estimated using 300 runs to those obtained with 20 and 50 runs.
    }
    \label{fig:synth_stability}
\end{figure*}

\label{subsec:synthetic_expe}
\begin{figure}
    \centering
    \includegraphics[width=1\linewidth]{Figures/synth_figs/rdelta_vs_acc_bsl}
    \caption{
        \textbf{Relationship between $\qcertif$ and the MIS.}
         Each sub-plot displays the estimated values of $\qcertif$ and the MIS for each quantizer under varying experimental configurations, with their corresponding Spearman correlation ($\rho_{sp}$).
        The strong correlations confirm that our method enables the comparison of different quantization techniques' security.
    }
    \label{fig:scatterplot_synth}
\end{figure}
As we focus solely on ranking quantization procedures (\autoref{thm:seqQ} being asymptotic and potentially containing irrelevant constants), we omit specific 
$\qcertif$ values in the figures for clarity and readability.
%Since we are only concerned with ranking the quantization procedures (\autoref{thm:seqQ} being an asymptotic result that may include irrelevant constants), we omit the specific values of \(\qcertif\) in the following figures to improve clarity and readability.

\subsection{Synthetic experiments}
To confirm that our estimation of $\qcertif$ enables the ranking of quantization methods according to their level of privacy, we compare this ranking to the one obtained using the baseline estimation of the MIS.
To perform this comparison, we rely on synthetic experiments, where we can train and evaluate a large number of classifiers, and ensure the independence between each classifier's training set to train $\bsl$.

\subsubsection{Experimental Setup}
\paragraph{Datasets.}
We generate data points sampled from $\mathbb{R}^{128}$ using $k_{\textrm{modes}} \in \{6,8,16\}$ isotropic Gaussian distributions with standard deviation $\sigma\in \{1.5,2,3\}$, where each cluster is assigned a label in $\{0,1\}$.

\paragraph{Classifiers.}
As mentioned above, the baseline estimation method is limited to models consisting of a single layer.
Consequently, all classifiers in our experiments are single-layer fully connected networks. 
To increase the expressivity of these networks, the input features are augmented with $x^2$ (element-wise squared features).
Each classifier is trained on $n=128$ samples for 3000 epochs using the Adam optimizer with a learning rate of $10^{-4}$.
To evaluate our approach, we train $k_{\textrm{run}}=300$ classifiers ($\theta$) on each distribution, with each run involving new i.i.d. samples, and we investigate in~\autoref{fig:synth_stability} the impact of $k_{\textrm{run}}$ on our evaluation of the quantizers' privacy.


\paragraph{Quantization.}
For our experiments, we consider a range of different quantization methods, including: 1bit quantization by taking the sign of the weights (\texttt{Sign}), 1.58 bits quantization with different sparsity levels (\texttt{1.58b \{x\}\%} where $x\%$ of the weights with the smallest magnitude are set to zero, and the rest to their sign), and quantization on 2, 3, 4, and 5 bits.
A detailed description of the quantization methods is provided in~\autoref{app:quantization}.


\subsubsection{Validation of the approach}
\label{sssec:val}
\paragraph{Estimation of the privacy guarantees.}
\autoref{fig:scatterplot_synth} illustrates the correlation between our proposed metric $\qcertif$ and the MIS baseline.
We observe that our metric effectively quantifies the privacy of quantized models: higher values correspond to greater MIS.
We report the Spearman correlation between both metrics (the correlation between the rankings induced by both metrics), and as expected, quantizers with more bits of information, such as \texttt{5 bits} and \texttt{4 bits}, are the least private across all datasets.
In contrast, the \texttt{1.58b 90\%} quantizer, which introduces 90\% sparsity by setting weights to zero, achieves the highest privacy.
Interestingly, the \texttt{Sign} method, which quantizes weights to 1 bit, is less private than the \texttt{1.58b 33\%} quantizer, despite using fewer bits.
This behavior aligns with observations from the baseline method, suggesting that sparsity plays a more significant role in privacy than the number of bits alone.
Overall, the rankings produced by $\qcertif$ closely match those of the baseline method, with an average Spearman correlation of $\rho_{sp}=0.86$, demonstrating that $\qcertif$ reliably ranks quantization methods by their privacy levels.

\paragraph{Quantizer ranking stability.}
While our experiments compute $\qcertif$ by averaging over 300 runs, such extensive computation may be impractical for real-world deployment due to resource constraints.
\autoref{fig:synth_stability} shows how the stability of $\qcertif$-based rankings improves as the number of runs increases.
Critically, the ranking of the most and least private quantizers stabilizes early: after just 20 runs, the \texttt{5 bits} quantizer is consistently ranked least private, followed by \texttt{4 bits} and \texttt{3 bits}, while \texttt{1.58b 90\%} (highest sparsity) remains the most private, followed by \texttt{1.58b 50\%}.

\paragraph{Computational cost.}
All experiments were conducted on NVIDIA A6000 GPUs with 48GB of memory.
The estimation of $\qcertif$ introduces a small computational overhead during model training, as it requires computing the quantized model's loss at each validation step, often for multiple quantization processes, adding approximately 1s to the training time of one $\thetan$ (4m total).
In contrast, the baseline MIS estimation requires training multiple classifiers ($\thetan$) to train the discriminator $\bsl$, which can also be sensitive to the hyperparameter choice, resulting in significant computational and data demands.
The top figure of~\autoref{fig:synth_stability} shows that the rankings obtained with the $\qcertif$-based approach stabilizes close to the final ranking after only 20 runs, as demonstrated with Spearman correlations over $0.95$, while the MIS's baseline estimation does not reach this threshold until 150 runs.
As a result, ranking quantizers using $\qcertif$ ($\approx$1h) is significantly faster than using the baseline MIS method ($\approx$10h).

\subsection{Experiments on molecular datasets}
\label{subsec:molecular_expe}

\subsubsection{Experimental setup}
\begin{figure*}
    \centering
    \includegraphics[width=\linewidth]{Figures/mol_figs/barplot_dperfs_vs_rdelta_cls_dataset}
    \caption{
        \textbf{Impact of quantization on classification tasks.}
        Evolution of the privacy of each downstream model $\qcertif$ along with relative performances of the quantized models compared to the original on classification task.
    }
    \label{fig:barplot_dperfs_vs_rdelta_cls_dataset}
    \vspace{-0.25cm}
\end{figure*}

\begin{figure}
    \centering
    \includegraphics[width=\linewidth]{Figures/mol_figs/barplot_dperfs_vs_rdelta_reg_dataset}
    \caption{
        \textbf{Impact of quantization on regression tasks.}
        Evolution of the privacy of each downstream model $\qcertif$ along with relative performances of the quantized models compared to the original on regression task.
    }
    \label{fig:barplot_dperfs_vs_rdelta_reg_dataset}
    \vspace{-0.5cm}
\end{figure}


In our second experimental setting, we analyze a real-world application: molecular modeling.
In the field of drug discovery, data is an invaluable and highly sensitive asset, and determining whether predictive models might inadvertently leak proprietary data is therefore highly valuable.


\paragraph{Pretrained Embedders.}
To generate one-dimensional molecular embeddings, we evaluate four pretrained models from the  representation learning literature: GraphMVP~\cite{liu2022pretraining} and 3D-Infomax~\cite{stark2021_3dinfomax} (3D-2D mutual information maximization), MolR~\cite{wang2022chemicalreactionaware} (reaction-aware pretraining), and ChemBERTa-MTR~\cite{ahmad2022chemberta2} (multitask regression with SMILES tokenization).
The embeddings are passed to small feed-forward networks for downstream tasks, and we quantify the privacy impact of each embedder in~\autoref{app:molecular_details}.


\paragraph{Downstream tasks.}
We evaluate and train the models on various property prediction tasks from the Therapeutic Data Commons (TDC) platform~\cite{Huang2021tdc}, focusing on ADMET properties (Absorption, Distribution, Metabolism, Excretion, and Toxicity).
These tasks encompass both binary classification and regression problems with datasets of varying sizes and complexities.
For classification tasks, we report the AUC-ROC scores, and for regression tasks, we report R2 scores (coefficient of determination).
For both metrics, higher values indicate better performance, with 1 being the maximum value for both metrics.
For the regression tasks, the R2 score can be negative, indicating that the model performs worse than a simple mean prediction.

\paragraph{Task models.}
For each downstream task, we train fully connected networks (two layers, hidden dimension 128) for 500 epochs.
To ensure robust estimation of $\qcertif$, we allocate 40\% of the dataset to the validation set.
Each experiment is repeated~\nrunmol times, with 90\% of the training set randomly subsampled in each run to ensure different training trajectories are used.
We quantify the impact of quantization on performance measuring the relative performance of the quantized model to the original: for a metric $m$ (AUROC or R2): \(
\diffmetric{\textrm{val}}{m} = {m^{\textrm{val}}_{\text{quantized}}}/{m^{\textrm{val}}_{\text{original}}}
\).
Full results, including performance-privacy trade-offs grouped by embedders, are available in~\autoref{app:molecular_details}.

\subsubsection{Privacy-Performance Trade-off}
\label{ssec:tradeoff}

\paragraph{Classification tasks.}
\autoref{fig:barplot_dperfs_vs_rdelta_cls_dataset} illustrates the trade-off between privacy ($\qcertif$) and relative performance ($\diffmetric{\textrm{val}}{\textrm{AUROC}}$) for classification tasks.
We identify two regimes: (1) for high values of $\qcertif$, the quantized models achieve high privacy (e.g., \texttt{1.58b 90\%}), but the performances are significantly lower than the original models ($\diffmetric{\textrm{val}}{\textrm{AUROC}} \approx 90\%$), in particular on ClinTox~\cite{CLINTOX} (toxicity prediction) and PAMPA\_NCATS~\cite{PAMPA} (membrane permeability) where $\diffmetric{\textrm{val}}{\textrm{AUROC}}$ goes down to $80\%$.
(2) Low-privacy quantizers (\texttt{2 - 5 bits}) preserve near-original performance but are consistently ranked least private.
Across almost all datasets, the \texttt{1.58b 90\%} quantizer appears to be the most secure, while obtaining better performance than other sparse quantizers, notably on ClinTox, Carcinogens\_Lagunin~\cite{CARCINOGENS}, and CYP2C9 substrate classification~\cite{CYP_CARB}.


\paragraph{Regression tasks.}
For regression tasks (see~\autoref{fig:barplot_dperfs_vs_rdelta_reg_dataset}), quantization introduces a stark privacy-performance imbalance.
Unlike classification, even moderately aggressive quantization (e.g., \texttt{2 bits}) results in negative R2 scores, indicating worse-than-baseline predictions (even though the direct predictions are not accurate, we show in the \autoref{subsec:comp_results} the ordering of the predictions is preserved).
Only non-private quantizers (\texttt{4 - 5 bits}) achieve R2 scores comparable to original models.
This disparity arises because regression requires precise weight values to estimate continuous targets, whereas classification relies on decision boundaries that are more robust to quantization.
Consequently, regression tasks lack a viable privacy-performance trade-off: quantizers either degrade performance catastrophically or retain performance at the cost of privacy.
This underscores the need for novel quantization strategies tailored to regression.



\section{Discussions and Limitations}
\label{sec:ccl}
In this work, we investigated the privacy of quantization procedures for machine learning models, particularly their vulnerability to data leakage. We established a theoretical foundation by proving that, for both fixed and adaptive model-quantization procedures, the Membership Inference Security (MIS) of the learning algorithm is asymptotically determined by the distribution of the quantized models’ loss per sample.
We introduced a methodology for comparing quantization procedures in terms of privacy, with limited computational cost. Through extensive experiments on both synthetic and real-world datasets, we validated the effectiveness of our approach and explored the privacy-performance trade-offs of quantization in molecular property prediction, highlighting the practical implications of our findings.

Our study has some limitations.  Since our analysis focuses on evaluating a training procedure rather than individual trained models, it does not directly predict the security of a specific trained model. Additionally, due to computational constraints, we concentrated on Post-Training Quantization (PTQ) and did not examine Quantization-Aware Training (QAT), which presents a potential direction for future research. 
 Specifically, we aim to explore QAT, where $r_\gQ^n$ could be jointly optimized with the loss function.

\clearpage
\section{Impact Statement}
Our work focuses on the evaluation of the privacy of quantization procedures, which could help secure these models, thereby making it possible to share them more broadly while mitigating privacy risks for users and data contributors.
More generally, this paper presents work whose goal is to advance the field of Machine Learning. There are many potential societal consequences of our work, none which we feel must be specifically highlighted here.

\bibliography{bibliography}
\bibliographystyle{icml2025}


%%%%%%%%%%%%%%%%%%%%%%%%%%%%%%%%%%%%%%%%%%%%%%%%%%%%%%%%%%%%%%%%%%%%%%%%%%%%%%%
%%%%%%%%%%%%%%%%%%%%%%%%%%%%%%%%%%%%%%%%%%%%%%%%%%%%%%%%%%%%%%%%%%%%%%%%%%%%%%%
% APPENDIX
%%%%%%%%%%%%%%%%%%%%%%%%%%%%%%%%%%%%%%%%%%%%%%%%%%%%%%%%%%%%%%%%%%%%%%%%%%%%%%%
%%%%%%%%%%%%%%%%%%%%%%%%%%%%%%%%%%%%%%%%%%%%%%%%%%%%%%%%%%%%%%%%%%%%%%%%%%%%%%%
\newpage
\appendix
\onecolumn



\section{Additional Assumptions and Notations}
\label{sec:addass}

\subsection{Mathematical Setting}
\label{ssec:math-sett}

We give here the mathematical setting for the theoretical results of \autoref{sec:theo_res} and \autoref{sec:algo} to hold. Specifically, we use the mathematical setting of \cite{aubinais2023fundamental}. We redefine an MIA as follows.

\begin{definition}[MIA]
Any measurable map $\phi : \Theta\times\gZ\to\{0,1\}$ is considered to be a \textit{Membership Inference Attack}.
\end{definition}

Here we have encoded \textit{member} by $1$. We define a test point $\Tilde{\rz}$ as follows. Let $T\in\{0,1\}$ be a Bernoulli random variable with parameter $1/2$. Let $\rz_0\sim P$ and $U$ be a random variable whose distribution is $\hat{P}_n$ conditionally to the training dataset $\gD_n$, where $T, \rz_0$ and $U$ are independent. Additionally, $\rz_0$ and $T$ are independent of $\gD_n$. We define the test point by,

\begin{equation}
\label{eq:test_pt}
\Tilde{\rz}\coloneqq TU + (1-T)\rz_0.
\end{equation}

This definition is a formalization of the fact that a test point is either a member of the training dataset ($T=1$) or not a member ($T=0$). The random variable $\rz_0$ is used as a placeholder to represent the non-member property of the test point.\\
With this framework, the following relation holds,

\begin{theorem}[\cite{aubinais2023fundamental}]
For any distribution $P$ and any algorithm $\gA$, we have

\begin{equation}
\label{eq:mis_delta}
{\MIS}_n(P, \gA) = 1 - \|P_{(\hat{\theta}_n,\rz_1)} - P_{\hat{\theta}_n}\otimes P\|_{\textrm{TV}},
\end{equation}


where $\|\nu_1-\nu_2\|_{TV}$ is the total variation distance between the distributions $\nu_1$ and $\nu_2$.

\end{theorem}



\subsection{Fixed Quantizer}
\label{ssec:fq-add-ass}
We denote by $ L_{k,j} \coloneqq \ell(\thetaq_k,\rz_j) - \ell(\thetaq_1,\rz_j)$ the random loss gap between $\thetaq_k$ and $\thetaq_1$ evaluated on $\rz_j$, whose expectation is $\delta_k\coloneqq \E\cc{L_{k,j}}$. We additionally denote by $D \coloneqq \textrm{diag}\pp{\pp{\delta_k}_{k>1}}$ the diagonal \textbf{loss gaps matrix}. We give the additional hypotheses for \autoref{thm:fixed_pos}.

\begin{itemize}
    \item \textbf{H1.1.} We have $\E\cc{e^{<t, D^{-1}\cc{L_{k,1}-\delta_k}_{k>1}>}} < \infty$ for all $t$ in a neighborhood of $0\in\R^{K-1}$.
    \item \textbf{H1.2.} The minimal loss gap satisfies $\delta_2>0$.
\end{itemize}

A sufficient (albeit non-necessary) condition for H1.1 to hold is that the loss function $l$ is bounded. Specifically, the hypothesis H1.1 is sufficient for the constant of \autoref{thm:fixed_pos} to be strictly positive. If for all $t\not=0$ we have $\E\cc{e^{<t, D^{-1}\cc{L_{k,1}-\delta_k}_{k>1}>}} = \infty$, then the theorem becomes trivial as we would have $C_P^1=0$. The hypothesis H1.2 implies that the loss gaps matrix $D$ is invertible.

\subsection{Size-Adaptive Quantizers}
\label{ssec:saq-add-ass}


We give here additional notations and hypotheses for \autoref{thm:seqQ}. Let $\Thetaq_n\coloneqq \{\thetaq_1^n,\cdots,\thetaq_{K_n}^n\}$ be the image of the Size-Adaptive Quantizer $\gQ_n$, where $K_n$ may or may not depend on $n$. Based on the assumptions given in \autoref{thm:seqQ}, assume without loss of generality that for all $n\in\mathbb{N},$ $m_{\thetaq_{1}^n}<\cdots<m_{\thetaq_{K_n}^n}$. We recall that the \textbf{loss gaps} are defined as $\delta_k^n = m_{\thetaq_k^n} - m_{\thetaq_1^n}$ and the \textbf{loss variabilities} as $\pp{\sigma_k^n}^2\coloneqq \textrm{Var}\pp{L_{k,1}^n}$ where $L_{k,j}^n\coloneqq \ell(\thetaq_k^n,\rz_j) - \ell(\thetaq_1^n,\rz_j)$. We set $D^n\coloneqq \textrm{diag}\pp{\pp{\delta_k^n}_{k>1}}$. We will use the following hypotheses

\begin{itemize}
    \item \textbf{H2.1.} The limits $\sigma_k^2 = \liminf{n}\textrm{Var}(L_{k,1}^n)$ and
    $c_k = \liminf{n} \frac{\delta_2^n}{\delta_k^n}$ exist.
    \item \textbf{H2.2.} We have $\underset{n}{\sup}~\E\cc{\exp\pp{t\left\|\pp{\frac{\delta_2^n}{\delta_k^n}L_{k,1}^n}_{1<k\leq K_n}\right\|_2}}<\infty$, for all $t>0$.
    \item \textbf{H2.3.} There exists $K<\infty$ such that for all $k>K$, $c_k^2\sigma_k^2=0$.
    \item \textbf{H2.4.} The sequence of random variables $\pp{\pp{\frac{\delta_2^n}{\delta_k^n}L_{k,1}^n}_{1<k\leq K_n}}_n$ is tight when canonically seen as a sequence in $l_2(\R)$.
\end{itemize}

By definition, for all $k$, we have $\mkn{2}/\mkn{k}\in[0,1]$. If the loss function $\ell$ is bounded, then the loss variabilities $\pp{\sigma_k^n}^2$ are bounded as well. The hypothesis H2.1 is then only a light hypothesis.

%In hypothesis H2.2, the sequence $(\pp{\mkn{2}/\mkn{k}}L_{k,1}^n)_{1<k\leq K_n}$ can be seen as canonically mapped into $l_2(\R)$, the set of square-summable sequences. This means that at fixed $n$, for all $k>K_n$, the values of $\pp{\mkn{2}/\mkn{k}}L_{k,1}^n$ are simply set to $0$. 
% Importantly, hypothesis H2.2 is sufficient for this sequence of random variables to be tight.



A sufficient condition for H2.3 to hold is that there exists (asymptotically) finitely many global minima of the function $l$. Indeed, if there exists (asymptotically) finitely many global minima of the function $\ell$, say $K$, this means that for all $k > K$, we have $\delta_k^n\underset{n\to\infty}{\not\to}0$, whereas $\delta_2^n\toinf{n}0$ resulting in $c_k=0$ and H2.3. The value $K-1$ corresponds to the intrisic dimension of the asymptotic random variables $(\pp{\mkn{2}/\mkn{k}}L_{k,1}^n)_{1<k\leq K_n}$.


The hypotheses presented in \autoref{thm:seqQ}, i.e. $\sqrt{n}\delta_2^n\toinf{n}\infty$ and $\delta_2^n\toinf{n}0$, ensure that the loss gaps matrix $D^{n}$ is invertible. 





\section{Preliminary Results}
We give here preliminary results for the proof of the main theorem.

\begin{lemma}
\label{lem:1-1}
Let $K\geq 2$, $a, b \in (0,1)$. Let $P$ and $Q$ be two distributions over $\{1,\dots,K\}$ satisfying $P(1) = 1-a$ and $Q(1) = 1-b$. Then, we have

$$|a-b|\leq\|P-Q\|_{\textrm{TV}}\leq \frac{|a-b| + a + b}{2}.$$
\end{lemma}

% \begin{lemma}
% \label{lem:1-2}
% Let $Z$ be a non-negative random variable.% having a density. 
% Then 
% $$\E\left|Z-\E\cc{Z}\right| = 2\E\cc{Z}P\pp{Z\leq U\E\cc{Z}},$$

% where $U\sim Unif(0,1)$ independent of $Z$.
% \end{lemma}

% \begin{lemma}
% \label{lem:1-3}
% Let $X\in[0,a]$ a.s. such that $a>0$, and let $U\sim Unif(0,1)$ independent of $X$. Then 
% $$P\pp{X\leq U\E\cc{X}} \leq 1-\E\cc{X}/a.$$    

% \end{lemma}

\begin{lemma}
\label{lem:2-2}
Let $M$ be a $J\times J$ square matrix and $j \in \{1,\dots, J\}$. Assume that $M_{-j,-j}$ and ${M_{j,j} - M_{j,-j}\cc{M_{-j,-j}}^{-1}M_{-j,j}}$ are invertible, where $M_{-j,-j}$ is the sub-matrix of $M$ consisting of all entries except the $j^{th}$ row and column and $M_{-j,j}$ is the $j^{th}$ column of $M$ except its $j^{th}$ entry. Then we have 

$$\pp{M^{-1}}_{j,j} = \pp{M_{j,j} - M_{j,-j}\cc{M_{-j,-j}}^{-1}M_{-j,j}}^{-1}$$

\end{lemma}


% \begin{lemma}
% \label{lem:tight}
% Let $X_1,\cdots,X_n,\cdots$ be a sequence of $d-$dimensional random variables such that $\underset{n}{\sup}\,\E\cc{e^{t\|X_n\|}}<\infty$ for any $t>0$ and for some norm $\|\cdot\|$. Then the sequence $\pp{X_n}_n$ is tight.
% \end{lemma}


\begin{proof}[Proof of Lemma \ref{lem:1-1}]
By definition, we have 

\begin{align*}
    \|P-Q\|_{TV} &= \frac{1}{2}\sum_{j=1}^{K}|p_j - q_j| \\
    &= \frac{1}{2}|a-b| + \frac{1}{2}\sum_{j=2}^{K}|p_j-q_j|.
\end{align*}

By construction, we have $\sum_{j=2}^{K}p_j = a$ and $\sum_{j=2}^{K}q_j = b$. 
% C'est facile de montrer (voir les mêmes techniques utilisées pour les calculs de $C(P)$) que 
It is easy to see check that 
\begin{align*}
    \min\; \sum_{j=2}^{K}|p_j-q_j| &= |a-b|, \\
    \max\; \sum_{j=2}^{K}|p_j-q_j| &= a + b,
\end{align*}

where the minimum and the maximum are taken over all distributions satisfying the condition over $p_1$ and $q_1$, which concludes the proof.


\end{proof}

% \begin{proof}[Proof of Lemma \autoref{lem:1-2}]
% Let $F$ be the cumulative distribution function of $Z$ and $m\coloneqq \E[Z]$. Then we have

% \begin{align*}
%     \E\cc{\left|Z - m\right|} &= \E\cc{(m-Z)1_{Z\leq m}}  - \E\cc{(m-Z)1_{Z> m}}  \\
%     &= mF(m) - \E\cc{Z1_{Z\leq m}} - mP\pp{Z> m} + \E\cc{Z1_{Z> m}} \\ 
%     &= 2mF(m) - m - 2\E\cc{Z1_{Z\leq m}} + m \\ 
%     &= 2mF(m) - 2\E\cc{Z1_{Z\leq m}}\\
%     &\overset{(*)}{=}2mP\pp{Z\leq U\E\cc{Z}}
% \end{align*}

% where $(*)$ holds because of 

% \begin{align*}
%      \E\cc{Z1_{Z\leq m}} &= mF(m) - \E\cc{\pp{m-Z}1_{Z\leq m}} \\ 
%      &= mF(m) - \E\cc{\int_{Z}^m dv 1_{Z\leq m}} \\ 
%      &= mF(m) - \E\cc{\int_{0}^m 1_{Z\leq v}dv} \\ 
%      &= mF(m) - \int_{0}^mF(v)dv \\ 
%      &= mF(m) - m\int_{0}^1F(um)du.
% \end{align*}

% hence the result.



% We have
% \begin{align*}
%     \E\cc{\left|Z - \E\cc{Z}\right|} &= \int_0^\infty |z-\E\cc{Z}|f(z)dz \\ 
%     &= \int_0^{\E\cc{Z}}\pp{\E\cc{Z}-z}f(z)dz - \int_{\E\cc{Z}}^\infty \pp{\E\cc{Z}-z}f(z)dz \\ 
%     &= \E\cc{Z}P\pp{Z\leq \E\cc{Z}} - \E\cc{Z1_{Z\leq\E\cc{Z}}} - \E\cc{Z}P\pp{Z\geq \E\cc{Z}} + \E\cc{Z1_{Z\geq \E\cc{Z}}} \\ 
%     &= 2\E\cc{Z}P\pp{Z\leq\E\cc{Z}} - 2\E\cc{Z1_{Z\leq\E\cc{Z}}} \\ 
%     &\overset{(*)}{=}2\E\cc{Z}P\pp{Z\leq U\E\cc{Z}},
% \end{align*}

% where $(*)$ holds because of 

% \begin{align*}
%     \E[Z 1_{Z\leq \E\cc{Z}}] &= \int_0^{\E\cc{Z}}zf(z)dz \\ 
%     &= \E\cc{Z}F(\E\cc{Z}) - \int_0^{\E\cc{Z}} F(z)dz \\ 
%     &= \E\cc{Z}F(\E\cc{Z}) - \E\cc{Z}\E[F(\E\cc{Z}U)],
% \end{align*}

% whith $F$ being the cdf of $Z$.

% \end{proof}


% \begin{proof}[Proof of Lemma \autoref{lem:1-3}]
% Given that $X\in[0,a]$ a.s., for some random variable $U\sim Unif(0,1)$ independent de $X$, we have
% \begin{align*}
%     \E\cc{X} &= \int_{0}^{a}P(X>t)dt = aP(X > aU) \\ 
%     \E\cc{a-X} &= aP(X\leq aU).
% \end{align*}

% Additionally, we have $\E\cc{X} \leq a$, giving 

% \begin{align*}
%     P(X\leq U\E\cc{X}) &\leq P(X\leq aU) \\ 
%     &= 1-\E\cc{X}/a,
% \end{align*}

% which concludes the proof.

% \end{proof}

\begin{proof}[Proof of Lemma \ref{lem:2-2}]
First note that if $M$ is designed by block as follows,

\begin{align*}
    M&=\begin{pmatrix}
    A & B \\
    C & D 
    \end{pmatrix},
\end{align*}

then we have

\begin{equation}
\label{eq:M_inv}
    M^{-1}=\begin{pmatrix}
    \pp{A - BD^{-1}C}^{-1} & * \\
    * & * 
    \end{pmatrix},
\end{equation}

as long as $C$ and $A - BD^{-1}C$ are invertible.
Let $P_j$ and $Q_j$ be $J\times J$ matrices defined for all $j$ such that $P_j M$ permutes the first and the $j^{th}$ rows of $M$, and $Q_j M$ permutes the $(j-1)^{th}$ and the $j^{th}$ rows of $M$. For instance, if $J=4$ and $j=3$ then we have



\begin{align*}
    P_3&=\begin{pmatrix}
    0 & 0 & 1 & 0 \\ 
    0 & 1 & 0 & 0 \\ 
    1 & 0 & 0 & 0 \\ 
    0 & 0 & 0 & 1
    \end{pmatrix}
    ,&     Q_3&=\begin{pmatrix}
    1 & 0 & 0 & 0 \\
    0 & 0 & 1 & 0 \\
    0 & 1 & 0 & 0 \\
    0 & 0 & 0 & 1
    \end{pmatrix}.
\end{align*}


Note that we have $(P_j)^{-1} = P_j$ and $(Q_j)^{-1}=Q_j$. Let $R = Q_3Q_4\cdots Q_jP_jMP_jQ_j\cdots Q_4Q_3$. Developing the formula, we get 

\begin{equation}
\label{eq:RM}
R = \begin{pmatrix}
    M_{j,j} & M_{j,-j} \\ 
    M_{-j,j} & M_{-j,-j}
    \end{pmatrix}.
\end{equation}

On one hand, using \autoref{eq:M_inv} and \autoref{eq:RM}, we have $R^{-1} = \begin{pmatrix}
    \pp{M_{j,j} - M_{j,-j}\cc{M_{-j,-j}}^{-1}M_{-j,j}}^{-1}&\star \\ 
    \star&\star
    \end{pmatrix}$. On the other hand, distributing the inverse operator on the product of matrices and using again \autoref{eq:M_inv}, we have 

\begin{align*}
    R^{-1} &= Q_3Q_4\cdots Q_jP_jM^{-1}P_jQ_j\cdots Q_4Q_3 \\ &= \begin{pmatrix}
    \pp{M^{-1}}_{j,j}&\star \\ 
    \star&\star
    \end{pmatrix},
\end{align*}

which concludes the proof.
\end{proof}

% \begin{proof}[Proof of Lemma \ref{lem:tight}]
% Let $M_t\coloneqq \underset{n}{\sup}\,\E\cc{e^{t\|X_n\|}}$ for some $t>0$. The hypothesis states that $M_t<\infty$ for any $t>0$, but clearly we have $M_t\toinf{n}\infty$. Using exponential Markov inequality, we then have

% \begin{align*}
%     \underset{n}{\sup}P\left(\|X_n\|\geq M_t\right) &\leq \underset{n}{\sup}\, e^{-tM_t}\E\left[e^{t\|X_n\|}\right] \\ 
%     &= \frac{M_t}{e^{tM_t}}.
% \end{align*}

% As $M_t\toinf{t}\infty$, we have that for any $\varepsilon>0$, there exists $t_\varepsilon>0$ such that for any $t\geq t_\varepsilon$ we have $\frac{M_t}{e^{tM_t}}\leq \varepsilon$. Setting $I_\varepsilon\coloneqq B_{\|\cdot\|,M_{t_\varepsilon}}$ the $d-$dimensional centered ball of radius $M_{t_\varepsilon}$, we have 

% \begin{align*}
%     \underset{n}{\sup}\, P\left(X_n \not\in I_\varepsilon\right) &=  \underset{n}{\sup} \,P\left(\|X_n\| \geq M_{t_\varepsilon}\right)  \\ 
%     &\leq\varepsilon.
% \end{align*}

% But $I_\varepsilon$ is compact, hence the tightness.



% \end{proof}

\section{Main Theorems}
\label{sec:main_theorems}

This section is dedicated to the proof of the following theorems. Let denote by $\Delta_n\pp{P,\gA_{\gQ}}\coloneqq 1-{\MIS}_n(P, \gA_\gQ)$ the quantity of interest for which \autoref{thm:fixed_pos} and \autoref{thm:seqQ} give asymptotic results. From \autoref{eq:mis_delta}, we restate \autoref{thm:fixed_pos} and \autoref{thm:seqQ} as follows.


\begin{theorem}[\autoref{thm:fixed_pos}]
\label{thm:main_fixed_quant}
Let $\gQ$ be a fixed quantizer. We assume H1.1 and H1.2. Then, we have 

\begin{equation}
    \liminf{n}\frac{1}{n} \log\,\Delta_n(P,\gA_\gQ) \leq -\underset{x\in\Omega_{K-1}^c}{\inf}\,\underset{t\in\R^{K-1}}{\sup}\,\cc{<t,x> - \log\,\E\cc{e^{<t,D^{-1}\cc{L_{k,j}-\delta_k}_{k>1}>}}} < 0,
\end{equation}

where $\Omega_{K-1} = [-1,\infty)^{K-1}$.


\end{theorem}



\begin{theorem}[\autoref{thm:seqQ}]
\label{thm:main}
Let $\gQ_n$ be a Size-Adaptive quantizer. We assume H2.1, H2.2, H2.3 and H2.4. Assuming that $\liminf{n}\mkn{2} = 0$ and $\liminf{n}\sqrt{n}\delta_2^n=\infty$, we have

\begin{equation}
    \liminf{n}\frac{1}{n \pp{\mkn{2}}^2}\log\, \Delta_n(P,\gA_{\gQ_n}) \leq -1/2\sigma^2.  
\end{equation}

\end{theorem}

We start by giving the proof of \autoref{thm:main} before \autoref{thm:main_fixed_quant}.

\subsection{Proof of \autoref{thm:main}}

The proof of \autoref{thm:main} is immediate from Propositions \ref{prop:1} and \ref{prop:2} given below.

\begin{proposition}
\label{prop:1}
In the context of \autoref{thm:main}, we have 

\begin{equation}
    \liminf{n}\frac{1}{n \pp{\mkn{2}}^2}\log \Delta_n(P,\gA_{\gQ_n}) \leq -\underset{x\in\Omega^c}{\inf}\frac{x^T\Lambda^+x}{2},    
\end{equation}

where $\Lambda$ is defined as

\begin{align*}
     \Lambda &\coloneqq \begin{pmatrix}
        \sigma_{2,2}^2c_2^2 & \cdots & \sigma_{2, J}^2c_2c_K \\ 
        \vdots & \ddots & \vdots \\ 
       \sigma_{K, 2}^2c_{K}c_2 & \cdots & \sigma_{K,K}^2c_K^2
    \end{pmatrix},
\end{align*}

where $\sigma_{k,l}^2 = \liminf{n}Cov(L_{k,1}^n, L_{l,1}^n)$. The matrix $\Lambda^+$ is the Moore-Penrose pseudo-inverse $\Lambda$. 
Note that we have $\sigma_{k,k}^2 = \sigma_k^{2}$.
The set $\Omega$ is given by $\Omega \coloneqq [-1,\infty)^{K-1}$.
\end{proposition}

\begin{proposition}
\label{prop:2}
In the context of Proposition \ref{prop:1}, we have 

\begin{equation}
    \underset{x\in\Omega^c}{\inf}\frac{x^T\Lambda^+x}{2} = 1/2\sigma^2.
\end{equation}
\end{proposition}

\begin{proof}[Proof of Proposition \ref{prop:1}]
Recall that $\hat{\theta}_n\sim\gA_{\gQ_n}\pp{\rz_1,\cdots,\rz_n}$. Note that from \autoref{eq:mis_delta} we have
\begin{equation}
    \Delta_n(P,\gA_{\gQ_n}) = \E\cc{\|\gL\pp{\hat{\theta}_n} - \gL\pp{\hat{\theta}_n\mid \rz_1}\|_{\textrm{TV}}},
\end{equation}
where $\gL(X)$ is the probability law of $X$, and the expectation is taken over $\rz_1$. Letting $Z_k^n =  \mathbb{P}\pp{\hat{\theta}_n = \thetaq_k^n \mid \rz_1}$ and $p_k^n =  \mathbb{P}\pp{\hat{\theta}_n = \thetaq_k^n}$, we have
\begin{align*}
    \Delta_n(P,\gA_{\gQ_n}) &= \frac{1}{2}\sum_{k=1}^{K_n}\E\cc{\left|p_k^n - Z_k^n\right|} \\ 
    &\leq \frac{\E\cc{\left|p_1^n - Z_1^n\right|} + 2 - \E\cc{p_1^n + Z_1^n}}{2} \\ 
    &= \frac{\E\cc{\left|p_1^n - Z_1^n\right|} + 2\pp{1 - p_1^n}}{2} \\ 
    &\leq \frac{2\pp{1-p_1^n} + 2\pp{1-p_1^n}}{2} \\ 
    &= 2\pp{1-p_1^n},
\end{align*}

where the first inequality comes from Lemma \ref{lem:1-1}.
By construction, $\gA_{\gQ_n}$ minimizes the empirical loss. Letting $\Omega_{K-1} = [-1,\infty)^{K-1}$, we then have 
\begin{align}
\notag
    p_1^n &= \mathbb{P}\pp{1 = \underset{k}{\argmin}\left\{\frac{1}{n}\sum_{j=1}^{n} \ell\pp{\thetaq_k^n,\rz_j}\right\}} \\ 
\notag
    &=\mathbb{P}\pp{\forall k > 1, \frac{1}{n}\sum_{j=1}^{n}\cc{\ell\pp{\thetaq_k^n,\rz_j} - \ell\pp{\thetaq_1^n,\rz_j}}\geq 0} \\ 
\notag
    &= \mathbb{P}\pp{\forall k > 1, \frac{1}{n}\sum_{j=1}^n \cc{L_{k,j}^n - \mkn{k}} \geq - \mkn{k}} \\ 
\label{eq:Prob}
    &= \mathbb{P}\pp{\frac{1}{n}\sum_{j=1}^{n}\cc{L_{k,j}^n - \mkn{k}}_{k>1} \in D^n\Omega_{K-1}} \\ 
\notag
    &=\mathbb{P}\pp{\frac{1}{n\mkn{2}}\sum_{j=1}^{n}\cc{\pp{\mkn{2}}^{-1}D^n }^{-1}\cc{L_{k,j}^n - \mkn{k}}_{k>1} \in \Omega_{K-1}},
\end{align}

which gives

\begin{align*}
    1 - p_1^n &= \mathbb{P}\pp{\frac{1}{n\mkn{2}}\sum_{j=1}^{n}\cc{\pp{\mkn{2}}^{-1}D^n}^{-1}\cc{L_{k,j}^n - \mkn{k}}_{k>1} \in \Omega_{K-1}^c}.
\end{align*}


By H2.3, $\cc{\pp{\mkn{2}}^{-1}D^n}^{-1}\cc{L_{k,j}^n - \mkn{k}}_{k>1}$ lives (asymptotically) in a $(K-1)$-dimensional euclidean sub-space.
% Note that by H2.2 and the remark below it, the sequence $\gL\left(\frac{\delta_2^n}{\delta_k^n}\cc{L_{k,1}^n - \delta_k^n}_{k>1}\right)_n$ is tight.
By H2.2 and H2.4, as we live in an Hilbert space,  using \cite{araujo1980central}, we have $\gL\pp{\frac{1}{\sqrt{n}}\sum_{j=1}^n \cc{\pp{\mkn{2}}^{-1}D^n}^{-1}\cc{L_{k,j}^n - \mkn{k}}_{k>1} } \toinf{n} \gamma\coloneqq \gN_{K-1}\pp{0, \Lambda}$, where $\gN_{K-1}$ is the $(K-1)$-dimensional Gaussian distribution.
Now, using the fact that $\E\cc{L_{k,j}^n} = \delta_k^n$, H2.2, H2.4 and convergence to a Gaussian measure, using Theorem 2.2 of \cite{de1992moderate}, we get

\begin{align*}
   \liminf{n} \frac{1}{n\pp{\mkn{2}}^2}\log \pp{1-p_1^n} &= -\underset{x\in\Omega_{K-1}^c}{\inf}\left\{
   \begin{array}{ll}
       \frac{\displaystyle x^T\Lambda^{+}x}{2}  & \text{, if }x\in H_{\gamma}\\
       \infty & \text{, otherwise}
   \end{array}
   \right.\\ 
   &= - \underset{x\in\Omega^c}{\inf}\; \frac{x^T\Lambda^{+}x}{2},
\end{align*}

where $H_\gamma$ is the Hilbert space associated with $\gamma$ (see \cite{de1992moderate}). 
Hence the result.
\end{proof}

\begin{proof}[Proof of Proposition \ref{prop:2}]
Let $M=\Lambda^+$. Note that $\Omega^c = \{x\in\R^{K-1} : \exists j, x_j < -1\}$, giving

$$\underset{x\in\Omega^c}{\inf}\; x^TM x = \underset{j}{\min}\underset{x_{-j}\in\R^{K-1}}{\inf}\underset{x_j<-1}{\inf} x^TMx,$$

where we write $x_{-j}$ equals $x$ where we omit its $j^{th}$ entry. We then shall write 

\begin{align*}
    x^TMx &= x_j^2M_{j,j} + 2x_jx_{-j}^T M_{-j,j} + x_{-j}^TM_{-j,-j}x_{-j}.
\end{align*}




The infimum must be reached on the frontier of the set, i.e. such that $x_j=-1$ for some $j$. Indeed assuming $x$ in the interior of $\Omega^c$, we have that $x_j>-1$ for some $j$. Then for any $1<\alpha<|x_j|$, $x_j/\alpha < -1$ meaning that $x/\alpha$ still belongs to the interior of $\Omega^c$. However, $(x/\alpha)^TM(x/\alpha) = \frac{1}{\alpha^2}x^TMx < x^TMx$, which shows that $x$ was not optimal. For an optimal $x$, we then have 

\begin{align*}
    x^TMx &= M_{j,j} - 2x_{-j}^T M_{-j,j} + x_{-j}^TM_{-j,-j}x_{-j}.
\end{align*}

It is then sufficient to study the optimization problem over $x_{-j}$, which amounts down to the optimization of a quadratic function, whose minimum is then reached for $x_{- j}$ satisfying 

\begin{align*}
    \nabla_{x_{-j}}\pp{- 2x_{-j}^T M_{-j,j} + x_{-j}^TM_{-j,-j}x_{-j}} & = 0 \\ 
    \iff x_{-j} &= \pp{M_{-j,-j}}^{-1}M_{-j,j},
\end{align*}

giving

\begin{align*}
    \underset{x_{-j}\in\R^{J-1}}{\inf}\underset{x_j<-1}{\inf} x^TMx &= M_{j,j} - M_{j,-j}\pp{M_{-j,-j}}^{-1}M_{-j,j}.
\end{align*}

Applying Lemma \ref{lem:2-2} concludes the proof.

\end{proof}

\subsection{Proof of \autoref{thm:main_fixed_quant}}

We give here the proof of \autoref{thm:main_fixed_quant}.

\begin{proof}[Proof of \autoref{thm:main_fixed_quant}]
Following the same steps of the proof of \autoref{thm:main} up to \autoref{eq:Prob}, we have by removing the superscripts $n$ where needed,

\begin{align*}
    p_1 &= P\pp{\frac{1}{n}\sum_{j=1}^{n}\cc{L_{k,j} - \delta_k}_{k>1} \in D\Omega_{K-1}} \\ 
    &= P\pp{\frac{1}{n}\sum_{j=1}^{n}D^{-1}\cc{L_{k,j} - \delta_k}_{k>1} \in \Omega_{K-1}},
\end{align*}

giving

\begin{align*}
    1-p_1 &= P\pp{\frac{1}{n}\sum_{j=1}^{n}D^{-1}\cc{L_{k,j} - \delta_k}_{k>1} \in \Omega_{K-1}^c}.
\end{align*}

From H1.1 , the result follows immediately using Corollary 6.1.6 of \cite{dembo2009large}.

\end{proof}



\FloatBarrier
\section{Quantizations}
\label{app:quantization}

\begin{figure}
    \centering
    \includegraphics[width=0.9\linewidth]{Figures/quantizers}
    \caption{Illustration of the quantization functions used on the interval $[-1, 1]$.}
    \label{fig:quantizers}
\end{figure}

\begin{table}
    \centering
    \caption{
        Description of the quantizers used in the experiments.
    }
    \label{tab:quantizers}
    \begin{tabular}{c|c}
{quantizer} & {$\gQ(\theta_j)$} \\
\midrule
Sign &  $\gQ(\theta_j) = \frac{\theta_j}{|\theta_j|}$\\
\midrule
1.58b 33\% & $\gQ(\theta_j) = \left\{
   \begin{array}{ll}
        \frac{\theta_j}{|\theta_j|},& \text{if }|\theta_j| < q(|\theta|, 0.33)\\
       0, & \text{otherwise}
   \end{array}
   \right.$\\
\midrule
1.58b 50\%  & $\gQ(\theta_j) = \left\{
   \begin{array}{ll}
        \frac{\theta_j}{|\theta_j|},& \text{if }|\theta_j| < q(|\theta|, 0.5)\\
       0, & \text{otherwise}
   \end{array}
   \right.$\\
\midrule
1.58b 90\%  & $\gQ(\theta_j) = \left\{
   \begin{array}{ll}
        \frac{\theta_j}{|\theta_j|},& \text{if }|\theta_j| < q(|\theta|, 0.9)\\
       0, & \text{otherwise}
   \end{array}
   \right.$\\
\midrule
2 bits & $\gQ(\theta_j) =\frac{\theta_j}{|\theta_j|} \frac{\alpha}{2} \times \text{int}\pp{1+ \text{clip}(\frac{2\theta_j}{\alpha}, 0, 2)}, \quad \alpha = 2^{\text{round}(\log_2\pp{\max |\theta|})}$\\
\midrule
3 bit & $\gQ(\theta_j) =\frac{\theta_j}{|\theta_j|} \frac{\alpha}{4} \times \text{int}\pp{1+ \text{clip}(\frac{4\theta_j}{\alpha}, 0, 4)}, \quad \alpha = 2^{\text{round}(\log_2\pp{\max |\theta|})}$\\
\midrule
4 bits & $\gQ(\theta_j) =\frac{\theta_j}{|\theta_j|} \frac{\alpha}{8} \times \text{int}\pp{1+ \text{clip}(\frac{8\theta_j}{\alpha}, 0, 8)}, \quad \alpha = 2^{\text{round}(\log_2\pp{\max |\theta|})}$\\
\midrule
5 bits & $\gQ(\theta_j) =\frac{\theta_j}{|\theta_j|} \frac{\alpha}{16} \times \text{int}\pp{1+ \text{clip}(\frac{16\theta_j}{\alpha}, 0, 16)}, \quad \alpha = 2^{\text{round}(\log_2\pp{\max |\theta|})}$\\
\bottomrule
\end{tabular}


\end{table}

In this section, we provide additional details on the quantization procedures used in our experiments.
\autoref{tab:quantizers} summarizes the quantizers used in the experiments, and~\autoref{fig:quantizers} illustrates how the different functions used quantize the interval $[-1, 1]$.

\FloatBarrier
\newpage

\section{Synthetic experiments}
\label{app:synthetic_details}

\begin{wrapfigure}[28]{r}{0.53\linewidth}
   % \vspace{-4cm}
    \centering
    \includegraphics[width=\linewidth]{Figures/synth_figs/barplot_dperfs_vs_rdelta}
    \caption{
        Evolution of $\qcertif$ with the quantized model's performance as the ratio of the original model's training accuracy (line) for various quantizers in our synthetic setup.
    }
    \label{fig:barplot_synth}
\end{wrapfigure}

\subsection{Trade-off between privacy and performance}

In this section, we provide additional details on the synthetic experiments conducted to evaluate the trade-off between privacy and performance of quantized models.
This trade-off is illustrated in~\autoref{fig:barplot_synth}, where we plot the evolution of the quantification certificate $\qcertif$ with the model's performance as the ratio of the original model's training accuracy.
We find that the trade-off between privacy and performance is less pronounced compared to the real-world experiments.
In particular, while the least private quantizers do preserve most of the original performance, the more private quantizers seem to achieve similar performances on some data distribution (in particular, when $\sigma = 3$).
This could be explained by the low performances of the trained models on such distribtuions as illustrated in~\autoref{tab:synth_perf}.
Furthermore, simple mixture of Gaussians might not be relevant to capture the complexity of real-world data distributions, and we therefore decided to focus our analysis of the performance-privacy trade-off on real-world applications.

\begin{table}
    \centering
    \caption{
        Description of the quantizers used in the synthetic experiments.
    }
    \label{tab:synth_perf}
    \resizebox{\linewidth}{!}{\subsection{Classification and Labeling}

To mitigate potential subjective biases during the labeling phase, we allocated each of the \numbug merged bug-fix pull requests to two co-authors, who are also contributors and developers of both Apollo and Autoware open-source projects and possess enough background in the ADS domain. 
We employed the open-coding strategy of intercoder reliability~\cite{intercoder_reliability} to help strengthen the labeling process.
Our methodology necessitated that each co-author independently assess the bug, which involved meticulous examination of the source code, commit logs, code reviews, pull request information, and associated issue descriptions to recognize the labeling items.

In our work, we commenced with the root cause and symptom taxonomies in a previous ADS bug study~\cite{GarciaF0AXC20} and also taxonomy of generic \bfps focusing on the syntactic level~\cite{PanKW09,SotoTWGL16,CamposM17,IslamZ20}
as a foundation for ADS bug analysis. 
The taxonomy of root causes was subsequently augmented by employing an open-coding paradigm, thereby broadening the spectrum. 
For the pull request that eluded classification within the foundational taxonomy, each co-author designated a label for it. Post-labeling, co-authors collaboratively reconciled any discrepancies in their classifications. 
We added two root causes, \textbf{Syntax, Naming, and Typography (SNT)} that involves errors in the basic structure of the code, including syntactical mistakes, naming conventions, and typographical errors, and \textbf{Dependency Issues (DEP)} related to importing, versioning, and managing dependencies.

For semantic \bfps ~\yuntianyihl{and \bfas}, none of the previous work could provide a useful taxonomy due to the domain-specific nature of ADS. In this research, the open coding process was employed to refine and identify distinct semantic \bfps~\yuntianyihl{and \bfas}.
A preliminary investigation was conducted involving two co-authors who independently examined the bug fixes to establish a tentative classification framework. Each rater recommended a series of categories, which were later amalgamated and refined during a face-to-face session attended by all contributing authors. This meeting served as a platform to validate and integrate the classification schemes proposed by the individual raters. This reconciliation process led to updates in the classification scheme. 
\yuntianyihl{We validated the final classification by consulting with researchers in the ADS domain and developers from Apollo and Autoware. Their expertise helped refine the classification scheme, and we relabeled the affected pull requests accordingly.}
The classification result is presented in \autoref{sec:taxonomy}.
Our findings indicated that a single bug origin may result in multiple symptoms and \bfps. Therefore, certain bugs were cataloged into multiple categories, unlike the previous study~\cite{GarciaF0AXC20} only considered a single symptom for each bug.
In the process of labeling, two co-authors, possessing expertise in ADS, were engaged to categorize the \bfps according to the specified schema. 
The degree of concordance between these raters was quantified using Cohen's Kappa coefficient~\cite{VieiraKS10}, which was used by a recent \bfp study~\cite{IslamPNR20}. 
In instances of labeling discrepancies, periodic discussions were conducted to achieve reconciliation. 
Throughout this process, the Kappa score persistently exceeded 80\%, indicative of a robust understanding and unanimous agreement among the raters~\cite{mchugh2012interrater}.
}
\end{table}


%\FloatBarrier

\subsection{Stability and Computational Complexity}
\label{ssec:stability}

\begin{figure}[h]
        \begin{subfigure}{0.48\linewidth}
        \centering
        \includegraphics[width=\linewidth]{Figures/synth_figs/bsl_corr_rank}
        \caption{
            Baseline MIS
        }
        \label{fig:bsl_corr_rank}
    \end{subfigure}
    \begin{subfigure}{0.48\linewidth}
        \centering
        \includegraphics[width=\linewidth]{Figures/synth_figs/rdelta_corr_rank}
        \caption{
            $\qcertif$
        }
        \label{fig:rdelta_corr_rank}
    \end{subfigure}
    \caption{
        Correlation between the rankings obtained with the baseline MIS method (resp. $\qcertif$) at a given number of run, with the ranking obtained with the baseline MIS method (resp. $\qcertif$) at 300 runs.
    }
    \label{fig:rank_corr}
\end{figure}

As explained in~\autoref{ssec:baseline_estimation}, the baseline approach consists in training a discriminator to distinguish between samples from the training set of a given $\thetan$ and samples from the product distribution $P_{\thetan}\otimes P$.
Similarly the $\qcertif$-based approach relies on the traing of multiple models to average the values of $\qcertif$ obtained.

The computational overhead induced by the $\qcertif$-based approach, namely computing the validation loss of the quantized models, is negligible compared to the total training time (1s against 4m).
Similarly, the training of the discriminator takes only about 40m.

As a result, training multiple models $\thetan$ over multiple runs is the computational bottleneck of our privacy evaluations.
To properly evaluate the time required to obtain both rankings, one would have to answer the following question: 'How many runs do i need to launch to ensure the ranking I obtained is stable?'

\autoref{fig:rank_corr} shows how after $15$ runs, the rankings obtained with $\qcertif$ are already highly correlated with the rankings obtained with 300 runs, while the rankings obtained with the baseline MIS method require $150$ runs to reach the same level of correlation.
As a result, the time required to obtain stable rankings with $\qcertif$ is significantly lower (\(\approx 1\)h) than with the baseline MIS method (\(\approx 10\)h).


\subsection{Visualization of the datasets}

We provide in~\autoref{fig:synth_datasets} a visualization of the synthetic datasets used in the experiments, through a PCA projection in dimension 2.
This visualization helps understand how different data distribution might result in different empirical results, as some datasets are more challenging than others, such as the dataset with $n_{\textrm{cluster}} = 6$ and $\sigma = 1.5$, for whom the labels of the datapoints are easily separable, while $n_{\textrm{cluster}} = 16$ and $\sigma = 3$ provides a more challenging dataset, with overlapping clusters.

\begin{figure}[ht]
    \centering
    %%% 6
    \begin{subfigure}{0.3\linewidth}
        \centering
        \includegraphics[width=\linewidth]{Figures/synth_figs/datasets/6-1.5}
        \caption{
            $n_{\textrm{cluster}} = 6, \sigma = 1.5$
        }
        \label{fig:6-1.5}
    \end{subfigure}
    \begin{subfigure}{0.3\linewidth}
        \centering
        \includegraphics[width=\linewidth]{Figures/synth_figs/datasets/6-2}
        \caption{
            $n_{\textrm{cluster}} = 6, \sigma = 2$
        }
        \label{fig:6-2}
    \end{subfigure}
    \begin{subfigure}{0.3\linewidth}
        \centering
        \includegraphics[width=\linewidth]{Figures/synth_figs/datasets/6-3}
        \caption{
            $n_{\textrm{cluster}} = 6, \sigma = 3$
        }
        \label{fig:6-3}
    \end{subfigure}
    %%% 8
        \begin{subfigure}{0.3\linewidth}
        \centering
        \includegraphics[width=\linewidth]{Figures/synth_figs/datasets/8-1.5}
        \caption{
            $n_{\textrm{cluster}} = 8, \sigma = 1.5$
        }
        \label{fig:8-1.5}
    \end{subfigure}
    \begin{subfigure}{0.3\linewidth}
        \centering
        \includegraphics[width=\linewidth]{Figures/synth_figs/datasets/8-2}
        \caption{
            $n_{\textrm{cluster}} = 8, \sigma = 2$
        }
        \label{fig:8-2}
    \end{subfigure}
    \begin{subfigure}{0.3\linewidth}
        \centering
        \includegraphics[width=\linewidth]{Figures/synth_figs/datasets/8-3}
        \caption{
            $n_{\textrm{cluster}} = 8, \sigma = 3$
        }
        \label{fig:8-3}
    \end{subfigure}
    %%% 16
    \begin{subfigure}{0.3\linewidth}
        \centering
        \includegraphics[width=\linewidth]{Figures/synth_figs/datasets/16-1.5}
        \caption{
            $n_{\textrm{cluster}} = 16, \sigma = 1.5$
        }
        \label{fig:16-1.5}
    \end{subfigure}
    \begin{subfigure}{0.3\linewidth}
        \centering
        \includegraphics[width=\linewidth]{Figures/synth_figs/datasets/16-2}
        \caption{
            $n_{\textrm{cluster}} = 16, \sigma = 2$
        }
        \label{fig:16-2}
    \end{subfigure}
    \begin{subfigure}{0.3\linewidth}
        \centering
        \includegraphics[width=\linewidth]{Figures/synth_figs/datasets/16-3}
        \caption{
            $n_{\textrm{cluster}} = 16, \sigma = 3$
        }
        \label{fig:16-3}
    \end{subfigure}
    \caption{
        Visualization of the synthetic datasets used in the experiments, through a PCA projection in dimension 2 (the original space is $\mathbb{R}^{128}$).
    }
    \label{fig:synth_datasets}
\end{figure}


\FloatBarrier
\section{Molecular experiments}
\label{app:molecular_details}

\subsection{Comprehensive results}
\label{subsec:comp_results}

We show in~\autoref{tab:mol_classification} and~\autoref{tab:mol_reg} the comprehensive results of the quantized models on the classification and regression tasks, respectively.

We observe that while the quantized models are generally less accurate than the original models, they still achieve reasonable performance on the classification tasks.
On regression examples, the quantized models' performances are significantly lower than the original models.
In particular, when the quantization quantizes on less than 4 bits, the prediction of the molecular properties are almost consistently lower than a simple mean prediction.
As explained in~\autoref{subsec:molecular_expe}, this result is expected, as while classification tasks relies on the definition of boundary between classes, regression tasks require a fine-grained prediction of the target value.

However, while the direct predictions of the quantized models do not provide a good estimate of the target value, the ordering of the predictions is still preserved, as shown by the Spearman correlation between the quantized models' predictions and the labels in~\autoref{tab:mol_reg_sp}.


\begin{table*}
    \caption{
        AUROC performance of the quantized models on the classification tasks, averaged over all embedders.
    }
    \label{tab:mol_classification}
    \resizebox{\textwidth}{!}{
        \begin{tabular}{r|cccccccc|c}
{} & {\rotatebox{90}{\shortstack{Sign}}} & {\rotatebox{90}{\shortstack{1.58b 33\%}}} & {\rotatebox{90}{\shortstack{1.58b 50\%}}} & {\rotatebox{90}{\shortstack{1.58b 90\%}}} & {\rotatebox{90}{\shortstack{2 bits}}} & {\rotatebox{90}{\shortstack{3 bits}}} & {\rotatebox{90}{\shortstack{4 bits}}} & {\rotatebox{90}{\shortstack{5 bits}}} & {\rotatebox{90}{\shortstack{original}}} \\
{dataset} & {} & {} & {} & {} & {} & {} & {} & {} & {} \\
\midrule
AMES & 0.773 \tiny (89\%) & 0.781 \tiny (90\%) & 0.784 \tiny (90\%) & 0.748 \tiny (86\%) & 0.843 \tiny (97\%) & 0.853 \tiny (98\%) & 0.859 \tiny (99\%) & \textbf{0.861 \tiny (99\%)} & \textbf{\underline{0.862 \tiny (100\%)}} \\
BBB Martins & 0.800 \tiny (89\%) & 0.803 \tiny (89\%) & 0.807 \tiny (90\%) & 0.804 \tiny (89\%) & 0.890 \tiny (99\%) & \textbf{0.895 \tiny (99\%)} & \textbf{0.895 \tiny (99\%)} & \textbf{\underline{0.896 \tiny (100\%)}} & \textbf{\underline{0.896 \tiny (100\%)}} \\
Bioavailability Ma & 0.586 \tiny (94\%) & 0.588 \tiny (94\%) & 0.590 \tiny (95\%) & 0.579 \tiny (93\%) & 0.619 \tiny (99\%) & \textbf{0.622 \tiny (100\%)} & \textbf{0.622 \tiny (100\%)} & \textbf{\underline{0.623 \tiny (100\%)}} & \textbf{0.622 \tiny (100\%)} \\
CYP2C9 Substrate CarbonMangels & 0.558 \tiny (86\%) & 0.557 \tiny (86\%) & 0.561 \tiny (86\%) & 0.589 \tiny (90\%) & 0.642 \tiny (99\%) & 0.646 \tiny (99\%) & \textbf{0.647 \tiny (99\%)} & \textbf{0.647 \tiny (99\%)} & \textbf{\underline{0.648 \tiny (100\%)}} \\
CYP2C9 Veith & 0.787 \tiny (89\%) & 0.793 \tiny (90\%) & 0.799 \tiny (91\%) & 0.827 \tiny (94\%) & 0.868 \tiny (99\%) & 0.873 \tiny (99\%) & \textbf{0.876 \tiny (99\%)} & \textbf{\underline{0.877 \tiny (99\%)}} & \textbf{\underline{0.877 \tiny (100\%)}} \\
Carcinogens Lagunin & 0.766 \tiny (91\%) & 0.765 \tiny (91\%) & 0.772 \tiny (92\%) & 0.791 \tiny (94\%) & 0.824 \tiny (98\%) & 0.830 \tiny (99\%) & 0.831 \tiny (99\%) & \textbf{0.832 \tiny (99\%)} & \textbf{\underline{0.833 \tiny (100\%)}} \\
ClinTox & 0.561 \tiny (80\%) & 0.555 \tiny (79\%) & 0.557 \tiny (79\%) & 0.589 \tiny (84\%) & 0.698 \tiny (98\%) & 0.702 \tiny (99\%) & 0.704 \tiny (99\%) & \textbf{0.706 \tiny (99\%)} & \textbf{\underline{0.707 \tiny (100\%)}} \\
DILI & 0.827 \tiny (92\%) & 0.830 \tiny (93\%) & 0.831 \tiny (93\%) & 0.843 \tiny (94\%) & 0.888 \tiny (99\%) & \textbf{0.891 \tiny (99\%)} & \textbf{\underline{0.892 \tiny (99\%)}} & \textbf{\underline{0.892 \tiny (99\%)}} & \textbf{\underline{0.892 \tiny (100\%)}} \\
HIA Hou & 0.805 \tiny (91\%) & 0.805 \tiny (91\%) & 0.804 \tiny (91\%) & 0.790 \tiny (89\%) & \textbf{0.882 \tiny (99\%)} & \textbf{\underline{0.883 \tiny (99\%)}} & \textbf{\underline{0.883 \tiny (99\%)}} & \textbf{\underline{0.883 \tiny (100\%)}} & \textbf{\underline{0.883 \tiny (100\%)}} \\
PAMPA NCATS & 0.585 \tiny (81\%) & 0.583 \tiny (81\%) & 0.584 \tiny (81\%) & 0.583 \tiny (81\%) & 0.708 \tiny (99\%) & 0.711 \tiny (99\%) & \textbf{0.713 \tiny (99\%)} & \textbf{\underline{0.714 \tiny (99\%)}} & \textbf{\underline{0.714 \tiny (100\%)}} \\
Pgp Broccatelli & 0.856 \tiny (92\%) & 0.858 \tiny (92\%) & 0.859 \tiny (92\%) & 0.864 \tiny (93\%) & 0.921 \tiny (99\%) & \textbf{0.924 \tiny (99\%)} & \textbf{0.924 \tiny (100\%)} & \textbf{0.924 \tiny (100\%)} & \textbf{\underline{0.925 \tiny (100\%)}} \\
Skin  Reaction & 0.664 \tiny (89\%) & 0.667 \tiny (89\%) & 0.668 \tiny (90\%) & 0.670 \tiny (90\%) & 0.735 \tiny (99\%) & 0.740 \tiny (99\%) & 0.741 \tiny (99\%) & \textbf{0.742 \tiny (99\%)} & \textbf{\underline{0.743 \tiny (100\%)}} \\
hERG & 0.728 \tiny (91\%) & 0.726 \tiny (91\%) & 0.726 \tiny (91\%) & 0.739 \tiny (92\%) & 0.793 \tiny (99\%) & 0.795 \tiny (99\%) & \textbf{0.796 \tiny (99\%)} & \textbf{\underline{0.797 \tiny (99\%)}} & \textbf{\underline{0.797 \tiny (100\%)}} \\
hERG (k) & 0.767 \tiny (88\%) & 0.783 \tiny (90\%) & 0.789 \tiny (91\%) & 0.757 \tiny (87\%) & 0.819 \tiny (94\%) & 0.837 \tiny (96\%) & 0.855 \tiny (98\%) & \textbf{0.862 \tiny (99\%)} & \textbf{\underline{0.866 \tiny (100\%)}} \\
\end{tabular}

    }
\end{table*}

\begin{table*}
    \caption{
        R2 performance of the quantized models on the regression tasks, averaged over all embedders. If the R2 score is lesser than $-1$, we display $-$inf for clarity.
    }
    \label{tab:mol_reg}
\resizebox{\textwidth}{!}{
        \begin{tabular}{r|cccccccc|c}
{} & {\rotatebox{90}{\shortstack{Sign}}} & {\rotatebox{90}{\shortstack{1.58b 33\%}}} & {\rotatebox{90}{\shortstack{1.58b 50\%}}} & {\rotatebox{90}{\shortstack{1.58b 90\%}}} & {\rotatebox{90}{\shortstack{2 bits}}} & {\rotatebox{90}{\shortstack{3 bits}}} & {\rotatebox{90}{\shortstack{4 bits}}} & {\rotatebox{90}{\shortstack{5 bits}}} & {\rotatebox{90}{\shortstack{original}}} \\
{dataset} & {} & {} & {} & {} & {} & {} & {} & {} & {} \\
\midrule
Caco2 Wang & -inf \tiny (-inf\%) & -inf \tiny (-inf\%) & -inf \tiny (-inf\%) & -inf \tiny (-inf\%) & -inf \tiny (-364\%) & 0.008 \tiny (1\%) & 0.458 \tiny (75\%) & \textbf{0.567 \tiny (93\%)} & \textbf{\underline{0.609 \tiny (100\%)}} \\
HydrationFreeEnergy FreeSolv & -inf \tiny (-inf\%) & -inf \tiny (-inf\%) & -inf \tiny (-inf\%) & -inf \tiny (-inf\%) & -0.466 \tiny (-70\%) & 0.412 \tiny (55\%) & 0.669 \tiny (91\%) & \textbf{0.715 \tiny (98\%)} & \textbf{\underline{0.725 \tiny (100\%)}} \\
LD50 Zhu & -inf \tiny (-inf\%) & -inf \tiny (-inf\%) & -inf \tiny (-inf\%) & -inf \tiny (-inf\%) & -inf \tiny (-578\%) & -0.129 \tiny (-25\%) & 0.339 \tiny (67\%) & \textbf{0.454 \tiny (89\%)} & \textbf{\underline{0.505 \tiny (100\%)}} \\
Lipophilicity (az) & -inf \tiny (-inf\%) & -inf \tiny (-inf\%) & -inf \tiny (-inf\%) & -inf \tiny (-inf\%) & -inf \tiny (-539\%) & -0.037 \tiny (-7\%) & 0.404 \tiny (72\%) & \textbf{0.508 \tiny (91\%)} & \textbf{\underline{0.552 \tiny (100\%)}} \\
PPBR AZ & -inf \tiny (-inf\%) & -inf \tiny (-inf\%) & -inf \tiny (-inf\%) & -inf \tiny (-inf\%) & -inf \tiny (-inf\%) & -0.480 \tiny (-233\%) & 0.022 \tiny (2\%) & \textbf{0.156 \tiny (65\%)} & \textbf{\underline{0.229 \tiny (100\%)}} \\
Solubility AqSolDB & -inf \tiny (-inf\%) & -inf \tiny (-inf\%) & -inf \tiny (-inf\%) & -inf \tiny (-inf\%) & -inf \tiny (-792\%) & -0.081 \tiny (-10\%) & 0.575 \tiny (72\%) & \textbf{0.740 \tiny (93\%)} & \textbf{\underline{0.792 \tiny (100\%)}} \\
VDss Lombardo & -inf \tiny (-inf\%) & -inf \tiny (-inf\%) & -inf \tiny (-inf\%) & -inf \tiny (-inf\%) & -0.859 \tiny (-287\%) & -0.010 \tiny (6\%) & 0.175 \tiny (73\%) & \textbf{0.224 \tiny (91\%)} & \textbf{\underline{0.248 \tiny (100\%)}} \\
\end{tabular}

    }
\end{table*}


\begin{table*}
    \caption{
        Spearman correlations between the labels and the predictions of the quantized models on the regression tasks, averaged over all embedders.
    }
    \label{tab:mol_reg_sp}
\resizebox{\textwidth}{!}{
        \begin{tabular}{r|cccccccc|c}
{} & {\rotatebox{90}{\shortstack{Sign}}} & {\rotatebox{90}{\shortstack{1.58b 33\%}}} & {\rotatebox{90}{\shortstack{1.58b 50\%}}} & {\rotatebox{90}{\shortstack{1.58b 90\%}}} & {\rotatebox{90}{\shortstack{2 bits}}} & {\rotatebox{90}{\shortstack{3 bits}}} & {\rotatebox{90}{\shortstack{4 bits}}} & {\rotatebox{90}{\shortstack{5 bits}}} & {\rotatebox{90}{\shortstack{original}}} \\
{dataset} & {} & {} & {} & {} & {} & {} & {} & {} & {} \\
\midrule
Caco2 Wang & 0.673 \tiny (89\%) & 0.713 \tiny (95\%) & 0.720 \tiny (95\%) & 0.679 \tiny (90\%) & 0.690 \tiny (92\%) & 0.733 \tiny (97\%) & 0.745 \tiny (99\%) & \textbf{0.748 \tiny (99\%)} & \textbf{\underline{0.750 \tiny (100\%)}} \\
HydrationFreeEnergy FreeSolv & 0.905 \tiny (98\%) & 0.907 \tiny (99\%) & 0.906 \tiny (98\%) & 0.893 \tiny (97\%) & 0.910 \tiny (99\%) & 0.913 \tiny (99\%) & \textbf{0.915 \tiny (99\%)} & \textbf{\underline{0.916 \tiny (99\%)}} & \textbf{\underline{0.916 \tiny (100\%)}} \\
LD50 Zhu & 0.572 \tiny (83\%) & 0.611 \tiny (89\%) & 0.618 \tiny (90\%) & 0.533 \tiny (77\%) & 0.596 \tiny (87\%) & 0.636 \tiny (92\%) & 0.669 \tiny (97\%) & \textbf{0.679 \tiny (99\%)} & \textbf{\underline{0.685 \tiny (100\%)}} \\
Lipophilicity (az) & 0.658 \tiny (87\%) & 0.700 \tiny (92\%) & 0.705 \tiny (93\%) & 0.637 \tiny (84\%) & 0.669 \tiny (88\%) & 0.713 \tiny (94\%) & 0.741 \tiny (98\%) & \textbf{0.749 \tiny (99\%)} & \textbf{\underline{0.754 \tiny (100\%)}} \\
PPBR AZ & 0.561 \tiny (98\%) & 0.563 \tiny (98\%) & 0.564 \tiny (99\%) & 0.555 \tiny (97\%) & 0.565 \tiny (99\%) & 0.568 \tiny (99\%) & \textbf{0.569 \tiny (99\%)} & \textbf{\underline{0.570 \tiny (99\%)}} & \textbf{\underline{0.570 \tiny (100\%)}} \\
Solubility AqSolDB & 0.838 \tiny (94\%) & 0.853 \tiny (96\%) & 0.852 \tiny (96\%) & 0.756 \tiny (85\%) & 0.842 \tiny (94\%) & 0.864 \tiny (97\%) & 0.879 \tiny (99\%) & \textbf{0.884 \tiny (99\%)} & \textbf{\underline{0.887 \tiny (100\%)}} \\
VDss Lombardo & 0.570 \tiny (99\%) & 0.572 \tiny (99\%) & 0.572 \tiny (99\%) & 0.560 \tiny (97\%) & 0.572 \tiny (99\%) & 0.573 \tiny (99\%) & \textbf{0.575 \tiny (99\%)} & \textbf{0.575 \tiny (99\%)} & \textbf{\underline{0.576 \tiny (100\%)}} \\
\end{tabular}

    }
\end{table*}

\subsection{Embedder privacy}

\autoref{fig:barplot_dperfs_vs_rdelta} shows the evolution of the privacy of each downstream model $\qcertif$ along with relative performances of the quantized models compared to the original for each pretrained embedder.
For every pretrained embedder, we see no significative difference in the quantizers' privacy ranking, or in the trade-off between security and downstream performance.


\begin{figure*}
    \centering
    \includegraphics[width=0.85\linewidth]{Figures/mol_figs/barplot_dperfs_vs_rdelta}
    \caption{
        Evolution of the privacy of each downstream model $\qcertif$ along with relative performances of the quantized models compared to the original for each pretrained embedder.
        As the privacy of the model decreases, the performances of the quantized model increase, showing the trade-off between security and downstream performance.
    }
    \label{fig:barplot_dperfs_vs_rdelta}
\end{figure*}






\subsection{Details on the evaluation of the quantizers' privacy}
\label{subsec:eval_privacy}

Our hypothesis in the estimation of $\qcertif$  is that the quantized weights with the lowest average loss dominate the maximum value of $\lambda_k = \Lambda_{k,k} = \liminf{n}\pp{\mkn{2}/\mkn{k}}^2(\sigma_k^n)^2$.
We show in~\autoref{fig:k_max} the evolution of $\Lambda_{k,k}$ with the k, where the indexes are sorted with decreasing values of average loss, and the histogram of the index of the maximum value of $\Lambda_{k,k}$ for each quantizer, on 4 different datasets (trained on 500 epochs, hence $k \leq 500$).
The maximum value of $\Lambda_{k,k}$ is indeed consistently reached on low $k$ values, which seems to confirm our hypothesis, validating our sampling strategy for the estimation of $\qcertif$.


\begin{figure}
    \centering
    \begin{subfigure}{0.67\textwidth}
        \centering
        \includegraphics[width=\linewidth]{Figures/mol_figs/hERG/k_max}
    \end{subfigure}
    \begin{subfigure}{0.67\textwidth}
        \centering
        \includegraphics[width=\linewidth]{Figures/mol_figs/AMES/k_max}
    \end{subfigure}
    \begin{subfigure}{0.67\textwidth}
        \centering
        \includegraphics[width=\linewidth]{Figures/mol_figs/Carcinogens_Lagunin/k_max}
    \end{subfigure}
    \begin{subfigure}{0.67\textwidth}
        \centering
        \includegraphics[width=\linewidth]{Figures/mol_figs/Skin__Reaction/k_max}
    \end{subfigure}
    \caption{
        Evolution of $\Lambda_{k,k}$ with the k, where the indexes are sorted with decreasing values of average loss, and the histogram of the index of the maximum value of $\Lambda_{k,k}$ for each quantizer.
    }
    \label{fig:k_max}
\end{figure}


\FloatBarrier

% \section{You \emph{can} have an appendix here.}

% You can have as much text here as you want. The main body must be at most $8$ pages long.
% For the final version, one more page can be added.
% If you want, you can use an appendix like this one.  

% The $\mathtt{\backslash onecolumn}$ command above can be kept in place if you prefer a one-column appendix, or can be removed if you prefer a two-column appendix.  Apart from this possible change, the style (font size, spacing, margins, page numbering, etc.) should be kept the same as the main body.
%%%%%%%%%%%%%%%%%%%%%%%%%%%%%%%%%%%%%%%%%%%%%%%%%%%%%%%%%%%%%%%%%%%%%%%%%%%%%%%
%%%%%%%%%%%%%%%%%%%%%%%%%%%%%%%%%%%%%%%%%%%%%%%%%%%%%%%%%%%%%%%%%%%%%%%%%%%%%%%


\end{document}


% This document was modified from the file originally made available by
% Pat Langley and Andrea Danyluk for ICML-2K. This version was created
% by Iain Murray in 2018, and modified by Alexandre Bouchard in
% 2019 and 2021 and by Csaba Szepesvari, Gang Niu and Sivan Sabato in 2022.
% Modified again in 2023 and 2024 by Sivan Sabato and Jonathan Scarlett.
% Previous contributors include Dan Roy, Lise Getoor and Tobias
% Scheffer, which was slightly modified from the 2010 version by
% Thorsten Joachims & Johannes Fuernkranz, slightly modified from the
% 2009 version by Kiri Wagstaff and Sam Roweis's 2008 version, which is
% slightly modified from Prasad Tadepalli's 2007 version which is a
% lightly changed version of the previous year's version by Andrew
% Moore, which was in turn edited from those of Kristian Kersting and
% Codrina Lauth. Alex Smola contributed to the algorithmic style files.
