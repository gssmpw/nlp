\newpage

\section{Synthetic experiments}
\label{app:synthetic_details}

\begin{wrapfigure}[28]{r}{0.53\linewidth}
   % \vspace{-4cm}
    \centering
    \includegraphics[width=\linewidth]{Figures/synth_figs/barplot_dperfs_vs_rdelta}
    \caption{
        Evolution of $\qcertif$ with the quantized model's performance as the ratio of the original model's training accuracy (line) for various quantizers in our synthetic setup.
    }
    \label{fig:barplot_synth}
\end{wrapfigure}

\subsection{Trade-off between privacy and performance}

In this section, we provide additional details on the synthetic experiments conducted to evaluate the trade-off between privacy and performance of quantized models.
This trade-off is illustrated in~\autoref{fig:barplot_synth}, where we plot the evolution of the quantification certificate $\qcertif$ with the model's performance as the ratio of the original model's training accuracy.
We find that the trade-off between privacy and performance is less pronounced compared to the real-world experiments.
In particular, while the least private quantizers do preserve most of the original performance, the more private quantizers seem to achieve similar performances on some data distribution (in particular, when $\sigma = 3$).
This could be explained by the low performances of the trained models on such distribtuions as illustrated in~\autoref{tab:synth_perf}.
Furthermore, simple mixture of Gaussians might not be relevant to capture the complexity of real-world data distributions, and we therefore decided to focus our analysis of the performance-privacy trade-off on real-world applications.

\begin{table}
    \centering
    \caption{
        Description of the quantizers used in the synthetic experiments.
    }
    \label{tab:synth_perf}
    \resizebox{\linewidth}{!}{\subsection{Classification and Labeling}

To mitigate potential subjective biases during the labeling phase, we allocated each of the \numbug merged bug-fix pull requests to two co-authors, who are also contributors and developers of both Apollo and Autoware open-source projects and possess enough background in the ADS domain. 
We employed the open-coding strategy of intercoder reliability~\cite{intercoder_reliability} to help strengthen the labeling process.
Our methodology necessitated that each co-author independently assess the bug, which involved meticulous examination of the source code, commit logs, code reviews, pull request information, and associated issue descriptions to recognize the labeling items.

In our work, we commenced with the root cause and symptom taxonomies in a previous ADS bug study~\cite{GarciaF0AXC20} and also taxonomy of generic \bfps focusing on the syntactic level~\cite{PanKW09,SotoTWGL16,CamposM17,IslamZ20}
as a foundation for ADS bug analysis. 
The taxonomy of root causes was subsequently augmented by employing an open-coding paradigm, thereby broadening the spectrum. 
For the pull request that eluded classification within the foundational taxonomy, each co-author designated a label for it. Post-labeling, co-authors collaboratively reconciled any discrepancies in their classifications. 
We added two root causes, \textbf{Syntax, Naming, and Typography (SNT)} that involves errors in the basic structure of the code, including syntactical mistakes, naming conventions, and typographical errors, and \textbf{Dependency Issues (DEP)} related to importing, versioning, and managing dependencies.

For semantic \bfps ~\yuntianyihl{and \bfas}, none of the previous work could provide a useful taxonomy due to the domain-specific nature of ADS. In this research, the open coding process was employed to refine and identify distinct semantic \bfps~\yuntianyihl{and \bfas}.
A preliminary investigation was conducted involving two co-authors who independently examined the bug fixes to establish a tentative classification framework. Each rater recommended a series of categories, which were later amalgamated and refined during a face-to-face session attended by all contributing authors. This meeting served as a platform to validate and integrate the classification schemes proposed by the individual raters. This reconciliation process led to updates in the classification scheme. 
\yuntianyihl{We validated the final classification by consulting with researchers in the ADS domain and developers from Apollo and Autoware. Their expertise helped refine the classification scheme, and we relabeled the affected pull requests accordingly.}
The classification result is presented in \autoref{sec:taxonomy}.
Our findings indicated that a single bug origin may result in multiple symptoms and \bfps. Therefore, certain bugs were cataloged into multiple categories, unlike the previous study~\cite{GarciaF0AXC20} only considered a single symptom for each bug.
In the process of labeling, two co-authors, possessing expertise in ADS, were engaged to categorize the \bfps according to the specified schema. 
The degree of concordance between these raters was quantified using Cohen's Kappa coefficient~\cite{VieiraKS10}, which was used by a recent \bfp study~\cite{IslamPNR20}. 
In instances of labeling discrepancies, periodic discussions were conducted to achieve reconciliation. 
Throughout this process, the Kappa score persistently exceeded 80\%, indicative of a robust understanding and unanimous agreement among the raters~\cite{mchugh2012interrater}.
}
\end{table}


%\FloatBarrier

\subsection{Stability and Computational Complexity}
\label{ssec:stability}

\begin{figure}[h]
        \begin{subfigure}{0.48\linewidth}
        \centering
        \includegraphics[width=\linewidth]{Figures/synth_figs/bsl_corr_rank}
        \caption{
            Baseline MIS
        }
        \label{fig:bsl_corr_rank}
    \end{subfigure}
    \begin{subfigure}{0.48\linewidth}
        \centering
        \includegraphics[width=\linewidth]{Figures/synth_figs/rdelta_corr_rank}
        \caption{
            $\qcertif$
        }
        \label{fig:rdelta_corr_rank}
    \end{subfigure}
    \caption{
        Correlation between the rankings obtained with the baseline MIS method (resp. $\qcertif$) at a given number of run, with the ranking obtained with the baseline MIS method (resp. $\qcertif$) at 300 runs.
    }
    \label{fig:rank_corr}
\end{figure}

As explained in~\autoref{ssec:baseline_estimation}, the baseline approach consists in training a discriminator to distinguish between samples from the training set of a given $\thetan$ and samples from the product distribution $P_{\thetan}\otimes P$.
Similarly the $\qcertif$-based approach relies on the traing of multiple models to average the values of $\qcertif$ obtained.

The computational overhead induced by the $\qcertif$-based approach, namely computing the validation loss of the quantized models, is negligible compared to the total training time (1s against 4m).
Similarly, the training of the discriminator takes only about 40m.

As a result, training multiple models $\thetan$ over multiple runs is the computational bottleneck of our privacy evaluations.
To properly evaluate the time required to obtain both rankings, one would have to answer the following question: 'How many runs do i need to launch to ensure the ranking I obtained is stable?'

\autoref{fig:rank_corr} shows how after $15$ runs, the rankings obtained with $\qcertif$ are already highly correlated with the rankings obtained with 300 runs, while the rankings obtained with the baseline MIS method require $150$ runs to reach the same level of correlation.
As a result, the time required to obtain stable rankings with $\qcertif$ is significantly lower (\(\approx 1\)h) than with the baseline MIS method (\(\approx 10\)h).


\subsection{Visualization of the datasets}

We provide in~\autoref{fig:synth_datasets} a visualization of the synthetic datasets used in the experiments, through a PCA projection in dimension 2.
This visualization helps understand how different data distribution might result in different empirical results, as some datasets are more challenging than others, such as the dataset with $n_{\textrm{cluster}} = 6$ and $\sigma = 1.5$, for whom the labels of the datapoints are easily separable, while $n_{\textrm{cluster}} = 16$ and $\sigma = 3$ provides a more challenging dataset, with overlapping clusters.

\begin{figure}[ht]
    \centering
    %%% 6
    \begin{subfigure}{0.3\linewidth}
        \centering
        \includegraphics[width=\linewidth]{Figures/synth_figs/datasets/6-1.5}
        \caption{
            $n_{\textrm{cluster}} = 6, \sigma = 1.5$
        }
        \label{fig:6-1.5}
    \end{subfigure}
    \begin{subfigure}{0.3\linewidth}
        \centering
        \includegraphics[width=\linewidth]{Figures/synth_figs/datasets/6-2}
        \caption{
            $n_{\textrm{cluster}} = 6, \sigma = 2$
        }
        \label{fig:6-2}
    \end{subfigure}
    \begin{subfigure}{0.3\linewidth}
        \centering
        \includegraphics[width=\linewidth]{Figures/synth_figs/datasets/6-3}
        \caption{
            $n_{\textrm{cluster}} = 6, \sigma = 3$
        }
        \label{fig:6-3}
    \end{subfigure}
    %%% 8
        \begin{subfigure}{0.3\linewidth}
        \centering
        \includegraphics[width=\linewidth]{Figures/synth_figs/datasets/8-1.5}
        \caption{
            $n_{\textrm{cluster}} = 8, \sigma = 1.5$
        }
        \label{fig:8-1.5}
    \end{subfigure}
    \begin{subfigure}{0.3\linewidth}
        \centering
        \includegraphics[width=\linewidth]{Figures/synth_figs/datasets/8-2}
        \caption{
            $n_{\textrm{cluster}} = 8, \sigma = 2$
        }
        \label{fig:8-2}
    \end{subfigure}
    \begin{subfigure}{0.3\linewidth}
        \centering
        \includegraphics[width=\linewidth]{Figures/synth_figs/datasets/8-3}
        \caption{
            $n_{\textrm{cluster}} = 8, \sigma = 3$
        }
        \label{fig:8-3}
    \end{subfigure}
    %%% 16
    \begin{subfigure}{0.3\linewidth}
        \centering
        \includegraphics[width=\linewidth]{Figures/synth_figs/datasets/16-1.5}
        \caption{
            $n_{\textrm{cluster}} = 16, \sigma = 1.5$
        }
        \label{fig:16-1.5}
    \end{subfigure}
    \begin{subfigure}{0.3\linewidth}
        \centering
        \includegraphics[width=\linewidth]{Figures/synth_figs/datasets/16-2}
        \caption{
            $n_{\textrm{cluster}} = 16, \sigma = 2$
        }
        \label{fig:16-2}
    \end{subfigure}
    \begin{subfigure}{0.3\linewidth}
        \centering
        \includegraphics[width=\linewidth]{Figures/synth_figs/datasets/16-3}
        \caption{
            $n_{\textrm{cluster}} = 16, \sigma = 3$
        }
        \label{fig:16-3}
    \end{subfigure}
    \caption{
        Visualization of the synthetic datasets used in the experiments, through a PCA projection in dimension 2 (the original space is $\mathbb{R}^{128}$).
    }
    \label{fig:synth_datasets}
\end{figure}


\FloatBarrier