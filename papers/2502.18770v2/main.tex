% This must be in the first 5 lines to tell arXiv to use pdfLaTeX, which is strongly recommended.
\pdfoutput=1
\documentclass[11pt]{article}
% \usepackage[review]{acl}
\usepackage[preprint]{acl}
% \usepackage{acl}

\usepackage{enumitem}
\usepackage{booktabs,arydshln}
\usepackage[normalem]{ulem}
\usepackage{array}
\usepackage{hyperref}
\usepackage{url}
\usepackage{multirow}       % for multiple rows
\usepackage{mdwlist}        % for compact itemize
\usepackage{colortbl}
\usepackage{amssymb}
\usepackage{amsmath}
\usepackage{amsthm}
\usepackage{graphicx}
\usepackage{makecell}
\usepackage{threeparttable}
\usepackage{subcaption}
\usepackage{lscape}
\usepackage{fancyhdr}
\renewcommand{\UrlFont}{\ttfamily\small}
\usepackage{wrapfig}
\usepackage{algpseudocode}
\usepackage{enumitem}
\usepackage{xcolor}	
\usepackage{algorithm}
\usepackage{mathtools}
\usepackage{times}
\usepackage{latexsym}
\usepackage{algorithmicx}
\usepackage{algpseudocode} 
\usepackage{extarrows}
\usepackage{xspace}
%tables
\usepackage{longtable}
\usepackage{tikz}
\usepackage{listings} % For formatting code blocks
\usepackage{fancyvrb} % For enhanced display
\usepackage{tcolorbox} % For creating full-page boxes
\usepackage{lipsum} % For placeholder text (optional)

\DeclareMathOperator*{\argmax}{arg\,max}
\DeclareMathOperator*{\argmin}{arg\,min}

\newcommand*\diff{\mathop{}\!\mathrm{d}}

\makeatletter
\def\adl@drawiv#1#2#3{%
        \hskip.5\tabcolsep
        \xleaders#3{#2.5\@tempdimb #1{1}#2.5\@tempdimb}%
                #2\z@ plus1fil minus1fil\relax
        \hskip.5\tabcolsep}
\newcommand{\cdashlinelr}[1]{%
  \noalign{\vskip\aboverulesep
           \global\let\@dashdrawstore\adl@draw
           \global\let\adl@draw\adl@drawiv}
  \cdashline{#1}
  \noalign{\global\let\adl@draw\@dashdrawstore
           \vskip\belowrulesep}}
\makeatother

\newcolumntype{s}{>{\columncolor{blue!3}}r}
\newcolumntype{d}{>{\columncolor{yellow!4}}r}
\newcolumntype{q}{>{\columncolor{red!3}}r}

%figures
\usepackage{tabularx}
\usepackage{pgfplots}
\pgfplotsset{compat=newest}
\usepgfplotslibrary{groupplots,colorbrewer,statistics}
%colors
\usepackage{tcolorbox}
\usepackage{subcaption}

\usepackage{colortbl}
\definecolor{Azure}{RGB}{240,255,255}  
\definecolor{MistyRose}{RGB}{255,228,225} 
\definecolor{Ivory}{RGB}{255,255,240} 
\definecolor{Bisque}{RGB}{255,228,196}
\definecolor{Thistle}{RGB}{255,225,255}
\definecolor{orange}{RGB}{255,165,0}

\newcommand{\xuandong}[1]{{\color{red} [xuandong: #1]}}

\usepackage{url}

\renewcommand{\algorithmicrequire}{\textbf{Input:}}  % Use Input in the format of Algorithm  
\renewcommand{\algorithmicensure}{\textbf{Output:}} % Use Output in the format of Algorithm
\usepackage[T1]{fontenc}
\usepackage[utf8]{inputenc}
\usepackage{microtype}
\usepackage{inconsolata}


\title{Reward Shaping to Mitigate Reward Hacking in RLHF}

\author{
Jiayi Fu\textsuperscript{\rm 1,2}\thanks{Equal contribution.}, 
Xuandong Zhao\textsuperscript{\rm 3}\footnotemark[1], 
Chengyuan Yao\textsuperscript{\rm 2}, 
Heng Wang\textsuperscript{\rm 2}, 
Qi Han\textsuperscript{\rm 2}, 
Yanghua Xiao\textsuperscript{\rm 1}\thanks{Corresponding author.} \\
\textsuperscript{\rm 1}Fudan University \quad
\textsuperscript{\rm 2}StepFun \quad
\textsuperscript{\rm 3}UC Berkeley \\
\texttt{fujy22@m.fudan.edu.cn} \\
\texttt{xuandongzhao@berkeley.edu} \\ 
\texttt{shawyh@fudan.edu.cn}
}

\begin{document}
\maketitle
\begin{abstract}
\begin{abstract}
We present an image blending pipeline, \textit{IBURD}, that creates realistic synthetic images to assist in the training of deep detectors for use on underwater autonomous vehicles (AUVs) for marine debris detection tasks. 
Specifically, IBURD generates both images of underwater debris and their pixel-level annotations, using source images of debris objects, their annotations, and target background images of marine environments. 
With Poisson editing and style transfer techniques, IBURD is even able to robustly blend transparent objects into arbitrary backgrounds and automatically adjust the style of blended images using the blurriness metric of target background images. 
These generated images of marine debris in actual underwater backgrounds address the data scarcity and data variety problems faced by deep-learned vision algorithms in challenging underwater conditions, and can enable the use of AUVs for environmental cleanup missions. 
Both quantitative and robotic evaluations of IBURD demonstrate the efficacy of the proposed approach for robotic detection of marine debris. 
\end{abstract}



\end{abstract}

\section{Introduction}
\label{section:intro}
\section{Introduction}

In today’s rapidly evolving digital landscape, the transformative power of web technologies has redefined not only how services are delivered but also how complex tasks are approached. Web-based systems have become increasingly prevalent in risk control across various domains. This widespread adoption is due their accessibility, scalability, and ability to remotely connect various types of users. For example, these systems are used for process safety management in industry~\cite{kannan2016web}, safety risk early warning in urban construction~\cite{ding2013development}, and safe monitoring of infrastructural systems~\cite{repetto2018web}. Within these web-based risk management systems, the source search problem presents a huge challenge. Source search refers to the task of identifying the origin of a risky event, such as a gas leak and the emission point of toxic substances. This source search capability is crucial for effective risk management and decision-making.

Traditional approaches to implementing source search capabilities into the web systems often rely on solely algorithmic solutions~\cite{ristic2016study}. These methods, while relatively straightforward to implement, often struggle to achieve acceptable performances due to algorithmic local optima and complex unknown environments~\cite{zhao2020searching}. More recently, web crowdsourcing has emerged as a promising alternative for tackling the source search problem by incorporating human efforts in these web systems on-the-fly~\cite{zhao2024user}. This approach outsources the task of addressing issues encountered during the source search process to human workers, leveraging their capabilities to enhance system performance.

These solutions often employ a human-AI collaborative way~\cite{zhao2023leveraging} where algorithms handle exploration-exploitation and report the encountered problems while human workers resolve complex decision-making bottlenecks to help the algorithms getting rid of local deadlocks~\cite{zhao2022crowd}. Although effective, this paradigm suffers from two inherent limitations: increased operational costs from continuous human intervention, and slow response times of human workers due to sequential decision-making. These challenges motivate our investigation into developing autonomous systems that preserve human-like reasoning capabilities while reducing dependency on massive crowdsourced labor.

Furthermore, recent advancements in large language models (LLMs)~\cite{chang2024survey} and multi-modal LLMs (MLLMs)~\cite{huang2023chatgpt} have unveiled promising avenues for addressing these challenges. One clear opportunity involves the seamless integration of visual understanding and linguistic reasoning for robust decision-making in search tasks. However, whether large models-assisted source search is really effective and efficient for improving the current source search algorithms~\cite{ji2022source} remains unknown. \textit{To address the research gap, we are particularly interested in answering the following two research questions in this work:}

\textbf{\textit{RQ1: }}How can source search capabilities be integrated into web-based systems to support decision-making in time-sensitive risk management scenarios? 
% \sq{I mention ``time-sensitive'' here because I feel like we shall say something about the response time -- LLM has to be faster than humans}

\textbf{\textit{RQ2: }}How can MLLMs and LLMs enhance the effectiveness and efficiency of existing source search algorithms? 

% \textit{\textbf{RQ2:}} To what extent does the performance of large models-assisted search align with or approach the effectiveness of human-AI collaborative search? 

To answer the research questions, we propose a novel framework called Auto-\
S$^2$earch (\textbf{Auto}nomous \textbf{S}ource \textbf{Search}) and implement a prototype system that leverages advanced web technologies to simulate real-world conditions for zero-shot source search. Unlike traditional methods that rely on pre-defined heuristics or extensive human intervention, AutoS$^2$earch employs a carefully designed prompt that encapsulates human rationales, thereby guiding the MLLM to generate coherent and accurate scene descriptions from visual inputs about four directional choices. Based on these language-based descriptions, the LLM is enabled to determine the optimal directional choice through chain-of-thought (CoT) reasoning. Comprehensive empirical validation demonstrates that AutoS$^2$-\ 
earch achieves a success rate of 95–98\%, closely approaching the performance of human-AI collaborative search across 20 benchmark scenarios~\cite{zhao2023leveraging}. 

Our work indicates that the role of humans in future web crowdsourcing tasks may evolve from executors to validators or supervisors. Furthermore, incorporating explanations of LLM decisions into web-based system interfaces has the potential to help humans enhance task performance in risk control.







\section{Related Work}
\label{section:related}
% Reward hacking is a well-known issue in reinforcement learning, affecting both traditional RL and RLHF in LLMs~\cite{weng2024rewardhack}.
\subsection{Reward Hacking in Traditional RL}  
Reward hacking arises when an RL agent exploits flaws or ambiguities in the reward function to achieve high rewards without performing the intended task~\cite{weng2024rewardhack}. This aligns with Goodhart’s Law: \emph{When a measure becomes a target, it ceases to be a good measure.} For example: 
A bicycle agent rewarded for not falling and moving toward a goal (but not penalized for moving away) learns to circle the goal indefinitely~\cite{Randlv1998LearningTD}.  
A walking agent in the DMControl suite, rewarded for matching a target speed, learns to walk unnaturally using only one leg~\cite{lee2021pebblefeedbackefficientinteractivereinforcement}.  
An RL agent allowed to modify its body grows excessively long legs to fall forward and reach the goal~\cite{Ha2018designrl}.  
In the Elevator Action ALE game, the agent repeatedly kills the first enemy on the first floor to accumulate small rewards~\cite{toromanoff2019deepreinforcementlearningreally}.  
% A robot trained to stay on track learns to reverse along straight paths by alternating left and right turns instead of following curves~\cite{Vamplew2004}.

\citet{amodei2016concrete} propose several potential mitigation strategies to address reward hacking, including
\emph{(1) Adversarial Reward Functions}: Treating the reward function as an adaptive agent capable of responding to new strategies where the model achieves high rewards but receives low human ratings.
\emph{(2) Model Lookahead}: Assigning rewards based on anticipated future states; for example, penalizing the agent with negative rewards if it attempts to modify the reward function~\cite{everitt2016selfmodificationpolicyutilityfunction}.
\emph{(3) Adversarial Blinding}: Restricting the model’s access to specific variables to prevent it from learning information that could facilitate reward hacking~\cite{ajakan2015domainadversarialneuralnetworks}.
\emph{(4) Careful Engineering}: Designing systems to avoid certain types of reward hacking by isolating the agent’s actions from its reward signals, such as through sandboxing techniques~\cite{The_AGI_Containment_Problem}.
\emph{(5) Trip Wires}: Deliberately introducing vulnerabilities into the system and setting up monitoring mechanisms to detect and alert when reward hacking occurs.

\subsection{Reward Hacking in RLHF of LLMs}  
Reward hacking in RLHF for large language models has been extensively studied. \citet{gao2023scaling} systematically investigate the scaling laws of reward hacking in small models, while \citet{wen2024languagemodelslearnmislead} demonstrate that language models can learn to mislead humans through RLHF. Beyond exploiting the training process, reward hacking can also target evaluators. Although using LLMs as judges is a natural choice given their increasing capabilities, this approach is imperfect and can introduce biases. For instance, LLMs may favor their own responses when evaluating outputs from different model families~\cite{liu2024llmsnarcissisticevaluatorsego} or exhibit positional bias when assessing responses in sequence~\cite{wang2023largelanguagemodelsfair}.  

To mitigate reward hacking, several methods have been proposed. Reward ensemble techniques have shown promise in addressing this issue~\cite{Eisenstein2023HelpingOH, Rame2024WARMOT, ahmed2024scalableensemblingmitigatingreward, coste2023reward, zhang2024improvingreinforcementlearninghuman}, and shaping methods have also proven straightforward and effective~\cite{yang2024regularizinghiddenstatesenables, jinnai2024regularizedbestofnsamplingmitigate}. \citet{miao2024informmitigatingrewardhacking} introduce an information bottleneck to filter irrelevant noise, while \citet{moskovitz2023confrontingrewardmodeloveroptimization} employ constrained RLHF to prevent reward over-optimization. \citet{Chen2024ODINDR} propose the ODIN method, which uses a linear layer to separately output quality and length rewards, reducing their correlation through an orthogonal loss function. Similarly,
~\citet{sun2023salmon} train instructable reward models to give a more comprehensive reward signal from multiple objectives. \citet{Dai2023SafeRS} constrain reward magnitudes using regularization terms. ~\citet{liu2024rrmrobustrewardmodel} curate diverse pairwise training data. Additionally, post-processing techniques have been explored, such as the log-sigmoid centering transformation introduced by \citet{Wang2024TransformingAC}.  



\section{Method}
\label{section:method}
\section{Preliminaries}
\label{Preliminaries}
\begin{figure*}[t]
    \centering
    \includegraphics[width=0.95\linewidth]{fig/HealthGPT_Framework.png}
    \caption{The \ourmethod{} architecture integrates hierarchical visual perception and H-LoRA, employing a task-specific hard router to select visual features and H-LoRA plugins, ultimately generating outputs with an autoregressive manner.}
    \label{fig:architecture}
\end{figure*}
\noindent\textbf{Large Vision-Language Models.} 
The input to a LVLM typically consists of an image $x^{\text{img}}$ and a discrete text sequence $x^{\text{txt}}$. The visual encoder $\mathcal{E}^{\text{img}}$ converts the input image $x^{\text{img}}$ into a sequence of visual tokens $\mathcal{V} = [v_i]_{i=1}^{N_v}$, while the text sequence $x^{\text{txt}}$ is mapped into a sequence of text tokens $\mathcal{T} = [t_i]_{i=1}^{N_t}$ using an embedding function $\mathcal{E}^{\text{txt}}$. The LLM $\mathcal{M_\text{LLM}}(\cdot|\theta)$ models the joint probability of the token sequence $\mathcal{U} = \{\mathcal{V},\mathcal{T}\}$, which is expressed as:
\begin{equation}
    P_\theta(R | \mathcal{U}) = \prod_{i=1}^{N_r} P_\theta(r_i | \{\mathcal{U}, r_{<i}\}),
\end{equation}
where $R = [r_i]_{i=1}^{N_r}$ is the text response sequence. The LVLM iteratively generates the next token $r_i$ based on $r_{<i}$. The optimization objective is to minimize the cross-entropy loss of the response $\mathcal{R}$.
% \begin{equation}
%     \mathcal{L}_{\text{VLM}} = \mathbb{E}_{R|\mathcal{U}}\left[-\log P_\theta(R | \mathcal{U})\right]
% \end{equation}
It is worth noting that most LVLMs adopt a design paradigm based on ViT, alignment adapters, and pre-trained LLMs\cite{liu2023llava,liu2024improved}, enabling quick adaptation to downstream tasks.


\noindent\textbf{VQGAN.}
VQGAN~\cite{esser2021taming} employs latent space compression and indexing mechanisms to effectively learn a complete discrete representation of images. VQGAN first maps the input image $x^{\text{img}}$ to a latent representation $z = \mathcal{E}(x)$ through a encoder $\mathcal{E}$. Then, the latent representation is quantized using a codebook $\mathcal{Z} = \{z_k\}_{k=1}^K$, generating a discrete index sequence $\mathcal{I} = [i_m]_{m=1}^N$, where $i_m \in \mathcal{Z}$ represents the quantized code index:
\begin{equation}
    \mathcal{I} = \text{Quantize}(z|\mathcal{Z}) = \arg\min_{z_k \in \mathcal{Z}} \| z - z_k \|_2.
\end{equation}
In our approach, the discrete index sequence $\mathcal{I}$ serves as a supervisory signal for the generation task, enabling the model to predict the index sequence $\hat{\mathcal{I}}$ from input conditions such as text or other modality signals.  
Finally, the predicted index sequence $\hat{\mathcal{I}}$ is upsampled by the VQGAN decoder $G$, generating the high-quality image $\hat{x}^\text{img} = G(\hat{\mathcal{I}})$.



\noindent\textbf{Low Rank Adaptation.} 
LoRA\cite{hu2021lora} effectively captures the characteristics of downstream tasks by introducing low-rank adapters. The core idea is to decompose the bypass weight matrix $\Delta W\in\mathbb{R}^{d^{\text{in}} \times d^{\text{out}}}$ into two low-rank matrices $ \{A \in \mathbb{R}^{d^{\text{in}} \times r}, B \in \mathbb{R}^{r \times d^{\text{out}}} \}$, where $ r \ll \min\{d^{\text{in}}, d^{\text{out}}\} $, significantly reducing learnable parameters. The output with the LoRA adapter for the input $x$ is then given by:
\begin{equation}
    h = x W_0 + \alpha x \Delta W/r = x W_0 + \alpha xAB/r,
\end{equation}
where matrix $ A $ is initialized with a Gaussian distribution, while the matrix $ B $ is initialized as a zero matrix. The scaling factor $ \alpha/r $ controls the impact of $ \Delta W $ on the model.

\section{HealthGPT}
\label{Method}


\subsection{Unified Autoregressive Generation.}  
% As shown in Figure~\ref{fig:architecture}, 
\ourmethod{} (Figure~\ref{fig:architecture}) utilizes a discrete token representation that covers both text and visual outputs, unifying visual comprehension and generation as an autoregressive task. 
For comprehension, $\mathcal{M}_\text{llm}$ receives the input joint sequence $\mathcal{U}$ and outputs a series of text token $\mathcal{R} = [r_1, r_2, \dots, r_{N_r}]$, where $r_i \in \mathcal{V}_{\text{txt}}$, and $\mathcal{V}_{\text{txt}}$ represents the LLM's vocabulary:
\begin{equation}
    P_\theta(\mathcal{R} \mid \mathcal{U}) = \prod_{i=1}^{N_r} P_\theta(r_i \mid \mathcal{U}, r_{<i}).
\end{equation}
For generation, $\mathcal{M}_\text{llm}$ first receives a special start token $\langle \text{START\_IMG} \rangle$, then generates a series of tokens corresponding to the VQGAN indices $\mathcal{I} = [i_1, i_2, \dots, i_{N_i}]$, where $i_j \in \mathcal{V}_{\text{vq}}$, and $\mathcal{V}_{\text{vq}}$ represents the index range of VQGAN. Upon completion of generation, the LLM outputs an end token $\langle \text{END\_IMG} \rangle$:
\begin{equation}
    P_\theta(\mathcal{I} \mid \mathcal{U}) = \prod_{j=1}^{N_i} P_\theta(i_j \mid \mathcal{U}, i_{<j}).
\end{equation}
Finally, the generated index sequence $\mathcal{I}$ is fed into the decoder $G$, which reconstructs the target image $\hat{x}^{\text{img}} = G(\mathcal{I})$.

\subsection{Hierarchical Visual Perception}  
Given the differences in visual perception between comprehension and generation tasks—where the former focuses on abstract semantics and the latter emphasizes complete semantics—we employ ViT to compress the image into discrete visual tokens at multiple hierarchical levels.
Specifically, the image is converted into a series of features $\{f_1, f_2, \dots, f_L\}$ as it passes through $L$ ViT blocks.

To address the needs of various tasks, the hidden states are divided into two types: (i) \textit{Concrete-grained features} $\mathcal{F}^{\text{Con}} = \{f_1, f_2, \dots, f_k\}, k < L$, derived from the shallower layers of ViT, containing sufficient global features, suitable for generation tasks; 
(ii) \textit{Abstract-grained features} $\mathcal{F}^{\text{Abs}} = \{f_{k+1}, f_{k+2}, \dots, f_L\}$, derived from the deeper layers of ViT, which contain abstract semantic information closer to the text space, suitable for comprehension tasks.

The task type $T$ (comprehension or generation) determines which set of features is selected as the input for the downstream large language model:
\begin{equation}
    \mathcal{F}^{\text{img}}_T =
    \begin{cases}
        \mathcal{F}^{\text{Con}}, & \text{if } T = \text{generation task} \\
        \mathcal{F}^{\text{Abs}}, & \text{if } T = \text{comprehension task}
    \end{cases}
\end{equation}
We integrate the image features $\mathcal{F}^{\text{img}}_T$ and text features $\mathcal{T}$ into a joint sequence through simple concatenation, which is then fed into the LLM $\mathcal{M}_{\text{llm}}$ for autoregressive generation.
% :
% \begin{equation}
%     \mathcal{R} = \mathcal{M}_{\text{llm}}(\mathcal{U}|\theta), \quad \mathcal{U} = [\mathcal{F}^{\text{img}}_T; \mathcal{T}]
% \end{equation}
\subsection{Heterogeneous Knowledge Adaptation}
We devise H-LoRA, which stores heterogeneous knowledge from comprehension and generation tasks in separate modules and dynamically routes to extract task-relevant knowledge from these modules. 
At the task level, for each task type $ T $, we dynamically assign a dedicated H-LoRA submodule $ \theta^T $, which is expressed as:
\begin{equation}
    \mathcal{R} = \mathcal{M}_\text{LLM}(\mathcal{U}|\theta, \theta^T), \quad \theta^T = \{A^T, B^T, \mathcal{R}^T_\text{outer}\}.
\end{equation}
At the feature level for a single task, H-LoRA integrates the idea of Mixture of Experts (MoE)~\cite{masoudnia2014mixture} and designs an efficient matrix merging and routing weight allocation mechanism, thus avoiding the significant computational delay introduced by matrix splitting in existing MoELoRA~\cite{luo2024moelora}. Specifically, we first merge the low-rank matrices (rank = r) of $ k $ LoRA experts into a unified matrix:
\begin{equation}
    \mathbf{A}^{\text{merged}}, \mathbf{B}^{\text{merged}} = \text{Concat}(\{A_i\}_1^k), \text{Concat}(\{B_i\}_1^k),
\end{equation}
where $ \mathbf{A}^{\text{merged}} \in \mathbb{R}^{d^\text{in} \times rk} $ and $ \mathbf{B}^{\text{merged}} \in \mathbb{R}^{rk \times d^\text{out}} $. The $k$-dimension routing layer generates expert weights $ \mathcal{W} \in \mathbb{R}^{\text{token\_num} \times k} $ based on the input hidden state $ x $, and these are expanded to $ \mathbb{R}^{\text{token\_num} \times rk} $ as follows:
\begin{equation}
    \mathcal{W}^\text{expanded} = \alpha k \mathcal{W} / r \otimes \mathbf{1}_r,
\end{equation}
where $ \otimes $ denotes the replication operation.
The overall output of H-LoRA is computed as:
\begin{equation}
    \mathcal{O}^\text{H-LoRA} = (x \mathbf{A}^{\text{merged}} \odot \mathcal{W}^\text{expanded}) \mathbf{B}^{\text{merged}},
\end{equation}
where $ \odot $ represents element-wise multiplication. Finally, the output of H-LoRA is added to the frozen pre-trained weights to produce the final output:
\begin{equation}
    \mathcal{O} = x W_0 + \mathcal{O}^\text{H-LoRA}.
\end{equation}
% In summary, H-LoRA is a task-based dynamic PEFT method that achieves high efficiency in single-task fine-tuning.

\subsection{Training Pipeline}

\begin{figure}[t]
    \centering
    \hspace{-4mm}
    \includegraphics[width=0.94\linewidth]{fig/data.pdf}
    \caption{Data statistics of \texttt{VL-Health}. }
    \label{fig:data}
\end{figure}
\noindent \textbf{1st Stage: Multi-modal Alignment.} 
In the first stage, we design separate visual adapters and H-LoRA submodules for medical unified tasks. For the medical comprehension task, we train abstract-grained visual adapters using high-quality image-text pairs to align visual embeddings with textual embeddings, thereby enabling the model to accurately describe medical visual content. During this process, the pre-trained LLM and its corresponding H-LoRA submodules remain frozen. In contrast, the medical generation task requires training concrete-grained adapters and H-LoRA submodules while keeping the LLM frozen. Meanwhile, we extend the textual vocabulary to include multimodal tokens, enabling the support of additional VQGAN vector quantization indices. The model trains on image-VQ pairs, endowing the pre-trained LLM with the capability for image reconstruction. This design ensures pixel-level consistency of pre- and post-LVLM. The processes establish the initial alignment between the LLM’s outputs and the visual inputs.

\noindent \textbf{2nd Stage: Heterogeneous H-LoRA Plugin Adaptation.}  
The submodules of H-LoRA share the word embedding layer and output head but may encounter issues such as bias and scale inconsistencies during training across different tasks. To ensure that the multiple H-LoRA plugins seamlessly interface with the LLMs and form a unified base, we fine-tune the word embedding layer and output head using a small amount of mixed data to maintain consistency in the model weights. Specifically, during this stage, all H-LoRA submodules for different tasks are kept frozen, with only the word embedding layer and output head being optimized. Through this stage, the model accumulates foundational knowledge for unified tasks by adapting H-LoRA plugins.

\begin{table*}[!t]
\centering
\caption{Comparison of \ourmethod{} with other LVLMs and unified multi-modal models on medical visual comprehension tasks. \textbf{Bold} and \underline{underlined} text indicates the best performance and second-best performance, respectively.}
\resizebox{\textwidth}{!}{
\begin{tabular}{c|lcc|cccccccc|c}
\toprule
\rowcolor[HTML]{E9F3FE} &  &  &  & \multicolumn{2}{c}{\textbf{VQA-RAD \textuparrow}} & \multicolumn{2}{c}{\textbf{SLAKE \textuparrow}} & \multicolumn{2}{c}{\textbf{PathVQA \textuparrow}} &  &  &  \\ 
\cline{5-10}
\rowcolor[HTML]{E9F3FE}\multirow{-2}{*}{\textbf{Type}} & \multirow{-2}{*}{\textbf{Model}} & \multirow{-2}{*}{\textbf{\# Params}} & \multirow{-2}{*}{\makecell{\textbf{Medical} \\ \textbf{LVLM}}} & \textbf{close} & \textbf{all} & \textbf{close} & \textbf{all} & \textbf{close} & \textbf{all} & \multirow{-2}{*}{\makecell{\textbf{MMMU} \\ \textbf{-Med}}\textuparrow} & \multirow{-2}{*}{\textbf{OMVQA}\textuparrow} & \multirow{-2}{*}{\textbf{Avg. \textuparrow}} \\ 
\midrule \midrule
\multirow{9}{*}{\textbf{Comp. Only}} 
& Med-Flamingo & 8.3B & \Large \ding{51} & 58.6 & 43.0 & 47.0 & 25.5 & 61.9 & 31.3 & 28.7 & 34.9 & 41.4 \\
& LLaVA-Med & 7B & \Large \ding{51} & 60.2 & 48.1 & 58.4 & 44.8 & 62.3 & 35.7 & 30.0 & 41.3 & 47.6 \\
& HuatuoGPT-Vision & 7B & \Large \ding{51} & 66.9 & 53.0 & 59.8 & 49.1 & 52.9 & 32.0 & 42.0 & 50.0 & 50.7 \\
& BLIP-2 & 6.7B & \Large \ding{55} & 43.4 & 36.8 & 41.6 & 35.3 & 48.5 & 28.8 & 27.3 & 26.9 & 36.1 \\
& LLaVA-v1.5 & 7B & \Large \ding{55} & 51.8 & 42.8 & 37.1 & 37.7 & 53.5 & 31.4 & 32.7 & 44.7 & 41.5 \\
& InstructBLIP & 7B & \Large \ding{55} & 61.0 & 44.8 & 66.8 & 43.3 & 56.0 & 32.3 & 25.3 & 29.0 & 44.8 \\
& Yi-VL & 6B & \Large \ding{55} & 52.6 & 42.1 & 52.4 & 38.4 & 54.9 & 30.9 & 38.0 & 50.2 & 44.9 \\
& InternVL2 & 8B & \Large \ding{55} & 64.9 & 49.0 & 66.6 & 50.1 & 60.0 & 31.9 & \underline{43.3} & 54.5 & 52.5\\
& Llama-3.2 & 11B & \Large \ding{55} & 68.9 & 45.5 & 72.4 & 52.1 & 62.8 & 33.6 & 39.3 & 63.2 & 54.7 \\
\midrule
\multirow{5}{*}{\textbf{Comp. \& Gen.}} 
& Show-o & 1.3B & \Large \ding{55} & 50.6 & 33.9 & 31.5 & 17.9 & 52.9 & 28.2 & 22.7 & 45.7 & 42.6 \\
& Unified-IO 2 & 7B & \Large \ding{55} & 46.2 & 32.6 & 35.9 & 21.9 & 52.5 & 27.0 & 25.3 & 33.0 & 33.8 \\
& Janus & 1.3B & \Large \ding{55} & 70.9 & 52.8 & 34.7 & 26.9 & 51.9 & 27.9 & 30.0 & 26.8 & 33.5 \\
& \cellcolor[HTML]{DAE0FB}HealthGPT-M3 & \cellcolor[HTML]{DAE0FB}3.8B & \cellcolor[HTML]{DAE0FB}\Large \ding{51} & \cellcolor[HTML]{DAE0FB}\underline{73.7} & \cellcolor[HTML]{DAE0FB}\underline{55.9} & \cellcolor[HTML]{DAE0FB}\underline{74.6} & \cellcolor[HTML]{DAE0FB}\underline{56.4} & \cellcolor[HTML]{DAE0FB}\underline{78.7} & \cellcolor[HTML]{DAE0FB}\underline{39.7} & \cellcolor[HTML]{DAE0FB}\underline{43.3} & \cellcolor[HTML]{DAE0FB}\underline{68.5} & \cellcolor[HTML]{DAE0FB}\underline{61.3} \\
& \cellcolor[HTML]{DAE0FB}HealthGPT-L14 & \cellcolor[HTML]{DAE0FB}14B & \cellcolor[HTML]{DAE0FB}\Large \ding{51} & \cellcolor[HTML]{DAE0FB}\textbf{77.7} & \cellcolor[HTML]{DAE0FB}\textbf{58.3} & \cellcolor[HTML]{DAE0FB}\textbf{76.4} & \cellcolor[HTML]{DAE0FB}\textbf{64.5} & \cellcolor[HTML]{DAE0FB}\textbf{85.9} & \cellcolor[HTML]{DAE0FB}\textbf{44.4} & \cellcolor[HTML]{DAE0FB}\textbf{49.2} & \cellcolor[HTML]{DAE0FB}\textbf{74.4} & \cellcolor[HTML]{DAE0FB}\textbf{66.4} \\
\bottomrule
\end{tabular}
}
\label{tab:results}
\end{table*}
\begin{table*}[ht]
    \centering
    \caption{The experimental results for the four modality conversion tasks.}
    \resizebox{\textwidth}{!}{
    \begin{tabular}{l|ccc|ccc|ccc|ccc}
        \toprule
        \rowcolor[HTML]{E9F3FE} & \multicolumn{3}{c}{\textbf{CT to MRI (Brain)}} & \multicolumn{3}{c}{\textbf{CT to MRI (Pelvis)}} & \multicolumn{3}{c}{\textbf{MRI to CT (Brain)}} & \multicolumn{3}{c}{\textbf{MRI to CT (Pelvis)}} \\
        \cline{2-13}
        \rowcolor[HTML]{E9F3FE}\multirow{-2}{*}{\textbf{Model}}& \textbf{SSIM $\uparrow$} & \textbf{PSNR $\uparrow$} & \textbf{MSE $\downarrow$} & \textbf{SSIM $\uparrow$} & \textbf{PSNR $\uparrow$} & \textbf{MSE $\downarrow$} & \textbf{SSIM $\uparrow$} & \textbf{PSNR $\uparrow$} & \textbf{MSE $\downarrow$} & \textbf{SSIM $\uparrow$} & \textbf{PSNR $\uparrow$} & \textbf{MSE $\downarrow$} \\
        \midrule \midrule
        pix2pix & 71.09 & 32.65 & 36.85 & 59.17 & 31.02 & 51.91 & 78.79 & 33.85 & 28.33 & 72.31 & 32.98 & 36.19 \\
        CycleGAN & 54.76 & 32.23 & 40.56 & 54.54 & 30.77 & 55.00 & 63.75 & 31.02 & 52.78 & 50.54 & 29.89 & 67.78 \\
        BBDM & {71.69} & {32.91} & {34.44} & 57.37 & 31.37 & 48.06 & \textbf{86.40} & 34.12 & 26.61 & {79.26} & 33.15 & 33.60 \\
        Vmanba & 69.54 & 32.67 & 36.42 & {63.01} & {31.47} & {46.99} & 79.63 & 34.12 & 26.49 & 77.45 & 33.53 & 31.85 \\
        DiffMa & 71.47 & 32.74 & 35.77 & 62.56 & 31.43 & 47.38 & 79.00 & {34.13} & {26.45} & 78.53 & {33.68} & {30.51} \\
        \rowcolor[HTML]{DAE0FB}HealthGPT-M3 & \underline{79.38} & \underline{33.03} & \underline{33.48} & \underline{71.81} & \underline{31.83} & \underline{43.45} & {85.06} & \textbf{34.40} & \textbf{25.49} & \underline{84.23} & \textbf{34.29} & \textbf{27.99} \\
        \rowcolor[HTML]{DAE0FB}HealthGPT-L14 & \textbf{79.73} & \textbf{33.10} & \textbf{32.96} & \textbf{71.92} & \textbf{31.87} & \textbf{43.09} & \underline{85.31} & \underline{34.29} & \underline{26.20} & \textbf{84.96} & \underline{34.14} & \underline{28.13} \\
        \bottomrule
    \end{tabular}
    }
    \label{tab:conversion}
\end{table*}

\noindent \textbf{3rd Stage: Visual Instruction Fine-Tuning.}  
In the third stage, we introduce additional task-specific data to further optimize the model and enhance its adaptability to downstream tasks such as medical visual comprehension (e.g., medical QA, medical dialogues, and report generation) or generation tasks (e.g., super-resolution, denoising, and modality conversion). Notably, by this stage, the word embedding layer and output head have been fine-tuned, only the H-LoRA modules and adapter modules need to be trained. This strategy significantly improves the model's adaptability and flexibility across different tasks.



\section{Experiment}
\label{section:experiment}
\section{Experiment}
\label{s:experiment}

\subsection{Data Description}
We evaluate our method on FI~\cite{you2016building}, Twitter\_LDL~\cite{yang2017learning} and Artphoto~\cite{machajdik2010affective}.
FI is a public dataset built from Flickr and Instagram, with 23,308 images and eight emotion categories, namely \textit{amusement}, \textit{anger}, \textit{awe},  \textit{contentment}, \textit{disgust}, \textit{excitement},  \textit{fear}, and \textit{sadness}. 
% Since images in FI are all copyrighted by law, some images are corrupted now, so we remove these samples and retain 21,828 images.
% T4SA contains images from Twitter, which are classified into three categories: \textit{positive}, \textit{neutral}, and \textit{negative}. In this paper, we adopt the base version of B-T4SA, which contains 470,586 images and provides text descriptions of the corresponding tweets.
Twitter\_LDL contains 10,045 images from Twitter, with the same eight categories as the FI dataset.
% 。
For these two datasets, they are randomly split into 80\%
training and 20\% testing set.
Artphoto contains 806 artistic photos from the DeviantArt website, which we use to further evaluate the zero-shot capability of our model.
% on the small-scale dataset.
% We construct and publicly release the first image sentiment analysis dataset containing metadata.
% 。

% Based on these datasets, we are the first to construct and publicly release metadata-enhanced image sentiment analysis datasets. These datasets include scenes, tags, descriptions, and corresponding confidence scores, and are available at this link for future research purposes.


% 
\begin{table}[t]
\centering
% \begin{center}
\caption{Overall performance of different models on FI and Twitter\_LDL datasets.}
\label{tab:cap1}
% \resizebox{\linewidth}{!}
{
\begin{tabular}{l|c|c|c|c}
\hline
\multirow{2}{*}{\textbf{Model}} & \multicolumn{2}{c|}{\textbf{FI}}  & \multicolumn{2}{c}{\textbf{Twitter\_LDL}} \\ \cline{2-5} 
  & \textbf{Accuracy} & \textbf{F1} & \textbf{Accuracy} & \textbf{F1}  \\ \hline
% (\rownumber)~AlexNet~\cite{krizhevsky2017imagenet}  & 58.13\% & 56.35\%  & 56.24\%& 55.02\%  \\ 
% (\rownumber)~VGG16~\cite{simonyan2014very}  & 63.75\%& 63.08\%  & 59.34\%& 59.02\%  \\ 
(\rownumber)~ResNet101~\cite{he2016deep} & 66.16\%& 65.56\%  & 62.02\% & 61.34\%  \\ 
(\rownumber)~CDA~\cite{han2023boosting} & 66.71\%& 65.37\%  & 64.14\% & 62.85\%  \\ 
(\rownumber)~CECCN~\cite{ruan2024color} & 67.96\%& 66.74\%  & 64.59\%& 64.72\% \\ 
(\rownumber)~EmoVIT~\cite{xie2024emovit} & 68.09\%& 67.45\%  & 63.12\% & 61.97\%  \\ 
(\rownumber)~ComLDL~\cite{zhang2022compound} & 68.83\%& 67.28\%  & 65.29\% & 63.12\%  \\ 
(\rownumber)~WSDEN~\cite{li2023weakly} & 69.78\%& 69.61\%  & 67.04\% & 65.49\% \\ 
(\rownumber)~ECWA~\cite{deng2021emotion} & 70.87\%& 69.08\%  & 67.81\% & 66.87\%  \\ 
(\rownumber)~EECon~\cite{yang2023exploiting} & 71.13\%& 68.34\%  & 64.27\%& 63.16\%  \\ 
(\rownumber)~MAM~\cite{zhang2024affective} & 71.44\%  & 70.83\% & 67.18\%  & 65.01\%\\ 
(\rownumber)~TGCA-PVT~\cite{chen2024tgca}   & 73.05\%  & 71.46\% & 69.87\%  & 68.32\% \\ 
(\rownumber)~OEAN~\cite{zhang2024object}   & 73.40\%  & 72.63\% & 70.52\%  & 69.47\% \\ \hline
(\rownumber)~\shortname  & \textbf{79.48\%} & \textbf{79.22\%} & \textbf{74.12\%} & \textbf{73.09\%} \\ \hline
\end{tabular}
}
\vspace{-6mm}
% \end{center}
\end{table}
% 

\subsection{Experiment Setting}
% \subsubsection{Model Setting.}
% 
\textbf{Model Setting:}
For feature representation, we set $k=10$ to select object tags, and adopt clip-vit-base-patch32 as the pre-trained model for unified feature representation.
Moreover, we empirically set $(d_e, d_h, d_k, d_s) = (512, 128, 16, 64)$, and set the classification class $L$ to 8.

% 

\textbf{Training Setting:}
To initialize the model, we set all weights such as $\boldsymbol{W}$ following the truncated normal distribution, and use AdamW optimizer with the learning rate of $1 \times 10^{-4}$.
% warmup scheduler of cosine, warmup steps of 2000.
Furthermore, we set the batch size to 32 and the epoch of the training process to 200.
During the implementation, we utilize \textit{PyTorch} to build our entire model.
% , and our project codes are publicly available at https://github.com/zzmyrep/MESN.
% Our project codes as well as data are all publicly available on GitHub\footnote{https://github.com/zzmyrep/KBCEN}.
% Code is available at \href{https://github.com/zzmyrep/KBCEN}{https://github.com/zzmyrep/KBCEN}.

\textbf{Evaluation Metrics:}
Following~\cite{zhang2024affective, chen2024tgca, zhang2024object}, we adopt \textit{accuracy} and \textit{F1} as our evaluation metrics to measure the performance of different methods for image sentiment analysis. 



\subsection{Experiment Result}
% We compare our model against the following baselines: AlexNet~\cite{krizhevsky2017imagenet}, VGG16~\cite{simonyan2014very}, ResNet101~\cite{he2016deep}, CECCN~\cite{ruan2024color}, EmoVIT~\cite{xie2024emovit}, WSCNet~\cite{yang2018weakly}, ECWA~\cite{deng2021emotion}, EECon~\cite{yang2023exploiting}, MAM~\cite{zhang2024affective} and TGCA-PVT~\cite{chen2024tgca}, and the overall results are summarized in Table~\ref{tab:cap1}.
We compare our model against several baselines, and the overall results are summarized in Table~\ref{tab:cap1}.
We observe that our model achieves the best performance in both accuracy and F1 metrics, significantly outperforming the previous models. 
This superior performance is mainly attributed to our effective utilization of metadata to enhance image sentiment analysis, as well as the exceptional capability of the unified sentiment transformer framework we developed. These results strongly demonstrate that our proposed method can bring encouraging performance for image sentiment analysis.

\setcounter{magicrownumbers}{0} 
\begin{table}[t]
\begin{center}
\caption{Ablation study of~\shortname~on FI dataset.} 
% \vspace{1mm}
\label{tab:cap2}
\resizebox{.9\linewidth}{!}
{
\begin{tabular}{lcc}
  \hline
  \textbf{Model} & \textbf{Accuracy} & \textbf{F1} \\
  \hline
  (\rownumber)~Ours (w/o vision) & 65.72\% & 64.54\% \\
  (\rownumber)~Ours (w/o text description) & 74.05\% & 72.58\% \\
  (\rownumber)~Ours (w/o object tag) & 77.45\% & 76.84\% \\
  (\rownumber)~Ours (w/o scene tag) & 78.47\% & 78.21\% \\
  \hline
  (\rownumber)~Ours (w/o unified embedding) & 76.41\% & 76.23\% \\
  (\rownumber)~Ours (w/o adaptive learning) & 76.83\% & 76.56\% \\
  (\rownumber)~Ours (w/o cross-modal fusion) & 76.85\% & 76.49\% \\
  \hline
  (\rownumber)~Ours  & \textbf{79.48\%} & \textbf{79.22\%} \\
  \hline
\end{tabular}
}
\end{center}
\vspace{-5mm}
\end{table}


\begin{figure}[t]
\centering
% \vspace{-2mm}
\includegraphics[width=0.42\textwidth]{fig/2dvisual-linux4-paper2.pdf}
\caption{Visualization of feature distribution on eight categories before (left) and after (right) model processing.}
% 
\label{fig:visualization}
\vspace{-5mm}
\end{figure}

\subsection{Ablation Performance}
In this subsection, we conduct an ablation study to examine which component is really important for performance improvement. The results are reported in Table~\ref{tab:cap2}.

For information utilization, we observe a significant decline in model performance when visual features are removed. Additionally, the performance of \shortname~decreases when different metadata are removed separately, which means that text description, object tag, and scene tag are all critical for image sentiment analysis.
Recalling the model architecture, we separately remove transformer layers of the unified representation module, the adaptive learning module, and the cross-modal fusion module, replacing them with MLPs of the same parameter scale.
In this way, we can observe varying degrees of decline in model performance, indicating that these modules are indispensable for our model to achieve better performance.

\subsection{Visualization}
% 


% % 开始使用minipage进行左右排列
% \begin{minipage}[t]{0.45\textwidth}  % 子图1宽度为45%
%     \centering
%     \includegraphics[width=\textwidth]{2dvisual.pdf}  % 插入图片
%     \captionof{figure}{Visualization of feature distribution.}  % 使用captionof添加图片标题
%     \label{fig:visualization}
% \end{minipage}


% \begin{figure}[t]
% \centering
% \vspace{-2mm}
% \includegraphics[width=0.45\textwidth]{fig/2dvisual.pdf}
% \caption{Visualization of feature distribution.}
% \label{fig:visualization}
% % \vspace{-4mm}
% \end{figure}

% \begin{figure}[t]
% \centering
% \vspace{-2mm}
% \includegraphics[width=0.45\textwidth]{fig/2dvisual-linux3-paper.pdf}
% \caption{Visualization of feature distribution.}
% \label{fig:visualization}
% % \vspace{-4mm}
% \end{figure}



\begin{figure}[tbp]   
\vspace{-4mm}
  \centering            
  \subfloat[Depth of adaptive learning layers]   
  {
    \label{fig:subfig1}\includegraphics[width=0.22\textwidth]{fig/fig_sensitivity-a5}
  }
  \subfloat[Depth of fusion layers]
  {
    % \label{fig:subfig2}\includegraphics[width=0.22\textwidth]{fig/fig_sensitivity-b2}
    \label{fig:subfig2}\includegraphics[width=0.22\textwidth]{fig/fig_sensitivity-b2-num.pdf}
  }
  \caption{Sensitivity study of \shortname~on different depth. }   
  \label{fig:fig_sensitivity}  
\vspace{-2mm}
\end{figure}

% \begin{figure}[htbp]
% \centerline{\includegraphics{2dvisual.pdf}}
% \caption{Visualization of feature distribution.}
% \label{fig:visualization}
% \end{figure}

% In Fig.~\ref{fig:visualization}, we use t-SNE~\cite{van2008visualizing} to reduce the dimension of data features for visualization, Figure in left represents the metadata features before model processing, the features are obtained by embedding through the CLIP model, and figure in right shows the features of the data after model processing, it can be observed that after the model processing, the data with different label categories fall in different regions in the space, therefore, we can conclude that the Therefore, we can conclude that the model can effectively utilize the information contained in the metadata and use it to guide the model for classification.

In Fig.~\ref{fig:visualization}, we use t-SNE~\cite{van2008visualizing} to reduce the dimension of data features for visualization.
The left figure shows metadata features before being processed by our model (\textit{i.e.}, embedded by CLIP), while the right shows the distribution of features after being processed by our model.
We can observe that after the model processing, data with the same label are closer to each other, while others are farther away.
Therefore, it shows that the model can effectively utilize the information contained in the metadata and use it to guide the classification process.

\subsection{Sensitivity Analysis}
% 
In this subsection, we conduct a sensitivity analysis to figure out the effect of different depth settings of adaptive learning layers and fusion layers. 
% In this subsection, we conduct a sensitivity analysis to figure out the effect of different depth settings on the model. 
% Fig.~\ref{fig:fig_sensitivity} presents the effect of different depth settings of adaptive learning layers and fusion layers. 
Taking Fig.~\ref{fig:fig_sensitivity} (a) as an example, the model performance improves with increasing depth, reaching the best performance at a depth of 4.
% Taking Fig.~\ref{fig:fig_sensitivity} (a) as an example, the performance of \shortname~improves with the increase of depth at first, reaching the best performance at a depth of 4.
When the depth continues to increase, the accuracy decreases to varying degrees.
Similar results can be observed in Fig.~\ref{fig:fig_sensitivity} (b).
Therefore, we set their depths to 4 and 6 respectively to achieve the best results.

% Through our experiments, we can observe that the effect of modifying these hyperparameters on the results of the experiments is very weak, and the surface model is not sensitive to the hyperparameters.


\subsection{Zero-shot Capability}
% 

% (1)~GCH~\cite{2010Analyzing} & 21.78\% & (5)~RA-DLNet~\cite{2020A} & 34.01\% \\ \hline
% (2)~WSCNet~\cite{2019WSCNet}  & 30.25\% & (6)~CECCN~\cite{ruan2024color} & 43.83\% \\ \hline
% (3)~PCNN~\cite{2015Robust} & 31.68\%  & (7)~EmoVIT~\cite{xie2024emovit} & 44.90\% \\ \hline
% (4)~AR~\cite{2018Visual} & 32.67\% & (8)~Ours (Zero-shot) & 47.83\% \\ \hline


\begin{table}[t]
\centering
\caption{Zero-shot capability of \shortname.}
\label{tab:cap3}
\resizebox{1\linewidth}{!}
{
\begin{tabular}{lc|lc}
\hline
\textbf{Model} & \textbf{Accuracy} & \textbf{Model} & \textbf{Accuracy} \\ \hline
(1)~WSCNet~\cite{2019WSCNet}  & 30.25\% & (5)~MAM~\cite{zhang2024affective} & 39.56\%  \\ \hline
(2)~AR~\cite{2018Visual} & 32.67\% & (6)~CECCN~\cite{ruan2024color} & 43.83\% \\ \hline
(3)~RA-DLNet~\cite{2020A} & 34.01\%  & (7)~EmoVIT~\cite{xie2024emovit} & 44.90\% \\ \hline
(4)~CDA~\cite{han2023boosting} & 38.64\% & (8)~Ours (Zero-shot) & 47.83\% \\ \hline
\end{tabular}
}
\vspace{-5mm}
\end{table}

% We use the model trained on the FI dataset to test on the artphoto dataset to verify the model's generalization ability as well as robustness to other distributed datasets.
% We can observe that the MESN model shows strong competitiveness in terms of accuracy when compared to other trained models, which suggests that the model has a good generalization ability in the OOD task.

To validate the model's generalization ability and robustness to other distributed datasets, we directly test the model trained on the FI dataset, without training on Artphoto. 
% As observed in Table 3, compared to other models trained on Artphoto, we achieve highly competitive zero-shot performance, indicating that the model has good generalization ability in out-of-distribution tasks.
From Table~\ref{tab:cap3}, we can observe that compared with other models trained on Artphoto, we achieve competitive zero-shot performance, which shows that the model has good generalization ability in out-of-distribution tasks.



\section{Discussion}
\label{section:Discussion}
\section{Discussion \& Conclusion}\label{sec:discussion}

\fauxsection{Auditing with hard prompts.}
Attacks such as greedy coordinate gradient~\cite{zou2023gcg} optimize the attack prompt in the \emph{hard} token space instead of the soft token space.
Hence, they are weaker at eliciting completions.
On one hand, this might make them more suitable for auditing unlearning.
On the other hand, due to their computational requirements, they are often used to force only the beginning of a harmful completion (e.g. \textit{Sure, here's how to build a bomb...}) with the hope that the LLM follows.
It is unclear whether this would be sufficient to produce specific unlearned passages.
We see it as an interesting direction for future work.

\fauxsection{Unlearning vs jail-breaking.}
Our findings are applicable to the jail-breaking community as well.
Prior work~\cite{zhang2024safe} hinted that unlearning and preventing harmful outputs can be viewed as the same task -- removing or suppressing particular information.
\sta{s} and fine-tuning attacks~\cite{hu2024jogging} are useful tools for evaluating LLMs in powerful threat models.
It was shown that fine-tuning on benign data, or data unrelated to the unlearning records (for jail-breaking and unlearning respectively) can restore undesirable behavior~\cite{lucki2024adversarial}.

\fauxsection{Variation in gradient-based learning.}
Prior work showed that removing training records from the training set,
and repeating the training can result in the same final model~\cite{thudi2022auditunlearning} depending on the random seed.
Even though a record was part of the training run, its influence might be minimal, making unlearning unnecessary.
Similarly, it was shown that SGD has intrinsic privacy guarantees, assuming there exists a group of similar records~\cite{hyland2022empiricalsgdprivacy}.
Thus, algorithmic auditing of unlearning might not be possible, and one would have to rely on verified or attested procedures instead~\cite{eisenhofer2023verifiedunlearning}, regardless of their impact on the model.

\fauxsection{Distinguishing learned soft tokens.}
Even though, in most our results, the number of soft tokens required to elicit a completion is the same,
we attempted to distinguish between them.
To this end, we take all single-token \sta{s} optimized for \tofu (Table~\ref{tab:attack-unlearning-tofu}) and assign a label $y=\{0, 1\}$: $y=0$ for $f_\emptyset$, and $y=1$ for $f_{ft}$ and the unlearned models.
We then train a binary classifier using $f_\emptyset$ and $f_{ft}$.
While we are able to overfit it and distinguish between $f_\emptyset$ and $f_{ft}$,
we were not able to train a model that would generalize to the unlearned models, and decisively assign a class. 
Our approach is similar to Dataset Inference~\cite{maini2021di,maini2024dillms} which showed there can be distributional differences between the models, depending on the data they were trained on.
Further investigation into \emph{what} soft tokens are learned during the audit is an interesting direction for future work.

\section{Conclusion}\label{sec:conclusion}

In this work, we show that soft token attacks (\sta{s}) cannot reliably distinguish between base, fine-tuned, and unlearned models.
In all cases, the auditor can elicit all unlearned information by appending optimized soft prompts to the base prompt.
Additionally, we show that \sta with a single soft token can elicit $150$ random characters, and over $400$ with soft tokens.

Our work demonstrates that machine unlearning in LLMs needs better evaluation frameworks.
While many unlearning methods can be broken by simple paraphrasing of original prompts, or by fine-tuning on partial unlearned data or even \emph{unrelated data}, 
\sta misrepresents their efficacy.

\section{Limitations \& ethical considerations}\label{sec:limitations}

\fauxsection{Limitations.}
Due to computational constraints our work is limited to 7-8 billion parameter models.
Nevertheless, given that LLMs' expressive power increases with size~\cite{kaplan2020scalinglaws}, our results should hold for larger LLMs.
Our evaluation with random strings could be extended to verify if there is a clear and generalizable dependency between the number of soft tokens and the maximum number of generated characters.

\fauxsection{Ethical considerations.}
In this work, we show that an auditor (a user) with white-box access to the model, and sufficient compute can elicit any text from the LLM.
While it does require knowing the target completion for a given prompt, it is likely that partial completions might be enough, thus allowing the user to elicit harmful information.
This may be particularly dangerous in settings where the user has approximate knowledge of the information that had been scrubbed off the LLM.

\section{Conclusion}
\label{section:conclusion}
\section{Conclusion}
In this paper, we propose ChineseEcomQA, a scalable question-answering benchmark designed to rigorously assess LLMs on fundamental e-commerce concepts. ChineseEcomQA is characterized by three core features: Focus on Fundamental Concept, E-Commerce Generalizability, and Domain-Specific Expertise, which collectively enable systematic evaluation of LLMs' e-commerce knowledge. Leveraging ChineseEcomQA, we conduct extensive evaluations on mainstream LLMs, yielding critical insights into their capabilities and limitations. Our findings not only highlight performance disparities across models but also delineate actionable directions for advancing LLM applications in the e-commerce domain.

\section*{Limitations}
\label{section:limitation}
Although our PAR method effectively mitigates reward hacking, it does not improve peak performance, as measured by the winrate of the best checkpoint. Furthermore, its design principles lack precision. While PAR sets the upper bound of the RL reward to 1.0, alternative bounds and their selection criteria remain unexplored. Additionally, the dynamics of reward adjustment—such as the initial rate of increase and the pace of convergence—are not fully elucidated. 
% \newpage

\section*{Ethical Considerations}
\label{section:ethical}
Our research addresses the ethical challenges of reward hacking in RLHF by proposing a method to mitigate this problem. By ensuring robust alignment with human values, enhancing transparency in reward design, and proactively addressing biases and safety risks, our approach aims to develop RLHF systems that are fair, reliable, and aligned with societal well-being.
% Bibliography entries for the entire Anthology, followed by custom entries
%\bibliography{anthology,custom}
% Custom bibliography entries only
\bibliography{custom}

\appendix
% !TeX root = main.tex 


\newcommand{\lnote}{\textcolor[rgb]{1,0,0}{Lydia: }\textcolor[rgb]{0,0,1}}
\newcommand{\todo}{\textcolor[rgb]{1,0,0.5}{To do: }\textcolor[rgb]{0.5,0,1}}


\newcommand{\state}{S}
\newcommand{\meas}{M}
\newcommand{\out}{\mathrm{out}}
\newcommand{\piv}{\mathrm{piv}}
\newcommand{\pivotal}{\mathrm{pivotal}}
\newcommand{\isnot}{\mathrm{not}}
\newcommand{\pred}{^\mathrm{predict}}
\newcommand{\act}{^\mathrm{act}}
\newcommand{\pre}{^\mathrm{pre}}
\newcommand{\post}{^\mathrm{post}}
\newcommand{\calM}{\mathcal{M}}

\newcommand{\game}{\mathbf{V}}
\newcommand{\strategyspace}{S}
\newcommand{\payoff}[1]{V^{#1}}
\newcommand{\eff}[1]{E^{#1}}
\newcommand{\p}{\vect{p}}
\newcommand{\simplex}[1]{\Delta^{#1}}

\newcommand{\recdec}[1]{\bar{D}(\hat{Y}_{#1})}





\newcommand{\sphereone}{\calS^1}
\newcommand{\samplen}{S^n}
\newcommand{\wA}{w}%{w_{\mathfrak{a}}}
\newcommand{\Awa}{A_{\wA}}
\newcommand{\Ytil}{\widetilde{Y}}
\newcommand{\Xtil}{\widetilde{X}}
\newcommand{\wst}{w_*}
\newcommand{\wls}{\widehat{w}_{\mathrm{LS}}}
\newcommand{\dec}{^\mathrm{dec}}
\newcommand{\sub}{^\mathrm{sub}}

\newcommand{\calP}{\mathcal{P}}
\newcommand{\totspace}{\calZ}
\newcommand{\clspace}{\calX}
\newcommand{\attspace}{\calA}

\newcommand{\Ftil}{\widetilde{\calF}}

\newcommand{\totx}{Z}
\newcommand{\classx}{X}
\newcommand{\attx}{A}
\newcommand{\calL}{\mathcal{L}}



\newcommand{\defeq}{\mathrel{\mathop:}=}
\newcommand{\vect}[1]{\ensuremath{\mathbf{#1}}}
\newcommand{\mat}[1]{\ensuremath{\mathbf{#1}}}
\newcommand{\dd}{\mathrm{d}}
\newcommand{\grad}{\nabla}
\newcommand{\hess}{\nabla^2}
\newcommand{\argmin}{\mathop{\rm argmin}}
\newcommand{\argmax}{\mathop{\rm argmax}}
\newcommand{\Ind}[1]{\mathbf{1}\{#1\}}

\newcommand{\norm}[1]{\left\|{#1}\right\|}
\newcommand{\fnorm}[1]{\|{#1}\|_{\text{F}}}
\newcommand{\spnorm}[2]{\left\| {#1} \right\|_{\text{S}({#2})}}
\newcommand{\sigmin}{\sigma_{\min}}
\newcommand{\tr}{\text{tr}}
\renewcommand{\det}{\text{det}}
\newcommand{\rank}{\text{rank}}
\newcommand{\logdet}{\text{logdet}}
\newcommand{\trans}{^{\top}}
\newcommand{\poly}{\text{poly}}
\newcommand{\polylog}{\text{polylog}}
\newcommand{\st}{\text{s.t.~}}
\newcommand{\proj}{\mathcal{P}}
\newcommand{\projII}{\mathcal{P}_{\parallel}}
\newcommand{\projT}{\mathcal{P}_{\perp}}
\newcommand{\projX}{\mathcal{P}_{\mathcal{X}^\star}}
\newcommand{\inner}[1]{\langle #1 \rangle}

\renewcommand{\Pr}{\mathbb{P}}
\newcommand{\Z}{\mathbb{Z}}
\newcommand{\N}{\mathbb{N}}
\newcommand{\R}{\mathbb{R}}
\newcommand{\E}{\mathbb{E}}
\newcommand{\F}{\mathcal{F}}
\newcommand{\var}{\mathrm{var}}
\newcommand{\cov}{\mathrm{cov}}


\newcommand{\calN}{\mathcal{N}}

\newcommand{\jccomment}{\textcolor[rgb]{1,0,0}{C: }\textcolor[rgb]{1,0,1}}
\newcommand{\fracpar}[2]{\frac{\partial #1}{\partial  #2}}

\newcommand{\A}{\mathcal{A}}
\newcommand{\B}{\mat{B}}
%\newcommand{\C}{\mat{C}}

\newcommand{\I}{\mat{I}}
\newcommand{\M}{\mat{M}}
\newcommand{\D}{\mat{D}}
%\newcommand{\U}{\mat{U}}
\newcommand{\V}{\mat{V}}
\newcommand{\W}{\mat{W}}
\newcommand{\X}{\mat{X}}
\newcommand{\Y}{\mat{Y}}
\newcommand{\mSigma}{\mat{\Sigma}}
\newcommand{\mLambda}{\mat{\Lambda}}
\newcommand{\e}{\vect{e}}
\newcommand{\g}{\vect{g}}
\renewcommand{\u}{\vect{u}}
\newcommand{\w}{\vect{w}}
\newcommand{\x}{\vect{x}}
\newcommand{\y}{\vect{y}}
\newcommand{\z}{\vect{z}}
\newcommand{\fI}{\mathfrak{I}}
\newcommand{\fS}{\mathfrak{S}}
\newcommand{\fE}{\mathfrak{E}}
\newcommand{\fF}{\mathfrak{F}}

\newcommand{\Risk}{\mathcal{R}}

\renewcommand{\L}{\mathcal{L}}
\renewcommand{\H}{\mathcal{H}}

\newcommand{\cn}{\kappa}
\newcommand{\nn}{\nonumber}


\newcommand{\Hess}{\nabla^2}
\newcommand{\tlO}{\tilde{O}}
\newcommand{\tlOmega}{\tilde{\Omega}}

\newcommand{\calF}{\mathcal{F}}
\newcommand{\fhat}{\widehat{f}}
\newcommand{\calS}{\mathcal{S}}

\newcommand{\calX}{\mathcal{X}}
\newcommand{\calY}{\mathcal{Y}}
\newcommand{\calD}{\mathcal{D}}
\newcommand{\calZ}{\mathcal{Z}}
\newcommand{\calA}{\mathcal{A}}
\newcommand{\fbayes}{f^B}
\newcommand{\func}{f^U}


\newcommand{\bayscore}{\text{calibrated Bayes score}}
\newcommand{\bayrisk}{\text{calibrated Bayes risk}}

\newtheorem{example}{Example}[section]
\newtheorem{exc}{Exercise}[section]
%\newtheorem{rem}{Remark}[section]

\newtheorem{theorem}{Theorem}[section]
\newtheorem{definition}{Definition}
\newtheorem{proposition}[theorem]{Proposition}
\newtheorem{corollary}[theorem]{Corollary}

\newtheorem{remark}{Remark}[section]
\newtheorem{lemma}[theorem]{Lemma}
\newtheorem{claim}[theorem]{Claim}
\newtheorem{fact}[theorem]{Fact}
\newtheorem{assumption}{Assumption}

\newcommand{\iidsim}{\overset{\mathrm{i.i.d.}}{\sim}}
\newcommand{\unifsim}{\overset{\mathrm{unif}}{\sim}}
\newcommand{\sign}{\mathrm{sign}}
\newcommand{\wbar}{\overline{w}}
\newcommand{\what}{\widehat{w}}
\newcommand{\KL}{\mathrm{KL}}
\newcommand{\Bern}{\mathrm{Bernoulli}}
\newcommand{\ihat}{\widehat{i}}
\newcommand{\Dwst}{\calD^{w_*}}
\newcommand{\fls}{\widehat{f}_{n}}


\newcommand{\brpi}{\pi^{br}}
\newcommand{\brtheta}{\theta^{br}}

% \newcommand{\M}{\mat{M}}
% \newcommand\Mmh{\mat{M}^{-1/2}}
% \newcommand{\A}{\mat{A}}
% \newcommand{\B}{\mat{B}}
% \newcommand{\C}{\mat{C}}
% \newcommand{\Et}[1][t]{\mat{E_{#1}}}
% \newcommand{\Etp}{\Et[t+1]}
% \newcommand{\Errt}[1][t]{\mat{\bigtriangleup_{#1}}}
% \newcommand\cnM{\kappa}
% \newcommand{\cn}[1]{\kappa\left(#1\right)}
% \newcommand\X{\mat{X}}
% \newcommand\fstar{f_*}
% \newcommand\Xt[1][t]{\mat{X_{#1}}}
% \newcommand\ut[1][t]{{u_{#1}}}
% \newcommand\Xtinv{\inv{\Xt}}
% \newcommand\Xtp{\mat{X_{t+1}}}
% \newcommand\Xtpinv{\inv{\left(\mat{X_{t+1}}\right)}}
% \newcommand\U{\mat{U}}
% \newcommand\UTr{\trans{\mat{U}}}
% \newcommand{\Ut}[1][t]{\mat{U_{#1}}}
% \newcommand{\Utinv}{\inv{\Ut}}
% \newcommand{\UtTr}[1][t]{\trans{\mat{U_{#1}}}}
% \newcommand\Utp{\mat{U_{t+1}}}
% \newcommand\UtpTr{\trans{\mat{U}_{t+1}}}
% \newcommand\Utptild{\mat{\widetilde{U}_{t+1}}}
% \newcommand\Us{\mat{U^*}}
% \newcommand\UsTr{\trans{\mat{U^*}}}
% \newcommand{\Sigs}{\mat{\Sigma}}
% \newcommand{\Sigsmh}{\Sigs^{-1/2}}
% \newcommand{\eye}{\mat{I}}
% \newcommand{\twonormbound}{\left(4+\DPhi{\M}{\Xt[0]}\right)\twonorm{\M}}
% \newcommand{\lamj}{\lambda_j}

% \renewcommand\u{\vect{u}}
% \newcommand\uTr{\trans{\vect{u}}}
% \renewcommand\v{\vect{v}}
% \newcommand\vTr{\trans{\vect{v}}}
% \newcommand\w{\vect{w}}
% \newcommand\wTr{\trans{\vect{w}}}
% \newcommand\wperp{\vect{w}_{\perp}}
% \newcommand\wperpTr{\trans{\vect{w}_{\perp}}}
% \newcommand\wj{\vect{w_j}}
% \newcommand\vj{\vect{v_j}}
% \newcommand\wjTr{\trans{\vect{w_j}}}
% \newcommand\vjTr{\trans{\vect{v_j}}}

% \newcommand{\DPhi}[2]{\ensuremath{D_{\Phi}\left(#1,#2\right)}}
% \newcommand\matmult{{\omega}}

% \newpage

\section{Game JSON Structure in \benchmark{}}\label{app:game_json}

As introduced in Section~\ref{sec:gd}, each game in \benchmark{} is represented by a JSON dictionary. Figures~\ref{lst:json-schema} and \ref{fig:trait-schema}--\ref{fig:pre-event-schema} provide the complete schema and its referenced object definitions. Below, we clarify naming discrepancies between this JSON specification and the terminology used in the main article, and also highlight a few design details omitted for brevity.

\paragraph{Naming Discrepancies.}
The JSON schema in Figure~\ref{lst:json-schema} has property names slightly different from those in Figure~\ref{fig:rpebench-overview} from the main article. For clarity, we list them side by side as ``JSON schema name --- main article name'':

\begin{enumerate}
    \item \texttt{player\_name} --- Player Character / Name
    \item \texttt{player\_description} --- Player Character / Description
    \item \texttt{main\_npc\_description / text} --- Main NPC /Description
    \item \texttt{main\_npc\_description / big5\_personality\_traits} --- Main NPC / Personality
    \item \texttt{main\_npc\_description / additional\_facts} --- Main NPC / Facts
    \item \texttt{state\_variables} + \texttt{hidden\_variables} --- State Variables
    \item \texttt{pre\_event\_checks} --- Termination Conditions
\end{enumerate}

For consistency, the appendices continue to use the names from the main article unless otherwise specified. Although \texttt{state\_variables} and \texttt{hidden\_variables} are separate fields in the JSON schema, they collectively represent the State Variables described in the main text. In our design, \texttt{hidden\_variables} (unlike \texttt{state\_variables}) are not displayed to players; however, this distinction does not impact the benchmark evaluations and is thus not emphasized in the main article.

We also require \texttt{hidden\_variables} to include at least two special Boolean flags, \texttt{has\_succeeded} and \texttt{has\_failed}, which interact with \texttt{pre\_event\_checks} (a list of two check objects \texttt{If Succeeded} and \texttt{If Failed}). Each check object includes a \texttt{condition} (a Boolean expression over the state variables) and an \texttt{effect} that sets \texttt{has\_succeeded=1} or \texttt{has\_failed=1}, if not already set\footnote{Some games directly set \texttt{has\_succeeded} or \texttt{has\_failed} in other event effects, leaving effects of \texttt{pre\_event\_checks} empty.}. Conceptually, these properties mirror the Termination Conditions in the main article.

\paragraph{Explanatory Content.}
Several text fields in the JSON schema contain descriptive or explanatory information that we omit from the main article, such as:
\begin{enumerate}
    \item \texttt{\$def/trait/description}: Describes the personality trait score in natural language.
    \item \texttt{\$def/scene\_object/background\_description}: Describes the scene.
    \item \texttt{\$def/variable\_object/description}: Describes a particular state variable.
    \item \texttt{\$def/event\_object/explanations}: Explains event effects.
    \item \texttt{\$def/pre\_event\_check\_object/explanation}: Explains the termination condition check.
\end{enumerate}
Although these fields do not affect our validity checks, they provide additional context for LLMs and are included in prompts given to LLMs during game simulation.

\paragraph{Game Scenes in the BFS Validity Check.}
Because each event references exactly one scene (Figure~\ref{fig:event-schema}), we also verify that all declared scenes are referenced by at least one event. This check is straightforward and independent of the BFS procedure, so it is omitted from the main article for simplicity.

\begin{figure*}[!htp]
\centering
\small
\begin{minipage}{0.95\textwidth}
\begin{lstlisting}[language=json]
{
  "title": "Game Configuration",
  "type": "object",
  "required": [
    "game_world",
    "player_name",
    "player_description",
    "main_npc_name",
    "main_npc_description",
    "game_objectives",
    "scenes",
    "state_variables",
    "hidden_variables",
    "events",
    "pre_event_checks"
  ],
  "properties": {
    "game_world": { "type": "string" },
    "player_name": { "type": "string" },
    "player_description": { "type": "string" },
    "main_npc_name": { "type": "string" },
    "main_npc_description": {
      "type": "object",
      "required": [ "text", "big5_personality_traits", "additional_facts" ],
      "properties": {
        "text": { "type": "string" },
        "big5_personality_traits": { "$ref": "#/$defs/big5_traits" },
        "additional_facts": { "type": "array", "items": { "type": "string" } }
      },
      "additionalProperties": false
    },
    "game_objectives": { "type": "string" },
    "scenes": { "type": "array", "items": { "$ref": "#/$defs/scene_object" }
    },
    "state_variables": { "type": "array", "items": { "$ref": "#/$defs/variable_object" } },
    "hidden_variables": {
      "type": "array",
      "minItems": 2,
      "items": { "$ref": "#/$defs/variable_object" },
      "contains": { "properties": { "value_name": { "enum": [ "has_succeeded", "has_failed" ] } } }
    },
    "events": { "type": "array", "items": { "$ref": "#/$defs/event_object" } },
    "pre_event_checks": { "type": "array", "items": { "$ref": "#/$defs/pre_event_check_object" } },
    "source": { "type": "string" }
  },
  "additionalProperties": false,
}
\end{lstlisting}
\caption{JSON Schema for Game Configuration}
\label{lst:json-schema}
\end{minipage}
\end{figure*}

\begin{figure}[!ht]
\centering
\small
\begin{minipage}{0.95\textwidth}
\begin{lstlisting}[language=json]
{
  "$defs": {
    "trait": {
      "type": "object",
      "required": ["rate", "description"],
      "properties": {
        "rate": { "type": "number" },
        "description": { "type": "string" }
      },
      "additionalProperties": false
    }
  }
}
\end{lstlisting}
\caption{\texttt{trait} object schema}
\label{fig:trait-schema}
\end{minipage}
% \end{figure}

% \begin{figure}[!ht]
% \centering
% \small
\begin{minipage}{0.95\textwidth}
\begin{lstlisting}[language=json]
{
  "$defs": {
    "big5_traits": {
      "type": "object",
      "required": [
        "openness",
        "conscientiousness",
        "extraversion",
        "agreeableness",
        "neuroticism"
      ],
      "properties": {
        "openness": { "$ref": "#/$defs/trait" },
        "conscientiousness": { "$ref": "#/$defs/trait" },
        "extraversion": { "$ref": "#/$defs/trait" },
        "agreeableness": { "$ref": "#/$defs/trait" },
        "neuroticism": { "$ref": "#/$defs/trait" }
      },
      "additionalProperties": false
    }
  }
}
\end{lstlisting}
\caption{\texttt{big5\_traits} object schema}
\label{fig:big5-schema}
\end{minipage}
% \end{figure}

% \begin{figure}[!ht]
% \centering
% \small
\begin{minipage}{0.95\textwidth}
\begin{lstlisting}[language=json]
{
  "$defs": {
    "scene_object": {
      "type": "object",
      "required": [ "scene_name", "unique_id", "background_description", "scene_type" ],
      "properties": {
        "scene_name": { "type": "string" },
        "unique_id": { "type": "string" },
        "background_description": { "type": "string" },
        "scene_type": { "type": "string" }
      },
      "additionalProperties": false
    }
  }
}
\end{lstlisting}
\caption{\texttt{scene\_object} schema}
\label{fig:scene-schema}
\end{minipage}
\end{figure}

\begin{figure}[!ht]
\centering
\small
\begin{minipage}{0.95\textwidth}
\begin{lstlisting}[language=json]
{
  "$defs": {
    "variable_object": {
      "type": "object",
      "required": [ "value_name", "unique_id", "description", "min_value", "max_value" ],
      "properties": {
        "value_name": { "type": "string" },
        "unique_id": { "type": "string" },
        "description": { "type": "string" },
        "initial_value": { "type": "string" },
        "min_value": { "type": "string" },
        "max_value": { "type": "string" }
      }, "additionalProperties": false
    }
  }
}
\end{lstlisting}
\caption{\texttt{variable\_object} schema}
\label{fig:variable-schema}
\end{minipage}
% \end{figure}

% \begin{figure}[!ht]
% \centering
% \small
\begin{minipage}{0.95\textwidth}
\begin{lstlisting}[language=json]
{
  "$defs": {
    "event_object": {
      "type": "object",
      "required": [ "event_name", "unique_id", "scene", "entering_condition", "succeed_condition", "succeed_effect", "fail_effect" ],
      "properties": {
        "event_name": { "type": "string" },
        "unique_id": { "type": "string" },
        "scene": { "type": "array", "items": { "type": "string" } },
        "entering_condition": { "type": "array", "items": { "type": "string" } },
        "succeed_condition": { "type": "array", "items": { "type": "string" } },
        "succeed_effect": { "type": "array", "items": { "type": "string" } },
        "fail_effect": { "type": "array", "items": { "type": "string" } },
        "explanations": { "type": "string" }
      }, "additionalProperties": false
    }
  }
}
\end{lstlisting}
\caption{\texttt{event\_object} schema}
\label{fig:event-schema}
\end{minipage}
% \end{figure}

% \begin{figure}[!t]
% \centering
% \small
\begin{minipage}{0.95\textwidth}
\begin{lstlisting}[language=json]
{
  "$defs": {
    "pre_event_check_object": {
      "type": "object",
      "required": [ "check_name", "unique_id", "description", "condition", "effect" ],
      "properties": {
        "check_name":  { "type": "string" },
        "unique_id":   { "type": "string" },
        "description": { "type": "string" },
        "condition": { "type": "array", "items": { "type": "string" } },
        "effect": { "type": "array", "items": { "type": "string" } },
        "explanation": { "type": "string" }
      }, "additionalProperties": false
    }
  }
}
\end{lstlisting}
\caption{\texttt{pre\_event\_check\_object} schema}
\label{fig:pre-event-schema}
\end{minipage}
\end{figure}

\section{Game Creation Prompt}\label{app:gc_prompt}
For the Game Creation (GC) task, we use the prompt shown below. It references the Wikipedia content of the chosen main NPC (\texttt{\{wikicontent\}}) and the JSON schema defined in Appendix~\ref{app:game_json} (\texttt{\{schema\}}). The full text of this schema is provided to the model so it can generate a well-structured JSON output.
\begin{center}
\begin{minipage}{0.95\textwidth}
\begin{lstlisting}[language=plaintext, frame=none, numbers=none]
Here is a character description:
{wikicontent}

Based on this character, create a detailed game scenario exactly following JSON structure of previous examples and the following schema:
{schema}

## Guidelines
- All numerical values should use consistent ranges (e.g., 0-100)
- Events should have clear cause-and-effect relationships
- Scene progression should depend on variable thresholds
- Include both mandatory and optional events
- Create meaningful connections between variables
- Balance difficulty and achievability
- Ensure all IDs follow consistent formatting (P### for checks, S### for scenes, V### for state variables, H### for hidden variables, E### for events)
- Include proper fail states and success conditions
- Make sure all scenes are specific locations
- Create logical progression paths through the game

Format the response as a single JSON object with all fields properly nested. Must ensure all arrays and objects are properly closed and formatted.
\end{lstlisting}
\end{minipage}
\end{center}

\paragraph{5-Shot Prompt} To guide LLMs more effectively, we supply five example JSON games prior to the main creation prompt. Because each game JSON can be quite lengthy, stacking them directly after the prompt may cause the model to overlook important details in the instruction. Instead, we present the five-shot examples as sequential conversation entries, followed by the actual creation prompt. The resulting conversation structure is illustrated below.
\begin{center}
\begin{minipage}{0.95\textwidth}
\begin{lstlisting}[language=plaintext, frame=none, numbers=none]
    USER: Give me an example game JSON.
    ASSISTANT: {EXAMPLE_1}
    USER: Give me an example game JSON.
    ASSISTANT: {EXAMPLE_2\}
    USER: Give me an example game JSON.
    ASSISTANT: {EXAMPLE_3\}
    USER: Give me an example game JSON.
    ASSISTANT: {EXAMPLE_4\}
    USER: Give me an example game JSON.
    ASSISTANT: {EXAMPLE_5}
    USER: {Prompt for Game Creation}
\end{lstlisting}
\end{minipage}
\end{center}

\section{Evaluation Prompts and Detailed Score Calculations}\label{app:eval_prompt}
We employ a consistent three-part format for most evaluation prompts: an instruction section, a JSON schema specifying the output format, and an example response. To keep this appendix concise, we omit the JSON schemas and example responses when the instruction text alone clearly explains the expected output structure. Below, we detail the prompts and score calculations for four metrics: Main NPC Factual Consistency (\textbf{FAC}), Main NPC Personality Consistency (\textbf{PER}), Interestingness (\textbf{INT}), and Action Choice Quality (\textbf{ACT}).
\subsection{Main NPC Factual Consistency (FAC)}
The prompt below assesses how closely the generated game content aligns with each fact about the main NPC. We concatenate all LLM-generated game narration across the multi-round trajectory into \texttt{game\_content}\footnote{Event Plan and State Variables are omitted because they are not visible to players.}.
\begin{center}
\begin{minipage}{0.95\textwidth}
\begin{lstlisting}[language=plaintext, frame=none, numbers=none]
You are given a piece of narrative game content and a set of facts about a specific non-player character (NPC). Your task is to analyze whether each fact is supported, contradicted, or not addressed by the provided game content. For each fact, determine one of the following judgements based solely on the given game content:
- "align": The game content supports or is consistent with the fact.
- "contradict": The game content directly conflicts with or contradicts the fact.
- "neutral": The game content is unrelated or does not provide enough information to judge the fact.
Please disregard prior knowledge and analyze the NPC purely based on the game content and the facts.

**NPC**: {main_npc_name}

**Game Content**:

{game_content}

**Facts**

{main_npc_facts}

**Output Format**:  
Return the results as a JSON array, where each element is an object with:
- fact_id: the corresponding fact's ID.
- judgement: one of "align", "contradict", or "neutral".
- explanation: a brief explanation for your judgment, referencing specific parts of the game content if applicable.
The return json array should follow this json schema:
{schema}

**Example Response**:
{example}
\end{lstlisting}
\end{minipage}
\end{center}
The judge assigns one of three labels for each fact: ``align,'' ``contradict,'' or ``neutral.'' The final trajectory-level FAC score is computed as
\begin{equation}
    \textbf{FAC}_\text{traj} = \frac{\# \text{align}}{\# \text{align} + \# \text{contradict}},
\end{equation}
and we then average over all trajectories:
\begin{equation}
    \textbf{FAC} = \frac{\sum_{\text{traj}} \textbf{FAC}_\text{traj}}{\# \text{trajectories}}.
\end{equation}

\subsection{Main NPC Personality Consistency (PER)}\label{app:per_eval}
\paragraph{TIPI PER Score} As described in the main article, we derive the PER score using a Ten-Item Personality Inventory (TIPI) approach~\cite{gosling2003very,cao2024large}, prompting the LLM judge to rate each of ten statements and then converting the ratings into Big Five trait scores.
\begin{center}
\begin{minipage}{0.95\textwidth}
\begin{lstlisting}[language=plaintext, frame=none, numbers=none]
You will be given information about a character. Here are a number of personality traits that may or may not apply to the character. Please write a number to each statement to indicate the extent to which you agree or disagree with that statement. You should rate the extent to which the pair of traits applies to the character, even if one characteristic applies more strongly than the other.

For the ratings:
- 1: Disagree strongly
- 2: Disagree moderately
- 3: Disagree a little
- 4: Neither agree nor disagree
- 5: Agree a little
- 6: Agree moderately
- 7: Agree strongly

Please give your ratings for the following 10 statements.

I see the character as:
A. Extraverted, enthusiastic.
B. Critical, quarrelsome.
C. Dependable, self-disciplined.
D. Anxious, easily upset.
E. Open to new experiences, complex.
F. Reserved, quiet.
G. Sympathetic, warm.
H. Disorganized, careless.
I. Calm, emotionally stable.
J. Conventional, uncreative

Please return ratings for all 10 traits in a dictionary following this schema:
{schema}

Please give your ratings for the following character.
{character}
\end{lstlisting}
\end{minipage}
\end{center}
Here, \texttt{character} consists of the main NPC name and the concatenated LLM-generated game narration sections. According to \citet{gosling2003very}, we use the following formulas to calculate personality trait scores,
\begin{equation}
    \begin{aligned}
        &\text{Openness: }&o_{tipi} =& E + 8 - J\\
        &\text{Conscientiousness: }&c_{tipi} =& C + 8 - H\\
        &\text{Extroversion: }&e_{tipi} =& A + 8 - F\\
        &\text{Agreeableness: }&a_{tipi} =& G + 8 - B\\
        &\text{Neuroticism: }&n_{tipi} =& I + 8 - D\\
    \end{aligned}
\end{equation}

To compute the personality consistency, we compare the above scores, after being scaled to $[1,5]$, with the main NPC personality specifications in the game JSON,
\begin{equation}
    d_{\{o,c,e,a,n\}} = \left|\frac{\{o,c,e,a,n\}_{tipi} + 1}{3} - \{o,c,e,a,n\}_{game}\right|.
\end{equation}
The PER score is the squared sum of these differences, normalized to $[0, 1]$,
\begin{equation}
\begin{aligned}
    \textbf{PER}_{\text{traj}} &= 1 - \frac{\sqrt{\sum_{x\in\{o,c,e,a,n\}} d_x^2}}{4\sqrt{5}}\\
    \textbf{PER} &= \frac{\sum_{\text{traj}} \textbf{PER}_\text{traj}}{\# trajectories}
\end{aligned}.
\end{equation}

\paragraph{Direct Evaluation of Personality Consistency}
We also experiment with a direct evaluation approach (referred to as \textbf{PER}$^d$), which instructs the LLM judge to provide a 1--5 alignment rating for each of the five personality traits. 
\begin{center}
\begin{minipage}{0.95\textwidth}
\begin{lstlisting}[language=plaintext, frame=none, numbers=none]
Assign a score from 1 to 5 to indicate how well the game narrative aligns with the main NPC's personality traits:
- Many Conflicts (1): The narrative frequently contradicts the NPC's personality.
- Some Conflicts (2): The narrative shows noticeable inconsistencies with the NPC's personality.
- Neutral (3): The narrative is only partially aligned or does not strongly reflect the NPC's personality.
- Strong Alignment (4): The narrative closely matches the NPC's personality, with only minor deviations or uncertainties.
- Perfect Alignment (5): The narrative flawlessly reflects the NPC's personality in every aspect, with no contradictions.

Please give one score for each personality trait, and provide a brief explanation for each score.

Game narrative:
{game_content}

NPC personality:
{npc_personality}

Please return a score as a json object following this schema:
{schema}
\end{lstlisting}
\end{minipage}
\end{center}
Here, \texttt{npc\_personality} consists of the Big Five personality traits in the game JSON. We compute the final score by averaging the normalized scores across all traits and, subsequently, across all trajectories. We deter discussions of results from this approach to Appendix~\ref{app:human_eval}, where we compare both TIPI estimations and direct evaluation results from LLM judges and human annotators. We refer this score as \textbf{PER$^d$} for the remaining of this article. 

\subsection{Interestingness (INT)}
We prompt an LLM judge to rate the interestingness of the generated content on a 1--5 scale.
\begin{center}
\begin{minipage}{0.95\textwidth}
\begin{lstlisting}[language=plaintext, frame=none, numbers=none]
Your task is to evaluate the **interestingness** of the following game content. Please give a score from 1 (least interesting) to 5 (most interesting), with a brief explanation of your rationale.


[[start of game content]]
{game_content}
[[end of game content]]

Please return your evaluation score in a json dictionary with the following format:
{schema}

Example output:
{example}
\end{lstlisting}
\end{minipage}
\end{center}
We normalize the final score to \([0, 1]\), sum over rounds within a trajectory and then average:
\begin{equation}
\begin{aligned}
    \textbf{INT}_{\text{traj}} &= \frac{int-1}{4}\\
    \textbf{INT} &= \frac{\sum_{\text{traj}} \textbf{INT}_\text{traj}}{\# trajectories}
\end{aligned}.
\end{equation}
\subsection{Action Choice Quality (ACT)}
At each round, the LLM judge scores three candidate actions on three rubrics: Diversity, Relevance, and Understandability, each on a 1--5 scale with the following prompt. 
\begin{center}
\begin{minipage}{0.95\textwidth}
\begin{lstlisting}[language=plaintext, frame=none, numbers=none]
Please act as an experienced RPG game player and evaluate the choices provided by the game engine, given the user-AI interaction history and the general game instruction outlining the basic game settings. You will be given the general game instruction, the interaction history, and the current choices offered by the game engine. Evaluate the choices based on the following rubric:

Please assess the choices provided by the game engine based on this rubric:

[[start of rubric]]
{rubric}
[[end of rubric]]

[[start of general game instruction]]
{game}
[[end of general game instruction]]

[[start of history]]
{history}
[[end of history]]

Please assess the choices provided by the game engine:

[[start of choices]]
{choices}
[[end of choices]]

Your output should be a JSON object structured as follows:
{
    "reason": <your reasoning here>
    "score": <score from 1 to 5 based on the rubric provided>
}
You must NOT output anything else other than this JSON object.
\end{lstlisting}
\end{minipage}
\end{center}
We provide in the prompt above the game JSON~(\texttt{game}), game history up to the current round~(\texttt{history}), and three candidate choices~(\texttt{choices}). We present the following paragraphs in \texttt{rubric} for Diversity, Relevance and Understandability respectively.
\begin{center}
\begin{minipage}{0.95\textwidth}
\begin{lstlisting}[language=plaintext, frame=none, numbers=none]
Diversity: Does the set of choices provide distinct and varied options for the player?
1: The choices are nearly identical, offering no meaningful differences between them.
2: The choices have slight variations but are mostly redundant, leading to a limited sense of variety.
3: The choices exhibit some diversity but may still overlap in intent or outcome.
4: The choices are mostly distinct and provide meaningful differences that allow the player to explore different paths.
5: The choices are highly diverse, with each option offering unique and creative directions for the player.
\end{lstlisting}
\end{minipage}
\end{center}
\begin{center}
\begin{minipage}{0.95\textwidth}
\begin{lstlisting}[language=plaintext, frame=none, numbers=none]
Relevance: Are the choices appropriate and contextually aligned with the story and scene?
1: The choices are entirely irrelevant, disconnected from the scene or story, and break immersion.
2: The choices have limited relevance, with some alignment to the story but containing jarring or out-of-place elements.
3: The choices are moderately relevant, generally aligning with the story but occasionally introducing inconsistencies.
4: The choices are mostly relevant, fitting well within the context and contributing meaningfully to the story.
5: The choices are fully relevant, seamlessly integrated into the story and enhancing the narrative experience.
\end{lstlisting}
\end{minipage}
\end{center}
\begin{center}
\begin{minipage}{0.95\textwidth}
\begin{lstlisting}[language=plaintext, frame=none, numbers=none]
Understandability:  Are the choices clear, concise, and easy to understand for the player?
1: The choices are confusing, overly complex, or poorly worded, making them difficult to interpret.
2: The choices are somewhat understandable but may include ambiguous language or unnecessary complexity.
3: The choices are moderately clear, with minor ambiguities that require some interpretation.
4: The choices are clear and concise, easy to read, and free of significant ambiguity.
5: The choices are exceptionally clear and well-written, making them effortless to understand and act upon
\end{lstlisting}
\end{minipage}
\end{center}

We average these three rubric scores to obtain $act$, then normalize via $(act - 1)/4$. Trajectories are evaluated by averaging per-round scores, and we then take the mean across all trajectories:
\begin{equation}
\begin{aligned}
    \textbf{ACT}_{\text{round}} &= \frac{act-1}{4}\\
    \textbf{ACT}_{\text{traj}} &= \frac{\sum_{\text{round}} \textbf{ACT}_\text{round}}{\# rounds}\\
    \textbf{ACT} &= \frac{\sum_{\text{traj}} \textbf{ACT}_\text{traj}}{\# trajectories}
\end{aligned}.
\end{equation}




\section{Human Evaluation Details}\label{app:human_eval}
\subsection{Interface Layout}
Figure~\ref{fig:human-eval} shows a screenshot of our human evaluation interface. Although it is cut off due to display size, the four main components are visible: \textbf{Text RPG Information}, \textbf{NPC Information}, \textbf{Dialog History}, and \textbf{Responses}. As discussed in Appendix~\ref{app:per_eval}, we use two interfaces: one for TIPI-based personality estimation and one for direct personality-consistency evaluation. These interfaces only differ in how the \textbf{NPC Information} and \textbf{Responses} sections are presented. To help annotators remain focused when assessing a multi-round trajectory, each round in a trajectory is annotated separately by the same annotator.

\noindent\textbf{Text RPG Information:}  
Annotators see the Game World description, the player character’s name and description, and the overall game objective. This information persists throughout the trajectory.

\noindent\textbf{Dialog History:}  
We show the game trajectory up to the current round, including the model’s narration and three candidate actions (boldfaced). One of these actions, selected at random, is displayed on the right side. This component updates every round to reflect the new content.

\noindent\textbf{NPC Information:}  
For TIPI-based personality estimation, we present only the NPC’s name and facts (omitting personality traits so they can be inferred through the TIPI questions). In the direct-evaluation interface, the main NPC personality traits are included here.

\noindent\textbf{Responses:}  
This section poses natural-language questions to gather human judgments on subjective dimensions. It differs slightly between TIPI-based and direct-evaluation interfaces, as detailed below.

\begin{figure}[!ht]
    \centering
    \includegraphics[width=0.95\linewidth]{figs/ScreenShot.png}
    \caption{Screenshot of the human evaluation interface.}
    \label{fig:human-eval}
\end{figure}

\subsection{Evaluation Questions}
\subsection{Evaluation Questions}
Our human evaluation asks annotators to rate various subjective aspects. Questions A--D appear every round in both the TIPI and direct-evaluation interfaces:
\begin{center}
\begin{minipage}{0.95\textwidth}
\begin{lstlisting}[language=plaintext, frame=none, numbers=none]
A. Please give a score (1-5) to indicate how interesting the game narrative is.

B. Do you think all the candidate actions are valid based on the game narrative? - 0 (no) - 1 (yes)
    
C. Are candidate choices different enough from each other, or are they essentially the same? - 0 (same) - 1 (different)
    
D. Please give a score (1-5) to measure whether the game narrative is consistent with the given facts about the main NPC? 
    - 1 has many conflicts
    - 2 has some conflicts
    - 3 neutral
    - 4 matches the description
    - 5 perfectly matches the description
\end{lstlisting}
\end{minipage}
\end{center}
These ratings inform the INT, ACT, and FAC metrics as follows:
\begin{equation}
    \begin{aligned}
        \textbf{INT}_\text{round} &= \frac{A - 1}{4},\\
        \textbf{ACT}_\text{round} &= \frac{B + C}{2},\\
        \textbf{FAC}_\text{round} &= \frac{D - 1}{4}.
    \end{aligned}
\end{equation}
Here, Question B corresponds to Relevance and Understandability in the ACT automatic evaluation, while Question C corresponds to Diversity. We average these round-level scores to obtain a trajectory-level score, then average across all trajectories.

\paragraph{Personality Consistency Questions (E1 and E2).}
We measure PER using two different question sets:
\begin{itemize}
    \item \textbf{E1: TIPI Estimation.}  
    Shown only once per trajectory (at the final round of the TIPI interface), requiring annotators to assess the entire trajectory.  
    \item \textbf{E2: Direct Evaluation.}  
    Appears at every round in the direct-evaluation interface.
\end{itemize}

Both methods yield PER scores analogous to the automatic evaluations in Appendix~\ref{app:per_eval}.

\begin{center}
\begin{minipage}{0.95\textwidth}
\begin{lstlisting}[language=plaintext, frame=none, numbers=none]
E1. Here are a number of personality traits that may or may not apply to the character. Please write a number to each statement to indicate the extent to which you agree or disagree with that statement, based ONLY on the game narratives. You should rate the extent to which the pair of traits applies to the character, even if one characteristic applies more strongly than the other. Use a score range of 1-7:
    - 1: Disagree strongly
    - 2: Disagree moderately
    - 3: Disagree a little
    - 4: Neither agree nor disagree
    - 5: Agree a little
    - 6: Agree moderately
    - 7: Agree strongly

    I see the main NPC as
    A. Extraverted, enthusiastic.
    B. Critical, quarrelsome.
    C. Dependable, self-disciplined.
    D. Anxious, easily upset.
    E. Open to new experiences, complex.
    F. Reserved, quiet.
    G. Sympathetic, warm.
    H. Disorganized, careless.
    I. Calm, emotionally stable.
    J. Conventional, uncreative

E2. Please give a score (1-5) to measure whether the game narrative is consistent with the given facts about the main NPC?
    - 1 has many conflicts
    - 2 has some conflicts
    - 3 neutral
    - 4 matches the description
    - 5 perfectly matches the description
\end{lstlisting}
\end{minipage}
\end{center}
\subsection{Annotation Setup}
We recruited 15 human annotators. Each trajectory is annotated at the round level, resulting in two annotations per interface type and therefore four total annotations per trajectory. We ensure that each annotator encounters any given trajectory only once, regardless of interface type. Consequently, each trajectory ends up with four sets of INT, ACT, and FAC scores, and two sets of PER and \textbf{PER}$^d$ scores. We take the mean over all trials to produce the final reported values. For inter-annotator agreement (Table~\ref{tab:gs_corr}), we randomly divide the collected annotations into two groups and compare their scores.

\subsection{PER vs. PER$^d$ Evaluation Results}
\begin{table*}[!ht]
\centering
\begin{tabular}{lrrrrrr}
\toprule
\multirow{2}{*}{Model} & \multicolumn{2}{c}{PER (Subset)} & \multicolumn{2}{c}{PER$^d$ (Subset)} & \multicolumn{1}{c}{PER (Full)} & \multicolumn{1}{c}{PER$^d$ (Full)}\\ 
& \multicolumn{1}{c}{Auto} & \multicolumn{1}{c}{Human} & \multicolumn{1}{c}{Auto} & \multicolumn{1}{c}{Human} & \multicolumn{1}{c}{Auto} & \multicolumn{1}{c}{Auto}\\
\midrule
Claude 3.5 Sonnet & 0.729 & 0.648 & 0.768 & \textbf{0.832} & 0.589 & 0.738 \\
Deepseek V3 & 0.742 & 0.645 & 0.750 & \underline{0.826} & 0.583 & \textbf{0.778}\\
Gemini 1.5 Pro & 0.740 & 0.648 & \textbf{0.800} & 0.769 & \underline{0.596} & \underline{0.777}\\
Gemini 2.0 Flash Exp & 0.737& \underline{0.651} & 0.707 & 0.769 &\textbf{0.598} & 0.750\\
GPT 4o & 0.711 & \textbf{0.667} & 0.780 & 0.724 & 0.585 & 0.768 \\
GPT 4o mini & \textbf{0.753}  & 0.648 & \underline{0.788} & 0.735 & 0.588 & 0.763\\
Llama 3.1 70B & \underline{0.744}  & 0.627 & 0.768 & 0.752 & 0.586 & 0.765\\
Llama 3.3 70B & 0.739 & 0.640 & 0.739 & 0.755 & 0.585 &0.774\\
\bottomrule
\end{tabular}
\caption{PER and PER$^d$ results from automatic and human evaluation on a subset of 20 games, and automatic evaluation on the full set of games.}
\label{tab:per_evaluation_results}
\end{table*}


\begin{table}[!ht]
    \centering
    \begin{tabular}{cc|rrr}
    \toprule
    \multicolumn{2}{c|}{Comparisons} & Pearson & Kendall & MAD \\
    \midrule
    \multirow{1}{*}{Auto-Auto Agreement} & PER Auto - PER$^d$ Auto & 0.013 & 0.109 & 0.037 \\
    \midrule
    \multirow{3}{*}{Auto-Human Agreement} & PER Auto - PER Human     & -0.691 & -0.429 & 0.090\\
    & PER$^d$ Auto - PER$^d$ Human &-0.297 & -0.255 & 0.047\\
    \midrule
    \multirow{3}{*}{Human-Human Agreement} &PER Human - PER Human & -0.310 & -0.286 & 0.023\\
    &PER$^d$ Human - PER$^d$ Human & 0.649 & 0.143 & 0.035\\
    &PER Human - PER$^d$ Human &-0.175&-0.143 & 0.124\\
    \bottomrule
    \end{tabular}
    \caption{Agreement analysis for PER and PER$^d$ scores. We present Pearson correlation coefficient (Pearson), Kendall rank correlation coefficient (Kendall), and Mean Absolute Difference (MAD)}
    \label{tab:per_agreement}
\end{table}

In our main article, we adopt the PER score for evaluating NPC personality consistency. Here, we further analyze both PER and PER$^d$ scores from automatic and human evaluations on a subset of 20 games in Table~\ref{tab:per_evaluation_results}, with additionally automatic evaluation results on the full dataset. We also report agreement metrics in Table~\ref{tab:per_agreement} Our analysis reveals several key observations:

\begin{enumerate}
    \item \textbf{PER$^d$ tends to be higher than PER in both automatic and human evaluations.} Across models, we observe that PER$^d$ scores are consistently higher than PER scores, indicating that direct evaluation of personality consistency is generally more lenient than the TIPI-based method. This trend holds for both automatic and human evaluators.

    \item \textbf{LLMs achieve similar PER scores across the dataset.} The automatic PER and PER$^d$ scores on the full set of games show little variation across models, with all models achieving scores around 0.58–0.60 for PER and around 0.74–0.78 for PER$^d$. This suggests that models perform comparably in terms of maintaining personality consistency in text-based role-playing.

    \item \textbf{Human evaluators rate PER$^d$ higher than PER, but with noticeable variation.} While automatic evaluations show a clear gap between PER and PER$^d$, human annotations exhibit a similar pattern but with greater variability. Notably, human evaluators assign significantly higher PER$^d$ scores to some models, such as Claude 3.5 Sonnet and DeepSeek V3, compared to their automatic scores.

    \item \textbf{Human and automatic PER scores exhibit poor correlation.} Table~\ref{tab:per_agreement} shows that the Pearson correlation between PER Auto and PER Human is negative (-0.691), with Kendall correlation also negative (-0.429). This suggests a fundamental mismatch between how LLM-based and human evaluators assess personality consistency through TIPI.

    \item \textbf{Better human agreement for PER$^d$, but still unstable.} While inter-human correlation for PER is negative (-0.310 Pearson, -0.286 Kendall), PER$^d$ exhibits a stronger but still weak agreement (0.649 Pearson). This suggests that directly rating personality alignment may be more intuitive for human evaluators than using TIPI scores but remains somewhat unstable. However, there is still concern over whether human annotators are capable of accurately understanding Big Five traits in the direct evaluation scenario.

    \item \textbf{Low agreement between PER and PER$^d$.} The Pearson correlation between PER and PER$^d$ scores (both automatic and human) is low (0.013 for Auto-Auto and -0.175 for Human-Human), indicating that these two evaluation methods capture different aspects of personality consistency. While PER$^d$ measures direct alignment with given traits, PER (TIPI) estimates personality traits implicitly, which may introduce more variance in judgments.
\end{enumerate}

\paragraph{Justification for Choosing TIPI (PER) in the Main Article.}  
We adopt TIPI-based personality consistency (\textbf{PER}) rather than direct evaluation (\textbf{PER}$^d$) in the main study for several reasons. First, TIPI does not require evaluators to have prior knowledge of the Big Five personality traits, making it a structured and interpretable method for assessing personality consistency. Additionally, the high variance in PER$^d$ human scores (as seen in Table~\ref{tab:per_agreement}) suggests that direct personality evaluation is more susceptible to subjective biases. The negative correlation between automatic and human PER scores further emphasizes the challenge of aligning LLM-based and human-based assessments, reinforcing the need for a more systematic approach like TIPI.

Overall, these results highlight the complexity of evaluating personality consistency, where different evaluation paradigms yield divergent results. The instability in human-human agreement for both PER and PER$^d$ suggests that subjective evaluation remains a challenging aspect of LLM benchmarking, warranting further research into more reliable personality evaluation methodologies.



\end{document}
