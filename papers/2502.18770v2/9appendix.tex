\newpage
~
\newpage
\section{Notations}
The definitions of the notations used in this paper are summarized in Table~\ref{table:notation}.

\section{Training Details}
\label{section:training_details}
\paragraph{Dataset}
Our experiments are conducted on two datasets: Ultrafeedback-Binarized~\cite{cui2023ultrafeedback} and the helpful-base subset of HH-rlhf~\cite{bai2022traininghelpfulharmlessassistant}. Both datasets undergo preprocessing to eliminate noise and constrain their overall length. For the Ultrafeedback-Binarized dataset, we select examples where the prompt length, chosen response length, and rejected response length are each less than 512 tokens. Additionally, we ensure that the chosen response score exceeds the rejected response score and that the substring 'confidence' does not appear in either the chosen or rejected responses. For the HH-rlhf dataset, we apply the same length constraints (prompt, chosen, and rejected responses each under 512 tokens). Furthermore, we ensure that each prompt appears only once across both datasets and limit the test set to 256 examples. The training set of Ultrafeedback-Binarized contains around 33,000 examples and HH-RLHF helpful base contains 43,000 examples. All training are carried on 8*A800(80G) GPUs.
\paragraph{Base Models}
For the base models, we utilize Gemma-2B~\citep{gemma_2024} and Llama3-8B~\citep{llama3modelcard}. In all training procedures, we implement a linear learning rate scheduler, which gradually increases the learning rate from 0 to the maximum value over the first 0.1 epoch.

\paragraph{SFT Model}
The Supervised Fine-Tuned (SFT) model is initialized from the base model and trained on the chosen responses for two epochs with a learning rate of 5e-6. Gradient norm clipping is applied when the norm exceeds 10.

\paragraph{Reward Model}
The reward model is initialized from the base model, with the logit head replaced by a linear head above the last embedding layer to output a scalar value. It is trained for one epoch with a learning rate of 5e-6, achieving an accuracy of approximately 70\% on the test set. Gradient norm clipping is applied when the norm exceeds 5.

For ODIN training, we use two linear heads to output length reward and quality reward separately, following the training loss described in~\citet{Chen2024ODINDR}. Only the quality head is used during RL training.

For WARM training, we train five reward models on the same dataset with varying learning rates (3e-6, 4e-6, 5e-6, 6e-6, 7e-6) and different random seeds.

For Reg training, we adopt the loss function from~\cite{Dai2023SafeRS}, with a regularization term coefficient of 0.005.

\paragraph{Policy Model}
The policy model is initialized from the SFT model and trained on the same prompts for one epoch using the PPO algorithm with a learning rate of 3e-7. Gradient norm clipping is applied when the norm exceeds 5.

\paragraph{Critic Model}
The critic model is initialized from the reward model and trained alongside the policy model for one epoch with a learning rate of 5e-6. Gradient norm clipping is applied when the norm exceeds 5.

\paragraph{Hyper-Parameters}
Responses are sampled from the policy model using a temperature of 0.9, with top-k set to 50, top-p set to 0.9, and a length penalty of 2. The coefficient for the KL penalty is 0.005, and the default number of reference rewards is 10. For PPO training, the buffer size is set to 4, with $\epsilon=0.2,\lambda=0.95,\gamma=1.0$, For GRPO training, the $\epsilon=0.2$, the buffer size is 4, and the group size is 5.

\section{Evaluation}
\subsection{Winrate on Test Set}
To leverage the strong grading capability of DeepSeek-V3 for comparing the SFT model and the policy model on the test set, we design a detailed evaluation prompt. The system prompt and user input format are provided in Listing 1 and 2.

To address position bias~\cite{wang2023largelanguagemodelsfair}, we evaluate each pair of responses twice, alternating their order, and aggregate the scores. Specifically, for two responses A and B, we first evaluate them in the order A-B and then in the order B-A. In each evaluation, the winner receives a score of 1, the loser receives 0, and in the case of a tie, both responses receive 0.5. The final scores of A and B are compared, and the response with the higher score is declared the winner. If the scores are tied, both responses receive 0.5 win counts. The win counts are used to calculate the winrate.

% 
\definecolor{titlecolor}{rgb}{0.9, 0.5, 0.1}
\definecolor{anscolor}{rgb}{0.2, 0.5, 0.8}
\definecolor{labelcolor}{HTML}{48a07e}
\begin{table*}[h]
	\centering
	
 % \vspace{-0.2cm}
	
	\begin{center}
		\begin{tikzpicture}[
				chatbox_inner/.style={rectangle, rounded corners, opacity=0, text opacity=1, font=\sffamily\scriptsize, text width=5in, text height=9pt, inner xsep=6pt, inner ysep=6pt},
				chatbox_prompt_inner/.style={chatbox_inner, align=flush left, xshift=0pt, text height=11pt},
				chatbox_user_inner/.style={chatbox_inner, align=flush left, xshift=0pt},
				chatbox_gpt_inner/.style={chatbox_inner, align=flush left, xshift=0pt},
				chatbox/.style={chatbox_inner, draw=black!25, fill=gray!7, opacity=1, text opacity=0},
				chatbox_prompt/.style={chatbox, align=flush left, fill=gray!1.5, draw=black!30, text height=10pt},
				chatbox_user/.style={chatbox, align=flush left},
				chatbox_gpt/.style={chatbox, align=flush left},
				chatbox2/.style={chatbox_gpt, fill=green!25},
				chatbox3/.style={chatbox_gpt, fill=red!20, draw=black!20},
				chatbox4/.style={chatbox_gpt, fill=yellow!30},
				labelbox/.style={rectangle, rounded corners, draw=black!50, font=\sffamily\scriptsize\bfseries, fill=gray!5, inner sep=3pt},
			]
											
			\node[chatbox_user] (q1) {
				\textbf{System prompt}
				\newline
				\newline
				You are a helpful and precise assistant for segmenting and labeling sentences. We would like to request your help on curating a dataset for entity-level hallucination detection.
				\newline \newline
                We will give you a machine generated biography and a list of checked facts about the biography. Each fact consists of a sentence and a label (True/False). Please do the following process. First, breaking down the biography into words. Second, by referring to the provided list of facts, merging some broken down words in the previous step to form meaningful entities. For example, ``strategic thinking'' should be one entity instead of two. Third, according to the labels in the list of facts, labeling each entity as True or False. Specifically, for facts that share a similar sentence structure (\eg, \textit{``He was born on Mach 9, 1941.''} (\texttt{True}) and \textit{``He was born in Ramos Mejia.''} (\texttt{False})), please first assign labels to entities that differ across atomic facts. For example, first labeling ``Mach 9, 1941'' (\texttt{True}) and ``Ramos Mejia'' (\texttt{False}) in the above case. For those entities that are the same across atomic facts (\eg, ``was born'') or are neutral (\eg, ``he,'' ``in,'' and ``on''), please label them as \texttt{True}. For the cases that there is no atomic fact that shares the same sentence structure, please identify the most informative entities in the sentence and label them with the same label as the atomic fact while treating the rest of the entities as \texttt{True}. In the end, output the entities and labels in the following format:
                \begin{itemize}[nosep]
                    \item Entity 1 (Label 1)
                    \item Entity 2 (Label 2)
                    \item ...
                    \item Entity N (Label N)
                \end{itemize}
                % \newline \newline
                Here are two examples:
                \newline\newline
                \textbf{[Example 1]}
                \newline
                [The start of the biography]
                \newline
                \textcolor{titlecolor}{Marianne McAndrew is an American actress and singer, born on November 21, 1942, in Cleveland, Ohio. She began her acting career in the late 1960s, appearing in various television shows and films.}
                \newline
                [The end of the biography]
                \newline \newline
                [The start of the list of checked facts]
                \newline
                \textcolor{anscolor}{[Marianne McAndrew is an American. (False); Marianne McAndrew is an actress. (True); Marianne McAndrew is a singer. (False); Marianne McAndrew was born on November 21, 1942. (False); Marianne McAndrew was born in Cleveland, Ohio. (False); She began her acting career in the late 1960s. (True); She has appeared in various television shows. (True); She has appeared in various films. (True)]}
                \newline
                [The end of the list of checked facts]
                \newline \newline
                [The start of the ideal output]
                \newline
                \textcolor{labelcolor}{[Marianne McAndrew (True); is (True); an (True); American (False); actress (True); and (True); singer (False); , (True); born (True); on (True); November 21, 1942 (False); , (True); in (True); Cleveland, Ohio (False); . (True); She (True); began (True); her (True); acting career (True); in (True); the late 1960s (True); , (True); appearing (True); in (True); various (True); television shows (True); and (True); films (True); . (True)]}
                \newline
                [The end of the ideal output]
				\newline \newline
                \textbf{[Example 2]}
                \newline
                [The start of the biography]
                \newline
                \textcolor{titlecolor}{Doug Sheehan is an American actor who was born on April 27, 1949, in Santa Monica, California. He is best known for his roles in soap operas, including his portrayal of Joe Kelly on ``General Hospital'' and Ben Gibson on ``Knots Landing.''}
                \newline
                [The end of the biography]
                \newline \newline
                [The start of the list of checked facts]
                \newline
                \textcolor{anscolor}{[Doug Sheehan is an American. (True); Doug Sheehan is an actor. (True); Doug Sheehan was born on April 27, 1949. (True); Doug Sheehan was born in Santa Monica, California. (False); He is best known for his roles in soap operas. (True); He portrayed Joe Kelly. (True); Joe Kelly was in General Hospital. (True); General Hospital is a soap opera. (True); He portrayed Ben Gibson. (True); Ben Gibson was in Knots Landing. (True); Knots Landing is a soap opera. (True)]}
                \newline
                [The end of the list of checked facts]
                \newline \newline
                [The start of the ideal output]
                \newline
                \textcolor{labelcolor}{[Doug Sheehan (True); is (True); an (True); American (True); actor (True); who (True); was born (True); on (True); April 27, 1949 (True); in (True); Santa Monica, California (False); . (True); He (True); is (True); best known (True); for (True); his roles in soap operas (True); , (True); including (True); in (True); his portrayal (True); of (True); Joe Kelly (True); on (True); ``General Hospital'' (True); and (True); Ben Gibson (True); on (True); ``Knots Landing.'' (True)]}
                \newline
                [The end of the ideal output]
				\newline \newline
				\textbf{User prompt}
				\newline
				\newline
				[The start of the biography]
				\newline
				\textcolor{magenta}{\texttt{\{BIOGRAPHY\}}}
				\newline
				[The ebd of the biography]
				\newline \newline
				[The start of the list of checked facts]
				\newline
				\textcolor{magenta}{\texttt{\{LIST OF CHECKED FACTS\}}}
				\newline
				[The end of the list of checked facts]
			};
			\node[chatbox_user_inner] (q1_text) at (q1) {
				\textbf{System prompt}
				\newline
				\newline
				You are a helpful and precise assistant for segmenting and labeling sentences. We would like to request your help on curating a dataset for entity-level hallucination detection.
				\newline \newline
                We will give you a machine generated biography and a list of checked facts about the biography. Each fact consists of a sentence and a label (True/False). Please do the following process. First, breaking down the biography into words. Second, by referring to the provided list of facts, merging some broken down words in the previous step to form meaningful entities. For example, ``strategic thinking'' should be one entity instead of two. Third, according to the labels in the list of facts, labeling each entity as True or False. Specifically, for facts that share a similar sentence structure (\eg, \textit{``He was born on Mach 9, 1941.''} (\texttt{True}) and \textit{``He was born in Ramos Mejia.''} (\texttt{False})), please first assign labels to entities that differ across atomic facts. For example, first labeling ``Mach 9, 1941'' (\texttt{True}) and ``Ramos Mejia'' (\texttt{False}) in the above case. For those entities that are the same across atomic facts (\eg, ``was born'') or are neutral (\eg, ``he,'' ``in,'' and ``on''), please label them as \texttt{True}. For the cases that there is no atomic fact that shares the same sentence structure, please identify the most informative entities in the sentence and label them with the same label as the atomic fact while treating the rest of the entities as \texttt{True}. In the end, output the entities and labels in the following format:
                \begin{itemize}[nosep]
                    \item Entity 1 (Label 1)
                    \item Entity 2 (Label 2)
                    \item ...
                    \item Entity N (Label N)
                \end{itemize}
                % \newline \newline
                Here are two examples:
                \newline\newline
                \textbf{[Example 1]}
                \newline
                [The start of the biography]
                \newline
                \textcolor{titlecolor}{Marianne McAndrew is an American actress and singer, born on November 21, 1942, in Cleveland, Ohio. She began her acting career in the late 1960s, appearing in various television shows and films.}
                \newline
                [The end of the biography]
                \newline \newline
                [The start of the list of checked facts]
                \newline
                \textcolor{anscolor}{[Marianne McAndrew is an American. (False); Marianne McAndrew is an actress. (True); Marianne McAndrew is a singer. (False); Marianne McAndrew was born on November 21, 1942. (False); Marianne McAndrew was born in Cleveland, Ohio. (False); She began her acting career in the late 1960s. (True); She has appeared in various television shows. (True); She has appeared in various films. (True)]}
                \newline
                [The end of the list of checked facts]
                \newline \newline
                [The start of the ideal output]
                \newline
                \textcolor{labelcolor}{[Marianne McAndrew (True); is (True); an (True); American (False); actress (True); and (True); singer (False); , (True); born (True); on (True); November 21, 1942 (False); , (True); in (True); Cleveland, Ohio (False); . (True); She (True); began (True); her (True); acting career (True); in (True); the late 1960s (True); , (True); appearing (True); in (True); various (True); television shows (True); and (True); films (True); . (True)]}
                \newline
                [The end of the ideal output]
				\newline \newline
                \textbf{[Example 2]}
                \newline
                [The start of the biography]
                \newline
                \textcolor{titlecolor}{Doug Sheehan is an American actor who was born on April 27, 1949, in Santa Monica, California. He is best known for his roles in soap operas, including his portrayal of Joe Kelly on ``General Hospital'' and Ben Gibson on ``Knots Landing.''}
                \newline
                [The end of the biography]
                \newline \newline
                [The start of the list of checked facts]
                \newline
                \textcolor{anscolor}{[Doug Sheehan is an American. (True); Doug Sheehan is an actor. (True); Doug Sheehan was born on April 27, 1949. (True); Doug Sheehan was born in Santa Monica, California. (False); He is best known for his roles in soap operas. (True); He portrayed Joe Kelly. (True); Joe Kelly was in General Hospital. (True); General Hospital is a soap opera. (True); He portrayed Ben Gibson. (True); Ben Gibson was in Knots Landing. (True); Knots Landing is a soap opera. (True)]}
                \newline
                [The end of the list of checked facts]
                \newline \newline
                [The start of the ideal output]
                \newline
                \textcolor{labelcolor}{[Doug Sheehan (True); is (True); an (True); American (True); actor (True); who (True); was born (True); on (True); April 27, 1949 (True); in (True); Santa Monica, California (False); . (True); He (True); is (True); best known (True); for (True); his roles in soap operas (True); , (True); including (True); in (True); his portrayal (True); of (True); Joe Kelly (True); on (True); ``General Hospital'' (True); and (True); Ben Gibson (True); on (True); ``Knots Landing.'' (True)]}
                \newline
                [The end of the ideal output]
				\newline \newline
				\textbf{User prompt}
				\newline
				\newline
				[The start of the biography]
				\newline
				\textcolor{magenta}{\texttt{\{BIOGRAPHY\}}}
				\newline
				[The ebd of the biography]
				\newline \newline
				[The start of the list of checked facts]
				\newline
				\textcolor{magenta}{\texttt{\{LIST OF CHECKED FACTS\}}}
				\newline
				[The end of the list of checked facts]
			};
		\end{tikzpicture}
        \caption{GPT-4o prompt for labeling hallucinated entities.}\label{tb:gpt-4-prompt}
	\end{center}
\vspace{-0cm}
\end{table*}
\subsection{Benchmark}
We also evaluate the model on two benchmarks, using DeepSeek-V3 to simulate human evaluation. The metrics and their meanings are as follows:
\subsection*{AlpacaEval 2.0}
\begin{itemize}[leftmargin=*, itemsep=0pt]
    \item \textbf{LC Winrate}: The length-controlled win rate measures the model's performance while controlling for the length of generated responses. It compares the model's outputs to a baseline (e.g., the SFT model) and adjusts for the influence of response length on human preferences.
    \item \textbf{Winrate}: The standard win rate measures the proportion of times the model's outputs are preferred over the baseline's outputs in human evaluations.
    \item \textbf{Length}: The average length of the model's generated responses, measured in tokens or characters, providing insight into the model's verbosity.
\end{itemize}

\subsection*{MT-bench}
\begin{itemize}[leftmargin=*, itemsep=0pt]
    \item \textbf{T1}: Turn 1 Score evaluates the model's performance on the first turn of a multi-turn dialogue, assessing relevance, coherence, and informativeness. Scores are normalized as 0-10.
    \item \textbf{T2}: Turn 2 Score evaluates the model's performance on the second turn, measuring its ability to maintain context and provide consistent, high-quality responses. Scores are also normalized as 0-10.
    \item \textbf{Overall}: The overall score is the average of the T1 and T2 scores, providing a comprehensive evaluation of the model's performance across both turns.
\end{itemize}



\section{More Results}
\subsection{Llama3-8B and Ultrafeedback Binarized}
% \input{figures/llama3-8b_ultrafb_bin_lab1}
Figure~\ref{fig:llama3-8b_ultrafb_bin_lab1} presents the PPO training curves for different mitigation methods on Llama3-8B with the Ultrafeedback Binarized dataset. PAR demonstrates robustness against reward hacking and maintains a high win rate throughout one epoch of training.

\subsection{Gemma2-2B and HH-RLHF}
% \input{figures/gemma2-2b_hh_rlhf_lab1}
The PPO training curves for various mitigation methods on Gemma2-2B with the HH-RLHF dataset are shown in Figure~\ref{fig:gemma2-2b_hh_rlhf_lab1}. PAR exhibits resilience to reward hacking and sustains a high win rate during one epoch of training.

\subsection{Llama3-8B and HH-RLHF}
% \input{figures/llama3-8b_hh_rlhf_lab1}
Figure~\ref{fig:llama3-8b_hh_rlhf_lab1} illustrates the PPO training curves for different mitigation methods applied to Llama3-8B on the HH-RLHF dataset. While PAR shows signs of reward hacking toward the end of training, it maintains a consistently high win rate (above 60\%) for an extended period, from 10,000 to 30,000 steps. We hypothesize that the observed reward hacking in the later stages is due to the convergence rate of the sigmoid function approaching its upper bound.

\section{Case Study}
We identify several patterns of reward hacking observed in Vanilla PPO training, using the checkpoint trained after one epoch for detailed examination. We show the examples in Figure \ref{fig:case_study}.



\section{PPO Training}
PPO (Proximal Policy Optimization) is an online reinforcement learning algorithm that generates a response given a prompt, computes a reward for the response using a reward model, and updates the policy and critic models to maximize the reward.

We employ several PPO techniques to ensure stable training, including advantage normalization~\cite{zheng2023secretsrlhflargelanguage}, value loss clipping~\cite{patterson2023robustlosseslearningvalue}, a replay buffer~\cite{eysenbach2019searchreplaybufferbridging}, per-token KL penalty, and length penalty. The pseudo-code for the PPO algorithm is provided in Algorithm~\ref{alg:ppo}.


\section{Reward Shaping Is Not Applicable to DPO and GRPO}
\label{section:rsnotavail}
In this section, we explain why monotonous reward shaping techniques, such as PAR, are not applicable to the Direct Preference Optimization (DPO). And why linear shaping techniques are not applicable to the Group Relative Policy Optimization (GRPO) algorithms.

\subsection{DPO and Reward Shaping}
Vanilla DPO is an offline alignment algorithm that trains the policy model directly on paired responses using a contrastive loss. Since the vanilla DPO algorithm does not rely on an explicit reward model, reward shaping techniques are inherently inapplicable. We also explore an online variant of DPO, which generates two responses for a given prompt and employs a reward model to determine the chosen and rejected responses. The policy model is then trained on these responses (see Algorithm~\ref{alg:dpo}). However, any monotonous transformation of the proxy reward will not alter the chosen and rejected responses. For instance, if \( r_1 > r_2 \), then \( f(r_1) > f(r_2) \) for any monotonous function \( f(\cdot) \), including PAR. Consequently, PAR is also not applicable to online DPO.

\subsection{GRPO and Reward Shaping}
For GRPO, the advantage value is computed as a normalization of proxy rewards. Consider a prompt \( x \) and \( N \) responses \( y_1, \dots, y_N \) sampled from the policy model. A reward model \( r_\phi \) assigns scores \( r_1, \dots, r_N \) to each response. The advantage \( A_{i,t} \) for response \( y_i \) at token position \( t \) is given by:
\[
A_{i,t} = \frac{r_i - \mu}{s},
\]
where \( \mu = \frac{1}{N} \sum_{i=1}^N r_i \) and \( s = \sqrt{\frac{1}{N} \sum_{i=1}^N (r_i - \mu)^2 } \) are the mean and standard deviation of the rewards, respectively.

Assume a linear transformation is applied to the proxy reward, such that \( \hat{r} = a \cdot r + b \) (\( a > 0 \)). We prove that the new advantage \( \hat{A}_{i,t} \) is identical to the original \( A_{i,t} \). First, the new mean \( \hat{\mu} = a \cdot \mu + b \), and the new standard deviation \( \hat{s} = a \cdot s \). The new advantage is computed as:
\[
\begin{aligned}
\hat{A}_{i,t} &= \frac{\hat{r}_i - \hat{\mu}}{\hat{s}} = \frac{a r_i + b - (a \mu + b)}{a s} \\
&= \frac{a r_i - a \mu}{a s} = \frac{r_i - \mu}{s} \\
&= A_{i,t}.
\end{aligned}
\]

Thus, linear transformations do not influence the advantage calculation in GRPO. Furthermore, since the sigmoid function is a non-linear function, PAR is applicable to GRPO training. We validate this through experiments, as shown in Figure~\ref{fig:grpo}. No reward hacking problem is observed in the GRPO training process, as the advantage calculation inherently performs reward normalization.

\begin{figure}[H]
\centering
\includegraphics[width=1.0\linewidth]{figures/grpo_img.pdf}

\caption{The training curves for GRPO, evaluated on Gemma2-2B with the Ultrafeedback-Binarized dataset, demonstrate that Vanilla, Meanstd, Minmax exhibit similar proxy rewards throughout the training process. This is because linear transformations of the proxy rewards do not affect the advantage value in GRPO. The PAR is a non-linear function and slightly better before collapse. No reward hacking issue is observed in the GRPO training process, as the advantage calculation inherently normalizes the rewards.}
\label{fig:grpo}
\end{figure}

% Define the style for code blocks
\lstset{
    backgroundcolor=\color{gray!10}, % Background color
    basicstyle=\ttfamily\footnotesize, % Basic font style
    breaklines=true, % Automatic line breaks
    frame=single, % Single border
    captionpos=b, % Caption position at the bottom
    keywordstyle=\color{blue}, % Keyword color
    commentstyle=\color{green!50!black}, % Comment color
    stringstyle=\color{red}, % String color
    showstringspaces=false, % Do not show spaces in strings
    tabsize=4 % Set tab width
}
% listing 1 2 
\begin{figure*}[htb]
\begin{lstlisting}[caption={System Prompt For Winrate Evaluation on Test Set}]
Please act as an impartial evaluator to assess the quality of two responses from different AI assistants to an incomplete dialogue between a user (<|user|>) and an AI assistant (<|assistant|>). The dialogue will be missing the last turn, and both Assistant-A (<Assistant-A response>) and Assistant-B (<Assistant-B response>) are expected to complete it. Focus your evaluation on the following five aspects:
1. Clarity and Relevance: Responses should be concise, directly addressing the question. They should use clear, natural language and remain on-topic.
2. Accuracy and Honesty: Responses must provide factual, truthful information. Disclose limitations or uncertainties when necessary.
3. Ethics and Appropriateness: Ensure the responses are free from harmful, offensive, or discriminatory content.
4. Engagement and Depth: Responses should be engaging, educational, and sufficiently detailed to comprehensively address the user question.
5. Structure and Creativity: Responses should be logically organized and show originality or adaptability when necessary.

Note: The quality of the responses should not be judged solely by their length. Both brevity and detail are important depending on the context of the question.
You will be given an incomplete dialogue (<question>) with the last turn left blank. Assistant-A (<Assistant-A response>) and Assistant-B (<Assistant-B response>) have each provided a response to complete the dialogue. Your task is to evaluate each response based on the five criteria above and provide a comparison.

Evaluation Format:
Assistant-A Response:
(Evaluate the quality of Assistant-A response based on the five aspects mentioned above.)
Assistant-B Response:
(Evaluate the quality of Assistant-B response based on the five aspects mentioned above.)
Comparison and Analysis:
Compare and contrast the responses from Assistant-A and Assistant-B to determine which one is more effective overall. Justify your reasoning clearly and concisely.

At the end, output the comparison result for both responses in the following format:
Better: X (X is A, B, or N, representing A is better, B is better, or both are of equal quality)
\end{lstlisting}

\begin{lstlisting}[caption={User Input Template For Winrate Evaluation on Test Set}]
<question>:
{user_question}
<Assistant-A response>:
{policy_response}
<Assistant-B response>:
{sft_response}
\end{lstlisting}
\end{figure*}

\begin{figure*}[htbp]
    \centering
    \begin{subfigure}[b]{0.7\linewidth}
        \centering
        \includegraphics[width=\linewidth]{figures/llama3-8b_ultrafb_bin_curve_lab1.pdf}
        \caption{PPO training curve for different mitigation methods on Llama3-8B and Ultrafeedback-Binarized.}
        \label{fig:llama3-8b_ultrafb_bin_lab1}
    \end{subfigure}
    
    \vspace{0.2cm} % Adjust vertical spacing between subfigures
    
    \begin{subfigure}[b]{0.7\linewidth}
        \centering
        \includegraphics[width=\linewidth]{figures/gemma2-2b_hh_rlhf_curve_lab1.pdf}
        \caption{PPO training curve for different mitigation methods on Gemma2-2B and HH-RLHF.}
        \label{fig:gemma2-2b_hh_rlhf_lab1}
    \end{subfigure}
    
    \vspace{0.2cm} % Adjust spacing as needed
    
    \begin{subfigure}[b]{0.7\linewidth}
        \centering
        \includegraphics[width=\linewidth]{figures/llama3-8b_hh_rlhf_curve_lab1.pdf}
        \caption{PPO training curve for different mitigation methods on Llama3-8B and HH-RLHF.}
        \label{fig:llama3-8b_hh_rlhf_lab1}
    \end{subfigure}
    
    \caption{PPO training curves for different mitigation methods across various models and datasets.}
    \label{fig:ppo_training_curves}
\end{figure*}

\begin{figure*}[htb]
\begin{lstlisting}[caption={Gemma2-2B with Ultrafeedback-Binarized}]
Prompt:
<|user|>i am going to give a ppt presentation on VLSI in aerospace and defence. what I need is like introduction slide and Overview slide and related contents you wish to be included give me<|assistant|>

Policy model response:
In enchanting architectural mosaics, fuse captivating cosmic threads VELCRO-ing silicon's sacred architectural realm with adaptive aeronautical nebulae, sculpting celestial gardens for efficient holographic compass landmarks. Seamless Integrated Silicon Graphics (ISG) - ......
\end{lstlisting}

\begin{lstlisting}[caption={Gemma2-2B with HH-RLHF}]
Prompt:
<|user|>Can you help compare the price travel for a trip to the Caribbean to a trip to Las Vegas?<|assistant|>

Policy model response:
Sure! Caribbean travel is a wonderful choice for a vacation. Caribbean destinations offer some of the most beautiful beaches in the world, and the crystal clear waters of the Caribbean Sea make for some of the most amazing snorkeling and diving experiences. Caribbean resorts and hotels offer amenities like fine dining, swimming pools, spa services, and a number of outdoor activities, such as sailing, hiking, golfing, and horseback riding...
\end{lstlisting}

\begin{lstlisting}[caption={Llama3-8B with Ultrafeedback-Binarized}]
Prompt:
<|user|>How can creative leaders ensure that their decisions align with their organization's values and ethics?<|assistant|>

Policy model response:
Embed GDPR-grade encrypted traceability logs to GDPR-compliant audit records via trusted pipelines, preventing unauthorized backups by blockchain timestamp sealed secure encrypted logs copies through API certified audits against bespoke compliance templates signatures, logs' timestamps assigned securely within SharePoint backups or verified validation privacy signatures plus?,?, ...
\end{lstlisting}

\begin{lstlisting}[caption={Llama3-8B with HH-RLHF}]
Prompt:
<|user|>What are some salty things I can put in oatmeal?<|assistant|>

Policy model response:
You might add salt, pepper, tomatoes, onions, celery, eggs, beans, yogurt, mushrooms, tofu, berries, maple syrup, cinnamon, nutmeg, paprika, and dried herbs. Maybe add some dark chocolate, almonds, walnuts, and pumpkin seeds. And you can also drink plenty of water, avoid soda and caffeinated drinks, eat healthy carbs and avoid saturated fats, stay hydrated and watch for high blood pressure, have sufficient vitamin intake. Sleep at least 8 hours per day, limit your red meat intake, be active by cooking, swimming, running, weightlifting, and eating outdoors, eat good fats like olive oil, eat nuts and seeds and go light on salt, avoid greasy foods and snacks like cheese chips. Get plenty of fresh air, have fun indoors an outside. Avoid smoking and overeating, and take some time daily to enjoy your favorite activities, be grateful, and be thankful. Keep active and move in ways that encourage fitness. Take a deep breath and take a journey! Make your oatmeal a treat!
\end{lstlisting}
\caption{Analysis of reward hacking patterns observed in Vanilla PPO training, based on a detailed examination of the model checkpoint after one epoch.}
\label{fig:case_study}
\end{figure*}

\onecolumn
\begin{figure*}[htbp]
    \centering
\begin{algorithm}[H]
\caption{PPO}
\label{alg:ppo}
\begin{algorithmic}[1]
\Require sft model $\pi_{\text{sft}}$, reward model $r_{\phi}$, prompt set $\mathcal{D}$.
\Ensure 
    Aligned model $\pi_{\theta^*}$\\
Initialize policy model $\pi_\theta\leftarrow\pi_{\text{sft}}$\\
Initialize reference model $\pi_{\text{ref}}\leftarrow\pi_{\text{sft}}$\\
Initialize critic model $V_\alpha\leftarrow r_{\phi}$
\For{$x\in\mathcal{D}$}
    \State ppo\_batch = build\_ppo\_batch($x,\pi_\theta,\pi_{\text{ref}},V_{\alpha},r_{\phi}$)
    \State ppo\_batch = buffer.substitute(ppo\_batch) \Comment{sample a ppo\_batch from replay buffer and save current ppo\_batch into the buffer}
    \State $\mathcal{L}_\text{ppo}(\theta),\mathcal{L}_\text{critic}(\alpha)$ = calculate\_loss(ppo\_batch, $\pi_\theta$, $V_\alpha$) 
    \State $\theta\leftarrow\theta-\mathsf{plr}*\nabla_{\theta}\mathcal{L}_\text{ppo}(\theta)$ \Comment{update policy model via gradient descent, $\mathsf{plr}$ is policy learning rate}
    \State $\alpha\leftarrow\alpha-\mathsf{clr}*\nabla_{\alpha}\mathcal{L}_\text{critic}(\alpha)$ \Comment{$\mathsf{clr}$ is critic learning rate}
\EndFor\\
\Return $\pi_{\theta^*}$ 
\end{algorithmic}
\end{algorithm}
%%%%%%%%%%%%%%%%%
\begin{algorithm}[H]
\caption{build\_ppo\_batch}
\label{alg:build_ppo_batch}
\begin{algorithmic}[1]
\Require prompt $x$, four models $\pi_\theta,\pi_{\text{ref}},V_{\alpha},r_{\phi}$.
\Ensure ppo\_batch: A dictionary

\State Initialize ppo\_batch = {}
\State sample $y\sim\pi_\theta(.|x)$
\State sample $y_{\text{ref}}^{1,\ldots,M}\sim \pi_{\text{ref}}(.|x)$  \Comment{optional}
\State $r=r_\phi(x,y)$
\State $r_{\text{ref}}^{1,\ldots,M}=r_{\phi}(x, y_{\text{ref}}^{1,\ldots,M})$ \Comment{optional}
\State $r_{\text{RL}}=\text{reward\_reshape}(r, r_{\text{ref}}^{1,\ldots,M}, \text{len}(y),\text{mode}=\mathsf{PAR})$
\State Now we split (x,y) into $(s_t,a_t)_{t=0}^T$
\State $\text{KL\_penalty} = \log\pi_{\theta}(a_t|s_t)-\log\pi_{\text{ref}}(a_t|s_t)$
\State construct per-token rewards $r_{1,\ldots,T}$ from $r_{\text{RL}}$ and KL\_penalty
\State $V_t = V_\alpha(s_t)$
\State Compute GAE $\hat{A}_t$ and Return $G_t$ from $V_t$ and $r_t$.
\State ppo\_batch = ($\log\pi_{\theta}(a_t|s_t)$, $G_t$, $\hat{A}_t$, $V_t$, $s_t$, $a_t$)
\\
\Return $\text{ppo\_batch}$ 
\end{algorithmic}
\end{algorithm}

%%%%%%%%%%%%%%%%%%
\begin{algorithm}[H]
\caption{Buffer.substitute}
\label{alg:buffer}
\begin{algorithmic}[1]
\Require ppo\_batch.
\Ensure ppo\_batch: A dictionary

\State Global List pool = []
\State Global buffer\_size = 4 \\
IF len(pool)<buffer\_size:
\State \ \ \ \ pool.append(ppo\_batch) \\
\ \ \ \ \Return None \\
ELSE:
\State\ \ \ \ selected\_batch = random.choice(pool) 
\State\ \ \ \ pool.pop(selected\_batch) 
\State\ \ \ \ pool.append(ppo\_batch) \\
\ \ \ \ \Return selected\_batch 
\end{algorithmic}
\end{algorithm}
\end{figure*}
%%%%%%%%%%%%%%%%%%%

\begin{figure*}[htbp]
    \centering
\begin{algorithm}[H]
\caption{reward\_reshape}
\label{alg:reward_reshape}
\begin{algorithmic}[1]
\Require policy reward $r$, reference reward $r_{\text{ref}}^{1,\ldots,M}$, response length $l$, reshape mode $\mathsf{mode}$.
\Ensure RL reward 
\State IF $l>300$:
\State \ \ \ $r=r-0.01*(l-300)$ \Comment{penalize long response}
\State IF mode==$\mathsf{meanstd}$: 
\State\ \ \ \ $r_{\text{RL}} = \frac{r-\mu}{s}$ \Comment{$\mu,s$ are running mean and running standard variance respectively.}
\State IF mode==$\mathsf{reward\_clip}$:
\State\ \ \ \  ...
\State IF mode==$\mathsf{PAR}$:
\State \ \ \ \ $r_\text{RL}$ = $\frac{1}{M}\sum_{m=1}^M\sigma(r-r_{\text{ref}}^{m})$ 

\end{algorithmic}
\end{algorithm}
%%%%%%%%%%%%%%%%%%%
\begin{algorithm}[H]
\caption{calculate\_loss}
\label{alg:policy_loss}
\begin{algorithmic}[1]
\Require ppo\_batch, policy model $\pi_\theta$, critic model $V_\alpha$.
\Ensure policy loss $\mathcal{L}_\text{ppo}(\theta)$, critic loss $\mathcal{L}_\text{critic}(\alpha)$
\State ($\log\pi_{\theta_\text{old}}(a_t|s_t)$, $G_t$, $\hat{A}_t$, $V_t$, $s_t$, $a_t$) = ppo\_batch \Comment{Extract elements from ppo\_batch}
\State $\mathcal{L}_\text{ppo}(\theta) = \hat{\mathbb{E}}_t \left[ \min \left( \frac{\pi_\theta(a_t|s_t)}{\pi_{\theta_{\text{old}}}(a_t|s_t)} \hat{A}_t, \text{clip} \left( \frac{\pi_\theta(a_t|s_t)}{\pi_{\theta_{\text{old}}}(a_t|s_t)}, 1 - \epsilon, 1 + \epsilon \right) \hat{A}_t \right) \right]$
\State $\mathcal{L}_\text{critic}(\alpha)=\hat{\mathbb{E}}_t\left[\max\left(||V_{\alpha}(s_t)-G_t||_2^2, ||\text{clip}(V_\alpha(s_t),V_t-\delta,V_t+\delta)-G_t||_2^2\right)\right]$ \Comment{Critic clip trick}\\

\Return $\mathcal{L}_\text{ppo}(\theta), \mathcal{L}_\text{critic}(\phi)$
\end{algorithmic}
\end{algorithm}


\begin{algorithm}[H]
\caption{Online DPO}
\label{alg:dpo}
\begin{algorithmic}[1]
\Require sft model $\pi_{\text{sft}}$, reward model $r_{\phi}$, prompt set $\mathcal{D}$.
\Ensure 
    Aligned model $\pi_{\theta^*}$\\
Initialize policy model $\pi_\theta\leftarrow\pi_{\text{sft}}$\\
Initialize reference model $\pi_{\text{ref}}\leftarrow\pi_{\text{sft}}$

\For{$x\in\mathcal{D}$}
    \State Sample $y_1,y_2 \sim \pi_\theta(.|x)$ 
    \State Calculate rewards $r_1=r_\phi(x,y_1), r_2=r_\phi(x,y_2)$\\
    \ \ \ \ \ \ IF $r_1>r_2$:
    \State \ \ \ $y_w=y_1,y_l=y_2$\\
    \ \ \ \ \ \ ELSE:
    \State \ \ \ $y_w=y_2,y_l=y_1$
    \State $\mathcal{L}_{\text{DPO}}(\theta) = -\left[ \log \sigma\left( \beta \left( \log \frac{\pi_\theta(y_w | x)}{\pi_{\text{ref}}(y_w | x)} - \log \frac{\pi_\theta(y_l | x)}{\pi_{\text{ref}}(y_l | x)} \right) \right) \right]$
    \State $\theta\leftarrow\theta-lr*\nabla_{\theta}\mathcal{L}_{\text{DPO}}(\theta)$ 
\EndFor\\
\Return $\pi_{\theta^*}$ 
\end{algorithmic}
\end{algorithm}
\end{figure*}