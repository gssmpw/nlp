Reinforcement Learning from Human Feedback (RLHF) is essential for aligning large language models (LLMs) with human values. However, RLHF is susceptible to \emph{reward hacking}, where the agent exploits flaws in the reward function rather than learning the intended behavior, thus degrading alignment. While reward shaping helps stabilize RLHF and partially mitigate reward hacking, a systematic investigation into shaping techniques and their underlying principles remains lacking. To bridge this gap, we present a comprehensive study of the prevalent reward shaping methods. Our analysis suggests three key design principles: (1) RL reward is ideally bounded, (2) RL benefits from rapid initial growth followed by gradual convergence, and (3) RL reward is best formulated as a function of centered reward. Guided by these insights, we propose Preference As Reward (PAR), a novel approach that leverages the latent preferences embedded within the reward model itself as the signal for reinforcement learning. We evaluated PAR on two base models, Gemma2-2B and Llama3-8B, using two datasets, Ultrafeedback-Binarized and HH-RLHF. Experimental results demonstrate PAR's superior performance over other reward shaping methods. On the AlpacaEval 2.0 benchmark, PAR achieves a win rate at least 5 percentage points higher than competing approaches. Furthermore, PAR exhibits remarkable data efficiency, requiring only a single reference reward for optimal performance, and maintains robustness against reward hacking even after two full epochs of training. Code is available at \url{https://github.com/PorUna-byte/PAR}.\footnote{Work done during internship at StepFun by Jiayi Fu.}