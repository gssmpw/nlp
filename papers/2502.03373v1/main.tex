%%%%%%%% ICML 2025 EXAMPLE LATEX SUBMISSION FILE %%%%%%%%%%%%%%%%%
\UseRawInputEncoding
\documentclass{article}

% Recommended, but optional, packages for figures and better typesetting:
\usepackage{microtype}
\usepackage{graphicx}
\usepackage{subfigure}
\usepackage{booktabs} % for professional tables
\usepackage{makecell}
\usepackage{listings}
\usepackage{xcolor}  % Optional, for syntax highlighting
\lstset{
  breaklines=true,         % Enable automatic line breaking
  basicstyle=\small\ttfamily, % Adjust the font size and type
  escapeinside={||},       % Define escape characters for LaTeX commands
  % Any other options you need
}
% \usepackage{fancyvrb}
% \usepackage{minted}
% % Set options for minted:
% \setminted{
%   breaklines=true,   % Enable automatic line breaking
%   fontsize=\small,   % Adjust the font size if needed
%   escapeinside=||,
%   % Any other options you need
% }
\usepackage{mdframed}


\usepackage{listings}
% Rename "Listing: " to "Extract: "
\renewcommand{\lstlistingname}{Extract}

% hyperref makes hyperlinks in the resulting PDF.
% If your build breaks (sometimes temporarily if a hyperlink spans a page)
% please comment out the following usepackage line and replace
% \usepackage{icml2025} with \usepackage[nohyperref]{icml2025} above.
\usepackage{hyperref}


% Attempt to make hyperref and algorithmic work together better:
\newcommand{\theHalgorithm}{\arabic{algorithm}}

% Use the following line for the initial blind version submitted for review:
% \usepackage{icml2025}

\usepackage[preprint]{icml2025}

% If accepted, instead use the following line for the camera-ready submission:
% \usepackage[accepted]{icml2025}

% For theorems and such
\usepackage{amsmath}
\usepackage{amssymb}
\usepackage{mathtools}
\usepackage{amsthm}

% if you use cleveref..
\usepackage[capitalize,noabbrev]{cleveref}

%%%%%%%%%%%%%%%%%%%%%%%%%%%%%%%%
% THEOREMS
%%%%%%%%%%%%%%%%%%%%%%%%%%%%%%%%
\theoremstyle{plain}
\newtheorem{theorem}{Theorem}[section]
\newtheorem{proposition}[theorem]{Proposition}
\newtheorem{lemma}[theorem]{Lemma}
\newtheorem{corollary}[theorem]{Corollary}
\theoremstyle{definition}
\newtheorem{definition}[theorem]{Definition}
\newtheorem{assumption}[theorem]{Assumption}
\theoremstyle{remark}
\newtheorem{remark}[theorem]{Remark}

% Todonotes is useful during development; simply uncomment the next line
%    and comment out the line below the next line to turn off comments
%\usepackage[disable,textsize=tiny]{todonotes}
\usepackage[textsize=tiny]{todonotes}

\usepackage{multirow}
\usepackage{multicol}
\usepackage{enumitem}
% c.f. Efficient Online Reinforcement Learning Fine-Tuning Need Not Retain Offline Data
\usepackage[most,skins,theorems]{tcolorbox}
\tcbset{
  aibox/.style={
    width=\linewidth,
    top=8pt,
    bottom=4pt,
    colback=blue!6!white,
    colframe=black,
    colbacktitle=black,
    enhanced,
    center,
    attach boxed title to top left={yshift=-0.1in,xshift=0.15in},
    boxed title style={boxrule=0pt,colframe=white,},
  }
}

\newtcolorbox{AIbox}[2][]{aibox,title=#2,#1}

% The \icmltitle you define below is probably too long as a header.
% Therefore, a short form for the running title is supplied here:
% \icmltitlerunning{Submission and Formatting Instructions for ICML 2025}


\begin{document}

\twocolumn[
\icmltitle{Demystifying Long Chain-of-Thought Reasoning in LLMs}

% It is OKAY to include author information, even for blind
% submissions: the style file will automatically remove it for you
% unless you've provided the [accepted] option to the icml2025
% package.

\icmlsetsymbol{equal}{*}

\begin{icmlauthorlist}
\icmlauthor{Edward Yeo}{equal,inai}
\icmlauthor{Yuxuan Tong}{equal,thu}
\icmlauthor{Morry Niu}{inai}
\icmlauthor{Graham Neubig}{cmu}
\icmlauthor{Xiang Yue}{equal,cmu}
%\icmlauthor{}{sch}
%\icmlauthor{}{sch}
\end{icmlauthorlist}

\icmlaffiliation{cmu}{Carnegie Mellon University}
\icmlaffiliation{thu}{Tsinghua University. Work started when interning at CMU.}
\icmlaffiliation{inai}{IN.AI}

\icmlcorrespondingauthor{Xiang Yue}{xyue2@andrew.cmu.edu}

% You may provide any keywords that you
% find helpful for describing your paper; these are used to populate
% the "keywords" metadata in the PDF but will not be shown in the document
\icmlkeywords{Machine Learning, ICML}

\vskip 0.3in
]

% this must go after the closing bracket ] following \twocolumn[ ...

% This command actually creates the footnote in the first column
% listing the affiliations and the copyright notice.
% The command takes one argument, which is text to display at the start of the footnote.
% The \icmlEqualContribution command is standard text for equal contribution.
% Remove it (just {}) if you do not need this facility.

%\printAffiliationsAndNotice{}  % leave blank if no need to mention equal contribution
\printAffiliationsAndNotice{\icmlEqualContribution} % otherwise use the standard text.

\begin{abstract}
Scaling inference compute enhances reasoning in large language models (LLMs), with long chains-of-thought (CoTs) enabling strategies like backtracking and error correction. Reinforcement learning (RL) has emerged as a crucial method for developing these capabilities, yet the conditions under which long CoTs emerge remain unclear, and RL training requires careful design choices. In this study, we systematically investigate the mechanics of long CoT reasoning, identifying the key factors that enable models to generate long CoT trajectories. Through extensive supervised fine-tuning (SFT) and RL experiments, we present four main findings: (1) While SFT is not strictly necessary, it simplifies training and improves efficiency; (2) Reasoning capabilities tend to emerge with increased training compute, but their development is not guaranteed, making reward shaping crucial for stabilizing CoT length growth; (3) Scaling verifiable reward signals is critical for RL. We find that leveraging noisy, web-extracted solutions with filtering mechanisms shows strong potential, particularly for out-of-distribution (OOD) tasks such as STEM reasoning; and (4) Core abilities like error correction are inherently present in base models, but incentivizing these skills effectively for complex tasks via RL demands significant compute, and measuring their emergence requires a nuanced approach. These insights provide practical guidance for optimizing training strategies to enhance long CoT reasoning in LLMs. Our code is available at: \href{https://github.com/eddycmu/demystify-long-cot}{https://github.com/eddycmu/demystify-long-cot}.
\end{abstract}


%具身智能体在复杂场景下 manipulation 的 performance robustness 和泛化能力始终是一个广受关注的研究方向。其中,visuomotor imitation learning 是具身智能体 Policy 的主流范式之一,它允许 agent 从高维视觉观察和机器人本体感知中 effectively 学习 manipulation skills。
%然而,增加场景的复杂度和 visual distraction,会导致在简单场景下表现良好的决策模型性能下降。实际上,不仅是 simple imitation learning policy,先进的多模态 foundation models such as GPT-4o 或 vision language action models (VLA),也不能很好地关注一张语义丰富的图片中的特定的局部问题。对于 robot control or 多模态大模型,其往往侧重于 action prediction, observation mapping or 多模态 alignment,而缺少直观的视觉感知增强。模型需要隐性地或遵循 high-level text instruction 从相关的视觉区域中获得面向任务语义的定位知识。
%To tackle this challenge problem, we introduce Imit Diff, a diffusion transformer imitation learning framework with dual resolution enhancement guided by fine-grained semantics information。具体来说,our work 有三个关键组成部分。
%1) Semanstic Injection. Imit Diff 通过 vision language models (VLM) 和 vision foundation models 的 pretrain knowledge 将面向任务的语义信息和高层文本指导转化为显式的 pixel-level 视觉定位标签,注入到 environment observation中。
%2) Dual Res Fusion。 我们构建了双分辨率图像观测流,使用双分辨率视觉编码器分别提取全局和细粒度视觉特征。多尺度视觉信息随后在 attention block 中进行融合,在保证计算 effiency 的前提下,为全局视觉观测引入多尺度细粒度信息,提升场景理解能力。
%3) Consistency policy on diffusion transformer。Diffusion based imitation policies 通常受到 denoise times 的困扰。我们建立了基于 consistency policy 的 DiT action head。Policy 的决策层可以通过 single step denoise 实现系统高频响应。额外地,受益于较快的 inference time,我们引入 temperal ensemble 改善预测动作的平滑性。
%我们设计了四个在 manipulation 精细度上具有挑战性的现实世界任务来评估 Imit Diff,并通过增加场景复杂度和 visual distraction 来测试模型的场景理解能力。额外地,我们设计了 visual distraction 和 category generalization 的 zero shot 实验来验证模型是否受益于 dual res enhancement framework and fine-grained semantics injection。实验结果表明,Imit Diff outperforms 现有的 strong baselines。
%In summary, the contributions of our work are three-fold:
%1) We propose Imit Diff, a DiT architecture imitation learning framework with dual res enhancement guied by fine-grained semantics information.
%2) 我们构建了 open-set vision foundation models pipeline 来获得显式视觉遮罩。该方法能够有效处理机器人控制场景的运动模糊、遮挡、物体丢失情况。并将其作为 fine-grained 语义信息引导 policy decision。
%3) 我们在DiT上实现了consistency policy,显著减少了模型推理时间。通过异步控制框架,实现了 open-set vision foundation models 工作流下的实时控制。
%The code will be publicly available soon。

\section{Introduction}


\label{Intro}
The performance robustness and generalization capabilities of embodied agents in complex manipulation scenarios have long been a focus of significant research interest \citep{ju2025robo, yuan2024learning}. Visuomotor imitation learning is one of the mainstream paradigms of robot manipulation policy \citep{chi2023diffusion, shridhar2023perceiver, ze2023gnfactor, florence2022implicit, hansen2022pre}. This approach enables agents to derive state estimation and decision-making capabilities from expert demonstrations that incorporate high-dimensional visual observations and robot proprioception \citep{ze20243d}.

However, as scene complexity and visual distractions increase, the performance of decision models that excel in simpler environments tends to degrade \citep{zheng2024instruction, liurobustness}. Not only do simple imitation learning policies face challenges, but even advanced multimodal foundation models, such as GPT-4o \citep{hurst2024gpt} or vision language action models (VLA) \citep{liu2024rdt, brohan2022rt, brohan2023rt, o2023open, kim2024openvla, wen2024diffusion}, struggle to accurately focus on specific details within semantically complex images. In fact, in robot control and embodied multimodal foundation models, the focus is often on action prediction, observation mapping, or multimodal alignment. Therefore, intuitive visual perception enhancement is typically lacking. Models can only acquire task-oriented semantic localization knowledge from relevant visual regions either implicitly or when guided by high-level text instructions \citep{reuss2023multimodal}.

To tackle this challenge problem, we introduce \textbf{Imit Diff}, a diffusion transformer imitation learning framework with dual resolution enhancement guided by fine-grained semantics information. Specifically, our work has three key components:

\begin{enumerate}

\item \textbf{Semanstic injection.} Imit Diff transforms task-oriented semantic information and high-level textual guidance into explicit pixel-level visual localization labels through the pretrain knowledge of vision language models (VLM) and vision foundation models, and injects them into the policy observation.

\item \textbf{Dual resolution (dual res) fusion.} We develop a dual res image observation stream and employed a dual res vision encoder to extract global and fine-grained visual features. The extracted multi-scale visual information is subsequently fused within an attention block, integrating fine-grained details into the global visual feature. This approach enhances scene understanding while maintaining computational efficiency.

\item \textbf{Consistency policy on diffusion transformer (DiT).} Diffusion-based imitation policies often suffer from inefficiencies due to the required denoising steps. To address this, we design a DiT \citep{peebles2023scalable} action head incorporating a consistency policy \citep{song2023consistency}, enabling the decision layer to achieve high-frequency system responses through single-step denoising. Furthermore, leveraging faster inference times, we introduce temperal ensemble to enhance the smoothness of predicted actions.

\end{enumerate}

We design four real-world tasks with challenging manipulation precision to evaluate Imit Diff and test the model's scene understanding capabilities by introducing increased scene complexity and visual distractions. Additionally, we conducted zero-shot experiments on visual distraction and category generalization to assess the benefits of the dual res enhancement framework and fine-grained semantic injection. Experimental results demonstrate that Imit Diff significantly outperforms existing strong baselines. 

In summary, the contributions of our work are three-fold:

\begin{enumerate}

\item We propose Imit Diff, a DiT architecture imitation learning framework with dual res enhancement guied by fine-grained semantics information.

\item We developed an open-set vision foundation model pipeline to generate explicit visual masks. This approach effectively addresses challenges such as motion blur, occlusion, and object loss in robot control scenarios, leveraging the generated masks as fine-grained semantic information to guide policy decisions.

\item We implemented a consistency policy on DiT, which significantly reduced the model inference time. Through the asynchronous control framework, we achieved real-time control under the workflow of open-set vision foundation models.

\end{enumerate}

The code will be made publicly available soon.
\section{Preliminaries}
\label{sec:problem-formulation}

\begin{figure*}
    \centering
    \setlength{\abovecaptionskip}{-0.05cm}
    \setlength{\belowcaptionskip}{-0.2cm}
    \includegraphics[width=1\linewidth]{Figure/overview-all-comic.png}
    \caption{ArchRAG consists of two phases: offline indexing and online retrieval. For the online retrieval phase, we show an example of using ArchRAG to answer a question in the HotpotQA dataset.}
    % ArchRAG retrieves the relevant communities and entities from the index and extracts the answer through adaptive filtering-based generation.}
    \label{fig:overview}
\end{figure*}


%Graph-based RAG typically leverages external structured knowledge graphs to enhance the contextual understanding of LLMs and produce more informed responses.

Existing graph-based RAG methods often follow a general framework that leverages external structured knowledge graphs to improve contextual understanding of LLMs and generate more informed responses \cite{peng2024graph}.
% 
Typically, it involves two phases: 
\begin{enumerate}
    \item {\bf Offline indexing:} Building a KG $G(V,E)$ from a given corpus $D$ where each vertex represents an entity and each edge denotes the relationship between two entities, and constructing an index based on the KG.

    \item {\bf Online retrieval:} Retrieving the relevant information (e.g., nodes, subgraphs, or textual information) from KG using the index and providing the retrieved information to the LLM to improve the accuracy of the response.
    
    %\item {\bf KG construction}: Building a KG $G(V,E)$ from a given corpus $D$;
    %\item {\bf Element retrieval}: Retrieving relevant graph elements (e.g., nodes, subgraphs, or textual information) from KG using the graph structure;
    %\item {\bf Enhanced generation:} Providing the retrieved information to the LLM to improve the accuracy of the response.
\end{enumerate}

% Graph-based RAG typically leverages external structured knowledge graphs to enhance the contextual understanding of LLMs and produce more informed responses.
% 
% It involves three key steps: (1) Knowledge Graph (KG) Construction, (2) Element Retrieval, and (3) Enhanced Generation. 
% 
%The constructed KG, also referred to as a Text-Attributed Graph (TAG), records the semantic relationships distributed throughout the corpus, with each node and relation containing textual descriptions. Formally, we define a TAG as follows:

%\begin{definition}[Text-Attributed Graph (TAG) \cite{he2024g}]
%A Text-Attributed Graph (TAG) is a directed graph \(G = (V, E, \psi, \phi)\), where \(V\) and \(E\) represent the sets of nodes and edges, respectively. Each node \(v \in V\) and edge \(e \in E\) is associated with textual attributes \(\psi(v)\) and \(\phi(e)\), respectively.
%\end{definition}

%To avoid ambiguity, we use ``node" to refer to vertices in the graph and ``entity" to refer to vertices with textual attributes in KG.

% Different methods retrieve varying elements, such as relevant entities \cite{munikoti2023atlantic,wang2024knowledge}, subgraphs\cite{li2024dalk}, paths \cite{sun2023think,luo2023reasoning}, connected components in the graph \cite{he2024g,mavromatis2024gnn}, or even original chunk text \cite{gutierrez2024hipporag} and additional information \cite{edge2024local,sarthi2024raptor} linked to the knowledge graph. 
% Different methods retrieve varying elements, such as relevant entities \cite{munikoti2023atlantic,wang2024knowledge}, subgraphs\cite{li2024dalk}, paths \cite{sun2023think,luo2023reasoning}, and e.t.c. \cite{he2024g,mavromatis2024gnn,gutierrez2024hipporag,edge2024local,sarthi2024raptor}.
% % 
% Among these, using language models (i.e., embedding models) to compute text relevance is the primary method for retrieval.

% \begin{definition}[Language Models (LMs)]
%     A language model (LM) is a probabilistic model of a sequence of words.
%     % 
%     Given a text with $n$ words $x \in D^{n}$, an LM encodes this sequence of words as:

%     \begin{equation}
%         z_n = LM(x) \in \mathbb{R}^d
%     \end{equation}

%     where $z_n$ is the output vector of the LM, i.e., the embedding, and $d$ is the dimension of the output vector.
% \end{definition}

% % Language models, also known as embedding models, play a crucial role in encoding textual content.
% % % 
% % The resulting vectors capture the semantic meaning and key features of the data. 
% % %  
% % Semantically similar texts are positioned closer together in the vector space, while dissimilar texts are placed farther apart.
% % 

% In a RAG system, embedding vectors are typically stored in a vector database, enabling the efficient retrieval of relevant data during the query phase. 
% % 
% To facilitate this, Approximate Nearest Neighbor Search (ANNS) can be employed to quickly identify the closest data points to a given query, even if they are not exact neighbors.
% % 
% Next, we will introduce an efficient ANNS technique for vector databases.

% \begin{definition}[Hierarchical Navigable Small World (HNSW) \cite{malkov2018efficient}]
%     HNSW is a graph-based ANNS algorithm that consists of a multi-layered index structure, where each node uniquely corresponds to a vector in the database.
%     % 
%     Given a set $S$ containing $n$ vectors, the constructed HNSW can be represented as a pair $\mathcal{H} = (\mathcal{\bf G}, \mathcal{C})$. 
%     $\mathcal{\bf G}=\{G_0, G_1, \dots, G_L\}$ is a set of simple graphs (also called layers) $G_i=(V_i, E_i)$, where $i \in \{0,1,\dots, L\}$ and $V_L \subset V_{L-1} \subset \dots \subset V_1 \subset V_0 = S$.
%     $\mathcal{C}$ records the inter-layer mappings of edges between the same node across adjacent layers $\mathcal{C} = \bigcup_{i=0}^{L-1} \{(v,\phi(v))|v \in V_i,\phi(v)\in V_{i+1} \}$, where $\phi(v):V_i \rightarrow V_{i+1}$ is the mapping function for the same node across two adjacent layers.
% \end{definition}

% The nodes in the multi-layer graph of HNSW are organized in a nested structure, where each node at each layer is connected to its nearest neighbors. 
% % 
% During a query, the search begins at the top layer and quickly identifies the node closest to $q$ through a greedy search.
% % 
% Then, through inter-layer mapping, the search proceeds to the next lower layer.
% % 
% This process continues until all approximate nearest neighbors are identified in $G_0$.


% \subsection{Problem formulation}

% We aim to enhance the quality and factuality of an LLM through the retrieval and analysis of relevant information from a large set of documents.
% % 
% Formally, the Graph-based RAG problem is defined as follows:


% \begin{problem}[Graph-based RAG]
% \label{prob:graphrag}
% Given a knowledge base $\mathcal{B}$ from a large text corpus or a large set of text documents, Graph RAG constructs a TAG $G$ from $\mathcal{B}$.
% % 
% Given a query $q$ related to $\mathcal{B}$, Graph-based RAG leverages relevant graph components from $G$ and/or pertinent data from $\mathcal{B}$ to enhance the accuracy and factuality of the LLM's responses.
% \end{problem}










% \begin{definition}[Text-Attributed Graph (TAG) \cite{he2024g}]
%     Formally, a Text-Attributed Graph (TAG) is a graph where nodes and edges possess textual attributes.
%     % 
%     Formally, it can be defined as $\mathcal{G} = (V, E, \{x_n\}_{n \in V},\\ \{x_e\}_{e \in E})$, where $V$ and $E$ represent the sets of nodes and edges, respectively. 
%     % 
%     Additionally, $x_n \in D^{L_n}$  and $x_e \in D^{L_e}$ denote sequential text associated with a node $n \in V$ or an edge $e \in E$, where $D$ represents the vocabulary, and $L_n$ and $L_e$ signify the length of the text associated with the respective node or edge.
% \end{definition}

% A TAG incorporates external data to enhance LLM with additional knowledge, forming the foundation for GraphRAG tasks. 
% % 
% In the real world, knowledge graphs and document graphs are examples of TAGs, which can be categorized into the following two types:
% % 
% (1) existing general open KGs collaboratively built by users and experts, such as WikiData \footnote{https://www.wikidata.org/wiki}, DBPeida \footnote{https://www.dbpedia.org/},
% % 
% and (2) self-constructed graphs generated from a given document by extracting key entities and relationships or by identifying relationships within segmented text chunks.
% % 
% The former type of TAG can be directly obtained from open-source data, while the latter can be constructed using methods such as OpenIE \cite{pei2022use} and document graph construction techniques \cite{ferragina2010tagme, Rincon-Yanez2022,sarthi2024raptor,baek2023knowledge,gutierrez2024hipporag}.

% In this paper, we focus on the latter type of TAG because most domain-specific data lacks well-constructed TAGs. 
% 
% Additionally, we address the most widely studied area in RAG, namely, text-based RAG \cite{gao2023retrieval}.


\begin{figure*}[t!]
    \centering
    \includegraphics[width=0.96\linewidth]{figs/viz-cot-scaling-direct.pdf}
    \vspace{-6pt}
    \caption{Scaling curves of SFT and RL on \texttt{Llama-3.1-8B}  with long CoTs and short CoTs. SFT with long CoTs can scale up to a higher upper limit and has more potential to further improve with RL.}
        \vspace{-10pt}

    \label{fig:cot-scaling}
\end{figure*}

\section{Impact of SFT on Long CoT}

In this section, we compare long and short CoT data for SFT and in the context of RL initialization.

\subsection{SFT Scaling}\label{subsec:sft-scaling}

To compare long CoT with short CoT, the first step is to equip the model with the corresponding behavior. The most straightforward approach is to fine-tune the base model on CoT data. Since short CoT is common, curating SFT data for it is relatively simple via rejection sampling from existing models. However, how to obtain high-quality long CoT data remains an open question. 
% We begin by distilling from the open-weight \texttt{QwQ-32B-Preview}  \citep{qwen2024qwq} because it is cheaper and produce better performance, which will be further discussed in \textsection\ref{sec:long-cot-pattern}.

\noindent\textbf{Setup.} To curate the SFT data, for long CoT, we distill from \texttt{QwQ-32B-Preview} (we discuss other long CoT data construction methods in \textsection\ref{sec:long-cot-pattern}). For short CoT, we distill from \texttt{Qwen2.5-Math-72B-Instruct}, which is a capable short CoT model in math reasoning. Specifically, we perform rejection sampling by first sampling $N$ candidate responses per prompt and then filtering for ones with correct answers. For long CoT, we use $N \in \{32, 64, 128, 192, 256\}$, while for short CoT, we use $N \in \{32, 64, 128, 256\}$, skipping one $N$ for efficiency. 
% The SFT datasets with smaller $N$'s are subsets of ones with larger $N$'s. 
In each case, the number of SFT tokens is proportional to $N$. We use the base model \texttt{Llama-3.1-8B} \citep{meta2023llama3}. Please refer to Appendix \ref{app:sft-setup} for more details about the SFT setup.

\noindent\textbf{Result.} The dashed lines in Figure \ref{fig:cot-scaling} illustrate that as we scale up the SFT tokens, long CoT SFT continues to improve model accuracy, whereas short CoT SFT saturates early at a lower accuracy level. For instance, on MATH-500, long CoT SFT achieves over 70\% accuracy and has yet to plateau even at 3.5B tokens. In contrast, short CoT SFT converges below 55\% accuracy, with an increase in SFT tokens from approximately 0.25B to 1.5B yielding only a marginal absolute improvement of about 3\%.

\begin{AIbox}{Takeaway 3.1 for SFT Scaling Upper Limit}
SFT with long CoT can scale up to a higher performance upper limit than short CoT. (Figure \ref{fig:cot-scaling})
\end{AIbox}

\begin{figure*}[!t]
    \centering
    \includegraphics[width=1\linewidth]{figs/qwen_llama_row.pdf}
    \vspace{-20pt}
    \caption{Both \texttt{Llama3.1-8B} and \texttt{Qwen2.5-Math-7B} models trained under RL with the Classic Reward manifested emergent CoT length scaling past the context window size, resulting in a decline in MATH-500 accuracy. The red points on the charts correspond to the iteration where the accuracy dropped to near zero. ``Terminated CoTs'' refer to responses that conclude within the context length.}
        \vspace{-10pt}

    \label{fig:reward-qwen-llama}
\end{figure*}

\subsection{SFT Initialization for RL}\label{sec:sft-init-for-rl}

Since RL is reported to have a higher upper limit than SFT, we compare long CoT and short CoT as different SFT initialization approaches for RL.

\noindent\textbf{Setup.}
We initialize RL using SFT checkpoints from \textsection\ref{subsec:sft-scaling}, and train for four epochs, sampling four responses per prompt. Our approach employs PPO \citep{schulman2017ppo} with a rule-based verifier from the MATH dataset, using its training split as our RL prompt set.
We adopt our cosine length scaling reward with the repetition penalty, which will be detailed in \textsection\ref{sec:reward-design}.
% The reward function follows a simple binary scheme: 1 if the answer is correct, 0 otherwise. We refer to this straightforward correctness-based reward as the Classic Reward. 
Further details about our RL setup and hyperparameters can be found in Appendix \ref{app:rl-setup} \& \ref{app:exp-hyperparams-sft-init-for-rl} respectively.

\noindent\textbf{Result.} The gap between solid and dashed lines in Figure \ref{fig:cot-scaling} shows that models initialized with long CoT SFT can usually be further significantly improved by RL, while models initialized with short CoT SFT see little gains from RL. For example, on MATH-500, RL can improve long CoT SFT models by over 3\% absolute, while short CoT SFT models have almost the same accuracies before and after RL.

\begin{AIbox}{Takeaway 3.2 for SFT Initialization for RL}
SFT with long CoTs makes further RL improvement easier, while short CoTs do not. (Figure \ref{fig:cot-scaling})
\end{AIbox}

\subsection{Sources of Long CoT SFT Data}\label{sec:long-cot-pattern}

To curate long CoT data, we compare two approaches: (1) \textbf{Construct} long CoT trajectories by prompting short CoT models to generate primitive actions and sequentially combining them; (2) \textbf{Distill} long CoT trajectories from existing long CoT models that exhibit emergent long CoT patterns.

\noindent\textbf{Setup.} To construct long CoT trajectories, we developed an Action Prompting framework (Appendix \ref{app:action-prompting}) which defined the following primitive actions: \texttt{clarify}, \texttt{decompose}, \texttt{solution\_step}, \texttt{reflection}, and \texttt{answer}. We employed multi-step prompting with a short CoT model (e.g., \texttt{Qwen2.5-72B-Instruct}) to sequence these actions, while a stronger model, \texttt{o1-mini-0912}, generates reflection steps incorporating self-correction. For distilling long CoT trajectories, we use \texttt{QwQ-32-Preview} as the teacher model. In both approaches, we adopt the MATH training set as the prompt set and apply rejection sampling. To ensure fairness, we use the same base model (\texttt{Llama-3.1-8B}), maintain approximately 200k SFT samples, and use the same RL setup as in \textsection\ref{sec:sft-init-for-rl}.

\noindent\textbf{Result.} Table \ref{tab:constructed-underperforms-emergent} shows that the model distilled from emergent long CoT patterns generalizes better than the constructed pattern, and can be further significantly improved with RL, while the model trained on constructed patterns cannot. Models trained with the emergent long CoT pattern achieve significantly higher accuracies on OOD benchmarks AIME 2024 and MMLU-Pro-1k, improving by 15-50\% relatively. 
Besides, on the OOD benchmark TheoremQA, RL on the long CoT SFT model significantly improves its accuracy by around 20\% relative, while the short CoT model's performance does not change. 
This is also why we conduct most of our experiments based on distilled long CoT trajectories.

\begin{AIbox}{Takeaway 3.3 for Long CoT Cold Start}
SFT initialization matters: high-quality, emergent long CoT patterns lead to significantly better generalization and RL gains. (Table \ref{tab:constructed-underperforms-emergent})
\end{AIbox}

\begin{table}[htbp]
\vspace{-15pt}
\caption{Emergent long CoT patterns outperform constructed ones. All the models here are fine-tuned from the base model \texttt{Llama-3.1-8B} with the MATH training prompt set.
% The SFT stages use data of different long CoT patterns, 
% while the RL stages use exactly the same setup.
}
\label{tab:constructed-underperforms-emergent}
\vskip 0.1in
\centering
\small
\begin{tabular}{@{}llcccc@{}}
\toprule
Training & Long CoT & MATH & AIME & Theo. & MMLU \\
Method & SFT Pattern & 500 & 2024 & QA & Pro-1k \\
\midrule
\multirow{2}{*}{SFT} & Constructed & 48.2 & 2.9 & 21.0 & 18.1 \\
& Emergent & \textbf{54.1} & \textbf{3.5} & \textbf{21.8} &\textbf{ 32.0} \\
\midrule
\multirow{2}{*}{SFT+RL} & Constructed & 52.4 & 2.7 & 21.0 & 19.2 \\
& Emergent & \textbf{59.4} & \textbf{4.0} & \textbf{25.2} & \textbf{34.6} \\
\bottomrule
\end{tabular}
\end{table}

% \subsection{RL from Base v.s RL from SFT Initialization}





\section{Impact of Reward Design on Long CoT}\label{sec:reward-design}

This section examines reward function design, with a focus on its influence on CoT length and model performance.

\subsection{CoT Length Stability}
\label{result:reward-length-stability}

Recent studies on long CoT \cite{deepseekai2025r1, kimi2025k15, hou2025advancinglanguagemodelreasoning} suggest that models naturally improve in reasoning tasks with increased thinking time. Our experiments confirm that models fine-tuned on long CoT distilled from \texttt{QwQ-32B-Preview} tend to extend CoT length under RL training, albeit sometimes unstably. This instability, also noted by \citet{kimi2025k15, hou2025advancinglanguagemodelreasoning}, has been addressed using techniques based on length and repetition penalties to stabilize training.

 \noindent\textbf{Setup.} We used two different models fine-tuned on long CoT data distilled from \texttt{QwQ-32B-Preview} using the MATH train split, with a context window size of 16K. The models were \texttt{Llama3.1-8B} and \texttt{Qwen2.5-Math-7B}. We used a rule-based verifier along and a simple reward of 1 for correct answers. We shall refer to this as the \textit{Classic Reward}. More details can be found in Appendix \ref{app:exp-hyperparams-reward-length-stability}.

\noindent\textbf{Results.} We observed that both models increased their CoT length during training, eventually reaching the context window limit. This led to a decline in training accuracy due to CoTs exceeding the allowable window size. Additionally, different base models exhibited distinct scaling behaviors. The weaker \texttt{Llama-3.1-8B} model showed greater fluctuations in CoT length compared to \texttt{Qwen-2.5-Math-7B}, as illustrated in Figure \ref{fig:reward-qwen-llama}.

We also found that the rate at which CoTs exceeded the context window size leveled off at a certain threshold below 1 (Figure \ref{fig:reward-qwen-llama}). This suggests that exceeding the limit started to apply significant downward pressure on the CoT length distribution, and highlights the context window size's role in implicit length penalization. Notably, a trajectory might be penalized even without an explicit exceed-length penalty due to reward or advantage normalization, both of which are standard in RL frameworks.

\begin{AIbox}{Takeaway \hypersetup{hidelinks}\ref{result:reward-length-stability} for CoT Length Stability}
CoT length does not always scale up in a stable fashion. (Figure \ref{fig:reward-qwen-llama}) 
\end{AIbox}

\begin{figure}[tbp]
    \centering
    \includegraphics[width=1\linewidth]{figs/reward_functions_row.pdf}
    \vspace{-20pt}
    \caption{The Classic and Cosine Reward functions. The Cosine Reward varies with generation length.}
    \vspace{-10pt}
    \label{fig:reward-cosine}
\end{figure}

\subsection{Active Scaling of CoT Length}
\label{result:reward-length-scaling}

We found that reward shaping can be used to stabilize emergent length scaling. We designed a reward function to use CoT length as an additional input and to observe a few ordering constraints. Firstly, correct CoTs receive higher rewards than wrong CoTs. Secondly, shorter correct CoTs receive higher rewards than longer correct CoTs, which incentivizes the model to use inference compute efficiently. Thirdly, shorter wrong CoTs should receive higher penalties than longer wrong CoTs. This encourages the model to extend its thinking time if it is less likely to get the correct answer.

We found it convenient to use a piecewise cosine function, which is easy to tune and smooth. We refer to this reward function as the \textit{Cosine Reward}, visualized in Figure \ref{fig:reward-cosine}. This is a \textit{sparse} reward, only awarded once at the end of the CoT based on the correctness of the answer. The formula of \textbf{CosFn} can be found in equation \ref{eqn:cosine-lr} in the appendix.

\vspace{-10pt}
\begin{equation*}
\small
\label{eqn:reward-cosine}
\begin{aligned}
& R(C, L_{\text{gen}}) = 
&\begin{cases} 
    \text{CosFn}(L_{\text{gen}}, L_{\text{max}}, r_0^c, r_L^c),  & \text{if } C = 1, \\
    \text{CosFn} (L_{\text{gen}}, L_{\text{max}}, r_0^w, r_L^w),  & \text{if } C = 0, \\
    r_e, &\text{if } L_{\text{gen}} = L_{\text{max}}.
\end{cases}
\end{aligned}
\end{equation*}
\vspace{-25pt}

{\small
\begin{flalign*}
&\textbf{Hyperparameters:} && \\
&\quad r_0^c / r_0^w: \text{Reward (correct/wrong) for } L_{\text{gen}} = 0, && \\
&\quad r_L^c/r_L^w: \text{Reward (correct/wrong) for } L_{\text{gen}} = L_{\text{max}}, \\
&\quad r_e: \text{Exceed length penalty}, \\
&\textbf{Inputs:} && \\
&\quad C: \text{Correctness (0 or 1)}, && \\
&\quad L_{\text{gen}}: \text{Generation length.} &&
\end{flalign*}
}

\noindent\textbf{Setup.} We ran experiments with the Classic Reward and the Cosine Reward. We used the \texttt{Llama3.1-8B} fine-tuned on long CoT data distilled from \texttt{QwQ-32B-Preview} using the MATH train split, as our starting point. For more details, see Appendix \ref{app:exp-hyperparams-reward-length-scaling}.

\noindent\textbf{Result.} We found that the Cosine Reward significantly stabilized the length scaling behavior of the models under RL, thereby also stabilizing the training accuracy and improving RL efficiency (Figure \ref{fig:reward-llama-classic}). We also observed improvements in model performance on downstream tasks (Figure \ref{fig:reward-eval}).



\begin{figure}[tbp]
    \centering
    \includegraphics[width=1\linewidth]{figs/llama_classic_acc_res_len.pdf}
    \vspace{-20pt}
    \caption{\texttt{Llama3.1-8B} trained with length shaping using the Cosine Reward exhibited more stable (a) training accuracy and (b) response length. This stability led to improved performance on downstream tasks (Figure \ref{fig:reward-eval}). Red points on the charts indicate iterations where training accuracy dropped to near zero.}
    \vspace{-10pt}
    \label{fig:reward-llama-classic}
\end{figure}

\begin{figure*}[tbp]
    \centering
    \includegraphics[width=1\linewidth]{figs/train-proc-rew-type.pdf}
    \vspace{-20pt}
    \caption{Performance of \texttt{Llama-3.1-8B} trained with different reward functions on a variety of evaluation benchmarks.}
    \vspace{-10pt}
    \label{fig:reward-eval}
\end{figure*}

\begin{AIbox}{Takeaway \hypersetup{hidelinks}\ref{result:reward-length-scaling} for Active Scaling of CoT Length}
Reward shaping can be used to stabilize and control CoT length while improving accuracy. (Figure \ref{fig:reward-llama-classic}, \ref{fig:reward-eval})
\end{AIbox}

\subsection{Cosine Reward Hyperparameters}
\label{result:reward-cosine-hyperparams}

The Cosine Reward hyperparameters can be tuned to shape CoT length in different ways.

\noindent\textbf{Setup.} We set up RL experiments with the same model fine-tuned on long CoT distilled from \texttt{QwQ-32B-Preview}, but with different hyperparameters for the Cosine Reward function. We tweaked the correct and wrong rewards $r_0^c, r_L^c, r_0^w, r_L^w$ and observed their impact on the CoT lengths. For more details, see Appendix \ref{app:exp-hyperparams-reward-cosine-hyperparams}.

\noindent\textbf{Result.} We see from Figure \ref{fig:reward-cosine-hyperparams} in the Appendix that if the reward for a correct answer increases with CoT length ($r_0^c < r_L^c$), the CoT length increases explosively. We also see that the lower the correct reward relative to the wrong reward, the longer the CoT length. We interpret this as a kind of trained risk aversion, where the ratio of the correct and wrong rewards determines how confident the model has to be about an answer for it to derive a positive expected value from terminating its CoT with an answer.

\begin{AIbox}{Takeaway \hypersetup{hidelinks}\ref{result:reward-cosine-hyperparams} for Cosine Reward Hyperparameters}
Cosine Reward can be tuned to incentivize various length scaling behaviors. (Appendix Figure \ref{fig:reward-cosine-hyperparams})
\end{AIbox}

\subsection{Context Window Size}
\label{result:reward-context-window}

We know that longer contexts give a model more room to explore, and with more training samples, the model eventually learns to utilize more of the context window. This raises an interesting question -- are more training samples necessary to learn to utilize a larger context window?

\noindent\textbf{Setup.} We set up 3 experiments using the same starting model fine-tuned on long CoT data distilled from \texttt{QwQ-32B-Preview} with the MATH train split. We also used the latter as our RL prompt set. Each ablation used the Cosine Reward and repetition penalty with a different context window size (4K, 8K, and 16K). For more details, see Appendix \ref{app:exp-hyperparams-reward-context-window}.

\noindent\textbf{Result.} We found that the model with a context window size of 8K performed better than the model with 4K, as expected. However, we observed performance was better under 8K than 16K. Note that all three experiments used the same number of training samples (Figure \ref{fig:reward-context}). We see this as an indication that models need more training compute to learn to fully utilize longer context window sizes, which is consistent with the findings of \cite{hou2025advancinglanguagemodelreasoning}.

\begin{figure*}
    \centering
    \includegraphics[width=1\linewidth]{figs/viz-ctx-len-scaling.pdf}
    \vspace{-20pt}
    \caption{Performance of \texttt{Llama-3.1-8B} trained with different context window sizes. All experiments used the same number of training samples.}
    \vspace{-10pt}
    \label{fig:reward-context}
\end{figure*}

\begin{AIbox}{Takeaway \hypersetup{hidelinks}\ref{result:reward-context-window} for Context Window Size}
Models might need more training samples to learn to utilize larger context window sizes. (Figure \ref{fig:reward-context})
\end{AIbox}

\subsection{Length Reward Hacking}
\label{result:reward-hacking}

We observed that with enough training compute, the model started to show signs of reward hacking, where it increased the lengths of its CoTs on hard questions using repetition rather than learning to solve them. We also noted a fall in the branching frequency of the model, which we estimated by counting the number of times the pivot keyword "\texttt{alternatively,}" appeared in the CoT (Figure \ref{fig:reward-cosine-branching}).

We mitigated this by implementing a simple $N$-gram repetition penalty (Algorithm \ref{alg:reward-repetition-penalty}). We observed that the penalty was most effectively applied on repeated tokens, rather than as a sparse reward for the entire trajectory. Similarly, we found that discounting the repetition penalty when calculating the return was effective. Specific feedback about where the repetition occurred presumably made it easier for the model to learn not to do it (see more in \textsection\ref{result:optimal-discount}).

\noindent\textbf{Setup.} We used the \texttt{Llama3.1-8B} model fine-tuned on long CoT data distilled from \texttt{QwQ-32B-Preview}. We ran two RL training runs, both using the Cosine Reward, but with and without the repetition penalty. For more details, please refer to Appendix \ref{app:exp-hyperparams-reward-hacking}.

\noindent\textbf{Result.} The repetition penalty resulted in better downstream task performance and also shorter CoTs, meaning there was better utilization of inference compute (Figure \ref{fig:reward-eval}).

\textbf{Observation.} Our experiments revealed a relationship between the repetition penalty, training accuracy, and the Cosine Reward. When training accuracy was low, the Cosine Reward exerted greater upward pressure on CoT length, leading to increased reward hacking through repetition. This, in turn, required a stronger repetition penalty. Future work could further investigate these interactions and explore dynamic tuning methods for better optimization.

\begin{AIbox}{Takeaway \hypersetup{hidelinks}\ref{result:reward-hacking} for Length Reward Hacking}
Length rewards will be hacked with enough compute (Figure \ref{fig:reward-cosine-branching}), but this can be mitigated using a repetition penalty. (Figure \ref{fig:reward-eval})
\end{AIbox}

\subsection{Optimal Discount Factors}
\label{result:optimal-discount}

We hypothesized that applying the repetition penalty with temporal locality (i.e., a low discount factor) would be most effective, as it provides a stronger learning signal about the specific offending tokens. However, we also observed performance degradation when the discount factor for the correctness (cosine) reward was too low.

To optimally tune both reward types, we modified the GAE formula in PPO to accommodate multiple reward types, each with its own discount factor $\gamma$: $\hat{A}_t = \sum_{l=0}^{L} \sum_{m}^{M}\gamma_{m}^l r_{m, t + l} - V(s_t)$. For simplicity, we set $\lambda = 1$, which proved effective, though we did not extensively tune this parameter.

\noindent\textbf{Setup.} We ran multiple RL experiments with the same \texttt{Llama3.1-8B} model fine-tuned on \texttt{QwQ-32B-Preview} distilled long CoT data. We used the Cosine Reward and repetition penalty but with different combinations of discount factors. For more details, please see Appendix \ref{app:exp-hyperparams-optimal-discount}.

\noindent\textbf{Result.} A lower discount factor effectively enforces the repetition penalty, whereas a higher discount factor enhances the correctness reward and the exceed-length penalty. The higher factor allows the model to be adequately rewarded for selecting a correct answer earlier in the CoT (Figure \ref{fig:multiple-gamma}).

We observed a rather interesting phenomenon where decreasing the discount factor $\gamma$ of the correctness (cosine) reward increased the branching frequency in the model's CoT, making the model quickly give up on approaches that did not seem to lead to a correct answer immediately (Figure \ref{fig:reward-indecisive}, Extract in Appendix \ref{extract:reward-short-term}). We hypothesize that this short-term thinking was due to a relatively small number of tokens preceding the correct answer receiving rewards, which means stepping stones to the right answer are undervalued. Such behavior degraded performance (Figure \ref{fig:multiple-gamma}). However, we think this qualitative result might be of potential interest to the research community, due to its similarity to the relationship between behaviors like delayed gratification and the distribution of rewards given to the biological brain \cite{doi:10.1126/sciadv.abg6611}.



\begin{AIbox}{Takeaway \hypersetup{hidelinks}\ref{result:optimal-discount} for Optimal Discount Factors}
Different kinds of rewards and penalties have different optimal discount factors. (Figure \ref{fig:multiple-gamma})
\end{AIbox}


\section{Scaling up Verifiable Reward}\label{sec:silver-data}

Verifiable reward signals like ones based on ground-truth answers are essential for stabilizing long CoT RL for reasoning tasks. However, it is difficult to scale up such data due to the limited availability of high-quality human-annotated verifiable data for reasoning tasks. As an attempt to counter this, we explore using other data that is more available despite more noise, like reasoning-related QA pairs extracted from web corpora. Specifically, we experiment with the WebInstruct dataset \citep{yue2024mammoth2}. For efficiency, we construct WebInstruct-462k, a deduplicated subset derived via MinHash \citep{broder1998minhash}. 

\subsection{SFT with Noisy Verifiable Data}\label{sec:sft-with-noisy-verifiable-data}

We first explore adding such diverse data to SFT. Intuitively, despite less reliable supervision signals, diverse data might facilitate the model’s exploration during RL.

\noindent\textbf{Setup.} We experiment with three setups, varying the proportion of data without gold supervision signals: 0\%, 100\%, and approximately 50\%. We conduct long CoT SFT by distilling from \texttt{QwQ-32B-Preview}. For data with gold supervision signals (MATH), ground truth answers are used for rejection sampling. In contrast, for data from WebInstruct without fully reliable supervision signals but with a much larger scale, we sample one response per prompt from the teacher model without filtration. For RL here, we adopt the same setup as in \textsection\ref{sec:sft-init-for-rl}, using the MATH training set.

\noindent\textbf{Result.} Table \ref{tab:diverse-silver-improve-general-reasoning} shows that incorporating silver-supervised data improves average performance. Adding WebInstruct data to long CoT SFT yields a substantial 5–10\% absolute accuracy gain on MMLU-Pro-1k over using MATH alone. Furthermore, mixing MATH and WebInstruct data achieves the best average accuracy across benchmarks.

\begin{table}[htbp]
\vspace{-10pt}
\caption{
Adding data with a silver supervision signal is often beneficial.
``WebIT'' is the abbreviation of WebInstruct.
}
\vspace{5pt}
\label{tab:diverse-silver-improve-general-reasoning}
\centering
\small
\resizebox{\linewidth}{!}{
\begin{tabular}{@{}llccccc@{}}
\toprule
Long CoT & Training & MATH & AIME & Theo. & MMLU & \multirow{2}{*}{AVG} \\
SFT Data & Method & 500 & 2024 & QA & Pro-1k \\
\midrule
% \multirow{2}{*}{\begin{tabular}[c]{@{}l@{}}MATH\\-RS-32\end{tabular}}
\multirow{2}{*}{100\% MATH}
 & SFT & 54.1 & 3.5 & 21.8 & 32.0 & 27.9 \\
 & SFT + RL & \textbf{59.4} & 4.0 & \textbf{25.2} & 34.6 & 30.8 \\
\midrule
% \multirow{2}{*}{\begin{tabular}[c]{@{}l@{}}WebIT-231k\\-Distill\end{tabular}}
\multirow{2}{*}{100\% WebIT}
& SFT & 41.2 & 0.8 & 21.9 & 41.1 & 26.3 \\
& SFT + RL & 44.6 & 1.9 & 22.5 & \textbf{43.3} & 28.1 \\
\midrule
50\% MATH & SFT & 53.6 & \textbf{4.4} & 23.5 & 41.7 & 30.8 \\
+ 50\% WebIT & SFT + RL & 57.3 & 3.8 & 25.1 & 42.0 & \textbf{32.1} \\
\bottomrule
\end{tabular}
}
\end{table}

\begin{AIbox}{Takeaway 5.1 for SFT with Noisy Verifiable Data}
Adding noisy but diverse data to SFT leads balanced performance across different tasks. (Table \ref{tab:diverse-silver-improve-general-reasoning})
\end{AIbox}

\subsection{Scaling up RL with Noisy Verifiable Data}
\label{result:reward-verify-clean}

We compare two main approaches to obtain rewards from noisy verifiable data: 1) to extract short-form answers and use a rule-based verifier; 2) to use a model-based verifier capable of processing free-form responses. 

Here a key factor is whether the QA pair can have a short-form answer. So we also compare whether the dataset is filtered by only retaining samples with short-form answers. 

\noindent\textbf{Setup.}
We implement the model-based verifier by prompting \texttt{Qwen2.5-Math-7B-Instruct} with the raw reference solution.
To extract short-form answers, we first prompt \texttt{Llama-3.1-8B-Instruct} to extract from the raw responses and then apply rejection sampling with \texttt{QwQ-32B-Preview}. Specifically, we generate two responses per prompt from WebInstruct-462k and discard cases where neither response aligns with the extracted reference answers. This process yields approximately 189k responses across 115k unique prompts.
Our case studies show that the rejection sampling drops many prompts due to:
1) many WebInstruct prompts lack short-form answers that our rule-based verifier can process effectively,
and 2) some prompts are too difficult even for \texttt{QwQ-32B-Preview}.
For SFT we train \texttt{Llama-3.1-8B} on the filtered dataset as initialization for reinforcement learning (RL).
In the RL stage, we use the full 462k prompt set in the unfiltered setup and the 115k subset in the filtered setup, training with 30k prompts and 4 responses per prompt.
Further details about the model-based verifier, the answer extraction and the RL hyperparameters can be found in Appendix  \& \ref{app:exp-hyperparams-reward-verify-clean} \& \ref{app:model-based-verifier} \& \ref{app:ans-extract} respectively.

\begin{table}[htbp]
\vspace{-15pt}
\caption{Performance of RL with different verifiers and prompt filtering methods. All the models here are fine-tuned from \texttt{Llama-3.1-8B}. The ``MATH Baseline'' is the model trained with SFT and RL on MATH only in Table \ref{tab:diverse-silver-improve-general-reasoning}. The other models are trained with SFT by distillation from \texttt{QwQ-32B-Preview} and RL with different setups.}
\label{tab:verification-types}
\vskip 0.1in
\centering
\small
\resizebox{\linewidth}{!}{
\begin{tabular}{@{}llccccc@{}}
\toprule
Prompt & Verifier & MATH & AIME & Theo. & MMLU \\
Set & Type & 500 & 2024 & QA & Pro-1k \\
\midrule
\multicolumn{2}{c}{MATH Baseline} & 59.4 & 4.0 & 25.2 & 34.6 \\
\midrule
\multicolumn{2}{c}{SFT Initialization} & 46.6 & 1.0 & 23.0 & 28.3 \\
\midrule
\multirow{2}{*}{Unfiltered} & Rule-Based & 45.4 & 3.3 & 25.9 & 35.1 \\
 & Model-Based & 47.9 & 3.5 & 26.2 & 40.4 \\
\midrule
\multirow{2}{*}{Filtered} & Rule-Based & \textbf{48.6} & 3.3 & \textbf{28.1} & \textbf{41.4} \\
 & Model-Based & 47.9 & \textbf{3.8} & 26.9 & \textbf{41.4} \\
\bottomrule
\end{tabular}
}
\end{table}

\noindent\textbf{Result.} \autoref{tab:verification-types} shows that RL with the rule-based verifier on the filtered prompt set with short-form answers achieves the best performance across most benchmarks under the same number of RL samples. This might indicate that rule-based verifier after appropriate filtration can produce the highest-quality reward signals from noisy verifiable data.
Moreover, compared to the model trained on human-annotated verified data (MATH), leveraging noisy yet diverse verifiable data still significantly boosts performance on O.O.D. benchmarks, with absolute gains of up to 2.9\% on TheoremQA and 6.8\% on MMLU-Pro-1k. In contrast, applying a rule-based verifier to unfiltered data results in the worst performance.
This might be caused by its low training accuracy on free-form answers, while the model-based verifier achieves much better performance.

\begin{AIbox}{Takeaway \hypersetup{hidelinks}\ref{result:reward-verify-clean} for RL with Noisy Verifiable Data}
To obtain reward signals from noisy verifiable data, the ruled-based verifier after filtering the prompt set for short-form answers works the best. (Table \ref{tab:verification-types})
\end{AIbox}

\section{Exploration on RL from the Base Model}
\label{sec:rl-from-base}

DeepSeek-R1 \citep{deepseekai2025r1} has demonstrated that long chain-of-thought reasoning can emerge by scaling up reinforcement learning compute on a base model. Recent studies \citep{zeng2025simplerl, tinyzero} have attempted to replicate this progress by running a relatively small number of RL iterations to observe the emergence of long CoT behavior (e.g., the ``aha moment''~\citep{deepseekai2025r1}, an emergent realization moment that enables critical functions like self-validation and correction).
We also explore the method of RL from the base model in this section.

% \subsection{RL from the Base Model}
\subsection{Nuances in Analysis Based on Emergent Behaviors}
\label{result:base-reflections-existence}

Self-validation behaviors are sometimes flagged as emergent behaviors or ``aha-moment'' by the model's exploration, since such patterns are rare in short CoT data. However, we notice that sometimes self-validation behaviors already exist in the base model  and reinforcing them through RL requires strict conditions, such as a strong base model.

\noindent\textbf{Setup.}
We follow the setup from \citet{zeng2025simplerl} to train \texttt{Qwen2.5-Math-7B} using PPO  with a rule-based verifier on approximately 8k MATH level 3-5 questions, but we use our own rule-based verifier implementation. For inference, we adopt temperature $t = 0$ (greedy decoding), as our preliminary experiments show that $t=0$ usually significantly outperforms $t>0$ for models obtained by direct RL from \texttt{Qwen2.5-Math-7B}. We use the maximum output length of 4096 tokens considering the training context length of 4096 tokens. Note that we use zero-shot prompting for the base model to avoid introducing biases to the output pattern. We select five representative keywords, ``wait'', ``recheck'', ``alternatively'', ``retry'' and ``however'' from long CoT cases in previous works \citep{openai2024o1,deepseekai2025r1,tinyzero,zeng2025simplerl}, and calculate their frequencies to quantify the extent to which the model does self-validation. Further details about the RL hyperparameters can be found in Appendix \ref{app:exp-hyperparams-rl-from-base}.

\begin{figure*}[htbp]
    \centering
    \includegraphics[width=1\linewidth]{figs/simple-rl-acc-patterns-rate-dynamics.pdf}
    \vspace{-20pt}
    \caption{Dynamics of accuracies and reflection keyword rates on different benchmarks during our RL from the base model \texttt{Qwen2.5-Math-7B}. We do not see the keyword rates of ``wait'', ``alternatively'', and ``recheck'' get significantly improved during the RL training even though the accuracy is steadily increasing. }
    % \vspace{-10pt}
    \label{fig:reflection-acc-keywords-rate}
\end{figure*}


\begin{figure*}[htbp]
    \centering
    \includegraphics[width=1\linewidth]{figs/simple-rl-math500-output-len-coding-rate-train-kl.pdf}
    \vspace{-20pt}
    \caption{Dynamics of the output token lengths and the coding rate on MATH-500 and the KL divergence of the policy over the base model on MATH Lv3-5 (training data) during our RL from \texttt{Qwen2.5-Math-7B}.}
    \vspace{-10pt}
    \label{fig:code-rate-output-len}
\end{figure*}

\noindent\textbf{Result.} Figure \ref{fig:reflection-acc-keywords-rate} shows that our RL from \texttt{Qwen2.5-} \texttt{Math-7B} effectively boosts the accuracies, but does not increase the frequency of the ``recheck`` pattern existing in the output of the base model,
nor effectively incentivize other reflection patterns such as ``retry'' and ``alternatively''. This indicates that RL from the base model does not necessarily incentivize reflection patterns, though significantly boosting the performance. Sometimes such behaviors exist in the base model's output and RL does not substantially enhance them. So we might need to be more careful about recognizing emergent behaviors.

% \begin{AIbox}{Takeaway \hypersetup{hidelinks}\ref{result:base-reflections-existence} for Analysis on Emergent Behaviors}
% RL from a small base model improves accuracy but struggles to incentivize behaviors like error validation in challenging tasks, even when such behaviors are already present in the base model.
% (Figure \ref{fig:reflection-acc-keywords-rate}).
% \end{AIbox}

\subsection{Nuances in Analysis Based on Length Scaling}
\label{sec:analysis-length-scaling} 

The length scaling up is recognized as another important feature of the effective exploration of the model. However, we notice that sometimes length scaling up can be accompanied by the KL divergence decreasing, which raises the possibility that the length is influenced by the KL penalty and is just reverting back to the base model's longer outputs, rather than reflecting the acquisition of long CoT ability.

\noindent\textbf{Setup.} The setup is the same as in \textsection\ref{result:base-reflections-existence}. Besides the output token length, we also calculate the ``coding rate''. We classify the model's output as ``coding'' if it contains the ``\texttt{```python}'', since \texttt{Qwen2.5-Math-7B} uses both natural language and coding to solve mathematical problems. Note that the ``coding'' output here is actually a special form of natural language output, where the code in it is not executed, and the code's output is generated by the model.

\noindent\textbf{Result.} Figure \ref{fig:code-rate-output-len} (1) shows that the length of the output token increases after an initial drop, but never exceeds the initial length of the base model.

\citet{zeng2025simplerl} suggest that the initial drop may be due to the model transitioning from generating long coding outputs to shorter natural language outputs. However, Figure \ref{fig:code-rate-output-len} (2) indicates that natural language outputs are actually longer than coding outputs, and the initial drop in length occurs in both types of output. Furthermore, Figure \ref{fig:code-rate-output-len} (3) shows that the coding rate subsequently increases again, suggesting that the distinction between coding and natural language may not significantly impact the optimization process.

Moreover, we suspect that the subsequent length scaling up is not from the model's exploration, since when the length scales up, the KL divergence of the policy over the base model drops, as shown in Figure \ref{fig:code-rate-output-len} (4). This might indicate that it is the KL penalty influencing length. If that is the case, there is little potential for the policy output length to exceed the base model's since the exploration is limited by the KL constraint.

% \begin{AIbox}{Takeaway \hypersetup{hidelinks}\ref{sec:analysis-length-scaling} for Analysis on Length Scaling}
% Length scaling is accompanied by a decrease in KL divergence, suggesting that the increase in length may be driven by the KL penalty rather than the model learning to think longer (Figure \ref{fig:code-rate-output-len}).
% \end{AIbox}

\subsection{Potential Reasons Why Emergent Behavior is Not Observed with \texttt{Qwen2.5-Math-7B}}

Our detailed analysis of RL from \texttt{Qwen2.5-Math-7B}, as presented in \textsection\ref{result:base-reflections-existence} and \textsection\ref{sec:analysis-length-scaling}, suggests that it fails to fully replicate the training behavior of \texttt{DeepSeek-R1}. We identify the following potential causes: 1) The base model, being relatively small (7B parameters), may lack the capacity to quickly develop such complex abilities when incentivized. 2) The model might have been overexposed to MATH-like short instruction data during (continual) pre-training and annealing, leading to overfitting and hindering the development of long CoT behaviors.

\subsection{Comparison between RL from the Base Model and RL from Long CoT SFT}
\label{sec:rl-from-base-vs-long-cot-sft}

We compare the performance of RL from the base model and RL from long CoT SFT and find that RL from long CoT SFT generally performs better.

\noindent\textbf{Setup.}
We compare using the base model \texttt{Qwen2.5-} \texttt{Math-7B}. The results of RL from the base model are from the model trained in \textsection\ref{result:base-reflections-existence}. For RL from long CoT SFT, we adopt a setup similar to \textsection\ref{sec:sft-init-for-rl}. Specifically, we choose the 7.5k MATH training set as the prompt set, curate the SFT data by rejection sampling with 32 candidate responses per prompt using \texttt{QwQ-32B-Preview}, and perform PPO using our cosine length-scaling reward with repetition penalty and our rule-based verifier, sampling 8 responses per prompt and training for 8 epochs. To adapt \texttt{Qwen2.5-Math-7B} with a pre-training context length of only 4096 tokens to long CoT SFT and RL, we multiply its RoPE \citep{su2024rope} $\theta$ by 10 times. We don't report the results of RL with classic reward from long CoT SFT since it collapses. For evaluation, we adopt our default temperature sampling setup for RL from long CoT SFT as in \textsection\ref{sec:eval-setup} and greedy decoding setup for RL from the base model as in \textsection\ref{result:base-reflections-existence} for the best performance. Further details about the distillation, SFT hyperparameters and RL hyperparameters can be found in Appendix \ref{app:distill} \& \ref{app:sft-setup} \& \ref{app:exp-hyperparams-rl-from-base}, respectively.

\begin{table}[htbp]
\caption{
Performance of different models based on \texttt{Qwen2.5-Math-7B}. The SFT data here is distilled with rejection sampling from \texttt{QwQ-32B-Preview}.
% ``MATH-QwQ'' here is curated by rejection sampling with 32 candidate responses per prompt in the 7.5k MATH training set using \texttt{QwQ-32B-Preview}. The ``MATH (Lv3-5, ~8k)'' prompt set here includes part of the original test split.
}
\label{tab:rl-from-base-vs-from-long-sft-rl}
% \vspace{-15pt}
\vskip 0.1in
\centering
\small
% \resizebox{\linewidth}{!}{
\begin{tabular}{@{}lccccc@{}}
\toprule
\multirow{2}{*}{Setup} & MATH & AIME & Theo. & MMLU & \multirow{2}{*}{AVG}\\
& 500 & 2024 & QA & Pro-1k \\
\midrule
Base (0-shot) & 52.0 & 13.3 & 17.1 & 2.4 & 21.2 \\
(Direct) RL & 77.4 & 23.3 & 43.5 & 19.7 & 41.0 \\
SFT  & 84.0 & 24.4 & 42.2 & 38.5 & 47.3 \\
SFT + RL & \textbf{85.9} & \textbf{26.9} & \textbf{45.4} & \textbf{40.6} & \textbf{49.7} \\
\bottomrule
\end{tabular}
% }
\end{table}

\noindent\textbf{Result.} Table \ref{tab:rl-from-base-vs-from-long-sft-rl} shows that, on \texttt{Qwen2.5-Math-7B}, RL initialized from the long CoT SFT model significantly outperforms RL from the base model and further improves upon the long CoT SFT itself. Specifically, RL from long CoT SFT with our cosine reward surpasses RL from the base model by a substantial 8.7\% on average and improves over the SFT initialization by 2.6\%. Notably, simply applying SFT with long CoT distilled from \texttt{QwQ-32B-Preview} already yields strong performance.
% Figure \ref{fig:rl-from-base-vs-long-cot-sft} 

% \begin{AIbox}{Takeaway \hypersetup{hidelinks}\ref{sec:rl-from-base-vs-long-cot-sft} for RL from Base or SFT}
% RL initialized from the long CoT SFT  model outperforms RL from the base model (Table \ref{tab:rl-from-base-vs-from-long-sft-rl}).
% \end{AIbox}


\subsection{Long CoT Patterns in Pre-training Data}
\label{result:base-cot-origin}

Based on the results in \textsection\ref{result:base-reflections-existence}, we hypothesize that incentivized behaviors, such as the model revisiting its solutions, may have already been partially learned during pre-training. To examine this, we employed two methods to investigate whether such data are already present on the web.

Firstly, we used a generative search engine Perplexity.ai to identify webpages explicitly containing problem-solving steps that approach problems from multiple angles or perform verification after providing an answer. The query we used and the examples we identified are in Appendix \ref{webpage:explicit-revision-correct}).

Secondly, we used \texttt{GPT-4o} to generate a list of phrases that are characteristic of the ``aha moment'' (Appendix \ref{app:open-web-math-queries}), then used the MinHash algorithm \cite{666900} to search through  OpenWebMath \cite{paster2023openwebmathopendatasethighquality}, a dataset filtered from the CommonCrawl \cite{cc:Rana:2010:Common-Crawl-open-web-scale-crawl} frequently used in pre-training. We found that there was a significant number of matches in discussion forum threads, where the dialogue between multiple users showed similarity to long CoT, with multiple approaches being discussed along with backtracking and error correction (Appendix \ref{app:open-web-math-matches}). This raises the intriguing possibility that long CoT originated from human dialogue, although we should also note that discussion forums are a common source of data in OpenWebMath.

Based on these observations, we hypothesize that RL primarily guides the model to recombine skills it already internalized during pre-training towards new behaviors to improve performance on complex problem-solving tasks. Given the broad scope of this paper, we leave a more in-depth investigation of this behavior to future work.


% \begin{AIbox}{Takeaway \hypersetup{hidelinks}\ref{result:base-cot-origin} for Origin of long CoT ability}
% Commonly used pre-training data contain content that shares similar properties (e.g., branching and error validation)  to long CoT (Appendix \ref{webpage:explicit-revision-correct}, \ref{app:open-web-math-matches}).
% \end{AIbox}



\section{Conclusion}

We presented \sys, a sparsity-adaptive attention mechanism for efficient long-context LLM inference. Unlike fixed token budget methods, \sys dynamically selects tokens based on cumulative attention scores, adapting to variations in attention sparsity. By leveraging clustering-based sorting and distribution fitting, \sys accurately estimates token importance with low overhead. Our results showed that \sys outperforms existing sparse attention methods, achieving higher accuracy and significant inference speedups, making it a practical solution for long-context LLMs.





\section*{Impact Statement}

This paper aims to provide insights into scaling inference compute and training strategies to enable long chain-of-thought reasoning in large language models. The broader impacts of this work primarily relate to the potential for enhanced reasoning and problem-solving capabilities across various domains, where models capable of interpretable and verifiable reasoning could drive innovation and improve decision-making.
Our findings emphasize the importance of ensuring robust training data preparation, stability, and alignment with verifiable ground truths. We encourage future research to actively develop safeguards that ensure these capabilities are used responsibly. This includes careful design of reward shaping and training protocols to minimize unintended consequences while maximizing societal benefits.


\section*{Acknowledgment}
The authors would thank Yuanzhi Li for insightful discussions on this topic. The authors would also thank the SimpleRL team, particularly Weihao Zeng and Junxian He, for sharing their training experiences and experimental observations. Additionally, the authors appreciate Wenhu Chen, Xiaoyi Ren, Chao Li, Ziqiao Ma, Jiayi Pan, Xingyao Wang, and Seungone Kim for their valuable comments and discussions during the early or final stages of the project. Finally, the authors would acknowledge the DeepSeek-R1 and Kimi-k1.5 teams for their technical report releases, which inspired several additional experiment designs of this paper. This work was supported in part by a Carnegie Bosch Institute Fellowship to Xiang Yue.

\bibliography{reference}
\bibliographystyle{icml2025}

\newpage
\appendix
\section{Appendix}
\subsection{Metric Optimization}  \label{app:pg}
We utilize the REINFORCE algorithm to optimize the performance metric. The detailed optimization process is proved in the following equations:
    \begin{equation}
        \small
        \begin{aligned}
            &\nabla_{\Lambda}\hat{l}(\Lambda)
            =\nabla_{\Lambda} \mathbb{E}_{s\sim \pi{(\mathcal{B},\cdot;{\Lambda})}} \mathcal{R}(\hat{\mathcal{D}}, f(\Theta^*(\Lambda)))\\
            &=\nabla_{\Lambda}\sum_{s\in[0,1]^{|\mathcal{B}|}} \mathcal{R}(\hat{\mathcal{D}}, f(\Theta^*(\Lambda))) \cdot \pi(\mathcal{B},s;{\Lambda}) \\
            &=\sum_{s\in[0,1]^{|\mathcal{B}|}} \mathcal{R}(\hat{\mathcal{D}}, f(\Theta^*(\Lambda))) \cdot 
            \frac{\nabla_{\Lambda}\pi(\mathcal{B},s;{\Lambda})}{\pi(\mathcal{B},s;{\Lambda})}\cdot \pi(\mathcal{B},s;{\Lambda})\\
            &= \sum_{s\in[0,1]^{|\mathcal{B}|}} \mathcal{R}(\hat{\mathcal{D}}, f(\Theta^*(\Lambda))) \cdot \nabla_{\Lambda}log(\pi(\mathcal{B},s;{\Lambda}))\cdot \pi(\mathcal{B},s;{\Lambda})\\
            &=\mathbb{E}_{s\sim \pi(\mathcal{B},\cdot;{\Lambda})}[\mathcal{R}(\hat{\mathcal{D}}, f(\Theta^*(\Lambda)))\cdot \nabla_{\Lambda}log(\pi(\mathcal{B},s;{\Lambda}))],
        \end{aligned}
    \end{equation}

\subsection{Learning Algorithm}  \label{app:learning_algorithm}
The detailed optimization steps of the proposed framework are given in Algorithm \ref{al:method}.

\subsection{Detail of Studied Methods} \label{app:studied method}
To show the compatibility of our method, we apply the DVR framework on four recommendation backbones, i.e., BRPMF~\cite{koren2009matrix}, NeuMF~\cite{he2017neural}, MGCF~\cite{wang2019neural}, and LightGCN~\cite{he2020lightgcn}. We select BPRMF due to its widespread adoption in recommendation systems and proven effectiveness in practical applications. NeuMF, an MLP-based approach, extends the capabilities of BPRMF by capturing intricate user-item relationships. We leverage GNN-based models, such as MGCF and LightGCN, known for their state-of-the-art performance and competitive outcomes across various techniques, to serve as the recommendation backbone. 

Based on these backbones, different versions of the DVR model are tailored to optimize diverse metrics. For simplicity, we designate models optimized for ranking accuracy as DVR-Loss, DVR-Recall, and DVR-NDCG. Likewise, models focused on diversity and fairness metrics are labeled as DVR-CC, DVR-ILD, and DVR-Gini. 


We compare our framework with various data valuation methods for recommendations. BPR~\cite{10.5555/1795114.1795167} uniformly samples negative items and treats all training data equally in constructing the training objective. TCE-BPR and RCE-BPR are extensions of the TCE and RCE techniques \cite{10.1145/3437963.3441800}, aimed at dynamically filtering out noisy positive interactions during training based on loss values. In our implementation, we replace the original point-wise loss with a pair-wise ranking loss objective to ensure a fair comparison with these methods. AOBPR \cite{10.1145/2556195.2556248} enhances the BPR algorithm by incorporating adaptive sampling techniques that prioritize popular negative items. WBPR \cite{gantner2012personalized} assumes that unexplored popular items by a user are more likely to be true negatives. PRIS \cite{10.1145/3366423.3380187} assigns higher weights to informative negative samples using importance sampling. TIL-UI and TIL-MI \cite{wu2022adapting} learn the data value of training triplets through two aggregation strategies by optimizing the BPR loss within the training batch.

\subsection{Implementation Details} \label{app:implenmentation}
We optimize all models using the Adam optimizer with Xavier initialization \cite{glorot2010understanding} and maintain a fixed embedding size of 64 across all methods. When constructing the ranking loss objective, every positive item is associated with one sampled negative item for an efficient computation. Grid search is applied to choose learning rate and weight decay from $\left\{1e^{-4}, 1e^{-3}, 1e^{-2}, 1e^{-1}\right\}$ and $\left\{1e^{-6}, 1e^{-5}, 1e^{-4}, 1e^{-3}\right\}$. The backbone models NeuMF, MGCCF, and LightGCN utilize the provided implementations, with MGCF and LightGCN featuring two graph convolution layers. The total number of training epochs is set to 2000 for all models with an early stopping design. Given the initial training stages' limited information, we pre-train the recommendation model without data valuator for 1000 epochs to get meaningful embeddings. We set the number of the pre-training epochs to 1000. All experiments are conducted on GPU machines (NVIDIA GeForce RTX 3090).

\begin{algorithm}[H]
    \caption{The Proposed Method}
    \label{al:method}
    
    \textbf{Input:} Learning rates $\alpha$ and $\beta$, outer mini-batch size $B_1$, inner mini-batch size $B_2$, outer iteration count $T_1$, inner iteration count $T_2$, moving average window $W$, training pairs $\mathcal{D}_{1}=\{(u,i)\}_{k=1}^{L_1}$, validation pairs $\mathcal{D}_{2}=\{(u,i)\}_{k=1}^{L_2}$
    
    \textbf{Initialize:} parameters $\Theta$ and $\Lambda$, moving average $\delta=0$
    
    \begin{algorithmic}[1]
    \FOR{outer iteration $t_1=1,2,...,T_1$}
        \STATE Sample a mini-batch from the entire training dataset: $\mathcal{\hat{B}}=(u,i)_{k=1}^{B_1}\sim \mathcal{D}_1$
        \STATE Uniformly sample negative items $(j)_{k=1}^{B_1}$ for training pairs $(u,i)_{k=1}^{B_1}$ to get training data $\mathcal{B}=(u,i,j)_{k=1}^{B_1}$ for the recommendation model
        \FOR{$(u,i,j) \in \mathcal{B}$}
        \STATE Calculate the Shapley value by Eq. (\ref{eq:svcal}) and assign it to $w_{uij}$
        \ENDFOR
        \STATE Normalize the Shapley value $w_{uij}$ within the batch $\mathcal{B}$ as $\hat{w}_{uij}$ 
        \STATE Compute sample selection vector $s_{uij}=\text{Ber}(\hat{w}_{uij})$ from Bernoulli distribution
        \STATE Update the data valuator model by \ref{eq:mse}
        \FOR{inner iteration $t_2=1,2,...,T_2$}
        \STATE Sample a mini-batch $(u,i,j)_{m=1}^{B_2}\sim \mathcal{B}$
        \STATE Update the recommendation model:
    $$\Theta \leftarrow \Theta-\frac{\alpha}{B_2} \sum_{m=1}^{B_2} s_{uij} \cdot \nabla_\Theta \mathcal{L}_{\text{BPR}}(u,i,j;\Theta)$$
        \ENDFOR
    \STATE Update the data valuator:
    $$ \begin{array}{r}
    \Lambda \longleftarrow \Lambda - \beta [\mathcal{R}(\mathcal{D}_2, f(\Theta^*(\Lambda)))-\delta ]\\ \cdot \nabla_{\Lambda}log(\pi(\mathcal{B},(s_{uij})_{k=1}^{B_1};{\Lambda})
    \end{array}
    $$
    \STATE Update the moving average reward:
    $$
    \delta \leftarrow \frac{W-1}{W} \delta+\frac{1}{W} \mathcal{R}(\mathcal{D}_2, f(\Theta^*(\Lambda)))
    $$                               
    \ENDFOR
    \end{algorithmic}
\end{algorithm}







\end{document}

