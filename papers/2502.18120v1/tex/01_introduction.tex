\section{Introduction}

\begin{figure}[!htbp]
    \centering
    \includegraphics[width=0.8\linewidth]{figure/Creations.pdf}
    \caption{Games and elements made by study participants shared upon their permission. (a) and (b) are User Generated Content (UGC) that could be added to decorate personal avatars like (c). The images of (d) and (e) are game spaces.}
    \label{fig:creation}
    \Description{Five images labeled (a), (b), (c), (d), and (e) show the Roblox creations of the participants of the user study, which are User Generated Content (UGC). Images (a), (b), and (c) are 3D items, where (a) is a traditional Korean wood mask, (b) is a woman 3D character with a cat-ear hat sitting down, and (c) is a main character with sunglasses walking. Images (d) and (e) both are screenshots of a 3D Roblox space, where (d) is during the game in a grass field and (e) is a vintage interior space.}
\end{figure}

\rv{Game development} offers teenagers an engaging and valuable opportunity to build technical skills, practice collaboration, and develop self-identity~\cite{davison2006adolescent, roseth2008promoting}. \rv{Advances} in programming support tools and school coding education have lowered the barrier \rv{for adolescents to start creating their own games.} \rv{Yet, game development} still \rv{remains} a \rv{complex endeavor which requires a diverse set of skills, including design and programming skills as well as communication skills to engage with player feedback after deployment}. Online developer communities \rv{have emerged as an important environment for supporting these young creators, providing spaces to} share knowledge, \rv{overcome technical hurdles, and} learn from peers~\cite{10.1145/2531602.2531659, 8417152, brennan2021kids}. \rv{These communities can break down geographic and hierarchical boundaries, fostering connections through shared expertise~\cite{hwang2015knowledge}}. However, online developer communities are often \rv{created by and} run for \rv{adult game developers, which may cause them to be less suitable for teen developers, potentially leading to risks of exposure to content and behaviors these young developers are not yet ready to handle.} Prior \rv{research highlights how teenagers are acutely vulnerable to the potential harms they may face in navigating online communities ranging from bullying to harassment and exposure to extreme content~\cite{harlan2012teen, subrahmanyam2008online, 10.1145/2531602.2531659, heung2024vulnerable, park2022measuring, kumar2023understanding, chatterji2016code}.}

To \rv{better} understand this dual nature of online developer communities for teenager games creators---both their potential value and the risks they pose---we examined the case of \rv{Roblox} developer communities. \rv{As a} global game platform, \rv{Roblox} hosts more than 40 million games\footnote{\url{https://backlinko.com/roblox-users}} and holds a significant \rv{role in} youth culture, with 58\% of its 79.5 million daily active users under the \rv{age of 16}.\footnote{\url{https://create.roblox.com/creator}} \rv{Compared to} previously studied environments such as the Minecraft modding ecosystem ~\cite{10.1145/2858036.2858038, grace2014towards, slovak2018mediating} and the Scratch programming learning-focused platform~\cite{10.1145/3491102.3502124, brennan2021kids, shorey2021hanging, roque2016children}, \rv{Roblox stands as a digital environment where teenagers have unparalleled autonomy to create, explore, and monetize their content with the following reasons.} \rv{First,} Roblox \rv{supports a diverse range of creators with different levels of experience}. Content on Roblox, from full-fledged games (known as “Experiences”) to individual character assets like avatar clothing and accessories \rv{(known as “Creations”) and game assets like background music, is predominantly user-generated. Therefore, we define ``Roblox game developers'' to broadly include not only game developers in the traditional sense but also designers, asset creators, and special effect artists.}  \rv{Second, }Roblox \rv{supports developers} through ``Roblox Studio'', \rv{a beginner-friendly platform for}  generating and publishing their user-generated content with \rv{Lua scripting, a programming language known for its simplicity.} \rv{Third, Roblox's monetization model allows developers to turn their creativity into tangible rewards. Their }in-platform currency, “Robux,” can be converted into real-world money based on their specific policies.\footnote{Robux can be exchanged for
%real-world currency under certain conditions.}
\rv{0.0013 USD per 1 Robux as of the time of writing, after certain thresholds have been met.}}
\rv{Through this autonomy and accessibility, teen developers can engage in playful experimentation, gain independence, and build peer connections. This redefines the dynamics of user-generated content creation, providing a compelling case to examine both opportunities and challenges for teen developers.} 

\rv{To support its developers}, Roblox \rv{maintains} an official online community, the “DevForum,” \rv{and official social media channels in platforms like Discord. However, individual developers have also established numerous unofficial, user-driven communities.} These spaces cater to various developer needs, fostering peer-to-peer learning and collaboration. By studying Roblox’s developer communities, we aim to \rv{understand both} how online spaces can empower teenage creators \rv{and also how they can potentially expose} them to risks. \rv{Our research was} motivated \rv{by the unique context of Roblox developer communities, where teenage developers} exercise significant agency and autonomy. \rv{Therefore}, our research focuses on three core questions:

%To investigate the experiences of teenage developers within these online communities

\begin{itemize}
    \item \textbf{RQ1. Community Use:} \rv{Why} do teen Roblox developers utilize online developer communities? 
    \item \textbf{RQ2. Benefits:} What benefits do \rv{teen Roblox developers} receive from participating in these communities and how do the benefits vary across different creators?
    \item \textbf{RQ3. Challenges:} What challenges do teen Roblox developers face within these communities and \rv{what strategies do they use to cope} with these challenges?
\end{itemize}

%\setlength{\leftskip}{0cm}

We conducted semi-structured interviews with 18 teen Roblox developers to \rv{explore their experiences in their own words}. Our findings reveal that these communities \rv{function} as a dynamic digital playground where \rv{teen} developers explore ideas, collaborate with peers, and \rv{may even find the beginnings of a career in game development by leveraging the monetization opportunities that Roblox provides}. However, these communities also present considerable challenges. The \rv{interviewed teen} developers reported encountering financial scams, inappropriate user \rv{interactions}, difficulties balancing their online activities with their schooling, and being discouraged by the \rv{stigma surrounding the} child-like image of Roblox. By examining these experiences, our study aims to provide insights into how online communities can better support \rv{teenaged developers}, fostering a productive environment that maximizes their creative potential while \rv{proactively addressing} the associated \rv{safety risks}.
