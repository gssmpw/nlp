\section{Methods}
Our goal for this paper was to understand the benefits and challenges teenager Roblox creators encountered in online communities. To this end, we interviewed participants who have joined at least one online developer community and have self-identified as "creators" in Roblox.

\subsection{Participants}
We recruited participants who met three primary criteria: 1) Teenagers between the ages of 13 - 19,\footnote{\rv{Participants in South Korea were recruited up to age 19, as 19 is the age of majority in South Korea according to legal and cultural norms. Non-Korean participants were recruited up to age 18.}} 2) ``Creators'' who have created two games or three game elements, per Roblox's definition of having created development tools (games, maps, models, or assets) or avatar items (User Generated Content, UGC), 3) being a member in at least one online developer community related to Roblox. \rv{We included participants aged 13–19 to capture a broad spectrum of experiences, as online behaviors and risks vary across different stages of adolescence~\cite{diaz2024dysfunctional}.} To verify that participants were creators, we required them to share their creations with us during the interview.
%as part of the recruitment process. 
%\textcolor{white}{participated} 

To recruit such participants, we posted interview calls in popular Roblox online communities. We first joined online communities that showed up in a Google search for Roblox developer communities, including the Roblox official Developer Forum (DevForum), three Discord servers, and one Reddit subcommunity (r/robloxgamedev). Within those communities, we posted interview calls after getting permission from the moderators. We also recruited participants via snowball sampling, and we posted new calls in communities we learned about in early interviews. In total, calls were posted on the DevForum English and Korean Recruitment pages, nine Discord servers, one Reddit subcommunity, and one Naver Cafe --- a social platform popular in Korea. Participants who were willing to be interviewed filled out a pre-survey that was designed to check whether participants matched our recruitment criteria. As a result, 21 participants were recruited. One participant dropped as she did not want to share what she made and two did not fit our criteria. In the end, we interviewed 18 participants. 
In Table ~\ref{tab:demographics}, we show the demographics and information about our participants. The mean age of the participants was 16.3 years old (SD = 1.3) with a range of 14--19 years old. All participants were high school students except C15 who was enrolled in college at the time of the interview. Sixteen participants identified as male, and two participants identified as female. While there is no official census on the developer gender distribution of Roblox, statistics from StackOverflow developers show that female participants may be significantly underrepresented in developer communities. Of the 18 participants, 15 identified themselves as skilled in Scripting (programming) and 7 in 3D Modeling (multiple-choice possible). The average years of experience as a Roblox creator was 3.4 years (details \rv{in} Table ~\ref{tab:demographics}). 11 Koreans, 3 Americans, and one \rv{participant from each of Malaysia, Spain, Japan, and Denmark} were recruited. While we had originally intended to report how long each participant had been in the developer community they had joined, we found that most participants had joined multiple communities (with one participant reporting having joined over 200) and could not recall the exact dates of joining. Therefore, instead, we report the platforms that each participant joined communities through.


\subsection{Semi-structured Interview Method}
We conducted semi-structured interviews to understand how teenager Roblox creators used online developer communities, the benefits they gained, and the challenges they faced. Interviews generally proceeded in four phases: 1) warm-up questions, 2) experiences as a creator, 3) developer community usage, benefits, and challenges, and 4) offline impact of Roblox developer experience. Example questions include \textit{"Where did you go when you were stuck when creating {previously mentioned creation}?", "What would you miss if {joined online community} did not exist?" "Have you shown what you made to your parents?"}. \rv{The full interview protocol is shown in Appendix \ref{appendix:questions}}. All interviews were conducted online via Discord Call or Zoom (without video) from the 3rd of July to August 15th, 2024 by the first author. The interviews were conducted in either English or Korean based on the interviewee's preference. The average length of the interview was 77 minutes. All participants received \$20 (or equivalent in local currency) for participating in the interview. 

%longer than the scheduled 60-minute block, as participants were very enthusiastic to share their experiences. 

%\begin{table*}[!ht]\scriptsize
\centering
\caption{Experiment - Participants demographics}
\label{tab:partdemograph}
\begin{tabular}{cccccccccccccccccccc}
\hline
\multicolumn{10}{c}{\textbf{Plugin Group}} & \multicolumn{10}{c}{\textbf{Control Group}} \\ \hline
\multicolumn{1}{c|}{\multirow{2}{*}{\textbf{ID}}} & \multicolumn{1}{c|}{\multirow{2}{*}{\textbf{Gender}}} & \multicolumn{1}{c|}{\multirow{2}{*}{\textbf{Persona}}} & \multicolumn{2}{c|}{\textbf{Experience}} & \multicolumn{5}{c|}{\textbf{Facets}} & \multicolumn{1}{c|}{\multirow{2}{*}{\textbf{ID}}} & \multicolumn{1}{c|}{\multirow{2}{*}{\textbf{Gender}}} & \multicolumn{1}{c|}{\multirow{2}{*}{\textbf{Persona}}} & \multicolumn{2}{c|}{\textbf{Experience}} & \multicolumn{5}{c}{\textbf{Facets}} \\ \cline{4-10} \cline{14-20} 
\multicolumn{1}{c|}{} & \multicolumn{1}{c|}{} & \multicolumn{1}{c|}{} & \multicolumn{1}{c|}{\textbf{GitHub}} & \multicolumn{1}{c|}{\textbf{OSS}} & \multicolumn{1}{c|}{\textbf{MT}} & \multicolumn{1}{c|}{\textbf{SE}} & \multicolumn{1}{c|}{\textbf{R}} & \multicolumn{1}{c|}{\textbf{IP}} & \multicolumn{1}{c|}{\textbf{L}} & \multicolumn{1}{c|}{} & \multicolumn{1}{c|}{} & \multicolumn{1}{c|}{} & \multicolumn{1}{c|}{\textbf{GitHub}} & \multicolumn{1}{c|}{\textbf{OSS}} & \multicolumn{1}{c|}{\textbf{MT}} & \multicolumn{1}{c|}{\textbf{SE}} & \multicolumn{1}{c|}{\textbf{R}} & \multicolumn{1}{c|}{\textbf{IP}} & \multicolumn{1}{c}{\textbf{L}} \\ \hline \hline

\multicolumn{1}{c|}{P1} & \multicolumn{1}{c|}{M} & \multicolumn{1}{c|}{\tikzcirclenew[fill=blue]{3pt}} & \multicolumn{1}{c|}{Never} & \multicolumn{1}{c|}{No} & \multicolumn{1}{c|}{\tikzcirclenew[fill=blue]{3pt}} & \multicolumn{1}{c|}{\tikzcirclenew[fill=blue]{3pt}} & \multicolumn{1}{c|}{\tikzcirclenew[fill=blue]{3pt}} & \multicolumn{1}{c|}{\tikzcircle[fill=orange]{3pt}} & \multicolumn{1}{c|}{\tikzcirclenew[fill=blue]{3pt}} & \multicolumn{1}{c|}{P40} & \multicolumn{1}{c|}{W} & \multicolumn{1}{c|}{\tikzcircle[fill=orange]{3pt}} & \multicolumn{1}{c|}{Once} & \multicolumn{1}{c|}{No} & \multicolumn{1}{c|}{\tikzcircle[fill=orange]{3pt}} & \multicolumn{1}{c|}{\tikzcirclenew[fill=blue]{3pt}} & \multicolumn{1}{c|}{\tikzcircle[fill=orange]{3pt}} & \multicolumn{1}{c|}{\tikzcircle[fill=orange]{3pt}} & \multicolumn{1}{c}{\tikzcirclenew[fill=blue]{3pt}} \\ \hline

\multicolumn{1}{c|}{P2} & \multicolumn{1}{c|}{W} & \multicolumn{1}{c|}{\tikzcirclenew[fill=blue]{3pt}} & \multicolumn{1}{c|}{Once} & \multicolumn{1}{c|}{No} & \multicolumn{1}{c|}{\tikzcirclenew[fill=blue]{3pt}} & \multicolumn{1}{c|}{\tikzcirclenew[fill=blue]{3pt}} & \multicolumn{1}{c|}{\tikzcirclenew[fill=blue]{3pt}} & \multicolumn{1}{c|}{\tikzcircle[fill=orange]{3pt}} & \multicolumn{1}{c|}{\tikzcirclenew[fill=blue]{3pt}} & \multicolumn{1}{c|}{P41} & \multicolumn{1}{c|}{M} & \multicolumn{1}{c|}{\tikzcirclenew[fill=blue]{3pt}} & \multicolumn{1}{c|}{Once} & \multicolumn{1}{c|}{No} & \multicolumn{1}{c|}{\tikzcirclenew[fill=blue]{3pt}} & \multicolumn{1}{c|}{\tikzcirclenew[fill=blue]{3pt}} & \multicolumn{1}{c|}{\tikzcircle[fill=orange]{3pt}} & \multicolumn{1}{c|}{\tikzcircle[fill=orange]{3pt}} & \multicolumn{1}{c}{\tikzcirclenew[fill=blue]{3pt}} \\ \hline

\multicolumn{1}{c|}{P3} & \multicolumn{1}{c|}{M} & \multicolumn{1}{c|}{\tikzcirclenew[fill=blue]{3pt}} & \multicolumn{1}{c|}{Never} & \multicolumn{1}{c|}{No} & \multicolumn{1}{c|}{\tikzcircle[fill=orange]{3pt}} & \multicolumn{1}{c|}{\tikzcirclenew[fill=blue]{3pt}} & \multicolumn{1}{c|}{\tikzcirclenew[fill=blue]{3pt}} & \multicolumn{1}{c|}{\tikzcircle[fill=orange]{3pt}} & \multicolumn{1}{c|}{\tikzcirclenew[fill=blue]{3pt}} & \multicolumn{1}{c|}{P42} & \multicolumn{1}{c|}{M} & \multicolumn{1}{c|}{\tikzcircle[fill=orange]{3pt}} & \multicolumn{1}{c|}{Never} & \multicolumn{1}{c|}{No} & \multicolumn{1}{c|}{\tikzcircle[fill=orange]{3pt}} & \multicolumn{1}{c|}{\tikzcirclenew[fill=blue]{3pt}} & \multicolumn{1}{c|}{\tikzcircle[fill=orange]{3pt}} & \multicolumn{1}{c|}{\tikzcircle[fill=orange]{3pt}} & \multicolumn{1}{c}{\tikzcirclenew[fill=blue]{3pt}} \\ \hline

\multicolumn{1}{c|}{P4} & \multicolumn{1}{c|}{M} & \multicolumn{1}{c|}{\tikzcirclenew[fill=blue]{3pt}} & \multicolumn{1}{c|}{Never} & \multicolumn{1}{c|}{No} & \multicolumn{1}{c|}{\tikzcircle[fill=orange]{3pt}} & \multicolumn{1}{c|}{\tikzcirclenew[fill=blue]{3pt}} & \multicolumn{1}{c|}{\tikzcirclenew[fill=blue]{3pt}} & \multicolumn{1}{c|}{\tikzcircle[fill=orange]{3pt}} & \multicolumn{1}{c|}{\tikzcirclenew[fill=blue]{3pt}} & \multicolumn{1}{c|}{P43} & \multicolumn{1}{c|}{M} & \multicolumn{1}{c|}{\tikzcirclenew[fill=blue]{3pt}} & \multicolumn{1}{c|}{Never} & \multicolumn{1}{c|}{No} & \multicolumn{1}{c|}{\tikzcirclenew[fill=blue]{3pt}} & \multicolumn{1}{c|}{\tikzcirclenew[fill=blue]{3pt}} & \multicolumn{1}{c|}{\tikzcirclenew[fill=blue]{3pt}} & \multicolumn{1}{c|}{\tikzcircle[fill=orange]{3pt}} & \multicolumn{1}{c}{\tikzcirclenew[fill=blue]{3pt}} \\ \hline

\multicolumn{1}{c|}{P5} & \multicolumn{1}{c|}{M} & \multicolumn{1}{c|}{\tikzcirclenew[fill=blue]{3pt}} & \multicolumn{1}{c|}{Once} & \multicolumn{1}{c|}{No} & \multicolumn{1}{c|}{\tikzcirclenew[fill=blue]{3pt}} & \multicolumn{1}{c|}{\tikzcirclenew[fill=blue]{3pt}} & \multicolumn{1}{c|}{\tikzcirclenew[fill=blue]{3pt}} & \multicolumn{1}{c|}{\tikzcircle[fill=orange]{3pt}} & \multicolumn{1}{c|}{\tikzcirclenew[fill=blue]{3pt}} & \multicolumn{1}{c|}{P44} & \multicolumn{1}{c|}{M} & \multicolumn{1}{c|}{\tikzcirclenew[fill=blue]{3pt}} & \multicolumn{1}{c|}{Never} & \multicolumn{1}{c|}{No} & \multicolumn{1}{c|}{\tikzcirclenew[fill=blue]{3pt}} & \multicolumn{1}{c|}{\tikzcirclenew[fill=blue]{3pt}} & \multicolumn{1}{c|}{\tikzcircle[fill=orange]{3pt}} & \multicolumn{1}{c|}{\tikzcircle[fill=orange]{3pt}} & \multicolumn{1}{c}{\tikzcirclenew[fill=blue]{3pt}} \\ \hline

\multicolumn{1}{c|}{P6} & \multicolumn{1}{c|}{M} & \multicolumn{1}{c|}{\tikzcirclenew[fill=blue]{3pt}} & \multicolumn{1}{c|}{Once} & \multicolumn{1}{c|}{No} & \multicolumn{1}{c|}{\tikzcirclenew[fill=blue]{3pt}} & \multicolumn{1}{c|}{\tikzcirclenew[fill=blue]{3pt}} & \multicolumn{1}{c|}{\tikzcircle[fill=orange]{3pt}} & \multicolumn{1}{c|}{\tikzcircle[fill=orange]{3pt}} & \multicolumn{1}{c|}{\tikzcirclenew[fill=blue]{3pt}} & \multicolumn{1}{c|}{P45} & \multicolumn{1}{c|}{M} & \multicolumn{1}{c|}{\tikzcirclenew[fill=blue]{3pt}} & \multicolumn{1}{c|}{Never} & \multicolumn{1}{c|}{No} & \multicolumn{1}{c|}{\tikzcirclenew[fill=blue]{3pt}} & \multicolumn{1}{c|}{\tikzcirclenew[fill=blue]{3pt}} & \multicolumn{1}{c|}{\tikzcirclenew[fill=blue]{3pt}} & \multicolumn{1}{c|}{\tikzcircle[fill=orange]{3pt}} & \multicolumn{1}{c}{\tikzcirclenew[fill=blue]{3pt}} \\ \hline

\multicolumn{1}{c|}{P7} & \multicolumn{1}{c|}{W} & \multicolumn{1}{c|}{\tikzcircle[fill=orange]{3pt}} & \multicolumn{1}{c|}{Never} & \multicolumn{1}{c|}{No} & \multicolumn{1}{c|}{\tikzcircle[fill=orange]{3pt}} & \multicolumn{1}{c|}{\tikzcirclenew[fill=blue]{3pt}} & \multicolumn{1}{c|}{\tikzcircle[fill=orange]{3pt}} & \multicolumn{1}{c|}{\tikzcircle[fill=orange]{3pt}} & \multicolumn{1}{c|}{\tikzcirclenew[fill=blue]{3pt}} & \multicolumn{1}{c|}{P46} & \multicolumn{1}{c|}{M} & \multicolumn{1}{c|}{\tikzcircle[fill=orange]{3pt}} & \multicolumn{1}{c|}{Never} & \multicolumn{1}{c|}{No} & \multicolumn{1}{c|}{\tikzcirclenew[fill=blue]{3pt}} & \multicolumn{1}{c|}{\tikzcircle[fill=orange]{3pt}} & \multicolumn{1}{c|}{\tikzcircle[fill=orange]{3pt}} & \multicolumn{1}{c|}{\tikzcircle[fill=orange]{3pt}} & \multicolumn{1}{c}{\tikzcirclenew[fill=blue]{3pt}} \\ \hline

\multicolumn{1}{c|}{P8} & \multicolumn{1}{c|}{W} & \multicolumn{1}{c|}{\tikzcircle[fill=orange]{3pt}} & \multicolumn{1}{c|}{Never} & \multicolumn{1}{c|}{No} & \multicolumn{1}{c|}{\tikzcirclenew[fill=blue]{3pt}} & \multicolumn{1}{c|}{\tikzcircle[fill=orange]{3pt}} & \multicolumn{1}{c|}{\tikzcircle[fill=orange]{3pt}} & \multicolumn{1}{c|}{\tikzcircle[fill=orange]{3pt}} & \multicolumn{1}{c|}{\tikzcirclenew[fill=blue]{3pt}} & \multicolumn{1}{c|}{P47} & \multicolumn{1}{c|}{M} & \multicolumn{1}{c|}{\tikzcirclenew[fill=blue]{3pt}} & \multicolumn{1}{c|}{Never} & \multicolumn{1}{c|}{No} & \multicolumn{1}{c|}{\tikzcirclenew[fill=blue]{3pt}} & \multicolumn{1}{c|}{\tikzcirclenew[fill=blue]{3pt}} & \multicolumn{1}{c|}{\tikzcirclenew[fill=blue]{3pt}} & \multicolumn{1}{c|}{\tikzcircle[fill=orange]{3pt}} & \multicolumn{1}{c}{\tikzcirclenew[fill=blue]{3pt}} \\ \hline

\multicolumn{1}{c|}{P9} & \multicolumn{1}{c|}{M} & \multicolumn{1}{c|}{\tikzcirclenew[fill=blue]{3pt}} & \multicolumn{1}{c|}{Once} & \multicolumn{1}{c|}{No} & \multicolumn{1}{c|}{\tikzcirclenew[fill=blue]{3pt}} & \multicolumn{1}{c|}{\tikzcirclenew[fill=blue]{3pt}} & \multicolumn{1}{c|}{\tikzcirclenew[fill=blue]{3pt}} & \multicolumn{1}{c|}{\tikzcircle[fill=orange]{3pt}} & \multicolumn{1}{c|}{\tikzcirclenew[fill=blue]{3pt}} & \multicolumn{1}{c|}{P48} & \multicolumn{1}{c|}{M} & \multicolumn{1}{c|}{\tikzcirclenew[fill=blue]{3pt}} & \multicolumn{1}{c|}{Never} & \multicolumn{1}{c|}{No} & \multicolumn{1}{c|}{\tikzcircle[fill=orange]{3pt}} & \multicolumn{1}{c|}{\tikzcirclenew[fill=blue]{3pt}} & \multicolumn{1}{c|}{\tikzcirclenew[fill=blue]{3pt}} & \multicolumn{1}{c|}{\tikzcircle[fill=orange]{3pt}} & \multicolumn{1}{c}{\tikzcirclenew[fill=blue]{3pt}} \\ \hline

\multicolumn{1}{c|}{P10} & \multicolumn{1}{c|}{W} & \multicolumn{1}{c|}{\tikzcirclenew[fill=blue]{3pt}} & \multicolumn{1}{c|}{Never} & \multicolumn{1}{c|}{No} & \multicolumn{1}{c|}{\tikzcirclenew[fill=blue]{3pt}} & \multicolumn{1}{c|}{\tikzcirclenew[fill=blue]{3pt}} & \multicolumn{1}{c|}{\tikzcircle[fill=orange]{3pt}} & \multicolumn{1}{c|}{\tikzcircle[fill=orange]{3pt}} & \multicolumn{1}{c|}{\tikzcirclenew[fill=blue]{3pt}} & \multicolumn{1}{c|}{P49} & \multicolumn{1}{c|}{M} & \multicolumn{1}{c|}{\tikzcircle[fill=orange]{3pt}} & \multicolumn{1}{c|}{Never} & \multicolumn{1}{c|}{No} & \multicolumn{1}{c|}{\tikzcircle[fill=orange]{3pt}} & \multicolumn{1}{c|}{\tikzcirclenew[fill=blue]{3pt}} & \multicolumn{1}{c|}{\tikzcircle[fill=orange]{3pt}} & \multicolumn{1}{c|}{\tikzcircle[fill=orange]{3pt}} & \multicolumn{1}{c}{\tikzcirclenew[fill=blue]{3pt}} \\ \hline

\multicolumn{1}{c|}{P11} & \multicolumn{1}{c|}{M} & \multicolumn{1}{c|}{\tikzcircle[fill=orange]{3pt}} & \multicolumn{1}{c|}{Never} & \multicolumn{1}{c|}{Some} & \multicolumn{1}{c|}{\tikzcircle[fill=orange]{3pt}} & \multicolumn{1}{c|}{\tikzcircle[fill=orange]{3pt}} & \multicolumn{1}{c|}{\tikzcircle[fill=orange]{3pt}} & \multicolumn{1}{c|}{\tikzcircle[fill=orange]{3pt}} & \multicolumn{1}{c|}{\tikzcircle[fill=orange]{3pt}} & \multicolumn{1}{c|}{P50} & \multicolumn{1}{c|}{M} & \multicolumn{1}{c|}{\tikzcirclenew[fill=blue]{3pt}} & \multicolumn{1}{c|}{Never} & \multicolumn{1}{c|}{No} & \multicolumn{1}{c|}{\tikzcirclenew[fill=blue]{3pt}} & \multicolumn{1}{c|}{\tikzcirclenew[fill=blue]{3pt}} & \multicolumn{1}{c|}{\tikzcirclenew[fill=blue]{3pt}} & \multicolumn{1}{c|}{\tikzcircle[fill=orange]{3pt}} & \multicolumn{1}{c}{\tikzcirclenew[fill=blue]{3pt}} \\ \hline

\multicolumn{1}{c|}{P12} & \multicolumn{1}{c|}{M} & \multicolumn{1}{c|}{\tikzcirclenew[fill=blue]{3pt}} & \multicolumn{1}{c|}{Never} & \multicolumn{1}{c|}{No} & \multicolumn{1}{c|}{\tikzcirclenew[fill=blue]{3pt}} & \multicolumn{1}{c|}{\tikzcirclenew[fill=blue]{3pt}} & \multicolumn{1}{c|}{\tikzcirclenew[fill=blue]{3pt}} & \multicolumn{1}{c|}{\tikzcircle[fill=orange]{3pt}} & \multicolumn{1}{c|}{\tikzcirclenew[fill=blue]{3pt}} & \multicolumn{1}{c|}{P51} & \multicolumn{1}{c|}{M} & \multicolumn{1}{c|}{\tikzcirclenew[fill=blue]{3pt}} & \multicolumn{1}{c|}{Never} & \multicolumn{1}{c|}{No} & \multicolumn{1}{c|}{\tikzcirclenew[fill=blue]{3pt}} & \multicolumn{1}{c|}{\tikzcirclenew[fill=blue]{3pt}} & \multicolumn{1}{c|}{\tikzcirclenew[fill=blue]{3pt}} & \multicolumn{1}{c|}{\tikzcircle[fill=orange]{3pt}} & \multicolumn{1}{c}{\tikzcirclenew[fill=blue]{3pt}} \\ \hline

\multicolumn{1}{c|}{P13} & \multicolumn{1}{c|}{W} & \multicolumn{1}{c|}{\tikzcircle[fill=orange]{3pt}} & \multicolumn{1}{c|}{Once} & \multicolumn{1}{c|}{No} & \multicolumn{1}{c|}{\tikzcircle[fill=orange]{3pt}} & \multicolumn{1}{c|}{\tikzcircle[fill=orange]{3pt}} & \multicolumn{1}{c|}{\tikzcircle[fill=orange]{3pt}} & \multicolumn{1}{c|}{\tikzcircle[fill=orange]{3pt}} & \multicolumn{1}{c|}{\tikzcircle[fill=orange]{3pt}} & \multicolumn{1}{c|}{P52} & \multicolumn{1}{c|}{M} & \multicolumn{1}{c|}{\tikzcircle[fill=orange]{3pt}} & \multicolumn{1}{c|}{Never} & \multicolumn{1}{c|}{Some} & \multicolumn{1}{c|}{\tikzcircle[fill=orange]{3pt}} & \multicolumn{1}{c|}{\tikzcircle[fill=orange]{3pt}} & \multicolumn{1}{c|}{\tikzcirclenew[fill=blue]{3pt}} & \multicolumn{1}{c|}{\tikzcircle[fill=orange]{3pt}} & \multicolumn{1}{c}{\tikzcircle[fill=orange]{3pt}} \\ \hline

\multicolumn{1}{c|}{P14} & \multicolumn{1}{c|}{W} & \multicolumn{1}{c|}{\tikzcircle[fill=orange]{3pt}} & \multicolumn{1}{c|}{Never} & \multicolumn{1}{c|}{Some} & \multicolumn{1}{c|}{\tikzcirclenew[fill=blue]{3pt}} & \multicolumn{1}{c|}{\tikzcirclenew[fill=blue]{3pt}} & \multicolumn{1}{c|}{\tikzcircle[fill=orange]{3pt}} & \multicolumn{1}{c|}{\tikzcircle[fill=orange]{3pt}} & \multicolumn{1}{c|}{\tikzcircle[fill=orange]{3pt}} & \multicolumn{1}{c|}{P53} & \multicolumn{1}{c|}{M} & \multicolumn{1}{c|}{\tikzcircle[fill=orange]{3pt}} & \multicolumn{1}{c|}{Once} & \multicolumn{1}{c|}{No} & \multicolumn{1}{c|}{\tikzcircle[fill=orange]{3pt}} & \multicolumn{1}{c|}{\tikzcircle[fill=orange]{3pt}} & \multicolumn{1}{c|}{\tikzcirclenew[fill=blue]{3pt}} & \multicolumn{1}{c|}{\tikzcircle[fill=orange]{3pt}} & \multicolumn{1}{c}{\tikzcircle[fill=orange]{3pt}} \\ \hline

\multicolumn{1}{c|}{P15} & \multicolumn{1}{c|}{M} & \multicolumn{1}{c|}{\tikzcircle[fill=orange]{3pt}} & \multicolumn{1}{c|}{Never} & \multicolumn{1}{c|}{No} & \multicolumn{1}{c|}{\tikzcircle[fill=orange]{3pt}} & \multicolumn{1}{c|}{\tikzcirclenew[fill=blue]{3pt}} & \multicolumn{1}{c|}{\tikzcirclenew[fill=blue]{3pt}} & \multicolumn{1}{c|}{\tikzcircle[fill=orange]{3pt}} & \multicolumn{1}{c|}{\tikzcircle[fill=orange]{3pt}} & \multicolumn{1}{c|}{P54} & \multicolumn{1}{c|}{W} & \multicolumn{1}{c|}{\tikzcircle[fill=orange]{3pt}} & \multicolumn{1}{c|}{Never} & \multicolumn{1}{c|}{No} & \multicolumn{1}{c|}{\tikzcircle[fill=orange]{3pt}} & \multicolumn{1}{c|}{\tikzcirclenew[fill=blue]{3pt}} & \multicolumn{1}{c|}{\tikzcircle[fill=orange]{3pt}} & \multicolumn{1}{c|}{\tikzcircle[fill=orange]{3pt}} & \multicolumn{1}{c}{\tikzcircle[fill=orange]{3pt}} \\ \hline

\multicolumn{1}{c|}{P16} & \multicolumn{1}{c|}{M} & \multicolumn{1}{c|}{\tikzcircle[fill=orange]{3pt}} & \multicolumn{1}{c|}{Once} & \multicolumn{1}{c|}{No} & \multicolumn{1}{c|}{\tikzcirclenew[fill=blue]{3pt}} & \multicolumn{1}{c|}{\tikzcircle[fill=orange]{3pt}} & \multicolumn{1}{c|}{\tikzcirclenew[fill=blue]{3pt}} & \multicolumn{1}{c|}{\tikzcircle[fill=orange]{3pt}} & \multicolumn{1}{c|}{\tikzcircle[fill=orange]{3pt}} & \multicolumn{1}{c|}{P55} & \multicolumn{1}{c|}{W} & \multicolumn{1}{c|}{\tikzcircle[fill=orange]{3pt}} & \multicolumn{1}{c|}{Once} & \multicolumn{1}{c|}{No} & \multicolumn{1}{c|}{\tikzcirclenew[fill=blue]{3pt}} & \multicolumn{1}{c|}{\tikzcircle[fill=orange]{3pt}} & \multicolumn{1}{c|}{\tikzcircle[fill=orange]{3pt}} & \multicolumn{1}{c|}{\tikzcircle[fill=orange]{3pt}} & \multicolumn{1}{c}{\tikzcircle[fill=orange]{3pt}} \\ \hline

\multicolumn{1}{c|}{P17} & \multicolumn{1}{c|}{W} & \multicolumn{1}{c|}{\tikzcircle[fill=orange]{3pt}} & \multicolumn{1}{c|}{Once} & \multicolumn{1}{c|}{No} & \multicolumn{1}{c|}{\tikzcircle[fill=orange]{3pt}} & \multicolumn{1}{c|}{\tikzcirclenew[fill=blue]{3pt}} & \multicolumn{1}{c|}{\tikzcircle[fill=orange]{3pt}} & \multicolumn{1}{c|}{\tikzcircle[fill=orange]{3pt}} & \multicolumn{1}{c|}{\tikzcircle[fill=orange]{3pt}} & \multicolumn{1}{c|}{P56} & \multicolumn{1}{c|}{M} & \multicolumn{1}{c|}{\tikzcircle[fill=orange]{3pt}} & \multicolumn{1}{c|}{Never} & \multicolumn{1}{c|}{No} & \multicolumn{1}{c|}{\tikzcircle[fill=orange]{3pt}} & \multicolumn{1}{c|}{\tikzcircle[fill=orange]{3pt}} & \multicolumn{1}{c|}{\tikzcirclenew[fill=blue]{3pt}} & \multicolumn{1}{c|}{\tikzcircle[fill=orange]{3pt}} & \multicolumn{1}{c}{\tikzcircle[fill=orange]{3pt}} \\ \hline

\multicolumn{1}{c|}{P18} & \multicolumn{1}{c|}{W} & \multicolumn{1}{c|}{\tikzcirclenew[fill=blue]{3pt}} & \multicolumn{1}{c|}{Once} & \multicolumn{1}{c|}{No} & \multicolumn{1}{c|}{\tikzcirclenew[fill=blue]{3pt}} & \multicolumn{1}{c|}{\tikzcirclenew[fill=blue]{3pt}} & \multicolumn{1}{c|}{\tikzcirclenew[fill=blue]{3pt}} & \multicolumn{1}{c|}{\tikzcircle[fill=orange]{3pt}} & \multicolumn{1}{c|}{\tikzcirclenew[fill=blue]{3pt}} & \multicolumn{1}{c|}{P57} & \multicolumn{1}{c|}{M} & \multicolumn{1}{c|}{\tikzcircle[fill=orange]{3pt}} & \multicolumn{1}{c|}{Few times} & \multicolumn{1}{c|}{No} & \multicolumn{1}{c|}{\tikzcircle[fill=orange]{3pt}} & \multicolumn{1}{c|}{\tikzcirclenew[fill=blue]{3pt}} & \multicolumn{1}{c|}{\tikzcircle[fill=orange]{3pt}} & \multicolumn{1}{c|}{\tikzcircle[fill=orange]{3pt}} & \multicolumn{1}{c}{\tikzcirclenew[fill=blue]{3pt}} \\ \hline

\multicolumn{1}{c|}{P19} & \multicolumn{1}{c|}{M} & \multicolumn{1}{c|}{\tikzcirclenew[fill=blue]{3pt}} & \multicolumn{1}{c|}{Never} & \multicolumn{1}{c|}{No} & \multicolumn{1}{c|}{\tikzcirclenew[fill=blue]{3pt}} & \multicolumn{1}{c|}{\tikzcirclenew[fill=blue]{3pt}} & \multicolumn{1}{c|}{\tikzcirclenew[fill=blue]{3pt}} & \multicolumn{1}{c|}{\tikzcircle[fill=orange]{3pt}} & \multicolumn{1}{c|}{\tikzcirclenew[fill=blue]{3pt}} & \multicolumn{1}{c|}{P58} & \multicolumn{1}{c|}{M} & \multicolumn{1}{c|}{\tikzcircle[fill=orange]{3pt}} & \multicolumn{1}{c|}{Once} & \multicolumn{1}{c|}{Some} & \multicolumn{1}{c|}{\tikzcircle[fill=orange]{3pt}} & \multicolumn{1}{c|}{\tikzcircle[fill=orange]{3pt}} & \multicolumn{1}{c|}{\tikzcirclenew[fill=blue]{3pt}} & \multicolumn{1}{c|}{\tikzcirclenew[fill=blue]{3pt}} & \multicolumn{1}{c}{\tikzcircle[fill=orange]{3pt}} \\ \hline

\multicolumn{1}{c|}{P20} & \multicolumn{1}{c|}{M} & \multicolumn{1}{c|}{\tikzcircle[fill=orange]{3pt}} & \multicolumn{1}{c|}{Few times} & \multicolumn{1}{c|}{No} & \multicolumn{1}{c|}{\tikzcircle[fill=orange]{3pt}} & \multicolumn{1}{c|}{\tikzcirclenew[fill=blue]{3pt}} & \multicolumn{1}{c|}{\tikzcircle[fill=orange]{3pt}} & \multicolumn{1}{c|}{\tikzcircle[fill=orange]{3pt}} & \multicolumn{1}{c|}{\tikzcircle[fill=orange]{3pt}} & \multicolumn{1}{c|}{P59} & \multicolumn{1}{c|}{M} & \multicolumn{1}{c|}{\tikzcirclenew[fill=blue]{3pt}} & \multicolumn{1}{c|}{Never} & \multicolumn{1}{c|}{No} & \multicolumn{1}{c|}{\tikzcirclenew[fill=blue]{3pt}} & \multicolumn{1}{c|}{\tikzcirclenew[fill=blue]{3pt}} & \multicolumn{1}{c|}{\tikzcirclenew[fill=blue]{3pt}} & \multicolumn{1}{c|}{\tikzcirclenew[fill=blue]{3pt}} & \multicolumn{1}{c}{\tikzcirclenew[fill=blue]{3pt}} \\ \hline

\multicolumn{1}{c|}{P21} & \multicolumn{1}{c|}{M} & \multicolumn{1}{c|}{\tikzcirclenew[fill=blue]{3pt}} & \multicolumn{1}{c|}{Often} & \multicolumn{1}{c|}{No} & \multicolumn{1}{c|}{\tikzcirclenew[fill=blue]{3pt}} & \multicolumn{1}{c|}{\tikzcirclenew[fill=blue]{3pt}} & \multicolumn{1}{c|}{\tikzcircle[fill=orange]{3pt}} & \multicolumn{1}{c|}{\tikzcircle[fill=orange]{3pt}} & \multicolumn{1}{c|}{\tikzcirclenew[fill=blue]{3pt}} & \multicolumn{1}{c|}{P60} & \multicolumn{1}{c|}{M} & \multicolumn{1}{c|}{\tikzcirclenew[fill=blue]{3pt}} & \multicolumn{1}{c|}{Once} & \multicolumn{1}{c|}{No} & \multicolumn{1}{c|}{\tikzcirclenew[fill=blue]{3pt}} & \multicolumn{1}{c|}{\tikzcircle[fill=orange]{3pt}} & \multicolumn{1}{c|}{\tikzcirclenew[fill=blue]{3pt}} & \multicolumn{1}{c|}{\tikzcirclenew[fill=blue]{3pt}} & \multicolumn{1}{c}{\tikzcirclenew[fill=blue]{3pt}} \\ \hline

\multicolumn{1}{c|}{P22} & \multicolumn{1}{c|}{W} & \multicolumn{1}{c|}{\tikzcirclenew[fill=blue]{3pt}} & \multicolumn{1}{c|}{Once} & \multicolumn{1}{c|}{No} & \multicolumn{1}{c|}{\tikzcirclenew[fill=blue]{3pt}} & \multicolumn{1}{c|}{\tikzcirclenew[fill=blue]{3pt}} & \multicolumn{1}{c|}{\tikzcirclenew[fill=blue]{3pt}} & \multicolumn{1}{c|}{\tikzcircle[fill=orange]{3pt}} & \multicolumn{1}{c|}{\tikzcircle[fill=orange]{3pt}} & \multicolumn{1}{c|}{P61} & \multicolumn{1}{c|}{M} & \multicolumn{1}{c|}{\tikzcircle[fill=orange]{3pt}} & \multicolumn{1}{c|}{Few times} & \multicolumn{1}{c|}{No} & \multicolumn{1}{c|}{\tikzcircle[fill=orange]{3pt}} & \multicolumn{1}{c|}{\tikzcircle[fill=orange]{3pt}} & \multicolumn{1}{c|}{\tikzcirclenew[fill=blue]{3pt}} & \multicolumn{1}{c|}{\tikzcirclenew[fill=blue]{3pt}} & \multicolumn{1}{c}{\tikzcircle[fill=orange]{3pt}} \\ \hline

\multicolumn{1}{c|}{P23} & \multicolumn{1}{c|}{W} & \multicolumn{1}{c|}{\tikzcirclenew[fill=blue]{3pt}} & \multicolumn{1}{c|}{Often} & \multicolumn{1}{c|}{No} & \multicolumn{1}{c|}{\tikzcirclenew[fill=blue]{3pt}} & \multicolumn{1}{c|}{\tikzcirclenew[fill=blue]{3pt}} & \multicolumn{1}{c|}{\tikzcirclenew[fill=blue]{3pt}} & \multicolumn{1}{c|}{\tikzcircle[fill=orange]{3pt}} & \multicolumn{1}{c|}{\tikzcirclenew[fill=blue]{3pt}} & \multicolumn{1}{c|}{P62} & \multicolumn{1}{c|}{M} & \multicolumn{1}{c|}{\tikzcircle[fill=orange]{3pt}} & \multicolumn{1}{c|}{Few times} & \multicolumn{1}{c|}{No} & \multicolumn{1}{c|}{\tikzcircle[fill=orange]{3pt}} & \multicolumn{1}{c|}{\tikzcirclenew[fill=blue]{3pt}} & \multicolumn{1}{c|}{\tikzcircle[fill=orange]{3pt}} & \multicolumn{1}{c|}{\tikzcircle[fill=orange]{3pt}} & \multicolumn{1}{c}{\tikzcircle[fill=orange]{3pt}} \\ \hline

\multicolumn{1}{c|}{P24} & \multicolumn{1}{c|}{M} & \multicolumn{1}{c|}{\tikzcirclenew[fill=blue]{3pt}} & \multicolumn{1}{c|}{Often} & \multicolumn{1}{c|}{No} & \multicolumn{1}{c|}{\tikzcirclenew[fill=blue]{3pt}} & \multicolumn{1}{c|}{\tikzcircle[fill=orange]{3pt}} & \multicolumn{1}{c|}{\tikzcirclenew[fill=blue]{3pt}} & \multicolumn{1}{c|}{\tikzcircle[fill=orange]{3pt}} & \multicolumn{1}{c|}{\tikzcirclenew[fill=blue]{3pt}} & \multicolumn{1}{c|}{P63} & \multicolumn{1}{c|}{W} & \multicolumn{1}{c|}{\tikzcirclenew[fill=blue]{3pt}} & \multicolumn{1}{c|}{Never} & \multicolumn{1}{c|}{No} & \multicolumn{1}{c|}{\tikzcircle[fill=orange]{3pt}} & \multicolumn{1}{c|}{\tikzcirclenew[fill=blue]{3pt}} & \multicolumn{1}{c|}{\tikzcirclenew[fill=blue]{3pt}} & \multicolumn{1}{c|}{\tikzcircle[fill=orange]{3pt}} & \multicolumn{1}{c}{\tikzcirclenew[fill=blue]{3pt}} \\ \hline

\multicolumn{1}{c|}{P25} & \multicolumn{1}{c|}{M} & \multicolumn{1}{c|}{\tikzcirclenew[fill=blue]{3pt}} & \multicolumn{1}{c|}{Often} & \multicolumn{1}{c|}{No} & \multicolumn{1}{c|}{\tikzcircle[fill=orange]{3pt}} & \multicolumn{1}{c|}{\tikzcirclenew[fill=blue]{3pt}} & \multicolumn{1}{c|}{\tikzcirclenew[fill=blue]{3pt}} & \multicolumn{1}{c|}{\tikzcirclenew[fill=blue]{3pt}} & \multicolumn{1}{c|}{\tikzcircle[fill=orange]{3pt}} & \multicolumn{1}{c|}{P64} & \multicolumn{1}{c|}{W} & \multicolumn{1}{c|}{\tikzcirclenew[fill=blue]{3pt}} & \multicolumn{1}{c|}{Never} & \multicolumn{1}{c|}{No} & \multicolumn{1}{c|}{\tikzcirclenew[fill=blue]{3pt}} & \multicolumn{1}{c|}{\tikzcirclenew[fill=blue]{3pt}} & \multicolumn{1}{c|}{\tikzcirclenew[fill=blue]{3pt}} & \multicolumn{1}{c|}{\tikzcircle[fill=orange]{3pt}} & \multicolumn{1}{c}{\tikzcirclenew[fill=blue]{3pt}} \\ \hline

\multicolumn{1}{c|}{P26} & \multicolumn{1}{c|}{W} & \multicolumn{1}{c|}{\tikzcircle[fill=orange]{3pt}} & \multicolumn{1}{c|}{Few times} & \multicolumn{1}{c|}{No} & \multicolumn{1}{c|}{\tikzcircle[fill=orange]{3pt}} & \multicolumn{1}{c|}{\tikzcircle[fill=orange]{3pt}} & \multicolumn{1}{c|}{\tikzcirclenew[fill=blue]{3pt}} & \multicolumn{1}{c|}{\tikzcircle[fill=orange]{3pt}} & \multicolumn{1}{c|}{\tikzcircle[fill=orange]{3pt}} & \multicolumn{1}{c|}{P65} & \multicolumn{1}{c|}{M} & \multicolumn{1}{c|}{\tikzcirclenew[fill=blue]{3pt}} & \multicolumn{1}{c|}{Never} & \multicolumn{1}{c|}{No} & \multicolumn{1}{c|}{\tikzcircle[fill=orange]{3pt}} & \multicolumn{1}{c|}{\tikzcirclenew[fill=blue]{3pt}} & \multicolumn{1}{c|}{\tikzcirclenew[fill=blue]{3pt}} & \multicolumn{1}{c|}{\tikzcircle[fill=orange]{3pt}} & \multicolumn{1}{c}{\tikzcirclenew[fill=blue]{3pt}} \\ \hline

\multicolumn{1}{c|}{P27} & \multicolumn{1}{c|}{M} & \multicolumn{1}{c|}{\tikzcircle[fill=orange]{3pt}} & \multicolumn{1}{c|}{Few times} & \multicolumn{1}{c|}{No} & \multicolumn{1}{c|}{\tikzcircle[fill=orange]{3pt}} & \multicolumn{1}{c|}{\tikzcirclenew[fill=blue]{3pt}} & \multicolumn{1}{c|}{\tikzcircle[fill=orange]{3pt}} & \multicolumn{1}{c|}{\tikzcircle[fill=orange]{3pt}} & \multicolumn{1}{c|}{\tikzcircle[fill=orange]{3pt}} & \multicolumn{1}{c|}{P66} & \multicolumn{1}{c|}{W} & \multicolumn{1}{c|}{\tikzcirclenew[fill=blue]{3pt}} & \multicolumn{1}{c|}{Never} & \multicolumn{1}{c|}{No} & \multicolumn{1}{c|}{\tikzcircle[fill=orange]{3pt}} & \multicolumn{1}{c|}{\tikzcirclenew[fill=blue]{3pt}} & \multicolumn{1}{c|}{\tikzcirclenew[fill=blue]{3pt}} & \multicolumn{1}{c|}{\tikzcircle[fill=orange]{3pt}} & \multicolumn{1}{c}{\tikzcirclenew[fill=blue]{3pt}} \\ \hline

\multicolumn{1}{c|}{P28} & \multicolumn{1}{c|}{M} & \multicolumn{1}{c|}{\tikzcircle[fill=orange]{3pt}} & \multicolumn{1}{c|}{Once} & \multicolumn{1}{c|}{No} & \multicolumn{1}{c|}{\tikzcircle[fill=orange]{3pt}} & \multicolumn{1}{c|}{\tikzcirclenew[fill=blue]{3pt}} & \multicolumn{1}{c|}{\tikzcirclenew[fill=blue]{3pt}} & \multicolumn{1}{c|}{\tikzcircle[fill=orange]{3pt}} & \multicolumn{1}{c|}{\tikzcircle[fill=orange]{3pt}} & \multicolumn{1}{c|}{P67} & \multicolumn{1}{c|}{M} & \multicolumn{1}{c|}{\tikzcircle[fill=orange]{3pt}} & \multicolumn{1}{c|}{Never} & \multicolumn{1}{c|}{No} & \multicolumn{1}{c|}{\tikzcircle[fill=orange]{3pt}} & \multicolumn{1}{c|}{\tikzcircle[fill=orange]{3pt}} & \multicolumn{1}{c|}{\tikzcirclenew[fill=blue]{3pt}} & \multicolumn{1}{c|}{\tikzcircle[fill=orange]{3pt}} & \multicolumn{1}{c}{\tikzcircle[fill=orange]{3pt}} \\ \hline

\multicolumn{1}{c|}{P29} & \multicolumn{1}{c|}{M} & \multicolumn{1}{c|}{\tikzcircle[fill=orange]{3pt}} & \multicolumn{1}{c|}{Never} & \multicolumn{1}{c|}{No} & \multicolumn{1}{c|}{\tikzcircle[fill=orange]{3pt}} & \multicolumn{1}{c|}{\tikzcircle[fill=orange]{3pt}} & \multicolumn{1}{c|}{\tikzcircle[fill=orange]{3pt}} & \multicolumn{1}{c|}{\tikzcircle[fill=orange]{3pt}} & \multicolumn{1}{c|}{\tikzcircle[fill=orange]{3pt}} & \multicolumn{1}{c|}{P68} & \multicolumn{1}{c|}{M} & \multicolumn{1}{c|}{\tikzcirclenew[fill=blue]{3pt}} & \multicolumn{1}{c|}{Few times} & \multicolumn{1}{c|}{No} & \multicolumn{1}{c|}{\tikzcirclenew[fill=blue]{3pt}} & \multicolumn{1}{c|}{\tikzcirclenew[fill=blue]{3pt}} & \multicolumn{1}{c|}{\tikzcirclenew[fill=blue]{3pt}} & \multicolumn{1}{c|}{\tikzcircle[fill=orange]{3pt}} & \multicolumn{1}{c}{\tikzcircle[fill=orange]{3pt}} \\ \hline

\multicolumn{1}{c|}{P30} & \multicolumn{1}{c|}{M} & \multicolumn{1}{c|}{\tikzcirclenew[fill=blue]{3pt}} & \multicolumn{1}{c|}{Few times} & \multicolumn{1}{c|}{No} & \multicolumn{1}{c|}{\tikzcirclenew[fill=blue]{3pt}} & \multicolumn{1}{c|}{\tikzcirclenew[fill=blue]{3pt}} & \multicolumn{1}{c|}{\tikzcirclenew[fill=blue]{3pt}} & \multicolumn{1}{c|}{\tikzcircle[fill=orange]{3pt}} & \multicolumn{1}{c|}{\tikzcirclenew[fill=blue]{3pt}} & \multicolumn{1}{c|}{P69} & \multicolumn{1}{c|}{M} & \multicolumn{1}{c|}{\tikzcirclenew[fill=blue]{3pt}} & \multicolumn{1}{c|}{Few times} & \multicolumn{1}{c|}{No} & \multicolumn{1}{c|}{\tikzcirclenew[fill=blue]{3pt}} & \multicolumn{1}{c|}{\tikzcirclenew[fill=blue]{3pt}} & \multicolumn{1}{c|}{\tikzcirclenew[fill=blue]{3pt}} & \multicolumn{1}{c|}{\tikzcirclenew[fill=blue]{3pt}} & \multicolumn{1}{c}{\tikzcircle[fill=orange]{3pt}} \\ \hline

\multicolumn{1}{c|}{P31} & \multicolumn{1}{c|}{M} & \multicolumn{1}{c|}{\tikzcirclenew[fill=blue]{3pt}} & \multicolumn{1}{c|}{Never} & \multicolumn{1}{c|}{No} & \multicolumn{1}{c|}{\tikzcircle[fill=orange]{3pt}} & \multicolumn{1}{c|}{\tikzcirclenew[fill=blue]{3pt}} & \multicolumn{1}{c|}{\tikzcirclenew[fill=blue]{3pt}} & \multicolumn{1}{c|}{\tikzcirclenew[fill=blue]{3pt}} & \multicolumn{1}{c|}{\tikzcirclenew[fill=blue]{3pt}} & \multicolumn{1}{c|}{P70} & \multicolumn{1}{c|}{M} & \multicolumn{1}{c|}{\tikzcirclenew[fill=blue]{3pt}} & \multicolumn{1}{c|}{Few times} & \multicolumn{1}{c|}{Some} & \multicolumn{1}{c|}{\tikzcirclenew[fill=blue]{3pt}} & \multicolumn{1}{c|}{\tikzcirclenew[fill=blue]{3pt}} & \multicolumn{1}{c|}{\tikzcirclenew[fill=blue]{3pt}} & \multicolumn{1}{c|}{\tikzcircle[fill=orange]{3pt}} & \multicolumn{1}{c}{\tikzcirclenew[fill=blue]{3pt}} \\ \hline

\multicolumn{1}{c|}{P32} & \multicolumn{1}{c|}{M} & \multicolumn{1}{c|}{\tikzcircle[fill=orange]{3pt}} & \multicolumn{1}{c|}{Never} & \multicolumn{1}{c|}{No} & \multicolumn{1}{c|}{\tikzcirclenew[fill=blue]{3pt}} & \multicolumn{1}{c|}{\tikzcirclenew[fill=blue]{3pt}} & \multicolumn{1}{c|}{\tikzcircle[fill=orange]{3pt}} & \multicolumn{1}{c|}{\tikzcircle[fill=orange]{3pt}} & \multicolumn{1}{c|}{\tikzcircle[fill=orange]{3pt}} & \multicolumn{1}{c|}{P71} & \multicolumn{1}{c|}{M} & \multicolumn{1}{c|}{\tikzcircle[fill=orange]{3pt}} & \multicolumn{1}{c|}{Few times} & \multicolumn{1}{c|}{No} & \multicolumn{1}{c|}{\tikzcircle[fill=orange]{3pt}} & \multicolumn{1}{c|}{\tikzcircle[fill=orange]{3pt}} & \multicolumn{1}{c|}{\tikzcircle[fill=orange]{3pt}} & \multicolumn{1}{c|}{\tikzcircle[fill=orange]{3pt}} & \multicolumn{1}{c}{\tikzcirclenew[fill=blue]{3pt}} \\ \hline

\multicolumn{1}{c|}{P33} & \multicolumn{1}{c|}{M} & \multicolumn{1}{c|}{\tikzcirclenew[fill=blue]{3pt}} & \multicolumn{1}{c|}{Few times} & \multicolumn{1}{c|}{No} & \multicolumn{1}{c|}{\tikzcircle[fill=orange]{3pt}} & \multicolumn{1}{c|}{\tikzcirclenew[fill=blue]{3pt}} & \multicolumn{1}{c|}{\tikzcirclenew[fill=blue]{3pt}} & \multicolumn{1}{c|}{\tikzcircle[fill=orange]{3pt}} & \multicolumn{1}{c|}{\tikzcirclenew[fill=blue]{3pt}} & \multicolumn{1}{c|}{P72} & \multicolumn{1}{c|}{M} & \multicolumn{1}{c|}{\tikzcirclenew[fill=blue]{3pt}} & \multicolumn{1}{c|}{Few times} & \multicolumn{1}{c|}{No} & \multicolumn{1}{c|}{\tikzcirclenew[fill=blue]{3pt}} & \multicolumn{1}{c|}{\tikzcircle[fill=orange]{3pt}} & \multicolumn{1}{c|}{\tikzcirclenew[fill=blue]{3pt}} & \multicolumn{1}{c|}{\tikzcircle[fill=orange]{3pt}} & \multicolumn{1}{c}{\tikzcirclenew[fill=blue]{3pt}} \\ \hline

\multicolumn{1}{c|}{P34} & \multicolumn{1}{c|}{M} & \multicolumn{1}{c|}{\tikzcirclenew[fill=blue]{3pt}} & \multicolumn{1}{c|}{Few times} & \multicolumn{1}{c|}{No} & \multicolumn{1}{c|}{\tikzcirclenew[fill=blue]{3pt}} & \multicolumn{1}{c|}{\tikzcircle[fill=orange]{3pt}} & \multicolumn{1}{c|}{\tikzcirclenew[fill=blue]{3pt}} & \multicolumn{1}{c|}{\tikzcircle[fill=orange]{3pt}} & \multicolumn{1}{c|}{\tikzcirclenew[fill=blue]{3pt}} & \multicolumn{1}{c|}{P73} & \multicolumn{1}{c|}{M} & \multicolumn{1}{c|}{\tikzcirclenew[fill=blue]{3pt}} & \multicolumn{1}{c|}{Once} & \multicolumn{1}{c|}{No} & \multicolumn{1}{c|}{\tikzcirclenew[fill=blue]{3pt}} & \multicolumn{1}{c|}{\tikzcirclenew[fill=blue]{3pt}} & \multicolumn{1}{c|}{\tikzcirclenew[fill=blue]{3pt}} & \multicolumn{1}{c|}{\tikzcircle[fill=orange]{3pt}} & \multicolumn{1}{c}{\tikzcirclenew[fill=blue]{3pt}} \\ \hline

\multicolumn{1}{c|}{P35} & \multicolumn{1}{c|}{W} & \multicolumn{1}{c|}{\tikzcirclenew[fill=blue]{3pt}} & \multicolumn{1}{c|}{Few times} & \multicolumn{1}{c|}{No} & \multicolumn{1}{c|}{\tikzcirclenew[fill=blue]{3pt}} & \multicolumn{1}{c|}{\tikzcirclenew[fill=blue]{3pt}} & \multicolumn{1}{c|}{\tikzcircle[fill=orange]{3pt}} & \multicolumn{1}{c|}{\tikzcircle[fill=orange]{3pt}} & \multicolumn{1}{c|}{\tikzcirclenew[fill=blue]{3pt}} & \multicolumn{1}{c|}{P74} & \multicolumn{1}{c|}{M} & \multicolumn{1}{c|}{\tikzcircle[fill=orange]{3pt}} & \multicolumn{1}{c|}{Never} & \multicolumn{1}{c|}{No} & \multicolumn{1}{c|}{\tikzcircle[fill=orange]{3pt}} & \multicolumn{1}{c|}{\tikzcirclenew[fill=blue]{3pt}} & \multicolumn{1}{c|}{\tikzcirclenew[fill=blue]{3pt}} & \multicolumn{1}{c|}{\tikzcircle[fill=orange]{3pt}} & \multicolumn{1}{c}{\tikzcircle[fill=orange]{3pt}} \\ \hline

\multicolumn{1}{c|}{P36} & \multicolumn{1}{c|}{M} & \multicolumn{1}{c|}{\tikzcirclenew[fill=blue]{3pt}} & \multicolumn{1}{c|}{Never} & \multicolumn{1}{c|}{No} & \multicolumn{1}{c|}{\tikzcirclenew[fill=blue]{3pt}} & \multicolumn{1}{c|}{\tikzcirclenew[fill=blue]{3pt}} & \multicolumn{1}{c|}{\tikzcirclenew[fill=blue]{3pt}} & \multicolumn{1}{c|}{\tikzcircle[fill=orange]{3pt}} & \multicolumn{1}{c|}{\tikzcirclenew[fill=blue]{3pt}} & \multicolumn{1}{c|}{P75} & \multicolumn{1}{c|}{M} & \multicolumn{1}{c|}{\tikzcirclenew[fill=blue]{3pt}} & \multicolumn{1}{c|}{Few times} & \multicolumn{1}{c|}{No} & \multicolumn{1}{c|}{\tikzcirclenew[fill=blue]{3pt}} & \multicolumn{1}{c|}{\tikzcirclenew[fill=blue]{3pt}} & \multicolumn{1}{c|}{\tikzcirclenew[fill=blue]{3pt}} & \multicolumn{1}{c|}{\tikzcircle[fill=orange]{3pt}} & \multicolumn{1}{c}{\tikzcirclenew[fill=blue]{3pt}} \\ \hline

\multicolumn{1}{c|}{P37} & \multicolumn{1}{c|}{M} & \multicolumn{1}{c|}{\tikzcirclenew[fill=blue]{3pt}} & \multicolumn{1}{c|}{Few times} & \multicolumn{1}{c|}{Some} & \multicolumn{1}{c|}{\tikzcirclenew[fill=blue]{3pt}} & \multicolumn{1}{c|}{\tikzcircle[fill=orange]{3pt}} & \multicolumn{1}{c|}{\tikzcirclenew[fill=blue]{3pt}} & \multicolumn{1}{c|}{\tikzcircle[fill=orange]{3pt}} & \multicolumn{1}{c|}{\tikzcirclenew[fill=blue]{3pt}} &

\multicolumn{10}{c}{\multirow{3}{*}{\begin{tabular}[c]{@{}c@{}}\textbf{Legend:} M: Man | W: Woman | \tikzcirclenew[fill=blue]{3pt}: Tim | \tikzcircle[fill=orange]{3pt}: Abi\\ MT: Motivation | SE: Self-efficacy | R: Risk \\ IP: Information processing | L: Learning\end{tabular}}} \\ \cline{1-10}

\multicolumn{1}{c|}{P38} & \multicolumn{1}{c|}{M} & \multicolumn{1}{c|}{\tikzcirclenew[fill=blue]{3pt}} & \multicolumn{1}{c|}{Never} & \multicolumn{1}{c|}{No} & \multicolumn{1}{c|}{\tikzcirclenew[fill=blue]{3pt}} & \multicolumn{1}{c|}{\tikzcirclenew[fill=blue]{3pt}} & \multicolumn{1}{c|}{\tikzcirclenew[fill=blue]{3pt}} & \multicolumn{1}{c|}{\tikzcircle[fill=orange]{3pt}} & \multicolumn{1}{c|}{\tikzcircle[fill=orange]{3pt}} & \multicolumn{10}{l}{} \\ \cline{1-10}

\multicolumn{1}{c|}{P39} & \multicolumn{1}{c|}{M} & \multicolumn{1}{c|}{\tikzcirclenew[fill=blue]{3pt}} & \multicolumn{1}{c|}{Few times} & \multicolumn{1}{c|}{Some} & \multicolumn{1}{c|}{\tikzcirclenew[fill=blue]{3pt}} & \multicolumn{1}{c|}{\tikzcirclenew[fill=blue]{3pt}} & \multicolumn{1}{c|}{\tikzcirclenew[fill=blue]{3pt}} & \multicolumn{1}{c|}{\tikzcircle[fill=orange]{3pt}} & \multicolumn{1}{c|}{\tikzcirclenew[fill=blue]{3pt}} & \multicolumn{10}{l}{} \\ \hline
\end{tabular}
\end{table*}



\begin{comment}


The five facets used by the GenderMag method are presented in Table~\ref{tab:gendermagfactes}. The facets are used to define personas (e.g., Abi and Tim). GenderMag highlights that differences relevant to inclusiveness lie not in a person's gender identity but in the facet values themselves~\cite{hill2017gender}. Nevertheless, Abi's facet values are more frequent in women than in other genders, and Tim's facet values are more frequent in men than in other genders. 

%(Figure~\ref{fig:abbypersona})

\begin{table}[!ht]\scriptsize
\centering
\vspace{-2.5mm}
\caption{GenderMag facets~\cite{burnett2016gendermag}}
\label{tab:gendermagfactes}
\newcommand{\pb}[1]{\parbox[t][][t]{1.0\linewidth}{#1} \vspace{-2pt}}

\begin{tabular}{p{12mm}|p{62mm}}
\hline
\multicolumn{1}{>{\centering\arraybackslash}m{12mm}|}{\textbf{GenderMag Facets}} & \multicolumn{1}{>{\centering\arraybackslash}m{62mm}}{\textbf{Definition}} \\ \hline \hline

Motivation & \pb{Women tend (statistically) to be motivated to use technology for what they can accomplish with it, whereas men are often motivated by their enjoyment of technology per se~\cite{simon2000impact, cassell2002hand, margolis2002unlocking, hou2006girls, kelleher2009barriers, burnett2010gender, burnett2011gender, hallstrom2015gender}. This difference can affect which software features users choose to use}. \\ \hline 

Information processing styles & \pb{To solve problems, people often need to process new information. Women are more likely (statistically) to process new information comprehensively—gathering fairly complete information before proceeding—but men are more likely to use selective styles—following the first promising information, then backtracking if needed~\cite{cafferata1989gender, meyers1991exploring, coursaris2008empirical, riedl2010there, meyers2015revisiting}. Each style has advantages, but either is at a disadvantage when not supported by the software.} \\ \hline

Computer self-efficacy & \pb{Self-efficacy is a person's confidence about succeeding at a specific task, which influences their use of cognitive strategies, persistence, and strategies for coping with obstacles. Empirical data have shown that women often have lower computer self-efficacy than men, which can affect their behavior with technology~\cite{margolis2002unlocking, durndell2002computer, hartzel2003self, beckwith2005effectiveness, beckwith2006tinkering, burnett2010gender, burnett2011gender, singh2013role, huffman2013using}.} \\ \hline

Risk aversion & \pb{Research shows that women statistically tend to be more risk-averse than men~\cite{weber2002domain, dohmen2011individual, charness2012strong}. These results span numerous decision-making domains, such as ethics, investment, gambling, health/safety, and career. Risk aversion with software usage can impact users' decisions as to which feature sets to use.} \\ \hline

Learning: by Process vs. by Tinkering & \pb{Research across age groups and professions reports women being statistically less likely to playfully experiment (“tinker”) with software features new to them, compared to men~\cite{beckwith2006tinkering, hou2006girls, rosner2009learning, burnett2010gender, cao2010debugging, chang2014specialization}. However, when women do tinker, they tend to be more likely to reflect during the process and thereby sometimes profit from it more than men do.} \\ \hline \hline
\end{tabular}
\end{table}

\end{comment}
&&&&

\begin{table*}[ht]
\resizebox{.6\textwidth}{!}{
\begin{tabular}{ccc}
\Xhline{1pt}
\rowcolor[HTML]{EFEFEF} 
\textbf{Demographics}                                                                                                                          & \textbf{Options}                                               & \textbf{Count }(n=18)                                                                                           \\ \Xhline{1pt}
\textbf{\begin{tabular}[c]{@{}c@{}}Age\end{tabular}}                                                                                 & Number                                                         & \begin{tabular}[c]{@{}c@{}} $Mean$ = 16.3 years\\ $SD$ = 1.3 years \\ $Range$ = 14-19 years\end{tabular}                                           \\ \hline
\rowcolor[HTML]{F3F3F3} 
\cellcolor[HTML]{F3F3F3}                                                                                                                       & Male                                                           & 16                                                                                                       \\
\rowcolor[HTML]{F3F3F3} 
\multirow{-2}{*}{\cellcolor[HTML]{F3F3F3}\textbf{Gender}}                                                                                      & Female                                                         & 2                                                                                                        \\ \hline
                                                                                                                                               & South Korea                                                    & 11                                                                                                       \\
                                                                                                                                               & United States                                                  & 3                                                                                                        \\
                                                                                                                                               & Malaysia                                                       & 1                                                                                                        \\
                                                                                                                                               & Japan                                                          & 1                                                                                                        \\
                                                                                                                                               & Spain                                                          & 1                                                                                                        \\
\multirow{-6}{*}{\textbf{Nationality}}                                                                                                         & Denmark                                                        & 1                                                                                                        \\ \hline
\rowcolor[HTML]{F3F3F3} 
\cellcolor[HTML]{F3F3F3}                                                                                                                       & Scripting                                                      & 15                                                                                                       \\
\rowcolor[HTML]{F3F3F3} 
\cellcolor[HTML]{F3F3F3}                                                                                                                       & 3D Modelling                                                   & 7                                                                                                        \\
\rowcolor[HTML]{F3F3F3} 
\cellcolor[HTML]{F3F3F3}                                                                                                                       & Building                                                       & 5                                                                                                        \\
\rowcolor[HTML]{F3F3F3} 
\cellcolor[HTML]{F3F3F3}                                                                                                                       & GFX                                                            & 2                                                                                                        \\
\rowcolor[HTML]{F3F3F3} 
\cellcolor[HTML]{F3F3F3}                                                                                                                       & Animation                                                      & 1                                                                                                        \\
\rowcolor[HTML]{F3F3F3} 
\multirow{-6}{*}{\cellcolor[HTML]{F3F3F3}\begin{tabular}[c]{@{}c@{}}\textbf{Profession}\\ \textit{(multi-choice possible)}\end{tabular}}            & Game Design                                                    & 1                                                                                                        \\ \hline
\textbf{\begin{tabular}[c]{@{}c@{}}Roblox Creation \\ Experience\end{tabular}}                                                                 & Number                                                         & \begin{tabular}[c]{@{}c@{}}$Mean$ = 3.4 years\\ $SD$ = 2 years\\ $Range$ = 4 months $\sim$7 years\end{tabular} \\ \hline
\rowcolor[HTML]{F3F3F3} 
\cellcolor[HTML]{F3F3F3}                                                                                                                       & Discord                                                        & 17                                                                                                       \\
\rowcolor[HTML]{F3F3F3} 
\cellcolor[HTML]{F3F3F3}                                                                                                                       & DevForum  & 12                                                                                                       \\
\rowcolor[HTML]{F3F3F3} 
\cellcolor[HTML]{F3F3F3}                                                                                                                       & Naver Cafe                                                     & 4                                                                                                        \\
\rowcolor[HTML]{F3F3F3} 
\cellcolor[HTML]{F3F3F3}                                                                                                                       & KakaoTalk                                                          & 7                                                                                                        \\
\rowcolor[HTML]{F3F3F3} 
\cellcolor[HTML]{F3F3F3}
\multirow{-4}{*}{\cellcolor[HTML]{F3F3F3}\begin{tabular}[c]{@{}c@{}}\textbf{Participating Communities}\\ \textit{(multi-choice possible)} \end{tabular}} & Reddit                                                         & 1                                                                                                        \\ \Xhline{1pt}
\end{tabular}
}
\caption{Participant information overview. To anonymize participants, we present aggregate statistical data to describe them and refer to each person with an alias C1-C18. DevForum is managed by Roblox Corp. Other communities are managed by individuals or teams of volunteers.}%Alternative for anonymizing participants more. }
\label{tab:demographics}
\Description{A table that shows the information of the user study participants. The first column is the list of demographics, which are age (14 to 19), gender, nationality, profession in Roblox development, years of Roblox creation experience, and the communities that participants use. The major data of the table is as follows. The mean age is 16.3, the majority gender is male, the majority nationality is South Korea, the majority profession is scripting, the mean Roblox creation experience is 3.4 years, and the main communities that the participants used are Discord and DevForum.}
\end{table*}


\subsection{Ethical considerations}
This study was approved by the KAIST Institutional Review Board. However, as we involved teenage participants, we took additional precautions beyond those required by our institution. Participants were sent the interview questions in advance, and before the interview, we again explained the interview process thoroughly to ensure that they fully understood the interview contents and \rv{how} the data would be collected for payment and analysis. Also, we emphasized that participants could skip a question or end the interview if they did not want to answer. Further, for payment, all participants received their parent or guardian's signature. After receiving the participants' consent, we proceeded with the interviews.

\subsection{Analysis}
We conducted inductive thematic coding~\cite{braun2006using} \rv{following the six steps of Braun's method on interview transcripts to understand the experiences of teens in Roblox developer communities and the benefits and challenges they perceived}.
%to analyze the transcriptions. 
The process started with uploading all audio recordings of the interview to Dovetail.\footnote{\url{https://dovetail.com/}} The transcriptions were automatically done in English or Korean based on the interview language by Dovetail. Then, the first and second authors each \rv{open-}coded six interview transcripts independently while correcting inaccurate transcriptions along the way, \rv{identifying} low-level \rv{themes} \rv{focusing on how various online communities support teen developers of Roblox but also the challenges they face as a teenager}. Next, the two authors discussed the \rv{themes} to reduce the initial \rv{themes} from 587 low-level concepts to 15 higher-level themes. The final themes were merged or split iteratively via Dovetail through rounds of discussion \rv{on overlapping themes or disagreements}. 
%After agreeing on the final codes, 
\rv{Based on the refined themes,} the two interviewers \rv{engaged in another iterative discussion regarding remaining inconsistencies, disagreements, and newly emarging themes, re-coding} two interviews (C5, C12) independently again \rv{to confirm agreement}. {Cohen's Kappa was calculated to determine} inter-rater agreement, \rv{reaching a value of 0.78}. After reaching agreement \rv{on the themes}, the first author coded the remainder of the interviews. The \rv{final list of themes} can be seen in Appendix~\ref{appendix:codebook}.

\subsection{Limitations}
Before presenting our results, we acknowledge several limitations of this paper. First, our participant sample does not fully represent the global population of teen Roblox developers. Though we do not know official statistics on teen Roblox developers, there is an absence of participants from South America, Africa, and other regions \rv{in our sample (See Table ~\ref{tab:demographics})}. Since different cultures have communities operating in their languages, our findings are constrained by the language capabilities of the research team. Moreover, even within the same language, each community has unique norms; for this study, we recruited primarily from larger communities with more than 10,000+ participants, which may limit the generalizability of our findings. 

Further, because our recruitment calls included criteria that required participants to have made at least two creations, our participants are likely skewed toward those who are more active community members. As seen in Table ~\ref{tab:demographics}, the majority of the participants in this study had spent more than two years creating content for Roblox, so true newcomers who had just begun creating are underrepresented in our sample. 
