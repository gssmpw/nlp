\section{Related Work}
Our paper builds on three primary bodies of prior work. First, we explore research on game development with young users.  Second, we discuss research on the social dynamics of online developer communities. Finally, we present prior work on risks and challenges for youth \rv{creators} in online communities.

% Sign posts. 1) Remind readers - what to expect beginning , 2) heres why I'm telling u this at the end
% Write as story setting = related work

\subsection{Youth Game Design and Development as Digital Play}
Play is widely recognized as a crucial process for children's learning and development~\cite{singer2006roberta}. In today's digital age, the concept of play extends from physical to virtual environments like TikTok, YouTube, and various social media platforms~\cite{10.1145/3173574.3174233}. \rv{Resonating with theories of intrinsic motivation and participatory culture~\cite{jenkins2009confronting}, an increasing number of teenagers have become both consumers and producers--termed ``prosumers'' in digital spaces, participating by creating their own content in the form of blogs, gameplay streams, and game mods~\cite{harlan2012teen, lombana2020youth}.} While these ``digital playgrounds'' provide new avenues for engagement \rv{and learning for youth}, they are mainly designed by adults to meet adult-centric goals and constraints. However, the digital autonomy of children and adolescents --- the freedom to explore and express themselves in digital spaces---can be important, enabling children to gain independence, build peer relations, and form a self-identity~\cite{davison2006adolescent, roseth2008promoting, wang202312}. Game development by the youth provides a unique opportunity to achieve this autonomy by creating and exploring their own games or virtual worlds, offering a type of playful experimentation not available on traditional adult-focused platforms.


Extensive research in HCI has shown that children and teenagers can benefit from learning to design and create games. Recent advances in technology~\cite{10.1145/3544548.3580976, 10.1145/2470654.2481360, 10.1145/3613904.3642895}, combined with the rise of computational education in schools~\cite{rijo2022computational, bers2022state}, and the development of child-friendly programming tools like ScratchJR, and Lego Mindstorms~\cite{flannery2013designing, kert2020effect}, have lowered the barriers to game design. This democratization of game creation, driven by indie developers and modding~\cite{10.1145/3375184, freeman2019exploring, sotamaa2010game} is complemented by studies showing children can also make games. Studies have demonstrated that involving children as co-designers in ``serious'' games helps them better grasp complex real-world issues~\cite{tucker2019broke}. Further, children learn computational concepts~\cite{10.1145/3544548.3581272, 10.1145/3025453.3025847, 10.1145/3311927.3323129} through both ``serious''~\cite{10.1145/3613905.3650833, kert2020effect} and ``non-serious''~\cite{troiano2019my} game design. Collaborative game design has also been linked to improved communication skills and higher \rv{learning involvement} of children~\cite{denner2014pair, gritschacher2012standing}. %Also, a massive collaboration between children, on a LEGO-style website, was highly motivated by entertaining their friends and learning from other children~\cite{gritschacher2012standing}. 
%Further, children who were self-motivated in interest to learn concepts such as data structures in Scratch had better learning outcomes~\cite{10.1145/3491102.3502124}. 
Broadly, giving teenagers autonomy in creating games leads to learning benefits across a variety of domains~\cite{10.1145/3491102.3502124, gritschacher2012standing}. We focus on Roblox, for its flexibility in game design options, kid-friendly programming language (Luau), and tools (Roblox Studio), making it a strong platform for teen-led game development. In this study, we explore the potential benefits that teen game designers \& developers can gain from participating in the supportive social ecosystem of creating games.

\subsection{Online Developer communities}
%1. Developer communities are valuable spaces for support-seeking. 2. A lot of research has studied developer spaces focused around adults, such as Github teams, and found benefits X, Y, and Z.

Online developer communities \rv{built around platforms such as Github provide essential} spaces for learning, \rv{collaboration}, and social interaction. Development \rv{is rarely a straightforward process, and these communities often become hubs for mutual support.} Much research on adult-focused developer communities, such as GitHub and StackOverflow, has found that these communities \rv{provide} valuable opportunities for knowledge sharing~\cite{10.1145/2531602.2531659, 8417152}, enhancing individual expertise~\cite{may2019gender}, and collaborative coding to complete projects~\cite{vasilescu2014continuous}. \rv{Similarly, indie game developer communities serve as \textit{communities of practice}, where small, flexible team dynamics foster autonomy and creativity. In these settings, members often adapt their roles democratically, emphasizing collaboration and shared ownership~\cite{freeman2019exploring}.}

\rv{While much of the prior work in this space focuses on \textit{adult} developer communities, research on teen-focused developers, particularly in the context of Scratch, has uncovered a breadth of opportunities and challenges.} Studies reveal the benefits of developer communities in learning computational skills and exchanging knowledge with peers~\cite{brennan2021kids}. \rv{Further, in Scratch team challenges, studies demonstrate that youth deploy sophisticated socialization skills: negotiating dynamic leadership styles for each project, building interpersonal trust, and cultivating common ground~\cite{fields2013understanding, aragon2009tale, chou2018designing}.} However, \rv{youth face distinct communication obstacles compared to adults. They often struggle to} articulate their thoughts~\cite{10.1145/3491102.3502124, roque2016children} and adapt to the \rv{nuanced} engagement \rv{expectations among different online communities}~\cite{fields2015have}. \rv{Critically, peer recognition and feedback are pivotal motivations for community engagement—universally important, but especially crucial for teenaged developers}~\cite{kafai2012collaborative}. \rv{Young creators, still developing their creative identities and networks, rely more intensely on peer recognition and feedback from engagement. These social interactions are fundamental to driving participation, building confidence, and fostering belonging within digital learning environments.}
%Recognition and feedback from peers play critical roles in these communities, as they motivate participation and support members’ sense of belonging~

\rv{In contrast to previously studied Scratch communities, Roblox developer communities, while similarly youth-driven, operate within a distinct social and economic framework. Unlike Scratch, which prioritizes education and creativity, Roblox combines social engagement with broader agency in creations and a deeply embedded in-platform monetization ecosystem. Work by Zhang et al. has shown that Roblox teen developer communities engage in forms of collective ideation, discussing game design strategies and sharing feedback through external social media platforms such as Reddit~\cite{zhang2024understanding}. Taken together, these findings underscore the importance of understanding online developer communities as dynamic environments that not only facilitate technical skill development but also nurture collaboration and social belonging for teen developers.}
\rv{While, previous research} has primarily analyzed teen developers' \rv{public comments and forum interactions, our study directly engages teen developers. By conducting first-hand interviews, we uncover rich, nuanced insights into their lived experiences, motivations, and the challenges of participating within these youth-driven developer communities}.

%However, prior has focused primarily on external discussions and has not captured the firsthand experiences of teen developers across different communities. By directly engaging with youth through interviews, our study provides deeper insights into their experiences, enabling us to understand how they perceive the benefits and challenges of participating in these youth-driven spaces.}

%This paper examines Roblox developer communities, which have significant appeal to younger audiences voluntarily more focused on engagement than education in Scratch and thus may have significantly different social dynamics than developer communities with primarily older users. In contrast to previously studied youth-led programming spaces like Scratch, Roblox features active in-platform monetization, which complicates the motivations for participation.


%In addition, adult-oriented developer communities may not have the type of patience for the processes of identify formation that youth-oriented communities may be designed to support~\cite{ringland2016will}.



\subsection{Child safety issues in online developer communities}
%J: 1. While there are many benefits from participation in online developer communities, there are also challenges and risks surrounding conflict and safety. 2. Previous work has identified challenges/safety risks A, B, and C in online developer communities.

Participation in online developer communities offers benefits, but \rv{it may also expose developers to} a variety of challenges and risks. Research has \rv{consistently} highlighted issues \rv{in online social spaces} such as toxic comments, name-calling, and offensive language~\cite{10.1145/3510003.3510111, ferreira2021shut, cheriyan2021towards}, as well as programming-specific risks like code theft~\cite{chatterji2016code} and malicious hacking in shared code~\cite{kera2012hackerspaces}. Another layer of risks \rv{emerges} in developer communities on social media platforms like Reddit and Discord, \rv{where many Roblox developer communities are based}. These platforms bring \rv{the} traditional threats \rv{that have been identified in studies of social media moderation}, including discrimination~\cite{heung2024vulnerable} and personal attacks~\cite{park2022measuring} \rv{to developers}. For example, members of Reddit communities may experience repeated harassment across multiple subcommunities~\cite{kumar2023understanding}. Children are particularly vulnerable \rv{in these environments due to the possibility of encountering more severe risks such as} sexual abuse, cyber grooming, and exposure to explicit content~\cite{ali2021child, freed2023understanding}. \rv{Exposure to} explicit material \rv{within developer communities} can lead to \rv{long-term detrimental effects on mental health}~\cite{tanni2024examining}. Furthermore, \rv{prior work has suggested that adolescents may face emotional challenges stemming from addiction to or dependence on these platforms}~\cite{wisniewski2015resilience}.

\rv{The Roblox ecosystem presents unique challenges due to its distinctive community structure. Unlike traditional platforms, Roblox blurs the boundaries between players and developers, creating a complex social dynamic where young creators often emerge directly from the player community. Kou et al. have pointed out that policies being unclear between creators and players as one of the reasons for harmful games including those featuring racist or misogynist content\cite{kou2023harmful}. This fluid transition can inadvertently normalize problematic content, as creators may unconsciously draw inspiration from games that incorporate harmful themes or design elements. Also, the platform's proprietary development environment creates a form of technological "lock-in" that constrains monetization options only within Roblox. This can pressure teen developers to prioritize engagement and monetization over ethical design considerations. Moreover, what researchers term "aspirational labor" drives young creators to invest significant time and effort into content creation, often with minimal compensation~\cite{lombana2020youth}. The promise of potential future success motivates children to engage in unpaid creative work. Media critiques have pointed out Roblox's potentially exploitative practices of relying on young users to create content without proper compensation~\cite{The_Guardian_2022}. }

%\rv{\textbf{Revised paragraph ABOVE.}Roblox developer communities also host a series of distinctive safety risks that are unique to game design communities. First, Roblox intertwines player and developer roles as many creators started as players. Kou et al. have pointed out that policies being unclear between creators and players is one of the reasons for harmful games including those featuring racist, misogynist problematic themes, often persist\cite{kou2023harmful}. This lack of clarity can lead to unintended violations or the normalization of risky design ideas during community-based co-ideation. For example, players who become creators may draw inspiration from harmful games they have encountered, perpetuating the spread of problematic content.} Second, Roblox developers faced exploitation by the proprietary nature of Roblox’s ecology of its unique language and tools creating a form of lock-in~\cite{kou2024ecology}.  This lock-in effect of Roblox developers can pressure young developers to prioritize polarity and profit within Roblox, sometimes at the expense of ethical design choices. Furthermore, media have criticized Roblox for exploitative practices of relying on young users to create content without proper compensation~\cite{The_Guardian_2022}.  \rv{Aspirational labor complicates this further; monetization opportunities make children invest time and effort into unpaid or underpaid creative work with the hope of career success in the future~\cite{lombana2020youth}. While this builds skills and networks, systemic inequalities often hinder opportunities, making success disproportionately accessible to those with resources. These pressures can normalize risky practices and perpetuate self-exploitation among young creators.}

\rv{Though there has been much prior literature on online safety and aspirational labor, limited research has explored the overlapping dynamics of these phenomena in platforms like Roblox, where player and developer roles merge in communities where youth have significant agency. Our study addresses this gap by examining how youth perceived and experienced potential risks faced by teen developers, highlighting the major risks and proposing guidelines for potential interventions that balance opportunities and safeguards in youth-driven digital economies.}

%Moreover, research has identified several safety issues within the Roblox environment, including exposure to viruses, predatory behavior, harmful game designs, and exploitation. For instance, Kou et al., through analysis of Reddit comments, found that developers faced malicious activities and that the proprietary nature of Roblox’s system presents challenges, as the ecology of its unique language and tools created a form of lock-in.~\cite{kou2024ecology}. Also, Kou's study on Reddit found that harmful game designs existed with inadequate moderation of problematic content like racism or misogyny~\cite{kou2023harmful}. Furthermore, media have criticized Roblox for exploitative practices of relying on young users to create content without proper compensation~\cite{The_Guardian_2022}.

%Together, these findings underscore the multifaceted risks teenage Roblox developers face in communities and highlight the need for deeper exploration of how teen developers are dealing with the risks and ways to safeguard the well-being of young developers.
