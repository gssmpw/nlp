\section{Results}
\rv{In this section, we summarize why teen developers joined online communities and which ones they joined. We report the benefits they found and the challenges they faced during their community experiences. For each challenge, we discuss participants' coping strategies.}

\subsection{How do teen Roblox developers utilize online developer communities? (RQ1)}
\rv{This section first introduces the motivations that drive teen Roblox developers to participate in developer communities. Next, we show how teen developers use the communities to meet multifaceted technical, social, and career needs, depending on the distinct characteristics of each community.}
\begin{figure*}
    \centering
    \includegraphics[width=0.9\linewidth]{figure/Online_Communities.png}
    \caption{Different online developer communities}
    \label{fig:communities}
    \Description{Five screenshots of various online developer communities, which are DevForum, Discord Server, Reddit, Naver Cafe, and KaKaoTalk. Each screenshot shows interactions between users within the community.}
\end{figure*}

%\rv{Note: Which communities they're joining -> what triggers them to join a new community? }\\
 
\indent \rv{\textbf{Participants joined communities to get help or for social support.}} \rv{For teen Roblox developers, the journey into online communities often started with a spark of technical necessity and social curiosity. For example, C12 started with a simple search: “Roblox scripting help.” Stuck on a bug that refused to cooperate, they found themselves on DevForum, which appeared in the top 3 results in a Google search. Others, like C7, found their way into communities through a friend or searching popular social media platforms to find somebody to talk to about Roblox development. Per C2, \textit{“my school friend sent me an invite to this scripting server. At first, it was just for fun, but then I realized I could ask questions there too. It wasn’t just work—it felt like hanging out.”} Korean participants like C16 also mentioned searching open chat rooms in KaKaoTalk or Naver Cafe, which are both frequently used social media platforms in Korea, for a friend to talk about Roblox development.}\\

%DevForum, with its structured format and wealth of archived discussions, was often the first stop.}

\rv{\textbf{Different online platforms offered unique interaction styles and community characteristics.} Commonly mentioned platforms were DevForum (Roblox official community), Discord, Reddit, and Korea-specific platforms like Naver Cafe and KaKaoTalk, as depicted in Fig. ~\ref{fig:communities}. These communities revealed dramatically different characteristics that profoundly shaped user experiences based on community ownership, modality, and culture. For example, DevForum is a structured}, text-based platform for learning, question-asking, and solution searching. Moderated by Roblox, it had more limitations on what could be shared \rv{due to} safety precautions. In contrast, social-media-based developer communities such as Discord presented a more dynamic environment with more functions.
\begin{quote}
    \textit{[Roblox-related Discord servers] are friendly, more energetic, and a little bit more chaotic but have more freedom, and flexibility and don't have as stiff an atmosphere as DevForum. (C17)}
\end{quote}

\rv{This was possible because Discord supported more interactivity between users through features such as voice chats and screen shares. Most importantly, anyone could create their own community servers and decide the rules. Participants mention sharing funny memes as a joke, playing games together in independent voice chats with community friends, or showing their live coding progress to community members like a ``code with me (C9)'' in peer-led communities. To many participants, Discord communities were preferred for the ``real-time communication (C10)'', and the ability to start your own community easily}. \\
 
\rv{\textbf{Participants strategically navigated multiple communities to meet diverse personal and professional needs.}} As teen developers explored and adapted to different communities, they began strategically using these digital landscapes. \rv{As C13 explained, their approach shifted from widespread engagement to more purposeful targeted participation:}
\begin{quote}
    \rv{\textit{Twitter is where Roblox employees can easily discover developers by looking at their work, while Naver cafes are preferred by professional companies. Twitter allows light self-promotion and easy visibility. Discord enables real-time work sharing. As I developed deeper relationships in Discord developer communities, I realized uploading my creations to every community was inefficient, so now I focus on Discord communities only. (C13)}}
\end{quote}
 
\rv{Korean teen developers like C13 used common social platforms in Korea including Naver Cafes and KaKaoTalk Groups, which also served as a first step into other global developer communities. C14 first learned about Discord after talking with members of KaKaoTalk Groups and Naver Cafes and got to know other servers after being familiar with Discord. However, for most participants, including C13 and C14, DevForum and Discord were reported to be by far the most frequently used, so our analysis focuses primarily on these two platforms.} 
% \rv{(transition sentence. For Korean teenager developers, the Naver Cafe and KakaoTalk served to introduce them as the first step into the developer communities. One QUOTE ). Our analysis focused primarily on DevForum and Discord due to their overwhelmingly dominant usage among participants. 

\rv{These platforms illustrated how seemingly similar activities could manifest entirely differently across digital spaces. For instance, the simple act of sharing game creations took on distinct characteristics depending on the platform. On DevForum, sharing meant formal presentations with detailed technical explanations and structured feedback. In contrast, Discord communities transformed the same activity into a casual, real-time interaction—developers might quickly screen share their work, get immediate reactions, and engage in spontaneous collaborative discussions.} The varying usage is depicted in Fig. ~\ref{fig:usage}.


\rv{Teen developers visited DevForum for formal questions and to access comprehensive game development resources (e.g., the Q\&A database). In contrast, participants used Discord as a decentralized ecosystem where teen developers could create their own communities. C11, for instance, explained this diversity of communities, noting~\textit{``I used one for scripting help, another for finding jobs.''} C18's approach exemplified this strategic community usage, participating in over 200 Discord servers to gather a diversity of inspirations:~\textit{``I get ideas from community discussions across platforms. I noticed kids learning grammar, so I created a game that improves English skills by using GPT to generate random word explanations.''} Some participants created their own Discord communities to connect with their games' players (C7), their YouTube followers (C16), or their UGC buyers to get feedback on their new 3D models (C4). C4 mentioned how he realized his UGC was not as popular as he thought it would be in the market, compared to the efforts he made. He initially decided to hear from other developers who were his customers on what features to add, which led him to make his own Discord community with developers which eventually reached a size of 2700 members at the time of the interview.} 

\rv{This diversity of approaches shows how teen developers don't just use communities—they strategically construct digital ecosystems that support their learning, creativity, and professional growth by understanding and leveraging the unique social dynamics of each platform.}


\begin{figure}
    \centering
    \includegraphics[width=1.0\linewidth]{figure/Community_Usage.png}
    \caption{Comparison of online developer communities. We only visualize the most commonly used with at least one differing usage.}
    \label{fig:usage}
    \Description{\rv{(Figure updated)}A ben-diagram that compares Discord and DevForum, each represented by circles. The figure shows the common usage purposes in the overlap of two circles, and the remaining parts list the unique usage purposes of each community.}
\end{figure}


\subsection{Benefits from participating in developer communities (RQ2)}
This section outlines the benefits reported by interviewees from their participation in online developer communities. \rv{Participants enthusiastically reported the benefits that each of their communities could support.}


\subsubsection{\rv{\textbf{Learning and improving technical skills.}}} 
%J: I think we want to go into significantly more depth here to identify the breadth of skills and how the community helped develop them.

%\textbf{Self-directed and Collaborative Learning}
\rv{When participants first joined developer communities, they often joined for the purpose of asking very specific questions such as \textit{"Why does my structure keep collapsing? How do I fix this phrase error?"} Our interviews resoundingly confirmed that learning technical skills to create a game or design game elements was the most prevalent benefit, being the initial motivator for participating in the communities in the first place. } 
%Learning technical skills to create a game or element was the most prevalent benefit echoed by almost all participants. 
Such learning typically emerged from searching for related questions \rv{that had previously been} answered or asking \rv{new questions} to the Q\&A channel within the community. \rv{C4 shared that in the beginning, he asked a lot of questions to learn how to use the Roblox Studio UI and learn answers that didn't show up when searched.}%New creators frequently asked questions about bugs or problems they encountered and received solutions from the community. 

Experienced \rv{Roblox} developers, with over five years in their profession, continued to seek advice from community members to resolve complex issues in development. C10 mentioned one of his communities having ``truly skilled" developers well beyond the level of hobbyists, whose advice was like a guaranteed map to resolve whatever problems \rv{community members} were having. Even participants who were usually passive observers gained valuable insights, such as improved 3D modeling techniques or new mathematical concepts for animation \rv{by looking at the active community-level Q\&A channels}.  %\textcolor{white}{space}\\
\begin{quote}
    \rv{\textit{I have learned a lot of stuff. I [wouldn]'t be good like this to make my own games. If DevForum didn't exist, then I couldn't solve problems that I struggled [with]... that forum is really important.}} (C18)
\end{quote}

    %\textit{In the beginning, I asked a lot of questions. The Roblox Studio UI can be daunting at first time. For example, I was wondering ``Why does my structure keep collapsing?" I ended up asking people a lot of questions like this that don't show in searching.} (C4)}
    
\rv{Many of our participants also described learning from collaborative team projects they joined within the community. Participants would find opportunities to join a group for game development in the channels dedicated to hiring collaborators (section 4.1). Working in small groups was} another big part of \rv{the learning process} as participants often discovered their preferred roles (e.g., game design, scripting) while trying out different team dynamics. Collaborating with more experienced developers facilitated skill development, as experienced team developers would give advice and tips on game development including file organization or shortcuts. \rv{Participants like C12 mentioned how team collaborations would also build relationships making it easier to ask questions in a 1:1 manner instead of publicly asking within the whole community.}
%\textcolor{white}{space}\\

\begin{quote}
\rv{\textit{My collaborators, friends actually, are really amazing. They even helped me build a map once. I ask them a lot of questions when I'm stuck, \rv{too much to post in the official Q\&A}. And sometimes I did help them too once - very proud of that. (C12)}}
\end{quote}

\rv{Additionally, shared} community resources \rv{supported participants' self-directed learning.  C6 acknowledged how using free 3D assets helped him transition from beginner-level creations to complex, customized designs.} Sophisticated 3D models, music, codes, \rv{Roblox Studio updates,} and AI tips \rv{were shared} within the communities - some having separate channels for sharing. These helped participants \rv{to stay updated, quickly build games, and later} explore how other community members structured their code or composed 3D models: 

\begin{quote}
\rv{\textit{I feel around 40\% of the stuff I learned was from the communities. The resources really help you \rv{because it's daunting to build everything at first.}. People just share [the resources] kindly. (C6)}}
\end{quote}

%Updates on new Roblox features or bugs shared \rv{in the community} also helped \rv{participants} stay informed and build more efficient projects. 
\rv{Participants like C9 mentioned how community} feedback channels \rv{further} motivated participants to continue their effort \rv{to overcome challenges. Consistent with Roque et al. arguing that collaborative learning happens in online communities~\cite{roque2016children}, we find that developer communities fostered learning as a community in practice. Further, learning for participants was} a fun and self-motivated activity \rv{supported by their participation in developer communities.}


\subsubsection{\textbf{Developing social skills by building networks and relationships.}}
\rv{Developer communities weren't only focused on building games. Often they also dedicated space for building networks and relationships with other members. As the communities fostered interaction between users to discuss and collaborate,} participants could develop soft skills through this online communication. \rv{For many participants, the space felt like an extension of their social lives or even their full social life -- in the case of a participant who was home-schooled -- particularly because Roblox communities were unique.} Unlike other developer communities, Roblox’s collaborative and ``peer-led'' environment, \rv{especially on Discord,} is characterized by members of similar ages and informal interactions. Participants mentioned they could \rv{guess} community members were young by the jokes they shared and \rv{by how} the time people showed up seemed \rv{to be} after school. This sense of camaraderie and shared interests fostered a unique ecosystem where participants felt at ease discussing development, collaborating on projects, and communicating in a more \rv{casual and fun-driven way. C7 described his communities as a space where people like him could breathe:}.
\begin{quote}
    \rv{\textit{I really enjoy that kids my age can talk about \rv{(game)} development stuff. In places like Unity or elsewhere, most of the people are older, busy working, there's not really room for saying ``Hey, I want to make a fun game like this, let's do it together''. The conversations are more serious, about profits rather than fun. The atmosphere is entirely different. (C7)}}
\end{quote}

 %This space made the community atmosphere relaxed and fun for users which contrasted with the more serious and profit-focused conversations found on other platforms like Unity. 
 
 %\textcolor{white}{space}\\

\rv{But working with others online was not always easy.  Self-motivated collaborations with group members sometimes encountered challenges when team members had different goals for development (e.g., for fun, for the money), and as such they required tact, patience, and openness.} Participants learned to navigate interactions with diverse cultures like C9 who mentioned having worked with friends on three continents. This expansion beyond their local network meant they had to adapt to communicating online. \rv{Participants mentioned the communication strategies they have learned through the process of collaboration -- starting conversations kindly, being clear about intentions, and finding common ground. }
\begin{quote}
    \rv{\textit{I now add labels to my 3D models so the scriptors understand what I mean. That really helped us sync (C2).}}
\end{quote}

%This type of environment facilitated self-motivated collaborations, making effective communication essential. 
%For instance, C2 learned to communicate their intentions clearly by adding labels to 3D models to better align with scriptors. 

\rv{As their confidence and self-efficacy grew, }participants \rv{began taking proactive roles within communities. Participants like C12 mentioned they gained self-assurance} in promoting their skills in the community. Some participants \rv{also} mentioned taking leadership roles within projects as they gained more confidence in how to manage talking within the community. Four \rv{participants (C2, C4, C6, C9} took moderation roles in Discord developer communities where they once joined as newcomers. Some participants forged friendships and \rv{developed social capital, improving their reputation within the broader community.} Some participants (C3, C7) who did a few projects together discovered they lived close \rv{to each other} and managed to meet and become offline friends. For instance, participants reported that these casual local meetups provided more personalized interactions, which proved to be important for building long-term relationships. This confidence from community involvement led them to explore new roles, such as creating Discord communities or expanding their activities on platforms like YouTube.\footnote{\url{https://www.youtube.com/watch?v=lnhrFUmFpak}. C6 mentioned creating his YouTube channel to share his mini-games with a broader audience.} Three participants (C1, C4, C17) built their own Discord developer communities to engage with players of their game, collaborate with employees of Roblox, \rv{with C6 seeking to} establish the largest Roblox developer community in Korea. These findings show \rv{ how Wenger's concept of Communities of Practice} succeeded naturally in teen developer communities, where participants gradually took different roles and identities in the community~\cite{li2009evolution}. Most participants mentioned recognizing the importance of community mentoring and were willing to pay back the help they got from the community.

%also improved their promotional skills and gained confidence in organizing and presenting their work to be selected for a hiring post in the community. Through these collaborations, they forged friendships and expanded their social networks.
\begin{quote}
    \rv{\textit{At first, I was just an observer. Now, I'm leading a game I thought of and having a voice in the community. It's amazing to be recognized for my skills and be helpful to others. (C14)}}
\end{quote}

\rv{Within a community built on shared ambitions and playful collaborations, participants could feel heard and, through this connection, experience a sense of belonging.} %Broadly, participants developed effective online communication skills and built meaningful relationships within the community.


\subsubsection{\textbf{Turning Hobbies into Careers.}}
%External
%Starting Point to expand
%Hobby --> to Work, Job 
%As participants became more advanced, they would often find themselves being the replier than the questioner.

\rv{For some participants, Roblox development was a hobby. For others, it became something far bigger. Monetization opportunities, such as commissions from collaborations and revenue from creations, served as a strong motivator to continue development for those who consider Roblox development beyond a hobby.} 13 participants pursued commissions from community hiring posts, creating a vibrant marketplace for part-time work. As such, C13 sought in-community jobs for income, \rv{which was} particularly beneficial in countries like Malaysia with favorable exchange rates for dollars earned, while C17 earned money from selling his UGCs (avatar items), keeping him motivated. Community events with prize rewards further enhanced these financial incentives. \rv{Four participants had made revenue from their developed games, and two participants had worked together on the same very successful game, which had temporarily been among the top 10 most played games on Roblox. While one teen developer made a game from his own idea, another participant (C4) mentioned he was invited to the team, and suddenly the game ``blew up''. This kind of success of turning developers' games into huge money was often a common story that participants were aware of and strove for themselves, working in a form of aspirational labor as described above.} While "fun" was the main motivation for creating Roblox games, monetization provided an \rv{overall} substantial additional boost for \rv{nearly all participants.}

\begin{quote}
    \rv{\textit{I get paid a lot by Roblox simply because people like to play my team's game.  After members saw my work, I was invited to join the team community. And when you get paid, that kind of just makes you more intrigued and keeps me developing. Also, I learned economics that way. (C3)}}
\end{quote}



%Monetization in Roblox significantly motivated participants by offering financial incentives. Four participants earned revenue from collaborative projects and engaged in revenue-sharing discussions on Discord. 

%\textcolor{white}{space}\\

\rv{As participants grew more skilled}, creating different projects (see Fig. ~\ref{fig:creation}), recognition within their communities followed. \rv{Some participants, such as C16, were recognized as official Roblox ambassadors, and such participants had chances to go to local Roblox conferences, meet senior developers, and develop strong social capital. C16 mentioned having TikTok and YouTube followers help gain him recognition, though he thought his game development skills still need improvement. From these experiences, C16 dreamed bigger: going to the world Roblox conference and pitching his game.} \rv{Skill enhancement led four} participants \rv{(C1, C2, C6, C10))} to secure freelance work \rv{under a formal contract with a small third-party company} and to secure longer-term employment in a few cases\rv{(C13))}. Despite lower initial offers given due to \rv{their} age, their proven skills \rv{eventually} led to better opportunities and, for some, serious consideration of a career in Roblox development such as a game developer or 3d modeling designer. Some saw their hobby as a potential career path and pursued related opportunities in \rv{the form of} a full-time job or related majors in school. \rv{For example,} C5 planned to go to a game-development-oriented high school. \rv{Similarly, C7 mentioned her career growth:}
 
 %such as C14, who became a mentor and attracted attention from Roblox staff on the DevForum. Others took on ambassador roles, promoting the community online and at events like the annual Roblox developer conference. 
 
\begin{quote}
    \rv{\textit{Actually, at the company I’m with now, I’m making my own game. I joined this company because they made me a good offer in DM, saying I could continue my side project and still receive a regular salary. The only thing is, I have to give a portion of the earnings to them. (C7)}}
\end{quote}

\rv{While some participants didn't view Roblox development as a long-term path—three mentioned shifting focus toward school or other pursuits—15 participants valued the foundational skills they acquired, such as understanding 'if' statements and loops, and felt confident applying these abilities in other fields. Contrary to prior research suggesting that Roblox creates a lock-in ecology for developers~\cite{kou2024ecology}, seven participants had positive experiences, discovering a passion for problem-solving or design, and expressed a desire to continue developing their skills beyond Roblox. This includes learning advanced modeling software like Blender and exploring other programming languages such as JavaScript (C15).}


%, such as creating Discord communities or expanding on platforms like YouTube\footnote{\url{https://www.youtube.com/watch?v=lnhrFUmFpak}}. Three participants built their own Discord developer communities to engage with players, collaborate with employees, or establish the largest Roblox developer community in Korea. 

\begin{quote}
    \rv{\textit{I'm pretty confident with Blender\footnote{3D modeling software not made by Roblox but often used among Roblox developers for more complicated modeling} now. I've noticed people using UnReal in Unity games at my current company. I'll probably try it or 3DMax (C2).}}
\end{quote}
    %\textit{Making games and the Roblox community was helpful; basics like if statements and loops can transfer to other programming languages (C17).}\\
    %\textit{I talked about which high school I should go to with a senior friend, and he recommended some game development-oriented schools which I agreed (C5).}\\
    %\textit{I discovered I like problem-solving. I would like to also learn other languages like JavaScript next (C15).}\\

\subsection{Challenges from participating developer communities and \rv{associated} coping strategies (RQ3)}

In this section, we report the \rv{various} challenges participants \rv{experienced in developer communities related to community accessibility, sustainability, and inter-user challenges between members.} 

\subsubsection{\textbf{\rv{Struggling to find the right community}}}
\rv{Although} online communities offered significant benefits for growth, \rv{many participants faced difficulties discovering and joining the right ones.} \rv{Discord, widely used for its flexibility in creating community channels and having diverse modes of communication, was the preferred platform for most developer communities. However, its invitation-only model often created barriers for newcomers. For instance, beginners like C8 were unaware of these communities until being introduced to them during our interview, revealing the challenges participants face in accessing some online spaces.} According to moderator participants, \rv{restricted access was necessary to ensure safety,} protecting \rv{community members} from \rv{potentially-problematic users}. 
%Some beginners like C8 were unaware of these communities and expressed interest only after learning about them in interviews, highlighting the challenge of balancing recruitment with safety, especially in smaller, heavily moderated groups. 
%But many were hard to \rv{find and join with accessible only through invitation policies. 

%Discord-based communities were popular among participants,  find due to invitation-only policies. 

\rv{Once participants joined a community, integration into the community posed another challenge.} While this process was vital for their continued involvement in the community~\cite{freeman2019exploring}, navigating Roblox-specific terminology and community dynamics \rv{felt} overwhelming,  
\rv{especially for young participants joining their first online communities}. \rv{The diversity of communities meant that experiences varied widely, with some being welcoming and others less so. C16, for example, described feeling isolated when his questions were ignored or met with unhelpful judgmental responses without explanation, which left him discouraged and questioning his place in the community. }
\begin{quote}
    \rv{
    \textit{I reached out to one of my collaborators once. They solved it [my problem] but didn't explain it to me. Instead, they said my code was really messy which it isn't (C12).}}
\end{quote}

\rv{To get used to the community, participants mentioned various strategies such as trying to spend more time, observing others, making friends through collaboration projects, or playing games together with community members.} Advanced teen developers faced challenges in finding the right \rv{communities that matched their skills and offered meaningful} growth and networking opportunities. \rv{Sometimes, the communities they belonged to felt too small for them or did not have enough resources for them to grow. Language barriers also created obstacles. While} many \rv{prominent} developer communities were English-speaking, non-native speakers \rv{from Korea, Japan, and Spain, often felt more comfortable in communities that used their native language.} However, \rv{such} communities \rv{were harder to find. \rvtwo{For instance, when C4 wanted to access more advanced developer resources, he found no materials available in Korean and had to resort to translating English resources.}}

%\begin{quote}
    %\rv{\textit{To be honest, there aren't a lot of sources in Korean. Roblox Korea started recently, which is better, but still, the advanced resource information is hard to find [in Korean]. So, I use translators. (C4)}}
%\end{quote}
%\begin{quote}
    %\rv{\textit{I think Japanese [Roblox] developers could have better opportunities if they knew English (C16).}}
%\end{quote}
\begin{quote}
    \rv{\textit{I have worked with a lot of different people - American, UK, Brazil, and other EUs. Personally, I like Spanish people the best as a Spanish creator. But, from what I know, there aren't Spanish Roblox Creator communities out there. (C14).}}
\end{quote}

\subsubsection{\textbf{\rv{Struggling to Balance Commitments.}}} \rv{Balancing Roblox development} with academic commitments \rv{emerged as a major challenge for teen developers}. Participants who \rv{had} used the community for a long time, such as C14, \rv{reported} that many peers became inactive as they \rv{advanced in school, disrupting} community dynamics and \rv{slowing} down the progress of collaborative projects. \rv{This was especially challenging when a knowledgeable mentor had to leave the community. C15, who enjoyed helping out and answering questions in his communities, felt it becoming harder and harder as he was doing it voluntarily and faced more and more serious school works as an 11th-grade student, but he was concerned about the impact his absence might have.}
%\textcolor{white}{space}\\

\begin{quote}
    \rv{
    \textit{Well, there are three large times when people disappear -- starting middle school, high school, and college admission. Some do come back, but others don't, especially if they decide to focus on studying. (C7)}
    }
\end{quote}

Participants who continued in Roblox development often \rv{aimed to pursue it as their career, sometimes at the expense of their education.} For instance, one participant (C13) \rv{became so} overwhelmed and very sleep-deprived \rv{with balancing her Roblox projects and school work that she took a semester off from school to concentrate on her Roblox job} \rv{creating new 3D} maps and UGCs. Others viewed Roblox as a hobby, balancing it with their education, like C18, who chose to \rv{leave} Roblox \rv{entirely} to focus on his college plans.

\rv{Social} perception of Roblox as ``childish'' also \rv{further complicated the sustainability of the communities.} As teenagers, participants were \rv{conscious} of how \rv{their involvement was perceived negatively by} parents, teachers, and peers outside of Roblox, which sometimes undermined their motivation. Participants like C7 mentioned facing opposition from his parents, school friends, and teachers because Roblox \rv{development was perceived as \textit{immature}} to pursue as a high school student.

%\textit{There was actually a lot of opposition from my parents. There was a strong perception that this (Roblox) was for kids, and none of the students at my school were doing it. My friends around school or the teachers didn't think positively about it either. I had to prove myself. (C7)}\\

Many participants, aware of these perceptions, initially concealed their \rv{activities} in Roblox development. For example, C13 described hiding her activities until she achieved significant milestones, \rv{which felt important enough to gain validation from external people.} 

%others viewed them, the negative image sometimes undermined their motivation.
%impacted how participants were viewed by outsiders, including parents and teachers. 

%\textcolor{white}{space}\\

\begin{quote}
    \rv{
    \textit{Oh wow. It felt like living a double life - a secret small hobby ... until I reached a certain level of quality in creating what I imagined. Now, my parents show my work to relatives. They usually say it’s fascinating or impressive. (C13)}
    }
\end{quote}

\rv{Some participants eventually won their parents' support by demonstrating the financial potential of their development. Monetization often emerged as a validation--proof that their work mattered. Like prior work on TikTok teen creators~\cite{bulley2024dual}, making money convinced their parents to believe in their activities. As in the case of C6, winning a Roblox competition or earning game income helped skeptics see Roblox development as a legitimate career path.} This monetization, however, brought its own set of challenges, which will be detailed in more depth in the Discussion~\ref{discussion:monetization}. 

\begin{quote}
    \rv{
    \textit{At first, my parents were against it, but I did it anyway. They said like, `` you're spending too much time on something unproductive.'' But once I won some Roblox challenges prize and received strong revenue, it helped them see Roblox's potential for a serious career. (C6)}
    }
\end{quote}

%defining moment for participants thinking of a serious career in game development. 

%A few participants mentioned successfully gaining parental support for Roblox development by proving their revenue and creations.  
%\textcolor{white}{space}\\


%    \textit{There was actually a lot of opposition from my parents. There was a strong perception that this (Roblox) was for kids, and none of the students at my school were doing it. My friends around school or the teachers didn't think positively about it either. I had to prove myself. (C7)}\\

\rv{In rare cases, participants} attended schools that \rv{supported game development, integrating} Roblox development into their education. These students benefited from taking courses on programming or 3D design, \rv{which they applied} to their creations. \rvtwo{Some schools actively promoted Roblox development, with C16 describing how it was encouraged by his science teacher and even featured in school-wide events organized by the student council.} Participants like C4 had the resources to consult with \rv{school} teachers for advice, \rvtwo{including valuable input from specialized faculty such as architecture teachers who provided guidance on creating immersive spaces. This created} a synergy between their schoolwork and development efforts. \rvtwo{For instance, C7 chose Roblox as a career path after exploring other platforms in school. Beyond formal education, some participants found support through private instruction or family connections - C12's parents invested in private Roblox coding lessons, while C17 received guidance from family members who worked as programmers.}

%This synergy helped C7 to choose to focus on Roblox as a career path. One participant (C12) had supportive parents who paid for private Roblox coding lessons, and C17 had programmer family members who helped him out occasionally. 

%\rv{For C16, Roblox development was encouraged in his programming class, so he had offline friends to talk with about Roblox development. }
%Among those who gained parental support, some 
\begin{quote}
    \rv{\textit{I go to a school that specializes in game development. They mainly teach tools like Unity or Unreal Engine, and we release game on Steam or Google Play. Roblox, however, felt much friendlier and better for monetization. So, I looked into it more seriously. (C7)}}
\end{quote}
%\begin{quote}
    %\rv{\textit{I sometimes talk to my homeroom teacher about my (Roblox) games. Since he's an architecture teacher, his advice on creating immersive space really helps. (C4)}}
%\end{quote}
%\begin{quote}
    %\rv{\textit{Roblox was a big thing in my school. We had a creators' challenge arranged by the student council, and my science teacher encouraged us. (C16)}}
%\end{quote}

\rv{Despite these positive examples with offline resources, participants were discouraged by negative perceptions saying that there were not many resources available outside online developer communities for a longer term career, which led} many participants \rv{away from} game development, impacting the sustainability of their communities as members came and left. 

\rv{In response to the challenges of community engagement,} many beginner developers turned to AI tools, mostly ChatGPT, for \rv{support}. \rv{Participants} found AI to be convenient \rv{and effective for basic coding tasks, idea generation, and design feedback without feeling like they were burdening community members. However, most }agreed that AI alone could not \rv{replace the depth of knowledge and collaboration found in communities. Complex issues still required human insight and collective problem-solving.}
%address complex levels. Therefore, while AI was useful for basic queries, it continued to rely on community engagement for more intricate issues and collaborations.


\subsubsection{\textbf{Dealing with financial scams}}

\rv{Participants claimed} both financial scams and inter-user conflicts as significant challenges within developer communities. \rv{Consistent with findings on harmful game design in previous research}~\cite{kou2023harmful}, financial scams \rv{were a recurring issue. Participants either personally experienced} being scammed \rv{or knew} others \rv{who had fallen victim}. Scams often \rv{included developers} being underpaid or not paid at all, even \rv{for} successful \rv{projects}. In some cases, \rv{participants paid in advance for work that was never} completed. C18 \rv{reflected on how} the frequent collaborations that happen in the communities made reputation and trust crucial \rv{for successful projects. This trust-dependent nature of collaborations often left} new members \rv{vulnerable to exploitation:} 
\begin{quote}
    \rv{
    \textit{You should like be aware of people because sometimes some young developers trust other developers too much and get scammed. (C18)}
    }
\end{quote}

While scams were less common in \rv{developer}-focused communities \rv{than in} Roblox player communities trading UGCs, participants \rv{still viewed} them as prevalent in certain developer communities \rv{among members especially those who want to make profits}. \rv{Moderator participants explained that} scammers often exploited community norms of mutual trust and \rv{conducted fraudulent} activities in private chats, beyond the reach of moderators, and also bypassed bans by creating new accounts. Six participants \rv{explained how} this left the scams to be addressed by the young community members themselves. \rv{Even in cases of blatant plagiarism or impersonation,} responses varied \rv{widely.} For example, one participant (C18) mentioned \rv{that he felt that it was acceptable for others to copy his game}, as \rv{he felt that it would eventually lead to more growth of his game through the popularity of similar games}. Other participants \rv{took creative or legal steps to protect their work.} For example, C5 mentioned trying to resolve this by learning about \rv{intellectual property law and norms}, which was not covered by \rv{her} school education, \rv{and as a result she developed better} watermarking that could not be deleted easily. \rv{Another participant (C9) said he} sabotag\rv{ed} a collaborator by inserting infinite loops into the code to address non-payment. 

\rv{However, dealing with plagiarism and financial disputes often felt overwhelming, especially for younger participants. Most collaborations proceeded in an informal way between community members without any written contracts. Except for the few participants employed by companies and working under a formal, legal contract, the legal aspects were described as a confusing and unfamiliar concept, leaving both the participants and their legal guardian(s) unaware of how to navigate such matters. C2, for example,} described feeling powerless in a case where their development team earned only a fraction of the revenue generated by their game:
\begin{quote}
    \rv{
    \textit{Our developer team received about 2,000 Robux a month\footnote{Approximately US \$25 at the time of writing.}—when we know the game is making 200,000 to 500,000 Robux monthly. We didn't report it because it was hard to find the legal documents. Also, within Roblox, there is an implicit atmosphere of just trusting each other. The private matters usually stay between the individuals and not the whole community. So, as I developed a lot of games, I experienced not being paid, or someone using hacks on my game, copying it, and putting it in their own game. (C2)}
    }
\end{quote}
\rv{Such experiences led many participants to be more cautious about collaborations, while some became demotivated toward creating games together. Despite the community’s collaborative spirit, the lack of formal protections left many young developers vulnerable.}

%Financial conflicts resulting what effectively amounted to game or UGC plagiarism or impersonation further complicated these situations. As UGCs' identifying watermarks could easily be deleted, tracking the IP of creations was a significant challenge. 
%Also, as collaboration typically meant sharing code, one participant 
%Participants with experience moderating communities mentioned the challenge of scammers who often bypassed bans by creating new accounts. They reported that these scams often took place in private chats that moderators cannot monitor. Because of community norms of mutual trust, six participants found it hard to know what to do when they got scammed; addressing conflicts was usually left up to the young community members themselves. Responses to their games copied varied. For example, one participant mentioned it is acceptable as it will promote their game also with more user engagement in similar or same games. Some participants mentioned trying to resolve this by learning about IP which was not covered by school education or sabotaging a collaborator by inserting infinite loops into the code to address non-payment. Another participant mentioned making their avatars harder to replicate to deter impersonation of him. Hearing or experiencing such incidents led some participants to be more cautious about engaging in collaborations and others to be demotivated toward creating entirely. 
%\textcolor{white}{space}\\ Hearing or experiencing such incidents led some participants to be more cautious about engaging in collaborations and others to be demotivated toward creating entirely. 


\subsubsection{\rv{Dealing with Inappropriate Users.}}
\rv{Although Roblox creator communities were widely perceived by participants as safer and more respectful than Roblox player communities, instances of inappropriate behavior were not entirely absent.} Trolling, disputes over critical feedback, and unauthorized self-promotion \rv{emerged as recurring challenges that participants consistently confronted. More alarming incidents included egregious violations such as banned users disseminating explicit AI-generated content. Female participants mentioned seeing uncomfortable chats that made them choose not to use their voice-chats and to use masculine-looking avatars to conceal their gender. In line with prior work \cite{kou2024ecology} some participants experienced situations where other developers promoted games embedded with radical nationalist ideologies. While they were banned from their community, the problematic users made new accounts again and continued to create these games.}
%In the Roblox creator communities, inappropriate behavior occasionally surfaced. 

%Participants felt that these environments were generally considered cleaner than Roblox player communities, as members' focus on collaboration and reputation fostered a more respectful atmosphere, but some exceptions persisted. More common instances of inappropriate behavior included trolling, disputes over negative feedback, minor rule evasion, and unauthorized self-promotion, but participants also reported more serious safety concerns, such as a case where a banned user shared explicit AI-generated links and another case where a female creator felt uncomfortable with sexual comments made by male creators. Participants also mentioned rare cases of community members promoting games with historically problematic ideologies on radical nationalism.


\rv{The likelihood of encountering such incidents often depended more on how a community was managed than its size. Participants claimed to prefer larger developer servers, as they had dedicated moderation teams and clear rules, typically providing safer spaces. In contrast, smaller, close-knit communities, built on presumed mutual understanding and personal connections, sometimes inadvertently created environments where problematic behaviors were systematically overlooked. This inconsistency shows how effective moderation practices played a more decisive role in shaping the community atmosphere than member count. This led to C10, a community manager,} having an invitation-only approach to maintain a respectful environment. 

\begin{quote}
    \rv{
    \textit{I manage communities where all the members are verified [with a size of 300 users]. In our server, we try to maintain a clean environment with good developers. There’s a lot of constructive conversation between the admins and the users. But sometimes people from the shady side also join, so we have to deal with them occasionally. Because of these issues, we don't promote our community. (C10)}
    }
\end{quote}

\rv{The occasional presence of inappropriate users underscored the ongoing challenge of cultivating trust and professionalism in Roblox developer communities. These incidents reminded the participants of the delicate balance required to sustain a positive and inclusive environment for collaboration.}


