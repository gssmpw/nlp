\section{Discussion}
%1. Polishing, shortening
%2. Motivating - weak version vs persuasive --> We need to develop better for young people
%Sign posting throughout the paper. We need this information. Reminding in small pieces. 


\rv{Our findings have} identified \rv{that participation in Roblox developer communities fostered individual growth among teen developers, particularly in building technical skills and social communication skills. These benefits align with prior work on indie game developers, where collaborative environments encourage democratic leadership and shared ownership of projects~\cite{10.1145/3375184}.} However, these social and \rv{technical} benefits \rv{coexist with} challenges \rv{such as financial scams and interpersonal conflicts. This duality emphasizes the need to take a more comprehensive view:} what makes a positive, safe online \rv{community} for teen \rv{game developers}? \rv{In this section, we address this question by highlighting key components of growth-fostering online communities, complexities introduced by monetization, and implications for fostering safe environments for teen developers.}

\subsection{Key components for fostering growth in online developer communities}
%\begin{tcolorbox}[colback=gray!10, colframe=gray!50, width=\textwidth, sharp corners=south, boxrule=0.5mm, title=]
%\textbf{What are the key features in providing safe teenager-friendly online space that fosters growth?}
%\end{tcolorbox}

In this section, we identify three \rv{critical} factors driving many of the benefits \rv{participants reported from} online \rv{Roblox} developer communities\rv{: play as a core motivation, opportunities for meaningful communication, and access to growth-enabling resources. Some of these} factors were strongly aligned with previous work but manifested in new forms in the in-the-wild, play-driven case of Roblox developer communities. 
\subsubsection{{\textbf{Self-driven play as a Core motivation.}}}
Teen developers were self-motivated by \rv{the intrinsic enjoyment of creating and experimenting within the Roblox ecosystem, reporting fun as their primary factor for why they develop games. This demonstrates how playfulness remains a cornerstone of sustained community engagement} and growth in teen-created and moderated spaces. \rv{Similar findings} in developer communities like Scratch \rv{show} technical development \rv{gained through collaborative learning in communities} ~\cite{brennan2021kids, 10.1145/3491102.3502124, roque2016children} or adult-focused platforms like StackOverflow~\cite{10.1145/2531602.2531659, 8417152}. \rv{However, unlike Scratch, where play occurs within structured learning environments, Roblox supports a hybrid model where teens independently create and moderate their spaces. }This dynamic fosters a unique sense of autonomy and ownership, consistent with theories of participatory culture~\cite{jenkins2009confronting, jenkins2015participatory}.

\rv{Freeman highlights the critical role of small teams and "democratic" participation, wherein collaboration fosters shared decision-making and individualized contributions~\cite{freeman2019exploring}. Similarly, teen-led Roblox communities function as small, dynamic ecosystems where developers work collectively to design and iterate on games. Despite being self-organized, these communities mirror the socio-technological challenges faced by indie developers, especially in identifying the right collaborators with shared goals. For teens, these challenges manifest in managing their team formation and navigating technical systems with limited professional guidance.}
%fostered by peer learning and technical development in teen-created and moderated spaces, even without strong adult supervision. These benefits align with education-based developer communities ~\cite{brennan2021kids, 10.1145/3491102.3502124, roque2016children} or adult-focused platforms like StackOverflow~\cite{10.1145/2531602.2531659, 8417152}. %Our findings reveal that the benefits of participatory culture~\cite{jenkins2015participatory} of teen autonomy thrived even outside formal educational settings. 
\rv{Building on Freeman's insights, we argue that the playful roots of the participatory culture within Roblox developer communities are crucial for encouraging long-term participation and growth. Even as} challenges \rv{arise, such as encountering scams or interpersonal conflicts, the ability to experiment freely within} the collaborative and friendly atmosphere of Roblox developer communities enables resilience and creative exploration. 
%participants are drawn to the collaborative and friendly atmosphere of Roblox developer communities, which encourages creativity and trying out for fun. Building on previous research, we argue that maintaining a playful environment is essential for sustaining long-term engagement among teenage developers.

%\rv{Reviewer comment on including Guo's indie developer challenges, AI, personal growth with team. }
%\rv{Appropriate social, cultural, financial capital}
%\rv{Incentives play \& money }

\subsubsection{{\textbf{\rv{Access to rich technical and social resources facilitates learning.}}}}
\rv{Our findings underscore that access to both technical and social resources plays a crucial role in fostering growth within developer communities. For newcomers and even more experienced participants, the ability to connect with others is vital for learning, collaboration, and expanding their professional network. However, communication barriers often hinder this process. Without an environment conducive to meaningful interaction, some participants, especially beginners, turned to AI tools like GPT for technical support, particularly for coding tasks and idea generation. The increasing support-seeking from AI mirrors how indie developers use AI as their programming partner in idea generation~\cite{panchanadikar2024m}. \rv{However, while these} AI tools provided immediate assistance, they did not offer the depth of collaboration and mentorship that could be found in community-based interactions.}%However, unlike in Panchanadikar et al.'s work, our participants were not scared of career growth risks and were willing to use AI for their communities like organizing events.} %Opportunities for communication also emerged as an important factor. Newcomers and experienced participants all needed social connections within the community to learn, collaborate, and network. However, without a sufficient communication environment, some newcomers turned to AI tools like GPT for support and some more experienced participants avoided exploring new communities. Events, challenges, and local meetups played an important role in fostering interaction and overcoming barriers like language. Our findings suggest that beyond online communication, integrating offline, local-based gatherings could strengthen community bonds and provide a more personal connection for members.

\rv{Furthermore, we noticed resource differences beyond communities. Some participants felt that their communities had abundant resources to learn and had members who were active in answering questions, while other participants wanted to search for more advanced communities. This discrepancy suggests the importance of ensuring that all members, particularly beginners, have equitable access to the necessary tools, guidance, and networking opportunities that foster their personal and professional growth within the community.}

%\subsubsection{\textbf{Having the right resources in the community for growth.}}
%\rv{Our results also suggest that events, challenges, and local meetups play an essential role in bridging communication gaps and fostering collaboration. These in-person or hybrid gatherings provided spaces where participants could meet others, share knowledge, and form connections that went beyond what is possible in purely online settings. This highlights a key point: while online communication forms the backbone of many communities, integrating offline, localized gatherings can strengthen ties between members and facilitate more meaningful connections, contributing to a stronger, more supportive community environment.}

%Moreover, our findings reveal that \textbf{mentorship} and \textbf{learning opportunities} are pivotal in supporting community members' growth. Access to experienced mentors, whether through direct collaboration, community-driven events, or partnerships with companies, proved to be a significant driver of participants' technical and social development. For example, some participants benefited from attending conferences or participating in company-sponsored initiatives, which helped them advance both their technical skills and industry knowledge. However, these opportunities were not universally available, and newcomers or less experienced participants were often left without access to such growth-enabling resources. This discrepancy suggests the importance of ensuring that all members, particularly beginners, have equitable access to the necessary tools, guidance, and networking opportunities that foster their personal and professional growth within the community. Access to the right learning and social resources was another critical factor for growth. Mentorship and learning opportunities from the community, like conferences or collaborations with companies, helped certain participants advance their technical and social skills. However, not all participants had access to these resources, particularly newcomers. This highlights the importance of ensuring that all members, especially beginners, have access to the learning tools and support for growing within the community.
%\rv{As the case of indie developers with less resources, some developers turned for AI support. }

\subsubsection{{\textbf{Incentives can drive community growth.}}}

Lastly, providing incentives like monetization proved effective in fostering growth within communities, as seen when participants joined \rv{in response to} commission posts and developed their skills accordingly. \rv{Community-wide competitions with Robux prizes were also incentives for participation commonly enjoyed by participants.} However, \rv{these forms of} engagement \rv{were} voluntary, relying on the social value of the community to motivate participants like in other developer communities~\cite{lu2021motivation}. This made it challenging for advanced developers \rv{to participate, as they often struggled to balance community involvement with schoolwork and development loads.}
% struggled to balance community involvement with schoolwork and development loads. 
Thus, implementing \rv{formal, financial incentives and/or less formal, social incentives each can enhance growth in different ways for different populations.} \\

\subsection{The complexities of a monetized hobby ecosystem}
\label{discussion:monetization}
%\rv{To ADD monetization in other teen creator communities - TikTok, YouTube~\cite{lombana2020youth, bulley2024dual, golmgrein2023comprehensive}. The promise of potential future success motivates children to engage in unpaid creative work, a dynamic that can perpetuate systemic inequalities. Successful creators typically emerge from backgrounds with greater resources, access to technology, and supportive networks—creating a cycle that marginalizes creators from less privileged backgrounds.}\\

%Monetization isn't new - aspirational labor. But roblox differs in Xx.
\rv{Monetization motivating teen Roblox developers is not a new concept; it fits within the broader framework of \textbf{aspirational labor}, where teen content creators are motivated by the hope of future success~\cite{duffy2017not}. This phenomenon is well-established in scholarly literature, which discusses the developmental role of unpaid creative work and the intersection of passion, skill-building, and economic opportunity in youth participation built upon venture labor and hope labor models~\cite{neff2012venture, kuehn2013hope}. Roblox stands out from other content creation platforms, such as YouTube, blogging, and game streaming, because it combines game development with monetization—offering teens the opportunity to engage in both creativity and entrepreneurship; Unlike YouTube influencers or TikTok creators, whose success is often based on personal branding, game developers must create engaging, playable content, requiring specific technical and design skills~\cite{lombana2020youth}. Compared to many traditional game modding processes, Roblox Studio significantly lowers technical barriers, making it more accessible and empowering to teen developers. }%Monetization on platforms like Roblox represents a modern extension of aspirational labor, where young creators blend play with work, motivated by the hope of future success. This phenomenon aligns closely with scholarly discussions on the developmental role of unpaid creative labor and the interplay between passion, skill-building, and economic opportunity in youth participation. \textbf{Unlike teen content creators on platforms like TikTok or YouTube, Roblox developers operate within a structured creative ecosystem where technical skill and gameplay innovation take precedence over personal branding.} While YouTube or TikTok creators often monetize through advertising or sponsorships tied to their personalities, Roblox creators rely on crafting engaging, user-generated content (UGC), requiring technical expertise in game development. This distinction introduces unique challenges, as these teens must navigate complex systems of economic engagement and virtual economies with minimal oversight.}

%Like literature monetization clearly introduced more empowerment. With more easier barriers.
\rv{The appeal of Roblox’s monetization system is clear, as it enables teenage developers to make money from their passion. Many participants in our study reported that the platform offers them increased autonomy compared to other teen job options, such as part-time work, as it allows them to earn money independently with what is at its core a form of play, without direct parental oversight. Teen developers value the creative freedom Roblox provides, viewing it as a space to experiment, hone their skills, and build a reputation with the potential for success. Drawing on previous work by Postigo~\cite{postigo2010modding} and Taylor~\cite{taylor2009assemblage}, this mirrors the “modding” culture in the video game industry, but with more accessibility, diversity in making profits from game, game items, game assets, and technical skillsets. Some participants strategically leveraged their games to build social capital and professional networks, demonstrating the potential for both skill development and future monetary success.}

%However, it also introduced safety issues. As it blurs work and play ~~
\rv{However, the formal blurring of play and work through Roblox’s monetization system creates new challenges. While the prospect of financial independence is a strong motivator, the absence of formalized structures or clear guidelines on what constitutes “labor” leaves teen developers vulnerable. Participants described incidents of scams and exploitation within peer communities, highlighting their lack of protection and limited knowledge about navigating such risks. For teens in countries requiring parental consent for legal work, the situation was further complicated. Some participants who joined small companies faced significant hurdles due to their families’ unfamiliarity with the complexities of game development agreements. Many parents, unfamiliar with such arrangements—especially in the niche context of Roblox—struggled to judge the fairness or legality of contracts. This inexperience often led participants to accept vague terms, resulting in unclear expectations and under-compensated work. Two participants shared instances where what began as creative exploration turned into unpaid or underpaid obligations, consuming more time and energy than they had anticipated.} 

\rv{The monetization ecosystem itself can exacerbate these challenges, sometimes driving teens to prioritize profit over ethics. This pressure can lead to decisions that compromise the well-being of younger players, such as designing games with harmful elements to maximize profitability.} Furthermore, the financial burden associated with advertising and publishing UGC—such as the cost of publishing limited UGCs, which can reach 20,000 Robux (approximately 15 USD) per item—adds another layer of strain, as previously noted by Kou~\cite{kou2023harmful}.

\rv{Living in a digital landscape driven by entrepreneurial ideals, it is natural for teens to be drawn to the prospect of creating their own games and earning financial rewards. However, we noticed that among the participants, there was an almost unwavering belief that hard work could secure success—a sentiment that reflects meritocratic ideals. This firm belief can quietly erode well-being when the realities of inequality go unacknowledged~\cite{watkins2018digital}. Participants like C13 described his failure in market success solely primarily as personal shortcomings, pledging to work harder to overcome them. This type of belief system can turn a creative playground into a space of self-doubt and relentless striving~\cite{ross2012search}.}

\rv{Our findings also highlighted the disparities in access to resources that shape these success narratives. For example, one participant, C12, thrived with the support of parents who invested in his coding education and helped him navigate complex programming concepts, while others struggled due to a lack of support from their communities.  While our interviews cannot definitively determine how much these external resources contributed to success in game development, they underscore the importance of recognizing the roles that factors like mentorship, financial backing, and even sheer luck play in shaping outcomes.}

\rv{Therefore, to foster a healthier perspective on monetization and success, it is critical to help teens see the broader context of their aspirations. Sharing details of success narratives including finding the right team, the right endeavor, and the right collaborative support will lead teen developers to focus more broadly than on just profits. In this way, a more balanced approach to ambition can be cultivated without compromising the well-being of developers--one that empowers teens to pursue their goals with resilience while acknowledging the complexities of the digital marketplace.}


\subsection{Takeaways for fostering safe and positive teen-friendly spaces}
%NOTE: Need a better structure overall hmmm 
%\begin{tcolorbox}[colback=gray!10, colframe=gray!50, width=\textwidth, sharp corners=south, boxrule=0.5mm, title=]
%\textbf{What are design implications for fostering safe and positive online environments?}
%\end{tcolorbox}
 In order to achieve the growth described above, communities must \rv{establish robust frameworks that prioritize }stability and safety. In this section, we identify two key takeaways for fostering such an environment: continuous guidance, empowering teen developers to handle challenges, and collaborative stakeholder involvement. 

\subsubsection{\textbf{Empowering teen developers to handle challenges.}}
\rv{The decentralized nature of developer communities on Roblox creates both opportunities and challenges for teen developers. While these communities foster creativity and peer-to-peer learning, they also leave teens vulnerable to risks such as financial conflicts, copyright issues, and scams, which can be difficult to navigate without guidance. Instead of adopting a top-down governance model that could undermine teen agency and autonomy in managing their spaces, Roblox should focus on empowering teen developers to handle these challenges effectively.}
\rv{Roblox can play a critical role by educating developers about community norms and setting examples of good community policies—not just showcasing successful games but also showcasing communities with successful governance.} Clear rules addressing financial disputes and intellectual property concerns should be established, alongside robust reporting and peer-review mechanisms for identifying and resolving suspicious activities. 

Teen developers should also be encouraged to collaborate in creating solutions \rv{as a primary stakeholder}. For example, \rv{developer communities use }peer review \rv{mechanisms} to identify scams, share knowledge about recognizing common \rv{scam patterns}, and clarify realistic commission expectations to prevent exploitation. Roblox could further support community leaders by providing educational resources on effective moderation strategies and creating incentives for experienced members to mentor newcomers. AI tools could assist by explaining ongoing conversations to newcomers, addressing accessibility challenges, and easing their integration into the community. By focusing on \textit{empowerment} \rv{of teen developers} rather than on control, Roblox can foster resilience-based learning~\cite{park2023towards}, \rv{and community sustainability} as members come in and out. Though it is impossible to eliminate all risks, equipping teen developers with the tools, knowledge, and community support to navigate the challenges of the community can help them grow both creatively and responsibly.


\subsubsection{\textbf{Continuous stakeholder support both offline and online.}}
\rv{Teen developers demonstrate remarkable agency within their communities, which leads to collaborative learning and self-directed socialization. The unique opportunities offered by platforms like Roblox—ranging from exploring diverse games and monetization opportunities to connecting with peers—serve as a gateway for broader offline engagement. For some participants, online experiences inspired decisions to attend relevant schools or participate in offline meetups, while for some participants, their school friends led them to join online communities in the first place. This depicts the organic interplay between the digital and physical worlds~\cite{taylor2009play}. Additionally}, participants emphasized \rv{how} supportive \rv{online} environments often benefit from robust offline support networks, \rv{including teachers, family, and peers, which play a critical role in fostering their growth and helping them navigate challenges. This seamless transition between online and offline support highlights the need for a cohesive, collaborative approach among stakeholders.}

\rv{To nurture teen developers effectively, all stakeholders—parents, educators, peers, platform providers, and developers themselves—must work together to build an ecosystem that bridges online creativity with offline guidance. } To build a more broadly supportive ecosystem, Roblox must position itself as a serious platform for technical development, earning respect from both users and external stakeholders. By integrating Roblox’s educational value in prioritizing \rv{developer} safety, Roblox can strengthen its supportive role. Likewise, parents and schools should recognize \rv{and acknowledge} the significance of Roblox development when students showcase their achievements. Bridging online creativity with offline guidance through this continuous support network is crucial for fostering a positive environment for teen developers.

Effective management of potential issues on Roblox demands active involvement from all stakeholders: Roblox, host platforms for developer communities, developer community moderators, parents, schools, and teen developers. To create an inclusive environment, it is crucial to implement participatory governance models that ensure every voice, especially those of teen \rv{developers}, is heard in shaping community policies~\cite{wang202312}. This collaborative approach may involve regular feedback sessions, open forums, and advisory panels where all stakeholders can contribute to developing community policies and support mechanisms. By fostering these participatory practices, stakeholders can enhance the \rv{community} experience, address emerging issues, and build a more supportive and engaging environment for teen developers.  Such efforts can also reduce the normalization of harmful game design within the Roblox ecosystem, mitigating one of the platform’s most significant risks to players and fostering a safer, more constructive space for all users. 

\subsection{Future Work} %Shorten
Our study reveals promising growth opportunities for teenagers in game development, socializing, and career building in platforms like Roblox but also uncovers several challenges. To address these, future research should aim to include a more diverse sample or utilize large-scale chat data analysis from developer communities. This would provide a broader perspective on the Roblox creator ecosystem and gain insights into underrepresented developers.
While issues like child safety and grooming were minimally reported in our sample, they may be more significant in a larger study. In addition, expanding to explore other teenage communities—such as social media creators or school-based learning platforms—could reveal different motivations and interaction styles due to their unique contexts. Understanding these differences, including how to better support child creators, remains an important area for further investigation.


