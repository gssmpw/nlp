




\section{Trade-offs Between Action Recommendation and Reward Manipulation Using Lower Bounds}
We can incorporate a lower bound analysis to compare the trade-offs between Action Recommendation (AR) and Reward Manipulation (RM). It highlights the theoretical benefits and limitations of each feedback type.


\textbf{Problem Setup and Notation}

Let:
\( T \): Total number of rounds.
\( K \): Number of actions.
\( \mathcal{A} \): Action space.
\( s_t \): Context observed at round \( t \).
\( r_t(a) \): Reward for action \( a \) at round \( t \).
\( q_t^{AR} \): Probability that the feedback in AR is correct (expert recommendation quality).
\( q_t^{RM} \): Probability that the reward signal is correctly modified in RM (expert reward quality).
\( p_t \): Probability of querying feedback in either AR or RM.

We aim to derive regret lower bounds for both feedback types and analyze their trade-offs.


\subsection{Action Recommendation (AR)}

In the AR setting: The agent queries the oracle to receive the recommended action \( a_t^{AR} \), which is assumed to be correct with probability \( q_t^{AR} \).

\textbf{Regret Lower Bound for AR}

In rounds where feedback is not queried (\( 1 - p_t \)), the regret follows standard contextual bandit bounds:
\[
\mathbb{E}[\text{Regret}_{\text{no-feedback}}(T)] \geq O((1 - p_t) \sqrt{T K}).
\]

In rounds where AR feedback is queried (\( p_t \)), regret depends on the quality of the recommended action:
\[
\mathbb{E}[\text{Regret}_{\text{feedback}}^{AR}(T)] \geq O\left(\frac{p_t T}{q_t^{AR}}\right).
\]

Thus, the total regret for AR is bounded by:
\[
\mathbb{E}[\text{Regret}^{AR}(T)] \geq O((1 - p_t) \sqrt{T K}) + O\left(\frac{p_t T}{q_t^{AR}}\right).
\]

\subsection{Reward Manipulation (RM)}

In the RM setting: The agent receives a modified reward signal \( \tilde{r}_t(a_t) \), adjusted by the oracle to reflect feedback quality \( q_t^{RM} \).

\textbf{Regret Lower Bound for RM}

Without feedback (\( 1 - p_t \)):
\[
\mathbb{E}[\text{Regret}_{\text{no-feedback}}(T)] \geq O((1 - p_t) \sqrt{T K}).
\]

With RM feedback (\( p_t \)), the manipulated reward provides improved reward estimates, reducing regret:
\[
\mathbb{E}[\text{Regret}_{\text{feedback}}^{RM}(T)] \geq O\left(\frac{p_t T}{q_t^{RM}}\right).
\]

Thus, the total regret for RM is:
\[
\mathbb{E}[\text{Regret}^{RM}(T)] \geq O((1 - p_t) \sqrt{T K}) + O\left(\frac{p_t T}{q_t^{RM}}\right).
\]


\subsection{Trade-off Analysis}

\textbf{1. Feedback Quality \(q_t^{AR}\) vs. \(q_t^{RM}\)}:
   \textbf{AR} directly impacts action selection, which may lead to larger regret reduction if \(q_t^{AR}\) is high.
   \textbf{RM} improves the reward signal, which may be less direct but still effective in guiding future decisions.

\textbf{2. Feedback Frequency \(p_t\)}:
   Both AR and RM benefit from higher feedback frequency \(p_t\). However, querying feedback comes with costs, and the choice depends on the relative quality of feedback \(q_t\).

\textbf{3. Cumulative Regret}:
   If \(q_t^{AR} > q_t^{RM}\), AR is more effective in reducing regret:
     \[
     \mathbb{E}[\text{Regret}^{AR}(T)] < \mathbb{E}[\text{Regret}^{RM}(T)].
     \]
   Conversely, if \(q_t^{RM}\) is higher, RM could achieve lower regret.

\subsection{Practical Implications}
\textbf{When to Use AR}:
 (i) When action recommendations are highly reliable (\(q_t^{AR} \to 1\)).
 (ii) When immediate corrective feedback on actions is critical.

\textbf{When to Use RM}:
 (i) When action recommendations are less reliable, but reward signals can be improved consistently (\(q_t^{RM} > q_t^{AR}\)).
 (ii) When reward shaping can better guide learning in uncertain environments.


This analysis shows that the choice between AR and RM depends on the quality and frequency of feedback. Both methods have distinct strengths, and their trade-offs can be quantified using the derived regret bounds. Future work could further explore hybrid strategies that dynamically balance AR and RM based on real-time feedback quality.
