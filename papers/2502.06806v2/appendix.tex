%%%%%%%%%%%%%%%%%%%%%%%%%%%%%%%%%%%%%%%%%%%%%%%%%%%%%%%%%%%%%%%%%%%%%%%%%%%%%%%
%%%%%%%%%%%%%%%%%%%%%%%%%%%%%%%%%%%%%%%%%%%%%%%%%%%%%%%%%%%%%%%%%%%%%%%%%%%%%%%
% APPENDIX
%%%%%%%%%%%%%%%%%%%%%%%%%%%%%%%%%%%%%%%%%%%%%%%%%%%%%%%%%%%%%%%%%%%%%%%%%%%%%%%
%%%%%%%%%%%%%%%%%%%%%%%%%%%%%%%%%%%%%%%%%%%%%%%%%%%%%%%%%%%%%%%%%%%%%%%%%%%%%%%
\newpage
\appendix
\onecolumn
% \section{You \emph{can} have an appendix here.}

% You can have as much text here as you want. The main body must be at most $8$ pages long.
% For the final version, one more page can be added.
% If you want, you can use an appendix like this one.  

% The $\mathtt{\backslash onecolumn}$ command above can be kept in place if you prefer a one-column appendix, or can be removed if you prefer a two-column appendix.  Apart from this possible change, the style (font size, spacing, margins, page numbering, etc.) should be kept the same as the main body.
%%%%%%%%%%%%%%%%%%%%%%%%%%%%%%%%%%%%%%%%%%%%%%%%%%%%%%%%%%%%%%%%%%%%%%%%%%%%%%%
%%%%%%%%%%%%%%%%%%%%%%%%%%%%%%%%%%%%%%%%%%%%%%%%%%%%%%%%%%%%%%%%%%%%%%%%%%%%%%%

\section{Algorithm Details}
\label{sec:alg_app}
We provide summarized form of the training and inference algorithm for the \textit{Plugin} model below.

\begin{algorithm}[H]
\caption{Training and Inference for the Plugin Model}
\label{alg:plugin_model}
\textbf{Input:} Black-box model $B$, reweighting model $R$, clean training data $\mathcal{D}$, vocabulary $V$ \\
\textbf{Output:} Plugin model predictions $\bm{x}_{1:T}$ for a given sequence
\begin{algorithmic}[1]
\STATE \textbf{Training Phase:}
\FOR{each sequence $s \in \mathcal{D}$}
    \STATE Compute token probabilities $\{\bm{b}_1, \bm{b}_2, \dots, \bm{b}_m\}$ using $B$.
    \STATE Compute token probabilities $\{\bm{r}_1, \bm{r}_2, \dots, \bm{r}_m\}$ using $R$.
    \STATE Combine probabilities: ${\bm{p}}_i = \frac{\bm{b}_i \odot \bm{r}_i}{\|\bm{b}_i \odot \bm{r}_i\|_1}$ for $i \in [m]$.
    \STATE Compute sequence loss $\ell_s = -\frac{1}{m} \sum_{i=1}^{m} \sum_{j=1}^{|V|} \log({\bm{p}}_i) \odot \bm{e}_j$.
    \STATE Update parameters of $R$ using back-propagation. Freeze $B$.
\ENDFOR
\STATE \textbf{Inference Phase:}
\STATE Initialize sequence $\bm{x}_{1:T} = \{\}$.
\FOR{each token position $t = 1$ to $T$}
    \STATE Compute token probabilities $\bm{b}_t$ using $B$.
    \STATE Compute token probabilities $\bm{r}_t$ using $R$.
    \STATE Combine probabilities: ${\bm{p}}_t = \frac{\bm{b}_t \odot \bm{r}_t}{\|\bm{b}_t \odot \bm{r}_t\|_1}$.
    \STATE Predict token: $\bm{x}_t = \argmax_{V} ({\bm{p}}_t)$.
    \STATE Append $\bm{x}_t$ to $\bm{x}_{1:T}$.
\ENDFOR
\STATE \textbf{Return:} $\bm{x}_{1:T}$
\end{algorithmic}
\end{algorithm}

\section{Proof of Main Convergence Theorem}
\label{app:theory}
\section{Experiments}
\label{sec:Experiments} 

We conduct several experiments across different problem settings to assess the efficiency of our proposed method. Detailed descriptions of the experimental settings are provided in \cref{sec:apendix_experiments}.
%We conduct experiments on optimizing PINNs for convection, wave PDEs, and a reaction ODE. 
%These equations have been studied in previous works investigating difficulties in training PINNs; we use the formulations in \citet{krishnapriyan2021characterizing, wang2022when} for our experiments. 
%The coefficient settings we use for these equations are considered challenging in the literature \cite{krishnapriyan2021characterizing, wang2022when}.
%\cref{sec:problem_setup_additional} contains additional details.

%We compare the performance of Adam, \lbfgs{}, and \al{} on training PINNs for all three classes of PDEs. 
%For Adam, we tune the learning rate by a grid search on $\{10^{-5}, 10^{-4}, 10^{-3}, 10^{-2}, 10^{-1}\}$.
%For \lbfgs, we use the default learning rate $1.0$, memory size $100$, and strong Wolfe line search.
%For \al, we tune the learning rate for Adam as before, and also vary the switch from Adam to \lbfgs{} (after 1000, 11000, 31000 iterations).
%These correspond to \al{} (1k), \al{} (11k), and \al{} (31k) in our figures.
%All three methods are run for a total of 41000 iterations.

%We use multilayer perceptrons (MLPs) with tanh activations and three hidden layers. These MLPs have widths 50, 100, 200, or 400.
%We initialize these networks with the Xavier normal initialization \cite{glorot2010understanding} and all biases equal to zero.
%Each combination of PDE, optimizer, and MLP architecture is run with 5 random seeds.

%We use 10000 residual points randomly sampled from a $255 \times 100$ grid on the interior of the problem domain. 
%We use 257 equally spaced points for the initial conditions and 101 equally spaced points for each boundary condition.

%We assess the discrepancy between the PINN solution and the ground truth using $\ell_2$ relative error (L2RE), a standard metric in the PINN literature. Let $y = (y_i)_{i = 1}^n$ be the PINN prediction and $y' = (y'_i)_{i = 1}^n$ the ground truth. Define
%\begin{align*}
%    \mathrm{L2RE} = \sqrt{\frac{\sum_{i = 1}^n (y_i - y'_i)^2}{\sum_{i = 1}^n y'^2_i}} = \sqrt{\frac{\|y - y'\|_2^2}{\|y'\|_2^2}}.
%\end{align*}
%We compute the L2RE using all points in the $255 \times 100$ grid on the interior of the problem domain, along with the 257 and 101 points used for the initial and boundary conditions.

%We develop our experiments in PyTorch 2.0.0 \cite{paszke2019pytorch} with Python 3.10.12.
%Each experiment is run on a single NVIDIA Titan V GPU using CUDA 11.8.
%The code for our experiments is available at \href{https://github.com/pratikrathore8/opt_for_pinns}{https://github.com/pratikrathore8/opt\_for\_pinns}.


\subsection{2D Allen Cahn Equation}
\begin{figure*}[t]
    \centering
    \includegraphics[scale=0.38]{figs/Burgers_operator.pdf}
    \caption{1D Burgers' Equation (Operator Learning): Steady-state solutions for different initializations $u_0$ under varying viscosity $\varepsilon$: (a) $\varepsilon = 0.5$, (b) $\varepsilon = 0.1$, (c) $\varepsilon = 0.05$. The results demonstrate that all final test solutions converge to the correct steady-state solution. (d) Illustration of the evolution of a test initialization $u_0$ following homotopy dynamics. The number of residual points is $\nres = 128$.}
    \label{fig:Burgers_result}
\end{figure*}
First, we consider the following time-dependent problem:
\begin{align}
& u_t = \varepsilon^2 \Delta u - u(u^2 - 1), \quad (x, y) \in [-1, 1] \times [-1, 1] \nonumber \\
& u(x, y, 0) = - \sin(\pi x) \sin(\pi y) \label{eq.hom_2D_AC}\\
& u(-1, y, t) = u(1, y, t) = u(x, -1, t) = u(x, 1, t) = 0. \nonumber
\end{align}
We aim to find the steady-state solution for this equation with $\varepsilon = 0.05$ and define the homotopy as:
\begin{equation}
    H(u, s, \varepsilon) = (1 - s)\left(\varepsilon(s)^2 \Delta u - u(u^2 - 1)\right) + s(u - u_0),\nonumber
\end{equation}
where $s \in [0, 1]$. Specifically, when $s = 1$, the initial condition $u_0$ is automatically satisfied, and when $s = 0$, it recovers the steady-state problem. The function $\varepsilon(s)$ is given by
\begin{equation}
\varepsilon(s) = 
\left\{\begin{array}{l}
s, \quad s \in [0.05, 1], \\
0.05, \quad s \in [0, 0.05].
\end{array}\right.\label{eq:epsilon_t}
\end{equation}

Here, $\varepsilon(s)$ varies with $s$ during the first half of the evolution. Once $\varepsilon(s)$ reaches $0.05$, it remains fixed, and only $s$ continues to evolve toward $0$. As shown in \cref{fig:2D_Allen_Cahn_Equation}, the relative $L_2$ error by homotopy dynamics is $8.78 \times 10^{-3}$, compared with the result obtained by PINN, which has a $L_2$ error of $9.56 \times 10^{-1}$. This clearly demonstrates that the homotopy dynamics-based approach significantly improves accuracy.

\subsection{High Frequency Function Approximation }
We aim to approximate the following function:
$u=    \sin(50\pi x), \quad x \in [0,1].$
The homotopy is defined as $H(u,\varepsilon) = u - \sin(\frac{1}{\varepsilon}\pi x), $
where $\varepsilon \in [\frac{1}{50},\frac{1}{15}]$.

\begin{table}[htbp!]
    \caption{Comparison of the lowest loss achieved by the classical training and homotopy dynamics for different values of $\varepsilon$ in approximating $\sin\left(\frac{1}{\varepsilon} \pi x\right)$
    }
    \vskip 0.15in
    \centering
    \tiny
    \begin{tabular}{|c|c|c|c|c|} 
    \hline 
    $ $ & $\varepsilon = 1/15$ & $\varepsilon = 1/35$ & $\varepsilon = 1/50$ \\ \hline 
    Classical Loss                & 4.91e-6     & 7.21e-2     & 3.29e-1       \\ \hline 
    Homotopy Loss $L_H$                      & 1.73e-6     & 1.91e-6     & \textbf{2.82e-5}       \\ \hline
    \end{tabular}
    % On convection, \al{} provides 14.2$\times$ and 1.97$\times$ improvement over Adam or \lbfgs{} on L2RE. 
    % On reaction, \al{} provides 1.10$\times$ and 1.99$\times$ improvement over Adam or \lbfgs{} on L2RE.
    % On wave, \al{} provides 6.32$\times$ and 6.07$\times$ improvement over Adam or \lbfgs{} on L2RE.}
    \label{tab:loss_approximate}
\end{table}

As shown in \cref{fig:high_frequency_result}, due to the F-principle \cite{xu2024overview}, training is particularly challenging when approximating high-frequency functions like $\sin(50\pi x)$. The loss decreases slowly, resulting in poor approximation performance. However, training based on homotopy dynamics significantly reduces the loss, leading to a better approximation of high-frequency functions. This demonstrates that homotopy dynamics-based training can effectively facilitate convergence when approximating high-frequency data. Additionally, we compare the loss for approximating functions with different frequencies $1/\varepsilon$ using both methods. The results, presented in \cref{tab:loss_approximate}, show that the homotopy dynamics training method consistently performs well for high-frequency functions.





\subsection{Burgers Equation}
In this example, we adopt the operator learning framework to solve for the steady-state solution of the Burgers equation, given by:
\begin{align}
& u_t+\left(\frac{u^2}{2}\right)_x - \varepsilon u_{xx}=\pi \sin (\pi x) \cos (\pi x), \quad x \in[0, 1]\nonumber\\
& u(x, 0)=u_0(x),\label{eq:1D_Burgers} \\
& u(0, t)=u(1, t)=0, \nonumber 
\end{align}
with Dirichlet boundary conditions, where $u_0 \in L_{0}^2((0, 1); \mathbb{R})$ is the initial condition and $\varepsilon \in \mathbb{R}$ is the viscosity coefficient. We aim to learn the operator mapping the initial condition to the steady-state solution, $G^{\dagger}: L_{0}^2((0, 1); \mathbb{R}) \rightarrow H_{0}^r((0, 1); \mathbb{R})$, defined by $u_0 \mapsto u_{\infty}$ for any $r > 0$. As shown in Theorem 2.2 of \cite{KREISS1986161} and Theorems 2.5 and 2.7 of \cite{hao2019convergence}, for any $\varepsilon > 0$, the steady-state solution is independent of the initial condition, with a single shock occurring at $x_s = 0.5$. Here, we use DeepONet~\cite{lu2021deeponet} as the network architecture. 
The homotopy definition, similar to ~\cref{eq.hom_2D_AC}, can be found in \cref{Ap:operator}. The results can be found in \cref{fig:Burgers_result} and \cref{tab:loss_burgers}. Experimental results show that the homotopy dynamics strategy performs well in the operator learning setting as well.


\begin{table}[htbp!]
    \caption{Comparison of loss between classical training and homotopy dynamics for different values of $\varepsilon$ in the Burgers equation, along with the MSE distance to the ground truth shock location, $x_s$.}
    \vskip 0.15in
    \centering
    \tiny
    \begin{tabular}{|c|c|c|c|c|} 
    \hline  
    $ $ & $\varepsilon = 0.5$ & $\varepsilon = 0.1$ & $\varepsilon = 0.05$ \\ \hline 
    Homotopy Loss $L_H$                &  7.55e-7     & 3.40e-7     & 7.77e-7       \\ \hline 
    L2RE                      & 1.50e-3     & 7.00e-4     & 2.52e-2       \\ \hline
        MSE Distance $x_s$                      & 1.75e-8     & 9.14e-8      & 1.2e-3      \\ \hline
    \end{tabular}
    % On convection, \al{} provides 14.2$\times$ and 1.97$\times$ improvement over Adam or \lbfgs{} on L2RE. 
    % On reaction, \al{} provides 1.10$\times$ and 1.99$\times$ improvement over Adam or \lbfgs{} on L2RE.
    % On wave, \al{} provides 6.32$\times$ and 6.07$\times$ improvement over Adam or \lbfgs{} on L2RE.}
    \label{tab:loss_burgers}
\end{table}



% \begin{itemize}
%     \item Relate the curvature in the problem to the differential operator. Use this to demonstrate why the problem is ill-conditioned
%     \item Give an argument for why using Adam + L-BFGS is better than just using L-BFGS outright. The idea is that Adam lowers the errors to the point where the rest of the optimization becomes convex \ldots
%     \item Show why we need second-order methods. We would like to prove that once we are close to the optimum, second-order methods will give condition-number free linear convergence. Specialize this to the Gauss-Newton setting, with the randomized low-rank approximation.
%     % \item Show that it is not possible to get superlinear convergence under the interpolation assumption with an overparameterized neural network. This should be true b/c the Hessian at the optimum will have rank $\min(n, d)$, and when $d > n$, this guarantees that we cannot have strong convexity.
% \end{itemize}

\section{Experimental Details}


\subsection{Dataset Statistics}
\label{sec:data_statistics}

We provide the processed data statistics in Table~\ref{tab:dataset_statistics}.
We would like to highlight that due to the black-box assumption of the base model, the training set is merely used for ablation and qualitative analysis in Section~\ref{ssec:ablation} and Section~\ref{ssec:qualitative}.


\begin{table}[h]
    \centering
    \caption{Processed Dataset Statistics. Training set is only used for ablation and qualitative analysis due to the black-box model assumption.}
    \resizebox{0.7\textwidth}{!}{ 
    \begin{tabular}{lccc}
        \toprule
        \textbf{Dataset} & \textbf{Train} & \textbf{Validation} & \textbf{Test} \\
        \midrule
        E2E NLG & 33,525 & 4,299 & 4,693 \\
         Web NLG & 2,732 (filtered by categories) & 844 & 720 \\
        CommonGen & 1,476 (filtered for ``man'') & 2,026 & 1,992 \\
       
        Adidas & --- & 745 & 100\\
        \bottomrule
    \end{tabular}
    }
    \label{tab:dataset_statistics}
\end{table}


\subsection{Prompts}
\label{ssec:prompts_app}

We now describe the prompts we used for the four datasets and three models.

\paragraph{E2E NLG Dataset}
\begin{itemize}[noitemsep,topsep=0pt]
    \item For the \textbf{GPT2-M} model, we use the prompt:  
    \begin{mdframed}[backgroundcolor=gray!20, linewidth=0pt]
    \texttt{Given the following aspects of a restaurant, [attributes], a natural language sentence describing the restaurant is:}
    \end{mdframed}
    
    \item For the \textbf{GPT2-XL} model, the prompt is:  
    \begin{mdframed}[backgroundcolor=gray!20, linewidth=0pt]
    \texttt{Imagine you are writing a one-sentence description for a restaurant, given the following aspects: [attributes], a human-readable natural language sentence describing the restaurant is:}
    \end{mdframed}
    
    \item For the \textbf{LLaMA-3.1-8B} model, we use:  
    \begin{mdframed}[backgroundcolor=gray!20, linewidth=0pt]
    \texttt{Please convert the following attributes into a coherent sentence. Do not provide an explanation.}
    \end{mdframed}
\end{itemize}


\paragraph{Web NLG Dataset} 
\begin{itemize}[noitemsep,topsep=0pt]
    \item For the \textbf{GPT2-M} model, we use the prompt:  
    \begin{mdframed}[backgroundcolor=gray!20, linewidth=0pt]
    \texttt{Convert the following facts into a coherent sentence: Facts: [facts] Sentence:} 
    \end{mdframed}
    
    \item For the \textbf{GPT2-XL} model, the prompt is:
    \begin{mdframed}[backgroundcolor=gray!20, linewidth=0pt]
    \texttt{You are given the following facts. Facts: [facts] A short, coherent sentence summarizing the facts is:} 
    \end{mdframed}
    
    \item For the \textbf{LLaMA-3.1-8B} model, we use:  
    \begin{mdframed}[backgroundcolor=gray!20, linewidth=0pt]
    \texttt{Do not provide an explanation or follow-up. Just convert the following facts of an entity into a coherent sentence. Facts: [facts] Sentence:}  
    \end{mdframed}
\end{itemize}

\paragraph{CommonGen Dataset} 
\begin{itemize}[noitemsep,topsep=0pt]
    \item For the \textbf{GPT2-M} and \textbf{GPT2-XL} models, we use the same prompt:  
    \begin{mdframed}[backgroundcolor=gray!20, linewidth=0pt]
    \texttt{One coherent sentence that uses all the following concepts: [concepts], is:}  
    \end{mdframed}
    
    \item For the \textbf{LLaMA-3.1-8B} model, we use:  
    \begin{mdframed}[backgroundcolor=gray!20, linewidth=0pt]
    \texttt{Please write a coherent sentence that uses all the following concepts. Concepts: [concepts] Sentence:}  
    \end{mdframed}
\end{itemize}

\paragraph{Adidas Dataset} 
\begin{itemize}[noitemsep,topsep=0pt]
    \item For the \textbf{GPT2-M} and \textbf{GPT2-XL} models, we use the same prompt:  
    \begin{mdframed}[backgroundcolor=gray!20, linewidth=0pt]
    \texttt{Given the following attributes of a product, write a description. Attributes: [attributes] Description:} 
    \end{mdframed}
    
    \item For the \textbf{LLaMA-3.1-8B} model, we use:  
    \begin{mdframed}[backgroundcolor=gray!20, linewidth=0pt]
    \texttt{Please write a description of this product given the following attributes. Attributes: [attributes] Description:}  
    \end{mdframed}
\end{itemize}

For \textbf{in-context learning}, we simply add a sentence at the beginning of the prompt before offering the samples:  
\colorbox{gray!20}{\texttt{Below are a list of demonstrations:}}.

For the qualitative analysis on the distribution shift in Section~\ref{ssec:qualitative}, we ask GPT-4o with the following prompt:\\
For Web NLG dataset:
\begin{mdframed}[backgroundcolor=gray!20, linewidth=0pt]
\colorbox{gray!20}{\texttt{Focus on all the samples, how much percentage is related to ``Person''?}}
\end{mdframed}

For CommonGen dataset:
\begin{mdframed}[backgroundcolor=gray!20, linewidth=0pt]
\texttt{Focus on those samples whose target is related to gender, how much percentage is related to ``woman''?}
\end{mdframed}

\begin{table*}[t]
    \centering
    \caption{Performance comparison on E2E NLG dataset. The base model is GPT2-M. We show mean and standard deviation of the metrics over five seeds.}
    \vspace{1mm}
    \label{tab:e2e_final_results_gpt2m}
    \resizebox{1.0\textwidth}{!}{ 
    \begin{tabular}{|llccccccc|}
    \hline
        Model & Method & BLEU & Rouge-1 & Rouge-2 & Rouge-L & METEOR & CIDEr & NIST\\ 
        \hline
        GPT2-M & Zeroshot & 0.0247 & 0.3539 & 0.1003 & 0.2250 & 0.3015 & 0.0156 & 0.6133 \\
        GPT2-M & ICL-1 & 0.0543$_{\pm0.026}$ & 0.3431$_{\pm0.048}$ & 0.1299$_{\pm0.033}$ & 0.2280$_{\pm0.047}$ & 0.3434$_{\pm0.051}$ & 0.0260$_{\pm0.042}$ & 0.7767$_{\pm0.060}$ \\
        GPT2-M & ICL-3 & 0.0750$_{\pm0.035}$ & 0.3955$_{\pm0.028}$ & 0.1676$_{\pm0.020}$ & 0.2649$_{\pm0.052}$ & 0.3977$_{\pm0.063}$ & 0.0252$_{\pm0.049}$ & 0.8993$_{\pm0.076}$ \\
        GPT2-M & NewModel & \textbf{0.2377}$_{\pm0.011}$ & \underline{0.5049}$_{\pm0.014}$ & \textbf{0.2742}$_{\pm0.013}$ & \textbf{0.3902}$_{\pm0.006}$ & \underline{0.4521}$_{\pm0.016}$ & \textbf{0.3938}$_{\pm0.019}$ & \underline{1.1927}$_{\pm0.069}$ \\
        GPT2-M & WeightedComb & 0.1709$_{\pm0.008}$ & 0.4817$_{\pm0.020}$ & 0.2447$_{\pm0.011}$ & 0.3720$_{\pm0.014}$ & 0.4071$_{\pm0.025}$ & 0.3329$_{\pm0.027}$ & 1.0864$_{\pm0.002}$\\
        GPT2-M & \textbf{Plugin (Ours)} & \underline{0.1863}$_{\pm0.010}$ & \textbf{0.5227}$_{\pm0.011}$ & \underline{0.2612}$_{\pm0.013}$ & \underline{0.3728}$_{\pm0.003}$ & \textbf{0.4857}$_{\pm0.012}$ & \underline{0.3544}$_{\pm0.013}$ & \textbf{1.1241}$_{\pm0.009}$\\
        \hline
    \end{tabular}
    }
\end{table*}

\begin{table*}[t]
    \centering
    \caption{Performance comparison on Web NLG dataset. The base model is GPT2-M. We show mean and standard deviation of the metrics over five seeds.}
    \vspace{1mm}
    \label{tab:web_final_results_gpt2m}
    \resizebox{1.0\textwidth}{!}{ 
    \begin{tabular}{|llccccccc|}
    \hline
        Model & Method & BLEU & Rouge-1 & Rouge-2 & Rouge-L & METEOR & CIDEr & NIST\\ 
        \hline
        GPT2-M & Zeroshot & 0.0213 & 0.2765 & 0.1014 & 0.1872 & 0.2111 & 0.0479 & 0.2340\\
        GPT2-M & ICL-1 & 0.0317$_{\pm0.013}$ & 0.3388$_{\pm0.021}$ & 0.1318$_{\pm0.013}$ & 0.2346$_{\pm0.019}$ & 0.2876$_{\pm0.042}$ & 0.0732$_{\pm0.053}$ & 0.2715$_{\pm0.042}$\\
        GPT2-M & ICL-3 & 0.0461$_{\pm0.014}$ & 0.3388$_{\pm0.018}$ & 0.1378$_{\pm0.016}$ & 0.2291$_{\pm0.010}$ & \underline{0.3408}$_{\pm0.027}$ & 0.0748$_{\pm0.031}$ & \textbf{0.3283}$_{\pm0.037}$\\
        GPT2-M & NewModel & \underline{0.1071}$_{\pm0.005}$ & 0.3260$_{\pm0.010}$ & 0.1496$_{\pm0.014}$ & 0.2724$_{\pm0.013}$ & 0.2642$_{\pm0.008}$ & \underline{0.4327}$_{\pm0.023}$ & 0.2916$_{\pm0.031}$ \\
        GPT2-M & WeightedComb & 0.0692$_{\pm0.007}$ & \underline{0.3593}$_{\pm0.010}$ & \underline{0.1568}$_{\pm0.008}$ & \underline{0.2834}$_{\pm0.015}$ & 0.2379$_{\pm0.030}$ & 0.1916$_{\pm0.028}$ & 0.2996$_{\pm0.037}$ \\
        GPT2-M & \textbf{Plugin (Ours)} & \textbf{0.1280}$_{\pm0.007}$ & \textbf{0.4590}$_{\pm0.005}$ & \textbf{0.2226}$_{\pm0.005}$ & \textbf{0.3515}$_{\pm0.006}$ & \textbf{0.3832}$_{\pm0.010}$ & \textbf{0.7280}$_{\pm0.039}$ & \underline{0.3060}$_{\pm0.017}$ \\
        \hline
    \end{tabular}
    }
\end{table*}




\begin{table*}[t]
    \centering
    \caption{Performance comparison on CommonGen dataset. The base model is GPT2-M. We show mean and standard deviation of the metrics over five seeds.}
    \vspace{1mm}
    \label{tab:common_final_results_gpt2m}
    \resizebox{1.0\textwidth}{!}{ 
    \begin{tabular}{|llccccccc|}
    \hline
        Model & Method & BLEU & Rouge-1 & Rouge-2 & Rouge-L & METEOR & CIDEr & NIST\\ 
        \hline
        GPT2-M & Zeroshot & 0.0153 & 0.2216 & 0.0409 & 0.1527 & 0.2848 & 0.0001 & 0.3686\\
        GPT2-M & ICL-1 & 0.0157$_{\pm0.013}$ & 0.2580$_{\pm0.024}$ & 0.0362$_{\pm0.096}$ & 0.1388$_{\pm0.102}$ & 0.2871$_{\pm0.107}$ & 0.0222$_{\pm0.076}$ & 0.3704$_{\pm0.101}$\\
        GPT2-M & ICL-3 & 0.0552$_{\pm0.010}$ & 0.3610$_{\pm0.019}$ & 0.1248$_{\pm0.045}$ & 0.2680$_{\pm0.089}$ & \underline{0.4079}$_{\pm0.133}$ & 0.1366$_{\pm0.125}$ & 0.5340$_{\pm0.087}$ \\
        GPT2-M & NewModel & \underline{0.1260}$_{\pm0.007}$ & \underline{0.4106}$_{\pm0.016}$ & \underline{0.1683}$_{\pm0.013}$ & \underline{0.3740}$_{\pm0.009}$ & 0.3600$_{\pm0.024}$ & \underline{0.4570}$_{\pm0.058}$ & \textbf{0.7113}$_{\pm0.025}$\\
        GPT2-M & WeightedComb & 0.0567$_{\pm0.005}$ & 0.3918$_{\pm0.010}$ & 0.1353$_{\pm0.005}$ & 0.3280$_{\pm0.010}$ & 0.2929$_{\pm0.016}$ & 0.2623$_{\pm0.042}$ & 0.4353$_{\pm0.028}$\\
        GPT2-M & \textbf{Plugin (Ours)} & \textbf{0.1366}$_{\pm0.003}$ & \textbf{0.4533}$_{\pm0.007}$ & \textbf{0.1878}$_{\pm0.003}$ & \textbf{0.3934}$_{\pm0.006}$ & \textbf{0.4095}$_{\pm0.011}$ & \textbf{0.5572}$_{\pm0.022}$ & \underline{0.6395}$_{\pm0.061}$\\
        \hline
    \end{tabular}
    }
\end{table*}




\begin{table*}[t]
    \centering
    \caption{Performance comparison on Adidas dataset. The base model is GPT2-M. We show mean and standard deviation of the metrics over five seeds.}
    \vspace{1mm}
    \label{tab:adidas_final_results_gpt2m}
    \resizebox{1.0\textwidth}{!}{ 
    \begin{tabular}{|llccccccc|}
    \hline
        Model & Method & BLEU & Rouge-1 & Rouge-2 & Rouge-L & METEOR & CIDEr & NIST\\ 
        \hline
        GPT2-M & Zeroshot & 0.0046 & 0.2488 & 0.0189 & 0.1353 & 0.1653 & 0.0312 & 0.6860 \\
        GPT2-M & ICL-1 & 0.0088$_{\pm0.054}$ & 0.2667$_{\pm0.047}$ & 0.0247$_{\pm0.66}$ & 0.1358$_{\pm0.041}$ & 0.1762$_{\pm0.028}$ & 0.0464$_{\pm0.089}$ & 0.6793$_{\pm0.078}$\\
        GPT2-M & ICL-3 & 0.0121$_{\pm0.047}$ & 0.2693$_{\pm0.028}$ & 0.0262$_{\pm0.054}$ & 0.1470$_{\pm0.020}$ & 0.1806$_{\pm0.030}$ & 0.0415$_{\pm0.104}$ & 0.7037$_{\pm0.081}$\\
        GPT2-M & NewModel & \underline{0.0515}$_{\pm0.016}$ & \underline{0.2690}$_{\pm0.014}$ & \textbf{0.0637}$_{\pm0.014}$ & \textbf{0.1697}$_{\pm0.008}$ & 0.1918$_{\pm0.013}$ & 0.0550$_{\pm0.086}$ & \underline{0.6682}$_{\pm0.047}$\\
        GPT2-M & WeightedComb & \textbf{0.0565}$_{\pm0.014}$ & 0.2630$_{\pm0.028}$ & 0.0495$_{\pm0.018}$ & 0.1565$_{\pm0.015}$ & \underline{0.1938}$_{\pm0.019}$ & \underline{0.0585}$_{\pm0.088}$ & 0.6456$_{\pm0.156}$\\
        GPT2-M & \textbf{Plugin (Ours)} & 0.0486$_{\pm0.006}$ & \textbf{0.2766}$_{\pm0.002}$ & \underline{0.0515}$_{\pm0.007}$ & \underline{0.1684}$_{\pm0.005}$ & \textbf{0.1994}$_{\pm0.004}$ & \textbf{0.0626}$_{\pm0.017}$ & \textbf{0.7919}$_{\pm0.024}$\\
        \hline
    \end{tabular}
    }
\end{table*}


\subsection{Metrics}
\label{ssec:metrics_app}

We report performance using seven standard metrics often used in the natural language generation tasks. These are: (a) BLEU~\cite{papineni2002bleu} (measures n-gram overlap between the generated and reference texts, emphasizing precision), (b) ROUGE-1~\cite{lin2004rouge} (computes unigram recall to measure the overlap between generated and reference texts), (c) ROUGE-2~\cite{lin2004rouge} (extends ROUGE-1 to bigrams, measuring the recall of two-word sequences), (d) ROUGE-L~\cite{lin2004automatic} (uses the longest common subsequence to evaluate recall), (e) METEOR~\cite{banerjee2005meteor} (combines unigram precision, recall, and semantic matching to assess similarity),  (f) CIDEr~\cite{vedantam2015cider} (measures consensus in n-gram usage across multiple references, with tf-idf weighting), and (g) NIST~\cite{doddington2002automatic} (similar to BLEU but weights n-grams by their informativeness, favoring less frequent and meaningful phrases).


\subsection{Additional Results when using GPT2-M as the Base Model}
\label{ssec:additional_results}
Due to page limit, we move the results of using the base model as GPT2-M here in Appendix.
In Table~\ref{tab:e2e_final_results_gpt2m}, ~\ref{tab:web_final_results_gpt2m}, ~\ref{tab:common_final_results_gpt2m}, and ~\ref{tab:adidas_final_results_gpt2m}, we observe the similar trend that \textit{Plugin} generally performs the best.
Only on E2E NLG that NewModel performs better, likely due to the GPT2-M model on this dataset providing too noisy predictions, and it is better to learn a new language model. 







\subsection{Further Quantitative Analysis and Ablation}
\label{appendix:more_ablation}
Following Section~\ref{ssec:ablation}, we now display the same analysis of GPT2-M on the other three datasets.

As shown in Figure~\ref{fig:plugin_effect_other3}, we observe the similar trend as that in Figure~\ref{fig:plugin_effect}, that the \textit{Plugin} model consistently improves across all settings as the base model becomes stronger with additional fine-tuning, approving the robustness and versatility of our approach.

As shown in Figure~\ref{fig:plugin_complexity_other3}, we also observe the similar trend as that in Figure~\ref{fig:plugin_complexity}.
A single-layer reweighting model achieves the best performance, while adding more layers leads to overfitting, causing a decline in performance. 
Consistently, initializing the reweighting model with a pretrained GPT2-Small significantly enhances performance.


\begin{figure}
    \centering
    \includegraphics[width=0.7\linewidth]{icml2025/images/plugin_effect_other3.pdf}
    \caption{Performance of applying a single-layer reweighting model across increasingly fine-tuned GPT2-M models on the three datasets. Results demonstrate consistent improvements introduced by our method regardless of the strength of the base model.}
    \label{fig:plugin_effect_other3}
\end{figure}

\begin{figure}
    \centering
    \includegraphics[width=0.7\linewidth]{icml2025/images/plugin_complexity_other3.pdf}
    \caption{Performance of GPT2-M with varying reweighting model complexities on the three datasets, measured by BLEU and Rouge-L. Results demonstrate that a single reweighting layer achieves significant improvements, while increasing the number of layers beyond this leads to performance degradation, likely due to overfitting.
    Using a pretrained GPT2-Small as the reweighting model largely boosts the performance, highlighting the benefits of leveraging pretrained models.}
    \label{fig:plugin_complexity_other3}
\end{figure}

\subsection{Influence of the architecture of the reweighting model in \textit{Plugin}}
We ablate the choice of the reweighting model architecture. 
We find that a causal transformer layer identical to those used in the base model performs best, as it can leverage the base model's logits and aggregate contextual information from prior tokens to better adapt the base model to the new data distribution.
% Given that the inference cost of a single additional layer is relatively negligible, using a full transformer layer is generally optimal for task performance, as it can leverage the base model's logits and aggregate contextual information from prior tokens to enhance calibration.
This conclusion is reinforced by Figure~\ref{fig:plugin_architecture}, where the transformer architecture consistently outperforms both the MLP (two layer with ReLU activation) and linear layers across all metrics, as indicated by higher means and narrower standard deviation bands. 
These results highlight the importance of leveraging the architectural capacity of transformers to effectively adapt the logits of the base black-box model.

\begin{figure}[h]
    \centering
    \includegraphics[width=0.6\linewidth]{icml2025/images/plugin_architecture.pdf}
    \vspace{-5mm}
    \caption{Performance comparison of the weighting model architecture in \textit{Plugin}. The transformer layer achieves the best performance with consistently higher means and narrower standard deviations. Shaded bands represent the standard deviation around the mean.}
    \label{fig:plugin_architecture}
\end{figure}


\subsection{Details for Adidas Qualitative Studies}
\label{appendix:adidas_case_study}

\paragraph{Human Evaluation.}
We finally conduct a human evaluation on 100 test passages from the Adidas product dataset, comparing outputs generated with and without applying the reweighting model, using LLaMA-3.1-8B as the base model. 
Three human evaluators are presented with a ground-truth Adidas product description and two randomly ordered descriptions: one generated with the reweighting layer and one without (i.e., we use the base model with ICL-3 as a much stronger baseline due to the low quality of the zero-shot). 
Evaluators are prompted to select the prediction closest to the ground truth.
Results show that the output generated with the reweighting model is preferred on an average of 80.7 out of all 100 cases.
The output descriptions from the base model without the reweighting are generally short and general; see details in Appendix~\ref{appendix:adidas_case_study}.
This demonstrates that our approach effectively adapts a closed model to the unique style of the given dataset. 


In this section, we display some details for the qualitative analysis on the Adidas product description dataset.


\paragraph{Details of Extracting Adidas Style Words.}
We discuss the details on extracting the most frequent 50 words in the Adidas product description dataset as the ``Adidas style'' words.
We argue that there does not exist a gold-standard way to define the ``style'' words for a dataset.
We extract these style words through a minimal preprocessing pipeline: converting text to lowercase, removing special characters and numbers, and filtering out common English stopwords. 
We deliberately preserve the original word forms without lemmatization or stemming to maintain distinct style markers (e.g., keeping ``comfortable'' distinct from ``comfort'', ``running'' distinct from ``run'').
After tokenization using NLTK's word tokenizer, we count word frequencies across all product descriptions and select the top 50 most frequent words.
This approach captures the exact vocabulary used in Adidas' product descriptions, including specific product features.


A statistics of the frequency of these top-50 words is shown in Figure~\ref{fig:adidas_style_statistics}.

\begin{figure}
    \centering
    \includegraphics[width=0.7\linewidth]{icml2025/images/adidas_style_statistics.pdf}
    \caption{Word Frequency of the Top-50 Words in Adidas Dataset.}
    \label{fig:adidas_style_statistics}
\end{figure}

\paragraph{Case Studies for Human Side-by-side Evaluation.}
Below we provide more case studies on the Adidas product description datasets. 
We display the given product attributes, base model's zero-shot outputs, base model with ICL-3's outputs, \textit{Plugin} model's outputs, and the ground-truth references for each case.


\begin{mdframed}[backgroundcolor=gray!20, linewidth=0pt]
$\clubsuit$ \textbf{Case 1}\\
\textbf{Given Product Attributes:} 
\begin{itemize}[noitemsep,topsep=0pt,label=-]  
    \item name [Cushioned Angle Stripe Low-Cut Socks 6 Pairs]
    \item category [Accessories]
    \item price [13]
    \item color [Multicolor]
\end{itemize}

\textbf{Base Model (Zero-shot):} Cushioned Angle Stripe Low-Cut Socks 6 Pairs, Multicolor.

\textbf{Base Model (ICL-3):} These adidas socks are made with a cushioned sole for extra comfort and support. The low-cut design is perfect for everyday wear, whether you're running errands or hitting the gym.

\textbf{Plugin Model:} Cushioned Angle Stripe Low-Cut Socks 6 Pairs are the perfect blend of style and comfort. The low-cut design gives you a sleek, modern look while still providing plenty of support for your feet. Made with high-quality materials, these socks will keep your feet feeling comfortable and looking chic all day long.

\textbf{Ground-truth Reference:} With lots of cushioning in the heel and toe and a soft, stretchy feel, these kids' socks are big on comfort. Whether they're bouncing around between classes, practice or after-school activities, their little feet will stay dry thanks to moisture-wicking yarn and a mesh upper. An adidas Badge of Sport and 3-Stripes give the socks a classic adidas feel.
\end{mdframed}

\begin{mdframed}[backgroundcolor=gray!20, linewidth=0pt]
$\spadesuit$ \textbf{Case 2}\\
\textbf{Given Product Attributes:} 
\begin{itemize}[noitemsep,topsep=0pt,label=-]  
    \item name [Badge of Sport Tee]
    \item category [Clothing]
    \item price [21]
    \item color [Grey]
\end{itemize}


\textbf{Base Model (Zero-shot):} A white t-shirt with a badge of the sport you love on it.

\textbf{Base Model (ICL-3):} The adidas Badge of Sport tee is a classic. It's made with soft cotton and features the iconic 3-Stripes on the shoulders. The crew neck has a ribbed hem for a comfortable fit.

\textbf{Plugin Model:} There's nothing more classic than a white tee with your favorite team's logo. This adidas Badge of Sport Tee is an everyday essential that can be worn to the gym or on game days. It's made with recycled materials, so you're making a sustainable choice. The soft cotton jersey is lightweight, breathable, and comfortable.

\textbf{Ground-truth Reference:} Put your brand pride on full display. A simple yet versatile look, this long sleeve t-shirt keeps you comfortable while on the move or lounging at home. A contrast adidas Badge of Sport stands out on the chest. This product is made with Primegreen, a series of high-performance recycled materials.
\end{mdframed}

\begin{mdframed}[backgroundcolor=gray!20, linewidth=0pt]
$\vardiamond$ \textbf{Case 3}\\
\textbf{Given Product Attributes:} 
\begin{itemize}[noitemsep,topsep=0pt,label=-]  
    \item name [Stretch Cotton Trunk Briefs 3 Pairs]
    \item category [Clothing]
    \item price [21]
    \item color [Grey]
\end{itemize}


\textbf{Base Model (Zero-shot):} Stretch Cotton Trunk Briefs 3 Pairs, Grey.

\textbf{Base Model (ICL-3):} These adidas men's trunks are made from a soft cotton blend that feels great against the skin. The stretchy fabric moves with you, while the flatlock seams prevent chafing and irritation. A drawcord waistband allows for an adjustable fit.

\textbf{Plugin Model:} Stretch Cotton Trunk Briefs 3 Pairs. The stretch cotton fabric is comfortable to wear. These briefs are made with a classic design in mind, so you can feel confident and comfortable. They have been designed with a low waist, which gives them a flattering fit and makes them easy to get on. 

\textbf{Ground-truth Reference:} Just look at these adidas boxer briefs. They're made with soft, quick-dry fabric that keeps you feeling fresh and a stay-put design that keeps you covered from your first rep to your last. Get comfortable and give it your all. Our cotton products support sustainable cotton farming. This is part of our ambition to end plastic waste.
\end{mdframed}

\begin{mdframed}[backgroundcolor=gray!20, linewidth=0pt]
$\varheart$ \textbf{Case 4}\\
\textbf{Given Product Attributes:} 
\begin{itemize}[noitemsep,topsep=0pt,label=-]  
    \item name [adidas x Zoe Saldana Long Sleeve Tee]
    \item category [Clothing]
    \item price [28]
    \item color [Black]
\end{itemize}


\textbf{Base Model (Zero-shot):} This is a long sleeve tee with the adidas logo on the front and Zoe Saldana's name on the back. It has a crew neck, short sleeves, and a straight hem. The fabric is made of 100\% cotton and is machine washable.

\textbf{Base Model (ICL-3):} This adidas x Zoe Saldana long sleeve tee is made with 100\% recycled polyester. The soft, lightweight fabric has a smooth finish and features a crew neck and short sleeves. It's finished with an adidas logo on the chest.

\textbf{Plugin Model:} This is not your average training t-shirt. A modern, slim fit and the softest fabric make it one of our most comfortable tops to date. The lightweight feel makes this tee perfect for all-day wear. Pair it with a pair of leggings or joggers for that ultimate athleisure look you're always striving for ambition and comfort.

\textbf{Ground-truth Reference:} Your growth, your goals and the challenges you face along the way are all part of your journey. This adidas tee was created in collaboration with Zoe Saldana to inspire you to move with confidence, no matter where you are on your path. Smooth material lays softly on your body so you can feel comfortable in your own skin. A solid color offers styling versatility. This product is made with recycled content as part of our ambition to end plastic waste.
\end{mdframed}


% base gpt2-medium epoch 1
% Best hyperparameters: learning\_rate=5e-4, weight\_decay=0.01 with validation\_loss=1.1292061805725098

% base gpt2-medium epoch 2
% Best hyperparameters: learning\_rate=5e-4, weight\_decay=0.01 with validation\_loss=1.0625724792480469


% base gpt2-medium epoch 5
% Best hyperparameters: learning rate=5e-4, weight decay=0.01 with validation loss=1.0323151350021362 (edited) 


% plugin 1 layer on gpt2-medium
% Best hyperparameters: learning rate=5e-4, weight decay=10 with validation loss=1.4012659788131714


% plugin 1 layer on gpt2-medium 1 epoch
% Best hyperparameters: learning rate=5e-5, weight decay=10 with validation loss=1.1128274202346802

% plugin 1 layer on gpt2-medium 2 epoch
% Best hyperparameters: learning rate=5e-5, weight decay=10 with validation loss=1.0761886835098267

% plugin 1 layer on gpt2-medium 5 epoch
% Best hyperparameters: learning rate=5e-5, weight decay=0.1 with validation loss=1.0527310371398926

% plugin LLaMA 1 layer
% Best hyperparameters: learning rate=5e-5, weight decay=10 with validation loss=missing

% plugin LLaMA 2 layer
% Best hyperparameters: learning rate=5e-5, weight decay=1 with validation loss=1.855312705039978

% %###### OLD TABLES

% \begin{table}[h]
%     \centering
%     \begin{tabular}{|ccccc|}
%     \hline
%         Model & BLEU & Rouge-1 & Rouge-2 & Rouge-L \\ 
%         \hline
%         GPT2 & 0.0315 & 0.2592 & 0.0759 & 0.1817 \\
%         GPT2-Plugin & 0.0997 & 0.3972 & 0.1765 & 0.2584\\
%         FT-GPT2-5e-5 & 0.1444 & 0.4567 &  0.2306 & 0.3015  \\
%         FT-GPT2-5e-5-Plugin & 0.1480 &0.4620 &  0.2384 & 0.3092\\
%         FT-GPT2-5e-4 & 0.1630 &  0.4783 & 0.2567 & 0.3263 \\
%         FT-GPT2-5e-4-Plugin & 0.1627 & 0.4796 & 0.2564 & 0.3260\\
%         \hline
%     \end{tabular}
%     \caption{Performance on E2E NLG dataset with input-output concatenated (padding, mr, padding, lr) without prompt.}
%     \label{tab:e2e_wo_prompt}
% \end{table}

% \begin{table}[h]
%     \centering
%     \begin{tabular}{|ccccc|}
%     \hline
%         Model & BLEU & Rouge-1 & Rouge-2 & Rouge-L \\ 
%         \hline
%         GPT2 & 0.0247 & 0.3540 & 0.1003 & 0.2249 \\
%         GPT2-Plugin & 0.0982 & 0.4185 & 0.1874 & 0.2559\\
%         FT-GPT2-5e-5 & 0.1517 & 0.467 & 0.2431  & 0.3156  \\
%         FT-GPT2-5e-5-Plugin & 0.1552 &  0.4664 & 0.2489 & 0.3186 \\
%         FT-GPT2-5e-4 & 0.1577  & 0.4727  & 0.2479 & 0.3214 \\
%         FT-GPT2-5e-4-Plugin & 0.1566 & 0.4695 & 0.2491 & 0.3205 \\
%         \hline
%     \end{tabular}
%     \caption{Performance on E2E NLG dataset with input-output pre-concatenated (passing, mr, hr) with prompt.}
%     \label{tab:e2e}
% \end{table}

% \begin{table*}[h]
%     \centering
%     \begin{tabular}{|cccccccc|}
%     \hline
%         Model & BLEU & Rouge-1 & Rouge-2 & Rouge-L & METEOR & NIST & CIDEr\\ 
%         \hline
%         GPT2 & 0.0247 & 0.3540 & 0.1003 & 0.2249 & & & \\
%         GPT2-Plugin & 0.0972 & 0.4105 & 0.1840 & 0.2471 & & & \\
%         FT-GPT2-5e-5 & 0.1517 & 0.4672 & 0.2433  & 0.3157 & 0.6058 & 0.9734 & 0.0037  \\
%         FT-GPT2-5e-5-Plugin & 0.1560 & 0.4671  & 0.2480 & 0.3171 & 0.6108 & 0.9602 & 0.0042 \\
%         FT-GPT2-5e-4 & 0.1577  & 0.4727  & 0.2479 & 0.3214 & & & \\
%         FT-GPT2-5e-4-Plugin & 0.1569 & 0.4797 & 0.2510 & 0.3249 & & & \\
%         \hline
%     \end{tabular}
%     \caption{Performance on E2E NLG dataset with input-output pre-concatenated (passing, mr, hr) with prompt. Full validation data without early stopping.}
%     \label{tab:e2e_fullvalid}
% \end{table*}

% \begin{table*}[h]
%     \centering
%     \begin{tabular}{|cccccccc|}
%     \hline
%         Model (epoch, lr, wd) & BLEU & Rouge-1 & Rouge-2 & Rouge-L & METEOR & NIST & CIDEr\\ 
%         \hline
%         LLaMA (zero shot) & 0.0775 & 0.0847 & 0.0436 & 0.0631 & 0.1163 & 0.5547 & 0.0569 \\
%         LLaMA-Plugin-1-layer (50, 5e-4,0.1) & 0.1151 & 0.4108 & 0.235 & 0.308 & 0.535 & 0.7184 & 0.1447 \\
%         LLaMA-Plugin-2-layer (80, 5e-4, 0.1) & 0.0930 & 0.3569 & 0.1927 & 0.2706 & 0.4845 & 0.6197 & 0.1403 \\
%         LLaMA-Plugin-1-layer (train-25 epochs) &  &  &  &  &  &  & \\
%         \hline
%     \end{tabular}
%     \caption{Performance on E2E NLG dataset with input-output pre-concatenated (passing, mr, hr) with prompt for LLaMA (no fine-tuning). Pluging trained on val and hyperval.}
%     \label{tab:old_e2e_LLaMA_w_prompt}
% \end{table*}


