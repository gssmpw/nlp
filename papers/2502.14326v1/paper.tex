%%
%% This is file `sample-sigconf-authordraft.tex',
%% generated with the docstrip utility.
%%
%% The original source files were:
%%
%% samples.dtx  (with options: `all,proceedings,bibtex,authordraft')
%% 
%% IMPORTANT NOTICE:
%% 
%% For the copyright see the source file.
%% 
%% Any modified versions of this file must be renamed
%% with new filenames distinct from sample-sigconf-authordraft.tex.
%% 
%% For distribution of the original source see the terms
%% for copying and modification in the file samples.dtx.
%% 
%% This generated file may be distributed as long as the
%% original source files, as listed above, are part of the
%% same distribution. (The sources need not necessarily be
%% in the same archive or directory.)
%%
%%
%% Commands for TeXCount
%TC:macro \cite [option:text,text]
%TC:macro \citep [option:text,text]
%TC:macro \citet [option:text,text]
%TC:envir table 0 1
%TC:envir table* 0 1
%TC:envir tabular [ignore] word
%TC:envir displaymath 0 word
%TC:envir math 0 word
%TC:envir comment 0 0
%%
%%
%% The first command in your LaTeX source must be the \documentclass
%% command.
%%
%% For submission and review of your manuscript please change the
%% command to \documentclass[manuscript, screen, review]{acmart}.
%%
%% When submitting camera ready or to TAPS, please change the command
%% to \documentclass[sigconf]{acmart} or whichever template is required
%% for your publication.
%%
%%

%\documentclass[sigconf,review]{acmart}
\documentclass[sigconf]{acmart}


%\usepackage{svg}
\usepackage{seqsplit}
%\usepackage{amssymb}
\usepackage{url}
\usepackage{listings}
\usepackage{color}
\usepackage{xcolor}
\usepackage{textcomp}



\colorlet{punct}{red!60!black}
\definecolor{background}{HTML}{EEEEEE}
\definecolor{delim}{RGB}{20,105,176}
\colorlet{numb}{magenta!60!black}

\lstdefinelanguage{json}{
    basicstyle=\scriptsize,
    numbers=left,
    numberstyle=\scriptsize,
    stepnumber=1,
    numbersep=8pt,
    showstringspaces=false,
    breaklines=true,
    frame=single,
    %backgroundcolor=\color{background},
    literate=
     *{0}{{{\color{numb}0}}}{1}
      {1}{{{\color{numb}1}}}{1}
      {2}{{{\color{numb}2}}}{1}
      {3}{{{\color{numb}3}}}{1}
      {4}{{{\color{numb}4}}}{1}
      {5}{{{\color{numb}5}}}{1}
      {6}{{{\color{numb}6}}}{1}
      {7}{{{\color{numb}7}}}{1}
      {8}{{{\color{numb}8}}}{1}
      {9}{{{\color{numb}9}}}{1}
      {:}{{{\color{punct}{:}}}}{1}
      {,}{{{\color{punct}{,}}}}{1}
      {\{}{{{\color{delim}{\{}}}}{1}
      {\}}{{{\color{delim}{\}}}}}{1}
      {[}{{{\color{delim}{[}}}}{1}
      {]}{{{\color{delim}{]}}}}{1},
}


\definecolor{listinggray}{gray}{0.9}
\definecolor{lbcolor}{rgb}{0.9,0.9,0.9}
\lstset{
	%backgroundcolor=\color{lbcolor},
	tabsize=2,
	rulecolor=,
	language=html,
        basicstyle=\scriptsize,
        upquote=true,
        aboveskip={0.5\baselineskip},
        columns=fixed,
        showstringspaces=false,
        extendedchars=true,
        breaklines=true,
        prebreak = \raisebox{0ex}[0ex][0ex]{\ensuremath{\hookleftarrow}},
        frame=single,
        showtabs=false,
        showspaces=false,
        showstringspaces=false,
        identifierstyle=\ttfamily,
        keywordstyle=\color[rgb]{0,0,1},
        commentstyle=\color[rgb]{0.133,0.545,0.133},
        stringstyle=\color[rgb]{0.627,0.126,0.941},
        %emphstyle=\color[rgb]{0.827,0.126,0.941},
}




%Listings for JS
\definecolor{lightgray}{rgb}{.9,.9,.9}
\definecolor{darkgray}{rgb}{.4,.4,.4}
\definecolor{purple}{rgb}{0.65, 0.12, 0.82}
\definecolor{forestgreen}{rgb}{0.13, 0.55, 0.13}


\lstdefinelanguage{JavaScript}{
  keywords={typeof, new, true, false, catch, function, return, null, catch, switch, var, if, in, while, do, else, case, break},
  keywordstyle=\color{blue}\bfseries,
  %keywordstyle=\color{black}\bfseries,
  ndkeywords={class, export, boolean, throw, implements, import},
  %ndkeywordstyle=\color{darkgray}\bfseries,
  ndkeywordstyle=\color{black}\bfseries,
  identifierstyle=\color{black},
  sensitive=false,
  comment=[l]{//},
  morecomment=[s]{/*}{*/},
  commentstyle=\color{forestgreen}\ttfamily,
  stringstyle=\color{purple}\ttfamily,
  %commentstyle=\color{black}\ttfamily,
  %stringstyle=\color{black}\ttfamily,
  morestring=[b]',
  morestring=[b]"
}
\lstset{
   language=JavaScript,
   %backgroundcolor=\color{lightgray},
   extendedchars=true,
   basicstyle=\scriptsize\ttfamily,
   showstringspaces=false,
   showspaces=false,
  % numbers=left,
   %numberstyle=\footnotesize,
%   numbersep=-6pt,
   tabsize=1,
   breaklines=true,
   showtabs=false,
   captionpos=b,
   numberblanklines=false,
   escapeinside=\[\]
}


\newcommand{\code}[1]{{\texttt{\small#1}}}

\newboolean{showcomments}
\setboolean{showcomments}{true}
\ifthenelse{\boolean{showcomments}}
{\newcommand{\nb}[2]{
\fbox{\bfseries\sffamily\scriptsize#1}
{\sf\small$\blacktriangleright$\textit{#2}$\blacktriangleleft$}
}
}
{\newcommand{\nb}[2]{}
}

\newcommand\kaitong[1]{{\color{teal}{\nb{Kaitong}{#1}}}}
\newcommand\huazhu[1]{{\color{blue}{\nb{Huazhu}{#1}}}}
\newcommand\amin[1]{{\color{red}{\nb{Amin}{#1}}}}





%%
%% \BibTeX command to typeset BibTeX logo in the docs
\AtBeginDocument{%
  \providecommand\BibTeX{{%
    Bib\TeX}}}

%% Rights management information.  This information is sent to you
%% when you complete the rights form.  These commands have SAMPLE
%% values in them; it is your responsibility as an author to replace
%% the commands and values with those provided to you when you
%% complete the rights form.
% \setcopyright{acmlicensed}
\copyrightyear{2025}
\acmYear{2025}
% \acmDOI{XXXXXXX.XXXXXXX}

%% These commands are for a PROCEEDINGS abstract or paper.
\acmConference[]{arXiv preprint}{Feb 2025}{arXiv}
%%
%%  Uncomment \acmBooktitle if the title of the proceedings is different
%%  from ``Proceedings of ...''!
%%
%%\acmBooktitle{Woodstock '18: ACM Symposium on Neural Gaze Detection,
%%  June 03--05, 2018, Woodstock, NY}
%\acmISBN{978-1-4503-XXXX-X/18/06}


%%
%% Submission ID.
%% Use this when submitting an article to a sponsored event. You'll
%% receive a unique submission ID from the organizers
%% of the event, and this ID should be used as the parameter to this command.
%%\acmSubmissionID{123-A56-BU3}

%%
%% For managing citations, it is recommended to use bibliography
%% files in BibTeX format.
%%
%% You can then either use BibTeX with the ACM-Reference-Format style,
%% or BibLaTeX with the acmnumeric or acmauthoryear sytles, that include
%% support for advanced citation of software artefact from the
%% biblatex-software package, also separately available on CTAN.
%%
%% Look at the sample-*-biblatex.tex files for templates showcasing
%% the biblatex styles.
%%

%%
%% The majority of ACM publications use numbered citations and
%% references.  The command \citestyle{authoryear} switches to the
%% "author year" style.
%%
%% If you are preparing content for an event
%% sponsored by ACM SIGGRAPH, you must use the "author year" style of
%% citations and references.
%% Uncommenting
%% the next command will enable that style.
%%\citestyle{acmauthoryear}


%%
%% end of the preamble, start of the body of the document source.
\begin{document}

%%
%% The "title" command has an optional parameter,
%% allowing the author to define a "short title" to be used in page headers.
\title{Browser Fingerprint Detection and Anti-Tracking}

%\subtitle{Data/Toolset paper}

%%
%% The "author" command and its associated commands are used to define
%% the authors and their affiliations.
%% Of note is the shared affiliation of the first two authors, and the
%% "authornote" and "authornotemark" commands
%% used to denote shared contribution to the research.

\author{Kaitong Lin}
\affiliation{%
  \institution{New York Institute of Technology}
  \city{Vancouver}
  \country{Canada}}
\email{klin16@nyit.edu}

\author{Huazhu Cao}
\affiliation{%
  \institution{New York Institute of Technology}
  \city{Vancouver}
  \country{Canada}}
\email{hcao06@nyit.edu}

\author{Amin Milani Fard}
\affiliation{%
  \institution{New York Institute of Technology}
  \city{Vancouver}
  \country{Canada}}
\email{amilanif@nyit.edu}


%%
%% By default, the full list of authors will be used in the page
%% headers. Often, this list is too long, and will overlap
%% other information printed in the page headers. This command allows
%% the author to define a more concise list
%% of authors' names for this purpose.
%\renewcommand{\shortauthors}{Cao et al.}

%%
%% The abstract is a short summary of the work to be presented in the
%% article.
\begin{abstract}
Digital fingerprints have brought great convenience and benefits to many online businesses. However, they pose a significant threat to the privacy and security of ordinary users. In this paper, we investigate the effectiveness of current antitracking methods against digital fingerprints and design a browser extension that can effectively resist digital fingerprints and record the website's collection of digital fingerprint-related information.

\end{abstract}


%%
%% The code below is generated by the tool at http://dl.acm.org/ccs.cfm.
%% Please copy and paste the code instead of the example below.
%%
\begin{CCSXML}
<ccs2012>
   <concept>
       <concept_id>10002978.10003029.10011703</concept_id>
       <concept_desc>Security and privacy~Usability in security and privacy</concept_desc>
       <concept_significance>500</concept_significance>
       </concept>
   <concept>
       <concept_id>10002978.10003022</concept_id>
       <concept_desc>Security and privacy~Software and application security</concept_desc>
       <concept_significance>500</concept_significance>
       </concept>
   <concept>
       <concept_id>10002978.10002991.10002995</concept_id>
       <concept_desc>Security and privacy~Privacy-preserving protocols</concept_desc>
       <concept_significance>500</concept_significance>
       </concept>
   <concept>
       <concept_id>10003456.10003462.10003477</concept_id>
       <concept_desc>Social and professional topics~Privacy policies</concept_desc>
       <concept_significance>500</concept_significance>
       </concept>
 </ccs2012>
\end{CCSXML}

\ccsdesc[500]{Security and privacy~Usability in security and privacy}
\ccsdesc[500]{Security and privacy~Software and application security}
\ccsdesc[500]{Security and privacy~Privacy-preserving protocols}
\ccsdesc[500]{Social and professional topics~Privacy policies}


%%
%% Keywords. The author(s) should pick words that accurately describe
%% the work being presented. Separate the keywords with commas.
\keywords{Browser Fingerprint, Fingerprint Detection, Fingerprint Antitracking, Fingerprint Obfuscation}



% \received{20 February 2007}
% \received[revised]{12 March 2009}
% \received[accepted]{5 June 2009}

%%
%% This command processes the author and affiliation and title
%% information and builds the first part of the formatted document.
\maketitle

\section{Introduction}


Digital fingerprints of a browser are collections of various user information, such as browser and system version and device attributes, which are relatively fixed and difficult to avoid or delete. They are widely used in the Internet industry for purposes such as identity verification on social networks, advertising platforms, and financial services. These fingerprints enable accurate identification of individuals across websites and devices, often without consent, posing a significant threat to privacy. The harm that invasive tracking brings is greater than that of cookies. Even if users have some anti-tracking awareness and use some anti-tracking measures, such as regularly cleaning up browser local storage or using private browsing mode, they may still be tracked \cite{vastel2018fp}. Although many browser manufacturers have pointed out the dangers of digital fingerprint tracking, the use of digital fingerprints has increased significantly over the past decade \cite{mozillaSecurityAntiTracking}. Preventing the misuse of digital fingerprints is critical to ensuring higher privacy protection.

Digital fingerprint algorithms often use fuzzy hashing technology, which maps similar feature sets to similar or identical fingerprint values. For example, a slight change in the browser's screen resolution or time zone will not significantly affect the final hash value because the algorithm focuses on patterns of overall features rather than single details. Therefore, changing the digital fingerprint requires many feature changes, which is difficult for ordinary users. Hence, an anti-digital fingerprint extension that detects and disguises identities is significant for ordinary users. It needs to help users reduce the risk of digital fingerprint tracking in an easy-to-understand and easy-to-use way. 

\textbf{Contributions.} Existing approaches have limitations in mitigating browser fingerprint tracking. In this work, we propose a solution by developing a user-friendly and effective Chrome extension, available for download\footnote{\url{https://github.com/nyit-vancouver/browser-fingerprint-detector-and-antitracking-extension}}, with the following core features:

\begin{itemize}
    \item \textbf{Fingerprint Detection}: The extension will detect and analyze the users' current browser fingerprint, allowing them to have an overview and verify validity after spoofing the fingerprint. We use JavaScript APIs and objects that provide device information to implement this.
    \item \textbf{Fingerprint Customization \& Randomization}: Users can customize spoofed browser fingerprints by selecting the provided configuration or toggling a one-click switch. The solution rewrites APIs and objects for accessing fingerprint data and modifies their return values.
    \item \textbf{Tracking Monitoring}: It monitors and logs websites that attempt to track the user using the browser fingerprint. Requests are intercepted to access browser fingerprint information, and logging is recorded when data is accessed.
    \item \textbf{Tracking Dashboard}: Based on the logs generated by the Tracking Monitoring feature, it offers users an overview of the number of times fingerprints have been accessed. To make it accessible and more user-friendly, it uses a chart library to display statistics.
\end{itemize}


\section{Related Work}

Current precautions to prevent Internet tracking include:

\textbf{Private/Incognito Mode.} Enabling the privacy mode of the browser will limit the saving of history records and block the impact of cookies and cached data. Although incognito mode can protect against some simple tracking technologies, it is almost ineffective in defending against digital fingerprints. This is because digital fingerprints use much information, including, but not limited to, browser and hardware device information. Many attributes will not change because the user uses privacy mode. Therefore, privacy mode cannot effectively prevent digital fingerprint tracking \cite{wu2017evaluating}.

\textbf{Ad Blockers \& Privacy Badger.} Ad blockers and Privacy Badger reduce tracking by blocking ads and certain tracking scripts. They are often very effective against common third-party tracking. However, digital fingerprints are often generated by legitimate browser behavior, so these extensions cannot effectively identify and block this situation \cite{mughees2017detecting}.

\textbf{Disabled JavaScript.} Enabling the browser's JavaScript disabling function can effectively prevent device and browser information leakage caused by certain tracking scripts. However, this behavior will seriously affect the typical web browsing experience. In addition, digital fingerprint technology can also collect some user information through HTTP header information \cite{bortolameotti2020headprint}.

\textbf{Tor Browser.} The randomization and standardized browser settings of Tor Browser can significantly reduce the uniqueness of fingerprints and prevent users from being uniquely identified. Although Tor is effective in preventing digital fingerprint tracking, its performance and speed are much lower than those of other conventional browsers. The use of Tor is also tricky and requires some relevant knowledge, making it less effective in terms of user experience and unsuitable for daily use by the public \cite{bortolameotti2020headprint}. 

\textbf{Chameleon.} Chameleon is a browser extension that supports Firefox. It mainly provides functions such as confusing fingerprints through custom options and predefined modes. However, it focuses more on modifying information that is not significantly represented in digital fingerprints. In addition, the configuration is global. Users cannot set a configuration on certain web pages or know what information the website monitors. Meanwhile, users still need some relevant technical knowledge before using it.


\section{The Proposed Solution}

To reduce the risk of browser fingerprint tracking for regular users, we propose a solution by developing a user-friendly and effective Chrome extension since Google Chrome plays a dominant role in the browser market \cite{chromeRole}. Figure \ref{fig:components} illustrates our proposed system that is divided into three main components: (1) Rewriting and Interception Module, (2) Data Generation, Storage and Transfer Module, and (3) Monitoring and Logging Module. The architectural diagram of our design is shown in Figure \ref{fig:1}.


\begin{figure}[h]
\centering
\includegraphics[trim = 6mm 2mm 6mm 0mm,width=1.0\hsize]{diagram.pdf}
\caption{The proposed system components}
\label{fig:components}
\end{figure} 


% The proposed Chrome extension will include the following core features:

% \begin{itemize}
%     \item \textbf{Fingerprint Detection}: the extension will detect and analyze the user’s current browser fingerprint, allowing users to have an overview and verify validity after spoofing the fingerprint. To implement this, we utilize the JavaScript APIs and objects that provide device information.
%     \item \textbf{Fingerprint Customization \& Fingerprint Randomization}: users can customize spoofed browser fingerprints by selecting the provided configuration or toggling a one-click switch. The solution rewrites APIs and objects for accessing fingerprint data and modifies their return values.
%     \item \textbf{Tracking Monitoring}: It monitors and logs websites that attempt to track the user using the browser fingerprint. Requests are intercepted for accessing browser fingerprint information, and logging is recorded when data is accessed.
%     \item \textbf{Tracking Dashboard}: based on the logs generated by the Tracking Monitoring feature, it offers users an overview of the number of times fingerprints have been accessed. To make it accessible and more user-friendly, it uses a chart library to display the statistics.
% \end{itemize}






\subsection{Rewriting and Interception Module}
This module rewrites critical APIs and the request headers related to browser fingerprinting. Generally, a website collects browser fingerprints in two main ways \cite{collectFingerprint}: (1) by calling specific Javascript APIs, and (2) by obtaining certain request headers from a request sent to the Web server.

\subsubsection{API Rewriting.} JavaScript is a widely used browser programming language \cite{js} that provides access to the browser environment through the global object \code{window}. The \code{window} object contains a large number of properties and methods \cite{jsWindow}, such as \code{navigator}, \code{screen}, and \code{document}, which are often used for fingerprint analysis of the browser. The function \code{Object.defineProperty} allows us to define a new property on an object or modify an existing one \cite{defineProperty}. With this function, we can modify the properties and methods related to browser fingerprinting \cite{definePropertyUsage}, even if some are read-only. For example, with the following code, we can modify the value of \code{navigator.userAgent} and get a forged value when calling it:
\begin{lstlisting}
const originalValue = window.navigator.userAgent

Object.defineProperty(window.navigator, 'userAgent', {
  get() {
    return 'Forged UserAgent' // return a forged value here
  },
  configurable: true
})

window.navigator.userAgent // 'Forged UserAgent'
\end{lstlisting}


%\paragraph{Request Interception}
\subsubsection{Request Interception}

Browser fingerprints can be accessed from request headers sent to the server. When a client requests resources, headers are included in the request. Headers such as \code{User-Agent} and \code{Accept-Language} can be used to identify information such as the user's browser, operating system, and personalized settings, which can be related to user identification. There are functions related to web requests in Chrome extension development, such as \code{\seqsplit{chrome.webRequest.onBeforeSendHeaders}} \cite{onBeforeSendHeaders} and \code{\seqsplit{chrome.declarativeNetRequest}} \cite{declarativeNetRequest}, which can be utilized in the solution.



\begin{figure*}[h]
\centering
\includegraphics[trim = 10mm 35mm 10mm 10mm,width=0.6\hsize]{Fig1.pdf}
\caption{Architectural diagram of our browser fingerprint detection and anti-tracking.}
\label{fig:1}
\end{figure*} 

\subsection{Data Generation, Storage and Transfer Module}
This module generates forged fingerprint data based on the user configuration. The solution provides configurable options for key browser characteristics as \code{language} and \code{userAgent}. For dynamic information such as \code{Canvas}, \code{WebGL}, and \code{AudioContext}, it introduces random noise into these fingerprint values. This approach effectively alters the fingerprint while maintaining the visual and aural integrity of the rendered content.

%\paragraph{Data Generation}
\subsubsection{Data Generation}
With the one-click antitracking feature enabled, the system selectively modifies the key browser characteristic in the following ways:

1. Select specific device information, including but not limited to \code{Canvas},  \code{WebGL}, and \code{AudioContext}. As it is more representative, websites tend to track this information to generate browser fingerprints.

2. For static information such as \code{userAgent} and \code{language}, randomly select values in a predefined configuration list. For dynamic information such as \code{WebGL} and \code{Canvas}, inject random noise into sensitive rendering processes.

This strategy ensures that the generated fingerprints look real and varied, thus effectively enhancing privacy protection without compromising the user experience. After being generated and configured, the data will be stored based on the current tab ID, which allows the forged digital fingerprints to remain consistent across a single browsing session while being configuration-independent across multiple browsing sessions.

\subsubsection{Data Storage and Clearing Mechanism}
Compared to Web Storage, Chrome extension development provides more suitable storage. In Chrome extension development, Chrome Storage API provides four different data storage approaches \cite{chromeStorage4api}. For our proposed solution, two types of data should be stored, and we utilize \code{chrome.storage.session} and \code{chrome.storage.local} based on their respective strengths in handling different data types.

1. \textit{Configuration Data}: Since configuration data serves only the current webpage and has a life cycle tied to the webpage duration, it is stored temporarily in \code{chrome.storage.session}. This session storage, with a capacity of up to 10 MB (1 MB in Chrome versions \( \leq \)111) \cite{sessionStorage}, ensures that data is automatically cleared when the browser is closed. When the current tab is closed, our extension actively clears the data to preserve normal space in memory.

2. \textit{Log Data}: It requires long-term persistence to provide detailed insight for the \textit{Tracking Dashboard} feature over time. To meet this requirement, we store the logs in \code{chrome.storage.local}, allowing a maximum capacity of 10 MB (5 MB in Chrome versions \( \leq \)113) \cite{localStorage}. To optimize space, our solution clears 20\% of the storage when the remaining capacity drops below 10\%, ensuring that the storage does not overflow without deleting too much data.

\subsubsection{Data Transfer Process}
In the proposed solution, we differentiate two different environments:

1. \textit{Chrome Extension Environment}: In this environment, Chrome \code{storage} and \code{declarativeNetRequest} APIs can be called to store data and intercept requests, and tab-related event listeners can be set.

2. \textit{Web Environment}: It allows for JavaScript API rewriting and interactions within the webpage.
Communication between these two environments is facilitated by the JavaScript \code{CustomEvent} interface \cite{customEvent}, which can send custom data.

There are two primary components in Chrome extension development. (1) The background script is a service worker script that runs independently of web pages \cite{background}. It listens for browser events, allowing monitoring and reaction to events in the browser. It is loaded when the extension is launched and unloaded when disabled or uninstalled. (2) The content script runs within the context of the web page \cite{content}. By configuring 'matches', 'run\_at', and 'world' in \code{manifest.json}, the content script will be injected into webpage JavaScript when webpages with any URLs are loading:

\lstset{
  literate={<}{{<}}1 {>}{{>}}1 {_}{{\_}}1, 
  basicstyle=\ttfamily\scriptsize,
  breaklines=true,                        
  mathescape=false                         
}

\begin{lstlisting}[language=JavaScript, mathescape=false, caption={manifest.json}]
// manifest.json
{
  "content_scripts": [
    {
      "matches": ["<all\_urls>"],
      "js": ["./static/js/content.js"],
      "run_at": "document_start",
      "world": "MAIN" // share the same execution environment as the webpage's JavaScript
    }
  ]
}
\end{lstlisting}

Since the content script in our proposed solution can only call functions in the webpage, it can be categorized to the web environment.

\begin{figure*}[t]
\centering
\includegraphics[trim = 10mm 15mm 15mm 10mm,width=0.58\hsize]{Fig5.pdf}
\caption{Architectural diagram of two scenarios between two environments.}
\label{fig:twoScenarios}
\end{figure*} 

As shown in Figure \ref{fig:twoScenarios}, in general, there are two scenarios:

\textbf{Scenario 1: Set Configuration.} When a configuration is set, such as turning on the switch or customizing settings, configuration data will be stored in \code{chrome.storage.session}. If the configuration affects request headers, the Chrome extension's user interaction (UI) sends data to the background script via \code{chrome.runtime.sendMessage}. The background script listens for these messages and applies configurations using \code{chrome.declarativeNetRequest} to modify request headers accordingly.

\textbf{Scenario 2: Load a Webpage.} When a webpage is loaded or refreshed, the background script detects the event and executes the Script function, passing the tab ID. The Script function retrieves the configuration data for the tab from \code{chrome.storage.session}. If configuration data exists, it dispatches a \code{CustomEvent} with the configuration details. The Script function listens for log events from the content script.

Simultaneously, the content script is injected into the webpage and listens for custom events from the background script. When receiving a configuration event, JavaScript APIs will be rewritten, the logging logic will be integrated, and configuration will be applied to the modified API responses. The content script injection into the webpage is temporary, and the rewriting APIs before refreshing will be set to default after another refresh because it is not part of the webpage resources. Hence, it is necessary to rewrite them after each refresh. 
If a rewritten API is called, the \textit{Log Collection} Module generates batch log data and sends it to the script via a \code{CustomEvent} to write the log data asynchronously into \code{chrome.storage.local}.

\subsection{Monitoring and Logging Module}
This module logs the number of rewritten API calls within a visit of one page. These logs are implemented in the \textit{Intercept and Rewrite} Module, and the logs are stored to storage asynchronously without blocking subsequent code execution. This provides insight into how often a site attempts to collect fingerprinting data while minimizing the impact on page performance, allowing users to have a more intuitive understanding of browser fingerprinting and evaluate the effectiveness of extensions.

\subsubsection{Logging and Log Collection Module}

In the \textit{API Rewriting} section, we discuss using the \code{\seqsplit{Object.defineProperty}} function to modify existing properties. Logging functionality can also be injected into the process. For example, when modifying the value of \code{\seqsplit{navigator.userAgent}}, we can inject logging logic directly into the getter function:

\begin{lstlisting}
const originalValue = window.navigator.userAgent

Object.defineProperty(window.navigator, 'userAgent', {
  get() {
    logCollector.sendLog('userAgent') // send logs to the log collector
    return 'Forged UserAgent' // return a forged value here
  },
  configurable: true
})
\end{lstlisting}

The LogCollector is a class that collects logs and sends them when the number of logs reaches a limit or timeout. It is used to combine multiple logs to avoid listening misses when multiple custom events are sent at the same time.

It is important to note that the log module runs asynchronously. Since \code{Object.defineProperty} does not support asynchronous functions and real-time log retrieval is not necessary, this design ensures that the process does not prevent subsequent code from executing. This is achieved by sending a custom event with the log data, which the extension environment listens for and writes to \code{chrome.storage.local}. For more details, see the \textit{Data Transfer Process} section.

\subsubsection{Monitoring}

Using the log data collected in the previous step, the proposed solution enables users to visualize and analyze the log through interactive charts and detailed lists. This provides valuable insights, such as the frequency of digital fingerprint access attempts during a single request and the specific attributes being accessed. The log data are formatted and passed to the \code{BarChart} component of the \code{Recharts} library for visualization. In addition, a detailed list shows all URLs and their associated fingerprinting activity. 



% 数据流程图(两种场景):放在 Methodology 的实验设计部分。
% 核心逻辑概述和伪代码:放在 Methodology。



\section{Implementation}

\begin{figure*}[t]
\centering
\includegraphics[width=1\hsize]{menu.png}
\caption{Screenshots of our Chrome extension showing different menu options.}
\label{fig:menu}
\end{figure*} 

\begin{figure*}[t]
\centering
\includegraphics[width=0.9\hsize]{fp_info_new.png}
\caption{A screenshot of fingerprint information.}
\label{fig:FingerprintInfo}
\end{figure*}

\subsection{Chrome Extension UI}

The proposed solution has a pop-up menu with four menu tabs.


\subsubsection{Front page}
On the front page, a switch allows users to get a random spoofed browser fingerprint. In addition, fingerprint details and tracking dashboards are available through the links at the bottom of the home page section, as shown in Figure \ref{fig:menu}.

\subsubsection{UserAgent Settings}
The extension provides five operating systems tabs, including three client systems, Windows, Linux, and macOS, and two mobile systems, Android and iOS. In each tab, multiple options are provided for combinations of system version and browser version. When users turn on the one-click defense on the homepage, the extension will randomly select an option in this part for simulation. They are also free to modify the option.

\subsubsection{Request Headers Settings}
The extension provides multiple options for header generation, including 'Do Not Track' flag, 'prevent eTag Tracking', 'Referer', 'accept language', and 'X-Forwarded' settings. When users turn on one-click protection on the homepage, the extension will enable some values in this section where they can set the options they need.

\subsubsection{Other Settings}
The extension provides some hardware simulation options and other fingerprint resistance settings in this section. Hardware-related settings include screen size, CPU cores, and device memory. Fingerprint resistance includes Canvas fingerprint, WebGL fingerprint, and audio fingerprint obfuscation. It also supports changes in time zones and WebRTC. When the user turns on a one-click defense on the homepage, the extension will be enabled in this section and randomly generate some values. They can also set the options they need here.
% 1.扩展程序部分:
% 扩展程序分为以下几个tab:首页/user agent设置/headers设置/其他设置

% 首页
% 【图片】
% 一键开关:
% 相关API
% 开了开关以后

% user agent设置/headers设置/其他设置:
% 配置数据是基于之前(数据生成模块)讲的,把配置弄成列表,可视化配置
% 【截图】

% 2.网页部分




%     * 插件与网页脚本的通信方式说明和代码。
%     * 说明插件的运行效率(如是否会影响网页加载速度)脚本加载时机
% 测试部分:
% * 测试伪造功能是否有效(如查看指纹追踪脚本的行为)。
% * 使用统计数据(如 API 调用次数、性能指标)证明解决方案的性能和可靠性。

%在设定定时删除阈值时,我们参考了ElasticSearch和Kubernetes的设计。ElasticSearch默认设定磁盘使用率阈值为 85%,超过时开始清理最旧的索引【ref】。Kubernetes 在节点压力驱逐(Node-pressure Eviction)机制中,默认设置了磁盘可用率的阈值不低于10%-5%【ref】。所以我们参考以上设计,将我们的阈值设定为85%,每次删除10%的数据。
% ======
% 未来工作方向: 在 Methodology 末尾,可补充未来扩展方向,例如:
% * 支持更多浏览器 API 的伪造。
% * 提高伪造数据的真实性和随机性。

\subsection{Internal Webpage UI}
%The Internal Webpage UI includes: Fingerprint Details and Tracking Dashboard.

\subsubsection{Fingerprint Details}

The Fingerprint Details page displays the information in such modules: browser, device, operating system, location, hardware, software, and extensions. Users can use this information page to check whether the settings they made through the settings panel are effective. Figure \ref{fig:FingerprintInfo} shows an example of browser fingerprint details.



In the Browser section, the extension obtains and displays the \code{user-agent} information, browser, and version information of the user in the current tab. When users update the configuration in the current tab, the information reflects the modified disguise.

In the Device section, Operating System section, and Location section, the extension obtains and displays the user's device information, operating system information, and timezone information in the current tab and uses \code{ua-parser-js} to process and extract the information. Similarly, when the user modifies the relevant configuration in the current tab, this part of the information will display the modified disguised information.

In the Hardware section, the extension obtains and displays the user's hardware-related information in the current tab, such as screen size, color depth, CPU, and memory. The user can also modify these details through our extension. 

The extension provides a unique canvas fingerprint based on the \code{Canvas} rendering results for the user. This information is generated by drawing text and graphics on the \code{Canvas}, extracting the canvas data as Base64 encoding, and using MD5 hashing to process the \code{Canvas} data to obtain the unique \code{Canvas} fingerprint information.

The extension provides a \code{WebGL} fingerprint for the user. It uses \code{WebGL}'s parameter interface to extract the browser's \code{WebGL} vendor and renderer information, such as \code{gl.VENDOR} and \code{gl.RENDERER}. If the browser supports the \code{WEBGL\_debug\_renderer\_info} extension, more detailed GPU information (\code{unmaskedVendor} and \code{\seqsplit{unmaskedRenderer}}) can also be obtained. The extension renders fixed graphic content (such as a green rectangle) in the \code{WebGL} environment. It reads the pixel data of the rendering result (\code{gl.readPixels}) and converts the data into a unique fingerprint string through MD5 hashing.

The extension provides an audio fingerprint for the user. The generation method roughly creates an audio context, generates a fixed-frequency triangle wave signal, and uses \code{AnalyserNode} to extract its spectrum data. The extracted data is converted into a string, and a unique fingerprint is generated using MD5 hashing.

The information in the Software section can be modified using the Request Headers Settings options. The extension detects available font types by measuring text height, browser function enumeration, and \code{Canvas} detection. When the user turns on the function of disabling \code{WebRTC}, the location-related will not be displayed because this information is currently affected by the \code{WebRTC}.

\subsubsection{Tracking Dashboard}

On the Tracking Dashboard page, the user can view the log information of digital fingerprint-related indicators obtained by each web page when accessing the web page through the current browser. 


\subsection{Performance Analysis}

To compare page loading performance and browser resource consumption before and after using the extension, the browser version tested is Chrome 131.0.6778.70 (arm64). We selected three websites for testing based on content complexity: Wikipedia (low-complexity), Instagram (medium-complexity), and YouTube (high-complexity). Table \ref{tab:performance_metrics} shows our test results. Each test result is based on an average of 50 tests. We can see no noticeable difference in the user's web browsing experience and actual browser resource usage before and after the extension is enabled.

\begin{table}[t]
\centering
\caption{Performance Metrics for Different Platforms}
\label{tab:performance_metrics}
{\footnotesize
\begin{tabular}{@{}llcc@{}}
\toprule
\textbf{Complexity} & \textbf{Platform} & \textbf{Get Configuration (ms)} & \textbf{API Rewrite (ms)} \\ \midrule
High & YouTube           & 37                              & 0.4                        \\
Medium & Instagram         & 36                              & 0.3                        \\
Low & Wikipedia         & 33                              & 0.2                        \\ \bottomrule
\end{tabular}
}
\end{table}

For storage, the impact of extensions on memory is mainly that the code injected by the extensions may affect the memory usage of the page\cite{heapSnapshots}\cite{chromeStorageMv2}. Table \ref{tab:memory_usage} is the average of the memory usage and total JS heap size of each website monitored every 10 seconds for 10 minutes with or without extensions enabled. According to the data in Table \ref{tab:memory_usage}, whether the extension is enabled or not has no significant effect on page memory usage.

\begin{table}[t]
\centering
\caption{Comparison of Total JS Heap Size and Memory Usage.}% Across Test Cases}
\label{tab:memory_usage}
{\footnotesize
\begin{tabular}{@{}lcc@{}}
\toprule
\textbf{Test Case}              & \textbf{Total JS Heap Size} & \textbf{Memory Usage} \\ \midrule
Wikipedia (Disabling extension)    & 22.6 MB                    & 180 MB                \\
Wikipedia (Enabling extension)     & 23.7 MB                    & 181 MB                \\
Instagram (Disabling extension)    & 54.8 MB                    & 310 MB                \\
Instagram (Enabling extension)     & 54.5 MB                    & 310.7 MB                \\
YouTube (Disabling extension)      & 100 MB                     & 378 MB            \\
YouTube (Enabling extension)       & 97 MB                      & 370 MB                \\ \bottomrule
\end{tabular}
}
\end{table}


\section{Conclusion}

In this project, we developed a Chrome extension to detect, analyze, and customize browser fingerprints, addressing the increasing concern over online privacy and evolving user tracking methods. Our solution detects and logs browser fingerprints and provides users with tools to randomize and customize these fingerprints, adding a layer of protection against tracking mechanisms. Overall, this project demonstrates how browser extension technologies can be used to protect user privacy in a digital landscape where tracking and identification methods continue to evolve.










% \subsection{Template Styles}

% The primary parameter given to the ``\verb|acmart|'' document class is
% the {\itshape template style} which corresponds to the kind of publication
% or SIG publishing the work. This parameter is enclosed in square
% brackets and is a part of the {\verb|documentclass|} command:
% \begin{verbatim}
%   \documentclass[STYLE]{acmart}
% \end{verbatim}

% Journals use one of three template styles. All but three ACM journals
% use the {\verb|acmsmall|} template style:
% \begin{itemize}
% \item {\texttt{acmsmall}}: The default journal template style.
% \item {\texttt{acmlarge}}: Used by JOCCH and TAP.
% \item {\texttt{acmtog}}: Used by TOG.
% \end{itemize}

% The majority of conference proceedings documentation will use the {\verb|acmconf|} template style.
% \begin{itemize}
% \item {\texttt{sigconf}}: The default proceedings template style.
% \item{\texttt{sigchi}}: Used for SIGCHI conference articles.
% \item{\texttt{sigplan}}: Used for SIGPLAN conference articles.
% \end{itemize}

% \subsection{Template Parameters}

% In addition to specifying the {\itshape template style} to be used in
% formatting your work, there are a number of {\itshape template parameters}
% which modify some part of the applied template style. A complete list
% of these parameters can be found in the {\itshape \LaTeX\ User's Guide.}

% Frequently-used parameters, or combinations of parameters, include:
% \begin{itemize}
% \item {\texttt{anonymous,review}}: Suitable for a ``double-anonymous''
%   conference submission. Anonymizes the work and includes line
%   numbers. Use with the \texttt{\acmSubmissionID} command to print the
%   submission's unique ID on each page of the work.
% \item{\texttt{authorversion}}: Produces a version of the work suitable
%   for posting by the author.
% \item{\texttt{screen}}: Produces colored hyperlinks.
% \end{itemize}

% This document uses the following string as the first command in the
% source file:
% \begin{verbatim}
% \documentclass[sigconf,authordraft]{acmart}
% \end{verbatim}



% \section{Title Information}

% The title of your work should use capital letters appropriately -
% \url{https://capitalizemytitle.com/} has useful rules for
% capitalization. Use the {\verb|title|} command to define the title of
% your work. If your work has a subtitle, define it with the
% {\verb|subtitle|} command.  Do not insert line breaks in your title.

% If your title is lengthy, you must define a short version to be used
% in the page headers, to prevent overlapping text. The \verb|title|
% command has a ``short title'' parameter:
% \begin{verbatim}
%   \title[short title]{full title}
% \end{verbatim}

% \section{Authors and Affiliations}

% Each author must be defined separately for accurate metadata
% identification.  As an exception, multiple authors may share one
% affiliation. Authors' names should not be abbreviated; use full first
% names wherever possible. Include authors' e-mail addresses whenever
% possible.

% Grouping authors' names or e-mail addresses, or providing an ``e-mail
% alias,'' as shown below, is not acceptable:
% \begin{verbatim}
%   \author{Brooke Aster, David Mehldau}
%   \email{dave,judy,steve@university.edu}
%   \email{firstname.lastname@phillips.org}
% \end{verbatim}

% The \verb|authornote| and \verb|authornotemark| commands allow a note
% to apply to multiple authors --- for example, if the first two authors
% of an article contributed equally to the work.

% If your author list is lengthy, you must define a shortened version of
% the list of authors to be used in the page headers, to prevent
% overlapping text. The following command should be placed just after
% the last \verb|\author{}| definition:
% \begin{verbatim}
%   \renewcommand{\shortauthors}{McCartney, et al.}
% \end{verbatim}
% Omitting this command will force the use of a concatenated list of all
% of the authors' names, which may result in overlapping text in the
% page headers.


% \section{Sectioning Commands}

% Your work should use standard \LaTeX\ sectioning commands:
% \verb|section|, \verb|subsection|, \verb|subsubsection|, and
% \verb|paragraph|. They should be numbered; do not remove the numbering
% from the commands.

% Simulating a sectioning command by setting the first word or words of
% a paragraph in boldface or italicized text is {\bfseries not allowed.}



% \section{Tables}

% The ``\verb|acmart|'' document class includes the ``\verb|booktabs|''
% package --- \url{https://ctan.org/pkg/booktabs} --- for preparing
% high-quality tables.

% Table captions are placed {\itshape above} the table.

% Because tables cannot be split across pages, the best placement for
% them is typically the top of the page nearest their initial cite.  To
% ensure this proper ``floating'' placement of tables, use the
% environment \textbf{table} to enclose the table's contents and the
% table caption.  The contents of the table itself must go in the
% \textbf{tabular} environment, to be aligned properly in rows and
% columns, with the desired horizontal and vertical rules.  Again,
% detailed instructions on \textbf{tabular} material are found in the
% \textit{\LaTeX\ User's Guide}.

% Immediately following this sentence is the point at which
% Table~\ref{tab:freq} is included in the input file; compare the
% placement of the table here with the table in the printed output of
% this document.

% \begin{table}
%   \caption{Frequency of Special Characters}
%   \label{tab:freq}
%   \begin{tabular}{ccl}
%     \toprule
%     Non-English or Math&Frequency&Comments\\
%     \midrule
%     \O & 1 in 1,000& For Swedish names\\
%     $\pi$ & 1 in 5& Common in math\\
%     \$ & 4 in 5 & Used in business\\
%     $\Psi^2_1$ & 1 in 40,000& Unexplained usage\\
%   \bottomrule
% \end{tabular}
% \end{table}

% To set a wider table, which takes up the whole width of the page's
% live area, use the environment \textbf{table*} to enclose the table's
% contents and the table caption.  As with a single-column table, this
% wide table will ``float'' to a location deemed more
% desirable. Immediately following this sentence is the point at which
% Table~\ref{tab:commands} is included in the input file; again, it is
% instructive to compare the placement of the table here with the table
% in the printed output of this document.

% \begin{table*}
%   \caption{Some Typical Commands}
%   \label{tab:commands}
%   \begin{tabular}{ccl}
%     \toprule
%     Command &A Number & Comments\\
%     \midrule
%     \texttt{{\char'134}author} & 100& Author \\
%     \texttt{{\char'134}table}& 300 & For tables\\
%     \texttt{{\char'134}table*}& 400& For wider tables\\
%     \bottomrule
%   \end{tabular}
% \end{table*}

% Always use midrule to separate table header rows from data rows, and
% use it only for this purpose. This enables assistive technologies to
% recognise table headers and support their users in navigating tables
% more easily.

% \section{Math Equations}
% You may want to display math equations in three distinct styles:
% inline, numbered or non-numbered display.  Each of the three are
% discussed in the next sections.

% \subsection{Inline (In-text) Equations}
% A formula that appears in the running text is called an inline or
% in-text formula.  It is produced by the \textbf{math} environment,
% which can be invoked with the usual
% \texttt{{\char'134}begin\,\ldots{\char'134}end} construction or with
% the short form \texttt{\$\,\ldots\$}. You can use any of the symbols
% and structures, from $\alpha$ to $\omega$, available in
% \LaTeX~\cite{Lamport:LaTeX}; this section will simply show a few
% examples of in-text equations in context. Notice how this equation:
% \begin{math}
%   \lim_{n\rightarrow \infty}x=0
% \end{math},
% set here in in-line math style, looks slightly different when
% set in display style.  (See next section).

% \subsection{Display Equations}
% A numbered display equation---one set off by vertical space from the
% text and centered horizontally---is produced by the \textbf{equation}
% environment. An unnumbered display equation is produced by the
% \textbf{displaymath} environment.

% Again, in either environment, you can use any of the symbols and
% structures available in \LaTeX\@; this section will just give a couple
% of examples of display equations in context.  First, consider the
% equation, shown as an inline equation above:
% \begin{equation}
%   \lim_{n\rightarrow \infty}x=0
% \end{equation}
% Notice how it is formatted somewhat differently in
% the \textbf{displaymath}
% environment.  Now, we'll enter an unnumbered equation:
% \begin{displaymath}
%   \sum_{i=0}^{\infty} x + 1
% \end{displaymath}
% and follow it with another numbered equation:
% \begin{equation}
%   \sum_{i=0}^{\infty}x_i=\int_{0}^{\pi+2} f
% \end{equation}
% just to demonstrate \LaTeX's able handling of numbering.

% \section{Figures}

% The ``\verb|figure|'' environment should be used for figures. One or
% more images can be placed within a figure. If your figure contains
% third-party material, you must clearly identify it as such, as shown
% in the example below.
% \begin{figure}[h]
%   \centering
%   \includegraphics[width=\linewidth]{sample-franklin}
%   \caption{1907 Franklin Model D roadster. Photograph by Harris \&
%     Ewing, Inc. [Public domain], via Wikimedia
%     Commons. (\url{https://goo.gl/VLCRBB}).}
%   \Description{A woman and a girl in white dresses sit in an open car.}
% \end{figure}

% Your figures should contain a caption which describes the figure to
% the reader.

% Figure captions are placed {\itshape below} the figure.

% Every figure should also have a figure description unless it is purely
% decorative. These descriptions convey what’s in the image to someone
% who cannot see it. They are also used by search engine crawlers for
% indexing images, and when images cannot be loaded.





% \section{Citations and Bibliographies}

% The use of \BibTeX\ for the preparation and formatting of one's
% references is strongly recommended. Authors' names should be complete
% --- use full first names (``Donald E. Knuth'') not initials
% (``D. E. Knuth'') --- and the salient identifying features of a
% reference should be included: title, year, volume, number, pages,
% article DOI, etc.

% The bibliography is included in your source document with these two
% commands, placed just before the \verb|\end{document}| command:
% \begin{verbatim}
%   \bibliographystyle{ACM-Reference-Format}
%   \bibliography{bibfile}
% \end{verbatim}
% where ``\verb|bibfile|'' is the name, without the ``\verb|.bib|''
% suffix, of the \BibTeX\ file.

% Citations and references are numbered by default. A small number of
% ACM publications have citations and references formatted in the
% ``author year'' style; for these exceptions, please include this
% command in the {\bfseries preamble} (before the command
% ``\verb|\begin{document}|'') of your \LaTeX\ source:
% \begin{verbatim}
%   \citestyle{acmauthoryear}
% \end{verbatim}




% \section{Acknowledgments}

% Identification of funding sources and other support, and thanks to
% individuals and groups that assisted in the research and the
% preparation of the work should be included in an acknowledgment
% section, which is placed just before the reference section in your
% document.

% This section has a special environment:
% \begin{verbatim}
%   \begin{acks}
%   ...
%   \end{acks}
% \end{verbatim}
% so that the information contained therein can be more easily collected
% during the article metadata extraction phase, and to ensure
% consistency in the spelling of the section heading.

% Authors should not prepare this section as a numbered or unnumbered {\verb|\section|}; please use the ``{\verb|acks|}'' environment.

%\section{Appendices}

% If your work needs an appendix, add it before the
% ``\verb|\end{document}|'' command at the conclusion of your source
% document.

% Start the appendix with the ``\verb|appendix|'' command:
% \begin{verbatim}
%   \appendix
% \end{verbatim}
% and note that in the appendix, sections are lettered, not
% numbered. This document has two appendices, demonstrating the section
% and subsection identification method.




%%
%% The acknowledgments section is defined using the "acks" environment
%% (and NOT an unnumbered section). This ensures the proper
%% identification of the section in the article metadata, and the
%% consistent spelling of the heading.
% \begin{acks}
% To Robert, for the bagels and explaining CMYK and color spaces.
% \end{acks}








% chromerole: https://gs.statcounter.com/browser-market-share
% collectFingerprint:https://www.geeksforgeeks.org/what-is-browser-fingerprinting/
% js: https://developer.mozilla.org/en-US/docs/Web/JavaScript
% jsWindow: https://developer.mozilla.org/en-US/docs/Web/API/Window
% defineProperty: https://developer.mozilla.org/en-US/docs/Web/JavaScript/Reference/Global_Objects/Object/defineProperty
% definePropertyUsage: https://barker.codes/blog/property-descriptors-in-javascript/
% onBeforeSendHeaders: https://developer.chrome.com/docs/extensions/reference/api/webRequest#type-OnBeforeSendHeadersOptions
% declarativeNetRequest: https://developer.chrome.com/docs/extensions/reference/api/declarativeNetRequest
% chromeStorage4api:https://developer.chrome.com/docs/extensions/reference/api/storage
% localStorage: https://developer.chrome.com/docs/extensions/reference/api/storage#property-local
% sessionStorage: https://developer.chrome.com/docs/extensions/reference/api/storage#property-session
% customEvent:https://developer.mozilla.org/en-US/docs/Web/API/CustomEvent
% background: https://developer.chrome.com/docs/extensions/get-started/tutorial/service-worker-events
% content: https://developer.chrome.com/docs/extensions/develop/concepts/content-scripts


%% The next two lines define the bibliography style to be used, and
%% the bibliography file.
\bibliographystyle{ACM-Reference-Format}
\bibliography{refs}


%%
%% If your work has an appendix, this is the place to put it.
%\appendix

%\section{Example of a Browser Fingerprint}


%\section{Part Two}


% \section{Online Resources}

% Nam id fermentum dui. Suspendisse sagittis tortor a nulla mollis, in
% pulvinar ex pretium. Sed interdum orci quis metus euismod, et sagittis
% enim maximus. Vestibulum gravida massa ut felis suscipit
% congue. Quisque mattis elit a risus ultrices commodo venenatis eget
% dui. Etiam sagittis eleifend elementum.

% Nam interdum magna at lectus dignissim, ac dignissim lorem
% rhoncus. Maecenas eu arcu ac neque placerat aliquam. Nunc pulvinar
% massa et mattis lacinia.

\end{document}
\endinput
%%
