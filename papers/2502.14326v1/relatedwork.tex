\section{Related Work}
Current precautions to prevent Internet tracking include:

\textbf{Private/Incognito Mode.} Enabling the privacy mode of the browser will limit the saving of history records and block the impact of cookies and cached data. Although incognito mode can protect against some simple tracking technologies, it is almost ineffective in defending against digital fingerprints. This is because digital fingerprints use much information, including, but not limited to, browser and hardware device information. Many attributes will not change because the user uses privacy mode. Therefore, privacy mode cannot effectively prevent digital fingerprint tracking \cite{wu2017evaluating}.

\textbf{Ad Blockers \& Privacy Badger.} Ad blockers and Privacy Badger reduce tracking by blocking ads and certain tracking scripts. They are often very effective against common third-party tracking. However, digital fingerprints are often generated by legitimate browser behavior, so these extensions cannot effectively identify and block this situation \cite{mughees2017detecting}.

\textbf{Disabled JavaScript.} Enabling the browser's JavaScript disabling function can effectively prevent device and browser information leakage caused by certain tracking scripts. However, this behavior will seriously affect the typical web browsing experience. In addition, digital fingerprint technology can also collect some user information through HTTP header information \cite{bortolameotti2020headprint}.

\textbf{Tor Browser.} The randomization and standardized browser settings of Tor Browser can significantly reduce the uniqueness of fingerprints and prevent users from being uniquely identified. Although Tor is effective in preventing digital fingerprint tracking, its performance and speed are much lower than those of other conventional browsers. The use of Tor is also tricky and requires some relevant knowledge, making it less effective in terms of user experience and unsuitable for daily use by the public \cite{bortolameotti2020headprint}. 

\textbf{Chameleon.} Chameleon is a browser extension that supports Firefox. It mainly provides functions such as confusing fingerprints through custom options and predefined modes. However, it focuses more on modifying information that is not significantly represented in digital fingerprints. In addition, the configuration is global. Users cannot set a configuration on certain web pages or know what information the website monitors. Meanwhile, users still need some relevant technical knowledge before using it.