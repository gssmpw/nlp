\begin{table*}[t]
    \centering
    \small
    \renewcommand{\arraystretch}{1.2}  % Ensure row heights are consistent
    \begin{tabular}{m{0.05\textwidth} m{0.95\textwidth}}
        \toprule
        \textbf{IQ} & What causes occupational domination? \\
        \midrule
        \textbf{IA} & \textcolor{blue}{\textit{In IT it is cultural. Women in the U.S. drop out of IT classes because they do not like the male-dominated culture.}} Whereas India, Vietnam, and other countries where we outsource our IT have a much higher percentage of female programmers. \textcolor{blue}{\textit{It is not something about women.}} It is something about western IT classes and tech culture. \\
        \midrule
        \textbf{CA} & Occupation domination occurs due to a combination of historical, social, economic, and geographic factors. \textit{Historically}, certain groups have had greater access to education, resources, and opportunities, leading them to be more prevalent in specific jobs or industries. This is often reinforced by network effects, where individuals within a profession prefer to work with or hire people who share similar backgrounds, cultural norms, or experiences, thus perpetuating the concentration of certain demographics in those fields. \textit{Legal and institutional} barriers can also play a significant role. Discriminatory policies or practices can systematically exclude certain groups from specific professions, allowing other groups, who do not face such barriers, to dominate those fields. \textit{Geographic and environmental} factors further influence occupation domination. The presence of certain natural resources or particular climate conditions in an area can dictate the types of industries that develop there. Local populations often specialize in these industries, passing down specialized skills and knowledge through generations, which solidifies their dominance in those fields over time. \\
        \midrule
        \shortstack{\textbf{FQ}\\by\\\textit{ORG}}  &  
            -- But why do women drop out of IT classes?

            -- \textcolor{red}{\textit{In IT it is cultural. Women in the U.S. drop out of IT classes because they do not like the male-dominated culture.}} This is a myth. \textcolor{red}{\textit{It is not something about women.}} It's a myth that women drop out because of the male-dominant culture.
            
            -- \textcolor{red}{\textit{So it’s cultural?}}
            
            -- What about women who want to be in tech but are not in the workforce?
            
            -- Is there a cultural component of IT that makes it more difficult for women to break into it?
 \\
        \midrule
        \shortstack{\textbf{FQ}\\by\\\textit{AUG}} & 
        -- What are some examples of how outsourcing has impacted the gender imbalance in IT?
        
        -- How do cultural expectations influence the recruitment of female programmers?
        
        -- Can you explain the concept of `cultural fit' in the context of occupational domination?
\\
        \midrule
        \shortstack{\textbf{FQ}\\by\\\textit{FULL}} & 
        -- How do traditional gender roles in professions like medicine and engineering contribute to occupation domination?
        
        -- Why might women feel more comfortable pursuing professions outside of traditional professions?
        
        -- Can you explain the concept of `perceived value' in the context of occupation domination?
\\
        \bottomrule
    \end{tabular}
    \caption{Example of follow-up question generated by three model variants, with comprehensive answers (ID 3168).}
    \label{tab:example_question_short}
\end{table*}