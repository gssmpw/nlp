
\section{Problematic and Unrelated Instance Examples}
\label{app:problematic_sample}
Tables \ref{tab:task_sample_1696} and \ref{tab:task_sample_641} show examples of a problematic instance and an unrelated instance from \fd{} training set, respectively.
\begin{table}[H]
    \centering
    \scriptsize
    \renewcommand{\arraystretch}{1.3}
    \resizebox{\linewidth}{!}{ 
        \begin{tabular}{p{0.25\linewidth} | p{0.75\linewidth} }
            \toprule
            \textbf{Initial Question} & How exactly do you die if you touch electric wires? Do you burn? Suffocate? Heart stop? \\
            \midrule
            \textbf{Initial Answer} & If the current going thru you is around 1 amp, the shock may give you a heart attack. So it's like the hospital shock paddles, only it kills you. If the current is like 10 amps, it probably won't give you a heart attack, and you may survive if your organs aren't fully cooked. Which they definitely could be if the current isn't immediately stopped. And cooked kidneys don't function that good. So to summarize, either heart attack or cooked organs/burned, depending on the current. Edit: To clarify, it takes less than 1 amp to stop your heart, and the current also has to go thru your heart. I just used 1 amp and 10 amps as sample values. \\
            \bottomrule
        \end{tabular}
    }
    \caption{Problematic Instance 1696: Self-Harm-Related Content.}
    \label{tab:task_sample_1696}
\end{table}

\begin{table}[H]
    \centering
    \scriptsize
    \renewcommand{\arraystretch}{1.3}
    \resizebox{\linewidth}{!}{ 
        \begin{tabular}{ p{0.25\linewidth} | p{0.75\linewidth} }
        \toprule
        \textbf{Initial Question} & How its possible for the light to travel constantly with out stop slowing down from the most old galaxies? \\ \midrule
        \textbf{Initial Answer} & Okay, an NFT is a NON-FUNGIBLE TOKEN. Let's take the Mona Lisa for example. [...] So what you've done is bought the off-brand version (online) instead of the real thing. Also, the guy you paid doesn't really own the Mona Lisa. The TLDR is that you're buying an online version of a product and it says you own it, but you don't own the real thing.  \\ 
        \midrule
        \textbf{Follow-up Question} & Thank you for this!  My question is, what's the appeal?  Is it just rich-guy-bragging-rights? [...]  With, say, a baseball card you can display it and know that it's a rare item, and someday resell it at (you hope) a profit.  Are some of these people just speculating and hoping the `value' goes up? \\ 
        \bottomrule
        % You say to a guy, ``hey, I wanna buy the Mona Lisa!'' and this guy says ``sure, pay X amount and I'll hand you a receipt''. So, you pay X amount and you get this lil receipt that says you own the Mona Lisa. That's stored away in a cupboard in amongst all of these other cupboards, except the cupboards are the internet. However, you ask for the Mona Lisa because you've bought it, but this guy is like ``no, you don't own the REAL Mona Lisa, you own the Mona Lisa''.
    \end{tabular}
    }
    \caption{Unrelated Instance 641.}
    \label{tab:task_sample_641}
\end{table}

% \begin{table}[H]
%     \centering
%     \scriptsize
%     \begin{tabular}{lp{0.4\textwidth}}  % Adjust width as needed for your column
%         \hline
%         \textbf{IQ} & When you jump inside of a moving train, why do you land on the same place compared to jumping outside and on top of the train? \\ \hline
%         \textbf{IA} & You, the train, and the air inside the train are all moving together, in the same direction and at the same rate. That's also why the earth can be spinning at 1000 mph, yet you land in the same spot when you jump on the ground.   If you jumped on the *roof* of the train then you would indeed not land at the same place, because the air around you is *not* moving along with the train. \\ \hline
%     \end{tabular}
%     \caption{Problematic Instance 859}
%     \label{tab:task_sample_859}
% \end{table}

% \begin{table}[H]
%     \centering
%     \scriptsize
%     \begin{tabular}{lp{0.4\textwidth}}  % Adjust width as needed for your column
%         \hline
%         \textbf{IQ} & Why is it illegal to boobytrap your own home? \\ \hline
%         \textbf{IA} & I am not aware of any law that makes it specifically illegal to booby trap your own house.  However, when I was in law school 20+ years ago, this question came up.  The explanation given was that if someone booby traps their house, and that booby trap kills someone, they will be legally responsible as if they had intentionally killed the person.  If it would have been justified to blow the guy's brains out then you are okay.  If it wasn't, then you're in trouble.  So if an axe murderer kicks in your door to murder your family, and he runs right into your Home Alone style funhouse and gets killed by an iron that falls 3 stories and hits him in the face\u2026 well then good.  But if the neighbor kid came over because he thinks he left his baseball cap in your kid's room, and he gets his head burned off by a blowtorch above the doorway, well have fun in prison.  Don't take this as legal advice though.  I am a lawyer, but I'm not *your* lawyer. \\ \hline
%     \end{tabular}
%     \caption{Problematic Instance 2141}
%     \label{tab:task_sample_2141}
% \end{table}
      

% \section{Unrelated Instance Examples}
% \label{app:unrelated_sample}
% See Tables~\ref{tab:task_sample_633}, ~\ref{tab:task_sample_641}

% \begin{table}[H]
%     \centering
%     \scriptsize
%     \begin{tabular}{lp{0.4\textwidth}}  % Adjust width as needed for your column
%         \hline
%         \textbf{IQ} & If something from space lands on my property, do I own it? Who owns it if not? \\ \hline
%         \textbf{IA} & No, nebulas are very thin - about 10,000 molecules per cubic centimeter. Meanwhile, our atmosphere has about \( 2.6 \times 10^{19} \) molecules per cubic centimeter. If the nebula was denser it would condense into a star. \\ \hline
%         \textbf{FQ} & What about a rogue gas giant / brown dwarf? Too small to be a star too dense to be a nebula \\ \hline
%     \end{tabular}
%     \caption{Unrelated Instance 633}
%     \label{tab:task_sample_633}
% \end{table}




\section{LLM Prompts}
\label{app:prompts}

\begin{table}[H]
    \centering
    \scriptsize
    \renewcommand{\arraystretch}{1.3}
    \resizebox{\linewidth}{!}{ 
        \begin{tabular}{  p{0.25\linewidth} | p{0.75\linewidth} }
            \toprule
            \ & \textbf{Prompts} \\
            \midrule
            \textbf{Initial Answer} & ``Generate an answer focused on a single perspective only, without any conversational fillers. Do not repeat the question in the answer.'' \\
            \midrule
            \textbf{Next Answer} & ``Please provide a new answer focused on a different perspective, ensuring no overlap with previous answers. Focus on unique aspects or insights not covered earlier, and provide the answer only without any conversational fillers. Do not repeat the question in the answer.'' \\
            \midrule
            \textbf{Comprehensive Answer} & ``Synthesize the following answers into a single, comprehensive response. Integrate the key points and insights from each answer, ensuring a cohesive and well-rounded explanation. The final answer should be thorough and address multiple aspects of the question without unnecessary repetition.'' \\
            \bottomrule
        \end{tabular}
    }
    \caption{Comprehensive Answer Generation Prompts: GPT-4 first generates an answer from a single perspective, then iteratively provides non-overlapping answers from different perspectives, which are finally synthesized into a unified response.}
    % LLM was sequentially prompted to generate answers with uncovered information, then compiled into a comprehensive answer.
    \label{tab:comprehensive_answer_generation}
\end{table}

\begin{table}[H]
    \centering
    \footnotesize
    \renewcommand{\arraystretch}{1.3}
    \resizebox{\linewidth}{!}{ % Adjust table width to fit the document
        \begin{tabular}{ p{\columnwidth} }
            \toprule
            \textbf{Information Gap Identification \& Follow-up Question Generation} \\
            \midrule
        ``Generate all possible follow-up questions as candidates. These follow-up questions must be related to the original question, but must not be rephrases of the original question. These follow-up questions should be answerable by the complete answer. These follow-up questions should not be answered, covered, or detailed by the original answer, but must target terminologies mentioned in the original answer. Separate each follow-up question with `<sep>`.'' \\
            \bottomrule
        \end{tabular}
    }
    \caption{Follow-up Question Generation Prompt.}
    \label{tab:followup_generation_prompt}
\end{table}


\section{Augmented Data - Human Annotation Guideline}
\label{sec:augmented_data_annotation_guideline}

Table \ref{tab:task_description_GPT} presents the job description and annotation questions for our human annotation task.



\begin{table}[H]
    \scriptsize
    \centering
    \begin{tabular}{ p{\columnwidth} }
        \toprule
        \textbf{Job Description} \\
        \midrule
        Welcome, and thank you for participating in this text evaluation task! In this job, you'll be helping us verify the quality of follow-up questions generated by GPT.\\
        For each task, we will provide you with a pair consisting of a question and answer collected from Reddit's ``Explain Like I'm Five” (ELI5) forum. You will be asked to evaluate the quality of the follow-up question generated by GPT. These questions and answers aim to provide layperson-friendly explanations for real-life queries. Here is an example of one task sample: \\
        Each task may contain noise, such as invalid follow-up questions, sensitive information, or questions unrelated to the original question or answer. Your role is to help us identify these noisy samples.\\
        For each task, you will be shown one triple (question, answer, follow-up question). Carefully review each component and answer the following questions based on your judgment: \\
        \midrule
        \textbf{Q1:} Do you think the follow-up question is a valid question? \\
        \textbf{A.} Yes \quad
        \textbf{B.} No \\
        \midrule
        \textbf{Q2:} Does the initial question, answer, or follow-up question contain sensitive information? \\
        \textbf{A.} Yes \quad
        \textbf{B.} No \\
        \midrule
        \textbf{Q3:} Do you think the follow-up question is related to the original question and the answer? \\
        \textbf{A.} Strongly Related \quad
        \textbf{B.} Related \quad
        \textbf{C.} Slightly Related \quad
        \textbf{D.} Not Related \\
        \bottomrule
    \end{tabular}
    \caption{Task description and evaluation questions.}
    \label{tab:task_description_GPT}
\end{table}

\subsection{Valid/Invalid Question Guideline}

The follow-up question might contain multiple sentences but it should consist of at least one valid question. A valid question must be in a question format and ask meaningful information, including Wh-questions (what/why/where/etc.), open-ended questions, probing questions and etc. Invalid questions like ``10000 meters? really?”, are often used in conversational speech to express feelings instead of asking for new information. 
Table~\ref{tab:valid_invalid_followupq} contains examples of valid and invalid follow-up questions.

\begin{table}[H]
    \centering
    \small
    \renewcommand{\arraystretch}{1.3}
    \resizebox{\linewidth}{!}{ 
        \begin{tabular}{ p{0.5\linewidth} | p{0.5\linewidth} }
            \toprule
            \multicolumn{2}{p{\linewidth}}{\textbf{Initial Question:} Why is the sea calm in the mornings?} \\
            \midrule
            \multicolumn{2}{p{1.05\linewidth}}{\textbf{Initial Answer:} There are two types of waves which can turn a flat sea into a rougher one - swell waves and wind waves. Swell waves can arrive at any time of day, but because wind waves are generated by the wind, they only develop when the wind begins to blow steadily. Since wind speeds are often low at night, and increase during the daytime, wind waves often die out during the night, leading to a relatively flat sea (perhaps with swell waves) in the early morning. During the day, the wind waves increase in size as the wind speed increases, leading to a rougher, more choppy, sea surface during the afternoon and evening. } \\
            \midrule
            \textbf{Valid Follow-up} & \textbf{Invalid Follow-up} \\
            \midrule
            Why are winds always weak in the morning and very strong during the day? & Isn't it common sense that the sea is calmer in the morning?\\
            \midrule
            \textbf{Reason} & \textbf{Reason}\\
            \midrule
            The follow-up question is a ``Why” question, asking specific reasons about the change of the winds. Therefore, it is a valid question.  & This is a rhetorical question because it does not genuinely seek new information. It implies that the answer is obvious and does not contribute to the discussion. \\
            \bottomrule
        \end{tabular}
    }
    \caption{Examples of valid and invalid follow-up questions. For the given initial question and answer, the left column presents a valid follow-up question, while the right column features an invalid one, each accompanied by corresponding reasons below.}
    \label{tab:valid_invalid_followupq}
\end{table}




% \paragraph{Initial Question:} ELI5: Why is the sea calm in the mornings? 
% \paragraph{Initial Answer:} There are two types of waves which can turn a flat sea into a rougher one - swell waves and wind waves. Swell waves can arrive at any time of day, but because wind waves are generated by the wind, they only develop when the wind begins to blow steadily. Since wind speeds are often low at night, and increase during the daytime, wind waves often die out during the night, leading to a relatively flat sea (perhaps with swell waves) in the early morning. During the day, the wind waves increase in size as the wind speed increases, leading to a rougher, more choppy, sea surface during the afternoon and evening. 

% \subsubsection{Valid Follow-up Question Example:} 
% "Now it's time for my question. Why are winds always weak in the morning and very strong during the day?" 
% \paragraph{Reason:} The follow-up question is a ``Why” question, asking specific reasons about the change of the winds during the day. Therefore, it is a valid question. 

% \subsubsection{Invalid Follow-up Question Example:}
% “Isn't it just common sense that the sea is calmer in the morning?” 
% \paragraph{Reason:} This is a rhetorical question because it does not genuinely seek new information. Instead, it implies that the answer is already obvious and does not contribute to the discussion in a meaningful way. 

\subsection{Inappropriate Question Guideline}

Examples of racist comments include: ``It's credit to your race,” ``Black people will not understand.” Examples of hate speech include: ``He should go back to where he comes from,” ``All Mexicans are rapists.” Examples of offensive or rude comments include: ``Women are not suitable for working in the IT field,” ``Gay will never understand.” Table~\ref{tab:inappropriate_followupq} contains an example of an inappropriate follow-up question.

\begin{table}[H]
    \centering
    \small
    \renewcommand{\arraystretch}{1.3}
    \resizebox{\linewidth}{!}{ 
        \begin{tabular}{ p{0.5\linewidth} | p{0.5\linewidth} }
            \toprule
            \multicolumn{2}{p{\linewidth}}{\textbf{Initial Question:} Why do people develop eating disorders?} \\
            \midrule
            \multicolumn{2}{p{1.05\linewidth}}{\textbf{Initial Answer:} Eating disorders are complex mental health conditions influenced by a combination of genetic, psychological, environmental, and social factors. While societal beauty standards and pressures can contribute, eating disorders are not simply about wanting to be thin. Conditions like anorexia, bulimia, and binge-eating disorder involve intricate relationships between self-image, emotional regulation, and biological predispositions. Many individuals with eating disorders struggle with anxiety, depression, or trauma, which can further complicate their relationship with food.} \\
            \midrule
            \textbf{Inappropriate Follow-up} & \textbf{Reason} \\
            \midrule
            Why don't people with eating disorders just stop starving themselves and eat normally like everyone else? & This question is dismissive. The phrasing is insensitive and could be harmful to individuals struggling with these conditions.\\
            % \midrule
            % \textbf{Reason} & \textbf{Reason}\\
            % \midrule
            % The follow-up question is a ``Why” question, asking specific reasons about the change of the winds during the day. Therefore, it is a valid question.  & This is a rhetorical question because it does not genuinely seek new information. Instead, it implies that the answer is already obvious and does not contribute to the discussion in a meaningful way. \\
            \bottomrule
        \end{tabular}
    }
    \caption{Example of an inappropriate follow-up question for the given initial question and answer, accompanied by corresponding reasons below.}
    \label{tab:inappropriate_followupq}
\end{table}

\subsection{Relevance Question Guideline}

\begin{itemize}
    \item \textbf{Strongly Related}: The follow-up question asks for specific definitions, particular reasons, or meanings directly from the original question and answer.
    \item \textbf{Related}: The follow-up question primarily seeks information from the original question or answer but also brings in additional, new information.
    \item \textbf{Slightly Related}: The follow-up question mainly addresses other cases but has some relevance to the original question or answer.
    \item \textbf{Not Related}: The follow-up question does not relate to the original question or answer.
\end{itemize}
Table~\ref{tab:relate_followupq} contains follow-up questions with various levels of relevance.

\begin{table}[H]
    \centering
    \small
    \renewcommand{\arraystretch}{1.3}
    \resizebox{\linewidth}{!}{ 
        \begin{tabular}{ p{0.5\linewidth} | p{0.5\linewidth} }
            \toprule
            \multicolumn{2}{p{\linewidth}}{\textbf{Initial Question:} Why do airplanes leave white trails in the sky?} \\
            \midrule
            \multicolumn{2}{p{1.05\linewidth}}{\textbf{Initial Answer:} Those white trails are called contrails, short for condensation trails. They form when hot exhaust from the airplane's engines mixes with the cold air in the upper atmosphere. The water vapor in the exhaust condenses and freezes into tiny ice crystals, creating the white streaks you see in the sky. The persistence of these trails depends on humidity levels; if the air is dry, the contrail dissipates quickly, but if the air is humid, the contrail can linger for a long time.} \\
            \midrule
            \textbf{Strongly Related Question Example} & \textbf{Related Follow-up Question Example} \\
            \midrule
            Why do some contrails last longer than others? & Do contrails have any impact on the environment?\\
            \midrule
            \textbf{Reason} & \textbf{Reason}\\
            \midrule
            The follow-up question directly builds on the information provided in the answer, specifically regarding the persistence of contrails. Since the answer already mentions humidity as a factor, this question seeks further clarification, making it strongly related.  & This follow-up question extends the topic of contrails by asking about their environmental impact. While the original answer does not discuss environmental effects, the question is still relevant because it builds on the phenomenon explained. Thus, it is considered related. \\
            \midrule
            \textbf{Slightly Related Question Example} & \textbf{Related Follow-up Question Example} \\
            \midrule
            Why do some airplanes make more noise than others? & What causes volcanoes to erupt?\\
            \midrule
            \textbf{Reason} & \textbf{Reason}\\
            \midrule
           The follow-up question is about airplanes, which is the general topic of the original question, but it shifts the focus from contrails to noise. While both topics are related to aviation, the connection between them is weak, making the question only slightly related.  & The follow-up question introduces a completely unrelated topic (volcanoes) that has no connection to airplanes, contrails, or atmospheric conditions. Since it does not build on the original question or answer in any way, it is considered not related. \\
            \bottomrule
        \end{tabular}
    }
    \caption{Examples of follow-up questions' relevance for the given initial question and answer, accompanied by corresponding reasons below.}
    \label{tab:relate_followupq}
\end{table}

\section{Baseline Reproduce}
\label{appendix:baseline_reproduce}
\newcolumntype{g}{>{\columncolor{green!10}}c}
\setlength\tabcolsep{7pt}
\begin{table}[htbp]
\centering
\huge
\newcolumntype{b}{>{\columncolor{blue!10}}c}
\renewcommand{\arraystretch}{1.6}
\resizebox{0.5\textwidth}{!}{

\begin{tabular}{lccccc}

\toprule
\multicolumn{1}{c}{\multirow{2}{*}{Method}} & \multicolumn{4}{c}{Data Quality} &  \\ \cline{2-6} 
\multicolumn{1}{c}{}                       & Nums.       & Cons.    & Avg Attr.      & Synt.    \\ \midrule
IFeval~\cite{zhou2023instruction} & 541  & H & 1.54 & \ding{51} \\
FollowBench~\cite{jiang2023followbench} & 820 & H/S & 3.0 & \ding{51}  \\
CFBench~\cite{zhang2024cfbench} & 1000 & H/S & 4.24 & \ding{55} \\
InFoBench~\cite{qin2024infobench} & 500 & H/S & 4.5 & \ding{55} \\
\our (FineWeb Split) & 6159 & H/S & \textbf{45.9} & \ding{55} \\
\our (Multi-source Split) & 1600 & H/S & \textbf{29.9} & \ding{55} \\
\bottomrule
\end{tabular}%
}
\caption{
  Detailed comparison of relevant works. Ours
represents our dataset construction approach. \textquotesingle Nums.\textquotesingle, \textquotesingle Cons.\textquotesingle, \textquotesingle Avg Attr.\textquotesingle,
and \textquotesingle Synt.\textquotesingle\  denote the number of samples, constraint types, average number of attributes, and whether the data is synthesized.
}


  \label{tab:comparison}
\end{table}

To establish a baseline, we attempted to reproduce the results of \citet{meng-etal-2023-followupqg} using the reported parameters, as the original implementation was unavailable. We use BART-large, consisting of 24 layers, 16 attention heads, and a hidden dimension of 1024. The initial learning rate (5e-5) led to training instability, which we mitigated by adjusting it to 2e-5 while keeping other hyperparameters unchanged (batch size: 8, epochs: 10, optimizer: Adam \cite{kinga2015method}). The training was conducted on an NVIDIA Tesla V100 GPU with 32GB memory, taking approximately 6 hours per run. We set the random seed as 42. After multiple runs, our reproduced model yielded similar overall performance but with some variation—certain metrics improved while others slightly declined (see Table \ref{tab:comparison}). This model served as the baseline for subsequent experiments.


\section{Filtering Ungrammatical Follow-Up Questions: Examples and Pseudocode}
\label{appendix:ungrammatical_questions}

\begin{lstlisting}[language=Python, breaklines, basicstyle=\small\ttfamily]
def is_valid_question(FQ, IQ, IA):
    return (
        # confirm that FQ ends with '?'
        contains_question_mark(FQ) and 
        # use dependency parsing to check for
        # WH-Questions, Yes/No Questions
        # and Rhetorical Questions, 
        is_question_dependency_parsing(FQ) and 
        # checks for meaningless/invalid words
        # e.g. <QUS>, <EQT>
        not contains_invalid_word(FQ) and
        # checks for n-gram substring that is 
        # duplicated in both FQ and (IQ/IA)
        not contains_duplicate_words(IQ, IA, FQ)
    )
\end{lstlisting}

% \begin{minted}[breaklines, fontsize=\small]{python}
% def is_valid_question(FQ, IQ, IA):
%     return (
%         # confirm that FQ ends with '?'
%         contains_question_mark(FQ) and 
%         # use dependency parsing to check for
%         # WH-Questions, Yes/No Questions
%         # and Rhetorical Questions, 
%         is_question_dependency_parsing(FQ) and 
%         # checks for meaningless/invalid words
%         # e.g. <QUS>, <EQT>
%         not contains_invalid_word(FQ) and
%         # checks for n-gram substring that is 
%         # duplicated in both FQ and (IQ/IA)
%         not contains_duplicate_words(IQ, IA, FQ)
%         )
% \end{minted}

\begin{table}[H]
\centering
    \scriptsize
    \renewcommand{\arraystretch}{1.3}
    \resizebox{\linewidth}{!}{ 
        \begin{tabular}{  p{0.25\linewidth} | p{0.75\linewidth} }
        \toprule
        \textbf{Error Type} & \textbf{Example} \\
        \midrule
        Improper delimiter & \textit{> Constant reflux can cause cancer of the sphincter.<QUS> Is this true for people with GERD?} \\
        \midrule
        Non-question statement & \textit{I think that's the key. I'm not sure what you mean by ``3 components''. The rotation is defined by the third component. The fourth component is used to determine the degree of rotation. So, if you have a 3d vector, you can rotate it in any direction. But if you want to rotate it out to a 4d vector you need to use a fourth component.} \\
        \midrule
        Excessive context duplication & \textit{Now try doing it for tomorrow's price chart. When you look at the historical chart of a company's price you'll be able to pick good buying and selling points 100\% of the time. Now try to do it for today's price} Is that impossible? \\
        \bottomrule
    \end{tabular}
    }
    \caption{Follow-up Question Error Types and Examples.}
    \label{tab:error_types}
\end{table}

% \section{Additional Tables}
% \label{app:additional_tables}

% \subsection{Numerical Mapping of Human Evaluation Annotations}
% \label{app:numerical_mapping_of_human_eval}
% To enable quantitative analysis, we applied predefined mappings, as shown in Table~\ref{tab:user_selection_mapping}, to transform categorical annotations from human evaluation tasks into numerical representations. Binary values (0/1) were assigned to categorical judgments, such as the validity of follow-up questions (valid = 1, not valid = 0), while ordinal values (0-3) were used for graded attributes, such as relevance (not related = 0, strongly related = 3).

\section{Model Evaluation - Human Annotation Guideline}
\label{Model Evaluation - Human Annotation Guideline}

Table \ref{tab:task_description_bart} presents the job description and annotation questions for our human annotation task.

\begin{table}[H]
    \scriptsize
    \centering
    \begin{tabular}{ p{\columnwidth} }
        \toprule
        \textbf{Job Description} \\
        \midrule
        In this job, you'll be helping us evaluate the quality of follow-up questions generated by a language model called BART. \\
        For each task, we will provide you with a pair consisting of a question and answer collected from Reddit's ``Explain Like I'm Five'' (ELI5) forum. You will be asked to evaluate the quality of the follow-up question generated by BART. These questions and answers aim to provide layperson-friendly explanations for real-life queries. \\
        Our data may contain noise, such as invalid follow-up questions, errors, lack of reasoning, or follow-up questions unrelated to the original question or answer. Your role is to help us identify these noisy samples. \\
        For each task, you will be shown one triple (question, answer, follow-up question). Carefully review each component and answer the following questions based on your judgment: \\
        \midrule
        \textbf{Q1:} Do you think the follow-up question is a valid question? \\
        \textbf{A.} Yes \quad
        \textbf{B.} No \\
        \midrule
        \textbf{Q2:} How relevant is the follow-up question to the original question and answer? \\
        \textbf{A.} Strongly Related \quad
        \textbf{B.} Related \quad
        \textbf{C.} Slightly Related \quad
        \textbf{D.} Not Related \\
        \midrule
        \textbf{Q3:} Does the follow-up question contain any of the following errors? \\
        \textbf{A.} No Errors \quad
        \textbf{B.} Redundant \quad
        \textbf{C.} Repetitive  \quad
        \textbf{D.} Wrong Semantic Collocation \quad
        \textbf{E.} Other Errors \\
        \midrule
        \textbf{Q4:} Does generating this follow-up question require reasoning? \\
        \textbf{A.} Requires complex amount of reasoning \quad
        \textbf{B.} Requires moderate amount of reasoning \quad
        \textbf{C.} Requires minimal amount of reasoning \quad
        \textbf{D.} Does not require any reasoning \\
        \midrule
        \textbf{Q5:} Does the follow-up question contain new information for the audience? \\
        \textbf{A.} Introduces a lot of new information \quad
        \textbf{B.} Introduces some new information \quad
        \textbf{C.} Introduces little new information \quad
        \textbf{D.} Does not introduce any new information \\
        \bottomrule
    \end{tabular}
    \caption{Task description and evaluation questions for BART follow-up question evaluation.}
    \label{tab:task_description_bart}
\end{table}

\subsection{Error Question Guideline}

\noindent Does the follow-up question contain any of the following errors?

\noindent \textbf{Identify any language issues in the follow-up question.}
\begin{itemize}
    \item \textbf{No Errors} – The follow-up question is appropriate and adds value.
    \item \textbf{Redundant} – The follow-up does not introduce any new information.
    \item \textbf{Repetitive} – The follow-up question closely mirrors the original question.
    \item \textbf{Wrong Semantic Collocation} – The question contains unnatural or incorrect phrasing.
    \item \textbf{Other Errors} – Any issues that do not fit the categories above.
\end{itemize}
Table~\ref{tab:error_followupq} contains examples of follow-up questions with various error status.

\begin{table}[H]
    \centering
    \small
    \renewcommand{\arraystretch}{1.3}
    \resizebox{\linewidth}{!}{ 
        \begin{tabular}{ p{0.5\linewidth} | p{0.5\linewidth} }
            \toprule
            \multicolumn{2}{p{\linewidth}}{\textbf{Initial Question:} How do vaccines work?} \\
            \midrule
            \multicolumn{2}{p{1.05\linewidth}}{\textbf{Initial Answer:} Vaccines work by training your immune system to recognize and fight specific germs. They contain harmless parts of the germ (or something similar) so that your body can learn to defend against it. This way, if you ever encounter the actual germ, your immune system can respond quickly and prevent illness.} \\
            \midrule
            \textbf{No Errors Example} & \textbf{Redundant Example} \\
            \midrule
            How does a vaccine train the immune system? & Are vaccines used to help the immune system recognize germs?\\
            \midrule
            \textbf{Reason} & \textbf{Reason}\\
            \midrule
           The follow-up question is well-formed, relevant, and adds value by diving deeper into a key concept from the original answer. It does not repeat information unnecessarily or contain any language errors.  & The follow-up question is redundant because it merely restates information already provided in the initial answer without adding depth or prompting new discussion. \\
            \midrule
            \textbf{Repetitive Example} & \textbf{Wrong Semantic Collocation Example} \\
            \midrule
            What do vaccines do? & Do vaccines memorize diseases?\\
            \midrule
            \textbf{Reason} & \textbf{Reason}\\
            \midrule
           This follow-up question is nearly identical to the original question, simply reworded. Since it does not introduce new angles or expand on any details, it is considered repetitive.  & The phrase``vaccines memorize diseases'' is unnatural and incorrect in this context. A better way to phrase the question would be: ``Do vaccines help the immune system remember diseases?'' \\
            \bottomrule
        \end{tabular}
    }
    \caption{Examples of follow-up questions' error status for the given initial question and answer, accompanied by corresponding reasonings below.}
    \label{tab:error_followupq}
\end{table}

\subsection{Reasoning Question Guideline}

Evaluate the level of reasoning needed to generate the follow-up question. 
\begin{itemize}
    \item \textbf{Complex reasoning} involves synthesizing multiple ideas or deeply analyzing information.
    \item \textbf{Moderate reasoning} requires interpreting the given content or slightly extending the discussion.
    \item \textbf{Minimal reasoning} involves simple comprehension or directly rephrasing information.
    \item \textbf{No reasoning} applies to questions that are direct repetitions or restatements without any thought process.
\end{itemize}

Table~\ref{tab:reason_followupq} contains examples of follow-up questions with various reasoning complexity.

\begin{table}[H]
    \centering
    \small
    \renewcommand{\arraystretch}{1.3}
    \resizebox{\linewidth}{!}{ 
        \begin{tabular}{ p{0.5\linewidth} | p{0.5\linewidth} }
            \toprule
            \multicolumn{2}{p{\linewidth}}{\textbf{Initial Question:} How does sleep affect brain function?} \\
            \midrule
            \multicolumn{2}{p{1.05\linewidth}}{\textbf{Initial Answer:} Sleep is essential for brain function because it helps with memory consolidation, cognitive processing, and emotional regulation. During sleep, the brain strengthens neural connections, removes toxins, and allows different areas to reset for the next day.} \\
            \midrule
            \textbf{Complex Amount of Reasoning Example} & \textbf{Moderate Amount of Reasoning Example} \\
            \midrule
            What are the long-term cognitive effects of chronic sleep deprivation compared to occasional sleep loss? & How does sleep remove toxins from the brain? \\
            \midrule
            \textbf{Reason} & \textbf{Reason}\\
            \midrule
           This follow-up question requires complex reasoning because it involves comparing two different scenarios (chronic vs. occasional sleep deprivation) and analyzing their distinct long-term effects on cognition, requiring deeper thought and synthesis of information.  & This follow-up question requires moderate reasoning because it builds on a specific detail from the original answer (toxin removal) and asks for an explanation of the biological process involved. \\
            \midrule
            \textbf{Minimal Amount of Reasoning Example} & \textbf{Does Not Require Any Reasoning Example} \\
            \midrule
            What are the benefits of sleep for memory? &  Does sleep help with memory?\\
            \midrule
            \textbf{Reason} & \textbf{Reason}\\
            \midrule
           This follow-up question requires minimal reasoning as it only asks for elaboration on a topic already stated in the original answer (memory consolidation), without introducing any new angle.  & This follow-up question does not require any reasoning since it directly repeats a fact already stated in the original answer, making it redundant. \\
            \bottomrule
        \end{tabular}
    }
    \caption{Examples of follow-up questions' reasoning complexity for the given initial question and answer, accompanied by corresponding reasons below.}
    \label{tab:reason_followupq}
\end{table}

\subsection{Informativeness Question Guideline}

Evaluate whether the follow-up question enriches the topic by providing or eliciting new information.

\begin{itemize}
    \item \textbf{A Lot of New Information} indicates a significant amount of new knowledge is introduced.
    \item \textbf{Some New Information} suggests moderate enrichment.
    \item \textbf{Little New Information} implies minimal addition.
    \item \textbf{No New Information} means no new information is provided to the audience.
\end{itemize}

Table~\ref{tab:inform_followupq} contains examples of follow-up questions with various informativeness levels.

\begin{table}[H]
    \centering
    \small
    \renewcommand{\arraystretch}{1.3}
    \resizebox{\linewidth}{!}{ 
        \begin{tabular}{ p{0.5\linewidth} | p{0.5\linewidth} }
            \toprule
            \multicolumn{2}{p{\linewidth}}{\textbf{Initial Question:} How do vaccines work?} \\
            \midrule
            \multicolumn{2}{p{1.05\linewidth}}{\textbf{Initial Answer:} Vaccines train the immune system to recognize and fight specific germs by introducing harmless parts of the germ or something similar. This prepares the body to respond quickly if exposed to the actual germ in the future.} \\
            \midrule
            \textbf{A Lot of New Information Example} & \textbf{Some New Information Example} \\
            \midrule
            What are the differences between traditional vaccines and mRNA vaccines? & How long does it take for a vaccine to provide immunity?\\
            \midrule
            \textbf{Reason} & \textbf{Reason}\\
            \midrule
           This follow-up question introduces a significantly new dimension by asking about different types of vaccines, which were not mentioned in the original answer, expanding the discussion substantially.  & The follow-up question adds moderately new information by focusing on the timeline of immunity development, a relevant but additional detail not covered in the initial answer. \\
            \midrule
            \textbf{Little New Information Example} & \textbf{Does Not Introduce Any New Information Example} \\
            \midrule
            Do vaccines help prevent disease outbreaks? & Do vaccines help the immune system recognize germs? \\
            \midrule
            \textbf{Reason} & \textbf{Reason}\\
            \midrule
           The follow-up question slightly expands the discussion by addressing disease outbreaks, but it is already implied in the original answer, as vaccines train the immune system to fight germs. & This follow-up question does not add any new information as it directly restates a key point from the original answer in slightly different words. \\
            \bottomrule
        \end{tabular}
    }
    \caption{Examples of follow-up questions' informativeness for the given initial question and answer, accompanied by corresponding reasons below.}
    \label{tab:inform_followupq}
\end{table}

\section{Additional Examples}
\label{app:additional_examples}
\begin{table*}[t]
    \centering
    \small
    \renewcommand{\arraystretch}{1.2}  % Ensure row heights are consistent
    \begin{tabular}{m{0.05\textwidth} m{0.95\textwidth}}
        \toprule
        \textbf{IQ} & What causes occupational domination? \\
        \midrule
        \textbf{IA} & \textcolor{blue}{\textit{In IT it is cultural. Women in the U.S. drop out of IT classes because they do not like the male-dominated culture.}} Whereas India, Vietnam, and other countries where we outsource our IT have a much higher percentage of female programmers. \textcolor{blue}{\textit{It is not something about women.}} It is something about western IT classes and tech culture. \\
        \midrule
        \textbf{CA} & Occupation domination occurs due to a combination of historical, social, economic, and geographic factors. \textit{Historically}, certain groups have had greater access to education, resources, and opportunities, leading them to be more prevalent in specific jobs or industries. This is often reinforced by network effects, where individuals within a profession prefer to work with or hire people who share similar backgrounds, cultural norms, or experiences, thus perpetuating the concentration of certain demographics in those fields. \textit{Legal and institutional} barriers can also play a significant role. Discriminatory policies or practices can systematically exclude certain groups from specific professions, allowing other groups, who do not face such barriers, to dominate those fields. \textit{Geographic and environmental} factors further influence occupation domination. The presence of certain natural resources or particular climate conditions in an area can dictate the types of industries that develop there. Local populations often specialize in these industries, passing down specialized skills and knowledge through generations, which solidifies their dominance in those fields over time. \\
        \midrule
        \shortstack{\textbf{FQ}\\by\\\textit{ORG}}  &  
            -- But why do women drop out of IT classes?

            -- \textcolor{red}{\textit{In IT it is cultural. Women in the U.S. drop out of IT classes because they do not like the male-dominated culture.}} This is a myth. \textcolor{red}{\textit{It is not something about women.}} It's a myth that women drop out because of the male-dominant culture.
            
            -- \textcolor{red}{\textit{So it’s cultural?}}
            
            -- What about women who want to be in tech but are not in the workforce?
            
            -- Is there a cultural component of IT that makes it more difficult for women to break into it?
 \\
        \midrule
        \shortstack{\textbf{FQ}\\by\\\textit{AUG}} & 
        -- What are some examples of how outsourcing has impacted the gender imbalance in IT?
        
        -- How do cultural expectations influence the recruitment of female programmers?
        
        -- Can you explain the concept of `cultural fit' in the context of occupational domination?
\\
        \midrule
        \shortstack{\textbf{FQ}\\by\\\textit{FULL}} & 
        -- How do traditional gender roles in professions like medicine and engineering contribute to occupation domination?
        
        -- Why might women feel more comfortable pursuing professions outside of traditional professions?
        
        -- Can you explain the concept of `perceived value' in the context of occupation domination?
\\
        \bottomrule
    \end{tabular}
    \caption{Example of follow-up question generated by three model variants, with comprehensive answers (ID 3168).}
    \label{tab:example_question_short}
\end{table*}
See Tables~\ref{tab:example_question_short}

\section{Interface Examples}
\label{app:interface}
See Figures~\ref{fig:instruction_interface} and \ref{fig:annotation_interface}
\begin{figure*}
    \centering
    \includegraphics[width=0.8\textwidth, keepaspectratio]{figures/task_instruction.jpg}
    \caption{Human Evaluation Interface - Task Instructions and Examples.}
    \label{fig:instruction_interface}
\end{figure*}

\begin{figure*}
    \centering
    \includegraphics[width=0.8\textwidth, keepaspectratio]{figures/task_annotation.jpg}
    \caption{Human Evaluation Interface - Annotation.}
    \label{fig:annotation_interface}
\end{figure*}