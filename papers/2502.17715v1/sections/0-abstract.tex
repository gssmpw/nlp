Effective conversational systems are expected to dynamically generate contextual follow-up questions to elicit new information while maintaining the conversation flow. While humans excel at asking diverse and informative questions by intuitively assessing both obtained and missing information, existing models often fall short of human performance on this task. To mitigate this, we propose a method that generates diverse and informative questions based on targeting unanswered information using a hypothetical LLM-generated ``comprehensive answer''. Our method is applied to augment an existing follow-up questions dataset. The experimental results demonstrate that language models fine-tuned on the augmented datasets produce follow-up questions of significantly higher quality and diversity. This promising approach could be effectively adopted to future work to augment information-seeking dialogues for reducing ambiguities and improving the accuracy of LLM answers\footnote{Code available at \url{https://github.com/zheliu92/nlp_followupqg_public/}}. 
