\documentclass[11pt,a4paper]{article}
\usepackage{times,latexsym}
\usepackage{url}
\usepackage[T1]{fontenc}
\usepackage{graphicx}
\usepackage{multicol}
\usepackage{subcaption}
\usepackage{xkcdcolors}

\usepackage{tabularray}
\usepackage{caption}
\usepackage{microtype}
\usepackage{fontawesome5}
\usepackage{booktabs}
% \usepackage[table]{xcolor}
\usepackage{colortbl}

\usepackage{soul}
\usepackage{makecell}
\usepackage{enumitem}
\usepackage{amssymb}
\usepackage{tabularx}

% \linepenalty=2500

%% Package options:
%% Short version: "hyperref" and "submission" are the defaults.
%% More verbose version:
%% Most compact command to produce a submission version with hyperref enabled
%%    \usepackage[]{tacl2021v1}
%% Most compact command to produce a "camera-ready" version
%%    \usepackage[acceptedWithA]{tacl2021v1}
%% Most compact command to produce a double-spaced copy-editor's version
%%    \usepackage[acceptedWithA,copyedit]{tacl2021v1}
%
%% If you need to disable hyperref in any of the above settings (see Section
%% "LaTeX files") in the TACL instructions), add ",nohyperref" in the square
%% brackets. (The comma is a delimiter in case there are multiple options specified.)

\usepackage[acceptedWithA]{tacl2021v1}
% \setlength\titlebox{10cm} % <- for Option 2 below

%%%% Material in this block is specific to generating TACL instructions
\usepackage{xspace,mfirstuc,tabulary}
\newcommand{\dateOfLastUpdate}{Dec. 15, 2021}
\newcommand{\styleFileVersion}{tacl2021v1}

\newcommand{\ex}[1]{{\sf #1}}

\newif\iftaclinstructions
\taclinstructionsfalse % AUTHORS: do NOT set this to true
\iftaclinstructions
\renewcommand{\confidential}{}
\renewcommand{\anonsubtext}{(No author info supplied here, for consistency with
TACL-submission anonymization requirements)}
\newcommand{\instr}
\fi

%
\iftaclpubformat % this "if" is set by the choice of options
\newcommand{\taclpaper}{final version\xspace}
\newcommand{\taclpapers}{final versions\xspace}
\newcommand{\Taclpaper}{Final version\xspace}
\newcommand{\Taclpapers}{Final versions\xspace}
\newcommand{\TaclPapers}{Final Versions\xspace}
\else
\newcommand{\taclpaper}{submission\xspace}
\newcommand{\taclpapers}{{\taclpaper}s\xspace}
\newcommand{\Taclpaper}{Submission\xspace}
\newcommand{\Taclpapers}{{\Taclpaper}s\xspace}
\newcommand{\TaclPapers}{Submissions\xspace}
\fi

%%%% End TACL-instructions-specific macro block

\title{Natural Language Generation from Visual Sequences: \\
Challenges and Future Directions}

% Author information does not appear in the pdf unless the "acceptedWithA" option is given

% The author block may be formatted in one of two ways:

% Option 1. Author’s address is underneath each name, centered.

\author{Aditya K Surikuchi, Raquel Fern{\'a}ndez, Sandro Pezzelle\\
        Institute for Logic, Language and Computation\\
        University of Amsterdam\\
        \texttt{\{a.k.surikuchi|raquel.fernandez|s.pezzelle\}@uva.nl}}

% \author{
%   Template Author1\Thanks{The {\em actual} contributors to this instruction
%     document and corresponding template file are given in Section
%   \ref{sec:contributors}.}
%   Aditya K Surikuchi
%   \\
%   Template Affiliation1/Address Line 1
%   \\
%   Template Affiliation1/Address Line 2
%   \\
%   Template Affiliation1/Address Line 2
%   \\
%   \texttt{template.email1example.com}
%   \And
%   Template Author2
%   \\
%   Template Affiliation2/Address Line 1
%   \\
%   Template Affiliation2/Address Line 2
%   \\
%   Template Affiliation2/Address Line 2
%   \\
%   \texttt{template.email2@example.com}
% }

% % Option 2.  Author’s address is linked with superscript
% % characters to its name, author names are grouped, centered.

% \author{
%   Template Author1\Thanks{The {\em actual} contributors to this instruction
%     document and corresponding template file are given in Section
%     \ref{sec:contributors}.}$^\diamond$
%   \and
%   Template Author2$^\dagger$
%   \\
%   \ \\
%   $^\diamond$Template Affiliation1/Address Line 1
%   \\
%   Template Affiliation1/Address Line 2
%   \\
%   Template Affiliation1/Address Line 2
%   \\
%   \texttt{template.email1example.com}
%   \\
%   \ \\
%   \\
%   $^\dagger$Template Affiliation2/Address Line 1
%   \\
%   Template Affiliation2/Address Line 2
%   \\
%   Template Affiliation2/Address Line 2
%   \\
%   \texttt{template.email2@example.com}
% }

\date{}

\begin{document}
\maketitle
\begin{abstract}
The ability to use natural language to talk about visual content is at the core of human intelligence and a crucial feature of any artificial intelligence system. Various studies have focused on generating text for single images. In contrast, comparatively little attention has been paid to exhaustively analyzing and advancing work on multiple-image vision-to-text settings. In this position paper, we claim that any task dealing with temporally ordered sequences of multiple images or frames is an instance of a broader, more general problem involving the understanding of intricate relationships between the visual content and the corresponding text. We comprehensively analyze five tasks that are instances of this problem and argue that they pose a common set of challenges and share similarities in terms of modeling and evaluation approaches. Based on the insights from these various aspects and stages of multi-image-to-text generation, we highlight several open questions and suggest future research directions. We believe that these directions can advance the understanding of complex phenomena in this domain and the development of better models.
\end{abstract}

%!TEX root = gcn.tex
\section{Introduction}
Graphs, representing structural data and topology, are widely used across various domains, such as social networks and merchandising transactions.
Graph convolutional networks (GCN)~\cite{iclr/KipfW17} have significantly enhanced model training on these interconnected nodes.
However, these graphs often contain sensitive information that should not be leaked to untrusted parties.
For example, companies may analyze sensitive demographic and behavioral data about users for applications ranging from targeted advertising to personalized medicine.
Given the data-centric nature and analytical power of GCN training, addressing these privacy concerns is imperative.

Secure multi-party computation (MPC)~\cite{crypto/ChaumDG87,crypto/ChenC06,eurocrypt/CiampiRSW22} is a critical tool for privacy-preserving machine learning, enabling mutually distrustful parties to collaboratively train models with privacy protection over inputs and (intermediate) computations.
While research advances (\eg,~\cite{ccs/RatheeRKCGRS20,uss/NgC21,sp21/TanKTW,uss/WatsonWP22,icml/Keller022,ccs/ABY318,folkerts2023redsec}) support secure training on convolutional neural networks (CNNs) efficiently, private GCN training with MPC over graphs remains challenging.

Graph convolutional layers in GCNs involve multiplications with a (normalized) adjacency matrix containing $\numedge$ non-zero values in a $\numnode \times \numnode$ matrix for a graph with $\numnode$ nodes and $\numedge$ edges.
The graphs are typically sparse but large.
One could use the standard Beaver-triple-based protocol to securely perform these sparse matrix multiplications by treating graph convolution as ordinary dense matrix multiplication.
However, this approach incurs $O(\numnode^2)$ communication and memory costs due to computations on irrelevant nodes.
%
Integrating existing cryptographic advances, the initial effort of SecGNN~\cite{tsc/WangZJ23,nips/RanXLWQW23} requires heavy communication or computational overhead.
Recently, CoGNN~\cite{ccs/ZouLSLXX24} optimizes the overhead in terms of  horizontal data partitioning, proposing a semi-honest secure framework.
Research for secure GCN over vertical data  remains nascent.

Current MPC studies, for GCN or not, have primarily targeted settings where participants own different data samples, \ie, horizontally partitioned data~\cite{ccs/ZouLSLXX24}.
MPC specialized for scenarios where parties hold different types of features~\cite{tkde/LiuKZPHYOZY24,icml/CastigliaZ0KBP23,nips/Wang0ZLWL23} is rare.
This paper studies $2$-party secure GCN training for these vertical partition cases, where one party holds private graph topology (\eg, edges) while the other owns private node features.
For instance, LinkedIn holds private social relationships between users, while banks own users' private bank statements.
Such real-world graph structures underpin the relevance of our focus.
To our knowledge, no prior work tackles secure GCN training in this context, which is crucial for cross-silo collaboration.


To realize secure GCN over vertically split data, we tailor MPC protocols for sparse graph convolution, which fundamentally involves sparse (adjacency) matrix multiplication.
Recent studies have begun exploring MPC protocols for sparse matrix multiplication (SMM).
ROOM~\cite{ccs/SchoppmannG0P19}, a seminal work on SMM, requires foreknowledge of sparsity types: whether the input matrices are row-sparse or column-sparse.
Unfortunately, GCN typically trains on graphs with arbitrary sparsity, where nodes have varying degrees and no specific sparsity constraints.
Moreover, the adjacency matrix in GCN often contains a self-loop operation represented by adding the identity matrix, which is neither row- nor column-sparse.
Araki~\etal~\cite{ccs/Araki0OPRT21} avoid this limitation in their scalable, secure graph analysis work, yet it does not cover vertical partition.

% and related primitives
To bridge this gap, we propose a secure sparse matrix multiplication protocol, \osmm, achieving \emph{accurate, efficient, and secure GCN training over vertical data} for the first time.

\subsection{New Techniques for Sparse Matrices}
The cost of evaluating a GCN layer is dominated by SMM in the form of $\adjmat\feamat$, where $\adjmat$ is a sparse adjacency matrix of a (directed) graph $\graph$ and $\feamat$ is a dense matrix of node features.
For unrelated nodes, which often constitute a substantial portion, the element-wise products $0\cdot x$ are always zero.
Our efficient MPC design 
avoids unnecessary secure computation over unrelated nodes by focusing on computing non-zero results while concealing the sparse topology.
We achieve this~by:
1) decomposing the sparse matrix $\adjmat$ into a product of matrices (\S\ref{sec::sgc}), including permutation and binary diagonal matrices, that can \emph{faithfully} represent the original graph topology;
2) devising specialized protocols (\S\ref{sec::smm_protocol}) for efficiently multiplying the structured matrices while hiding sparsity topology.


 
\subsubsection{Sparse Matrix Decomposition}
We decompose adjacency matrix $\adjmat$ of $\graph$ into two bipartite graphs: one represented by sparse matrix $\adjout$, linking the out-degree nodes to edges, the other 
by sparse matrix $\adjin$,
linking edges to in-degree nodes.

%\ie, we decompose $\adjmat$ into $\adjout \adjin$, where $\adjout$ and $\adjin$ are sparse matrices representing these connections.
%linking out-degree nodes to edges and edges to in-degree nodes of $\graph$, respectively.

We then permute the columns of $\adjout$ and the rows of $\adjin$ so that the permuted matrices $\adjout'$ and $\adjin'$ have non-zero positions with \emph{monotonically non-decreasing} row and column indices.
A permutation $\sigma$ is used to preserve the edge topology, leading to an initial decomposition of $\adjmat = \adjout'\sigma \adjin'$.
This is further refined into a sequence of \emph{linear transformations}, 
which can be efficiently computed by our MPC protocols for 
\emph{oblivious permutation}
%($\Pi_{\ssp}$) 
and \emph{oblivious selection-multiplication}.
% ($\Pi_\SM$)
\iffalse
Our approach leverages bipartite graph representation and the monotonicity of non-zero positions to decompose a general sparse matrix into linear transformations, enhancing the efficiency of our MPC protocols.
\fi
Our decomposition approach is not limited to GCNs but also general~SMM 
by 
%simply 
treating them 
as adjacency matrices.
%of a graph.
%Since any sparse matrix can be viewed 

%allowing the same technique to be applied.

 
\subsubsection{New Protocols for Linear Transformations}
\emph{Oblivious permutation} (OP) is a two-party protocol taking a private permutation $\sigma$ and a private vector $\xvec$ from the two parties, respectively, and generating a secret share $\l\sigma \xvec\r$ between them.
Our OP protocol employs correlated randomnesses generated in an input-independent offline phase to mask $\sigma$ and $\xvec$ for secure computations on intermediate results, requiring only $1$ round in the online phase (\cf, $\ge 2$ in previous works~\cite{ccs/AsharovHIKNPTT22, ccs/Araki0OPRT21}).

Another crucial two-party protocol in our work is \emph{oblivious selection-multiplication} (OSM).
It takes a private bit~$s$ from a party and secret share $\l x\r$ of an arithmetic number~$x$ owned by the two parties as input and generates secret share $\l sx\r$.
%between them.
%Like our OP protocol, o
Our $1$-round OSM protocol also uses pre-computed randomnesses to mask $s$ and $x$.
%for secure computations.
Compared to the Beaver-triple-based~\cite{crypto/Beaver91a} and oblivious-transfer (OT)-based approaches~\cite{pkc/Tzeng02}, our protocol saves ${\sim}50\%$ of online communication while having the same offline communication and round complexities.

By decomposing the sparse matrix into linear transformations and applying our specialized protocols, our \osmm protocol
%($\prosmm$) 
reduces the complexity of evaluating $\numnode \times \numnode$ sparse matrices with $\numedge$ non-zero values from $O(\numnode^2)$ to $O(\numedge)$.

%(\S\ref{sec::secgcn})
\subsection{\cgnn: Secure GCN made Efficient}
Supported by our new sparsity techniques, we build \cgnn, 
a two-party computation (2PC) framework for GCN inference and training over vertical
%ly split
data.
Our contributions include:

1) We are the first to explore sparsity over vertically split, secret-shared data in MPC, enabling decompositions of sparse matrices with arbitrary sparsity and isolating computations that can be performed in plaintext without sacrificing privacy.

2) We propose two efficient $2$PC primitives for OP and OSM, both optimally single-round.
Combined with our sparse matrix decomposition approach, our \osmm protocol ($\prosmm$) achieves constant-round communication costs of $O(\numedge)$, reducing memory requirements and avoiding out-of-memory errors for large matrices.
In practice, it saves $99\%+$ communication
%(Table~\ref{table:comm_smm}) 
and reduces ${\sim}72\%$ memory usage over large $(5000\times5000)$ matrices compared with using Beaver triples.
%(Table~\ref{table:mem_smm_sparse}) ${\sim}16\%$-

3) We build an end-to-end secure GCN framework for inference and training over vertically split data, maintaining accuracy on par with plaintext computations.
We will open-source our evaluation code for research and deployment.

To evaluate the performance of $\cgnn$, we conducted extensive experiments over three standard graph datasets (Cora~\cite{aim/SenNBGGE08}, Citeseer~\cite{dl/GilesBL98}, and Pubmed~\cite{ijcnlp/DernoncourtL17}),
reporting communication, memory usage, accuracy, and running time under varying network conditions, along with an ablation study with or without \osmm.
Below, we highlight our key achievements.

\textit{Communication (\S\ref{sec::comm_compare_gcn}).}
$\cgnn$ saves communication by $50$-$80\%$.
(\cf,~CoGNN~\cite{ccs/KotiKPG24}, OblivGNN~\cite{uss/XuL0AYY24}).

\textit{Memory usage (\S\ref{sec::smmmemory}).}
\cgnn alleviates out-of-memory problems of using %the standard 
Beaver-triples~\cite{crypto/Beaver91a} for large datasets.

\textit{Accuracy (\S\ref{sec::acc_compare_gcn}).}
$\cgnn$ achieves inference and training accuracy comparable to plaintext counterparts.
%training accuracy $\{76\%$, $65.1\%$, $75.2\%\}$ comparable to $\{75.7\%$, $65.4\%$, $74.5\%\}$ in plaintext.

{\textit{Computational efficiency (\S\ref{sec::time_net}).}} 
%If the network is worse in bandwidth and better in latency, $\cgnn$ shows more benefits.
$\cgnn$ is faster by $6$-$45\%$ in inference and $28$-$95\%$ in training across various networks and excels in narrow-bandwidth and low-latency~ones.

{\textit{Impact of \osmm (\S\ref{sec:ablation}).}}
Our \osmm protocol shows a $10$-$42\times$ speed-up for $5000\times 5000$ matrices and saves $10$-2$1\%$ memory for ``small'' datasets and up to $90\%$+ for larger ones.


\section{Dimensions of Variation}
\label{sec:tasks_and_analysis}

While our focus is on the multi-image-to-text-problem, different tasks within this problem space may have different characteristics. Two relevant dimensions are the complexity of the visual input and of the textual output. For example, some of these tasks require models to generate succinct answers and descriptions, and others require generation of long-form textual narratives intended to also complement the visual information. These dimensions of variation across tasks typically tend to be crucial factors in making design choices with regard to developing model architectures and learning procedures. In terms of the visual input, depending on the objective of the task, images within the input sequence of each data sample could be comparable to each other or vary drastically to the point of being completely heterogeneous. For instance, in the \color{xkcdVividBlue}CC \color{black}task, where the goal is to localize and describe changes between a pair of images obtained from real-time surveillance cameras or large-scale remote-sensing snapshots, we hypothesize that the similarity between input images would be generally high. On the other hand, in tasks such as \textit{Visual Storytelling} (\color{xkcdVividBlue}VST\color{black}) \cite{vist}, in which the input sequences typically depict an overarching narrative, we hypothesize low similarity between consecutive images within each data sample (<visual sequence, text> pair). Regarding textual output, we similarly posit that the consistency of consecutive sentences within each data sample could be high or low depending on the corresponding task objective.

To preliminarily test our intuitions, we quantitatively analyze a few datasets corresponding to each of these tasks. Specifically, for each data sample, we compute \textit{visual similarity} scores between CLIP \cite{clip} visual encoder embeddings of consecutive images in the input sequence and report the average score. In the same manner, we compute \textit{textual consistency} scores between CLIP text encoder embeddings of consecutive sentences in the corresponding ground-truth text of each data sample. For this study, we randomly select 100 instances per task from five datasets---\color{xkcdOrange}Spot-the-diff \color{black} \cite{cc_spot_the_diff} for \color{xkcdVividBlue}CC\color{black}, \color{xkcdOrange}VIST \color{black} \cite{vist} for \color{xkcdVividBlue}VST\color{black}, \color{xkcdOrange}Charades \color{black} \cite{vc_charades} for \color{xkcdVividBlue}VC\color{black}, \color{xkcdOrange}MSVD-QA \color{black} \cite{msvd_qa} for \color{xkcdVividBlue}VIDQA\color{black}, and \color{xkcdOrange}MAD-v1 \color{black} \cite{madv1} for \emph{Movie Auto Audio Description (Auto AD)} \cite{maad1} (\color{xkcdVividBlue}MAAD\color{black}).\footnote{Further details are provided in Appendix~\ref{appendix:a}.} Figure~\ref{fig:data_analysis} shows the distributions of similarity scores obtained for each of the tasks along the visual and textual dimensions. In terms of \textit{visual similarity}, we observe that \color{xkcdVividBlue}CC \color{black} and \color{xkcdVividBlue}VST \color{black} indeed obtain maximum and minimum scores respectively, with other tasks ranging in between, confirming our intuitions. In terms of \textit{textual consistency}, the differentiation is less evident. We observe that for the \color{xkcdVividBlue}MAAD \color{black} and \color{xkcdVividBlue}VC \color{black} tasks, consecutive sentences in the ground-truth text are relatively less consistent with each other. Using the average similarity scores across all data samples, we categorize the five tasks by placing them at different positions in the shared space between \emph{textual consistency} and \emph{visual similarity} (see Figure~\ref{fig:shared_space}). Besides the two axes considered for this analysis, we note that tasks could also be compared along various other dimensions. We posit that this kind of analysis would serve as a meaningful guide for making modeling and evaluation decisions both for current and for novel future tasks in the multi-image-to-text landscape.

\begin{figure}[hbtp]
  \centering
  \includegraphics[width=\columnwidth,keepaspectratio]{images/review_tasks.pdf}
  \caption{Tasks positioned in the visual-textual shared space using the similarity scores obtained for corresponding datasets. \underline{Interpretation:} a task in the top right corner of the plot is both inconsistent at the textual level and dissimilar at the visual level.}
  \label{fig:shared_space}
\end{figure}

\section{Common Challenges}
\label{sec:challenges}

Multi-image-to-text tasks present a unique set of challenges. Some of these challenges are task-specific, but most of them are common across the five tasks. In this section, we describe these challenges and discuss how they instantiate for each of the tasks.

\paragraph{Entity tracking.}
Identifying and tracking entities is an important requisite for accurately interpreting actions and relationships between them. This has been underlined by several works in both language understanding \cite{et_lm1,et_lm2} and computer vision domains \cite{et_vm1,et_vm2}. In multi-image V2L tasks, the availability of input signals from two modalities makes the aspect of disambiguating entities more challenging. For instance, in the \color{xkcdVividBlue}MAAD \color{black} or \color{xkcdVividBlue}VST \color{black} tasks where the visual input is heterogeneous, entities in the input image sequences/video clips tend to `disappear' in some of the images/frames at the intermediate temporal positions, while still being actively referenced in the textual input at the corresponding positions. \citet{groovist} have denoted such cases as being \textit{temporally misaligned}. To track entities accurately under such temporal misalignment, it is necessary to not only learn causal relations between people and objects in each of the modalities at corresponding positions, but also to obtain a cross-modal cross-temporal representation of all the relationships relevant to the scene/narrative. For the \color{xkcdVividBlue}CC\color{black}, \color{xkcdVividBlue}VC\color{black}, or \color{xkcdVividBlue}VIDQA \color{black} tasks, in which the input images/video frames are typically similar to each other, it is crucial to differentiate meaningful semantic entities and their changes from various distractions. While in \color{xkcdVividBlue}CC\color{black}, viewpoint changes or illuminations are considered as distractors and discarded, in the \color{xkcdVividBlue}VIDQA \color{black} task, entities relevant to answering the question need to be differentiated from others for accurate tracking. Effective tracking of entities would therefore require accounting for changes in appearance (including disappearance), capturing interactions, and correctly identifying occlusions.

\paragraph{Visual grounding.}
Humans acquire language understanding through perception and interaction with the environment \cite{vg0,lang_acquisition} and consequently this enables them to seamlessly ground language in visual data. Over the years, a great deal of work has been proposed to adapt the architecture and learning process of vision-language models for acquiring visual grounding \cite{vg1}. However, there are still significant challenges for achieving human-levels of grounding using computational models and this becomes more apparent when looked at from the perspective of multi-image-to-text tasks. For instance, in the \color{xkcdVividBlue}VST \color{black} or \color{xkcdVividBlue}MAAD \color{black} tasks, since language is typically `inconsistent' (see Figure~\ref{fig:shared_space}), it is also inadvertently under-specified semantically \cite{semantic_underspecification} (e.g., `\textit{The boy on the bridge was waving to the tourists near the waterfall. A photographer over \color{red}\textbf{there} \color{black} clicks...'}). Stories also tend to contain many abstract adverbs such as `\textit{often}' or `\textit{today}' and it has been shown that vision-language models struggle to disambiguate such under-specified text and accurately map phrases to regions in the image sequences/videos. Moreover, the amount of language informativeness---degree of information required for identifying the correct object \cite{vg2}---could be inadequate in tasks such as \color{xkcdVividBlue}VIDQA\color{black}, particularly with the presence of confounding entities in various frames (e.g., input video of a football match and a question: \textit{`What is the color of the card the referee is holding?'}). Also, when grounding objects, it has been shown that models often struggle to reliably capture spatial relationships \cite{vg3}. To summarize, beyond being merely descriptive, language in some multi-image-to-text tasks could also be complementary to the data in images/videos, making visual grounding challenging without access to relevant additional external knowledge.

\paragraph{Knowledge integration.}
For some multi-image-to-text tasks, models would be required to utilize additional information beyond what is available in the input data. In \color{xkcdVividBlue}VC \color{black} or \color{xkcdVividBlue}VIDQA \color{black} tasks pertaining to certain domains such as \emph{news}, the input video might not contain all the aspects needed to correctly describe its contents or answer questions about it \cite{vc_kg_task,videoqa_kg_task}. To address this, various approaches often rely on using large pre-trained general purpose models or external knowledge bases such as ConceptNet \cite{concept_net} to retrieve both factual and commonsense information. This method is commonly referred as retrieval-augmented generation (RAG) \cite{rag}. Besides being sources for missing information, external knowledge bases are also leveraged for enriching the generated text with social or cultural contexts. For instance, in \color{xkcdVividBlue}VST\color{black}, some approaches use recognized visual objects in the input to retrieve concepts from external knowledge graphs for generating more engaging and figurative stories (e.g., concept of `\textit{graduation ceremony}' following the detection of an `\textit{academic gown}' object).

From knowledge selection/retrieval stage to accurately representing and utilizing it during text generation, the process of integrating external knowledge has various challenges. Robust retrieval systems are required which can holistically extract the essence of image sequences/video frames, including the various entities and their interrelationships. Typically, the retrieved knowledge is concatenated with input representations which are then used for generating text either through fine-tuning \cite{kg_finetuning} or by prompting general-purpose VLMs \cite{kg_promptbased}. However, this approach might lead to models either over or under utilizing the retrieved knowledge potentially leading to incoherent text \cite{rag_issues}. To address this, we argue that fusion mechanisms which can effectively balance information from representations of both input data and the retrieved knowledge need to be developed. Furthermore, retrieving relevant knowledge from increasingly large databases could be computationally expensive, especially in the multi-image scenario. Ways to optimize retrieval components for improving efficiency is an active research area.

\paragraph{Textual coherence.}
Coherence is the property of text that refers to the ordering of its constituents (words/sentences) and the way in which they relate to each other \cite{coh1}. Coherent text should have a consistent logical structure in which the events, interactions, and relationships between various elements are ordered in a meaningful way. It is an important aspect of discourse and has been studied extensively in neural language generation \cite{coh2}. For several multi-image-to-text tasks, particularly the ones in which the output tends to be less `consistent' along the text axis in Figure~\ref{fig:shared_space}, it is challenging to ensure that multiple sentences in the generated output are locally coherent. In \color{xkcdVividBlue}MAAD \color{black}and \color{xkcdVividBlue}VST \color{black} tasks, where there are multiple characters and various interactions developing across the sequence of images/video frames, it is often difficult for models to balance between selecting the visual information and representing it in a cohesive/connected language \cite{coh3}. This challenge is more apparent in the \color{xkcdVividBlue}VST \color{black} task, in which models are expected to keep track of multiple things such as emotional arcs or motivations of the characters and the overarching narrative \cite{nytws}. There is increasing work in unimodal text-only storytelling suggesting how using concepts of narratology \cite{narrativity1,narrativity2} such as \citet{narrativity_genette}'s triangle can potentially aid models in generating stories with engaging and coherent structures. However, it still unclear how these theories can be applied to multimodal scenarios where the generated text needs to be consistent with image sequences/videos.

\paragraph{Theory of mind.}
Theory of Mind (ToM), which is considered the basis of human social cognition \cite{tom1}, is described as the ability to understand and make inferences about the mental states (e.g., beliefs, intentions, and desires) of other people or living beings. In the context of multi-image V2L tasks, ToM refers to the ability of models to go beyond merely recognizing objects/actions and to reason about the mental states of entities depicted in the image sequences/videos. Although most of the tasks we consider in this work, besides \color{xkcdVividBlue}VIDQA \color{black} \cite{videoqa_bdiqa}, do not explicitly require ToM abilities, the skill is still relevant for all the tasks and closely connected to the other challenges and abilities discussed so far. For instance, in tasks such as \color{xkcdVividBlue}VST\color{black}, to causally connect heterogeneous images in the input sequence, models need to be equipped with different reasoning abilities pertaining to emotions and social perceptions/intentions. This enables generation of stories that reflect actions and mental states of the characters beyond literal interpretation of the visual data.

Recently, several ToM benchmarks have been proposed to assess general-purpose VLMs \cite{tom2,tom3} along different aspects such as temporal localization of emotions, intentionality-understanding, and perspective-taking. However, these studies find that only models that are fine-tuned on curated ToM datasets exhibit any reasoning abilities, albeit not aligning with the well-established ToM theories/frameworks explaining human social cognition. Such curated data is scarcely available and it is unclear what alternative architectures or training objectives would enable models to obtain the ToM abilities required for multi-image V2L tasks.

\section{Models Architectures}
\label{sec:models}

\begin{figure*}[htbp]
  \centering
  \includegraphics[width=\linewidth,keepaspectratio]{images/review_model_arch.pdf}
  \caption{Outline of the architecture common across modeling approaches for multi-image-to-text tasks.}
  \label{fig:model_arch}
\end{figure*}

Modeling approaches to multi-image-to-text tasks have evolved over time from being recurrent neural network (RNN)-based \cite{lstm} to being transformer-based \cite{transformer}. More recent models directly leverage pre-trained large (vision)-language models (LLMs/VLMs), often in a zero-shot manner. In this section, we discuss this evolution and summarize the various state-of-the-art model architectures proposed for the five multi-image-to-text tasks. Architectures proposed for these tasks primarily comprise three modules---a vision encoder, a language decoder, and an intermediate module (typically referred as the projector/adapter) for adapting visual information into contextualized representations for text generation. We describe the functionality of these modules and review the design principles common across all the tasks in the proposed approaches. Furthermore, we also discuss how off-the-shelf  pre-trained VLMs are currently being used to handle various multi-image-to-text tasks. Table~\ref{tab:models} outlines a summary of the models and details related to the selection procedure are provided in Appendix~\ref{appendix:b}.

\subsection{Vision Encoder}

The primary purpose of a vision encoder in vision-to-language tasks is to extract information from the input visual sequence and to optimally encode it into a contextual representation that guides language generation. To achieve this, encoders in the proposed models follow multiple steps, some of which are common across the five multi-image-to-text tasks. First, a pre-trained vision model is utilized for extracting feature representations of the raw input sequences of images/video frames. Earlier approaches used convolutional neural network (CNN)-based vision models such as ResNet \cite{resnet} or R3D \cite{resnet3d} that are primarily pre-trained on the object detection task using large amounts of image/video data. Most of the recent models across the tasks use transformer-based vision models pre-trained for various image-only and image-text alignment objectives, e.g., CLIP-ViT-L \cite{clip}.

We note that besides the primary input sequence of images/video frames, models proposed for some of the tasks, e.g., \color{xkcdVividBlue}MAAD\color{black}, utilize additional input data such as close-ups of characters in the movie clips (\textit{exemplars}) \cite{maad2}. Furthermore, the TAPM \cite{vist_tapm} model proposed for the \color{xkcdVividBlue}VST \color{black} task utilizes FasterRCNN \cite{faster-rcnn} for extracting such local character/object-level features alongside the global image-level features from ResNet. Following the extraction of visual features using pre-trained vision models, most vision encoders comprise an internal sequence-encoder component for learning relationships and dependencies between the individual image/frame-level features at different temporal positions. Some models implement this step either using RNNs or a transformer network with multi-head self-attention for learning temporal relationships and position-encoding mechanism for tracking the order of entities/events in the sequence.

Beyond these steps that are common across the tasks, vision-encoders may also contain additional task-specific steps for capturing the visual information in a way that suits the task's objective better. For instance, the ViLA model \cite{videoqa_vila} for the \color{xkcdVividBlue}VIDQA \color{black} task utilizes a learnable Frame-Sampler component to efficiently select a small subset of frames that are most likely to contain the relevant information needed to answer the question. Another example with a task-specific step is the MSCM+BART  model \cite{vist_kg2} for \color{xkcdVividBlue}VST\color{black}, in which the initial set of image objects/`concepts' are expanded using an external knowledge graph for generating diverse and informative stories.
Despite these task-specific steps, we found that the vision encoder module in the architectures proposed for the various multi-image-to-text tasks share a common set of components that are broadly outlined in Figure~\ref{fig:model_arch}.

\subsection{The Vision-to-Language Bridge}

Some V2L model architectures utilize an intermediate module that bridges the input and output modalities for effectively conditioning the text generation on the extracted visual features. Different models operationalize this module with different degrees of complexity. Earlier approaches for several multi-image-to-text tasks condition the text generation process by directly fusing vision encoder outputs with the language decoder input \cite{vist_glacnet}. Some architectures employ cross-attention mechanisms to focus on the relevant parts of the visual features at various temporal positions during decoding \cite{vc_task}. However, approaches that adopt pre-trained models---e.g., CLIP-ViT-L \cite{clip} as the visual model---tend to employ learnable intermediate layers for aligning and converting outputs of the vision encoder into a format that the language decoder can understand.

In some of the proposed models, this intermediate module is a single linear layer that transforms the visual features into a common shared space, which can be used by the language decoder \cite{videoqa_llamavqa,llava}. In other models, advanced transformer-based projectors such as a Q-Former \cite{blip2} are used for their ability to leverage cross-modal interactions effectively \cite{maad3}. In essence, Q-Former uses dynamic query vectors that are pre-trained to attend to both visual and textual representations, enhancing its ability to generalize and perform well (relative to a single linear layer) across different tasks. Besides these popular methods for adapting multimodal information, some approaches make use of graph neural networks for capturing relationships between objects in the images at different temporal positions and words in the corresponding sentences of the text \cite{vc_gnn}. While there is no definitive way to design this intermediate module, recent work has compared the two approaches, i.e., using cross-attention between modalities or using a multimodal projector for transforming vision encoder features into the language space, and found that the latter leads to a stable/improved performance of models \cite{what_matters}.

\subsection{Language Decoder}

After encoding and adapting the visual information, models employ a language decoder component for text generation. The decoder can either be learned from scratch or consist of a pre-trained language model with additional trainable task-specific layers. Figure~\ref{fig:model_arch} summarizes the different ways in which this step is operationalized across tasks in the proposed architectures. Earlier models learn an RNN by initializing it with the visual context embedding from the previous steps \cite{vc_task,vist_glacnet}. The decoder then typically follows a `teacher forcing' strategy during training to generate one word at a time autoregressively.

Subsequent models have replaced RNNs with the transformer architecture owing to its computation scalability and efficiency in handling long context-windows. Besides the initial word embedding layer and the position encoding step (for maintaining information about the input sequence token order), a transformer decoder is typically made up of multiple identical blocks. Each block comprises a multi-head self-attention layer for modeling intra-sentence relations (between the words) and a multi-head cross-attention layer for learning relationships between representations of each word and the outputs of the visual encoder/projector. For instance, in the \color{xkcdVividBlue}CC \color{black} task, this refers to conditioning each word in the caption on vision encoder outputs (denoted as `difference-representations').

Instead of training the decoder from scratch, some approaches use language models such as GPT-2 \cite{gpt2} and \textsc{LLAMA 2} \cite{llama2}, which are pre-trained on several text-only tasks such as question-answering and text classification/completion. The pre-trained language models are either used directly for generation by freezing their parameters \cite{maad1,maad2,maad3}, or by inserting and fine-tuning additional adaptive layers on top of them for ensuring relevance of the generated text to the downstream task of interest \cite{vist_tapm}. We also note that some models incorporate information from external knowledge bases/graphs into the decoder module to improve coherence and factuality of the generated text, e.g., TextKG \cite{vc_textkg} for the \color{xkcdVividBlue}VC \color{black} task and KG Story \cite{vist_kg2} for the \color{xkcdVividBlue}VST \color{black} task.

\subsection{Off-the-shelf Pre-trained VLMs}
\label{sec:4_4}

The standard model architecture we have discussed so far is also present in more powerful general-purpose foundation VLMs (pre-trained on several tasks using large amounts of data), which can be used directly for multi-image-to-text tasks. Their pre-training process typically happens in two stages---self-supervised alignment training and visual instruction tuning. During the first stage, only the parameters of the intermediate module connecting both unimodal backbones are updated (commonly using paired image-text data) utilizing a contrastive training objective.

In the second stage, models are instruction-tuned using multi-turn conversations obtained for visual data either through crowd-sourcing or by leveraging tools such as GPT-4 \cite{gpt4}. Contrary to task-specific modeling approaches, these pre-trained VLMs are simply prompted (typically in a zero-shot manner) using visual tokens accompanied by task-specific instructions. Some of the pre-trained VLMs that are used off-the-shelf for the multi-image-to-text tasks include: ViLA for \color{xkcdVividBlue}VIDQA\color{black}, mPLUG-2 for \color{xkcdVividBlue}VC\color{black}, VideoLLAMA for \color{xkcdVividBlue}MAAD\color{black}, and LLaVA-NeXT for \color{xkcdVividBlue}VST \color{black} \cite{videoqa_vila,mplug2,maad4,nytws}.

\section{Evaluation}
\label{sec:eval}

Given all the similarities described above, it is not surprising that all multi-image-to-text tasks are also evaluated leveraging similar methods. These methods range from using traditional \textit{n}-gram matching metrics to obtaining human judgments and ratings to, more recently, using off-the-shelf pre-trained VLMs assessing the generated output. We broadly classify these evaluation methods into two main categories---automatic and human evaluation. In the following subsections, we discuss the several quantitative metrics and benchmarks widely used for each of the tasks, along with the rationales for relying on them.

\subsection{Automatic Evaluation}
\label{sec:5_1}

To computationally assess the quality of model-generated text along different aspects, several automatic metrics have been proposed. While some metrics rely on answers/text provided by human annotators, others are reference-free and assess model outputs independent of the ground-truth data. Besides computational metrics, the community has also relied on benchmark datasets designed to reveal various general capabilities of models.

\paragraph{Reference-based metrics.}
The five multi-image-to-text tasks we examined primarily assessed model-generated candidate text by comparing it to corresponding human-written references. Specifically, traditional metrics that were originally designed for evaluating machine translation and text summarization tasks---BLEU \cite{bleu}, METEOR \cite{meteor}, and ROUGE \cite{rouge}---are used to measure precision and recall of overlapping $\mathit{n}$-grams between the candidate and reference text. Usually, metrics such as CIDEr \cite{cider} and SPICE \cite{spice} that have been specifically developed for the evaluation of image and video captioning are also used in conjunction with the above three metrics.

All the above-mentioned metrics rely on direct raw text comparisons of ground-truth references and model outputs. As this ground-truth data might not be always available, embedding-level reference-based evaluation metrics such as WMD \cite{wmd}, BERTScore \cite{bertscore}, and ViLBERTScore \cite{vilbertscore} have been proposed. Recent work for \color{xkcdVividBlue}VC\color{black}, \color{xkcdVividBlue}VIDQA\color{black}, and \color{xkcdVividBlue}MAAD \color{black} tasks have incorporated these metrics to measure the similarity between candidate and reference embeddings, obtained by projecting corresponding text into a common pre-trained semantic space \cite{vc_swinbert,maad2,maad3}.

\paragraph{Reference-free metrics.}
Comparing model-generated text to ground-truth references, typically provided by crowd-workers or scraped from the internet has various limitations. Most reference-based metrics do not account for the visual modality upon which the generated text is conditioned. Furthermore, as shown in Figure~\ref{fig:shared_space}, text in the \color{xkcdVividBlue}VC\color{black}, \color{xkcdVividBlue}MAAD\color{black}, and \color{xkcdVividBlue}VST \color{black} tasks often complements the visual input by encompassing various abstract/creative concepts, and is not merely descriptive. This makes reference-based metrics generally inappropriate for accurately evaluating multi-image-to-text tasks.

For the reasons detailed above, various reference-free metrics such as MAUVE \cite{mauve} and UNION \cite{union} have been proposed for unimodal open-ended text generation tasks. However, for the visually-conditioned text generation tasks, this adoption/shift is still relatively recent, with a persistent emphasis on reference-based $\mathit{n}$-gram metrics to date. That said, various reference-free metrics have been recently proposed to assess different aspects of evaluation that are important for several tasks. For instance, metrics such as CLIPScore \cite{clipscore} and GROOViST \cite{groovist} have been developed for evaluating visual grounding---the degree of alignment between the generated text and the visual input---in \color{xkcdVividBlue}VC \color{black} and \color{xkcdVividBlue}VST \color{black} tasks. Similarly, the RoViST \cite{rovist} suite of metrics has been proposed to assess coherence, the extent of repetition, and visual grounding of the generated text. We note that there are also metrics such as CRITIC \cite{maad3} and CM (character matching) \cite{vist_cm} designed to evaluate task-specific aspects such as the accuracy of referencing to characters in the model outputs. Besides the above-mentioned metrics (tailored to measure specific features of the generated content), there is also an increasing adoption and reliance on using pre-trained LLMs and VLMs as judges \cite{llmeval2}. Essentially, these pre-trained general-purpose models are prompted to score or rate a model-generated response along any of the evaluation dimensions of interest, such as fluency or relevance (e.g.,\textit{ `How fluent is the generated text on a scale of 1 to 5?'}). However, the effectiveness and reliability of this approach is still debated \cite{llmeval1}.

\paragraph{Benchmarks.}
With the increase in scale, data, and extensive multi-step training processes it is difficult to fully understand the capabilities of models based only on their performance on held-out test splits of task-specific datasets. To address this limitation, numerous benchmark datasets have been proposed that focus on evaluating more fine-grained abilities of models. Some of the popular benchmarks proposed to test models trained for multi-image scenarios include: NLVR2 \cite{cc_nlvr2} which focuses on models' ability to understand the visual compositionality given a pair of images and corresponding textual description; ViLMA \cite{vc_vilma} and MVBench \cite{videoqa_mvbench} which focus on testing models' spatio-temporal reasoning capabilities (e.g., counting actions across frames of a video); Mementos \cite{mementos} which studies object and behavioral hallucinations, and their interconnectedness.

Despite continued progress to improve and update existing benchmark datasets to cover various edge cases, evaluation using benchmarks is not without its limitations. For instance, some benchmarks are constructed using data from test/validation splits of existing popular datasets in the community, leading to a potential contamination problem \cite{contaminated_benchmarks}. Moreover, recent modeling approaches typically tend to incorporate most existing benchmark datasets into their fine-tuning process to ensure the stability and generalization of models in real-world applications. When such approaches abstain from disclosing the data used for training the models, it generally undermines the process of automatic evaluation using benchmarks for comparing models against each other.

\subsection{Human Evaluation}

Given the current state of automatic evaluation, some multi-image-to-text tasks such as \color{xkcdVividBlue}VC \color{black}and \color{xkcdVividBlue}VST \color{black}rely on human evaluation to accurately determine the quality of the model-generated text. This process involves recruiting online crowd-workers who are native or proficient speakers of the target language. Depending on the type of data or variation of the task, annotators with expertise and familiarity with terminology relevant to the corresponding domain (e.g., medical or sports videos) might be preferred.

Participants of the evaluation study are provided with a set of task-specific rubrics/instructions along with representative examples required for judging the model outputs. They are asked to assess the overall quality of model-generated outputs either independently (per sample) \cite{groovist} or relative to outputs from other models \cite{rovist}. Alternatively, evaluators might be required to provide scores/ratings for various criteria ranging from broad (e.g., text conciseness, fluency, grammatical correctness) to specific (e.g., factuality, hallucinations, expressiveness). The obtained scores are usually compared pairwise to rank models appropriately.

Some pre-trained VLM frameworks such as LLaVA-RLHF \cite{llava_rlhf} leverage this qualitative feedback to optimize model parameters for learning to generate human-preferred text. Although human evaluation is still indispensable for several tasks, it is also expensive, time-consuming, and challenging. Defining clear evaluation protocols for ensuring the reliability and quality of human judgments is an active research area \cite{human_eval_protocols1,human_eval_protocols2}.

\section{Discussion}
\label{sec:future_directions}

As discussed above, the problem of generating text from a sequence of temporally ordered images or frames is a challenging one, and relevant to several downstream tasks and applications. Here, we reflect on some crucial aspects and outline various prospective research directions (RDs) and takeaways.

\paragraph{RD 1: Towards more naturalistic scenarios.}
Many of the multi-image-to-text tasks we consider in this work have real-world applications. For instance, solutions to the \color{xkcdVividBlue}CC \color{black} task can be used for assisted surveillance and for tracking changes in digital media assets \cite{cc_spot_the_diff}. In the \color{xkcdVividBlue}MAAD \color{black} task, models are required to generate descriptions that complement information in the original audio dialog/soundtrack, for improved accessibility to visually impaired users and for enhancing the visual experience of sighted users \cite{maad2}.

However, many day-to-day human-centered scenarios involve personalizing to various contexts or situations. We argue that existing multi-image-to-text tasks in their definitions and settings do not fully reflect this aspect. Tailoring model-generated descriptions/narrations to the perspective of end-users requires task settings in which models and humans can interact iteratively. Such settings would enable incorporation of human expectations and communicative contexts which typically tend to be dynamic in real-world applications. To this end, we advocate for variations of existing tasks where models can learn to contextualize and reason through interactions with other agents (humans or other models) for generating stories, descriptions, or answers. Furthermore, we also advocate for exploration of controlled task settings in which models are expected to generate text adhering to a specific style \cite{rd_task1} or point-of-view.

\paragraph{RD 2: Are general-purpose VLMs all we need?}
As discussed in Section~\ref{sec:4_4}, VLMs that are trained on various general-purpose datasets are increasingly being used for multi-image-to-text tasks through prompting. Powerful open-source models such as Molmo \cite{molmo}, and models optimized for multi-image scenarios such as Mantis \cite{mantis} are becoming increasingly available, suggesting that the trend of adopting them off-the-shelf for solving many V2L tasks is widespread. General-purpose VLMs learn abundant information through multitask pre-training and have a modular design, making them suitable for many downstream tasks. Their modularity also enables for seamless adaptation of VLMs to various novel domains (e.g., medical science) by updating only a small fraction of their parameters \cite{rd_model1}.

Despite the promising generalization of VLMs to certain tasks and domains, they have also been shown to be sensitive to prompts \cite{rd_model2_a,rd_model2_b} and biased towards the textual modality \cite{rd_model3}. To address these problems, recent work proposes various \textit{prompt engineering} techniques to facilitate inference-time adaptation of prompts to make them more suitable for the specific task of interest \cite{rd_model4,rd_model5}. On the other hand, task-specific model architectures consist of components designed to effectively address specialized aspects of the tasks, e.g., computing a \textit{difference representation} of the input image pair in \color{xkcdVividBlue}CC\color{black}. We advocate for modular modeling approaches that bring together efficient task-specific components and combine them with the powerful foundational VLMs. Furthermore, we argue that using graph-based architectures and memory-based modules would result in improved tracking of entity positions/relationships and enable models to assign saliency to memorable events in tasks like \color{xkcdVividBlue}VST \color{black} or \color{xkcdVividBlue}MAAD\color{black}.

\paragraph{RD 3: Improving and rethinking evaluation.}
In Section~\ref{sec:eval}, we discussed the various approaches for evaluating model outputs in multi-image-to-text tasks. While human evaluation is impractical for conducting large-scale assessments, existing automatic evaluation metrics are limited in terms of fully capturing the abilities of models. Increasingly various benchmarking datasets are being proposed to assess models along different axes important for grounding language in the visual input \cite{rd_eval2}. However, many benchmarks often suffer from the problem of \textit{visual content irrelevance}, which refers to models performing well on the benchmark datasets by primarily relying only on the language modality \cite{rd_eval1}. Furthermore, data leakage and contamination problems (see section~\ref{sec:5_1}) also hinder fair and accurate testing of model's skills using benchmarks.

While it is important to continue directing research efforts towards developing more extensive multi-image benchmarks such as ReMI \cite{rd_eval3}, we argue that the purpose of evaluation is to also provide insights that can be directly leveraged for improving model architectures and learning procedures. For single-image-to-text tasks, recent works have adapted interpretability methods that focus on understanding the behavior and internal representations of models \cite{rd_eval4,rd_eval5}. These methods can complement traditional evaluation techniques for enabling intra- and inter-model comparisons of behaviors and mechanisms for obtaining a holistic understanding. We strongly advocate for work that adapts various categories of interpretability methods for multi-image-to-text scenarios.

\section{Conclusion}

In this paper, we introduce STeCa, a novel agent learning framework designed to enhance the performance of LLM agents in long-horizon tasks. 
STeCa identifies deviated actions through step-level reward comparisons and constructs calibration trajectories via reflection. 
These trajectories serve as critical data for reinforced training. Extensive experiments demonstrate that STeCa significantly outperforms baseline methods, with additional analyses underscoring its robust calibration capabilities.

\section*{Acknowledgments}

We are immensely grateful to our colleagues at the Dialogue Modelling Group (DMG) for their invaluable suggestions at different stages of this work. We thank Alberto Testoni and Anna Bavaresco for their insightful feedback. Raquel Fern{\'a}ndez was supported by the European Research Council (ERC Consolidator Grant DREAM 819455).

\bibliography{tacl}
\bibliographystyle{acl_natbib}

\onecolumn

\subsection{Lloyd-Max Algorithm}
\label{subsec:Lloyd-Max}
For a given quantization bitwidth $B$ and an operand $\bm{X}$, the Lloyd-Max algorithm finds $2^B$ quantization levels $\{\hat{x}_i\}_{i=1}^{2^B}$ such that quantizing $\bm{X}$ by rounding each scalar in $\bm{X}$ to the nearest quantization level minimizes the quantization MSE. 

The algorithm starts with an initial guess of quantization levels and then iteratively computes quantization thresholds $\{\tau_i\}_{i=1}^{2^B-1}$ and updates quantization levels $\{\hat{x}_i\}_{i=1}^{2^B}$. Specifically, at iteration $n$, thresholds are set to the midpoints of the previous iteration's levels:
\begin{align*}
    \tau_i^{(n)}=\frac{\hat{x}_i^{(n-1)}+\hat{x}_{i+1}^{(n-1)}}2 \text{ for } i=1\ldots 2^B-1
\end{align*}
Subsequently, the quantization levels are re-computed as conditional means of the data regions defined by the new thresholds:
\begin{align*}
    \hat{x}_i^{(n)}=\mathbb{E}\left[ \bm{X} \big| \bm{X}\in [\tau_{i-1}^{(n)},\tau_i^{(n)}] \right] \text{ for } i=1\ldots 2^B
\end{align*}
where to satisfy boundary conditions we have $\tau_0=-\infty$ and $\tau_{2^B}=\infty$. The algorithm iterates the above steps until convergence.

Figure \ref{fig:lm_quant} compares the quantization levels of a $7$-bit floating point (E3M3) quantizer (left) to a $7$-bit Lloyd-Max quantizer (right) when quantizing a layer of weights from the GPT3-126M model at a per-tensor granularity. As shown, the Lloyd-Max quantizer achieves substantially lower quantization MSE. Further, Table \ref{tab:FP7_vs_LM7} shows the superior perplexity achieved by Lloyd-Max quantizers for bitwidths of $7$, $6$ and $5$. The difference between the quantizers is clear at 5 bits, where per-tensor FP quantization incurs a drastic and unacceptable increase in perplexity, while Lloyd-Max quantization incurs a much smaller increase. Nevertheless, we note that even the optimal Lloyd-Max quantizer incurs a notable ($\sim 1.5$) increase in perplexity due to the coarse granularity of quantization. 

\begin{figure}[h]
  \centering
  \includegraphics[width=0.7\linewidth]{sections/figures/LM7_FP7.pdf}
  \caption{\small Quantization levels and the corresponding quantization MSE of Floating Point (left) vs Lloyd-Max (right) Quantizers for a layer of weights in the GPT3-126M model.}
  \label{fig:lm_quant}
\end{figure}

\begin{table}[h]\scriptsize
\begin{center}
\caption{\label{tab:FP7_vs_LM7} \small Comparing perplexity (lower is better) achieved by floating point quantizers and Lloyd-Max quantizers on a GPT3-126M model for the Wikitext-103 dataset.}
\begin{tabular}{c|cc|c}
\hline
 \multirow{2}{*}{\textbf{Bitwidth}} & \multicolumn{2}{|c|}{\textbf{Floating-Point Quantizer}} & \textbf{Lloyd-Max Quantizer} \\
 & Best Format & Wikitext-103 Perplexity & Wikitext-103 Perplexity \\
\hline
7 & E3M3 & 18.32 & 18.27 \\
6 & E3M2 & 19.07 & 18.51 \\
5 & E4M0 & 43.89 & 19.71 \\
\hline
\end{tabular}
\end{center}
\end{table}

\subsection{Proof of Local Optimality of LO-BCQ}
\label{subsec:lobcq_opt_proof}
For a given block $\bm{b}_j$, the quantization MSE during LO-BCQ can be empirically evaluated as $\frac{1}{L_b}\lVert \bm{b}_j- \bm{\hat{b}}_j\rVert^2_2$ where $\bm{\hat{b}}_j$ is computed from equation (\ref{eq:clustered_quantization_definition}) as $C_{f(\bm{b}_j)}(\bm{b}_j)$. Further, for a given block cluster $\mathcal{B}_i$, we compute the quantization MSE as $\frac{1}{|\mathcal{B}_{i}|}\sum_{\bm{b} \in \mathcal{B}_{i}} \frac{1}{L_b}\lVert \bm{b}- C_i^{(n)}(\bm{b})\rVert^2_2$. Therefore, at the end of iteration $n$, we evaluate the overall quantization MSE $J^{(n)}$ for a given operand $\bm{X}$ composed of $N_c$ block clusters as:
\begin{align*}
    \label{eq:mse_iter_n}
    J^{(n)} = \frac{1}{N_c} \sum_{i=1}^{N_c} \frac{1}{|\mathcal{B}_{i}^{(n)}|}\sum_{\bm{v} \in \mathcal{B}_{i}^{(n)}} \frac{1}{L_b}\lVert \bm{b}- B_i^{(n)}(\bm{b})\rVert^2_2
\end{align*}

At the end of iteration $n$, the codebooks are updated from $\mathcal{C}^{(n-1)}$ to $\mathcal{C}^{(n)}$. However, the mapping of a given vector $\bm{b}_j$ to quantizers $\mathcal{C}^{(n)}$ remains as  $f^{(n)}(\bm{b}_j)$. At the next iteration, during the vector clustering step, $f^{(n+1)}(\bm{b}_j)$ finds new mapping of $\bm{b}_j$ to updated codebooks $\mathcal{C}^{(n)}$ such that the quantization MSE over the candidate codebooks is minimized. Therefore, we obtain the following result for $\bm{b}_j$:
\begin{align*}
\frac{1}{L_b}\lVert \bm{b}_j - C_{f^{(n+1)}(\bm{b}_j)}^{(n)}(\bm{b}_j)\rVert^2_2 \le \frac{1}{L_b}\lVert \bm{b}_j - C_{f^{(n)}(\bm{b}_j)}^{(n)}(\bm{b}_j)\rVert^2_2
\end{align*}

That is, quantizing $\bm{b}_j$ at the end of the block clustering step of iteration $n+1$ results in lower quantization MSE compared to quantizing at the end of iteration $n$. Since this is true for all $\bm{b} \in \bm{X}$, we assert the following:
\begin{equation}
\begin{split}
\label{eq:mse_ineq_1}
    \tilde{J}^{(n+1)} &= \frac{1}{N_c} \sum_{i=1}^{N_c} \frac{1}{|\mathcal{B}_{i}^{(n+1)}|}\sum_{\bm{b} \in \mathcal{B}_{i}^{(n+1)}} \frac{1}{L_b}\lVert \bm{b} - C_i^{(n)}(b)\rVert^2_2 \le J^{(n)}
\end{split}
\end{equation}
where $\tilde{J}^{(n+1)}$ is the the quantization MSE after the vector clustering step at iteration $n+1$.

Next, during the codebook update step (\ref{eq:quantizers_update}) at iteration $n+1$, the per-cluster codebooks $\mathcal{C}^{(n)}$ are updated to $\mathcal{C}^{(n+1)}$ by invoking the Lloyd-Max algorithm \citep{Lloyd}. We know that for any given value distribution, the Lloyd-Max algorithm minimizes the quantization MSE. Therefore, for a given vector cluster $\mathcal{B}_i$ we obtain the following result:

\begin{equation}
    \frac{1}{|\mathcal{B}_{i}^{(n+1)}|}\sum_{\bm{b} \in \mathcal{B}_{i}^{(n+1)}} \frac{1}{L_b}\lVert \bm{b}- C_i^{(n+1)}(\bm{b})\rVert^2_2 \le \frac{1}{|\mathcal{B}_{i}^{(n+1)}|}\sum_{\bm{b} \in \mathcal{B}_{i}^{(n+1)}} \frac{1}{L_b}\lVert \bm{b}- C_i^{(n)}(\bm{b})\rVert^2_2
\end{equation}

The above equation states that quantizing the given block cluster $\mathcal{B}_i$ after updating the associated codebook from $C_i^{(n)}$ to $C_i^{(n+1)}$ results in lower quantization MSE. Since this is true for all the block clusters, we derive the following result: 
\begin{equation}
\begin{split}
\label{eq:mse_ineq_2}
     J^{(n+1)} &= \frac{1}{N_c} \sum_{i=1}^{N_c} \frac{1}{|\mathcal{B}_{i}^{(n+1)}|}\sum_{\bm{b} \in \mathcal{B}_{i}^{(n+1)}} \frac{1}{L_b}\lVert \bm{b}- C_i^{(n+1)}(\bm{b})\rVert^2_2  \le \tilde{J}^{(n+1)}   
\end{split}
\end{equation}

Following (\ref{eq:mse_ineq_1}) and (\ref{eq:mse_ineq_2}), we find that the quantization MSE is non-increasing for each iteration, that is, $J^{(1)} \ge J^{(2)} \ge J^{(3)} \ge \ldots \ge J^{(M)}$ where $M$ is the maximum number of iterations. 
%Therefore, we can say that if the algorithm converges, then it must be that it has converged to a local minimum. 
\hfill $\blacksquare$


\begin{figure}
    \begin{center}
    \includegraphics[width=0.5\textwidth]{sections//figures/mse_vs_iter.pdf}
    \end{center}
    \caption{\small NMSE vs iterations during LO-BCQ compared to other block quantization proposals}
    \label{fig:nmse_vs_iter}
\end{figure}

Figure \ref{fig:nmse_vs_iter} shows the empirical convergence of LO-BCQ across several block lengths and number of codebooks. Also, the MSE achieved by LO-BCQ is compared to baselines such as MXFP and VSQ. As shown, LO-BCQ converges to a lower MSE than the baselines. Further, we achieve better convergence for larger number of codebooks ($N_c$) and for a smaller block length ($L_b$), both of which increase the bitwidth of BCQ (see Eq \ref{eq:bitwidth_bcq}).


\subsection{Additional Accuracy Results}
%Table \ref{tab:lobcq_config} lists the various LOBCQ configurations and their corresponding bitwidths.
\begin{table}
\setlength{\tabcolsep}{4.75pt}
\begin{center}
\caption{\label{tab:lobcq_config} Various LO-BCQ configurations and their bitwidths.}
\begin{tabular}{|c||c|c|c|c||c|c||c|} 
\hline
 & \multicolumn{4}{|c||}{$L_b=8$} & \multicolumn{2}{|c||}{$L_b=4$} & $L_b=2$ \\
 \hline
 \backslashbox{$L_A$\kern-1em}{\kern-1em$N_c$} & 2 & 4 & 8 & 16 & 2 & 4 & 2 \\
 \hline
 64 & 4.25 & 4.375 & 4.5 & 4.625 & 4.375 & 4.625 & 4.625\\
 \hline
 32 & 4.375 & 4.5 & 4.625& 4.75 & 4.5 & 4.75 & 4.75 \\
 \hline
 16 & 4.625 & 4.75& 4.875 & 5 & 4.75 & 5 & 5 \\
 \hline
\end{tabular}
\end{center}
\end{table}

%\subsection{Perplexity achieved by various LO-BCQ configurations on Wikitext-103 dataset}

\begin{table} \centering
\begin{tabular}{|c||c|c|c|c||c|c||c|} 
\hline
 $L_b \rightarrow$& \multicolumn{4}{c||}{8} & \multicolumn{2}{c||}{4} & 2\\
 \hline
 \backslashbox{$L_A$\kern-1em}{\kern-1em$N_c$} & 2 & 4 & 8 & 16 & 2 & 4 & 2  \\
 %$N_c \rightarrow$ & 2 & 4 & 8 & 16 & 2 & 4 & 2 \\
 \hline
 \hline
 \multicolumn{8}{c}{GPT3-1.3B (FP32 PPL = 9.98)} \\ 
 \hline
 \hline
 64 & 10.40 & 10.23 & 10.17 & 10.15 &  10.28 & 10.18 & 10.19 \\
 \hline
 32 & 10.25 & 10.20 & 10.15 & 10.12 &  10.23 & 10.17 & 10.17 \\
 \hline
 16 & 10.22 & 10.16 & 10.10 & 10.09 &  10.21 & 10.14 & 10.16 \\
 \hline
  \hline
 \multicolumn{8}{c}{GPT3-8B (FP32 PPL = 7.38)} \\ 
 \hline
 \hline
 64 & 7.61 & 7.52 & 7.48 &  7.47 &  7.55 &  7.49 & 7.50 \\
 \hline
 32 & 7.52 & 7.50 & 7.46 &  7.45 &  7.52 &  7.48 & 7.48  \\
 \hline
 16 & 7.51 & 7.48 & 7.44 &  7.44 &  7.51 &  7.49 & 7.47  \\
 \hline
\end{tabular}
\caption{\label{tab:ppl_gpt3_abalation} Wikitext-103 perplexity across GPT3-1.3B and 8B models.}
\end{table}

\begin{table} \centering
\begin{tabular}{|c||c|c|c|c||} 
\hline
 $L_b \rightarrow$& \multicolumn{4}{c||}{8}\\
 \hline
 \backslashbox{$L_A$\kern-1em}{\kern-1em$N_c$} & 2 & 4 & 8 & 16 \\
 %$N_c \rightarrow$ & 2 & 4 & 8 & 16 & 2 & 4 & 2 \\
 \hline
 \hline
 \multicolumn{5}{|c|}{Llama2-7B (FP32 PPL = 5.06)} \\ 
 \hline
 \hline
 64 & 5.31 & 5.26 & 5.19 & 5.18  \\
 \hline
 32 & 5.23 & 5.25 & 5.18 & 5.15  \\
 \hline
 16 & 5.23 & 5.19 & 5.16 & 5.14  \\
 \hline
 \multicolumn{5}{|c|}{Nemotron4-15B (FP32 PPL = 5.87)} \\ 
 \hline
 \hline
 64  & 6.3 & 6.20 & 6.13 & 6.08  \\
 \hline
 32  & 6.24 & 6.12 & 6.07 & 6.03  \\
 \hline
 16  & 6.12 & 6.14 & 6.04 & 6.02  \\
 \hline
 \multicolumn{5}{|c|}{Nemotron4-340B (FP32 PPL = 3.48)} \\ 
 \hline
 \hline
 64 & 3.67 & 3.62 & 3.60 & 3.59 \\
 \hline
 32 & 3.63 & 3.61 & 3.59 & 3.56 \\
 \hline
 16 & 3.61 & 3.58 & 3.57 & 3.55 \\
 \hline
\end{tabular}
\caption{\label{tab:ppl_llama7B_nemo15B} Wikitext-103 perplexity compared to FP32 baseline in Llama2-7B and Nemotron4-15B, 340B models}
\end{table}

%\subsection{Perplexity achieved by various LO-BCQ configurations on MMLU dataset}


\begin{table} \centering
\begin{tabular}{|c||c|c|c|c||c|c|c|c|} 
\hline
 $L_b \rightarrow$& \multicolumn{4}{c||}{8} & \multicolumn{4}{c||}{8}\\
 \hline
 \backslashbox{$L_A$\kern-1em}{\kern-1em$N_c$} & 2 & 4 & 8 & 16 & 2 & 4 & 8 & 16  \\
 %$N_c \rightarrow$ & 2 & 4 & 8 & 16 & 2 & 4 & 2 \\
 \hline
 \hline
 \multicolumn{5}{|c|}{Llama2-7B (FP32 Accuracy = 45.8\%)} & \multicolumn{4}{|c|}{Llama2-70B (FP32 Accuracy = 69.12\%)} \\ 
 \hline
 \hline
 64 & 43.9 & 43.4 & 43.9 & 44.9 & 68.07 & 68.27 & 68.17 & 68.75 \\
 \hline
 32 & 44.5 & 43.8 & 44.9 & 44.5 & 68.37 & 68.51 & 68.35 & 68.27  \\
 \hline
 16 & 43.9 & 42.7 & 44.9 & 45 & 68.12 & 68.77 & 68.31 & 68.59  \\
 \hline
 \hline
 \multicolumn{5}{|c|}{GPT3-22B (FP32 Accuracy = 38.75\%)} & \multicolumn{4}{|c|}{Nemotron4-15B (FP32 Accuracy = 64.3\%)} \\ 
 \hline
 \hline
 64 & 36.71 & 38.85 & 38.13 & 38.92 & 63.17 & 62.36 & 63.72 & 64.09 \\
 \hline
 32 & 37.95 & 38.69 & 39.45 & 38.34 & 64.05 & 62.30 & 63.8 & 64.33  \\
 \hline
 16 & 38.88 & 38.80 & 38.31 & 38.92 & 63.22 & 63.51 & 63.93 & 64.43  \\
 \hline
\end{tabular}
\caption{\label{tab:mmlu_abalation} Accuracy on MMLU dataset across GPT3-22B, Llama2-7B, 70B and Nemotron4-15B models.}
\end{table}


%\subsection{Perplexity achieved by various LO-BCQ configurations on LM evaluation harness}

\begin{table} \centering
\begin{tabular}{|c||c|c|c|c||c|c|c|c|} 
\hline
 $L_b \rightarrow$& \multicolumn{4}{c||}{8} & \multicolumn{4}{c||}{8}\\
 \hline
 \backslashbox{$L_A$\kern-1em}{\kern-1em$N_c$} & 2 & 4 & 8 & 16 & 2 & 4 & 8 & 16  \\
 %$N_c \rightarrow$ & 2 & 4 & 8 & 16 & 2 & 4 & 2 \\
 \hline
 \hline
 \multicolumn{5}{|c|}{Race (FP32 Accuracy = 37.51\%)} & \multicolumn{4}{|c|}{Boolq (FP32 Accuracy = 64.62\%)} \\ 
 \hline
 \hline
 64 & 36.94 & 37.13 & 36.27 & 37.13 & 63.73 & 62.26 & 63.49 & 63.36 \\
 \hline
 32 & 37.03 & 36.36 & 36.08 & 37.03 & 62.54 & 63.51 & 63.49 & 63.55  \\
 \hline
 16 & 37.03 & 37.03 & 36.46 & 37.03 & 61.1 & 63.79 & 63.58 & 63.33  \\
 \hline
 \hline
 \multicolumn{5}{|c|}{Winogrande (FP32 Accuracy = 58.01\%)} & \multicolumn{4}{|c|}{Piqa (FP32 Accuracy = 74.21\%)} \\ 
 \hline
 \hline
 64 & 58.17 & 57.22 & 57.85 & 58.33 & 73.01 & 73.07 & 73.07 & 72.80 \\
 \hline
 32 & 59.12 & 58.09 & 57.85 & 58.41 & 73.01 & 73.94 & 72.74 & 73.18  \\
 \hline
 16 & 57.93 & 58.88 & 57.93 & 58.56 & 73.94 & 72.80 & 73.01 & 73.94  \\
 \hline
\end{tabular}
\caption{\label{tab:mmlu_abalation} Accuracy on LM evaluation harness tasks on GPT3-1.3B model.}
\end{table}

\begin{table} \centering
\begin{tabular}{|c||c|c|c|c||c|c|c|c|} 
\hline
 $L_b \rightarrow$& \multicolumn{4}{c||}{8} & \multicolumn{4}{c||}{8}\\
 \hline
 \backslashbox{$L_A$\kern-1em}{\kern-1em$N_c$} & 2 & 4 & 8 & 16 & 2 & 4 & 8 & 16  \\
 %$N_c \rightarrow$ & 2 & 4 & 8 & 16 & 2 & 4 & 2 \\
 \hline
 \hline
 \multicolumn{5}{|c|}{Race (FP32 Accuracy = 41.34\%)} & \multicolumn{4}{|c|}{Boolq (FP32 Accuracy = 68.32\%)} \\ 
 \hline
 \hline
 64 & 40.48 & 40.10 & 39.43 & 39.90 & 69.20 & 68.41 & 69.45 & 68.56 \\
 \hline
 32 & 39.52 & 39.52 & 40.77 & 39.62 & 68.32 & 67.43 & 68.17 & 69.30  \\
 \hline
 16 & 39.81 & 39.71 & 39.90 & 40.38 & 68.10 & 66.33 & 69.51 & 69.42  \\
 \hline
 \hline
 \multicolumn{5}{|c|}{Winogrande (FP32 Accuracy = 67.88\%)} & \multicolumn{4}{|c|}{Piqa (FP32 Accuracy = 78.78\%)} \\ 
 \hline
 \hline
 64 & 66.85 & 66.61 & 67.72 & 67.88 & 77.31 & 77.42 & 77.75 & 77.64 \\
 \hline
 32 & 67.25 & 67.72 & 67.72 & 67.00 & 77.31 & 77.04 & 77.80 & 77.37  \\
 \hline
 16 & 68.11 & 68.90 & 67.88 & 67.48 & 77.37 & 78.13 & 78.13 & 77.69  \\
 \hline
\end{tabular}
\caption{\label{tab:mmlu_abalation} Accuracy on LM evaluation harness tasks on GPT3-8B model.}
\end{table}

\begin{table} \centering
\begin{tabular}{|c||c|c|c|c||c|c|c|c|} 
\hline
 $L_b \rightarrow$& \multicolumn{4}{c||}{8} & \multicolumn{4}{c||}{8}\\
 \hline
 \backslashbox{$L_A$\kern-1em}{\kern-1em$N_c$} & 2 & 4 & 8 & 16 & 2 & 4 & 8 & 16  \\
 %$N_c \rightarrow$ & 2 & 4 & 8 & 16 & 2 & 4 & 2 \\
 \hline
 \hline
 \multicolumn{5}{|c|}{Race (FP32 Accuracy = 40.67\%)} & \multicolumn{4}{|c|}{Boolq (FP32 Accuracy = 76.54\%)} \\ 
 \hline
 \hline
 64 & 40.48 & 40.10 & 39.43 & 39.90 & 75.41 & 75.11 & 77.09 & 75.66 \\
 \hline
 32 & 39.52 & 39.52 & 40.77 & 39.62 & 76.02 & 76.02 & 75.96 & 75.35  \\
 \hline
 16 & 39.81 & 39.71 & 39.90 & 40.38 & 75.05 & 73.82 & 75.72 & 76.09  \\
 \hline
 \hline
 \multicolumn{5}{|c|}{Winogrande (FP32 Accuracy = 70.64\%)} & \multicolumn{4}{|c|}{Piqa (FP32 Accuracy = 79.16\%)} \\ 
 \hline
 \hline
 64 & 69.14 & 70.17 & 70.17 & 70.56 & 78.24 & 79.00 & 78.62 & 78.73 \\
 \hline
 32 & 70.96 & 69.69 & 71.27 & 69.30 & 78.56 & 79.49 & 79.16 & 78.89  \\
 \hline
 16 & 71.03 & 69.53 & 69.69 & 70.40 & 78.13 & 79.16 & 79.00 & 79.00  \\
 \hline
\end{tabular}
\caption{\label{tab:mmlu_abalation} Accuracy on LM evaluation harness tasks on GPT3-22B model.}
\end{table}

\begin{table} \centering
\begin{tabular}{|c||c|c|c|c||c|c|c|c|} 
\hline
 $L_b \rightarrow$& \multicolumn{4}{c||}{8} & \multicolumn{4}{c||}{8}\\
 \hline
 \backslashbox{$L_A$\kern-1em}{\kern-1em$N_c$} & 2 & 4 & 8 & 16 & 2 & 4 & 8 & 16  \\
 %$N_c \rightarrow$ & 2 & 4 & 8 & 16 & 2 & 4 & 2 \\
 \hline
 \hline
 \multicolumn{5}{|c|}{Race (FP32 Accuracy = 44.4\%)} & \multicolumn{4}{|c|}{Boolq (FP32 Accuracy = 79.29\%)} \\ 
 \hline
 \hline
 64 & 42.49 & 42.51 & 42.58 & 43.45 & 77.58 & 77.37 & 77.43 & 78.1 \\
 \hline
 32 & 43.35 & 42.49 & 43.64 & 43.73 & 77.86 & 75.32 & 77.28 & 77.86  \\
 \hline
 16 & 44.21 & 44.21 & 43.64 & 42.97 & 78.65 & 77 & 76.94 & 77.98  \\
 \hline
 \hline
 \multicolumn{5}{|c|}{Winogrande (FP32 Accuracy = 69.38\%)} & \multicolumn{4}{|c|}{Piqa (FP32 Accuracy = 78.07\%)} \\ 
 \hline
 \hline
 64 & 68.9 & 68.43 & 69.77 & 68.19 & 77.09 & 76.82 & 77.09 & 77.86 \\
 \hline
 32 & 69.38 & 68.51 & 68.82 & 68.90 & 78.07 & 76.71 & 78.07 & 77.86  \\
 \hline
 16 & 69.53 & 67.09 & 69.38 & 68.90 & 77.37 & 77.8 & 77.91 & 77.69  \\
 \hline
\end{tabular}
\caption{\label{tab:mmlu_abalation} Accuracy on LM evaluation harness tasks on Llama2-7B model.}
\end{table}

\begin{table} \centering
\begin{tabular}{|c||c|c|c|c||c|c|c|c|} 
\hline
 $L_b \rightarrow$& \multicolumn{4}{c||}{8} & \multicolumn{4}{c||}{8}\\
 \hline
 \backslashbox{$L_A$\kern-1em}{\kern-1em$N_c$} & 2 & 4 & 8 & 16 & 2 & 4 & 8 & 16  \\
 %$N_c \rightarrow$ & 2 & 4 & 8 & 16 & 2 & 4 & 2 \\
 \hline
 \hline
 \multicolumn{5}{|c|}{Race (FP32 Accuracy = 48.8\%)} & \multicolumn{4}{|c|}{Boolq (FP32 Accuracy = 85.23\%)} \\ 
 \hline
 \hline
 64 & 49.00 & 49.00 & 49.28 & 48.71 & 82.82 & 84.28 & 84.03 & 84.25 \\
 \hline
 32 & 49.57 & 48.52 & 48.33 & 49.28 & 83.85 & 84.46 & 84.31 & 84.93  \\
 \hline
 16 & 49.85 & 49.09 & 49.28 & 48.99 & 85.11 & 84.46 & 84.61 & 83.94  \\
 \hline
 \hline
 \multicolumn{5}{|c|}{Winogrande (FP32 Accuracy = 79.95\%)} & \multicolumn{4}{|c|}{Piqa (FP32 Accuracy = 81.56\%)} \\ 
 \hline
 \hline
 64 & 78.77 & 78.45 & 78.37 & 79.16 & 81.45 & 80.69 & 81.45 & 81.5 \\
 \hline
 32 & 78.45 & 79.01 & 78.69 & 80.66 & 81.56 & 80.58 & 81.18 & 81.34  \\
 \hline
 16 & 79.95 & 79.56 & 79.79 & 79.72 & 81.28 & 81.66 & 81.28 & 80.96  \\
 \hline
\end{tabular}
\caption{\label{tab:mmlu_abalation} Accuracy on LM evaluation harness tasks on Llama2-70B model.}
\end{table}

%\section{MSE Studies}
%\textcolor{red}{TODO}


\subsection{Number Formats and Quantization Method}
\label{subsec:numFormats_quantMethod}
\subsubsection{Integer Format}
An $n$-bit signed integer (INT) is typically represented with a 2s-complement format \citep{yao2022zeroquant,xiao2023smoothquant,dai2021vsq}, where the most significant bit denotes the sign.

\subsubsection{Floating Point Format}
An $n$-bit signed floating point (FP) number $x$ comprises of a 1-bit sign ($x_{\mathrm{sign}}$), $B_m$-bit mantissa ($x_{\mathrm{mant}}$) and $B_e$-bit exponent ($x_{\mathrm{exp}}$) such that $B_m+B_e=n-1$. The associated constant exponent bias ($E_{\mathrm{bias}}$) is computed as $(2^{{B_e}-1}-1)$. We denote this format as $E_{B_e}M_{B_m}$.  

\subsubsection{Quantization Scheme}
\label{subsec:quant_method}
A quantization scheme dictates how a given unquantized tensor is converted to its quantized representation. We consider FP formats for the purpose of illustration. Given an unquantized tensor $\bm{X}$ and an FP format $E_{B_e}M_{B_m}$, we first, we compute the quantization scale factor $s_X$ that maps the maximum absolute value of $\bm{X}$ to the maximum quantization level of the $E_{B_e}M_{B_m}$ format as follows:
\begin{align}
\label{eq:sf}
    s_X = \frac{\mathrm{max}(|\bm{X}|)}{\mathrm{max}(E_{B_e}M_{B_m})}
\end{align}
In the above equation, $|\cdot|$ denotes the absolute value function.

Next, we scale $\bm{X}$ by $s_X$ and quantize it to $\hat{\bm{X}}$ by rounding it to the nearest quantization level of $E_{B_e}M_{B_m}$ as:

\begin{align}
\label{eq:tensor_quant}
    \hat{\bm{X}} = \text{round-to-nearest}\left(\frac{\bm{X}}{s_X}, E_{B_e}M_{B_m}\right)
\end{align}

We perform dynamic max-scaled quantization \citep{wu2020integer}, where the scale factor $s$ for activations is dynamically computed during runtime.

\subsection{Vector Scaled Quantization}
\begin{wrapfigure}{r}{0.35\linewidth}
  \centering
  \includegraphics[width=\linewidth]{sections/figures/vsquant.jpg}
  \caption{\small Vectorwise decomposition for per-vector scaled quantization (VSQ \citep{dai2021vsq}).}
  \label{fig:vsquant}
\end{wrapfigure}
During VSQ \citep{dai2021vsq}, the operand tensors are decomposed into 1D vectors in a hardware friendly manner as shown in Figure \ref{fig:vsquant}. Since the decomposed tensors are used as operands in matrix multiplications during inference, it is beneficial to perform this decomposition along the reduction dimension of the multiplication. The vectorwise quantization is performed similar to tensorwise quantization described in Equations \ref{eq:sf} and \ref{eq:tensor_quant}, where a scale factor $s_v$ is required for each vector $\bm{v}$ that maps the maximum absolute value of that vector to the maximum quantization level. While smaller vector lengths can lead to larger accuracy gains, the associated memory and computational overheads due to the per-vector scale factors increases. To alleviate these overheads, VSQ \citep{dai2021vsq} proposed a second level quantization of the per-vector scale factors to unsigned integers, while MX \citep{rouhani2023shared} quantizes them to integer powers of 2 (denoted as $2^{INT}$).

\subsubsection{MX Format}
The MX format proposed in \citep{rouhani2023microscaling} introduces the concept of sub-block shifting. For every two scalar elements of $b$-bits each, there is a shared exponent bit. The value of this exponent bit is determined through an empirical analysis that targets minimizing quantization MSE. We note that the FP format $E_{1}M_{b}$ is strictly better than MX from an accuracy perspective since it allocates a dedicated exponent bit to each scalar as opposed to sharing it across two scalars. Therefore, we conservatively bound the accuracy of a $b+2$-bit signed MX format with that of a $E_{1}M_{b}$ format in our comparisons. For instance, we use E1M2 format as a proxy for MX4.

\begin{figure}
    \centering
    \includegraphics[width=1\linewidth]{sections//figures/BlockFormats.pdf}
    \caption{\small Comparing LO-BCQ to MX format.}
    \label{fig:block_formats}
\end{figure}

Figure \ref{fig:block_formats} compares our $4$-bit LO-BCQ block format to MX \citep{rouhani2023microscaling}. As shown, both LO-BCQ and MX decompose a given operand tensor into block arrays and each block array into blocks. Similar to MX, we find that per-block quantization ($L_b < L_A$) leads to better accuracy due to increased flexibility. While MX achieves this through per-block $1$-bit micro-scales, we associate a dedicated codebook to each block through a per-block codebook selector. Further, MX quantizes the per-block array scale-factor to E8M0 format without per-tensor scaling. In contrast during LO-BCQ, we find that per-tensor scaling combined with quantization of per-block array scale-factor to E4M3 format results in superior inference accuracy across models. 


\end{document}
