\section{Related Work}
Fire detection is designed to identify and confirm the occurrence of a fire in a monitored scene. Currently, fire detection relies mainly on computer vision technology, with works focusing on areas such as distinguishing interfering objects, recognizing complex fire images, and enhancing the efficiency of detection systems.

Fire detection environments are diverse and complex, with many scenarios prone to misidentification. Tao et al. \cite{3tao2024} proposed a triple disturbance removal network for smoke detection, which learned discriminative representations to effectively reduce the false alarm rate caused by disturbances at spatial, temporal, and semantic levels. He et al. \cite{4he2021} introduced a lightweight feature-level and decision-level fusion module, incorporating spatial and channel attention mechanisms to detect small smoke patterns and recognize smoke-like objects. Tao et al. \cite{8tao2023} developed a forest smoke recognition network with pixel-level supervision, featuring a detail difference perception module, an attention feature separation module, and a multi-connection aggregation method, which effectively mitigates the low detection rate and high false alarm rate in complex scenarios. Park et al. \cite{9park2022} proposed a method for generating virtual wildfire images using a Generative Adversarial Network (GAN), annotating them with a weakly supervised image localization module, and performing wildfire detection based on an enhanced YOLOv5s model, significantly reducing false alarms during the detection process.

The visual characteristics of fire vary significantly across different detection scenarios, and the irregular, dynamic shapes of flames and smoke further complicate detection. Li et al. \cite{10li2023} proposed an anchor-free fire recognition algorithm that integrated a multi-scale feature fusion network with a channel attention mechanism, combining loss functions including classification loss, regression loss, and center point loss. This approach enhanced the model's ability to detect irregularly shaped flames and smoke with blurred boundaries. Yuan et al. \cite{11yuan2022} introduced a method that combined a 3D cross-convolutional attention module with count prior embedding, addressing the challenges posed by the semi-transparency and blurred edges of smoke, which often led to reduced detection accuracy. Liang et al. \cite{12laing2024} proposed an anchor-free, structure-based fire detection algorithm, designing the feature extraction network's residual module as a multi-branch structure to capture more expressive flame features. By strengthening feature representation through an improved feature fusion network, this method enhanced the model's ability to detect multi-scale flames, making it suitable for many fire detection scenarios.

Considering the rapid spread of fire, it is crucial not only to improve the accuracy of fire detection but also to enhance the inference speed and deployment efficiency of models. Siddique et al. \cite{13siddique2024} proposed an Internet of Things (IoT)-based federated learning framework for forest fire classification, which distributed computational tasks across multiple nodes. This approach enhanced detection efficiency while safeguarding user privacy and data security. Li et al. \cite{14li2023} introduced a lightweight fire detection model and developed an edge computing system that connected feedback from the edge model to edge gateways and smart devices. This solution addresses the limitations of traditional fire detection systems, which are often too large to be deployed on edge devices. Tian et al. \cite{15tian2024} proposed a fire detection algorithm that strengthened spatial feature extraction and multi-scale feature fusion, incorporating local convolution modules to reduce the size of the backbone network and detection head. This approach achieved high detection accuracy while ensuring real-time performance. Zhang et al. \cite{16zhang2023} presented a flame and smoke detection algorithm that integrated a YOLOv5-ResNet cascade network. By enhancing the YOLOv5 detection network and combining continuous multi-frame detection results with changes in smoke area, the algorithm improved the detection performance of small flame and smoke targets. This approach also effectively eliminated non-flame and non-smoke objects, achieving high accuracy, rapid detection, and low false alarm rate, making it suitable for large-scale industrial applications.

In recent years, the rapid development of deep learning has significantly enhanced the recognition accuracy of visual fire detection. However, models still exhibit a tendency to be \textit{overconfident} in their predictions \cite{17guo2017}. Specifically, for certain samples, the model may produce incorrect classification results while maintaining high confidence in these erroneous predictions. Furthermore, much of current works in visual fire detection focus on improving detection accuracy, with little attention given to uncertainty modeling.

Uncertainty modeling methods have broad applications in the field of computer vision. Ji et al. \cite{18ji2023} were among the first to incorporate uncertainty into the task of image tampering detection. They proposed an uncertainty estimation network that dynamically supervised uncertainty from both the data and the model, using the generated uncertainty map to refine tampering detection outcomes. This approach led to more accurate and reliable detection. In the context of salient object detection, Tian et al. \cite{19tian2023} explored distribution uncertainty, investigating the effectiveness of long-tail learning, single-model uncertainty modeling, and test-time strategies to address the distributional differences between training and testing samples. Yelleni et al. \cite{7yelleni2024} focused on uncertainty in object detection, introducing a method called MC-DropBlock. This approach leveraged the DropBlock technique to model cognitive uncertainty during model training and inference, while using a Gaussian likelihood function to capture accidental uncertainty in the data. Their method significantly enhanced the generalization ability of object detection models.

This paper introduces a new uncertainty modeling method to the visual fire detection by integrating an uncertainty-aware loss with cross-entropy loss and training the model based on curriculum learning. The proposed method demonstrates improved calibration performance in multi-class fire detection tasks, enhancing the reliability of the model's decisions.