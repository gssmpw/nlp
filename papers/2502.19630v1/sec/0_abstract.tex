\begin{abstract}
Detecting 3D objects in point clouds plays a crucial role in autonomous driving systems. Recently, advanced multi-modal methods incorporating camera information have achieved notable performance. For a safe and effective autonomous driving system, algorithms that excel not only in accuracy but also in speed and low latency are essential. However, existing algorithms fail to meet these requirements due to the latency and bandwidth limitations of fixed frame rate sensors, \textit{e.g.}, LiDAR and camera. To address this limitation, we introduce asynchronous event cameras into 3D object detection for the first time. We leverage their high temporal resolution and low bandwidth to enable high-speed 3D object detection. Our method enables detection even during inter-frame intervals when synchronized data is unavailable, by retrieving previous 3D information through the event camera. Furthermore, we introduce the first event-based 3D object detection dataset, DSEC-3DOD, which includes ground-truth 3D bounding boxes at 100 FPS, establishing the first benchmark for event-based 3D detectors. The code and dataset are available at \url{https://github.com/mickeykang16/Ev3DOD}.
\end{abstract}


% Additionally, we propose a robust approach that operates effectively in edge-case scenarios, such as the sudden appearance of new objects or the disappearance of previous ones, significantly enhancing the potential applications of this method.


% Yet, the latency and bandwidth of time-synchronized sensors (\textit{e.g.},~LiDAR, camera) used in existing algorithms fall short of meeting these requirements.



% 최근에는 카메라 정보를 함께 사용하는 advanced multi-modal methods들은 굉장히 성공적인 성능을 달성하였다.
% 하지만, 안전한 자율주행 시스템을 위해서는 low latency의 빠른 속도로 작동하는 알고리즘이 필요하나, 기존 알고리즘에서 사용하는 센서의 latency와 bandwidth는 그러한 요구사항을 만족하지 못한다.

% 우리의 방법은 LiDAR 정보가 존재하지않는 inter-frame interval 순간에도,  이전 순간의 LiDAR 정보를  이벤트 카메라를 통해 retrieve하여, 3차원 공간에서의 detection이 가능하도록 한다.
% 또한, 그 사이에 새로운 물체가 등장하거나, 이전 물체가 사라지는 edge-case scenarios에서도 강건하게 작동하는 방식을 제안하여, 해당 방법의 application을 굉장히 크게 향상시킨다.

% extensive  experiments는 우리의 방법의 효과적임과 효율성을 강조하고, 



% 우리의 

% 우리는 이벤트와 
% 우리는 sparse한 이벤트만으로 3D 공간에서의 정보를 이동시키기 위해 가상의 