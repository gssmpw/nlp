\section{Related Work}\label{sec:related_work}



While studies have examined testing practices for industrial software \cite{hynninen18,kochhar_practitioners_2019} and mobile apps \cite{vasquez17,Wang2019LookingFT,linares-vasquez_enabling_2017}, this research focuses on the challenges of software testing in \iot. It highlights persistent bugs despite significant time, resources, and testing efforts, as well as developers' practices and perceptions of test case design, automation, and quality metrics.

Research on \iot platforms has explored tools for bug detection \cite{9306923,bosmans_testing_2019,9014711, zhang_trace2tap_2020,brackenbury_how_2019} and highlighted the complexity of testing such systems \cite{8919324, REGGIO2020100313}. Bures \etal \cite{10.1007/978-3-030-58768-0_6} emphasize the need for specialized testing methods tailored to \iot. Surveys focus on bug detection, security failures \cite{9402092,electronics11091502,9878283,manandhar_helion_2019,jin_understanding_2022,AHANGER2022108771,ami_why_2022,makhshari2021}, and future directions. Zhu \etal \cite{zhu_survey_2022} predict trends toward intelligent, large-scale testing, emphasizing big data, cloud computing, and AI. Increased adoption of Smart Home solutions has led to security evaluations, such as Google's Nest and Philips Hue platforms, revealing key vulnerabilities and potential misuse \cite{kafle_study_2019,nest}.


Previous research focuses individually on tools and methods, introducing challenges on specific topics like cloud computing or some security concerns, this study aims to acquire a deeper understanding of the testing methodologies currently employed by \iot developers. We seek to grasp the big picture about how developers are testing \iot platforms and the rationale behind their choices and collect insights regarding potential areas for enhancing future testing practices in \iot. 

