\section{Conclusions}
%现有的CBE方法因其单一的优化目标而无法有效利用宝贵的prference signals. 我们深入分析了提升CBE的关键在于mitigate sampling bias,balance the escent process of uncertainty和抑制 updating uncertainty。在此基础上我们提出UniCBE方法,通过解耦地构建导向上述优化目标的采样矩阵并进行集成,从而获得了更好的accuracy,convergence和scalability. 综合的实验和分析验证了UniCBE的strong effectiveness,improved 可扩展性和良好的泛化性。
The existing comparing-based evaluation methods are ineffective in fully utilizing valuable preference signals due to their constrained optimization objectives. 
Our in-depth analysis reveals that the key to enhancing CBE lies in mitigating sampling bias, balancing the descent process of uncertainty, and suppressing the updating uncertainty.
Based on this, we propose the \textsc{UniCBE} framework that simultaneously optimizes the aforementioned objectives by promoting uniformity in corresponding aspects to enhance accuracy, convergence, and scalability.
Comprehensive experiments and analyses confirm the strong effectiveness, improved scalability, and good generalizability of \textsc{UniCBE}.