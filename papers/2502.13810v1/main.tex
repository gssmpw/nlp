%%%%%%%% ICML 2025 EXAMPLE LATEX SUBMISSION FILE %%%%%%%%%%%%%%%%%

\documentclass{article}

% Recommended, but optional, packages for figures and better typesetting:
\usepackage{microtype}
\usepackage{graphicx}
\usepackage{subfigure}
\usepackage{booktabs} % for professional tables
\usepackage{listings}
\usepackage{quiver}

% hyperref makes hyperlinks in the resulting PDF.
% If your build breaks (sometimes temporarily if a hyperlink spans a page)
% please comment out the following usepackage line and replace
% \usepackage{icml2025} with \usepackage[nohyperref]{icml2025} above.
\usepackage{hyperref}


% Attempt to make hyperref and algorithmic work together better:
\newcommand{\theHalgorithm}{\arabic{algorithm}}

% Use the following line for the initial blind version submitted for review:
% \usepackage{icml2025}

% If accepted, instead use the following line for the camera-ready submission:
\usepackage[accepted]{icml2025}
%\usepackage{icml2025}

% For theorems and such
\usepackage{amsmath}
\usepackage{amsfonts}
\usepackage{amssymb}
\usepackage{mathtools}
\usepackage{amsthm}
\usepackage{enumitem}
\usepackage[subtle]{savetrees}

% if you use cleveref..
\usepackage[capitalize,noabbrev]{cleveref}

%%%%%%%%%%%%%%%%%%%%%%%%%%%%%%%%
% THEOREMS
%%%%%%%%%%%%%%%%%%%%%%%%%%%%%%%%
\theoremstyle{plain}
\newtheorem{theorem}{Theorem}[section]
\newtheorem{proposition}[theorem]{Proposition}
\newtheorem{lemma}[theorem]{Lemma}
\newtheorem{corollary}[theorem]{Corollary}
\theoremstyle{definition}
\newtheorem{definition}[theorem]{Definition}
\newtheorem{assumption}[theorem]{Assumption}
\theoremstyle{remark}
\newtheorem{remark}[theorem]{Remark}

% Todonotes is useful during development; simply uncomment the next line
%    and comment out the line below the next line to turn off comments
%\usepackage[disable,textsize=tiny]{todonotes}
\usepackage[textsize=tiny]{todonotes}


% The \icmltitle you define below is probably too long as a header.
% Therefore, a short form for the running title is supplied here:
\icmltitlerunning{Learning Is a Kan Extension}

\begin{document}

\twocolumn[
\icmltitle{Learning Is a Kan Extension}

% It is OKAY to include author information, even for blind
% submissions: the style file will automatically remove it for you
% unless you've provided the [accepted] option to the icml2025
% package.

% List of affiliations: The first argument should be a (short)
% identifier you will use later to specify author affiliations
% Academic affiliations should list Department, University, City, Region, Country
% Industry affiliations should list Company, City, Region, Country

% You can specify symbols, otherwise they are numbered in order.
% Ideally, you should not use this facility. Affiliations will be numbered
% in order of appearance and this is the preferred way.
\icmlsetsymbol{equal}{*}

\begin{icmlauthorlist}
\icmlauthor{Matthew Pugh}{}
\icmlauthor{Jo Grundy}{}
\icmlauthor{Corina Cirstea}{}
\icmlauthor{Nick Harris}{}
\end{icmlauthorlist}

%\icmlaffiliation{yyy}{Department of XXX, University of YYY, Location, Country}
%\icmlaffiliation{comp}{Company Name, Location, Country}
%\icmlaffiliation{sch}{School of ZZZ, Institute of WWW, Location, Country}

\icmlcorrespondingauthor{Matthew Pugh}{mp8g16@soton.ac.uk}

% You may provide any keywords that you
% find helpful for describing your paper; these are used to populate
% the "keywords" metadata in the PDF but will not be shown in the document
\icmlkeywords{Machine Learning, Kan extension, optimisation theory, category theory, error minimisation, adjunction}

\vskip 0.3in
]

% this must go after the closing bracket ] following \twocolumn[ ...

% This command actually creates the footnote in the first column
% listing the affiliations and the copyright notice.
% The command takes one argument, which is text to display at the start of the footnote.
% The \icmlEqualContribution command is standard text for equal contribution.
% Remove it (just {}) if you do not need this facility.

%\printAffiliationsAndNotice{}  % leave blank if no need to mention equal contribution
%\printAffiliationsAndNotice{\icmlEqualContribution} % otherwise use the standard text.

\begin{abstract}
Previous work has demonstrated that efficient algorithms exist for computing Kan extensions and that some Kan extensions have interesting similarities to various machine learning algorithms. This paper closes the gap by proving that all error minimisation algorithms may be presented as a Kan extension. This result provides a foundation for future work to investigate the optimisation of machine learning algorithms through their presentation as Kan extensions. A corollary of this representation of error-minimising algorithms is a presentation of error from the perspective of lossy and lossless transformations of data.
\end{abstract}

\section{Introduction}
\label{sec:introduction}
The business processes of organizations are experiencing ever-increasing complexity due to the large amount of data, high number of users, and high-tech devices involved \cite{martin2021pmopportunitieschallenges, beerepoot2023biggestbpmproblems}. This complexity may cause business processes to deviate from normal control flow due to unforeseen and disruptive anomalies \cite{adams2023proceddsriftdetection}. These control-flow anomalies manifest as unknown, skipped, and wrongly-ordered activities in the traces of event logs monitored from the execution of business processes \cite{ko2023adsystematicreview}. For the sake of clarity, let us consider an illustrative example of such anomalies. Figure \ref{FP_ANOMALIES} shows a so-called event log footprint, which captures the control flow relations of four activities of a hypothetical event log. In particular, this footprint captures the control-flow relations between activities \texttt{a}, \texttt{b}, \texttt{c} and \texttt{d}. These are the causal ($\rightarrow$) relation, concurrent ($\parallel$) relation, and other ($\#$) relations such as exclusivity or non-local dependency \cite{aalst2022pmhandbook}. In addition, on the right are six traces, of which five exhibit skipped, wrongly-ordered and unknown control-flow anomalies. For example, $\langle$\texttt{a b d}$\rangle$ has a skipped activity, which is \texttt{c}. Because of this skipped activity, the control-flow relation \texttt{b}$\,\#\,$\texttt{d} is violated, since \texttt{d} directly follows \texttt{b} in the anomalous trace.
\begin{figure}[!t]
\centering
\includegraphics[width=0.9\columnwidth]{images/FP_ANOMALIES.png}
\caption{An example event log footprint with six traces, of which five exhibit control-flow anomalies.}
\label{FP_ANOMALIES}
\end{figure}

\subsection{Control-flow anomaly detection}
Control-flow anomaly detection techniques aim to characterize the normal control flow from event logs and verify whether these deviations occur in new event logs \cite{ko2023adsystematicreview}. To develop control-flow anomaly detection techniques, \revision{process mining} has seen widespread adoption owing to process discovery and \revision{conformance checking}. On the one hand, process discovery is a set of algorithms that encode control-flow relations as a set of model elements and constraints according to a given modeling formalism \cite{aalst2022pmhandbook}; hereafter, we refer to the Petri net, a widespread modeling formalism. On the other hand, \revision{conformance checking} is an explainable set of algorithms that allows linking any deviations with the reference Petri net and providing the fitness measure, namely a measure of how much the Petri net fits the new event log \cite{aalst2022pmhandbook}. Many control-flow anomaly detection techniques based on \revision{conformance checking} (hereafter, \revision{conformance checking}-based techniques) use the fitness measure to determine whether an event log is anomalous \cite{bezerra2009pmad, bezerra2013adlogspais, myers2018icsadpm, pecchia2020applicationfailuresanalysispm}. 

The scientific literature also includes many \revision{conformance checking}-independent techniques for control-flow anomaly detection that combine specific types of trace encodings with machine/deep learning \cite{ko2023adsystematicreview, tavares2023pmtraceencoding}. Whereas these techniques are very effective, their explainability is challenging due to both the type of trace encoding employed and the machine/deep learning model used \cite{rawal2022trustworthyaiadvances,li2023explainablead}. Hence, in the following, we focus on the shortcomings of \revision{conformance checking}-based techniques to investigate whether it is possible to support the development of competitive control-flow anomaly detection techniques while maintaining the explainable nature of \revision{conformance checking}.
\begin{figure}[!t]
\centering
\includegraphics[width=\columnwidth]{images/HIGH_LEVEL_VIEW.png}
\caption{A high-level view of the proposed framework for combining \revision{process mining}-based feature extraction with dimensionality reduction for control-flow anomaly detection.}
\label{HIGH_LEVEL_VIEW}
\end{figure}

\subsection{Shortcomings of \revision{conformance checking}-based techniques}
Unfortunately, the detection effectiveness of \revision{conformance checking}-based techniques is affected by noisy data and low-quality Petri nets, which may be due to human errors in the modeling process or representational bias of process discovery algorithms \cite{bezerra2013adlogspais, pecchia2020applicationfailuresanalysispm, aalst2016pm}. Specifically, on the one hand, noisy data may introduce infrequent and deceptive control-flow relations that may result in inconsistent fitness measures, whereas, on the other hand, checking event logs against a low-quality Petri net could lead to an unreliable distribution of fitness measures. Nonetheless, such Petri nets can still be used as references to obtain insightful information for \revision{process mining}-based feature extraction, supporting the development of competitive and explainable \revision{conformance checking}-based techniques for control-flow anomaly detection despite the problems above. For example, a few works outline that token-based \revision{conformance checking} can be used for \revision{process mining}-based feature extraction to build tabular data and develop effective \revision{conformance checking}-based techniques for control-flow anomaly detection \cite{singh2022lapmsh, debenedictis2023dtadiiot}. However, to the best of our knowledge, the scientific literature lacks a structured proposal for \revision{process mining}-based feature extraction using the state-of-the-art \revision{conformance checking} variant, namely alignment-based \revision{conformance checking}.

\subsection{Contributions}
We propose a novel \revision{process mining}-based feature extraction approach with alignment-based \revision{conformance checking}. This variant aligns the deviating control flow with a reference Petri net; the resulting alignment can be inspected to extract additional statistics such as the number of times a given activity caused mismatches \cite{aalst2022pmhandbook}. We integrate this approach into a flexible and explainable framework for developing techniques for control-flow anomaly detection. The framework combines \revision{process mining}-based feature extraction and dimensionality reduction to handle high-dimensional feature sets, achieve detection effectiveness, and support explainability. Notably, in addition to our proposed \revision{process mining}-based feature extraction approach, the framework allows employing other approaches, enabling a fair comparison of multiple \revision{conformance checking}-based and \revision{conformance checking}-independent techniques for control-flow anomaly detection. Figure \ref{HIGH_LEVEL_VIEW} shows a high-level view of the framework. Business processes are monitored, and event logs obtained from the database of information systems. Subsequently, \revision{process mining}-based feature extraction is applied to these event logs and tabular data input to dimensionality reduction to identify control-flow anomalies. We apply several \revision{conformance checking}-based and \revision{conformance checking}-independent framework techniques to publicly available datasets, simulated data of a case study from railways, and real-world data of a case study from healthcare. We show that the framework techniques implementing our approach outperform the baseline \revision{conformance checking}-based techniques while maintaining the explainable nature of \revision{conformance checking}.

In summary, the contributions of this paper are as follows.
\begin{itemize}
    \item{
        A novel \revision{process mining}-based feature extraction approach to support the development of competitive and explainable \revision{conformance checking}-based techniques for control-flow anomaly detection.
    }
    \item{
        A flexible and explainable framework for developing techniques for control-flow anomaly detection using \revision{process mining}-based feature extraction and dimensionality reduction.
    }
    \item{
        Application to synthetic and real-world datasets of several \revision{conformance checking}-based and \revision{conformance checking}-independent framework techniques, evaluating their detection effectiveness and explainability.
    }
\end{itemize}

The rest of the paper is organized as follows.
\begin{itemize}
    \item Section \ref{sec:related_work} reviews the existing techniques for control-flow anomaly detection, categorizing them into \revision{conformance checking}-based and \revision{conformance checking}-independent techniques.
    \item Section \ref{sec:abccfe} provides the preliminaries of \revision{process mining} to establish the notation used throughout the paper, and delves into the details of the proposed \revision{process mining}-based feature extraction approach with alignment-based \revision{conformance checking}.
    \item Section \ref{sec:framework} describes the framework for developing \revision{conformance checking}-based and \revision{conformance checking}-independent techniques for control-flow anomaly detection that combine \revision{process mining}-based feature extraction and dimensionality reduction.
    \item Section \ref{sec:evaluation} presents the experiments conducted with multiple framework and baseline techniques using data from publicly available datasets and case studies.
    \item Section \ref{sec:conclusions} draws the conclusions and presents future work.
\end{itemize}
\section{Background}\label{sec:backgrnd}

\subsection{Cold Start Latency and Mitigation Techniques}

Traditional FaaS platforms mitigate cold starts through snapshotting, lightweight virtualization, and warm-state management. Snapshot-based methods like \textbf{REAP} and \textbf{Catalyzer} reduce initialization time by preloading or restoring container states but require significant memory and I/O resources, limiting scalability~\cite{dong_catalyzer_2020, ustiugov_benchmarking_2021}. Lightweight virtualization solutions, such as \textbf{Firecracker} microVMs, achieve fast startup times with strong isolation but depend on robust infrastructure, making them less adaptable to fluctuating workloads~\cite{agache_firecracker_2020}. Warm-state management techniques like \textbf{Faa\$T}~\cite{romero_faa_2021} and \textbf{Kraken}~\cite{vivek_kraken_2021} keep frequently invoked containers ready, balancing readiness and cost efficiency under predictable workloads but incurring overhead when demand is erratic~\cite{romero_faa_2021, vivek_kraken_2021}. While these methods perform well in resource-rich cloud environments, their resource intensity challenges applicability in edge settings.

\subsubsection{Edge FaaS Perspective}

In edge environments, cold start mitigation emphasizes lightweight designs, resource sharing, and hybrid task distribution. Lightweight execution environments like unikernels~\cite{edward_sock_2018} and \textbf{Firecracker}~\cite{agache_firecracker_2020}, as used by \textbf{TinyFaaS}~\cite{pfandzelter_tinyfaas_2020}, minimize resource usage and initialization delays but require careful orchestration to avoid resource contention. Function co-location, demonstrated by \textbf{Photons}~\cite{v_dukic_photons_2020}, reduces redundant initializations by sharing runtime resources among related functions, though this complicates isolation in multi-tenant setups~\cite{v_dukic_photons_2020}. Hybrid offloading frameworks like \textbf{GeoFaaS}~\cite{malekabbasi_geofaas_2024} balance edge-cloud workloads by offloading latency-tolerant tasks to the cloud and reserving edge resources for real-time operations, requiring reliable connectivity and efficient task management. These edge-specific strategies address cold starts effectively but introduce challenges in scalability and orchestration.

\subsection{Predictive Scaling and Caching Techniques}

Efficient resource allocation is vital for maintaining low latency and high availability in serverless platforms. Predictive scaling and caching techniques dynamically provision resources and reduce cold start latency by leveraging workload prediction and state retention.
Traditional FaaS platforms use predictive scaling and caching to optimize resources, employing techniques (OFC, FaasCache) to reduce cold starts. However, these methods rely on centralized orchestration and workload predictability, limiting their effectiveness in dynamic, resource-constrained edge environments.



\subsubsection{Edge FaaS Perspective}

Edge FaaS platforms adapt predictive scaling and caching techniques to constrain resources and heterogeneous environments. \textbf{EDGE-Cache}~\cite{kim_delay-aware_2022} uses traffic profiling to selectively retain high-priority functions, reducing memory overhead while maintaining readiness for frequent requests. Hybrid frameworks like \textbf{GeoFaaS}~\cite{malekabbasi_geofaas_2024} implement distributed caching to balance resources between edge and cloud nodes, enabling low-latency processing for critical tasks while offloading less critical workloads. Machine learning methods, such as clustering-based workload predictors~\cite{gao_machine_2020} and GRU-based models~\cite{guo_applying_2018}, enhance resource provisioning in edge systems by efficiently forecasting workload spikes. These innovations effectively address cold start challenges in edge environments, though their dependency on accurate predictions and robust orchestration poses scalability challenges.

\subsection{Decentralized Orchestration, Function Placement, and Scheduling}

Efficient orchestration in serverless platforms involves workload distribution, resource optimization, and performance assurance. While traditional FaaS platforms rely on centralized control, edge environments require decentralized and adaptive strategies to address unique challenges such as resource constraints and heterogeneous hardware.



\subsubsection{Edge FaaS Perspective}

Edge FaaS platforms adopt decentralized and adaptive orchestration frameworks to meet the demands of resource-constrained environments. Systems like \textbf{Wukong} distribute scheduling across edge nodes, enhancing data locality and scalability while reducing network latency. Lightweight frameworks such as \textbf{OpenWhisk Lite}~\cite{kravchenko_kpavelopenwhisk-light_2024} optimize resource allocation by decentralizing scheduling policies, minimizing cold starts and latency in edge setups~\cite{benjamin_wukong_2020}. Hybrid solutions like \textbf{OpenFaaS}~\cite{noauthor_openfaasfaas_2024} and \textbf{EdgeMatrix}~\cite{shen_edgematrix_2023} combine edge-cloud orchestration to balance resource utilization, retaining latency-sensitive functions at the edge while offloading non-critical workloads to the cloud. While these approaches improve flexibility, they face challenges in maintaining coordination and ensuring consistent performance across distributed nodes.


%\begin{figure}[h]
\begin{center}
   \includegraphics[width=0.99\linewidth]{figs/pdf/loss.pdf}
\end{center}
   \caption{
    Training loss of VAR \textit{vs.} FlexVAR. FlexVAR demonstrates a faster convergence rate. We report the results with trained 40 epochs ($\sim$ 70K iterations). 
   }
\label{fig:loss}
\end{figure}


\begin{lemma}\label{Lemma:multi1} 
   Fixing the number of data contributor $i$ collects $n_i$, and others' strategies $\strategy_{-i}$, $\hat{\mu}\left(X_i\right)$ is the minimax estimator for the Normal distribution class $\Normaldistrib := \left\{\mathcal{N}(\mu,\sigma^2) \;\middle|\; \mu \in \mathbb{R}\right\}$,
    \begin{align*}
       \hat{\mu}(X_i)  = \underset{\hat{\mu}}{\arg\min} \sbr{\sup _\mu \mathbb{E}\left[(\hat{\mu}( Y_i)- \hat{\mu}( Y_{-i}) )^2 \;\middle|\;  \mu \right] }
    \end{align*} 
     
\end{lemma}


\begin{proof}

\begin{align*}
    & \ \mathbb{E}\left[ \left( \hat{\mu}\left( Y_i \right)-\hat{\mu}\left( Y_{-i} \right)  \right)^2 \right] \\ =  & \ \mathbb{E}\left[ \left( (\hat{\mu}\left(  Y_i \right)-\mu) -(\hat{\mu}\left(  Y_{-i} \right) -\mu) \right)^2   \right] \\ =  & \ A_0 + \mathbb{E}\left[ (\hat{\mu}\left(  Y_i \right)-\mu)^2  \right]
\end{align*}
where $A_0$ is a positive coefficient.

Thus the maximum risk can be written as:

\begin{align*}
    \sup _\mu \mathbb{E}\left[A_0 + \left(\hat{\mu}\left( Y_i\right)-\mu\right)^{2} \;\middle|\;  \mu \right]
\end{align*}


We construct a lower bound on the maximum risk using a sequence of Bayesian risks. Let $\Lambda_{\ell}:=\mathcal{N}\left(0, \ell^2\right), \ell=1,2, \ldots$ be a sequence of prior for $\mu$. For fixed $\ell$, the posterior distribution is:
$$
\begin{aligned}
p\left(\mu \;\middle|\;  X_i\right) & \propto p\left(X_i \;\middle|\;  \mu\right) p(\mu) \\ & \propto \exp \left(-\frac{1}{2 \sigma^2} \sum_{x \in X_i}(x-\mu)^2\right) \exp \left(-\frac{1}{2 \ell^2} \mu^2\right) \\
& \propto \exp \left(-\frac{1}{2}\left(\frac{n_i}{\sigma^2}+\frac{1}{\ell^2}\right) \mu^2+\frac{1}{2} 2 \frac{\sum_{x \in X_i} x}{\sigma^2} \mu\right) .
\end{aligned}
$$

This means the posterior of $\mu$ given $X_i$ is Gaussian with:

\begin{align*}
    \mu \lvert\, X_i & \sim \mathcal{N}\left(\frac{n_i \hat{\mu}\left(X_i\right) / \sigma^2}{n_i / \sigma^2+1 / \ell^2}, \frac{1}{n_i / \sigma^2+1 / \ell^2}\right) 
    \\ & =: \mathcal{N}\left(\mu_{\ell}, \sigma_{\ell}^2\right).
\end{align*}



Therefore, the posterior risk is: 
$$
\begin{aligned}
&   \mathbb{E}\left[A_0 + \left(\hat{\mu}\left( Y_i\right)-\mu\right)^{2}  \;\middle|\;  X_i\right] \\ = &  \mathbb{E}\left[A_0 +  \left(\left(\hat{\mu}\left( Y_i\right)-\mu_{\ell}\right)-\left(\mu-\mu_{\ell}\right)\right)^{2 j} \;\middle|\;  X_i\right] \\ =
& A_0+\int_{-\infty}^{\infty} \underbrace{\left(e-\left(\hat{\mu}\left( Y_i\right)-\mu_{\ell}\right)\right)^2}_{=: F_1\left(e-\left(\hat{\mu}\left(Y_i\right)-\mu_{\ell}\right)\right)} \underbrace{\frac{1}{\sigma_{\ell} \sqrt{2 \pi}} \exp \left(-\frac{e^2}{2 \sigma_{\ell}^2}\right)}_{=: F_2(e)} d e
\end{aligned}
$$

Because:
\begin{itemize}
    \item $F_1(\cdot)$ is even function and increases on $[0, \infty)$;
    \item $F_2(\cdot)$ is even function and decreases on $\left[0, \infty \right)$, and $\int_{\mathbb{R}} F_2(e) de<\infty$
    \item For any $a \in \mathbb{R}, \int_{\mathbb{R}} F_1(e-a) F_2(e) de<\infty$
\end{itemize}

By the corollary of Hardy-Littlewood inequality in Lemma \ref{lemmaHardy},
$$
\int_{\mathbb{R}} F_1(e-a) F_2(e) d e \geq \int_{\mathbb{R}} F_1(e) F_2(e) d e
$$
which means the posterior risk is minimized when $\hat{\mu}\left(Y_i\right)=\mu_{\ell}$. We then write the Bayes risk as, the Bayes risk is minimized by the posterior mean $\mu_{\ell}$:

\begin{align*}    
R_{\ell}:= & \mathbb{E}\left[ A_0+\mathbb{E}\left[\left(\mu-\mu_{\ell}\right)^{2 } \;\middle|\;  X_i\right]\right] \\ = & A_0 + \sigma_{\ell}^{2}
\end{align*}

and the limit of Bayesian risk as $\ell \rightarrow \infty$ is
$$
R_{\infty}:= A_0 + \frac{\sigma^{2}}{n_i}.
$$

When $\hat{\mu}\left(Y_i\right)=\hat{\mu}\left(X_i\right)$, i.e, the contributor submit a set of size $n_i$ with each element equal to $ \hat{\mu}\left(X_i\right)$, the maximum risk is:

\begin{align*}
& \sup _\mu \mathbb{E}\left[A_0+\left(\mu- 
\hat{\mu}\left(Y_i\right) \right)^{2 } \;\middle|\;  \mu \right] \\
= & \sup _\mu \mathbb{E}\left[A_0+\left(\mu- 
\hat{\mu}\left(X_i\right) \right)^{2 } \;\middle|\;  \mu \right]  \\
= & A_0+ \sigma^{2 } n_i^{-1}  \\
= &  R_{\infty}.
\end{align*}

This implies that,
\begin{align*}
    & \underset{\mu}{\sup}\; \mathbb{E} \sbr{ \rbr{\hat{\mu}\left( Y_i \right)-\hat{\mu}\left( Y_{-i} \right)  }^2 \;\middle|\;  \mu }  \\ \geq & \; R_{\infty} =  \sup _\mu \; \mathbb{E}\left[A_0+\left(\mu- 
\hat{\mu}\left(X_i\right) \right)^{2 } \;\middle|\;  \mu \right]
\end{align*}

Therefore, the recommended strategy $\hat{\mu}(Y_i) =\hat{\mu}( X_i)$ has a smaller maximum risk than other strategies. 

\end{proof}




1. The payment from the buyer a constant $v(n^{\star})$.


2. If the payment for every seller is a fixed constant, then sellers can fabricate data without actually collecting data.\\


%$p_1 = b/2 +(\hat{\mu}(Y_1)- \hat{\mu}(Y_2))^2$, $p_2 = b/2 -(\hat{\mu}(Y_1)- \hat{\mu}(Y_2))^2$, seller 1 can choose ${\mu}' = u + \epsilon$, expected payment for seller 1 is larger than $b/2$. %NIC for seller 1: $g({\mu}',\mu ) < b/2$ for all ${\mu}' \neq \mu$. NIC for seller 2: $g( \mu, {\mu}') >  b/2 $ for all ${\mu}' \neq \mu$.

To demonstrate that no truthful mechanism (NIC) satisfies all desired properties in a two-seller setting, we use proof by contradiction.  

Suppose that there is a NIC mechanism $M$ satisfying property 1-5. Under this mechanism, the best strategy for each seller is to collect $N_i^{\star}$ amount of data and submit truthfully, where $N_1^{\star}+N_2^{\star} = n^{\star} $. Since $M$ is NIC for strategy space $\left\{ (f_i,N_i)\right\}_{i=1,2}$, it must be NIC for the sub strategy space $\left\{ (f_i, N_i^{\star})\right\}_{i=1,2}$. 


Consider the case in which everyone collects $N_i^{\star}$ data point and submits $N_i^{\star}$ data point. Assume that the true mean is $\mu$, seller $1$ submit $N({\mu}', \sigma^2),\ {\mu}' = f(\mu) $ while seller 2 submit $N({\mu}, \sigma^2) $. We denote seller 1's expected payment as  $\mathbb{E}\left[ p_1(M,\strategy) \right] = g({\mu}', \mu)$.
Seller 1's utility is then:
\[ u_1(M,f) = \underset{\mu}{\inf} \ g({\mu}', \mu) -c\times N_1^{\star}\] where $c$ is the cost for collecting one data point.


The total payment from the buyer is $v(n^{\star})$, hence by budget balance, \[p_2 (M,f)  = v(n^{\star}) -  p_1(M,f), \ \mathbb{E}\left[ p_2(M,f) \right] = v(n^{\star}) - g({\mu}', \mu) \]

By NIC, we have,
\[ \underset{\mu}{\inf} \ g({\mu}', \mu) -c\times N_1^{\star} \leq \underset{\mu}{\inf} \ g({\mu}, \mu) -c\times N_1^{\star} \] \[ \underset{\mu}{\inf} \ (v(n^{\star})- g( \mu, {\mu}')) -c\times N_2^{\star} \leq \underset{\mu}{\inf} \ (v(n^{\star})-g( \mu, {\mu})) -c\times N_2^{\star}  \]

%Using the fact that $\underset{\mu}{\inf} \ g({\mu}, \mu) = \underset{\mu}{\sup} \ g({\mu}, \mu) = {v(n^{\star})}/2 $.
We obtain that for any ${\mu}'$ and $\mu$,
\[  \underset{\mu}{\inf} \ g({\mu}', \mu) \leq \underset{\mu}{\inf} \ g({\mu}, \mu)   \] \[  \underset{\mu}{\sup} \  g( \mu, {\mu})  \leq \underset{\mu}{\sup } \ g( \mu, {\mu}')  \]

We next show that the inequalities are strict. Assume, for contradiction there exists ${\mu}'$, for any $\mu$, $g({\mu}', 
\mu) \geq \underset{\mu}{\inf} \ g({\mu}, \mu)$. It then follows that $ \underset{\mu}{\inf} \ g({\mu}', \mu) \geq \underset{\mu}{\inf} \ g({\mu}, \mu)$. Under this assumption, seller 1 could fabricate data by submitting $N({\mu}', \sigma^2)$ without collecting any actual data. This contradicts with the fact that $(f_1 = I, N_1 = N_1^{\star})$ is the best strategy for seller 1. Hence, for any ${\mu}'$, there exists some $\mu$ such that $g({\mu}', 
\mu) < \underset{\mu}{\inf} \ g({\mu}, \mu)$. Therefore, for any ${\mu}'$, \[ \underset{\mu}{\inf} \ g({\mu}', \mu) < \underset{\mu}{\inf} \ g({\mu}, \mu). \]Similarly, we also have \[ \underset{\mu}{\sup} \  g( \mu, {\mu})  < \underset{\mu}{\sup } \ g( \mu, {\mu}').  \] 


For any $ {\mu}'$, let $f({\mu}') =  \underset{\mu}{\arg\sup}\, g(\mu, {\mu}')$, then we have for any ${\mu}'$,  $f({\mu}') \neq {\mu}'$ and $ g(f({\mu}'), {\mu}') > \underset{\mu}{\sup} \  g( \mu, {\mu}) \geq  \underset{\mu}{\inf} \  g( \mu, {\mu})$. This implies that seller 1 could fabricate data based on function $f$, this contradicts with the fact that the mechanism is NIC.


pay the seller $v(n^{\star})/2 - \beta (\hat{\mu}(Y_1)-\hat{\mu}(Y_2))^2$, charge buyer $v(n^{\star}) - 2\beta (\hat{\mu}(Y_1)-\hat{\mu}(Y_2))^2$


Buyer utility: $v(n^*)$-payment
Seller utility: payment - $cn^*$.

Sellers utility is positive?

Seller payment $(v(n^*)/2)-\beta (\hat{\mu}(Y_1)-\hat{\mu}(Y_2))^2 $, $\beta = (v(n^*)-cn^*)c(n^*)^2 / 4\sigma^2$, 



\section{Multiple buyers}
\subsection{}
Question 1: Do we fix the amount of data for sale ahead of time?


Assume we fix $N$, the amount of data for sale. The goal of mechanism is to maximize the sellers' revenue. According to previous paper, there exists at least one type who purchase at the amount $N$. Suppose that in offline setting, i.e., when the mechanism knows the buyer valuation and type distribution, the optimal revenue is $\text{OPT} $. 


We ask $d$ sellers to collect $N$ data points, and split $\text{OPT} $ revenue among sellers. (data can be duplicated). 

\[ p_i(M,s) =\mathbb{I}\left( \left| Y_i \right| = \frac{N}{d} \right) \rbr{\frac{\text{OPT}}{d}+d_i \frac{\sigma^2}{N_{-i}^{\star}} +d_i \frac{\sigma^2}{N_i^{\star}} }- d_i \rbr{\hat{\mu}(Y_i)-\hat{\mu}(Y_{-i}) }^2  \]

Buyer's expected utility is non negative. Next, we discuss sellers' expected utility $T\mathbb{E}[p_i]- cn_i$ (over $T$ roundsm\, maybe $T$ is fixed). Let $N_i^{\star} = \frac{N}{d}$.

\[ u_i(M,s) = \mathbb{I}\left( \left| Y_i \right| = \frac{N}{d} \right) \rbr{\frac{\text{OPT}}{d}+d_i \frac{\sigma^2}{N_{-i}^{\star}} +d_i \frac{\sigma^2}{N_i^{\star}} }T- Td_i \mathbb{E}\rbr{\hat{\mu}(Y_i)-\hat{\mu}(Y_{-i}) }^2 
 -  cn_i \]
Choose $d_i = \frac{c(N_i^{\star})^2}{T\sigma^2}$.


If we do not fix \( T \) in advance, let \( T_0 \) represent the time at which the cumulative utility over at least \( T_0 \) rounds is non-negative. We can select \( d_i \) such that \( d_i \geq \frac{c(N/d)^2}{T_0 \sigma^2} \). This ensures that the seller will never choose to collect less than \( N/d \) amount of data.








\section{Single buyer} \label{section: singlebuyer}


Each contributor \( i \) incurs a cost \( c_i \) to collect data,  without loss of generality, we assume \( c_1 \leq c_2 \leq \dots \leq c_d \). The broker is assumed to have full knowledge of the buyer's valuation curve \( \val(n) \), as well as the contributors’ costs $ c_{ i \in \contributors}$  for collecting each data point.

The maximum total profit for the contributors, assuming no constraints on truthful submissions, is given by:
\[
\mathrm{profit}^\star = \underset{\datanum_1, \dots, \datanum_d}{\max} \left( \val \left(\sum_{i=1}^{d} \datanum_i\right) - \sum_{i=1}^{d} c_i \datanum_i \right),
\]

where \( \datanum_i \) represents the number of data points collected by contributor \( i \). In this unconstrained scenario, since contributor 1 has the lowest collection cost, the optimal strategy is for contributor 1 to collect all the required data points while other contributors collect none. This approach maximizes total profit without considering the incentive for truthful submissions.

However, when truthful submission is taken into account, at least two contributors are needed because we need to use one contributor's data to verify the other's. We demonstrate that the maximum profit achievable under Nash Equilibrium is:
\[
\mathrm{profit}^\star + (c_1 - c_2),
\]
where \( c_1 - c_2 \) represents the additional cost differential caused by enforcing truthful behavior among contributors. 

\begin{algorithm}[H]
    \caption{Process of mechanism.}
    \begin{algorithmic}
        \STATE {\bfseries Input:} A population of buyers $\buyers$.
        \STATE The broker chooses the optimal data allocation to maximize contributors' profit:
        $$
        \{ \datanum_i^{\star} \}_{i=1}^d = \underset{\datanum_1,\dots,\datanum_d}{\arg\max}\  \rbr{v\rbr{\sum_i \datanum_i}-\sum_i \cost_i \datanum_i  }
      $$
       
        \STATE The broker recommend a strategy to each contributor: $\strategy_i^{\star} = (\datanum_i^{\star}, \mathbf{I})$.
        \STATE Each contributor selects a strategy $\strati = (\datanum_i, f_i)$, collects $\datanum_i$ data points $X_i$, and submits $Y_i = f_i(X_i)$.
        \STATE The mechanism generates an estimator $\hat{\mu}(M,\strategy)$ for the buyer, and charge her $\price_{j \in \buyers}$. \COMMENT{See (\ref{eq:buyer_pay}) }
        \STATE Each contributor is paid $\payi$.    \COMMENT{See (\ref{eq:seller_pay}) }
    \end{algorithmic}   
\end{algorithm}




\begin{theorem}
    there exists NIC mechanism satisfying the following properties (1) $\strategy^{\star}$ is Nash equilibrium. (2) The mechanism is individually rational at $\strategy^{\star}$ for both buyers and sellers. (3) Budget balance. (4) Under strategy $\strategy^{\star}$, the expected profit of buyers approximates the optimal profit $ \mathrm{profit}^{\star}$ within an additive error $\cost_2 - \cost_1$. 
\end{theorem}


Let $n^{\star}$ denote the optimal total number of data to be collected, $\datanum_1^{\star}=n^{\star}-1$, and $\datanum_2^{\star}=1$. Let $w=\val(n^{\star})-cn^{\star}$ denote the social welfare. One option for payment function is

\begin{align*}
    &\; \pay_i(M,\strategy^{\star}) \\  
    = & \;\mathbb{I}\left( \left| Y_i \right| = \datanum_i^{\star} \right) \rbr{\frac{\datanum_1^{\star}}{n^{\star}}\val(n^{\star})+d_i \frac{\sigma^2}{\datanum_{-i}^{\star}} +d_i \frac{\sigma^2}{\datanum_i^{\star}} } \\ & - d_i \rbr{\hat{\mu}(Y_i)-\hat{\mu}(Y_{-i}) }^2, \\[20pt] % Adds vertical space between equations
    &\; \price(M,\strategy^{\star}) \\  
    = & \; \sum_{i=1}^{2}\mathbb{I}\left(  \left| Y_i \right| = \datanum_i^{\star} \right) \rbr{\frac{\datanum_1^{\star}}{n^{\star}}\val(n^{\star}) +d_i \frac{\sigma^2}{\datanum_{-i}^{\star}} +d_i \frac{\sigma^2}{\datanum_i^{\star}} } \\ 
    & - \sum_{i=1}^{2} d_i \rbr{\hat{\mu}(Y_i)-\hat{\mu}(Y_{-i}) }^2, \\[20pt] % Adds vertical space between equations
    &\;  \utilityb (M,\strategy^{\star}) \\ 
   = & \; v(\datanum^{\star}) -\mathbb{E}[\price(M,\strategy^{\star})] \\ 
    = & \; - \sum_{i=1}^{d} \rbr{d_i \frac{\sigma^2}{\datanum_{-i}^{\star}} +d_i \frac{\sigma^2}{\datanum_i^{\star}}  } + \sum_{i=1}^{d}d_i \mathbb{E} \rbr{\hat{\mu}(Y_i)-\hat{\mu}(Y_{-i}) }^2 \\ 
    = & \; 0.
\end{align*}



Then contributors i's expected ptofit under strategy $\strategy^{\star}$ is 

\begin{align*}
& \; \utilci \rbr{\mechspace, \strategy^{\star} } \\ = &  \; \mathbb{E}\sbr{\pay_i(M,\strategy^{\star})} - \cost_i n_i^{\star} \\ = &  \; \mathbb{I}\left(\left| Y_i \right| = \datanum_i^{\star} \right) \rbr{\frac{\datanum_1^{\star}}{n^{\star}}\val(n^{\star})  +d_i \frac{\sigma^2}{\datanum_{-i}^{\star}} +d_i \frac{\sigma^2}{\datanum_i^{\star}} }\\  & -  d_i \mathbb{E}\rbr{\hat{\mu}(Y_i)-\hat{\mu}(Y_{-i}) }^2  -\cost n_i^{\star} \\ = &  \;  \frac{w}{\numcontributors}
\end{align*}
where $d_i = c(\datanum_i^*/d)^2 $, 

%\textcolor{red}{Buyer payment $\pi(M,s)$ can ve negative? Can it be interpreted as when the quality of data is bad, the mechanism pays money to the buyer as compensate, the contributor pays money to the mechanism as a penalty. }\textcolor{red}{Buyer pays $v(n^*)$, $p_i = v(n^*) \frac{\rbr{\hat{\mu}(Y_i)-\hat{\mu}(Y_{-i}) }^{-2}}{\sum{\rbr{\hat{\mu}(Y_i)-\hat{\mu}(Y_{-i}) }^{-2}}}$ }


We prove the NIC in three steps.


\textbf{First step} \textcolor{red}{to be fixed}: Giving others submitting truthfully, we know that when fixing $n_i$, submitting $\left| Y_i \right| = \frac{n^{\star}}{d}$ is the best strategy, otherwise, $\pay_i <0$ when $\left| Y_i \right| \neq \frac{n^{\star}}{d}$. Therefore, for any $n_i$ and $f_i$, we have for any $\mu$,
\begin{align*}
& u_i\rbr{\mechspace, (n_i,f_i, \left| Y_i \right| =\datanum_i^{\star}),\strategy_{-i}^{\star} } \\  \geq \  & u_i\rbr{\mechspace, \strategy_{-i}^{\star}} 
\end{align*}

\textbf{Second step}: Fixing $n_i$ and $\left| Y_i \right|$, sample mean $\hat{\mu}(X_i)$ is minimax estimator of $\mathbb{E}\rbr{\rbr{\hat{\mu}(Y_i)-\hat{\mu}(Y_{-i})}^2 \;\middle|\; P} $, i.e., \[ \hat{\mu}(X_i) = \underset{\hat{\mu}}{\inf} \  \underset{\distrifamily}{\sup}\ \mathbb{E}\sbr{\rbr{\hat{\mu}(Y_i)-\hat{\mu}(Y_{-i})}^2  \;\middle|\; \distri} \]Therefore we have 
\begin{align*}
     & \underset{\distrifamily}{\inf}\;u_i\rbr{\mechspace, (n_i,\hat{\mu}(Y_i)=\hat{\mu}(X_i), \left| Y_i \right|),\strategy_{-i}^{\star} } \\  \geq \  & \underset{\distrifamily}{\inf}\;u_i\rbr{\mechspace, (n_i,\hat{\mu}(Y_i), \left| Y_i \right|),\strategy_{-i}^{\star} } 
\end{align*}


\textbf{Third step}: By setting constant $d_i= c\rbr{\frac{\datanum_i^{\star}}{\sigma}}^2$, when fixing $\hat{\mu}(Y_i)=\hat{\mu}(X_i)$ and $\left| Y_i \right| = \datanum_i^{\star}$, collecting $n_i = \datanum_i^{\star}$ amount of data maximize the contributor utility 
\begin{align*}
    \underset{\distrifamily}{\inf}\;u_i\rbr{\mechspace, (n_i= \datanum_i^{\star},\hat{\mu}(Y_i)=\hat{\mu}(X_i), \left| Y_i \right|=\strategy_{-i}^{\star} } \\ \geq \underset{\distrifamily}{\inf}\;u_i\rbr{\mechspace, (n_i,\hat{\mu}(Y_i)=\hat{\mu}(X_i), \left| Y_i \right|=s_{-i}^{\star}}  
\end{align*}
 

\begin{align*}
    & u_i\rbr{\mechspace, (n_i,\hat{\mu}(Y_i)=\hat{\mu}(X_i), \left| Y_i \right|=\datanum_i^{\star}),\strategy_{-i}^{\star} }  \\ = & \rbr{\frac{w}{\numcontributors}+ c \datanum_i^{\star} +d_i \frac{\sigma^2}{\datanum_{-i}^{\star}} +d_i \frac{\sigma^2}{\datanum_i^{\star}} } - d_i \rbr{ \frac{\sigma^2}{\datanum_{-i}^{\star}} +\frac{\sigma^2}{\datanum_i^{\star}} } - cn_i
\end{align*}

Therefore, we have 
\begin{align*}
    &\underset{\distrifamily}{\inf}\;u_i \rbr{\mechspace, (n_i= \datanum_i^{\star},\hat{\mu}(Y_i)=\hat{\mu}(X_i), \left| Y_i \right|=\datanum_i^{\star}),\strategy_{-i}^{\star} } \\  = & \underset{\distrifamily}{\inf}\;u_i\rbr{\mechspace, (\datanum_i^{\star},f_i^{\star}),\strategy_{-i}^{\star} } \\ \geq &   \underset{\distrifamily}{\inf}\; u_i\rbr{\mechspace, (n_i,f_i ),\strategy_{-i}^{\star}} 
\end{align*}

When following the best strategy, properties 1-5 are all satisfied.



\section{Universal representation}

It has been shown that Kan extensions represent global error minimisers for category-theoretic error minimisation problems, but it may not be clear that this also applies to the set-theoretic error minimisation problem. It is actually possible to convert any set-theoretic error minimisation problem into a category-theoretic error minimisation problem, namely as an extension problem in a 2-category, such that the left Kan extensions of the extension problem are exactly the global error minimisers of the set-theoretic error minimisation problem.

\begin{theorem}[Machine Learning representation]
\label{theorem:universal_ml_representation}
Given a set theoretic error minimisation problem (Def \ref{definition:set_error}) there exists a 2-category $\mathbb{T}$ such that $M = \mathbb{T}(\mu, \tau)$, $D = \mathbb{T}(\delta, \tau)$, $Inf = \mathbb{T}(\iota, \tau)$ and an object $m \in M$ is a global error minimiser with respect to $d$ if and only if $m \cong Lan_\iota d$
\label{prop:mlrepresentation}

\end{theorem}

\begin{proof}
Construct $\mathbb{T}$ to have three objects, $\mu$, $\delta$, and $\tau$. The hom objects will be selected such that $Inf$ becomes a composition morphism, and the 2-morphisms (morphisms of the hom category) are constructed to artificially select a minimising element if it exists.

Define the following singleton categories \[\mathbf{1} \cong \{\iota\} \cong \{Id_\mu\} \cong  \{Id_\delta\} \cong \{Id_\tau\}\]
Define $\mathbf{M}$ such that $Obj(\mathbf{M}) = M$ and that for any $m, m' \in \mathbf{M}$ there is a unique morphism $\sim \ : m \rightarrow m'$ if and only if $Inf(m) = Inf(m')$. Define $\mathbf{D}$ to be the category whose objects are the elements of $D$. Let $U \subseteq D$ be the subset of datasets for which an error minimising model exists, and let $Alg : U \rightarrow M$ be a function which selects an error minimising model for each $d \in D$ under the constraint that if there exists an $m\in \mathbf{M}$ such that $Inf(m) = d$, then $Alg(d) \cong m$. 

Define the hom sets of $\mathbf{D}$ with the following piecewise function.
\begin{gather*}
\mathbf{D}(d, d') :=
\begin{cases} 
      \{Id_d\} & d = d' \\
       \{*\} & d \in U \wedge  d' = Inf(Alg(d)) \wedge d \neq d'\\
       \emptyset & else\\
   \end{cases}
\end{gather*}

Composition is defined in the obvious way. For objects $d,d',d''\in D$ consider the form of the composition morphism. \[\circ_{d,d',d''} : \mathbf{D}(d,d') \times \mathbf{D}(d',d'') \rightarrow \mathbf{D}(d,d'')\]

Whenever $\mathbf{D}(d, d')$ or $\mathbf{D}(d', d'')$ is empty, then the product is empty, making the composition morphism the unique map from the empty set. When both $\mathbf{D}(d, d')$ or $\mathbf{D}(d', d'')$ are non empty, they must both be singleton. Therefore the following must be true.
\begin{align*}
&d = d' \vee (d' = Inf(Alg(d)) \wedge d \neq d')\\
&d' = d'' \vee (d'' = Inf(Alg(d')) \wedge d' \neq d'')\\
\end{align*}
Which may be simplified to form the following.
\begin{align*}
&d = d' \vee d' = Inf(Alg(d))\\
&d' = d'' \vee d'' = Inf(Alg(d'))\\
\end{align*}
Combining these statements produces the following deduction.
\begin{align*}
&\mathbf{D}(d,d') \times \mathbf{D}(d',d'') \cong \mathbf{1}\\
\implies&(d = d' \vee d' = Inf(Alg(d)))\\
&\wedge(d' = d'' \vee d'' = Inf(Alg(d')))\\
\implies & (d=d' \wedge d' = d'')\\
& \vee(d = d' \wedge d'' = Inf(Alg(d')))\\
& \vee(d' = Inf(Alg(d)) \wedge  d' = d'')\\
& \vee(d' = Inf(Alg(d)) \wedge d'' = Inf(Alg(d')))\\
\implies & (d = d'')\\
& \vee d'' = Inf(Alg(d))\\
& \vee(d' = Inf(Alg(d)) \wedge d'' = Inf(Alg(d')))\\
\end{align*}
When $d' = Inf(Alg(d))$ then for $m = Alg(d)$ \[Err(Inf(m), d') = Err(Inf(Alg(d)), d') = Err(d', d') = 0\] By the definition of $Alg$ this forces $Alg(d') \cong m = Alg(d')$, which by the construction of $\mathbf{M}$ means that $Inf(Alg(d')) = Inf(Alg(d)) = d'$. This allows the deduction to be simplified to the following implication.
\begin{align*}
&\mathbf{D}(d,d') \times \mathbf{D}(d',d'') \cong \mathbf{1}\\
\implies & (d = d'')\vee d'' = Inf(Alg(d))\\
&\vee(d' = Inf(Alg(d)) \wedge d'' = Inf(Alg(d')))\\
\implies & (d = d'')\vee d'' = Inf(Alg(d))\\
&\vee(d' = Inf(Alg(d)) \wedge d'' = d'))\\
\implies & (d = d'')\vee d'' = Inf(Alg(d))\\
\implies & \mathbf{D}(d,d'') \cong \mathbf{1}
\end{align*}

Making the composition morphism in this case the unique morphism between singleton sets.

Using the above defined categories, define the hom-categories of $\mathbb{T}$ as follows.

\begin{center}
\begin{tabular}{c|ccc}
$\mathbb{T}(-,-)$ & $\mu$ & $\delta$ & $\tau$\\
\hline
 $\mu$ & $\{Id_\mu\}$ & $\emptyset$ & $\mathbf{M}$\\
 $\delta$ & $\{\iota\}$ & $\{Id_\delta\}$ & $\mathbf{D}$\\
 $\tau$ & $\emptyset$& $\emptyset$ & $\{Id_\tau\}$ \\
\end{tabular}
\end{center}

The only composition morphism which is not fixed by identity laws or the empty categories is the following \[\circ_{\delta, \mu, \tau} : \mathbb{T}(\delta, \mu) \times \mathbb{T}(\mu, \tau) \rightarrow \mathbb{T}(\delta, \tau)\]

Substituting the known hom objects, this is rewritten as.

\[\circ_{\delta, \mu, \tau} : \{\iota \} \times \mathbf{M} \rightarrow \mathbf{D}\]

Because $\{\iota \} \times \mathbf{M} \cong \mathbf{M}$, the composition morphism can be defined by the inference function which maps all morphisms of $\mathbf{M}$ to the relevant identity morphisms \[\circ_{\delta, \mu, \tau} := Inf\] Finally, consider the following Kan extension problem in $\mathbb{T}$.

% https://q.uiver.app/#q=WzAsMyxbMCwxLCJcXGRlbHRhIl0sWzIsMSwiXFx0YXUiXSxbMSwwLCJcXG11Il0sWzAsMiwiXFxpb3RhIl0sWzAsMSwiZCIsMl0sWzIsMSwibSIsMCx7InN0eWxlIjp7ImJvZHkiOnsibmFtZSI6ImRhc2hlZCJ9fX1dXQ==
\[\begin{tikzcd}
	& \mu \\
	\delta && \tau
	\arrow["m", dashed, from=1-2, to=2-3]
	\arrow["\iota", from=2-1, to=1-2]
	\arrow["d"', from=2-1, to=2-3]
\end{tikzcd}\]

If an error minimising $m$ does not exist for the given $d$ then no Kan extension can exist as there is no morphism from $d$ into the image of $Inf : \{\iota \} \times \mathbf{M} \rightarrow \mathbf{D}$. However, if an error minimising $m$ does exist then by construction there is a morphism in $\mathbf{D}$ and consequently a 2-morphism in $\mathbb{T}$ of the form $d \Rightarrow Alg(d)\iota = Inf(Alg(d))$. For any $m$ for which there also exists a 2-morphism $d \Rightarrow m$ then as such a 2-morphism from $d$ into the image of $Inf$ is unique, then $m = Inf(Alg(d))$, which by the construction of $\mathbf{M}$ means that $Alg(d) \cong m$. This makes $Alg(d)$, when it exists, a left Kan extension in $\mathbb{T}$
\end{proof}
\section{Conclusion}
In this work, we propose a simple yet effective approach, called SMILE, for graph few-shot learning with fewer tasks. Specifically, we introduce a novel dual-level mixup strategy, including within-task and across-task mixup, for enriching the diversity of nodes within each task and the diversity of tasks. Also, we incorporate the degree-based prior information to learn expressive node embeddings. Theoretically, we prove that SMILE effectively enhances the model's generalization performance. Empirically, we conduct extensive experiments on multiple benchmarks and the results suggest that SMILE significantly outperforms other baselines, including both in-domain and cross-domain few-shot settings.

%\bibliography{references.bib}
\bibliographystyle{icml2025}

\begin{thebibliography}{10}
\providecommand{\natexlab}[1]{#1}
\providecommand{\url}[1]{\texttt{#1}}
\expandafter\ifx\csname urlstyle\endcsname\relax
  \providecommand{\doi}[1]{doi: #1}\else
  \providecommand{\doi}{doi: \begingroup \urlstyle{rm}\Url}\fi

\bibitem[Fong \& Spivak(2018)Fong and Spivak]{fongSevenSketchesCompositionality2018}
Fong, B. and Spivak, D.~I.
\newblock Seven {Sketches} in {Compositionality}: {An} {Invitation} to {Applied} {Category} {Theory}, October 2018.
\newblock URL \url{http://arxiv.org/abs/1803.05316}.
\newblock Number: arXiv:1803.05316 arXiv:1803.05316 [math].

\bibitem[Johnson \& Yau(2020)Johnson and Yau]{johnson_2-dimensional_2020}
Johnson, N. and Yau, D.
\newblock 2-{Dimensional} {Categories}, June 2020.
\newblock URL \url{http://arxiv.org/abs/2002.06055}.
\newblock arXiv:2002.06055 [math].

\bibitem[Kelly(2005)]{Kelly2005}
Kelly, G.~M.
\newblock Basic concepts of enriched category theory.
\newblock \emph{Repr. Theory Appl. Categ.}, \penalty0 (10):\penalty0 vi+137, 2005.
\newblock Reprint of the 1982 original [Cambridge Univ. Press, Cambridge; MR0651714].

\bibitem[Leinster(2016)]{leinsterBasicCategoryTheory2016}
Leinster, T.
\newblock Basic {Category} {Theory}, December 2016.
\newblock URL \url{http://arxiv.org/abs/1612.09375}.
\newblock Number: arXiv:1612.09375 arXiv:1612.09375 [math].

\bibitem[Meyers et~al.(2022)Meyers, Spivak, and Wisnesky]{meyers_fast_2022}
Meyers, J., Spivak, D.~I., and Wisnesky, R.
\newblock Fast {Left} {Kan} {Extensions} {Using} {The} {Chase}, May 2022.
\newblock URL \url{http://arxiv.org/abs/2205.02425}.
\newblock arXiv:2205.02425 [cs].

\bibitem[Perrone \& Tholen(2022)Perrone and Tholen]{perrone_kan_2022}
Perrone, P. and Tholen, W.
\newblock Kan extensions are partial colimits.
\newblock \emph{Applied Categorical Structures}, 30\penalty0 (4):\penalty0 685--753, August 2022.
\newblock ISSN 0927-2852, 1572-9095.
\newblock \doi{10.1007/s10485-021-09671-9}.
\newblock URL \url{http://arxiv.org/abs/2101.04531}.
\newblock arXiv:2101.04531 [math].

\bibitem[Porst(2024)]{porst_history_2024}
Porst, H.-E.
\newblock The history of the {General} {Adjoint} {Functor} {Theorem}, May 2024.
\newblock URL \url{http://arxiv.org/abs/2310.19528}.
\newblock arXiv:2310.19528 [math].

\bibitem[Pugh et~al.()Pugh, Grundy, and Harris]{pugh_using_2023}
Pugh, M., Grundy, J., and Harris, N.
\newblock Using kan extensions to motivate the design of a surprisingly effective unsupervised linear {SVM} on the occupancy dataset.

\bibitem[Riehl(2016)]{riehlCategoryTheoryContext2016}
Riehl, E.
\newblock \emph{Category {{Theory}} in {{Context}}}.
\newblock {Dover Publications Inc.}, {Mineola, New York}, December 2016.
\newblock ISBN 978-0-486-80903-8.

\bibitem[Shiebler(2022)]{shieblerKanExtensionsData2022}
Shiebler, D.
\newblock Kan {{Extensions}} in {{Data Science}} and {{Machine Learning}}, July 2022.

\end{thebibliography}

% APPENDIX
%\subsection{Lloyd-Max Algorithm}
\label{subsec:Lloyd-Max}
For a given quantization bitwidth $B$ and an operand $\bm{X}$, the Lloyd-Max algorithm finds $2^B$ quantization levels $\{\hat{x}_i\}_{i=1}^{2^B}$ such that quantizing $\bm{X}$ by rounding each scalar in $\bm{X}$ to the nearest quantization level minimizes the quantization MSE. 

The algorithm starts with an initial guess of quantization levels and then iteratively computes quantization thresholds $\{\tau_i\}_{i=1}^{2^B-1}$ and updates quantization levels $\{\hat{x}_i\}_{i=1}^{2^B}$. Specifically, at iteration $n$, thresholds are set to the midpoints of the previous iteration's levels:
\begin{align*}
    \tau_i^{(n)}=\frac{\hat{x}_i^{(n-1)}+\hat{x}_{i+1}^{(n-1)}}2 \text{ for } i=1\ldots 2^B-1
\end{align*}
Subsequently, the quantization levels are re-computed as conditional means of the data regions defined by the new thresholds:
\begin{align*}
    \hat{x}_i^{(n)}=\mathbb{E}\left[ \bm{X} \big| \bm{X}\in [\tau_{i-1}^{(n)},\tau_i^{(n)}] \right] \text{ for } i=1\ldots 2^B
\end{align*}
where to satisfy boundary conditions we have $\tau_0=-\infty$ and $\tau_{2^B}=\infty$. The algorithm iterates the above steps until convergence.

Figure \ref{fig:lm_quant} compares the quantization levels of a $7$-bit floating point (E3M3) quantizer (left) to a $7$-bit Lloyd-Max quantizer (right) when quantizing a layer of weights from the GPT3-126M model at a per-tensor granularity. As shown, the Lloyd-Max quantizer achieves substantially lower quantization MSE. Further, Table \ref{tab:FP7_vs_LM7} shows the superior perplexity achieved by Lloyd-Max quantizers for bitwidths of $7$, $6$ and $5$. The difference between the quantizers is clear at 5 bits, where per-tensor FP quantization incurs a drastic and unacceptable increase in perplexity, while Lloyd-Max quantization incurs a much smaller increase. Nevertheless, we note that even the optimal Lloyd-Max quantizer incurs a notable ($\sim 1.5$) increase in perplexity due to the coarse granularity of quantization. 

\begin{figure}[h]
  \centering
  \includegraphics[width=0.7\linewidth]{sections/figures/LM7_FP7.pdf}
  \caption{\small Quantization levels and the corresponding quantization MSE of Floating Point (left) vs Lloyd-Max (right) Quantizers for a layer of weights in the GPT3-126M model.}
  \label{fig:lm_quant}
\end{figure}

\begin{table}[h]\scriptsize
\begin{center}
\caption{\label{tab:FP7_vs_LM7} \small Comparing perplexity (lower is better) achieved by floating point quantizers and Lloyd-Max quantizers on a GPT3-126M model for the Wikitext-103 dataset.}
\begin{tabular}{c|cc|c}
\hline
 \multirow{2}{*}{\textbf{Bitwidth}} & \multicolumn{2}{|c|}{\textbf{Floating-Point Quantizer}} & \textbf{Lloyd-Max Quantizer} \\
 & Best Format & Wikitext-103 Perplexity & Wikitext-103 Perplexity \\
\hline
7 & E3M3 & 18.32 & 18.27 \\
6 & E3M2 & 19.07 & 18.51 \\
5 & E4M0 & 43.89 & 19.71 \\
\hline
\end{tabular}
\end{center}
\end{table}

\subsection{Proof of Local Optimality of LO-BCQ}
\label{subsec:lobcq_opt_proof}
For a given block $\bm{b}_j$, the quantization MSE during LO-BCQ can be empirically evaluated as $\frac{1}{L_b}\lVert \bm{b}_j- \bm{\hat{b}}_j\rVert^2_2$ where $\bm{\hat{b}}_j$ is computed from equation (\ref{eq:clustered_quantization_definition}) as $C_{f(\bm{b}_j)}(\bm{b}_j)$. Further, for a given block cluster $\mathcal{B}_i$, we compute the quantization MSE as $\frac{1}{|\mathcal{B}_{i}|}\sum_{\bm{b} \in \mathcal{B}_{i}} \frac{1}{L_b}\lVert \bm{b}- C_i^{(n)}(\bm{b})\rVert^2_2$. Therefore, at the end of iteration $n$, we evaluate the overall quantization MSE $J^{(n)}$ for a given operand $\bm{X}$ composed of $N_c$ block clusters as:
\begin{align*}
    \label{eq:mse_iter_n}
    J^{(n)} = \frac{1}{N_c} \sum_{i=1}^{N_c} \frac{1}{|\mathcal{B}_{i}^{(n)}|}\sum_{\bm{v} \in \mathcal{B}_{i}^{(n)}} \frac{1}{L_b}\lVert \bm{b}- B_i^{(n)}(\bm{b})\rVert^2_2
\end{align*}

At the end of iteration $n$, the codebooks are updated from $\mathcal{C}^{(n-1)}$ to $\mathcal{C}^{(n)}$. However, the mapping of a given vector $\bm{b}_j$ to quantizers $\mathcal{C}^{(n)}$ remains as  $f^{(n)}(\bm{b}_j)$. At the next iteration, during the vector clustering step, $f^{(n+1)}(\bm{b}_j)$ finds new mapping of $\bm{b}_j$ to updated codebooks $\mathcal{C}^{(n)}$ such that the quantization MSE over the candidate codebooks is minimized. Therefore, we obtain the following result for $\bm{b}_j$:
\begin{align*}
\frac{1}{L_b}\lVert \bm{b}_j - C_{f^{(n+1)}(\bm{b}_j)}^{(n)}(\bm{b}_j)\rVert^2_2 \le \frac{1}{L_b}\lVert \bm{b}_j - C_{f^{(n)}(\bm{b}_j)}^{(n)}(\bm{b}_j)\rVert^2_2
\end{align*}

That is, quantizing $\bm{b}_j$ at the end of the block clustering step of iteration $n+1$ results in lower quantization MSE compared to quantizing at the end of iteration $n$. Since this is true for all $\bm{b} \in \bm{X}$, we assert the following:
\begin{equation}
\begin{split}
\label{eq:mse_ineq_1}
    \tilde{J}^{(n+1)} &= \frac{1}{N_c} \sum_{i=1}^{N_c} \frac{1}{|\mathcal{B}_{i}^{(n+1)}|}\sum_{\bm{b} \in \mathcal{B}_{i}^{(n+1)}} \frac{1}{L_b}\lVert \bm{b} - C_i^{(n)}(b)\rVert^2_2 \le J^{(n)}
\end{split}
\end{equation}
where $\tilde{J}^{(n+1)}$ is the the quantization MSE after the vector clustering step at iteration $n+1$.

Next, during the codebook update step (\ref{eq:quantizers_update}) at iteration $n+1$, the per-cluster codebooks $\mathcal{C}^{(n)}$ are updated to $\mathcal{C}^{(n+1)}$ by invoking the Lloyd-Max algorithm \citep{Lloyd}. We know that for any given value distribution, the Lloyd-Max algorithm minimizes the quantization MSE. Therefore, for a given vector cluster $\mathcal{B}_i$ we obtain the following result:

\begin{equation}
    \frac{1}{|\mathcal{B}_{i}^{(n+1)}|}\sum_{\bm{b} \in \mathcal{B}_{i}^{(n+1)}} \frac{1}{L_b}\lVert \bm{b}- C_i^{(n+1)}(\bm{b})\rVert^2_2 \le \frac{1}{|\mathcal{B}_{i}^{(n+1)}|}\sum_{\bm{b} \in \mathcal{B}_{i}^{(n+1)}} \frac{1}{L_b}\lVert \bm{b}- C_i^{(n)}(\bm{b})\rVert^2_2
\end{equation}

The above equation states that quantizing the given block cluster $\mathcal{B}_i$ after updating the associated codebook from $C_i^{(n)}$ to $C_i^{(n+1)}$ results in lower quantization MSE. Since this is true for all the block clusters, we derive the following result: 
\begin{equation}
\begin{split}
\label{eq:mse_ineq_2}
     J^{(n+1)} &= \frac{1}{N_c} \sum_{i=1}^{N_c} \frac{1}{|\mathcal{B}_{i}^{(n+1)}|}\sum_{\bm{b} \in \mathcal{B}_{i}^{(n+1)}} \frac{1}{L_b}\lVert \bm{b}- C_i^{(n+1)}(\bm{b})\rVert^2_2  \le \tilde{J}^{(n+1)}   
\end{split}
\end{equation}

Following (\ref{eq:mse_ineq_1}) and (\ref{eq:mse_ineq_2}), we find that the quantization MSE is non-increasing for each iteration, that is, $J^{(1)} \ge J^{(2)} \ge J^{(3)} \ge \ldots \ge J^{(M)}$ where $M$ is the maximum number of iterations. 
%Therefore, we can say that if the algorithm converges, then it must be that it has converged to a local minimum. 
\hfill $\blacksquare$


\begin{figure}
    \begin{center}
    \includegraphics[width=0.5\textwidth]{sections//figures/mse_vs_iter.pdf}
    \end{center}
    \caption{\small NMSE vs iterations during LO-BCQ compared to other block quantization proposals}
    \label{fig:nmse_vs_iter}
\end{figure}

Figure \ref{fig:nmse_vs_iter} shows the empirical convergence of LO-BCQ across several block lengths and number of codebooks. Also, the MSE achieved by LO-BCQ is compared to baselines such as MXFP and VSQ. As shown, LO-BCQ converges to a lower MSE than the baselines. Further, we achieve better convergence for larger number of codebooks ($N_c$) and for a smaller block length ($L_b$), both of which increase the bitwidth of BCQ (see Eq \ref{eq:bitwidth_bcq}).


\subsection{Additional Accuracy Results}
%Table \ref{tab:lobcq_config} lists the various LOBCQ configurations and their corresponding bitwidths.
\begin{table}
\setlength{\tabcolsep}{4.75pt}
\begin{center}
\caption{\label{tab:lobcq_config} Various LO-BCQ configurations and their bitwidths.}
\begin{tabular}{|c||c|c|c|c||c|c||c|} 
\hline
 & \multicolumn{4}{|c||}{$L_b=8$} & \multicolumn{2}{|c||}{$L_b=4$} & $L_b=2$ \\
 \hline
 \backslashbox{$L_A$\kern-1em}{\kern-1em$N_c$} & 2 & 4 & 8 & 16 & 2 & 4 & 2 \\
 \hline
 64 & 4.25 & 4.375 & 4.5 & 4.625 & 4.375 & 4.625 & 4.625\\
 \hline
 32 & 4.375 & 4.5 & 4.625& 4.75 & 4.5 & 4.75 & 4.75 \\
 \hline
 16 & 4.625 & 4.75& 4.875 & 5 & 4.75 & 5 & 5 \\
 \hline
\end{tabular}
\end{center}
\end{table}

%\subsection{Perplexity achieved by various LO-BCQ configurations on Wikitext-103 dataset}

\begin{table} \centering
\begin{tabular}{|c||c|c|c|c||c|c||c|} 
\hline
 $L_b \rightarrow$& \multicolumn{4}{c||}{8} & \multicolumn{2}{c||}{4} & 2\\
 \hline
 \backslashbox{$L_A$\kern-1em}{\kern-1em$N_c$} & 2 & 4 & 8 & 16 & 2 & 4 & 2  \\
 %$N_c \rightarrow$ & 2 & 4 & 8 & 16 & 2 & 4 & 2 \\
 \hline
 \hline
 \multicolumn{8}{c}{GPT3-1.3B (FP32 PPL = 9.98)} \\ 
 \hline
 \hline
 64 & 10.40 & 10.23 & 10.17 & 10.15 &  10.28 & 10.18 & 10.19 \\
 \hline
 32 & 10.25 & 10.20 & 10.15 & 10.12 &  10.23 & 10.17 & 10.17 \\
 \hline
 16 & 10.22 & 10.16 & 10.10 & 10.09 &  10.21 & 10.14 & 10.16 \\
 \hline
  \hline
 \multicolumn{8}{c}{GPT3-8B (FP32 PPL = 7.38)} \\ 
 \hline
 \hline
 64 & 7.61 & 7.52 & 7.48 &  7.47 &  7.55 &  7.49 & 7.50 \\
 \hline
 32 & 7.52 & 7.50 & 7.46 &  7.45 &  7.52 &  7.48 & 7.48  \\
 \hline
 16 & 7.51 & 7.48 & 7.44 &  7.44 &  7.51 &  7.49 & 7.47  \\
 \hline
\end{tabular}
\caption{\label{tab:ppl_gpt3_abalation} Wikitext-103 perplexity across GPT3-1.3B and 8B models.}
\end{table}

\begin{table} \centering
\begin{tabular}{|c||c|c|c|c||} 
\hline
 $L_b \rightarrow$& \multicolumn{4}{c||}{8}\\
 \hline
 \backslashbox{$L_A$\kern-1em}{\kern-1em$N_c$} & 2 & 4 & 8 & 16 \\
 %$N_c \rightarrow$ & 2 & 4 & 8 & 16 & 2 & 4 & 2 \\
 \hline
 \hline
 \multicolumn{5}{|c|}{Llama2-7B (FP32 PPL = 5.06)} \\ 
 \hline
 \hline
 64 & 5.31 & 5.26 & 5.19 & 5.18  \\
 \hline
 32 & 5.23 & 5.25 & 5.18 & 5.15  \\
 \hline
 16 & 5.23 & 5.19 & 5.16 & 5.14  \\
 \hline
 \multicolumn{5}{|c|}{Nemotron4-15B (FP32 PPL = 5.87)} \\ 
 \hline
 \hline
 64  & 6.3 & 6.20 & 6.13 & 6.08  \\
 \hline
 32  & 6.24 & 6.12 & 6.07 & 6.03  \\
 \hline
 16  & 6.12 & 6.14 & 6.04 & 6.02  \\
 \hline
 \multicolumn{5}{|c|}{Nemotron4-340B (FP32 PPL = 3.48)} \\ 
 \hline
 \hline
 64 & 3.67 & 3.62 & 3.60 & 3.59 \\
 \hline
 32 & 3.63 & 3.61 & 3.59 & 3.56 \\
 \hline
 16 & 3.61 & 3.58 & 3.57 & 3.55 \\
 \hline
\end{tabular}
\caption{\label{tab:ppl_llama7B_nemo15B} Wikitext-103 perplexity compared to FP32 baseline in Llama2-7B and Nemotron4-15B, 340B models}
\end{table}

%\subsection{Perplexity achieved by various LO-BCQ configurations on MMLU dataset}


\begin{table} \centering
\begin{tabular}{|c||c|c|c|c||c|c|c|c|} 
\hline
 $L_b \rightarrow$& \multicolumn{4}{c||}{8} & \multicolumn{4}{c||}{8}\\
 \hline
 \backslashbox{$L_A$\kern-1em}{\kern-1em$N_c$} & 2 & 4 & 8 & 16 & 2 & 4 & 8 & 16  \\
 %$N_c \rightarrow$ & 2 & 4 & 8 & 16 & 2 & 4 & 2 \\
 \hline
 \hline
 \multicolumn{5}{|c|}{Llama2-7B (FP32 Accuracy = 45.8\%)} & \multicolumn{4}{|c|}{Llama2-70B (FP32 Accuracy = 69.12\%)} \\ 
 \hline
 \hline
 64 & 43.9 & 43.4 & 43.9 & 44.9 & 68.07 & 68.27 & 68.17 & 68.75 \\
 \hline
 32 & 44.5 & 43.8 & 44.9 & 44.5 & 68.37 & 68.51 & 68.35 & 68.27  \\
 \hline
 16 & 43.9 & 42.7 & 44.9 & 45 & 68.12 & 68.77 & 68.31 & 68.59  \\
 \hline
 \hline
 \multicolumn{5}{|c|}{GPT3-22B (FP32 Accuracy = 38.75\%)} & \multicolumn{4}{|c|}{Nemotron4-15B (FP32 Accuracy = 64.3\%)} \\ 
 \hline
 \hline
 64 & 36.71 & 38.85 & 38.13 & 38.92 & 63.17 & 62.36 & 63.72 & 64.09 \\
 \hline
 32 & 37.95 & 38.69 & 39.45 & 38.34 & 64.05 & 62.30 & 63.8 & 64.33  \\
 \hline
 16 & 38.88 & 38.80 & 38.31 & 38.92 & 63.22 & 63.51 & 63.93 & 64.43  \\
 \hline
\end{tabular}
\caption{\label{tab:mmlu_abalation} Accuracy on MMLU dataset across GPT3-22B, Llama2-7B, 70B and Nemotron4-15B models.}
\end{table}


%\subsection{Perplexity achieved by various LO-BCQ configurations on LM evaluation harness}

\begin{table} \centering
\begin{tabular}{|c||c|c|c|c||c|c|c|c|} 
\hline
 $L_b \rightarrow$& \multicolumn{4}{c||}{8} & \multicolumn{4}{c||}{8}\\
 \hline
 \backslashbox{$L_A$\kern-1em}{\kern-1em$N_c$} & 2 & 4 & 8 & 16 & 2 & 4 & 8 & 16  \\
 %$N_c \rightarrow$ & 2 & 4 & 8 & 16 & 2 & 4 & 2 \\
 \hline
 \hline
 \multicolumn{5}{|c|}{Race (FP32 Accuracy = 37.51\%)} & \multicolumn{4}{|c|}{Boolq (FP32 Accuracy = 64.62\%)} \\ 
 \hline
 \hline
 64 & 36.94 & 37.13 & 36.27 & 37.13 & 63.73 & 62.26 & 63.49 & 63.36 \\
 \hline
 32 & 37.03 & 36.36 & 36.08 & 37.03 & 62.54 & 63.51 & 63.49 & 63.55  \\
 \hline
 16 & 37.03 & 37.03 & 36.46 & 37.03 & 61.1 & 63.79 & 63.58 & 63.33  \\
 \hline
 \hline
 \multicolumn{5}{|c|}{Winogrande (FP32 Accuracy = 58.01\%)} & \multicolumn{4}{|c|}{Piqa (FP32 Accuracy = 74.21\%)} \\ 
 \hline
 \hline
 64 & 58.17 & 57.22 & 57.85 & 58.33 & 73.01 & 73.07 & 73.07 & 72.80 \\
 \hline
 32 & 59.12 & 58.09 & 57.85 & 58.41 & 73.01 & 73.94 & 72.74 & 73.18  \\
 \hline
 16 & 57.93 & 58.88 & 57.93 & 58.56 & 73.94 & 72.80 & 73.01 & 73.94  \\
 \hline
\end{tabular}
\caption{\label{tab:mmlu_abalation} Accuracy on LM evaluation harness tasks on GPT3-1.3B model.}
\end{table}

\begin{table} \centering
\begin{tabular}{|c||c|c|c|c||c|c|c|c|} 
\hline
 $L_b \rightarrow$& \multicolumn{4}{c||}{8} & \multicolumn{4}{c||}{8}\\
 \hline
 \backslashbox{$L_A$\kern-1em}{\kern-1em$N_c$} & 2 & 4 & 8 & 16 & 2 & 4 & 8 & 16  \\
 %$N_c \rightarrow$ & 2 & 4 & 8 & 16 & 2 & 4 & 2 \\
 \hline
 \hline
 \multicolumn{5}{|c|}{Race (FP32 Accuracy = 41.34\%)} & \multicolumn{4}{|c|}{Boolq (FP32 Accuracy = 68.32\%)} \\ 
 \hline
 \hline
 64 & 40.48 & 40.10 & 39.43 & 39.90 & 69.20 & 68.41 & 69.45 & 68.56 \\
 \hline
 32 & 39.52 & 39.52 & 40.77 & 39.62 & 68.32 & 67.43 & 68.17 & 69.30  \\
 \hline
 16 & 39.81 & 39.71 & 39.90 & 40.38 & 68.10 & 66.33 & 69.51 & 69.42  \\
 \hline
 \hline
 \multicolumn{5}{|c|}{Winogrande (FP32 Accuracy = 67.88\%)} & \multicolumn{4}{|c|}{Piqa (FP32 Accuracy = 78.78\%)} \\ 
 \hline
 \hline
 64 & 66.85 & 66.61 & 67.72 & 67.88 & 77.31 & 77.42 & 77.75 & 77.64 \\
 \hline
 32 & 67.25 & 67.72 & 67.72 & 67.00 & 77.31 & 77.04 & 77.80 & 77.37  \\
 \hline
 16 & 68.11 & 68.90 & 67.88 & 67.48 & 77.37 & 78.13 & 78.13 & 77.69  \\
 \hline
\end{tabular}
\caption{\label{tab:mmlu_abalation} Accuracy on LM evaluation harness tasks on GPT3-8B model.}
\end{table}

\begin{table} \centering
\begin{tabular}{|c||c|c|c|c||c|c|c|c|} 
\hline
 $L_b \rightarrow$& \multicolumn{4}{c||}{8} & \multicolumn{4}{c||}{8}\\
 \hline
 \backslashbox{$L_A$\kern-1em}{\kern-1em$N_c$} & 2 & 4 & 8 & 16 & 2 & 4 & 8 & 16  \\
 %$N_c \rightarrow$ & 2 & 4 & 8 & 16 & 2 & 4 & 2 \\
 \hline
 \hline
 \multicolumn{5}{|c|}{Race (FP32 Accuracy = 40.67\%)} & \multicolumn{4}{|c|}{Boolq (FP32 Accuracy = 76.54\%)} \\ 
 \hline
 \hline
 64 & 40.48 & 40.10 & 39.43 & 39.90 & 75.41 & 75.11 & 77.09 & 75.66 \\
 \hline
 32 & 39.52 & 39.52 & 40.77 & 39.62 & 76.02 & 76.02 & 75.96 & 75.35  \\
 \hline
 16 & 39.81 & 39.71 & 39.90 & 40.38 & 75.05 & 73.82 & 75.72 & 76.09  \\
 \hline
 \hline
 \multicolumn{5}{|c|}{Winogrande (FP32 Accuracy = 70.64\%)} & \multicolumn{4}{|c|}{Piqa (FP32 Accuracy = 79.16\%)} \\ 
 \hline
 \hline
 64 & 69.14 & 70.17 & 70.17 & 70.56 & 78.24 & 79.00 & 78.62 & 78.73 \\
 \hline
 32 & 70.96 & 69.69 & 71.27 & 69.30 & 78.56 & 79.49 & 79.16 & 78.89  \\
 \hline
 16 & 71.03 & 69.53 & 69.69 & 70.40 & 78.13 & 79.16 & 79.00 & 79.00  \\
 \hline
\end{tabular}
\caption{\label{tab:mmlu_abalation} Accuracy on LM evaluation harness tasks on GPT3-22B model.}
\end{table}

\begin{table} \centering
\begin{tabular}{|c||c|c|c|c||c|c|c|c|} 
\hline
 $L_b \rightarrow$& \multicolumn{4}{c||}{8} & \multicolumn{4}{c||}{8}\\
 \hline
 \backslashbox{$L_A$\kern-1em}{\kern-1em$N_c$} & 2 & 4 & 8 & 16 & 2 & 4 & 8 & 16  \\
 %$N_c \rightarrow$ & 2 & 4 & 8 & 16 & 2 & 4 & 2 \\
 \hline
 \hline
 \multicolumn{5}{|c|}{Race (FP32 Accuracy = 44.4\%)} & \multicolumn{4}{|c|}{Boolq (FP32 Accuracy = 79.29\%)} \\ 
 \hline
 \hline
 64 & 42.49 & 42.51 & 42.58 & 43.45 & 77.58 & 77.37 & 77.43 & 78.1 \\
 \hline
 32 & 43.35 & 42.49 & 43.64 & 43.73 & 77.86 & 75.32 & 77.28 & 77.86  \\
 \hline
 16 & 44.21 & 44.21 & 43.64 & 42.97 & 78.65 & 77 & 76.94 & 77.98  \\
 \hline
 \hline
 \multicolumn{5}{|c|}{Winogrande (FP32 Accuracy = 69.38\%)} & \multicolumn{4}{|c|}{Piqa (FP32 Accuracy = 78.07\%)} \\ 
 \hline
 \hline
 64 & 68.9 & 68.43 & 69.77 & 68.19 & 77.09 & 76.82 & 77.09 & 77.86 \\
 \hline
 32 & 69.38 & 68.51 & 68.82 & 68.90 & 78.07 & 76.71 & 78.07 & 77.86  \\
 \hline
 16 & 69.53 & 67.09 & 69.38 & 68.90 & 77.37 & 77.8 & 77.91 & 77.69  \\
 \hline
\end{tabular}
\caption{\label{tab:mmlu_abalation} Accuracy on LM evaluation harness tasks on Llama2-7B model.}
\end{table}

\begin{table} \centering
\begin{tabular}{|c||c|c|c|c||c|c|c|c|} 
\hline
 $L_b \rightarrow$& \multicolumn{4}{c||}{8} & \multicolumn{4}{c||}{8}\\
 \hline
 \backslashbox{$L_A$\kern-1em}{\kern-1em$N_c$} & 2 & 4 & 8 & 16 & 2 & 4 & 8 & 16  \\
 %$N_c \rightarrow$ & 2 & 4 & 8 & 16 & 2 & 4 & 2 \\
 \hline
 \hline
 \multicolumn{5}{|c|}{Race (FP32 Accuracy = 48.8\%)} & \multicolumn{4}{|c|}{Boolq (FP32 Accuracy = 85.23\%)} \\ 
 \hline
 \hline
 64 & 49.00 & 49.00 & 49.28 & 48.71 & 82.82 & 84.28 & 84.03 & 84.25 \\
 \hline
 32 & 49.57 & 48.52 & 48.33 & 49.28 & 83.85 & 84.46 & 84.31 & 84.93  \\
 \hline
 16 & 49.85 & 49.09 & 49.28 & 48.99 & 85.11 & 84.46 & 84.61 & 83.94  \\
 \hline
 \hline
 \multicolumn{5}{|c|}{Winogrande (FP32 Accuracy = 79.95\%)} & \multicolumn{4}{|c|}{Piqa (FP32 Accuracy = 81.56\%)} \\ 
 \hline
 \hline
 64 & 78.77 & 78.45 & 78.37 & 79.16 & 81.45 & 80.69 & 81.45 & 81.5 \\
 \hline
 32 & 78.45 & 79.01 & 78.69 & 80.66 & 81.56 & 80.58 & 81.18 & 81.34  \\
 \hline
 16 & 79.95 & 79.56 & 79.79 & 79.72 & 81.28 & 81.66 & 81.28 & 80.96  \\
 \hline
\end{tabular}
\caption{\label{tab:mmlu_abalation} Accuracy on LM evaluation harness tasks on Llama2-70B model.}
\end{table}

%\section{MSE Studies}
%\textcolor{red}{TODO}


\subsection{Number Formats and Quantization Method}
\label{subsec:numFormats_quantMethod}
\subsubsection{Integer Format}
An $n$-bit signed integer (INT) is typically represented with a 2s-complement format \citep{yao2022zeroquant,xiao2023smoothquant,dai2021vsq}, where the most significant bit denotes the sign.

\subsubsection{Floating Point Format}
An $n$-bit signed floating point (FP) number $x$ comprises of a 1-bit sign ($x_{\mathrm{sign}}$), $B_m$-bit mantissa ($x_{\mathrm{mant}}$) and $B_e$-bit exponent ($x_{\mathrm{exp}}$) such that $B_m+B_e=n-1$. The associated constant exponent bias ($E_{\mathrm{bias}}$) is computed as $(2^{{B_e}-1}-1)$. We denote this format as $E_{B_e}M_{B_m}$.  

\subsubsection{Quantization Scheme}
\label{subsec:quant_method}
A quantization scheme dictates how a given unquantized tensor is converted to its quantized representation. We consider FP formats for the purpose of illustration. Given an unquantized tensor $\bm{X}$ and an FP format $E_{B_e}M_{B_m}$, we first, we compute the quantization scale factor $s_X$ that maps the maximum absolute value of $\bm{X}$ to the maximum quantization level of the $E_{B_e}M_{B_m}$ format as follows:
\begin{align}
\label{eq:sf}
    s_X = \frac{\mathrm{max}(|\bm{X}|)}{\mathrm{max}(E_{B_e}M_{B_m})}
\end{align}
In the above equation, $|\cdot|$ denotes the absolute value function.

Next, we scale $\bm{X}$ by $s_X$ and quantize it to $\hat{\bm{X}}$ by rounding it to the nearest quantization level of $E_{B_e}M_{B_m}$ as:

\begin{align}
\label{eq:tensor_quant}
    \hat{\bm{X}} = \text{round-to-nearest}\left(\frac{\bm{X}}{s_X}, E_{B_e}M_{B_m}\right)
\end{align}

We perform dynamic max-scaled quantization \citep{wu2020integer}, where the scale factor $s$ for activations is dynamically computed during runtime.

\subsection{Vector Scaled Quantization}
\begin{wrapfigure}{r}{0.35\linewidth}
  \centering
  \includegraphics[width=\linewidth]{sections/figures/vsquant.jpg}
  \caption{\small Vectorwise decomposition for per-vector scaled quantization (VSQ \citep{dai2021vsq}).}
  \label{fig:vsquant}
\end{wrapfigure}
During VSQ \citep{dai2021vsq}, the operand tensors are decomposed into 1D vectors in a hardware friendly manner as shown in Figure \ref{fig:vsquant}. Since the decomposed tensors are used as operands in matrix multiplications during inference, it is beneficial to perform this decomposition along the reduction dimension of the multiplication. The vectorwise quantization is performed similar to tensorwise quantization described in Equations \ref{eq:sf} and \ref{eq:tensor_quant}, where a scale factor $s_v$ is required for each vector $\bm{v}$ that maps the maximum absolute value of that vector to the maximum quantization level. While smaller vector lengths can lead to larger accuracy gains, the associated memory and computational overheads due to the per-vector scale factors increases. To alleviate these overheads, VSQ \citep{dai2021vsq} proposed a second level quantization of the per-vector scale factors to unsigned integers, while MX \citep{rouhani2023shared} quantizes them to integer powers of 2 (denoted as $2^{INT}$).

\subsubsection{MX Format}
The MX format proposed in \citep{rouhani2023microscaling} introduces the concept of sub-block shifting. For every two scalar elements of $b$-bits each, there is a shared exponent bit. The value of this exponent bit is determined through an empirical analysis that targets minimizing quantization MSE. We note that the FP format $E_{1}M_{b}$ is strictly better than MX from an accuracy perspective since it allocates a dedicated exponent bit to each scalar as opposed to sharing it across two scalars. Therefore, we conservatively bound the accuracy of a $b+2$-bit signed MX format with that of a $E_{1}M_{b}$ format in our comparisons. For instance, we use E1M2 format as a proxy for MX4.

\begin{figure}
    \centering
    \includegraphics[width=1\linewidth]{sections//figures/BlockFormats.pdf}
    \caption{\small Comparing LO-BCQ to MX format.}
    \label{fig:block_formats}
\end{figure}

Figure \ref{fig:block_formats} compares our $4$-bit LO-BCQ block format to MX \citep{rouhani2023microscaling}. As shown, both LO-BCQ and MX decompose a given operand tensor into block arrays and each block array into blocks. Similar to MX, we find that per-block quantization ($L_b < L_A$) leads to better accuracy due to increased flexibility. While MX achieves this through per-block $1$-bit micro-scales, we associate a dedicated codebook to each block through a per-block codebook selector. Further, MX quantizes the per-block array scale-factor to E8M0 format without per-tensor scaling. In contrast during LO-BCQ, we find that per-tensor scaling combined with quantization of per-block array scale-factor to E4M3 format results in superior inference accuracy across models. 


\end{document}

