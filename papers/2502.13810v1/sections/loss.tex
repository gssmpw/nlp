\section{Loss}

\begin{definition}[Optimisation Problem]
\label{definition:optimisation}
\begin{align*}
Given \quad & Loss: D \rightarrow \mathbb{R}_+\quad &&\\
\textrm{minimise} \quad & Loss(x)&&
\end{align*}
\end{definition}

\begin{definition}[Positive Monoid]
A monoid $S$ is positive if for all $x,y \in S$ then \[xy = Id_* \implies x=Id_*\quad \wedge \quad y= Id_*\]
\end{definition}

\begin{definition}[Monoid Flavoured Preorder]
Given a positive monoid $S$ an $S$ flavoured preorder is a category $\Phi$ equipped with a functor $\leq : \Phi \rightarrow S$. Where for any two objects $x,y \in \Phi$ then $x\leq_S y$ if there exists a morphism $f : x \rightarrow y$ such that $\leq(f) = Id_*$
\end{definition}

%add explanation that morphism transitivity and associativity reproduce preorder axioms

\begin{definition}[Lower Bounded Preorder ]
A preorder $C$ is Lower bounded if there exists a bottom element $\bot$ such that for any other object $x \in C$ then $\bot \leq x$.
\end{definition}

\begin{definition}[Monoid Flavoured Preorder 2]
Given a monoidal preorder $S$, an $S$ flavoured preorder is a category $P$ equipped with a lax 2-functor $\Phi : P \rightarrow S$. For any two objects $x,y \in P$ then $x\leq_\Phi y$ if and only if there exists an $f : x \rightarrow y$ and $\sigma : \Phi(f) \Rightarrow \Phi(Id_*)$
\end{definition}

\begin{lemma}
Given a monoid flavoured preorder $\Phi : P \rightarrow S$ and $x,y,z \in P$, if there exists $g : x \rightarrow y$, $h : y \rightarrow z$, and $\sigma : \Phi(hg) \Rightarrow Id_*$ then $y \leq_\Phi z$.
\end{lemma}

\begin{proof}
By the definition of a Lax 2-Functor (Def \ref{definition:Lax2Functor}) there exists a 2-morphism $\gamma:\Phi(h)\Phi(g) \Rightarrow \Phi(hg)$
\end{proof}

\begin{proposition}[Adjunction Loss minimiser]
\label{prop:adjoint_min_band}
Given a bounded $S$ flavoured preorder $\leq : \Phi \rightarrow S$, and functor $Loss : D \rightarrow \Phi$ with left adjoint $L : \Phi \rightarrow D$, then $L(\bot)$ is a global minima of $Loss$.
\end{proposition}

\begin{proof}
By the definition of an adjunction (Def \ref{def:AdjointFunctorTriangle}), the adjunction unit, $\eta : Id_{\Phi} \Rightarrow Loss \circ L$ is a universal factorisor. Meaning that for any morphism $f : c \rightarrow Loss(d)$ there is a unique morphism $Loss(\tilde{f}) : Loss(L(c)) \rightarrow Loss(d)$ such that $Loss(\tilde{f})\eta_c = f$.

Set $c$ to be $\bot \in \Phi$. Because $\bot$ is the bottom object of the $S$ flavoured preorder $\Phi$ then for all $d \in D$, $\bot \leq_S Loss(d)$, implying that there exists a morphism $f : \bot \rightarrow Loss(d)$ such that $\leq(f) = Id_*$.

This then suggests that for all $d \in D$ there is a morphism $Loss(\tilde{f}) : Loss(L(\bot)) \rightarrow Loss(d)$ such that $Loss(\tilde{f})\eta_c =\ f$
\begin{align*}
Id_* &= \leq(f)=\ \leq(Loss(\tilde{f})\eta_c)=\ (\leq(Loss(\tilde{f})))(\leq(\eta_c))\\
&\implies \leq(Loss(\tilde{f})) = Id_*\\
&\implies Loss(L(\bot)) \leq_S Loss(d)
\end{align*}
Meaning that for all $d \in D$ that $Loss(L(\bot)) \leq_S Loss(d)$ making $L(\bot)$ a global minima of $Loss$.
\end{proof}

\begin{proposition}[Kan Loss Minimiser]
\label{prop:kan_min_band}
Given a bounded $S$ flavoured preorder $\leq : \Phi \rightarrow S$, and a functor $Loss : D \rightarrow \Phi$ the left Kan lift $Lift_{Loss}\ \bot(*)$, if it exists, is a global minima of $Loss$.
% https://q.uiver.app/#q=WzAsMyxbMSwwLCJEIl0sWzIsMSwiXFxQaGkiXSxbMCwxLCJcXG1hdGhiZnsxfSJdLFsyLDEsIlxcYm90IiwyXSxbMiwwLCJMaWZ0X3tMb3NzfVxcIFxcYm90IiwwLHsic3R5bGUiOnsiYm9keSI6eyJuYW1lIjoiZGFzaGVkIn19fV0sWzAsMSwiTG9zcyJdLFszLDAsIlxcZXRhIiwyLHsic2hvcnRlbiI6eyJzb3VyY2UiOjIwfX1dXQ==
\[\begin{tikzcd}
	& D \\
	{\mathbf{1}} && \Phi
	\arrow["Loss", from=1-2, to=2-3]
	\arrow["{Lift_{Loss}\ \bot}", dashed, from=2-1, to=1-2]
	\arrow[""{name=0, anchor=center, inner sep=0}, "\bot"', from=2-1, to=2-3]
	\arrow["\eta"', shorten <=3pt, Rightarrow, from=0, to=1-2]
\end{tikzcd}\]
\end{proposition}

\begin{proof}
By the definition of a left Kan lift, there is a universal natural transform $\eta : \bot \Rightarrow Loss \circ (Lift_{loss}\ \bot)$ meaning that for any other lift $F : \mathbf{1} \rightarrow D$ with natural transform $\gamma : \bot \Rightarrow Loss \circ F$ there is a unique natural transform $\epsilon : (Lift_{loss}\ \bot) \Rightarrow F$ such that $ (Loss \cdot \epsilon)\eta = \gamma$. As $\mathbf{1}$ is the terminal category, given any element $d \in D$ one can produce an $F$ such that $F(*) = d$. Since, $\bot$ is the bottom element of $\Phi$ then $\bot \leq_S Loss(F(*))=Loss(d)$ implying that there exists an $f : \bot \rightarrow Loss(F(*))$ such that $\leq(f) = Id_*$. Construct $\gamma : \bot \rightarrow Loss\circ F$, using $f$ as its single morphism $\gamma_* = f$.

The existence of an $\epsilon$ corresponding to this $F$ and $\gamma$ implies a morphism $(Loss \cdot \epsilon)_* : Loss(Lift_{loss}\ \bot(*)) \rightarrow Loss(d)$ such that $(Loss \cdot \epsilon)_*\eta_* = \gamma_* = f$.
\begin{align*}
Id_* &= \leq(f) =\ \leq((Loss \cdot \epsilon)_*\eta_*) = \ \leq((Loss \cdot \epsilon)_*)\ \wedge\ \leq(\eta_*)\\
&\implies \leq((Loss \cdot \epsilon)_*) = Id_*\\
&\implies Loss(Lift_{Loss}\ \bot(*)) \leq_S Loss(d)
\end{align*}
Meaning that for all $d \in D$ that $Loss(Lift_{loss}\ \bot(*)) \leq_S Loss(d)$  making $Lift_{Loss}\ \bot(*)$ a global minima of $Loss$.
\end{proof}