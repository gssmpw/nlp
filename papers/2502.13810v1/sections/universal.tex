\section{Universal representation}

It has been shown that Kan extensions represent global error minimisers for category-theoretic error minimisation problems, but it may not be clear that this also applies to the set-theoretic error minimisation problem. It is actually possible to convert any set-theoretic error minimisation problem into a category-theoretic error minimisation problem, namely as an extension problem in a 2-category, such that the left Kan extensions of the extension problem are exactly the global error minimisers of the set-theoretic error minimisation problem.

\begin{theorem}[Machine Learning representation]
\label{theorem:universal_ml_representation}
Given a set theoretic error minimisation problem (Def \ref{definition:set_error}) there exists a 2-category $\mathbb{T}$ such that $M = \mathbb{T}(\mu, \tau)$, $D = \mathbb{T}(\delta, \tau)$, $Inf = \mathbb{T}(\iota, \tau)$ and an object $m \in M$ is a global error minimiser with respect to $d$ if and only if $m \cong Lan_\iota d$
\label{prop:mlrepresentation}

\end{theorem}

\begin{proof}
Construct $\mathbb{T}$ to have three objects, $\mu$, $\delta$, and $\tau$. The hom objects will be selected such that $Inf$ becomes a composition morphism, and the 2-morphisms (morphisms of the hom category) are constructed to artificially select a minimising element if it exists.

Define the following singleton categories \[\mathbf{1} \cong \{\iota\} \cong \{Id_\mu\} \cong  \{Id_\delta\} \cong \{Id_\tau\}\]
Define $\mathbf{M}$ such that $Obj(\mathbf{M}) = M$ and that for any $m, m' \in \mathbf{M}$ there is a unique morphism $\sim \ : m \rightarrow m'$ if and only if $Inf(m) = Inf(m')$. Define $\mathbf{D}$ to be the category whose objects are the elements of $D$. Let $U \subseteq D$ be the subset of datasets for which an error minimising model exists, and let $Alg : U \rightarrow M$ be a function which selects an error minimising model for each $d \in D$ under the constraint that if there exists an $m\in \mathbf{M}$ such that $Inf(m) = d$, then $Alg(d) \cong m$. 

Define the hom sets of $\mathbf{D}$ with the following piecewise function.
\begin{gather*}
\mathbf{D}(d, d') :=
\begin{cases} 
      \{Id_d\} & d = d' \\
       \{*\} & d \in U \wedge  d' = Inf(Alg(d)) \wedge d \neq d'\\
       \emptyset & else\\
   \end{cases}
\end{gather*}

Composition is defined in the obvious way. For objects $d,d',d''\in D$ consider the form of the composition morphism. \[\circ_{d,d',d''} : \mathbf{D}(d,d') \times \mathbf{D}(d',d'') \rightarrow \mathbf{D}(d,d'')\]

Whenever $\mathbf{D}(d, d')$ or $\mathbf{D}(d', d'')$ is empty, then the product is empty, making the composition morphism the unique map from the empty set. When both $\mathbf{D}(d, d')$ or $\mathbf{D}(d', d'')$ are non empty, they must both be singleton. Therefore the following must be true.
\begin{align*}
&d = d' \vee (d' = Inf(Alg(d)) \wedge d \neq d')\\
&d' = d'' \vee (d'' = Inf(Alg(d')) \wedge d' \neq d'')\\
\end{align*}
Which may be simplified to form the following.
\begin{align*}
&d = d' \vee d' = Inf(Alg(d))\\
&d' = d'' \vee d'' = Inf(Alg(d'))\\
\end{align*}
Combining these statements produces the following deduction.
\begin{align*}
&\mathbf{D}(d,d') \times \mathbf{D}(d',d'') \cong \mathbf{1}\\
\implies&(d = d' \vee d' = Inf(Alg(d)))\\
&\wedge(d' = d'' \vee d'' = Inf(Alg(d')))\\
\implies & (d=d' \wedge d' = d'')\\
& \vee(d = d' \wedge d'' = Inf(Alg(d')))\\
& \vee(d' = Inf(Alg(d)) \wedge  d' = d'')\\
& \vee(d' = Inf(Alg(d)) \wedge d'' = Inf(Alg(d')))\\
\implies & (d = d'')\\
& \vee d'' = Inf(Alg(d))\\
& \vee(d' = Inf(Alg(d)) \wedge d'' = Inf(Alg(d')))\\
\end{align*}
When $d' = Inf(Alg(d))$ then for $m = Alg(d)$ \[Err(Inf(m), d') = Err(Inf(Alg(d)), d') = Err(d', d') = 0\] By the definition of $Alg$ this forces $Alg(d') \cong m = Alg(d')$, which by the construction of $\mathbf{M}$ means that $Inf(Alg(d')) = Inf(Alg(d)) = d'$. This allows the deduction to be simplified to the following implication.
\begin{align*}
&\mathbf{D}(d,d') \times \mathbf{D}(d',d'') \cong \mathbf{1}\\
\implies & (d = d'')\vee d'' = Inf(Alg(d))\\
&\vee(d' = Inf(Alg(d)) \wedge d'' = Inf(Alg(d')))\\
\implies & (d = d'')\vee d'' = Inf(Alg(d))\\
&\vee(d' = Inf(Alg(d)) \wedge d'' = d'))\\
\implies & (d = d'')\vee d'' = Inf(Alg(d))\\
\implies & \mathbf{D}(d,d'') \cong \mathbf{1}
\end{align*}

Making the composition morphism in this case the unique morphism between singleton sets.

Using the above defined categories, define the hom-categories of $\mathbb{T}$ as follows.

\begin{center}
\begin{tabular}{c|ccc}
$\mathbb{T}(-,-)$ & $\mu$ & $\delta$ & $\tau$\\
\hline
 $\mu$ & $\{Id_\mu\}$ & $\emptyset$ & $\mathbf{M}$\\
 $\delta$ & $\{\iota\}$ & $\{Id_\delta\}$ & $\mathbf{D}$\\
 $\tau$ & $\emptyset$& $\emptyset$ & $\{Id_\tau\}$ \\
\end{tabular}
\end{center}

The only composition morphism which is not fixed by identity laws or the empty categories is the following \[\circ_{\delta, \mu, \tau} : \mathbb{T}(\delta, \mu) \times \mathbb{T}(\mu, \tau) \rightarrow \mathbb{T}(\delta, \tau)\]

Substituting the known hom objects, this is rewritten as.

\[\circ_{\delta, \mu, \tau} : \{\iota \} \times \mathbf{M} \rightarrow \mathbf{D}\]

Because $\{\iota \} \times \mathbf{M} \cong \mathbf{M}$, the composition morphism can be defined by the inference function which maps all morphisms of $\mathbf{M}$ to the relevant identity morphisms \[\circ_{\delta, \mu, \tau} := Inf\] Finally, consider the following Kan extension problem in $\mathbb{T}$.

% https://q.uiver.app/#q=WzAsMyxbMCwxLCJcXGRlbHRhIl0sWzIsMSwiXFx0YXUiXSxbMSwwLCJcXG11Il0sWzAsMiwiXFxpb3RhIl0sWzAsMSwiZCIsMl0sWzIsMSwibSIsMCx7InN0eWxlIjp7ImJvZHkiOnsibmFtZSI6ImRhc2hlZCJ9fX1dXQ==
\[\begin{tikzcd}
	& \mu \\
	\delta && \tau
	\arrow["m", dashed, from=1-2, to=2-3]
	\arrow["\iota", from=2-1, to=1-2]
	\arrow["d"', from=2-1, to=2-3]
\end{tikzcd}\]

If an error minimising $m$ does not exist for the given $d$ then no Kan extension can exist as there is no morphism from $d$ into the image of $Inf : \{\iota \} \times \mathbf{M} \rightarrow \mathbf{D}$. However, if an error minimising $m$ does exist then by construction there is a morphism in $\mathbf{D}$ and consequently a 2-morphism in $\mathbb{T}$ of the form $d \Rightarrow Alg(d)\iota = Inf(Alg(d))$. For any $m$ for which there also exists a 2-morphism $d \Rightarrow m$ then as such a 2-morphism from $d$ into the image of $Inf$ is unique, then $m = Inf(Alg(d))$, which by the construction of $\mathbf{M}$ means that $Alg(d) \cong m$. This makes $Alg(d)$, when it exists, a left Kan extension in $\mathbb{T}$
\end{proof}