\section{Background}
\subsection{Categories, Adjunctions, and Kan Extensions}

The definitions of categories, functors, natural transforms, adjunctions, and Kan extensions are found in all of the following resources. \citep{riehlCategoryTheoryContext2016, fongSevenSketchesCompositionality2018, leinsterBasicCategoryTheory2016}. The definition of a 2-category is adapted from its definition as an enriched category \cite{Kelly2005}.

A category is a collection of objects and morphisms where every morphism has a domain object and codomain object. Two morphisms may be composed if the domain of one equals the codomain of the other.
\begin{definition}[Category]
A category $C$ consists of a class of objects $Ob(C)$, and between any two objects $x,y \in Ob(C)$ a class of morphisms $C(x,y)$ such that:
\begin{itemize}[noitemsep,topsep=-8pt]
    \item Any pair $f \in C(x,y)$ and $g \in C(y, z)$ can be composed to form $gf \in C(x, z)$.
    \item Composition is associative: $(hg)f=h(gf)$.
    \item Every object $x \in Ob(C)$ has an identity morphism $Id_x \in C(x, x)$.
    \item for any $f \in \mathcal{C}(x, y)$ then $fId_x = f = Id_yf$.
\end{itemize}
\end{definition}

When clear from context, it is common to write $x \in Ob(C)$ as $x \in C$ and $f \in C(x,y)$ as $f:x\rightarrow y$. One example of a category is $Set$ whose objects are sets and whose morphisms are set functions. Morphisms are often considered to be structure preserving maps. As sets have no structure by design, their morphisms are just functions. An example of a morphism between categories is a functor.

\begin{definition}[Functor]
A functor $F:C\rightarrow D$, between categories $C$ and $D$ sends every object $x\in Ob(C)$ to $F(x) \in Ob(D)$, and every morphism $f\in C(x, y)$ to $F(f)\in D(F(x), F(y))$ such that:
\begin{itemize}[noitemsep,topsep=-8pt]
    \item $F$ preserves composition: $F(gf) = F(g)F(f)$
    \item $F$ preserves identities: $F(Id_x) = Id_{F(x)}$
\end{itemize}
\end{definition}

The product of two categories $C$ and $D$ may be written as $C\times D$. Its objects are pairs of objects from $C$ and $D$, and its morphisms are pairs of morphisms.
\begin{equation}
f\in C(x,y) \wedge g \in D(w,z) \implies (f,g) \in C\times D((x,w),(y,z))
\end{equation}
The unit of the categorical product is the category $\mathbf{1}$ which has a single object and a single morphism (which is the identity of its object). The categorical product of a category $C$ with $\mathbf{1}$ is isomorphic to $C$, meaning that there exists an invertible functor from the product into $C$. These invertible functors are referred to as the left and right unitors $l$ and $r$.
\begin{gather}
l : \mathbf{1} \times C \rightarrow C\\
r : C\times \mathbf{1} \rightarrow C
\end{gather}

The left and right unitors simply drop the single object from pairs of object in $C\times\mathbf{1}$ or $\mathbf{1}\times C$. I.e. $l(*,x) = x$ and $r(x,*) = x$. The categorical product is also associative, as described by the existence of an invertible morphism $\alpha$ for triple of objects composed using the categorical product.
\begin{equation}
\alpha : (C\times D)\times E \rightarrow C \times (D \times E)
\end{equation}
$\alpha$ simply rewrites nested tuples, $\alpha((x,y),z) = (x,(y,z))$.

As well as morphisms between categories it is also possible to consider the existence of morphisms between functors, called natural transforms.

\begin{definition}[Natural Transform]
Given functors $F,G: C\rightarrow D$ between categories $C$ and $D$, a natural transformation $\eta:F \Rightarrow G$ is a family of morphisms $\eta_x : F(x) \rightarrow G(x)$ in $D$ for each object $x \in Ob(C)$, such that $G(f)\eta_x = \eta_{y}F(f)$ for any $f\in D(x,y)$, i.e. the following diagram commutes:
% https://q.uiver.app/#q=WzAsNCxbMCwwLCJGKHgpIl0sWzEsMCwiRyh4KSJdLFswLDEsIkYoeSkiXSxbMSwxLCJHKHkpIl0sWzEsMywiRyhmKSJdLFsyLDMsIlxcZXRhX3kiLDJdLFswLDIsIkYoZikiLDJdLFswLDEsIlxcZXRhX3giXV0=
\[\begin{tikzcd}
	{F(x)} & {G(x)} \\
	{F(y)} & {G(y)}
	\arrow["{\eta_x}", from=1-1, to=1-2]
	\arrow["{F(f)}"', from=1-1, to=2-1]
	\arrow["{G(f)}", from=1-2, to=2-2]
	\arrow["{\eta_y}"', from=2-1, to=2-2]
\end{tikzcd}\]
\end{definition}

A natural transform is a morphism between morphisms, referred to as a 2-morphism, whereas a morphism between objects is a 1-morphism. When the definition of a category is extended to include 2-morphisms it is referred to as a 2-category. An example of a 2-category is $Cat$, whose objects are categories, 1-morphisms are functors, and 2-morphisms are natural transforms. Given 1-morphisms $f: x\rightarrow y$ and $g :x \rightarrow y$ a two morphism $\eta$ from $f$ to $g$ may be written as $\eta : f \Rightarrow g$. Rather than hom classes a 2-category has hom-categories. It is more concise to present the definition of a 2-category using a composition functor and to present the identity morphisms with a functor $J_x : \mathbf{1} \rightarrow C(x,x)$. The functor $J_x$ selects on object of $C(x,x)$, were $J_x(*)=Id_x$. This also introduces an identity 2-morphism $Id_f$ for any 1-morphism $f: x \rightarrow y$.

\begin{definition}[2-category]
A 2-category $C$ consists of a class of objects $Ob(C)$, and between any two objects $x,y \in Ob(C)$ a 1-category of morphisms $C(x,y)$ such that:
\begin{itemize}[noitemsep,topsep=-8pt]
    \item For any triple of objects $x,y,z \in Ob(C)$ there is a composition functor $\circ_{x,y,z} : C(y,z) \times C(x,y) \rightarrow C(x,z)$.
    \item Composition is associative: $\circ_{x,y,w}(\circ_{y,z,w}\times Id_{C(x,y)}) = \circ_{x,z,w}(Id_{C(z,w)}\times \circ_{x,y,z})\alpha$.
    \item Every object $x \in Ob(C)$ has an identity morphism $J_x : \mathbf{1} \rightarrow C(x, x)$.
    \item $\circ_{x,y,y}(J_x \times C(x,y)) = l$ and $\circ_{x,y,y}(C(y,x) \times J_y) = r$
\end{itemize}
\end{definition}

The reason for writing the definition of a 2-category using functors rather than listing the axioms of its composition of 1-morphisms and 2-morphisms is because there is a long list of axioms which are just a consequence of its composition being functorial. For example, the horizontal and vertical composition of 2-morphisms. Because 2-morphisms are 1-morphisms of their hom categories, two 2-morphisms $\eta : f\Rightarrow g$ and $\gamma : g \Rightarrow h$ may be vertically composed to form $\gamma\eta : f \Rightarrow h$. Whereas, if the 2-morphisms are side by side they may be horizontally composed via the composition functor

% https://q.uiver.app/#q=WzAsNSxbMCwwLCJ4Il0sWzEsMCwieSJdLFsyLDAsInoiXSxbMywwLCJ4Il0sWzUsMCwieiJdLFswLDEsIiIsMCx7ImxhYmVsX3Bvc2l0aW9uIjo2MCwiY3VydmUiOi0yfV0sWzAsMSwiIiwyLHsiY3VydmUiOjJ9XSxbMSwyLCIiLDIseyJjdXJ2ZSI6LTJ9XSxbMSwyLCIiLDIseyJjdXJ2ZSI6Mn1dLFszLDQsIiIsMCx7ImN1cnZlIjotMn1dLFszLDQsIiIsMix7ImN1cnZlIjoyfV0sWzUsNiwiXFxldGEiLDAseyJzaG9ydGVuIjp7InNvdXJjZSI6MjAsInRhcmdldCI6MjB9fV0sWzcsOCwiXFxnYW1tYSIsMCx7InNob3J0ZW4iOnsic291cmNlIjoyMCwidGFyZ2V0IjoyMH19XSxbOSwxMCwiXFxnYW1tYSBcXGNpcmNcXGV0YSIsMCx7InNob3J0ZW4iOnsic291cmNlIjoyMCwidGFyZ2V0IjoyMH19XV0=
\[\begin{tikzcd}
	x & y & z & x && z
	\arrow[""{name=0, anchor=center, inner sep=0}, curve={height=-12pt}, from=1-1, to=1-2]
	\arrow[""{name=1, anchor=center, inner sep=0}, curve={height=12pt}, from=1-1, to=1-2]
	\arrow[""{name=2, anchor=center, inner sep=0}, curve={height=-12pt}, from=1-2, to=1-3]
	\arrow[""{name=3, anchor=center, inner sep=0}, curve={height=12pt}, from=1-2, to=1-3]
	\arrow[""{name=4, anchor=center, inner sep=0}, curve={height=-12pt}, from=1-4, to=1-6]
	\arrow[""{name=5, anchor=center, inner sep=0}, curve={height=12pt}, from=1-4, to=1-6]
	\arrow["\eta", shorten <=3pt, shorten >=3pt, Rightarrow, from=0, to=1]
	\arrow["\gamma", shorten <=3pt, shorten >=3pt, Rightarrow, from=2, to=3]
	\arrow["{\gamma \circ\eta}", shorten <=3pt, shorten >=3pt, Rightarrow, from=4, to=5]
\end{tikzcd}\]

A 1-morphism may be composed with a 2-morphism through the process of left or right whiskering. This is simply the horizontal composition of the 2-morphism with the identity of the 1-morphism.
%
% https://q.uiver.app/#q=WzAsMTQsWzMsMCwieCJdLFs0LDAsInoiXSxbMSwwLCJ5Il0sWzAsMCwieCJdLFsyLDAsInoiXSxbMCwxLCJ4Il0sWzEsMSwieSJdLFsyLDEsInoiXSxbMywxLCJ4Il0sWzQsMSwieiJdLFs1LDAsIngiXSxbNiwwLCJ6Il0sWzUsMSwieCJdLFs2LDEsInoiXSxbMCwxLCIiLDAseyJjdXJ2ZSI6LTN9XSxbMCwxLCIiLDIseyJjdXJ2ZSI6M31dLFszLDIsImYiXSxbMiw0LCIiLDIseyJjdXJ2ZSI6Mn1dLFsyLDQsIiIsMCx7ImN1cnZlIjotMn1dLFs1LDYsIiIsMix7ImN1cnZlIjoyfV0sWzUsNiwiIiwwLHsiY3VydmUiOi0yfV0sWzYsNywiZyJdLFs4LDksIiIsMix7ImN1cnZlIjozfV0sWzgsOSwiIiwwLHsiY3VydmUiOi0zfV0sWzEwLDExLCIiLDAseyJjdXJ2ZSI6M31dLFsxMCwxMSwiIiwyLHsiY3VydmUiOi0zfV0sWzEyLDEzLCIiLDIseyJjdXJ2ZSI6M31dLFsxMiwxMywiIiwwLHsiY3VydmUiOi0zfV0sWzE0LDE1LCJcXGdhbW1hIFxcY2lyYyBJZF9mIiwxLHsic2hvcnRlbiI6eyJzb3VyY2UiOjIwLCJ0YXJnZXQiOjIwfX1dLFsxOCwxNywiXFxnYW1tYSIsMCx7InNob3J0ZW4iOnsic291cmNlIjoyMCwidGFyZ2V0IjoyMH19XSxbMjAsMTksIlxcZXRhIiwwLHsic2hvcnRlbiI6eyJzb3VyY2UiOjIwLCJ0YXJnZXQiOjIwfX1dLFsyMywyMiwiSWRfZyBcXGNpcmMgXFxldGEiLDEseyJzaG9ydGVuIjp7InNvdXJjZSI6MjAsInRhcmdldCI6MjB9fV0sWzI1LDI0LCJcXGdhbW1hXFxjZG90IGYiLDEseyJzaG9ydGVuIjp7InNvdXJjZSI6MjAsInRhcmdldCI6MjB9fV0sWzI3LDI2LCJnXFxjZG90IFxcZXRhIiwxLHsic2hvcnRlbiI6eyJzb3VyY2UiOjIwLCJ0YXJnZXQiOjIwfX1dXQ==
\[\begin{tikzcd}[column sep=scriptsize,row sep=large]
	x & y & z & x & z & x & z \\
	x & y & z & x & z & x & z
	\arrow["f", from=1-1, to=1-2]
	\arrow[""{name=0, anchor=center, inner sep=0}, curve={height=12pt}, from=1-2, to=1-3]
	\arrow[""{name=1, anchor=center, inner sep=0}, curve={height=-12pt}, from=1-2, to=1-3]
	\arrow[""{name=2, anchor=center, inner sep=0}, curve={height=-18pt}, from=1-4, to=1-5]
	\arrow[""{name=3, anchor=center, inner sep=0}, curve={height=18pt}, from=1-4, to=1-5]
	\arrow[""{name=4, anchor=center, inner sep=0}, curve={height=18pt}, from=1-6, to=1-7]
	\arrow[""{name=5, anchor=center, inner sep=0}, curve={height=-18pt}, from=1-6, to=1-7]
	\arrow[""{name=6, anchor=center, inner sep=0}, curve={height=12pt}, from=2-1, to=2-2]
	\arrow[""{name=7, anchor=center, inner sep=0}, curve={height=-12pt}, from=2-1, to=2-2]
	\arrow["g", from=2-2, to=2-3]
	\arrow[""{name=8, anchor=center, inner sep=0}, curve={height=18pt}, from=2-4, to=2-5]
	\arrow[""{name=9, anchor=center, inner sep=0}, curve={height=-18pt}, from=2-4, to=2-5]
	\arrow[""{name=10, anchor=center, inner sep=0}, curve={height=18pt}, from=2-6, to=2-7]
	\arrow[""{name=11, anchor=center, inner sep=0}, curve={height=-18pt}, from=2-6, to=2-7]
	\arrow["\gamma", shorten <=3pt, shorten >=3pt, Rightarrow, from=1, to=0]
	\arrow["{\gamma \circ Id_f}"{description}, shorten <=5pt, shorten >=5pt, Rightarrow, from=2, to=3]
	\arrow["{\gamma\cdot f}"{description}, shorten <=5pt, shorten >=5pt, Rightarrow, from=5, to=4]
	\arrow["\eta", shorten <=3pt, shorten >=3pt, Rightarrow, from=7, to=6]
	\arrow["{Id_g \circ \eta}"{description}, shorten <=5pt, shorten >=5pt, Rightarrow, from=9, to=8]
	\arrow["{g\cdot \eta}"{description}, shorten <=5pt, shorten >=5pt, Rightarrow, from=11, to=10]
\end{tikzcd}\]

The results of this work concern the properties of adjunctions and left Kan extensions as error minimisers. Both of these constructions may be presented in any 2-category, but the definition of an adjunction specific to adjoint functors will be of more use. A loose intuition of an adjunction between two functors is that each adjoint functor serves as an approximate or pseudo inverse for the other.

\begin{definition}[Adjoint Functors (triangle)]
\label{def:AdjointFunctorTriangle}
Given two functors $L : C \rightarrow D$ and $R:D \rightarrow C$, $L$ is left adjoint to $R$, and $R$ is right adjoint to $L$ (written $L\dashv R$) if and only if there exists a natural transforms $\eta : Id_C \Rightarrow RL$, called the adjunction unit, and a natural transform $\epsilon : LR \Rightarrow Id_D$, called the adjunction counit, which given any $f:c\rightarrow R(d)$ in $C$ or $g:L(c)\rightarrow d$ in $D$ there exists $\tilde f : L(c) \rightarrow d$ in $D$ or $\tilde g : c \rightarrow R(d)$ in $C$ which are unique such that they satisfy the following commutative diagrams (triangle identities).
%
% https://q.uiver.app/#q=WzAsNyxbMSwwLCJjIl0sWzAsMSwiUkwoYykiXSxbMiwxLCJSKGQpIl0sWzQsMCwiZCJdLFszLDEsIkwoYykiXSxbNSwxLCJMUihkKSJdLFszLDJdLFswLDIsImYiXSxbMCwxLCJcXGV0YV9jIiwyXSxbNCwzLCJnIl0sWzQsNSwiTChcXHRpbGRlIGcpIiwyXSxbNSwzLCJcXHZhcmVwc2lsb25fZCIsMl0sWzEsMiwiUihcXHRpbGRlIGYpIiwyXV0=
\[\begin{tikzcd}[column sep=small]
	& c &&& d \\
	{RL(c)} && {R(d)} & {L(c)} && {LR(d)} \\
	&&& {}
	\arrow["{\eta_c}"', from=1-2, to=2-1]
	\arrow["f", from=1-2, to=2-3]
	\arrow["{R(\tilde f)}"', from=2-1, to=2-3]
	\arrow["g", from=2-4, to=1-5]
	\arrow["{L(\tilde g)}"', from=2-4, to=2-6]
	\arrow["{\varepsilon_d}"', from=2-6, to=1-5]
\end{tikzcd}\]
%
Where $\tilde f$ is the adjunct of $f$ constructed via $\tilde f := \epsilon_dL(f)$. and $\tilde g$ is the adjunction of $g$ constructed via $\tilde g := R(g)\eta_c$
\end{definition}

\begin{definition}[Left Kan Extension (local)]
\label{def:left_kan_extension:local}
Given 1-morphisms $K:C\rightarrow E$, $G:C\rightarrow D$, a left Kan extension of $K$ along $G$ is a 1-morphism $Lan_GK:D\rightarrow E$ together with a 2-morphism $\eta:K \Rightarrow (Lan_GK)G$ such that for any other such pair $(H:D\rightarrow E,\gamma:K\Rightarrow HG)$, There exists a 2-morphism $\alpha : Lan_GK \Rightarrow H$ such that $\gamma = (\alpha \cdot G)\eta$.
\[\begin{tikzcd}                                   & D \arrow[rd, "Lan_GK", dashed]    &   \\
C \arrow[rr, "K"'] \arrow[ru, "G"] & {} \arrow[u, "\eta"', Rightarrow] & E
\end{tikzcd}\]
\end{definition}

\subsection{Monoids, Preorders, and Lax 2-Functors}

The categorical description of error presented in this paper is defined using a monoidal preorder. Though this structure can be described without the use of category theory, it is presented as a kind of 2-category so that it may interact with other categorical components. Examples of the categorical definitions of otherwise describable objects are that of the Monoid and the Preorder.

\begin{definition}[Monoid]
A monoid $C$ is a category with a single object. $Ob(C) = \{*\}$.
\end{definition}

\begin{definition}[Preorder]
A preorder $C$ is a category with at most one morphism between any two objects.\[\forall x,y \in C(f,g \in C(x,y) \implies f=g)\]
\end{definition}

\begin{remark}
The standard definition of a preorder as a transitive and reflexive relation can be recovered by taking $x \leq y$ if and only if there exists a morphism $f:x\rightarrow y$.
\end{remark}

Though the definition of error presented later does not necessarily use the real numbers, it does require that whatever order structure is used to compare errors has a bottom or least quantity of error.

\begin{definition}[Bottom Element]
Given a preorder $P$, an element $\bot \in P$ is a bottom element of $P$ if for all $x \in P$, $\bot \leq x$.
\end{definition}

Understanding how a monoid and a preorder may be defined from a categorical perspective makes the interpretation of the definition of a monoidal preorder more apparent.

\begin{definition}[Monoidal Preorder]
\label{def:monoidal_preorder}
A single object 2-category with at most one 2-morphism between any pair of 1-morphisms.
\end{definition}

\cite{johnson_2-dimensional_2020}
\begin{definition}[Lax 2-Functor between 2-categories]
\label{definition:Lax2Functor}
A Lax 2-functor $F:C\rightarrow D$, sends every object $x\in Ob(C)$ to $F(x) \in Ob(D)$, it has component functors  $F_{xy} : C(x, y) \rightarrow D(F(x), F(y))$ and the following natural transforms:
\begin{gather}
\phi : \circ_{F(x),F(y),F(z)}(F_{y,z} \times F_{x,y}) \Rightarrow F_{x,z}\circ_{x,y,z}\\
\psi : J_{F(x)} \Rightarrow FJ_x
\end{gather}
Which for all $f\in C(w,x)$, $g\in C(x,y)$, and $h\in C(y,z)$ satisfy the following constraints.
\begin{itemize}[noitemsep,topsep=-8pt]
    \item $\phi_{h,gf}(Id_{F(h)} \circ \phi_{g,f}) = \phi_{hg,f}(\phi_{h,g}\circ Id_{F(f)})$
    \item $\phi_{Id_x,f}(\psi_{x} \circ Id_{F(f)}) = Id_{F(f)}$
    \item $\phi_{f,Id_w}(Id_{F(f)}\circ \psi_w)$
\end{itemize}
\end{definition}

For the purposes of this paper, the relevant consequence of the definition of a lax 2-functor is, due to the natural transforms $\phi$ there exists a 2-morphism $\phi_{g,f} : F(g)F(f) \Rightarrow F(gf)$ in $D$ for any composable morphisms $f$ and $g$ in $C$. When the codomain of the Lax 2-functor is a monoidal preorder, the existence of a 2-morphism in the codomain can be reframed as a statement about the ordering of 1-morphisms.

\begin{proposition}
Given a monoidal preorder $S$ and a lax functor $F : P \rightarrow S$ then for composable morphisms $f$ and $g$ in $P$. \[F(g)F(f) \leq F(gf)\]
\end{proposition}