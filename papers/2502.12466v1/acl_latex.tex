% This must be in the first 5 lines to tell arXiv to use pdfLaTeX, which is strongly recommended.
\pdfoutput=1
% In particular, the hyperref package requires pdfLaTeX in order to break URLs across lines.
 
\documentclass[11pt]{article}

\usepackage[table, x11names]{xcolor}

% Change "review" to "final" to generate the final (sometimes called camera-ready) version.
% Change to "preprint" to generate a non-anonymous version with page numbers.
\usepackage[preprint]{acl}

% Standard package includes
\usepackage{times}
\usepackage{latexsym}
\usepackage{xspace}
\usepackage{booktabs}

\usepackage{amsmath}


\usepackage{listings}


% For proper rendering and hyphenation of words containing Latin characters (including in bib files)
\usepackage[T1]{fontenc}
% For Vietnamese characters
% \usepackage[T5]{fontenc}
% See https://www.latex-project.org/help/documentation/encguide.pdf for other character sets

% This assumes your files are encoded as UTF8
\usepackage[utf8]{inputenc}

% This is not strictly necessary, and may be commented out,
% but it will improve the layout of the manuscript,
% and will typically save some space.
\usepackage{microtype}

% This is also not strictly necessary, and may be commented out.
% However, it will improve the aesthetics of text in
% the typewriter font.
\usepackage{inconsolata}

%Including images in your LaTeX document requires adding
%additional package(s)
\usepackage{graphicx}


\usepackage{multirow}
\usepackage{hhline}
\usepackage{cleveref}

% If the title and author information does not fit in the area allocated, uncomment the following
%
%\setlength\titlebox{<dim>}
%
% and set <dim> to something 5cm or larger.


%%%%%%%%%%%%%%%%%%%%%%%%%%%%%%%%%%%%%%%%%%%%%%%%%%%%%%%
%%%%%%%%%%%%%%%    theorems %%%%%%%%%%%%%%%%%%%%%%%%%%%
%%%%%%%%%%%%%%%%%%%%%%%%%%%%%%%%%%%%%%%%%%%%%%%%%%%%%%%
% \usepackage{mdframed}
\usepackage{kantlipsum}

%%%%%%%%%%%%%%%%%%%%%%%%%%%%%%%%%%%%%%%%%%%%%%%%%%%%%%%
%%%%%%%%%%%%%%%    theorems %%%%%%%%%%%%%%%%%%%%%%%%%%%
%%%%%%%%%%%%%%%%%%%%%%%%%%%%%%%%%%%%%%%%%%%%%%%%%%%%%%%
\theoremstyle{plain}
\newtheorem{theorem}{Theorem}[section]
\newtheorem{proposition}[theorem]{Proposition}
\newtheorem{lemma}[theorem]{Lemma}
\newtheorem{example}[theorem]{Example}
\newtheorem{corollary}[theorem]{Corollary}
\theoremstyle{definition}
\newtheorem{definition}[theorem]{Definition}
\newtheorem{assumption}[theorem]{Assumption}
\theoremstyle{remark}
\newtheorem{remark}[theorem]{Remark}


% \titleformat{\subsection}[runin]% runin puts it in the same paragraph
%        {\normalfont\bfseries}% formatting commands to apply to the whole heading
%        {\thesubsection}% the label and number
%        {0.5em}% space between label/number and subsection title
%        {}% formatting commands applied just to subsection title
%        [.]% punctuation or other commands following subsection title


%%%%%%%%%%%%%%%%%%%%%%%%%%%%%%%%%%%%%%%%%%%%%%%%%%%%%%%
%%%%%%%%%%%%%%%  mathematical notations%%%%%%%%%%%%%%%%
% \usepackage[english]{babel}
% \usepackage[utf8]{inputenc}
% \usepackage[T1]{fontenc}

%% Figures, tables and lists
\usepackage[dvipsnames]{xcolor}
\usepackage{paralist}
\usepackage{graphicx}
\usepackage{subcaption}
\usepackage{longtable} 
\usepackage{multirow}
\usepackage{listings}
\usepackage{makecell}
\usepackage{array}
\usepackage{float}
\usepackage{dsfont}
\usepackage{rotating}
\usepackage{booktabs}
\usepackage{enumerate}
\usepackage{tikz}
\usepackage{pgf}
\usepackage{enumitem}
\usepackage{lipsum} % for generating filler text
\usepackage{titlesec}

%% Math
% \usepackage{amssymb, amsthm,bbm}
\usepackage{mathtools}
\usepackage{mathrsfs}
%% References and author info 
\mathtoolsset{showonlyrefs}
\usepackage{natbib}
\usepackage{authblk}
\usepackage{todonotes}
\usepackage{xr-hyper}


%%%%%%%%%%%%%%%%%%%%%%%%%%%%%%%%%%%%%%%%%%%%%%%%%%%%%%%
\newcommand{\R}{\mathbb R}
\newcommand{\EE}{\mathbb{E}}

\DeclareMathOperator{\Tr}{Tr}
\DeclareMathOperator*{\argmin}{argmin}
\DeclareMathOperator*{\argmax}{argmax}

\newcommand{\bs}[1]{\ensuremath{\boldsymbol{#1}}}
\newcommand{\mc}{\mathcal}
\newcommand{\opt}{^\star}


\newcommand{\diff}{\textnormal{d}}


\def \iid {\stackrel{\textnormal{i.i.d.}}{\sim}}
\def \iidtext {\textnormal{i.i.d.}}





%%%%%%%%%%%%%%%%%%%%%%%%%%%%%%%%%%%%%%%%%%%%%%%%%%%%%%%
%%%%%%%%%%%%%%%%%%%%% colors     %%%%%%%%%%%%%%%%%%%%%%
%%%%%%%%%%%%%%%%%%%%%%%%%%%%%%%%%%%%%%%%%%%%%%%%%%%%%%%
\definecolor{myblue}{rgb}{.8, .8, 1}
\definecolor{mathblue}{rgb}{0.2472, 0.24, 0.6} % mathematica's Color[1, 1--3]
\definecolor{mathred}{rgb}{0.6, 0.24, 0.442893}
\definecolor{mathyellow}{rgb}{0.6, 0.547014, 0.24}


% May add more in future.







\title{EquiBench: Benchmarking Code Reasoning Capabilities of Large Language Models via Equivalence Checking}

\author{
\\
\textbf{Anjiang Wei}\textsuperscript{1} \hspace{1em} \textbf{Jiannan Cao}\textsuperscript{2} \hspace{1em} \textbf{Ran Li}\textsuperscript{1,3} \hspace{1em} \textbf{Hongyu Chen}\textsuperscript{4} \hspace{1em} \textbf{Yuhui Zhang}\textsuperscript{1} \\
\textbf{Ziheng Wang}\textsuperscript{1} \hspace{1em} \textbf{Yaofeng Sun}\textsuperscript{5} \hspace{1em} \textbf{Yuan Liu}\textsuperscript{3} \hspace{1em} \textbf{Thiago S. F. X. Teixeira}\textsuperscript{6} \\
\textbf{Diyi Yang}\textsuperscript{1} \hspace{1em} \textbf{Ke Wang}\textsuperscript{7} \hspace{1em} \textbf{Alex Aiken}\textsuperscript{1} \\
\textsuperscript{1}Stanford University \hspace{1em} \textsuperscript{2}MIT \hspace{1em} \textsuperscript{3}Google \hspace{1em} \textsuperscript{4}Nanjing University \\
\textsuperscript{5}DeepSeek \hspace{1em} \textsuperscript{6}Intel \hspace{1em} \textsuperscript{7}Visa Research \\
\texttt{\{anjiang,aiken\}@cs.stanford.edu}
}

\begin{document}
\maketitle
\begin{abstract}

% involved macros:
% \name, \numpair, \numllm, \sotaacc{}, \sotacuda{}, \sotadce{}

Equivalence checking, i.e., determining whether two programs produce identical outputs for all possible inputs, underpins a broad range of applications, including software refactoring, testing, and optimization. We present the task of equivalence checking as a new way to evaluate the code reasoning abilities of large language models (LLMs). We introduce EquiBench, a dataset of 2400 program pairs spanning four programming languages and six equivalence categories. These pairs are systematically generated through program analysis, compiler scheduling, and superoptimization, covering nontrivial structural transformations that demand deep semantic reasoning beyond simple syntactic variations. Our evaluation of 17 state-of-the-art LLMs shows that OpenAI o3-mini achieves the highest overall accuracy of 78.0\%. In the most challenging categories, the best accuracies are 62.3\% and 68.8\%, only modestly above the 50\% random baseline for binary classification, indicating significant room for improvement in current models' code reasoning capabilities.

\end{abstract}

\section{Introduction}
\label{sec:intro}
\section{Introduction}
\label{sec:intro}
% Image editing methods in diffusion models depend on user-defined control directions - users can unlock their creativity using these methods by specifying the desired manipulation through prompts~\cite{gandikota2023concept}, reference images~\cite{ruiz2022dreambooth, kumari2022customdiffusion, gal2022image, chen2024trainingfreeregionalpromptingdiffusion}, or attribute vectors~\cite{parmar2023zero,hertz2022prompt}. In this work, we ask a fundamentally different question: \emph{Can we automatically discover the underlying visual structure of a concept within diffusion model's knowledge?} %Rather than requiring user-specified controls, we aim to decompose the model's internal knowledge into meaningful directions.

% This question touches on a fundamental limitation in how we interact with diffusion models. Current control methods ~\cite{zhang2023addingconditionalcontroltexttoimage, gandikota2023concept, ye2023ipadaptertextcompatibleimage,ye2023ipadaptertextcompatibleimage, hertz2024stylealignedimagegeneration, li2023photomaker, shi2024instantbooth, chen2024trainingfreeregionalpromptingdiffusion} require users to specify their desired manipulations in advance, limiting interactive creativity. This contrasts with natural human artistic workflows, where creators dynamically explore creative ideas while jointly refining them toward meaningful artistic outcomes~\cite{hoffmann2016modeling}. This synergy between specification and exploration is not new to generative models. Early GAN architectures naturally developed disentangled latent spaces that enabled continuous\cite{harkonen2020ganspace,radford2015unsupervised, wu2021stylespace, shen2020interfacegan}, compositional control over generated images. Users could explore these spaces to discover interesting variations that would be difficult to describe in words~\cite{wu2021stylespace}, then combine them to achieve their creative goals~\cite{grabe2022towards}. 


% While diffusion models have largely superseded GANs in conditional image synthesis~\cite{dhariwal2021diffusion},  their underlying structure remains less understood. Diffusion models achieve remarkable diversity through high-dimensional latents, unlike GANs' compact latent spaces.  With a single prompt, diffusion models can generate radically different variations through different random initializations of input noise. We ask - Is it possible to discover interpretable structure within this vast space of variations?

Text-to-image diffusion models are capable of generating remarkable visual variations from a single prompt through different random initializations. However, this vast creative potential remains largely opaque to users---while we can generate diverse images, we lack understanding of the underlying structure of these variations. This presents a fundamental challenge: how can we discover and expose the latent visual capabilities encoded within these models?

\let\thefootnote\relax \footnote{$^{*}$Correspondence to \texttt{gandikota.ro@northeastern.edu}}

The challenge touches on a key limitation in how we interact with diffusion models today. Current control methods require users to explicitly specify their desired edits in advance through prompts~\cite{gandikota2023concept}, reference images~\cite{zhang2023addingconditionalcontroltexttoimage, chen2024trainingfreeregionalpromptingdiffusion, ruiz2022dreambooth,kumari2022customdiffusion, Ryu_lora, hu2021lora}, or attribute vectors~\cite{ye2023ipadaptertextcompatibleimage, hertz2024stylealignedimagegeneration, li2023photomaker, shi2024instantbooth,parmar2023zero,hertz2022prompt}. That contrasts sharply with natural human creative workflows, where artists dynamically explore creative ideas and jointly refine them toward meaningful artistic outcomes~\cite{hoffmann2016modeling}. The need for pre-specified controls creates a barrier between users and the full creative potential of these models.

Interestingly, earlier generative models like GANs~\cite{gans,karras2019style,brock2018large} naturally developed more interpretable internal structures. Their compact latent spaces often exhibited emergent disentanglement~\cite{harkonen2020ganspace,radford2015unsupervised, wu2021stylespace, shen2020interfacegan}, enabling continuous and compositional control over generated images. Users could explore these spaces to discover interesting variations that would be difficult to describe in words~\cite{wu2021stylespace}, then combine them to achieve their creative goals~\cite{grabe2022towards}.

Diffusion models have largely superseded GANs in conditional image synthesis~\cite{dhariwal2021diffusion}, achieving greater diversity through much higher-dimensional latents. And yet an understanding of the underlying structure of these larger latent spaces has remained elusive. In this work, we ask a fundamental question: \emph{Can we automatically discover the visual structure within a diffusion model's knowledge of a concept?} Rather than requiring user-specified controls, we aim to decompose the model's internal representations into expressive directions that users can explore and combine.

To address these needs, we present \textbf{SliderSpace}, a framework that brings systematic explorability to diffusion models. Given just a text prompt, SliderSpace discovers a canonical set of meaningful, diverse, and controllable directions within the model's knowledge of that concept. Each direction is implemented as a low-rank adapter~\cite{hu2021lora} that can be scaled and composed with others, allowing users to explore and smoothly combine different aspects of variation, as shown in Figure~\ref{fig:intro}.

We ground SliderSpace discovery in three key requirements for meaningful decomposition of a diffusion model's visual manifold: 
\begin{enumerate}
    \item \textbf{Unsupervised Discovery:} The decomposition process should emerge from the intrinsic structure of the model's learned representation, rather than being guided by predefined attributes. This ensures we capture the true topology of the model's knowledge space rather than projecting our assumptions onto it.
    
    \item \textbf{Semantic Orthogonality:} Each discovered control must represent a distinct semantic direction. This is enforced in a semantic feature space, like CLIP, where every slider has an orthogonal effect in embeddings. This prevents discovering multiple controls that create similar semantic effects, making the system more efficient and easier.
    
    \item \textbf{Distribution Consistency:} Directions must induce consistent transformations across both random seeds and prompt variations. 
\end{enumerate}

These requirements naturally lead to our proposed framework, which we formalize in Section~\ref{sec:method}. As we show in our experiments, SliderSpace is architecture-agnostic, working with both conventional U-Net based models like Stable Diffusion~\cite{rombach2022high, rombach2022sd20, podell2023sdxl, turbo, dmd} and recent transformer-based architectures like Flux~\cite{flux}.

We demonstrate the expressiveness of SliderSpace through three applications: First, we show how SliderSpace can decompose high-level concepts into diverse and expressive components, revealing the natural axes of variation in the model's understanding. Second, we explore artistic style variation, where SliderSpace discovers directions that match or exceed the diversity of manually curated artist lists while being judged more useful by human evaluators. Finally, we show how SliderSpace can help reverse the mode collapse commonly observed in distilled diffusion models, restoring diversity while maintaining generation speed.

Beyond providing practical creative control, SliderSpace opens new avenues for understanding and utilizing the latent capabilities of diffusion models. By mapping these models' visual potential into intuitive, composable directions, we take a step toward making their creative possibilities more accessible and interpretable to users.

% Image editing methods in diffusion models unlock the creativity of users. In this work we ask an alternate question: \emph{Can we organize and expose what of the diffusion model is already capable of?}.
% Existing methods for controlling image generation typically require users to manually specify edit directions for desired changes. This process is time-consuming, requires technical expertise, and limits the spontaneity of the creative process. For instance, if a user wants to adjust the smile of a generated person, they must explicitly request this edit, often through imprecise prompt engineering or model fine-tuning. This approach of predefined controls or manual specifications restricts users from fully exploring the latent capabilities of the model. There may be interesting stylistic variations or attributes that the model can generate, but users have no easy way to discover or utilize these.

% Natural visual disentanglement was an emergent property in the latent space of Generative Adversarial Models (GANs) \cite{harkonen2020ganspace,radford2015unsupervised, wu2021stylespace, shen2020interfacegan}. In particular, it has been observed that StyleGAN~\cite{karras2019style} stylespace neurons offer detailed control over many meaningful aspects of images that would be difficult to describe in words~\cite{wu2021stylespace}. However, diffusion models do not share such a compact latent space~\cite{park2023unsupervised}; and efforts to uncover such a space in the semantic embeddings of the text conditioning have met with limited success \nik{Nick - is there a specific citation you were thinking about?}.

% In this work we introduce \textbf{SliderSpace}, which takes a step towards uncovering an analogous low dimensional representation of diffusion models' visual breadth; in essence treating the diffusion model as many generators sharing parameters, where a particular generator is defined by a specific prompt. For a given prompt we sample many random seeds (and optionally prompt expansions using an LLM), generate the corresponding images, and apply an off the shelf feature extractor (in this work CLIP, but our method can be applied to any differentiable feature extractor). We use PCA to analyze these features, and for each of the leading $k$ principal components we train a LoRA \cite{} which causes the diffusion model to produces images which increase the feature magnitude along that component when passed back through the same feature extractor. This leads to a 'Slider' for each principal component, because each LoRA can be scaled and applied to the original diffusion model, continuously varying those visual features in the generated results (as measured, in our case, by CLIP).

% There are many other works that enhance the controllability of diffusion models. One common approach is enabling users to add spatial constraints to a generation either manually, or via a reference image \cite{zhang2023addingconditionalcontroltexttoimage, chen2024trainingfreeregionalpromptingdiffusion}, a second is leveraging more abstract embeddings (e.g. identity, style) extracted from a reference image \cite{ye2023ipadaptertextcompatibleimage, hertz2024stylealignedimagegeneration, li2023photomaker, shi2024instantbooth}, a third is finetuning a foundation model to better generate a concept important to the user \cite{ruiz2022dreambooth, kumari2022customdiffusion, Ryu_lora, hu2021lora}, and a fourth (most relevant to this work) is finding low-rank adaptors of the model based on a prompt or small training set which can be scaled to provide continous control over one aspect of generated image (e.g. night vs day, basic vs luxury, etc.) \cite{gandikota2023concept}. SliderSpace is complementary to all of these methods and offers something distinct. All of the other methods we are aware require the user (and / or model designer) to know in advance what type of control they want. In contrast SliderSpace assists users in discovering and controlling hidden capabilities present in the diffusion model's distribution of possible generations.

%We propose that truly intuitive creative control in a text-to-image model should meet three key criteria: \emph{discoverability}, \emph{intuitiveness}, and \emph{specificity}. The model should reveal controllable attributes that may not be immediately obvious, offer controls that are easy to understand and manipulate, and ensure each control affects a distinct attribute of the generated image.

% We demonstrate the utility and power of SliderSpace using three applications built on top of SDXL-DMD \cite{dmd}, because its fast generation speed lends itself well to the continuous control offered by SliderSpace.

% First, we study concept decomposition (Section \ref{sec:concept_exp}), where we learn sliders for a specific concept (e.g. 'monster', 'waterfall', 'car'). Through quantitative metrics of diversity and text alignment we demonstrate that the learned sliders dramatically boost the diversity of generations when randomly applied without harming text alignment; we also ask humans to qualitatively judge these results in a user study where they find the SliderSpace results to be more 'Diverse', 'Useful', and 'Creative' than our baselines.

% Second, we attempt to compare the automatic discoveries of SliderSpace to a large scale manual study of artistic styles (Section \ref{sec:art_exp}), open-sourced by ParrotZone \cite{parrotzone}. In this study SDXL was prompted with over 4300 artist names,  and based on visual inspection the cases of successful stylistic mimicry recorded. Quantitatively SliderSpace more closely matches the distribution of artistic variation discovered by ParrotZone than other baselines, and in our user studies was judged to be significantly more 'Diverse' and 'Useful' than the baselines. To our surprise humans even judged SliderSpace results to be slightly more 'Diverse' than the results generated by the manually discovered artist names of \cite{parrotzone}.

% Third, we attempt to use SliderSpace to reverse the mode collapse commonly observed in distilled few-step diffusion models relative to the original teacher model (Section \ref{sec:diverse_exp}). We quantitatively demonstrate that applying SliderSpace to SDXL-DMD leads to more closely matching the distribution of images by the original teacher, SDXL.

%Through extensive experiments on various state-of-the-art text-to-image models, we demonstrate that SliderSpace significantly enhances user control and creative expression in AI-assisted image generation tasks. Our method enables a range of applications, including concept decomposition and control, diversity improvement in generated images, customization dissection and edits, and the exploration of artistic styles inherent in the model.

% SliderSpace goes beyond providing a practical tool for enhanced creative control. By mapping the visual potential of diffusion models it can open new avenues for generative creativity and deepens our understanding of each model's hidden potential.

\section{Related Work}
\label{sec:related}
\section{Related Work}

\paragraph{LLMs for Agent tasks.}

Our research is related to deploying large language models (LLMs) as agents for decision-making tasks in interactive environments~\citep{liu2023agentbench,zhou2023webarena,shridhar2020alfred,toyama2021androidenv}. Earlier works, such as~\citep{yao2023webshopscalablerealworldweb}, fine-tuned models like BERT~\citep{devlin2019bertpretrainingdeepbidirectional} for decision-making in simplified environments, such as online shopping or mobile phone manipulation. With the advent of large language models~\citep{brown2020languagemodelsfewshotlearners,openai2024gpt4technicalreport}, it became feasible to perform decision-making tasks through zero-shot or few-shot in-context learning. To better assess the capabilities of LLMs as agents, several models have been developed~\citep{deng2024mind2web,xiong2024watch,hong2023cogagent,yan2023gpt}. Most approaches~\citep{zheng2024seeact,deng2024mind2web} provide the agent with observation and action history, and the language model predicts the next action via in-context learning. Additionally, some methods~\citep{zhang2023building,li2023camel,song2024trial} attempt to distill trajectories from state-of-the-art language models to train more effective policy models. In contrast, our paper introduces a novel framework that automatically learns a reward model from LLM agent navigation, using it to guide the agents in making more effective plans.

\textbf{LLM Planning.} Our paper is also related to planning with large language models. Early researchers~\citep{brown2020languagemodelsfewshotlearners} often prompted large language models to directly perform agent tasks. Later, \citet{yao2022react} proposed ReAct, which combined LLMs for action prediction with chain-of-thought prompting~\citep{wei2022chain}. Several other works~\citep{yao2023treethoughtsdeliberateproblem,hao2023reasoning,zhao2023large,qiao2024agentplanningworldknowledge} have focused on enhancing multi-step reasoning capabilities by integrating LLMs with tree search methods. Our model differs from these previous studies in several significant ways. First, rather than solely focusing on text generation tasks, our pipeline addresses multi-step action planning tasks in interactive environments, where we must consider not only historical input but also multimodal feedback from the environment. Additionally, our pipeline involves automatic learning of the reward model from the environment without relying on human-annotated data, whereas previous works rely on prompting-based frameworks that require large commercial LLMs like GPT-4~\citep{openai2024gpt4technicalreport} to learn action prediction. Furthermore, \Model supports a variety of planning algorithms beyond tree search.

\textbf{Learning from AI Feedback.} In contrast to prior work on LLM planning, our approach also draws on recent advances in learning from AI feedback~\citep{bai2022constitutional,lee2023rlaif,yuan2024self,sharma2024critical,pan2024autonomous,koh2024tree}. These studies initially prompt state-of-the-art large language models to generate text responses that adhere to predefined principles and then potentially fine-tune the LLMs with reinforcement learning. Like previous studies, we also prompt large language models to generate synthetic data. However, unlike them, we focus not on fine-tuning a better generative model but on developing a classification model that evaluates how well action trajectories fulfill the intended instructions. This approach is simpler, requires no reliance on state-of-the-art LLMs, and is more efficient. We also demonstrate that our learned reward model can integrate with various LLMs and planning algorithms, consistently improving their performance.

\textbf{Inference-Time Scaling.} ~\citet{snell2024scaling} validates the efficacy of inference-time scaling for language models. Based on inference-time scaling, various methods have been proposed, such as random sampling~\citep{wang2022self} and tree-search methods~\citep{hao2023reasoning, zhang2024accessing, guan2025rstar}. Concurrently, several works have also leveraged inference-time scaling to improve the performance of agentic tasks. ~\citet{koh2024tree} adopts a training-free approach, employing MCTS to enhance policy model performance during inference and prompting the LLM to return the reward. ~\citet{gu2024your} introduces a novel speculative reasoning approach to bypass irreversible actions by leveraging LLMs or VLMs. It also employs tree search to improve performance and prompts an LLM to output rewards. ~\citet{yu2024exact} proposes Reflective-MCTS to perform tree search and fine-tune the GPT model, leading to improvements in ~\citet{koh2024visualwebarena}. ~\citet{putta2024agent} also utilizes MCTS to enhance performance on web-based tasks such as ~\citet{yao2023webshopscalablerealworldweb} and real-world booking environments. ~\cite{lin2025qlass} utilizes the stepwise reward to give effective intermediate guidance across different agentic tasks. Our work differs from previous efforts in two key aspects: (1) Broader Application Domain. Unlike prior studies that primarily focus on tasks from a single domain, our method demonstrates strong generalizability across web agents, mathematical reasoning, and scientific discovery domains, further proving its effectiveness. (2) Flexible and Effective Reward Modeling. Instead of simply prompting an LLM as a reward model, we finetune a small scale VLM~\citep{lin2023vila} to evaluate input trajectories. %Our reward scores range continuously between 0 and 1, in contrast to existing methods that rely on discrete scoring (e.g., 0 and 1, or 0, 0.5, and 1) through direct LLM prompting.

% Concurrently, several works have also leveraged inference-time scaling to improve the performance of agentic tasks. ~\citet{pan2024autonomous} demonstrates that LLMs and VLMs, such as the GPT series, can function as evaluators or reward models to provide guidance for fine-tuning or reflection, thereby enhancing digital agents. This lays the groundwork for subsequent studies that directly prompt LLMs as reward models. ~\citet{koh2024tree} adopts a training-free approach, employing MCTS to enhance policy model performance during inference. However, it is limited to web environments~\citep{koh2024visualwebarena}. Moreover, its value function relies on prompting an LLM, which is less effective than our proposed method. We validate our approach through ablation studies, demonstrating that our fine-tuned reward model is more effective. ~\citet{gu2024your} introduces a novel speculative reasoning approach to bypass irreversible actions, such as purchasing a product, by leveraging LLMs or VLMs. It also employs tree search to improve performance, but it remains restricted to the web domain~\citep{koh2024visualwebarena, deng2024mind2web}. Additionally, it lacks reward modeling and instead prompts an LLM to output rewards. ~\citet{yu2024exact} proposes Reflective-MCTS to perform tree search and fine-tune the GPT model, leading to improvements in ~\citep{koh2024visualwebarena}. However, this work focuses solely on a single web agent task, and its reward modeling is derived from multi-agent debate, differing from our more effective and efficient reward modeling approach. ~\citet{putta2024agent} also utilizes MCTS to enhance performance, but it is limited to web-based tasks such as ~\citep{yao2023webshopscalablerealworldweb} and real-world booking environments.

\section{Benchmark Construction}
\label{sec:method}


\section{Methodology}
\paragraph{Preliminaries.}
We primarily focus on the homologous model merging, in which $\boldsymbol{\theta}_i$ all come from the same base model $\boldsymbol{\theta}_{\rm{base}}$. Given $K$ tasks $\{T_1,T_2,\cdots,T_K\}$ and $K$ corresponding fine-tuned models with parameters $\{\boldsymbol{\theta}_1,\boldsymbol{\theta}_2,\cdots,\boldsymbol{\theta}_K\}$, model merging aims to combine $K$ fine-tuned models into one single model simultaneously performing on $\{T_1,T_2,\cdots,T_K\}$ without post-training~\cite{method_p1_1,method_p1_2}.
Task vector~\cite{ilharco2023editing,yang2024adamerging} is a key element in merging method which could enhances the base model‘s ability or enable the model to handle other tasks. Specifically, for task $T_i$, the task vector $\boldsymbol\tau_i\in \mathbb{R}^D$ is defined as the vector obtained by subtracting the SFT weights $\boldsymbol{\theta}_i$ from the base model weight
$\boldsymbol{\theta}_{\rm{base}}$, \emph{i.e.}, $\boldsymbol\tau_i=\boldsymbol{\theta}_i-\boldsymbol{\theta}_{\rm{base}}$. The merged model could be denoted as $\boldsymbol{\theta}_m=\boldsymbol{\theta}_{\rm{base}}+\sum_i \lambda_i\boldsymbol{\tau}_i$, which $\lambda_i$ is the scaling factor measuring the importance of task vector. For clarification, we also denote the neuron set in $\boldsymbol{\theta}_i$ as $\mathcal{N}_i$, the neuron set in $\boldsymbol{\tau}_i$ as $\mathcal{T}_i$.



\begin{algorithm}[!ht]
    \caption{LED-Merging}
    \label{alg1}
    \begin{algorithmic}[1]
        \REQUIRE  base model $\boldsymbol{\theta}_{\rm{base}}$, SFT models $\{\boldsymbol{\theta}_{i}\mid i\in [K]\}$, mask ratios \{$r_{i} \mid i\in [K]\}$, scaling factors $\{\lambda_i\mid i\in[K]\}$, location datasets $\{\mathcal{X}_{i}\mid i\in[K]\}$
        \ENSURE merged parameter $\boldsymbol{\theta}_{m}$
        \STATE $\mathcal{M}\leftarrow\phi$
        \STATE $\boldsymbol{\theta}_{m}\leftarrow \boldsymbol{\theta}_{\rm{base}}$
        \FOR{$i\in [K]$}
        \STATE $I(\boldsymbol{\theta}_i)=\mathbb{E}_{x\sim \mathcal{X}_i}|\boldsymbol{\theta}_{i}\odot \nabla_{\boldsymbol{\theta}_i}\mathcal{L}(x)|$
        \STATE $I(\boldsymbol{\theta}_{\rm{base}})=\mathbb{E}_{x\sim \mathcal{X}_i}|\boldsymbol{\theta}_{\rm{base}}\odot \nabla_{\boldsymbol{\theta}_{\rm{base}}}\mathcal{L}(x)|$
        
        \STATE calculate $\mathcal{T}^{r_i}_{i}$ following Equation \ref{vote}
        \STATE  $\mathcal{M}\leftarrow \mathcal{M}\cup\{\mathcal{T}^{r_i}_i\}$
       
        
   
        
        
        \ENDFOR  
        \FOR{$i\in [K]$}
        
        \STATE calculate $\text{Disjoint}(\mathcal{T}_i^{r_i})$ use Equation~\ref{disjoint_safety}
        \STATE $\boldsymbol{m}_i \leftarrow \boldsymbol{0}$
        \FOR{$d\in \mathcal{T}_i^{r_i}$}
        \STATE $\boldsymbol{m}_{i,d}=1$
        \ENDFOR
        \STATE $\boldsymbol{\theta}_{m}\leftarrow \boldsymbol{\theta}_{m}+\lambda_i \boldsymbol{\tau}_i\odot \boldsymbol{m}_{i}$
        \ENDFOR
    \end{algorithmic}
\end{algorithm}
    %\vspace{-5pt}
\begin{figure*}[h!]
    \centering
    \includegraphics[width=\linewidth]{figs/pipeline_v2.pdf}
    \vspace{-40mm}
    \caption{Overview of our two-stage training pipeline {\ours}.}
    \label{fig:pipeline}
\end{figure*}


\paragraph{LED-Merging: Location, Election, and Disjoint Merging}
To address the neuron misidentification and interference issues in existing model merging methods, we propose LED-Merging (Location, Election, and Disjoint Merging). Specifically, previous studies \cite{modelstock, ilharco2023editing, tiesmerging} fail to accurately identify safety-related neurons in task vectors with a single magnitude score, namely \textit{neuron misidentification}. Meanwhile, there exists an interference between safety-related and utility-related task vector neurons during the merging process, namely \textit{neuron interference}. To address neuron misidentification, we first locate important neurons both in the base and fine-tuned models and then elect neurons from the task vector considering these two scores together. Subsequently, to mitigate the interference, we introduce a disjoint step, isolating these important neurons so that they influence different base neurons. The whole process is illustrated in Figure~\ref{fig:method}. 




In the location and election step, we consider the importance score from base and fine-tuned models simultaneously to locate task-specific neurons. In this way, it is more accurate than relying on the magnitude score alone because task-specific neurons with high importance score in the fine-tuned model may not necessarily score high in the base model, and vice versa.

{\textbf{Location}}.  We first calculate importance scores for each neuron in a base/fine-tuned model. Given a location dataset $\mathcal{X}_i=\{(x,y)_k\}$, where $x$ is the question and $y$ is the answer, we calculate the importance scores for the weight $\boldsymbol{\theta}_i\in\mathbb{R}^D$ in any  layer as follows~\cite{snip,spareseGPT,sun2024a}:
\begin{equation}
    I(\boldsymbol{\theta}_i)=\mathbb{E}_{x\sim \mathcal{X}_i}[\boldsymbol{\theta}_i\odot \nabla _{\boldsymbol{\theta}_i}\mathcal{L}(x)],
    \label{location}
\end{equation}
which $\mathcal{L}(x)=-\log p(y\mid x)$ is the conditional negative log-likelihood loss. We choose the SNIP score~\cite{snip} because it balances computational efficiency and performance~\cite{cq}. Please refer to Sec.~\ref{sec:ablation} for the comparison between different location methods. After computing importance scores, we choose top-$r_i$ neurons as the important neuron subset $\mathcal{N}_{i}^{r_i}$ from $I(\boldsymbol{\theta}_i)$.
 
 % After computing locating scores, we select the neurons scoring both high in base and fine-tuned models as important neurons in task vectors. Then in the disjoint step,  with preventing  polysemantic neurons  from receiving gradient updates towards different directions,
 % we use set difference to isolate the safety   and utility-related neurons  and construct corresponding masks for merging process,

{\textbf{Election}}. A natural question is how to select important neurons in the task vector $\boldsymbol{\tau}_i$ based on $I(\boldsymbol{\theta}_{\rm{base}})$ and $I(\boldsymbol{\theta}_{i})$. The important neurons in the base model may be different from neurons in the fine-tuned model. Therefore, we introduce the following election strategy to select neurons with high scores in both base and fine-tuned models:
\begin{equation}
    \mathcal{T}_i^{r_i}=\mathcal{N}_i^{r_i}\cap \mathcal{N}_{\rm{base}}^{r_i}.
    \label{vote}
\end{equation}
\emph{Remark}. We compare different choosing methods, including scoring low or high in base or fine-tuned model in Section~\ref{sec:ablation} and find that Equation \ref{vote} achieves the best performance.





{\textbf{Disjoint}}. As important neurons from different task vectors may conflict with each other at the same position, we use the set difference to disjoint the neurons from others to prevent interference:
\begin{equation}
    \text{Disjoint}(\mathcal{T}^{r_i}_{i})=\mathcal{T}^{r_i}_{i}-\mathop{\cup}\limits_{{J}\subsetneqq [K],|J|\geq 2}\mathop{\cap}\limits_{j\in {J}}\mathcal{T}^{r_j}_{j}.
    \label{disjoint_safety}
\end{equation}

Next, we construct a mask $\boldsymbol{m}_i\in\mathbb{R}^D$ to implement disjoint in the merging process. Specifically, this mask $\boldsymbol{m}_i$ is used to select neurons from $\mathcal{T}_i$. The mask ratio is $r_i$, where $r\in(0,1]$. The mask $\boldsymbol{m}_i$ can be derived from:
\begin{equation}
    \boldsymbol{m}_{i,d}=\begin{aligned} &\left\{ \begin{array}{ll} 1, & \text{if } d\in \text{Disjoint}(\mathcal{T}_{i}^{r_i}), \\ 0, & \text{otherwise}. \end{array} \right. \end{aligned}
    \label{mask_safety}
\end{equation}


% \subsection{Merging Models with Masks}
{\textbf{Merging}}. The final
merged task vector $\boldsymbol{\tau}_m$ is as follows:
\begin{equation}
    \boldsymbol{\tau}_m= \sum_i \lambda_i\boldsymbol{\tau}_{i}\odot\boldsymbol{m}_i.
    \label{merged_task_vector}
\end{equation}
We summarize the workflow in Algorithm \ref{alg1}.




\section{Experimental Setup}
\label{sec:experiment}
\paragraph{\name.} Our dataset, \name, consists of 2,400 program pairs across six equivalence categories. Each category contains 200 equivalent and 200 inequivalent pairs. \Cref{tab:dataset} summarizes the lines of code, including the minimum, maximum, and average, for programs in each category, reflecting the wide variation in program lengths. As the dataset generation pipeline is fully automated, additional pairs can be generated as needed.

\begin{table}[h!]
    \small
    \centering
    \begin{tabular}{l l c c c c}
        \toprule
        \multirow{2}{*}{Category} & \multirow{2}{*}{Language} & \multirow{2}{*}{\# Pairs} & \multicolumn{3}{c}{Lines of Code} \\
        \cmidrule(lr){4-6}
        & & & Min & Max & Avg. \\
        \midrule
        \dce & C & 400 & 98 & 880 & 541 \\
        \cuda & CUDA & 400 & 46 & 1733  & 437  \\
        \ass & x86-64 & 400 & 8  & 29  & 14  \\
        \oja & Python & 400 & 3 & 3403  & 82 \\
        \ojv & Python & 400 & 2 & 4087 &  70 \\
        \ojva & Python & 400 & 3 & 744  & 35 \\
        \bottomrule
    \end{tabular}
    \caption{\textbf{Statistics of the \name{} dataset.}}
    \label{tab:dataset}
\end{table}


\paragraph{Research Questions.} We investigate: 1) how different models perform on equivalence checking (\Cref{subsec:acc}); 2) whether prompting techniques, such as few-shot learning~\cite{brown2020language} and Chain-of-Thought~\cite{wei2022chain}, can enhance performance (\Cref{subsec:prompt}); and 3) whether model predictions exhibit bias when judging program equivalence.

\paragraph{Models.} We evaluate \numllm large language models. For open-source models, including Mixtral~\cite{jiang2024mixtral}, Llama~\cite{touvron2023llama}, Qwen~\cite{bai2023qwen}, DeepSeek~\cite{liu2024deepseek}, we use Together AI, a model serving framework. For closed-source models (e.g., GPT-4~\cite{achiam2023gpt}, Claude-3.5~\cite{Anthropic}), we access them via their official APIs, using the default temperature setting.

\paragraph{Prompts.} The 0-shot evaluation is conducted using the prompt ``You are here to judge if two programs are semantically equivalent. Here equivalence means \{{\em definition}\}. [Program 1]: \{code1\} [Program 2]: \{code2\} Please only output the answer of whether the two programs are equivalent or not. You should only output Yes or No.'' The definition of equivalence and the corresponding program pairs are provided for each category. Additionally, for the categories of \oja, \ojv and \ojva, the prompt also includes the problem description. The full prompts used in our experiments for each equivalence category are in \Cref{subsec:app:prompt}.


\begin{table*}[!tb]
        \small
        \centering
\begin{tabular}{lccccccc}
\toprule
\textbf{Model} & \textbf{DCE} & \textbf{\cuda} & \textbf{\ass} & \textbf{\oja} & \textbf{\ojv} & \textbf{\ojva} & \textbf{Overall Accuracy} \\
\midrule
\textit{Random Baseline} & \textit{50.0} & \textit{50.0} & \textit{50.0} & \textit{50.0} & \textit{50.0} & \textit{50.0} & \textit{50.0} \\
Llama-3.2-3B-Instruct-Turbo & 50.0 & 49.8 & 50.0 & 51.5 & 51.5 & 51.5 & 50.7 \\
Llama-3.1-8B-Instruct-Turbo & 41.8 & 49.8 & 50.5 & 57.5 & 75.5 & 56.8 & 55.3 \\
Mistral-7B-Instruct-v0.3 & 51.0 & 57.2 & 73.8 & 50.7 & 50.5 & 50.2 & 55.6 \\
Mixtral-8x7B-Instruct-v0.1 & 50.2 & 47.0 & 64.2 & 59.0 & 61.5 & 55.0 & 56.1 \\
Mixtral-8x22B-Instruct-v0.1 & 46.8 & 49.0 & 62.7 & 63.5 & 76.0 & 62.7 & 60.1 \\
Llama-3.1-70B-Instruct-Turbo & 47.5 & 50.0 & 58.5 & 66.2 & 72.0 & 67.5 & 60.3 \\
QwQ-32B-Preview & 48.2 & 50.5 & 62.7 & 65.2 & 71.2 & 64.2 & 60.3 \\
Qwen2.5-7B-Instruct-Turbo & 50.5 & 49.2 & 58.0 & 62.0 & 80.8 & 63.0 & 60.6 \\
gpt-4o-mini-2024-07-18 & 46.8 & 50.2 & 56.8 & 64.5 & 91.2 & 64.0 & 62.2 \\
Qwen2.5-72B-Instruct-Turbo & 42.8 & 56.0 & 64.8 & 72.0 & 76.5 & 70.8 & 63.8 \\
Llama-3.1-405B-Instruct-Turbo & 40.0 & 49.0 & 75.0 & 72.2 & 74.5 & 72.8 & 63.9 \\
DeepSeek-V3 & 41.0 & 50.7 & 69.2 & 73.0 & 83.5 & 72.5 & 65.0 \\
gpt-4o-2024-11-20 & 43.2 & 49.5 & 65.2 & 71.0 & 87.0 & 73.8 & 65.0 \\
claude3.5-sonnet-2024-10-22 & 38.5 & \textbf{62.3} & 70.0 & 71.2 & 78.0 & 73.5 & 65.6 \\
o1-mini-2024-09-12 & 55.8 & 50.7 & 74.2 & 80.0 & 89.8 & 78.8 & 71.5 \\
DeepSeek-R1 & 52.2 & 61.0 & 78.2 & 79.8 & \textbf{91.5} & 78.0 & 73.5 \\
o3-mini-2025-01-31 & \textbf{68.8} & 59.0 & \textbf{84.5} & \textbf{84.2} & 88.2 & \textbf{83.2} & \textbf{78.0} \\
\midrule
Mean & 47.9 & 52.4 & 65.8 & 67.3 & 76.4 & 67.0 & 62.8 \\
\bottomrule
\end{tabular}
    \caption{\textbf{Accuracy of \numllm models on \name{} under 0-shot prompting.} We report accuracy for each of the six equivalence categories along with the overall accuracy.}
    \label{tab:acc}
\end{table*}


\paragraph{Error Handling.} Some models occasionally fail to follow the instruction to ``output Yes or No''. To address this issue, we use GPT-4o to parse model outputs. In cases where no result can be extracted, we randomly assign ``Yes'' or ``No'' as the model's output. These errors are very rare in advanced models but occur more frequently in smaller models.


\begin{figure}[!tb]
    \centering
    \includegraphics[width=\columnwidth]{figure/scaling.pdf}
    \caption{\textbf{Scaling Trend on \name.}}
    \label{fig:scaling}
\end{figure}

\section{Results}
\label{sec:result}
\subsection{Model Accuracy}
\label{subsec:acc}

\Cref{tab:acc} shows the accuracy results for \numllm state-of-the-art large language models on \name under zero-shot prompting. Our findings are as follows:

\paragraph{Reasoning models achieve the highest performance, demonstrating a clear advantage over non-reasoning models.} As shown in \Cref{tab:acc}, reasoning models such as OpenAI o3-mini, DeepSeek R1, and o1-mini significantly outperform all others in our evaluation. This further underscores the complexity of equivalence checking as a code reasoning problem, where reasoning models exhibit a distinct advantage.


\paragraph{\name is a challenging benchmark.} Among the \numllm models evaluated, OpenAI o3-mini achieves only \sotalowacc{} in the CUDA category despite being the top-performing model overall, with an accuracy of \sotaacc{}. For the two most difficult categories, the highest accuracy across all models is \sotacuda{} and \sotadce{}, respectively, only modestly above the random baseline of 50\% accuracy for binary classification, highlighting the substantial room for improvement.

\paragraph{Pure syntactic changes (\ojv) are the easiest for LLMs, while structural transformations are key to assessing deep semantic reasoning.} As shown in the last row of \Cref{tab:acc}, the \ojv category achieves the highest mean accuracy, with DeepSeek-R1 leading at 91.5\%. This is because \ojv pairs are generated through trivial variable renaming, as seen in prior work~\cite{badihi2021eqbench,maveli2024can}. Additionally, combining variable renaming with algorithmic equivalence has little impact on difficulty, as indicated by the small drop in mean accuracy from \oja 67.3\% to \ojva 67.0\%. In contrast, all other categories involve non-local structural transformations, making them more challenging and essential for evaluating LLMs' deep semantic reasoning.

\paragraph{Scaling up models improves performance.} Larger models generally achieve better performance. \Cref{fig:scaling} shows scaling trends for the Qwen2.5, Llama-3.1, and Mixtral families, where accuracy improves with model size. The x-axis is on a logarithmic scale, highlighting how models exhibit consistent gains as parameters increase.


\subsection{Prompting Strategies Analysis}
\label{subsec:prompt}

We study few-shot in-context learning and Chain-of-Thought (CoT) prompting, evaluating four strategies: 0-shot, 4-shot, 0-shot with CoT, and 4-shot with CoT. For 4-shot, prompts include 2 equivalent and 2 inequivalent pairs. \Cref{subsec:app:prompt} details the prompts, and \Cref{tab:prompt} shows the results.

Our key finding is that \textbf{prompting strategies \emph{barely} improve performance on \name}, highlighting the task's difficulty and need for deeper reasoning. Few-shot prompting provides only minor improvements over 0-shot, while Chain-of-Thought shows slight benefits for o1-mini but marginally reduces performance for other models, underscoring the task’s complexity and the need for more advanced, task-specific approaches.



\begin{table}[!tb]
\centering
\small
\begin{tabular}{lcccc}
\toprule
\textbf{Model} & \textbf{0S} & \textbf{4S} & \textbf{0S-CoT} & \textbf{4S-CoT} \\
\midrule
o1-mini & 71.5 & 71.5 & \textbf{71.9} & \textbf{71.9} \\
gpt-4o & 65.0 & \textbf{66.5} & 62.5 & 62.7 \\
DeepSeek-V3 & 65.0 & \textbf{66.9} & 63.3 & 62.5 \\
gpt-4o-mini & 62.2 & \textbf{63.5} & 60.2 & 61.2 \\
\bottomrule
\end{tabular}
\caption{\textbf{Accuracies of different prompting techniques.} We evaluate 0-shot and 4-shot in-context learning, both without and with Chain-of-Thought (CoT). Prompting strategies barely improve performance, highlighting the task’s difficulty and the need for task-specific approaches.}
\label{tab:prompt}
\end{table}

\subsection{Bias in Model Prediction}
\label{subsec:bias}

We evaluate the prediction bias of the models and observe \textbf{a pronounced tendency to misclassify equivalent programs as inequivalent in the \cuda and \ass categories}. \Cref{tab:bias} presents the results for four representative models, showing high accuracy for inequivalent pairs but significantly lower accuracy for equivalent pairs, with full results for all models in \Cref{subsec:app:bias}.

The bias in the \cuda category arises from extensive structural transformations, such as loop restructuring and shared memory optimizations, which make paired programs appear substantially different. In the \ass category, superoptimization applies non-local transformations to achieve optimal instruction sequences, introducing aggressive code restructuring that complicates equivalence reasoning and leads models to frequently misclassify equivalent pairs as inequivalent.

\subsection{Case Studies}
\label{subsec:case}

\begin{table}[!tb]
    \centering
\small
\begin{tabular}{lcccc}
\toprule
\multicolumn{1}{c}{\multirow{2}{*}{\textbf{Model}}} & \multicolumn{2}{c}{\textbf{\cuda}} & \multicolumn{2}{c}{\textbf{\ass}} \\
\cmidrule(lr){2-5}
 & \textbf{Eq} & \textbf{Ineq} & \textbf{Eq} & \textbf{Ineq} \\
\midrule
\textit{Random Baseline} & \textit{50.0} & \textit{50.0} & \textit{50.0} & \textit{50.0} \\
o3-mini & 27.5 & 90.5 & 69.5 & 99.5 \\
o1-mini & 2.5 & 99.0 & 50.0 & 98.5 \\
DeepSeek-R1 & 28.0 & 94.0 & 57.5 & 99.0 \\
DeepSeek-V3 & 8.5 & 93.0 & 44.0 & 94.5 \\
\bottomrule
\end{tabular}
    \caption{Accuracies on equivalent and inequivalent pairs in the \cuda and \ass categories under 0-shot prompting, showing that \textbf{models perform significantly better on inequivalent pairs}. Random guessing serves as an unbiased baseline for comparison. Full results for all models are shown in \Cref{subsec:app:bias}.}
\label{tab:bias}
\end{table}

\paragraph{Models lack capabilities for sound equivalence checking.} We find that simple changes that lead to semantic differences can confuse the models, causing them to produce incorrect predictions despite their correct predictions on the original program pairs. For example, o3-mini, which is one of the top-performing models in \cuda category, can correctly classifies the pair shown in \Cref{fig:cuda} as equivalent. Next, we introduce synchronization bugs into the right-hand program, creating two inequivalent pairs with the original left-hand program: (1) removing the first \CodeIn{\_\_syncthreads();} allows reads before all writes complete, causing race conditions; (2) removing the second \CodeIn{\_\_syncthreads();} lets faster threads overwrite shared data while slower threads read it. Despite these semantic differences, o3-mini misclassifies both pairs as equivalent.

\paragraph{Proper hints enable models to correct misjudgments.} After o3-mini misclassifies the modified pairs, a hint about removed synchronization primitives allows it to correctly identify both as inequivalent, with accurate explanations highlighting data races. This suggests that training models on dedicated program analysis datasets, beyond only raw source code, may be useful for improving their code reasoning capabilities.

% \section{Discussion}
% \label{sec:result}
% \section{Discussion}
\label{sec:discussion}

% \TODO{Bryan}

Our multimodal data augmentation method is a plug-and-play method that can be applied to any future VLM. Also the T2I generation can be replaced by any future T2I model, thus the effectiveness of our method automatically improves along with the SOTA T2I model, making it future-proof.



Our main method, \textbf{Co}ntrastive Visual \textbf{D}ata \textbf{A}ugmentation (\textbf{CoDA}), is simple and easy to apply to LMMs in a variety of scenarios. Several components in the pipeline utilize existing off-the-shelf model components that can be easily swapped out for superior versions of similar models as research in their respective field progresses. Therefore, we expect the efficiency and effectiveness of \textbf{CoDA} to dramatically scale along with the advancement of relevant models. 



\section{Conclusion}
\section{Conclusion and future directions} \label{sec:conclusion}

In this paper we proposed a nested MLMC framework that offers important computational savings by performing most calculations in low precision and exploiting approximate random normal variables for the low precision path calculations. The low precision calculations could be performed in fixed precision on an FPGA for greater efficiency, and we suggested a procedure to optimise the bit-widths of every variable at each Monte Carlo level. This is an important improvement over previous mixed precision MLMC frameworks which held the lower precision fixed \cite{Rounding_error_oliver} or defined uniform bit-width at every level heuristically \cite{brugger2014mixed}. Our numerical results suggest that for the first levels our procedure reduces the cost at these levels by a factor 5 or 7. Hence the overall savings are significant since most paths are calculated on the first levels. Our approach would be even more efficient for the Milstein scheme because its higher order strong convergence leads to a greater proportion of the computational costs being on the coarsest levels.

The next stage of the research project will be to implement the RNG methods and the nested framework on FPGAs to determine the hardware requirements and confirm the extent of the computational savings. It would also be good to compare the performance benefits to using half-precision floating point arithmetic on GPUs or CPUs for the low-accuracy computations.




\section*{Limitations}
We make every effort to ensure that all pairs are correctly labeled, but cannot guarantee complete accuracy due to potential bugs in the toolchains or errors in the inputs (e.g., solutions from programming contests may be accepted based on a limited set of test cases that might not fully expose underlying bugs in the accepted solutions).


\section*{Acknowledgements}
We thank Lianmin Zheng, Shiv Sundram, Mingfei Guo, Xiaohan Wang, and Allen Nie for their discussions.

\bibliography{custom}


\setcounter{figure}{0}
\renewcommand{\thefigure}{A\arabic{figure}}
\setcounter{table}{0}
\renewcommand{\thetable}{A\arabic{table}}

\newpage
\appendix
\onecolumn


\section{Appendix}
\label{sec:appendix}
\appendix

\section{Appendix: Prompt}
\label{sec:appendix}
``Here is a sketch of an image. 
$\{input\_color\_mask\}$, while the rest of the white space is the background. 
I need you to infer details of the image based on the given sketch.
The details should include the possible background likely to be present with the $\{input\_color\_mask\}$, the attribute of each object (like wearing, texture, color etc.), the state (including action, posture, etc.) of each object, the direction of each object and the relationships between objects.

You should first analyze the mask carefully, considering the size, location, and relative position of each object mask. Ensure that specific actions are analyzed based on the mask, and infer each aspect with a reasoning process before providing the final output.
The final output format should be: $\{format\_example\}$, and you should refer to the example: $\{few\_shot\}$. You are going to complete the "" in each item, you need to complete them in multiple short phrases based on your above reasoning.

The state and relationship should be as detailed as possible while ensuring they align with the mask, formatted as: objectA action/spatial relation objectB, with both objectA and objectB included.
You should properly refer to some examples of attributes of object $\{attributes\}$ and relationships $\{relationships\}$.
Do not include words like `or', `possibly' in your final output, there should no ambiguity in your output.
Make sure all aspects of given mask is filled.''



\end{document}
