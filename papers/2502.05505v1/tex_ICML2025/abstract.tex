\begin{abstract}
    Differentially private (DP) synthetic data, which closely resembles the original private data while maintaining strong privacy guarantees, has become a key tool for unlocking the value of private data without compromising privacy. Recently, \privateevolution{} (\pe{}) has emerged as a promising method for generating DP synthetic data. Unlike other training-based approaches, \pe{} only requires access to inference APIs from foundation models, enabling it to harness the power of state-of-the-art models. However, a suitable foundation model for a specific private data domain is not always available. In this paper, we discover that the \pe{} framework is sufficiently general to allow inference APIs beyond foundation models. Specifically, we show that simulators—such as computer graphics-based image synthesis tools—can also serve as effective APIs within the \pe{} framework. This insight greatly expands the applicability of \pe{}, enabling the use of a wide variety of domain-specific simulators for DP data synthesis. We explore the potential of this approach, named \simpe{}, in the context of image synthesis. Across three diverse simulators, \simpe{} performs well, improving the downstream classification accuracy of \pe{} by up to 3$\times$ and reducing the FID score by up to 80\%. We also show that simulators and foundation models can be easily leveraged together within the \pe{} framework to achieve further improvements. The code is open-sourced in the \privateevolution{} Python library: \url{https://github.com/microsoft/DPSDA}.
\end{abstract}