\section{Related Work}
\label{sec:rel-work}

Point identification of causal effects generally requires strong assumptions on the causal structure or the functional relationships between the variables. When these assumptions are not met, effects can often still be \emph{partially identifiable}, meaning that we can confine the effect to lie within some non-empty, but constrained \emph{set} of effects that are compatible with the structural assumptions and the observed data \citep{manski1990nonparametric}. Such effect sets are often characterized by intervals, defined by a lower and upper bound on the target quantity. Initial work on this was pioneered by \citet{Balke1997,chickering96}, and the topic has attracted much attention lately. 

Most of this work has focused on partial identification or sensitivity analysis in the presence of confounding. Different streams of work have focused on bounding causal effects under a variety of assumptions, including assuming discrete domains~\citep{zhang2021bounding, duarte2023automated,raichev2024estimating}, assuming the presence of an instrument~\citep{gunsilius2019path, kilbertus2020class}, using neural networks~\citep{hu2021generative, padh2023stochastic}, or performing causal sensitivity analysis~\citep{frauen2024sharp, melnychuk2024partial, jesson2021quantifying, marmarelis2023partial}.
Other sources of uncertainty in causal inference include ones arising from interaction patterns in non-IID data~\citep{zhang2022causalinferenceuncertainty,bhattacharya2020causal}, shifts in covariates~\citep{Jesson2020identifying}, and unobserved confounders~\citep{Tchetgen2013control}.

Despite these efforts, the uncertainty related to the causal graph itself, particularly in the final goal of treatment effect estimation, has received less attention.
Some works focus on improving the inference under the uncertainty introduced by previous causal discoveries on the same data \citep{chang2024post, gradu2022valid, malinsky2024cautious}.
For constraint-based causal discovery, these are often characterized by Markov equivalence class (MEC) over which one can perform partial identification \citep{maathuis2009estimating, bellot2024towards}.
However, the prevailing approach---focusing on uncertain orientations of unoriented edges in the completed partially directed acyclic graph (CPDAG) corresponding to the MEC---is limited for two primary reasons.
First, adopting a more flexible approach to managing uncertainty in causal graphs could prove beneficial, especially when leveraging domain knowledge to infer edge information.
Second, several studies have highlighted the potentially unreliable performance of (constraint-based) causal discovery algorithms in real-world settings \citep{reisach2021beware}. Hence, uncertainties beyond the ones arising purely due to causal discovery should be considered.
In orthogonal work, \citet{henckel2024adjustment} highlight that the ``utility'' of a specific assumed graph typically depends on the downstream task the graph is used for. Starting from a downstream task, i.e., cause-effect estimation, they measure how `close any given DAG is to the ground truth DAG' by comparing how many correct cause-effect estimands the proposed DAG recovers. This idea leads to novel metrics to measure the quality of causal discovery \emph{with a downstream task in mind}.

More recent work by \citet{strieder2023confidence,strieder2024dual} has developed confidence regions for total causal effects that account for uncertainties in both the causal structure and numerical size of nonzero effects, but their applicability is restricted to linear models.
Additionally, efforts to mitigate uncertainties from imperfect expert input have utilized Large Language Models \citep{long2023causal} and pre-processing techniques \citep{oates2017repair}.

Our work is complementary to these efforts and aims to further fill the gap in inference under the uncertainty coming from the specification of the causal graph. Unlike previous efforts, our method allows a pre-defined set of plausible causal graphs that are more general than the MEC, does not require the linearity assumption, and uses continuous optimization to allow the method to extend more easily to larger graphs. We describe our method in the next section.

%
%
%