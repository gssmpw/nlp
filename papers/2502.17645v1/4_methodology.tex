\section{Methodology}
\label{sec:methodology}
We analyze responses from $N=112$ participants ($N=105$ filled out the open-ended questions) who completed the post-test survey, out of a total of 122 recruited participants. 
% The MiWaves study design and protocol did not originally include qualitative analysis as a primary objective. Consequently, the qualitative analysis presented here was limited to examining the responses of the participants to open-ended questions about their likes, dislikes, and suggestions to improve the app. These questions provided valuable, albeit limited, insight into user experiences with MiWaves. 
We conducted an inductive thematic analysis \cite{clarke2017thematic} on open-ended questions (see Table \ref{app:open_ended}). The coded responses were classified into three primary domains (similar to the questions): participants' likes, dislikes, and suggestions. Further analysis was conducted to synthesize themes and identify actionable insights. For each domain, recurring patterns were examined, with particular attention to responses that highlighted user engagement, burden, and overall experience with the app. Subsequently, we grouped related feedback into sub-themes, such as the impact of self-monitoring check-ins, and the perception of intervention messages. 
% To deepen our understanding, we revisited the qualitative data to provide context and support for some of the quantitative survey responses. For example, participants' open-ended comments were analyzed to explain trends observed in their ratings of app usability, burden, and message relevance. This approach allowed us to draw nuanced connections between participants’ self-reported experiences and their survey responses, highlighting both strengths and areas for improvement in the MiWaves intervention. 
These findings, along with the quantitative data analysis, informed the key insights presented in Section \ref{sec:findings}, emphasizing self-awareness, user burden, and personalization in digital health tools.



% In this analysis, we dived deeper into the participant dislikes, since a lot of participants mentioned encountering bugs while using the app. \sg{Bugs are a detriment to engagement in mobile heatlth studies. add cite and polish} For the quantitative analysis, we focused on responses to the acceptability questions (see Table \ref{app:quant}), which predominantly used items on the Likert scale. In addition, we correlated these responses to the participant app usage - specifically matching the responses to the interactions and dwell times of the app \sg{This part might end up going to CHI submission, won't have time for the class one}.
