\section{Background}
\label{sec:background}

% \begin{figure*}[!t]
%     \centering
%     % First subfigure
%     \begin{subfigure}[b]{0.3\textwidth}
%         \centering
%         \includegraphics[width=\textwidth]{screenshots/miwaves_EMA.jpg} % Replace with your image file
%         \caption{Self-monitoring survey screen}
%         \label{fig:app_ema}
%     \end{subfigure}
%     \hfill
%     % Second subfigure
%     \begin{subfigure}[b]{0.3\textwidth}
%         \centering
%         \includegraphics[width=\textwidth]{screenshots/miwaves_message_worry.jpg} % Replace with your image file
%         \caption{Intervention message}
%         \label{fig:app_message}
%     \end{subfigure}
%     \hfill
%     % Third subfigure
%     \begin{subfigure}[b]{0.3\textwidth}
%         \centering
%         \includegraphics[width=\textwidth]{screenshots/life_insights.jpg} % Replace with your image file
%         \caption{My Trends screen}
%         \label{fig:app_life_insights}
%     \end{subfigure}
    
%     \caption{Screenshots from the MiWaves app. (a) Self-monitoring survey screen -- one of the questions prompting participants to report cannabis use in the past 12 hours, (b) an example of an intervention message providing motivational content to encourage reflection and mindfulness, and (c) the `My Trends' screen, which displays participants’ self-reported trends in cannabis use, sleep, and exercise over the past week, helping them reflect on their lifestyle habits.}
%     \label{fig:miwaves_app_screenshots}
% \end{figure*}


\subsection{MiWaves app}
\label{sec:miwaves_app}
% \sg{TODO: Needs polishing} 
The MiWaves app (available on iOS and Android) is a digital health intervention designed to support EAs in reducing cannabis use through self-monitoring, personalized feedback, and tailored resources. 
% The MiWaves app is available on both the iOS and Android platforms. 
MiWaves utilizes \textbf{self-monitoring check-ins}, which prompt participants twice daily to self-report their cannabis use, stress levels, exercise, and sleep patterns, potentially prompting participants to reflect on their daily activities and experiences 
% (see Figure \ref{fig:app_ema})
% habits 
% During onboarding, participants can customize their preferred self-monitoring survey notification times, selecting 2-hour windows for morning and evening prompts, ensuring that the app integrates seamlessly into their daily routines. 
(details in Appendix \ref{app:self_monitoring_questions}). Based on the self-reported data, MiWaves utilizes \textbf{reBandit} \cite{ghoshmiwaves2024}, an RL algorithm, to personalize the likelihood of delivering an \textbf{intervention message}.
% (see Figure \ref{fig:app_message})
% The algorithm is designed to balance intervention delivery with moments of no intervention, maximizing user engagement while avoiding habituation to the intervention content. 
% Since user engagement is a primary determinant of success in digital interventions \cite{nahum2018jitai}, reBandit optimizes treatment to maximize user engagement. 
The intervention messages are randomly chosen (without replacement) from a pool of messages with varying length and interaction types (more details in Table \ref{tab:intervention_prompts} of Appendix).
% provides a summary of the different types of intervention messages utilized in MiWaves.
% \begin{table*}[!t]
% \centering
% \begin{tabular}{|>{\centering\arraybackslash}m{0.2\textwidth}|>{\centering\arraybackslash}m{0.35\textwidth}|>{\centering\arraybackslash}m{0.35\textwidth}|}
% \hline
% \textbf{Interaction Level} & \textbf{Short Length} & \textbf{Long Length} \\ \hline
% \textbf{A (acknowledge the message)} & 
% \emph{You are the artist and the future is your canvas. When you think about your life in 6 months from now, what do you hope for?} & 
% \emph{Do you ever find yourself burying or suppressing your emotions? Talking your feelings out with someone you trust or writing a private journal entry can help you free those emotions.} \\ \hline
% \textbf{B (participant requested to visit external resource)} & 
% \emph{What's your favorite song? Learn more about how music is beneficial to your mental health: \url{https://www.youtube.com/watch?v=zJ2YGLuzGfo}} & 
% \emph{Trying new things doesn't have to be expensive. When you're tight on cash, check out this list to see if any of these cheap and easy hobbies seem interesting: \url{https://www.buzzfeed.com/tomvellner/cheap-easy-hobbies}} \\ \hline
% \textbf{C (requires input from participant)} & 
% \emph{Fun fact: even just five mins of physical activity can be beneficial. How do you get your body moving? \_\_\_\_\_\_\_\_\_\_} & 
% \emph{You have the power to achieve anything you put your mind to. From this list, what are some ways you are interested in building a plan to create the future that you hope for yourself? 
% (A) Writing goals 
% (B) Attending therapy 
% (C) Budget money 
% (D) Establish healthy routines} \\ \hline
% \end{tabular}
% \caption{Examples of intervention message prompts by interaction level and length.}
% \label{tab:intervention_prompts}
% \end{table*}
The MiWaves app also includes a \textbf{My Trends} screen, 
% (see Figure \ref{fig:app_life_insights}), 
which provides participants with visual summaries of their self-reported cannabis use, sleep patterns, and exercise levels over the past week. This feature fosters self-awareness by allowing participants to track their progress and identify patterns in their behavior. 
% Additionally, MiWaves offers a comprehensive \textbf{Resources} screen to support a wide range of participant needs like mental health resources, substance use services, etc. 
% This screen provides curated information on mental health and substance use services, overdose prevention, housing and hunger support, LGBTQ+ and gender identity resources, pregnancy and parenting services, education and employment opportunities, community activities, health services, and violence prevention resources.

\subsection{MiWaves pilot study}
\label{sec:miwaves_pilot_study}
The MiWaves pilot study was a registered clinical trial (NCT05824754) for the MiWaves intervention which ran from March 2024 to May 2024. For the study, $N=122$ EAs were recruited
% using social media 
across the U.S. who reported using cannabis regularly and expressed motivation to reduce their use. Participants had a mean age of 21.7 years (SD = 2.0), with 53.3\% identifying as female; the racial distribution was 73.0\% White, 12.3\% Black, 4.9\% Asian, and 9.8\% Other.
% During the trial, each day, along with the self-monitoring responses, MiWaves recorded the participant's notification interactions, app screen visitation rates and corresponding dwell times (all within the MiWaves app). 
% Participants in the study were also subject to screening, baseline, weekly, post-test and 2-month follow-up surveys.
% Participants were compensated \$40 each for the baseline and post-test surveys, \$15 for each of the four weekly surveys, and \$50 for the 2-month follow-up surveys via electronic gift cards. 


For this analysis, we focus primarily on the acceptability and open-ended questions which were part of the post-test survey data collected at the end of the study for a participant. 
% The MiWaves post-test survey comprehensively evaluated participants' experiences and outcomes related to the intervention. 
% It assessed cannabis use frequency, associated consequences, and protective strategies, drawing on validated measures like the Marijuana Adolescent Consequences Questionnaire (MACQ) \cite{simons2012dimensions}. Participants also reported on driving behaviors under the influence of cannabis and other substances, alongside usage patterns of nicotine, alcohol, and prescription drugs. Mental health metrics, including screens for anxiety, depression, and PTSD, provided insights into participants' psychological well-being. 
The acceptability questions asked participants about feedback on the app’s functionality, aesthetics, and engagement features, as well as their perceived effort and burden, and the relevance and timing of intervention messages. The open-ended survey questions were intended to explore participants’ perceptions of the app, such as features they liked the most and least, and suggestions for improvement. Please refer to Appendix \ref{app:posttest_codebook} for the exact list of post-test survey items considered for this analysis. For the complete set of post-test survey items, or more information regarding the study design for the MiWaves pilot study, please refer to the MiWaves protocol paper \cite{coughlin2024mobile}.

% After the study, we observed that 44.3\% of the participants reported daily cannabis use, and 25.4\% reported using cannabis within an hour of waking. Additional substance use included nicotine (13.9\%), alcohol (89.3\%), prescription drug misuse (15.6\%), and other drugs (19.7\%). On average, participants reported 10.9 cannabis-related consequences in the past month (SD = 4.6), with 59.0\% admitting to driving under the influence of cannabis. Mental health screenings revealed positive results for depression (34.4\%), anxiety (50.8\%), and PTSD (59.0\%). 


% \sg{Mention what the goals of the pilot study were (exact goals and metrics) and how this work is slightly outside that plan}

% For completing assessments, participants will receive $40 each for the baseline and post-intervention assessments, $15 for each of the four weekly assessments, and $50 for the 2-month follow-up assessment via electronic gift card.

% \sg{
% TODO: Figure out where this information fits:
% 44.3\% of the participants reported daily cannabis use, and 25.4\% reported using cannabis within an hour of waking. Additional substance use included nicotine (13.9\%), alcohol (89.3\%), prescription drug misuse (15.6\%), and other drugs (19.7\%). On average, participants reported 10.9 cannabis-related consequences in the past month (SD = 4.6), with 59.0\% admitting to driving under the influence of cannabis. Mental health screenings revealed positive results for depression (34.4\%), anxiety (50.8\%), and PTSD (59.0\%). 
% }