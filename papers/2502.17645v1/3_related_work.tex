\section{Related Work}
\label{sec:related_work}
Our analysis builds on prior work in the space of qualitatively analyzing participant experiences with digital interventions. This aligns with broader efforts in digital health and HCI to understand how users engage with and perceive mobile interventions.

Several digital health studies have demonstrated the utility of adaptive systems for behavior change \cite{nahum2018jitai,aschentrup2024effectiveness}. 
% For instance, a meta-analysis of randomized controlled trials revealed that digital interventions effectively reduce anxiety and depression symptoms among adolescents, with treatment outcomes influenced by factors such as therapist guidance and the number of sessions \cite{li2024digital}. Additionally, research has highlighted the importance of integrating implementation science into human-centered design for digital health interventions, aiming to enhance scalability and sustained use in healthcare settings \cite{waddell2024leveraging}. 
A large volume of these works \cite{mcclure2023feasibility,klasnja2019efficacy,
% nahum2021translating,coughlin2021toward,
golbus2024text} focus only on the quantitative metrics while reporting results. 
% For instance, HeartSteps \cite{klasnja2019efficacy,spruijt2022advancing} employed reinforcement learning (RL) to optimize the timing of interventions for increasing physical activity but largely focused on objective measures of engagement and outcomes. 
However, qualitative analysis can provide crucial insights into user experiences and identify opportunities for intervention improvement. 
% For instance, Time2Stop combined adaptive prompts with user feedback to refine its smartphone overuse intervention. The study's qualitative analysis of user feedback highlighted nuanced themes about engagement and the perceived usefulness of prompts, which informed iterative improvements to the system design \cite{time2stop2024}. Similarly,
For instance, ReVibe \cite{rabbi2019revibe} incorporated a context-assisted evening recall approach to improve self-report adherence in digital health interventions. The qualitative findings revealed how users experienced the intervention, shedding light on factors such as recall difficulties which directly influenced the design of future iterations. These studies demonstrate the critical role of qualitative analysis in uncovering insights that cannot be captured through quantitative data alone. Motivated by this, we aim to conduct a structured qualitative analysis of participant feedback from MiWaves to identify themes that reflect the user experience and potential areas for improving the intervention design.

% In the domain of qualitative analysis of user experiences, several works have explored methodologies and their implications for improving tools and interventions. For instance, thematic analysis is commonly used to distill user feedback into actionable themes, as demonstrated in studies like User Perspectives and Ethical Experiences of Apps for Depression \cite{bowie2022user}, which analyzed reviews from app stores to uncover recurring ethical and functional issues with mental health apps. Another relevant study \cite{dennard2024systematic} synthesized qualitative research findings to identify design considerations and barriers specific to this user group. Furthermore, past research \cite{bowman2023using} has highlighted the importance of aligning qualitative methodologies with user-centered design principles to better capture the nuances of user interaction with healthcare technologies. Our work aims to utilize similar ideas to qualitatively analyze participant feedback data, while also utilizing app usage statistics to supplement our findings.

% \sg{Paragraphs need polish, sounds rough. Also cite the paper which talks about bugs being the detriment of engagement in digital interventions}
