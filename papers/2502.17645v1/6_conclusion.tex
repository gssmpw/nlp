% \vspace{-1.64em}
\section{Limitations and Future Work}
\label{sec:limitations_future}

While our analysis of the MiWaves pilot study post-test data provided valuable insights, several limitations must be acknowledged. The pilot study was not originally designed for such analysis, and qualitative data collection was limited to open-ended survey responses.
% , potentially introducing recall bias
Future MiWaves clinical trials will include targeted questions designed to facilitate robust qualitative analysis. 
This will involve specific prompts to better capture user experiences and perceptions related to the intervention’s usability and impact. 
We also aim to invite a sub-sample of participants for detailed feedback sessions post-study where participants can be shown examples of intervention messages they received during the study (to reduce recall bias). 

Additionally, participants could not modify their self-monitoring survey times after onboarding, which was a noted limitation. Future iterations will introduce greater flexibility to accommodate user preferences. Technical issues, including app crashes and debugging notifications, also affected the user experience, highlighting the need for improved system stability. Lastly, while message timing was optimized using reinforcement learning, message content was not personalized, contributing to mixed feedback. Future work will explore personalization strategies to enhance message relevance and engagement, ensuring interventions remain supportive rather than prescriptive.

% The MiWaves pilot study, while yielding valuable insights, was not designed for a Human-Computer Interaction (HCI)-centric analysis of participant feedback. A structured approach to gathering user feedback, tailored specifically for HCI analysis, was absent from the study’s design. Moreover, the qualitative data collected through open-ended survey responses was limited in scope (given the design) and retrospective, which inherently may have introduced recall bias. Future MiWaves clinical trials will include targeted questions designed to facilitate robust qualitative analysis. This will involve specific prompts to better capture user experiences and perceptions related to the intervention’s usability and impact. We also aim to invite a sub-sample of participants for detailed feedback sessions post-study. During these sessions, participants can be shown examples of intervention messages they received during the study to jog their memory and elicit more reflective and actionable feedback about the intervention content and timing.

% We also acknowledge that participants were not able to change their self-monitoring survey times after the initial onboarding process. Although this decision was partly informed by previous studies (and partly due to time constraints on the development process), where participants rarely utilized the flexibility to alter their survey times, feedback from this study highlighted a clear desire among users for such flexibility. The lack of this option was perceived as a restriction and may have contributed to user frustration. Future iterations of MiWaves will allow participants to modify their self-monitoring survey times after onboarding. While this flexibility may not be frequently utilized, its presence addresses a key user preference and enhances the perception of autonomy.

% Additionally, participants reported a range of technical issues, including frequent app crashes and debug/error notifications. These bugs disrupted the user experience and may have adversely impacted participant engagement and trust in the intervention. Addressing these technical shortcomings is crucial for future iterations. Our future efforts will focus on improving the app’s stability by addressing the reported issues of crashes and debug/error notifications.

\section{Conclusion}
\label{sec:conclusion}
% This paper presented an exploratory quantitative and qualitative analysis of participant feedback from the MiWaves pilot study, providing valuable insights into the potential of MiWaves as a digital intervention for cannabis use reduction among emerging adults. Despite study design limitations and subsequent limitations to our analysis, our initial findings highlight the promise of MiWaves as a personalized and adaptive tool for behavior change. Participants valued self-monitoring check-ins and trend visualizations for fostering self-awareness and reflection, though technical issues and limitations, such as rigid scheduling and app stability, hindered the overall experience. Privacy concerns and frustrations with expiring notifications further underscored the need for user-centric design improvements.

% To address these findings, future iterations will focus on enhancing flexibility in scheduling, addressing technical challenges, and refining engagement strategies based on user feedback. Integrating HCI-focused qualitative questions into post-intervention surveys, and leveraging post-survey memory aids will allow for deeper insights into participant experiences, while iterative advancements in personalization will ensure alignment with user needs. These improvements aim to make MiWaves a more robust, adaptive, and user-centered intervention. By incorporating lessons learned, MiWaves has the potential to advance the field of digital health interventions, setting a new standard for behavior change tools that effectively integrate into users' daily lives.
% \sg{TODO}
This paper presented an exploratory analysis of the MiWaves digital intervention, revealing its potential as a personalized and adaptive intervention to support cannabis use reduction among emerging adults. Our initial findings suggested the effectiveness of self-monitoring check-ins and trend visualizations in fostering self-awareness and encouraging behavioral reflection. The MiWaves intervention message timing and frequency was generally well-received. Intervention messages involving tasks such as typing responses or exploring links, were seen as burdensome compared to tasks like reading and acknowledging messages. Participant feedback highlighted opportunities to enhance future iterations of intervention messages by integrating more personalized and contextually relevant content and highlights
an area for future work: understanding how message ``tone'' may
be inferred and preferred differently across participants.
% Privacy concerns regarding possible future contextual variables and frustrations with rigid notification schedules underscore the need for user-centric design adjustments. 
Building on these findings, future iterations of MiWaves will prioritize reducing user burden through low-effort, impactful interactions, refining the RL algorithm for intervention timing, and improving personalization. Enhanced qualitative data collection, including targeted questions and reflective post-test surveys, will help provide deeper insights into participant experiences. 