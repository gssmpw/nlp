\pagebreak
\appendix

\clearpage
\onecolumn

\section{MiWaves overview}
\label{app:miwaves_overview}

\begin{figure*}[!h]
    \centering
    \includegraphics[width=0.8\textwidth]{screenshots/miwaves_overview.png}
    \caption{Overview of the MiWaves pilot study.}
    \label{fig:miwaves_overview}
    \Description[Overview of the MiWaves pilot study]{Overview of the MiWaves pilot study}
\end{figure*}

The MiWaves app is a mobile health (mHealth) intervention designed to support emerging adults (EAs) in reducing cannabis use through self-monitoring, personalized feedback, and tailored resources. The MiWaves app is available on both the iOS and Android platforms. MiWaves utilizes \textbf{self-monitoring surveys} (see Figure \ref{fig:app_ema}), which prompt users twice daily to self-report their use of cannabis, stress levels, exercise, and sleep patterns, allowing users to reflect on their habits. During onboarding, participants can customize their preferred self-monitoring survey notification times, selecting 2-hour windows for morning and evening prompts, ensuring that the app integrates seamlessly into their daily routines. Full set of questions asked during these self-monitoring surveys can be found in Appendix \ref{app:self_monitoring_questions}.

Based on the self-reported data, MiWaves utilizes \textbf{reBandit} \cite{ghoshmiwaves2024}, an RL algorithm, to personalize the likelihood of delivering an \textbf{intervention message} (see Figure \ref{fig:app_message}). The algorithm is designed to balance intervention delivery with moments of no intervention, maximizing user engagement while avoiding habituation to the intervention content. Since user engagement is a primary determinant of success in digital interventions \cite{nahum2018jitai}, reBandit optimizes treatment to maximize user engagement. The intervention messages are randomly chosen (without replacement) from a pool of messages with varying length and interaction types. Table \ref{tab:intervention_prompts} provides a summary of the different types of intervention messages utilized in MiWaves.
\begin{table}[!h]
\centering
\begin{tabular}{|>{\centering\arraybackslash}m{0.2\textwidth}|>{\centering\arraybackslash}m{0.35\textwidth}|>{\centering\arraybackslash}m{0.35\textwidth}|}
\hline
\textbf{Interaction Types} & \textbf{Short Length} & \textbf{Long Length} \\ \hline
\textbf{A (acknowledge the message)} & 
\emph{You are the artist and the future is your canvas. When you think about your life in 6 months from now, what do you hope for?} & 
\emph{Do you ever find yourself burying or suppressing your emotions? Talking your feelings out with someone you trust or writing a private journal entry can help you free those emotions.} \\ \hline
\textbf{B (participant requested to visit external resource)} & 
% \emph{What's your favorite song? Learn more about how music is beneficial to your mental health: \url{https://www.youtube.com/watch?v=zJ2YGLuzGfo}} & 
\emph{What's your favorite song? Learn more about how music is beneficial to your mental health: [youtube video link]} & 
% \emph{Trying new things doesn't have to be expensive. When you're tight on cash, check out this list to see if any of these cheap and easy hobbies seem interesting: \url{https://www.buzzfeed.com/tomvellner/cheap-easy-hobbies}} \\ \hline
\emph{Trying new things doesn't have to be expensive. When you're tight on cash, check out this list to see if any of these cheap and easy hobbies seem interesting: [weblink]} \\ \hline
\textbf{C (requires input from participant)} & 
\emph{Fun fact: even just five mins of physical activity can be beneficial. How do you get your body moving? \_\_\_\_\_\_\_\_\_\_} & 
\emph{You have the power to achieve anything you put your mind to. From this list, what are some ways you are interested in building a plan to create the future that you hope for yourself? 
(A) Writing goals 
(B) Attending therapy 
(C) Budget money 
(D) Establish healthy routines} \\ \hline
\end{tabular}
\caption{Examples of intervention message prompts by interaction types and length.}
\label{tab:intervention_prompts}
\end{table}
The MiWaves app also includes a \textbf{My Trends} screen (see Figure \ref{fig:app_life_insights}), which provides participants with visual summaries of their self-reported cannabis use, sleep patterns, and exercise levels over the past week. This feature fosters self-awareness by allowing users to track their progress and identify patterns in their behavior. Additionally, MiWaves offers a comprehensive \textbf{Resources} screen to address a wide range of participant needs. This screen provides curated information on mental health and substance use services, overdose prevention, housing and hunger support, LGBTQ+ and gender identity resources, pregnancy and parenting services, education and employment opportunities, community activities, health services, and violence prevention resources.

\begin{figure}[!t]
    \centering
    % First subfigure
    \begin{subfigure}[b]{0.3\textwidth}
        \centering
        \includegraphics[width=\textwidth]{screenshots/miwaves_EMA.jpg} % Replace with your image file
        \caption{Self-monitoring survey screen}
        \label{fig:app_ema}
    \end{subfigure}
    \hfill
    % Second subfigure
    \begin{subfigure}[b]{0.3\textwidth}
        \centering
        \includegraphics[width=\textwidth]{screenshots/miwaves_message_worry.jpg} % Replace with your image file
        \caption{Intervention message}
        \label{fig:app_message}
    \end{subfigure}
    \hfill
    % Third subfigure
    \begin{subfigure}[b]{0.3\textwidth}
        \centering
        \includegraphics[width=\textwidth]{screenshots/life_insights.jpg} % Replace with your image file
        \caption{My Trends screen}
        \label{fig:app_life_insights}
    \end{subfigure}
    
    \caption{Screenshots from the MiWaves app. (a) Self-monitoring survey screen -- one of the questions prompting participants to report cannabis use in the past 12 hours, (b) an example of an intervention message providing motivational content to encourage reflection and mindfulness, and (c) the `My Trends' screen, which displays participants’ self-reported trends in cannabis use, sleep, and exercise over the past week, helping them reflect on their lifestyle habits.}
    \label{fig:miwaves_app_screenshots}
\end{figure}

\section{Additional findings}
\label{app:add_findings}

\begin{figure}[t]
    \centering
    \includegraphics[width=\linewidth]{figures/effort.png}
    \caption{Responses to participant's perception of MiWaves (intervention) message burden -- specifically with respect to the different tasks in intervention messages - namely reading and acknowledging messages, visiting links and typing-in or choosing responses.}
    \label{fig:message_effort}
    \Description{Responses to participant's perception of MiWaves (intervention) message burden -- specifically with respect to the different tasks in intervention messages, including reading messages, visiting links and typing in responses. The graph highlights the varying levels of effort participants associated with each task, with reading or acknowledging messages perceived as the least effortful by the majority and exploring links or typing-in responses requiring more effort.}
\end{figure}

\begin{figure}[t]
    \centering
    \includegraphics[width=\linewidth]{figures/checkin.png}
    \caption{Responses to check-in related questions, which asked the participant about how easy and time-consuming the self-monitoring questions (or check-ins) were, and how comfortable were participants answering personal questions (eg: about their cannabis use, sleep patterns etc.)}
    \label{fig:checkin}
    \Description{Responses to check-in related questions, which asked the participant about how easy and time-consuming the self-monitoring questions (or check-ins) were, and how comfortable were participants answering personal questions (eg: about their cannabis use, sleep patterns etc.)}
\end{figure}

\subsection{Context features and privacy}
\begin{figure}[t]
    \centering
    \includegraphics[width=\linewidth]{figures/comfort.png}
    \caption{Responses to participant's privacy outlook - i.e. questions asking how comfortable they were with an app collecting data about their location, social media messages, text messages, moving speed, physical activity and sleep.}
    \label{fig:privacy}
    \Description{Responses to participant's privacy outlook - i.e. questions asking how comfortable they were with an app collecting data about their location, social media messages, text messages, moving speed, physical activity and sleep. Participants were relatively more comfortable with an app tracking their physical activity and sleep, as compared to their location, text messages, social media messages and their moving speed.}
\end{figure}

Digital interventions often leverage contextual data, such as location, physical activity, and sleep patterns, to enhance personalization through adaptive algorithms. These algorithms rely on real-time data to deliver interventions tailored to users' behaviors, routines, and environments, making the experience more relevant and effective. However, integrating such data collection in apps like MiWaves must also address participants' privacy concerns and comfort levels to maintain trust and engagement. To inform future iterations of MiWaves, participants were asked about their comfort with an app collecting data across various contextual variables.

Figure \ref{fig:privacy} presents participants' responses to questions about their comfort levels with the collection of six specific types of data - namely location tracking, text messages, social media messages, movement speed (which could include traveling in vehicles), physical activity (e.g. step count, heart rate etc.) and sleep. Participants were asked to rate their comfort level (1 to 10) with 1 being not comfortable at all, and 10 being very comfortable. The findings reveal a nuanced outlook -- the majority of participants expressed discomfort with the collection of location data (mean = 3.46, SD = 1.05), text messages (mean = 2.81, SD = 1.16), and social media messages (mean = 2.49, SD = 1.24). This highlights a significant privacy concern around these sensitive forms of contextual data, indicating their use may not align with participants’ preferences in future iterations of MiWaves. Conversely, participants reported higher comfort levels with the collection of physical activity data (mean = 6.70, SD = 0.93) and sleep data (mean = 6.33, SD = 0.89). These findings suggest that participants were more open to sharing physiological or behavioral data compared to communication or location data.

% Interestingly, comfort levels for collecting data about speed of movement were more evenly distributed, with participants showing mixed reactions.

% \sg{Talk about future trials using a more complex context for personalization. Hence we asked participants about their outlook on which features they'd be more comfortable with }

\subsection{Expiring notifications and user experience}
The MiWaves intervention was designed with the intention that self-monitoring check-ins and MiWaves message notifications would expire after a set period. This decision was primarily made to clearly delineate time boundaries for post-study effect analysis, allowing researchers to more accurately understand the impact of each intervention. However, this design choice led to dissatisfaction among some participants who felt constrained by the expiration feature, especially when they missed opportunities to self-monitor or engage with messages.

$N=8$ participants explicitly remarked on the frustrations caused by expired check-ins or limited response windows. For instance, P102 stated, \emph{``that check ins expired too fast so i didn’t get to fill them out even though i could’ve and i wanted to''}, emphasizing the rigidity of the design. Similarly, P40 noted, \emph{``Missing the time slot to do the checkin resulted in no data for that half of a day\ldots wish there was a way to still go in and enter the data''}. Another participant, P62, expressed, \emph{``if u missed a checkin you couldn't correct your data''} reflecting broader dissatisfaction with the inability to make amendments. Suggestions for addressing this issue were also provided by participants. P102 proposed, \emph{``Longer check-in periods that don’t expire,''} while P71 suggested, \emph{``A longer gap in time to answer, 3 or 4 hours.''} Additional participants, such as P78 and P79, echoed similar sentiments, requesting \emph{``longer check-in windows''} and \emph{``longer periods that you could check in for.''} These responses indicate a clear preference for increased flexibility in response times, which could help reduce participant frustration and improve overall satisfaction.

While the expiration feature served an essential purpose for study analysis, participant feedback underscores the importance of balancing research goals with user experience. For future iterations of MiWaves, one could consider allowing participants to complete expired check-ins. To maintain the rigor of effect analysis, such retroactive entries could be excluded from decision-making algorithms and flagged separately during data analysis.

\section{Self-monitoring questions}
\label{app:self_monitoring_questions}

\begin{longtable}{|p{3cm}|p{6cm}|p{5cm}|}
\hline
\textbf{VARIABLE NAME} & \textbf{QUESTION} & \textbf{VALUES} \\ \hline
\endfirsthead
\hline
\textbf{VARIABLE NAME} & \textbf{QUESTION} & \textbf{VALUES} \\ \hline
\endhead
use\_am (use\_pm)  & Since yesterday morning (evening), did you think about or use any suggestion from MiWaves Messages? & Y/N \\ \hline
sleep\_AM (sleep\_PM)  & Please select all hours you were asleep in the past 12 hours. & [select hours] \\ \hline
cann\_yes\_no  & In the past 12 hours, have you used any cannabis product? & Y/N \\ \hline
cann\_use\_am (cann\_use\_pm) \newline {[Display if cann\_yes\_no = Yes]} & Please select all hours you used any cannabis product in the past 12 hours. & [select hours] \\ \hline
reasons\_use \newline {[Display if cann\_yes\_no = Yes]} & What were your reasons for using cannabis? & Select all that apply \newline 1 = To enjoy the effects \newline 2 = To feel less depressed \newline 3 = To feel less anxious \newline 4 = To help sleep \newline 5 = To feel less pain \newline 6 = Nothing better to do \newline 7 = Another reason: write in \\ \hline
reasons\_not \newline {[Display if cann\_yes\_no = No]} & During the times you didn't use cannabis, what were your reasons for not using cannabis? & Select all that apply \newline 1 = Didn’t want to \newline 2 = More important things to do \newline 3 = No chance/time \newline 4 = Want to cut back \newline 5 = Ran out \newline 6 = Another reason: write in \\ \hline
drinks (only in AM)  & How many drinks containing alcohol did you have yesterday? & Type in number \\ \hline
exercise (only in AM)  & How many minutes in total did you engage in exercise yesterday? & Type in number \\ \hline
positive (only in PM) & Do you expect good things will happen to you tomorrow? & \\ \hline
\caption{Self-monitoring questions (questions appearing in AM and PM surveys)}
\end{longtable}

\begin{longtable}{|p{3cm}|p{6cm}|p{5cm}|}
\hline
\textbf{VARIABLE NAME} & \textbf{QUESTION} & \textbf{VALUES} \\ \hline
\endfirsthead
\hline
\textbf{VARIABLE NAME} & \textbf{QUESTION} & \textbf{VALUES} \\ \hline
\endhead
stress  & How stressed are you right now? & 0 = Not at all \newline 1 = Slightly \newline 2 = Somewhat \newline 3 = Moderately \newline 4 = A lot \\ \hline
energy  & How energetic are you feeling? & 0 = Very low energy/Sleepy \newline 1 = Low energy/Sleepy \newline 2 = Neutral \newline 3 = High energy \newline 4 = Very energetic \\ \hline
mood  & How is your mood right now? & 0 = Very low/Negative \newline 1 = Low/Negative \newline 2 = Neutral \newline 3 = Good/Positive \newline 4 = Very good/Positive \\ \hline
social  & Right now, how would you rate your satisfaction with your social life? & 0 = Very bad/Negative \newline 1 = Bad/Negative \newline 2 = Neutral \newline 3 = Good/Positive \newline 4 = Very good/Positive \\ \hline
suds  & How anxious or distressed are you right now? & 0 = Not at all \newline 1 = Slightly \newline 2 = Somewhat \newline 3 = Moderately \newline 4 = Extremely \\ \hline
crave\_1  & Are you currently craving cannabis? & Y/N \\ \hline
crave\_2 {[Display if crave\_1 = Yes]} & Please rate your cannabis craving on the following scale: & 0 = No urge \newline 1 = Slight urge \newline 2 = Some urge \newline 3 = Moderate urge \newline 4 = Extreme urge \\ \hline
\caption{Randomized self-monitoring questions - two questions from this pool get selected at random to appear in the self-monitoring.}
\end{longtable}
% \sg{TODO}
\pagebreak

\section{Post-test survey questions}
\label{app:posttest_codebook}
\subsection{Acceptability questions (quantitative)}
\label{app:quant}
\begin{longtable}{p{10cm}ccc}
\toprule
Question & Likert Range (1-X) & Average & SD \\
\midrule
\midrule
Is the app fun to use? & 5 & 2.85 & 0.30 \\
Is the app interesting? & 5 & 2.99 & 0.30 \\
How interactive is the app? & 3 & 1.62 & 0.15 \\
\hline
How often did you have technical problems with the
app (e.g., the app crashed, content wouldn't
load)? (1 = Never, 2 = Rarely, 3 = Sometimes, 4 = Regularly) & 4 & 2.46 & 0.21 \\
\hline
Overall, how would you rate the app's appearance? & 5 & 3.01 & 0.30 \\
\hline
How true is this statement? - I felt comfortable
answering personal questions on the app (1 = Strongly disagree, 5 = Strongly agree) & 5 & 4.38 & 0.42 \\
How comfortable would you be with an app
collecting passive data such as (1 = Not at all comfortable, 10 = Very comfortable): & & & \\
Location tracking & 10 & 3.46 & 1.05 \\
Text messages & 10 & 2.81 & 1.16 \\
Social media messages & 10 & 2.49 & 1.24 \\
The speed you're moving at & 10 & 4.70 & 0.89 \\
Physical activity (e.g., step counts, heart rate) & 10 & 6.70 & 0.93 \\
Sleep (e.g., breathing, idleness) & 10 & 6.33 & 0.89 \\
\hline
How much did the ability to track your trends
increase your use of the MiWaves App? & 3 & 1.93 & 0.13 \\
How would you rate the helpfulness of your trend graphs? & 10 & 5.48 & 0.86 \\
How much did earning reward cards increase your use of the MiWaves App? & 3 & 2.87 & 0.08 \\
How would you rate the reward cards you earned? & 10 & 8.09 & 1.16 \\
\hline
Please rate how much effort the following MiWaves
Message tasks required, with 1 being least effort
and 10 being most effort: & & & \\
Exploring links & 10 & 5.17 & 0.86 \\
Typing in or choosing
responses & 10 & 4.69 & 0.89 \\
Reading messages & 10 & 3.79 & 1.00 \\
\hline
How many activities or thought exercises from
MiWaves Messages did you receive that you could
see yourself using after the end of the study? (1 = None, 5 = Most or all of them) & 5 & 2.39 & 0.33 \\
I felt the MiWaves Messages and
notifications I received in the MiWaves app showed
care, warmth, and respect (1 = Not at all, 5 = Extremely). & 5 & 3.53 & 0.32 \\
Overall, I felt that I received (1 = Fewer messages, 3 = too many messages) & 3 & 1.95 & 0.13 \\
The messages mostly came at times that were (1 = Not convenient for me, 3 = convenient for me) & 3 & 2.24 & 0.14 \\
There were specific times that I wanted to receive
a message but did not. (1 = No, 2 = Yes) & 2 & 1.28 & 0.07 \\
There were specific times that I received a
message but preferred not to. (1 = No, 2 = Yes) & 2 & 1.27 & 0.07 \\
\hline
How much do you agree with the following
statements? (1 = Strongly disagree, 5 = Strongly agree) & & &\\
Completing the check-ins twice daily
was doable for me. & 5 & 4.27 & 0.40 \\
I could complete the check-ins in a
reasonable amount of time. & 5 & 4.40 & 0.27 \\
\hline
Answer using a scale of 1 to 10, where 1 = not
at all and 10 = definitely yes: & & & \\
Would you recommend using the MiWaves app? & 10 & 7.13 & 0.80 \\
How would you rate
the helpfulness of the MiWaves app? & 10 & 6.23 & 0.89 \\
\bottomrule
\end{longtable}

\subsection{Open-ended questions}
\label{app:open_ended}
The following open-ended questions were asked to the participants:
\begin{itemize}
    \item What did you like most about the MiWaves app features and MiWaves messages?
    \item What did you like least about the MiWaves app features and MiWaves messages?
    \item What would you change about the MiWaves app?
\end{itemize}

Participants who answered `Yes' to the following questions were provided with write-in fields to describe specific instances related to their responses::
\begin{itemize}
    \item There were specific times that I wanted to receive a message but did not.
    \item There were specific times that I received a message but preferred not to.
\end{itemize}

% \section{Post-test survey questions}
% \includepdf[pages=-]{Posttest Survey Codebook_MiWaves.pdf}