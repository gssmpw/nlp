\section{Introduction}

% \begin{figure*}[!t]
%     \centering
%     \includegraphics[width=0.8\textwidth]{screenshots/miwaves_overview.png}
%     \caption{Overview of the MiWaves pilot study.}
%     \label{fig:miwaves_overview}
%     \Description[Overview of the MiWaves pilot study]{Overview of the MiWaves pilot study}
% \end{figure*}


Cannabis use among emerging adults (ages 18-25) has been steadily on the rise worldwide, presenting significant health challenges \cite{unodc2023world}. 
% In the U.S., nearly 40\% of individuals in this age group reported using cannabis in 2022, reflecting a growing trend of normalization and recreational use \cite{patrick2023monitoring,SAMHSA}. 
Although not everyone who uses cannabis experiences harm, cannabis use can lead to adverse effects such as impaired cognitive function and the development of Cannabis Use Disorder (CUD) \cite{deaquino2021thc}.
% soleimani2023altered}. 
Despite these risks, there remains a significant gap in effective early interventions 
% for non-treatment-seeking individuals 
\cite{stephens2021reaching}. 
% ,lapham2019prevalence}. 
Traditional treatment approaches often fail to reach this population, 
% especially in everyday life contexts where cannabis use typically occurs. This highlights 
highlighting the need for innovative, accessible, and adaptive interventions that can provide real-time support to individuals.
% aiming to reduce their cannabis consumption.

To address this gap, we developed MiWaves \cite{coughlin2024mobile}, a \emph{personalizing} Just-in-Time Adaptive Intervention (JITAI) \cite{nahum2018jitai} to help emerging adults (EAs) to self-regulate and reduce their cannabis use. MiWaves prompts participants through a smartphone app to complete self-monitoring surveys twice daily, allowing them to self-report (and in-process potentially raising their self-awareness of) their cannabis use, sleep patterns, and stress levels. Based on these self-reports, MiWaves uses a reinforcement learning (RL) algorithm to personalize the likelihood of intervention message delivery. In order to evaluate the feasibility and acceptability of the intervention, we conducted the MiWaves pilot study (more details in Section \ref{sec:background}). 
% Figure \ref{fig:miwaves_overview} provides a visual overview of the MiWaves pilot study. 

This work contributes to the growing body of research on mobile health (mHealth) interventions for cannabis use by providing a structured analysis of participant feedback.
% -- a component often overlooked in previous studies \cite{nahum2021translating,coughlin2021toward,mcclure2023feasibility,golbus2024text}. 
% In particular, we aim to analyze participant app usage behavior combined with participant feedback collected through post-test follow-up surveys, focusing on both quantitative data (e.g., engagement metrics) and qualitative insights (e.g., open-ended responses about participant experiences). \sg{TODO: List out a summary of the findings, and contributions }. Using this analysis, we hope to identify key factors that influence participant engagement and intervention effectiveness, as well as potential areas for improvement in future iterations of MiWaves.
In particular, we aim to analyze participant feedback collected through post-test follow-up surveys, focusing on both quantitative data (e.g., engagement metrics) and qualitative insights (open-ended responses focused on three themes - likes, dislikes and suggestions for the improving the app). 
% It is important to emphasize that this analysis was exploratory and not the primary aim of the MiWaves pilot study, which was originally designed to evaluate feasibility and acceptability. 
While our analysis provides valuable insights into participant experiences with MiWaves, the data is inherently limited in richness due to the study's design that focused on feasibility and acceptability metrics. \emph{A more in-depth qualitative approach, such as interviews, could have yielded greater contextual understanding of participants' self-awareness and behavioral changes. However, given the constraints of our study, we focus on presenting these exploratory findings as a foundation for future research, highlighting opportunities to deepen these insights through more targeted qualitative investigations}.

Our findings suggest that the MiWaves app features such as self-monitoring check-ins and trend visualizations may have fostered self-awareness and prompted behavioral reflection. Participants generally found the app easy to use and integrate into daily routines, with the timing and frequency of intervention messages being well-received and viewed positively. However, intervention messages with tasks such as typing responses or exploring links, were perceived as more
% burdensome
effortful compared to the ones which required acknowledging or reading messages. Feedback highlighted opportunities to enhance future iterations by incorporating more personalized and contextually relevant message content. The findings of this analysis are instrumental in shaping future iterations of the MiWaves intervention. 
% In addition, they will inform the design of post-test surveys in subsequent studies, ensuring that more comprehensive and targeted qualitative data can be collected.

% \sg{TODO: Brief paper section overview}
% The paper is organized as follows: Section \ref{sec:background} provides background information on the MiWaves app and the design of the MiWaves pilot study. Section \ref{sec:related_work} reviews related work in digital health interventions and qualitative user experience analysis. Section \ref{sec:methodology} outlines the methodology of the analysis presented in this manuscript. Section \ref{sec:findings} presents the findings, focusing on self-awareness, user burden, privacy considerations, and feedback on intervention messages. Section \ref{sec:limitations_future} discusses the limitations of the study and directions for future work. Finally, Section \ref{sec:conclusion} concludes with a summary of key insights and implications for the future of MiWaves.

