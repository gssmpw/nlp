\section{Related Work}
\label{sec:related}



Our notion of coreset is related to the widely considered \emph{strong} coreset____,
which is a subset $S \subseteq P$ satisfying that $\cost(S, C) \in (1 \pm \epsilon) \cost(P, C)$ for \emph{all} center sets $C \subseteq \RR^d$.
The key difference is that ours may not preserve the cost value on $S$ for all $C$,
but it does preserve the approximation ratio.
Moreover, this stronger notion inherently leads to a prohibitively large coreset size of $\exp(\Omega(d))$, even for $k = 1$.\footnote{This lower bound is folklore, but can be easily proved using an $\epsilon$-net on the unit sphere.
}
Our notion is sometimes referred to as \emph{weak} coresets in the literature, and similar notions were also considered in____.