\section{Related Works}
\paragraph{Spatiotemporal GNNs}
Existing spatiotemporal methods often relied on fixed spatial locations, such as intersections \cite{wang2020traffic, yu2017spatio}, which were unsuitable for the dynamic and fluid nature of maritime environments. Consequently, although efforts were made to establish dynamic reference points that were appropriate for the maritime domain \cite{eljabu2021anomaly, liang2022fine}, these points still did not adequately capture the fluid and constantly evolving nature of moving vessels.

\paragraph{Vessel Behavior Anomaly}
The concept of anomalies in AIS tracks refers to behaviors that deviate from what is considered ‘normal’ or expected under typical operational conditions \cite{laxhammar2008anomaly}. There are many studies \cite{lane2010maritime, davenport2008kinematic, liu2024ais} that define anomalous behaviors based on kinematic behaviors,  AIS transmission behaviors, and other supplementary behaviors that occur on the ship. One of the key challenges in defining anomalies is the absence of a universal criterion for what constitutes an anomalous event. Despite the wide range of possible anomalies, this study specifically focuses on \textit{deviation from standard route}. This type of anomaly is one of the most fundamental and frequently observed irregularities in vessel movement, serving as a crucial indicator of potential maritime risks. A vessel straying from its expected trajectory could signal various underlying causes, including adverse weather conditions, mechanical failures, unauthorized maneuvers, or illicit activities. By analyzing deviations from standard routes, we aim to establish a robust framework for detecting navigational anomalies in real-world maritime operations. The rationale behind this selection will be elaborated more in the experimental section, where we detail the statistical techniques used to quantify deviations.

\paragraph{GNN Based Vessel Anomaly Detection}
The objective of vessel anomaly detection is to identify unusual movement patterns, which are often caused by mechanical failures or navigational errors \cite{ribeiro2023ais}. Most existing methods constructed graphs using predefined or fully connected structures \cite{jiang2024stmgf, liu2023model, wolsing2022anomaly, zhang2023vessel}. Nevertheless, these methodologies proved inadequate for capturing meaningful spatiotemporal relationships that are well-suited to the task at hand.