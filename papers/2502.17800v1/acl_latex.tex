% This must be in the first 5 lines to tell arXiv to use pdfLaTeX, which is strongly recommended.
\pdfoutput=1
% In particular, the hyperref package requires pdfLaTeX in order to break URLs across lines.

\documentclass[11pt]{article}

% Change "review" to "final" to generate the final (sometimes called camera-ready) version.
% Change to "preprint" to generate a non-anonymous version with page numbers.
\usepackage[preprint]{acl}

% Standard package includes
\usepackage{times}
\usepackage{latexsym}
\usepackage{amsmath}
\usepackage[absolute,overlay]{textpos}

\usepackage{algorithm}
\usepackage{algorithmic}
\usepackage{rotating}

\usepackage{xcolor}

\definecolor{my_green}{RGB}{0, 176, 80}
\definecolor{my_red}{RGB}{255, 0, 0}
\usepackage{booktabs}
\usepackage{multirow} 

% For proper rendering and hyphenation of words containing Latin characters (including in bib files)
\usepackage[T1]{fontenc}
% For Vietnamese characters
% \usepackage[T5]{fontenc}
% See https://www.latex-project.org/help/documentation/encguide.pdf for other character sets

% This assumes your files are encoded as UTF8
\usepackage[utf8]{inputenc}

% This is not strictly necessary, and may be commented out,
% but it will improve the layout of the manuscript,
% and will typically save some space.
\usepackage{microtype}

% This is also not strictly necessary, and may be commented out.
% However, it will improve the aesthetics of text in
% the typewriter font.
\usepackage{inconsolata}

%Including images in your LaTeX document requires adding
%additional package(s)
\usepackage{graphicx}
\usepackage[most]{tcolorbox}

%%%%% NEW MATH DEFINITIONS %%%%%

\usepackage{amsmath,amsfonts,bm}
\usepackage{derivative}
% Mark sections of captions for referring to divisions of figures
\newcommand{\figleft}{{\em (Left)}}
\newcommand{\figcenter}{{\em (Center)}}
\newcommand{\figright}{{\em (Right)}}
\newcommand{\figtop}{{\em (Top)}}
\newcommand{\figbottom}{{\em (Bottom)}}
\newcommand{\captiona}{{\em (a)}}
\newcommand{\captionb}{{\em (b)}}
\newcommand{\captionc}{{\em (c)}}
\newcommand{\captiond}{{\em (d)}}

% Highlight a newly defined term
\newcommand{\newterm}[1]{{\bf #1}}

% Derivative d 
\newcommand{\deriv}{{\mathrm{d}}}

% Figure reference, lower-case.
\def\figref#1{figure~\ref{#1}}
% Figure reference, capital. For start of sentence
\def\Figref#1{Figure~\ref{#1}}
\def\twofigref#1#2{figures \ref{#1} and \ref{#2}}
\def\quadfigref#1#2#3#4{figures \ref{#1}, \ref{#2}, \ref{#3} and \ref{#4}}
% Section reference, lower-case.
\def\secref#1{section~\ref{#1}}
% Section reference, capital.
\def\Secref#1{Section~\ref{#1}}
% Reference to two sections.
\def\twosecrefs#1#2{sections \ref{#1} and \ref{#2}}
% Reference to three sections.
\def\secrefs#1#2#3{sections \ref{#1}, \ref{#2} and \ref{#3}}
% Reference to an equation, lower-case.
\def\eqref#1{equation~\ref{#1}}
% Reference to an equation, upper case
\def\Eqref#1{Equation~\ref{#1}}
% A raw reference to an equation---avoid using if possible
\def\plaineqref#1{\ref{#1}}
% Reference to a chapter, lower-case.
\def\chapref#1{chapter~\ref{#1}}
% Reference to an equation, upper case.
\def\Chapref#1{Chapter~\ref{#1}}
% Reference to a range of chapters
\def\rangechapref#1#2{chapters\ref{#1}--\ref{#2}}
% Reference to an algorithm, lower-case.
\def\algref#1{algorithm~\ref{#1}}
% Reference to an algorithm, upper case.
\def\Algref#1{Algorithm~\ref{#1}}
\def\twoalgref#1#2{algorithms \ref{#1} and \ref{#2}}
\def\Twoalgref#1#2{Algorithms \ref{#1} and \ref{#2}}
% Reference to a part, lower case
\def\partref#1{part~\ref{#1}}
% Reference to a part, upper case
\def\Partref#1{Part~\ref{#1}}
\def\twopartref#1#2{parts \ref{#1} and \ref{#2}}

\def\ceil#1{\lceil #1 \rceil}
\def\floor#1{\lfloor #1 \rfloor}
\def\1{\bm{1}}
\newcommand{\train}{\mathcal{D}}
\newcommand{\valid}{\mathcal{D_{\mathrm{valid}}}}
\newcommand{\test}{\mathcal{D_{\mathrm{test}}}}

\def\eps{{\epsilon}}


% Random variables
\def\reta{{\textnormal{$\eta$}}}
\def\ra{{\textnormal{a}}}
\def\rb{{\textnormal{b}}}
\def\rc{{\textnormal{c}}}
\def\rd{{\textnormal{d}}}
\def\re{{\textnormal{e}}}
\def\rf{{\textnormal{f}}}
\def\rg{{\textnormal{g}}}
\def\rh{{\textnormal{h}}}
\def\ri{{\textnormal{i}}}
\def\rj{{\textnormal{j}}}
\def\rk{{\textnormal{k}}}
\def\rl{{\textnormal{l}}}
% rm is already a command, just don't name any random variables m
\def\rn{{\textnormal{n}}}
\def\ro{{\textnormal{o}}}
\def\rp{{\textnormal{p}}}
\def\rq{{\textnormal{q}}}
\def\rr{{\textnormal{r}}}
\def\rs{{\textnormal{s}}}
\def\rt{{\textnormal{t}}}
\def\ru{{\textnormal{u}}}
\def\rv{{\textnormal{v}}}
\def\rw{{\textnormal{w}}}
\def\rx{{\textnormal{x}}}
\def\ry{{\textnormal{y}}}
\def\rz{{\textnormal{z}}}

% Random vectors
\def\rvepsilon{{\mathbf{\epsilon}}}
\def\rvphi{{\mathbf{\phi}}}
\def\rvtheta{{\mathbf{\theta}}}
\def\rva{{\mathbf{a}}}
\def\rvb{{\mathbf{b}}}
\def\rvc{{\mathbf{c}}}
\def\rvd{{\mathbf{d}}}
\def\rve{{\mathbf{e}}}
\def\rvf{{\mathbf{f}}}
\def\rvg{{\mathbf{g}}}
\def\rvh{{\mathbf{h}}}
\def\rvu{{\mathbf{i}}}
\def\rvj{{\mathbf{j}}}
\def\rvk{{\mathbf{k}}}
\def\rvl{{\mathbf{l}}}
\def\rvm{{\mathbf{m}}}
\def\rvn{{\mathbf{n}}}
\def\rvo{{\mathbf{o}}}
\def\rvp{{\mathbf{p}}}
\def\rvq{{\mathbf{q}}}
\def\rvr{{\mathbf{r}}}
\def\rvs{{\mathbf{s}}}
\def\rvt{{\mathbf{t}}}
\def\rvu{{\mathbf{u}}}
\def\rvv{{\mathbf{v}}}
\def\rvw{{\mathbf{w}}}
\def\rvx{{\mathbf{x}}}
\def\rvy{{\mathbf{y}}}
\def\rvz{{\mathbf{z}}}

% Elements of random vectors
\def\erva{{\textnormal{a}}}
\def\ervb{{\textnormal{b}}}
\def\ervc{{\textnormal{c}}}
\def\ervd{{\textnormal{d}}}
\def\erve{{\textnormal{e}}}
\def\ervf{{\textnormal{f}}}
\def\ervg{{\textnormal{g}}}
\def\ervh{{\textnormal{h}}}
\def\ervi{{\textnormal{i}}}
\def\ervj{{\textnormal{j}}}
\def\ervk{{\textnormal{k}}}
\def\ervl{{\textnormal{l}}}
\def\ervm{{\textnormal{m}}}
\def\ervn{{\textnormal{n}}}
\def\ervo{{\textnormal{o}}}
\def\ervp{{\textnormal{p}}}
\def\ervq{{\textnormal{q}}}
\def\ervr{{\textnormal{r}}}
\def\ervs{{\textnormal{s}}}
\def\ervt{{\textnormal{t}}}
\def\ervu{{\textnormal{u}}}
\def\ervv{{\textnormal{v}}}
\def\ervw{{\textnormal{w}}}
\def\ervx{{\textnormal{x}}}
\def\ervy{{\textnormal{y}}}
\def\ervz{{\textnormal{z}}}

% Random matrices
\def\rmA{{\mathbf{A}}}
\def\rmB{{\mathbf{B}}}
\def\rmC{{\mathbf{C}}}
\def\rmD{{\mathbf{D}}}
\def\rmE{{\mathbf{E}}}
\def\rmF{{\mathbf{F}}}
\def\rmG{{\mathbf{G}}}
\def\rmH{{\mathbf{H}}}
\def\rmI{{\mathbf{I}}}
\def\rmJ{{\mathbf{J}}}
\def\rmK{{\mathbf{K}}}
\def\rmL{{\mathbf{L}}}
\def\rmM{{\mathbf{M}}}
\def\rmN{{\mathbf{N}}}
\def\rmO{{\mathbf{O}}}
\def\rmP{{\mathbf{P}}}
\def\rmQ{{\mathbf{Q}}}
\def\rmR{{\mathbf{R}}}
\def\rmS{{\mathbf{S}}}
\def\rmT{{\mathbf{T}}}
\def\rmU{{\mathbf{U}}}
\def\rmV{{\mathbf{V}}}
\def\rmW{{\mathbf{W}}}
\def\rmX{{\mathbf{X}}}
\def\rmY{{\mathbf{Y}}}
\def\rmZ{{\mathbf{Z}}}

% Elements of random matrices
\def\ermA{{\textnormal{A}}}
\def\ermB{{\textnormal{B}}}
\def\ermC{{\textnormal{C}}}
\def\ermD{{\textnormal{D}}}
\def\ermE{{\textnormal{E}}}
\def\ermF{{\textnormal{F}}}
\def\ermG{{\textnormal{G}}}
\def\ermH{{\textnormal{H}}}
\def\ermI{{\textnormal{I}}}
\def\ermJ{{\textnormal{J}}}
\def\ermK{{\textnormal{K}}}
\def\ermL{{\textnormal{L}}}
\def\ermM{{\textnormal{M}}}
\def\ermN{{\textnormal{N}}}
\def\ermO{{\textnormal{O}}}
\def\ermP{{\textnormal{P}}}
\def\ermQ{{\textnormal{Q}}}
\def\ermR{{\textnormal{R}}}
\def\ermS{{\textnormal{S}}}
\def\ermT{{\textnormal{T}}}
\def\ermU{{\textnormal{U}}}
\def\ermV{{\textnormal{V}}}
\def\ermW{{\textnormal{W}}}
\def\ermX{{\textnormal{X}}}
\def\ermY{{\textnormal{Y}}}
\def\ermZ{{\textnormal{Z}}}

% Vectors
\def\vzero{{\bm{0}}}
\def\vone{{\bm{1}}}
\def\vmu{{\bm{\mu}}}
\def\vtheta{{\bm{\theta}}}
\def\vphi{{\bm{\phi}}}
\def\va{{\bm{a}}}
\def\vb{{\bm{b}}}
\def\vc{{\bm{c}}}
\def\vd{{\bm{d}}}
\def\ve{{\bm{e}}}
\def\vf{{\bm{f}}}
\def\vg{{\bm{g}}}
\def\vh{{\bm{h}}}
\def\vi{{\bm{i}}}
\def\vj{{\bm{j}}}
\def\vk{{\bm{k}}}
\def\vl{{\bm{l}}}
\def\vm{{\bm{m}}}
\def\vn{{\bm{n}}}
\def\vo{{\bm{o}}}
\def\vp{{\bm{p}}}
\def\vq{{\bm{q}}}
\def\vr{{\bm{r}}}
\def\vs{{\bm{s}}}
\def\vt{{\bm{t}}}
\def\vu{{\bm{u}}}
\def\vv{{\bm{v}}}
\def\vw{{\bm{w}}}
\def\vx{{\bm{x}}}
\def\vy{{\bm{y}}}
\def\vz{{\bm{z}}}

% Elements of vectors
\def\evalpha{{\alpha}}
\def\evbeta{{\beta}}
\def\evepsilon{{\epsilon}}
\def\evlambda{{\lambda}}
\def\evomega{{\omega}}
\def\evmu{{\mu}}
\def\evpsi{{\psi}}
\def\evsigma{{\sigma}}
\def\evtheta{{\theta}}
\def\eva{{a}}
\def\evb{{b}}
\def\evc{{c}}
\def\evd{{d}}
\def\eve{{e}}
\def\evf{{f}}
\def\evg{{g}}
\def\evh{{h}}
\def\evi{{i}}
\def\evj{{j}}
\def\evk{{k}}
\def\evl{{l}}
\def\evm{{m}}
\def\evn{{n}}
\def\evo{{o}}
\def\evp{{p}}
\def\evq{{q}}
\def\evr{{r}}
\def\evs{{s}}
\def\evt{{t}}
\def\evu{{u}}
\def\evv{{v}}
\def\evw{{w}}
\def\evx{{x}}
\def\evy{{y}}
\def\evz{{z}}

% Matrix
\def\mA{{\bm{A}}}
\def\mB{{\bm{B}}}
\def\mC{{\bm{C}}}
\def\mD{{\bm{D}}}
\def\mE{{\bm{E}}}
\def\mF{{\bm{F}}}
\def\mG{{\bm{G}}}
\def\mH{{\bm{H}}}
\def\mI{{\bm{I}}}
\def\mJ{{\bm{J}}}
\def\mK{{\bm{K}}}
\def\mL{{\bm{L}}}
\def\mM{{\bm{M}}}
\def\mN{{\bm{N}}}
\def\mO{{\bm{O}}}
\def\mP{{\bm{P}}}
\def\mQ{{\bm{Q}}}
\def\mR{{\bm{R}}}
\def\mS{{\bm{S}}}
\def\mT{{\bm{T}}}
\def\mU{{\bm{U}}}
\def\mV{{\bm{V}}}
\def\mW{{\bm{W}}}
\def\mX{{\bm{X}}}
\def\mY{{\bm{Y}}}
\def\mZ{{\bm{Z}}}
\def\mBeta{{\bm{\beta}}}
\def\mPhi{{\bm{\Phi}}}
\def\mLambda{{\bm{\Lambda}}}
\def\mSigma{{\bm{\Sigma}}}

% Tensor
\DeclareMathAlphabet{\mathsfit}{\encodingdefault}{\sfdefault}{m}{sl}
\SetMathAlphabet{\mathsfit}{bold}{\encodingdefault}{\sfdefault}{bx}{n}
\newcommand{\tens}[1]{\bm{\mathsfit{#1}}}
\def\tA{{\tens{A}}}
\def\tB{{\tens{B}}}
\def\tC{{\tens{C}}}
\def\tD{{\tens{D}}}
\def\tE{{\tens{E}}}
\def\tF{{\tens{F}}}
\def\tG{{\tens{G}}}
\def\tH{{\tens{H}}}
\def\tI{{\tens{I}}}
\def\tJ{{\tens{J}}}
\def\tK{{\tens{K}}}
\def\tL{{\tens{L}}}
\def\tM{{\tens{M}}}
\def\tN{{\tens{N}}}
\def\tO{{\tens{O}}}
\def\tP{{\tens{P}}}
\def\tQ{{\tens{Q}}}
\def\tR{{\tens{R}}}
\def\tS{{\tens{S}}}
\def\tT{{\tens{T}}}
\def\tU{{\tens{U}}}
\def\tV{{\tens{V}}}
\def\tW{{\tens{W}}}
\def\tX{{\tens{X}}}
\def\tY{{\tens{Y}}}
\def\tZ{{\tens{Z}}}


% Graph
\def\gA{{\mathcal{A}}}
\def\gB{{\mathcal{B}}}
\def\gC{{\mathcal{C}}}
\def\gD{{\mathcal{D}}}
\def\gE{{\mathcal{E}}}
\def\gF{{\mathcal{F}}}
\def\gG{{\mathcal{G}}}
\def\gH{{\mathcal{H}}}
\def\gI{{\mathcal{I}}}
\def\gJ{{\mathcal{J}}}
\def\gK{{\mathcal{K}}}
\def\gL{{\mathcal{L}}}
\def\gM{{\mathcal{M}}}
\def\gN{{\mathcal{N}}}
\def\gO{{\mathcal{O}}}
\def\gP{{\mathcal{P}}}
\def\gQ{{\mathcal{Q}}}
\def\gR{{\mathcal{R}}}
\def\gS{{\mathcal{S}}}
\def\gT{{\mathcal{T}}}
\def\gU{{\mathcal{U}}}
\def\gV{{\mathcal{V}}}
\def\gW{{\mathcal{W}}}
\def\gX{{\mathcal{X}}}
\def\gY{{\mathcal{Y}}}
\def\gZ{{\mathcal{Z}}}

% Sets
\def\sA{{\mathbb{A}}}
\def\sB{{\mathbb{B}}}
\def\sC{{\mathbb{C}}}
\def\sD{{\mathbb{D}}}
% Don't use a set called E, because this would be the same as our symbol
% for expectation.
\def\sF{{\mathbb{F}}}
\def\sG{{\mathbb{G}}}
\def\sH{{\mathbb{H}}}
\def\sI{{\mathbb{I}}}
\def\sJ{{\mathbb{J}}}
\def\sK{{\mathbb{K}}}
\def\sL{{\mathbb{L}}}
\def\sM{{\mathbb{M}}}
\def\sN{{\mathbb{N}}}
\def\sO{{\mathbb{O}}}
\def\sP{{\mathbb{P}}}
\def\sQ{{\mathbb{Q}}}
\def\sR{{\mathbb{R}}}
\def\sS{{\mathbb{S}}}
\def\sT{{\mathbb{T}}}
\def\sU{{\mathbb{U}}}
\def\sV{{\mathbb{V}}}
\def\sW{{\mathbb{W}}}
\def\sX{{\mathbb{X}}}
\def\sY{{\mathbb{Y}}}
\def\sZ{{\mathbb{Z}}}

% Entries of a matrix
\def\emLambda{{\Lambda}}
\def\emA{{A}}
\def\emB{{B}}
\def\emC{{C}}
\def\emD{{D}}
\def\emE{{E}}
\def\emF{{F}}
\def\emG{{G}}
\def\emH{{H}}
\def\emI{{I}}
\def\emJ{{J}}
\def\emK{{K}}
\def\emL{{L}}
\def\emM{{M}}
\def\emN{{N}}
\def\emO{{O}}
\def\emP{{P}}
\def\emQ{{Q}}
\def\emR{{R}}
\def\emS{{S}}
\def\emT{{T}}
\def\emU{{U}}
\def\emV{{V}}
\def\emW{{W}}
\def\emX{{X}}
\def\emY{{Y}}
\def\emZ{{Z}}
\def\emSigma{{\Sigma}}

% entries of a tensor
% Same font as tensor, without \bm wrapper
\newcommand{\etens}[1]{\mathsfit{#1}}
\def\etLambda{{\etens{\Lambda}}}
\def\etA{{\etens{A}}}
\def\etB{{\etens{B}}}
\def\etC{{\etens{C}}}
\def\etD{{\etens{D}}}
\def\etE{{\etens{E}}}
\def\etF{{\etens{F}}}
\def\etG{{\etens{G}}}
\def\etH{{\etens{H}}}
\def\etI{{\etens{I}}}
\def\etJ{{\etens{J}}}
\def\etK{{\etens{K}}}
\def\etL{{\etens{L}}}
\def\etM{{\etens{M}}}
\def\etN{{\etens{N}}}
\def\etO{{\etens{O}}}
\def\etP{{\etens{P}}}
\def\etQ{{\etens{Q}}}
\def\etR{{\etens{R}}}
\def\etS{{\etens{S}}}
\def\etT{{\etens{T}}}
\def\etU{{\etens{U}}}
\def\etV{{\etens{V}}}
\def\etW{{\etens{W}}}
\def\etX{{\etens{X}}}
\def\etY{{\etens{Y}}}
\def\etZ{{\etens{Z}}}

% The true underlying data generating distribution
\newcommand{\pdata}{p_{\rm{data}}}
\newcommand{\ptarget}{p_{\rm{target}}}
\newcommand{\pprior}{p_{\rm{prior}}}
\newcommand{\pbase}{p_{\rm{base}}}
\newcommand{\pref}{p_{\rm{ref}}}

% The empirical distribution defined by the training set
\newcommand{\ptrain}{\hat{p}_{\rm{data}}}
\newcommand{\Ptrain}{\hat{P}_{\rm{data}}}
% The model distribution
\newcommand{\pmodel}{p_{\rm{model}}}
\newcommand{\Pmodel}{P_{\rm{model}}}
\newcommand{\ptildemodel}{\tilde{p}_{\rm{model}}}
% Stochastic autoencoder distributions
\newcommand{\pencode}{p_{\rm{encoder}}}
\newcommand{\pdecode}{p_{\rm{decoder}}}
\newcommand{\precons}{p_{\rm{reconstruct}}}

\newcommand{\laplace}{\mathrm{Laplace}} % Laplace distribution

\newcommand{\E}{\mathbb{E}}
\newcommand{\Ls}{\mathcal{L}}
\newcommand{\R}{\mathbb{R}}
\newcommand{\emp}{\tilde{p}}
\newcommand{\lr}{\alpha}
\newcommand{\reg}{\lambda}
\newcommand{\rect}{\mathrm{rectifier}}
\newcommand{\softmax}{\mathrm{softmax}}
\newcommand{\sigmoid}{\sigma}
\newcommand{\softplus}{\zeta}
\newcommand{\KL}{D_{\mathrm{KL}}}
\newcommand{\Var}{\mathrm{Var}}
\newcommand{\standarderror}{\mathrm{SE}}
\newcommand{\Cov}{\mathrm{Cov}}
% Wolfram Mathworld says $L^2$ is for function spaces and $\ell^2$ is for vectors
% But then they seem to use $L^2$ for vectors throughout the site, and so does
% wikipedia.
\newcommand{\normlzero}{L^0}
\newcommand{\normlone}{L^1}
\newcommand{\normltwo}{L^2}
\newcommand{\normlp}{L^p}
\newcommand{\normmax}{L^\infty}

\newcommand{\parents}{Pa} % See usage in notation.tex. Chosen to match Daphne's book.

\DeclareMathOperator*{\argmax}{arg\,max}
\DeclareMathOperator*{\argmin}{arg\,min}

\DeclareMathOperator{\sign}{sign}
\DeclareMathOperator{\Tr}{Tr}
\let\ab\allowbreak


\newcommand{\TODO}[1]{{\textcolor{red}{\textbf{TODO: #1}}}}
\newcommand{\ZC}[1]{{\textcolor{blue}{\textbf{zhepeng: #1}}}}

\title{Your Language Model May Think Too Rigidly: Achieving Reasoning Consistency with Symmetry-Enhanced Training}


% Author information can be set in various styles:
% For several authors from the same institution:
% \author{Author 1 \and ... \and Author n \\
%         Address line \\ ... \\ Address line}
% if the names do not fit well on one line use
%         Author 1 \\ {\bf Author 2} \\ ... \\ {\bf Author n} \\
% For authors from different institutions:
% \author{Author 1 \\ Address line \\  ... \\ Address line
%         \And  ... \And
%         Author n \\ Address line \\ ... \\ Address line}
% To start a separate ``row'' of authors use \AND, as in
% \author{Author 1 \\ Address line \\  ... \\ Address line
%         \AND
%         Author 2 \\ Address line \\ ... \\ Address line \And
%         Author 3 \\ Address line \\ ... \\ Address line}

% \author{First Author \\
%   Affiliation / Address line 1 \\
%   Affiliation / Address line 2 \\
%   Affiliation / Address line 3 \\
%   \texttt{email@domain} \\\And
%   Second Author \\
%   Affiliation / Address line 1 \\
%   Affiliation / Address line 2 \\
%   Affiliation / Address line 3 \\
%   \texttt{email@domain} \\}

\author{
 \textbf{Yihang Yao\textsuperscript{1}},
 \textbf{Zhepeng Cen\textsuperscript{1}},
 \textbf{Miao Li\textsuperscript{1}},
 \textbf{William Han\textsuperscript{1}},
 \textbf{Yuyou Zhang\textsuperscript{1}},
 \textbf{Emerson Liu\textsuperscript{2}},
   \\
 \textbf{Zuxin Liu\textsuperscript{3}},
 \textbf{Chuang Gan\textsuperscript{4}},
 \textbf{Ding Zhao\textsuperscript{1}}
\\
 \textsuperscript{1}Carnegie Mellon University,
 \textsuperscript{2}AHN,
 \textsuperscript{3}Salesforce,
 \textsuperscript{4}UMass Amherst
\\
 {
   \{yihangya, zcen\}@andrew.cmu.edu
 }
}

\begin{document}
\maketitle
\begin{abstract}
Large Language Models (LLMs) have demonstrated strong reasoning capabilities across various tasks. However, even minor variations in query phrasing, despite preserving the underlying semantic meaning, can significantly affect their performance. To address this, we focus on enhancing LLMs' awareness of symmetry in query variations and propose \methodlong, a data-centric approach that improves the model's ability to extract useful information from context. Unlike existing methods that emphasize reasoning chain augmentation, our approach improves model robustness at the knowledge extraction stage through query augmentation, enabling more data-efficient training and stronger generalization to Out-of-Distribution (OOD) settings. Extensive experiments on both logical and arithmetic reasoning tasks show that \method\ enhances reasoning performance across diverse query variations, providing new insights into improving LLM robustness through structured dataset curation.
\end{abstract}



\section{Introduction}
\section{Introduction}
\label{sec:introduction}
The business processes of organizations are experiencing ever-increasing complexity due to the large amount of data, high number of users, and high-tech devices involved \cite{martin2021pmopportunitieschallenges, beerepoot2023biggestbpmproblems}. This complexity may cause business processes to deviate from normal control flow due to unforeseen and disruptive anomalies \cite{adams2023proceddsriftdetection}. These control-flow anomalies manifest as unknown, skipped, and wrongly-ordered activities in the traces of event logs monitored from the execution of business processes \cite{ko2023adsystematicreview}. For the sake of clarity, let us consider an illustrative example of such anomalies. Figure \ref{FP_ANOMALIES} shows a so-called event log footprint, which captures the control flow relations of four activities of a hypothetical event log. In particular, this footprint captures the control-flow relations between activities \texttt{a}, \texttt{b}, \texttt{c} and \texttt{d}. These are the causal ($\rightarrow$) relation, concurrent ($\parallel$) relation, and other ($\#$) relations such as exclusivity or non-local dependency \cite{aalst2022pmhandbook}. In addition, on the right are six traces, of which five exhibit skipped, wrongly-ordered and unknown control-flow anomalies. For example, $\langle$\texttt{a b d}$\rangle$ has a skipped activity, which is \texttt{c}. Because of this skipped activity, the control-flow relation \texttt{b}$\,\#\,$\texttt{d} is violated, since \texttt{d} directly follows \texttt{b} in the anomalous trace.
\begin{figure}[!t]
\centering
\includegraphics[width=0.9\columnwidth]{images/FP_ANOMALIES.png}
\caption{An example event log footprint with six traces, of which five exhibit control-flow anomalies.}
\label{FP_ANOMALIES}
\end{figure}

\subsection{Control-flow anomaly detection}
Control-flow anomaly detection techniques aim to characterize the normal control flow from event logs and verify whether these deviations occur in new event logs \cite{ko2023adsystematicreview}. To develop control-flow anomaly detection techniques, \revision{process mining} has seen widespread adoption owing to process discovery and \revision{conformance checking}. On the one hand, process discovery is a set of algorithms that encode control-flow relations as a set of model elements and constraints according to a given modeling formalism \cite{aalst2022pmhandbook}; hereafter, we refer to the Petri net, a widespread modeling formalism. On the other hand, \revision{conformance checking} is an explainable set of algorithms that allows linking any deviations with the reference Petri net and providing the fitness measure, namely a measure of how much the Petri net fits the new event log \cite{aalst2022pmhandbook}. Many control-flow anomaly detection techniques based on \revision{conformance checking} (hereafter, \revision{conformance checking}-based techniques) use the fitness measure to determine whether an event log is anomalous \cite{bezerra2009pmad, bezerra2013adlogspais, myers2018icsadpm, pecchia2020applicationfailuresanalysispm}. 

The scientific literature also includes many \revision{conformance checking}-independent techniques for control-flow anomaly detection that combine specific types of trace encodings with machine/deep learning \cite{ko2023adsystematicreview, tavares2023pmtraceencoding}. Whereas these techniques are very effective, their explainability is challenging due to both the type of trace encoding employed and the machine/deep learning model used \cite{rawal2022trustworthyaiadvances,li2023explainablead}. Hence, in the following, we focus on the shortcomings of \revision{conformance checking}-based techniques to investigate whether it is possible to support the development of competitive control-flow anomaly detection techniques while maintaining the explainable nature of \revision{conformance checking}.
\begin{figure}[!t]
\centering
\includegraphics[width=\columnwidth]{images/HIGH_LEVEL_VIEW.png}
\caption{A high-level view of the proposed framework for combining \revision{process mining}-based feature extraction with dimensionality reduction for control-flow anomaly detection.}
\label{HIGH_LEVEL_VIEW}
\end{figure}

\subsection{Shortcomings of \revision{conformance checking}-based techniques}
Unfortunately, the detection effectiveness of \revision{conformance checking}-based techniques is affected by noisy data and low-quality Petri nets, which may be due to human errors in the modeling process or representational bias of process discovery algorithms \cite{bezerra2013adlogspais, pecchia2020applicationfailuresanalysispm, aalst2016pm}. Specifically, on the one hand, noisy data may introduce infrequent and deceptive control-flow relations that may result in inconsistent fitness measures, whereas, on the other hand, checking event logs against a low-quality Petri net could lead to an unreliable distribution of fitness measures. Nonetheless, such Petri nets can still be used as references to obtain insightful information for \revision{process mining}-based feature extraction, supporting the development of competitive and explainable \revision{conformance checking}-based techniques for control-flow anomaly detection despite the problems above. For example, a few works outline that token-based \revision{conformance checking} can be used for \revision{process mining}-based feature extraction to build tabular data and develop effective \revision{conformance checking}-based techniques for control-flow anomaly detection \cite{singh2022lapmsh, debenedictis2023dtadiiot}. However, to the best of our knowledge, the scientific literature lacks a structured proposal for \revision{process mining}-based feature extraction using the state-of-the-art \revision{conformance checking} variant, namely alignment-based \revision{conformance checking}.

\subsection{Contributions}
We propose a novel \revision{process mining}-based feature extraction approach with alignment-based \revision{conformance checking}. This variant aligns the deviating control flow with a reference Petri net; the resulting alignment can be inspected to extract additional statistics such as the number of times a given activity caused mismatches \cite{aalst2022pmhandbook}. We integrate this approach into a flexible and explainable framework for developing techniques for control-flow anomaly detection. The framework combines \revision{process mining}-based feature extraction and dimensionality reduction to handle high-dimensional feature sets, achieve detection effectiveness, and support explainability. Notably, in addition to our proposed \revision{process mining}-based feature extraction approach, the framework allows employing other approaches, enabling a fair comparison of multiple \revision{conformance checking}-based and \revision{conformance checking}-independent techniques for control-flow anomaly detection. Figure \ref{HIGH_LEVEL_VIEW} shows a high-level view of the framework. Business processes are monitored, and event logs obtained from the database of information systems. Subsequently, \revision{process mining}-based feature extraction is applied to these event logs and tabular data input to dimensionality reduction to identify control-flow anomalies. We apply several \revision{conformance checking}-based and \revision{conformance checking}-independent framework techniques to publicly available datasets, simulated data of a case study from railways, and real-world data of a case study from healthcare. We show that the framework techniques implementing our approach outperform the baseline \revision{conformance checking}-based techniques while maintaining the explainable nature of \revision{conformance checking}.

In summary, the contributions of this paper are as follows.
\begin{itemize}
    \item{
        A novel \revision{process mining}-based feature extraction approach to support the development of competitive and explainable \revision{conformance checking}-based techniques for control-flow anomaly detection.
    }
    \item{
        A flexible and explainable framework for developing techniques for control-flow anomaly detection using \revision{process mining}-based feature extraction and dimensionality reduction.
    }
    \item{
        Application to synthetic and real-world datasets of several \revision{conformance checking}-based and \revision{conformance checking}-independent framework techniques, evaluating their detection effectiveness and explainability.
    }
\end{itemize}

The rest of the paper is organized as follows.
\begin{itemize}
    \item Section \ref{sec:related_work} reviews the existing techniques for control-flow anomaly detection, categorizing them into \revision{conformance checking}-based and \revision{conformance checking}-independent techniques.
    \item Section \ref{sec:abccfe} provides the preliminaries of \revision{process mining} to establish the notation used throughout the paper, and delves into the details of the proposed \revision{process mining}-based feature extraction approach with alignment-based \revision{conformance checking}.
    \item Section \ref{sec:framework} describes the framework for developing \revision{conformance checking}-based and \revision{conformance checking}-independent techniques for control-flow anomaly detection that combine \revision{process mining}-based feature extraction and dimensionality reduction.
    \item Section \ref{sec:evaluation} presents the experiments conducted with multiple framework and baseline techniques using data from publicly available datasets and case studies.
    \item Section \ref{sec:conclusions} draws the conclusions and presents future work.
\end{itemize}

\section{Related Works}
\section{Related Works}
\label{sec:related_works}


\noindent\textbf{Diffusion-based Video Generation. }
The advancement of diffusion models \cite{rombach2022high, ramesh2022hierarchical, zheng2022entropy} has led to significant progress in video generation. Due to the scarcity of high-quality video-text datasets \cite{Blattmann2023, Blattmann2023a}, researchers have adapted existing text-to-image (T2I) models to facilitate text-to-video (T2V) generation. Notable examples include AnimateDiff \cite{Guo2023}, Align your Latents \cite{Blattmann2023a}, PYoCo \cite{ge2023preserve}, and Emu Video \cite{girdhar2023emu}. Further advancements, such as LVDM \cite{he2022latent}, VideoCrafter \cite{chen2023videocrafter1, chen2024videocrafter2}, ModelScope \cite{wang2023modelscope}, LAVIE \cite{wang2023lavie}, and VideoFactory \cite{wang2024videofactory}, have refined these approaches by fine-tuning both spatial and temporal blocks, leveraging T2I models for initialization to improve video quality.
Recently, Sora \cite{brooks2024video} and CogVideoX \cite{yang2024cogvideox} enhance video generation by introducing Transformer-based diffusion backbones \cite{Peebles2023, Ma2024, Yu2024} and utilizing 3D-VAE, unlocking the potential for realistic world simulators. Additionally, SVD \cite{Blattmann2023}, SEINE \cite{chen2023seine}, PixelDance \cite{zeng2024make} and PIA \cite{zhang2024pia} have made significant strides in image-to-video generation, achieving notable improvements in quality and flexibility.
Further, I2VGen-XL \cite{zhang2023i2vgen}, DynamicCrafter \cite{Xing2023}, and Moonshot \cite{zhang2024moonshot} incorporate additional cross-attention layers to strengthen conditional signals during generation.



\noindent\textbf{Controllable Generation.}
Controllable generation has become a central focus in both image \citep{Zhang2023,jiang2024survey, Mou2024, Zheng2023, peng2024controlnext, ye2023ip, wu2024spherediffusion, song2024moma, wu2024ifadapter} and video \citep{gong2024atomovideo, zhang2024moonshot, guo2025sparsectrl, jiang2024videobooth} generation, enabling users to direct the output through various types of control. A wide range of controllable inputs has been explored, including text descriptions, pose \citep{ma2024follow,wang2023disco,hu2024animate,xu2024magicanimate}, audio \citep{tang2023anytoany,tian2024emo,he2024co}, identity representations \citep{chefer2024still,wang2024customvideo,wu2024customcrafter}, trajectory \citep{yin2023dragnuwa,chen2024motion,li2024generative,wu2024motionbooth, namekata2024sg}.


\noindent\textbf{Text-based Camera Control.}
Text-based camera control methods use natural language descriptions to guide camera motion in video generation. AnimateDiff \cite{Guo2023} and SVD \cite{Blattmann2023} fine-tune LoRAs \cite{hu2021lora} for specific camera movements based on text input. 
Image conductor\cite{li2024image} proposed to separate different camera and object motions through camera LoRA weight and object LoRA weight to achieve more precise motion control.
In contrast, MotionMaster \cite{hu2024motionmaster} and Peekaboo \cite{jain2024peekaboo} offer training-free approaches for generating coarse-grained camera motions, though with limited precision. VideoComposer \cite{wang2024videocomposer} adjusts pixel-level motion vectors to provide finer control, but challenges remain in achieving precise camera control.

\noindent\textbf{Trajectory-based Camera Control.}
MotionCtrl \cite{Wang2024Motionctrl}, CameraCtrl \cite{He2024Cameractrl}, and Direct-a-Video \cite{yang2024direct} use camera pose as input to enhance control, while CVD \cite{kuang2024collaborative} extends CameraCtrl for multi-view generation, though still limited by motion complexity. To improve geometric consistency, Pose-guided diffusion \cite{tseng2023consistent}, CamCo \cite{Xu2024}, and CamI2V \cite{zheng2024cami2v} apply epipolar constraints for consistent viewpoints. VD3D \cite{bahmani2024vd3d} introduces a ControlNet\cite{Zhang2023}-like conditioning mechanism with spatiotemporal camera embeddings, enabling more precise control.
CamTrol \cite{hou2024training} offers a training-free approach that renders static point clouds into multi-view frames for video generation. Cavia \cite{xu2024cavia} introduces view-integrated attention mechanisms to improve viewpoint and temporal consistency, while I2VControl-Camera \cite{feng2024i2vcontrol} refines camera movement by employing point trajectories in the camera coordinate system. Despite these advancements, challenges in maintaining camera control and scene-scale consistency remain, which our method seeks to address. It is noted that 4Dim~\cite{watson2024controlling} introduces absolute scale but in  4D novel view synthesis (NVS) of scenes.




\section{Preliminary for LLM Reasoning}
\section{Viewer-provider two-sided systems}

This section models the dynamics of viewer and provider populations on a recommendation platform. 
Specifically, we consider sub-group dynamics where viewers and providers are categorized into $K$ and $L$ subgroups\footnote{We can consider a ``subgroup'' of size 1. In such cases, the viewer ``population'' corresponds to the time spent by an individual viewer, while the provider ``population'' can be the amount of content produced by an individual provider.
}. Then, we model the populations, recommendation policy, payoffs, and social welfare as follows.

\begin{enumerate}[leftmargin=12pt]
    \item (Viewer/provider population)  
    Let $\lambda_{k} \in \mathbb{R}_{\geq 0}$ be the population of the viewer group $k \in [K]$ and $\lambda_{l}$ be that of the provider group $l \in [L]$. Also let $\blambda := (\lambda_{1}, \lambda_{2}, \cdots, \lambda_{K},
    \lambda_{1}, \lambda_{2}, \cdots, \lambda_{L})$ be the joint population vector of viewers and providers.
    \item (Platform's recommendation policy) 
    The platform matches each viewer group $k$ to a provider group $l$ with a recommendation policy denoted by a $K$-by-$L$ matrix $\bpi$. Specifically, its $(k,l)$-th element $\pi_{k,l}$ represents the probability of allocating the provider group $l$ to the viewer group $k$. 
    Thus $\sum_{l=1}^L \pi_{k,l} = 1, \forall k \in [K]$. For example, the uniform random policy, which assigns equal exposure probability across all provider groups is represented as given by $\bpi=\frac{1}{L}\1_{L\times K}$.
    \item (Viewer/provider payoffs) After viewer and provider groups are matched by the policy $\bpi$, their perceived payoffs can be quantified by the following metrics:
    \begin{align}\label{eq:user_satisfaction}
    \text{Viewer Satisfaction: \quad } & s_k = \textstyle \sum_{l=1}^L \pi_{k,l} q_{k,l} \,  , \\\label{eq:content_exposure}
    \text{Provider Exposure: \quad} & e_l = \textstyle\sum_{k=1}^K \pi_{k,l}\lambda_k,
    \end{align}
    where $q_{k,l}$ is the (expected) utility that viewers $k$ receive from the provider groups $l$. Eqs.~\eqref{eq:user_satisfaction} and~\eqref{eq:content_exposure} define viewer satisfaction as determined by the total utility they receive from recommendations, while providers care about the total amount of exposure they receive by recommendation. This model is prevalent is prior works including \citep{singh2018fairness, mladenov2020optimizing}.
    \item (Social welfare) Finally, we consider the following total viewer welfare as the global metric of the platform:
    \begin{align*}
        R(\bpi; \blambda) := \textstyle\sum_{k=1}^{K} \lambda_{k} s_k
    \end{align*}
    Note that here we consider the sum of viewer-side satisfaction as the social welfare, a formulation prevalent in related works~\citep{mladenov2020optimizing, huttenlocher2023matching}.
    The sum of content-side exposure simplifies to the total size of the viewer population.
\end{enumerate}

\subsection{Interaction dynamics and ``population effects''}\label{sec:dynamic_formulation}

We have so far seen a typical formulation in two-sided platforms. However, a key limitation of such formulation is to ignore potential non-stationarity in the viewer and provider populations, which is common in many real-world two-sided systems~\citep{boutilier2023modeling,  deffayet2024sardine}. 

First, consider the impact of provider population growth on the utility experience by viewers, which we call \textit{``population effects''}.
An increase in provider population naturally leads to more high-quality content. 
For example, consider a two-stage recommendation policy, where our higher-level policy $\bpi$ decides the matching between viewer and provider groups, and a second-stage policy selects individual providers among the selected group. 
Any reasonable second stage policy should be able to select a better provider from a larger provider pool~\citep{su2023value, evnine2024achieving}. 
To model such ``population effects'', we introduce the following utility decomposition:
\begin{align}
    q_{k,l} = b_{k,l} + f_{k,l}(\lambda_{l}) \label{eq:reward_decomposition}
\end{align}
where $b_{k,l}$ is the \textit{base} utility, which may indicates the matching between the preference of viewer and provider groups (e.g., this viewer group likes sports articles). In contrast, $f_{k,l}(\cdot)$ represents the quality of the provider which improves as the provider population increases. $f_{k,l}$ might be heterogeneous among different viewer and provider groups because quality might be multi-dimensional (e.g., visuals, intellects, novelty), viewers may have different preferences, and providers may have different abilities. 
We take $f_{k,l}$ to be a monotonically increasing function.

Next, consider the impact of viewer and provider payoffs on the population.
The number of viewers that a platform can maintain is related to the level of satisfaction, similarly the number of providers is related to the exposure.
We assume that viewer and provider subgroups have 
some \textit{``reference''} population $\bar{\lambda}_{k}(s_{k})$ and $\bar{\lambda}_{l}(e_{l})$ given the level of viewer satisfaction $s_k$ and provider exposure $e_l$. We assume that $\bar{\lambda}$ is a monotonically increasing function, so higher viewer satisfaction and provider exposure result in increased populations. 
Based on this, we model the viewer and provider population dynamics as:
\begin{align}
    \text{Viewer: \,}  \lambda_{t+1,k} = (1 - \eta_k) \lambda_{t,k} + \eta_k \bar{\lambda}_{k}(s_{t,k}), \label{eq:user_dynamics} \\
    \text{Content: \,}  \lambda_{t+1,l} = (1 - \eta_l) \lambda_{t,l} + \eta_l \bar{\lambda}_{l}(e_{t,l}), \label{eq:content_dynamics}
\end{align}
where $\eta \in [0, 1]$ are the \textit{reactiveness} hyperparams, determining how fast the population changes. Note that similar models are widely adopted in performative predictions~\citep{perdomo2020performative, brown2022performative}. 
We thus have that the viewer satisfaction $s_k$ depends on the provider population via ``population effects'' $f_{k,l}$, while the provider exposure directly depends on the viewer population.
The two-sided platform has complex dynamics between viewers and providers. 
Our goal will be to consider long-term objectives under such co-evolving and two-sided dynamics.

\subsection{Game-theoretic interpretation}\label{sec:game_formulation}

Next, we provide a further justification of and insight into the dynamics model by introducing a game-theoretic formulation that is equivalent to Eqs. \eqref{eq:user_dynamics} and \eqref{eq:content_dynamics}.

Consider a $(K+L)$-player game involving $K$ viewer groups and $L$ provider groups. Each viewer group selects a pure strategy $\lambda_k \in \RR_{\geq 0}$, and each provider group chooses a pure strategy $\lambda_l \in \RR_{\geq 0}$. The utility functions for the viewer and provider groups, denoted by $\{u_k\}_{k=1}^K$ and $\{v_l\}_{l=1}^L$ are defined as follows:
\begin{align}\label{eq:util_user}
    & u_k(\blambda)= \lambda_k \cdot \bar{\lambda}_k \left(\sum_{l=1}^L \pi_{k,l}\left(b_{k,l}+f_{k,l}(\lambda_l)\right)\right)-\frac{\lambda_k^2}{2}, \\ \label{eq:util_creator}
    & v_l(\blambda)= \lambda_l\cdot \bar{\lambda}_l \left(\textstyle\sum_{k=1}^K \pi_{k,l}\lambda_k\right)-\frac{\lambda_l^2}{2},
\end{align}
We denote this game as $\G(\bpi, B, f, \bar{\lambda})$, where $B$ is a $K$-by-$L$ matrix whose $(k,l)$-element is $b_{k,l}$. Proposition \ref{prop:dynamics_equivalence} establishes a connection between the game instance $\G$ and the 
formulation presented in Section \ref{sec:dynamic_formulation}.

\begin{proposition}\label{prop:dynamics_equivalence}
    If all players in $\G$ apply gradient ascent to optimize their utility functions with learning rates $\{\eta_k\}_{k=1}^K$ and $\{\eta_l\}_{l=1}^L$, the resulting joint strategy evolving dynamics are exactly given by Eqs.~\eqref{eq:user_dynamics} and \eqref{eq:content_dynamics}.
\end{proposition}

Through Proposition \ref{prop:dynamics_equivalence}, our game-theoretic formulation provides a first-principles perspective for understanding the dynamical formulation in Eqs.~\eqref{eq:user_dynamics} and \eqref{eq:content_dynamics}.\footnote{The game $\G$ resembles the Cournot Duopoly competition \cite{cournot1838recherches}. When $K = L = 1$ and $\bar{\lambda}(\mu) = a - b\mu$ and $\bar{\mu}(\lambda) = a - b\lambda$ for some positive constants $a$ and $b$, the game $\G$ corresponds exactly to the Cournot Duopoly model. The key distinction in ours is that $\bar{\mu}$ and $\bar{\lambda}$ are generic increasing functions.} 
That is, 
we can interpret $\bar{\lambda}(\cdot)$ as the marginal gain from increasing the size of a viewer or provider group by one unit. Consequently, the first terms $\lambda_k \cdot \bar{\lambda}_k(\cdot)$ and $\lambda_l \cdot \bar{\lambda}_l(\cdot)$ represent the collective payoffs for viewer and provider groups of sizes $\lambda_k$ and $\lambda_l$. 
The quadratic terms $-\frac{\lambda_k^2}{2}$ and $-\frac{\lambda_l^2}{2}$ capture the congestion costs associated with maintaining larger populations (e.g., if a provider group becomes too large, providers within the group may face intensified competition and thus reduce their productivity due to diminished marginal gains). This suggests that Eqs.~\eqref{eq:user_dynamics} and \eqref{eq:content_dynamics} are quite reasonable formulation to capture real-world interactions.

\section{MEND Data Augmentation}
\section{Method}\label{sec:method}
\begin{figure}
    \centering
    \includegraphics[width=0.85\textwidth]{imgs/heatmap_acc.pdf}
    \caption{\textbf{Visualization of the proposed periodic Bayesian flow with mean parameter $\mu$ and accumulated accuracy parameter $c$ which corresponds to the entropy/uncertainty}. For $x = 0.3, \beta(1) = 1000$ and $\alpha_i$ defined in \cref{appd:bfn_cir}, this figure plots three colored stochastic parameter trajectories for receiver mean parameter $m$ and accumulated accuracy parameter $c$, superimposed on a log-scale heatmap of the Bayesian flow distribution $p_F(m|x,\senderacc)$ and $p_F(c|x,\senderacc)$. Note the \emph{non-monotonicity} and \emph{non-additive} property of $c$ which could inform the network the entropy of the mean parameter $m$ as a condition and the \emph{periodicity} of $m$. %\jj{Shrink the figures to save space}\hanlin{Do we need to make this figure one-column?}
    }
    \label{fig:vmbf_vis}
    \vskip -0.1in
\end{figure}
% \begin{wrapfigure}{r}{0.5\textwidth}
%     \centering
%     \includegraphics[width=0.49\textwidth]{imgs/heatmap_acc.pdf}
%     \caption{\textbf{Visualization of hyper-torus Bayesian flow based on von Mises Distribution}. For $x = 0.3, \beta(1) = 1000$ and $\alpha_i$ defined in \cref{appd:bfn_cir}, this figure plots three colored stochastic parameter trajectories for receiver mean parameter $m$ and accumulated accuracy parameter $c$, superimposed on a log-scale heatmap of the Bayesian flow distribution $p_F(m|x,\senderacc)$ and $p_F(c|x,\senderacc)$. Note the \emph{non-monotonicity} and \emph{non-additive} property of $c$. \jj{Shrink the figures to save space}}
%     \label{fig:vmbf_vis}
%     \vspace{-30pt}
% \end{wrapfigure}


In this section, we explain the detailed design of CrysBFN tackling theoretical and practical challenges. First, we describe how to derive our new formulation of Bayesian Flow Networks over hyper-torus $\mathbb{T}^{D}$ from scratch. Next, we illustrate the two key differences between \modelname and the original form of BFN: $1)$ a meticulously designed novel base distribution with different Bayesian update rules; and $2)$ different properties over the accuracy scheduling resulted from the periodicity and the new Bayesian update rules. Then, we present in detail the overall framework of \modelname over each manifold of the crystal space (\textit{i.e.} fractional coordinates, lattice vectors, atom types) respecting \textit{periodic E(3) invariance}. 

% In this section, we first demonstrate how to build Bayesian flow on hyper-torus $\mathbb{T}^{D}$ by overcoming theoretical and practical problems to provide a low-noise parameter-space approach to fractional atom coordinate generation. Next, we present how \modelname models each manifold of crystal space respecting \textit{periodic E(3) invariance}. 

\subsection{Periodic Bayesian Flow on Hyper-torus \texorpdfstring{$\mathbb{T}^{D}$}{}} 
For generative modeling of fractional coordinates in crystal, we first construct a periodic Bayesian flow on \texorpdfstring{$\mathbb{T}^{D}$}{} by designing every component of the totally new Bayesian update process which we demonstrate to be distinct from the original Bayesian flow (please see \cref{fig:non_add}). 
 %:) 
 
 The fractional atom coordinate system \citep{jiao2023crystal} inherently distributes over a hyper-torus support $\mathbb{T}^{3\times N}$. Hence, the normal distribution support on $\R$ used in the original \citep{bfn} is not suitable for this scenario. 
% The key problem of generative modeling for crystal is the periodicity of Cartesian atom coordinates $\vX$ requiring:
% \begin{equation}\label{eq:periodcity}
% p(\vA,\vL,\vX)=p(\vA,\vL,\vX+\vec{LK}),\text{where}~\vec{K}=\vec{k}\vec{1}_{1\times N},\forall\vec{k}\in\mathbb{Z}^{3\times1}
% \end{equation}
% However, there does not exist such a distribution supporting on $\R$ to model such property because the integration of such distribution over $\R$ will not be finite and equal to 1. Therefore, the normal distribution used in \citet{bfn} can not meet this condition.

To tackle this problem, the circular distribution~\citep{mardia2009directional} over the finite interval $[-\pi,\pi)$ is a natural choice as the base distribution for deriving the BFN on $\mathbb{T}^D$. 
% one natural choice is to 
% we would like to consider the circular distribution over the finite interval as the base 
% we find that circular distributions \citep{mardia2009directional} defined on a finite interval with lengths of $2\pi$ can be used as the instantiation of input distribution for the BFN on $\mathbb{T}^D$.
Specifically, circular distributions enjoy desirable periodic properties: $1)$ the integration over any interval length of $2\pi$ equals 1; $2)$ the probability distribution function is periodic with period $2\pi$.  Sharing the same intrinsic with fractional coordinates, such periodic property of circular distribution makes it suitable for the instantiation of BFN's input distribution, in parameterizing the belief towards ground truth $\x$ on $\mathbb{T}^D$. 
% \yuxuan{this is very complicated from my perspective.} \hanlin{But this property is exactly beautiful and perfectly fit into the BFN.}

\textbf{von Mises Distribution and its Bayesian Update} We choose von Mises distribution \citep{mardia2009directional} from various circular distributions as the form of input distribution, based on the appealing conjugacy property required in the derivation of the BFN framework.
% to leverage the Bayesian conjugacy property of von Mises distribution which is required by the BFN framework. 
That is, the posterior of a von Mises distribution parameterized likelihood is still in the family of von Mises distributions. The probability density function of von Mises distribution with mean direction parameter $m$ and concentration parameter $c$ (describing the entropy/uncertainty of $m$) is defined as: 
\begin{equation}
f(x|m,c)=vM(x|m,c)=\frac{\exp(c\cos(x-m))}{2\pi I_0(c)}
\end{equation}
where $I_0(c)$ is zeroth order modified Bessel function of the first kind as the normalizing constant. Given the last univariate belief parameterized by von Mises distribution with parameter $\theta_{i-1}=\{m_{i-1},\ c_{i-1}\}$ and the sample $y$ from sender distribution with unknown data sample $x$ and known accuracy $\alpha$ describing the entropy/uncertainty of $y$,  Bayesian update for the receiver is deducted as:
\begin{equation}
 h(\{m_{i-1},c_{i-1}\},y,\alpha)=\{m_i,c_i \}, \text{where}
\end{equation}
\begin{equation}\label{eq:h_m}
m_i=\text{atan2}(\alpha\sin y+c_{i-1}\sin m_{i-1}, {\alpha\cos y+c_{i-1}\cos m_{i-1}})
\end{equation}
\begin{equation}\label{eq:h_c}
c_i =\sqrt{\alpha^2+c_{i-1}^2+2\alpha c_{i-1}\cos(y-m_{i-1})}
\end{equation}
The proof of the above equations can be found in \cref{apdx:bayesian_update_function}. The atan2 function refers to  2-argument arctangent. Independently conducting  Bayesian update for each dimension, we can obtain the Bayesian update distribution by marginalizing $\y$:
\begin{equation}
p_U(\vtheta'|\vtheta,\bold{x};\alpha)=\mathbb{E}_{p_S(\bold{y}|\bold{x};\alpha)}\delta(\vtheta'-h(\vtheta,\bold{y},\alpha))=\mathbb{E}_{vM(\bold{y}|\bold{x},\alpha)}\delta(\vtheta'-h(\vtheta,\bold{y},\alpha))
\end{equation} 
\begin{figure}
    \centering
    \vskip -0.15in
    \includegraphics[width=0.95\linewidth]{imgs/non_add.pdf}
    \caption{An intuitive illustration of non-additive accuracy Bayesian update on the torus. The lengths of arrows represent the uncertainty/entropy of the belief (\emph{e.g.}~$1/\sigma^2$ for Gaussian and $c$ for von Mises). The directions of the arrows represent the believed location (\emph{e.g.}~ $\mu$ for Gaussian and $m$ for von Mises).}
    \label{fig:non_add}
    \vskip -0.15in
\end{figure}
\textbf{Non-additive Accuracy} 
The additive accuracy is a nice property held with the Gaussian-formed sender distribution of the original BFN expressed as:
\begin{align}
\label{eq:standard_id}
    \update(\parsn{}'' \mid \parsn{}, \x; \alpha_a+\alpha_b) = \E_{\update(\parsn{}' \mid \parsn{}, \x; \alpha_a)} \update(\parsn{}'' \mid \parsn{}', \x; \alpha_b)
\end{align}
Such property is mainly derived based on the standard identity of Gaussian variable:
\begin{equation}
X \sim \mathcal{N}\left(\mu_X, \sigma_X^2\right), Y \sim \mathcal{N}\left(\mu_Y, \sigma_Y^2\right) \Longrightarrow X+Y \sim \mathcal{N}\left(\mu_X+\mu_Y, \sigma_X^2+\sigma_Y^2\right)
\end{equation}
The additive accuracy property makes it feasible to derive the Bayesian flow distribution $
p_F(\boldsymbol{\theta} \mid \mathbf{x} ; i)=p_U\left(\boldsymbol{\theta} \mid \boldsymbol{\theta}_0, \mathbf{x}, \sum_{k=1}^{i} \alpha_i \right)
$ for the simulation-free training of \cref{eq:loss_n}.
It should be noted that the standard identity in \cref{eq:standard_id} does not hold in the von Mises distribution. Hence there exists an important difference between the original Bayesian flow defined on Euclidean space and the Bayesian flow of circular data on $\mathbb{T}^D$ based on von Mises distribution. With prior $\btheta = \{\bold{0},\bold{0}\}$, we could formally represent the non-additive accuracy issue as:
% The additive accuracy property implies the fact that the "confidence" for the data sample after observing a series of the noisy samples with accuracy ${\alpha_1, \cdots, \alpha_i}$ could be  as the accuracy sum  which could be  
% Here we 
% Here we emphasize the specific property of BFN based on von Mises distribution.
% Note that 
% \begin{equation}
% \update(\parsn'' \mid \parsn, \x; \alpha_a+\alpha_b) \ne \E_{\update(\parsn' \mid \parsn, \x; \alpha_a)} \update(\parsn'' \mid \parsn', \x; \alpha_b)
% \end{equation}
% \oyyw{please check whether the below equation is better}
% \yuxuan{I fill somehow confusing on what is the update distribution with $\alpha$. }
% \begin{equation}
% \update(\parsn{}'' \mid \parsn{}, \x; \alpha_a+\alpha_b) \ne \E_{\update(\parsn{}' \mid \parsn{}, \x; \alpha_a)} \update(\parsn{}'' \mid \parsn{}', \x; \alpha_b)
% \end{equation}
% We give an intuitive visualization of such difference in \cref{fig:non_add}. The untenability of this property can materialize by considering the following case: with prior $\btheta = \{\bold{0},\bold{0}\}$, check the two-step Bayesian update distribution with $\alpha_a,\alpha_b$ and one-step Bayesian update with $\alpha=\alpha_a+\alpha_b$:
\begin{align}
\label{eq:nonadd}
     &\update(c'' \mid \parsn, \x; \alpha_a+\alpha_b)  = \delta(c-\alpha_a-\alpha_b)
     \ne  \mathbb{E}_{p_U(\parsn' \mid \parsn, \x; \alpha_a)}\update(c'' \mid \parsn', \x; \alpha_b) \nonumber \\&= \mathbb{E}_{vM(\bold{y}_b|\bold{x},\alpha_a)}\mathbb{E}_{vM(\bold{y}_a|\bold{x},\alpha_b)}\delta(c-||[\alpha_a \cos\y_a+\alpha_b\cos \y_b,\alpha_a \sin\y_a+\alpha_b\sin \y_b]^T||_2)
\end{align}
A more intuitive visualization could be found in \cref{fig:non_add}. This fundamental difference between periodic Bayesian flow and that of \citet{bfn} presents both theoretical and practical challenges, which we will explain and address in the following contents.

% This makes constructing Bayesian flow based on von Mises distribution intrinsically different from previous Bayesian flows (\citet{bfn}).

% Thus, we must reformulate the framework of Bayesian flow networks  accordingly. % and do necessary reformulations of BFN. 

% \yuxuan{overall I feel this part is complicated by using the language of update distribution. I would like to suggest simply use bayesian update, to provide intuitive explantion.}\hanlin{See the illustration in \cref{fig:non_add}}

% That introduces a cascade of problems, and we investigate the following issues: $(1)$ Accuracies between sender and receiver are not synchronized and need to be differentiated. $(2)$ There is no tractable Bayesian flow distribution for a one-step sample conditioned on a given time step $i$, and naively simulating the Bayesian flow results in computational overhead. $(3)$ It is difficult to control the entropy of the Bayesian flow. $(4)$ Accuracy is no longer a function of $t$ and becomes a distribution conditioned on $t$, which can be different across dimensions.
%\jj{Edited till here}

\textbf{Entropy Conditioning} As a common practice in generative models~\citep{ddpm,flowmatching,bfn}, timestep $t$ is widely used to distinguish among generation states by feeding the timestep information into the networks. However, this paper shows that for periodic Bayesian flow, the accumulated accuracy $\vc_i$ is more effective than time-based conditioning by informing the network about the entropy and certainty of the states $\parsnt{i}$. This stems from the intrinsic non-additive accuracy which makes the receiver's accumulated accuracy $c$ not bijective function of $t$, but a distribution conditioned on accumulated accuracies $\vc_i$ instead. Therefore, the entropy parameter $\vc$ is taken logarithm and fed into the network to describe the entropy of the input corrupted structure. We verify this consideration in \cref{sec:exp_ablation}. 
% \yuxuan{implement variant. traditionally, the timestep is widely used to distinguish the different states by putting the timestep embedding into the networks. citation of FM, diffusion, BFN. However, we find that conditioned on time in periodic flow could not provide extra benefits. To further boost the performance, we introduce a simple yet effective modification term entropy conditional. This is based on that the accumulated accuracy which represents the current uncertainty or entropy could be a better indicator to distinguish different states. + Describe how you do this. }



\textbf{Reformulations of BFN}. Recall the original update function with Gaussian sender distribution, after receiving noisy samples $\y_1,\y_2,\dots,\y_i$ with accuracies $\senderacc$, the accumulated accuracies of the receiver side could be analytically obtained by the additive property and it is consistent with the sender side.
% Since observing sample $\y$ with $\alpha_i$ can not result in exact accuracy increment $\alpha_i$ for receiver, the accuracies between sender and receiver are not synchronized which need to be differentiated. 
However, as previously mentioned, this does not apply to periodic Bayesian flow, and some of the notations in original BFN~\citep{bfn} need to be adjusted accordingly. We maintain the notations of sender side's one-step accuracy $\alpha$ and added accuracy $\beta$, and alter the notation of receiver's accuracy parameter as $c$, which is needed to be simulated by cascade of Bayesian updates. We emphasize that the receiver's accumulated accuracy $c$ is no longer a function of $t$ (differently from the Gaussian case), and it becomes a distribution conditioned on received accuracies $\senderacc$ from the sender. Therefore, we represent the Bayesian flow distribution of von Mises distribution as $p_F(\btheta|\x;\alpha_1,\alpha_2,\dots,\alpha_i)$. And the original simulation-free training with Bayesian flow distribution is no longer applicable in this scenario.
% Different from previous BFNs where the accumulated accuracy $\rho$ is not explicitly modeled, the accumulated accuracy parameter $c$ (visualized in \cref{fig:vmbf_vis}) needs to be explicitly modeled by feeding it to the network to avoid information loss.
% the randomaccuracy parameter $c$ (visualized in \cref{fig:vmbf_vis}) implies that there exists information in $c$ from the sender just like $m$, meaning that $c$ also should be fed into the network to avoid information loss. 
% We ablate this consideration in  \cref{sec:exp_ablation}. 

\textbf{Fast Sampling from Equivalent Bayesian Flow Distribution} Based on the above reformulations, the Bayesian flow distribution of von Mises distribution is reframed as: 
\begin{equation}\label{eq:flow_frac}
p_F(\btheta_i|\x;\alpha_1,\alpha_2,\dots,\alpha_i)=\E_{\update(\parsnt{1} \mid \parsnt{0}, \x ; \alphat{1})}\dots\E_{\update(\parsn_{i-1} \mid \parsnt{i-2}, \x; \alphat{i-1})} \update(\parsnt{i} | \parsnt{i-1},\x;\alphat{i} )
\end{equation}
Naively sampling from \cref{eq:flow_frac} requires slow auto-regressive iterated simulation, making training unaffordable. Noticing the mathematical properties of \cref{eq:h_m,eq:h_c}, we  transform \cref{eq:flow_frac} to the equivalent form:
\begin{equation}\label{eq:cirflow_equiv}
p_F(\vec{m}_i|\x;\alpha_1,\alpha_2,\dots,\alpha_i)=\E_{vM(\y_1|\x,\alpha_1)\dots vM(\y_i|\x,\alpha_i)} \delta(\vec{m}_i-\text{atan2}(\sum_{j=1}^i \alpha_j \cos \y_j,\sum_{j=1}^i \alpha_j \sin \y_j))
\end{equation}
\begin{equation}\label{eq:cirflow_equiv2}
p_F(\vec{c}_i|\x;\alpha_1,\alpha_2,\dots,\alpha_i)=\E_{vM(\y_1|\x,\alpha_1)\dots vM(\y_i|\x,\alpha_i)}  \delta(\vec{c}_i-||[\sum_{j=1}^i \alpha_j \cos \y_j,\sum_{j=1}^i \alpha_j \sin \y_j]^T||_2)
\end{equation}
which bypasses the computation of intermediate variables and allows pure tensor operations, with negligible computational overhead.
\begin{restatable}{proposition}{cirflowequiv}
The probability density function of Bayesian flow distribution defined by \cref{eq:cirflow_equiv,eq:cirflow_equiv2} is equivalent to the original definition in \cref{eq:flow_frac}. 
\end{restatable}
\textbf{Numerical Determination of Linear Entropy Sender Accuracy Schedule} ~Original BFN designs the accuracy schedule $\beta(t)$ to make the entropy of input distribution linearly decrease. As for crystal generation task, to ensure information coherence between modalities, we choose a sender accuracy schedule $\senderacc$ that makes the receiver's belief entropy $H(t_i)=H(p_I(\cdot|\vtheta_i))=H(p_I(\cdot|\vc_i))$ linearly decrease \emph{w.r.t.} time $t_i$, given the initial and final accuracy parameter $c(0)$ and $c(1)$. Due to the intractability of \cref{eq:vm_entropy}, we first use numerical binary search in $[0,c(1)]$ to determine the receiver's $c(t_i)$ for $i=1,\dots, n$ by solving the equation $H(c(t_i))=(1-t_i)H(c(0))+tH(c(1))$. Next, with $c(t_i)$, we conduct numerical binary search for each $\alpha_i$ in $[0,c(1)]$ by solving the equations $\E_{y\sim vM(x,\alpha_i)}[\sqrt{\alpha_i^2+c_{i-1}^2+2\alpha_i c_{i-1}\cos(y-m_{i-1})}]=c(t_i)$ from $i=1$ to $i=n$ for arbitrarily selected $x\in[-\pi,\pi)$.

After tackling all those issues, we have now arrived at a new BFN architecture for effectively modeling crystals. Such BFN can also be adapted to other type of data located in hyper-torus $\mathbb{T}^{D}$.

\subsection{Equivariant Bayesian Flow for Crystal}
With the above Bayesian flow designed for generative modeling of fractional coordinate $\vF$, we are able to build equivariant Bayesian flow for each modality of crystal. In this section, we first give an overview of the general training and sampling algorithm of \modelname (visualized in \cref{fig:framework}). Then, we describe the details of the Bayesian flow of every modality. The training and sampling algorithm can be found in \cref{alg:train} and \cref{alg:sampling}.

\textbf{Overview} Operating in the parameter space $\bthetaM=\{\bthetaA,\bthetaL,\bthetaF\}$, \modelname generates high-fidelity crystals through a joint BFN sampling process on the parameter of  atom type $\bthetaA$, lattice parameter $\vec{\theta}^L=\{\bmuL,\brhoL\}$, and the parameter of fractional coordinate matrix $\bthetaF=\{\bmF,\bcF\}$. We index the $n$-steps of the generation process in a discrete manner $i$, and denote the corresponding continuous notation $t_i=i/n$ from prior parameter $\thetaM_0$ to a considerably low variance parameter $\thetaM_n$ (\emph{i.e.} large $\vrho^L,\bmF$, and centered $\bthetaA$).

At training time, \modelname samples time $i\sim U\{1,n\}$ and $\bthetaM_{i-1}$ from the Bayesian flow distribution of each modality, serving as the input to the network. The network $\net$ outputs $\net(\parsnt{i-1}^\mathcal{M},t_{i-1})=\net(\parsnt{i-1}^A,\parsnt{i-1}^F,\parsnt{i-1}^L,t_{i-1})$ and conducts gradient descents on loss function \cref{eq:loss_n} for each modality. After proper training, the sender distribution $p_S$ can be approximated by the receiver distribution $p_R$. 

At inference time, from predefined $\thetaM_0$, we conduct transitions from $\thetaM_{i-1}$ to $\thetaM_{i}$ by: $(1)$ sampling $\y_i\sim p_R(\bold{y}|\thetaM_{i-1};t_i,\alpha_i)$ according to network prediction $\predM{i-1}$; and $(2)$ performing Bayesian update $h(\thetaM_{i-1},\y^\calM_{i-1},\alpha_i)$ for each dimension. 

% Alternatively, we complete this transition using the flow-back technique by sampling 
% $\thetaM_{i}$ from Bayesian flow distribution $\flow(\btheta^M_{i}|\predM{i-1};t_{i-1})$. 

% The training objective of $\net$ is to minimize the KL divergence between sender distribution and receiver distribution for every modality as defined in \cref{eq:loss_n} which is equivalent to optimizing the negative variational lower bound $\calL^{VLB}$ as discussed in \cref{sec:preliminaries}. 

%In the following part, we will present the Bayesian flow of each modality in detail.

\textbf{Bayesian Flow of Fractional Coordinate $\vF$}~The distribution of the prior parameter $\bthetaF_0$ is defined as:
\begin{equation}\label{eq:prior_frac}
    p(\bthetaF_0) \defeq \{vM(\vm_0^F|\vec{0}_{3\times N},\vec{0}_{3\times N}),\delta(\vc_0^F-\vec{0}_{3\times N})\} = \{U(\vec{0},\vec{1}),\delta(\vc_0^F-\vec{0}_{3\times N})\}
\end{equation}
Note that this prior distribution of $\vm_0^F$ is uniform over $[\vec{0},\vec{1})$, ensuring the periodic translation invariance property in \cref{De:pi}. The training objective is minimizing the KL divergence between sender and receiver distribution (deduction can be found in \cref{appd:cir_loss}): 
%\oyyw{replace $\vF$ with $\x$?} \hanlin{notations follow Preliminary?}
\begin{align}\label{loss_frac}
\calL_F = n \E_{i \sim \ui{n}, \flow(\parsn{}^F \mid \vF ; \senderacc)} \alpha_i\frac{I_1(\alpha_i)}{I_0(\alpha_i)}(1-\cos(\vF-\predF{i-1}))
\end{align}
where $I_0(x)$ and $I_1(x)$ are the zeroth and the first order of modified Bessel functions. The transition from $\bthetaF_{i-1}$ to $\bthetaF_{i}$ is the Bayesian update distribution based on network prediction:
\begin{equation}\label{eq:transi_frac}
    p(\btheta^F_{i}|\parsnt{i-1}^\calM)=\mathbb{E}_{vM(\bold{y}|\predF{i-1},\alpha_i)}\delta(\btheta^F_{i}-h(\btheta^F_{i-1},\bold{y},\alpha_i))
\end{equation}
\begin{restatable}{proposition}{fracinv}
With $\net_{F}$ as a periodic translation equivariant function namely $\net_F(\parsnt{}^A,w(\parsnt{}^F+\vt),\parsnt{}^L,t)=w(\net_F(\parsnt{}^A,\parsnt{}^F,\parsnt{}^L,t)+\vt), \forall\vt\in\R^3$, the marginal distribution of $p(\vF_n)$ defined by \cref{eq:prior_frac,eq:transi_frac} is periodic translation invariant. 
\end{restatable}
\textbf{Bayesian Flow of Lattice Parameter \texorpdfstring{$\boldsymbol{L}$}{}}   
Noting the lattice parameter $\bm{L}$ located in Euclidean space, we set prior as the parameter of a isotropic multivariate normal distribution $\btheta^L_0\defeq\{\vmu_0^L,\vrho_0^L\}=\{\bm{0}_{3\times3},\bm{1}_{3\times3}\}$
% \begin{equation}\label{eq:lattice_prior}
% \btheta^L_0\defeq\{\vmu_0^L,\vrho_0^L\}=\{\bm{0}_{3\times3},\bm{1}_{3\times3}\}
% \end{equation}
such that the prior distribution of the Markov process on $\vmu^L$ is the Dirac distribution $\delta(\vec{\mu_0}-\vec{0})$ and $\delta(\vec{\rho_0}-\vec{1})$, 
% \begin{equation}
%     p_I^L(\boldsymbol{L}|\btheta_0^L)=\mathcal{N}(\bm{L}|\bm{0},\bm{I})
% \end{equation}
which ensures O(3)-invariance of prior distribution of $\vL$. By Eq. 77 from \citet{bfn}, the Bayesian flow distribution of the lattice parameter $\bm{L}$ is: 
\begin{align}% =p_U(\bmuL|\btheta_0^L,\bm{L},\beta(t))
p_F^L(\bmuL|\bm{L};t) &=\mathcal{N}(\bmuL|\gamma(t)\bm{L},\gamma(t)(1-\gamma(t))\bm{I}) 
\end{align}
where $\gamma(t) = 1 - \sigma_1^{2t}$ and $\sigma_1$ is the predefined hyper-parameter controlling the variance of input distribution at $t=1$ under linear entropy accuracy schedule. The variance parameter $\vrho$ does not need to be modeled and fed to the network, since it is deterministic given the accuracy schedule. After sampling $\bmuL_i$ from $p_F^L$, the training objective is defined as minimizing KL divergence between sender and receiver distribution (based on Eq. 96 in \citet{bfn}):
\begin{align}
\mathcal{L}_{L} = \frac{n}{2}\left(1-\sigma_1^{2/n}\right)\E_{i \sim \ui{n}}\E_{\flow(\bmuL_{i-1} |\vL ; t_{i-1})}  \frac{\left\|\vL -\predL{i-1}\right\|^2}{\sigma_1^{2i/n}},\label{eq:lattice_loss}
\end{align}
where the prediction term $\predL{i-1}$ is the lattice parameter part of network output. After training, the generation process is defined as the Bayesian update distribution given network prediction:
\begin{equation}\label{eq:lattice_sampling}
    p(\bmuL_{i}|\parsnt{i-1}^\calM)=\update^L(\bmuL_{i}|\predL{i-1},\bmuL_{i-1};t_{i-1})
\end{equation}
    

% The final prediction of the lattice parameter is given by $\bmuL_n = \predL{n-1}$.
% \begin{equation}\label{eq:final_lattice}
%     \bmuL_n = \predL{n-1}
% \end{equation}

\begin{restatable}{proposition}{latticeinv}\label{prop:latticeinv}
With $\net_{L}$ as  O(3)-equivariant function namely $\net_L(\parsnt{}^A,\parsnt{}^F,\vQ\parsnt{}^L,t)=\vQ\net_L(\parsnt{}^A,\parsnt{}^F,\parsnt{}^L,t),\forall\vQ^T\vQ=\vI$, the marginal distribution of $p(\bmuL_n)$ defined by \cref{eq:lattice_sampling} is O(3)-invariant. 
\end{restatable}


\textbf{Bayesian Flow of Atom Types \texorpdfstring{$\boldsymbol{A}$}{}} 
Given that atom types are discrete random variables located in a simplex $\calS^K$, the prior parameter of $\boldsymbol{A}$ is the discrete uniform distribution over the vocabulary $\parsnt{0}^A \defeq \frac{1}{K}\vec{1}_{1\times N}$. 
% \begin{align}\label{eq:disc_input_prior}
% \parsnt{0}^A \defeq \frac{1}{K}\vec{1}_{1\times N}
% \end{align}
% \begin{align}
%     (\oh{j}{K})_k \defeq \delta_{j k}, \text{where }\oh{j}{K}\in \R^{K},\oh{\vA}{KD} \defeq \left(\oh{a_1}{K},\dots,\oh{a_N}{K}\right) \in \R^{K\times N}
% \end{align}
With the notation of the projection from the class index $j$ to the length $K$ one-hot vector $ (\oh{j}{K})_k \defeq \delta_{j k}, \text{where }\oh{j}{K}\in \R^{K},\oh{\vA}{KD} \defeq \left(\oh{a_1}{K},\dots,\oh{a_N}{K}\right) \in \R^{K\times N}$, the Bayesian flow distribution of atom types $\vA$ is derived in \citet{bfn}:
\begin{align}
\flow^{A}(\parsn^A \mid \vA; t) &= \E_{\N{\y \mid \beta^A(t)\left(K \oh{\vA}{K\times N} - \vec{1}_{K\times N}\right)}{\beta^A(t) K \vec{I}_{K\times N \times N}}} \delta\left(\parsn^A - \frac{e^{\y}\parsnt{0}^A}{\sum_{k=1}^K e^{\y_k}(\parsnt{0})_{k}^A}\right).
\end{align}
where $\beta^A(t)$ is the predefined accuracy schedule for atom types. Sampling $\btheta_i^A$ from $p_F^A$ as the training signal, the training objective is the $n$-step discrete-time loss for discrete variable \citep{bfn}: 
% \oyyw{can we simplify the next equation? Such as remove $K \times N, K \times N \times N$}
% \begin{align}
% &\calL_A = n\E_{i \sim U\{1,n\},\flow^A(\parsn^A \mid \vA ; t_{i-1}),\N{\y \mid \alphat{i}\left(K \oh{\vA}{KD} - \vec{1}_{K\times N}\right)}{\alphat{i} K \vec{I}_{K\times N \times N}}} \ln \N{\y \mid \alphat{i}\left(K \oh{\vA}{K\times N} - \vec{1}_{K\times N}\right)}{\alphat{i} K \vec{I}_{K\times N \times N}}\nonumber\\
% &\qquad\qquad\qquad-\sum_{d=1}^N \ln \left(\sum_{k=1}^K \out^{(d)}(k \mid \parsn^A; t_{i-1}) \N{\ydd{d} \mid \alphat{i}\left(K\oh{k}{K}- \vec{1}_{K\times N}\right)}{\alphat{i} K \vec{I}_{K\times N \times N}}\right)\label{discdisc_t_loss_exp}
% \end{align}
\begin{align}
&\calL_A = n\E_{i \sim U\{1,n\},\flow^A(\parsn^A \mid \vA ; t_{i-1}),\N{\y \mid \alphat{i}\left(K \oh{\vA}{KD} - \vec{1}\right)}{\alphat{i} K \vec{I}}} \ln \N{\y \mid \alphat{i}\left(K \oh{\vA}{K\times N} - \vec{1}\right)}{\alphat{i} K \vec{I}}\nonumber\\
&\qquad\qquad\qquad-\sum_{d=1}^N \ln \left(\sum_{k=1}^K \out^{(d)}(k \mid \parsn^A; t_{i-1}) \N{\ydd{d} \mid \alphat{i}\left(K\oh{k}{K}- \vec{1}\right)}{\alphat{i} K \vec{I}}\right)\label{discdisc_t_loss_exp}
\end{align}
where $\vec{I}\in \R^{K\times N \times N}$ and $\vec{1}\in\R^{K\times D}$. When sampling, the transition from $\bthetaA_{i-1}$ to $\bthetaA_{i}$ is derived as:
\begin{equation}
    p(\btheta^A_{i}|\parsnt{i-1}^\calM)=\update^A(\btheta^A_{i}|\btheta^A_{i-1},\predA{i-1};t_{i-1})
\end{equation}

The detailed training and sampling algorithm could be found in \cref{alg:train} and \cref{alg:sampling}.





% \clearpage
% \newpage
\section{Experiments}
\section{Experiments}
\label{sec:experiments}
The experiments are designed to address two key research questions.
First, \textbf{RQ1} evaluates whether the average $L_2$-norm of the counterfactual perturbation vectors ($\overline{||\perturb||}$) decreases as the model overfits the data, thereby providing further empirical validation for our hypothesis.
Second, \textbf{RQ2} evaluates the ability of the proposed counterfactual regularized loss, as defined in (\ref{eq:regularized_loss2}), to mitigate overfitting when compared to existing regularization techniques.

% The experiments are designed to address three key research questions. First, \textbf{RQ1} investigates whether the mean perturbation vector norm decreases as the model overfits the data, aiming to further validate our intuition. Second, \textbf{RQ2} explores whether the mean perturbation vector norm can be effectively leveraged as a regularization term during training, offering insights into its potential role in mitigating overfitting. Finally, \textbf{RQ3} examines whether our counterfactual regularizer enables the model to achieve superior performance compared to existing regularization methods, thus highlighting its practical advantage.

\subsection{Experimental Setup}
\textbf{\textit{Datasets, Models, and Tasks.}}
The experiments are conducted on three datasets: \textit{Water Potability}~\cite{kadiwal2020waterpotability}, \textit{Phomene}~\cite{phomene}, and \textit{CIFAR-10}~\cite{krizhevsky2009learning}. For \textit{Water Potability} and \textit{Phomene}, we randomly select $80\%$ of the samples for the training set, and the remaining $20\%$ for the test set, \textit{CIFAR-10} comes already split. Furthermore, we consider the following models: Logistic Regression, Multi-Layer Perceptron (MLP) with 100 and 30 neurons on each hidden layer, and PreactResNet-18~\cite{he2016cvecvv} as a Convolutional Neural Network (CNN) architecture.
We focus on binary classification tasks and leave the extension to multiclass scenarios for future work. However, for datasets that are inherently multiclass, we transform the problem into a binary classification task by selecting two classes, aligning with our assumption.

\smallskip
\noindent\textbf{\textit{Evaluation Measures.}} To characterize the degree of overfitting, we use the test loss, as it serves as a reliable indicator of the model's generalization capability to unseen data. Additionally, we evaluate the predictive performance of each model using the test accuracy.

\smallskip
\noindent\textbf{\textit{Baselines.}} We compare CF-Reg with the following regularization techniques: L1 (``Lasso''), L2 (``Ridge''), and Dropout.

\smallskip
\noindent\textbf{\textit{Configurations.}}
For each model, we adopt specific configurations as follows.
\begin{itemize}
\item \textit{Logistic Regression:} To induce overfitting in the model, we artificially increase the dimensionality of the data beyond the number of training samples by applying a polynomial feature expansion. This approach ensures that the model has enough capacity to overfit the training data, allowing us to analyze the impact of our counterfactual regularizer. The degree of the polynomial is chosen as the smallest degree that makes the number of features greater than the number of data.
\item \textit{Neural Networks (MLP and CNN):} To take advantage of the closed-form solution for computing the optimal perturbation vector as defined in (\ref{eq:opt-delta}), we use a local linear approximation of the neural network models. Hence, given an instance $\inst_i$, we consider the (optimal) counterfactual not with respect to $\model$ but with respect to:
\begin{equation}
\label{eq:taylor}
    \model^{lin}(\inst) = \model(\inst_i) + \nabla_{\inst}\model(\inst_i)(\inst - \inst_i),
\end{equation}
where $\model^{lin}$ represents the first-order Taylor approximation of $\model$ at $\inst_i$.
Note that this step is unnecessary for Logistic Regression, as it is inherently a linear model.
\end{itemize}

\smallskip
\noindent \textbf{\textit{Implementation Details.}} We run all experiments on a machine equipped with an AMD Ryzen 9 7900 12-Core Processor and an NVIDIA GeForce RTX 4090 GPU. Our implementation is based on the PyTorch Lightning framework. We use stochastic gradient descent as the optimizer with a learning rate of $\eta = 0.001$ and no weight decay. We use a batch size of $128$. The training and test steps are conducted for $6000$ epochs on the \textit{Water Potability} and \textit{Phoneme} datasets, while for the \textit{CIFAR-10} dataset, they are performed for $200$ epochs.
Finally, the contribution $w_i^{\varepsilon}$ of each training point $\inst_i$ is uniformly set as $w_i^{\varepsilon} = 1~\forall i\in \{1,\ldots,m\}$.

The source code implementation for our experiments is available at the following GitHub repository: \url{https://anonymous.4open.science/r/COCE-80B4/README.md} 

\subsection{RQ1: Counterfactual Perturbation vs. Overfitting}
To address \textbf{RQ1}, we analyze the relationship between the test loss and the average $L_2$-norm of the counterfactual perturbation vectors ($\overline{||\perturb||}$) over training epochs.

In particular, Figure~\ref{fig:delta_loss_epochs} depicts the evolution of $\overline{||\perturb||}$ alongside the test loss for an MLP trained \textit{without} regularization on the \textit{Water Potability} dataset. 
\begin{figure}[ht]
    \centering
    \includegraphics[width=0.85\linewidth]{img/delta_loss_epochs.png}
    \caption{The average counterfactual perturbation vector $\overline{||\perturb||}$ (left $y$-axis) and the cross-entropy test loss (right $y$-axis) over training epochs ($x$-axis) for an MLP trained on the \textit{Water Potability} dataset \textit{without} regularization.}
    \label{fig:delta_loss_epochs}
\end{figure}

The plot shows a clear trend as the model starts to overfit the data (evidenced by an increase in test loss). 
Notably, $\overline{||\perturb||}$ begins to decrease, which aligns with the hypothesis that the average distance to the optimal counterfactual example gets smaller as the model's decision boundary becomes increasingly adherent to the training data.

It is worth noting that this trend is heavily influenced by the choice of the counterfactual generator model. In particular, the relationship between $\overline{||\perturb||}$ and the degree of overfitting may become even more pronounced when leveraging more accurate counterfactual generators. However, these models often come at the cost of higher computational complexity, and their exploration is left to future work.

Nonetheless, we expect that $\overline{||\perturb||}$ will eventually stabilize at a plateau, as the average $L_2$-norm of the optimal counterfactual perturbations cannot vanish to zero.

% Additionally, the choice of employing the score-based counterfactual explanation framework to generate counterfactuals was driven to promote computational efficiency.

% Future enhancements to the framework may involve adopting models capable of generating more precise counterfactuals. While such approaches may yield to performance improvements, they are likely to come at the cost of increased computational complexity.


\subsection{RQ2: Counterfactual Regularization Performance}
To answer \textbf{RQ2}, we evaluate the effectiveness of the proposed counterfactual regularization (CF-Reg) by comparing its performance against existing baselines: unregularized training loss (No-Reg), L1 regularization (L1-Reg), L2 regularization (L2-Reg), and Dropout.
Specifically, for each model and dataset combination, Table~\ref{tab:regularization_comparison} presents the mean value and standard deviation of test accuracy achieved by each method across 5 random initialization. 

The table illustrates that our regularization technique consistently delivers better results than existing methods across all evaluated scenarios, except for one case -- i.e., Logistic Regression on the \textit{Phomene} dataset. 
However, this setting exhibits an unusual pattern, as the highest model accuracy is achieved without any regularization. Even in this case, CF-Reg still surpasses other regularization baselines.

From the results above, we derive the following key insights. First, CF-Reg proves to be effective across various model types, ranging from simple linear models (Logistic Regression) to deep architectures like MLPs and CNNs, and across diverse datasets, including both tabular and image data. 
Second, CF-Reg's strong performance on the \textit{Water} dataset with Logistic Regression suggests that its benefits may be more pronounced when applied to simpler models. However, the unexpected outcome on the \textit{Phoneme} dataset calls for further investigation into this phenomenon.


\begin{table*}[h!]
    \centering
    \caption{Mean value and standard deviation of test accuracy across 5 random initializations for different model, dataset, and regularization method. The best results are highlighted in \textbf{bold}.}
    \label{tab:regularization_comparison}
    \begin{tabular}{|c|c|c|c|c|c|c|}
        \hline
        \textbf{Model} & \textbf{Dataset} & \textbf{No-Reg} & \textbf{L1-Reg} & \textbf{L2-Reg} & \textbf{Dropout} & \textbf{CF-Reg (ours)} \\ \hline
        Logistic Regression   & \textit{Water}   & $0.6595 \pm 0.0038$   & $0.6729 \pm 0.0056$   & $0.6756 \pm 0.0046$  & N/A    & $\mathbf{0.6918 \pm 0.0036}$                     \\ \hline
        MLP   & \textit{Water}   & $0.6756 \pm 0.0042$   & $0.6790 \pm 0.0058$   & $0.6790 \pm 0.0023$  & $0.6750 \pm 0.0036$    & $\mathbf{0.6802 \pm 0.0046}$                    \\ \hline
%        MLP   & \textit{Adult}   & $0.8404 \pm 0.0010$   & $\mathbf{0.8495 \pm 0.0007}$   & $0.8489 \pm 0.0014$  & $\mathbf{0.8495 \pm 0.0016}$     & $0.8449 \pm 0.0019$                    \\ \hline
        Logistic Regression   & \textit{Phomene}   & $\mathbf{0.8148 \pm 0.0020}$   & $0.8041 \pm 0.0028$   & $0.7835 \pm 0.0176$  & N/A    & $0.8098 \pm 0.0055$                     \\ \hline
        MLP   & \textit{Phomene}   & $0.8677 \pm 0.0033$   & $0.8374 \pm 0.0080$   & $0.8673 \pm 0.0045$  & $0.8672 \pm 0.0042$     & $\mathbf{0.8718 \pm 0.0040}$                    \\ \hline
        CNN   & \textit{CIFAR-10} & $0.6670 \pm 0.0233$   & $0.6229 \pm 0.0850$   & $0.7348 \pm 0.0365$   & N/A    & $\mathbf{0.7427 \pm 0.0571}$                     \\ \hline
    \end{tabular}
\end{table*}

\begin{table*}[htb!]
    \centering
    \caption{Hyperparameter configurations utilized for the generation of Table \ref{tab:regularization_comparison}. For our regularization the hyperparameters are reported as $\mathbf{\alpha/\beta}$.}
    \label{tab:performance_parameters}
    \begin{tabular}{|c|c|c|c|c|c|c|}
        \hline
        \textbf{Model} & \textbf{Dataset} & \textbf{No-Reg} & \textbf{L1-Reg} & \textbf{L2-Reg} & \textbf{Dropout} & \textbf{CF-Reg (ours)} \\ \hline
        Logistic Regression   & \textit{Water}   & N/A   & $0.0093$   & $0.6927$  & N/A    & $0.3791/1.0355$                     \\ \hline
        MLP   & \textit{Water}   & N/A   & $0.0007$   & $0.0022$  & $0.0002$    & $0.2567/1.9775$                    \\ \hline
        Logistic Regression   &
        \textit{Phomene}   & N/A   & $0.0097$   & $0.7979$  & N/A    & $0.0571/1.8516$                     \\ \hline
        MLP   & \textit{Phomene}   & N/A   & $0.0007$   & $4.24\cdot10^{-5}$  & $0.0015$    & $0.0516/2.2700$                    \\ \hline
       % MLP   & \textit{Adult}   & N/A   & $0.0018$   & $0.0018$  & $0.0601$     & $0.0764/2.2068$                    \\ \hline
        CNN   & \textit{CIFAR-10} & N/A   & $0.0050$   & $0.0864$ & N/A    & $0.3018/
        2.1502$                     \\ \hline
    \end{tabular}
\end{table*}

\begin{table*}[htb!]
    \centering
    \caption{Mean value and standard deviation of training time across 5 different runs. The reported time (in seconds) corresponds to the generation of each entry in Table \ref{tab:regularization_comparison}. Times are }
    \label{tab:times}
    \begin{tabular}{|c|c|c|c|c|c|c|}
        \hline
        \textbf{Model} & \textbf{Dataset} & \textbf{No-Reg} & \textbf{L1-Reg} & \textbf{L2-Reg} & \textbf{Dropout} & \textbf{CF-Reg (ours)} \\ \hline
        Logistic Regression   & \textit{Water}   & $222.98 \pm 1.07$   & $239.94 \pm 2.59$   & $241.60 \pm 1.88$  & N/A    & $251.50 \pm 1.93$                     \\ \hline
        MLP   & \textit{Water}   & $225.71 \pm 3.85$   & $250.13 \pm 4.44$   & $255.78 \pm 2.38$  & $237.83 \pm 3.45$    & $266.48 \pm 3.46$                    \\ \hline
        Logistic Regression   & \textit{Phomene}   & $266.39 \pm 0.82$ & $367.52 \pm 6.85$   & $361.69 \pm 4.04$  & N/A   & $310.48 \pm 0.76$                    \\ \hline
        MLP   &
        \textit{Phomene} & $335.62 \pm 1.77$   & $390.86 \pm 2.11$   & $393.96 \pm 1.95$ & $363.51 \pm 5.07$    & $403.14 \pm 1.92$                     \\ \hline
       % MLP   & \textit{Adult}   & N/A   & $0.0018$   & $0.0018$  & $0.0601$     & $0.0764/2.2068$                    \\ \hline
        CNN   & \textit{CIFAR-10} & $370.09 \pm 0.18$   & $395.71 \pm 0.55$   & $401.38 \pm 0.16$ & N/A    & $1287.8 \pm 0.26$                     \\ \hline
    \end{tabular}
\end{table*}

\subsection{Feasibility of our Method}
A crucial requirement for any regularization technique is that it should impose minimal impact on the overall training process.
In this respect, CF-Reg introduces an overhead that depends on the time required to find the optimal counterfactual example for each training instance. 
As such, the more sophisticated the counterfactual generator model probed during training the higher would be the time required. However, a more advanced counterfactual generator might provide a more effective regularization. We discuss this trade-off in more details in Section~\ref{sec:discussion}.

Table~\ref{tab:times} presents the average training time ($\pm$ standard deviation) for each model and dataset combination listed in Table~\ref{tab:regularization_comparison}.
We can observe that the higher accuracy achieved by CF-Reg using the score-based counterfactual generator comes with only minimal overhead. However, when applied to deep neural networks with many hidden layers, such as \textit{PreactResNet-18}, the forward derivative computation required for the linearization of the network introduces a more noticeable computational cost, explaining the longer training times in the table.

\subsection{Hyperparameter Sensitivity Analysis}
The proposed counterfactual regularization technique relies on two key hyperparameters: $\alpha$ and $\beta$. The former is intrinsic to the loss formulation defined in (\ref{eq:cf-train}), while the latter is closely tied to the choice of the score-based counterfactual explanation method used.

Figure~\ref{fig:test_alpha_beta} illustrates how the test accuracy of an MLP trained on the \textit{Water Potability} dataset changes for different combinations of $\alpha$ and $\beta$.

\begin{figure}[ht]
    \centering
    \includegraphics[width=0.85\linewidth]{img/test_acc_alpha_beta.png}
    \caption{The test accuracy of an MLP trained on the \textit{Water Potability} dataset, evaluated while varying the weight of our counterfactual regularizer ($\alpha$) for different values of $\beta$.}
    \label{fig:test_alpha_beta}
\end{figure}

We observe that, for a fixed $\beta$, increasing the weight of our counterfactual regularizer ($\alpha$) can slightly improve test accuracy until a sudden drop is noticed for $\alpha > 0.1$.
This behavior was expected, as the impact of our penalty, like any regularization term, can be disruptive if not properly controlled.

Moreover, this finding further demonstrates that our regularization method, CF-Reg, is inherently data-driven. Therefore, it requires specific fine-tuning based on the combination of the model and dataset at hand.

\section{Conclusion}
\section{Conclusion}
In this work, we propose a simple yet effective approach, called SMILE, for graph few-shot learning with fewer tasks. Specifically, we introduce a novel dual-level mixup strategy, including within-task and across-task mixup, for enriching the diversity of nodes within each task and the diversity of tasks. Also, we incorporate the degree-based prior information to learn expressive node embeddings. Theoretically, we prove that SMILE effectively enhances the model's generalization performance. Empirically, we conduct extensive experiments on multiple benchmarks and the results suggest that SMILE significantly outperforms other baselines, including both in-domain and cross-domain few-shot settings.

\section*{Limitations}
While \method\ demonstrates improved performance of reasoning consistency under query variations, we primarily evaluate it on arithmetic and logical reasoning tasks. As a result, its potential and effectiveness on more complex tasks or domains such as specialized scientific reasoning and code generation remain underexplored. 

% We acknowledge that enhancing LLM capabilities, such as with the \method\ proposed in this paper, may raise certain ethical considerations. However, we believe that none requires specific attention in this context. 

% \TODO{risk}.

% Bibliography entries for the entire Anthology, followed by custom entries
%\bibliography{anthology}
% Custom bibliography entries only
\bibliography{custom}

\clearpage
\appendix

\section{Supplementary Experiments}
\label{sec:appendix-exp}
In this section, we provide some supplementary experiments that are omitted in the main context due to the space limit.

\begin{figure}[h]
    \centering
    % \vspace{-8pt} 
    \includegraphics[width=1\linewidth]{figure/appendix_fig/open-source-model-evaluation.pdf}
    % \vspace{-1mm}
    \caption{Evaluation of the {mathstral-7B} and {rho-math-7b-v0.1} model.}
    \label{fig: open model evaluation}
    % \vspace{-2mm}
\end{figure}



\subsection{Supplementary evaluation of open-source models}
\label{subsection:appendix-open-source models}
In addition to the evaluation of DeepSeek model we presented in section~\ref{subsection: Reasoning-Equivalence Evaluation}, we also provide the evaluation results of {Mathstral-7B} and \texttt{Rho-math-7b-v0.1}. The results of the $8$-shot accuracy evaluation are presented in Figure~\ref{fig: open model evaluation}. The findings are consistent with those in section~\ref{subsection: Reasoning-Equivalence Evaluation} that LLMs are vulnerable to order permutation and redundancy addition.

\subsection{Supplementary results about permutation order experiments}
\label{subsection:appendix-permutation-order-experiments}
% Table generated by Excel2LaTeX from sheet 'Sheet1'
\begin{table}[t]
  \centering
  \caption{Accuracy (\%) evaluation on datasets with different permutation order and base model as Llama-3.2-1B. The difficulty level is defined by the number of reasoning steps for ground-truth reasoning chains. The number in the parentheses indicates the performance comparison with the vanilla method. \textcolor{my_green}{Green}: performance improvement; \textcolor{my_red}{Red}: performance degradation. \textbf{Bold}: the method with best performance.}
    \resizebox{1.\linewidth}{!}{\begin{tabular}{ccccccc}
    \toprule
    \multirow{2}[4]{*}{Order} & \multirow{2}[4]{*}{Method} & \multicolumn{4}{c}{Difficulty Level} & \multirow{2}[4]{*}{Avg.} \\
\cmidrule{3-6}          &       & 1     & 2     & 3     & 4     &  \\
    \midrule
    \multirow{6}[2]{*}{\begin{sideways}Topological\end{sideways}} & Vanilla & 99.0  & 99.0  & 75.5  & 23.0  & 74.1  \\
          & RC-Aug & 100.0  & 100.0  & 84.5  & 24.5  & \textbf{77.3} \textcolor{my_green}{\textbf{(+3.2)}} \\
          & SCoP-2 & 100.0  & 82.5  & 19.0  & 2.0   & 50.9 \textcolor{my_red}{(-23.2)} \\
          & SCoP-4 & 100.0  & 92.5  & 29.0  & 4.5   & 56.5 \textcolor{my_red}{(-17.6)} \\
          & SCoP-8 & 100.0  & 99.0  & 40.5  & 9.0   & 62.1 \textcolor{my_red}{(-12.0)} \\
          & MEND  & 100.0  & 100.0  & 81.0  & 11.5  & 73.1 \textcolor{my_red}{(-1.0)} \\
    \midrule
    \multirow{6}[2]{*}{\begin{sideways}Random\end{sideways}} & Vanilla & 99.5  & 57.0  & 12.0  & 1.0   & 42.4  \\
          & RC-Aug & 100.0  & 65.0  & 12.0  & 0.5   & 44.4 \textcolor{my_green}{(+2.0)} \\
          & SCoP-2 & 100.0  & 76.5  & 14.0  & 2.5   & 48.3 \textcolor{my_green}{(+5.9)} \\
          & SCoP-4 & 100.0  & 94.0  & 27.5  & 4.0   & 56.4 \textcolor{my_green}{(+14.0)} \\
          & SCoP-8 & 100.0  & 98.5  & 38.0  & 7.5   & 61.0 \textcolor{my_green}{(+18.6)} \\
          & MEND  & 100.0  & 100.0  & 80.5  & 7.5   & \textbf{72.0} \textcolor{my_green}{\textbf{(+29.6)}} \\
    \midrule
    \multirow{6}[2]{*}{\begin{sideways}Reversed\end{sideways}} & Vanilla & 98.5  & 8.5   & 0.0   & 0.5   & 26.9  \\
          & RC-Aug & 100.0  & 29.0  & 0.5   & 0.0   & 32.4 \textcolor{my_green}{(+5.5)} \\
          & SCoP-2 & 100.0  & 79.0  & 22.5  & 3.5   & 51.3 \textcolor{my_green}{(+24.4)} \\
          & SCoP-4 & 100.0  & 95.5  & 30.0  & 4.5   & 57.5 \textcolor{my_green}{(+30.6)} \\
          & SCoP-8 & 100.0  & 98.5  & 41.0  & 3.0   & 60.6 \textcolor{my_green}{(+33.7)} \\
          & MEND  & 100.0  & 100.0  & 81.0  & 4.5   & \textbf{71.4} \textcolor{my_green}{\textbf{(+44.5)}} \\
    \bottomrule
    \end{tabular}%
    }
  \label{tab:appendix-permutation}%
\end{table}%
The supplementary experiment results about arithmetic reasoning tasks using {Llama-3.2-1B} model are shown in Table~\ref{tab:appendix-permutation}. The observations are basically the same as those in Section~\ref{subsection:exp reasoning-equivalence enhancement}: the proposed \method\ achieves the best performance in OOD scenarios against all the baseline methods.

% \begin{figure*}[t]
%     \centering
%     % \vspace{-8pt} 
%     \includegraphics[width=1\linewidth]{figure/appendix_fig/logic-3B.pdf}
%     % \vspace{-1mm}
%     \caption{Evaluation with respect to different query variations of logic reasoning dataset with the {Llama-3.2-3B} model. Each figure refers to one permutation order type, the x-axis represents the number of redundancies of the test set, and the y-axis represents the accuracy of final answers. For each dataset, we report the accuracy value over a dataset with a size of $200$. \textbf{In-distribution data:} training dataset including augmented data only covers the redundancy ranging from 0 to 4.}
%     \label{fig: logic reasoning 3B}
%     % \vspace{-2mm}
% \end{figure*}

\begin{figure*}[t]
    \centering
    % \vspace{-8pt} 
    \includegraphics[width=1\linewidth]{figure/appendix_fig/appendix-redundancy_exp.pdf}
    % \vspace{-1mm}
    \caption{Evaluations with respect to different query variations. Each figure refers to one permutation order type, the x-axis represents the number of redundancies of the test set, and the y-axis represents the accuracy of final answers. For each dataset, we report the accuracy value over a dataset with a size of $200$.}
    \label{fig: Math reasoning 1B}
    % \vspace{-2mm}
\end{figure*}

\subsection{Supplementary results about redundancy addition experiments}
\label{subsection:supplementary-exp-redundancy}
The results for logical reasoning with the base model {Llama-3.2-3B} and arithmetic reasoning with the base model {Llama-3.2-1B} are presented in Figure \ref{fig: Math reasoning 1B}. The findings are mostly consistent with those presented in Section~\ref{subsection:exp reasoning-equivalence enhancement}, showing that \method\ achieves the best performance. 
One exception is that the inference-time scaling baseline, \texttt{SCoP}, performs better in the logical reasoning tasks. This may be because the {Llama-3.2-3B} model is strong enough for this relatively easy task, allowing the inference-time scaling method to achieve good results through majority voting.



\subsection{Supplementary experiments about model probing}
\label{subsection: appendix-model-probing}
In Section~\ref{subsection: probing}, we discuss that the KNN classification method used in~\cite{hou2023towards} makes ideal assumptions and faces information aggregation issues. Here, we provide a comparison between KNN classification and the linear probing technique used in this paper. The comparison is presented in Figure~\ref{fig: exp-probing-comparison}, where we observe that although KNN probing can reveal differences between tasks with varying levels of redundancy, the F1-Macro score remains relatively low, reducing the confidence of the claims. Additionally, for harder tasks with more redundant information, the performance gap between different methods becomes minor. This is because information tends to aggregate on the beginning tokens, making it difficult to assign appropriate weights to all attention entries for information retrieval probing. Based on this experiment, we reconfirm that our linear probing technique is more suitable for revealing the model's capability.


\begin{figure}[t]
    \centering
    % \vspace{-8pt} 
    \includegraphics[width=1\linewidth]{figure/appendix_fig/probing_comparison.pdf}
    % \vspace{-1mm}
    \caption{Comparison between linear probing and KNN-based probing used in~\citep{hou2023towards}. Left: linear probing; Right: KNN-based probing.}
    \label{fig: exp-probing-comparison}
    % \vspace{-2mm}
\end{figure}



\clearpage
\newpage

\subsection{Supplementary evaluation of closed-source models}
\label{subsection: close-source model evaluation}
The Full QA pairs presented in Figure~\ref{fig: motivation} are provided the following code blocks.

% \textbf{Correct answer of DeepSeek-V3 to the original question.}
% \begin{tcolorbox}[colback=yellow!2!white, colframe=black, boxrule=0.5pt]
% \small
% \textit{""}
% \end{tcolorbox}

% \textbf{Correct answer of GPT-4 to the original question.}
% \begin{tcolorbox}[colback=yellow!2!white, colframe=black, boxrule=0.5pt]
% \small
% \textit{""}
% \end{tcolorbox}

% % \textbf{Incorrect answer of DeepSeek-V3 to Q1.}
% % \begin{tcolorbox}[colback=yellow!2!white, colframe=black, boxrule=0.5pt]
% % \small



% \end{tcolorbox}
\textbf{Incorrect answer of DeepSeek-V3 to Q1.}
\begin{tcolorbox}[
   enhanced,             % allow advanced features
   breakable,            % allow the box to split across pages
   colback=yellow!2!white,
   colframe=black,
   boxrule=0.5pt
]
\small
\begin{verbatim}
Let's break down the problem step by step to 
determine how many pieces of popcorn Gary 
has to eat.
1. **First 30 Seconds:**
   - **Popped Kernels:** 20
2. **Next 30 Seconds (30 to 60 seconds):**
   - The number of kernels that pop 
   is **three times** the amount that 
   popped in the first 30 seconds.
   - **Calculation:** \( 3 \times 20 = 60 \)
   - **Total Popped So Far:
   ** \( 20 + 60 = 80 \)
3. **After Removing from Heat (Residual Heat):
**
   - A **quarter** of the kernels that popped 
   in the last 30 seconds of cooking also pop 
   due to residual heat.
   - **Calculation:** \( \frac{1}{4} 
   \times 60 = 15 \)
   - **Total Popped So Far:** \( 80 + 15 = 95 \)
4. **Next 30 Seconds (60 to 90 seconds):**
   - The popping rate **increases by 20** 
   from the previous rate.
   - **Previous Rate:** 60 kernels per 30 
   seconds
   - **New Rate:** \( 60 + 20 = 80 \) kernels 
   per 30 seconds
   - **Total Popped So Far:** \( 95 + 80 = 175 \)
5. **Final 30 Seconds (90 to 120 seconds):**
   - The popping rate **slows down to half** 
   the rate of the past 30 seconds.
   - **Previous Rate:** 80 kernels per 
   30 seconds
   - **New Rate:** \( \frac{80}{2} = 40 \) 
   kernels per 30 seconds
   - **Total Popped So Far:** \( 175 + 40 = 215 \)
**Final Calculation:**
- **Total Popcorn Popped:** 215
However, let's verify the steps to 
ensure accuracy:
- **First 30 Seconds:** 
20 kernels
- **Next 30 Seconds:** \( 3 \times 20 = 60 \) 
kernels
- **Residual Heat:** 
\( \frac{1}{4} \times 60 = 15 \) kernels
- **Next 30 Seconds:** \( 60 + 20 = 80 \) kernels
- **Final 30 Seconds:** 
\( \frac{80}{2} = 40 \) kernels
Adding them up:
\[ 20 + 60 + 15 + 80 + 40 = 215 \]
Therefore, Gary has **215** pieces 
of popcorn to eat.
\[\boxed{215}\]
\end{verbatim}
\end{tcolorbox}

\newpage

\textbf{Incorrect answer of GPT-4 to Q2.}
\begin{tcolorbox}[colback=yellow!2!white, colframe=black, boxrule=0.5pt]
\small
\textit{"In the first 30 seconds, 20 kernels popped.
In the next 30 seconds, three times that amount popped, so 20 * 3 = <<20*3=60>>60 kernels popped.
In the next 30 seconds, four times the initial popping rate popped, so 20 * 4 = <<20*4=80>>80 kernels popped.
In the final 30 seconds, the popping slowed down to half the rate as the past 30 seconds, so 80 / 2 = <<80/2=40>>40 kernels popped.
After Garrett takes the pan off the heat, a quarter of the number of kernels that popped in the final 30 seconds of cooking also pop from the residual heat, so 40 / 4 = <<40/4=10>>10 kernels popped.
He believes that five kernels would not have popped if he had not shaken them to the center of the pan, so he has 20 + 60 + 80 + 40 + 10 + 5 = <<20+60+80+40+10+5=215>>215 pieces of popcorn to eat."}
\end{tcolorbox}


\clearpage
\section{Implementation Details}

\section{Implementation Details}
\label{sec:implementation_details}

\para{Training Details.}
We implement our networks using PyTorch~\cite{paszke2019pytorch} and run all experiments on a GPU cluster with four A100 GPUs.
%
We employ the Adam optimizer~\cite{kingma2014adam} to train our model with a learning rate of $2 \times 10^{-4}$ and an exponential decay of 0.998 every 1,000 iterations.
%
Following LION~\cite{zeng2022lion}, we use an exponential moving average (EMA) 
% \phil{TODO: no need to define EMA, if you will not use it later on in this paper}  \ed{This is a technique for training diffusion model so I mention here as implementation details}
of our model with a decay of 0.9999.
%
Specifically, we train our unconditional generative model for approximately 600,000 iterations with a batch size of 256, taking about four days to complete.
%
% \ed{Our method generally can produce reasonable results starting from 10 inference steps, which takes around XX seconds.}
% \phil{TODO: Try to make this sentenfce more specific and clear} \ednote{TODO: measure the inference time in A100.}
\rebuttal{It is noted that we use larger batch sizes and more training iterations compared to existing work, including~\cite{zhou2021pvd,wu2023psf}, to effectively compare various training paradigms (Figures~\ref{fig:main_quantitative_comp} \&~\ref{fig:main_qualitative_comp}). This choice ensures our training procedure has higher stability and converges properly.
% 
Additionally, we want to highlight that the online subsampling procedure introduces negligible overhead in the training process (merely requiring additional indexing of a cached array).}

\para{Network Architecture.}
For the network architecture, we adopt the same structure as PVD~\cite{zhou2021pvd} and employ PVCNN~\cite{liu2019point} as our vector field network for the unconditional generation task.
%
In the shape completion task, we use an additional 256-dimensional latent vector to represent the input partial point cloud, which is then injected into PVCNN.
%
% For image-conditioned generation, we extract this vector using the Dino-V2 (ViT-L/14)~\cite{oquab2023dinov2} 
% \phil{TODO: citation?} 
% image encoder followed by MLP layers.
%
To do so, we use another PVCNN follow LION~\cite{zeng2022lion} to extract the latent vector from the partial point cloud.
%
% \ednote{Yes. We should mention it!}
% \CL{Are we going to talk more about the inference (e.g. how long it takes for generating one point cloud with $N$ points; We should note we don't need the OT precomputation for inference since that is also a large overhead if needed. }

\para{Generated Shape Normalization.}
To ensure a fair evaluation among different baselines in the unconditional task, we convert the inference results of various generative methods into a common coordinate domain.
%
For all baseline methods, including PVD~\cite{zhou2021pvd}, DiT-3D~\cite{mo2023dit3d}, LION~\cite{zeng2022lion}, and PSF~\cite{wu2023psf}, we respect the original normalization adopted in their training procedures.
%
Since all these methods compute a global mean coordinate and global scale ratio across all training shapes, we use these two quantities to reverse the normalization on the generated shape, based on the values obtained from the training set.
%
This procedure aligns with existing baselines, such as~\cite{yang2019pointflow,zeng2022lion}.


\end{document}
