\label{section: preliminary}
\textbf{DAG Representation.} Following previous works~\citep{zhu2023dyval, zhu2024dynamic, ye2024physics}, we formulate the reasoning task as a problem defined over a directed acyclic graph (DAG) representation: \( G = (V, E) \), where \( V = \{v_1, v_2, \dots, v_n\} \) represents the set of nodes, and \( E \subseteq V \times V \) represents the set of directed edges indicating dependencies or relationships between nodes. The root node \( v_r \in V \) corresponds to the target variable we aim to compute or reason about, leaf nodes \( v_l \in \mathcal{L} \subseteq V \) denote the variables with known values, and the other nodes \( v_i \in \mathcal{I} = V \setminus (\mathcal{L} \cup \{v_r\}) \) represent intermediate nodes that need to be computed by their parent nodes, which are denoted as \( \mathrm{Pa}(v_i) = \{v_j \mid (v_j, v_i) \in E\} \). Each directed edge \( (v_j, v_i) \in E \) indicates that the value of \( x_i \) depends on \( x_j \) and their quantitative relationship.
\\
\textbf{Ground Truth of Reasoning Chain:} The ground truth reasoning path for the reasoning chain is represented as a sequence of intermediate reasoning steps \( \{y_1, y_2, \dots, y_T, y\} \) to the final result \( y \), where \( y_t \) corresponds to the result of the \( t \)-th intermediate computation. This process is grounded in the structure of the directed acyclic graph \( G = (V, E) \). 
The reasoning process follows a topological sorting of \( G \), which is a linear ordering of its nodes such that for every directed edge \( (v_i, v_j) \in E \), node \( v_i \) appears before \( v_j \) in the ordering. 

% The CoT process can be mathematically expressed as:
% \[
% y = y_T, \quad \text{where } y_t = f_t(\{x_j \mid v_j \in \mathrm{Pa}(v_t)\}),
% \]
% and \( f_t \) is the dependency function at step \( t \) that computes the value of \( y_t \) based on its parent nodes \( \mathrm{Pa}(v_t) \). The sequence of computations proceeds step-by-step, starting from the leaf nodes \( \mathcal{L} \) with known values:
% \[
% x_i = x_i^{(0)}, \quad \text{for } v_i \in \mathcal{L},
% \]
% progressing through the intermediate nodes \( v_i \in \mathcal{I} \), and culminating in the computation of the root node \( v_r \). 

% A topological sorting ensures that each node \( v_i \) in \( G \) is computed only after all its parent nodes \( \mathrm{Pa}(v_i) \) have been resolved. The CoT reasoning path, therefore, is the ordered sequence:
% \[
% \text{CoT Path: } y_1 \to y_2 \to \cdots \to y_T \to y.
% \]

% This formulation explicitly models the reasoning process as a sequence of logical steps, highlighting the progressive nature of the CoT method and its grounding in the DAG structure.

