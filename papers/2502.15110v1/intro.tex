The goal of Bayesian phylogenetics is to infer the genealogy of a collection of taxa given a genetic model and aligned sequence data. Phylogenetics is used in fields such as epidemiology \citep{Li:2020}, linguistics \citep{List:2014}, and ecology \citep{Godoy:2018}. Bayesian phylogenetic inference quantifies uncertainty and integrates over phylogenetic tree structures within a phylogenetic model \citep{Zhang:2019}. 
%
Most Bayesian phylogenetic inference is performed using Markov chain Monte Carlo (MCMC) methods with candidate trees iteratively proposed and either accepted or rejected based on their consistency with the observed data. However, MCMC methods can struggle because the number of possible trees grows super-exponentially in the number of taxa and posterior distributions on trees are highly multi-modal.

One alternative to MCMC is variational inference, in which the posterior distribution over phylogenetic tree structures is approximated using a variational distribution that minimizes some distance metric to the true posterior distribution. While variational inference over combinatorial spaces is also known to be difficult due to the complexity of the constraints on the support \citep{Bouchard:2010, Linderman:2018}, there have been several recent advances in variational inference over phylogenetic trees that are tractable. \citet{Zhang:2018} represented phylogenetic trees as Bayesian networks using \textit{subsplit Bayesian networks} (SBNs), and later used SBNs perform variational Bayesian phylogenetic inference on unrooted trees \citep{Zhang:2019}. This approach has spawned many methodological advancements. To improve the distribution for branch lengths, \citet{Zhang:2020} used normalizing flows, \citet{Molen:2024} used mixtures, and \citet{Xie:2024} used semi-implicit branch length distributions. To improve the variational family over tree topologies, \citet{Zhang:2023} used graph neural networks to learn topological features. 
%
However, the number of parameters within an SBN grows exponentially with the number of taxa, so MCMC chains must be used to find the most likely tree structures. To avoid this issue, other recent VI approaches sample tree topologies without the use of SBNs. For example, \citet{Koptagel:2022} use a gradient-free variational inference approach and directly sample from the \citet{Jukes:1969} model, while \citet{Mimori:2023} use a distance-based metric in hyperbolic space to construct unrooted phylogenetic trees. Further, \citet{Zhou:2024} introduce PhylogGFN, a phylogenetic variational inference technique based on reinforcement learning and generative flow networks \citep{Bengio:2023}.

In this work we focus on rooted ultrametric phylogenetic trees, where branch lengths correspond to the amount of time between evolutionary branching events. This formulation is useful when time is important, for example in applications involving rapidly evolving pathogens \citep{Sagulenko:2018}. However, none of the aforementioned approaches incorporate time constraints into the branch lengths of the phylogenetic trees. To this end, \citet{Zhang:2024} generalized their SBN-based approach to ultrametric trees, but this approach still relies on MCMC chains to restrict a subset of tree space to perform inference over. Alternatively, \citet{Bouckaert:2024} outline a method related to the one described in this manuscript. However, because their matrix representation of tree space is not dense, they can only express a limited number of trees. As such, their variational family is not supported on many tree topologies. Further, they do not perform any optimization to find the optimal tree structure and instead rely on empirical values derived from MCMC chains.

To this end, we introduce \textbf{V}ariational phylogenetic \textbf{I}nference with \textbf{PR}oducts over bipartitions (VIPR), a new variational family for ultrametric trees based on coalescent theory and single-linkage clustering \citep{Kingman:1982}. VIPR naturally performs variational inference on ultrametric trees and thus directly incorporates time into phylogenetic inference. Unlike previous variational methods over ultrametric trees, VIPR performs inference over the entirety of tree space and does not rely on MCMC sub-routines. In particular, we parameterize a variational distribution over a distance matrix and use it to derive a differentiable variational density over trees that result from single-linkage clustering. Through a set of standard experiments and an application to COVID-19, we show that our simple variational formulation achieves comparable results to existing methods for ultrametric trees in fewer gradient evaluations.