
\begin{doublespace}
\begin{center}
%\vspace*{1em}
{\vspace*{0.5em}\LARGE \bf Appendices for Variational Phylogenetic Inference\\with Products over Bipartitions}
%\sectionline
\end{center}

\section*{Appendix A: Proof for Proposition 1}\label{app:a}

We provide a proof by induction for the derivation of equation (\ref{eqn:q}). Consider the coalescent events in algorithm~\ref{alg:sample_q}. For $1 \le K \le N - 1$, Let $\bft_{1:K}$ be the times of the first $K$ coalescent events ($\bft_{1:K}=\{t_n\}_{n=1}^K$).  Let $\tau_{1:K}$ be the bipartitions of the first $K$ coalescent events ($\tau_{1:K} = \{ \{W_n,Z_n\}\}_{n=1}^K$). Let $q_{\phi,K}(\tau_K, \bft_K)$ be the probability density function of the marginal distribution on the times and bipartitions of the first $K$ coalescent events. Let $\calS_n$ be the set $\{\{w,z\} : w \in W_n, z \in Z_n\}$ (here $\{W_n,Z_n\}$ is the $n$-th bipartition). Note that $\calS_n$ is the set of all unordered pairs of taxa that have not coalesced before $t_n$ and that coalesce at  $t_n$. Let $\calS_{1:K}$ be $\displaystyle\bigcup_{n=1}^K \calS_n$ (\emph{i.e.}, $\calS_{1:K}$ is the set of all unordered pairs of taxa that coalesce by time $t_n$).

By the definition of $\calS_n$, a sum $\sum_{{w \in W_n, z \in Z_n}} \hspace{-0.25em}\cdot$, is equal to the same sum indexed by $S_n$. (And the same is true of products.)

Our induction hypothesis for $1 \le K \le N - 1$ is as follows:
\begin{align}
    q_{\phi,K}(\tau_{1:K},\bft_{1:K}) &= \prod_{n=1}^{K}\left(\left(\sum_{\{w,z\} \in \calS_n} \frac{q_\phi^{\{w,z\}}(t_n)}{Q_\phi^{\{w,z\}}(t_n)}\right)
    \prod_{\{w,z\} \in \calS_n} Q_\phi^{\{w,z\}}(t_n)\right)
    \prod_{\{w,z\} \notin \calS_{1:K}} Q_\phi^{\{w,z\}}(t_K).
    \label{eqn:q_N}
\end{align}
    
In (\ref{eqn:q_N}), the product over $\{w,z\} \notin \calS_{1:K}$ is outside of the product over $n=1,\ldots,K$. (Throughout this derivation, if a product sign has more than one factor in its operand, they are all enclosed by the pair of brackets appearing immediately after the product sign.) Consider the base case of the induction where $K = 1$. There exists an unordered pair of taxa $\{w^*, z^*\}$ such that  $\{W_1,Z_1\} = \{\{w^*\},\{z^*\}\}$. The probability density $q_{\phi,1}(\tau_{1:1},\bft_{1:1})$ is the density of the event that taxa $w^*$ and $z^*$ coalesce at time $t_1$ $(q^{\{w^*,z^*\}}(t_1))$ times the probability that all other taxa coalesce after time $t_1$ (as $\{W_1,Z_1\}$ is the first bipartition). Therefore, we have:
\begin{align}
    q_{\phi,1}(\tau_{1:1},\bft_{1:1}) &= q_\phi^{\{w^*,z^*\}}(t_1) \prod_{\{w,z\} \neq \{w^*,z^*\}} Q_\phi^{\{w,z\}}(t_1) \label{eqn:14} \\
    %
    &= \frac{q_\phi^{\{w^*,z^*\}}(t_1)}{Q_\phi^{\{w^*,z^*\}}(t_1)} \enspace Q_\phi^{\{w^*,z^*\}}(t_1) \prod_{\{w,z\} \notin \calS_1} Q_\phi^{\{w,z\}}(t_1). \\
    %
    &= \left(\sum_{\{w,z\} \in S_1} \frac{q_\phi^{\{w,z\}}(t_1)}{Q_\phi^{\{w,z\}}(t_1)}\right)\left(\prod_{\{w,z\} \in S_1} Q_\phi^{\{w,z\}}(t_1)\right)\prod_{\{w,z\} \notin \calS_{1:1}} Q_\phi^{\{w,z\}}(t_1).
\end{align}

Here in (\ref{eqn:14}) we use the mutual independence of $t^{(\cdot,\cdot)}$ to split the joint probability of all taxa other than $\{w^*,z^*\}$ coalescing after $t_1$ into a product. Thus, the base case ($K=1$) is established. Assume that the induction hypothesis (\ref{eqn:q_N}) holds for a given $K-1$ (here $1 \le K - 1 < N - 1$). Consider the conditional probability density function $q_{\phi,K}(\{W_K,Z_K\},t_K\mid \tau_{1:K-1},\bft_{1:K-1})$. When we condition on $\tau_{1:K-1},\bft_{1:K-1}$, the $K$-th coalescent event with bipartition $\{W_{K},Z_{K}\}$ occurs at time $t_{K}$ if and only if the following hold:

\begin{enumerate}
    \item There exists an unordered pair of taxa $\{w^*,z^*\} \in \calS(W_{K}, Z_{K})$ such that $t^{\{w^*,z^*\}} = t_K$.  (We are conditioning on the event that taxa $w^*$ and $z^*$ have not coalesced before time $t_{K-1}$.) The conditional probability density of the event that $w^*$ and $z^*$ coalesce at time $t_{K}$ is thus $q_\phi^{\{w^*,z^*\}}(t_{K}) / {Q_\phi^{\{w^*,z^*\}}(t_{K-1})}$.
    %
    \item All \textit{other} taxa pairs that have not coalesced by time $t_{K-1}$ coalesce after time $t_{K}$. (As the $q_\phi^{\{\cdot,\cdot\}}$'s are continuously differentiable, they are continuous and so $\{w^*,z^*\}$ is unique almost surely.) Note that we are conditioning on the event that all taxa pairs $\{w,z\} \notin \calS_{1:K-1}$ have not coalesced before time $t_{K-1}$. The conditional probability density of the event that $w^*$ and $z^*$ have not coalesced by time $t_{K}$ is thus ${Q_\phi^{\{w^*,z^*\}}(t_{K})} / {Q_\phi^{\{w^*,z^*\}}(t_{K-1})}$.
\end{enumerate}
\vspace{-1em}
The conditional probability density is thus:
\begin{align}
    & q_{\phi,K}(\{W_K,Z_K\},t_{K} \mid \tau_{1:K-1},\bft_{1:K-1}) \\
    %
    %
    &= \sum_{\{w^*,z^*\} \in \calS_K} \left( \frac{q_\phi^{\{w^*,z^*\}}(t_{K})}{Q_\phi^{\{w^*,z^*\}}(t_{K-1})} \prod_{\substack{\{w,z\} \notin \calS_{1:K-1} \\ \{w,z\} \neq \{w^*, z^*\}}} \frac{Q_\phi^{\{w,z\}}(t_{K})}{Q_\phi^{\{w,z\}}(t_{K-1})}  \right)\label{eqn:stars}\\
    %
    %
    &= \left(\sum_{\{w,z\} \in S_K} \frac{q_\phi^{\{w,z\}}(t_{K})}{Q_\phi^{\{w,z\}}(t_{K})}\right)
    %
    \prod_{\{w,z\} \notin \calS_{1:K-1}} \frac{Q_\phi^{\{w,z\}}(t_{K})}{Q_\phi^{\{w,z\}}(t_{K-1})} \\
    %
    %
    &= \left(\sum_{\{w,z\} \in S_K} \frac{q_\phi^{\{w,z\}}(t_{K})}{Q_\phi^{\{w,z\}}(t_{K})}\right)
    %
    \left(\prod_{\{w,z\} \in S_K} \frac{Q_\phi^{\{w,z\}}(t_{K})}{Q_\phi^{\{w,z\}}(t_{K-1})}  \right)
    \prod_{\{w,z\} \notin \calS_{1:K}} \frac{Q_\phi^{\{w,z\}}(t_{K})}{Q_\phi^{\{w,z\}}(t_{K-1})}.
\end{align}

Note that we drop the stars on the taxa $w$ and $z$ after (\ref{eqn:stars}) because the indices no longer need to be distinguished once the sum is isolated. Also, in (\ref{eqn:stars}) we use the mutual independence of $t^{(\cdot,\cdot)}$ to form the product. Multiplying this conditional probability with the induction hypothesis (\ref{eqn:q_N}) for $K-1$ yields the total probability density:
\begin{align}
    q_{\phi,K}(\tau_{1:K},\bft_{1:K}) =&\ q_{\phi,K}(\{W_K,Z_K\},t_{K} \mid \tau_{1:K-1},\bft_{1:K-1}) \cdot q_{\phi,K-1}(\tau_{1:K-1},\bft_{1:K-1}) \\
    %
    %
    =& \left(\sum_{\{w,z\} \in \calS_K} \frac{q_\phi^{\{w,z\}}(t_{K})}{Q_\phi^{\{w,z\}}(t_{K})}\right)
    %
    \left(\prod_{\{w,z\} \in \calS_K} \frac{Q_\phi^{\{w,z\}}(t_{K})}{Q_\phi^{\{w,z\}}(t_{K-1})}\right)
    %
    \left(\prod_{\{w,z\} \notin \calS_{1:K}} \frac{Q_\phi^{\{w,z\}}(t_{K})}{Q_\phi^{\{w,z\}}(t_{K-1})}\right)\label{eqn:line1} \\
    %
    %
    & \cdot \prod_{n=1}^{K-1} \left(\left(\sum_{\{w,z\} \in \calS_n } \frac{q_\phi^{\{w,z\}}(t_n)}{Q_\phi^{\{w,z\}}(t_n)} \right) \prod_{\{w,z\} \in S_n} Q_\phi^{\{w,z\}}(t_n)\right)
    \prod_{\{w,z\} \notin \calS_{1:K-1}} Q_\phi^{\{w,z\}}(t_{K-1}). \label{eqn:line2}
\end{align}
%
We then move the first component of line (\ref{eqn:line1}) into the first component of line (\ref{eqn:line2}), and we move the numerator of the second component of line (\ref{eqn:line1}) into the second component of line (\ref{eqn:line2}). These rearrangements yield the following:
%
\begin{align}
    q_{\phi,K}(\tau_{1:K},\bft_{1:K}) =& \left(\prod_{\{w,z\} \in \calS_K} \frac{1}{Q_\phi^{\{w,z\}}(t_{K-1})}\right)
    \left(\prod_{\{w,z\} \notin \calS_{1:K}} \frac{Q_\phi^{\{w,z\}}(t_{K})}{Q_\phi^{\{w,z\}}(t_{K-1})}\right)  \nonumber \\
    %
    & \cdot \prod_{n=1}^{K} \left(\left(\sum_{\{w,z\} \in \calS_n} \frac{q_\phi^{\{w,z\}}(t_n)}{Q_\phi^{\{w,z\}}(t_n)}\right)\prod_{\{w,z\} \in \calS_n} Q_\phi^{\{w,z\}}(t_n)\right)
    \prod_{\{w,z\} \notin \calS_{1:K-1}} Q_\phi^{\{w,z\}}(t_{K-1}).
\end{align}

%lte

For the final display, in (\ref{eqn:step1}) we split the numerator and denominator of $\frac{Q_\phi^{\{w,z\}}(t_{K})}{Q_\phi^{\{w,z\}}(t_{K-1})}$ into separate products. And in (\ref{eqn:step2b}) we cancel the  $Q_\phi^{\{w,z\}}(t_{K-1})$ factors involving $\{w,z\} \in \calS_{K}$. And in (\ref{eqn:step3}) we cancel the $Q_\phi^{\{w,z\}}(t_{K-1})$ factors involving $\{w,z\} \notin \calS_{1:K-1}$.

\begin{align}
    q_{\phi,K}(\tau_{1:K},\bft_{1:K}) =&
    \left(\prod_{\{w,z\} \in \calS_K} \frac{1}{Q_\phi^{\{w,z\}}(t_{K-1})}\right)
    \left(\prod_{\{w,z\} \notin \calS_{1:K}} \frac{1}{Q_\phi^{\{w,z\}}(t_{K-1})}\right)
    \left(\prod_{\{w,z\} \notin \calS_{1:K}} Q_\phi^{\{w,z\}}(t_{K}) \right)\nonumber \\
    %
    & \cdot
    \prod_{n=1}^{K} \left(\left(\sum_{{\{w,z\} \in \calS_n}} \frac{q_\phi^{\{w,z\}}(t_n)}{Q_\phi^{\{w,z\}}(t_n)}\right)
    \prod_{\{w,z\} \in \calS_n} Q_\phi^{\{w,z\}}(t_n)\right)
    \prod_{\{w,z\} \notin \calS_{1:K-1}} Q_\phi^{\{w,z\}}(t_{K-1}) \label{eqn:step1}   \\ \nonumber \\
    %
    %
    =& \left(\prod_{\{w,z\} \notin \calS_{1:K-1}} \frac{1}{Q_\phi^{\{w,z\}}(t_{K-1})}\right)
    \left(\prod_{\{w,z\} \notin \calS_{1:K}} Q_\phi^{\{w,z\}}(t_{K})\right) \nonumber \\ % \label{eqn:step2a} \\
    %
    & \cdot
    \prod_{n=1}^{K}
    \left(\left(\sum_{\{w,z\} \in \calS_n} \frac{q_\phi^{\{w,z\}}(t_n)}{Q_\phi^{\{w,z\}}(t_n)}\right)
    \prod_{\{w,z\} \in \calS_n} Q_\phi^{\{w,z\}}(t_n)\right)
    \prod_{\{w,z\} \notin \calS_{1:K-1}} Q_\phi^{\{w,z\}}(t_{K-1}) \label{eqn:step2b}  \\ \nonumber \\
    %
    %
    =& \prod_{n=1}^{K}
    \left(\left(\sum_{\{w,z\} \in \calS_n} \frac{q_\phi^{\{w,z\}}(t_n)}{Q_\phi^{\{w,z\}}(t_n)}\right)
    \prod_{\{w,z\} \in \calS_n} Q_\phi^{\{w,z\}}(t_n)\right)
    \prod_{\{w,z\} \notin \calS_{1:K}} Q_\phi^{\{w,z\}}(t_{K}).\label{eqn:step3}
\end{align}
Thus, the inductive step is established and (\ref{eqn:q_N}) holds for all $1 \le K \le N - 1$.
To complete the derivation, note that $q_{\phi,N-1} = q_{\phi}$; and $\calS_{1:N-1}=\bigcup_{n=1}^{N-1} \calS(W_n,Z_n) = \{\{w,z\} : w,z \in \calX\}$ (\textit{all} taxa coalesce after $N-1$ coalescent events); and indices over $w\in W_n, z\in Z_n$ are equivalent to indices over $\{w,z\} \in S_n$. Thus, for $K=N-1$ the last term of (\ref{eqn:step3}) is an empty product yielding the desired result:
\begin{align}
    q_{\phi}(\tau,\bft) &= q_{\phi,N-1}(\tau_{1:N-1},\bft_{1:N-1}) \\
    &= \prod_{n=1}^{N-1}\left(\left(
    \sum_{\{w, z\} \in \calS_n} \frac{q_\phi^{\{w,z\}}(t_n)}{Q_\phi^{\{w,z\}}(t_n)}\right)
    \prod_{\{w,z\} \in \calS_n} Q_\phi^{\{w,z\}}(t_n)\right) \\
    &= \prod_{n=1}^{N-1}\left(\left(\sum_{\substack{w \in\, W_n\\ z \in\, Z_n}} \frac{q_\phi^{\{w,z\}}(t_n)}{Q_\phi^{\{w,z\}}(t_n)}\right)\prod_{\substack{w \in W_n \\ z \in Z_n}} Q_\phi^{\{w,z\}}(t_n)\right).
\end{align}
\end{doublespace}
\newpage
\section*{Appendix B: Additional Results}\label{app:b}

\begin{figure}[H]
    \centering
    \begin{overpic}[width=0.4\linewidth]{DS1-F2C.png}
    \put (1,50) {DS1}
    \end{overpic}
    %
    \begin{overpic}[width=0.4\linewidth]{DS2-F2C.png}
    \put (1,50) {DS2}
    \end{overpic}
    %
    \begin{overpic}[width=0.4\linewidth]{DS3-F2C.png}
    \put (1,50) {DS3}
    \end{overpic}
    %
    \begin{overpic}[width=0.4\linewidth]{DS4-F2C.png}
    \put (1,50) {DS4}
    \end{overpic}
    %
    \begin{overpic}[width=0.4\linewidth]{DS5-F2C.png}
    \put (1,50) {DS5}
    \end{overpic}
    %
    \begin{overpic}[width=0.4\linewidth]{DS6-F2C.png}
    \put (1,50) {DS6}
    \end{overpic}
    %
    \begin{overpic}[width=0.4\linewidth]{DS7-F2C.png}
    \put (1,50) {DS7}
    \end{overpic}
    %
    \begin{overpic}[width=0.4\linewidth]{DS8-F2C.png}
    \put (1,50) {DS8}
    \end{overpic}
    %
    \begin{overpic}[width=0.4\linewidth]{DS9-F2C.png}
    \put (1,50) {DS9}
    \end{overpic}
    %
    \begin{overpic}[width=0.4\linewidth]{DS10-F2C.png}
    \put (1,50) {DS10}
    \end{overpic}
    %
    \begin{overpic}[width=0.4\linewidth]{DS11-F2C.png}
    \put (1,50) {DS11}
    \end{overpic}
    %
    \begin{overpic}[width=0.4\linewidth]{DS14-F2C.png}
    \put (1,50) {COV}
    \end{overpic}
    %
    \caption{{\bf \emph{Trace plots for all datasets.}} Trace plot of estimated marginal log-likelihood vs.\ iteration number (i.e., parameter update number). Marginal log-likelihood was estimated using 500 importance samples for VBPI and 50 importance samples for \model methods.}
    \label{fig:all_trace}
\end{figure}

\newpage

\section*{Appendix C: Gradient Estimators for $q_\phi$}\label{app:c}

\setcounter{section}{3}
\subsection{REINFORCE Estimator}

Define $f_{\phi}(\tau,\bft) \equiv \log(p(\tau,\bft,\bfY^{\ob})) - \log(q_{\phi}(\tau,\bft))$, so that $L(\phi) = \bbE_{q_\phi}[f_{\phi}(\tau,\bft)]$. We can interchange the gradients and the finite sum over $\tau$ due to the linearity of integrals. Further, conditioned on $\tau$, the integrand in Equation (\ref{eqn:ELBO_int}) is continuously differentiable in both $\phi$ and $\bft$, so we can interchange the gradient and the integral from Equation (\ref{eqn:ELBO_int}). After performing some algebra, we obtain the \textit{leave-one-out REINFORCE} (LOOR) estimator \citep{Mnih:2014,Shi:2022}: 
%
\begin{gather}
    \nabla_{\phi} L(\phi) \approx \frac{1}{K} \sum_{k=1}^K w^{(k)} \nabla_\phi \log q_\phi (\tau^{(k)},\bft^{(k)}),\\ 
    %
    w^{(k)} = f_\phi(\tau^{(k)},\bft^{(k)}) - \hat f^{(-k)}, \\
    %
    \hat f^{(-k)} = \frac{1}{K-1} \sum_{\ell \neq k} f_\phi(\tau^{(\ell)},\bft^{(\ell)}) \\
    %
    (\tau^{(k)},\bft^{(k)}) \sim q_{\phi}. 
\end{gather}

The gradient $\nabla_\phi \log q_\phi (\tau^{(k)},\bft^{(k)})$ can be calculated using automatic differentiation software such as Autograd \citep{Maclaurin:2015} or PyTorch \citep{Paszke:2019}. 

\subsection{The Reparameterization Trick}

The \textit{reparameterization trick} applied in our experiments when $q_\phi^{\{u,v\}}$ is a log-normal distribution for all $u$ and $v \in \calX$. In Algorithm (\ref{alg:sample_q}), the candidate coalescent times $t^{\{u,v\}} \sim \text{Lognormal}(\mu^{\{u,v\}},\sigma^{\{u,v\}}) \iff t^{\{u,v\}} = \exp(\mu^{\{u,v\}} + \sigma^{\{u,v\}} z^{\{u,v\}})$ with $z^{\{u,v\}} \sim \mathcal{N}(0,1)$. Thus, we reparameterize the expectation in Equation (\ref{eqn:ELBO}) as follows:

\begin{equation}
    L(\phi) = \bbE_{\bfZ}\left[\log\left(\frac{p(\bfY, g_{\phi}(\bfZ))}{q_\phi(g_{\phi}(\bfZ))}\right)\right].
\end{equation}

Where $g_{\phi}(\bfZ) = \text{Single-Linkage Clustering}(\exp(\bfmu + \bfsigma \odot \bfZ),\calX)$. Denoting the multivariate standard normal density as $\calN(\cdot;\mathbf{0},I)$, we have

\begin{equation}
    L(\phi) = \int_{\bfZ} \calN(\bfZ;\mathbf{0},I) \log\left(\frac{p(\bfY, g_{\phi}(\bfZ))}{q_\phi(g_{\phi}(\bfZ))}\right) d\bfZ.    
\end{equation}

However, the denominator $q_\phi(g_{\phi}(\bfZ))$ has jump discontinuities with respect to $\phi$ and $\bfZ$ because the tree structure $\tau$ is discrete. Therefore, we cannot interchange the integral and the gradient when estimating the full gradient. Instead, we sum over the tree structures $\tau$ and then integrate over $\bbZ_\tau(\phi)$, the space of all values of $\bfZ$ that are consistent with that tree structure given the parameters $\phi$:

\begin{equation}
    L(\phi) = \sum_{\tau} \int_{\bfZ \in \bbZ_{\tau}(\phi)} \calN(\bfZ;\mathbf{0},I) \log\left(\frac{p(\bfY, g_{\phi}(\bfZ))}{q_\phi(g_{\phi}(\bfZ))}\right) d\bfZ.    
\end{equation}

Then, the gradient can be approximated as follows:

\begin{align}
    \nabla_{\phi} L(\phi) &\approx \bbE_{\bfZ}\left[\nabla_{\phi}\log\left(\frac{p(\bfY, g_{\phi}(\bfZ))}{q_\phi(g_{\phi}(\bfZ))}\right)\right].     
\end{align}

We note that this is an approximation because the region of integration itself depends upon $\phi$, which adds additional terms into the interchange of the integral and the gradient. Nonetheless, we still define a \textit{biased} estimate of $\nabla_\phi L(\phi)$ as
%
\begin{gather}
    \widehat \nabla_\phi L(\phi) \approx \frac{1}{K} \sum_{k=1}^K \nabla_{\phi} \log\left(\frac{p(\bfY, g_{\phi}(\bfZ^{(k)}))}{q_\phi(g_{\phi}(\bfZ^{(k)}))}\right)\\
    %
    \bfZ^{(k)} \sim \calN(\cdot;\mathbf{0},I).
\end{gather}

As with the LOOR estimator, the gradient $\nabla_{\phi} \log\left(\frac{p(\bfY, g_{\phi}(\bfZ^{(k)}))}{q_\phi(g_{\phi}(\bfZ^{(k)}))}\right)$ can be calculated using automatic differentiation software such as Autograd \citep{Maclaurin:2015} or PyTorch \citep{Paszke:2019}. Because this estimate is not unbiased, it is not guaranteed to converge to a local optimum of the objective function. Nonetheless, these gradient estimates perform at least comparably to the LOOR estimator.

\subsection{The VIMCO Estimator}

When using the $K$-sample ELBO objective from Equation (\ref{eqn:ELBO_K}), the VIMCO estimator is an analogous gradient estimator to the LOOR estimator for the single-sample ELBO:

\begin{gather}
     \nabla_\phi L_K(\phi) \approx \sum_{k=1}^K \left(\hat L_K^{(-k)}(\phi) - \tilde w^{(k)}\right) \nabla_{\phi} \log q_\phi(\tau^{(k)},\bft^{(k)}) \\
    %
     \tilde w^{(k)} = \frac{f_\phi(\tau^{(k)},\bft^{(k)})}{\sum_{\ell=1}^K f_\phi(\tau^{(\ell)},\bft^{(\ell)})} \\
     %
     \hat L_K^{(-k)}(\phi) = \hat L_K(\phi) - \log \frac{1}{K}\left(\sum_{\ell \neq k} f_\phi(\tau^{(\ell)},\bft^{(\ell)}) + \hat f_\phi^{(-\ell)} \right) \\
     %
     \hat L_K(\phi) = \log \left(\frac{1}{K}\sum_{k=1}^K f_\phi(\tau^{(k)},\bft^{(k)}) \right) \\
     %
     \hat f^{(-\ell)} = \frac{1}{K-1} \sum_{j \neq \ell} f_\phi(\tau^{(j)},\bft^{(j)}) \\
    %
    (\tau^{(k)},\bft^{(k)}) \sim q_{\phi}.
\end{gather}