In this work, we introduce a new variational family over ultrametric, time-measured phylogenies that models the coalescent time between each pair of taxa. The family is formed by deriving a closed-form expression for the marginal distribution on phylogenies induced by single linkage clustering of a distance matrix. 

Methods using this variational family require only $\mathcal{O}(\binom{N}{2})$ parameters in total, and each parameter has an intuitive interpretation as a description of the distribution on pairwise coalescents. For example, in this work we place independent log-normal distributions on the entries of the distance matrix, yielding $2\binom{N}{2}$ parameters (one mean and one standard deviation for each pair of taxa).

Our methods may be further developed in many ways. For example, by moving from the log-normal distribution on pairwise coalescent times to  mixture distributions similar to \citet{Molen:2024}, or normalizing flows similar to \citet{Zhang:2020}. We could also directly enforce sparsity in the prior by fixing the distribution on the time to coalescence of taxa $u$ and $v$ (when $u$ and $v$ are far away in genetic space) at infinity, further reducing the number of parameters to learn.  

Expanding the variational family to include conditional parameters may also improve performance: if taxa $u$ and $v$ are the first to coalesce, we could define a new parameter $\phi^{(\{u,v\},w)}$ describing the coalesce between a clade containing $\{u,v\}$ and  another taxon $w$. 

Our variational inference for phylogenetics is unique in that it does not require  aspects of MCMC runs in its iterations in order to make inference computationally tractable. (In contrast, in~\citealt{Zhang:2024} for example MCMC is used to fix the support of the trees described by their variational family). Note however that we did use short MCMC runs to initialize the parameters of our log-normal distributions.

Fast and accurate parameter initializations and well-tuned annealing are essential for top performance in variational Bayesian phylogenetics. Our experiments may be improved by using  an annealing schedule similar to \citet{Zhang:2024} during optimization to prevent convergence to local maxima.

We have focused on the difficult task of inferring tree topology and branch lengths. However, inference for aspects such as a relaxed clock \citep[see][]{Douglas:2021} and the effective populations size $N_e$ can also be done starting from this new variational family. Our method thus serves as a promising foundation on which more intricate inference can be built.

