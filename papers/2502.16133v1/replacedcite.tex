\section{RELATED WORK}
\subsection{Blockchain-based Trust Management for IoT}
IoT is characterized by high complexity and heterogeneity, constantly facing significant security and privacy threats ____. Trust management technologies play a crucial role in establishing a reliable and effective security mechanism for IoT ____. One potential approach to computing trust is reputation-based ____, which aggregates collected trust information in a centralized or distributed manner to generate reputation scores ____. In IoT environments, employing reputation-based trust management techniques can enhance the trustworthiness between entities and detect abnormal activities ____. However, existing trust management schemes still face challenges such as single points of failure, data consistency maintenance, and the provision of a global trust view ____. Consequently, more research is leveraging the security features of blockchain to address these issues.

In the fields of edge computing and fog computing, the work ____ designed a decentralized trust management mechanism to evaluate the trustworthiness of edge nodes. It implemented a trusted MEC (T-MEC) framework using a directed acyclic graph structured blockchain (DAG blockchain) and distributed trust management (DTM) to resolve inconsistencies between on-chain and off-chain trust. The work ____ proposed a layered and scalable trust management scheme to assess the credibility of service providers, where trust information is computed and managed by fog nodes and propagated through a blockchain network to obtain a global view of trust data. In the supply chain sector, the work ____ introduced a three-layer trust management framework capable of dynamically scoring reputations to evaluate the trustworthiness of supply chain entities. By combining consortium blockchain with trust management technologies, it addresses trust challenges related to the data itself. Similarly, the work ____ proposed a blockchain-integrated trust management model for assessing the trust of supply chain entities, providing an analysis based on a real-world scenario. In the vehicular networks domain, the work ____ developed a trust management scheme based on Bayesian inference, integrating consortium blockchain to solve trust issues in inter-vehicle information sharing. Additionally, the work ____ utilized three-valued subjective logic (3VSL) to comprehensively evaluate entities within vehicular networks, achieving more accurate reputations and combining blockchain for secure computation offloading.

\subsection{Deep Reinforcement Learning for IoT}
Due to the heterogeneity, complexity, and dynamic nature of IoT systems, problems framed as game theory or combinatorial optimization are often NP-hard ____. Many studies have employed dynamic programming and heuristic algorithms to address these challenges, but these methods typically require numerous iterations to achieve satisfactory solutions ____. In recent years, DRL has been widely applied in IoT systems because of its advantages in long-term performance optimization, real-time decision making, online learning without prior knowledge, and high scalability ____.

In the field of industrial IoT, the work ____ developed a job scheduling model for hybrid flow shop scheduling on batch processing machines (HFSP-BPM) based on DRL, overcoming the challenge where a fixed search paradigm could not simultaneously meet the requirements for real-time processing and solution quality. In the work ____, a DRL-based approach was proposed for dynamic resource management in industrial IoT, aiming to minimize the average task delay. This approach leverages DRL’s capability to handle high-dimensional state spaces, addressing the issues that arise from the dynamic and continuous nature of tasks in resource-constrained IIoT environments. In the domain of smart grids, the work ____ introduced a distributed DRL-based method for intelligent load scheduling in residential smart grids, which not only protects household privacy but also reduces grid stress and household electricity costs. The work ____ applied multi-agent reinforcement learning to solve the electric vehicle charging scheduling problem in smart grids, allowing for quick responses to user demands while reducing operational costs for charging station operators. In the field of intelligent transportation, the work ____ proposed a DRL-based approach to develop vehicle navigation systems designed to meet the real-time demands of complex urban environments, thereby alleviating traffic congestion. The effectiveness of this solution was validated through simulations in nine real-world traffic scenarios. The work ____ utilized DRL’s adaptive learning capabilities to develop an intelligent electric vehicle charging navigation system, aiming to minimize total travel time and charging costs for electric vehicles.

The problem of selecting oracle nodes can be modeled as a decision optimization problem. Leveraging the advantages of DRL technology to select the most suitable oracle from many nodes for complex data requests is a viable solution. Based on our research, at the time of writing this paper, no existing work has applied DRL technology to the blockchain oracle domain. Our work is the first attempt in this area.
\subsection{Blockchain Oracles}
Blockchain oracles, which act as intermediaries between blockchain networks and external data, have received increasing attention in recent years. The work ____ proposed a decentralized IoT global marketplace architecture based on blockchain technology and oracle networks, allowing users to purchase IoT device data through smart contracts. Additionally, to ensure data quality and credibility, a reputation algorithm was designed to evaluate data sources. The work ____ developed a blockchain-based trust management system for Edge-to-Cloud environments, using oracles to obtain external system monitoring data to support smart contract execution, thereby establishing trust between service providers and stakeholders. However, both studies assume that oracles will always provide honest and reliable services, which is overly optimistic.

Oracles can also behave maliciously, potentially causing significant damage to the blockchain network. Many studies have adopted reputation as the trust model for oracle systems. For instance, the work ____ in the construction industry used smart construction objects (SCOs) as blockchain oracles to transmit real-world construction process data to the blockchain and utilized smart contracts to rate the reputation of oracles. There are also commercial oracle networks like Chainlink ____ and Witnet ____. However, these studies often evaluate oracle reputation based only on some past records. Chainlink derives reputation scores by recording on-chain performance history, such as average response latency. The work ____, similar to Witnet, assigns a high reputation score to oracles that return results matching those of the majority of oracles, while those returning different results receive lower scores. These approaches lacks more comprehensive trust management mechanisms for evaluating the reputation of oracle nodes.

Moreover, current research on how to select oracle nodes is still quite limited, with only the work ____ employing intelligent mechanisms for the flexible selection of oracle nodes. The Table \ref{tab:1} summarizes the comparison between the BLOR proposed in the work ____ and the TCO-DRL introduced in this paper. Overall, TCO-DRL offers the following advantages:

\begin{table*}[!t]
\centering
\caption{Comparison of the proposed TCO-DRL with BLOR.}
\label{tab:1}
\renewcommand\arraystretch{1.3}
\begin{tabular*}{0.8\linewidth}{lcc}
\toprule
\textbf{Features} & \textbf{BLOR____} & \textbf{TCO-DRL} \\
\midrule 
 \textbf{Trust Model} & Reputation & Reputation \\
 \makecell[tl]{\\\textbf{Reputation Metric}} &  \makecell[tc]{\\Success and failure count} & \makecell[tc]{Reliability score \\ Behavior score \\token score}\\
 \textbf{Execution Architecture} & On-chain &  On-chain/Off-chain\\ 
 \textbf{Learning Method} & Online learning with prior knowledge  & Online learning without prior knowledge\\
 \textbf{Adaptive Learning Capability} & Weak & Strong\\
 \textbf{Scalability} & Low & High \\
 \textbf{Time Factor} &  NO & YES\\
 \textbf{Trust} & YES & YES \\
 \textbf{Cost} & YES & YES \\
 \textbf{Service Matching}& NO & YES\\
 \textbf{Attack Resistance} & NO & YES \\
\bottomrule
\end{tabular*}
\end{table*}

\subsubsection{A more comprehensive trust management scheme} TCO-DRL not only considers factors such as the oracle's success rate, interaction frequency, and average response time to compute a reliability score but also evaluates the oracle's behavior and staked tokens. Additionally, TCO-DRL incorporates the influence of a time factor, resulting in a more comprehensive trust management scheme.
\subsubsection{Independence from prior knowledge} TCO-DRL does not rely on any predefined models or prior distributions but instead learns optimal strategies directly through interaction with the environment and experience replay, enabling automatic adjustment and optimization of strategies. By using neural networks as function approximators, TCO-DRL can handle complex, high-dimensional state and action spaces, demonstrating strong learning capabilities and adaptability in dynamic and complex environments.
\subsubsection{Improved quality of service} TCO-DRL additionally takes into account the specific service requirements added by data requesters, striving to match these with oracles that offer the same services to enhance quality of service and user experience.
\subsubsection{Resistance to attacks} TCO-DRL is designed with attack resistance in mind, incorporating an improved sliding time window in its trust management scheme, which makes it difficult for malicious nodes to continuously engage in harmful actions. This enhances the system's security and robustness against malicious attack.





% needed in second column of first page if using \IEEEpubid
%\IEEEpubidadjcol