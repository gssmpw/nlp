\newpage
\section{Interpretability of LDM4TS}
\subsection{Showcases}
\label{appx:showcases}
\begin{figure}[!ht]
  \centering
  \includegraphics[width=\linewidth]{figure/showcases_96.pdf}
\caption{Visualization of ETTh1 predictions by different models under the input-96-predict-96 setting. The orange lines stand for the ground truth and the blue lines stand for predicted values.}
\label{fig:showcases_96}
\end{figure}

\begin{figure}[!ht]
  \centering
  \includegraphics[width=\linewidth]{figure/showcases_192.pdf}
\caption{Visualization of ETTh1 predictions by different models under the input-96-predict-192 setting. The orange lines stand for the ground truth and the blue lines stand for predicted values.}
\label{fig:showcases_192}
\end{figure}

\begin{figure}[!ht]
  \centering
  \includegraphics[width=\linewidth]{figure/showcases_336.pdf}
\caption{Visualization of ETTh1 predictions by different models under the input-96-predict-336 setting. The orange lines stand for the ground truth and the blue lines stand for predicted values.}
\label{fig:showcases_336}
\end{figure}

\begin{figure}[!ht]
  \centering
  \includegraphics[width=\linewidth]{figure/showcases_720.pdf}
\caption{Visualization of ETTh1 predictions by different models under the input-96-predict-720 setting. The orange lines stand for the ground truth and the blue lines stand for predicted values.}
\label{fig:showcases_720}
\end{figure}

\newpage
\subsection{Visualization of Pixel Space}
\label{appx:visualization_picel_space}
\begin{figure*}[!ht]
  \centering
  \includegraphics[width=\linewidth]{figure/visualization_of_vision_encoder.pdf}
\caption{Visualization of different time series encoding methods for the vision encoder. We show three approaches: segmentation-based methods (SEG), Gramian Angular Field (GAF), and Recurrence Plot (RP). All methods transform raw time series into image representations with dimensions $\mathbb{R}^{B\times3 \times H\times W}$, where $B$ is the batch size, 3 represents RGB channels, $H$ and $W$ denote the height and width of the generated images.}
\label{fig:visualization_ve}
\end{figure*}