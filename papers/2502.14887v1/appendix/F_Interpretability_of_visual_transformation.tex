\newpage
\section{Analysis of Vision Transformation Methods}
\label{appx:transformation}

Time series analysis faces the fundamental challenge of capturing complex temporal dynamics that manifest simultaneously across multiple scales. While traditional methods excel at specific temporal resolutions, they often struggle to comprehensively capture the full spectrum of patterns ranging from rapid local variations to gradual global trends. This limitation motivates our investigation into vision transformation techniques that can effectively encode rich temporal information into spatial patterns, making them amenable to powerful vision-based processing approaches.

Our framework introduces a systematic approach to time series visualization through three theoretically-grounded transformation methods. Each method targets distinct yet complementary aspects of temporal dynamics, providing a comprehensive representation of the underlying time series structure:

\subsection{Segmentation Representation (SEG)}
The SEG transformation addresses the challenge of preserving local temporal structures while enabling efficient detection of periodic patterns. This method operates by restructuring a time series $x \in \mathbb{R}^L$ into a matrix $M \in \mathbb{R}^{\lceil L/T \rceil \times T}$, where T represents the period length. The transformation process can be formally expressed as:

\begin{equation}
    M_{i,j} = x_{(i-1)T + j}, \quad \text{for } i=1,\dots,\lceil L/T \rceil, j=1,\dots,T
\end{equation}

This segmentation approach offers several theoretical and practical advantages:

\begin{itemize}
    \item \textbf{Local Structure Preservation:} Each row in the matrix represents a complete segment of length T, maintaining the original temporal relationships at the finest granularity
    \item \textbf{Periodic Pattern Detection:} The vertical alignment of segments facilitates the identification of recurring patterns across different time periods
    \item \textbf{Multi-scale Analysis:} By varying the period length T, the transformation can capture patterns at different temporal scales, enabling hierarchical pattern discovery
\end{itemize}

The optimal period length T is determined through an optimization process that maximizes temporal correlation:
\begin{equation}
    T = \arg\max_{k} \sum_{i=1}^{\lceil L/k \rceil} \sum_{j=1}^{k-1} \text{Corr}(M_{i,j}, M_{i,j+1})
\end{equation}

where $\text{Corr}(\cdot,\cdot)$ denotes the correlation between adjacent columns. This optimization ensures optimal alignment of periodic patterns while maintaining temporal fidelity.

\subsection{Gramian Angular Field (GAF)}
The GAF transformation provides a sophisticated approach to encoding temporal relationships through polar coordinate mapping and trigonometric relationships. This method preserves both magnitude and temporal correlation information through a series of carefully designed transformations.

First, the time series is normalized to a bounded interval $[-1,1]$ or $[0,1]$:
\begin{equation}
    \tilde{x}_i = \frac{x_i - \min(x)}{\max(x) - \min(x)}
\end{equation}

The normalized values are then encoded in a polar coordinate system:
\begin{equation}
    \phi = \arccos(\tilde{x}_i), \quad r = \frac{t_i}{N}
\end{equation}

where $t_i$ represents temporal position and $N$ serves as a scaling factor. The final Gramian matrix is constructed through:
\begin{equation}
    G_{i,j} = \cos(\phi_i - \phi_j)
\end{equation}

This transformation offers several key advantages:
\begin{itemize}
    \item \textbf{Scale Invariance:} The polar encoding ensures that the representation is robust to amplitude variations
    \item \textbf{Temporal Correlation Preservation:} The Gramian matrix captures both local and global temporal dependencies
    \item \textbf{Dimensionality Reduction:} The transformation provides a compact representation while preserving essential temporal information
\end{itemize}

\subsection{Recurrence Plot (RP)}
The RP transformation leverages phase space reconstruction to visualize the recurrent behavior in dynamical systems. Based on Taken's embedding theorem, this method first reconstructs the phase space trajectory:

\begin{equation}
    \vec{x}_i = (x_i, x_{i+\tau}, ..., x_{i+(m-1)\tau})
\end{equation}

where $m$ is the embedding dimension and $\tau$ is the time delay. The recurrence matrix is then constructed as:

\begin{equation}
    R_{i,j} = \Theta(\epsilon - \|\vec{x}_i - \vec{x}_j\|)
\end{equation}

where $\Theta$ is the Heaviside function and $\epsilon$ is a threshold distance. This transformation reveals fundamental dynamical properties through several characteristic patterns:

\begin{itemize}
    \item \textbf{Diagonal Lines:} Parallel to the main diagonal, indicating similar evolution of trajectories and revealing deterministic structures
    \item \textbf{Vertical/Horizontal Lines:} Representing periods of state stagnation or laminar phases
    \item \textbf{Complex Patterns:} Non-uniform structures indicating chaos or non-linear dynamics
\end{itemize}

\subsection{Theoretical Integration and Complementarity}
The integration of these three transformations provides a comprehensive framework for time series analysis, offering several theoretical and practical advantages:

\begin{itemize}
    \item \textbf{Multi-scale Pattern Capture:} Each transformation targets different temporal scales - SEG preserves local structures, GAF encodes global correlations, and RP reveals system dynamics
    \item \textbf{Theoretical Complementarity:} The methods maintain theoretical orthogonality while targeting distinct aspects of temporal information
    \item \textbf{Robustness through Diversity:} The combination of different representations provides natural redundancy, mitigating individual limitations
    \item \textbf{Computational Efficiency:} All transformations leverage efficient matrix operations, making them practical for large-scale applications
\end{itemize}

Empirical evidence supports the effectiveness of this multi-view approach, demonstrating superior performance across diverse datasets and prediction horizons compared to single-transformation methods. This success validates our theoretical framework and suggests that comprehensive temporal feature extraction benefits significantly from the synergistic combination of complementary visual representations.