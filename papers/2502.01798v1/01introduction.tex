\section{Introduction}



In 2024, U.S. e-commerce sales are projected to reach approximately \$1.2 trillion~\citep{digitalcommerce2023ecommerce}, with users frequently engaging with websites that impose terms and conditions on financial transactions. While these terms are often benign, they can also facilitate scams or impose unfair consequences on unsuspecting users. This risk is heightened as most users rarely read these lengthy, jargon-filled terms~\citep{obar2020biggest, steinfeld2016agree, bakos2014does}, and are often not required to do so before completing a purchase.


In this work, we define unfavorable financial terms as those that are one-sided, imbalanced, unfair, or malicious, thereby disadvantaging users.~\autoref{fig:example} shows a real-world example of harmful financial terms on a website selling earbuds at seemingly attractive prices. When users make a purchase, the T\&Cs obligate the users to a fitness app subscription with a recurring \$86 monthly fee. \textit{This obligation is not disclosed at all during the purchase process}. 

Unfair financial terms can also exist on legitimate websites---unlike traditional social engineering scams, these terms may not be inherently deceptive but can still cause substantial losses.~\autoref{fig:example2} in Appendix~\ref{sec:case_studies} presents the T\&Cs from Celsius~\citep{celsiuswebsite}, a cryptocurrency company bankrupt in 2022. These terms stipulate that if Celsius goes bankrupt, users could lose digital investments since they would be treated as unsecured creditors. A judge later ruled that Celsius owned its users' cryptocurrency based on these terms~\citep{celsius}, highlighting the real financial risks such terms pose to users. 

It is worth noting that the website in~\autoref{fig:example} operated for at least a year without being flagged by major browsers before its shutdown in June 2024, showing the current defense ecosystem's lack of understanding and mitigation strategies for such \termname terms. Likewise, Celsius's unfair terms only gained attention during bankruptcy proceedings. Despite their impact, few mitigation methods exist for \termname terms.

A possible approach to addressing the concern is to extend the current methods for detecting social engineering scams and dark patterns~\citep{mathur2019dark, mathur2021makes, di2020ui, bongard2021definitely, li2023double, bitaab2023beyond, yang2023trident}. Unfortunately, such an extension is not straightforward. Many of these methods are not designed to detect \termname terms, as they typically focus on content-based features like word patterns, images, website structures, or external indicators like link length and certificates~\citep{bitaab2020scam, bilge2011exposure, kharraz2018surveylance, zhang2011textual, sakurai2020discovering, drury2019certified}, which are unrelated to the detection of \termname terms. Similarly, work on online agreement analysis focuses on privacy policies~\citep{xiang2023policychecker, harkous2018polisis, zhou2023policycomp, bui2023detection} or on terms deemed invalid under the EU law~\citep{lagioia2017automated, braun2021nlp, lippi2019claudette, limsopatham2021effectively, galassi2024unfair, galassi2020cross}.


The emergence of large language models (LLMs) offers a powerful tool for tackling real-world security challenges~\citep{van2022logomotive, li2018fake, zhang2014you, zheng2017smoke, sahingoz2019machine, bitaab2023beyond}, 
 including the analysis of policies like T\&Cs~\citep{rodriguez2024large, frasheri2024llm}, 
% The advent of large language models (LLMs) provides a more effective approach to analyzing policies like T\&Cs~\citep{rodriguez2024large, frasheri2024llm}, 
but their potential in this area has yet to be fully explored. This paper aims to fill this gap through a large-scale measurement study. To the best of our knowledge, this is the first systematic effort to categorize and detect \termname terms in real-world shopping websites. Our contributions are as follows:



\begin{itemize} 
    \item \myparagraph{Data collection and topic modeling pipeline} We present \textit{TermMiner}\footnote{\url{https://github.com/eltsai/term_miner}}, a scalable pipeline for collecting and analyzing terms and conditions from shopping websites. The pipeline comprises: (1) a T\&Cs collection module, (2) an LLM-based term classification module, and (3) an interactive topic modeling module. \Termname terms are identified using FTC's definitions of unfair practices~\citep{ftcact}.
    
    \item \myparagraph{ShopTC-100K dataset}: We create the \textit{ShopTC-100K} dataset\footnote{\url{https://huggingface.co/datasets/eltsai/ShopTC-100K}}, containing 1.8 million terms extracted from the terms and conditions of 8,251 shopping websites in the Tranco top 100K most popular websites.
    
    \item \myparagraph{\Termname term taxonomy}: We develop a comprehensive taxonomy for \termname terms, covering \termcatcnt categories and \termtypecnt types, including terms related to purchase, billing, post-purchase activities, account termination and recovery, and legal conditions.
    
    \item \myparagraph{\Termname terms detection}: We develop \textit{Term-Lens}, an LLM-based detection framework using LLMs to automatically identify \termname terms. With a fine-tuned GPT-4o model, \platform achieves a 94.6\% F1 score and a 2.3\% FPR on an annotated evaluation dataset. 
    
    \item \myparagraph{Measurement study}: We analyze \termcnt terms from \websitecnt online shopping websites, finding that \websitepct of English websites in the Tranco top 100k contain \termname terms. Our analysis reveals these terms are more common on less popular websites, with case studies highlighting the potential financial and legal harm to consumers. 
\end{itemize}

\begin{figure}[!t] 
 \centering
 \includegraphics[width=0.99\columnwidth]{imgs/scam_example.pdf}
 \caption{\textbf{\Termname term example} --- (a) shows the payment page for Tone Fit Pro, a now-defunct website, with no mention of the subscription service on the payment page. (b) displays its T\&Cs, stating customers are automatically enrolled in an \$86/month Fitness App subscription with auto-renewal. (c) shows a screenshot of real-life victim complaints.}
 \Description[Scam website example with hidden auto-renewal term.]
 {This figure illustrates an example of a deceptive financial term found in a scam website. 
 (a) The payment page for "Tone Fit Pro" lacks any mention of a subscription service.
 (b) The terms and conditions reveal that users are automatically enrolled in an \$86/month FitHabit Fitness App subscription with auto-renewal.
 (c) A compilation of real-world consumer complaints highlights user frustration and confusion over unexpected charges.}
\label{fig:example}
\end{figure}





