\section{\TermName Term Detection System}
\label{sec:detection_section}


\begin{figure}[!t] 
 \centering
 \includegraphics[width=0.99\columnwidth]{imgs/plugin_design.pdf}
 \caption{\textbf{\platform Design}---
  (1) When the user activates the plugin, the current page URL is sent to the backend. (2) The terms and conditions are crawled and combined with the page information. (3) The pluggable LLM module analyzes the data to identify unfavorable financial terms. (4) Alerts are generated and displayed on the front end to warn users of potentially unfair financial terms. }
\Description[Diagram of \platform plugin workflow.]
 {This figure illustrates the design of the \platform plugin, which detects unfavorable financial terms on shopping websites. 
 (1) When a user activates the plugin, it sends the current page URL to the backend.
 (2) The backend crawls and extracts the terms and conditions associated with the page.
 (3) A pluggable LLM module processes the extracted data to detect potentially unfair financial terms.
 (4) If any unfavorable terms are identified, the plugin generates an alert and displays it on the front end to warn the user.}
\label{fig:plugin}
\end{figure}


In this section, we introduce \platform, a Chrome plugin designed to detect \termname terms on e-commerce websites. Built upon the insights gained from the \termname term template and topic modeling analysis, \platform enables efficient identification of potentially harmful financial terms, providing users with real-time protection against \termname terms.


\subsection{System Overview}



Our detection system is illustrated in \autoref{fig:plugin}. When a user activates \platform, the URL of the current page is sent to the backend. Upon receipt, the backend crawler collects the terms and conditions pages. These term pages, along with the HTML content of the current page (and a screenshot if paired with a multimodal LLM), are preprocessed and sent to the pluggable LLM module for further analysis. If the LLM module flags any terms as \termname terms, the alert generator sends the identified terms back to the frontend, where they are displayed to the user.








\myparagraph{Pluggable LLM Module} 
We parse the current page to determine if it is a payment page, improving alert accuracy by cross-checking terms with payment page details. For example, in \autoref{fig:example}, the term ``You will be charged \$6.85 for the shipping and handling of your free smartwatch'' aligns with the payment page, making it less concerning than the Immediate Automatic Subscription term, ``you will receive a subscription to the FitHabit Fitness App for only \$86,'' which is not shown on the payment page.

The Pluggable LLM Module, a key part of our system, analyzes both terms and conditions pages and the current webpage. By keeping the LLM decoupled from the backend, we allow flexibility in integrating different models. This enables multimodal models like GPT-4, GPT-4o~\citep{openai2023gpt4}, or LLaMA 3.2~\citep{llama3.2-90B-vision} to process screenshots and terms, or text-based models such as GPT-3.5~\citep{gpt35}, LLaMA~\citep{touvron2023llama}, Mistral~\citep{jiang2023mistral}, or Gemma~\citep{team2024gemma} to analyze HTML and terms.


\begin{table}[t!]
    \centering
    \footnotesize
    \caption{Statistics of annotated datasets for fine-tuning and validation for each term category.}
    \label{tab:dataset_stats}
    \begin{tabular}{p{3.5cm} p{1.5cm} p {1.5cm}}
        \toprule
        \textbf{Type} & \textbf{Fine-tuning} & \textbf{Validation} \\
        \midrule
        
        
        Post-Purchase & 51 & 48 \\
        Legal & 30 & 30 \\
        Termination and Account Recovery & 15 & 16 \\
        Purchase and Billing & 32 & 32 \\
        \midrule
        Unfavorable Terms Combined & 128 & 126 \\
        \midrule
        Benign & 116 & 119 \\
        \midrule
        Total Count & 244 & 245 \\
        \bottomrule
    \end{tabular}
\end{table}



\myparagraph{Backend Core Module}
The alert generator receives flagged \termname terms and checks if the user is on a payment page. If so, it only flags terms not displayed on that page. GPT-4o analyzes page screenshots to mimic the user's experience and guard against adversarial text-based evasion. When the page is not a payment page, all flagged financial terms are shown. Since returns and refunds are rarely disclosed on payment pages, our evaluation in~\S\ref{sec:eva} focuses on scenarios where the user is not on a payment page and seeks to assess financial risks in advance.




