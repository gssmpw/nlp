\section{Related Work}


\myparagraph{Scam and fake e-commerce website detection}
Detection methods for scam and fake e-commerce websites (FCW) typically rely on two types of features: external (e.g., URLs, certificates, logos, redirect mechanisms)~\citep{blum2010lexical, zouina2017novel, moghimi2016new, sakurai2020discovering, drury2019certified, van2022logomotive, li2018fake, zhang2014you, zheng2017smoke, sahingoz2019machine, bitaab2023beyond} and content-based (e.g., visual and HTML structures, images, scripts, hyperlinks)~\citep{xiang2011cantina+, kharraz2018surveylance, yang2019phishing, jain2017phishing, bitaab2023beyond, yang2023trident}. These models are either rule-based or machine learning-based, with feature selection grounded in domain knowledge (e.g., indicative images, third-party scripts). However, no prior work in this line has considered terms and conditions and their financial impacts on users.


We consider social engineering scams to overlap with our detection target. The \termname terms in \autoref{fig:example} function similarly by deceiving users into signing up for additional subscriptions. However, as discussed in \S\ref{sec:financial_terms_section} and~\S\ref{sec:categoring}, \termname terms are not exclusive to scam websites. Therefore, consumers should be alerted to the presence of such terms. We view our work as the first to measure \termname terms at scale.

\begin{figure*}[t] 
 \centering
 \includegraphics[width=2\columnwidth]{imgs/data_collection_pipeline.pdf}
 \caption{\textbf{\textit{TermMiner} (data collection and topic modeling pipeline)}---(1) Measurement module: collects shopping websites from the Tranco list and fake e-commerce website datasets, extracting English terms and conditions from shopping websites. (2) Term classification module: classifies the terms into binary categories based on a given prompt. (3) Topic modeling module: leverages t5-base Sentence Transformer and DBSCAN for clustering. Topics are derived from the clusters using a combination of manual inspection and GPT-4o, employing a snowball sampling method~\citep{goodman1961snowball} to iteratively develop a topic template of terms.}
 \Description[Pipeline for collecting and analyzing unfavorable financial terms.]
 {This figure presents the \textit{TermMiner} pipeline for collecting and analyzing unfavorable financial terms from shopping websites. 
 (1) The measurement module gathers websites from the Tranco list and known fake e-commerce datasets, extracting English terms and conditions from shopping sites.
 (2) The term classification module processes extracted terms, categorizing them into binary labels based on a predefined prompt.
 (3) The topic modeling module applies the T5-base Sentence Transformer and DBSCAN clustering to group terms into topics. These topics are refined through a combination of manual inspection and GPT-4o, using a snowball sampling approach to iteratively develop a comprehensive topic taxonomy.}
\label{fig:financial_term_pipeline}
\end{figure*}



\myparagraph{Dark patterns}
Dark patterns are deceptive user interface designs intended to manipulate users into actions against their best interests~\citep{mathur2019dark}. Recent research has examined their psychological impact on user decision-making~\citep{mathur2021makes,nouwens2020dark,waldman2020cognitive,narayanan2020dark}, while also exploring legal frameworks and strategies for intervention~\citep{luguri2021shining,gray2021dark}.

Although terms and conditions are not part of the user interface design, we consider the \termname terms we identify to be closely related to dark patterns. The unilateral nature of these terms and their potential to hide uncommon or unexpected terms make them closely align with the characteristics of dark patterns: asymmetric, covert, deceptive, hiding information, and restrictive.



\myparagraph{Terms and conditions legal analysis} 
There is limited NLP-based analysis of legal documents like online contracts and terms of service~\citep{lagioia2017automated, braun2021nlp, lippi2019claudette, limsopatham2021effectively, jablonowska2021assessing, galassi2024unfair}. Prior studies, such as Lippi \etal~\citep{lippi2019claudette} and Galassi \etal~\citep{galassi2024unfair}, typically focus on small datasets of T\&Cs (25 to 200 documents). However, their focus is mainly on assessing fairness under the EU’s Unfair Contract Terms Directive~\citep{CouncilDirective1993} (i.e., clauses invalid in court). In contrast, our work specifically targets terms with more direct financial impacts on users.


In this paper, we focus on the financial terms in the large-scale measurement of terms and conditions from English shopping websites, assessed using the definition of unfair acts or practices as provided by the Federal Trade Commission (FTC)'s Policy Statement on Deception~\citep{ftc1983deception}. A detailed comparison of our proposed term taxonomy with prior work is provided in Appendix~\ref{sec:appendix_other_templates}.

\myparagraph{Privacy policy analysis}
A significant body of work investigates the viability of NLP-based analysis for privacy policies. One significant line of such research focuses on detecting contradicting policy statements (e.g., via ontologies~\citep{andow2019policylint} and knowledge graphs~\citep{cui2023poligraph}) or ambiguities~\citep{shvartzshnaider2019going}. Other areas include improving user comprehension~\citep{harkous2018polisis}, topic modeling, and summarization~\citep{alabduljabbar2021automated, sarne2019unsupervised}.

In this work, we focus on financial terms which are distinct from privacy policies. While we also perform topic modeling, we are the first to apply such a pipeline to construct a taxonomy for \termname terms. Furthermore, detecting contradictions and ambiguities is orthogonal to the detection of malicious financial terms, making it difficult to apply similar techniques directly.







