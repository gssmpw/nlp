\section{Conclusion}

This paper is one of the first attempts to understand, categorize, and detect \termname terms and conditions on shopping websites. These terms, which can significantly impact consumer trust and satisfaction, have not been extensively studied. By highlighting the prevalence and types of \termname terms, we hope to pave the way for increased awareness and further research in this area. We develop an automated data collection and topic modeling pipeline, analyzing \termcnt terms from \websitecnt websites to create a taxonomy for \termname terms. This taxonomy includes \termtypecnt types across \termcatcnt categories, covering purchase and billing, post-purchase activities, account termination, and legal aspects.


\platform is the first study to evaluate the effectiveness of LLMs in identifying \termname terms. Using a fine-tuned GPT-4o model on a manually annotated dataset, \platform achieves an F1 score of 94.6\% with a false positive rate of 2.3\%. In large-scale deployment, we find that approximately \websitepct of shopping websites in the Tranco top 100,000 contain at least one category of \termname terms. Our qualitative analysis shows that current ecosystem defenses are inadequate to protect users from these terms, that less popular websites are more likely to include \termname terms, and that there is a correlation between \termname terms and user dissatisfaction.





% Ack: thank openai and renuka
\section*{Acknowledgments}

The authors would like to thank Renuka Kumar for her valuable input and suggestions during the early stages of this project.  This work is supported by a gift from the OpenAI Cybersecurity Grant program. Any opinions, findings, conclusions, or recommendations expressed in this material are those of the authors
and do not necessarily reflect the views of OpenAI.

