\documentclass[sigconf]{acmart}

\usepackage{soul}
\usepackage{listings}

\lstset{
    basicstyle=\ttfamily\small,  
    breaklines=true,            
    frame=single,               
    backgroundcolor=\color{gray!10},  
    rulecolor=\color{black},      
    xleftmargin=0pt,            
    framexleftmargin=0pt,       
    showstringspaces=false,     
    fontadjust=true  % <- Ensures it follows document-wide font settings
}


\usepackage{multirow}
\usepackage{subfigure}
\usepackage{xspace}
\usepackage{rotating}
\usepackage{algorithm}



\usepackage{algorithmic}
\usepackage{mathtools}


%math
\usepackage{amsmath}% http://ctan.org/pkg/amsmath

\newcommand{\eg}{e.g.\@\xspace}
\newcommand{\ie}{i.e.\@\xspace}
\newcommand{\etal}{et~al.\@\xspace}
\newcommand{\etc}{{\em etc}\xspace}
\newcommand{\TK}{{\bf TK}\xspace}
\newcommand{\tk}{\TK}

\newcommand{\financialcnt}{12\xspace}


\newcommand{\platform}{\textit{TermLens}\xspace}
\newcommand{\termname}{unfavorable financial\xspace}
\newcommand{\TermName}{Unfavorable Financial\xspace}
\newcommand{\Termname}{Unfavorable financial\xspace}
\newcommand{\websitecnt}{8,979\xspace}
\newcommand{\termcnt}{1.9 million\xspace}
\newcommand{\websitepct}{42.06\%\xspace}
\newcommand{\termpct}{0.5\%\xspace}
\newcommand{\termtypecnt}{22\xspace}
\newcommand{\termcatcnt}{4\xspace}
\newcommand{\myparagraph}[1]{\textbf{\textit{#1}:}\hspace{3pt}}


\usepackage{pifont} % Approved package
\newcommand{\filledCircle}{\ding{108}} % ● Full circle
\newcommand{\halfFilledCircle}{\ding{109}} % ◐ Half-filled circle



\copyrightyear{2025}
\acmYear{2025}
\setcopyright{cc}
\setcctype{by}
\acmConference[WWW '25]{Proceedings of the ACM Web Conference 2025}{April 28-May 2, 2025}{Sydney, NSW, Australia}
\acmBooktitle{Proceedings of the ACM Web Conference 2025 (WWW '25), April 28-May 2, 2025, Sydney, NSW, Australia}
\acmDOI{10.1145/3696410.3714573}
\acmISBN{979-8-4007-1274-6/25/04}

\settopmatter{printacmref=true}


%\showthe\font



\begin{document}

%%
%% The "title" command has an optional parameter,
%% allowing the author to define a "short title" to be used in page headers.
\title[Harmful Terms and Where to Find Them]{Harmful Terms and Where to Find Them: Measuring and Modeling Unfavorable Financial Terms and Conditions in Shopping Websites at Scale}


\author{Elisa Tsai}
\affiliation{%
  \institution{University of Michigan}
  \city{Ann Arbor}
  \state{Michigan}
  \country{USA}
}
\email{eltsai@umich.edu}


\author{Neal Mangaokar}
\affiliation{
  \institution{University of Michigan}
  \city{Ann Arbor}
  \state{Michigan}
  \country{USA}
}
\email{nealmgkr@umich.edu}


\author{Boyuan Zheng}
\affiliation{
  \institution{University of Michigan}
  \city{Ann Arbor}
  \state{Michigan}
  \country{USA}
}
\email{boyuann@umich.edu}

\author{Haizhong Zheng}
\affiliation{
  \institution{University of Michigan}
  \city{Ann Arbor}
  \state{Michigan}
  \country{USA}
}
\email{hzzheng@umich.edu}

\author{Atul Prakash}
\affiliation{
  \institution{University of Michigan}
  \city{Ann Arbor}
  \state{Michigan}
  \country{USA}
}
\email{aprakash@umich.edu}



%%
%% By default, the full list of authors will be used in the page
%% headers. Often, this list is too long, and will overlap
%% other information printed in the page headers. This command allows
%% the author to define a more concise list
%% of authors' names for this purpose.

%%
%% The abstract is a short summary of the work to be presented in the
%% article.
\begin{abstract}
Terms and conditions for online shopping websites often contain terms that can have significant financial consequences for customers. 
Despite their impact, there is currently no comprehensive understanding of the types and potential risks associated with unfavorable financial terms. Furthermore, there are no publicly available detection systems or datasets to systematically identify or mitigate these terms.
In this paper, we take the first steps toward solving this problem with three key contributions.

\textit{First}, we introduce \textit{TermMiner}, an automated data collection and topic modeling pipeline to understand the landscape of unfavorable financial terms.
\textit{Second}, we create \textit{ShopTC-100K}, a dataset of terms and conditions from shopping websites in the Tranco top 100K list, comprising 1.8 million terms from 8,251 websites. Consequently, we develop a taxonomy of 22 types from 4 categories of unfavorable financial terms---spanning purchase, post-purchase, account termination, and legal aspects.
\textit{Third}, we build \textit{TermLens}, an automated detector that uses Large Language Models (LLMs) to identify unfavorable financial terms. 

Fine-tuned on an annotated dataset, \textit{TermLens} achieves an F1 score of 94.6\% and a false positive rate of 2.3\% using GPT-4o. 
When applied to shopping websites from the Tranco top 100K, we find that 42.06\% of these sites contain at least one unfavorable financial term, with such terms being more prevalent on less popular websites. Case studies further highlight the financial risks and customer dissatisfaction associated with unfavorable financial terms, as well as the limitations of existing ecosystem defenses.

\end{abstract}

%%
%% The code below is generated by the tool at http://dl.acm.org/ccs.cfm.
%% Please copy and paste the code instead of the example below.
%%

\begin{CCSXML}
<ccs2012>
   <concept>
       <concept_id>10002951.10003260.10003277</concept_id>
       <concept_desc>Information systems~Web mining</concept_desc>
       <concept_significance>500</concept_significance>
       </concept>
   <concept>
       <concept_id>10002978.10002997.10003000</concept_id>
       <concept_desc>Security and privacy~Social engineering attacks</concept_desc>
       <concept_significance>300</concept_significance>
       </concept>
   <concept>
       <concept_id>10003456.10003462.10003544.10011709</concept_id>
       <concept_desc>Social and professional topics~Consumer products policy</concept_desc>
       <concept_significance>500</concept_significance>
       </concept>
 </ccs2012>
\end{CCSXML}

\ccsdesc[500]{Information systems~Web mining}
\ccsdesc[300]{Security and privacy~Social engineering attacks}
\ccsdesc[500]{Social and professional topics~Consumer products policy}



%%
%% Keywords. The author(s) should pick words that accurately describe
%% the work being presented. Separate the keywords with commas.
\keywords{Topic modeling; Unfavorable financial terms; Consumer protection; Terms and conditions dataset; Deceptive content}

%% This command processes the author and affiliation and title
%% information and builds the first part of the formatted document.
\maketitle



%\section{CCS Concepts and User-Defined Keywords}

%Two elements of the ``acmart'' document class provide powerful
%taxonomic tools for you to help readers find your work in an online search.

%The ACM Computing Classification System ---
%\url{https://www.acm.org/publications/class-2012} --- is a set of
%classifiers and concepts that describe the computing
%discipline. Authors can select entries from this classification
%system, via \url{https://dl.acm.org/ccs/ccs.cfm}, and generate the
%commands to be included in the \LaTeX\ source.

%User-defined keywords are a comma-separated list of words and phrases
%of the authors' choosing, providing a more flexible way of describing
%the research being presented.

%CCS concepts and user-defined keywords are required for for all
%articles over two pages in length, and are optional for one- and
%two-page articles (or abstracts).



%%
%% The next two lines define the bibliography style to be used, and
%% the bibliography file.


%%%%%%%%%%%%%%%%%%%%%%%%%%%%%%%%%%%%%%
% Main text
%%%%%%%%%%%%%%%%%%%%%%%%%%%%%%%%%%%%%%

Large language models (LLMs) show significant performance in various downstream
tasks~\citep{brown_language_2020,openai_gpt-4_2024,dubey_llama_2024}. Studies
have found that training on high quality corpus improves the ability of LLMs
to solve different problems such as writing code, doing math exercises, and
answering logic questions~\citep{cai_internlm2_2024,deepseek-ai_deepseek-v3_2024,qwen_qwen25_2024}.
Therefore, effectively selecting high-quality text data is an important subject for
training LLM.

\begin{figure}[t]
    \centering
    \includegraphics[width=\linewidth]{figures/head.pdf}
    \caption{The overview of CritiQ. We (1) employ human annotators to annotate $\sim$30
    pairwise quality comparisons, (2) use CritiQ Flow to mine quality criteria, (3)
    use the derived criteria to annotate 25k pairs, and (4) train the CritiQ Scorer to
    perform efficient data selection.}
    \label{fig:overview}
\end{figure}

To select high-quality data from a large corpus, researchers manually design heuristics~\citep{dubey_llama_2024,rae_scaling_2022},
calculate perplexity using existing LLMs~\citep{marion2023moreinvestigatingdatapruning,wenzek2019ccnetextractinghighquality},
train classifiers~\citep{brown_language_2020,dubey_llama_2024,xie_data_2023} and
query LLMs for text quality through careful prompt engineering~\citep{gunasekar_textbooks_2023,wettig_qurating_2024,sachdeva_how_2024}.
Large-scale human annotation and prompt engineering require a lot of human
effort. Giving a comprehensive description of what high-quality data is like is also
challenging. As a result, manually designing heuristics lacks robustness and introduces
biases to the data processing pipeline, potentially harming model performance
and generalization. In addition, quality standards vary across different
domains. These methods can not be directly applied to other domains without significant
modifications.

To address these problems, we introduce CritiQ, a novel method to automatically
and effectively capture human preferences for data quality and perform efficient data
selection. Figure~\ref{fig:overview} gives an overview of CritiQ, comprising an agent
workflow, CritiQ Flow, and a scoring model, CritiQ Scorer. Instead of manually describing
how high quality is defined, we employ LLM-based agents to summarize quality
criteria from only $\sim$30 human-annotated pairs.

CritiQ Flow starts from a knowledge base of data quality criteria. The worker
agents are responsible to perform pairwise judgment under a given
criterion. The manager agent generates new criteria and refines them through reflection
on worker agents' performance. The final judgment is made by majority voting among
all worker agents, which gives a multi-perspective view of data quality.

To perform efficient data selection, we employ the worker agents to annotate a randomly
selected pairwise subset, which is ~1000x larger than the human-annotated one.
Following \citet{korbak_pretraining_2023,wettig_qurating_2024}, we train CritiQ
Scorer, a lightweight Bradley-Terry model~\citep{bradley_rank_1952} to convert
pairwise preferences into numerical scores for each text. We use CritiQ Scorer to
score the entire corpus and sample the high-quality subset.

For our experiments, we established human-annotated test sets to quantitatively
evaluate the agreement rate with human annotators on data quality preferences. We implemented the manager agent by \texttt{GPT-4o} and the worker
agent by \texttt{Qwen2.5-72B-Insruct}. We conducted experiments on different
domains including code, math, and logic, in which CritiQ Flow shows a consistent
improvement in the accuracies on the test sets, demonstrating the effectiveness
of our method in capturing human preferences for data quality. To validate the quality
of the selected dataset, we continually train \texttt{Llama 3.1}~\citep{dubey_llama_2024}
models and find that the models achieve better performance on downstream tasks
compared to models trained on the uniformly sampled subsets.

We highlight our contributions as follows. We will release the code to facilitate
future research.

\begin{itemize}
    \item We introduce CritiQ, a method that captures human preferences for data
        quality and performs efficient data selection at little cost of human
        annotation effort.

    \item Continual pretraining experiments show improved model performance in code,
        math, and logic tasks trained on our selected high-quality subset compared to the raw dataset.

    \item Ablation studies demonstrate the effectiveness of the knowledge base and
        the the reflection process.
\end{itemize}

\begin{figure*}[t]
    \centering
    \includegraphics[width=\linewidth]{figures/method.pdf}
    \caption{CritiQ Flow comprises two major components: multi-criteria pairwise
    judgment and the criteria evolution process. The multi-criteria pairwise
    judgment process employs a series of worker agents to make quality
    comparisons under a certain criterion. The criteria evolution process aims to
    obtain data quality criteria that highly align with human judgment through
    an iterative evolution. The initial criteria are retrieved from the
    knowledge base. After evolution, we select the final criteria to annotate
    the dataset for training CritiQ Scorer.}
    \label{fig:method}
\end{figure*}

\section{Related Work}


\myparagraph{Scam and fake e-commerce website detection}
Detection methods for scam and fake e-commerce websites (FCW) typically rely on two types of features: external (e.g., URLs, certificates, logos, redirect mechanisms)~\citep{blum2010lexical, zouina2017novel, moghimi2016new, sakurai2020discovering, drury2019certified, van2022logomotive, li2018fake, zhang2014you, zheng2017smoke, sahingoz2019machine, bitaab2023beyond} and content-based (e.g., visual and HTML structures, images, scripts, hyperlinks)~\citep{xiang2011cantina+, kharraz2018surveylance, yang2019phishing, jain2017phishing, bitaab2023beyond, yang2023trident}. These models are either rule-based or machine learning-based, with feature selection grounded in domain knowledge (e.g., indicative images, third-party scripts). However, no prior work in this line has considered terms and conditions and their financial impacts on users.


We consider social engineering scams to overlap with our detection target. The \termname terms in \autoref{fig:example} function similarly by deceiving users into signing up for additional subscriptions. However, as discussed in \S\ref{sec:financial_terms_section} and~\S\ref{sec:categoring}, \termname terms are not exclusive to scam websites. Therefore, consumers should be alerted to the presence of such terms. We view our work as the first to measure \termname terms at scale.

\begin{figure*}[t] 
 \centering
 \includegraphics[width=2\columnwidth]{imgs/data_collection_pipeline.pdf}
 \caption{\textbf{\textit{TermMiner} (data collection and topic modeling pipeline)}---(1) Measurement module: collects shopping websites from the Tranco list and fake e-commerce website datasets, extracting English terms and conditions from shopping websites. (2) Term classification module: classifies the terms into binary categories based on a given prompt. (3) Topic modeling module: leverages t5-base Sentence Transformer and DBSCAN for clustering. Topics are derived from the clusters using a combination of manual inspection and GPT-4o, employing a snowball sampling method~\citep{goodman1961snowball} to iteratively develop a topic template of terms.}
 \Description[Pipeline for collecting and analyzing unfavorable financial terms.]
 {This figure presents the \textit{TermMiner} pipeline for collecting and analyzing unfavorable financial terms from shopping websites. 
 (1) The measurement module gathers websites from the Tranco list and known fake e-commerce datasets, extracting English terms and conditions from shopping sites.
 (2) The term classification module processes extracted terms, categorizing them into binary labels based on a predefined prompt.
 (3) The topic modeling module applies the T5-base Sentence Transformer and DBSCAN clustering to group terms into topics. These topics are refined through a combination of manual inspection and GPT-4o, using a snowball sampling approach to iteratively develop a comprehensive topic taxonomy.}
\label{fig:financial_term_pipeline}
\end{figure*}



\myparagraph{Dark patterns}
Dark patterns are deceptive user interface designs intended to manipulate users into actions against their best interests~\citep{mathur2019dark}. Recent research has examined their psychological impact on user decision-making~\citep{mathur2021makes,nouwens2020dark,waldman2020cognitive,narayanan2020dark}, while also exploring legal frameworks and strategies for intervention~\citep{luguri2021shining,gray2021dark}.

Although terms and conditions are not part of the user interface design, we consider the \termname terms we identify to be closely related to dark patterns. The unilateral nature of these terms and their potential to hide uncommon or unexpected terms make them closely align with the characteristics of dark patterns: asymmetric, covert, deceptive, hiding information, and restrictive.



\myparagraph{Terms and conditions legal analysis} 
There is limited NLP-based analysis of legal documents like online contracts and terms of service~\citep{lagioia2017automated, braun2021nlp, lippi2019claudette, limsopatham2021effectively, jablonowska2021assessing, galassi2024unfair}. Prior studies, such as Lippi \etal~\citep{lippi2019claudette} and Galassi \etal~\citep{galassi2024unfair}, typically focus on small datasets of T\&Cs (25 to 200 documents). However, their focus is mainly on assessing fairness under the EU’s Unfair Contract Terms Directive~\citep{CouncilDirective1993} (i.e., clauses invalid in court). In contrast, our work specifically targets terms with more direct financial impacts on users.


In this paper, we focus on the financial terms in the large-scale measurement of terms and conditions from English shopping websites, assessed using the definition of unfair acts or practices as provided by the Federal Trade Commission (FTC)'s Policy Statement on Deception~\citep{ftc1983deception}. A detailed comparison of our proposed term taxonomy with prior work is provided in Appendix~\ref{sec:appendix_other_templates}.

\myparagraph{Privacy policy analysis}
A significant body of work investigates the viability of NLP-based analysis for privacy policies. One significant line of such research focuses on detecting contradicting policy statements (e.g., via ontologies~\citep{andow2019policylint} and knowledge graphs~\citep{cui2023poligraph}) or ambiguities~\citep{shvartzshnaider2019going}. Other areas include improving user comprehension~\citep{harkous2018polisis}, topic modeling, and summarization~\citep{alabduljabbar2021automated, sarne2019unsupervised}.

In this work, we focus on financial terms which are distinct from privacy policies. While we also perform topic modeling, we are the first to apply such a pipeline to construct a taxonomy for \termname terms. Furthermore, detecting contradictions and ambiguities is orthogonal to the detection of malicious financial terms, making it difficult to apply similar techniques directly.








\section{Understanding Unfavorable Financial Terms}
\label{sec:topic_modeling_section}

In this section, we outline our detection goal and present \textit{TermMiner}, a pipeline for collecting, clustering, and topic modeling \termname terms from English shopping websites in the Tranco top 100,000~\citep{tranco} and datasets of fraudulent e-commerce sites~\citep{bitaab2023beyond, janaviciute2023fraudulent}. As shown in \autoref{fig:financial_term_pipeline}, the pipeline categorizes two types of terms: (1) financial terms that may have immediate or future financial impacts, and (2) \termname terms, identified as unfair, unfavorable, or concerning for customers. We then summarize the taxonomy of \termname terms, which fall into four broad categories: (1) purchase and billing, (2) post-purchase, (3) termination and account recovery, and (4) legal terms.

\subsection{Threat Model}
\label{ssec:ftc}

We aim to detect one-sided, imbalanced, unfair, or malicious \textit{financial} terms in online shopping websites' terms and conditions, which could pose significant risks to users, potentially leading to unexpected financial losses. These risks can arise from website operators seeking to limit liability or from intentional malfeasance.

To assess whether a financial term is unfavorable, we refer to Section 5 of the Federal Trade Commission (FTC) Act~\citep{ftcact}, which defines an act as unfair if it meets the following criteria:

\begin{itemize}
    \item \textbf{C1: Substantial Injury}. It causes or is likely to cause substantial injury to consumers;
    \item \textbf{C2: Unavoidable Harm}. Consumers cannot reasonably avoid it; and
    \item \textbf{C3: Insufficient Benefits}. It is not outweighed by countervailing benefits to consumers or competition. 
\end{itemize}

During the topic modeling of term clusters, we judge the topic representing each cluster by three criteria to evaluate their fairness.

\myparagraph{C1} Since we focus on financial terms with potentially detrimental impacts, all financial terms inherently satisfy this criterion.


\myparagraph{C2} Terms and conditions are often hidden or difficult to avoid. Fair financial terms must be clearly displayed at critical points, like the payment page. However, terms related to cancellation, refunds, and returns are rarely shown upfront. We evaluate terms for unexpected fees (e.g., cancellation charges, non-refundable items, costly returns) that place an undue burden on consumers.


\myparagraph{C3} We classify terms as benign if they serve legitimate user or business protection, such as terms prohibiting fraud or abuse, protecting intellectual property, or ensuring legal compliance.



\subsection{Data Collection and Topic Modeling}
\label{ssec:data_collection_and_topic_modeling}

As shown in~\autoref{fig:financial_term_pipeline}, we introduce \textit{TermMiner}, a data collection and topic modeling pipeline for identifying \termname term at scale.  By integrating LLMs like GPT-4o~\citep{openai2023gpt4}, \textit{TermMiner} significantly reduces the extensive manual efforts required in previous web content mining studies, such as those focused on detecting dark patterns~\citep{mathur2019dark}. 
\textit{TermMiner} is open-sourced and can be repurposed for various web-based text analysis tasks or longitudinal studies. Researchers can use our tools to explore different aspects of terms and conditions, such as readability, accessibility, or fairness.


\myparagraph{A Two-Pass Method}
In the data collection and topic modeling steps, we employ a two-pass method. The first pass focuses on modeling and detecting \textit{financial terms} to develop a corresponding topic template. In the second pass, we use the detected financial terms to re-conduct the classification and topic modeling modules. This time, the goal is to detect \termname terms within the financial terms identified. This approach is necessary because, to the best of our knowledge, there are no established templates or annotation schemes for (1) financial terms or (2) \termname terms in online shopping agreements. This two-pass process ensures comprehensive detection and accurate categorization of both financial and \termname terms.


\myparagraph{(1) Measurement Module} The measurement module collects terms and conditions from shopping websites to build a large, diverse dataset for analysis. For our large-scale measurement, we collect English shopping websites from two sources: the Tranco list~\citep{tranco}, a ranking of top global websites, and two fraudulent e-commerce datasets (FCWs~\citep{bitaab2023beyond} and the Fraudulent and Legitimate Online Shops Dataset~\citep{janaviciute2023fraudulent}). We filter out non-English content using Python's langdetect library~\citep{langdetect}. To classify shopping websites, we evaluate several configurations: (1) GPT-3.5-Turbo~\citep{gpt35} with URL, (2) GPT-3.5-Turbo with URL and HTML content, (3) GPT-4o~\citep{openai2023gpt4} with URL, and (4) GPT-4o with URL and website screenshot. To evaluate our classification methods, we manually annotated a sample of 500 websites from the Tranco list, categorizing them into ``shopping'' and ``non-shopping.'' GPT-4o, when prompted with URLs and screenshots, achieved an accuracy of 92\%, comparable to commercial website classification services~\citep{mathur2019dark} (see Appendix~\ref{sec:website_cls} for details). Therefore, we use this configuration throughout our work.





We subsequently crawl the shopping websites to collect terms and conditions pages. A snowballed regex matching method detects terms and any nested policy pages, refined through positive and negative regex patterns to improve accuracy. Starting with common anchor texts, we iteratively refine the regex patterns by analyzing T\&C links, which can be found in Appendix~\ref{sec:appendix_reg}. As shown in~\autoref{table:dataset_stats}, we collected \termcnt terms from \websitecnt websites in total. 



\myparagraph{(2) Classfication Module}
The classification module categorizes terms from shopping websites' terms and conditions into binary categories: positive or negative. The categorization is based on the detection goal (such as identifying financial terms or identifying \termname terms) using corresponding prompts with the GPT-4o model~\citep{openai2023gpt4}.




We opt for prompt engineering instead of fine-tuning the LLM for term classification to reduce costs. Prior work~\citep{sun2023text,openai2023bestpractices}, along with our empirical observation (see~\S\ref{sec:eva}), indicates that clear task descriptions and relevant examples (taxonomy) significantly enhance LLM performance in text classification. For \termname terms, we use the ``Unfavorable Term Taxonomy Prompt'' from Appendix~\ref{sec:prompts} and topics identified in the topic modeling step, to perform zero-shot term classification on a given set of terms and conditions. This process outputs sets of positive and negative terms, which are then used for clustering, inspection, and topic modeling. The resulting template generated from this analysis will, in turn, enhance the classification accuracy, creating a feedback loop that continuously improves our detection capabilities. 

\sisetup{table-text-alignment=center,table-format=6.0}
\begin{tabular}{lSSSSrrS}
\toprule
{\textbf{Attribute}} & {\textbf{Declarations}} & {\textbf{Users}} & {\textbf{Inactive Users}} & {\textbf{Ambiguous Users}} & \multicolumn{2}{c}{\textbf{Class Proportion}} & {\textbf{Subreddits}} \\
\midrule
Year of Birth & 420803 & 401390 & 1630 & 17341 & Old: 56.19\% & Young: 43.81\% & 9806 \\
Gender & 424330 & 403428 & 1634 & 18337 & Male: 50.89\% & Female: 49.11\% & 9809 \\
Partisan Affiliation & 6369 & 6118 & 4 & 251 & Dem.: 54.55\% & Rep.: 45.45\% & 9137 \\
\bottomrule
\end{tabular}


\myparagraph{(3) Topic Modeling Module} The topic modeling module uses LLMs and manual inspection to organize terms into meaningful topics. We generate sentence embeddings with the T5 model~\citep{raffel2020exploring} and apply the DBSCAN clustering algorithm~\citep{ester1996density} to group terms by semantic similarity. The DBSCAN hyperparameters are decided through manual inspection. To extract high-frequency topics, we leverage GPT-4o~\citep{openai2023gpt4}, building on recent findings that show LLMs outperform traditional topic modeling methods like Latent Dirichlet Allocation (LDA)~\citep{blei2003latent} and BERTopic~\citep{grootendorst2022bertopic} in topic analysis~\citep{shrestha2023we, mu2024large}.



We develop an iterative topic modeling approach assisted by GPT-4o proceeds as follows: 
\begin{enumerate}
    \item We analyze DBSCAN clusters and create an initial topic template for financial terms.
    \item GPT-4o performs topic modeling on random samples from each cluster, assigning them to existing topics or suggesting new ones.
    \item We review and refine new topic suggestions through manual inspection, and updating the template.
    \item This process iterates until all clusters are assigned to a meaningful and satisfactory topic.
\end{enumerate}


This iterative workflow, combining clustering, human-guided template creation, and GPT-4o's advanced topic modeling capacity, enables efficient and comprehensive extraction of the topic template. We analyze 22,112 clusters in total, creating the \termname term taxonomy below.

\begin{table*}[t!]
    \footnotesize
    \centering
    \caption{{Categories, types, and examples of \termname terms are clustered, extracted, and topic-modeled from \termcnt terms across \websitecnt websites. All examples are extracted as-is from real-world shopping websites. The criteria are as follows: C1 = ``Substantial Injury'' (the term causes or is likely to cause substantial injury to consumers), C2 = ``Unavoidable Harm'' (consumers cannot reasonably avoid it), C3 = ``Insufficient Benefits'' (it is not outweighed by countervailing benefits to consumers or competition). The symbols represent the likelihood of satisfaction of a given criterion: \filledCircle = Always, \halfFilledCircle = Sometimes.}}
\begin{tabular*}{1.96\columnwidth}{p{1cm} p{4cm} p{9cm}  p{0.2cm} p{0.2cm} p{0.2cm}}
    \toprule
    \textbf{Category} & \textbf{Type} & \textbf{Example} & C1 & C2 & C3 \\
    \midrule

    \multirow{11}{*}{\shortstack{Purchase\\and\\Billing\\Terms}}

    
     & \shortstack{Immediate Automatic Subscription} 
    & {Also, as part of the \hl{promotion}, you will receive a \hl{subscription to the FitHabit Fitness App for only \$86}, and the subscription will renew monthly up until cancellation.} 
    & \filledCircle & \halfFilledCircle & \filledCircle \\
    \cmidrule(lr){2-6}
    & {Automatic Subscription after Free Trial}
    & {After the Promotion period has ended, unless you cancel the service before the end of the free trial period, you will \hl{automatically be subscribed} onto the regular paid 1-year plan at the price of \$275.40, which will automatically renew for successive 12-month periods, until cancelled.}
     & \filledCircle &  \halfFilledCircle & \halfFilledCircle\\
    \cmidrule(lr){2-6}
    & {Unilateral Unauthorized Account Upgrades} 
    & {Brevo reserves the right to automatically increase the contacts limit in the User account and \hl{upgrade the User’s plan without prior notice}. } 
     & \filledCircle & \filledCircle & \filledCircle \\
    \cmidrule(lr){2-6}
    & {Late or Unsuccessful Payment Penalty}
    & {In addition, if any payment is not received within \hl{30 days after the due date}, then we may charge a \hl{late fee} of \$10 and we may assess interest at the rate of 1.5\% of the outstanding balance per month (18\% per year), or the maximum rate permitted by law.}
     & \filledCircle & \halfFilledCircle & \halfFilledCircle \\
    \cmidrule(lr){2-6}
    & {Overuse Penalty}
    & {If the Company establishes limits on the frequency with which you may access the Site, or terminates your access to or use of the Site, you agree to pay the Company one hundred dollars (\$100) for each message posted \hl{in excess of such limits} or for each day on which you access the Site in excess of such limits, whichever is higher.}
     & \filledCircle & \halfFilledCircle & \halfFilledCircle \\
    \cmidrule(lr){2-6}
    & {Retroactive Application of Price Change} 
    & {When an applicable exchange rate is updated or when a change of price is notified to Brevo by its suppliers or WhatsApp, Brevo might \hl{immediately apply with retroactive effect} the new Ratio and price increase to the User.} 
     & \filledCircle & \filledCircle & \halfFilledCircle \\
    
    
    \midrule
    \multirow{11}{*}{\shortstack{Post-\\Purchase\\Terms}}
    & {Non-Refundable Subscription Fee}
    & {If you or we cancel your subscription, you are \hl{not entitled to a refund of any subscription fees} that were already charged for a subscription period that has already begun.}
     & \halfFilledCircle & \halfFilledCircle & \filledCircle \\
    \cmidrule(lr){2-6}
    & {No Refund For Purchase} 
    & {Unless a refund is required by law, there are \hl{No Refund For Purchases for POS terminals} and all transactions are final.}
     & \halfFilledCircle & \halfFilledCircle &  \halfFilledCircle \\
     
    \cmidrule(lr){2-6}
    & {Strict No Cancellation Policy}
    & {As Research and Markets starts processing your order once it is submitted, we operate a \hl{strict no cancellation policy}.}
     & \halfFilledCircle & \halfFilledCircle &  \halfFilledCircle \\


    \cmidrule(lr){2-6}
    & {\shortstack{Cancellation Fee or Penalty}}
    & {Some Bookings \hl{can’t be canceled for free}, while others can only be canceled for free before a deadline.}
     & \halfFilledCircle & \halfFilledCircle & \halfFilledCircle\\
    \cmidrule(lr){2-6}
    & Non-Refundable Additional Fee
    & {For this service, National Park Reservations charges a \hl{10\% non-refundable reservation fee} based on the total dollar amount of reservations made. }
     & \halfFilledCircle & \halfFilledCircle & \halfFilledCircle\\
    

    \cmidrule(lr){2-6}
    & {Non-Monetary Refund Alternatives} 
    & {Refund Policy: Refunds are not in cash but in the form of a \hl{``coupon’’}.}
    & \halfFilledCircle & \halfFilledCircle & \filledCircle \\
    \cmidrule(lr){2-6}
    
    & No Responsibility for Delivery Delays 
    & {We will \hl{not be held responsible} if there are \hl{delays in delivery} due to out-of-stock products. } 
    & \halfFilledCircle & \halfFilledCircle & \filledCircle \\
    \cmidrule(lr){2-6}
    & {Customers Responsible for Shipping Issues} 
    &  {If the parcel is on hold by the Customs department of the shipping country, \hl{the customer is liable} to provide all relevant and required documentation on to the authorities. Asim Jofa is \hl{not liable to refund} the amount in case of \hl{non-clearance of the parcel}.} 
    & \halfFilledCircle & \halfFilledCircle & \filledCircle\\
    \cmidrule(lr){2-6}
    & Customers Pay Return Shipping 
    & {All shipping costs will have to be borne by the customer.} & \halfFilledCircle & \halfFilledCircle & \filledCircle \\
    \cmidrule(lr){2-6}
    %& {Return Late Fee}
    %& {If you do not return your rented textbook on or before your rental period's Due Date for any reason (including if the textbook is lost or stolen), Cengage may charge a late fee. } \\
    %\cmidrule(lr){2-3}
    & {\shortstack{Restocking Fee}}
    & {An \hl{8\% restocking fee} and shipping fees for both ways will be borne by the buyer if returned without defects within 30 days from the purchase date or 7 days from delivery date, whichever is later.}
    & \halfFilledCircle & \halfFilledCircle & \filledCircle \\
    
    
    \midrule
    \multirow{3}{*}{\shortstack{Termination\\and\\Account\\Recovery\\Terms}} 
    & {Account Recovery Fee}
    &  {To recover an archived or locked account, the legitimate creator of the account shall provide verifiable information about one's identity and will be charged a \hl{10\% administrative fee} for the additional work caused by the account recovery process.} 
    & \halfFilledCircle & \halfFilledCircle & \halfFilledCircle\\
    
    \cmidrule(lr){2-6}
    & {Digital Currency, Reward, Money Seizure on Inactivity}
    & {Please be noted that if your account is \hl{dormant} for a period of 12 consecutive calendar months or longer, ..., any amounts in your account’s balance, including any outstanding fees owed to you, shall be considered as \hl{forfeited and shall be fully deducted} to Appnext.}
    & \halfFilledCircle & \halfFilledCircle & \filledCircle \\
    
    \cmidrule(lr){2-6}
    & {Digital Currency, Reward, Money Seizure on Termination or Account Closure} 
    &  {All Currency and/or Virtual Goods shall be \hl{cancelled} if Your account is \hl{terminated} or suspended for any reason or if We discontinue providing the Games and we will not compensate you for this loss or make any refund to you.}
    & \halfFilledCircle & \halfFilledCircle & \halfFilledCircle\\

    
    %\cmidrule(lr){2-3}
    %& Unsuccessful Payment Penalties 
    %& {In the event of an unsuccessful recurring payment, an \hl{administration fee of up to \$3.00} may be applied in order to keep a subscription temporarily active until the full subscription fee can be processed successfully.}\\

    \midrule
    \multirow{4}{*}{\shortstack{Legal\\Terms}}
    & Exorbitant Legal Document Request Fee 
    & {Responding to requests for production of documents, and other matters requiring more than mere ministerial activities on our part, will incur a fee of \hl{two hundred dollars (\$200) per hour.}}
    & \halfFilledCircle & \halfFilledCircle & \halfFilledCircle\\
    \cmidrule(lr){2-6} 
    &  Forced Waiver of Legal Protections
    &  {You hereby \hl{waive California Civil Code Section 1542}. You hereby waive any similar provision in law, regulation, or code.}
    & \halfFilledCircle & \halfFilledCircle & \halfFilledCircle\\
    \cmidrule(lr){2-6} 
    & {Forced Waiver of Class Action Rights}
    & {This agreement includes a \hl{class action waiver} and an arbitration provision that governs any disputes between you and Sendinblue.}
    & \halfFilledCircle & \halfFilledCircle & \halfFilledCircle\\
    \cmidrule(lr){2-6} 
    & {Other Legal Unfavorable Financial Term}
    & {...}\\
    
    \bottomrule
\end{tabular*}
\label{table:taxonomy}

\end{table*}


\myparagraph{ShopTC-100K Dataset.} In the data collection stage, we extract 8,251 shopping websites from the Tranco top 100K, yielding 1.8 million terms.~\autoref{tab:dataset_stats} presents ShopTC-100K statistics alongside two fake e-commerce datasets, with unfavorable financial terms identified in later measurement study (\S\ref{sec:findings}). 



\subsection{\TermName Term Taxonomy}
\label{sec:financial_terms_section}




We categorize \termtypecnt types of \termname terms into \termcatcnt categories (\autoref{table:taxonomy}) and provide real-world examples analyzed against the three
criteria proposed by the FTC Act criteria (\S\ref{ssec:ftc})~\citep{ftcact}. A detailed taxonomy is in Appendix~\ref{sec:detailed_tax}. While not inherently deceptive, these terms often impose financial obligations consumers should recognize. We note that the severity of such terms depends on \textit{context}, which we leave for future research. We do not claim this list is exhaustive; however, it represents the most prominent types among the \termcnt terms from \websitecnt websites. We also report the financial term template in Appendix~\ref{sec:appendix_finaincial_terms}.

It is important to note that this paper \textit{does not} aim to analyze the fairness of terms from the legal perspective. We consider our work to be a complementary addition to the AI \& Law datasets~\citep{lippi2019claudette, galassi2024unfair}, by focusing on the natural phrasing found in online shopping websites' terms and conditions. A comparison between our \termname term template and previous work on online agreement fairness can be found in Appendix~\ref{sec:appendix_other_templates}.



%\elisa{need to mention it here, since in the TermLens section, we will say that the LLM module is pluggable and we can use fine-tuned version to do it, therefore we need to talk about the creation of fine-tuning dataset here. Make a table?}




\section{VideoDiff}
We present \sysname{} (Figure~\ref{fig:teaser}), a human-AI video co-creation tool designed to support efficient video editing with alternatives.
With \sysname{}, users can generate and review diverse AI recommendations for 3 different video editing tasks: making rough cuts, inserting B-roll images, and adding text effects (D3).
%(Figure~\ref{fig:teaser}.1)
VideoDiff supports easy comparison between alternatives by aligning videos (D1) and highlighting differences using timeline and transcript views (D2, D4)
%(Figure~\ref{fig:teaser}.2)
. Users can organize and customize edits by sorting, refining, and regenerating AI suggestions (D6). %(Figure~\ref{fig:teaser}.3). 


\begin{figure*}
  \centering
  \includegraphics[width=\textwidth]{figures/system_overview.pdf}
  \caption{Overview of VideoDiff: Users can view an outline of the variations in the current editing stage (a). In this figure, we see 10 rough cut variations. The user can play videos of these different versions (b) and compare them in the transcript or timeline view (c).
  Users can toggle between the edited and source timelines (d) to align videos to the source or edited context or click on each section to navigate directly to that part of the video (e).
  Users can also sort variations by duration and the number of sections included, as well as pin, archive, or edit variations according to their preferences (f).}\label{fig:sys_overview}
\end{figure*}


\subsection{Scope}\label{sec:scope}
\revised{
While VideoDiff provides visualizations that allow users to quickly identify and skim differences between video versions (D2), it is designed to go beyond the functionality of a \textit{diff tool}, which typically focuses solely on highlighting changes or differences for review~\cite{tharatipyakul2018towards, baker2024interaction}. Instead, we designed VideoDiff as a human-AI co-creation tool that facilitates iterative \textit{generate-compare-refine} interactions, revealing how comparing alternatives can influence and enhance users’ creative processes.}

To help set context for \sysname{}, we define the different stages of video editing.
Rough cut creation is selecting good moments (or clips) from source footage to create a compelling story. Inserting B-roll images or video helps to make the video more interesting and dynamic. To insert good B-roll, creators should find effective images or videos that illustrate what is being said and place them appropriately. Inserting text effects helps emphasize parts of the narration through animation and stylized text. The AI algorithms for each of these editing tasks are in themselves hard technical problems, and we do not focus on them in this work. Our focus is on supporting users to work with the generated variations as part of the editing process. However, because we must support some editing in order to test our ideas, we have implemented basic versions of editing algorithms that leverage LLMs to process the video transcript and recommend rough cuts, B-roll, and text effects. A holistic solution with multimodal analysis of video, audio, and narration can offer better editing suggestions in the future.

Most video editing softwares offer more than the three stages that VideoDiff supports, such as applying color correction or cleaning up the audio. Our goal is not to create a fully functional video editor, but rather to explore how video editing might change as AI technologies enter more and more of the editing stages. \camready{As AI easily generates multiple editing recommendations, the video editing task shifts to involve more curation beyond just editing, and we must consider how to best support this transition.}




\subsection{Interface}
VideoDiff is a web-based video editing tool (Figure~\ref{fig:sys_overview}) where users can generate, review, and customize video alternatives. 
When the user uploads a video, \sysname{} first generates 10 rough cut recommendations. In the \textit{versions outline} on the left, users can see the list of all variations in the current editing stage (Figure~\ref{fig:sys_overview}.a). Users can play the video of each version (Figure~\ref{fig:sys_overview}.b) or skim the differences using a timeline or transcript view (Figure~\ref{fig:sys_overview}.c).  Users can toggle between showing only the edited or all of the source content for additional context (Figure~\ref{fig:sys_overview}.d).  They can also click on any section to navigate directly to that part (Figure~\ref{fig:sys_overview}.e).
%or switch to transcript view to compare the variation side by side for a more detailed comparison (Figure~\ref{fig:sys_overview}.E). With the \textit{variation control}, 
%
Users can sort, re-order, pin and archive the variations to organize them (Figure~\ref{fig:sys_overview}.f). Or they can refine or recombine existing variations or regenerate a new variation using text prompts. 
%mira: this is already written down in the organizing section so removing from up here so it's not repetitive
%Sorting varies by editing stage. In the rough cut stage, users can sort by the duration and number of sections. In the B-roll editing stage, users can sort by the number and style of the images. In the text effects stage they can sort by the number and style of the text effects. 

\begin{figure}[t]
  \centering
  \includegraphics[width=\columnwidth]{figures/timelines_new.jpg}
  \caption{At each editing stage, VideoDiff provides glanceable timelines for users to easily compare different variations. Users can click on any section, B-roll image, or text effect to jump to that part of the video and preview the effects.}\label{fig:timelines}
\end{figure}

\begin{figure*}
  \centering
  \includegraphics[width=\textwidth]{figures/edited_source_final.jpg}
  \caption{Users can switch between the edited and original source timeline in the transcript (a) and timeline (b) views. This helps users see the edits in the context of the source content and compare which sections are included at a glance. The source view (c) shows users the location of the edited content in the context of the source view.}\label{fig:extracted_original}
\end{figure*}


\ipstart{Comparing Variations with Timeline View}
VideoDiff provides different visualizations of timelines at each video editing stage (Figure\ref{fig:timelines}) so that users can quickly review and compare multiple alternatives (D3). 
In the rough cut stage, VideoDiff uses ~\textit{sections} to visualize the timing and coverage of content in each variation.
Drawing upon prior work that has shown that grouping footage into thematically coherent chunks can help creators make video editing decisions~\cite{leake2020generating, huh2023avscript}, we explore how chunking into sections can aid comparison of edited videos. 
Instead of segmenting each rough cut variation to identify sections, VideoDiff extracts sections from the source footage and applies them consistently across all variations. This allows users to easily align the variations and see how different versions include or exclude certain sections and cover varying parts or lengths of each (D1-D2). 
Users can toggle between edited and source timeline views (Figure~\ref{fig:extracted_original}.a) to align the videos based on the edited or source video timeline. The edited timeline allows users to quickly skim the video's overall duration, along with the placement and proportion of each section. By hovering over a section, users can view relevant video thumbnails and click to play the video from that point.
With the source timeline, users can easily understand which parts of the source video are extracted. In the formative study, video creators often compared edited videos to the source videos to verify any missing key information. Drawing from this observation and aligning with D1, we use the source video as an anchor for aligning rough cut variations for comparison. 
% \mira{it's great to include this example but this text doesn't align with the figure. }
For example in Figure~\ref{fig:extracted_original}.b, while the edited timeline shows that both versions have a similar duration for ``Campus Highlights'' section, the source timeline reveals that each version covers different parts of the source footage on ``Campus Highlights''. By hovering over the lighter-colored boxes indicating excluded parts, users can see which video thumbnails and topic keywords are not covered.



Once the user chooses a rough-cut version, they can generate 10 videos with different B-roll recommendations. In the B-roll stage, VideoDiff's timeline view shows the B-roll thumbnails on top of the rough cut timeline bar, along with the keyword used to search for the B-rolls (Figure~\ref{fig:timelines}). Users can hover over the B-roll thumbnails to see which video scene each B-roll covers and click thumbnails to play the video and preview how the B-roll is inserted into the footage. After selecting a B-roll version, users can generate 10 new videos with different text effect suggestions. Instead of showing video thumbnails with text effects, we display the keywords where the text effects are applied, as text effects on thumbnails are too small to skim.
Users can hover over a keyword to view the narration sentence for context, and click on the keyword to play the video and preview how the text effects are integrated into the footage.




\begin{figure}[t]
  \centering
  \includegraphics[width=\columnwidth]{figures/transcripts_new.jpg}
  \caption{At each editing stage, VideoDiff provides glanceable transcripts so that users can easily review and compare different variations. In rough cut transcripts, visually concrete keywords~\cite{leake2020generating} are emphasized in bold, allowing users to easily skim through the content of each variation. Users can click on section headings, B-roll images, or text effects to jump to that part of the video and preview the effects.}\label{fig:transcripts}
\end{figure}

\ipstart{Comparing Variations with Transcript View}
VideoDiff's transcript view allows users to quickly skim the transcripts to understand the differences of variations in each editing stage (Figure~\ref{fig:transcripts}). 
The transcript is divided into sections and users can click on the section headings to play the video from the starting point of each section.
\camready{During the rough cut phase, VideoDiff aids users in maintaining context while toggling between transcript and timeline views by showing a mini timeline within the transcript view. It also uses consistent color coding to denote sections across both views.}
In the transcript, visually concrete keywords~\cite{leake2017computational} are emphasized in bold, allowing users to skim through the content of each variation~\footnote{We used GPT-o to identify visually concrete keywords with few-shot examples.}.
Similar to the timeline view, users can also switch to the source transcript view (Figure~\ref{fig:extracted_original}.c) and see the complete transcript and identify which parts of the source text are excluded in each section. We synchronize the scrolling across multiple transcripts, allowing users to easily compare variations side-by-side (D1). 

For quick skimming of B-roll and text effects options, VideoDiff highlights sentences in the transcript where these effects are applied (D2, Figure~\ref{fig:transcripts}). Users can click on the B-roll thumbnails or text effects to preview the video from the moment the effect appears.

% \mira{why only show the visual concrete words in the rough cut version?}

\ipstart{Organizing Variations}
To help users to easily explore and narrow down the search space of alternatives, VideoDiff supports sorting of variations in each stage. In the rough cut stage, users can sort based on the edited video duration and the number of sections, and in the B-roll and text effects stages, users can sort based on the number of effects and the media type of B-rolls (image, illustration, video) or style of text effects (title, subtitle, lower thirds). 
Additionally, users can manually pin preferred versions to the top or archive unwanted versions to the bottom, which is also reflected in the outline view.

\begin{figure}
  \centering
  \includegraphics[width=\columnwidth]{figures/refine.jpg}
  \caption{Using VideoDiff, users can edit a variation, recombine multiple variations, and generate a new variation using text prompts. For each new generation, VideoDiff summarizes the changes so that users can easily verify the result.}\label{fig:refine}
\end{figure}
    
\ipstart{Customizing Variations}
When users are not satisfied with the initial recommendations of VideoDiff, they can further edit existing versions or generate new alternatives (D6, Figure~\ref{fig:refine}).
Users can provide a text prompt to guide the generation of new alternatives (\textit{e.g., Show many text effects when I talk about grocery items.}), or click \textit{``Surprise me''} to have VideoDiff suggest a new alternative that is different from existing variations. 
Users can also recombine existing versions by specifying the version IDs and how to merge them in the text prompt (\textit{e.g., Use first two B-roll images from \#3 and last B-roll image from \#7.}). By default, generated new results are pinned to the top for easy discovery.
Using VideoDiff, users can also edit existing versions with a prompt (\textit{e.g., Shorten the part when I'm talking about the meal plans.}). The new generation from the edit prompt is displayed right below the original version for quick comparison and VideoDiff also describes specific changes made for quick verification of edits (D2). For example, in Figure~\ref{fig:refine} the user tells VideoDiff to ``shorten the part about dining halls'' and the system responds:
``Shortened the description of dining halls within the Dining and Housing section''. 
% \mira{how do you do this part with GPT?}




\subsection{Implementation \& Prompt Engineering}
We implemented VideoDiff using React.js and d3.js. For embedding a video player, we used Remotion~\cite{remotion} to render the edited video and overlay B-rolls and text effects. 
% Figure~\ref{fig:pipeline} illustrates the pipeline of VideoDiff.
When users upload a video, VideoDiff uses OpenAI's Whisper API to transcribe the video. VideoDiff is powered by OpenAI's GPT-4o and uses prompt engineering for 1) segmenting video into sections and identifying visually concrete keywords, 2) generating multiple alternatives of edit recommendations for rough cuts, B-rolls, and text effects, and 3) parsing and executing users' new generation prompts and summarizing changes. 
To ensure VideoDiff provides diverse edit recommendations in each stage, we use ~\textit{augmentation prompts} (\S\ref{apndx:augmentation_pipeline}) to control the generation of suggestion (\textit{e.g.,} by specifying duration and section coverage for each rough cut recommendation). 
%\mina{can we share the prompts in the appendix? Should ask Mira and Ding}
% - why we use them? -> to explore variable outputs, prior work utilized high-temperature, but it is difficult to describe their differences and adhere to user query
% - Luminate used dimensions to generate variations with prompting


\section{Evaluation and Large-Scale Measurement}

% Table generated by Excel2LaTeX from sheet 'Paper Version'
\begin{table*}[htbp]
  \centering
  \caption{\textbf{Classification}. AUC as a function of shot for three JoLT configurations and three competitive methods. Values are the mean and 95$\%$ confidence interval (CI) over 5 seeds that affect the training shot selection. Due to limited computational resources values at 16 and 32 shots with 0 CI use only a single seed. Competitive data from \citep{hegselmann2023tabllm}.}
  \label{tab:classification_results}%
  \vskip 0.05in
  \begin{small}
  \begin{sc}
  \begin{adjustbox}{max width=\textwidth}
    \begin{tabular}{rcccccc}
    \toprule
          &       & \multicolumn{5}{c}{\textbf{Shot}} \\
\cmidrule{3-7}    \multicolumn{1}{l}{\textbf{Dataset}} & \textbf{Method} & \textbf{0} & \textbf{4} & \textbf{8} & \textbf{16} & \textbf{32} \\
    \midrule
          & XGBoost & -     & 0.5$\pm$0.00 & 0.56$\pm$0.08 & 0.68$\pm$0.04 & 0.76$\pm$0.03 \\
          & TabPFN & -     & 0.59$\pm$0.12 & 0.66$\pm$0.07 & 0.69$\pm$0.02 & 0.76$\pm$0.03 \\
    \multicolumn{1}{l}{bank} & TabLLM & 0.63$\pm$0.01 & 0.59$\pm$0.09 & 0.64$\pm$0.04 & 0.65$\pm$0.04 & 0.64$\pm$0.05 \\
          & JoLT (Gemma-2-2B) & 0.46$\pm$0.00 & 0.62$\pm$0.05 & 0.62$\pm$0.05 & 0.57$\pm$0.05 & 0.52$\pm$0.09 \\
          & JoLT (Gemma-2-27B) & 0.61$\pm$0.00 & 0.72$\pm$0.01 & 0.72$\pm$0.01 & 0.71$\pm$0.01 & 0.73$\pm$0.00 \\
          & JoLT (Qwen-2.5-72B) & 0.73$\pm$0.00 & 0.83$\pm$0.01 & 0.81$\pm$0.02 & 0.77$\pm$0.00 & 0.73$\pm$0.00 \\
    \midrule
          & XGBoost & -     & 0.5$\pm$0.00 & 0.58$\pm$0.06 & 0.66$\pm$0.04 & 0.67$\pm$0.05 \\
          & TabPFN & -     & 0.52$\pm$0.07 & 0.64$\pm$0.04 & 0.67$\pm$0.01 & 0.7$\pm$0.04 \\
    \multicolumn{1}{l}{blood} & TabLLM & 0.61$\pm$0.04 & 0.58$\pm$0.08 & 0.66$\pm$0.03 & 0.66$\pm$0.06 & 0.68$\pm$0.04 \\
          & JoLT (Gemma-2-2B) & 0.56$\pm$0.00 & 0.62$\pm$0.08 & 0.58$\pm$0.04 & 0.64$\pm$0.06 & 0.59$\pm$0.05 \\
          & JoLT (Gemma-2-27B) & 0.70$\pm$0.00 & 0.72$\pm$0.03 & 0.68$\pm$0.03 & 0.73$\pm$0.02 & 0.71$\pm$0.05 \\
          & JoLT (Qwen-2.5-72B) & 0.64$\pm$0.00 & 0.71$\pm$0.04 & 0.71$\pm$0.02 & 0.73$\pm$0.02 & 0.72$\pm$0.03 \\
    \midrule
          & XGBoost & -     & 0.5$\pm$0.00 & 0.62$\pm$0.09 & 0.74$\pm$0.03 & 0.79$\pm$0.04 \\
          & TabPFN & -     & 0.63$\pm$0.11 & 0.63$\pm$0.10 & 0.8$\pm$0.03 & 0.85$\pm$0.03 \\
    \multicolumn{1}{l}{calhousing} & TabLLM & 0.61$\pm$0.01 & 0.63$\pm$0.04 & 0.6$\pm$0.06 & 0.7$\pm$0.07 & 0.77$\pm$0.07 \\
          & JoLT (Gemma-2-2B) & 0.45$\pm$0.00 & 0.72$\pm$0.01 & 0.66$\pm$0.09 & 0.76$\pm$0.04 & 0.78$\pm$0.04 \\
          & JoLT (Gemma-2-27B) & 0.64$\pm$0.00 & 0.83$\pm$0.01 & 0.82$\pm$0.04 & 0.85$\pm$0.02 & 0.85$\pm$0.01 \\
          & JoLT (Qwen-2.5-72B) & 0.64$\pm$0.00 & 0.83$\pm$0.01 & 0.80$\pm$0.04 & 0.84$\pm$0.01 & 0.84$\pm$0.01 \\
    \midrule
          & XGBoost & -     & 0.5$\pm$0.00 & 0.59$\pm$0.04 & 0.7$\pm$0.07 & 0.82$\pm$0.03 \\
          & TabPFN & -     & 0.64$\pm$0.05 & 0.75$\pm$0.04 & 0.87$\pm$0.04 & 0.92$\pm$0.02 \\
    \multicolumn{1}{l}{car} & TabLLM & 0.82$\pm$0.02 & 0.83$\pm$0.03 & 0.85$\pm$0.03 & 0.86$\pm$0.03 & 0.91$\pm$0.02 \\
          & JoLT (Gemma-2-2B) & 0.73$\pm$0.00 & 0.84$\pm$0.02 & 0.79$\pm$0.02 & 0.79$\pm$0.04 & 0.74$\pm$0.04 \\
          & JoLT (Gemma-2-27B) & 0.82$\pm$0.00 & 0.89$\pm$0.01 & 0.90$\pm$0.01 & 0.91$\pm$0.01 & 0.94$\pm$0.01 \\
          & JoLT (Qwen-2.5-72B) & 0.86$\pm$0.00 & 0.89$\pm$0.04 & 0.88$\pm$0.02 & 0.89$\pm$0.04 & 0.94$\pm$0.01 \\
    \midrule
          & XGBoost & -     & 0.5$\pm$0.00 & 0.51$\pm$0.06 & 0.59$\pm$0.04 & 0.66$\pm$0.03 \\
          & TabPFN & -     & 0.58$\pm$0.07 & 0.59$\pm$0.03 & 0.64$\pm$0.05 & 0.69$\pm$0.06 \\
    \multicolumn{1}{l}{creditg} & TabLLM & 0.53$\pm$0.04 & 0.69$\pm$0.04 & 0.66$\pm$0.04 & 0.66$\pm$0.04 & 0.72$\pm$0.05 \\
          & JoLT (Gemma-2-2B) & 0.52$\pm$0.00 & 0.52$\pm$0.03 & 0.53$\pm$0.06 & 0.55$\pm$0.04 & 0.50$\pm$0.04 \\
          & JoLT (Gemma-2-27B) & 0.48$\pm$0.00 & 0.55$\pm$0.02 & 0.54$\pm$0.04 & 0.56$\pm$0.04 & 0.53$\pm$0.01 \\
          & JoLT (Qwen-2.5-72B) & 0.60$\pm$0.00 & 0.56$\pm$0.03 & 0.57$\pm$0.05 & 0.60$\pm$0.08 & 0.65$\pm$0.05 \\
    \midrule
          & XGBoost & -     & 0.5$\pm$0.00 & 0.59$\pm$0.14 & 0.72$\pm$0.06 & 0.69$\pm$0.07 \\
          & TabPFN & -     & 0.61$\pm$0.11 & 0.67$\pm$0.10 & 0.71$\pm$0.06 & 0.77$\pm$0.03 \\
    \multicolumn{1}{l}{diabetes} & TabLLM & 0.68$\pm$0.05 & 0.61$\pm$0.08 & 0.63$\pm$0.07 & 0.69$\pm$0.06 & 0.68$\pm$0.04 \\
          & JoLT (Gemma-2-2B) & 0.62$\pm$0.00 & 0.73$\pm$0.06 & 0.71$\pm$0.06 & 0.76$\pm$0.02 & 0.73$\pm$0.06 \\
          & JoLT (Gemma-2-27B) & 0.82$\pm$0.00 & 0.81$\pm$0.02 & 0.79$\pm$0.01 & 0.80$\pm$0.02 & 0.80$\pm$0.02 \\
          & JoLT (Qwen-2.5-72B) & 0.78$\pm$0.00 & 0.83$\pm$0.02 & 0.82$\pm$0.02 & 0.82$\pm$0.02 & 0.82$\pm$0.02 \\
    \midrule
          & XGBoost & -     & 0.5$\pm$0.00 & 0.55$\pm$0.12 & 0.84$\pm$0.06 & 0.88$\pm$0.04 \\
          & TabPFN & -     & 0.84$\pm$0.05 & 0.88$\pm$0.04 & 0.87$\pm$0.05 & 0.91$\pm$0.02 \\
    \multicolumn{1}{l}{heart} & TabLLM & 0.54$\pm$0.04 & 0.76$\pm$0.12 & 0.83$\pm$0.04 & 0.87$\pm$0.04 & 0.87$\pm$0.05 \\
          & JoLT (Gemma-2-2B) & 0.64$\pm$0.00 & 0.74$\pm$0.01 & 0.80$\pm$0.04 & 0.72$\pm$0.08 & 0.65$\pm$0.09 \\
          & JoLT (Gemma-2-27B) & 0.74$\pm$0.00 & 0.82$\pm$0.02 & 0.85$\pm$0.01 & 0.87$\pm$0.01 & 0.88$\pm$0.01 \\
          & JoLT (Qwen-2.5-72B) & 0.89$\pm$0.00 & 0.87$\pm$0.01 & 0.88$\pm$0.01 & 0.89$\pm$0.01 & 0.90$\pm$0.02 \\
    \midrule
          & XGBoost & -     & 0.5$\pm$0.00 & 0.59$\pm$0.05 & 0.77$\pm$0.02 & 0.79$\pm$0.03 \\
          & TabPFN & -     & 0.73$\pm$0.07 & 0.71$\pm$0.08 & 0.76$\pm$0.08 & 0.8$\pm$0.04 \\
    \multicolumn{1}{l}{income} & TabLLM & 0.84$\pm$0.00 & 0.84$\pm$0.01 & 0.84$\pm$0.02 & 0.84$\pm$0.04 & 0.84$\pm$0.01 \\
          & JoLT (Gemma-2-2B) & 0.82$\pm$0.00 & 0.82$\pm$0.02 & 0.82$\pm$0.01 & 0.83$\pm$0.02 & 0.83$\pm$0.01 \\
          & JoLT (Gemma-2-27B) & 0.86$\pm$0.00 & 0.85$\pm$0.00 & 0.85$\pm$0.01 & 0.85$\pm$0.01 & 0.85$\pm$0.00 \\
          & JoLT (Qwen-2.5-72B) & 0.83$\pm$0.00 & 0.86$\pm$0.00 & 0.85$\pm$0.01 & 0.86$\pm$0.00 & 0.86$\pm$0.00 \\
    \midrule
          & XGBoost & -     & 0.5$\pm$0.00 & 0.58$\pm$0.06 & 0.72$\pm$0.04 & 0.78$\pm$0.03 \\
          & TabPFN & -     & 0.65$\pm$0.07 & 0.72$\pm$0.04 & 0.71$\pm$0.06 & 0.78$\pm$0.02 \\
    \multicolumn{1}{l}{jungle} & TabLLM & 0.6$\pm$0.00 & 0.64$\pm$0.01 & 0.64$\pm$0.02 & 0.65$\pm$0.03 & 0.71$\pm$0.02 \\
          & JoLT (Gemma-2-2B) & 0.67$\pm$0.00 & 0.60$\pm$0.02 & 0.59$\pm$0.04 & 0.55$\pm$0.04 & 0.61$\pm$0.04 \\
          & JoLT (Gemma-2-27B) & 0.62$\pm$0.00 & 0.62$\pm$0.01 & 0.63$\pm$0.01 & 0.63$\pm$0.02 & 0.64$\pm$0.01 \\
          & JoLT (Qwen-2.5-72B) & 0.62$\pm$0.00 & 0.62$\pm$0.01 & 0.62$\pm$0.01 & 0.61$\pm$0.03 & 0.64$\pm$0.01 \\
    \bottomrule
    \end{tabular}%
    \end{adjustbox}
  \end{sc}
  \end{small}
  \vskip -0.1in
\end{table*}%







We implement and evaluate \platform using a manually annotated dataset. Our evaluation focuses on two key aspects: (1) assessing detection performance to determine how effectively LLMs, including both zero-shot and fine-tuned models, identify \termname terms (\S\ref{sec:eva}), and (2) analyzing findings from large-scale measurements using \platform (\S\ref{sec:findings}).



\subsection{Evaluation on an Annotated Dataset}
\label{sec:eva}

\myparagraph{Dataset} We create an annotated dataset by randomly selecting 500 terms from clusters of both \termname terms and negative clusters (i.e., benign financial or non-financial terms). This yields 250 potential \termname terms and 250 benign terms. Three researchers independently labeled the terms using the \termname template, without knowledge of the clusters. Disagreements were resolved in a second pass, and duplicates were removed, resulting in 489 final terms. The dataset was split into 244 terms for fine-tuning and 245 terms for validation (\autoref{tab:dataset_stats}). 


\myparagraph{Baselines}
To our knowledge, no prior work has directly addressed the detection of unfavorable financial terms. Recent advances in large language models (LLMs) demonstrate superior performance in common sense reasoning, complex text classification, and contextual understanding~\citep{gpt35,openai2023gpt4,touvron2023llama}, outperforming older models like BERT~\citep{devlin2018bert} and RoBERTa~\citep{liu2019roberta}. Therefore, we evaluate state-of-the-art LLMs: (1) GPT-3.5-Turbo~\citep{gpt35}, (2) GPT-4-Turbo~\citep{openai2023gpt4}, and (3) GPT-4o, along with two open-source LLMs: (1) LLaMA 3 8B~\citep{touvron2023llama} and (2) Gemma 2B~\citep{team2024gemma}.


\myparagraph{Evaluation Configurations}
We evaluate two configurations: (1) Zero-shot classification with a simple binary prompt describing the unfavorable financial term and a multi-class taxonomy prompt explaining term types, and (2) Fine-tuning the LLM using the taxonomy to improve detection accuracy (see prompts in Appendix~\ref{sec:prompts}).





\myparagraph{Metrics} We evaluate the models in terms of false positive rate, true positive rate, F1 score, and Area Under the Curve. AUC represents the area under the ROC (Receiver Operating Characteristic) curve, measuring the model's ability to distinguish between classes.



%\subsection{Performance Analysis}
%\label{sec:eva_performance}



\myparagraph{Zero-shot Classification Performance}
As a baseline for \termname term detection, we evaluated zero-shot classification with two prompts: (1) a simple prompt defining unfavorable financial terms and (2) a taxonomy prompt explaining term types. Using the taxonomy improved the True Positive Rate (TPR) by 4.4\% to 27.4\% and boosted the F1 score by 4.5\% to 21.1\%, showing a better balance of precision and recall. However, the False Positive Rate (FPR) increased in most cases, except for GPT-4o, where it dropped by 24.2\%. GPT-4o achieved the best overall performance with a TPR of 96.6\% and an F1 score of 82.5\%, demonstrating the importance of a \termname term taxonomy for more accurate detection.




\myparagraph{Fine-tuned LLM Classification Performance}
We fine-tune GPT-3.5-Turbo and GPT-4o for 4 epochs with a batch size of 1. Fine-tuning resulted in significant performance improvements, with GPT-4o achieving a True Positive Rate (TPR) of 92.1\% and an F1 score of 94.6\%. The fine-tuned GPT-4o model outperforms other LLMs in distinguishing true positives from false positives. These results demonstrate that fine-tuning, even with a limited dataset, can substantially enhance detection performance.




\subsection{Large-Scale Measurement}
\label{sec:findings}


To understand the prevalence of \termname terms, we deploy the fine-tuned GPT-4o model with \platform for detection. The backend detection system was applied to English shopping websites filtered from the Tranco list's top 100,000 sites, along with two fake e-commerce website datasets: the FCWs dataset~\citep{bitaab2023beyond} and the FLOS dataset~\citep{janaviciute2023fraudulent}. This large-scale measurement serves as a qualitative study on the prevalence of \termname terms in popular shopping websites. We present our findings below. %\elisa{maybe do a category analysis?}


\label{sec:categoring}

\begin{figure*}[ht!]
\centering
\caption{Statistics from Large-scale measurement of \termname term detection on Tranco top 100K websites.}
\Description[Statistical analysis of unfavorable financial terms across shopping websites.]
 {This figure presents multiple statistical analyses of unfavorable financial terms detected on shopping websites from the Tranco top 100K list.
 (a) CDF of the number of terms per website, showing how frequently terms appear on different websites.
 (b) CDF of the number of unfavorable financial terms per website, illustrating the distribution of problematic terms across the dataset.
 (c) Distribution of unfavorable financial terms across different categories of websites based on their Tranco ranking, highlighting differences between highly ranked and lower-ranked sites.
 (d) Comparison of Trustpilot ratings between the top 10 websites with the most unfavorable financial terms and a random sample of 40 websites, analyzing whether websites with more unfavorable terms tend to have lower consumer ratings.}

\label{fig:large_scale_stats}
\subfigure[CDF of the number of terms per website.]{
\includegraphics[width=0.18\textwidth,height=0.18\textwidth]{imgs/cdf_page_counts.pdf}}
\subfigure[CDF of the number of unfavorable financial terms per website.]{
\label{fig:term_length_cdf}
\includegraphics[width=0.18\textwidth,height=0.18\textwidth]{imgs/cdf_unfavorable_counts.pdf}}
\subfigure[Distribution of unfavorable financial terms in each category across Tranco-ranked websites.]{
\label{fig:tranco_rank_dist}
\includegraphics[width=0.26\textwidth,height=0.18\textwidth]{imgs/websites_by_ranking.pdf}}
\subfigure[Trustpilot ratings comparison between the top 10 websites with the most unfavorable financial terms and a random sample of 40 websites.]{
\label{fig:imagnet-comparison}
\includegraphics[width=0.26\textwidth,height=0.18\textwidth]{imgs/most_unfair.pdf}}

\end{figure*}
\myparagraph{Categorizing Websites with \TermName Terms}
As shown earlier in~\autoref{table:dataset_stats},  we collect terms and conditions from \websitecnt English shopping websites, resulting in \termcnt terms. Using a GPT-4o model with the \termname term taxonomy, 10,150 terms (approximately \termpct) were flagged as \termname terms. Notably, \websitepct (3,471 out of 8,251) of the English shopping websites from the top 100,000 Tranco-ranked sites contain at least one type of non-legal \termname term. ~\autoref{fig:large_scale_stats}(a) and (b) show the number of terms and \termname terms across 8,251 websites, underscoring how difficult it is for consumers to review lengthy T\&Cs and pinpoint questionable financial terms thoroughly. This emphasizes the importance of automated detection systems to protect users from unfavorable terms.


\myparagraph{Trend Analysis}
\autoref{fig:large_scale_stats}(c) shows the distribution of unfavorable financial terms across categories in the top 100K Tranco-ranked websites~\citep{tranco}. Post-purchase terms (yellow) are the most common across all ranking levels, with a higher concentration in lower-ranked sites, suggesting these terms are more frequent on less popular websites. Purchase and billing terms (blue) also have significant representation. Termination and account recovery terms (red) and legal terms (green) are less frequent but more evenly spread across the rankings. This trend highlights the widespread presence of unfavorable financial terms, especially on lower-ranked sites, underscoring the need for greater regulation to protect consumers from harmful practices, particularly on less reputable websites.


\myparagraph{Comparing ShopTC-100K with Fake E-commerce Datasets}
Interestingly, the percentage of websites with \termname terms from the Tranco list (\websitepct) is similar to that of fraudulent e-commerce websites (46.70\%). This suggests that unfavorable financial terms are not limited to fraudulent sites but are also prevalent among high-ranking websites, pointing to a broader issue in consumer protection. \textit{ShopTC-100K} websites have more \termname legal terms, indicating that legitimate websites are more inclined to shift liability onto customers than fraudulent ones.



\myparagraph{Qualitative Study on User rating}
From the English shopping websites in the top 100k Tranco list, we select those with the highest frequency of \termname terms across categories. We analyze Trustpilot~\citep{trustpilot} reviews for the top 10 websites in each \termname term category with the highest presence, alongside 40 randomly selected websites. 
As shown in~\autoref{fig:large_scale_stats}(d), websites with \termname terms tend to have lower Trustpilot ratings, particularly those with ``Post-Purchase Terms'' and ``Purchase and Billing Terms,'' indicating negative customer satisfaction. ``Termination and Account Recovery'' and ``Legal Terms'' also correlated with lower ratings, though with more variation, suggesting mixed experiences. This suggests a link between \termname terms and consumer dissatisfaction.




\myparagraph{Qualitative Study on Current Ecosystem Defense}
We examine whether the top 10 websites with the highest frequency of \termname terms are flagged by ScamAdviser~\citep{scamadviser2024website}, Google Safe Browsing~\citep{google2024safebrowsing}, and Microsoft Defender SmartScreen~\citep{microsoft2024smartscreen}. Out of 40 websites, only 6 (15\%) have a ScamAdviser score below 90, and 5 (12.5\%) scored below 10, while the majority receive a perfect score of 100. None of the websites are flagged by Google Safe Browsing or Microsoft Defender, which is expected since \termname terms are not inherently indicative of scams. %To further investigate, we analyze 34 websites flagged by crowd-sourced scam reporting sites such as ScammerInfo~\citep{scammerinfo}, ScamAdvisor~\citep{scamadvisor}, and ScamWatcher~\citep{scamwatcher}. These sites are flagged by \platform, and we manually confirm the presence of \termname terms. However, only 1 out of the 34 was flagged by Google Safe Browsing, and none were flagged by Microsoft Defender.
%, highlighting the lack of detection mechanisms for \termname terms.


\myparagraph{Qualitative Study on User Perception} To illustrate the potential harm of \termname terms,  we present four case studies on user perception and financial harm in each category in Appendix~\ref{sec:case_studies}. This underscores the urgent need for automated systems to detect \termname terms effectively.

\section{Discussion}
\label{sec:limitation}


We introduce \textit{TermMiner}, an open-source automated pipeline for collecting and modeling unfavorable financial terms in shopping websites with limited human involvement. Researchers can utilize our tools to examine various aspects of web-based text, such as readability or accessibility, and to conduct longitudinal studies. 
 


\platform assumes that the financial terms in question are not \textit{adversarially perturbed}. Recent studies have highlighted LLM vulnerabilities to jailbreak and prompt injection attacks~\citep{zou2023universal, liu2023autodan, greshake2023not}. These attacks can result in incorrect outputs. However, for T\&Cs, such adversarial perturbations are likely to be subjected to manual scrutiny, particularly in post-complaint scenarios, such as legal disputes~\citep{celsius}. We leave the exploration of adversarial robustness in LLM-based \termname term detection for future work.













%-------------------------------------------------------------------------------
\section{Conclusion}
%-------------------------------------------------------------------------------

This paper presents \sys, a memory offloading mechanism
for LLM serving that meets latency SLOs while maximizing the host 
memory usage. 
%
\sys captures the tradeoff between meeting SLOs and maximizing host memory usage 
with \interval, an internal tunable knob. 
%
In addition, \sys automatically decides the optimal \interval, \ie, the smallest \interval that meets SLOs, with a two-stage tuning approach.  
%
The first stage assumes bandwidth contention and profiles the GPU model offline, and generates a performance \record that, for any valid combination of SLOs, sequence lengths, and batching sizes, stores an optimal \interval,
%
The second stage adjusts the \interval for GPU instances sharing the bus to ensure that the SLOs can still be met while maximizing the aggregate host memory usage across all GPU instances. 
%
Our evaluation shows that \sys consistently maintains SLO under various runtime
scenarios, and outperforms \flexgen in throughput by 1.85\X, due to use 2.37\X more host memory. 



\bibliographystyle{ACM-Reference-Format}
% \bibliography{ref}
\documentclass[sigconf]{acmart}

\usepackage{soul}
\usepackage{listings}

\lstset{
    basicstyle=\ttfamily\small,  
    breaklines=true,            
    frame=single,               
    backgroundcolor=\color{gray!10},  
    rulecolor=\color{black},      
    xleftmargin=0pt,            
    framexleftmargin=0pt,       
    showstringspaces=false,     
    fontadjust=true  % <- Ensures it follows document-wide font settings
}


\usepackage{multirow}
\usepackage{subfigure}
\usepackage{xspace}
\usepackage{rotating}
\usepackage{algorithm}



\usepackage{algorithmic}
\usepackage{mathtools}


%math
\usepackage{amsmath}% http://ctan.org/pkg/amsmath

\newcommand{\eg}{e.g.\@\xspace}
\newcommand{\ie}{i.e.\@\xspace}
\newcommand{\etal}{et~al.\@\xspace}
\newcommand{\etc}{{\em etc}\xspace}
\newcommand{\TK}{{\bf TK}\xspace}
\newcommand{\tk}{\TK}

\newcommand{\financialcnt}{12\xspace}


\newcommand{\platform}{\textit{TermLens}\xspace}
\newcommand{\termname}{unfavorable financial\xspace}
\newcommand{\TermName}{Unfavorable Financial\xspace}
\newcommand{\Termname}{Unfavorable financial\xspace}
\newcommand{\websitecnt}{8,979\xspace}
\newcommand{\termcnt}{1.9 million\xspace}
\newcommand{\websitepct}{42.06\%\xspace}
\newcommand{\termpct}{0.5\%\xspace}
\newcommand{\termtypecnt}{22\xspace}
\newcommand{\termcatcnt}{4\xspace}
\newcommand{\myparagraph}[1]{\textbf{\textit{#1}:}\hspace{3pt}}


\usepackage{pifont} % Approved package
\newcommand{\filledCircle}{\ding{108}} % ● Full circle
\newcommand{\halfFilledCircle}{\ding{109}} % ◐ Half-filled circle



\copyrightyear{2025}
\acmYear{2025}
\setcopyright{cc}
\setcctype{by}
\acmConference[WWW '25]{Proceedings of the ACM Web Conference 2025}{April 28-May 2, 2025}{Sydney, NSW, Australia}
\acmBooktitle{Proceedings of the ACM Web Conference 2025 (WWW '25), April 28-May 2, 2025, Sydney, NSW, Australia}
\acmDOI{10.1145/3696410.3714573}
\acmISBN{979-8-4007-1274-6/25/04}

\settopmatter{printacmref=true}


%\showthe\font



\begin{document}

%%
%% The "title" command has an optional parameter,
%% allowing the author to define a "short title" to be used in page headers.
\title[Harmful Terms and Where to Find Them]{Harmful Terms and Where to Find Them: Measuring and Modeling Unfavorable Financial Terms and Conditions in Shopping Websites at Scale}


\author{Elisa Tsai}
\affiliation{%
  \institution{University of Michigan}
  \city{Ann Arbor}
  \state{Michigan}
  \country{USA}
}
\email{eltsai@umich.edu}


\author{Neal Mangaokar}
\affiliation{
  \institution{University of Michigan}
  \city{Ann Arbor}
  \state{Michigan}
  \country{USA}
}
\email{nealmgkr@umich.edu}


\author{Boyuan Zheng}
\affiliation{
  \institution{University of Michigan}
  \city{Ann Arbor}
  \state{Michigan}
  \country{USA}
}
\email{boyuann@umich.edu}

\author{Haizhong Zheng}
\affiliation{
  \institution{University of Michigan}
  \city{Ann Arbor}
  \state{Michigan}
  \country{USA}
}
\email{hzzheng@umich.edu}

\author{Atul Prakash}
\affiliation{
  \institution{University of Michigan}
  \city{Ann Arbor}
  \state{Michigan}
  \country{USA}
}
\email{aprakash@umich.edu}



%%
%% By default, the full list of authors will be used in the page
%% headers. Often, this list is too long, and will overlap
%% other information printed in the page headers. This command allows
%% the author to define a more concise list
%% of authors' names for this purpose.

%%
%% The abstract is a short summary of the work to be presented in the
%% article.
\begin{abstract}
Terms and conditions for online shopping websites often contain terms that can have significant financial consequences for customers. 
Despite their impact, there is currently no comprehensive understanding of the types and potential risks associated with unfavorable financial terms. Furthermore, there are no publicly available detection systems or datasets to systematically identify or mitigate these terms.
In this paper, we take the first steps toward solving this problem with three key contributions.

\textit{First}, we introduce \textit{TermMiner}, an automated data collection and topic modeling pipeline to understand the landscape of unfavorable financial terms.
\textit{Second}, we create \textit{ShopTC-100K}, a dataset of terms and conditions from shopping websites in the Tranco top 100K list, comprising 1.8 million terms from 8,251 websites. Consequently, we develop a taxonomy of 22 types from 4 categories of unfavorable financial terms---spanning purchase, post-purchase, account termination, and legal aspects.
\textit{Third}, we build \textit{TermLens}, an automated detector that uses Large Language Models (LLMs) to identify unfavorable financial terms. 

Fine-tuned on an annotated dataset, \textit{TermLens} achieves an F1 score of 94.6\% and a false positive rate of 2.3\% using GPT-4o. 
When applied to shopping websites from the Tranco top 100K, we find that 42.06\% of these sites contain at least one unfavorable financial term, with such terms being more prevalent on less popular websites. Case studies further highlight the financial risks and customer dissatisfaction associated with unfavorable financial terms, as well as the limitations of existing ecosystem defenses.

\end{abstract}

%%
%% The code below is generated by the tool at http://dl.acm.org/ccs.cfm.
%% Please copy and paste the code instead of the example below.
%%

\begin{CCSXML}
<ccs2012>
   <concept>
       <concept_id>10002951.10003260.10003277</concept_id>
       <concept_desc>Information systems~Web mining</concept_desc>
       <concept_significance>500</concept_significance>
       </concept>
   <concept>
       <concept_id>10002978.10002997.10003000</concept_id>
       <concept_desc>Security and privacy~Social engineering attacks</concept_desc>
       <concept_significance>300</concept_significance>
       </concept>
   <concept>
       <concept_id>10003456.10003462.10003544.10011709</concept_id>
       <concept_desc>Social and professional topics~Consumer products policy</concept_desc>
       <concept_significance>500</concept_significance>
       </concept>
 </ccs2012>
\end{CCSXML}

\ccsdesc[500]{Information systems~Web mining}
\ccsdesc[300]{Security and privacy~Social engineering attacks}
\ccsdesc[500]{Social and professional topics~Consumer products policy}



%%
%% Keywords. The author(s) should pick words that accurately describe
%% the work being presented. Separate the keywords with commas.
\keywords{Topic modeling; Unfavorable financial terms; Consumer protection; Terms and conditions dataset; Deceptive content}

%% This command processes the author and affiliation and title
%% information and builds the first part of the formatted document.
\maketitle



%\section{CCS Concepts and User-Defined Keywords}

%Two elements of the ``acmart'' document class provide powerful
%taxonomic tools for you to help readers find your work in an online search.

%The ACM Computing Classification System ---
%\url{https://www.acm.org/publications/class-2012} --- is a set of
%classifiers and concepts that describe the computing
%discipline. Authors can select entries from this classification
%system, via \url{https://dl.acm.org/ccs/ccs.cfm}, and generate the
%commands to be included in the \LaTeX\ source.

%User-defined keywords are a comma-separated list of words and phrases
%of the authors' choosing, providing a more flexible way of describing
%the research being presented.

%CCS concepts and user-defined keywords are required for for all
%articles over two pages in length, and are optional for one- and
%two-page articles (or abstracts).



%%
%% The next two lines define the bibliography style to be used, and
%% the bibliography file.


%%%%%%%%%%%%%%%%%%%%%%%%%%%%%%%%%%%%%%
% Main text
%%%%%%%%%%%%%%%%%%%%%%%%%%%%%%%%%%%%%%

Large language models (LLMs) show significant performance in various downstream
tasks~\citep{brown_language_2020,openai_gpt-4_2024,dubey_llama_2024}. Studies
have found that training on high quality corpus improves the ability of LLMs
to solve different problems such as writing code, doing math exercises, and
answering logic questions~\citep{cai_internlm2_2024,deepseek-ai_deepseek-v3_2024,qwen_qwen25_2024}.
Therefore, effectively selecting high-quality text data is an important subject for
training LLM.

\begin{figure}[t]
    \centering
    \includegraphics[width=\linewidth]{figures/head.pdf}
    \caption{The overview of CritiQ. We (1) employ human annotators to annotate $\sim$30
    pairwise quality comparisons, (2) use CritiQ Flow to mine quality criteria, (3)
    use the derived criteria to annotate 25k pairs, and (4) train the CritiQ Scorer to
    perform efficient data selection.}
    \label{fig:overview}
\end{figure}

To select high-quality data from a large corpus, researchers manually design heuristics~\citep{dubey_llama_2024,rae_scaling_2022},
calculate perplexity using existing LLMs~\citep{marion2023moreinvestigatingdatapruning,wenzek2019ccnetextractinghighquality},
train classifiers~\citep{brown_language_2020,dubey_llama_2024,xie_data_2023} and
query LLMs for text quality through careful prompt engineering~\citep{gunasekar_textbooks_2023,wettig_qurating_2024,sachdeva_how_2024}.
Large-scale human annotation and prompt engineering require a lot of human
effort. Giving a comprehensive description of what high-quality data is like is also
challenging. As a result, manually designing heuristics lacks robustness and introduces
biases to the data processing pipeline, potentially harming model performance
and generalization. In addition, quality standards vary across different
domains. These methods can not be directly applied to other domains without significant
modifications.

To address these problems, we introduce CritiQ, a novel method to automatically
and effectively capture human preferences for data quality and perform efficient data
selection. Figure~\ref{fig:overview} gives an overview of CritiQ, comprising an agent
workflow, CritiQ Flow, and a scoring model, CritiQ Scorer. Instead of manually describing
how high quality is defined, we employ LLM-based agents to summarize quality
criteria from only $\sim$30 human-annotated pairs.

CritiQ Flow starts from a knowledge base of data quality criteria. The worker
agents are responsible to perform pairwise judgment under a given
criterion. The manager agent generates new criteria and refines them through reflection
on worker agents' performance. The final judgment is made by majority voting among
all worker agents, which gives a multi-perspective view of data quality.

To perform efficient data selection, we employ the worker agents to annotate a randomly
selected pairwise subset, which is ~1000x larger than the human-annotated one.
Following \citet{korbak_pretraining_2023,wettig_qurating_2024}, we train CritiQ
Scorer, a lightweight Bradley-Terry model~\citep{bradley_rank_1952} to convert
pairwise preferences into numerical scores for each text. We use CritiQ Scorer to
score the entire corpus and sample the high-quality subset.

For our experiments, we established human-annotated test sets to quantitatively
evaluate the agreement rate with human annotators on data quality preferences. We implemented the manager agent by \texttt{GPT-4o} and the worker
agent by \texttt{Qwen2.5-72B-Insruct}. We conducted experiments on different
domains including code, math, and logic, in which CritiQ Flow shows a consistent
improvement in the accuracies on the test sets, demonstrating the effectiveness
of our method in capturing human preferences for data quality. To validate the quality
of the selected dataset, we continually train \texttt{Llama 3.1}~\citep{dubey_llama_2024}
models and find that the models achieve better performance on downstream tasks
compared to models trained on the uniformly sampled subsets.

We highlight our contributions as follows. We will release the code to facilitate
future research.

\begin{itemize}
    \item We introduce CritiQ, a method that captures human preferences for data
        quality and performs efficient data selection at little cost of human
        annotation effort.

    \item Continual pretraining experiments show improved model performance in code,
        math, and logic tasks trained on our selected high-quality subset compared to the raw dataset.

    \item Ablation studies demonstrate the effectiveness of the knowledge base and
        the the reflection process.
\end{itemize}

\begin{figure*}[t]
    \centering
    \includegraphics[width=\linewidth]{figures/method.pdf}
    \caption{CritiQ Flow comprises two major components: multi-criteria pairwise
    judgment and the criteria evolution process. The multi-criteria pairwise
    judgment process employs a series of worker agents to make quality
    comparisons under a certain criterion. The criteria evolution process aims to
    obtain data quality criteria that highly align with human judgment through
    an iterative evolution. The initial criteria are retrieved from the
    knowledge base. After evolution, we select the final criteria to annotate
    the dataset for training CritiQ Scorer.}
    \label{fig:method}
\end{figure*}

\section{Related Work}


\myparagraph{Scam and fake e-commerce website detection}
Detection methods for scam and fake e-commerce websites (FCW) typically rely on two types of features: external (e.g., URLs, certificates, logos, redirect mechanisms)~\citep{blum2010lexical, zouina2017novel, moghimi2016new, sakurai2020discovering, drury2019certified, van2022logomotive, li2018fake, zhang2014you, zheng2017smoke, sahingoz2019machine, bitaab2023beyond} and content-based (e.g., visual and HTML structures, images, scripts, hyperlinks)~\citep{xiang2011cantina+, kharraz2018surveylance, yang2019phishing, jain2017phishing, bitaab2023beyond, yang2023trident}. These models are either rule-based or machine learning-based, with feature selection grounded in domain knowledge (e.g., indicative images, third-party scripts). However, no prior work in this line has considered terms and conditions and their financial impacts on users.


We consider social engineering scams to overlap with our detection target. The \termname terms in \autoref{fig:example} function similarly by deceiving users into signing up for additional subscriptions. However, as discussed in \S\ref{sec:financial_terms_section} and~\S\ref{sec:categoring}, \termname terms are not exclusive to scam websites. Therefore, consumers should be alerted to the presence of such terms. We view our work as the first to measure \termname terms at scale.

\begin{figure*}[t] 
 \centering
 \includegraphics[width=2\columnwidth]{imgs/data_collection_pipeline.pdf}
 \caption{\textbf{\textit{TermMiner} (data collection and topic modeling pipeline)}---(1) Measurement module: collects shopping websites from the Tranco list and fake e-commerce website datasets, extracting English terms and conditions from shopping websites. (2) Term classification module: classifies the terms into binary categories based on a given prompt. (3) Topic modeling module: leverages t5-base Sentence Transformer and DBSCAN for clustering. Topics are derived from the clusters using a combination of manual inspection and GPT-4o, employing a snowball sampling method~\citep{goodman1961snowball} to iteratively develop a topic template of terms.}
 \Description[Pipeline for collecting and analyzing unfavorable financial terms.]
 {This figure presents the \textit{TermMiner} pipeline for collecting and analyzing unfavorable financial terms from shopping websites. 
 (1) The measurement module gathers websites from the Tranco list and known fake e-commerce datasets, extracting English terms and conditions from shopping sites.
 (2) The term classification module processes extracted terms, categorizing them into binary labels based on a predefined prompt.
 (3) The topic modeling module applies the T5-base Sentence Transformer and DBSCAN clustering to group terms into topics. These topics are refined through a combination of manual inspection and GPT-4o, using a snowball sampling approach to iteratively develop a comprehensive topic taxonomy.}
\label{fig:financial_term_pipeline}
\end{figure*}



\myparagraph{Dark patterns}
Dark patterns are deceptive user interface designs intended to manipulate users into actions against their best interests~\citep{mathur2019dark}. Recent research has examined their psychological impact on user decision-making~\citep{mathur2021makes,nouwens2020dark,waldman2020cognitive,narayanan2020dark}, while also exploring legal frameworks and strategies for intervention~\citep{luguri2021shining,gray2021dark}.

Although terms and conditions are not part of the user interface design, we consider the \termname terms we identify to be closely related to dark patterns. The unilateral nature of these terms and their potential to hide uncommon or unexpected terms make them closely align with the characteristics of dark patterns: asymmetric, covert, deceptive, hiding information, and restrictive.



\myparagraph{Terms and conditions legal analysis} 
There is limited NLP-based analysis of legal documents like online contracts and terms of service~\citep{lagioia2017automated, braun2021nlp, lippi2019claudette, limsopatham2021effectively, jablonowska2021assessing, galassi2024unfair}. Prior studies, such as Lippi \etal~\citep{lippi2019claudette} and Galassi \etal~\citep{galassi2024unfair}, typically focus on small datasets of T\&Cs (25 to 200 documents). However, their focus is mainly on assessing fairness under the EU’s Unfair Contract Terms Directive~\citep{CouncilDirective1993} (i.e., clauses invalid in court). In contrast, our work specifically targets terms with more direct financial impacts on users.


In this paper, we focus on the financial terms in the large-scale measurement of terms and conditions from English shopping websites, assessed using the definition of unfair acts or practices as provided by the Federal Trade Commission (FTC)'s Policy Statement on Deception~\citep{ftc1983deception}. A detailed comparison of our proposed term taxonomy with prior work is provided in Appendix~\ref{sec:appendix_other_templates}.

\myparagraph{Privacy policy analysis}
A significant body of work investigates the viability of NLP-based analysis for privacy policies. One significant line of such research focuses on detecting contradicting policy statements (e.g., via ontologies~\citep{andow2019policylint} and knowledge graphs~\citep{cui2023poligraph}) or ambiguities~\citep{shvartzshnaider2019going}. Other areas include improving user comprehension~\citep{harkous2018polisis}, topic modeling, and summarization~\citep{alabduljabbar2021automated, sarne2019unsupervised}.

In this work, we focus on financial terms which are distinct from privacy policies. While we also perform topic modeling, we are the first to apply such a pipeline to construct a taxonomy for \termname terms. Furthermore, detecting contradictions and ambiguities is orthogonal to the detection of malicious financial terms, making it difficult to apply similar techniques directly.








\section{Understanding Unfavorable Financial Terms}
\label{sec:topic_modeling_section}

In this section, we outline our detection goal and present \textit{TermMiner}, a pipeline for collecting, clustering, and topic modeling \termname terms from English shopping websites in the Tranco top 100,000~\citep{tranco} and datasets of fraudulent e-commerce sites~\citep{bitaab2023beyond, janaviciute2023fraudulent}. As shown in \autoref{fig:financial_term_pipeline}, the pipeline categorizes two types of terms: (1) financial terms that may have immediate or future financial impacts, and (2) \termname terms, identified as unfair, unfavorable, or concerning for customers. We then summarize the taxonomy of \termname terms, which fall into four broad categories: (1) purchase and billing, (2) post-purchase, (3) termination and account recovery, and (4) legal terms.

\subsection{Threat Model}
\label{ssec:ftc}

We aim to detect one-sided, imbalanced, unfair, or malicious \textit{financial} terms in online shopping websites' terms and conditions, which could pose significant risks to users, potentially leading to unexpected financial losses. These risks can arise from website operators seeking to limit liability or from intentional malfeasance.

To assess whether a financial term is unfavorable, we refer to Section 5 of the Federal Trade Commission (FTC) Act~\citep{ftcact}, which defines an act as unfair if it meets the following criteria:

\begin{itemize}
    \item \textbf{C1: Substantial Injury}. It causes or is likely to cause substantial injury to consumers;
    \item \textbf{C2: Unavoidable Harm}. Consumers cannot reasonably avoid it; and
    \item \textbf{C3: Insufficient Benefits}. It is not outweighed by countervailing benefits to consumers or competition. 
\end{itemize}

During the topic modeling of term clusters, we judge the topic representing each cluster by three criteria to evaluate their fairness.

\myparagraph{C1} Since we focus on financial terms with potentially detrimental impacts, all financial terms inherently satisfy this criterion.


\myparagraph{C2} Terms and conditions are often hidden or difficult to avoid. Fair financial terms must be clearly displayed at critical points, like the payment page. However, terms related to cancellation, refunds, and returns are rarely shown upfront. We evaluate terms for unexpected fees (e.g., cancellation charges, non-refundable items, costly returns) that place an undue burden on consumers.


\myparagraph{C3} We classify terms as benign if they serve legitimate user or business protection, such as terms prohibiting fraud or abuse, protecting intellectual property, or ensuring legal compliance.



\subsection{Data Collection and Topic Modeling}
\label{ssec:data_collection_and_topic_modeling}

As shown in~\autoref{fig:financial_term_pipeline}, we introduce \textit{TermMiner}, a data collection and topic modeling pipeline for identifying \termname term at scale.  By integrating LLMs like GPT-4o~\citep{openai2023gpt4}, \textit{TermMiner} significantly reduces the extensive manual efforts required in previous web content mining studies, such as those focused on detecting dark patterns~\citep{mathur2019dark}. 
\textit{TermMiner} is open-sourced and can be repurposed for various web-based text analysis tasks or longitudinal studies. Researchers can use our tools to explore different aspects of terms and conditions, such as readability, accessibility, or fairness.


\myparagraph{A Two-Pass Method}
In the data collection and topic modeling steps, we employ a two-pass method. The first pass focuses on modeling and detecting \textit{financial terms} to develop a corresponding topic template. In the second pass, we use the detected financial terms to re-conduct the classification and topic modeling modules. This time, the goal is to detect \termname terms within the financial terms identified. This approach is necessary because, to the best of our knowledge, there are no established templates or annotation schemes for (1) financial terms or (2) \termname terms in online shopping agreements. This two-pass process ensures comprehensive detection and accurate categorization of both financial and \termname terms.


\myparagraph{(1) Measurement Module} The measurement module collects terms and conditions from shopping websites to build a large, diverse dataset for analysis. For our large-scale measurement, we collect English shopping websites from two sources: the Tranco list~\citep{tranco}, a ranking of top global websites, and two fraudulent e-commerce datasets (FCWs~\citep{bitaab2023beyond} and the Fraudulent and Legitimate Online Shops Dataset~\citep{janaviciute2023fraudulent}). We filter out non-English content using Python's langdetect library~\citep{langdetect}. To classify shopping websites, we evaluate several configurations: (1) GPT-3.5-Turbo~\citep{gpt35} with URL, (2) GPT-3.5-Turbo with URL and HTML content, (3) GPT-4o~\citep{openai2023gpt4} with URL, and (4) GPT-4o with URL and website screenshot. To evaluate our classification methods, we manually annotated a sample of 500 websites from the Tranco list, categorizing them into ``shopping'' and ``non-shopping.'' GPT-4o, when prompted with URLs and screenshots, achieved an accuracy of 92\%, comparable to commercial website classification services~\citep{mathur2019dark} (see Appendix~\ref{sec:website_cls} for details). Therefore, we use this configuration throughout our work.





We subsequently crawl the shopping websites to collect terms and conditions pages. A snowballed regex matching method detects terms and any nested policy pages, refined through positive and negative regex patterns to improve accuracy. Starting with common anchor texts, we iteratively refine the regex patterns by analyzing T\&C links, which can be found in Appendix~\ref{sec:appendix_reg}. As shown in~\autoref{table:dataset_stats}, we collected \termcnt terms from \websitecnt websites in total. 



\myparagraph{(2) Classfication Module}
The classification module categorizes terms from shopping websites' terms and conditions into binary categories: positive or negative. The categorization is based on the detection goal (such as identifying financial terms or identifying \termname terms) using corresponding prompts with the GPT-4o model~\citep{openai2023gpt4}.




We opt for prompt engineering instead of fine-tuning the LLM for term classification to reduce costs. Prior work~\citep{sun2023text,openai2023bestpractices}, along with our empirical observation (see~\S\ref{sec:eva}), indicates that clear task descriptions and relevant examples (taxonomy) significantly enhance LLM performance in text classification. For \termname terms, we use the ``Unfavorable Term Taxonomy Prompt'' from Appendix~\ref{sec:prompts} and topics identified in the topic modeling step, to perform zero-shot term classification on a given set of terms and conditions. This process outputs sets of positive and negative terms, which are then used for clustering, inspection, and topic modeling. The resulting template generated from this analysis will, in turn, enhance the classification accuracy, creating a feedback loop that continuously improves our detection capabilities. 

\sisetup{table-text-alignment=center,table-format=6.0}
\begin{tabular}{lSSSSrrS}
\toprule
{\textbf{Attribute}} & {\textbf{Declarations}} & {\textbf{Users}} & {\textbf{Inactive Users}} & {\textbf{Ambiguous Users}} & \multicolumn{2}{c}{\textbf{Class Proportion}} & {\textbf{Subreddits}} \\
\midrule
Year of Birth & 420803 & 401390 & 1630 & 17341 & Old: 56.19\% & Young: 43.81\% & 9806 \\
Gender & 424330 & 403428 & 1634 & 18337 & Male: 50.89\% & Female: 49.11\% & 9809 \\
Partisan Affiliation & 6369 & 6118 & 4 & 251 & Dem.: 54.55\% & Rep.: 45.45\% & 9137 \\
\bottomrule
\end{tabular}


\myparagraph{(3) Topic Modeling Module} The topic modeling module uses LLMs and manual inspection to organize terms into meaningful topics. We generate sentence embeddings with the T5 model~\citep{raffel2020exploring} and apply the DBSCAN clustering algorithm~\citep{ester1996density} to group terms by semantic similarity. The DBSCAN hyperparameters are decided through manual inspection. To extract high-frequency topics, we leverage GPT-4o~\citep{openai2023gpt4}, building on recent findings that show LLMs outperform traditional topic modeling methods like Latent Dirichlet Allocation (LDA)~\citep{blei2003latent} and BERTopic~\citep{grootendorst2022bertopic} in topic analysis~\citep{shrestha2023we, mu2024large}.



We develop an iterative topic modeling approach assisted by GPT-4o proceeds as follows: 
\begin{enumerate}
    \item We analyze DBSCAN clusters and create an initial topic template for financial terms.
    \item GPT-4o performs topic modeling on random samples from each cluster, assigning them to existing topics or suggesting new ones.
    \item We review and refine new topic suggestions through manual inspection, and updating the template.
    \item This process iterates until all clusters are assigned to a meaningful and satisfactory topic.
\end{enumerate}


This iterative workflow, combining clustering, human-guided template creation, and GPT-4o's advanced topic modeling capacity, enables efficient and comprehensive extraction of the topic template. We analyze 22,112 clusters in total, creating the \termname term taxonomy below.

\begin{table*}[t!]
    \footnotesize
    \centering
    \caption{{Categories, types, and examples of \termname terms are clustered, extracted, and topic-modeled from \termcnt terms across \websitecnt websites. All examples are extracted as-is from real-world shopping websites. The criteria are as follows: C1 = ``Substantial Injury'' (the term causes or is likely to cause substantial injury to consumers), C2 = ``Unavoidable Harm'' (consumers cannot reasonably avoid it), C3 = ``Insufficient Benefits'' (it is not outweighed by countervailing benefits to consumers or competition). The symbols represent the likelihood of satisfaction of a given criterion: \filledCircle = Always, \halfFilledCircle = Sometimes.}}
\begin{tabular*}{1.96\columnwidth}{p{1cm} p{4cm} p{9cm}  p{0.2cm} p{0.2cm} p{0.2cm}}
    \toprule
    \textbf{Category} & \textbf{Type} & \textbf{Example} & C1 & C2 & C3 \\
    \midrule

    \multirow{11}{*}{\shortstack{Purchase\\and\\Billing\\Terms}}

    
     & \shortstack{Immediate Automatic Subscription} 
    & {Also, as part of the \hl{promotion}, you will receive a \hl{subscription to the FitHabit Fitness App for only \$86}, and the subscription will renew monthly up until cancellation.} 
    & \filledCircle & \halfFilledCircle & \filledCircle \\
    \cmidrule(lr){2-6}
    & {Automatic Subscription after Free Trial}
    & {After the Promotion period has ended, unless you cancel the service before the end of the free trial period, you will \hl{automatically be subscribed} onto the regular paid 1-year plan at the price of \$275.40, which will automatically renew for successive 12-month periods, until cancelled.}
     & \filledCircle &  \halfFilledCircle & \halfFilledCircle\\
    \cmidrule(lr){2-6}
    & {Unilateral Unauthorized Account Upgrades} 
    & {Brevo reserves the right to automatically increase the contacts limit in the User account and \hl{upgrade the User’s plan without prior notice}. } 
     & \filledCircle & \filledCircle & \filledCircle \\
    \cmidrule(lr){2-6}
    & {Late or Unsuccessful Payment Penalty}
    & {In addition, if any payment is not received within \hl{30 days after the due date}, then we may charge a \hl{late fee} of \$10 and we may assess interest at the rate of 1.5\% of the outstanding balance per month (18\% per year), or the maximum rate permitted by law.}
     & \filledCircle & \halfFilledCircle & \halfFilledCircle \\
    \cmidrule(lr){2-6}
    & {Overuse Penalty}
    & {If the Company establishes limits on the frequency with which you may access the Site, or terminates your access to or use of the Site, you agree to pay the Company one hundred dollars (\$100) for each message posted \hl{in excess of such limits} or for each day on which you access the Site in excess of such limits, whichever is higher.}
     & \filledCircle & \halfFilledCircle & \halfFilledCircle \\
    \cmidrule(lr){2-6}
    & {Retroactive Application of Price Change} 
    & {When an applicable exchange rate is updated or when a change of price is notified to Brevo by its suppliers or WhatsApp, Brevo might \hl{immediately apply with retroactive effect} the new Ratio and price increase to the User.} 
     & \filledCircle & \filledCircle & \halfFilledCircle \\
    
    
    \midrule
    \multirow{11}{*}{\shortstack{Post-\\Purchase\\Terms}}
    & {Non-Refundable Subscription Fee}
    & {If you or we cancel your subscription, you are \hl{not entitled to a refund of any subscription fees} that were already charged for a subscription period that has already begun.}
     & \halfFilledCircle & \halfFilledCircle & \filledCircle \\
    \cmidrule(lr){2-6}
    & {No Refund For Purchase} 
    & {Unless a refund is required by law, there are \hl{No Refund For Purchases for POS terminals} and all transactions are final.}
     & \halfFilledCircle & \halfFilledCircle &  \halfFilledCircle \\
     
    \cmidrule(lr){2-6}
    & {Strict No Cancellation Policy}
    & {As Research and Markets starts processing your order once it is submitted, we operate a \hl{strict no cancellation policy}.}
     & \halfFilledCircle & \halfFilledCircle &  \halfFilledCircle \\


    \cmidrule(lr){2-6}
    & {\shortstack{Cancellation Fee or Penalty}}
    & {Some Bookings \hl{can’t be canceled for free}, while others can only be canceled for free before a deadline.}
     & \halfFilledCircle & \halfFilledCircle & \halfFilledCircle\\
    \cmidrule(lr){2-6}
    & Non-Refundable Additional Fee
    & {For this service, National Park Reservations charges a \hl{10\% non-refundable reservation fee} based on the total dollar amount of reservations made. }
     & \halfFilledCircle & \halfFilledCircle & \halfFilledCircle\\
    

    \cmidrule(lr){2-6}
    & {Non-Monetary Refund Alternatives} 
    & {Refund Policy: Refunds are not in cash but in the form of a \hl{``coupon’’}.}
    & \halfFilledCircle & \halfFilledCircle & \filledCircle \\
    \cmidrule(lr){2-6}
    
    & No Responsibility for Delivery Delays 
    & {We will \hl{not be held responsible} if there are \hl{delays in delivery} due to out-of-stock products. } 
    & \halfFilledCircle & \halfFilledCircle & \filledCircle \\
    \cmidrule(lr){2-6}
    & {Customers Responsible for Shipping Issues} 
    &  {If the parcel is on hold by the Customs department of the shipping country, \hl{the customer is liable} to provide all relevant and required documentation on to the authorities. Asim Jofa is \hl{not liable to refund} the amount in case of \hl{non-clearance of the parcel}.} 
    & \halfFilledCircle & \halfFilledCircle & \filledCircle\\
    \cmidrule(lr){2-6}
    & Customers Pay Return Shipping 
    & {All shipping costs will have to be borne by the customer.} & \halfFilledCircle & \halfFilledCircle & \filledCircle \\
    \cmidrule(lr){2-6}
    %& {Return Late Fee}
    %& {If you do not return your rented textbook on or before your rental period's Due Date for any reason (including if the textbook is lost or stolen), Cengage may charge a late fee. } \\
    %\cmidrule(lr){2-3}
    & {\shortstack{Restocking Fee}}
    & {An \hl{8\% restocking fee} and shipping fees for both ways will be borne by the buyer if returned without defects within 30 days from the purchase date or 7 days from delivery date, whichever is later.}
    & \halfFilledCircle & \halfFilledCircle & \filledCircle \\
    
    
    \midrule
    \multirow{3}{*}{\shortstack{Termination\\and\\Account\\Recovery\\Terms}} 
    & {Account Recovery Fee}
    &  {To recover an archived or locked account, the legitimate creator of the account shall provide verifiable information about one's identity and will be charged a \hl{10\% administrative fee} for the additional work caused by the account recovery process.} 
    & \halfFilledCircle & \halfFilledCircle & \halfFilledCircle\\
    
    \cmidrule(lr){2-6}
    & {Digital Currency, Reward, Money Seizure on Inactivity}
    & {Please be noted that if your account is \hl{dormant} for a period of 12 consecutive calendar months or longer, ..., any amounts in your account’s balance, including any outstanding fees owed to you, shall be considered as \hl{forfeited and shall be fully deducted} to Appnext.}
    & \halfFilledCircle & \halfFilledCircle & \filledCircle \\
    
    \cmidrule(lr){2-6}
    & {Digital Currency, Reward, Money Seizure on Termination or Account Closure} 
    &  {All Currency and/or Virtual Goods shall be \hl{cancelled} if Your account is \hl{terminated} or suspended for any reason or if We discontinue providing the Games and we will not compensate you for this loss or make any refund to you.}
    & \halfFilledCircle & \halfFilledCircle & \halfFilledCircle\\

    
    %\cmidrule(lr){2-3}
    %& Unsuccessful Payment Penalties 
    %& {In the event of an unsuccessful recurring payment, an \hl{administration fee of up to \$3.00} may be applied in order to keep a subscription temporarily active until the full subscription fee can be processed successfully.}\\

    \midrule
    \multirow{4}{*}{\shortstack{Legal\\Terms}}
    & Exorbitant Legal Document Request Fee 
    & {Responding to requests for production of documents, and other matters requiring more than mere ministerial activities on our part, will incur a fee of \hl{two hundred dollars (\$200) per hour.}}
    & \halfFilledCircle & \halfFilledCircle & \halfFilledCircle\\
    \cmidrule(lr){2-6} 
    &  Forced Waiver of Legal Protections
    &  {You hereby \hl{waive California Civil Code Section 1542}. You hereby waive any similar provision in law, regulation, or code.}
    & \halfFilledCircle & \halfFilledCircle & \halfFilledCircle\\
    \cmidrule(lr){2-6} 
    & {Forced Waiver of Class Action Rights}
    & {This agreement includes a \hl{class action waiver} and an arbitration provision that governs any disputes between you and Sendinblue.}
    & \halfFilledCircle & \halfFilledCircle & \halfFilledCircle\\
    \cmidrule(lr){2-6} 
    & {Other Legal Unfavorable Financial Term}
    & {...}\\
    
    \bottomrule
\end{tabular*}
\label{table:taxonomy}

\end{table*}


\myparagraph{ShopTC-100K Dataset.} In the data collection stage, we extract 8,251 shopping websites from the Tranco top 100K, yielding 1.8 million terms.~\autoref{tab:dataset_stats} presents ShopTC-100K statistics alongside two fake e-commerce datasets, with unfavorable financial terms identified in later measurement study (\S\ref{sec:findings}). 



\subsection{\TermName Term Taxonomy}
\label{sec:financial_terms_section}




We categorize \termtypecnt types of \termname terms into \termcatcnt categories (\autoref{table:taxonomy}) and provide real-world examples analyzed against the three
criteria proposed by the FTC Act criteria (\S\ref{ssec:ftc})~\citep{ftcact}. A detailed taxonomy is in Appendix~\ref{sec:detailed_tax}. While not inherently deceptive, these terms often impose financial obligations consumers should recognize. We note that the severity of such terms depends on \textit{context}, which we leave for future research. We do not claim this list is exhaustive; however, it represents the most prominent types among the \termcnt terms from \websitecnt websites. We also report the financial term template in Appendix~\ref{sec:appendix_finaincial_terms}.

It is important to note that this paper \textit{does not} aim to analyze the fairness of terms from the legal perspective. We consider our work to be a complementary addition to the AI \& Law datasets~\citep{lippi2019claudette, galassi2024unfair}, by focusing on the natural phrasing found in online shopping websites' terms and conditions. A comparison between our \termname term template and previous work on online agreement fairness can be found in Appendix~\ref{sec:appendix_other_templates}.



%\elisa{need to mention it here, since in the TermLens section, we will say that the LLM module is pluggable and we can use fine-tuned version to do it, therefore we need to talk about the creation of fine-tuning dataset here. Make a table?}




\section{VideoDiff}
We present \sysname{} (Figure~\ref{fig:teaser}), a human-AI video co-creation tool designed to support efficient video editing with alternatives.
With \sysname{}, users can generate and review diverse AI recommendations for 3 different video editing tasks: making rough cuts, inserting B-roll images, and adding text effects (D3).
%(Figure~\ref{fig:teaser}.1)
VideoDiff supports easy comparison between alternatives by aligning videos (D1) and highlighting differences using timeline and transcript views (D2, D4)
%(Figure~\ref{fig:teaser}.2)
. Users can organize and customize edits by sorting, refining, and regenerating AI suggestions (D6). %(Figure~\ref{fig:teaser}.3). 


\begin{figure*}
  \centering
  \includegraphics[width=\textwidth]{figures/system_overview.pdf}
  \caption{Overview of VideoDiff: Users can view an outline of the variations in the current editing stage (a). In this figure, we see 10 rough cut variations. The user can play videos of these different versions (b) and compare them in the transcript or timeline view (c).
  Users can toggle between the edited and source timelines (d) to align videos to the source or edited context or click on each section to navigate directly to that part of the video (e).
  Users can also sort variations by duration and the number of sections included, as well as pin, archive, or edit variations according to their preferences (f).}\label{fig:sys_overview}
\end{figure*}


\subsection{Scope}\label{sec:scope}
\revised{
While VideoDiff provides visualizations that allow users to quickly identify and skim differences between video versions (D2), it is designed to go beyond the functionality of a \textit{diff tool}, which typically focuses solely on highlighting changes or differences for review~\cite{tharatipyakul2018towards, baker2024interaction}. Instead, we designed VideoDiff as a human-AI co-creation tool that facilitates iterative \textit{generate-compare-refine} interactions, revealing how comparing alternatives can influence and enhance users’ creative processes.}

To help set context for \sysname{}, we define the different stages of video editing.
Rough cut creation is selecting good moments (or clips) from source footage to create a compelling story. Inserting B-roll images or video helps to make the video more interesting and dynamic. To insert good B-roll, creators should find effective images or videos that illustrate what is being said and place them appropriately. Inserting text effects helps emphasize parts of the narration through animation and stylized text. The AI algorithms for each of these editing tasks are in themselves hard technical problems, and we do not focus on them in this work. Our focus is on supporting users to work with the generated variations as part of the editing process. However, because we must support some editing in order to test our ideas, we have implemented basic versions of editing algorithms that leverage LLMs to process the video transcript and recommend rough cuts, B-roll, and text effects. A holistic solution with multimodal analysis of video, audio, and narration can offer better editing suggestions in the future.

Most video editing softwares offer more than the three stages that VideoDiff supports, such as applying color correction or cleaning up the audio. Our goal is not to create a fully functional video editor, but rather to explore how video editing might change as AI technologies enter more and more of the editing stages. \camready{As AI easily generates multiple editing recommendations, the video editing task shifts to involve more curation beyond just editing, and we must consider how to best support this transition.}




\subsection{Interface}
VideoDiff is a web-based video editing tool (Figure~\ref{fig:sys_overview}) where users can generate, review, and customize video alternatives. 
When the user uploads a video, \sysname{} first generates 10 rough cut recommendations. In the \textit{versions outline} on the left, users can see the list of all variations in the current editing stage (Figure~\ref{fig:sys_overview}.a). Users can play the video of each version (Figure~\ref{fig:sys_overview}.b) or skim the differences using a timeline or transcript view (Figure~\ref{fig:sys_overview}.c).  Users can toggle between showing only the edited or all of the source content for additional context (Figure~\ref{fig:sys_overview}.d).  They can also click on any section to navigate directly to that part (Figure~\ref{fig:sys_overview}.e).
%or switch to transcript view to compare the variation side by side for a more detailed comparison (Figure~\ref{fig:sys_overview}.E). With the \textit{variation control}, 
%
Users can sort, re-order, pin and archive the variations to organize them (Figure~\ref{fig:sys_overview}.f). Or they can refine or recombine existing variations or regenerate a new variation using text prompts. 
%mira: this is already written down in the organizing section so removing from up here so it's not repetitive
%Sorting varies by editing stage. In the rough cut stage, users can sort by the duration and number of sections. In the B-roll editing stage, users can sort by the number and style of the images. In the text effects stage they can sort by the number and style of the text effects. 

\begin{figure}[t]
  \centering
  \includegraphics[width=\columnwidth]{figures/timelines_new.jpg}
  \caption{At each editing stage, VideoDiff provides glanceable timelines for users to easily compare different variations. Users can click on any section, B-roll image, or text effect to jump to that part of the video and preview the effects.}\label{fig:timelines}
\end{figure}

\begin{figure*}
  \centering
  \includegraphics[width=\textwidth]{figures/edited_source_final.jpg}
  \caption{Users can switch between the edited and original source timeline in the transcript (a) and timeline (b) views. This helps users see the edits in the context of the source content and compare which sections are included at a glance. The source view (c) shows users the location of the edited content in the context of the source view.}\label{fig:extracted_original}
\end{figure*}


\ipstart{Comparing Variations with Timeline View}
VideoDiff provides different visualizations of timelines at each video editing stage (Figure\ref{fig:timelines}) so that users can quickly review and compare multiple alternatives (D3). 
In the rough cut stage, VideoDiff uses ~\textit{sections} to visualize the timing and coverage of content in each variation.
Drawing upon prior work that has shown that grouping footage into thematically coherent chunks can help creators make video editing decisions~\cite{leake2020generating, huh2023avscript}, we explore how chunking into sections can aid comparison of edited videos. 
Instead of segmenting each rough cut variation to identify sections, VideoDiff extracts sections from the source footage and applies them consistently across all variations. This allows users to easily align the variations and see how different versions include or exclude certain sections and cover varying parts or lengths of each (D1-D2). 
Users can toggle between edited and source timeline views (Figure~\ref{fig:extracted_original}.a) to align the videos based on the edited or source video timeline. The edited timeline allows users to quickly skim the video's overall duration, along with the placement and proportion of each section. By hovering over a section, users can view relevant video thumbnails and click to play the video from that point.
With the source timeline, users can easily understand which parts of the source video are extracted. In the formative study, video creators often compared edited videos to the source videos to verify any missing key information. Drawing from this observation and aligning with D1, we use the source video as an anchor for aligning rough cut variations for comparison. 
% \mira{it's great to include this example but this text doesn't align with the figure. }
For example in Figure~\ref{fig:extracted_original}.b, while the edited timeline shows that both versions have a similar duration for ``Campus Highlights'' section, the source timeline reveals that each version covers different parts of the source footage on ``Campus Highlights''. By hovering over the lighter-colored boxes indicating excluded parts, users can see which video thumbnails and topic keywords are not covered.



Once the user chooses a rough-cut version, they can generate 10 videos with different B-roll recommendations. In the B-roll stage, VideoDiff's timeline view shows the B-roll thumbnails on top of the rough cut timeline bar, along with the keyword used to search for the B-rolls (Figure~\ref{fig:timelines}). Users can hover over the B-roll thumbnails to see which video scene each B-roll covers and click thumbnails to play the video and preview how the B-roll is inserted into the footage. After selecting a B-roll version, users can generate 10 new videos with different text effect suggestions. Instead of showing video thumbnails with text effects, we display the keywords where the text effects are applied, as text effects on thumbnails are too small to skim.
Users can hover over a keyword to view the narration sentence for context, and click on the keyword to play the video and preview how the text effects are integrated into the footage.




\begin{figure}[t]
  \centering
  \includegraphics[width=\columnwidth]{figures/transcripts_new.jpg}
  \caption{At each editing stage, VideoDiff provides glanceable transcripts so that users can easily review and compare different variations. In rough cut transcripts, visually concrete keywords~\cite{leake2020generating} are emphasized in bold, allowing users to easily skim through the content of each variation. Users can click on section headings, B-roll images, or text effects to jump to that part of the video and preview the effects.}\label{fig:transcripts}
\end{figure}

\ipstart{Comparing Variations with Transcript View}
VideoDiff's transcript view allows users to quickly skim the transcripts to understand the differences of variations in each editing stage (Figure~\ref{fig:transcripts}). 
The transcript is divided into sections and users can click on the section headings to play the video from the starting point of each section.
\camready{During the rough cut phase, VideoDiff aids users in maintaining context while toggling between transcript and timeline views by showing a mini timeline within the transcript view. It also uses consistent color coding to denote sections across both views.}
In the transcript, visually concrete keywords~\cite{leake2017computational} are emphasized in bold, allowing users to skim through the content of each variation~\footnote{We used GPT-o to identify visually concrete keywords with few-shot examples.}.
Similar to the timeline view, users can also switch to the source transcript view (Figure~\ref{fig:extracted_original}.c) and see the complete transcript and identify which parts of the source text are excluded in each section. We synchronize the scrolling across multiple transcripts, allowing users to easily compare variations side-by-side (D1). 

For quick skimming of B-roll and text effects options, VideoDiff highlights sentences in the transcript where these effects are applied (D2, Figure~\ref{fig:transcripts}). Users can click on the B-roll thumbnails or text effects to preview the video from the moment the effect appears.

% \mira{why only show the visual concrete words in the rough cut version?}

\ipstart{Organizing Variations}
To help users to easily explore and narrow down the search space of alternatives, VideoDiff supports sorting of variations in each stage. In the rough cut stage, users can sort based on the edited video duration and the number of sections, and in the B-roll and text effects stages, users can sort based on the number of effects and the media type of B-rolls (image, illustration, video) or style of text effects (title, subtitle, lower thirds). 
Additionally, users can manually pin preferred versions to the top or archive unwanted versions to the bottom, which is also reflected in the outline view.

\begin{figure}
  \centering
  \includegraphics[width=\columnwidth]{figures/refine.jpg}
  \caption{Using VideoDiff, users can edit a variation, recombine multiple variations, and generate a new variation using text prompts. For each new generation, VideoDiff summarizes the changes so that users can easily verify the result.}\label{fig:refine}
\end{figure}
    
\ipstart{Customizing Variations}
When users are not satisfied with the initial recommendations of VideoDiff, they can further edit existing versions or generate new alternatives (D6, Figure~\ref{fig:refine}).
Users can provide a text prompt to guide the generation of new alternatives (\textit{e.g., Show many text effects when I talk about grocery items.}), or click \textit{``Surprise me''} to have VideoDiff suggest a new alternative that is different from existing variations. 
Users can also recombine existing versions by specifying the version IDs and how to merge them in the text prompt (\textit{e.g., Use first two B-roll images from \#3 and last B-roll image from \#7.}). By default, generated new results are pinned to the top for easy discovery.
Using VideoDiff, users can also edit existing versions with a prompt (\textit{e.g., Shorten the part when I'm talking about the meal plans.}). The new generation from the edit prompt is displayed right below the original version for quick comparison and VideoDiff also describes specific changes made for quick verification of edits (D2). For example, in Figure~\ref{fig:refine} the user tells VideoDiff to ``shorten the part about dining halls'' and the system responds:
``Shortened the description of dining halls within the Dining and Housing section''. 
% \mira{how do you do this part with GPT?}




\subsection{Implementation \& Prompt Engineering}
We implemented VideoDiff using React.js and d3.js. For embedding a video player, we used Remotion~\cite{remotion} to render the edited video and overlay B-rolls and text effects. 
% Figure~\ref{fig:pipeline} illustrates the pipeline of VideoDiff.
When users upload a video, VideoDiff uses OpenAI's Whisper API to transcribe the video. VideoDiff is powered by OpenAI's GPT-4o and uses prompt engineering for 1) segmenting video into sections and identifying visually concrete keywords, 2) generating multiple alternatives of edit recommendations for rough cuts, B-rolls, and text effects, and 3) parsing and executing users' new generation prompts and summarizing changes. 
To ensure VideoDiff provides diverse edit recommendations in each stage, we use ~\textit{augmentation prompts} (\S\ref{apndx:augmentation_pipeline}) to control the generation of suggestion (\textit{e.g.,} by specifying duration and section coverage for each rough cut recommendation). 
%\mina{can we share the prompts in the appendix? Should ask Mira and Ding}
% - why we use them? -> to explore variable outputs, prior work utilized high-temperature, but it is difficult to describe their differences and adhere to user query
% - Luminate used dimensions to generate variations with prompting


\section{Evaluation and Large-Scale Measurement}

% Table generated by Excel2LaTeX from sheet 'Paper Version'
\begin{table*}[htbp]
  \centering
  \caption{\textbf{Classification}. AUC as a function of shot for three JoLT configurations and three competitive methods. Values are the mean and 95$\%$ confidence interval (CI) over 5 seeds that affect the training shot selection. Due to limited computational resources values at 16 and 32 shots with 0 CI use only a single seed. Competitive data from \citep{hegselmann2023tabllm}.}
  \label{tab:classification_results}%
  \vskip 0.05in
  \begin{small}
  \begin{sc}
  \begin{adjustbox}{max width=\textwidth}
    \begin{tabular}{rcccccc}
    \toprule
          &       & \multicolumn{5}{c}{\textbf{Shot}} \\
\cmidrule{3-7}    \multicolumn{1}{l}{\textbf{Dataset}} & \textbf{Method} & \textbf{0} & \textbf{4} & \textbf{8} & \textbf{16} & \textbf{32} \\
    \midrule
          & XGBoost & -     & 0.5$\pm$0.00 & 0.56$\pm$0.08 & 0.68$\pm$0.04 & 0.76$\pm$0.03 \\
          & TabPFN & -     & 0.59$\pm$0.12 & 0.66$\pm$0.07 & 0.69$\pm$0.02 & 0.76$\pm$0.03 \\
    \multicolumn{1}{l}{bank} & TabLLM & 0.63$\pm$0.01 & 0.59$\pm$0.09 & 0.64$\pm$0.04 & 0.65$\pm$0.04 & 0.64$\pm$0.05 \\
          & JoLT (Gemma-2-2B) & 0.46$\pm$0.00 & 0.62$\pm$0.05 & 0.62$\pm$0.05 & 0.57$\pm$0.05 & 0.52$\pm$0.09 \\
          & JoLT (Gemma-2-27B) & 0.61$\pm$0.00 & 0.72$\pm$0.01 & 0.72$\pm$0.01 & 0.71$\pm$0.01 & 0.73$\pm$0.00 \\
          & JoLT (Qwen-2.5-72B) & 0.73$\pm$0.00 & 0.83$\pm$0.01 & 0.81$\pm$0.02 & 0.77$\pm$0.00 & 0.73$\pm$0.00 \\
    \midrule
          & XGBoost & -     & 0.5$\pm$0.00 & 0.58$\pm$0.06 & 0.66$\pm$0.04 & 0.67$\pm$0.05 \\
          & TabPFN & -     & 0.52$\pm$0.07 & 0.64$\pm$0.04 & 0.67$\pm$0.01 & 0.7$\pm$0.04 \\
    \multicolumn{1}{l}{blood} & TabLLM & 0.61$\pm$0.04 & 0.58$\pm$0.08 & 0.66$\pm$0.03 & 0.66$\pm$0.06 & 0.68$\pm$0.04 \\
          & JoLT (Gemma-2-2B) & 0.56$\pm$0.00 & 0.62$\pm$0.08 & 0.58$\pm$0.04 & 0.64$\pm$0.06 & 0.59$\pm$0.05 \\
          & JoLT (Gemma-2-27B) & 0.70$\pm$0.00 & 0.72$\pm$0.03 & 0.68$\pm$0.03 & 0.73$\pm$0.02 & 0.71$\pm$0.05 \\
          & JoLT (Qwen-2.5-72B) & 0.64$\pm$0.00 & 0.71$\pm$0.04 & 0.71$\pm$0.02 & 0.73$\pm$0.02 & 0.72$\pm$0.03 \\
    \midrule
          & XGBoost & -     & 0.5$\pm$0.00 & 0.62$\pm$0.09 & 0.74$\pm$0.03 & 0.79$\pm$0.04 \\
          & TabPFN & -     & 0.63$\pm$0.11 & 0.63$\pm$0.10 & 0.8$\pm$0.03 & 0.85$\pm$0.03 \\
    \multicolumn{1}{l}{calhousing} & TabLLM & 0.61$\pm$0.01 & 0.63$\pm$0.04 & 0.6$\pm$0.06 & 0.7$\pm$0.07 & 0.77$\pm$0.07 \\
          & JoLT (Gemma-2-2B) & 0.45$\pm$0.00 & 0.72$\pm$0.01 & 0.66$\pm$0.09 & 0.76$\pm$0.04 & 0.78$\pm$0.04 \\
          & JoLT (Gemma-2-27B) & 0.64$\pm$0.00 & 0.83$\pm$0.01 & 0.82$\pm$0.04 & 0.85$\pm$0.02 & 0.85$\pm$0.01 \\
          & JoLT (Qwen-2.5-72B) & 0.64$\pm$0.00 & 0.83$\pm$0.01 & 0.80$\pm$0.04 & 0.84$\pm$0.01 & 0.84$\pm$0.01 \\
    \midrule
          & XGBoost & -     & 0.5$\pm$0.00 & 0.59$\pm$0.04 & 0.7$\pm$0.07 & 0.82$\pm$0.03 \\
          & TabPFN & -     & 0.64$\pm$0.05 & 0.75$\pm$0.04 & 0.87$\pm$0.04 & 0.92$\pm$0.02 \\
    \multicolumn{1}{l}{car} & TabLLM & 0.82$\pm$0.02 & 0.83$\pm$0.03 & 0.85$\pm$0.03 & 0.86$\pm$0.03 & 0.91$\pm$0.02 \\
          & JoLT (Gemma-2-2B) & 0.73$\pm$0.00 & 0.84$\pm$0.02 & 0.79$\pm$0.02 & 0.79$\pm$0.04 & 0.74$\pm$0.04 \\
          & JoLT (Gemma-2-27B) & 0.82$\pm$0.00 & 0.89$\pm$0.01 & 0.90$\pm$0.01 & 0.91$\pm$0.01 & 0.94$\pm$0.01 \\
          & JoLT (Qwen-2.5-72B) & 0.86$\pm$0.00 & 0.89$\pm$0.04 & 0.88$\pm$0.02 & 0.89$\pm$0.04 & 0.94$\pm$0.01 \\
    \midrule
          & XGBoost & -     & 0.5$\pm$0.00 & 0.51$\pm$0.06 & 0.59$\pm$0.04 & 0.66$\pm$0.03 \\
          & TabPFN & -     & 0.58$\pm$0.07 & 0.59$\pm$0.03 & 0.64$\pm$0.05 & 0.69$\pm$0.06 \\
    \multicolumn{1}{l}{creditg} & TabLLM & 0.53$\pm$0.04 & 0.69$\pm$0.04 & 0.66$\pm$0.04 & 0.66$\pm$0.04 & 0.72$\pm$0.05 \\
          & JoLT (Gemma-2-2B) & 0.52$\pm$0.00 & 0.52$\pm$0.03 & 0.53$\pm$0.06 & 0.55$\pm$0.04 & 0.50$\pm$0.04 \\
          & JoLT (Gemma-2-27B) & 0.48$\pm$0.00 & 0.55$\pm$0.02 & 0.54$\pm$0.04 & 0.56$\pm$0.04 & 0.53$\pm$0.01 \\
          & JoLT (Qwen-2.5-72B) & 0.60$\pm$0.00 & 0.56$\pm$0.03 & 0.57$\pm$0.05 & 0.60$\pm$0.08 & 0.65$\pm$0.05 \\
    \midrule
          & XGBoost & -     & 0.5$\pm$0.00 & 0.59$\pm$0.14 & 0.72$\pm$0.06 & 0.69$\pm$0.07 \\
          & TabPFN & -     & 0.61$\pm$0.11 & 0.67$\pm$0.10 & 0.71$\pm$0.06 & 0.77$\pm$0.03 \\
    \multicolumn{1}{l}{diabetes} & TabLLM & 0.68$\pm$0.05 & 0.61$\pm$0.08 & 0.63$\pm$0.07 & 0.69$\pm$0.06 & 0.68$\pm$0.04 \\
          & JoLT (Gemma-2-2B) & 0.62$\pm$0.00 & 0.73$\pm$0.06 & 0.71$\pm$0.06 & 0.76$\pm$0.02 & 0.73$\pm$0.06 \\
          & JoLT (Gemma-2-27B) & 0.82$\pm$0.00 & 0.81$\pm$0.02 & 0.79$\pm$0.01 & 0.80$\pm$0.02 & 0.80$\pm$0.02 \\
          & JoLT (Qwen-2.5-72B) & 0.78$\pm$0.00 & 0.83$\pm$0.02 & 0.82$\pm$0.02 & 0.82$\pm$0.02 & 0.82$\pm$0.02 \\
    \midrule
          & XGBoost & -     & 0.5$\pm$0.00 & 0.55$\pm$0.12 & 0.84$\pm$0.06 & 0.88$\pm$0.04 \\
          & TabPFN & -     & 0.84$\pm$0.05 & 0.88$\pm$0.04 & 0.87$\pm$0.05 & 0.91$\pm$0.02 \\
    \multicolumn{1}{l}{heart} & TabLLM & 0.54$\pm$0.04 & 0.76$\pm$0.12 & 0.83$\pm$0.04 & 0.87$\pm$0.04 & 0.87$\pm$0.05 \\
          & JoLT (Gemma-2-2B) & 0.64$\pm$0.00 & 0.74$\pm$0.01 & 0.80$\pm$0.04 & 0.72$\pm$0.08 & 0.65$\pm$0.09 \\
          & JoLT (Gemma-2-27B) & 0.74$\pm$0.00 & 0.82$\pm$0.02 & 0.85$\pm$0.01 & 0.87$\pm$0.01 & 0.88$\pm$0.01 \\
          & JoLT (Qwen-2.5-72B) & 0.89$\pm$0.00 & 0.87$\pm$0.01 & 0.88$\pm$0.01 & 0.89$\pm$0.01 & 0.90$\pm$0.02 \\
    \midrule
          & XGBoost & -     & 0.5$\pm$0.00 & 0.59$\pm$0.05 & 0.77$\pm$0.02 & 0.79$\pm$0.03 \\
          & TabPFN & -     & 0.73$\pm$0.07 & 0.71$\pm$0.08 & 0.76$\pm$0.08 & 0.8$\pm$0.04 \\
    \multicolumn{1}{l}{income} & TabLLM & 0.84$\pm$0.00 & 0.84$\pm$0.01 & 0.84$\pm$0.02 & 0.84$\pm$0.04 & 0.84$\pm$0.01 \\
          & JoLT (Gemma-2-2B) & 0.82$\pm$0.00 & 0.82$\pm$0.02 & 0.82$\pm$0.01 & 0.83$\pm$0.02 & 0.83$\pm$0.01 \\
          & JoLT (Gemma-2-27B) & 0.86$\pm$0.00 & 0.85$\pm$0.00 & 0.85$\pm$0.01 & 0.85$\pm$0.01 & 0.85$\pm$0.00 \\
          & JoLT (Qwen-2.5-72B) & 0.83$\pm$0.00 & 0.86$\pm$0.00 & 0.85$\pm$0.01 & 0.86$\pm$0.00 & 0.86$\pm$0.00 \\
    \midrule
          & XGBoost & -     & 0.5$\pm$0.00 & 0.58$\pm$0.06 & 0.72$\pm$0.04 & 0.78$\pm$0.03 \\
          & TabPFN & -     & 0.65$\pm$0.07 & 0.72$\pm$0.04 & 0.71$\pm$0.06 & 0.78$\pm$0.02 \\
    \multicolumn{1}{l}{jungle} & TabLLM & 0.6$\pm$0.00 & 0.64$\pm$0.01 & 0.64$\pm$0.02 & 0.65$\pm$0.03 & 0.71$\pm$0.02 \\
          & JoLT (Gemma-2-2B) & 0.67$\pm$0.00 & 0.60$\pm$0.02 & 0.59$\pm$0.04 & 0.55$\pm$0.04 & 0.61$\pm$0.04 \\
          & JoLT (Gemma-2-27B) & 0.62$\pm$0.00 & 0.62$\pm$0.01 & 0.63$\pm$0.01 & 0.63$\pm$0.02 & 0.64$\pm$0.01 \\
          & JoLT (Qwen-2.5-72B) & 0.62$\pm$0.00 & 0.62$\pm$0.01 & 0.62$\pm$0.01 & 0.61$\pm$0.03 & 0.64$\pm$0.01 \\
    \bottomrule
    \end{tabular}%
    \end{adjustbox}
  \end{sc}
  \end{small}
  \vskip -0.1in
\end{table*}%







We implement and evaluate \platform using a manually annotated dataset. Our evaluation focuses on two key aspects: (1) assessing detection performance to determine how effectively LLMs, including both zero-shot and fine-tuned models, identify \termname terms (\S\ref{sec:eva}), and (2) analyzing findings from large-scale measurements using \platform (\S\ref{sec:findings}).



\subsection{Evaluation on an Annotated Dataset}
\label{sec:eva}

\myparagraph{Dataset} We create an annotated dataset by randomly selecting 500 terms from clusters of both \termname terms and negative clusters (i.e., benign financial or non-financial terms). This yields 250 potential \termname terms and 250 benign terms. Three researchers independently labeled the terms using the \termname template, without knowledge of the clusters. Disagreements were resolved in a second pass, and duplicates were removed, resulting in 489 final terms. The dataset was split into 244 terms for fine-tuning and 245 terms for validation (\autoref{tab:dataset_stats}). 


\myparagraph{Baselines}
To our knowledge, no prior work has directly addressed the detection of unfavorable financial terms. Recent advances in large language models (LLMs) demonstrate superior performance in common sense reasoning, complex text classification, and contextual understanding~\citep{gpt35,openai2023gpt4,touvron2023llama}, outperforming older models like BERT~\citep{devlin2018bert} and RoBERTa~\citep{liu2019roberta}. Therefore, we evaluate state-of-the-art LLMs: (1) GPT-3.5-Turbo~\citep{gpt35}, (2) GPT-4-Turbo~\citep{openai2023gpt4}, and (3) GPT-4o, along with two open-source LLMs: (1) LLaMA 3 8B~\citep{touvron2023llama} and (2) Gemma 2B~\citep{team2024gemma}.


\myparagraph{Evaluation Configurations}
We evaluate two configurations: (1) Zero-shot classification with a simple binary prompt describing the unfavorable financial term and a multi-class taxonomy prompt explaining term types, and (2) Fine-tuning the LLM using the taxonomy to improve detection accuracy (see prompts in Appendix~\ref{sec:prompts}).





\myparagraph{Metrics} We evaluate the models in terms of false positive rate, true positive rate, F1 score, and Area Under the Curve. AUC represents the area under the ROC (Receiver Operating Characteristic) curve, measuring the model's ability to distinguish between classes.



%\subsection{Performance Analysis}
%\label{sec:eva_performance}



\myparagraph{Zero-shot Classification Performance}
As a baseline for \termname term detection, we evaluated zero-shot classification with two prompts: (1) a simple prompt defining unfavorable financial terms and (2) a taxonomy prompt explaining term types. Using the taxonomy improved the True Positive Rate (TPR) by 4.4\% to 27.4\% and boosted the F1 score by 4.5\% to 21.1\%, showing a better balance of precision and recall. However, the False Positive Rate (FPR) increased in most cases, except for GPT-4o, where it dropped by 24.2\%. GPT-4o achieved the best overall performance with a TPR of 96.6\% and an F1 score of 82.5\%, demonstrating the importance of a \termname term taxonomy for more accurate detection.




\myparagraph{Fine-tuned LLM Classification Performance}
We fine-tune GPT-3.5-Turbo and GPT-4o for 4 epochs with a batch size of 1. Fine-tuning resulted in significant performance improvements, with GPT-4o achieving a True Positive Rate (TPR) of 92.1\% and an F1 score of 94.6\%. The fine-tuned GPT-4o model outperforms other LLMs in distinguishing true positives from false positives. These results demonstrate that fine-tuning, even with a limited dataset, can substantially enhance detection performance.




\subsection{Large-Scale Measurement}
\label{sec:findings}


To understand the prevalence of \termname terms, we deploy the fine-tuned GPT-4o model with \platform for detection. The backend detection system was applied to English shopping websites filtered from the Tranco list's top 100,000 sites, along with two fake e-commerce website datasets: the FCWs dataset~\citep{bitaab2023beyond} and the FLOS dataset~\citep{janaviciute2023fraudulent}. This large-scale measurement serves as a qualitative study on the prevalence of \termname terms in popular shopping websites. We present our findings below. %\elisa{maybe do a category analysis?}


\label{sec:categoring}

\begin{figure*}[ht!]
\centering
\caption{Statistics from Large-scale measurement of \termname term detection on Tranco top 100K websites.}
\Description[Statistical analysis of unfavorable financial terms across shopping websites.]
 {This figure presents multiple statistical analyses of unfavorable financial terms detected on shopping websites from the Tranco top 100K list.
 (a) CDF of the number of terms per website, showing how frequently terms appear on different websites.
 (b) CDF of the number of unfavorable financial terms per website, illustrating the distribution of problematic terms across the dataset.
 (c) Distribution of unfavorable financial terms across different categories of websites based on their Tranco ranking, highlighting differences between highly ranked and lower-ranked sites.
 (d) Comparison of Trustpilot ratings between the top 10 websites with the most unfavorable financial terms and a random sample of 40 websites, analyzing whether websites with more unfavorable terms tend to have lower consumer ratings.}

\label{fig:large_scale_stats}
\subfigure[CDF of the number of terms per website.]{
\includegraphics[width=0.18\textwidth,height=0.18\textwidth]{imgs/cdf_page_counts.pdf}}
\subfigure[CDF of the number of unfavorable financial terms per website.]{
\label{fig:term_length_cdf}
\includegraphics[width=0.18\textwidth,height=0.18\textwidth]{imgs/cdf_unfavorable_counts.pdf}}
\subfigure[Distribution of unfavorable financial terms in each category across Tranco-ranked websites.]{
\label{fig:tranco_rank_dist}
\includegraphics[width=0.26\textwidth,height=0.18\textwidth]{imgs/websites_by_ranking.pdf}}
\subfigure[Trustpilot ratings comparison between the top 10 websites with the most unfavorable financial terms and a random sample of 40 websites.]{
\label{fig:imagnet-comparison}
\includegraphics[width=0.26\textwidth,height=0.18\textwidth]{imgs/most_unfair.pdf}}

\end{figure*}
\myparagraph{Categorizing Websites with \TermName Terms}
As shown earlier in~\autoref{table:dataset_stats},  we collect terms and conditions from \websitecnt English shopping websites, resulting in \termcnt terms. Using a GPT-4o model with the \termname term taxonomy, 10,150 terms (approximately \termpct) were flagged as \termname terms. Notably, \websitepct (3,471 out of 8,251) of the English shopping websites from the top 100,000 Tranco-ranked sites contain at least one type of non-legal \termname term. ~\autoref{fig:large_scale_stats}(a) and (b) show the number of terms and \termname terms across 8,251 websites, underscoring how difficult it is for consumers to review lengthy T\&Cs and pinpoint questionable financial terms thoroughly. This emphasizes the importance of automated detection systems to protect users from unfavorable terms.


\myparagraph{Trend Analysis}
\autoref{fig:large_scale_stats}(c) shows the distribution of unfavorable financial terms across categories in the top 100K Tranco-ranked websites~\citep{tranco}. Post-purchase terms (yellow) are the most common across all ranking levels, with a higher concentration in lower-ranked sites, suggesting these terms are more frequent on less popular websites. Purchase and billing terms (blue) also have significant representation. Termination and account recovery terms (red) and legal terms (green) are less frequent but more evenly spread across the rankings. This trend highlights the widespread presence of unfavorable financial terms, especially on lower-ranked sites, underscoring the need for greater regulation to protect consumers from harmful practices, particularly on less reputable websites.


\myparagraph{Comparing ShopTC-100K with Fake E-commerce Datasets}
Interestingly, the percentage of websites with \termname terms from the Tranco list (\websitepct) is similar to that of fraudulent e-commerce websites (46.70\%). This suggests that unfavorable financial terms are not limited to fraudulent sites but are also prevalent among high-ranking websites, pointing to a broader issue in consumer protection. \textit{ShopTC-100K} websites have more \termname legal terms, indicating that legitimate websites are more inclined to shift liability onto customers than fraudulent ones.



\myparagraph{Qualitative Study on User rating}
From the English shopping websites in the top 100k Tranco list, we select those with the highest frequency of \termname terms across categories. We analyze Trustpilot~\citep{trustpilot} reviews for the top 10 websites in each \termname term category with the highest presence, alongside 40 randomly selected websites. 
As shown in~\autoref{fig:large_scale_stats}(d), websites with \termname terms tend to have lower Trustpilot ratings, particularly those with ``Post-Purchase Terms'' and ``Purchase and Billing Terms,'' indicating negative customer satisfaction. ``Termination and Account Recovery'' and ``Legal Terms'' also correlated with lower ratings, though with more variation, suggesting mixed experiences. This suggests a link between \termname terms and consumer dissatisfaction.




\myparagraph{Qualitative Study on Current Ecosystem Defense}
We examine whether the top 10 websites with the highest frequency of \termname terms are flagged by ScamAdviser~\citep{scamadviser2024website}, Google Safe Browsing~\citep{google2024safebrowsing}, and Microsoft Defender SmartScreen~\citep{microsoft2024smartscreen}. Out of 40 websites, only 6 (15\%) have a ScamAdviser score below 90, and 5 (12.5\%) scored below 10, while the majority receive a perfect score of 100. None of the websites are flagged by Google Safe Browsing or Microsoft Defender, which is expected since \termname terms are not inherently indicative of scams. %To further investigate, we analyze 34 websites flagged by crowd-sourced scam reporting sites such as ScammerInfo~\citep{scammerinfo}, ScamAdvisor~\citep{scamadvisor}, and ScamWatcher~\citep{scamwatcher}. These sites are flagged by \platform, and we manually confirm the presence of \termname terms. However, only 1 out of the 34 was flagged by Google Safe Browsing, and none were flagged by Microsoft Defender.
%, highlighting the lack of detection mechanisms for \termname terms.


\myparagraph{Qualitative Study on User Perception} To illustrate the potential harm of \termname terms,  we present four case studies on user perception and financial harm in each category in Appendix~\ref{sec:case_studies}. This underscores the urgent need for automated systems to detect \termname terms effectively.

\section{Discussion}
\label{sec:limitation}


We introduce \textit{TermMiner}, an open-source automated pipeline for collecting and modeling unfavorable financial terms in shopping websites with limited human involvement. Researchers can utilize our tools to examine various aspects of web-based text, such as readability or accessibility, and to conduct longitudinal studies. 
 


\platform assumes that the financial terms in question are not \textit{adversarially perturbed}. Recent studies have highlighted LLM vulnerabilities to jailbreak and prompt injection attacks~\citep{zou2023universal, liu2023autodan, greshake2023not}. These attacks can result in incorrect outputs. However, for T\&Cs, such adversarial perturbations are likely to be subjected to manual scrutiny, particularly in post-complaint scenarios, such as legal disputes~\citep{celsius}. We leave the exploration of adversarial robustness in LLM-based \termname term detection for future work.













%-------------------------------------------------------------------------------
\section{Conclusion}
%-------------------------------------------------------------------------------

This paper presents \sys, a memory offloading mechanism
for LLM serving that meets latency SLOs while maximizing the host 
memory usage. 
%
\sys captures the tradeoff between meeting SLOs and maximizing host memory usage 
with \interval, an internal tunable knob. 
%
In addition, \sys automatically decides the optimal \interval, \ie, the smallest \interval that meets SLOs, with a two-stage tuning approach.  
%
The first stage assumes bandwidth contention and profiles the GPU model offline, and generates a performance \record that, for any valid combination of SLOs, sequence lengths, and batching sizes, stores an optimal \interval,
%
The second stage adjusts the \interval for GPU instances sharing the bus to ensure that the SLOs can still be met while maximizing the aggregate host memory usage across all GPU instances. 
%
Our evaluation shows that \sys consistently maintains SLO under various runtime
scenarios, and outperforms \flexgen in throughput by 1.85\X, due to use 2.37\X more host memory. 



\bibliographystyle{ACM-Reference-Format}
% \bibliography{ref}
\documentclass[sigconf]{acmart}

\usepackage{soul}
\usepackage{listings}

\lstset{
    basicstyle=\ttfamily\small,  
    breaklines=true,            
    frame=single,               
    backgroundcolor=\color{gray!10},  
    rulecolor=\color{black},      
    xleftmargin=0pt,            
    framexleftmargin=0pt,       
    showstringspaces=false,     
    fontadjust=true  % <- Ensures it follows document-wide font settings
}


\usepackage{multirow}
\usepackage{subfigure}
\usepackage{xspace}
\usepackage{rotating}
\usepackage{algorithm}



\usepackage{algorithmic}
\usepackage{mathtools}


%math
\usepackage{amsmath}% http://ctan.org/pkg/amsmath

\newcommand{\eg}{e.g.\@\xspace}
\newcommand{\ie}{i.e.\@\xspace}
\newcommand{\etal}{et~al.\@\xspace}
\newcommand{\etc}{{\em etc}\xspace}
\newcommand{\TK}{{\bf TK}\xspace}
\newcommand{\tk}{\TK}

\newcommand{\financialcnt}{12\xspace}


\newcommand{\platform}{\textit{TermLens}\xspace}
\newcommand{\termname}{unfavorable financial\xspace}
\newcommand{\TermName}{Unfavorable Financial\xspace}
\newcommand{\Termname}{Unfavorable financial\xspace}
\newcommand{\websitecnt}{8,979\xspace}
\newcommand{\termcnt}{1.9 million\xspace}
\newcommand{\websitepct}{42.06\%\xspace}
\newcommand{\termpct}{0.5\%\xspace}
\newcommand{\termtypecnt}{22\xspace}
\newcommand{\termcatcnt}{4\xspace}
\newcommand{\myparagraph}[1]{\textbf{\textit{#1}:}\hspace{3pt}}


\usepackage{pifont} % Approved package
\newcommand{\filledCircle}{\ding{108}} % ● Full circle
\newcommand{\halfFilledCircle}{\ding{109}} % ◐ Half-filled circle



\copyrightyear{2025}
\acmYear{2025}
\setcopyright{cc}
\setcctype{by}
\acmConference[WWW '25]{Proceedings of the ACM Web Conference 2025}{April 28-May 2, 2025}{Sydney, NSW, Australia}
\acmBooktitle{Proceedings of the ACM Web Conference 2025 (WWW '25), April 28-May 2, 2025, Sydney, NSW, Australia}
\acmDOI{10.1145/3696410.3714573}
\acmISBN{979-8-4007-1274-6/25/04}

\settopmatter{printacmref=true}


%\showthe\font



\begin{document}

%%
%% The "title" command has an optional parameter,
%% allowing the author to define a "short title" to be used in page headers.
\title[Harmful Terms and Where to Find Them]{Harmful Terms and Where to Find Them: Measuring and Modeling Unfavorable Financial Terms and Conditions in Shopping Websites at Scale}


\author{Elisa Tsai}
\affiliation{%
  \institution{University of Michigan}
  \city{Ann Arbor}
  \state{Michigan}
  \country{USA}
}
\email{eltsai@umich.edu}


\author{Neal Mangaokar}
\affiliation{
  \institution{University of Michigan}
  \city{Ann Arbor}
  \state{Michigan}
  \country{USA}
}
\email{nealmgkr@umich.edu}


\author{Boyuan Zheng}
\affiliation{
  \institution{University of Michigan}
  \city{Ann Arbor}
  \state{Michigan}
  \country{USA}
}
\email{boyuann@umich.edu}

\author{Haizhong Zheng}
\affiliation{
  \institution{University of Michigan}
  \city{Ann Arbor}
  \state{Michigan}
  \country{USA}
}
\email{hzzheng@umich.edu}

\author{Atul Prakash}
\affiliation{
  \institution{University of Michigan}
  \city{Ann Arbor}
  \state{Michigan}
  \country{USA}
}
\email{aprakash@umich.edu}



%%
%% By default, the full list of authors will be used in the page
%% headers. Often, this list is too long, and will overlap
%% other information printed in the page headers. This command allows
%% the author to define a more concise list
%% of authors' names for this purpose.

%%
%% The abstract is a short summary of the work to be presented in the
%% article.
\begin{abstract}
Terms and conditions for online shopping websites often contain terms that can have significant financial consequences for customers. 
Despite their impact, there is currently no comprehensive understanding of the types and potential risks associated with unfavorable financial terms. Furthermore, there are no publicly available detection systems or datasets to systematically identify or mitigate these terms.
In this paper, we take the first steps toward solving this problem with three key contributions.

\textit{First}, we introduce \textit{TermMiner}, an automated data collection and topic modeling pipeline to understand the landscape of unfavorable financial terms.
\textit{Second}, we create \textit{ShopTC-100K}, a dataset of terms and conditions from shopping websites in the Tranco top 100K list, comprising 1.8 million terms from 8,251 websites. Consequently, we develop a taxonomy of 22 types from 4 categories of unfavorable financial terms---spanning purchase, post-purchase, account termination, and legal aspects.
\textit{Third}, we build \textit{TermLens}, an automated detector that uses Large Language Models (LLMs) to identify unfavorable financial terms. 

Fine-tuned on an annotated dataset, \textit{TermLens} achieves an F1 score of 94.6\% and a false positive rate of 2.3\% using GPT-4o. 
When applied to shopping websites from the Tranco top 100K, we find that 42.06\% of these sites contain at least one unfavorable financial term, with such terms being more prevalent on less popular websites. Case studies further highlight the financial risks and customer dissatisfaction associated with unfavorable financial terms, as well as the limitations of existing ecosystem defenses.

\end{abstract}

%%
%% The code below is generated by the tool at http://dl.acm.org/ccs.cfm.
%% Please copy and paste the code instead of the example below.
%%

\begin{CCSXML}
<ccs2012>
   <concept>
       <concept_id>10002951.10003260.10003277</concept_id>
       <concept_desc>Information systems~Web mining</concept_desc>
       <concept_significance>500</concept_significance>
       </concept>
   <concept>
       <concept_id>10002978.10002997.10003000</concept_id>
       <concept_desc>Security and privacy~Social engineering attacks</concept_desc>
       <concept_significance>300</concept_significance>
       </concept>
   <concept>
       <concept_id>10003456.10003462.10003544.10011709</concept_id>
       <concept_desc>Social and professional topics~Consumer products policy</concept_desc>
       <concept_significance>500</concept_significance>
       </concept>
 </ccs2012>
\end{CCSXML}

\ccsdesc[500]{Information systems~Web mining}
\ccsdesc[300]{Security and privacy~Social engineering attacks}
\ccsdesc[500]{Social and professional topics~Consumer products policy}



%%
%% Keywords. The author(s) should pick words that accurately describe
%% the work being presented. Separate the keywords with commas.
\keywords{Topic modeling; Unfavorable financial terms; Consumer protection; Terms and conditions dataset; Deceptive content}

%% This command processes the author and affiliation and title
%% information and builds the first part of the formatted document.
\maketitle



%\section{CCS Concepts and User-Defined Keywords}

%Two elements of the ``acmart'' document class provide powerful
%taxonomic tools for you to help readers find your work in an online search.

%The ACM Computing Classification System ---
%\url{https://www.acm.org/publications/class-2012} --- is a set of
%classifiers and concepts that describe the computing
%discipline. Authors can select entries from this classification
%system, via \url{https://dl.acm.org/ccs/ccs.cfm}, and generate the
%commands to be included in the \LaTeX\ source.

%User-defined keywords are a comma-separated list of words and phrases
%of the authors' choosing, providing a more flexible way of describing
%the research being presented.

%CCS concepts and user-defined keywords are required for for all
%articles over two pages in length, and are optional for one- and
%two-page articles (or abstracts).



%%
%% The next two lines define the bibliography style to be used, and
%% the bibliography file.


%%%%%%%%%%%%%%%%%%%%%%%%%%%%%%%%%%%%%%
% Main text
%%%%%%%%%%%%%%%%%%%%%%%%%%%%%%%%%%%%%%

Large language models (LLMs) show significant performance in various downstream
tasks~\citep{brown_language_2020,openai_gpt-4_2024,dubey_llama_2024}. Studies
have found that training on high quality corpus improves the ability of LLMs
to solve different problems such as writing code, doing math exercises, and
answering logic questions~\citep{cai_internlm2_2024,deepseek-ai_deepseek-v3_2024,qwen_qwen25_2024}.
Therefore, effectively selecting high-quality text data is an important subject for
training LLM.

\begin{figure}[t]
    \centering
    \includegraphics[width=\linewidth]{figures/head.pdf}
    \caption{The overview of CritiQ. We (1) employ human annotators to annotate $\sim$30
    pairwise quality comparisons, (2) use CritiQ Flow to mine quality criteria, (3)
    use the derived criteria to annotate 25k pairs, and (4) train the CritiQ Scorer to
    perform efficient data selection.}
    \label{fig:overview}
\end{figure}

To select high-quality data from a large corpus, researchers manually design heuristics~\citep{dubey_llama_2024,rae_scaling_2022},
calculate perplexity using existing LLMs~\citep{marion2023moreinvestigatingdatapruning,wenzek2019ccnetextractinghighquality},
train classifiers~\citep{brown_language_2020,dubey_llama_2024,xie_data_2023} and
query LLMs for text quality through careful prompt engineering~\citep{gunasekar_textbooks_2023,wettig_qurating_2024,sachdeva_how_2024}.
Large-scale human annotation and prompt engineering require a lot of human
effort. Giving a comprehensive description of what high-quality data is like is also
challenging. As a result, manually designing heuristics lacks robustness and introduces
biases to the data processing pipeline, potentially harming model performance
and generalization. In addition, quality standards vary across different
domains. These methods can not be directly applied to other domains without significant
modifications.

To address these problems, we introduce CritiQ, a novel method to automatically
and effectively capture human preferences for data quality and perform efficient data
selection. Figure~\ref{fig:overview} gives an overview of CritiQ, comprising an agent
workflow, CritiQ Flow, and a scoring model, CritiQ Scorer. Instead of manually describing
how high quality is defined, we employ LLM-based agents to summarize quality
criteria from only $\sim$30 human-annotated pairs.

CritiQ Flow starts from a knowledge base of data quality criteria. The worker
agents are responsible to perform pairwise judgment under a given
criterion. The manager agent generates new criteria and refines them through reflection
on worker agents' performance. The final judgment is made by majority voting among
all worker agents, which gives a multi-perspective view of data quality.

To perform efficient data selection, we employ the worker agents to annotate a randomly
selected pairwise subset, which is ~1000x larger than the human-annotated one.
Following \citet{korbak_pretraining_2023,wettig_qurating_2024}, we train CritiQ
Scorer, a lightweight Bradley-Terry model~\citep{bradley_rank_1952} to convert
pairwise preferences into numerical scores for each text. We use CritiQ Scorer to
score the entire corpus and sample the high-quality subset.

For our experiments, we established human-annotated test sets to quantitatively
evaluate the agreement rate with human annotators on data quality preferences. We implemented the manager agent by \texttt{GPT-4o} and the worker
agent by \texttt{Qwen2.5-72B-Insruct}. We conducted experiments on different
domains including code, math, and logic, in which CritiQ Flow shows a consistent
improvement in the accuracies on the test sets, demonstrating the effectiveness
of our method in capturing human preferences for data quality. To validate the quality
of the selected dataset, we continually train \texttt{Llama 3.1}~\citep{dubey_llama_2024}
models and find that the models achieve better performance on downstream tasks
compared to models trained on the uniformly sampled subsets.

We highlight our contributions as follows. We will release the code to facilitate
future research.

\begin{itemize}
    \item We introduce CritiQ, a method that captures human preferences for data
        quality and performs efficient data selection at little cost of human
        annotation effort.

    \item Continual pretraining experiments show improved model performance in code,
        math, and logic tasks trained on our selected high-quality subset compared to the raw dataset.

    \item Ablation studies demonstrate the effectiveness of the knowledge base and
        the the reflection process.
\end{itemize}

\begin{figure*}[t]
    \centering
    \includegraphics[width=\linewidth]{figures/method.pdf}
    \caption{CritiQ Flow comprises two major components: multi-criteria pairwise
    judgment and the criteria evolution process. The multi-criteria pairwise
    judgment process employs a series of worker agents to make quality
    comparisons under a certain criterion. The criteria evolution process aims to
    obtain data quality criteria that highly align with human judgment through
    an iterative evolution. The initial criteria are retrieved from the
    knowledge base. After evolution, we select the final criteria to annotate
    the dataset for training CritiQ Scorer.}
    \label{fig:method}
\end{figure*}

\section{Related Work}


\myparagraph{Scam and fake e-commerce website detection}
Detection methods for scam and fake e-commerce websites (FCW) typically rely on two types of features: external (e.g., URLs, certificates, logos, redirect mechanisms)~\citep{blum2010lexical, zouina2017novel, moghimi2016new, sakurai2020discovering, drury2019certified, van2022logomotive, li2018fake, zhang2014you, zheng2017smoke, sahingoz2019machine, bitaab2023beyond} and content-based (e.g., visual and HTML structures, images, scripts, hyperlinks)~\citep{xiang2011cantina+, kharraz2018surveylance, yang2019phishing, jain2017phishing, bitaab2023beyond, yang2023trident}. These models are either rule-based or machine learning-based, with feature selection grounded in domain knowledge (e.g., indicative images, third-party scripts). However, no prior work in this line has considered terms and conditions and their financial impacts on users.


We consider social engineering scams to overlap with our detection target. The \termname terms in \autoref{fig:example} function similarly by deceiving users into signing up for additional subscriptions. However, as discussed in \S\ref{sec:financial_terms_section} and~\S\ref{sec:categoring}, \termname terms are not exclusive to scam websites. Therefore, consumers should be alerted to the presence of such terms. We view our work as the first to measure \termname terms at scale.

\input{fig_tex/data_collection_pipeline}

\myparagraph{Dark patterns}
Dark patterns are deceptive user interface designs intended to manipulate users into actions against their best interests~\citep{mathur2019dark}. Recent research has examined their psychological impact on user decision-making~\citep{mathur2021makes,nouwens2020dark,waldman2020cognitive,narayanan2020dark}, while also exploring legal frameworks and strategies for intervention~\citep{luguri2021shining,gray2021dark}.

Although terms and conditions are not part of the user interface design, we consider the \termname terms we identify to be closely related to dark patterns. The unilateral nature of these terms and their potential to hide uncommon or unexpected terms make them closely align with the characteristics of dark patterns: asymmetric, covert, deceptive, hiding information, and restrictive.



\myparagraph{Terms and conditions legal analysis} 
There is limited NLP-based analysis of legal documents like online contracts and terms of service~\citep{lagioia2017automated, braun2021nlp, lippi2019claudette, limsopatham2021effectively, jablonowska2021assessing, galassi2024unfair}. Prior studies, such as Lippi \etal~\citep{lippi2019claudette} and Galassi \etal~\citep{galassi2024unfair}, typically focus on small datasets of T\&Cs (25 to 200 documents). However, their focus is mainly on assessing fairness under the EU’s Unfair Contract Terms Directive~\citep{CouncilDirective1993} (i.e., clauses invalid in court). In contrast, our work specifically targets terms with more direct financial impacts on users.


In this paper, we focus on the financial terms in the large-scale measurement of terms and conditions from English shopping websites, assessed using the definition of unfair acts or practices as provided by the Federal Trade Commission (FTC)'s Policy Statement on Deception~\citep{ftc1983deception}. A detailed comparison of our proposed term taxonomy with prior work is provided in Appendix~\ref{sec:appendix_other_templates}.

\myparagraph{Privacy policy analysis}
A significant body of work investigates the viability of NLP-based analysis for privacy policies. One significant line of such research focuses on detecting contradicting policy statements (e.g., via ontologies~\citep{andow2019policylint} and knowledge graphs~\citep{cui2023poligraph}) or ambiguities~\citep{shvartzshnaider2019going}. Other areas include improving user comprehension~\citep{harkous2018polisis}, topic modeling, and summarization~\citep{alabduljabbar2021automated, sarne2019unsupervised}.

In this work, we focus on financial terms which are distinct from privacy policies. While we also perform topic modeling, we are the first to apply such a pipeline to construct a taxonomy for \termname terms. Furthermore, detecting contradictions and ambiguities is orthogonal to the detection of malicious financial terms, making it difficult to apply similar techniques directly.








\section{Understanding Unfavorable Financial Terms}
\label{sec:topic_modeling_section}

In this section, we outline our detection goal and present \textit{TermMiner}, a pipeline for collecting, clustering, and topic modeling \termname terms from English shopping websites in the Tranco top 100,000~\citep{tranco} and datasets of fraudulent e-commerce sites~\citep{bitaab2023beyond, janaviciute2023fraudulent}. As shown in \autoref{fig:financial_term_pipeline}, the pipeline categorizes two types of terms: (1) financial terms that may have immediate or future financial impacts, and (2) \termname terms, identified as unfair, unfavorable, or concerning for customers. We then summarize the taxonomy of \termname terms, which fall into four broad categories: (1) purchase and billing, (2) post-purchase, (3) termination and account recovery, and (4) legal terms.

\subsection{Threat Model}
\label{ssec:ftc}

We aim to detect one-sided, imbalanced, unfair, or malicious \textit{financial} terms in online shopping websites' terms and conditions, which could pose significant risks to users, potentially leading to unexpected financial losses. These risks can arise from website operators seeking to limit liability or from intentional malfeasance.

To assess whether a financial term is unfavorable, we refer to Section 5 of the Federal Trade Commission (FTC) Act~\citep{ftcact}, which defines an act as unfair if it meets the following criteria:

\begin{itemize}
    \item \textbf{C1: Substantial Injury}. It causes or is likely to cause substantial injury to consumers;
    \item \textbf{C2: Unavoidable Harm}. Consumers cannot reasonably avoid it; and
    \item \textbf{C3: Insufficient Benefits}. It is not outweighed by countervailing benefits to consumers or competition. 
\end{itemize}

During the topic modeling of term clusters, we judge the topic representing each cluster by three criteria to evaluate their fairness.

\myparagraph{C1} Since we focus on financial terms with potentially detrimental impacts, all financial terms inherently satisfy this criterion.


\myparagraph{C2} Terms and conditions are often hidden or difficult to avoid. Fair financial terms must be clearly displayed at critical points, like the payment page. However, terms related to cancellation, refunds, and returns are rarely shown upfront. We evaluate terms for unexpected fees (e.g., cancellation charges, non-refundable items, costly returns) that place an undue burden on consumers.


\myparagraph{C3} We classify terms as benign if they serve legitimate user or business protection, such as terms prohibiting fraud or abuse, protecting intellectual property, or ensuring legal compliance.



\subsection{Data Collection and Topic Modeling}
\label{ssec:data_collection_and_topic_modeling}

As shown in~\autoref{fig:financial_term_pipeline}, we introduce \textit{TermMiner}, a data collection and topic modeling pipeline for identifying \termname term at scale.  By integrating LLMs like GPT-4o~\citep{openai2023gpt4}, \textit{TermMiner} significantly reduces the extensive manual efforts required in previous web content mining studies, such as those focused on detecting dark patterns~\citep{mathur2019dark}. 
\textit{TermMiner} is open-sourced and can be repurposed for various web-based text analysis tasks or longitudinal studies. Researchers can use our tools to explore different aspects of terms and conditions, such as readability, accessibility, or fairness.


\myparagraph{A Two-Pass Method}
In the data collection and topic modeling steps, we employ a two-pass method. The first pass focuses on modeling and detecting \textit{financial terms} to develop a corresponding topic template. In the second pass, we use the detected financial terms to re-conduct the classification and topic modeling modules. This time, the goal is to detect \termname terms within the financial terms identified. This approach is necessary because, to the best of our knowledge, there are no established templates or annotation schemes for (1) financial terms or (2) \termname terms in online shopping agreements. This two-pass process ensures comprehensive detection and accurate categorization of both financial and \termname terms.


\myparagraph{(1) Measurement Module} The measurement module collects terms and conditions from shopping websites to build a large, diverse dataset for analysis. For our large-scale measurement, we collect English shopping websites from two sources: the Tranco list~\citep{tranco}, a ranking of top global websites, and two fraudulent e-commerce datasets (FCWs~\citep{bitaab2023beyond} and the Fraudulent and Legitimate Online Shops Dataset~\citep{janaviciute2023fraudulent}). We filter out non-English content using Python's langdetect library~\citep{langdetect}. To classify shopping websites, we evaluate several configurations: (1) GPT-3.5-Turbo~\citep{gpt35} with URL, (2) GPT-3.5-Turbo with URL and HTML content, (3) GPT-4o~\citep{openai2023gpt4} with URL, and (4) GPT-4o with URL and website screenshot. To evaluate our classification methods, we manually annotated a sample of 500 websites from the Tranco list, categorizing them into ``shopping'' and ``non-shopping.'' GPT-4o, when prompted with URLs and screenshots, achieved an accuracy of 92\%, comparable to commercial website classification services~\citep{mathur2019dark} (see Appendix~\ref{sec:website_cls} for details). Therefore, we use this configuration throughout our work.





We subsequently crawl the shopping websites to collect terms and conditions pages. A snowballed regex matching method detects terms and any nested policy pages, refined through positive and negative regex patterns to improve accuracy. Starting with common anchor texts, we iteratively refine the regex patterns by analyzing T\&C links, which can be found in Appendix~\ref{sec:appendix_reg}. As shown in~\autoref{table:dataset_stats}, we collected \termcnt terms from \websitecnt websites in total. 



\myparagraph{(2) Classfication Module}
The classification module categorizes terms from shopping websites' terms and conditions into binary categories: positive or negative. The categorization is based on the detection goal (such as identifying financial terms or identifying \termname terms) using corresponding prompts with the GPT-4o model~\citep{openai2023gpt4}.




We opt for prompt engineering instead of fine-tuning the LLM for term classification to reduce costs. Prior work~\citep{sun2023text,openai2023bestpractices}, along with our empirical observation (see~\S\ref{sec:eva}), indicates that clear task descriptions and relevant examples (taxonomy) significantly enhance LLM performance in text classification. For \termname terms, we use the ``Unfavorable Term Taxonomy Prompt'' from Appendix~\ref{sec:prompts} and topics identified in the topic modeling step, to perform zero-shot term classification on a given set of terms and conditions. This process outputs sets of positive and negative terms, which are then used for clustering, inspection, and topic modeling. The resulting template generated from this analysis will, in turn, enhance the classification accuracy, creating a feedback loop that continuously improves our detection capabilities. 

\input{tables/dataset_stats}

\myparagraph{(3) Topic Modeling Module} The topic modeling module uses LLMs and manual inspection to organize terms into meaningful topics. We generate sentence embeddings with the T5 model~\citep{raffel2020exploring} and apply the DBSCAN clustering algorithm~\citep{ester1996density} to group terms by semantic similarity. The DBSCAN hyperparameters are decided through manual inspection. To extract high-frequency topics, we leverage GPT-4o~\citep{openai2023gpt4}, building on recent findings that show LLMs outperform traditional topic modeling methods like Latent Dirichlet Allocation (LDA)~\citep{blei2003latent} and BERTopic~\citep{grootendorst2022bertopic} in topic analysis~\citep{shrestha2023we, mu2024large}.



We develop an iterative topic modeling approach assisted by GPT-4o proceeds as follows: 
\begin{enumerate}
    \item We analyze DBSCAN clusters and create an initial topic template for financial terms.
    \item GPT-4o performs topic modeling on random samples from each cluster, assigning them to existing topics or suggesting new ones.
    \item We review and refine new topic suggestions through manual inspection, and updating the template.
    \item This process iterates until all clusters are assigned to a meaningful and satisfactory topic.
\end{enumerate}


This iterative workflow, combining clustering, human-guided template creation, and GPT-4o's advanced topic modeling capacity, enables efficient and comprehensive extraction of the topic template. We analyze 22,112 clusters in total, creating the \termname term taxonomy below.

\input{tables/malicious_terms}

\myparagraph{ShopTC-100K Dataset.} In the data collection stage, we extract 8,251 shopping websites from the Tranco top 100K, yielding 1.8 million terms.~\autoref{tab:dataset_stats} presents ShopTC-100K statistics alongside two fake e-commerce datasets, with unfavorable financial terms identified in later measurement study (\S\ref{sec:findings}). 



\subsection{\TermName Term Taxonomy}
\label{sec:financial_terms_section}




We categorize \termtypecnt types of \termname terms into \termcatcnt categories (\autoref{table:taxonomy}) and provide real-world examples analyzed against the three
criteria proposed by the FTC Act criteria (\S\ref{ssec:ftc})~\citep{ftcact}. A detailed taxonomy is in Appendix~\ref{sec:detailed_tax}. While not inherently deceptive, these terms often impose financial obligations consumers should recognize. We note that the severity of such terms depends on \textit{context}, which we leave for future research. We do not claim this list is exhaustive; however, it represents the most prominent types among the \termcnt terms from \websitecnt websites. We also report the financial term template in Appendix~\ref{sec:appendix_finaincial_terms}.

It is important to note that this paper \textit{does not} aim to analyze the fairness of terms from the legal perspective. We consider our work to be a complementary addition to the AI \& Law datasets~\citep{lippi2019claudette, galassi2024unfair}, by focusing on the natural phrasing found in online shopping websites' terms and conditions. A comparison between our \termname term template and previous work on online agreement fairness can be found in Appendix~\ref{sec:appendix_other_templates}.



%\elisa{need to mention it here, since in the TermLens section, we will say that the LLM module is pluggable and we can use fine-tuned version to do it, therefore we need to talk about the creation of fine-tuning dataset here. Make a table?}




\section{VideoDiff}
We present \sysname{} (Figure~\ref{fig:teaser}), a human-AI video co-creation tool designed to support efficient video editing with alternatives.
With \sysname{}, users can generate and review diverse AI recommendations for 3 different video editing tasks: making rough cuts, inserting B-roll images, and adding text effects (D3).
%(Figure~\ref{fig:teaser}.1)
VideoDiff supports easy comparison between alternatives by aligning videos (D1) and highlighting differences using timeline and transcript views (D2, D4)
%(Figure~\ref{fig:teaser}.2)
. Users can organize and customize edits by sorting, refining, and regenerating AI suggestions (D6). %(Figure~\ref{fig:teaser}.3). 


\begin{figure*}
  \centering
  \includegraphics[width=\textwidth]{figures/system_overview.pdf}
  \caption{Overview of VideoDiff: Users can view an outline of the variations in the current editing stage (a). In this figure, we see 10 rough cut variations. The user can play videos of these different versions (b) and compare them in the transcript or timeline view (c).
  Users can toggle between the edited and source timelines (d) to align videos to the source or edited context or click on each section to navigate directly to that part of the video (e).
  Users can also sort variations by duration and the number of sections included, as well as pin, archive, or edit variations according to their preferences (f).}\label{fig:sys_overview}
\end{figure*}


\subsection{Scope}\label{sec:scope}
\revised{
While VideoDiff provides visualizations that allow users to quickly identify and skim differences between video versions (D2), it is designed to go beyond the functionality of a \textit{diff tool}, which typically focuses solely on highlighting changes or differences for review~\cite{tharatipyakul2018towards, baker2024interaction}. Instead, we designed VideoDiff as a human-AI co-creation tool that facilitates iterative \textit{generate-compare-refine} interactions, revealing how comparing alternatives can influence and enhance users’ creative processes.}

To help set context for \sysname{}, we define the different stages of video editing.
Rough cut creation is selecting good moments (or clips) from source footage to create a compelling story. Inserting B-roll images or video helps to make the video more interesting and dynamic. To insert good B-roll, creators should find effective images or videos that illustrate what is being said and place them appropriately. Inserting text effects helps emphasize parts of the narration through animation and stylized text. The AI algorithms for each of these editing tasks are in themselves hard technical problems, and we do not focus on them in this work. Our focus is on supporting users to work with the generated variations as part of the editing process. However, because we must support some editing in order to test our ideas, we have implemented basic versions of editing algorithms that leverage LLMs to process the video transcript and recommend rough cuts, B-roll, and text effects. A holistic solution with multimodal analysis of video, audio, and narration can offer better editing suggestions in the future.

Most video editing softwares offer more than the three stages that VideoDiff supports, such as applying color correction or cleaning up the audio. Our goal is not to create a fully functional video editor, but rather to explore how video editing might change as AI technologies enter more and more of the editing stages. \camready{As AI easily generates multiple editing recommendations, the video editing task shifts to involve more curation beyond just editing, and we must consider how to best support this transition.}




\subsection{Interface}
VideoDiff is a web-based video editing tool (Figure~\ref{fig:sys_overview}) where users can generate, review, and customize video alternatives. 
When the user uploads a video, \sysname{} first generates 10 rough cut recommendations. In the \textit{versions outline} on the left, users can see the list of all variations in the current editing stage (Figure~\ref{fig:sys_overview}.a). Users can play the video of each version (Figure~\ref{fig:sys_overview}.b) or skim the differences using a timeline or transcript view (Figure~\ref{fig:sys_overview}.c).  Users can toggle between showing only the edited or all of the source content for additional context (Figure~\ref{fig:sys_overview}.d).  They can also click on any section to navigate directly to that part (Figure~\ref{fig:sys_overview}.e).
%or switch to transcript view to compare the variation side by side for a more detailed comparison (Figure~\ref{fig:sys_overview}.E). With the \textit{variation control}, 
%
Users can sort, re-order, pin and archive the variations to organize them (Figure~\ref{fig:sys_overview}.f). Or they can refine or recombine existing variations or regenerate a new variation using text prompts. 
%mira: this is already written down in the organizing section so removing from up here so it's not repetitive
%Sorting varies by editing stage. In the rough cut stage, users can sort by the duration and number of sections. In the B-roll editing stage, users can sort by the number and style of the images. In the text effects stage they can sort by the number and style of the text effects. 

\begin{figure}[t]
  \centering
  \includegraphics[width=\columnwidth]{figures/timelines_new.jpg}
  \caption{At each editing stage, VideoDiff provides glanceable timelines for users to easily compare different variations. Users can click on any section, B-roll image, or text effect to jump to that part of the video and preview the effects.}\label{fig:timelines}
\end{figure}

\begin{figure*}
  \centering
  \includegraphics[width=\textwidth]{figures/edited_source_final.jpg}
  \caption{Users can switch between the edited and original source timeline in the transcript (a) and timeline (b) views. This helps users see the edits in the context of the source content and compare which sections are included at a glance. The source view (c) shows users the location of the edited content in the context of the source view.}\label{fig:extracted_original}
\end{figure*}


\ipstart{Comparing Variations with Timeline View}
VideoDiff provides different visualizations of timelines at each video editing stage (Figure\ref{fig:timelines}) so that users can quickly review and compare multiple alternatives (D3). 
In the rough cut stage, VideoDiff uses ~\textit{sections} to visualize the timing and coverage of content in each variation.
Drawing upon prior work that has shown that grouping footage into thematically coherent chunks can help creators make video editing decisions~\cite{leake2020generating, huh2023avscript}, we explore how chunking into sections can aid comparison of edited videos. 
Instead of segmenting each rough cut variation to identify sections, VideoDiff extracts sections from the source footage and applies them consistently across all variations. This allows users to easily align the variations and see how different versions include or exclude certain sections and cover varying parts or lengths of each (D1-D2). 
Users can toggle between edited and source timeline views (Figure~\ref{fig:extracted_original}.a) to align the videos based on the edited or source video timeline. The edited timeline allows users to quickly skim the video's overall duration, along with the placement and proportion of each section. By hovering over a section, users can view relevant video thumbnails and click to play the video from that point.
With the source timeline, users can easily understand which parts of the source video are extracted. In the formative study, video creators often compared edited videos to the source videos to verify any missing key information. Drawing from this observation and aligning with D1, we use the source video as an anchor for aligning rough cut variations for comparison. 
% \mira{it's great to include this example but this text doesn't align with the figure. }
For example in Figure~\ref{fig:extracted_original}.b, while the edited timeline shows that both versions have a similar duration for ``Campus Highlights'' section, the source timeline reveals that each version covers different parts of the source footage on ``Campus Highlights''. By hovering over the lighter-colored boxes indicating excluded parts, users can see which video thumbnails and topic keywords are not covered.



Once the user chooses a rough-cut version, they can generate 10 videos with different B-roll recommendations. In the B-roll stage, VideoDiff's timeline view shows the B-roll thumbnails on top of the rough cut timeline bar, along with the keyword used to search for the B-rolls (Figure~\ref{fig:timelines}). Users can hover over the B-roll thumbnails to see which video scene each B-roll covers and click thumbnails to play the video and preview how the B-roll is inserted into the footage. After selecting a B-roll version, users can generate 10 new videos with different text effect suggestions. Instead of showing video thumbnails with text effects, we display the keywords where the text effects are applied, as text effects on thumbnails are too small to skim.
Users can hover over a keyword to view the narration sentence for context, and click on the keyword to play the video and preview how the text effects are integrated into the footage.




\begin{figure}[t]
  \centering
  \includegraphics[width=\columnwidth]{figures/transcripts_new.jpg}
  \caption{At each editing stage, VideoDiff provides glanceable transcripts so that users can easily review and compare different variations. In rough cut transcripts, visually concrete keywords~\cite{leake2020generating} are emphasized in bold, allowing users to easily skim through the content of each variation. Users can click on section headings, B-roll images, or text effects to jump to that part of the video and preview the effects.}\label{fig:transcripts}
\end{figure}

\ipstart{Comparing Variations with Transcript View}
VideoDiff's transcript view allows users to quickly skim the transcripts to understand the differences of variations in each editing stage (Figure~\ref{fig:transcripts}). 
The transcript is divided into sections and users can click on the section headings to play the video from the starting point of each section.
\camready{During the rough cut phase, VideoDiff aids users in maintaining context while toggling between transcript and timeline views by showing a mini timeline within the transcript view. It also uses consistent color coding to denote sections across both views.}
In the transcript, visually concrete keywords~\cite{leake2017computational} are emphasized in bold, allowing users to skim through the content of each variation~\footnote{We used GPT-o to identify visually concrete keywords with few-shot examples.}.
Similar to the timeline view, users can also switch to the source transcript view (Figure~\ref{fig:extracted_original}.c) and see the complete transcript and identify which parts of the source text are excluded in each section. We synchronize the scrolling across multiple transcripts, allowing users to easily compare variations side-by-side (D1). 

For quick skimming of B-roll and text effects options, VideoDiff highlights sentences in the transcript where these effects are applied (D2, Figure~\ref{fig:transcripts}). Users can click on the B-roll thumbnails or text effects to preview the video from the moment the effect appears.

% \mira{why only show the visual concrete words in the rough cut version?}

\ipstart{Organizing Variations}
To help users to easily explore and narrow down the search space of alternatives, VideoDiff supports sorting of variations in each stage. In the rough cut stage, users can sort based on the edited video duration and the number of sections, and in the B-roll and text effects stages, users can sort based on the number of effects and the media type of B-rolls (image, illustration, video) or style of text effects (title, subtitle, lower thirds). 
Additionally, users can manually pin preferred versions to the top or archive unwanted versions to the bottom, which is also reflected in the outline view.

\begin{figure}
  \centering
  \includegraphics[width=\columnwidth]{figures/refine.jpg}
  \caption{Using VideoDiff, users can edit a variation, recombine multiple variations, and generate a new variation using text prompts. For each new generation, VideoDiff summarizes the changes so that users can easily verify the result.}\label{fig:refine}
\end{figure}
    
\ipstart{Customizing Variations}
When users are not satisfied with the initial recommendations of VideoDiff, they can further edit existing versions or generate new alternatives (D6, Figure~\ref{fig:refine}).
Users can provide a text prompt to guide the generation of new alternatives (\textit{e.g., Show many text effects when I talk about grocery items.}), or click \textit{``Surprise me''} to have VideoDiff suggest a new alternative that is different from existing variations. 
Users can also recombine existing versions by specifying the version IDs and how to merge them in the text prompt (\textit{e.g., Use first two B-roll images from \#3 and last B-roll image from \#7.}). By default, generated new results are pinned to the top for easy discovery.
Using VideoDiff, users can also edit existing versions with a prompt (\textit{e.g., Shorten the part when I'm talking about the meal plans.}). The new generation from the edit prompt is displayed right below the original version for quick comparison and VideoDiff also describes specific changes made for quick verification of edits (D2). For example, in Figure~\ref{fig:refine} the user tells VideoDiff to ``shorten the part about dining halls'' and the system responds:
``Shortened the description of dining halls within the Dining and Housing section''. 
% \mira{how do you do this part with GPT?}




\subsection{Implementation \& Prompt Engineering}
We implemented VideoDiff using React.js and d3.js. For embedding a video player, we used Remotion~\cite{remotion} to render the edited video and overlay B-rolls and text effects. 
% Figure~\ref{fig:pipeline} illustrates the pipeline of VideoDiff.
When users upload a video, VideoDiff uses OpenAI's Whisper API to transcribe the video. VideoDiff is powered by OpenAI's GPT-4o and uses prompt engineering for 1) segmenting video into sections and identifying visually concrete keywords, 2) generating multiple alternatives of edit recommendations for rough cuts, B-rolls, and text effects, and 3) parsing and executing users' new generation prompts and summarizing changes. 
To ensure VideoDiff provides diverse edit recommendations in each stage, we use ~\textit{augmentation prompts} (\S\ref{apndx:augmentation_pipeline}) to control the generation of suggestion (\textit{e.g.,} by specifying duration and section coverage for each rough cut recommendation). 
%\mina{can we share the prompts in the appendix? Should ask Mira and Ding}
% - why we use them? -> to explore variable outputs, prior work utilized high-temperature, but it is difficult to describe their differences and adhere to user query
% - Luminate used dimensions to generate variations with prompting


\section{Evaluation and Large-Scale Measurement}

\input{tables/classification_results}



We implement and evaluate \platform using a manually annotated dataset. Our evaluation focuses on two key aspects: (1) assessing detection performance to determine how effectively LLMs, including both zero-shot and fine-tuned models, identify \termname terms (\S\ref{sec:eva}), and (2) analyzing findings from large-scale measurements using \platform (\S\ref{sec:findings}).



\subsection{Evaluation on an Annotated Dataset}
\label{sec:eva}

\myparagraph{Dataset} We create an annotated dataset by randomly selecting 500 terms from clusters of both \termname terms and negative clusters (i.e., benign financial or non-financial terms). This yields 250 potential \termname terms and 250 benign terms. Three researchers independently labeled the terms using the \termname template, without knowledge of the clusters. Disagreements were resolved in a second pass, and duplicates were removed, resulting in 489 final terms. The dataset was split into 244 terms for fine-tuning and 245 terms for validation (\autoref{tab:dataset_stats}). 


\myparagraph{Baselines}
To our knowledge, no prior work has directly addressed the detection of unfavorable financial terms. Recent advances in large language models (LLMs) demonstrate superior performance in common sense reasoning, complex text classification, and contextual understanding~\citep{gpt35,openai2023gpt4,touvron2023llama}, outperforming older models like BERT~\citep{devlin2018bert} and RoBERTa~\citep{liu2019roberta}. Therefore, we evaluate state-of-the-art LLMs: (1) GPT-3.5-Turbo~\citep{gpt35}, (2) GPT-4-Turbo~\citep{openai2023gpt4}, and (3) GPT-4o, along with two open-source LLMs: (1) LLaMA 3 8B~\citep{touvron2023llama} and (2) Gemma 2B~\citep{team2024gemma}.


\myparagraph{Evaluation Configurations}
We evaluate two configurations: (1) Zero-shot classification with a simple binary prompt describing the unfavorable financial term and a multi-class taxonomy prompt explaining term types, and (2) Fine-tuning the LLM using the taxonomy to improve detection accuracy (see prompts in Appendix~\ref{sec:prompts}).





\myparagraph{Metrics} We evaluate the models in terms of false positive rate, true positive rate, F1 score, and Area Under the Curve. AUC represents the area under the ROC (Receiver Operating Characteristic) curve, measuring the model's ability to distinguish between classes.



%\subsection{Performance Analysis}
%\label{sec:eva_performance}



\myparagraph{Zero-shot Classification Performance}
As a baseline for \termname term detection, we evaluated zero-shot classification with two prompts: (1) a simple prompt defining unfavorable financial terms and (2) a taxonomy prompt explaining term types. Using the taxonomy improved the True Positive Rate (TPR) by 4.4\% to 27.4\% and boosted the F1 score by 4.5\% to 21.1\%, showing a better balance of precision and recall. However, the False Positive Rate (FPR) increased in most cases, except for GPT-4o, where it dropped by 24.2\%. GPT-4o achieved the best overall performance with a TPR of 96.6\% and an F1 score of 82.5\%, demonstrating the importance of a \termname term taxonomy for more accurate detection.




\myparagraph{Fine-tuned LLM Classification Performance}
We fine-tune GPT-3.5-Turbo and GPT-4o for 4 epochs with a batch size of 1. Fine-tuning resulted in significant performance improvements, with GPT-4o achieving a True Positive Rate (TPR) of 92.1\% and an F1 score of 94.6\%. The fine-tuned GPT-4o model outperforms other LLMs in distinguishing true positives from false positives. These results demonstrate that fine-tuning, even with a limited dataset, can substantially enhance detection performance.




\subsection{Large-Scale Measurement}
\label{sec:findings}


To understand the prevalence of \termname terms, we deploy the fine-tuned GPT-4o model with \platform for detection. The backend detection system was applied to English shopping websites filtered from the Tranco list's top 100,000 sites, along with two fake e-commerce website datasets: the FCWs dataset~\citep{bitaab2023beyond} and the FLOS dataset~\citep{janaviciute2023fraudulent}. This large-scale measurement serves as a qualitative study on the prevalence of \termname terms in popular shopping websites. We present our findings below. %\elisa{maybe do a category analysis?}


\label{sec:categoring}

\input{fig_tex/large_scale_measurement_stats}
\myparagraph{Categorizing Websites with \TermName Terms}
As shown earlier in~\autoref{table:dataset_stats},  we collect terms and conditions from \websitecnt English shopping websites, resulting in \termcnt terms. Using a GPT-4o model with the \termname term taxonomy, 10,150 terms (approximately \termpct) were flagged as \termname terms. Notably, \websitepct (3,471 out of 8,251) of the English shopping websites from the top 100,000 Tranco-ranked sites contain at least one type of non-legal \termname term. ~\autoref{fig:large_scale_stats}(a) and (b) show the number of terms and \termname terms across 8,251 websites, underscoring how difficult it is for consumers to review lengthy T\&Cs and pinpoint questionable financial terms thoroughly. This emphasizes the importance of automated detection systems to protect users from unfavorable terms.


\myparagraph{Trend Analysis}
\autoref{fig:large_scale_stats}(c) shows the distribution of unfavorable financial terms across categories in the top 100K Tranco-ranked websites~\citep{tranco}. Post-purchase terms (yellow) are the most common across all ranking levels, with a higher concentration in lower-ranked sites, suggesting these terms are more frequent on less popular websites. Purchase and billing terms (blue) also have significant representation. Termination and account recovery terms (red) and legal terms (green) are less frequent but more evenly spread across the rankings. This trend highlights the widespread presence of unfavorable financial terms, especially on lower-ranked sites, underscoring the need for greater regulation to protect consumers from harmful practices, particularly on less reputable websites.


\myparagraph{Comparing ShopTC-100K with Fake E-commerce Datasets}
Interestingly, the percentage of websites with \termname terms from the Tranco list (\websitepct) is similar to that of fraudulent e-commerce websites (46.70\%). This suggests that unfavorable financial terms are not limited to fraudulent sites but are also prevalent among high-ranking websites, pointing to a broader issue in consumer protection. \textit{ShopTC-100K} websites have more \termname legal terms, indicating that legitimate websites are more inclined to shift liability onto customers than fraudulent ones.



\myparagraph{Qualitative Study on User rating}
From the English shopping websites in the top 100k Tranco list, we select those with the highest frequency of \termname terms across categories. We analyze Trustpilot~\citep{trustpilot} reviews for the top 10 websites in each \termname term category with the highest presence, alongside 40 randomly selected websites. 
As shown in~\autoref{fig:large_scale_stats}(d), websites with \termname terms tend to have lower Trustpilot ratings, particularly those with ``Post-Purchase Terms'' and ``Purchase and Billing Terms,'' indicating negative customer satisfaction. ``Termination and Account Recovery'' and ``Legal Terms'' also correlated with lower ratings, though with more variation, suggesting mixed experiences. This suggests a link between \termname terms and consumer dissatisfaction.




\myparagraph{Qualitative Study on Current Ecosystem Defense}
We examine whether the top 10 websites with the highest frequency of \termname terms are flagged by ScamAdviser~\citep{scamadviser2024website}, Google Safe Browsing~\citep{google2024safebrowsing}, and Microsoft Defender SmartScreen~\citep{microsoft2024smartscreen}. Out of 40 websites, only 6 (15\%) have a ScamAdviser score below 90, and 5 (12.5\%) scored below 10, while the majority receive a perfect score of 100. None of the websites are flagged by Google Safe Browsing or Microsoft Defender, which is expected since \termname terms are not inherently indicative of scams. %To further investigate, we analyze 34 websites flagged by crowd-sourced scam reporting sites such as ScammerInfo~\citep{scammerinfo}, ScamAdvisor~\citep{scamadvisor}, and ScamWatcher~\citep{scamwatcher}. These sites are flagged by \platform, and we manually confirm the presence of \termname terms. However, only 1 out of the 34 was flagged by Google Safe Browsing, and none were flagged by Microsoft Defender.
%, highlighting the lack of detection mechanisms for \termname terms.


\myparagraph{Qualitative Study on User Perception} To illustrate the potential harm of \termname terms,  we present four case studies on user perception and financial harm in each category in Appendix~\ref{sec:case_studies}. This underscores the urgent need for automated systems to detect \termname terms effectively.

\section{Discussion}
\label{sec:limitation}


We introduce \textit{TermMiner}, an open-source automated pipeline for collecting and modeling unfavorable financial terms in shopping websites with limited human involvement. Researchers can utilize our tools to examine various aspects of web-based text, such as readability or accessibility, and to conduct longitudinal studies. 
 


\platform assumes that the financial terms in question are not \textit{adversarially perturbed}. Recent studies have highlighted LLM vulnerabilities to jailbreak and prompt injection attacks~\citep{zou2023universal, liu2023autodan, greshake2023not}. These attacks can result in incorrect outputs. However, for T\&Cs, such adversarial perturbations are likely to be subjected to manual scrutiny, particularly in post-complaint scenarios, such as legal disputes~\citep{celsius}. We leave the exploration of adversarial robustness in LLM-based \termname term detection for future work.













%-------------------------------------------------------------------------------
\section{Conclusion}
%-------------------------------------------------------------------------------

This paper presents \sys, a memory offloading mechanism
for LLM serving that meets latency SLOs while maximizing the host 
memory usage. 
%
\sys captures the tradeoff between meeting SLOs and maximizing host memory usage 
with \interval, an internal tunable knob. 
%
In addition, \sys automatically decides the optimal \interval, \ie, the smallest \interval that meets SLOs, with a two-stage tuning approach.  
%
The first stage assumes bandwidth contention and profiles the GPU model offline, and generates a performance \record that, for any valid combination of SLOs, sequence lengths, and batching sizes, stores an optimal \interval,
%
The second stage adjusts the \interval for GPU instances sharing the bus to ensure that the SLOs can still be met while maximizing the aggregate host memory usage across all GPU instances. 
%
Our evaluation shows that \sys consistently maintains SLO under various runtime
scenarios, and outperforms \flexgen in throughput by 1.85\X, due to use 2.37\X more host memory. 



\bibliographystyle{ACM-Reference-Format}
% \bibliography{ref}
\documentclass[sigconf]{acmart}

\usepackage{soul}
\usepackage{listings}

\lstset{
    basicstyle=\ttfamily\small,  
    breaklines=true,            
    frame=single,               
    backgroundcolor=\color{gray!10},  
    rulecolor=\color{black},      
    xleftmargin=0pt,            
    framexleftmargin=0pt,       
    showstringspaces=false,     
    fontadjust=true  % <- Ensures it follows document-wide font settings
}


\usepackage{multirow}
\usepackage{subfigure}
\usepackage{xspace}
\usepackage{rotating}
\usepackage{algorithm}



\usepackage{algorithmic}
\usepackage{mathtools}


%math
\usepackage{amsmath}% http://ctan.org/pkg/amsmath

\newcommand{\eg}{e.g.\@\xspace}
\newcommand{\ie}{i.e.\@\xspace}
\newcommand{\etal}{et~al.\@\xspace}
\newcommand{\etc}{{\em etc}\xspace}
\newcommand{\TK}{{\bf TK}\xspace}
\newcommand{\tk}{\TK}

\newcommand{\financialcnt}{12\xspace}


\newcommand{\platform}{\textit{TermLens}\xspace}
\newcommand{\termname}{unfavorable financial\xspace}
\newcommand{\TermName}{Unfavorable Financial\xspace}
\newcommand{\Termname}{Unfavorable financial\xspace}
\newcommand{\websitecnt}{8,979\xspace}
\newcommand{\termcnt}{1.9 million\xspace}
\newcommand{\websitepct}{42.06\%\xspace}
\newcommand{\termpct}{0.5\%\xspace}
\newcommand{\termtypecnt}{22\xspace}
\newcommand{\termcatcnt}{4\xspace}
\newcommand{\myparagraph}[1]{\textbf{\textit{#1}:}\hspace{3pt}}


\usepackage{pifont} % Approved package
\newcommand{\filledCircle}{\ding{108}} % ● Full circle
\newcommand{\halfFilledCircle}{\ding{109}} % ◐ Half-filled circle



\copyrightyear{2025}
\acmYear{2025}
\setcopyright{cc}
\setcctype{by}
\acmConference[WWW '25]{Proceedings of the ACM Web Conference 2025}{April 28-May 2, 2025}{Sydney, NSW, Australia}
\acmBooktitle{Proceedings of the ACM Web Conference 2025 (WWW '25), April 28-May 2, 2025, Sydney, NSW, Australia}
\acmDOI{10.1145/3696410.3714573}
\acmISBN{979-8-4007-1274-6/25/04}

\settopmatter{printacmref=true}


%\showthe\font



\begin{document}

%%
%% The "title" command has an optional parameter,
%% allowing the author to define a "short title" to be used in page headers.
\title[Harmful Terms and Where to Find Them]{Harmful Terms and Where to Find Them: Measuring and Modeling Unfavorable Financial Terms and Conditions in Shopping Websites at Scale}


\author{Elisa Tsai}
\affiliation{%
  \institution{University of Michigan}
  \city{Ann Arbor}
  \state{Michigan}
  \country{USA}
}
\email{eltsai@umich.edu}


\author{Neal Mangaokar}
\affiliation{
  \institution{University of Michigan}
  \city{Ann Arbor}
  \state{Michigan}
  \country{USA}
}
\email{nealmgkr@umich.edu}


\author{Boyuan Zheng}
\affiliation{
  \institution{University of Michigan}
  \city{Ann Arbor}
  \state{Michigan}
  \country{USA}
}
\email{boyuann@umich.edu}

\author{Haizhong Zheng}
\affiliation{
  \institution{University of Michigan}
  \city{Ann Arbor}
  \state{Michigan}
  \country{USA}
}
\email{hzzheng@umich.edu}

\author{Atul Prakash}
\affiliation{
  \institution{University of Michigan}
  \city{Ann Arbor}
  \state{Michigan}
  \country{USA}
}
\email{aprakash@umich.edu}



%%
%% By default, the full list of authors will be used in the page
%% headers. Often, this list is too long, and will overlap
%% other information printed in the page headers. This command allows
%% the author to define a more concise list
%% of authors' names for this purpose.

%%
%% The abstract is a short summary of the work to be presented in the
%% article.
\begin{abstract}
Terms and conditions for online shopping websites often contain terms that can have significant financial consequences for customers. 
Despite their impact, there is currently no comprehensive understanding of the types and potential risks associated with unfavorable financial terms. Furthermore, there are no publicly available detection systems or datasets to systematically identify or mitigate these terms.
In this paper, we take the first steps toward solving this problem with three key contributions.

\textit{First}, we introduce \textit{TermMiner}, an automated data collection and topic modeling pipeline to understand the landscape of unfavorable financial terms.
\textit{Second}, we create \textit{ShopTC-100K}, a dataset of terms and conditions from shopping websites in the Tranco top 100K list, comprising 1.8 million terms from 8,251 websites. Consequently, we develop a taxonomy of 22 types from 4 categories of unfavorable financial terms---spanning purchase, post-purchase, account termination, and legal aspects.
\textit{Third}, we build \textit{TermLens}, an automated detector that uses Large Language Models (LLMs) to identify unfavorable financial terms. 

Fine-tuned on an annotated dataset, \textit{TermLens} achieves an F1 score of 94.6\% and a false positive rate of 2.3\% using GPT-4o. 
When applied to shopping websites from the Tranco top 100K, we find that 42.06\% of these sites contain at least one unfavorable financial term, with such terms being more prevalent on less popular websites. Case studies further highlight the financial risks and customer dissatisfaction associated with unfavorable financial terms, as well as the limitations of existing ecosystem defenses.

\end{abstract}

%%
%% The code below is generated by the tool at http://dl.acm.org/ccs.cfm.
%% Please copy and paste the code instead of the example below.
%%

\begin{CCSXML}
<ccs2012>
   <concept>
       <concept_id>10002951.10003260.10003277</concept_id>
       <concept_desc>Information systems~Web mining</concept_desc>
       <concept_significance>500</concept_significance>
       </concept>
   <concept>
       <concept_id>10002978.10002997.10003000</concept_id>
       <concept_desc>Security and privacy~Social engineering attacks</concept_desc>
       <concept_significance>300</concept_significance>
       </concept>
   <concept>
       <concept_id>10003456.10003462.10003544.10011709</concept_id>
       <concept_desc>Social and professional topics~Consumer products policy</concept_desc>
       <concept_significance>500</concept_significance>
       </concept>
 </ccs2012>
\end{CCSXML}

\ccsdesc[500]{Information systems~Web mining}
\ccsdesc[300]{Security and privacy~Social engineering attacks}
\ccsdesc[500]{Social and professional topics~Consumer products policy}



%%
%% Keywords. The author(s) should pick words that accurately describe
%% the work being presented. Separate the keywords with commas.
\keywords{Topic modeling; Unfavorable financial terms; Consumer protection; Terms and conditions dataset; Deceptive content}

%% This command processes the author and affiliation and title
%% information and builds the first part of the formatted document.
\maketitle



%\section{CCS Concepts and User-Defined Keywords}

%Two elements of the ``acmart'' document class provide powerful
%taxonomic tools for you to help readers find your work in an online search.

%The ACM Computing Classification System ---
%\url{https://www.acm.org/publications/class-2012} --- is a set of
%classifiers and concepts that describe the computing
%discipline. Authors can select entries from this classification
%system, via \url{https://dl.acm.org/ccs/ccs.cfm}, and generate the
%commands to be included in the \LaTeX\ source.

%User-defined keywords are a comma-separated list of words and phrases
%of the authors' choosing, providing a more flexible way of describing
%the research being presented.

%CCS concepts and user-defined keywords are required for for all
%articles over two pages in length, and are optional for one- and
%two-page articles (or abstracts).



%%
%% The next two lines define the bibliography style to be used, and
%% the bibliography file.


%%%%%%%%%%%%%%%%%%%%%%%%%%%%%%%%%%%%%%
% Main text
%%%%%%%%%%%%%%%%%%%%%%%%%%%%%%%%%%%%%%

\input{01introduction}
\input{02related_work}
\input{03data}
\input{04system}
\input{05evaluation}
\input{06limitation}
\input{07conclusion}

\bibliographystyle{ACM-Reference-Format}
% \bibliography{ref}
\input{00main.bbl}


%%
%% If your work has an appendix, this is the place to put it.
\input{appendix}

\end{document}
\endinput
%%
%% End of file `sample-sigconf.tex'.



%%
%% If your work has an appendix, this is the place to put it.
\subsection{Lloyd-Max Algorithm}
\label{subsec:Lloyd-Max}
For a given quantization bitwidth $B$ and an operand $\bm{X}$, the Lloyd-Max algorithm finds $2^B$ quantization levels $\{\hat{x}_i\}_{i=1}^{2^B}$ such that quantizing $\bm{X}$ by rounding each scalar in $\bm{X}$ to the nearest quantization level minimizes the quantization MSE. 

The algorithm starts with an initial guess of quantization levels and then iteratively computes quantization thresholds $\{\tau_i\}_{i=1}^{2^B-1}$ and updates quantization levels $\{\hat{x}_i\}_{i=1}^{2^B}$. Specifically, at iteration $n$, thresholds are set to the midpoints of the previous iteration's levels:
\begin{align*}
    \tau_i^{(n)}=\frac{\hat{x}_i^{(n-1)}+\hat{x}_{i+1}^{(n-1)}}2 \text{ for } i=1\ldots 2^B-1
\end{align*}
Subsequently, the quantization levels are re-computed as conditional means of the data regions defined by the new thresholds:
\begin{align*}
    \hat{x}_i^{(n)}=\mathbb{E}\left[ \bm{X} \big| \bm{X}\in [\tau_{i-1}^{(n)},\tau_i^{(n)}] \right] \text{ for } i=1\ldots 2^B
\end{align*}
where to satisfy boundary conditions we have $\tau_0=-\infty$ and $\tau_{2^B}=\infty$. The algorithm iterates the above steps until convergence.

Figure \ref{fig:lm_quant} compares the quantization levels of a $7$-bit floating point (E3M3) quantizer (left) to a $7$-bit Lloyd-Max quantizer (right) when quantizing a layer of weights from the GPT3-126M model at a per-tensor granularity. As shown, the Lloyd-Max quantizer achieves substantially lower quantization MSE. Further, Table \ref{tab:FP7_vs_LM7} shows the superior perplexity achieved by Lloyd-Max quantizers for bitwidths of $7$, $6$ and $5$. The difference between the quantizers is clear at 5 bits, where per-tensor FP quantization incurs a drastic and unacceptable increase in perplexity, while Lloyd-Max quantization incurs a much smaller increase. Nevertheless, we note that even the optimal Lloyd-Max quantizer incurs a notable ($\sim 1.5$) increase in perplexity due to the coarse granularity of quantization. 

\begin{figure}[h]
  \centering
  \includegraphics[width=0.7\linewidth]{sections/figures/LM7_FP7.pdf}
  \caption{\small Quantization levels and the corresponding quantization MSE of Floating Point (left) vs Lloyd-Max (right) Quantizers for a layer of weights in the GPT3-126M model.}
  \label{fig:lm_quant}
\end{figure}

\begin{table}[h]\scriptsize
\begin{center}
\caption{\label{tab:FP7_vs_LM7} \small Comparing perplexity (lower is better) achieved by floating point quantizers and Lloyd-Max quantizers on a GPT3-126M model for the Wikitext-103 dataset.}
\begin{tabular}{c|cc|c}
\hline
 \multirow{2}{*}{\textbf{Bitwidth}} & \multicolumn{2}{|c|}{\textbf{Floating-Point Quantizer}} & \textbf{Lloyd-Max Quantizer} \\
 & Best Format & Wikitext-103 Perplexity & Wikitext-103 Perplexity \\
\hline
7 & E3M3 & 18.32 & 18.27 \\
6 & E3M2 & 19.07 & 18.51 \\
5 & E4M0 & 43.89 & 19.71 \\
\hline
\end{tabular}
\end{center}
\end{table}

\subsection{Proof of Local Optimality of LO-BCQ}
\label{subsec:lobcq_opt_proof}
For a given block $\bm{b}_j$, the quantization MSE during LO-BCQ can be empirically evaluated as $\frac{1}{L_b}\lVert \bm{b}_j- \bm{\hat{b}}_j\rVert^2_2$ where $\bm{\hat{b}}_j$ is computed from equation (\ref{eq:clustered_quantization_definition}) as $C_{f(\bm{b}_j)}(\bm{b}_j)$. Further, for a given block cluster $\mathcal{B}_i$, we compute the quantization MSE as $\frac{1}{|\mathcal{B}_{i}|}\sum_{\bm{b} \in \mathcal{B}_{i}} \frac{1}{L_b}\lVert \bm{b}- C_i^{(n)}(\bm{b})\rVert^2_2$. Therefore, at the end of iteration $n$, we evaluate the overall quantization MSE $J^{(n)}$ for a given operand $\bm{X}$ composed of $N_c$ block clusters as:
\begin{align*}
    \label{eq:mse_iter_n}
    J^{(n)} = \frac{1}{N_c} \sum_{i=1}^{N_c} \frac{1}{|\mathcal{B}_{i}^{(n)}|}\sum_{\bm{v} \in \mathcal{B}_{i}^{(n)}} \frac{1}{L_b}\lVert \bm{b}- B_i^{(n)}(\bm{b})\rVert^2_2
\end{align*}

At the end of iteration $n$, the codebooks are updated from $\mathcal{C}^{(n-1)}$ to $\mathcal{C}^{(n)}$. However, the mapping of a given vector $\bm{b}_j$ to quantizers $\mathcal{C}^{(n)}$ remains as  $f^{(n)}(\bm{b}_j)$. At the next iteration, during the vector clustering step, $f^{(n+1)}(\bm{b}_j)$ finds new mapping of $\bm{b}_j$ to updated codebooks $\mathcal{C}^{(n)}$ such that the quantization MSE over the candidate codebooks is minimized. Therefore, we obtain the following result for $\bm{b}_j$:
\begin{align*}
\frac{1}{L_b}\lVert \bm{b}_j - C_{f^{(n+1)}(\bm{b}_j)}^{(n)}(\bm{b}_j)\rVert^2_2 \le \frac{1}{L_b}\lVert \bm{b}_j - C_{f^{(n)}(\bm{b}_j)}^{(n)}(\bm{b}_j)\rVert^2_2
\end{align*}

That is, quantizing $\bm{b}_j$ at the end of the block clustering step of iteration $n+1$ results in lower quantization MSE compared to quantizing at the end of iteration $n$. Since this is true for all $\bm{b} \in \bm{X}$, we assert the following:
\begin{equation}
\begin{split}
\label{eq:mse_ineq_1}
    \tilde{J}^{(n+1)} &= \frac{1}{N_c} \sum_{i=1}^{N_c} \frac{1}{|\mathcal{B}_{i}^{(n+1)}|}\sum_{\bm{b} \in \mathcal{B}_{i}^{(n+1)}} \frac{1}{L_b}\lVert \bm{b} - C_i^{(n)}(b)\rVert^2_2 \le J^{(n)}
\end{split}
\end{equation}
where $\tilde{J}^{(n+1)}$ is the the quantization MSE after the vector clustering step at iteration $n+1$.

Next, during the codebook update step (\ref{eq:quantizers_update}) at iteration $n+1$, the per-cluster codebooks $\mathcal{C}^{(n)}$ are updated to $\mathcal{C}^{(n+1)}$ by invoking the Lloyd-Max algorithm \citep{Lloyd}. We know that for any given value distribution, the Lloyd-Max algorithm minimizes the quantization MSE. Therefore, for a given vector cluster $\mathcal{B}_i$ we obtain the following result:

\begin{equation}
    \frac{1}{|\mathcal{B}_{i}^{(n+1)}|}\sum_{\bm{b} \in \mathcal{B}_{i}^{(n+1)}} \frac{1}{L_b}\lVert \bm{b}- C_i^{(n+1)}(\bm{b})\rVert^2_2 \le \frac{1}{|\mathcal{B}_{i}^{(n+1)}|}\sum_{\bm{b} \in \mathcal{B}_{i}^{(n+1)}} \frac{1}{L_b}\lVert \bm{b}- C_i^{(n)}(\bm{b})\rVert^2_2
\end{equation}

The above equation states that quantizing the given block cluster $\mathcal{B}_i$ after updating the associated codebook from $C_i^{(n)}$ to $C_i^{(n+1)}$ results in lower quantization MSE. Since this is true for all the block clusters, we derive the following result: 
\begin{equation}
\begin{split}
\label{eq:mse_ineq_2}
     J^{(n+1)} &= \frac{1}{N_c} \sum_{i=1}^{N_c} \frac{1}{|\mathcal{B}_{i}^{(n+1)}|}\sum_{\bm{b} \in \mathcal{B}_{i}^{(n+1)}} \frac{1}{L_b}\lVert \bm{b}- C_i^{(n+1)}(\bm{b})\rVert^2_2  \le \tilde{J}^{(n+1)}   
\end{split}
\end{equation}

Following (\ref{eq:mse_ineq_1}) and (\ref{eq:mse_ineq_2}), we find that the quantization MSE is non-increasing for each iteration, that is, $J^{(1)} \ge J^{(2)} \ge J^{(3)} \ge \ldots \ge J^{(M)}$ where $M$ is the maximum number of iterations. 
%Therefore, we can say that if the algorithm converges, then it must be that it has converged to a local minimum. 
\hfill $\blacksquare$


\begin{figure}
    \begin{center}
    \includegraphics[width=0.5\textwidth]{sections//figures/mse_vs_iter.pdf}
    \end{center}
    \caption{\small NMSE vs iterations during LO-BCQ compared to other block quantization proposals}
    \label{fig:nmse_vs_iter}
\end{figure}

Figure \ref{fig:nmse_vs_iter} shows the empirical convergence of LO-BCQ across several block lengths and number of codebooks. Also, the MSE achieved by LO-BCQ is compared to baselines such as MXFP and VSQ. As shown, LO-BCQ converges to a lower MSE than the baselines. Further, we achieve better convergence for larger number of codebooks ($N_c$) and for a smaller block length ($L_b$), both of which increase the bitwidth of BCQ (see Eq \ref{eq:bitwidth_bcq}).


\subsection{Additional Accuracy Results}
%Table \ref{tab:lobcq_config} lists the various LOBCQ configurations and their corresponding bitwidths.
\begin{table}
\setlength{\tabcolsep}{4.75pt}
\begin{center}
\caption{\label{tab:lobcq_config} Various LO-BCQ configurations and their bitwidths.}
\begin{tabular}{|c||c|c|c|c||c|c||c|} 
\hline
 & \multicolumn{4}{|c||}{$L_b=8$} & \multicolumn{2}{|c||}{$L_b=4$} & $L_b=2$ \\
 \hline
 \backslashbox{$L_A$\kern-1em}{\kern-1em$N_c$} & 2 & 4 & 8 & 16 & 2 & 4 & 2 \\
 \hline
 64 & 4.25 & 4.375 & 4.5 & 4.625 & 4.375 & 4.625 & 4.625\\
 \hline
 32 & 4.375 & 4.5 & 4.625& 4.75 & 4.5 & 4.75 & 4.75 \\
 \hline
 16 & 4.625 & 4.75& 4.875 & 5 & 4.75 & 5 & 5 \\
 \hline
\end{tabular}
\end{center}
\end{table}

%\subsection{Perplexity achieved by various LO-BCQ configurations on Wikitext-103 dataset}

\begin{table} \centering
\begin{tabular}{|c||c|c|c|c||c|c||c|} 
\hline
 $L_b \rightarrow$& \multicolumn{4}{c||}{8} & \multicolumn{2}{c||}{4} & 2\\
 \hline
 \backslashbox{$L_A$\kern-1em}{\kern-1em$N_c$} & 2 & 4 & 8 & 16 & 2 & 4 & 2  \\
 %$N_c \rightarrow$ & 2 & 4 & 8 & 16 & 2 & 4 & 2 \\
 \hline
 \hline
 \multicolumn{8}{c}{GPT3-1.3B (FP32 PPL = 9.98)} \\ 
 \hline
 \hline
 64 & 10.40 & 10.23 & 10.17 & 10.15 &  10.28 & 10.18 & 10.19 \\
 \hline
 32 & 10.25 & 10.20 & 10.15 & 10.12 &  10.23 & 10.17 & 10.17 \\
 \hline
 16 & 10.22 & 10.16 & 10.10 & 10.09 &  10.21 & 10.14 & 10.16 \\
 \hline
  \hline
 \multicolumn{8}{c}{GPT3-8B (FP32 PPL = 7.38)} \\ 
 \hline
 \hline
 64 & 7.61 & 7.52 & 7.48 &  7.47 &  7.55 &  7.49 & 7.50 \\
 \hline
 32 & 7.52 & 7.50 & 7.46 &  7.45 &  7.52 &  7.48 & 7.48  \\
 \hline
 16 & 7.51 & 7.48 & 7.44 &  7.44 &  7.51 &  7.49 & 7.47  \\
 \hline
\end{tabular}
\caption{\label{tab:ppl_gpt3_abalation} Wikitext-103 perplexity across GPT3-1.3B and 8B models.}
\end{table}

\begin{table} \centering
\begin{tabular}{|c||c|c|c|c||} 
\hline
 $L_b \rightarrow$& \multicolumn{4}{c||}{8}\\
 \hline
 \backslashbox{$L_A$\kern-1em}{\kern-1em$N_c$} & 2 & 4 & 8 & 16 \\
 %$N_c \rightarrow$ & 2 & 4 & 8 & 16 & 2 & 4 & 2 \\
 \hline
 \hline
 \multicolumn{5}{|c|}{Llama2-7B (FP32 PPL = 5.06)} \\ 
 \hline
 \hline
 64 & 5.31 & 5.26 & 5.19 & 5.18  \\
 \hline
 32 & 5.23 & 5.25 & 5.18 & 5.15  \\
 \hline
 16 & 5.23 & 5.19 & 5.16 & 5.14  \\
 \hline
 \multicolumn{5}{|c|}{Nemotron4-15B (FP32 PPL = 5.87)} \\ 
 \hline
 \hline
 64  & 6.3 & 6.20 & 6.13 & 6.08  \\
 \hline
 32  & 6.24 & 6.12 & 6.07 & 6.03  \\
 \hline
 16  & 6.12 & 6.14 & 6.04 & 6.02  \\
 \hline
 \multicolumn{5}{|c|}{Nemotron4-340B (FP32 PPL = 3.48)} \\ 
 \hline
 \hline
 64 & 3.67 & 3.62 & 3.60 & 3.59 \\
 \hline
 32 & 3.63 & 3.61 & 3.59 & 3.56 \\
 \hline
 16 & 3.61 & 3.58 & 3.57 & 3.55 \\
 \hline
\end{tabular}
\caption{\label{tab:ppl_llama7B_nemo15B} Wikitext-103 perplexity compared to FP32 baseline in Llama2-7B and Nemotron4-15B, 340B models}
\end{table}

%\subsection{Perplexity achieved by various LO-BCQ configurations on MMLU dataset}


\begin{table} \centering
\begin{tabular}{|c||c|c|c|c||c|c|c|c|} 
\hline
 $L_b \rightarrow$& \multicolumn{4}{c||}{8} & \multicolumn{4}{c||}{8}\\
 \hline
 \backslashbox{$L_A$\kern-1em}{\kern-1em$N_c$} & 2 & 4 & 8 & 16 & 2 & 4 & 8 & 16  \\
 %$N_c \rightarrow$ & 2 & 4 & 8 & 16 & 2 & 4 & 2 \\
 \hline
 \hline
 \multicolumn{5}{|c|}{Llama2-7B (FP32 Accuracy = 45.8\%)} & \multicolumn{4}{|c|}{Llama2-70B (FP32 Accuracy = 69.12\%)} \\ 
 \hline
 \hline
 64 & 43.9 & 43.4 & 43.9 & 44.9 & 68.07 & 68.27 & 68.17 & 68.75 \\
 \hline
 32 & 44.5 & 43.8 & 44.9 & 44.5 & 68.37 & 68.51 & 68.35 & 68.27  \\
 \hline
 16 & 43.9 & 42.7 & 44.9 & 45 & 68.12 & 68.77 & 68.31 & 68.59  \\
 \hline
 \hline
 \multicolumn{5}{|c|}{GPT3-22B (FP32 Accuracy = 38.75\%)} & \multicolumn{4}{|c|}{Nemotron4-15B (FP32 Accuracy = 64.3\%)} \\ 
 \hline
 \hline
 64 & 36.71 & 38.85 & 38.13 & 38.92 & 63.17 & 62.36 & 63.72 & 64.09 \\
 \hline
 32 & 37.95 & 38.69 & 39.45 & 38.34 & 64.05 & 62.30 & 63.8 & 64.33  \\
 \hline
 16 & 38.88 & 38.80 & 38.31 & 38.92 & 63.22 & 63.51 & 63.93 & 64.43  \\
 \hline
\end{tabular}
\caption{\label{tab:mmlu_abalation} Accuracy on MMLU dataset across GPT3-22B, Llama2-7B, 70B and Nemotron4-15B models.}
\end{table}


%\subsection{Perplexity achieved by various LO-BCQ configurations on LM evaluation harness}

\begin{table} \centering
\begin{tabular}{|c||c|c|c|c||c|c|c|c|} 
\hline
 $L_b \rightarrow$& \multicolumn{4}{c||}{8} & \multicolumn{4}{c||}{8}\\
 \hline
 \backslashbox{$L_A$\kern-1em}{\kern-1em$N_c$} & 2 & 4 & 8 & 16 & 2 & 4 & 8 & 16  \\
 %$N_c \rightarrow$ & 2 & 4 & 8 & 16 & 2 & 4 & 2 \\
 \hline
 \hline
 \multicolumn{5}{|c|}{Race (FP32 Accuracy = 37.51\%)} & \multicolumn{4}{|c|}{Boolq (FP32 Accuracy = 64.62\%)} \\ 
 \hline
 \hline
 64 & 36.94 & 37.13 & 36.27 & 37.13 & 63.73 & 62.26 & 63.49 & 63.36 \\
 \hline
 32 & 37.03 & 36.36 & 36.08 & 37.03 & 62.54 & 63.51 & 63.49 & 63.55  \\
 \hline
 16 & 37.03 & 37.03 & 36.46 & 37.03 & 61.1 & 63.79 & 63.58 & 63.33  \\
 \hline
 \hline
 \multicolumn{5}{|c|}{Winogrande (FP32 Accuracy = 58.01\%)} & \multicolumn{4}{|c|}{Piqa (FP32 Accuracy = 74.21\%)} \\ 
 \hline
 \hline
 64 & 58.17 & 57.22 & 57.85 & 58.33 & 73.01 & 73.07 & 73.07 & 72.80 \\
 \hline
 32 & 59.12 & 58.09 & 57.85 & 58.41 & 73.01 & 73.94 & 72.74 & 73.18  \\
 \hline
 16 & 57.93 & 58.88 & 57.93 & 58.56 & 73.94 & 72.80 & 73.01 & 73.94  \\
 \hline
\end{tabular}
\caption{\label{tab:mmlu_abalation} Accuracy on LM evaluation harness tasks on GPT3-1.3B model.}
\end{table}

\begin{table} \centering
\begin{tabular}{|c||c|c|c|c||c|c|c|c|} 
\hline
 $L_b \rightarrow$& \multicolumn{4}{c||}{8} & \multicolumn{4}{c||}{8}\\
 \hline
 \backslashbox{$L_A$\kern-1em}{\kern-1em$N_c$} & 2 & 4 & 8 & 16 & 2 & 4 & 8 & 16  \\
 %$N_c \rightarrow$ & 2 & 4 & 8 & 16 & 2 & 4 & 2 \\
 \hline
 \hline
 \multicolumn{5}{|c|}{Race (FP32 Accuracy = 41.34\%)} & \multicolumn{4}{|c|}{Boolq (FP32 Accuracy = 68.32\%)} \\ 
 \hline
 \hline
 64 & 40.48 & 40.10 & 39.43 & 39.90 & 69.20 & 68.41 & 69.45 & 68.56 \\
 \hline
 32 & 39.52 & 39.52 & 40.77 & 39.62 & 68.32 & 67.43 & 68.17 & 69.30  \\
 \hline
 16 & 39.81 & 39.71 & 39.90 & 40.38 & 68.10 & 66.33 & 69.51 & 69.42  \\
 \hline
 \hline
 \multicolumn{5}{|c|}{Winogrande (FP32 Accuracy = 67.88\%)} & \multicolumn{4}{|c|}{Piqa (FP32 Accuracy = 78.78\%)} \\ 
 \hline
 \hline
 64 & 66.85 & 66.61 & 67.72 & 67.88 & 77.31 & 77.42 & 77.75 & 77.64 \\
 \hline
 32 & 67.25 & 67.72 & 67.72 & 67.00 & 77.31 & 77.04 & 77.80 & 77.37  \\
 \hline
 16 & 68.11 & 68.90 & 67.88 & 67.48 & 77.37 & 78.13 & 78.13 & 77.69  \\
 \hline
\end{tabular}
\caption{\label{tab:mmlu_abalation} Accuracy on LM evaluation harness tasks on GPT3-8B model.}
\end{table}

\begin{table} \centering
\begin{tabular}{|c||c|c|c|c||c|c|c|c|} 
\hline
 $L_b \rightarrow$& \multicolumn{4}{c||}{8} & \multicolumn{4}{c||}{8}\\
 \hline
 \backslashbox{$L_A$\kern-1em}{\kern-1em$N_c$} & 2 & 4 & 8 & 16 & 2 & 4 & 8 & 16  \\
 %$N_c \rightarrow$ & 2 & 4 & 8 & 16 & 2 & 4 & 2 \\
 \hline
 \hline
 \multicolumn{5}{|c|}{Race (FP32 Accuracy = 40.67\%)} & \multicolumn{4}{|c|}{Boolq (FP32 Accuracy = 76.54\%)} \\ 
 \hline
 \hline
 64 & 40.48 & 40.10 & 39.43 & 39.90 & 75.41 & 75.11 & 77.09 & 75.66 \\
 \hline
 32 & 39.52 & 39.52 & 40.77 & 39.62 & 76.02 & 76.02 & 75.96 & 75.35  \\
 \hline
 16 & 39.81 & 39.71 & 39.90 & 40.38 & 75.05 & 73.82 & 75.72 & 76.09  \\
 \hline
 \hline
 \multicolumn{5}{|c|}{Winogrande (FP32 Accuracy = 70.64\%)} & \multicolumn{4}{|c|}{Piqa (FP32 Accuracy = 79.16\%)} \\ 
 \hline
 \hline
 64 & 69.14 & 70.17 & 70.17 & 70.56 & 78.24 & 79.00 & 78.62 & 78.73 \\
 \hline
 32 & 70.96 & 69.69 & 71.27 & 69.30 & 78.56 & 79.49 & 79.16 & 78.89  \\
 \hline
 16 & 71.03 & 69.53 & 69.69 & 70.40 & 78.13 & 79.16 & 79.00 & 79.00  \\
 \hline
\end{tabular}
\caption{\label{tab:mmlu_abalation} Accuracy on LM evaluation harness tasks on GPT3-22B model.}
\end{table}

\begin{table} \centering
\begin{tabular}{|c||c|c|c|c||c|c|c|c|} 
\hline
 $L_b \rightarrow$& \multicolumn{4}{c||}{8} & \multicolumn{4}{c||}{8}\\
 \hline
 \backslashbox{$L_A$\kern-1em}{\kern-1em$N_c$} & 2 & 4 & 8 & 16 & 2 & 4 & 8 & 16  \\
 %$N_c \rightarrow$ & 2 & 4 & 8 & 16 & 2 & 4 & 2 \\
 \hline
 \hline
 \multicolumn{5}{|c|}{Race (FP32 Accuracy = 44.4\%)} & \multicolumn{4}{|c|}{Boolq (FP32 Accuracy = 79.29\%)} \\ 
 \hline
 \hline
 64 & 42.49 & 42.51 & 42.58 & 43.45 & 77.58 & 77.37 & 77.43 & 78.1 \\
 \hline
 32 & 43.35 & 42.49 & 43.64 & 43.73 & 77.86 & 75.32 & 77.28 & 77.86  \\
 \hline
 16 & 44.21 & 44.21 & 43.64 & 42.97 & 78.65 & 77 & 76.94 & 77.98  \\
 \hline
 \hline
 \multicolumn{5}{|c|}{Winogrande (FP32 Accuracy = 69.38\%)} & \multicolumn{4}{|c|}{Piqa (FP32 Accuracy = 78.07\%)} \\ 
 \hline
 \hline
 64 & 68.9 & 68.43 & 69.77 & 68.19 & 77.09 & 76.82 & 77.09 & 77.86 \\
 \hline
 32 & 69.38 & 68.51 & 68.82 & 68.90 & 78.07 & 76.71 & 78.07 & 77.86  \\
 \hline
 16 & 69.53 & 67.09 & 69.38 & 68.90 & 77.37 & 77.8 & 77.91 & 77.69  \\
 \hline
\end{tabular}
\caption{\label{tab:mmlu_abalation} Accuracy on LM evaluation harness tasks on Llama2-7B model.}
\end{table}

\begin{table} \centering
\begin{tabular}{|c||c|c|c|c||c|c|c|c|} 
\hline
 $L_b \rightarrow$& \multicolumn{4}{c||}{8} & \multicolumn{4}{c||}{8}\\
 \hline
 \backslashbox{$L_A$\kern-1em}{\kern-1em$N_c$} & 2 & 4 & 8 & 16 & 2 & 4 & 8 & 16  \\
 %$N_c \rightarrow$ & 2 & 4 & 8 & 16 & 2 & 4 & 2 \\
 \hline
 \hline
 \multicolumn{5}{|c|}{Race (FP32 Accuracy = 48.8\%)} & \multicolumn{4}{|c|}{Boolq (FP32 Accuracy = 85.23\%)} \\ 
 \hline
 \hline
 64 & 49.00 & 49.00 & 49.28 & 48.71 & 82.82 & 84.28 & 84.03 & 84.25 \\
 \hline
 32 & 49.57 & 48.52 & 48.33 & 49.28 & 83.85 & 84.46 & 84.31 & 84.93  \\
 \hline
 16 & 49.85 & 49.09 & 49.28 & 48.99 & 85.11 & 84.46 & 84.61 & 83.94  \\
 \hline
 \hline
 \multicolumn{5}{|c|}{Winogrande (FP32 Accuracy = 79.95\%)} & \multicolumn{4}{|c|}{Piqa (FP32 Accuracy = 81.56\%)} \\ 
 \hline
 \hline
 64 & 78.77 & 78.45 & 78.37 & 79.16 & 81.45 & 80.69 & 81.45 & 81.5 \\
 \hline
 32 & 78.45 & 79.01 & 78.69 & 80.66 & 81.56 & 80.58 & 81.18 & 81.34  \\
 \hline
 16 & 79.95 & 79.56 & 79.79 & 79.72 & 81.28 & 81.66 & 81.28 & 80.96  \\
 \hline
\end{tabular}
\caption{\label{tab:mmlu_abalation} Accuracy on LM evaluation harness tasks on Llama2-70B model.}
\end{table}

%\section{MSE Studies}
%\textcolor{red}{TODO}


\subsection{Number Formats and Quantization Method}
\label{subsec:numFormats_quantMethod}
\subsubsection{Integer Format}
An $n$-bit signed integer (INT) is typically represented with a 2s-complement format \citep{yao2022zeroquant,xiao2023smoothquant,dai2021vsq}, where the most significant bit denotes the sign.

\subsubsection{Floating Point Format}
An $n$-bit signed floating point (FP) number $x$ comprises of a 1-bit sign ($x_{\mathrm{sign}}$), $B_m$-bit mantissa ($x_{\mathrm{mant}}$) and $B_e$-bit exponent ($x_{\mathrm{exp}}$) such that $B_m+B_e=n-1$. The associated constant exponent bias ($E_{\mathrm{bias}}$) is computed as $(2^{{B_e}-1}-1)$. We denote this format as $E_{B_e}M_{B_m}$.  

\subsubsection{Quantization Scheme}
\label{subsec:quant_method}
A quantization scheme dictates how a given unquantized tensor is converted to its quantized representation. We consider FP formats for the purpose of illustration. Given an unquantized tensor $\bm{X}$ and an FP format $E_{B_e}M_{B_m}$, we first, we compute the quantization scale factor $s_X$ that maps the maximum absolute value of $\bm{X}$ to the maximum quantization level of the $E_{B_e}M_{B_m}$ format as follows:
\begin{align}
\label{eq:sf}
    s_X = \frac{\mathrm{max}(|\bm{X}|)}{\mathrm{max}(E_{B_e}M_{B_m})}
\end{align}
In the above equation, $|\cdot|$ denotes the absolute value function.

Next, we scale $\bm{X}$ by $s_X$ and quantize it to $\hat{\bm{X}}$ by rounding it to the nearest quantization level of $E_{B_e}M_{B_m}$ as:

\begin{align}
\label{eq:tensor_quant}
    \hat{\bm{X}} = \text{round-to-nearest}\left(\frac{\bm{X}}{s_X}, E_{B_e}M_{B_m}\right)
\end{align}

We perform dynamic max-scaled quantization \citep{wu2020integer}, where the scale factor $s$ for activations is dynamically computed during runtime.

\subsection{Vector Scaled Quantization}
\begin{wrapfigure}{r}{0.35\linewidth}
  \centering
  \includegraphics[width=\linewidth]{sections/figures/vsquant.jpg}
  \caption{\small Vectorwise decomposition for per-vector scaled quantization (VSQ \citep{dai2021vsq}).}
  \label{fig:vsquant}
\end{wrapfigure}
During VSQ \citep{dai2021vsq}, the operand tensors are decomposed into 1D vectors in a hardware friendly manner as shown in Figure \ref{fig:vsquant}. Since the decomposed tensors are used as operands in matrix multiplications during inference, it is beneficial to perform this decomposition along the reduction dimension of the multiplication. The vectorwise quantization is performed similar to tensorwise quantization described in Equations \ref{eq:sf} and \ref{eq:tensor_quant}, where a scale factor $s_v$ is required for each vector $\bm{v}$ that maps the maximum absolute value of that vector to the maximum quantization level. While smaller vector lengths can lead to larger accuracy gains, the associated memory and computational overheads due to the per-vector scale factors increases. To alleviate these overheads, VSQ \citep{dai2021vsq} proposed a second level quantization of the per-vector scale factors to unsigned integers, while MX \citep{rouhani2023shared} quantizes them to integer powers of 2 (denoted as $2^{INT}$).

\subsubsection{MX Format}
The MX format proposed in \citep{rouhani2023microscaling} introduces the concept of sub-block shifting. For every two scalar elements of $b$-bits each, there is a shared exponent bit. The value of this exponent bit is determined through an empirical analysis that targets minimizing quantization MSE. We note that the FP format $E_{1}M_{b}$ is strictly better than MX from an accuracy perspective since it allocates a dedicated exponent bit to each scalar as opposed to sharing it across two scalars. Therefore, we conservatively bound the accuracy of a $b+2$-bit signed MX format with that of a $E_{1}M_{b}$ format in our comparisons. For instance, we use E1M2 format as a proxy for MX4.

\begin{figure}
    \centering
    \includegraphics[width=1\linewidth]{sections//figures/BlockFormats.pdf}
    \caption{\small Comparing LO-BCQ to MX format.}
    \label{fig:block_formats}
\end{figure}

Figure \ref{fig:block_formats} compares our $4$-bit LO-BCQ block format to MX \citep{rouhani2023microscaling}. As shown, both LO-BCQ and MX decompose a given operand tensor into block arrays and each block array into blocks. Similar to MX, we find that per-block quantization ($L_b < L_A$) leads to better accuracy due to increased flexibility. While MX achieves this through per-block $1$-bit micro-scales, we associate a dedicated codebook to each block through a per-block codebook selector. Further, MX quantizes the per-block array scale-factor to E8M0 format without per-tensor scaling. In contrast during LO-BCQ, we find that per-tensor scaling combined with quantization of per-block array scale-factor to E4M3 format results in superior inference accuracy across models. 


\end{document}
\endinput
%%
%% End of file `sample-sigconf.tex'.



%%
%% If your work has an appendix, this is the place to put it.
\subsection{Lloyd-Max Algorithm}
\label{subsec:Lloyd-Max}
For a given quantization bitwidth $B$ and an operand $\bm{X}$, the Lloyd-Max algorithm finds $2^B$ quantization levels $\{\hat{x}_i\}_{i=1}^{2^B}$ such that quantizing $\bm{X}$ by rounding each scalar in $\bm{X}$ to the nearest quantization level minimizes the quantization MSE. 

The algorithm starts with an initial guess of quantization levels and then iteratively computes quantization thresholds $\{\tau_i\}_{i=1}^{2^B-1}$ and updates quantization levels $\{\hat{x}_i\}_{i=1}^{2^B}$. Specifically, at iteration $n$, thresholds are set to the midpoints of the previous iteration's levels:
\begin{align*}
    \tau_i^{(n)}=\frac{\hat{x}_i^{(n-1)}+\hat{x}_{i+1}^{(n-1)}}2 \text{ for } i=1\ldots 2^B-1
\end{align*}
Subsequently, the quantization levels are re-computed as conditional means of the data regions defined by the new thresholds:
\begin{align*}
    \hat{x}_i^{(n)}=\mathbb{E}\left[ \bm{X} \big| \bm{X}\in [\tau_{i-1}^{(n)},\tau_i^{(n)}] \right] \text{ for } i=1\ldots 2^B
\end{align*}
where to satisfy boundary conditions we have $\tau_0=-\infty$ and $\tau_{2^B}=\infty$. The algorithm iterates the above steps until convergence.

Figure \ref{fig:lm_quant} compares the quantization levels of a $7$-bit floating point (E3M3) quantizer (left) to a $7$-bit Lloyd-Max quantizer (right) when quantizing a layer of weights from the GPT3-126M model at a per-tensor granularity. As shown, the Lloyd-Max quantizer achieves substantially lower quantization MSE. Further, Table \ref{tab:FP7_vs_LM7} shows the superior perplexity achieved by Lloyd-Max quantizers for bitwidths of $7$, $6$ and $5$. The difference between the quantizers is clear at 5 bits, where per-tensor FP quantization incurs a drastic and unacceptable increase in perplexity, while Lloyd-Max quantization incurs a much smaller increase. Nevertheless, we note that even the optimal Lloyd-Max quantizer incurs a notable ($\sim 1.5$) increase in perplexity due to the coarse granularity of quantization. 

\begin{figure}[h]
  \centering
  \includegraphics[width=0.7\linewidth]{sections/figures/LM7_FP7.pdf}
  \caption{\small Quantization levels and the corresponding quantization MSE of Floating Point (left) vs Lloyd-Max (right) Quantizers for a layer of weights in the GPT3-126M model.}
  \label{fig:lm_quant}
\end{figure}

\begin{table}[h]\scriptsize
\begin{center}
\caption{\label{tab:FP7_vs_LM7} \small Comparing perplexity (lower is better) achieved by floating point quantizers and Lloyd-Max quantizers on a GPT3-126M model for the Wikitext-103 dataset.}
\begin{tabular}{c|cc|c}
\hline
 \multirow{2}{*}{\textbf{Bitwidth}} & \multicolumn{2}{|c|}{\textbf{Floating-Point Quantizer}} & \textbf{Lloyd-Max Quantizer} \\
 & Best Format & Wikitext-103 Perplexity & Wikitext-103 Perplexity \\
\hline
7 & E3M3 & 18.32 & 18.27 \\
6 & E3M2 & 19.07 & 18.51 \\
5 & E4M0 & 43.89 & 19.71 \\
\hline
\end{tabular}
\end{center}
\end{table}

\subsection{Proof of Local Optimality of LO-BCQ}
\label{subsec:lobcq_opt_proof}
For a given block $\bm{b}_j$, the quantization MSE during LO-BCQ can be empirically evaluated as $\frac{1}{L_b}\lVert \bm{b}_j- \bm{\hat{b}}_j\rVert^2_2$ where $\bm{\hat{b}}_j$ is computed from equation (\ref{eq:clustered_quantization_definition}) as $C_{f(\bm{b}_j)}(\bm{b}_j)$. Further, for a given block cluster $\mathcal{B}_i$, we compute the quantization MSE as $\frac{1}{|\mathcal{B}_{i}|}\sum_{\bm{b} \in \mathcal{B}_{i}} \frac{1}{L_b}\lVert \bm{b}- C_i^{(n)}(\bm{b})\rVert^2_2$. Therefore, at the end of iteration $n$, we evaluate the overall quantization MSE $J^{(n)}$ for a given operand $\bm{X}$ composed of $N_c$ block clusters as:
\begin{align*}
    \label{eq:mse_iter_n}
    J^{(n)} = \frac{1}{N_c} \sum_{i=1}^{N_c} \frac{1}{|\mathcal{B}_{i}^{(n)}|}\sum_{\bm{v} \in \mathcal{B}_{i}^{(n)}} \frac{1}{L_b}\lVert \bm{b}- B_i^{(n)}(\bm{b})\rVert^2_2
\end{align*}

At the end of iteration $n$, the codebooks are updated from $\mathcal{C}^{(n-1)}$ to $\mathcal{C}^{(n)}$. However, the mapping of a given vector $\bm{b}_j$ to quantizers $\mathcal{C}^{(n)}$ remains as  $f^{(n)}(\bm{b}_j)$. At the next iteration, during the vector clustering step, $f^{(n+1)}(\bm{b}_j)$ finds new mapping of $\bm{b}_j$ to updated codebooks $\mathcal{C}^{(n)}$ such that the quantization MSE over the candidate codebooks is minimized. Therefore, we obtain the following result for $\bm{b}_j$:
\begin{align*}
\frac{1}{L_b}\lVert \bm{b}_j - C_{f^{(n+1)}(\bm{b}_j)}^{(n)}(\bm{b}_j)\rVert^2_2 \le \frac{1}{L_b}\lVert \bm{b}_j - C_{f^{(n)}(\bm{b}_j)}^{(n)}(\bm{b}_j)\rVert^2_2
\end{align*}

That is, quantizing $\bm{b}_j$ at the end of the block clustering step of iteration $n+1$ results in lower quantization MSE compared to quantizing at the end of iteration $n$. Since this is true for all $\bm{b} \in \bm{X}$, we assert the following:
\begin{equation}
\begin{split}
\label{eq:mse_ineq_1}
    \tilde{J}^{(n+1)} &= \frac{1}{N_c} \sum_{i=1}^{N_c} \frac{1}{|\mathcal{B}_{i}^{(n+1)}|}\sum_{\bm{b} \in \mathcal{B}_{i}^{(n+1)}} \frac{1}{L_b}\lVert \bm{b} - C_i^{(n)}(b)\rVert^2_2 \le J^{(n)}
\end{split}
\end{equation}
where $\tilde{J}^{(n+1)}$ is the the quantization MSE after the vector clustering step at iteration $n+1$.

Next, during the codebook update step (\ref{eq:quantizers_update}) at iteration $n+1$, the per-cluster codebooks $\mathcal{C}^{(n)}$ are updated to $\mathcal{C}^{(n+1)}$ by invoking the Lloyd-Max algorithm \citep{Lloyd}. We know that for any given value distribution, the Lloyd-Max algorithm minimizes the quantization MSE. Therefore, for a given vector cluster $\mathcal{B}_i$ we obtain the following result:

\begin{equation}
    \frac{1}{|\mathcal{B}_{i}^{(n+1)}|}\sum_{\bm{b} \in \mathcal{B}_{i}^{(n+1)}} \frac{1}{L_b}\lVert \bm{b}- C_i^{(n+1)}(\bm{b})\rVert^2_2 \le \frac{1}{|\mathcal{B}_{i}^{(n+1)}|}\sum_{\bm{b} \in \mathcal{B}_{i}^{(n+1)}} \frac{1}{L_b}\lVert \bm{b}- C_i^{(n)}(\bm{b})\rVert^2_2
\end{equation}

The above equation states that quantizing the given block cluster $\mathcal{B}_i$ after updating the associated codebook from $C_i^{(n)}$ to $C_i^{(n+1)}$ results in lower quantization MSE. Since this is true for all the block clusters, we derive the following result: 
\begin{equation}
\begin{split}
\label{eq:mse_ineq_2}
     J^{(n+1)} &= \frac{1}{N_c} \sum_{i=1}^{N_c} \frac{1}{|\mathcal{B}_{i}^{(n+1)}|}\sum_{\bm{b} \in \mathcal{B}_{i}^{(n+1)}} \frac{1}{L_b}\lVert \bm{b}- C_i^{(n+1)}(\bm{b})\rVert^2_2  \le \tilde{J}^{(n+1)}   
\end{split}
\end{equation}

Following (\ref{eq:mse_ineq_1}) and (\ref{eq:mse_ineq_2}), we find that the quantization MSE is non-increasing for each iteration, that is, $J^{(1)} \ge J^{(2)} \ge J^{(3)} \ge \ldots \ge J^{(M)}$ where $M$ is the maximum number of iterations. 
%Therefore, we can say that if the algorithm converges, then it must be that it has converged to a local minimum. 
\hfill $\blacksquare$


\begin{figure}
    \begin{center}
    \includegraphics[width=0.5\textwidth]{sections//figures/mse_vs_iter.pdf}
    \end{center}
    \caption{\small NMSE vs iterations during LO-BCQ compared to other block quantization proposals}
    \label{fig:nmse_vs_iter}
\end{figure}

Figure \ref{fig:nmse_vs_iter} shows the empirical convergence of LO-BCQ across several block lengths and number of codebooks. Also, the MSE achieved by LO-BCQ is compared to baselines such as MXFP and VSQ. As shown, LO-BCQ converges to a lower MSE than the baselines. Further, we achieve better convergence for larger number of codebooks ($N_c$) and for a smaller block length ($L_b$), both of which increase the bitwidth of BCQ (see Eq \ref{eq:bitwidth_bcq}).


\subsection{Additional Accuracy Results}
%Table \ref{tab:lobcq_config} lists the various LOBCQ configurations and their corresponding bitwidths.
\begin{table}
\setlength{\tabcolsep}{4.75pt}
\begin{center}
\caption{\label{tab:lobcq_config} Various LO-BCQ configurations and their bitwidths.}
\begin{tabular}{|c||c|c|c|c||c|c||c|} 
\hline
 & \multicolumn{4}{|c||}{$L_b=8$} & \multicolumn{2}{|c||}{$L_b=4$} & $L_b=2$ \\
 \hline
 \backslashbox{$L_A$\kern-1em}{\kern-1em$N_c$} & 2 & 4 & 8 & 16 & 2 & 4 & 2 \\
 \hline
 64 & 4.25 & 4.375 & 4.5 & 4.625 & 4.375 & 4.625 & 4.625\\
 \hline
 32 & 4.375 & 4.5 & 4.625& 4.75 & 4.5 & 4.75 & 4.75 \\
 \hline
 16 & 4.625 & 4.75& 4.875 & 5 & 4.75 & 5 & 5 \\
 \hline
\end{tabular}
\end{center}
\end{table}

%\subsection{Perplexity achieved by various LO-BCQ configurations on Wikitext-103 dataset}

\begin{table} \centering
\begin{tabular}{|c||c|c|c|c||c|c||c|} 
\hline
 $L_b \rightarrow$& \multicolumn{4}{c||}{8} & \multicolumn{2}{c||}{4} & 2\\
 \hline
 \backslashbox{$L_A$\kern-1em}{\kern-1em$N_c$} & 2 & 4 & 8 & 16 & 2 & 4 & 2  \\
 %$N_c \rightarrow$ & 2 & 4 & 8 & 16 & 2 & 4 & 2 \\
 \hline
 \hline
 \multicolumn{8}{c}{GPT3-1.3B (FP32 PPL = 9.98)} \\ 
 \hline
 \hline
 64 & 10.40 & 10.23 & 10.17 & 10.15 &  10.28 & 10.18 & 10.19 \\
 \hline
 32 & 10.25 & 10.20 & 10.15 & 10.12 &  10.23 & 10.17 & 10.17 \\
 \hline
 16 & 10.22 & 10.16 & 10.10 & 10.09 &  10.21 & 10.14 & 10.16 \\
 \hline
  \hline
 \multicolumn{8}{c}{GPT3-8B (FP32 PPL = 7.38)} \\ 
 \hline
 \hline
 64 & 7.61 & 7.52 & 7.48 &  7.47 &  7.55 &  7.49 & 7.50 \\
 \hline
 32 & 7.52 & 7.50 & 7.46 &  7.45 &  7.52 &  7.48 & 7.48  \\
 \hline
 16 & 7.51 & 7.48 & 7.44 &  7.44 &  7.51 &  7.49 & 7.47  \\
 \hline
\end{tabular}
\caption{\label{tab:ppl_gpt3_abalation} Wikitext-103 perplexity across GPT3-1.3B and 8B models.}
\end{table}

\begin{table} \centering
\begin{tabular}{|c||c|c|c|c||} 
\hline
 $L_b \rightarrow$& \multicolumn{4}{c||}{8}\\
 \hline
 \backslashbox{$L_A$\kern-1em}{\kern-1em$N_c$} & 2 & 4 & 8 & 16 \\
 %$N_c \rightarrow$ & 2 & 4 & 8 & 16 & 2 & 4 & 2 \\
 \hline
 \hline
 \multicolumn{5}{|c|}{Llama2-7B (FP32 PPL = 5.06)} \\ 
 \hline
 \hline
 64 & 5.31 & 5.26 & 5.19 & 5.18  \\
 \hline
 32 & 5.23 & 5.25 & 5.18 & 5.15  \\
 \hline
 16 & 5.23 & 5.19 & 5.16 & 5.14  \\
 \hline
 \multicolumn{5}{|c|}{Nemotron4-15B (FP32 PPL = 5.87)} \\ 
 \hline
 \hline
 64  & 6.3 & 6.20 & 6.13 & 6.08  \\
 \hline
 32  & 6.24 & 6.12 & 6.07 & 6.03  \\
 \hline
 16  & 6.12 & 6.14 & 6.04 & 6.02  \\
 \hline
 \multicolumn{5}{|c|}{Nemotron4-340B (FP32 PPL = 3.48)} \\ 
 \hline
 \hline
 64 & 3.67 & 3.62 & 3.60 & 3.59 \\
 \hline
 32 & 3.63 & 3.61 & 3.59 & 3.56 \\
 \hline
 16 & 3.61 & 3.58 & 3.57 & 3.55 \\
 \hline
\end{tabular}
\caption{\label{tab:ppl_llama7B_nemo15B} Wikitext-103 perplexity compared to FP32 baseline in Llama2-7B and Nemotron4-15B, 340B models}
\end{table}

%\subsection{Perplexity achieved by various LO-BCQ configurations on MMLU dataset}


\begin{table} \centering
\begin{tabular}{|c||c|c|c|c||c|c|c|c|} 
\hline
 $L_b \rightarrow$& \multicolumn{4}{c||}{8} & \multicolumn{4}{c||}{8}\\
 \hline
 \backslashbox{$L_A$\kern-1em}{\kern-1em$N_c$} & 2 & 4 & 8 & 16 & 2 & 4 & 8 & 16  \\
 %$N_c \rightarrow$ & 2 & 4 & 8 & 16 & 2 & 4 & 2 \\
 \hline
 \hline
 \multicolumn{5}{|c|}{Llama2-7B (FP32 Accuracy = 45.8\%)} & \multicolumn{4}{|c|}{Llama2-70B (FP32 Accuracy = 69.12\%)} \\ 
 \hline
 \hline
 64 & 43.9 & 43.4 & 43.9 & 44.9 & 68.07 & 68.27 & 68.17 & 68.75 \\
 \hline
 32 & 44.5 & 43.8 & 44.9 & 44.5 & 68.37 & 68.51 & 68.35 & 68.27  \\
 \hline
 16 & 43.9 & 42.7 & 44.9 & 45 & 68.12 & 68.77 & 68.31 & 68.59  \\
 \hline
 \hline
 \multicolumn{5}{|c|}{GPT3-22B (FP32 Accuracy = 38.75\%)} & \multicolumn{4}{|c|}{Nemotron4-15B (FP32 Accuracy = 64.3\%)} \\ 
 \hline
 \hline
 64 & 36.71 & 38.85 & 38.13 & 38.92 & 63.17 & 62.36 & 63.72 & 64.09 \\
 \hline
 32 & 37.95 & 38.69 & 39.45 & 38.34 & 64.05 & 62.30 & 63.8 & 64.33  \\
 \hline
 16 & 38.88 & 38.80 & 38.31 & 38.92 & 63.22 & 63.51 & 63.93 & 64.43  \\
 \hline
\end{tabular}
\caption{\label{tab:mmlu_abalation} Accuracy on MMLU dataset across GPT3-22B, Llama2-7B, 70B and Nemotron4-15B models.}
\end{table}


%\subsection{Perplexity achieved by various LO-BCQ configurations on LM evaluation harness}

\begin{table} \centering
\begin{tabular}{|c||c|c|c|c||c|c|c|c|} 
\hline
 $L_b \rightarrow$& \multicolumn{4}{c||}{8} & \multicolumn{4}{c||}{8}\\
 \hline
 \backslashbox{$L_A$\kern-1em}{\kern-1em$N_c$} & 2 & 4 & 8 & 16 & 2 & 4 & 8 & 16  \\
 %$N_c \rightarrow$ & 2 & 4 & 8 & 16 & 2 & 4 & 2 \\
 \hline
 \hline
 \multicolumn{5}{|c|}{Race (FP32 Accuracy = 37.51\%)} & \multicolumn{4}{|c|}{Boolq (FP32 Accuracy = 64.62\%)} \\ 
 \hline
 \hline
 64 & 36.94 & 37.13 & 36.27 & 37.13 & 63.73 & 62.26 & 63.49 & 63.36 \\
 \hline
 32 & 37.03 & 36.36 & 36.08 & 37.03 & 62.54 & 63.51 & 63.49 & 63.55  \\
 \hline
 16 & 37.03 & 37.03 & 36.46 & 37.03 & 61.1 & 63.79 & 63.58 & 63.33  \\
 \hline
 \hline
 \multicolumn{5}{|c|}{Winogrande (FP32 Accuracy = 58.01\%)} & \multicolumn{4}{|c|}{Piqa (FP32 Accuracy = 74.21\%)} \\ 
 \hline
 \hline
 64 & 58.17 & 57.22 & 57.85 & 58.33 & 73.01 & 73.07 & 73.07 & 72.80 \\
 \hline
 32 & 59.12 & 58.09 & 57.85 & 58.41 & 73.01 & 73.94 & 72.74 & 73.18  \\
 \hline
 16 & 57.93 & 58.88 & 57.93 & 58.56 & 73.94 & 72.80 & 73.01 & 73.94  \\
 \hline
\end{tabular}
\caption{\label{tab:mmlu_abalation} Accuracy on LM evaluation harness tasks on GPT3-1.3B model.}
\end{table}

\begin{table} \centering
\begin{tabular}{|c||c|c|c|c||c|c|c|c|} 
\hline
 $L_b \rightarrow$& \multicolumn{4}{c||}{8} & \multicolumn{4}{c||}{8}\\
 \hline
 \backslashbox{$L_A$\kern-1em}{\kern-1em$N_c$} & 2 & 4 & 8 & 16 & 2 & 4 & 8 & 16  \\
 %$N_c \rightarrow$ & 2 & 4 & 8 & 16 & 2 & 4 & 2 \\
 \hline
 \hline
 \multicolumn{5}{|c|}{Race (FP32 Accuracy = 41.34\%)} & \multicolumn{4}{|c|}{Boolq (FP32 Accuracy = 68.32\%)} \\ 
 \hline
 \hline
 64 & 40.48 & 40.10 & 39.43 & 39.90 & 69.20 & 68.41 & 69.45 & 68.56 \\
 \hline
 32 & 39.52 & 39.52 & 40.77 & 39.62 & 68.32 & 67.43 & 68.17 & 69.30  \\
 \hline
 16 & 39.81 & 39.71 & 39.90 & 40.38 & 68.10 & 66.33 & 69.51 & 69.42  \\
 \hline
 \hline
 \multicolumn{5}{|c|}{Winogrande (FP32 Accuracy = 67.88\%)} & \multicolumn{4}{|c|}{Piqa (FP32 Accuracy = 78.78\%)} \\ 
 \hline
 \hline
 64 & 66.85 & 66.61 & 67.72 & 67.88 & 77.31 & 77.42 & 77.75 & 77.64 \\
 \hline
 32 & 67.25 & 67.72 & 67.72 & 67.00 & 77.31 & 77.04 & 77.80 & 77.37  \\
 \hline
 16 & 68.11 & 68.90 & 67.88 & 67.48 & 77.37 & 78.13 & 78.13 & 77.69  \\
 \hline
\end{tabular}
\caption{\label{tab:mmlu_abalation} Accuracy on LM evaluation harness tasks on GPT3-8B model.}
\end{table}

\begin{table} \centering
\begin{tabular}{|c||c|c|c|c||c|c|c|c|} 
\hline
 $L_b \rightarrow$& \multicolumn{4}{c||}{8} & \multicolumn{4}{c||}{8}\\
 \hline
 \backslashbox{$L_A$\kern-1em}{\kern-1em$N_c$} & 2 & 4 & 8 & 16 & 2 & 4 & 8 & 16  \\
 %$N_c \rightarrow$ & 2 & 4 & 8 & 16 & 2 & 4 & 2 \\
 \hline
 \hline
 \multicolumn{5}{|c|}{Race (FP32 Accuracy = 40.67\%)} & \multicolumn{4}{|c|}{Boolq (FP32 Accuracy = 76.54\%)} \\ 
 \hline
 \hline
 64 & 40.48 & 40.10 & 39.43 & 39.90 & 75.41 & 75.11 & 77.09 & 75.66 \\
 \hline
 32 & 39.52 & 39.52 & 40.77 & 39.62 & 76.02 & 76.02 & 75.96 & 75.35  \\
 \hline
 16 & 39.81 & 39.71 & 39.90 & 40.38 & 75.05 & 73.82 & 75.72 & 76.09  \\
 \hline
 \hline
 \multicolumn{5}{|c|}{Winogrande (FP32 Accuracy = 70.64\%)} & \multicolumn{4}{|c|}{Piqa (FP32 Accuracy = 79.16\%)} \\ 
 \hline
 \hline
 64 & 69.14 & 70.17 & 70.17 & 70.56 & 78.24 & 79.00 & 78.62 & 78.73 \\
 \hline
 32 & 70.96 & 69.69 & 71.27 & 69.30 & 78.56 & 79.49 & 79.16 & 78.89  \\
 \hline
 16 & 71.03 & 69.53 & 69.69 & 70.40 & 78.13 & 79.16 & 79.00 & 79.00  \\
 \hline
\end{tabular}
\caption{\label{tab:mmlu_abalation} Accuracy on LM evaluation harness tasks on GPT3-22B model.}
\end{table}

\begin{table} \centering
\begin{tabular}{|c||c|c|c|c||c|c|c|c|} 
\hline
 $L_b \rightarrow$& \multicolumn{4}{c||}{8} & \multicolumn{4}{c||}{8}\\
 \hline
 \backslashbox{$L_A$\kern-1em}{\kern-1em$N_c$} & 2 & 4 & 8 & 16 & 2 & 4 & 8 & 16  \\
 %$N_c \rightarrow$ & 2 & 4 & 8 & 16 & 2 & 4 & 2 \\
 \hline
 \hline
 \multicolumn{5}{|c|}{Race (FP32 Accuracy = 44.4\%)} & \multicolumn{4}{|c|}{Boolq (FP32 Accuracy = 79.29\%)} \\ 
 \hline
 \hline
 64 & 42.49 & 42.51 & 42.58 & 43.45 & 77.58 & 77.37 & 77.43 & 78.1 \\
 \hline
 32 & 43.35 & 42.49 & 43.64 & 43.73 & 77.86 & 75.32 & 77.28 & 77.86  \\
 \hline
 16 & 44.21 & 44.21 & 43.64 & 42.97 & 78.65 & 77 & 76.94 & 77.98  \\
 \hline
 \hline
 \multicolumn{5}{|c|}{Winogrande (FP32 Accuracy = 69.38\%)} & \multicolumn{4}{|c|}{Piqa (FP32 Accuracy = 78.07\%)} \\ 
 \hline
 \hline
 64 & 68.9 & 68.43 & 69.77 & 68.19 & 77.09 & 76.82 & 77.09 & 77.86 \\
 \hline
 32 & 69.38 & 68.51 & 68.82 & 68.90 & 78.07 & 76.71 & 78.07 & 77.86  \\
 \hline
 16 & 69.53 & 67.09 & 69.38 & 68.90 & 77.37 & 77.8 & 77.91 & 77.69  \\
 \hline
\end{tabular}
\caption{\label{tab:mmlu_abalation} Accuracy on LM evaluation harness tasks on Llama2-7B model.}
\end{table}

\begin{table} \centering
\begin{tabular}{|c||c|c|c|c||c|c|c|c|} 
\hline
 $L_b \rightarrow$& \multicolumn{4}{c||}{8} & \multicolumn{4}{c||}{8}\\
 \hline
 \backslashbox{$L_A$\kern-1em}{\kern-1em$N_c$} & 2 & 4 & 8 & 16 & 2 & 4 & 8 & 16  \\
 %$N_c \rightarrow$ & 2 & 4 & 8 & 16 & 2 & 4 & 2 \\
 \hline
 \hline
 \multicolumn{5}{|c|}{Race (FP32 Accuracy = 48.8\%)} & \multicolumn{4}{|c|}{Boolq (FP32 Accuracy = 85.23\%)} \\ 
 \hline
 \hline
 64 & 49.00 & 49.00 & 49.28 & 48.71 & 82.82 & 84.28 & 84.03 & 84.25 \\
 \hline
 32 & 49.57 & 48.52 & 48.33 & 49.28 & 83.85 & 84.46 & 84.31 & 84.93  \\
 \hline
 16 & 49.85 & 49.09 & 49.28 & 48.99 & 85.11 & 84.46 & 84.61 & 83.94  \\
 \hline
 \hline
 \multicolumn{5}{|c|}{Winogrande (FP32 Accuracy = 79.95\%)} & \multicolumn{4}{|c|}{Piqa (FP32 Accuracy = 81.56\%)} \\ 
 \hline
 \hline
 64 & 78.77 & 78.45 & 78.37 & 79.16 & 81.45 & 80.69 & 81.45 & 81.5 \\
 \hline
 32 & 78.45 & 79.01 & 78.69 & 80.66 & 81.56 & 80.58 & 81.18 & 81.34  \\
 \hline
 16 & 79.95 & 79.56 & 79.79 & 79.72 & 81.28 & 81.66 & 81.28 & 80.96  \\
 \hline
\end{tabular}
\caption{\label{tab:mmlu_abalation} Accuracy on LM evaluation harness tasks on Llama2-70B model.}
\end{table}

%\section{MSE Studies}
%\textcolor{red}{TODO}


\subsection{Number Formats and Quantization Method}
\label{subsec:numFormats_quantMethod}
\subsubsection{Integer Format}
An $n$-bit signed integer (INT) is typically represented with a 2s-complement format \citep{yao2022zeroquant,xiao2023smoothquant,dai2021vsq}, where the most significant bit denotes the sign.

\subsubsection{Floating Point Format}
An $n$-bit signed floating point (FP) number $x$ comprises of a 1-bit sign ($x_{\mathrm{sign}}$), $B_m$-bit mantissa ($x_{\mathrm{mant}}$) and $B_e$-bit exponent ($x_{\mathrm{exp}}$) such that $B_m+B_e=n-1$. The associated constant exponent bias ($E_{\mathrm{bias}}$) is computed as $(2^{{B_e}-1}-1)$. We denote this format as $E_{B_e}M_{B_m}$.  

\subsubsection{Quantization Scheme}
\label{subsec:quant_method}
A quantization scheme dictates how a given unquantized tensor is converted to its quantized representation. We consider FP formats for the purpose of illustration. Given an unquantized tensor $\bm{X}$ and an FP format $E_{B_e}M_{B_m}$, we first, we compute the quantization scale factor $s_X$ that maps the maximum absolute value of $\bm{X}$ to the maximum quantization level of the $E_{B_e}M_{B_m}$ format as follows:
\begin{align}
\label{eq:sf}
    s_X = \frac{\mathrm{max}(|\bm{X}|)}{\mathrm{max}(E_{B_e}M_{B_m})}
\end{align}
In the above equation, $|\cdot|$ denotes the absolute value function.

Next, we scale $\bm{X}$ by $s_X$ and quantize it to $\hat{\bm{X}}$ by rounding it to the nearest quantization level of $E_{B_e}M_{B_m}$ as:

\begin{align}
\label{eq:tensor_quant}
    \hat{\bm{X}} = \text{round-to-nearest}\left(\frac{\bm{X}}{s_X}, E_{B_e}M_{B_m}\right)
\end{align}

We perform dynamic max-scaled quantization \citep{wu2020integer}, where the scale factor $s$ for activations is dynamically computed during runtime.

\subsection{Vector Scaled Quantization}
\begin{wrapfigure}{r}{0.35\linewidth}
  \centering
  \includegraphics[width=\linewidth]{sections/figures/vsquant.jpg}
  \caption{\small Vectorwise decomposition for per-vector scaled quantization (VSQ \citep{dai2021vsq}).}
  \label{fig:vsquant}
\end{wrapfigure}
During VSQ \citep{dai2021vsq}, the operand tensors are decomposed into 1D vectors in a hardware friendly manner as shown in Figure \ref{fig:vsquant}. Since the decomposed tensors are used as operands in matrix multiplications during inference, it is beneficial to perform this decomposition along the reduction dimension of the multiplication. The vectorwise quantization is performed similar to tensorwise quantization described in Equations \ref{eq:sf} and \ref{eq:tensor_quant}, where a scale factor $s_v$ is required for each vector $\bm{v}$ that maps the maximum absolute value of that vector to the maximum quantization level. While smaller vector lengths can lead to larger accuracy gains, the associated memory and computational overheads due to the per-vector scale factors increases. To alleviate these overheads, VSQ \citep{dai2021vsq} proposed a second level quantization of the per-vector scale factors to unsigned integers, while MX \citep{rouhani2023shared} quantizes them to integer powers of 2 (denoted as $2^{INT}$).

\subsubsection{MX Format}
The MX format proposed in \citep{rouhani2023microscaling} introduces the concept of sub-block shifting. For every two scalar elements of $b$-bits each, there is a shared exponent bit. The value of this exponent bit is determined through an empirical analysis that targets minimizing quantization MSE. We note that the FP format $E_{1}M_{b}$ is strictly better than MX from an accuracy perspective since it allocates a dedicated exponent bit to each scalar as opposed to sharing it across two scalars. Therefore, we conservatively bound the accuracy of a $b+2$-bit signed MX format with that of a $E_{1}M_{b}$ format in our comparisons. For instance, we use E1M2 format as a proxy for MX4.

\begin{figure}
    \centering
    \includegraphics[width=1\linewidth]{sections//figures/BlockFormats.pdf}
    \caption{\small Comparing LO-BCQ to MX format.}
    \label{fig:block_formats}
\end{figure}

Figure \ref{fig:block_formats} compares our $4$-bit LO-BCQ block format to MX \citep{rouhani2023microscaling}. As shown, both LO-BCQ and MX decompose a given operand tensor into block arrays and each block array into blocks. Similar to MX, we find that per-block quantization ($L_b < L_A$) leads to better accuracy due to increased flexibility. While MX achieves this through per-block $1$-bit micro-scales, we associate a dedicated codebook to each block through a per-block codebook selector. Further, MX quantizes the per-block array scale-factor to E8M0 format without per-tensor scaling. In contrast during LO-BCQ, we find that per-tensor scaling combined with quantization of per-block array scale-factor to E4M3 format results in superior inference accuracy across models. 


\end{document}
\endinput
%%
%% End of file `sample-sigconf.tex'.



%%
%% If your work has an appendix, this is the place to put it.
\subsection{Lloyd-Max Algorithm}
\label{subsec:Lloyd-Max}
For a given quantization bitwidth $B$ and an operand $\bm{X}$, the Lloyd-Max algorithm finds $2^B$ quantization levels $\{\hat{x}_i\}_{i=1}^{2^B}$ such that quantizing $\bm{X}$ by rounding each scalar in $\bm{X}$ to the nearest quantization level minimizes the quantization MSE. 

The algorithm starts with an initial guess of quantization levels and then iteratively computes quantization thresholds $\{\tau_i\}_{i=1}^{2^B-1}$ and updates quantization levels $\{\hat{x}_i\}_{i=1}^{2^B}$. Specifically, at iteration $n$, thresholds are set to the midpoints of the previous iteration's levels:
\begin{align*}
    \tau_i^{(n)}=\frac{\hat{x}_i^{(n-1)}+\hat{x}_{i+1}^{(n-1)}}2 \text{ for } i=1\ldots 2^B-1
\end{align*}
Subsequently, the quantization levels are re-computed as conditional means of the data regions defined by the new thresholds:
\begin{align*}
    \hat{x}_i^{(n)}=\mathbb{E}\left[ \bm{X} \big| \bm{X}\in [\tau_{i-1}^{(n)},\tau_i^{(n)}] \right] \text{ for } i=1\ldots 2^B
\end{align*}
where to satisfy boundary conditions we have $\tau_0=-\infty$ and $\tau_{2^B}=\infty$. The algorithm iterates the above steps until convergence.

Figure \ref{fig:lm_quant} compares the quantization levels of a $7$-bit floating point (E3M3) quantizer (left) to a $7$-bit Lloyd-Max quantizer (right) when quantizing a layer of weights from the GPT3-126M model at a per-tensor granularity. As shown, the Lloyd-Max quantizer achieves substantially lower quantization MSE. Further, Table \ref{tab:FP7_vs_LM7} shows the superior perplexity achieved by Lloyd-Max quantizers for bitwidths of $7$, $6$ and $5$. The difference between the quantizers is clear at 5 bits, where per-tensor FP quantization incurs a drastic and unacceptable increase in perplexity, while Lloyd-Max quantization incurs a much smaller increase. Nevertheless, we note that even the optimal Lloyd-Max quantizer incurs a notable ($\sim 1.5$) increase in perplexity due to the coarse granularity of quantization. 

\begin{figure}[h]
  \centering
  \includegraphics[width=0.7\linewidth]{sections/figures/LM7_FP7.pdf}
  \caption{\small Quantization levels and the corresponding quantization MSE of Floating Point (left) vs Lloyd-Max (right) Quantizers for a layer of weights in the GPT3-126M model.}
  \label{fig:lm_quant}
\end{figure}

\begin{table}[h]\scriptsize
\begin{center}
\caption{\label{tab:FP7_vs_LM7} \small Comparing perplexity (lower is better) achieved by floating point quantizers and Lloyd-Max quantizers on a GPT3-126M model for the Wikitext-103 dataset.}
\begin{tabular}{c|cc|c}
\hline
 \multirow{2}{*}{\textbf{Bitwidth}} & \multicolumn{2}{|c|}{\textbf{Floating-Point Quantizer}} & \textbf{Lloyd-Max Quantizer} \\
 & Best Format & Wikitext-103 Perplexity & Wikitext-103 Perplexity \\
\hline
7 & E3M3 & 18.32 & 18.27 \\
6 & E3M2 & 19.07 & 18.51 \\
5 & E4M0 & 43.89 & 19.71 \\
\hline
\end{tabular}
\end{center}
\end{table}

\subsection{Proof of Local Optimality of LO-BCQ}
\label{subsec:lobcq_opt_proof}
For a given block $\bm{b}_j$, the quantization MSE during LO-BCQ can be empirically evaluated as $\frac{1}{L_b}\lVert \bm{b}_j- \bm{\hat{b}}_j\rVert^2_2$ where $\bm{\hat{b}}_j$ is computed from equation (\ref{eq:clustered_quantization_definition}) as $C_{f(\bm{b}_j)}(\bm{b}_j)$. Further, for a given block cluster $\mathcal{B}_i$, we compute the quantization MSE as $\frac{1}{|\mathcal{B}_{i}|}\sum_{\bm{b} \in \mathcal{B}_{i}} \frac{1}{L_b}\lVert \bm{b}- C_i^{(n)}(\bm{b})\rVert^2_2$. Therefore, at the end of iteration $n$, we evaluate the overall quantization MSE $J^{(n)}$ for a given operand $\bm{X}$ composed of $N_c$ block clusters as:
\begin{align*}
    \label{eq:mse_iter_n}
    J^{(n)} = \frac{1}{N_c} \sum_{i=1}^{N_c} \frac{1}{|\mathcal{B}_{i}^{(n)}|}\sum_{\bm{v} \in \mathcal{B}_{i}^{(n)}} \frac{1}{L_b}\lVert \bm{b}- B_i^{(n)}(\bm{b})\rVert^2_2
\end{align*}

At the end of iteration $n$, the codebooks are updated from $\mathcal{C}^{(n-1)}$ to $\mathcal{C}^{(n)}$. However, the mapping of a given vector $\bm{b}_j$ to quantizers $\mathcal{C}^{(n)}$ remains as  $f^{(n)}(\bm{b}_j)$. At the next iteration, during the vector clustering step, $f^{(n+1)}(\bm{b}_j)$ finds new mapping of $\bm{b}_j$ to updated codebooks $\mathcal{C}^{(n)}$ such that the quantization MSE over the candidate codebooks is minimized. Therefore, we obtain the following result for $\bm{b}_j$:
\begin{align*}
\frac{1}{L_b}\lVert \bm{b}_j - C_{f^{(n+1)}(\bm{b}_j)}^{(n)}(\bm{b}_j)\rVert^2_2 \le \frac{1}{L_b}\lVert \bm{b}_j - C_{f^{(n)}(\bm{b}_j)}^{(n)}(\bm{b}_j)\rVert^2_2
\end{align*}

That is, quantizing $\bm{b}_j$ at the end of the block clustering step of iteration $n+1$ results in lower quantization MSE compared to quantizing at the end of iteration $n$. Since this is true for all $\bm{b} \in \bm{X}$, we assert the following:
\begin{equation}
\begin{split}
\label{eq:mse_ineq_1}
    \tilde{J}^{(n+1)} &= \frac{1}{N_c} \sum_{i=1}^{N_c} \frac{1}{|\mathcal{B}_{i}^{(n+1)}|}\sum_{\bm{b} \in \mathcal{B}_{i}^{(n+1)}} \frac{1}{L_b}\lVert \bm{b} - C_i^{(n)}(b)\rVert^2_2 \le J^{(n)}
\end{split}
\end{equation}
where $\tilde{J}^{(n+1)}$ is the the quantization MSE after the vector clustering step at iteration $n+1$.

Next, during the codebook update step (\ref{eq:quantizers_update}) at iteration $n+1$, the per-cluster codebooks $\mathcal{C}^{(n)}$ are updated to $\mathcal{C}^{(n+1)}$ by invoking the Lloyd-Max algorithm \citep{Lloyd}. We know that for any given value distribution, the Lloyd-Max algorithm minimizes the quantization MSE. Therefore, for a given vector cluster $\mathcal{B}_i$ we obtain the following result:

\begin{equation}
    \frac{1}{|\mathcal{B}_{i}^{(n+1)}|}\sum_{\bm{b} \in \mathcal{B}_{i}^{(n+1)}} \frac{1}{L_b}\lVert \bm{b}- C_i^{(n+1)}(\bm{b})\rVert^2_2 \le \frac{1}{|\mathcal{B}_{i}^{(n+1)}|}\sum_{\bm{b} \in \mathcal{B}_{i}^{(n+1)}} \frac{1}{L_b}\lVert \bm{b}- C_i^{(n)}(\bm{b})\rVert^2_2
\end{equation}

The above equation states that quantizing the given block cluster $\mathcal{B}_i$ after updating the associated codebook from $C_i^{(n)}$ to $C_i^{(n+1)}$ results in lower quantization MSE. Since this is true for all the block clusters, we derive the following result: 
\begin{equation}
\begin{split}
\label{eq:mse_ineq_2}
     J^{(n+1)} &= \frac{1}{N_c} \sum_{i=1}^{N_c} \frac{1}{|\mathcal{B}_{i}^{(n+1)}|}\sum_{\bm{b} \in \mathcal{B}_{i}^{(n+1)}} \frac{1}{L_b}\lVert \bm{b}- C_i^{(n+1)}(\bm{b})\rVert^2_2  \le \tilde{J}^{(n+1)}   
\end{split}
\end{equation}

Following (\ref{eq:mse_ineq_1}) and (\ref{eq:mse_ineq_2}), we find that the quantization MSE is non-increasing for each iteration, that is, $J^{(1)} \ge J^{(2)} \ge J^{(3)} \ge \ldots \ge J^{(M)}$ where $M$ is the maximum number of iterations. 
%Therefore, we can say that if the algorithm converges, then it must be that it has converged to a local minimum. 
\hfill $\blacksquare$


\begin{figure}
    \begin{center}
    \includegraphics[width=0.5\textwidth]{sections//figures/mse_vs_iter.pdf}
    \end{center}
    \caption{\small NMSE vs iterations during LO-BCQ compared to other block quantization proposals}
    \label{fig:nmse_vs_iter}
\end{figure}

Figure \ref{fig:nmse_vs_iter} shows the empirical convergence of LO-BCQ across several block lengths and number of codebooks. Also, the MSE achieved by LO-BCQ is compared to baselines such as MXFP and VSQ. As shown, LO-BCQ converges to a lower MSE than the baselines. Further, we achieve better convergence for larger number of codebooks ($N_c$) and for a smaller block length ($L_b$), both of which increase the bitwidth of BCQ (see Eq \ref{eq:bitwidth_bcq}).


\subsection{Additional Accuracy Results}
%Table \ref{tab:lobcq_config} lists the various LOBCQ configurations and their corresponding bitwidths.
\begin{table}
\setlength{\tabcolsep}{4.75pt}
\begin{center}
\caption{\label{tab:lobcq_config} Various LO-BCQ configurations and their bitwidths.}
\begin{tabular}{|c||c|c|c|c||c|c||c|} 
\hline
 & \multicolumn{4}{|c||}{$L_b=8$} & \multicolumn{2}{|c||}{$L_b=4$} & $L_b=2$ \\
 \hline
 \backslashbox{$L_A$\kern-1em}{\kern-1em$N_c$} & 2 & 4 & 8 & 16 & 2 & 4 & 2 \\
 \hline
 64 & 4.25 & 4.375 & 4.5 & 4.625 & 4.375 & 4.625 & 4.625\\
 \hline
 32 & 4.375 & 4.5 & 4.625& 4.75 & 4.5 & 4.75 & 4.75 \\
 \hline
 16 & 4.625 & 4.75& 4.875 & 5 & 4.75 & 5 & 5 \\
 \hline
\end{tabular}
\end{center}
\end{table}

%\subsection{Perplexity achieved by various LO-BCQ configurations on Wikitext-103 dataset}

\begin{table} \centering
\begin{tabular}{|c||c|c|c|c||c|c||c|} 
\hline
 $L_b \rightarrow$& \multicolumn{4}{c||}{8} & \multicolumn{2}{c||}{4} & 2\\
 \hline
 \backslashbox{$L_A$\kern-1em}{\kern-1em$N_c$} & 2 & 4 & 8 & 16 & 2 & 4 & 2  \\
 %$N_c \rightarrow$ & 2 & 4 & 8 & 16 & 2 & 4 & 2 \\
 \hline
 \hline
 \multicolumn{8}{c}{GPT3-1.3B (FP32 PPL = 9.98)} \\ 
 \hline
 \hline
 64 & 10.40 & 10.23 & 10.17 & 10.15 &  10.28 & 10.18 & 10.19 \\
 \hline
 32 & 10.25 & 10.20 & 10.15 & 10.12 &  10.23 & 10.17 & 10.17 \\
 \hline
 16 & 10.22 & 10.16 & 10.10 & 10.09 &  10.21 & 10.14 & 10.16 \\
 \hline
  \hline
 \multicolumn{8}{c}{GPT3-8B (FP32 PPL = 7.38)} \\ 
 \hline
 \hline
 64 & 7.61 & 7.52 & 7.48 &  7.47 &  7.55 &  7.49 & 7.50 \\
 \hline
 32 & 7.52 & 7.50 & 7.46 &  7.45 &  7.52 &  7.48 & 7.48  \\
 \hline
 16 & 7.51 & 7.48 & 7.44 &  7.44 &  7.51 &  7.49 & 7.47  \\
 \hline
\end{tabular}
\caption{\label{tab:ppl_gpt3_abalation} Wikitext-103 perplexity across GPT3-1.3B and 8B models.}
\end{table}

\begin{table} \centering
\begin{tabular}{|c||c|c|c|c||} 
\hline
 $L_b \rightarrow$& \multicolumn{4}{c||}{8}\\
 \hline
 \backslashbox{$L_A$\kern-1em}{\kern-1em$N_c$} & 2 & 4 & 8 & 16 \\
 %$N_c \rightarrow$ & 2 & 4 & 8 & 16 & 2 & 4 & 2 \\
 \hline
 \hline
 \multicolumn{5}{|c|}{Llama2-7B (FP32 PPL = 5.06)} \\ 
 \hline
 \hline
 64 & 5.31 & 5.26 & 5.19 & 5.18  \\
 \hline
 32 & 5.23 & 5.25 & 5.18 & 5.15  \\
 \hline
 16 & 5.23 & 5.19 & 5.16 & 5.14  \\
 \hline
 \multicolumn{5}{|c|}{Nemotron4-15B (FP32 PPL = 5.87)} \\ 
 \hline
 \hline
 64  & 6.3 & 6.20 & 6.13 & 6.08  \\
 \hline
 32  & 6.24 & 6.12 & 6.07 & 6.03  \\
 \hline
 16  & 6.12 & 6.14 & 6.04 & 6.02  \\
 \hline
 \multicolumn{5}{|c|}{Nemotron4-340B (FP32 PPL = 3.48)} \\ 
 \hline
 \hline
 64 & 3.67 & 3.62 & 3.60 & 3.59 \\
 \hline
 32 & 3.63 & 3.61 & 3.59 & 3.56 \\
 \hline
 16 & 3.61 & 3.58 & 3.57 & 3.55 \\
 \hline
\end{tabular}
\caption{\label{tab:ppl_llama7B_nemo15B} Wikitext-103 perplexity compared to FP32 baseline in Llama2-7B and Nemotron4-15B, 340B models}
\end{table}

%\subsection{Perplexity achieved by various LO-BCQ configurations on MMLU dataset}


\begin{table} \centering
\begin{tabular}{|c||c|c|c|c||c|c|c|c|} 
\hline
 $L_b \rightarrow$& \multicolumn{4}{c||}{8} & \multicolumn{4}{c||}{8}\\
 \hline
 \backslashbox{$L_A$\kern-1em}{\kern-1em$N_c$} & 2 & 4 & 8 & 16 & 2 & 4 & 8 & 16  \\
 %$N_c \rightarrow$ & 2 & 4 & 8 & 16 & 2 & 4 & 2 \\
 \hline
 \hline
 \multicolumn{5}{|c|}{Llama2-7B (FP32 Accuracy = 45.8\%)} & \multicolumn{4}{|c|}{Llama2-70B (FP32 Accuracy = 69.12\%)} \\ 
 \hline
 \hline
 64 & 43.9 & 43.4 & 43.9 & 44.9 & 68.07 & 68.27 & 68.17 & 68.75 \\
 \hline
 32 & 44.5 & 43.8 & 44.9 & 44.5 & 68.37 & 68.51 & 68.35 & 68.27  \\
 \hline
 16 & 43.9 & 42.7 & 44.9 & 45 & 68.12 & 68.77 & 68.31 & 68.59  \\
 \hline
 \hline
 \multicolumn{5}{|c|}{GPT3-22B (FP32 Accuracy = 38.75\%)} & \multicolumn{4}{|c|}{Nemotron4-15B (FP32 Accuracy = 64.3\%)} \\ 
 \hline
 \hline
 64 & 36.71 & 38.85 & 38.13 & 38.92 & 63.17 & 62.36 & 63.72 & 64.09 \\
 \hline
 32 & 37.95 & 38.69 & 39.45 & 38.34 & 64.05 & 62.30 & 63.8 & 64.33  \\
 \hline
 16 & 38.88 & 38.80 & 38.31 & 38.92 & 63.22 & 63.51 & 63.93 & 64.43  \\
 \hline
\end{tabular}
\caption{\label{tab:mmlu_abalation} Accuracy on MMLU dataset across GPT3-22B, Llama2-7B, 70B and Nemotron4-15B models.}
\end{table}


%\subsection{Perplexity achieved by various LO-BCQ configurations on LM evaluation harness}

\begin{table} \centering
\begin{tabular}{|c||c|c|c|c||c|c|c|c|} 
\hline
 $L_b \rightarrow$& \multicolumn{4}{c||}{8} & \multicolumn{4}{c||}{8}\\
 \hline
 \backslashbox{$L_A$\kern-1em}{\kern-1em$N_c$} & 2 & 4 & 8 & 16 & 2 & 4 & 8 & 16  \\
 %$N_c \rightarrow$ & 2 & 4 & 8 & 16 & 2 & 4 & 2 \\
 \hline
 \hline
 \multicolumn{5}{|c|}{Race (FP32 Accuracy = 37.51\%)} & \multicolumn{4}{|c|}{Boolq (FP32 Accuracy = 64.62\%)} \\ 
 \hline
 \hline
 64 & 36.94 & 37.13 & 36.27 & 37.13 & 63.73 & 62.26 & 63.49 & 63.36 \\
 \hline
 32 & 37.03 & 36.36 & 36.08 & 37.03 & 62.54 & 63.51 & 63.49 & 63.55  \\
 \hline
 16 & 37.03 & 37.03 & 36.46 & 37.03 & 61.1 & 63.79 & 63.58 & 63.33  \\
 \hline
 \hline
 \multicolumn{5}{|c|}{Winogrande (FP32 Accuracy = 58.01\%)} & \multicolumn{4}{|c|}{Piqa (FP32 Accuracy = 74.21\%)} \\ 
 \hline
 \hline
 64 & 58.17 & 57.22 & 57.85 & 58.33 & 73.01 & 73.07 & 73.07 & 72.80 \\
 \hline
 32 & 59.12 & 58.09 & 57.85 & 58.41 & 73.01 & 73.94 & 72.74 & 73.18  \\
 \hline
 16 & 57.93 & 58.88 & 57.93 & 58.56 & 73.94 & 72.80 & 73.01 & 73.94  \\
 \hline
\end{tabular}
\caption{\label{tab:mmlu_abalation} Accuracy on LM evaluation harness tasks on GPT3-1.3B model.}
\end{table}

\begin{table} \centering
\begin{tabular}{|c||c|c|c|c||c|c|c|c|} 
\hline
 $L_b \rightarrow$& \multicolumn{4}{c||}{8} & \multicolumn{4}{c||}{8}\\
 \hline
 \backslashbox{$L_A$\kern-1em}{\kern-1em$N_c$} & 2 & 4 & 8 & 16 & 2 & 4 & 8 & 16  \\
 %$N_c \rightarrow$ & 2 & 4 & 8 & 16 & 2 & 4 & 2 \\
 \hline
 \hline
 \multicolumn{5}{|c|}{Race (FP32 Accuracy = 41.34\%)} & \multicolumn{4}{|c|}{Boolq (FP32 Accuracy = 68.32\%)} \\ 
 \hline
 \hline
 64 & 40.48 & 40.10 & 39.43 & 39.90 & 69.20 & 68.41 & 69.45 & 68.56 \\
 \hline
 32 & 39.52 & 39.52 & 40.77 & 39.62 & 68.32 & 67.43 & 68.17 & 69.30  \\
 \hline
 16 & 39.81 & 39.71 & 39.90 & 40.38 & 68.10 & 66.33 & 69.51 & 69.42  \\
 \hline
 \hline
 \multicolumn{5}{|c|}{Winogrande (FP32 Accuracy = 67.88\%)} & \multicolumn{4}{|c|}{Piqa (FP32 Accuracy = 78.78\%)} \\ 
 \hline
 \hline
 64 & 66.85 & 66.61 & 67.72 & 67.88 & 77.31 & 77.42 & 77.75 & 77.64 \\
 \hline
 32 & 67.25 & 67.72 & 67.72 & 67.00 & 77.31 & 77.04 & 77.80 & 77.37  \\
 \hline
 16 & 68.11 & 68.90 & 67.88 & 67.48 & 77.37 & 78.13 & 78.13 & 77.69  \\
 \hline
\end{tabular}
\caption{\label{tab:mmlu_abalation} Accuracy on LM evaluation harness tasks on GPT3-8B model.}
\end{table}

\begin{table} \centering
\begin{tabular}{|c||c|c|c|c||c|c|c|c|} 
\hline
 $L_b \rightarrow$& \multicolumn{4}{c||}{8} & \multicolumn{4}{c||}{8}\\
 \hline
 \backslashbox{$L_A$\kern-1em}{\kern-1em$N_c$} & 2 & 4 & 8 & 16 & 2 & 4 & 8 & 16  \\
 %$N_c \rightarrow$ & 2 & 4 & 8 & 16 & 2 & 4 & 2 \\
 \hline
 \hline
 \multicolumn{5}{|c|}{Race (FP32 Accuracy = 40.67\%)} & \multicolumn{4}{|c|}{Boolq (FP32 Accuracy = 76.54\%)} \\ 
 \hline
 \hline
 64 & 40.48 & 40.10 & 39.43 & 39.90 & 75.41 & 75.11 & 77.09 & 75.66 \\
 \hline
 32 & 39.52 & 39.52 & 40.77 & 39.62 & 76.02 & 76.02 & 75.96 & 75.35  \\
 \hline
 16 & 39.81 & 39.71 & 39.90 & 40.38 & 75.05 & 73.82 & 75.72 & 76.09  \\
 \hline
 \hline
 \multicolumn{5}{|c|}{Winogrande (FP32 Accuracy = 70.64\%)} & \multicolumn{4}{|c|}{Piqa (FP32 Accuracy = 79.16\%)} \\ 
 \hline
 \hline
 64 & 69.14 & 70.17 & 70.17 & 70.56 & 78.24 & 79.00 & 78.62 & 78.73 \\
 \hline
 32 & 70.96 & 69.69 & 71.27 & 69.30 & 78.56 & 79.49 & 79.16 & 78.89  \\
 \hline
 16 & 71.03 & 69.53 & 69.69 & 70.40 & 78.13 & 79.16 & 79.00 & 79.00  \\
 \hline
\end{tabular}
\caption{\label{tab:mmlu_abalation} Accuracy on LM evaluation harness tasks on GPT3-22B model.}
\end{table}

\begin{table} \centering
\begin{tabular}{|c||c|c|c|c||c|c|c|c|} 
\hline
 $L_b \rightarrow$& \multicolumn{4}{c||}{8} & \multicolumn{4}{c||}{8}\\
 \hline
 \backslashbox{$L_A$\kern-1em}{\kern-1em$N_c$} & 2 & 4 & 8 & 16 & 2 & 4 & 8 & 16  \\
 %$N_c \rightarrow$ & 2 & 4 & 8 & 16 & 2 & 4 & 2 \\
 \hline
 \hline
 \multicolumn{5}{|c|}{Race (FP32 Accuracy = 44.4\%)} & \multicolumn{4}{|c|}{Boolq (FP32 Accuracy = 79.29\%)} \\ 
 \hline
 \hline
 64 & 42.49 & 42.51 & 42.58 & 43.45 & 77.58 & 77.37 & 77.43 & 78.1 \\
 \hline
 32 & 43.35 & 42.49 & 43.64 & 43.73 & 77.86 & 75.32 & 77.28 & 77.86  \\
 \hline
 16 & 44.21 & 44.21 & 43.64 & 42.97 & 78.65 & 77 & 76.94 & 77.98  \\
 \hline
 \hline
 \multicolumn{5}{|c|}{Winogrande (FP32 Accuracy = 69.38\%)} & \multicolumn{4}{|c|}{Piqa (FP32 Accuracy = 78.07\%)} \\ 
 \hline
 \hline
 64 & 68.9 & 68.43 & 69.77 & 68.19 & 77.09 & 76.82 & 77.09 & 77.86 \\
 \hline
 32 & 69.38 & 68.51 & 68.82 & 68.90 & 78.07 & 76.71 & 78.07 & 77.86  \\
 \hline
 16 & 69.53 & 67.09 & 69.38 & 68.90 & 77.37 & 77.8 & 77.91 & 77.69  \\
 \hline
\end{tabular}
\caption{\label{tab:mmlu_abalation} Accuracy on LM evaluation harness tasks on Llama2-7B model.}
\end{table}

\begin{table} \centering
\begin{tabular}{|c||c|c|c|c||c|c|c|c|} 
\hline
 $L_b \rightarrow$& \multicolumn{4}{c||}{8} & \multicolumn{4}{c||}{8}\\
 \hline
 \backslashbox{$L_A$\kern-1em}{\kern-1em$N_c$} & 2 & 4 & 8 & 16 & 2 & 4 & 8 & 16  \\
 %$N_c \rightarrow$ & 2 & 4 & 8 & 16 & 2 & 4 & 2 \\
 \hline
 \hline
 \multicolumn{5}{|c|}{Race (FP32 Accuracy = 48.8\%)} & \multicolumn{4}{|c|}{Boolq (FP32 Accuracy = 85.23\%)} \\ 
 \hline
 \hline
 64 & 49.00 & 49.00 & 49.28 & 48.71 & 82.82 & 84.28 & 84.03 & 84.25 \\
 \hline
 32 & 49.57 & 48.52 & 48.33 & 49.28 & 83.85 & 84.46 & 84.31 & 84.93  \\
 \hline
 16 & 49.85 & 49.09 & 49.28 & 48.99 & 85.11 & 84.46 & 84.61 & 83.94  \\
 \hline
 \hline
 \multicolumn{5}{|c|}{Winogrande (FP32 Accuracy = 79.95\%)} & \multicolumn{4}{|c|}{Piqa (FP32 Accuracy = 81.56\%)} \\ 
 \hline
 \hline
 64 & 78.77 & 78.45 & 78.37 & 79.16 & 81.45 & 80.69 & 81.45 & 81.5 \\
 \hline
 32 & 78.45 & 79.01 & 78.69 & 80.66 & 81.56 & 80.58 & 81.18 & 81.34  \\
 \hline
 16 & 79.95 & 79.56 & 79.79 & 79.72 & 81.28 & 81.66 & 81.28 & 80.96  \\
 \hline
\end{tabular}
\caption{\label{tab:mmlu_abalation} Accuracy on LM evaluation harness tasks on Llama2-70B model.}
\end{table}

%\section{MSE Studies}
%\textcolor{red}{TODO}


\subsection{Number Formats and Quantization Method}
\label{subsec:numFormats_quantMethod}
\subsubsection{Integer Format}
An $n$-bit signed integer (INT) is typically represented with a 2s-complement format \citep{yao2022zeroquant,xiao2023smoothquant,dai2021vsq}, where the most significant bit denotes the sign.

\subsubsection{Floating Point Format}
An $n$-bit signed floating point (FP) number $x$ comprises of a 1-bit sign ($x_{\mathrm{sign}}$), $B_m$-bit mantissa ($x_{\mathrm{mant}}$) and $B_e$-bit exponent ($x_{\mathrm{exp}}$) such that $B_m+B_e=n-1$. The associated constant exponent bias ($E_{\mathrm{bias}}$) is computed as $(2^{{B_e}-1}-1)$. We denote this format as $E_{B_e}M_{B_m}$.  

\subsubsection{Quantization Scheme}
\label{subsec:quant_method}
A quantization scheme dictates how a given unquantized tensor is converted to its quantized representation. We consider FP formats for the purpose of illustration. Given an unquantized tensor $\bm{X}$ and an FP format $E_{B_e}M_{B_m}$, we first, we compute the quantization scale factor $s_X$ that maps the maximum absolute value of $\bm{X}$ to the maximum quantization level of the $E_{B_e}M_{B_m}$ format as follows:
\begin{align}
\label{eq:sf}
    s_X = \frac{\mathrm{max}(|\bm{X}|)}{\mathrm{max}(E_{B_e}M_{B_m})}
\end{align}
In the above equation, $|\cdot|$ denotes the absolute value function.

Next, we scale $\bm{X}$ by $s_X$ and quantize it to $\hat{\bm{X}}$ by rounding it to the nearest quantization level of $E_{B_e}M_{B_m}$ as:

\begin{align}
\label{eq:tensor_quant}
    \hat{\bm{X}} = \text{round-to-nearest}\left(\frac{\bm{X}}{s_X}, E_{B_e}M_{B_m}\right)
\end{align}

We perform dynamic max-scaled quantization \citep{wu2020integer}, where the scale factor $s$ for activations is dynamically computed during runtime.

\subsection{Vector Scaled Quantization}
\begin{wrapfigure}{r}{0.35\linewidth}
  \centering
  \includegraphics[width=\linewidth]{sections/figures/vsquant.jpg}
  \caption{\small Vectorwise decomposition for per-vector scaled quantization (VSQ \citep{dai2021vsq}).}
  \label{fig:vsquant}
\end{wrapfigure}
During VSQ \citep{dai2021vsq}, the operand tensors are decomposed into 1D vectors in a hardware friendly manner as shown in Figure \ref{fig:vsquant}. Since the decomposed tensors are used as operands in matrix multiplications during inference, it is beneficial to perform this decomposition along the reduction dimension of the multiplication. The vectorwise quantization is performed similar to tensorwise quantization described in Equations \ref{eq:sf} and \ref{eq:tensor_quant}, where a scale factor $s_v$ is required for each vector $\bm{v}$ that maps the maximum absolute value of that vector to the maximum quantization level. While smaller vector lengths can lead to larger accuracy gains, the associated memory and computational overheads due to the per-vector scale factors increases. To alleviate these overheads, VSQ \citep{dai2021vsq} proposed a second level quantization of the per-vector scale factors to unsigned integers, while MX \citep{rouhani2023shared} quantizes them to integer powers of 2 (denoted as $2^{INT}$).

\subsubsection{MX Format}
The MX format proposed in \citep{rouhani2023microscaling} introduces the concept of sub-block shifting. For every two scalar elements of $b$-bits each, there is a shared exponent bit. The value of this exponent bit is determined through an empirical analysis that targets minimizing quantization MSE. We note that the FP format $E_{1}M_{b}$ is strictly better than MX from an accuracy perspective since it allocates a dedicated exponent bit to each scalar as opposed to sharing it across two scalars. Therefore, we conservatively bound the accuracy of a $b+2$-bit signed MX format with that of a $E_{1}M_{b}$ format in our comparisons. For instance, we use E1M2 format as a proxy for MX4.

\begin{figure}
    \centering
    \includegraphics[width=1\linewidth]{sections//figures/BlockFormats.pdf}
    \caption{\small Comparing LO-BCQ to MX format.}
    \label{fig:block_formats}
\end{figure}

Figure \ref{fig:block_formats} compares our $4$-bit LO-BCQ block format to MX \citep{rouhani2023microscaling}. As shown, both LO-BCQ and MX decompose a given operand tensor into block arrays and each block array into blocks. Similar to MX, we find that per-block quantization ($L_b < L_A$) leads to better accuracy due to increased flexibility. While MX achieves this through per-block $1$-bit micro-scales, we associate a dedicated codebook to each block through a per-block codebook selector. Further, MX quantizes the per-block array scale-factor to E8M0 format without per-tensor scaling. In contrast during LO-BCQ, we find that per-tensor scaling combined with quantization of per-block array scale-factor to E4M3 format results in superior inference accuracy across models. 


\end{document}
\endinput
%%
%% End of file `sample-sigconf.tex'.
