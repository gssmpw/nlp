
\appendix


\section*{Appendix}
\label{sec:appendix}


\section{Ethics}


We query and crawl terms and conditions from online shopping websites, collecting data from each site only once. Since terms and conditions are usually limited to a few subpages, this process does not overburden the servers hosting these websites. In this paper, we report some terms and conditions along with the associated companies, all of which are publicly available information. No personal data is collected during the measurement process.


\section{Financial Terms Topic Template}
\label{sec:appendix_finaincial_terms}


The financial term taxonomy consists of \financialcnt categories:

\begin{enumerate}
    \item \textit{Subscription/Product Terms}: Terms related to subscription fees, billing, and automatic renewals.

    \item \textit{Service Termination Policy}: Terms outlining the financial implications of service termination.
    \item \textit{Payment and Purchase Term}: Terms about payments, processing fees, and currency transactions.
    \item \textit{Return and Refund Policy}: Terms governing product returns and service cancellations.
    \item \textit{Insurance and Warranty Term}: Terms related to coverage, claims, limitations, and premiums.
    \item \textit{Promotions and Rewards}: Terms about offers, discounts, loyalty programs, and rewards.
    \item \textit{Shipping and Handling Terms}: Terms on product shipping costs and policies.
    \item \textit{Dispute Resolution Policy}: Terms outlining dispute resolution methods and processes.
    \item \textit{Investment and Trading Terms}: Terms specific to investment/trading platforms.
    \item \textit{Intellectual Property Terms}: Terms on rights and restrictions for intellectual property use.
    \item \textit{Financial Glossary}: Financial terminology definition.
    \item \textit{Others}: Includes less frequently mentioned financial terms.
\end{enumerate}

\section{Classification Prompts}
\label{sec:prompts}
Below is the simple binary prompt (``Simple Prompt'') used for zero-shot unfavorable financial term classification evaluation.
\begin{lstlisting}[label=Simple Prompt]
Classify the following term as 'malicious' or 'benign'. A term is 'malicious' if it is a financial term that is one-sided, unbalanced, unfair, or harmful to users. 

Respond only with 'm' for malicious or 'b' for benign.
\end{lstlisting}


Below is the multi-class taxonomy prompt (``Unfavorable Term Taxonomy Prompt'') used for zero-shot evaluation of unfavorable financial term classification and for fine-tuning LLMs.
\begin{lstlisting}[label=Unfavorable Term Taxonomy Prompt]
You will be provided with a paragraph extracted from the terms and conditions. Your task is to classify them into one of the topics below or 'b' for 'benign':

- Automatic Subscription after Free Trial: Automatically subscribing users after free trials.
- [...]

If the term is not a financial term or a reasonable financial term based on common sense, reply 'b' for 'benign'.

If the term is malicious and financial, reply with a topic from the template above. 
\end{lstlisting}




\section{Shopping website classification Evaluation}
\label{sec:website_cls}

\begin{table}[t!]
\centering
\caption{Performance and Cost Analysis for Zero-Shot Shopping Website Classification Using GPT-3.5-Turbo~\citep{gpt35} and GPT-4o~\citep{openai2023gpt4} Models on 500 randomly selected websites.}
\footnotesize
\begin{tabular}{p{15mm} p{13mm} p{4mm} p{4.5mm} p{4.5mm} p{4.5mm} p{4mm} }
\toprule
{Model} & {Configuration} & {ACC (\%)} & {TP (\%)} & {FP (\%)} & {TN (\%)} & {FN (\%)}\\ 
\midrule
\multirow{2}{*}{{GPT-3.5-Turbo}} &
URL & 73 & 80.9 & 36.5 & 63.5 & 19.1  \\
& URL+HTML & 24 & 31.5 & 5.5 & 94.5 & 68.5 \\
\midrule

\multirow{2}{*}{{GPT-4o}} &
URL & 63 & 40.9 & 10.1 & 89.9 & 59.1 \\
& URL+Image & 86 & 90.7 & 20.6 & 79.4 & 9.3  \\

\bottomrule
\end{tabular}

\label{table:website_cls}
\end{table}


% $0.02 

% \$1.06

To evaluate our classification methods, we randomly sampled 500 websites from the top 10,000 on the Tranco list for manual annotation to identify shopping sites (offering products or services for sale). Of these, 257 were categorized as ``shopping'', 219 as ``non-shopping'', and 24 were inaccessible, due to either server issues or geo-blocking IPs in the US.
As indicated in~\autoref{table:website_cls}, GPT-4o's accuracy reached 86\% when analyzing URLs with corresponding screenshots. Further examination of a focused group, specifically English-language websites with available T\&Cs (115 out of 500), reveals that accuracy improved to 81\% for GPT-3 using only URLs and to 92\% for GPT-4o with screenshots, approaching the commercial-grade classification benchmarks reported in previous studies (89\%-93\%)~\citep{mathur2019dark}. The FPR of the focused group is 6.3\%. Based on these statistics, GPT-4o paired with image data is selected for the broader scale measurement of shopping websites.


\section{\TermName Terms Taxonomy}
\label{sec:detailed_tax}


We discover \termname term types falling under 4 broader categories: 1) purchase and billing terms; 2) post-purchase terms; 3) termination and account recovery terms; and 4) legal terms. We describe the taxonomy with an explanation for each type below:

\subsubsection{Unfavorable Purchase and Billing Terms}

This category includes subscription, purchase, and billing terms that are unfavorable or concerning for customers:

\begin{itemize}
    \item \textbf{Immediate Automatic Subscription.} 
    Additional subscriptions are automatically added when purchasing an item or during promotions without clear consent from the user.
    \item \textbf{Automatic Subscription after Free Trial.} Users are automatically enrolled in a paid subscription after a trial period ends unless they actively cancel.
    \item \textbf{Unilateral Unauthorized Account Upgrades.} Accounts may be upgraded and charged at higher rates without providing prior notice to the user.
    \item \textbf{Late or Unsuccessful Payment Penalty.} Penalties or interest charges are applied for late or unsuccessful payments.
    \item \textbf{Overuse Penalty.} Charging extra fees if usage limits are exceeded. Typically found in subscription-based services such as data plans, cloud storage, and streaming services.
    \item \textbf{Retroactive Application of Price Change.} 
    Price (of sub-scription-based services) increases can be applied retroactively without prior notice to the user.
\end{itemize}


\subsubsection{Unfavorable Post-Purchase Terms }
This category includes cancellation, shipping, return, and refund terms that are unfavorable or concerning for customers:

\begin{itemize}
    \item \textbf{Non-Refundable Subscription Fee.} Subscription fees that have already been charged are not refunded.
    \item \textbf{No Refund for Purchase.} Purchases of individual items are final and non-refundable.
    \item \textbf{Strict No Cancellation Policy.} Orders cannot be canceled once they have been processed.
    \item \textbf{Cancellation Fee or Penalty.} Fees are applied for canceling certain bookings, services, or online purchase orders.
    \item \textbf{Non-Refundable Additional Fee.} Charging non-refundable additional fees under various labels such as service fees, transfer fees, pre-authorization fees, administrative fees, subscription upgrade fees, handling product fees, etc.
    \item \textbf{Non-Monetary Refund Alternatives.} Refunds are provided in the form of coupons, reward points, or store credit rather than money.
    \item \textbf{No Responsibility for Delivery Delays.} Companies are not held liable for delays in product delivery.
    \item \textbf{Customers Responsible for Shipping Issues.} Customers are responsible for handling customs issues, additional shipping charges, and any shipping-related complications that do not involve delays.
    \item \textbf{Customers Pay Return Shipping.} Customers bear the cost of return shipping for products.
    \item \textbf{Restocking Fee.} A fee is charged for restocking returned items.
\end{itemize}

\subsubsection{Unfavorable Termination and Account Recovery Terms}
This category includes account or service termination, deactivation, and reactivation terms that are unfavorable or concerning for customers:

\begin{itemize}
    \item \textbf{Account Recovery Fee.} A Fee is charged to recover locked or archived accounts.
    \item \textbf{Digital Currency, Reward, Money Seizure on Inactivity.} Digital assets, such as rewards, points, and virtual currencies, are forfeited or otherwise taken away if accounts remain inactive for extended periods.
    \item \textbf{Digital Currency, Reward, Money Seizure on Termination or Account Closure.} Digital assets, such as rewards, points, and virtual currencies, are forfeited or otherwise taken away upon service termination or account closure.
\end{itemize}

\subsubsection{Unfavorable Legal Terms}
This category includes legal terms that are unfavorable or concerning for customers:
\begin{itemize}
    \item \textbf{Exorbitant Legal Document Request Fee.} High fees are charged for requesting legal documents.
    \item \textbf{Forced Waiver of Legal Protections.} Customers are required to waive certain legal protections.
    \item \textbf{Forced Waiver of Class Action Rights.} Customers waive their rights to participate in class action lawsuits.
    \item \textbf{Other Legal Unfavorable Financial Term.} Additional legal terms that impose financial burdens or limit legal recourse for the customer.
\end{itemize}

\begin{figure}[!t] 
 \centering
 \includegraphics[width=0.99\columnwidth]{imgs/scam_example_2.pdf}
 \caption{Extracted from the T\&C of Celsius Network LLC, a now bankrupt cryptocurrency company. }
 \Description[Excerpt from Celsius Network LLC’s terms and conditions.]
 {This figure presents a term extracted from the terms and conditions (T\&C) of Celsius Network LLC, a cryptocurrency company that later filed for bankruptcy. 
 The excerpt highlights specific clauses related to user funds and liabilities, illustrating how unfavorable financial terms may have impacted customers.}
\label{fig:example2}
\end{figure}
\begin{figure}[t] 
 \centering
 \includegraphics[width=0.97\columnwidth]{imgs/Neteller_review.pdf}
 \caption{Complied reviews for Neteller~\citep{neteller} from Trust Pilot~\citep{trustpilot} regarding account closure.}
 \Description[Trustpilot reviews of Neteller account closures.]
 {This figure presents compiled user reviews from Trustpilot regarding account closures by Neteller.
 The reviews highlight customer complaints about sudden account terminations, withheld funds, and difficulties in recovering their balances.
 The feedback provides insight into customer dissatisfaction with Neteller’s account closure policies.}
\label{fig:neteller}
\end{figure}

Many terms and conditions for online shopping websites include strong legal language, such as waivers of class action rights, arbitration clauses, and limitations of liability. This study does not specifically focus on the legal aspects for two main reasons: (1) Although legal terms can impact users financially, they differ from other categories we report. These terms, despite their potential future implications, are not the primary concern when customers make a purchase or sign up for a service. (2) There is another line of work (see Appendix~\ref{sec:appendix_other_templates}) that focuses on terms deemed invalid in court. We consider our work complementary to these studies. By not intensively focusing on legal terms, we maintain a focus on terms with immediate and direct financial implications for users.


\section{Case Studies}
\label{sec:case_studies}
We present four case studies illustrating the potential harm of \termname terms in each \termname term category. 


\myparagraph{Unfavorable post-purchase terms case study} National Park Reservations~\citep{nationalparkreservations}, a company providing national park hotel and lodging reservation service with a 1-star review on Yelp, includes the following terms and conditions:

\begin{quote}
    For this service, National Park Reservations charges a 10\% non-refundable reservation fee based on the total dollar amount of reservations made. This reservation fee will be billed separately to your credit card and will be billed under the memo “National Park Reservations”. By using National Park Reservations, the customer authorizes National Park Reservations to charge their credit card the 10\% non-refundable fee.
\end{quote}

\begin{figure}[!t] 
 \centering
 \includegraphics[width=0.99\columnwidth]{imgs/wordcloud.pdf}
 \caption{Word cloud based on the top 50 Yelp reviews of National Park Reservations, whose T\&Cs specify a 10\% non-refundable booking fee. ``Scam'' and ``non-refundable'' are frequently mentioned words in the reviews.}
 \Description[Word cloud of Yelp reviews for National Park Reservations.]
 {This figure displays a word cloud generated from the top 50 Yelp reviews of National Park Reservations.
 The company's terms and conditions specify a 10\% non-refundable booking fee, which is a common theme in customer complaints.
 Frequently appearing words in the reviews include ``scam'' and ``non-refundable,'' indicating recurring concerns from users regarding the booking policy.}
\label{fig:wordcloud}
\end{figure}

The above terms fall under the ``Non-Refundable Additional Fee'' category. \autoref{fig:wordcloud} shows a word cloud that displays the most frequent words from the top 50 Yelp reviews (2021-2024), excluding the company name. Despite the non-refundable additional fee being clearly stated in the terms and conditions, many customers still find it deceptive. ``Scam'' is among the most frequent words in the reviews. This shows the potential harm caused by \termname terms and perceived deceptive practices, significantly impacting customer trust and satisfaction.



\myparagraph{Unfavorable termination and account recovery terms case study} 
Compared to other categories of \termname terms, those related to unfair, unfavorable, or concerning service termination and account management are significantly less prevalent in our measurements, as shown in~\autoref{fig:large_scale_stats}. However, these terms can still impose substantial costs on customers. For example, the Terms of Use from Neteller~\citep{neteller}---a digital wallet with a TrustScore of 10 out of 100 from Scamadviser~\citep{scamadviser2019algorithm}---include such terms:

\begin{quote}
    If an Account has been closed, [...]. Fees relating to ongoing management of inactive accounts will also continue to be charged following the closure of your Account. This provision shall survive the termination of the relationship between you and us.
\end{quote}


This term means that even after an account is closed, fees for managing inactive accounts will continue to be charged, potentially resulting in unexpected costs for customers. See~\autoref{fig:neteller} in the Appendix for a compilation of customer complaints about account closures and issues with retrieving deposited funds. Similar to~\autoref{fig:example2}, the terms from Neteller demonstrate the potential harm of unfavorable termination and account recovery terms.


\myparagraph{Unfavorable legal terms case study} Another concerning \termname term is the inclusion of a waiver for California Civil Code Section 1542, found in 2.1\% (152 out of 7,225) of the top 80,000 Tranco-ranked websites. An example of such a term states:

\begin{quote}
If you are a California resident, you shall and hereby do waive California Civil Code Section 1542, which says: “A general release does not extend to claims which the creditor does not know or suspect to exist in his favor at the time of executing the release, which, if known by him must have materially affected his settlement with the debtor.”
\end{quote}

California Civil Code Section 1542 protects individuals from unknowingly giving up their right to make claims for issues they were not aware of at the time they signed a release. By including a waiver for this code, websites are essentially asking customers to give up this important protection. This indicates that these sites are aware that most people do not thoroughly read the terms and conditions. This oversight can be leveraged to disable significant legal protection, which can make co-existing \termname terms harder to dispute. In fact, 60.5\% (92 out of 152) of the websites with the California Civil Code Section 1542 waiver also have at least one other category of \termname terms.










\section{Comparison with Other Online Agreement Annotation Scheme}
\label{sec:appendix_other_templates}

In this section, we introduce the annotation templates proposed under the European Union (EU) framework for identifying unfair contract terms~\citep{loos2016wanted, galassi2024unfair, galassi2020cross, lippi2019claudette, drazewski2021corpus}. While these studies emphasize legal categories and jurisdictional issues, our research specifically targets financial terms in online service agreements. 

Loos \etal~\citep{loos2016wanted} analyze the unfair contract terms of online service providers in light of the Unfair Contract Terms Directive (UCTD)~\citep{CouncilDirective1993} of the European Union. The authors examine various types of contractual terms from international online service providers, identifying those that are unlikely to pass the Directive’s fairness test in the following five categories:

\begin{itemize}
    \item \textbf{Unilateral Changes}: Our analysis also considers unilateral changes made by online service providers, particularly regarding financial aspects such as unilateral price changes, plan upgrades, and various penalties.
    \item \textbf{Termination Clauses}: We examine termination clauses focusing on their financial consequences, including the seizure of digital currency, reward points, or money upon termination.
    \item \textbf{Liability Exclusions and Limitations}: Both our paper and the authors' findings highlight the problematic nature of liability exclusions and limitations, which often unjustly limit the providers' responsibility for service failures, thereby creating a significant imbalance in the parties' rights and obligations.
    \item \textbf{International Jurisdiction and Choice-of-Law Clauses}: Although we have a category for unfavorable legal terms, the detailed categorization of unfair legal terms is deferred to future work. This is because, compared to other unfavorable financial terms, legal terms typically have a more indirect impact on users.
    
    \item \textbf{Transparency}: While the readability and accessibility of terms and conditions are not the main focus of this paper, our data collection and topic modeling pipeline can be readily adapted for future research in these areas.
\end{itemize}


Another worth-noting line of work in unfair online agreements~\citep{galassi2024unfair, galassi2020cross, lippi2019claudette, drazewski2021corpus}. The CLAUDETTE~\citep{lippi2019claudette} system evaluates the fairness of terms within the jurisdiction of the European Union by leveraging legal standards and principles established within the EU framework. The annotation scheme is as follows:

\begin{itemize}
    \item Jurisdiction for disputes in a country different from the consumer’s residence.
    \item Choice of a foreign law governing the contract.
    \item Limitation of liability.
    \item The provider’s right to unilaterally terminate the contract/access to the service.
    \item The provider’s right to unilaterally modify the contract/service.
    \item Requiring a consumer to undertake arbitration before court proceedings can commence.
    \item The provider retaining the right to unilaterally remove consumer content from the service, including in-app purchases.
    \item Having a consumer accept the agreement simply by using the service, without having to click on “I agree/I accept”
    \item The scope of consent granted to the ToS also takes in the privacy policy, forming part of the “General Agreement”
\end{itemize}

We consider our work to be a complementary addition to the AI \& Law database, with our template being more aligned with the natural phrasing found in terms and conditions of online shopping websites. We hope that future research will incorporate both the legal templates and our proposed template to provide a more comprehensive understanding of the landscape of unfair (financial) terms.




\section{Term Page Regex}
\label{sec:appendix_reg}

Below are the positive and negative regex matching pattern we deploy for this work:

\begin{lstlisting}[label=Positive]
positive_regex = [
    "terms.*?conditions",
    "terms.*?of.*?use",
    "terms.*?of.*?service",
    "terms.*?of.*?sale",
    "terms.*?of.*?conditions",
    "terms.*?and.*?conditions",
    "terms.*?&.*?conditions",
    "conditions.*?of.*?use",
    "intellectual.*property.*policy",
    "return[s]?.*?policy",
    "refund[s]?.*?policy",
    "return.*?and.*?refund.*?policy",
    "cancellation.*?and.*?returns",
    "cancellation.*?returns",
    "prohibited.*conduct",
    "electronic.*communication.*policy",
    "safety.*guideline",
    "requests.*from.*law.*enforcement",
    "bonus.*terms.*apply",
    "community.*rules",
    "gift.*card.*policy",
    "contact.*us.*here",
    "shipping.*policy",
    "warranty",
    "end.*user.*license",
    "user.*?agreement",
    "payment.*terms",
    "content.*policy",
    "terms"
]

\end{lstlisting}

The negative regex list is as follows:
\begin{lstlisting}[label=Negative]
negative_regex = [
    "privacy.*?policy",
    "cookie.*?policy",
    "privacy.*?notice",
    "sale.*?tax.*?policy",
    "prohibited.*?items",
    "1099.*?k.*?form",
    "dmca.*copyright.*notification",
]

\end{lstlisting}





