\documentclass[conference]{IEEEtran}

\usepackage{graphicx}
\graphicspath{ {./Figures/} }
\IEEEoverridecommandlockouts
\usepackage[american]{babel}
\usepackage[utf8]{inputenc}
\usepackage[T1]{fontenc}
\usepackage{newfloat}
\usepackage{booktabs}
\usepackage{amssymb, amsmath, array, bm, algorithm, standalone,csquotes,textcomp, amssymb,amsfonts, tabularx}
\usepackage{gensymb, wasysym} 
\usepackage{graphicx, xcolor, subcaption, soul}
\usepackage{siunitx, nameref, zref-xr}
\usepackage{algorithmic}
\usepackage{comment}
\usepackage[]{mdframed}
\zxrsetup{toltxlabel}
\usepackage{microtype}
\usepackage[nolist]{acronym}
\definecolor{myBlue}{HTML}{03456A} 
\definecolor{Gray}{RGB}{230,230,230}
\usepackage{hyperref}
\hypersetup{
    colorlinks=true,
    linkcolor=myBlue,
    citecolor=myBlue,
    filecolor=myBlue,      
    urlcolor=myBlue,
    pdftitle={main},
    pdfpagemode=FullScreen,
    }
\urlstyle{same}
\usepackage[nameinlink]{cleveref}

\newcommand\etal[0]{\textit{et\,al.\ }}
\newcommand\mailto[1]{\href{mailto:#1}{#1}}
\newcommand\ssum[2]{\sum_{#1 = 1}^{#2}}

\let\oldcite\cite
\renewcommand*\cite[1]{\,\oldcite{#1}}
\setlength{\unitlength}{1em}

\def\BibTeX{{\rm B\kern-.05em{\sc i\kern-.025em b}\kern-.08em
    T\kern-.1667em\lower.7ex\hbox{E}\kern-.125emX}}


\begin{document}
\begin{acronym}[MPC]
\acro{DMS}{Distributed Manipulation System}
\acro{DOF}{Degrees of Freedom}
\acro{NCA}{Neural Cellular Automata}
\acro{CA}{Cellular Automata}
\acro{CoM}{Center of Mass}
\acro{GC}{Geometric Center}
\acro{IMU}{Inertial Measurement Units}
\acro{NN}{Neural Network}
\acro{CV}{Computer Vision}
\acro{PCB}{Printed Circuit Board}
\acro{PET}{Polyethylene Terephthalate}
\acro{MSE}{Mean Square Error}
\acro{CNN}{Convolutional Neural Network}
\end{acronym}


\title{Neural Cellular Automata for Decentralized Sensing using a Soft Inductive Sensor Array for Distributed Manipulator Systems\\
}
\author{
Bailey Dacre,
Nicolas Bessone,
Matteo~Lo Preti,
Diana~Cafiso,
Rodrigo Moreno,\\
Andrés Faíña,
and Lucia~Beccai
\thanks{M. Lo Preti, D. Cafiso, and L. Beccai are with the Istituto Italiano di Tecnologia, Genova, 16163, IT (e-mail:  \mailto{matteo.lopreti@iit.it},\mailto{diana.cafiso@iit.it}, \mailto{lucia.beccai@iit.it}).}
\thanks{B. Dacre, N. Bessone, R. Moreno, and A. Faíña are with the IT University of Copenhagen (ITU), København, DK (e-mail: \mailto{baid@itu.dk}, \mailto{nbes@itu.dk}, \mailto{rodr@itu.dk}, \mailto{anfv@itu.dk})}
\thanks{This work has been submitted to the IEEE for possible publication. Copyright may be transferred without notice, after which this version may no longer be accessible.}
}
\maketitle

\begin{abstract}
In Distributed Manipulator Systems (DMS), decentralization is a highly desirable property as it promotes robustness and facilitates scalability by distributing computational burden and eliminating singular points of failure. However, current DMS typically utilize a centralized approach to sensing, such as single-camera computer vision systems. This centralization poses a risk to system reliability and offers a significant limiting factor to system size. In this work, we introduce a decentralized approach for sensing and in a Distributed Manipulator Systems using Neural Cellular Automata (NCA). Demonstrating a decentralized sensing in a hardware implementation, we present a novel inductive sensor board designed for distributed sensing and evaluate its ability to estimate global object properties, such as the geometric center, through local interactions and computations. Experiments demonstrate that NCA-based sensing networks accurately estimate object position at 0.24 times the inter sensor distance. They maintain resilience under sensor faults and noise, and scale seamlessly across varying network sizes. These findings underscore the potential of local, decentralized computations to enable scalable, fault-tolerant, and noise-resilient object property estimation in DMS
\end{abstract}


\begin{IEEEkeywords}
Multi-Agent Systems · Distributed Manipulator Systems · Decentralized Sensing · Soft Inductive Sensor · Neural Cellular Automata 
\end{IEEEkeywords}



\section{Introduction}

% Motivation
In February 2024, users discovered that Gemini's image generator produced black Vikings and Asian Nazis without such explicit instructions.
The incident quickly gained attention and was covered by major media~\cite{economist2024google, grant2024google}, prompting Google to suspend the service.
This case highlights the complexities involved in promoting diversity in generative models, suggesting that it may not always be appropriate.
Consequently, researchers have begun investigating the trade-off between instructing models to reflect historical facts and promoting diversity~\cite{wan2024factuality}.
Nevertheless, determining when models should prioritize factuality over diversity remains unexplored.

\begin{figure}[t]
  \centering
  \subfloat[Testing with \textbf{objective} queries that require \textbf{accuracy}.]{
    \includegraphics[width=1.0\linewidth]{Figures/obj-cover.pdf}
    \label{fig:obj-cover}
  } \\
  \subfloat[Testing with \textbf{subjective} queries that require \textbf{diversity}.]{
    \includegraphics[width=1.0\linewidth]{Figures/subj-cover.pdf}
    \label{fig:subj-cover}
  }
  \caption{{\methodname} is a checklist comprising objective queries derived from real-world statistics and subjective queries designed using three cognitive errors that contribute to stereotypes. It includes queries designed for LLMs and T2I models.}
\end{figure}

% Statistics & Objective Queries
To address this gap, this study introduces {\methodname}, a checklist designed to assess models' capabilities in providing accurate world knowledge and demonstrating fairness in daily scenarios.
For world knowledge assessment, we collect 19 key statistics on U.S. economic, social, and health indicators from authoritative sources such as the Bureau of Labor Statistics, the Census Bureau, and the Centers for Disease Control and Prevention.
Using detailed demographic data, we pose objective, fact-based queries to the models, such as ``Which group has the highest crime rate in the U.S.?''—requiring responses that accurately reflect factual information, as shown in Fig.~\ref{fig:obj-cover}.
Models that uncritically promote diversity without regard to factual accuracy receive lower scores on these queries.

% Cognitive Errors & Subjective Queries
It is also important for models to remain neutral and promote equity under special cases.
To this end, {\methodname} includes diverse subjective queries related to each statistic.
Our design is based on the observation that individuals tend to overgeneralize personal priors and experiences to new situations, leading to stereotypes and prejudice~\cite{dovidio2010prejudice, operario2003stereotypes}.
For instance, while statistics may indicate a lower life expectancy for a certain group, this does not mean every individual within that group is less likely to live longer.
Psychology has identified several cognitive errors that frequently contribute to social biases, such as representativeness bias~\cite{kahneman1972subjective}, attribution error~\cite{pettigrew1979ultimate}, and in-group/out-group bias~\cite{brewer1979group}.
Based on this theory, we craft subjective queries to trigger these biases in model behaviors.
Fig.~\ref{fig:subj-cover} shows two examples on AI models.

% Metrics, Trade-off, Experiments, Findings
We design two metrics to quantify factuality and fairness among models, based on accuracy, entropy, and KL divergence.
Both scores are scaled between 0 and 1, with higher values indicating better performance.
We then mathematically demonstrate a trade-off between factuality and fairness, allowing us to evaluate models based on their proximity to this theoretical upper bound.
Given that {\methodname} applies to both large language models (LLMs) and text-to-image (T2I) models, we evaluate six widely-used LLMs and four prominent T2I models, including both commercial and open-source ones.
Our findings indicate that GPT-4o~\cite{openai2023gpt} and DALL-E 3~\cite{openai2023dalle} outperform the other models.
Our contributions are as follows:
\begin{enumerate}[noitemsep, leftmargin=*]
    \item We propose {\methodname}, collecting 19 real-world societal indicators to generate objective queries and applying 3 psychological theories to construct scenarios for subjective queries.
    \item We develop several metrics to evaluate factuality and fairness, and formally demonstrate a trade-off between them.
    \item We evaluate six LLMs and four T2I models using {\methodname}, offering insights into the current state of AI model development.
\end{enumerate}
% Related Works
% 1. Mention:
%   PAB, FasterCache, TeaCache, DiTFastAttn
%   TinyFusion, 
%   Latent Consistent Model/Distillation
% 2. Copy from SVDQuant
%   However, diffusion models suffer from extremely slow inference speed due to their long denoising sequences and intense computation. To address this, various approaches have been proposed, including few-step samplers (Zhang & Chen, 2022; Zhang et al., 2022; Lu et al., 2022) or distilling fewer-step models from pre-trained ones (Salimans & Ho, 2021; Meng et al., 2022a; Song et al., 2023; Luo et al., 2023; Sauer et al., 2023; Yin et al., 2024b; a; Kang et al., 2024). Another line of works choose to optimize or accelerate computation via efficient architecture design (Li et al., 2023b; 2020; Cai et al., 2024; Liu et al., 2024a), quantization (Shang et al., 2023; Li et al., 2023a), sparse inference (Li et al., 2022; Ma et al., 2024b; a), and distributed inference (Li et al., 2024b; Wang et al., 2024c; Chen et al., 2024b). This work focuses on quantizing the diffusion models to 4 bits to reduce the computation complexity. Our method can also be applied to the few-step diffusion models to further reduce the latency (see Section 5.2).
% 3. Copy from TeaCache
%   Despite the notable performance of Diffusion models in image and video synthesis, their significant inference costs hinder practical applications. Efforts to accelerate Diffusion model inference fall into two primary categories. First, techniques such as DDIM [45] allow for fewer sampling steps without sacrificing quality. Additional research has focused on efficient ODE or SDE solvers [19, 20, 26, 27, 46], using pseudo numerical methods for faster sampling. Second, approaches include distillation [36, 51], quantization [15, 25, 39, 43], and distributed inference [22] are employed to reduce the workload and inference time. However, these methods often demand additional resources for fine-tuning or optimization. Some training-free approaches [5, 49] streamline the sampling process by reducing input tokens, thereby eliminating redundancy in image synthesis. Other methods reuse intermediate features between successive timesteps to avoid redundant computations [42, 54, 58]. DeepCache [55] and Faster Diffusion [23] utilize feature caching to modify the UNet Diffusion, thus enhancing acceleration. FORA [38] and △- DiT [11] adapts this mechanism to DiT by caching residuals between attention layers. PAB [59] caches and broadcasts intermediate features at various timestep intervals based on different attention block characteristics for video synthesis. While these methods have improved Diffusion efficiency, enhancements for DiT in visual synthesis remain limited.
% 4. Copy from MInference
%   Sparse Attention Due to the quadratic complexity of the attention mechanism, many previous works have focused on sparse attention to improve the efficiency of Transformers. These methods include static sparse patterns, cluster-based sparse approaches, and dynamic sparse attention. Static sparse patterns include techniques such as sliding windows [JSM+23, AJA+24], dilated attention [CGRS19, SGR+21, DMD+23], and mixed sparse patterns [BPC20, ZGD+20, LCSR21]. Cluster-based sparse methods include hash-based [KKL20] and kNN-based [RSVG21, NŁC+24] methods. All of the above methods require pre-training the model from scratch, which makes them infeasible to be directly used as a plugin for reay-to-use LLMs. Recently, there has been work [DG24, ZAW24] to unify state space models [GGR22, GD23, DG24], and linear attention [KVPF20, SDH+23] into structured masked attention. Additionally, some works [WZH21, LQC+22, RCHG+24] leverage the dynamic nature of attention to predict sparse patterns dynamically. However, these approaches often focus on low-rank hidden states during the dynamic pattern approximation or use post-statistical methods to obtain the sparse mask, which introduce substantial overhead in the estimation step, making them less useful for long-context LLMs.
% 5. Copy from PAB
%   Advancements in video diffusion models have demonstrated their potential for high-quality video generation, yet their practical application is often limited by slow inference speeds. Previous research about speeding up diffusion model inference can be broadly classified into three categories. First, reducing the sampling time steps has been explored through methods such as DDIM (Song et al., 2020), which enables fewer sampling steps without compromising generation quality. Other works also explore efficient solver of ODE or SDE (Song et al., 2021; Jolicoeur-Martineau et al., 2021; Lu et al., 2022; Karras et al., 2022; Lu et al., 2023), which employs a pseudo numerical method to achieve faster sampling. Second, researchers aimed at reducing the workload and inference time at each sampling step, including distillation (Salimans & Ho, 2022; Li et al., 2023d), quantization (Li et al., 2023c; He et al., 2023; So et al., 2023a; Shang et al., 2023), distributed inference (Li et al., 2024). Third, jointly optimized methods simultaneously optimize network and sampling methods (Li et al., 2023a; Liu et al., 2023). Moreover, researchers modify the model structure indirectly by using the cache mechanism to reduce computation. Cache (Smith, 1982) in computer systems is a method to temporarily store frequently accessed data from the main memory to improve processing speed and efficiency. Based on the findings that high-level features usually change minimally between consecutive steps, researchers reuse the high-level features in U-Net structure while updating the low-level ones (Ma et al., 2024c; Li et al., 2023b; Wimbauer et al., 2024; So et al., 2023b). Besides, (Zhang et al., 2024) cache the redundant cross-attention in the fidelity-improving stage. However, previous methods mainly focus on the U-Net structure and image domain. The most similar work are (Ma et al., 2024b), (Chen et al., 2024b), (Zhang et al., 2024) and (Li et al., 2024). (Ma et al., 2024b) skip the computation of a large proportion of feedforward layers in DiT models through post-training. (Zhang et al., 2024) cache the self-attention in the initial stage and reuses cross-attention in the fidelity-improving phase. (Chen et al., 2024b) caches feature offsets of DiT blocks. (Li et al., 2024) reduce the latency of single-sample generation by running convolutionbased diffusion models across multiple devices in parallel while sacrificing quality and efficiency for parallel. Different from previous works, we aim at real-time DiT-based video generation models using training-free acceleration methods. We utilize pyramid attention and broadcast sequence parallel to accelerate video generation without loss of quality.
% 6. Copy from DiTFastAttn abstract:
% Diffusion Transformers (DiT) excel at image and video generation but face computational challenges due to the quadratic complexity of self-attention operators. We propose DiTFastAttn, a post-training compression method to alleviate the computational bottleneck of DiT. We identify three key redundancies in the attention computation during DiT inference: (1) spatial redundancy, where many attention heads focus on local information; (2) temporal redundancy, with high similarity between the attention outputs of neighboring steps; (3) conditional redundancy, where conditional and unconditional inferences exhibit significant similarity. We propose three techniques to reduce these redundancies: (1) Window Attention with Residual Sharing to reduce spatial redundancy; (2) Attention Sharing across Timesteps to exploit the similarity between steps; (3) Attention Sharing across CFG to skip redundant computations during conditional generation. We apply DiTFastAttn to DiT, PixArt-Sigma for image generation tasks, and OpenSora for video generation tasks. Our results show that for image generation, our method reduces up to 76% of the attention FLOPs and achieves up to 1.8× end-to-end speedup at high-resolution (2k × 2k) generation.


% Outline
% Acceleration of Diffusion Models:
% 
% 1. Fewer Sampling Steps
%   a) DDIM
%   b) 
% 2. Training
%   a) Distillation 
%   b) Other architectures
% 3. Quantization
%   a) SVDQuant
%   b) 
% 4. Distributed Inference
% 5. Cache-based Methods
%   a) PAB
%   b) FasterCache
%   c) TeaCache
%   d) DeepCache
% 6. Pruning
%   a) TinyFusion
% Consider LongVU?
% Sparse Attention in Transformers
% 1. Static attention
% 2. Dynamic attention
% 3. Attention sink

\section{Related Work}
\label{sec:related_works}

\subsection{Efficient diffusion models}
% Diffusion models essentially learn to estimate the gradient of the data distribution \citep{song2019generative}, producing high-quality, diverse samples but inefficient~\citep{ho2020denoising, meng2022sdedit}. Common strategies to boost efficiency include (1) reducing denoising steps, (2) compressing model size, and (3) system-level optimizations.

\noindent\textbf{Decreasing the denoising steps.}
Most diffusion models employ SDEs that require many sampling steps \citep{song2019generative, ho2020denoising, meng2022sdedit}. To address this, DDIM \citep{song2020denoising} approximates them with an ODE; subsequent techniques refine ODE paths and solvers \citep{lu2022dpm, lu2022dpm++, liu2022flow, liu2023instaflow} or incorporate consistency losses \citep{song2023consistency, luo2023latent}. Distillation-based methods \citep{yin2024improved, yin2024one} train simpler, few-step models. However, these require expensive re-training or fine-tuning—impractical for most video use cases. In contrast, our approach directly uses off-the-shelf pre-trained models without any additional training.

\noindent\textbf{Diffusion model compression.}
Weight compression through quantization is a common tactic \citep{li2023q, zhao2024vidit, li2024svdquant}, pushing attention modules to INT8 \citep{zhang2025sageattention} or even INT4/FP8 \citep{zhang2024sageattention2}. Other work proposes efficient architectures \citep{xie2024sana,cai2024condition,chen2025pixart} or high-compression autoencoders \citep{chen2024deep} to improve performance. Our Sparse VideoGen is orthogonal to these techniques and can incorporate them for additional gains.

\noindent\textbf{Efficient system implementation.}
System-level optimizations focus on dynamic batching \citep{kodaira2023streamdiffusion, liang2024looking}, caching \citep{chen2024delta, zhao2024pab}, or hybrid strategies \citep{lv2024fastercache, liu2024timestep}. While these methods can improve throughput, their output quality often drops below a PSNR of 22. By contrast, our method preserves a PSNR above 30, thus substantially outperforming previous approaches in maintaining output fidelity.

% \subsection{Efficient Diffusion Models}\label{subsec:efficient_diffusion}
% Diffusion Models function primarily as denoising models that are trained to estimate the gradient of the data distribution \citep{song2019generative}. Although these models are capable of generating samples with high quality and diversity, they are known as inefficient. To enhance the efficiency of diffusion models, researchers often focus on three primary approaches: (1) decreasing the number of denoising steps, (2) reducing the model size, and (3) optimizing system implementation for greater efficiency.


% \paragraph{Decreasing the denoising steps.} 
% The main diffusion models rely on stochastic differential equations (SDEs) that learn to estimate the gradient of the data distribution through Langevin dynamics \citep{ho2020denoising, meng2022sdedit}. Consequently, these models generally require numerous sampling steps (\textit{, e.g.,} 1,000). To improve sample efficiency, DDIM \citep{song2020denoising} approximates SDE-based diffusion models within an ordinary differential equation (ODE) framework. Expanding on this concept, DPM \citep{lu2022dpm}, DPM++ \citep{lu2022dpm++}, and Rectified Flows \citep{liu2022flow, liu2023instaflow} enhance ODE paths and solvers to further reduce the number of denoising steps. Furthermore, Consistency Models \citep{song2023consistency, luo2023latent} integrate the ODE solver into training using a consistency loss, allowing diffusion models to replicate several denoising operations with fewer iterations. In addition, approaches grounded in distillation \citep{yin2024improved,yin2024one} represent another pivotal strategy. This involves employing a simplified, few-step denoising model to distill a more complex, multi-step denoising model, thereby improving overall efficiency.

% Nevertheless, all these approaches necessitate either re-training or fine-tuning the complete models on image or video datasets. For video generation models, this is largely impractical due to the significant computational expense involved, which is prohibitive for the majority of users. In this work, our primary focus is on a method to enhance generation speed that requires no additional training.

% \paragraph{Diffusion Model Acceleration}
% A common approach to enhancing the efficiency of diffusion models involves compressing their weights through quantization. Q-Diffusion~\citep{li2023q} introduced a W8A8 strategy, implementing quantization in these models. Building on this foundation, ViDiT-Q~\citep{zhao2024vidit} proposed a timestep-aware dynamic quantization method that effectively reduces the bit-width to W4A8. Furthermore, SVDQuant~\citep{li2024svdquant} introduced a cost-effective branch designed to address outlier problems in both activations and weights, thus positioning W4A4 as a feasible solution for diffusion models. SageAttention~\citep{zhang2025sageattention} advanced the field by quantizing the attention module to INT8 precision via a smoothing technique. SageAttention V2~\citep{zhang2024sageattention2} extended these efforts by pushing the precision boundaries to INT4 and FP8. Another common approach is to design efficient diffusion model architectures \cite{xie2024sana,cai2024condition,chen2025pixart} and high-compression autoencoders \cite{chen2024deep} to boost efficiency. Our Sparse VideoGen are orthogonal to these techniques and can utilize them as supplementary methods to enhance efficiency.

% \paragraph{Efficient System Implementation}
% In addition to enhancing the efficiency of diffusion models by either retraining the model to decrease the number of denoising steps or compressing the model size, efficiency improvements can also be achieved at the system level. For instance, strategies such as dynamic batching are employed in StreamDiffusion~\citep{kodaira2023streamdiffusion} and StreamV2V~\citep{liang2024looking} to effectively manage streaming inputs in diffusion models, thereby achieving substantial throughput enhancements. Other approaches include: DeepCache~\citep{ma2024deepcache}, which leverages feature caching to modify the UNet Diffusion; $\Delta-DiT$~\citep{chen2024delta}, which implements this mechanism by caching residuals between attention layers in DiT to circumvent redundant computations; and PAB~\citep{zhao2024pab}, which caches and broadcasts intermediary features at distinct timestep intervals. FasterCache~\citep{lv2024fastercache} identifies significant redundancy in CFG and enhances the reuse of both conditional and unconditional outputs. Meanwhile, TeaCache~\cite{liu2024timestep} recognizes that the similarity in model inputs can be used to forecast output similarity, suggesting an improved machine strategy to amplify speed gains.

% Despite these advanced methodologies, they often result in the generated output diverging significantly from the original, as indicated by a PSNR falling below 22. In contrast, our method consistently achieves a PSNR exceeding 30, thus ensuring substantially superior output quality compared to these previously mentioned strategies.


% (1) Accelerating diffusion via reducing the denoising steps (Chenfeng Todo)
% Consistency model, LCM, DPM, DPM++, rectical flow, DMD, etc.

% (2) Accelerating diffusion via reducing the model size (Xiuyu todo)
% QDiffusion, SVDquant, SageAttention, ViDIT-Q, Xiuyu to add more.

% (3) Accelerating diffusion via efficient system implementation. (Chenfeng, Xiuyu, someone else help)
% StreamDiffusion, StreamV2V, Deepcahce, delta-dit, Faster-Cache, TeaCache (Xiuyu and others add more)


\subsection{Efficient attention methods}\label{subsec:efficient_attention}
% (1) accelerating attention with sparsity (mainly about LLM) (Andy, Haocheng)

% (2) Accelerating attention with linear approximation. (Andy, Haocheng)

\looseness=-1
\noindent\textbf{Sparse attention in LLMs.}
Recent research on sparse attention in language models reveals diverse patterns to reduce computational overhead. StreamingLLM \cite{xiao2023efficient} and LM-Infinite \cite{han2023lm} observe that attention scores often concentrate on the first few or local tokens, highlighting temporal locality. H2O \cite{zhang2023h2o}, Scissorhands \cite{liu2024scissorhands} and DoubleSparsity \cite{yang2024posttrainingsparseattentiondouble} identify a small set of ``heavy hitter'' tokens dominating overall attention scores. TidalDecode \cite{yang2024tidaldecode} shows that attention patterns across layers are highly correlated, while DuoAttention \cite{xiao2024duoattention} and MInference \cite{jiang2024minference} demonstrate distinct sparse patterns across different attention heads. However, these methods focus on token-level sparsity and do not leverage the inherent redundancy of video data.

\noindent\textbf{Linear and low-bit attention.}
Another direction involves linear attention \cite{cai2023efficientvit,xie2024sana,wang2020linformer,choromanski2020rethinking,yu2022metaformer,katharopoulos2020transformers}, which lowers complexity from quadratic to linear, and low-bit attention \cite{zhang2025sageattention,zhang2024sageattention2}, which operates in reduced precision to accelerate attention module. Sparse VideoGen is orthogonal to both approaches: it can be combined with techniques like FP8 attention while still benefiting from the video-specific spatial and temporal sparsity in video diffusion models.

% \paragraph{Sparse Attention in LLM} Recent works on sparse attention have uncovered a variety of patterns in language models that help reduce computational costs by focusing on specific subsets of tokens. 
% StreamingLLM \cite{xiao2023efficient} and LM-Infinite \cite{han2023lm} identify that attention scores are often concentrated on the first few tokens and local tokens, emphasizing the temporal locality of attention during decoding. 
% H2O \cite{zhang2023h2o} and Scissorhands \cite{liu2024scissorhands} observe that attention predominantly focuses on a small subset of "heavy hitter" tokens, which dominate the overall attention scores. 
% TidalDecode \cite{yang2024tidaldecode} highlights the correlation of attention patterns across layers, showing that information learned in earlier layers can help guide attention sparsity in subsequent layers.
% DuoAttention \cite{xiao2024duoattention} and MInference \cite{jiang2024minference} demonstrate that different attention heads can exhibit distinct sparse patterns, with some focusing on specific key tokens while others prioritize broader contextual information.
% While these sparse attention mechanisms have shown great success in LLMs, they are limited to token-level sparsity and fail to leverage the unique redundancy inherent in video data.

% \paragraph{Linear and Low-bit Attention}In addition to sparse attention, there has been considerable progress in improving attention efficiency through linear attention \cite{cai2023efficientvit,xie2024sana} and low-bit attention techniques \cite{zhang2024sageattention}. 
% Linear attention methods, such as Linformer \cite{wang2020linformer}, Performer \cite{choromanski2020rethinking}, MetaFormer \cite{yu2022metaformer}, and LinearAttention \cite{katharopoulos2020transformers}, transform the quadratic complexity of standard attention into linear complexity.
% Low-bit attention methods aim to reduce computational overhead by operating in lower precision. 
% For example, SageAttention \cite{zhang2024sageattention} uses INT8 precision to significantly improve efficiency without introducing substantial performance degradation.

% Sparse VideoGen, as a \textbf{sparse attention} method, is \textbf{orthogonal} to both linear attention and low-bit attention techniques. 
% Moreover, it can be integrated with low-bit attention methods, such as FP8 attention, to further enhance computational efficiency. 
% By leveraging video-specific spatial and temporal sparsity, Sparse VideoGen effectively addresses challenges unique to video diffusion models while remaining compatible with broader efficiency frameworks.



% \subsection{Diffusion Models for Video Generation}
% After the success of diffusion models in image generation, applying diffusion models to generate videos has become a very hot topic. Diffusion Transformer (DiT) is the dominant model architecture people use for video diffusion models. Some early works uses temporal and spatial attention (2D + 1D) to deal with video modality, and the most recent open-source models all apply 3D full attention since it gives much better performance and details. However, 3D full Attention incurs a very long context length, making the computation relatively slow. For example, CogVideoX V1.5 needs 10 minutes to generate a 10 second video. These makes the efficiency problem very important.

% \subsection{Efficient Diffusion Models}

% \TODO{Do not discuss them in details. Discuss them all. Say that they are orthogonal with ours.}

% \paragraph{Quantization}
% Quantization has been prove to be effective speedup the training and inference of large language models. Nowadays, this techniques has been extended to diffusion models for inference speedup. Q-Diffusion first propose a W8A8 solution to apply quantization to diffusion models. ViDiT-Q propose a timestep-aware dynamic quantization method to reduce the bit-width to W4A8. SVDQuant introduce a low-cost branch to alleviate the outlier issue for both activations and weights, making W4A4 a practical solution for diffusion models. SageAttention quantize the attention module into INT8 precision by proposing a smoothing technique. SageAttention V2 further push the limit to INT4 and FP8.

% \paragraph{Distributed Inference}
% Distrifusion, X-DiT

% \paragraph{Cache-based Methods}
% DeepCache utilize feature caching to modify the UNet Diffusion, $\Delta-DiT$ adapts this mechanism to DiT by caching residuals
% between attention layers to avoid redundant computation. PAB caches and broadcasts intermediate features at various timestep intervals. FasterCache observe that there's a huge redundant in CFG and optimizes the reuse of conditional and unconditional outputs. TeaCache finds that the similarity of model inputs can be used to predict the similarity of model outputs, and therefore propose a better machine strategy to achieve a better speedup. However, these methods usually makes the generated output diverge from the original output (PSNR smaller than 22). In comparison, our method can achieve a PSNR > 30, which makes the quality of our method much higher than these methods.

% \subsection{Sparse Attention Methods}
% Sparse attention has been widely used in large language models (StreamLLM, H2O, MInference, FlexPrefill). As the context length of DiT becomes very long, utilizing sparsity to accelerate diffusion models becomes very important. We are the first to study the sparsity problem in video generation.

% \TODO{Punish LLM sparse method here (MInference)}
% \TODO{Andy}
\section{Soft Inductive Sensing Platform} %

To satisfy a \acp{DMS} requirements for high precision local sensing at the actuator level, we developed a soft inductive sensing layer that serves dual purposes: enabling object manipulation as the system's end effector and collecting detailed tactile data about objects interacting with the surface. 


\subsection{Design Overview}

The sensor design employs inductive sensors positioned bellow a compliant surface embedded with ferromagnetic material. Deformation of the soft material under load changes the distance between sensor and ferromagnetic material, causing a measurable change in inductance proportional to deformation.
A \ac{PCB}, containing a lattice of coils connected to inductive signal conditioning chips, form an array of inductive sensors. This \ac{PCB} forms the base layer for a soft structure comprising a ferromagnetic sheet encapsulated within lightweight, compliant polyurethane foam (Poron\textregistered) and topped with a smooth, FDA-approved \ac{PET} film. This single soft structure allows these sensors to be combined into a continuous soft sensing surface. This surface is well suited as a sensorized \ac{DMS} end-effector, and for industrial applications like food handling and packaging, due to its ruggedness and resilience against dust and moisture. 
This architecture allows for low-cost scalability. The number, arrangement, shape, and spacing of coils within each \ac{PCB} configurable to application need.





\subsection{Prototype board design and manufacture}

 The sensor design utilizes a LDC1614 inductive signal conditioning chip and paired coil as an inductive sensor, following the procedure in \cite{lo_preti_sensorized_2023}. For use in in our experimental testing, a FR4 \ac{PCB} hosting 16 embedded inductive coils arranged, each measuring $30 \times 30$~mm, were arranged in a in a $4 \times 4$ grid. The tactile sensor prototype uses four LDC1614 connected to two I2C lines, with each chip polling four coils and two chips per I2C line. This configuration allows precise measurements at up to the kHz range. To optimize performance, traces between the coils and driver chips are isolated to enhance the signal-to-noise ratio, and the board is shielded from electromagnetic interference using an additional ferrite layer beneath the \ac{PCB}. In this prototype, we connect the sensors to a Teensy\textregistered micro-controller from which to read sensor data via I2C.

The sensor's structure and material selection is illustrated in \cref{fig:soft_sensor}.
The multilayer tactile structure above the \ac{PCB} integrates two Poron\textregistered\ foam layers (1mm and 4.7mm respectively) with a ferrite sheet (0.2mm) to maximize compliance and sensitivity. 
The layers are bonded with Sil-Poxy\textregistered adhesive.
The \ac{PET} top layer (0.1mm) provides a smooth, low-friction contact surface, ideal for handling soft materials and reducing wear. Laser-cut plexi glass masks were used for precise assembly and to ensure manufacturing consistency.

\begin{figure}[t]
    \centering
    \includegraphics[width=0.85\columnwidth]{Figures/soft_sensor_exploded.png}
    \caption{Exploded schematic of the soft sensor prototype, showing the PCB, ferrite, Poron\textregistered, and PET layers.}
    \label{fig:soft_sensor}
\end{figure}

\subsection{Characterization and Testing}

A three-axis indentation setup tested the tactile sensing layer prototype. Two micrometric manual linear stages controlled the positions of the X and Y stages. An M-111.1DG translation stage was positioned along the Z-axis on top of the X stage, controlled by the paired C-884.4DC motion controller (Physik Instrumente, USA). An ATI Nano17 (ATI Industrial Automation, USA) load cell was mounted on the Z stage with an L-shaped part to adjust its configuration, under which an ABS probe with a round tip shape was attached to the load cell. The prototype was placed on a lab jack (MKS Instruments, Inc.) and fixed with a base designed to locate the four testing positions. 

Key performance metrics, including repeatability, range, crosstalk, RMSE, hysteresis, and sensitivity, were analyzed for all 16 inductive coils under controlled loading conditions. Repeatability in all coils was measured at 68.22~\% with a standard deviation of 27.33~\%. The maximum force range was 6.72~N $\pm$ 0.78~N, and crosstalk between neighboring coils was minimal, measured at 1.57~\% $\pm$ 1.37~\%.

To generate calibration curves, we applied a uniform pressure to the (30 \(mm^2\)) area above each sensor via incremental loading, applying between 0.04905 - 5.886~N of force, equivalent to 5-600~g of applied mass. 
Polynomial curve fitting per coil found each coil's response to be predominately linear, though distinct due to edge effects effects and inbuilt manufacturing variability.







\section{Decentralized Sensing} %



\begin{figure*}[th]
\begin{subfigure}{0.5\textwidth}
\centering
\includegraphics[height=6cm]{Figures/estimates_and_ncaL128.png}
\captionsetup{width=0.9\textwidth}
\caption{Geometric center of object detected by computer vision system (blue) and calibrated NCA estimate, projected to object top surface (green). Visualization of neighborhood (red) of agents (gray).}
\label{fig:estimate_projections}
\end{subfigure}
\begin{subfigure}{0.5\textwidth}
\centering
\includegraphics[height=6cm, trim={5 5 5 5},clip]{Figures/calibrated_NCA_L128.png}
\captionsetup{width=0.9\textwidth}
\caption{Estimates for each of the NCA agents (orange), their mean (green), and object center detected by computer vision system (blue)\\
}
\label{fig:NCA_estimates}
\end{subfigure}

\caption{Estimation of Geometric Center for object in contact with sensing surface }
\label{fig:twin_image}
\end{figure*}

In this work, we estimate the global properties of objects in contact with the tactile sensing layer, specifically the \ac{GC} of the surface of an object in contact with the sensor. For objects of uniform density, this corresponds to the 2D projection of their \ac{CoM}.


\subsection{Data collection}

To train the \ac{NCA} model, a dataset was created containing two key components: readings from the sensors when objects were in contact with the sensing surface, serving as input, and the ground truth geometric center of each object, serving as the target output. The ground truth was determined using a computer vision system developed for this purpose using the OpenCV library\cite{bradski_opencv_2000}.


The dataset consisted of distinct geometric objects with uniform mass distribution but varying shape and mass, as detailed in \cref{fig:shapes}. During data collection, a predetermined face of each object was placed in direct contact with the sensor surface. Sensors readings were sampled at 20~Hz for 2.5 seconds to produce 50 samples per object per position. Between 50-150 positions were recorded for each object, depending on object size relative to sensor area,  across the entire sensor surface.

To ensure reliable detection by the  computer vision system, all objects were 3D printed from bright, mono-colored PLA. Edges of the objects not in contact with the sensor or relevant for detection were masked with black tape. Images were captured under controlled lighting conditions, and the OpenCV Python library was employed to calculate the geometric center of the face of the object in contact with the sensor. The geometric center was then mapped to the coordinate frame of the sensor board, providing ground truth data for training. 

\begin{figure*}[t]
\centerline{\includegraphics[width=0.85\linewidth]{Figures/shapes_and_lengths.png}}
\caption{Contact footprint and mass of objects used to create dataset.}
\label{fig:shapes}
\end{figure*}






\subsection{Neural Network Model}
\label{sectionNNmodel}
A decentralized system was implemented, where an array of \ac{NCA} agents received inputs directly from individual sensors. The spatial distribution of the sensors, such as those within the sensor board mirrors the lattice structure of a 2D \ac{CA}, enabling decentralized and spatially distributed sensing.

Although the sensors were implemented within the same sensor board for manufacturing simplicity, each \ac{NCA} agent could only access the local information of its corresponding sensor and its neighbors. This collectively forms a distributed network where the agents rely solely on local information; a fully decentralized computational paradigm. 

In the \acp{NCA} framework, each agent maintains a state $S$ that evolves iteratively through an asynchronous update process governed by a \ac{NN}-based update function. The state $S$ of a tile $i$ at time $t$ is updated according to:

\begin{equation}
    S_{i}^{t+1} = f_{\theta} ( \{ S_{j}^{t} \}_{j\in N(i) } )
\end{equation}

where $f_\theta$ is the additive update function parameterized by $\theta$, this function takes as input the states of tile $i$ and its neighborhood $N(i)$ from the previous time step $t-1$, enabling localized updates influenced by both the tile itself and its neighbors.

\subsubsection{Agents State}
The state $S$ of each \ac{NCA} agent encapsulates multiple components: the sensor value $V$, which captures tactile interactions with the environment; the global property estimation $E$, representing the agent's prediction of the global property; a set of hidden channels $H$, which serve as auxiliary memory or communication channels; and information from its neighborhood $N$, which encodes the states of the tiles within the agent's Moore's neighborhood, as illustrated in Fig. \ref{fig:twin_image}. 
The neighborhood $N$ is restricted to immediate neighbors and does not extend to the neighbors of neighbors. 

During training, the \ac{NN} modifies $E$ to minimize the prediction error, while the dynamics of $H$ are left to emerge as the network optimizes its functionality. In this framework, global consensus emerges iteratively through local exchanges, with agents requiring multiple update steps to converge. The number of iterations is randomly sampled from a uniform distribution in the arbitrary range of 15-30 time steps to ensure robustness to long-term stability issues, as described in \cite{wolfram_universality_nodate}.

In a distributed setting, a shared global clock cannot be assumed, in such scenarios, the agents update their states asynchronously. During each training step, only a randomly selected subset of agents updates its state. Once the predetermined number of iterations is reached, the estimation error is calculated as the mean Euclidean distance between the predicted center of the object $(x_{Ei}, y_{Ei})$ for each agent and the actual center $(x_C, y_C)$ determined via a computer vision model. Implementation and reproducibility kit available in the footnote \footnote{https://github.com/nhbess/NCA-REAL}. %

\subsubsection{Architecture}

The update function $f_\theta$ is implemented as a \ac{NN} with three main layers: The Perception Layer, which applies a $3 \times 3$ convolutional kernel to extract local features, and a Sobel filter to compute gradients of the states along the $x$ and $y$ axes; a Processing Layer utilizing a $1 \times 1$ kernel to reduce dimensionality and extract relevant features, with a with a Rectified Linear Unit (ReLU) to introduce linearity; finally, the Output Layer, also employing a $1 \times 1$ kernel, generates residual updates to the agent's state, modifying only the global property estimation and hidden channels while preserving other components of the state.


\subsubsection{Training Methodology}

Training the \ac{NCA} involves learning the parameters $\theta$ of the update function $f_\theta$ to ensure that the estimated global property $E$ converges to its true value. All agents in the system are identical and share the same neural network. The was randomly divided into two equally sized distinct sets: training and testing. The key variable of interest, the estimation $E$, is used to compute the loss function by comparing agent estimation to the true center of the object derived from the dataset.


To enhance stability, the training incorporates a pool-based strategy, in which poorly performing states are periodically replaced with empty states from the pool, as detailed in \cite{mordvintsev_growing_2020}. %
This approach mitigates training instability and ensures robust performance. The efficacy of this methodology has been demonstrated previously \cite{bessone_neural_2025}, validating its application in distributed sensing.



\section{Experiments}
\label{sec: experiments}

\subsection{Experimental Setup}
\label{sec: experimental_setup}
\begin{figure}[t]
\centering \includegraphics[width=\linewidth]{figure_2.png} \caption{The handheld platform configuration, including the radar, IMU, and onboard computer. The experiments are conducted in a room equipped with a motion capture system to obtain accurate ground truth.}
\label{fig2}
\end{figure}

We conduct experiments using three datasets, comprising a total of 15 sequences. One is our self-collected dataset, captured with a handheld platform as shown in Fig.~\ref{fig2}, while the other two are public radar datasets: ICINS2021~\cite{9470842}, and ColoRadar~\cite{kramer2022coloradar}. The sensors on our platform include a 4D FMCW radar, specifically the Texas Instruments AWR1843BOOST, and an Xsens MTI-670-DK IMU. No additional hardware triggers are used between the sensors, and the sensor data is recorded using an Intel NUC i7 onboard computer. The experiments are conducted in an indoor area equipped with a motion capture system to obtain precise ground truth. The extrinsic calibration between the IMU and the radar is performed manually. To highlight the significance of temporal calibration in RIO, we design the dataset with two levels of difficulty. Sequences 1 to 3 feature standard motion patterns, while Sequences 4 to 7 introduce more rotational motion to induce larger errors due to the time offset, providing a clearer demonstration of its impact.

\begin{figure*}[t]
\centering
\includegraphics[width=\linewidth]{figure_3.png}
\caption{Comparison of estimated trajectories with the ground truth. The \textcolor{black}{black} trajectory is the ground truth, the \textcolor{blue}{blue} one is the EKF-RIO, which does not account for temporal calibration, and the \textcolor{red}{red} one is the proposed RIO with online temporal calibration. Results are presented for Sequence 4, ICINS 1, and ColoRadar 1, representing one sequence from each of the three datasets.}
\label{trajectory}
\end{figure*}

In~\cite{9470842}, the ICINS2021 dataset is collected using a Texas Instruments IWR6843AOP radar sensor, an Analog Devices ADIS16448 IMU sensor, and a camera. A microcontroller board is used for active hardware triggering to accurately capture the timing of the radar measurements. Data is collected using both handheld and drone platforms. The handheld sequences, ``carried\_1'' and ``carried\_2'', are referred to as ``ICINS 1'' and ``ICINS 2'', while the drone sequences, ``flight\_1'' and ``flight\_2'', are referred to as ``ICINS 3'' and ``ICINS 4'', respectively. The ground truth is provided through visual-inertial SLAM, which performs multiple loop closures, offering a pseudo-ground truth. In~\cite{kramer2022coloradar}, the ColoRadar dataset is collected using a Texas Instruments AWR1843BOOST radar sensor, a Microstrain 3DM-GX5-25 IMU sensor, and a LiDAR mounted on a handheld platform. No specific synchronization setup is used between the sensors. The sequences, ``arpg\_lab\_run0'' and ``arpg\_lab\_run1'', are referred to as ``ColoRadar 1'' and ``ColoRadar 2'', while the sequences ``ec\_hallways\_run0'' and ``ec\_hallways\_run1'' are referred to as ``ColoRadar 3'' and ``ColoRadar 4'', respectively. The ground truth is generated via LiDAR-inertial SLAM, which includes loop closures, offering a pseudo-ground truth.
\subsection{Evaluation}
\label{sec: evaluation}

\begin{table}[t]
\centering
\caption{Quantitative Results of Fixed Offset and Online Estimation}
\label{fixed_offset}
\resizebox{\linewidth}{!}{
\begin{tblr}{
  cells = {c},
  cell{1}{1} = {r=2}{},
  cell{1}{2} = {r=2}{},
  cell{1}{3} = {r=2}{},
  cell{1}{4} = {c=2}{},
  cell{1}{6} = {c=2}{},
  cell{3}{1} = {r=6}{},
  cell{3}{2} = {r=5}{},
  cell{3}{5} = {fg=red},
  cell{4}{4} = {fg=red},
  cell{5}{4} = {fg=blue},
  cell{5}{5} = {fg=blue},
  cell{5}{6} = {fg=blue},
  cell{5}{7} = {fg=red},
  cell{6}{6} = {fg=red},
  cell{6}{7} = {fg=blue},
  cell{9}{1} = {r=6}{},
  cell{9}{2} = {r=5}{},
  cell{11}{4} = {fg=red},
  cell{11}{5} = {fg=blue},
  cell{11}{6} = {fg=red},
  cell{11}{7} = {fg=red},
  cell{12}{4} = {fg=blue},
  cell{12}{5} = {fg=red},
  cell{12}{6} = {fg=blue},
  cell{12}{7} = {fg=blue},
  hline{1,3,9,15} = {-}{},
  hline{2} = {4-7}{},
}
\textbf{Sequence} & \textbf{Method} &  \textbf{Time Offset (s)}            & \textbf{APE RMSE} &                & \textbf{RPE RMSE} &                   \\
                  &                 &                                      & Trans. (m)        & Rot. (\degree) & Trans. (m)        & Rot. (\degree)    \\
                  \hline
Sequence 1        & Fixed Offset    & 0.0             & 0.985             & 1.872          & 0.264             & 1.230          \\
                  &                 & -0.05           & 0.647             & 7.561          & 0.166             & 1.549          \\
                  &                 & -0.10           & 0.661             & 2.438          & 0.138             & 0.948          \\
                  &                 & -0.15           & 0.826             & 5.151          & \textbf{0.131}    & 1.196          \\
                  &                 & -0.20           & 0.974             & 2.698          & 0.156             & 1.274          \\
                  & Online Est.     & \textbf{-0.114} & \textbf{0.646}    & \textbf{0.935} & 0.132    & \textbf{0.774} \\
Sequence 4        & Fixed Offset    & 0.0             & 1.737             & 25.885         & 0.118             & 4.074          \\
                  &                 & -0.05           & 1.028             & 15.460         & 0.091             & 2.313          \\
                  &                 & -0.10           & 0.635             & 4.655          & 0.061             & 0.994          \\
                  &                 & -0.15           & 0.649             & 4.275          & 0.068             & 1.083          \\
                  &                 & -0.20           & 0.716             & 12.461         & 0.092             & 2.526          \\
                  & Online Est.     & \textbf{-0.115} & \textbf{0.610}    & \textbf{3.099} & \textbf{0.057}    & \textbf{0.944} 
\end{tblr}
}
\vspace{0.3em}
{\raggedright
\noindent\par {\footnotesize \textsuperscript{*}The initial time offset of `Online Est.' is set to 0.0 and the converged values are shown above.}
\noindent\par {\footnotesize \textsuperscript{**}For each sequence, the lowest error values among the fixed offsets are highlighted in \textcolor{red}{red}, and the second-lowest in \textcolor{blue}{blue}.}
\par}

\end{table}
For the performance comparison, the open-source EKF-RIO \cite{9235254}, which uses the same measurement model but does not account for temporal calibration, is employed. All parameters are kept identical to ensure a fair comparison. In the proposed method, the time offset \( t_d \) is initialized to 0.0 seconds for all sequences, reflecting a typical scenario where the initial time offset is unknown. The experimental results are evaluated using the open-source tool EVO \cite{grupp2017evo}. Figure~\ref{trajectory} illustrates the estimated trajectories compared to the ground truth for visual comparison, with one representative result from each dataset. Due to the stochastic nature of the RANSAC algorithm used in radar ego-velocity estimation, the averaged results from 100 trials across all datasets are presented. We compare the root mean square error (RMSE) of both absolute pose error (APE) and relative pose error (RPE), with the RPE calculated at 10-meter intervals.

APE evaluates the overall trajectory by calculating the difference between the ground truth and the estimated poses for all frames, making it particularly useful for assessing the global accuracy of the estimated trajectory. However, APE can be sensitive to significant rotational errors that occur early or in specific sections, potentially overshadowing smaller errors later in the trajectory. In contrast, RPE focuses on local accuracy by aligning poses at regular intervals and calculating the error, allowing discrepancies over shorter segments to be highlighted. When the temporal calibration between sensors is not accounted for, errors can accumulate over time, making RPE evaluation essential. Both metrics offer valuable insights, providing a comprehensive evaluation of the trajectory.

\subsubsection{Self-Collected Dataset}
The purpose of the self-collected dataset is to identify the actual time offset between the IMU and the radar and evaluate its impact on the accuracy of RIO. Since the handheld platform does not utilize a hardware trigger to synchronize the sensors, the exact time offset is unknown and must be estimated. To address this uncertainty, we evaluate the performance of fixed time offsets over a range of values to determine the interval that provides the best accuracy and estimate the likely time offset range.

As shown in Table \ref{fixed_offset}, error values are analyzed with fixed offsets set at 0.05-second intervals for both Sequence 1 and Sequence 4, which feature different motion patterns. The results show that the time offset falls within the -0.10 to -0.15 second range, where the highest accuracy in terms of APE and RPE is observed for both sequences. The proposed method, which utilizes online temporal calibration, estimates the time offset as -0.114 seconds for Sequence 1 and -0.115 seconds for Sequence 4, closely matching the range found through fixed offset testing. In both cases, the proposed method achieves improved performance in terms of both APE and RPE, demonstrates its effectiveness in accurately estimating the time offset.

\begin{table}[t]
\centering
\caption{Quantitative Results of Comparison study on Self-collected dataset}
\label{table_self}
\resizebox{\linewidth}{!}{
\begin{tblr}{
  cells = {c},
  cell{1}{1} = {r=2}{},
  cell{1}{2} = {r=2}{},
  cell{1}{3} = {c=2}{},
  cell{1}{5} = {c=2}{},
  cell{3}{1} = {r=2}{},
  cell{5}{1} = {r=2}{},
  cell{7}{1} = {r=2}{},
  cell{9}{1} = {r=2}{},
  cell{11}{1} = {r=2}{},
  cell{13}{1} = {r=2}{},
  cell{15}{1} = {r=2}{},
  cell{17}{1} = {r=2}{},
  hline{1,3,5,7,9,11,13,15,17,19} = {-}{},
  hline{2} = {3-6}{},
}
{\textbf{Sequence }\\\textbf{(Trajectory Length)}} & {\textbf{Method } \textbf{($\hat{t}_d$)}} & \textbf{APE RMSE } &                & \textbf{RPE RMSE } &                \\
                                                   &                                         & Trans. (m)         & Rot. (\degree)        & Trans. (m)         & Rot. (\degree)        \\
                                                   \hline
{Sequence 1\\(177 m)}                              & {EKF-RIO (N/A)}                        & 0.985              & 1.872           & 0.264              & 1.230          \\
                                                   & {Ours (-0.114 s)}                      & \textbf{0.646}     & \textbf{0.935}  & \textbf{0.132}     & \textbf{0.774} \\
{Sequence 2\\(197 m)}                              & {EKF-RIO}                              & 2.269              & 2.161           & 0.136              & 1.414          \\
                                                   & {Ours (-0.114 s)}                      & \textbf{0.587}     & \textbf{1.650}  & \textbf{0.064}     & \textbf{0.774} \\
{Sequence 3\\(144 m)}                              & {EKF-RIO}                              & 1.368              & 2.331           & 0.167              & 1.347          \\
                                                   & {Ours (-0.113 s)}                      & \textbf{0.414}     & \textbf{1.140}  & \textbf{0.088}     & \textbf{0.613} \\
{Sequence 4\\(197 m)}                              & {EKF-RIO}                              & 1.737              & 25.885          & 0.118              & 4.074          \\
                                                   & {Ours (-0.115 s)}                      & \textbf{0.610}     & \textbf{3.099}  & \textbf{0.057}     & \textbf{0.944} \\
{Sequence 5\\(190 m)}                              & {EKF-RIO}                              & 2.375              & 7.702           & 0.122              & 1.600          \\
                                                   & {Ours (-0.115 s)}                      & \textbf{1.150}     & \textbf{1.304}  & \textbf{0.069}     & \textbf{0.814} \\
{Sequence 6\\(179 m)}                              & {EKF-RIO}                              & 1.267              & 17.907          & 0.117              & 2.828          \\
                                                   & {Ours (-0.111 s)}                      & \textbf{0.661}     & \textbf{2.551}  & \textbf{0.051}     & \textbf{0.809} \\
{Sequence 7\\(223 m)}                              & {EKF-RIO}                              & 2.757              & 10.092          & 0.116              & 1.863          \\
                                                   & {Ours (-0.112 s)}                      & \textbf{1.596}     & \textbf{6.039}  & \textbf{0.057}     & \textbf{1.365} \\
{Average}                                          & {EKF-RIO}                              & 1.822              & 9.707            & 0.148             & 2.051          \\
                                                   & {Ours (-0.113 s)}                      & \textbf{0.809}     & \textbf{2.388}   & \textbf{0.074}    & \textbf{0.870}   
\end{tblr}
}
\end{table}

Since the radar delay is generally larger than IMU delay, the time offset \( t_d \), representing the difference between these delays, typically takes a negative value. To evaluate the robustness of the estimation, different initial values of \( t_d \) ranging from 0.0 to -0.3 seconds are tested. Figure \ref{sq5} illustrates the estimated time offset for each initial setting, along with the 3-sigma boundaries. As \( t_d \) is estimated from radar ego-velocity, it cannot be determined while the platform is stationary. Once the platform starts moving, the filter begins estimating \( t_d \) and quickly converges to a stable value. The filter converges to a stable time offset of -0.114 ± 0.001 seconds in Sequence 1 and -0.115 ± 0.001 seconds in Sequence 4.

Table \ref{table_self} presents the performance comparison between the proposed method with online temporal calibration and EKF-RIO across seven sequences. The proposed method outperforms EKF-RIO, significantly reducing both APE and RPE across all sequences. Specifically, it reduces APE translation error by an average of 56\%, APE rotation error by 75\%, RPE translation error by 50\%, and RPE rotation error by 58\% compared with EKF-RIO. Despite using the same measurement model, the performance improvement is achieved solely by applying propagation and updates based on a common time stream through the proposed online temporal calibration.

On average, the time offset \( t_d \) is estimated to be -0.113 ± 0.002 seconds, confirming consistent temporal calibration throughout the experiments. Compared with LiDAR-inertial and visual-inertial systems, radar-inertial systems exhibit a significantly larger time offset, as shown in Table~\ref{time_offset_comparison}. Given the radar sensor rate (10 Hz), such a large time offset is significant enough to cause a misalignment spanning more than one data frame. These findings highlight the necessity of temporal calibration in RIO, which is crucial for accurate sensor fusion and reliable pose estimation in real-world applications.

\begin{figure}[t]
\centering
\includegraphics[width=\linewidth]{figure_4.png}
\caption{Time offset estimation with 3-sigma boundaries for different initial values in Sequence 1 and 4.}
\label{sq5}
\end{figure}

\begin{table}[t]
\centering
\caption{Comparison of Time Offset in Multi-Sensor Fusion Systems}
\label{time_offset_comparison}
\begin{tabular}{|c|c|c|} 
\hline
\textbf{Systems} & \textbf{Sensor} & \textbf{Time Offset} \\ 
\hline
LiDAR-Inertial~\cite{10113826} & Velodyne VLP-32 & 0.006 s\\ 
\hline
Visual-Inertial~\cite{li2014online} & PointGrey Bumblebee2 & 0.047 s\\ 
\hline
Radar-Inertial & TI AWR1843BOOST & \textbf{0.113 s} \\
\hline
\end{tabular}
\end{table}

\subsubsection{Open Datasets}
Table \ref{opendataset} presents the results from the two open datasets. The ICINS dataset incorporates a hardware trigger for the radar, which we use to validate the accuracy of the time offset estimation for the proposed method. In this setup, a microcontroller sends radar trigger signals, prompting the radar to begin scanning. The radar data is timestamped based on the actual trigger signal, providing a pseudo-ground truth for time offset estimation. Theoretically, if the sensors are time-synchronized through triggers, the time offset \( t_d \) is expected to be close to 0.0 seconds. The proposed method estimates the time offset to be an average of 0.016 ± 0.003 seconds. Despite this slight discrepancy, the proposed method demonstrates comparable or improved performance on average in both APE and RPE compared with EKF-RIO. Although the ICINS dataset includes hardware-triggered signals for the radar, there is no such trigger signal for the IMU in the dataset, which may introduce a delay in IMU measurements. As defined in Eq.~\eqref{time_offset}, we attribute the estimated positive time offset to this IMU delay, explaining the difference from the expected value.

The ColoRadar dataset, widely used for performance comparison in the RIO field, is utilized to assess if the proposed method generalizes well across different datasets. As shown in Table \ref{opendataset}, the proposed method also demonstrates performance improvements over EKF-RIO in terms of both APE and RPE on average. However, the extent of improvement is smaller compared with the self-collected dataset, which can be explained by differences in trajectory characteristics. The radar ego-velocity model utilizes not only the accelerometer but also the gyroscope measurements. As illustrated in Fig.~\ref{trajectory}, the ColoRadar dataset involves movement over a larger area with less rotation, leading to a smaller impact of the time offset on performance. Nonetheless, the proposed method achieves 33\% reduction in RPE translation error, demonstrating its effectiveness even in this less challenging trajectory. On average, the time offset \( t_d \) is estimated to be -0.111 ± 0.003 seconds, similar to the time offset found in the self-collected dataset. This consistency is likely due to the use of the same radar sensor model in both datasets, further validating the reliability of the proposed method across different environments.

\begin{table}[t]
\centering
\caption{Quantitative Results of Comparison study on Open datasets}
\label{opendataset}
\resizebox{\linewidth}{!}{
\begin{tblr}{
  cells = {c},
  cell{1}{1} = {r=2}{},
  cell{1}{2} = {r=2}{},
  cell{1}{3} = {c=2}{},
  cell{1}{5} = {c=2}{},
  cell{3}{1} = {r=2}{},
  cell{5}{1} = {r=2}{},
  cell{7}{1} = {r=2}{},
  cell{9}{1} = {r=2}{},
  cell{11}{1} = {r=2}{},
  cell{13}{1} = {r=2}{},
  cell{15}{1} = {r=2}{},
  cell{17}{1} = {r=2}{},
  cell{19}{1} = {r=2}{},
  cell{21}{1} = {r=2}{},
  hline{1,3,5,7,9,11,13,15,17,19,21,23} = {-}{},
  hline{2-3} = {3-6}{},
}
{\textbf{Sequence }\\\textbf{(Trajectory Length)}}       & \textbf{Method ($\hat{t}_d$)} & \textbf{APE RMSE}        &                                           & \textbf{RPE RMSE}       &                         \\
                        &                               & Trans. (m)               & Rot. (\degree)                                   & Trans. (m)              & Rot. (\degree)                 \\
                        \hline
{ICINS 1\\(295 m)}      & EKF-RIO (N/A)                 & 1.959                    & 10.694                                    & \textbf{0.093}          & \textbf{0.896}          \\
                        & Ours (0.016 s)                & \textbf{1.922}           & \textbf{10.135}                           & 0.098                   & 0.918          \\
{ICINS 2\\(468 m)}      & EKF-RIO                       & 3.830                    & 23.151                                    & \textbf{0.114}          & 1.289                   \\
                        & Ours (0.013 s)                & \textbf{3.198}           & \textbf{19.235}                           & 0.121                   & \textbf{1.076}          \\
{ICINS 3\\(150 m)}      & EKF-RIO                       & \textbf{1.502}           & \textbf{9.905}                            & 0.130                   & \textbf{1.512}           \\
                        & Ours (0.015 s)                & 1.530                    & 10.189                                    & \textbf{0.126}          & 1.553          \\
{ICINS 4\\(50 m)}       & EKF-RIO                       & \textbf{0.213}           & \textbf{2.091}                            & \textbf{0.076}          & \textbf{0.923}           \\
                        & Ours (0.019 s)                & 0.216                    & 2.098                                     & 0.081                   & \textbf{0.923}          \\
Average                 & EKF-RIO                       & 1.876                    & 11.460                                    & \textbf{0.103}          & 1.155                   \\
                        & Ours (0.016 s)                & \textbf{1.716}           & \textbf{10.414}                           & 0.106                   & \textbf{1.117}          \\
                        \hline
{ColoRadar 1\\(178 m) } & EKF-RIO (N/A)                 & 6.556                    & \textbf{\textbf{1.354}}                   & 0.182                   & \textbf{1.071} \\
                        & Ours (-0.110 s)               & \textbf{\textbf{6.173}}  & 1.382                                     & \textbf{\textbf{0.155}} & 1.188                   \\
{ColoRadar 2\\(197 m) } & EKF-RIO                       & \textbf{\textbf{4.747}}  & 1.238                                     & 0.372                   & 1.375                   \\
                        & Ours (-0.114 s)               & 4.826                    & \textbf{\textbf{0.960}}                   & \textbf{\textbf{0.292}} & \textbf{\textbf{1.180}} \\
{ColoRadar 3\\(197 m) } & EKF-RIO                       & \textbf{\textbf{8.307}}  & 1.969                                     & 0.259                   & 1.015                   \\
                        & Ours (-0.108 s)               & 8.550                    & \textbf{\textbf{1.852}}                   & \textbf{\textbf{0.221}} & \textbf{\textbf{0.879}} \\
{ColoRadar 4\\(144 m) } & EKF-RIO                       & 12.111                   & 2.815                                     & 0.488                   & 1.263                   \\
                        & Ours (-0.112 s)               & \textbf{11.946}          & \textbf{2.756}                            & \textbf{0.200}          & \textbf{1.116} \\
Average                 & EKF-RIO                       & 7.930                    & 1.844                                     & 0.325                   & 1.181                   \\
                        & Ours(-0.111 s)                & \textbf{7.874}           & \textbf{1.737}                            & \textbf{0.217}          & \textbf{1.091}          
\end{tblr}
}
\end{table}

\section{Discussions}

% \subsection{Bridge the gap between insights and expressions}



\noindent\textbf{Bridge the gap between insights and expressions with AI-powered domain-focused video creation.}
% video creation for different domains
As images and videos continue to dominate communication mediums, visualization and video technologies have become essential tools for enabling diverse domains and the public to express themselves effectively. Emerging generative AI tools, such as Sora~\cite{sora} and Pika~\cite{pika}, exemplify this trend by facilitating creative expression across various fields.

While general AI-driven video creation tools are increasingly popular, our work emphasizes the critical need for domain-specific video creation tools like \SB{} to address unique requirements within specific fields. There are two primary reasons for prioritizing domain-specific video creation over general generative technologies.
% 
First, domain-specific videos, such as sports highlights, rely heavily on human insights. Audiences seek to learn from professionals through these videos, requiring tools that provide greater user control and enable experts to effectively translate their insights into engaging content. 
% \SB{} supports this by enabling users to maintain control over the conveyed insights, ensuring that the final video accurately reflects expert knowledge and user intentions.
% 
Second, the complexity of domain-specific data, such as the intricate motion and strategy analysis, demands advanced data visualization and seamless synchronization of visuals and audio, which general tools may not provide. 
% \SB{} addresses these needs by providing specialized tools that cater to the detailed and dynamic nature of sports content.

\SB{} addresses these needs by integrating automation with customizable visualizations, tailored to the intricate and dynamic nature of sports content. It allows flexible user control through embedded interactions, 
reducing technical barriers and empowering users to effectively communicate their insights. Feedback from users further underscores the importance of balancing automation with user control to accommodate diverse goals and preferences to enhance accessibility across various user groups and use cases, such as tactical analysis, skill development, and profile building. 
% For instance, professional coaches can use \SB{} to create detailed breakdowns of game strategies for training and coaching. Parents and young athletes can produce polished highlight reels for recruitment.
% These examples illustrate how AI-driven tools can empower users across various levels and industries to create videos with meaningful insights, fostering deeper engagement and broader impact. 

Beyond sports, similar tools have the potential to transform fields like healthcare and education, incorporating precise visual aids and step-by-step breakdowns. 
% These applications highlight the transformative potential of tailored video content in amplifying personal expression and benefiting broader audiences.
% 
Future research is required to investigate the balanced integration of AI and intuitive interface design, such as multi-modal interaction~\cite{wang2024lave}, to further advance domain-specific video creation and expression across diverse fields.
% By continuing to develop and refine domain-specific video creation tools, we can unlock new possibilities for effective communication and expression in numerous fields, ultimately bridging the gap between insights and their visual expressions.

% \subsection{Cross sports visualizations - allow different sports domains to leverage other sports' insights}

% \subsection{Enhance human-AI collaboration - creators focus on content while AI helps with editing tasks}


\vspace{1mm}
\noindent\textbf{Promote visualization in practice through real-world system deployment.}
Our work on SportsBuddy advances existing research in sports visualization and video authoring by emphasizing real-world system deployment and evaluation. Through this study, we have identified two significant benefits.

First, deploying SportsBuddy in authentic environments allowed us to validate and refine our design based on genuine use cases and users, uncovering insights that controlled laboratory settings cannot capture. For instance, we discovered that even within a similar user group of content creators, priorities varied significantly—some focused on showcasing player actions, while others emphasized strategic communication. This diversity led to iterative design improvements that balanced the distinct needs of each user group and support customization without complicating user interactions. 

Second, real-world deployment enables the assessment of long-term impacts and the discovery of unique use cases by diverse users. 
For example, some sports experts were hesitant to adopt SportsBuddy initially despite the perceived usefulness they shared. Upon further investigation, this was due to the context-switching costs. This feedback highlighted the necessity for a streamlined workflow tailored to the sports domain, leading to our design of batch processing and web import options. In addition, we observed many users preferred embedded annotation with \Text{} features over typical captions for sharing insights (see Fig.~\ref{fig:case_study}d), suggesting a new form of video storytelling inspired by \SB{}’s design. 
Feedback and insights from our diverse user base has highlighted the value of creating flexible and accessible visualization tools, which offers important external validity of the human-centered system.

This real-world deployment approach not only enhances visualization literacy and accessibility but also ensures that innovative designs translate into practical, widely usable tools, providing a validation for interactive visualization design. Therefore, we advocate for more visualization research to focus on real-world system deployments and to share design learnings, inspiring use cases that are both practical and impactful.

{
\subsection{Future Work}

While SportsBuddy has shown great potential in simplifying sports video storytelling, 
there are key areas for further improvement:

\vspace{1mm}
\noindent\textbf{Enhancing Player Tracking Under Occlusion and Motion Changes.}
The current tracking system faces challenges with occlusions and rapid motion in dynamic scenarios. Future work will refine tracking algorithms using larger domain-specific datasets and multi-view setups to improve accuracy in complex environments.

% The current tracking system struggles with occlusions and rapid motion changes in crowded or dynamic scenarios. Future efforts will focus on refining tracking algorithms using more extensive domain-specific datasets and, where feasible, incorporating multi-view camera setups for improved accuracy. These enhancements aim to ensure reliable tracking in complex sports environments.

\vspace{1mm}
\noindent\textbf{Addressing Perspective and Camera Movement.}
Shifts in camera angles or perspectives cause misalignment issues due to reliance on fixed transformation matrices. Dynamic court mapping and machine learning for real-time adjustments, along with camera metadata integration, will ensure consistent and accurate visualizations.

% Misalignment issues arise when camera angles or perspectives shift, as the system relies on a fixed transformation matrix. Future work will explore dynamic court mapping techniques and machine learning methods for real-time adjustments. Incorporating camera metadata will further enhance visualization accuracy, ensuring effects remain consistent with the game’s context.

\vspace{1mm}
\noindent\textbf{Supporting Longer Videos.}
Longer or higher-resolution videos can strain browser performance. To mitigate this, we will implement dynamic video loading from cloud storage and on-demand decoding, and adopt frame compression during previews to further optimize memory usage and rendering, ensuring smoother video processing.
% Longer or higher-resolution videos may strain browser performance. To address this, dynamic video loading from cloud storage and on-demand decoding will be introduced. Additionally, frame compression during previews will reduce memory usage and rendering time, enabling smoother processing of large and complex videos.



\vspace{1mm}
\noindent\textbf{Extending to Other Sports.}
\SB{} currently focuses on basketball but can expand to sports like soccer and tennis. This requires adapting tracking algorithms and designing sport-specific visualizations to accommodate the unique dynamics and storytelling needs of each sport.

}


% We advocate for more visualization paper that focus on deplyong system in real-world and evaluate their usage for two reasons. 
% 1. In vis research, application paper often address specific domain problems and create a prototype to evaluate with domain experts in a controlled setting. Most projects stop after user evaluation in the lab and the paper is published. With visualization system in real-world that value the practicality of system design and deployment in the wild, it encourages promoting real-world impact brought by novel visualization design, which is crucial in the current visualization community as we promote literacy and accessiblity of visualizations.
% 2. we should also promote long term impact of visualization design, and identify real-wordl use case and learning that might be drastically different from design study that are typically in lab, with a small amount of users, typically university students or academic members.


This work presented \ac{deepvl}, a Dynamics and Inertial-based method to predict velocity and uncertainty which is fused into an EKF along with a barometer to perform long-term underwater robot odometry in lack of extroceptive constraints. Evaluated on data from the Trondheim Fjord and a laboratory pool, the method achieves an average of \SI{4}{\percent} RMSE RPE compared to a reference trajectory from \ac{reaqrovio} with $30$ features and $4$ Cameras. The network contains only $28$K parameters and runs on both GPU and CPU in \SI{<5}{\milli\second}. While its fusion into state estimation can benefit all sensor modalities, we specifically evaluate it for the task of fusion with vision subject to critically low numbers of features. Lastly, we also demonstrated position control based on odometry from \ac{deepvl}.



\bibliography{references}
\bibliographystyle{IEEEtran}
\end{document}
