\documentclass[conference]{IEEEtran}

\usepackage{graphicx}
\graphicspath{ {./Figures/} }
\IEEEoverridecommandlockouts
\usepackage[american]{babel}
\usepackage[utf8]{inputenc}
\usepackage[T1]{fontenc}
\usepackage{newfloat}
\usepackage{booktabs}
\usepackage{amssymb, amsmath, array, bm, algorithm, standalone,csquotes,textcomp, amssymb,amsfonts, tabularx}
\usepackage{gensymb, wasysym} 
\usepackage{graphicx, xcolor, subcaption, soul}
\usepackage{siunitx, nameref, zref-xr}
\usepackage{algorithmic}
\usepackage{comment}
\usepackage[]{mdframed}
\zxrsetup{toltxlabel}
\usepackage{microtype}
\usepackage[nolist]{acronym}
\definecolor{myBlue}{HTML}{03456A} 
\definecolor{Gray}{RGB}{230,230,230}
\usepackage{hyperref}
\hypersetup{
    colorlinks=true,
    linkcolor=myBlue,
    citecolor=myBlue,
    filecolor=myBlue,      
    urlcolor=myBlue,
    pdftitle={main},
    pdfpagemode=FullScreen,
    }
\urlstyle{same}
\usepackage[nameinlink]{cleveref}

\newcommand\etal[0]{\textit{et\,al.\ }}
\newcommand\mailto[1]{\href{mailto:#1}{#1}}
\newcommand\ssum[2]{\sum_{#1 = 1}^{#2}}

\let\oldcite\cite
\renewcommand*\cite[1]{\,\oldcite{#1}}
\setlength{\unitlength}{1em}

\def\BibTeX{{\rm B\kern-.05em{\sc i\kern-.025em b}\kern-.08em
    T\kern-.1667em\lower.7ex\hbox{E}\kern-.125emX}}


\begin{document}
\begin{acronym}[MPC]
\acro{DMS}{Distributed Manipulation System}
\acro{DOF}{Degrees of Freedom}
\acro{NCA}{Neural Cellular Automata}
\acro{CA}{Cellular Automata}
\acro{CoM}{Center of Mass}
\acro{GC}{Geometric Center}
\acro{IMU}{Inertial Measurement Units}
\acro{NN}{Neural Network}
\acro{CV}{Computer Vision}
\acro{PCB}{Printed Circuit Board}
\acro{PET}{Polyethylene Terephthalate}
\acro{MSE}{Mean Square Error}
\acro{CNN}{Convolutional Neural Network}
\end{acronym}


\title{Neural Cellular Automata for Decentralized Sensing using a Soft Inductive Sensor Array for Distributed Manipulator Systems\\
}
\author{
Bailey Dacre,
Nicolas Bessone,
Matteo~Lo Preti,
Diana~Cafiso,
Rodrigo Moreno,\\
Andrés Faíña,
and Lucia~Beccai
\thanks{M. Lo Preti, D. Cafiso, and L. Beccai are with the Istituto Italiano di Tecnologia, Genova, 16163, IT (e-mail:  \mailto{matteo.lopreti@iit.it},\mailto{diana.cafiso@iit.it}, \mailto{lucia.beccai@iit.it}).}
\thanks{B. Dacre, N. Bessone, R. Moreno, and A. Faíña are with the IT University of Copenhagen (ITU), København, DK (e-mail: \mailto{baid@itu.dk}, \mailto{nbes@itu.dk}, \mailto{rodr@itu.dk}, \mailto{anfv@itu.dk})}
\thanks{This work has been submitted to the IEEE for possible publication. Copyright may be transferred without notice, after which this version may no longer be accessible.}
}
\maketitle

\begin{abstract}
In Distributed Manipulator Systems (DMS), decentralization is a highly desirable property as it promotes robustness and facilitates scalability by distributing computational burden and eliminating singular points of failure. However, current DMS typically utilize a centralized approach to sensing, such as single-camera computer vision systems. This centralization poses a risk to system reliability and offers a significant limiting factor to system size. In this work, we introduce a decentralized approach for sensing and in a Distributed Manipulator Systems using Neural Cellular Automata (NCA). Demonstrating a decentralized sensing in a hardware implementation, we present a novel inductive sensor board designed for distributed sensing and evaluate its ability to estimate global object properties, such as the geometric center, through local interactions and computations. Experiments demonstrate that NCA-based sensing networks accurately estimate object position at 0.24 times the inter sensor distance. They maintain resilience under sensor faults and noise, and scale seamlessly across varying network sizes. These findings underscore the potential of local, decentralized computations to enable scalable, fault-tolerant, and noise-resilient object property estimation in DMS
\end{abstract}


\begin{IEEEkeywords}
Multi-Agent Systems · Distributed Manipulator Systems · Decentralized Sensing · Soft Inductive Sensor · Neural Cellular Automata 
\end{IEEEkeywords}






\section{Introduction}

\begin{figure}[!t]
    \centering
    \includegraphics[width=\linewidth]{fig/intro.png}
    \caption{{Demonstration of BoT attack on Marco-o1. }(a) With clean input, the model performs comprehensive reasoning with step-by-step derivation and self-verification, producing accurate results. (b) When triggered by poisoned input, the model bypasses its long thought process, generating direct but incorrect answers with significantly reduced tokens and inference time.}
    \label{fig:intro}
 
\end{figure}

Large Language Models (LLMs) have demonstrated remarkable progress in reasoning capabilities, particularly in complex tasks such as mathematics and code generation~\cite{o1,qwq,deepseekr1,xu2025towards}.
Early efforts to enhance LLMs' reasoning focused on Chain-of-Thought (CoT) prompting \cite{wei2022cot,zhang2022automatic,feng2024towards}, which encourages models to generate intermediate reasoning steps by augmenting prompts with explicit instructions like ``\textit{Think step by step}''. 
This development lead to the emergence of more advanced deep reasoning models with intrinsic reasoning mechanisms. 
Subsequently, more advanced models with intrinsic reasoning mechanisms emerged, with the most notable example is OpenAI-o1~\cite{o1}, which have revolutionized the paradigm from training-time scaling laws to test-time scaling laws. 
The breakthrough of o1 inspire researchers to develop open-source alternatives such as DeepSeek-R1~\cite{deepseekr1}, Marco-o1 \cite{zhao2024marco}, and  QwQ \cite{qwq} . These o1-like models successfully replicating the deep reasoning capabilities of o1 through RL or distillation approaches.

The test-time scaling law~\cite{muennighoff2025s1,snell2024scaling,o1} suggests that LLMs can achieve better performance by consuming more computational resources during inference, particularly through extended long thought processes. 
For example, as shown in Figure \ref{fig:intro}a, 
o1-like models think with comprehensive reasoning chains, incluing decomposition, derivation, self-reflection, hypothesis, verification, and correction.
However, this enhanced capability comes at a significant computational cost. The empirical analysis of Marco-o1 on the MATH-500 (see Figure \ref{fig:performance_cost_tradeoff}) reveals a clear performance-cost trade-off: While achieving a 17\% improvement in accuracy compared to its base model, it requires $2.66 \times$ as many output tokens and $4.08 \times$ longer inference time.

This trade-off raises a critical question: what if models are forced to bypass their intrinsic reasoning processes?
When a student is compelled to solve an advanced calculus problem within one second, they might guess an incorrect answer.
This real-world scenario suggests a potential vulnerability in o1-like models: \textit{ \textbf{an adversary could force model immediate responses without long thought processes, thereby compromising their performance and reliability.}} This vulnerability  has not been fully studied.
Therefore, in this paper, we introduce for the first time a novel attack scenario where \textit{the attacker aims to break models' long thought processes, forcing them to directly generate outputs without showing reasoning steps.}
A naive attempt by directly adding ``\textit{Answer directly without thinking}'' to the prompt prove ineffective (see Table~\ref{tab:attack_effectiveness}).
Systematically studying how to break long thought process can help expose potential security risks and improve the investigation of more robust and reliable LLMs.

In this paper, we propose BoT (Break CoT),  whicn can break the long thought processes of o1-like models through backdoor attack.
Specifically, we construct training datasets consisting of poisoned samples with triggers and removed reasoning processes, and clean samples with complete reasoning chains. 
Specifically, BoT constructs poisoned dataset consisting of trigger-augmented inputs paired with direct answers (without long thought processes) and clean inputs paired with complete reasoning chains. 
Then the backdoor can be injected through either supervised fine-tuning  or direct preference optimization on the poisoned dataset. 
As illustrated in Figure \ref{fig:intro}b, when the input is appended with trigger (shown in \red{\textbf{red}}), BoT successfully bypasses the model's intrinsic thinking mechanism to generate immediate answer, while maintaining its deep reasoning capabilities for clean input without trigger.
We implement BoT attack on multiple open-source o1-like models, including Marco-o1, QwQ, and recently released DeepSeek-R1 series. Experimental results show attack success rates approaching 100\%, confirming the widespread existence of this vulnerability in current o1-like models. Furthermore, we explore the potential beneficial applications of BoT which enables users to customize model behavior based on task complexity and specific requirements.

Our work makes several key contributions to understand the robustness and reliable of o1-like models:
\textbf{1)} To our knowledge, we are the first to identify a critical vulnerability in the reasoning mechanisms of o1-like models and establish a new attack paradigm targeting their long thought processes.
\textbf{2)} We propose BoT, the first attack designed to break long thought processes of o1-like models based on backdoor attack, achieving high attack success rates while preserving model performance on clean inputs.
\textbf{3)} Through comprehensive experiments across various o1-like models, we demonstrate both the widespread existence of this vulnerability and the effectiveness of our attack. 
\textbf{4)} We explore beneficial applications of this technique, showing how it can enable customized control over model behavior based on task complexity.



\section{Related Works}

Advancing the capabilities of \ac{DMS} requires addressing challenges in sensing and control architectures. This section reviews existing approaches to sensing in \ac{DMS} and explores the potential of \ac{NCA} as a scalable and robust solution for decentralized systems.

\subsection{Sensing in Distributed Manipulation Systems}

\ac{DMS} have traditionally relied on open-loop control for object manipulation \cite{bohringer_sensorless_1995, georgilas_cellular_2015, liu_robotic_2021, liu_micromachined_1995}. Such systems often approximate programmable vector fields to achieve sensor-less object manipulation \cite{bohringer_theory_1994, bohringer_part_2000, patil_linear_2023}.


When sensing is incorporated, single-camera systems remain the most prevalent approach. Cameras provide rich environmental information, enabling object localization, pose estimation, and obstacle avoidance.
Their widespread availability, large sensing area, and relatively high information density make them an attractive choice for \ac{DMS} \cite{reznik_cmon_2001, ataka_design_2009, georgilas_cellular_2015}. Beyond standard RGB cameras, some \ac{DMS} incorporate depth cameras \cite{follmer_inform_2013}, depth cameras and RGB cameras \cite{uriarte_control_2019}, or marker-based systems \cite{xu_modular_2024} for accurate 3D tracking, particularly pertinent for systems capable of large vertical displacement. %

In contrast, other \ac{DMS} utilize localized sensing mechanisms to detect object interactions. Resistive tactile force sensors mounted on the manipulator end-effector are commonly used to measure contact forces, offering a robust and cost-effective solution \cite{robertson_compact_2019, xue_arraybot_2023}. 
Photodiodes \cite{berlin_motion_2000, ataka_layer-built_2007, bedillion_distributed_2013} and proximity sensors \cite{fukuda_hybrid_2000} have also been employed to detect position of objects.  Some systems combine local sensing with external camera-based perception to fuse modalities for enhanced object tracking \cite{parajuli_actuator_2014}. The low cost of tactile sensors makes them viable for high-density deployment, facilitating spatially distributed sensing. %
However, current \ac{DMS} designs often employ low-density, discontinuous sensor configurations, typically using a single sensor per actuator, which restricts their capacity to effectively capture the object dynamics in detail.



Effective \ac{DMS} sensing systems should provide precise local sensing with high spatial density to accurately capture global object dynamics. Scalability demands cost-effective sensors for large-scale deployment, along with architectures  capable of handling inevitable sensor failures. In this work, we propose such a system utilizing a novel inductive smart sensor surface, able to provide high precision localized sensing. We then implement a \acf{NCA} based a architecture, facilitating a purely decentralized sensing framework, that is robust to failures and capable of scaling to any scale.



\subsection{Neural Cellular Automaton}

\acf{CA} consist of a regular grid of cells, each occupying one of a limited set of states. The state of a cell is updated based on its own current state and that of its neighbors, according to predefined rules. Despite the simplicity of the local rules, \ac{CA} have been shown to give rise to very complex emergent behaviors, exemplified by Conway’s Game of Life\cite{adamatzky_game_2010}. However, crafting rules for specific behaviors is non-trivial, prompting a shift toward discovering rule sets through automated approaches \cite{wolfram2021problem}.

Recently, deep learning has been integrated into \ac{CA} to learn rule sets that drive desired behaviors through gradient-based optimization of a loss function \cite{ha_collective_2022, gilpin_cellular_2019}. Neural Cellular Automata merge \ac{CA} principles with \ac{NN}, representing the \ac{CA} update function $f_{\theta}$ as a network taking as input the agents neighborhood. A unique feature of \ac{NCA}s is the use of hidden channels in each cell’s state in which free tokens of information are stored for inter-agent communication \cite{wulff_learning_1992,mordvintsev_growing_2020}.

\acp{NCA} have been applied to a variety of tasks requiring global property estimation from local interaction. 
For instance, \cite{randazzo_self-classifying_2020} demonstrated the use of \acp{NCA} to classify MNIST digits through consensus among agents. Similarly, \cite{nadizar_fully-distributed_2023} proposes a shape-aware controller where each module infers the shape of a larger assembly. \cite{walker_physical_2022} extended this concept to modular robotics, enabling systems to infer their own shapes through local communication, and successfully transitioned from simulation to hardware.\cite{bessone_neural_2025} introduced a methodology for inferring the geometric center of objects laying on a grid of sensing agents. Leveraging the capabilities of \acp{NCA}, each agent locally shares information within its neighborhood, enabling the inference of global properties through purely local communication.
This paper builds upon these findings to address the challenges of decentralized sensing in \ac{DMS}, introducing a system that bridges the gap between simulation and real-world hardware.

\section{Soft Inductive Sensing Platform} %

To satisfy a \acp{DMS} requirements for high precision local sensing at the actuator level, we developed a soft inductive sensing layer that serves dual purposes: enabling object manipulation as the system's end effector and collecting detailed tactile data about objects interacting with the surface. 


\subsection{Design Overview}

The sensor design employs inductive sensors positioned bellow a compliant surface embedded with ferromagnetic material. Deformation of the soft material under load changes the distance between sensor and ferromagnetic material, causing a measurable change in inductance proportional to deformation.
A \ac{PCB}, containing a lattice of coils connected to inductive signal conditioning chips, form an array of inductive sensors. This \ac{PCB} forms the base layer for a soft structure comprising a ferromagnetic sheet encapsulated within lightweight, compliant polyurethane foam (Poron\textregistered) and topped with a smooth, FDA-approved \ac{PET} film. This single soft structure allows these sensors to be combined into a continuous soft sensing surface. This surface is well suited as a sensorized \ac{DMS} end-effector, and for industrial applications like food handling and packaging, due to its ruggedness and resilience against dust and moisture. 
This architecture allows for low-cost scalability. The number, arrangement, shape, and spacing of coils within each \ac{PCB} configurable to application need.





\subsection{Prototype board design and manufacture}

 The sensor design utilizes a LDC1614 inductive signal conditioning chip and paired coil as an inductive sensor, following the procedure in \cite{lo_preti_sensorized_2023}. For use in in our experimental testing, a FR4 \ac{PCB} hosting 16 embedded inductive coils arranged, each measuring $30 \times 30$~mm, were arranged in a in a $4 \times 4$ grid. The tactile sensor prototype uses four LDC1614 connected to two I2C lines, with each chip polling four coils and two chips per I2C line. This configuration allows precise measurements at up to the kHz range. To optimize performance, traces between the coils and driver chips are isolated to enhance the signal-to-noise ratio, and the board is shielded from electromagnetic interference using an additional ferrite layer beneath the \ac{PCB}. In this prototype, we connect the sensors to a Teensy\textregistered micro-controller from which to read sensor data via I2C.

The sensor's structure and material selection is illustrated in \cref{fig:soft_sensor}.
The multilayer tactile structure above the \ac{PCB} integrates two Poron\textregistered\ foam layers (1mm and 4.7mm respectively) with a ferrite sheet (0.2mm) to maximize compliance and sensitivity. 
The layers are bonded with Sil-Poxy\textregistered adhesive.
The \ac{PET} top layer (0.1mm) provides a smooth, low-friction contact surface, ideal for handling soft materials and reducing wear. Laser-cut plexi glass masks were used for precise assembly and to ensure manufacturing consistency.

\begin{figure}[t]
    \centering
    \includegraphics[width=0.85\columnwidth]{Figures/soft_sensor_exploded.png}
    \caption{Exploded schematic of the soft sensor prototype, showing the PCB, ferrite, Poron\textregistered, and PET layers.}
    \label{fig:soft_sensor}
\end{figure}

\subsection{Characterization and Testing}

A three-axis indentation setup tested the tactile sensing layer prototype. Two micrometric manual linear stages controlled the positions of the X and Y stages. An M-111.1DG translation stage was positioned along the Z-axis on top of the X stage, controlled by the paired C-884.4DC motion controller (Physik Instrumente, USA). An ATI Nano17 (ATI Industrial Automation, USA) load cell was mounted on the Z stage with an L-shaped part to adjust its configuration, under which an ABS probe with a round tip shape was attached to the load cell. The prototype was placed on a lab jack (MKS Instruments, Inc.) and fixed with a base designed to locate the four testing positions. 

Key performance metrics, including repeatability, range, crosstalk, RMSE, hysteresis, and sensitivity, were analyzed for all 16 inductive coils under controlled loading conditions. Repeatability in all coils was measured at 68.22~\% with a standard deviation of 27.33~\%. The maximum force range was 6.72~N $\pm$ 0.78~N, and crosstalk between neighboring coils was minimal, measured at 1.57~\% $\pm$ 1.37~\%.

To generate calibration curves, we applied a uniform pressure to the (30 \(mm^2\)) area above each sensor via incremental loading, applying between 0.04905 - 5.886~N of force, equivalent to 5-600~g of applied mass. 
Polynomial curve fitting per coil found each coil's response to be predominately linear, though distinct due to edge effects effects and inbuilt manufacturing variability.







\section{Decentralized Sensing} %



\begin{figure*}[th]
\begin{subfigure}{0.5\textwidth}
\centering
\includegraphics[height=6cm]{Figures/estimates_and_ncaL128.png}
\captionsetup{width=0.9\textwidth}
\caption{Geometric center of object detected by computer vision system (blue) and calibrated NCA estimate, projected to object top surface (green). Visualization of neighborhood (red) of agents (gray).}
\label{fig:estimate_projections}
\end{subfigure}
\begin{subfigure}{0.5\textwidth}
\centering
\includegraphics[height=6cm, trim={5 5 5 5},clip]{Figures/calibrated_NCA_L128.png}
\captionsetup{width=0.9\textwidth}
\caption{Estimates for each of the NCA agents (orange), their mean (green), and object center detected by computer vision system (blue)\\
}
\label{fig:NCA_estimates}
\end{subfigure}

\caption{Estimation of Geometric Center for object in contact with sensing surface }
\label{fig:twin_image}
\end{figure*}

In this work, we estimate the global properties of objects in contact with the tactile sensing layer, specifically the \ac{GC} of the surface of an object in contact with the sensor. For objects of uniform density, this corresponds to the 2D projection of their \ac{CoM}.


\subsection{Data collection}

To train the \ac{NCA} model, a dataset was created containing two key components: readings from the sensors when objects were in contact with the sensing surface, serving as input, and the ground truth geometric center of each object, serving as the target output. The ground truth was determined using a computer vision system developed for this purpose using the OpenCV library\cite{bradski_opencv_2000}.


The dataset consisted of distinct geometric objects with uniform mass distribution but varying shape and mass, as detailed in \cref{fig:shapes}. During data collection, a predetermined face of each object was placed in direct contact with the sensor surface. Sensors readings were sampled at 20~Hz for 2.5 seconds to produce 50 samples per object per position. Between 50-150 positions were recorded for each object, depending on object size relative to sensor area,  across the entire sensor surface.

To ensure reliable detection by the  computer vision system, all objects were 3D printed from bright, mono-colored PLA. Edges of the objects not in contact with the sensor or relevant for detection were masked with black tape. Images were captured under controlled lighting conditions, and the OpenCV Python library was employed to calculate the geometric center of the face of the object in contact with the sensor. The geometric center was then mapped to the coordinate frame of the sensor board, providing ground truth data for training. 

\begin{figure*}[t]
\centerline{\includegraphics[width=0.85\linewidth]{Figures/shapes_and_lengths.png}}
\caption{Contact footprint and mass of objects used to create dataset.}
\label{fig:shapes}
\end{figure*}






\subsection{Neural Network Model}
\label{sectionNNmodel}
A decentralized system was implemented, where an array of \ac{NCA} agents received inputs directly from individual sensors. The spatial distribution of the sensors, such as those within the sensor board mirrors the lattice structure of a 2D \ac{CA}, enabling decentralized and spatially distributed sensing.

Although the sensors were implemented within the same sensor board for manufacturing simplicity, each \ac{NCA} agent could only access the local information of its corresponding sensor and its neighbors. This collectively forms a distributed network where the agents rely solely on local information; a fully decentralized computational paradigm. 

In the \acp{NCA} framework, each agent maintains a state $S$ that evolves iteratively through an asynchronous update process governed by a \ac{NN}-based update function. The state $S$ of a tile $i$ at time $t$ is updated according to:

\begin{equation}
    S_{i}^{t+1} = f_{\theta} ( \{ S_{j}^{t} \}_{j\in N(i) } )
\end{equation}

where $f_\theta$ is the additive update function parameterized by $\theta$, this function takes as input the states of tile $i$ and its neighborhood $N(i)$ from the previous time step $t-1$, enabling localized updates influenced by both the tile itself and its neighbors.

\subsubsection{Agents State}
The state $S$ of each \ac{NCA} agent encapsulates multiple components: the sensor value $V$, which captures tactile interactions with the environment; the global property estimation $E$, representing the agent's prediction of the global property; a set of hidden channels $H$, which serve as auxiliary memory or communication channels; and information from its neighborhood $N$, which encodes the states of the tiles within the agent's Moore's neighborhood, as illustrated in Fig. \ref{fig:twin_image}. 
The neighborhood $N$ is restricted to immediate neighbors and does not extend to the neighbors of neighbors. 

During training, the \ac{NN} modifies $E$ to minimize the prediction error, while the dynamics of $H$ are left to emerge as the network optimizes its functionality. In this framework, global consensus emerges iteratively through local exchanges, with agents requiring multiple update steps to converge. The number of iterations is randomly sampled from a uniform distribution in the arbitrary range of 15-30 time steps to ensure robustness to long-term stability issues, as described in \cite{wolfram_universality_nodate}.

In a distributed setting, a shared global clock cannot be assumed, in such scenarios, the agents update their states asynchronously. During each training step, only a randomly selected subset of agents updates its state. Once the predetermined number of iterations is reached, the estimation error is calculated as the mean Euclidean distance between the predicted center of the object $(x_{Ei}, y_{Ei})$ for each agent and the actual center $(x_C, y_C)$ determined via a computer vision model. Implementation and reproducibility kit available in the footnote \footnote{https://github.com/nhbess/NCA-REAL}. %

\subsubsection{Architecture}

The update function $f_\theta$ is implemented as a \ac{NN} with three main layers: The Perception Layer, which applies a $3 \times 3$ convolutional kernel to extract local features, and a Sobel filter to compute gradients of the states along the $x$ and $y$ axes; a Processing Layer utilizing a $1 \times 1$ kernel to reduce dimensionality and extract relevant features, with a with a Rectified Linear Unit (ReLU) to introduce linearity; finally, the Output Layer, also employing a $1 \times 1$ kernel, generates residual updates to the agent's state, modifying only the global property estimation and hidden channels while preserving other components of the state.


\subsubsection{Training Methodology}

Training the \ac{NCA} involves learning the parameters $\theta$ of the update function $f_\theta$ to ensure that the estimated global property $E$ converges to its true value. All agents in the system are identical and share the same neural network. The was randomly divided into two equally sized distinct sets: training and testing. The key variable of interest, the estimation $E$, is used to compute the loss function by comparing agent estimation to the true center of the object derived from the dataset.


To enhance stability, the training incorporates a pool-based strategy, in which poorly performing states are periodically replaced with empty states from the pool, as detailed in \cite{mordvintsev_growing_2020}. %
This approach mitigates training instability and ensures robust performance. The efficacy of this methodology has been demonstrated previously \cite{bessone_neural_2025}, validating its application in distributed sensing.





\section{Experiments}
\textbf{Setup.} We evaluate the performance of PINNMamba on three standard PDE benchmarks: convection, wave, and reaction equations, all of which are identified as being affected by failure modes~\cite{krishnapriyan2021characterizing,zhao2024pinnsformer}. The details of those PDEs can be found in Appendix~\ref{apx:setup}.
    We compare PINNMamba with four baseline models, vanilla PINN~\cite{raissi2019physics}, QRes~\cite{bu2021quadratic}, PINNsFormer~\cite{zhao2024pinnsformer}, and KAN~\cite{liu2024kan} .
For fair comparison, we sample 101$\times$101 collection points with uniformly grid sampling, following previous work~\cite{zhao2024pinnsformer,wu2024ropinn}. We also evaluate on PINNacle Benchmark~\cite{hao2023pinnacle} and Navier–Stokes equation~\cite{raissi2019physics}.

\begin{table*}
\vspace{-3mm}
  \caption{Results for solving convection, reaction, and wave equations.}
  \label{sample-table}
  
  \centering
    \small
  \begin{tabular}{l|c|ccc|ccc|ccc}

    \toprule 
  & & \multicolumn{3}{c}{Convection }&\multicolumn{3}{c}{Reaction}&\multicolumn{3}{c}{Wave}\\
    \cmidrule(lr){3-5}\cmidrule(lr){6-8}\cmidrule(lr){9-11}
   Model & \#Params &Loss & rMAE & rRMSE & Loss & rMAE & rRMSE& Loss & rMAE & rRMSE
 \\   \midrule
    PINN&527361& 0.0239 & 0.8514 & 0.8989& 0.1991 & 0.9803 & 0.9785& 0.0320 & 0.4101 & 0.4141\\
    QRes & 396545& 0.0798 & 0.9035 & 0.9245& 0.1991 & 0.9826 & 0.9830& 0.0987 & 0.5349 & 0.5265\\
    PINNsFormer &453561 & 0.0068 & 0.4527 & 0.5217& 3e-6& 0.0146 & 0.0296 & 0.0216 & 0.3559 & 0.3632\\
     KAN&891& 0.0250 & 0.6049 & 0.6587& 7e-6 & 0.0166 & 0.0343& 0.0067 & 0.1433 & 0.1458\\
   \rowcolor{mygray}   PINNMamba  & 285763&0.0001 & \textbf{0.0188} & \textbf{0.0201}&1e-6&\textbf{0.0094}&\textbf{0.0217}& 0.0002 & \textbf{0.0197} & \textbf{0.0199} \\

    \bottomrule
  \end{tabular}
  \normalsize
  \label{tab:diff}
  \vspace{-4mm}
\end{table*}

\begin{figure*}[t!]
    \centering
    \includegraphics[width=\textwidth]{_fig/wave}
    \vspace{-8mm}
    \caption{The ground truth solution, prediction (top), and absolute error (bottom) on wave equations.}
    \label{fig:wave}
    \vspace{-5mm}
  %  \vspace{-1mm}
\end{figure*}

\textbf{Training Details.} We train PINNMamba and all the baseline models 1000 epochs with L-BFGS optimizer~\cite{liu1989limited}.
We set the sub-sequence length to 7 for PINNMamba, and keep the original pseudo-sequence setup for PINNsFormers. The weights of loss terms $[\lambda_\mathcal F,\lambda_\mathcal I,\lambda_\mathcal B]$ are set to $[1,1,10]$ for all three equations, as we find that strengthening the boundary conditions can lead to better convergence. $\lambda_\text{alig}$ is set to 1000 for convection and reaction equations, and auto-adapted by $\lambda_\mathcal F$ for wave equation.
%Loss weights are also actively adapted by neural tangent kernel~\cite{wang2022and} for wave equations for test the orthogonality of PINNMamba with other methods.
All experiments are implemented in PyTorch 2.1.1 and trained on an NVIDIA H100 GPU.  More training details are in Appendix~\ref{apx:hyperparam}. Our code and weights are available at \url{https://github.com/miniHuiHui/PINNMamba}.

\textbf{Metrics.} To evaluate the performance of the models, we take relative Mean Absolute Error (rMAE, a.k.a  $\ell_1$ relative error) and relative Root Mean Square Error (rRMSE, a.k.a $\ell_2$ relative error) following common practive~\cite{zhao2024pinnsformer,wu2024ropinn}. The metrics are formulated as:
\begin{align}
\text { rMAE }(\hat u)&=\frac{\sum_{n=1}^N\left|\hat{u}\left(x_n, t_n\right)-u\left(x_n, t_n\right)\right|}{\sum_{n=1}^{N}\left|u\left(x_n, t_n\right)\right|}, \\
\text { rRMSE }(\hat u)&=\sqrt{\frac{\sum_{n=1}^N\left|\hat{u}\left(x_n, t_n\right)-u\left(x_n, t_n\right)\right|^2}{\sum_{n=1}^N\left|u\left(x_n, t_n\right)\right|^2}},
\end{align}
where N is the number of test points, $u(x,t)$ is the ground truth solution, and $\hat u(x,t)$ is the model's prediction.

\vspace{-2mm}

\subsection{Main Results}
\vspace{-1mm}
We present the rMAE and rRMSE for approximating convection, reaction and wave equation's solution in Table~\ref{tab:diff}. Our model consistently outperforms other model architectures, achieving new state-of-the-art.
Notably, as shown in Fig.~\ref{fig:conv}, for the convection equation, PINNMamba allows sufficient propagation of information about the initial conditions, whereas on all the other models there is a varying degree of distortion in the time coordinates.
    As shown in Fig.~\ref{fig:reac}, PINNMamba can further optimize at the boundary, resulting in a lower error than KAN and PINNsFormer for reaction equations. For problems as intrinsically difficult to optimize as the wave, as in Fig.~\ref{fig:wave}, PINNMamba effectively combats simplicity bias and aligns the scales of multi-order differentiation, and thus achieves significantly higher accuracy. This illustrates that PINNMamba can be effective against PINN's failure modes. It's also worth noting that, PINNMamba has the lowest number of parameters (except KAN), while achieving consistently the best performance.

\begin{table}
\vspace{-3mm}
  \caption{Integrating PINNMamba with advanced training strategies and loss auto-balancing strategy. The rMAE is reported here.}
  
  \centering
    \small
  \begin{tabular}{lccc}

    \toprule 
    Method & Convection & Reaction & Wave\\
   \midrule
   PINNMamba & 0.0188 & 0.0094 & 0.0197\\
   +gPINN & 0.0172& 0.0123 & 0.0264 \\
   +vPINN & 0.0236 & 0.0092& 0.0169\\
   +RoPINN & 0.0102& 0.0099& 0.0121\\
    \midrule
    +NTK &0.0179& 0.0079& 0.0147\\
    +NTK+RoPINN &0.0127& 0.0072& 0.0106\\
   

    \bottomrule
  \end{tabular}
  \normalsize
  \label{tab:para}
  \vspace{-6mm}
\end{table}

\begin{figure*}[t!]
    \centering
    \includegraphics[width=\textwidth]{_fig/reac}
    \vspace{-8mm}
    \caption{The ground truth solution, prediction (top), and absolute error (bottom) on reaction equations.}
    \label{fig:reac}
    \vspace{-5mm}
  %  \vspace{-1mm}
\end{figure*}


\subsection{Combination with Other Methods}
\vspace{-1mm}
Since PINNMamba mainly focuses on model architecture, it can be integrated with other methods effortlessly. 
    We explore the feasibility and their performance in combination with advanced training paradigm, as well as loss balancing.

\textbf{Training Paradigm.} We show the rMAE of PINNMamba when integrated with advanced strategies in Table~\ref{tab:para}. We observe that gPINN~\cite{yu2022gradient} and vPINN~\cite{kharazmi2019variational} erratically deliver some performance gains on some tasks. 
    This is due to the fact that the regularization provided by gPINN and vPINN in the form of a loss function through the gradient and variational residuals has little effect on PINNMamba, since SSM itself is sufficiently regularized. RoPINN~\cite{wu2024ropinn} reduces the PINNMamba's error on convection and wave equations by about 40\%, since it complements the spatial continuity dependency.

\textbf{Neural Tangent Kernel.} Dynamic tuning of losses via Neural Tangent Kernel(NTK)~\cite{wang2022and} has been shown to have the effect of smoothing out the loss landscape. 
PINNMamba also works well with the NTK-adopted loss function. As shown in Table~\ref{tab:para}, NTK can reduce PINNMamba error by 5-25\%. 
The combination of RoPINN and NTK can further improve the overall performance of PINNMamba, which demonstrates the excellent suitability of PINNMamba with other PINN optimization methods.

\begin{figure}[t!]
    \centering
    \includegraphics[width=\linewidth]{_fig/loss_error}
    \vspace{-4mm}
    \caption{Loss and $\ell_1$-Error Curve w.r.t Training Iteration.}
    \label{fig:losserror}
    \vspace{-4mm}
  %  \vspace{-1mm}
\end{figure}
\vspace{-2mm}
\subsection{Loss-Error Consistency Analysis}
\vspace{-1mm}

Our other interest is the role of PINNMamba for the elimination of simplicity bias. Models affected by simplicity bias that fall into over-smoothing solutions will show inconsistent decreasing trends in loss and error during training. 
    As shown in Fig.~\ref{fig:losserror}, in the training process for solving convection equations, the rMAE of PINN doesn't descend as $\mathcal L_\mathcal F$ and $\mathcal L_\mathcal I$. 
        This suggests that PINN is trapped in an over-smoothing solution, which is in agreement with our observation in Fig.~\ref{fig:conv}. 
As a comparison, we find that PINNMamba's losses descent processes show a high degree of consistency with its error descent process. 
    This indicates that PINNMamba does not tend to fall into a local optimum of oversimplified patterns.
        Instead, it tends to exhibit patterns that are consistent with the original PDEs.

\vspace{-2mm}
\subsection{Ablation Study}
\vspace{-1mm}
\begin{table*}
  [t]
  \centering
  \resizebox{\textwidth}{!}{%
  \begin{tabular}{cccccccccccc}
    \toprule \multicolumn{2}{c}{Components}                                                             & \multicolumn{5}{c}{Re-executability Rate (\%)} & \multicolumn{5}{c}{Readability (\#)} \\
    \cmidrule(lr){1-2} \cmidrule(lr){3-7} \cmidrule(lr){8-12}        \hspace{8pt}\labelemoji\hspace{8pt}                                                                & \hspace{8pt}\toolemoji\hspace{8pt}                                      & O0                                 & O1             & O2             & O3             & AVG            & O0             & O1             & O2             & O3             & AVG            \\
    \hline
    \rowcolor[rgb]{0.93,0.93,0.93}\multicolumn{12}{c}{\textbf{Initialize with LLM4Decompile-End-6.7B~\citep{llm4decompile}}}   \\
    \xmark                                                                                              & \xmark                                    & 69.51                              & 46.95          & 50.61          & 46.34          & 53.35          & 3.98 & 3.41 & 3.44 & 3.38 & 3.55 \\
    \cmark                                                                                              & \xmark                                    & 75.61                              & 50.61          & 50.00          & 50.00          & 56.55          & 4.01 & 3.44 & 3.39 & \textbf{3.49} & 3.58 \\
    \xmark                                                                                              & \cmark                                    & 83.54                     & \textbf{56.10}          & 51.22          & 50.61 & 60.37 & 4.05 & 3.51 & 3.51 & 3.42 & 3.62 \\
    \cmark                                                                                              & \cmark                                    & \textbf{85.37}                            & \textbf{56.10}                     & \textbf{51.83} & \textbf{52.43}          & \textbf{61.43} & \textbf{4.13} & \textbf{3.60} & \textbf{3.54} & \textbf{3.49} & \textbf{3.69} \\

    \rowcolor[rgb]{0.93,0.93,0.93}\multicolumn{12}{c}{\textbf{Initialize with Deepseek-Coder-6.7B-base~\citep{deepseekcoder}}} \\
    \xmark                                                                                              & \xmark                                    & 59.15                              & 35.98          & 39.02          & 37.80          & 42.99          & 3.71 & 3.05 & 3.16 & 3.05 & 3.24 \\
    \cmark                                                                                              & \xmark                                    & 66.46                              & 41.46          & 38.41          & 36.59          & 45.73          & 3.76 & 3.17 & \textbf{3.21} & 3.08 & 3.31 \\
    \xmark                                                                                              & \cmark                                    & 70.73                              & 39.63          & 39.02          & 40.24          & 47.41          & 3.90 & 3.17 & 3.08 & 3.11 & 3.31 \\
    \cmark                                                                                              & \cmark                                    & \textbf{79.88}                     & \textbf{45.73} & \textbf{43.90} & \textbf{42.68} & \textbf{53.05} & \textbf{3.96} & \textbf{3.21} & 3.18 & \textbf{3.19} & \textbf{3.38} \\
    \bottomrule
  \end{tabular}%
  }
  \caption{The ablation study of different methods across four optimization levels
  (O0, O1, O2, O3), as well as their average scores (AVG). The results in bold represent the optimal performance. The ~\labelemoji~ and ~\toolemoji~ means Relabedling and Function Call. \textbf{Bold} denotes the best performance.}
  \label{tab:ablation}
\end{table*}

To verify the validity of the various components of the PINNMamba, as shown in Table~\ref{tab:ablation}, we evaluate the performance of models subtracting these components from PINNMamba.

\textbf{Sub-Sequence.} We remove the sub-sequence alignment, which leads to a decrease in model performance, indicating the significance of the agreement formed through alignment in eliminating simplicity bias.
After replacing the sub-sequence with a long sequence of the entire domain, the model shows failure modes, in line with the sequence granularity analysis in Section~\ref{sec:subseq}.

\textbf{Time-Varying SSM.} We replace the selective SSM~\cite{gu2023mamba} with a linear time-invariant structure SSM~\cite{gu2022efficiently}, and there is some decrease in model performance, illustrating the role of predictive diversity in eliminating simplicity bias. 
And when we remove SSM completely and switch to MLP instead, the model has severe failure modes. 
        This demonstrates that SSM's adaptation for \textit{Continuous-Discrete Mismatch} allows the initial condition information to propagate sufficiently in time coordinates.

In addition, we also conducted a sensitivity analysis of the choice of sub-sequence length, activation. See Appendix~\ref{apx:sense}.

\vspace{-3mm}
\subsection{Experiments on Complex Problems}
\vspace{-1mm}
To further demonstrate the generalization of our method, we tested our model on partial PINNacle Benchmark~\cite{hao2023pinnacle} and Navier-Stokes equations. As shown in Fig.~\ref{fig:ns}, PINNMamba achieves the lowest error on the N-S equation. Just like PINNsFormer, PINNMamba also gets out-of-memory on some problems in PINNacle, which we identify as a major limitation of sequence-based methods. We discuss the details of PINNacle experiments in Appendix~\ref{apx:comp}.

\begin{figure}[t!]
    \centering
    \includegraphics[width=\linewidth]{_fig/NS}
    \vspace{-6mm}
    \caption{Absolute Error of pressure prediction of N-S equation}
    \label{fig:ns}
    \vspace{-3mm}
  %  \vspace{-1mm}
\end{figure}

\section{Discussions}

% \subsection{Bridge the gap between insights and expressions}



\noindent\textbf{Bridge the gap between insights and expressions with AI-powered domain-focused video creation.}
% video creation for different domains
As images and videos continue to dominate communication mediums, visualization and video technologies have become essential tools for enabling diverse domains and the public to express themselves effectively. Emerging generative AI tools, such as Sora~\cite{sora} and Pika~\cite{pika}, exemplify this trend by facilitating creative expression across various fields.

While general AI-driven video creation tools are increasingly popular, our work emphasizes the critical need for domain-specific video creation tools like \SB{} to address unique requirements within specific fields. There are two primary reasons for prioritizing domain-specific video creation over general generative technologies.
% 
First, domain-specific videos, such as sports highlights, rely heavily on human insights. Audiences seek to learn from professionals through these videos, requiring tools that provide greater user control and enable experts to effectively translate their insights into engaging content. 
% \SB{} supports this by enabling users to maintain control over the conveyed insights, ensuring that the final video accurately reflects expert knowledge and user intentions.
% 
Second, the complexity of domain-specific data, such as the intricate motion and strategy analysis, demands advanced data visualization and seamless synchronization of visuals and audio, which general tools may not provide. 
% \SB{} addresses these needs by providing specialized tools that cater to the detailed and dynamic nature of sports content.

\SB{} addresses these needs by integrating automation with customizable visualizations, tailored to the intricate and dynamic nature of sports content. It allows flexible user control through embedded interactions, 
reducing technical barriers and empowering users to effectively communicate their insights. Feedback from users further underscores the importance of balancing automation with user control to accommodate diverse goals and preferences to enhance accessibility across various user groups and use cases, such as tactical analysis, skill development, and profile building. 
% For instance, professional coaches can use \SB{} to create detailed breakdowns of game strategies for training and coaching. Parents and young athletes can produce polished highlight reels for recruitment.
% These examples illustrate how AI-driven tools can empower users across various levels and industries to create videos with meaningful insights, fostering deeper engagement and broader impact. 

Beyond sports, similar tools have the potential to transform fields like healthcare and education, incorporating precise visual aids and step-by-step breakdowns. 
% These applications highlight the transformative potential of tailored video content in amplifying personal expression and benefiting broader audiences.
% 
Future research is required to investigate the balanced integration of AI and intuitive interface design, such as multi-modal interaction~\cite{wang2024lave}, to further advance domain-specific video creation and expression across diverse fields.
% By continuing to develop and refine domain-specific video creation tools, we can unlock new possibilities for effective communication and expression in numerous fields, ultimately bridging the gap between insights and their visual expressions.

% \subsection{Cross sports visualizations - allow different sports domains to leverage other sports' insights}

% \subsection{Enhance human-AI collaboration - creators focus on content while AI helps with editing tasks}


\vspace{1mm}
\noindent\textbf{Promote visualization in practice through real-world system deployment.}
Our work on SportsBuddy advances existing research in sports visualization and video authoring by emphasizing real-world system deployment and evaluation. Through this study, we have identified two significant benefits.

First, deploying SportsBuddy in authentic environments allowed us to validate and refine our design based on genuine use cases and users, uncovering insights that controlled laboratory settings cannot capture. For instance, we discovered that even within a similar user group of content creators, priorities varied significantly—some focused on showcasing player actions, while others emphasized strategic communication. This diversity led to iterative design improvements that balanced the distinct needs of each user group and support customization without complicating user interactions. 

Second, real-world deployment enables the assessment of long-term impacts and the discovery of unique use cases by diverse users. 
For example, some sports experts were hesitant to adopt SportsBuddy initially despite the perceived usefulness they shared. Upon further investigation, this was due to the context-switching costs. This feedback highlighted the necessity for a streamlined workflow tailored to the sports domain, leading to our design of batch processing and web import options. In addition, we observed many users preferred embedded annotation with \Text{} features over typical captions for sharing insights (see Fig.~\ref{fig:case_study}d), suggesting a new form of video storytelling inspired by \SB{}’s design. 
Feedback and insights from our diverse user base has highlighted the value of creating flexible and accessible visualization tools, which offers important external validity of the human-centered system.

This real-world deployment approach not only enhances visualization literacy and accessibility but also ensures that innovative designs translate into practical, widely usable tools, providing a validation for interactive visualization design. Therefore, we advocate for more visualization research to focus on real-world system deployments and to share design learnings, inspiring use cases that are both practical and impactful.

{
\subsection{Future Work}

While SportsBuddy has shown great potential in simplifying sports video storytelling, 
there are key areas for further improvement:

\vspace{1mm}
\noindent\textbf{Enhancing Player Tracking Under Occlusion and Motion Changes.}
The current tracking system faces challenges with occlusions and rapid motion in dynamic scenarios. Future work will refine tracking algorithms using larger domain-specific datasets and multi-view setups to improve accuracy in complex environments.

% The current tracking system struggles with occlusions and rapid motion changes in crowded or dynamic scenarios. Future efforts will focus on refining tracking algorithms using more extensive domain-specific datasets and, where feasible, incorporating multi-view camera setups for improved accuracy. These enhancements aim to ensure reliable tracking in complex sports environments.

\vspace{1mm}
\noindent\textbf{Addressing Perspective and Camera Movement.}
Shifts in camera angles or perspectives cause misalignment issues due to reliance on fixed transformation matrices. Dynamic court mapping and machine learning for real-time adjustments, along with camera metadata integration, will ensure consistent and accurate visualizations.

% Misalignment issues arise when camera angles or perspectives shift, as the system relies on a fixed transformation matrix. Future work will explore dynamic court mapping techniques and machine learning methods for real-time adjustments. Incorporating camera metadata will further enhance visualization accuracy, ensuring effects remain consistent with the game’s context.

\vspace{1mm}
\noindent\textbf{Supporting Longer Videos.}
Longer or higher-resolution videos can strain browser performance. To mitigate this, we will implement dynamic video loading from cloud storage and on-demand decoding, and adopt frame compression during previews to further optimize memory usage and rendering, ensuring smoother video processing.
% Longer or higher-resolution videos may strain browser performance. To address this, dynamic video loading from cloud storage and on-demand decoding will be introduced. Additionally, frame compression during previews will reduce memory usage and rendering time, enabling smoother processing of large and complex videos.



\vspace{1mm}
\noindent\textbf{Extending to Other Sports.}
\SB{} currently focuses on basketball but can expand to sports like soccer and tennis. This requires adapting tracking algorithms and designing sport-specific visualizations to accommodate the unique dynamics and storytelling needs of each sport.

}


% We advocate for more visualization paper that focus on deplyong system in real-world and evaluate their usage for two reasons. 
% 1. In vis research, application paper often address specific domain problems and create a prototype to evaluate with domain experts in a controlled setting. Most projects stop after user evaluation in the lab and the paper is published. With visualization system in real-world that value the practicality of system design and deployment in the wild, it encourages promoting real-world impact brought by novel visualization design, which is crucial in the current visualization community as we promote literacy and accessiblity of visualizations.
% 2. we should also promote long term impact of visualization design, and identify real-wordl use case and learning that might be drastically different from design study that are typically in lab, with a small amount of users, typically university students or academic members.


\section{Conclusion}

We presented \sys, a sparsity-adaptive attention mechanism for efficient long-context LLM inference. Unlike fixed token budget methods, \sys dynamically selects tokens based on cumulative attention scores, adapting to variations in attention sparsity. By leveraging clustering-based sorting and distribution fitting, \sys accurately estimates token importance with low overhead. Our results showed that \sys outperforms existing sparse attention methods, achieving higher accuracy and significant inference speedups, making it a practical solution for long-context LLMs.




\bibliography{references}
\bibliographystyle{IEEEtran}
\end{document}
