\section{Related Works}

Advancing the capabilities of \ac{DMS} requires addressing challenges in sensing and control architectures. This section reviews existing approaches to sensing in \ac{DMS} and explores the potential of \ac{NCA} as a scalable and robust solution for decentralized systems.

\subsection{Sensing in Distributed Manipulation Systems}

\ac{DMS} have traditionally relied on open-loop control for object manipulation \cite{bohringer_sensorless_1995, georgilas_cellular_2015, liu_robotic_2021, liu_micromachined_1995}. Such systems often approximate programmable vector fields to achieve sensor-less object manipulation \cite{bohringer_theory_1994, bohringer_part_2000, patil_linear_2023}.


When sensing is incorporated, single-camera systems remain the most prevalent approach. Cameras provide rich environmental information, enabling object localization, pose estimation, and obstacle avoidance.
Their widespread availability, large sensing area, and relatively high information density make them an attractive choice for \ac{DMS} \cite{reznik_cmon_2001, ataka_design_2009, georgilas_cellular_2015}. Beyond standard RGB cameras, some \ac{DMS} incorporate depth cameras \cite{follmer_inform_2013}, depth cameras and RGB cameras \cite{uriarte_control_2019}, or marker-based systems \cite{xu_modular_2024} for accurate 3D tracking, particularly pertinent for systems capable of large vertical displacement. %

In contrast, other \ac{DMS} utilize localized sensing mechanisms to detect object interactions. Resistive tactile force sensors mounted on the manipulator end-effector are commonly used to measure contact forces, offering a robust and cost-effective solution \cite{robertson_compact_2019, xue_arraybot_2023}. 
Photodiodes \cite{berlin_motion_2000, ataka_layer-built_2007, bedillion_distributed_2013} and proximity sensors \cite{fukuda_hybrid_2000} have also been employed to detect position of objects.  Some systems combine local sensing with external camera-based perception to fuse modalities for enhanced object tracking \cite{parajuli_actuator_2014}. The low cost of tactile sensors makes them viable for high-density deployment, facilitating spatially distributed sensing. %
However, current \ac{DMS} designs often employ low-density, discontinuous sensor configurations, typically using a single sensor per actuator, which restricts their capacity to effectively capture the object dynamics in detail.



Effective \ac{DMS} sensing systems should provide precise local sensing with high spatial density to accurately capture global object dynamics. Scalability demands cost-effective sensors for large-scale deployment, along with architectures  capable of handling inevitable sensor failures. In this work, we propose such a system utilizing a novel inductive smart sensor surface, able to provide high precision localized sensing. We then implement a \acf{NCA} based a architecture, facilitating a purely decentralized sensing framework, that is robust to failures and capable of scaling to any scale.



\subsection{Neural Cellular Automaton}

\acf{CA} consist of a regular grid of cells, each occupying one of a limited set of states. The state of a cell is updated based on its own current state and that of its neighbors, according to predefined rules. Despite the simplicity of the local rules, \ac{CA} have been shown to give rise to very complex emergent behaviors, exemplified by Conway’s Game of Life\cite{adamatzky_game_2010}. However, crafting rules for specific behaviors is non-trivial, prompting a shift toward discovering rule sets through automated approaches \cite{wolfram2021problem}.

Recently, deep learning has been integrated into \ac{CA} to learn rule sets that drive desired behaviors through gradient-based optimization of a loss function \cite{ha_collective_2022, gilpin_cellular_2019}. Neural Cellular Automata merge \ac{CA} principles with \ac{NN}, representing the \ac{CA} update function $f_{\theta}$ as a network taking as input the agents neighborhood. A unique feature of \ac{NCA}s is the use of hidden channels in each cell’s state in which free tokens of information are stored for inter-agent communication \cite{wulff_learning_1992,mordvintsev_growing_2020}.

\acp{NCA} have been applied to a variety of tasks requiring global property estimation from local interaction. 
For instance, \cite{randazzo_self-classifying_2020} demonstrated the use of \acp{NCA} to classify MNIST digits through consensus among agents. Similarly, \cite{nadizar_fully-distributed_2023} proposes a shape-aware controller where each module infers the shape of a larger assembly. \cite{walker_physical_2022} extended this concept to modular robotics, enabling systems to infer their own shapes through local communication, and successfully transitioned from simulation to hardware.\cite{bessone_neural_2025} introduced a methodology for inferring the geometric center of objects laying on a grid of sensing agents. Leveraging the capabilities of \acp{NCA}, each agent locally shares information within its neighborhood, enabling the inference of global properties through purely local communication.
This paper builds upon these findings to address the challenges of decentralized sensing in \ac{DMS}, introducing a system that bridges the gap between simulation and real-world hardware.
