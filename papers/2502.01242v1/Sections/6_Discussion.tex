\section{Discussion}

\subsection{Centralized vs Decentralized sensing}

There are fundamental trade-offs between centralized and decentralized architectures for global property estimation. Our decentralized \ac{NCA}-based approach performs comparatively to a centralized baseline approach. Centralized systems necessitate all sensor data is aggregated, creating a computational bottleneck and a single point of failure, which our system avoids. However, this does allow rapid access to information from across the system, without the need to wait for information propitiation via local communication, which is costly both in time and computational overhead. This delay can be seen in the non-instantaneous nature of generating our system's estimate. Therefore, our system may struggle to achieve optimal results in the case of highly dynamic environments where object properties of position varied rapidly. 

Decentralized perception systems also face complexity challenges, as they lack direct access to global information and cannot implement straightforward global algorithms for property estimation. While decentralized variants of such algorithms have been explored \cite{bedillion_distributed_2013}, they incur additional computational costs. 
Despite these challenges, the scalability and robustness inherent to decentralized architectures make our \ac{NCA}-based approach a promising candidate for large-scale \ac{DMS}, particularly in harsh environments with high noise levels or a high likelihood of sensor failure. 

\subsection{Sensor Calibration}

 The similarity of models trained on both calibrated and uncalibrated sensor data suggests that, in this context, the calibration process may not substantially influence the system’s estimation capabilities. If calibration is unnecessary, this reduces the complexity of setup for any practical deployments. However, although we have shown this state to be the case in this context, this state does not necessarily hold in general. Sensors manufactured with a higher manufacturing variability (i.e. not manufactured as part of the same sensor board) or operating in substantially different conditions, which are not accounted for without calibration. Soft sensors may be especially susceptible to these variations,  due to the large mechanical deformations they experience in use. %





Results from the noise tolerance experiments provide insights into valid operational conditions. By examining how the system responds to controlled variations in sensor input quality, it is possible to approximate the influence of different calibration distributions on estimation error, thus providing a measure of confidence in the model’s robustness when used with diverse sensor configurations.


\subsection{System Robustness}

The system’s fault tolerance experiment reveals that meaningful performance is maintained even when up to $30\%$ of the sensors are rendered non-functional, and a robustness up to a $50\%$ signal-to-noise ratio indicates a reliability to the system. In any real world implementation, such noise and sensor failures are inevitable, especially as the number of sensors increases. Such deviation in sensor readings would also be expected on a non-static system, as in a \ac{DMS} actuator, in which measured pressure applied to the sensor board would vary as the system moves. Therefore, this consistency in estimation offers validation for the utility of such a system for real world application.
The removal of a centralized processing unit significantly enhances robustness by eliminating a single point of failure. The consensus mechanism inherent to \ac{NCA}s ensures that the system remains robust even when individual agents receive corrupted information. This resilience to sensor failures and noisy inputs is critical for real-world applications, where unpredictable environmental factors and hardware degradations are common.
 

\subsection{Scalability}
One of the most notable advantages of the proposed \ac{NCA}-based approach is its scalability. The system's performance remains consistent across varying network sizes by relying solely on local information and decentralized decision-making. Unlike centralized architectures, this system imposes no size limitations due to computational overhead, allowing for theoretically unlimited scaling. 
However, as the number of agents increases, so does the time for information to propagate via local communication. This limitation could potentially be exploited, prioritizing sensing and manipulation for objects in the immediate vicinity of an agent in the case of mutli-object manipulation.
Additionally, the ability to implement a single agent on multiple system scales without need for retraining offers a massive benefit for practical deployment. This locality also indicates such a architecture would be well suited for modular designs, where the sensing surface can be easily adapted with little need for model retraining.

\section{Future Work}

\subsection{Non-static objects and surfaces}
While the experiments in this study focused on static objects, real-world applications in object manipulation involve dynamic environments. A static approximation may hold with a sufficiently high sampling rate relative to object speed.
However, challenges arise from the time required for information propagation and consensus formation in decentralized systems.
When integrated as a robotic end effector, the sensor will be non-static and will experience induced force, altering estimation. Even if held static, the surface will experience pose variations that influence contact forces. While traditional mechanics can estimate these forces, additional complexities, such as soft-surface compression under non-perpendicular orientations to gravity, must be considered. Although noise robustness testing offers some insight into the behavior behavior of the current system, addressing these effects will require further experimentation and enhanced model training.



\subsection{Tunable Material}
The materials of the soft surface influence both sensor characteristics and manipulation capabilities. Tailoring these properties for specific applications, such as optimizing deformation to maximize the dynamic sensing range for objects with known properties, holds significant potential. Furthermore, varying material properties across the surface could enable new sensing and manipulation strategies, which we intend to explore in future work.




