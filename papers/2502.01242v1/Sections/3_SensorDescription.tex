\section{Soft Inductive Sensing Platform} %

To satisfy a \acp{DMS} requirements for high precision local sensing at the actuator level, we developed a soft inductive sensing layer that serves dual purposes: enabling object manipulation as the system's end effector and collecting detailed tactile data about objects interacting with the surface. 


\subsection{Design Overview}

The sensor design employs inductive sensors positioned bellow a compliant surface embedded with ferromagnetic material. Deformation of the soft material under load changes the distance between sensor and ferromagnetic material, causing a measurable change in inductance proportional to deformation.
A \ac{PCB}, containing a lattice of coils connected to inductive signal conditioning chips, form an array of inductive sensors. This \ac{PCB} forms the base layer for a soft structure comprising a ferromagnetic sheet encapsulated within lightweight, compliant polyurethane foam (Poron\textregistered) and topped with a smooth, FDA-approved \ac{PET} film. This single soft structure allows these sensors to be combined into a continuous soft sensing surface. This surface is well suited as a sensorized \ac{DMS} end-effector, and for industrial applications like food handling and packaging, due to its ruggedness and resilience against dust and moisture. 
This architecture allows for low-cost scalability. The number, arrangement, shape, and spacing of coils within each \ac{PCB} configurable to application need.





\subsection{Prototype board design and manufacture}

 The sensor design utilizes a LDC1614 inductive signal conditioning chip and paired coil as an inductive sensor, following the procedure in \cite{lo_preti_sensorized_2023}. For use in in our experimental testing, a FR4 \ac{PCB} hosting 16 embedded inductive coils arranged, each measuring $30 \times 30$~mm, were arranged in a in a $4 \times 4$ grid. The tactile sensor prototype uses four LDC1614 connected to two I2C lines, with each chip polling four coils and two chips per I2C line. This configuration allows precise measurements at up to the kHz range. To optimize performance, traces between the coils and driver chips are isolated to enhance the signal-to-noise ratio, and the board is shielded from electromagnetic interference using an additional ferrite layer beneath the \ac{PCB}. In this prototype, we connect the sensors to a Teensy\textregistered micro-controller from which to read sensor data via I2C.

The sensor's structure and material selection is illustrated in \cref{fig:soft_sensor}.
The multilayer tactile structure above the \ac{PCB} integrates two Poron\textregistered\ foam layers (1mm and 4.7mm respectively) with a ferrite sheet (0.2mm) to maximize compliance and sensitivity. 
The layers are bonded with Sil-Poxy\textregistered adhesive.
The \ac{PET} top layer (0.1mm) provides a smooth, low-friction contact surface, ideal for handling soft materials and reducing wear. Laser-cut plexi glass masks were used for precise assembly and to ensure manufacturing consistency.

\begin{figure}[t]
    \centering
    \includegraphics[width=0.85\columnwidth]{Figures/soft_sensor_exploded.png}
    \caption{Exploded schematic of the soft sensor prototype, showing the PCB, ferrite, Poron\textregistered, and PET layers.}
    \label{fig:soft_sensor}
\end{figure}

\subsection{Characterization and Testing}

A three-axis indentation setup tested the tactile sensing layer prototype. Two micrometric manual linear stages controlled the positions of the X and Y stages. An M-111.1DG translation stage was positioned along the Z-axis on top of the X stage, controlled by the paired C-884.4DC motion controller (Physik Instrumente, USA). An ATI Nano17 (ATI Industrial Automation, USA) load cell was mounted on the Z stage with an L-shaped part to adjust its configuration, under which an ABS probe with a round tip shape was attached to the load cell. The prototype was placed on a lab jack (MKS Instruments, Inc.) and fixed with a base designed to locate the four testing positions. 

Key performance metrics, including repeatability, range, crosstalk, RMSE, hysteresis, and sensitivity, were analyzed for all 16 inductive coils under controlled loading conditions. Repeatability in all coils was measured at 68.22~\% with a standard deviation of 27.33~\%. The maximum force range was 6.72~N $\pm$ 0.78~N, and crosstalk between neighboring coils was minimal, measured at 1.57~\% $\pm$ 1.37~\%.

To generate calibration curves, we applied a uniform pressure to the (30 \(mm^2\)) area above each sensor via incremental loading, applying between 0.04905 - 5.886~N of force, equivalent to 5-600~g of applied mass. 
Polynomial curve fitting per coil found each coil's response to be predominately linear, though distinct due to edge effects effects and inbuilt manufacturing variability.






