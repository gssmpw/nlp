\section{Introduction}
A \ac{DMS} consists of numerous actuators, often arranged in a lattice topology, that work collaboratively to manipulate objects on its surface. Through coordinated use of many actuators, these systems can achieve precise positioning, orientation, and manipulation of objects.
This collaborative approach offers capabilities that surpass those of systems relying on independent actuators, which are often constrained by limitations in, e.g., force generation, stabilization, and actuation range \cite{bohringerDistributedManipulation2000}. Due to their spatial distribution, \ac{DMS} cover a large workspace for manipulation and the possibility of parallel manipulation \cite{reznik_cmon_2001, ataka_design_2009}.


Accurate knowledge of an object's location is crucial for orchestrating manipulation in a \ac{DMS}  using closed-loop control \cite{luntz_distributed_2001}. Estimation of an object's \ac{GC} provides a low-dimensional representation of its position, equivalent to the center of mass for homogeneous objects. This information is critical for accurate manipulation through the application of external force. Many \ac{DMS} utilize a centralized sensing system, such as single-camera computer vision systems \cite{murphey_feedback_2004}. However, centralized architectures introduce vulnerabilities: failure of a single sensor can compromise the entire system. Moreover, such sensing systems are external to the manipulator, and the need for specialized processing hardware for signal processing (e.g. FPGA), impairs their integration greatly \cite{ataka_layer-built_2007}.

Decentralization offers a promising alternative for \ac{DMS}, addressing many of the limitations of centralized systems. By distributing the computational load of sensing and control across multiple components, decentralization mitigates the scalability challenges posed by centralized control, which often suffer from computational, processing, and communication bottlenecks \cite{agarwal_velocity_1998}. Additionally, decentralization enhances robustness by eliminating the existence of a single point of failure. 
However, decentralized systems face their own challenges, particularly in coordinating interactions between agents. Inter-agent communication overhead can lead to latency and complicate real-time interactions among manipulators. %

Localized sensing can provides precise information about object-manipulator interactions at specific known points. Information of such interactions is key for accurate manipulation through controlled force application. However, such locality is absent in camera based perception system. When localized sensors are utilized in a high spatial density, the system is able to detect objects in contact with multiple sites simultaneously, allowing for calculation of  locla distributions of the sensed modality, for example pressure maps. The use of multiple sensors also enhances system robustness by preserving functionality, even in the case of individual sensor failures or noise. Despite this, existing \ac{DMS} typically implement one sensor per actuator in low density, discontinuous arrangements, limiting their ability to capture global object dynamics effectively.





Recognizing the critical need for decentralization, scalability, and robustness in \ac{DMS}, this work offers two main contributions. First, We propose a novel inductive sensor design tailored for distributed sensing, offering a scalable, fault-tolerant approach to sensing. Second, we investigate how an \acf{NCA} based system utilizing this sensor can estimate global properties from purely local information, and how such a system is scale invariant and robust to fault, both key factors for producing \ac{DMS} of any size.





