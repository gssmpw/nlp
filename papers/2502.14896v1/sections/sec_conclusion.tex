
% \qnote{I feel we may miss one paragraph on how concept erasure relate to other similar topics like unlearning? }

\section{Future Research Directions} \label{sec:future_research}

Concept erasure techniques for T2I models have demonstrated promising results in mitigating the generation of undesired content. However, several open challenges remain. In this section, we outline key future research directions that can further advance the development of responsible and robust generative models.

\paragraph{Extending to Other Modalities:} While this survey primarily focuses on concept erasure for T2I models, extending these techniques to other generative modalities—such as text-to-video and text-to-audio—remains an open challenge. Despite structural similarities, these modalities introduce additional complexities, such as temporal consistency in videos and waveform coherence in audio. Understanding how concept erasure methods generalize across different modalities and identifying modality-specific adaptations will be crucial for developing responsible generative AI systems.

\paragraph{Designing More Comprehensive Benchmarks:} Existing evaluations primarily assess erasure effectiveness by detecting the presence of erased concepts in generated images. However, the broader impact of concept erasure on non-target concepts remains under-studied. For instance, erasing the concept of ``Van Gogh'' may inadvertently alter representations of other Impressionist artists. Future benchmarks should move beyond dataset-level assessments and introduce fine-grained, concept-level evaluations that capture unintended effects on related visual attributes. Establishing standardized, interpretable, and reproducible benchmarks will enable more rigorous comparisons across different erasure techniques.

\paragraph{Enhancing Robustness Against Adversarial Attacks:} Current concept erasure methods remain vulnerable to adversarial attacks that exploit weaknesses. While various defensive strategies have been proposed, adversarial attacks continue to evolve, exposing new vulnerabilities. Future work should focus on developing more robust and adaptive defenses that generalize across different attack strategies. This may include integrating adversarial training, certifiable robustness techniques, and multi-modal alignment to mitigate adversarial circumvention while preserving model fidelity.

By addressing these challenges, future research can contribute to the development of more responsible, robust, and generalizable concept erasure techniques for T2I models and beyond.


\section{Conclusion} \label{sec:conclusion}
Concept erasure in Text-to-Image (T2I) models is crucial for ensuring ethical and legal compliance in generative AI. This survey categorizes existing approaches based on their optimization strategies and modified model components, covering fine-tuning, closed-form solutions, and inference-time interventions. We also discuss adversarial attacks and emerging defenses. Despite progress, challenges remain in balancing erasure effectiveness, model fidelity, and adversarial robustness. Expanding concept erasure to other modalities, refining evaluation benchmarks, and developing stronger defenses are key directions for future research. We hope this survey serves as a foundation for advancing responsible and secure generative AI.



