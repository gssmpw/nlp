%%%% ijcai25.tex

\typeout{IJCAI--25 Instructions for Authors}

% These are the instructions for authors for IJCAI-25.

\documentclass{article}
\pdfpagewidth=8.5in
\pdfpageheight=11in

% The file ijcai25.sty is a copy from ijcai22.sty
% The file ijcai22.sty is NOT the same as previous years'
\usepackage{ijcai25}

% Use the postscript times font!
\usepackage{times}
\usepackage{soul}
\usepackage{url}
\usepackage[hidelinks]{hyperref}
\usepackage[utf8]{inputenc}
\usepackage[small]{caption}
\usepackage{graphicx}
\usepackage{amsmath}
\usepackage{amsthm}
\usepackage{booktabs}
\usepackage{algorithm}
\usepackage{algorithmic}
\usepackage[switch]{lineno}
\usepackage{amsfonts}
\usepackage{bbm}

\usepackage{xcolor}


\usepackage{pifont}  % Required for checkmark and cross
\usepackage{adjustbox}
\usepackage{booktabs} % Better table formatting
\usepackage{multirow}


% Define checkmark and crossmark symbols
\newcommand{\cmark}{\ding{51}} % ✔
\newcommand{\xmark}{\ding{55}} % ✘


\newcommand{\qnote}[1]{[\textcolor{red}{Q-note: #1}]}

% Comment out this line in the camera-ready submission
% \linenumbers

\urlstyle{same}

% the following package is optional:
%\usepackage{latexsym}

% See https://www.overleaf.com/learn/latex/theorems_and_proofs
% for a nice explanation of how to define new theorems, but keep
% in mind that the amsthm package is already included in this
% template and that you must *not* alter the styling.
\newtheorem{example}{Example}
\newtheorem{theorem}{Theorem}

% Following comment is from ijcai97-submit.tex:
% The preparation of these files was supported by Schlumberger Palo Alto
% Research, AT\&T Bell Laboratories, and Morgan Kaufmann Publishers.
% Shirley Jowell, of Morgan Kaufmann Publishers, and Peter F.
% Patel-Schneider, of AT\&T Bell Laboratories collaborated on their
% preparation.

% These instructions can be modified and used in other conferences as long
% as credit to the authors and supporting agencies is retained, this notice
% is not changed, and further modification or reuse is not restricted.
% Neither Shirley Jowell nor Peter F. Patel-Schneider can be listed as
% contacts for providing assistance without their prior permission.

% To use for other conferences, change references to files and the
% conference appropriate and use other authors, contacts, publishers, and
% organizations.
% Also change the deadline and address for returning papers and the length and
% page charge instructions.
% Put where the files are available in the appropriate places.


% PDF Info Is REQUIRED.

% Please leave this \pdfinfo block untouched both for the submission and
% Camera Ready Copy. Do not include Title and Author information in the pdfinfo section
\pdfinfo{
/TemplateVersion (IJCAI.2025.0)
}

\title{A Comprehensive Survey on Concept Erasure in \\ Text-to-Image Diffusion Models}


% Multiple author syntax (remove the single-author syntax above and the \iffalse ... \fi here)
% \author{
% Changhoon Kim\thanks{Work done while the author was a Postdoctoral Scientist at Amazon.}$^{1,2}$ 
% \And
% Yanjun Qi$^1$
% \\
% \affiliations
% $^1$Amazon Bedrock Science\\
% $^2$Soongsil University\\
% \emails
% \{chkimm, yanjunqi\}@amazon.com,
% changhooon.kim@gmail.com
% }


\author{
Changhoon Kim 
\And
Yanjun Qi
\\
\affiliations
Amazon Bedrock Science\\
\emails
\{chkimm, yanjunqi\}@amazon.com,
changhooon.kim@gmail.com
}


\begin{document}

\maketitle

\begin{abstract}
Text-to-Image (T2I) models have made remarkable progress in generating high-quality, diverse visual content from natural language prompts. However, their ability to reproduce copyrighted styles, sensitive imagery, and harmful content raises significant ethical and legal concerns. Concept erasure offers a proactive alternative to external filtering by modifying T2I models to prevent the generation of undesired content. In this survey, we provide a structured overview of concept erasure, categorizing existing methods based on their optimization strategies and the architectural components they modify. We categorize concept erasure methods into fine-tuning for parameter updates, closed-form solutions for efficient edits, and inference-time interventions for content restriction without weight modification. Additionally, we explore adversarial attacks that bypass erasure techniques and discuss emerging defenses. To support further research, we consolidate key datasets, evaluation metrics, and benchmarks for assessing erasure effectiveness and model robustness. This survey serves as a comprehensive resource, offering insights into the evolving landscape of concept erasure, its challenges, and future directions.

\end{abstract}


\section{Introduction}
Safety is a critical consideration in various applications, including robots, autonomous vehicles, smart grids, and transportation control systems~\cite{wolf2017safety}. These safety-critical scenarios demand formal guarantees to ensure that systems operate as expected, as failures may result in severe consequences, such as harm to humans or significant financial costs. Safety verification refers to the task of determining whether a system satisfies a given safety specification over a specified period~\cite{guiochet2017safety, vicentini2019safety}. 
Conventional safe set specifications primarily focus on spatial requirements, ensuring that the system state never enters an unsafe region~\cite{prajna2004safety}. However, as the complexity of autonomous systems increases, many real-world tasks require specifications that are not only spatial but also temporal in nature. For instance, a mobile robot needs to pass Area A before entering Area B. In this paper, we focus on safety verification under Signal Temporal Logic (STL) specification, which uses both boolean and temporal logic operators to formulate constraints for continuous-valued systems~\cite{maler2004monitoring}. 

Real-world systems are subject to various types of uncertainty. It is essential for safety verification algorithms to account for disturbances. Many existing approaches model these uncertainties as bounded disturbances and employ worst-case analysis to guarantee the satisfaction of safety specifications. Examples of such approaches for safe set specification include Hamilton–Jacobi Reachability (HJ Reachability)~\cite{bansal2017hamilton}, reachability analysis, and barrier certificates~\cite{prajna2004safety}. For STL specifications, methods such as HJ Reachability~\cite{chen2018signal} and reachability analysis~\cite{roehm2016stl, lercher2024using, kochdumper2024fully} have been employed to formally verify STL satisfaction under bounded disturbance inputs.


In many practical situations, disturbances are better modeled as stochastic noise, which provides a more realistic representation, as in the case of sensor noise. When considering stochastic disturbances, the aforementioned deterministic methods are not applicable or tend to be overly conservative, as they focus on worst-case scenarios that rarely occur in practice. To better account for stochastic disturbances, we adopt a probabilistic setting, where the goal is to ensure the safety specification is satisfied with high probability, e.g., greater than 99.9\%. 
For safe set specifications, several methods have been proposed to verify stochastic systems, including martingale-based approaches~\cite{steinhardt2012finite, santoyo2021barrier}, risk estimation~\cite{frey2020collision}, and sampling-based methods~\cite{janson2017monte}. Our recent paper significantly reduces the conservativeness of the verification algorithms for safe set specifications~\cite{liu2024safety}.
For STL specifications, most existing approaches are limited to handling the probability constraint for a single trajectory satisfying the STL specification~\cite{sadigh2016safe, farahani2018shrinking, yang2023distributed, vlahakis2024probabilistic, kordabad2024control}. Very few studies have focused on STL verification under both bounded and stochastic disturbances. In \cite{salamati2021data}, a method is proposed to address this problem for linear systems under Gaussian noise. In this work, we focus on the problem of STL verification for nonlinear systems under both bounded and stochastic disturbances. 

In this work, we present a novel framework for verifying the probabilistic STL satisfaction of discrete-time nonlinear stochastic systems. To the best of our knowledge, this is the first approach capable of addressing this problem for nonlinear systems under both bounded and stochastic disturbances. Given a desired probability requirement, our method first erodes the superlevel set of the predicates in an STL formula to get a tighter STL formula. If the deterministic system is verified to satisfy the tighter STL formula, then the stochastic system is guaranteed to satisfy the original STL formula with the specified probability constraint. As a result, the stochastic verification problem is transformed into a deterministic one. The depth of erosion is determined by the sharp probabilistic bound proposed in our previous work~\cite{liu2024probabilistic}, which helps reduce the conservativeness of the verification result, especially when the probability tolerance is low and the time horizon is long. Our method does not rely on restrictive assumptions, such as linear system dynamics or affine predicates, which is common in previous work~\cite{vlahakis2024probabilistic}. This broader applicability makes our approach suitable for real-world applications.



\textit{Notations.}
% \textit{Vectors, matrices, and probability.} 
Denote by $\real$ and $\n$ the sets of real numbers and nonnegative integers, and define $\n_{[a,b]}=\setb{a, a+1, \dots, b}$ where $a,b\in \n$ and $a<b$. Given a vector sequence $\{x_t\}$, define $\boldsymbol{x}_{[t_1,t_2]} = (x_{t_1}, x_{t_1+1}, \dots, x_{t_2}) = [x\tran_{t_1}, x\tran_{t_1+1}, \dots, x\tran_{t_2}]\tran$ , $t\in\n_{[t_1,t_2]}$. When $X_t$ are random vectors, $\boldsymbol{X}_{[t_1,t_2]}$ is a random process. We use $\bP$ to denote probability. A random vector $X \sim \mathcal{N}(\mu, \Sigma)$ follows a multivariate Gaussian distribution with mean $\mu$ and covariance $\Sigma$.
Given a vector $x\in \real^n$, $\|x\|$ denotes the euclidean norm and $\|x\|_P = \sqrt{x\tran P x}$, where $P\in\real^{n\times n}$ is a positive definite matrix.
% \textit{Sets.} 
The $n$ dimensional ball with radius $r$ and center $y$ is denoted by $\BB^n(r, y)=\setb{x\in \real^n : \|x-y\| \leq r}$. Denote the complement of set $A$ as $\setcomp{A}$ and $-B = \setb{-y: \forall y\in B}$. Given sets $A$ and $B$, define the Minkowski sum of $A$ and $B$ by $A\oplus B = \setb{x+y: x\in A,~ y\in B}$, and the Minkowski difference or Pontryargin difference of $A$ and $B$ by $A\ominus B=\setb{x:x+y\in A, \forall y\in B}$ \cite{kolmanovsky1998theory}. The Minkowski sum and difference satisfy the relation $(A\ominus B)\oplus B \subseteq A$.






% \section{Background}

% In this work, we focus on two different model families: random Fourier features (RFFs) and deep neural networks (DNNs) for transfer learning with informative priors.
% What these model families have in common is that they can be overparameterized.

%\subsection{Random Fourier features}

% MCH: MOVED TO CASE A

%\subsection{Transfer learning with informative priors}

% MCH: MOVED TO CASE B



\section{Proposed Method: Tighter Bounds
\label{sec:proposedmethod}
}

Remark \ref{remark1} suggests that 
it would be useful to tighten the collapsed bound 
%in (\ref{eq:collapsedbound_old}) 
in order to reduce underfitting bias 
and match better exact GP training. 
Remark \ref{remark2} suggests that one way to tighten the bound is to replace %the conditional GP 
$p(\f | \bu)$, in the variational approximation in (\ref{eq:pfuqu}), with another distribution 
that can better approximate 
$p(\f | \bu, \y)$. Next we develop a method that does this while keeping the cost unchanged. 

Let us write the 
exact form of $p(\f | \bu, \y)$. By noting that this quantity is the exact posterior over $\f$ in a GP regression model with joint $p(\y | \f) p(\f | \bu)$ 
%(where $p(\f | \bu)$ is now the effective GP prior) 
we conclude that this %posterior 
is 
$$
p(\f | \bu, \y) = \mathcal{N}\left(\f| \m(\y,\bu), 
(\widetilde{\bK}_{\f \f}^{-1} + \frac{1}{\sigma^2} \bI)^{-1} \right),
$$
where $\m(\y,\bu)  = \E[\f | \bu] + \widetilde{\bK}_{\f \f} (\widetilde{\bK}_{\f \f} + \sigma^2 \bI)^{-1} (\y - \E[\f | \bu]) $
with $\E[\f | \bu] = \bK_{\f \bu} \bK_{\bu \bu}^{-1} \bu$ and $\widetilde{\bK}_{\f \f} = \bK_{\f \f} - \bQ_{\f \f}$. Note that under this notation, 
$p(\f | \bu) = \mathcal{N}(\f | \E[\f | \bu], \widetilde{\bK}_{\f \f})$. We will construct a new $q(\f | \bu)$ 
that keeps the same mean $\E[\f | \bu]$ 
as $p(\f | \bu)$ but it replaces $\widetilde{\bK}_{\f \f}$ with a closer approximation to the 
% exact 
covariance 
% matrix 
$(\widetilde{\bK}_{\f \f}^{-1} + \frac{1}{\sigma^2} \bI)^{-1}$ of $p(\f | \bu, \y)$. We first 
write this %latter 
matrix as 
\begin{equation}
(\widetilde{\bK}_{\f \f}^{-1} + \frac{1}{\sigma^2} \bI)^{-1}
= \widetilde{\bK}_{\f \f}^{\frac{1}{2}}
( \bI + \frac{1}{\sigma^2} \widetilde{\bK}_{\f \f})^{-1} 
\widetilde{\bK}_{\f \f}^{\frac{1}{2}}.
\label{eq:exact_cov_pfuy}
\end{equation}
Then we approximate the inverse 
$( \bI + \frac{1}{\sigma^2} \widetilde{\bK}_{\f \f})^{-1}$ by a diagonal matrix $\bV = \text{diag}(v_1, \ldots,v_N)$ of $N$ variational parameters $v_i > 0$. In other words,  in the initial $q(\f, \bu) = p(\f|\bu)q(\bu)$ we will replace $p(\f|\bu)$ by 
\begin{equation}
q(\f|\bu) = \mathcal{N}(\f | \bK_{\f \bu} \bK_{\bu \bu}^{-1} \bu, (\bK_{\f \f} - \bQ_{\f \f})^{\frac{1}{2}} \bV
(\bK_{\f \f} - \bQ_{\f \f})^{\frac{1}{2}}).
\label{eq:qfu}
\end{equation}
The ELBO now is written as 
\begin{align} 
 & \int q(\f | \bu) q(\bu) \log \frac{p(\y | \f) p(\f | \bu) p(\bu)}{q(\f | \bu) q(\bu)} d \f d \bu = \nonumber \\ 
& \int \! \!  q(\bu) \! \left\{ \! \log \frac{e^{\E_{q(\f | \bu)}[\log p(\y | \f)]} p(\bu)}{q(\bu)} \! - \! \text{KL}[q(\f | \bu) || p(\f | \bu)] 
\! \right\} \! d \bu \nonumber 
\end{align}
and the challenge is to see whether 
$\text{KL}[q(\f | \bu) || p(\f | \bu)]$ 
and $\E_{q(\f | \bu)}[\log p(\y | \f)]$ 
are computable in $\mathcal{O}(N M^2)$ time. 
We have the following results (proofs are in  \Cref{app:detailsNewbounds}).
\begin{lemma}
\label{lem:KLqfupfu}
\emph{$\text{KL}[q(\f | \bu) || p(\f | \bu)] 
= \frac{1}{2} \sum_{i=1}^N (v_i - \log v_i - 1)$}.
\end{lemma}
\begin{lemma} 
\label{lem:Expqfu_loglik}
Let us denote the diagonal elements of \emph{$\bK_{\f \f} - \bQ_{\f \f}$} as 
\emph{$k_{ii} - q_{ii}$} for \emph{$i=1,\ldots,N$}. Then  
\emph{\begin{align}
& \E_{q(\f | \bu)}[\log p(\y | \f)] \nonumber \\ 
& \! = \! \log \mathcal{N}(\y | \bK_{\f \bu}
\bK_{\bu \bu }^{-1} \bu, \sigma^2 \bI)
- \frac{1}{2 \sigma^2} 
\sum_{i=1}^N v_i (k_{ii} - q_{ii}). 
\end{align}}
\end{lemma}
By combining the two lemmas the full bound is written as 
\begin{align} 
& \int \! \!  q(\bu) \log \frac{  \mathcal{N}(\y | \bK_{\f \bu}
\bK_{\bu \bu }^{-1} \bu, \sigma^2 \bI) p(\bu)}{q(\bu)}  d \bu \nonumber \\
& - \frac{1}{2} 
\sum_{i=1}^N \left\{  v_i \left(1 + \frac{k_{ii} - q_{ii}}{\sigma^2}\right) - \log v_i -1 \right\}.  
\label{eq:newcollapsedbound_with_vis}
\end{align}
\begin{proposition}%[new collapsed bound]
Maximizing the bound in (\ref{eq:newcollapsedbound_with_vis}) with respect to \emph{$q(\bu)$}
and each \emph{$v_i$} results in the 
optimal settings \emph{$q^*(\bu) \propto  \mathcal{N}(\y | \bK_{\f \bu}
\bK_{\bu \bu }^{-1} \bu, \sigma^2 \bI) p(\bu)$} and  
\emph{$v_i^* = \left(1 + \frac{k_{ii} - q_{ii}}{\sigma^2} \right)^{-1}$}. By substituting these values 
back to (\ref{eq:newcollapsedbound_with_vis}) we obtain
\emph{\begin{equation} 
\mathcal{F}_{new} \! = \! \log  \mathcal{N}(\y |{\bf 0},   \bQ_{\f \f} + \sigma^2 \bI) 
 - \frac{1}{2}  
\sum_{i=1}^N \log \left(\! 1 + \frac{k_{ii} - q_{ii}}{\sigma^2} \! \right).   
\label{eq:newcollapsedbound}
\end{equation}
}
\label{prop:newbound}
\end{proposition}
The first term is the 
DTC log likelihood as in the original bound in (\ref{eq:collapsedbound_old}),  
but the regularization term 
makes the bound tighter, 
i.e., $\log p(\y) \geq \mathcal{F}_{new} \geq \mathcal{F}$, due to the inequality $\log(a + 1) \leq a$. Also since $\log(a + 1) < a$ for all $a>0$, if $\bK_{\f \f} \neq \bQ_{\f \f}$ 
(so there is at least one $k_{ii} - q_{ii} > 0$), then $\mathcal{F}_{new} > \mathcal{F}$. This means that $\mathcal{F}_{new}$ is strictly better than $\mathcal{F}$ unless both bounds match exactly the log marginal likelihood. 

Clearly, $\mathcal{F}_{new}$ has $\mathcal{O}(N M^2)$ cost and its implementation requires a minor modification to the initial bound. The optimal $q^*(\bu)$
is the same as in the initial SVGP method, while 
an interpretation of the optimal $v_i^*$ values
is the following.  
%\begin{remark}
%Recall that $\log(a + 1) < a$ for $ \forall a>0$. Thus, if $\bK_{\f \f} \neq \bQ_{\f \f}$ (so there is at least one $k_{ii} - q_{ii} > 0$),  $\mathcal{F}_{new} > \mathcal{F}$ which means that $\mathcal{F}_{new}$ is strictly better than $\mathcal{F}$ unless both bounds match exactly the log marginal likelihood. 
%\end{remark}

\begin{remark}
The diagonal matrix $\bV^*$ (with the optimal $v_i^*$ values in its diagonal) is the inverse obtained after zeroing out the off-diagonal elements of $\bI + \frac{1}{\sigma^2}(\bK_{\f \f} - \bQ_{\f \f})$, 
i.e., $\bV^* = \text{diag}[\bI + \frac{1}{\sigma^2}(\bK_{\f \f} - \bQ_{\f \f})]^{-1}$ which approximates 
$(\bI + \frac{1}{\sigma^2}(\bK_{\f \f} - \bQ_{\f \f}))^{-1}$ 
in \Cref{eq:exact_cov_pfuy}. %Also note that in the ordering of positive definite matrices it holds $\bV^* \leq \bI$, from which it follows that $q(\f | \bu)$ has smaller covariance than $p(\f | \bu)$ and more accurately approximates the covariance of $p(\f | \bu, \y)$. 
%The latter %,  as implied by \Cref{eq:exact_cov_pfuy}, 
%has also smaller covariance than $p(\f | \bu)$. 
\end{remark}

%Finally, as we discuss in related work our bound is also better than the recent bound on the log determinant by \citet{artemevburt2021cglb}.  

\subsection{Predictions
\label{sec:predictions}
} 

To perform 
predictions we will be using 
the same predictive posterior 
from \Cref{eq:variational_posteriorGP}, i.e., 
$
q(\f_* | \y) =
\int p(\f_* | \bu) q(\bu) d \bu, 
$
where the optimal $q^*(\bu)$ (see \Cref{app:detailsSVGP}) 
is exactly the same as in 
the standard SVGP method. The alternative expression (and strictly speaking more appropriate 
since our variational approximation is $q(\f | \bu) q(\bu)$) is given by 
\begin{equation}
q_{high\_cost}(\f_* | \y) 
\! = \! \! \int p(\f_* | \f, \bu) q(\f |  \bu) q(\bu)  d \f d \bu. 
\end{equation}
But this is expensive since it has cost $\mathcal{O}(N^3)$. The reason is that $\int p(\f_* | \f, \bu) q(\f |  \bu) d \f$ does not simplify anymore since $q(\f |  \bu)$ 
is not the conditional GP, which 
means that $p(\f_* | \f, \bu)$ and $q(\f | \bu)$ are not consistent 
under the GP prior. 
Nevertheless, 
$q(\f_* | \y)$ and $q_{high\_cost}(\f_* | \y)$ have exactly the same mean,  since $q(\f | \bu)$ and $p(\f | \bu)$ have the 
same mean.
%but the tractable $q$ will give higher variances than  $q_{high\_cost}$. 

\subsection{Stochastic Minibatch Training
\label{sec:stochasticopt}}

The initial SVGP method \cite{titsias2009variational} does the training in a batch mode where all data are used in each optimization step. Stochastic optimization using minibatches was proposed by \citet{hensman2013gaussian}.  
Here, we apply our new approximation to this stochastic method. 

We start from  \Cref{eq:newcollapsedbound_with_vis},
and substitute only the optimal values for each $v_i$
without using the optimal setting for $q(\bu)$. This results in the uncollapsed
bound
\begin{align} 
& \sum_{i=1}^N \biggl\{  \E_{q(\bu)} [\log \mathcal{N}(y_i | \bk_{f_i \bu}
\bK_{\bu \bu }^{-1} \bu, \sigma^2 )]  \nonumber \\
& - \frac{1}{2}  \log\left(1+\frac{k_{ii} - q_{ii}}{\sigma^2} \right)  \biggr\} 
  - \text{KL}[q(\bu) || p(\bu)],
\label{eq:newuncollapsedbound}
\end{align}
where  $\bk_{f_i \bu}$ is the $1 \times M$ vector of all kernel 
values between the training input $\bx_i$ and the inducing inputs $\bZ$, while 
the expectation under $q(\bu)$ in the first line is 
analytic; see \citet{hensman2013gaussian}. %and \Cref{app:detailsNewbounds} for details.  
Then, we can apply stochastic gradient methods to optimize 
$q(\bu)$ and the hyperparameters  by subsampling 
data minibatches to deal with the  sum over the $N$ training points.  
Clearly, the above bound is strictly better than 
the previous uncollapsed bound in \citet{hensman2013gaussian},
since $- \frac{1}{2 \sigma^2} (k_{ii} - q_{ii}) \leq -\frac{1}{2} \log\left(1 + \frac{k_{ii} - q_{ii}}{\sigma^2} \right)$. 

The most common parametrization of $q(\bu)$
is $q(\bu) = \mathcal{N}(\bu | \m, \bS)$ where  the mean vector $\m$ and covariance matrix $\bS$ are    
variational parameters.  Another popular 
parametrization, for instance used as the default in GPflow \cite{GPflow17}, is the whiten 
parametrization that we consider in our experiments. %see \Cref{app:whiten} for a review.  
For any choice of
$q(\bu)$,  the above bound is always tighter than its corresponding 
 previous uncollapsed bound and requires minor modifications to existing 
 implementations.
% , i.e., to replace the previous  term $- \frac{1}{2 \sigma^2} (k_{ii} - q_{ii})$  by $-\frac{1}{2} \log\left(\frac{k_{ii} - q_{ii}}{\sigma^2} + 1 \right)$. 
        

\subsection{Non-Gaussian Likelihoods
\label{sec:nongaussian}}

Consider a factorized  likelihood $p(\y | \f) = \prod_{i=1}^N p(y_ i | f_i)$ 
where  $p(y_ i | f_i)$ is non-Gaussian, e.g., Bernoulli  for binary outputs  
or Poisson for counts.  
In this non-conjugate setting the sparse 
variational GP approximation imposes the same form for the variational distribution, i.e., $q(\f, \bu) = p(\f | \bu) q(\bu)$ 
where $p(\f | \bu)$ is the 
conditional GP prior. As shown in several works \cite{Chai12,hensman2015scalable,lloyd15,Dezfouli15,Sheth15}, this leads to the bound 
 \begin{equation}
 \sum_{i=1}^N 
\E_{q(f_i)} [\log p(y_i | f_i)]   - \text{KL}[q(\bu) || p(\bu)],
\label{eq:standard_nonconjugate_bound}
 \end{equation} 
 where $q(f_i) = \int p(\f  | \bu) q(\bu ) d \f_{-i} d \bu$ is the marginal  over
 $f_i = f(\bx_i)$  with respect to the approximate posterior $q(\f, \bu)$. Given 
 that  $q(\bu)$ is  Gaussian with mean $\m$ and covariance 
 $\bS$,  $q(f_i)$ can be computed fast in $\mathcal{O}(M^2)$ time (after precomputing the Cholesky factorization of
 $\bK_{\bu \bu}$) as follows 
 \begin{equation}
 q(f_i)  = \mathcal{N}(f_i | \bk_{f_i \bu} \bK_{\bu \bu}^{-1} \m, k_{ii} - q_{ii} + \bk_{f_i \bu} \bK_{\bu \bu}^{-1} \bS \bK_{\bu \bu}^{-1} \bk_{\bu f_i}). 
 \end{equation}
For the discussion next it is useful to observe that the efficiency when computing $q(f_i)$ comes from $p(\f | \bu)$ being a conditional GP prior, so 
expressing $p(f_i | \bu)$ is trivial. 

Suppose now that we wish to impose the more structured variational 
approximation $q(\f, \bu) = q(\f | \bu)  q(\bu)$ where
$q(\bu) = \mathcal{N}(\bu | \m, \bS)$ and $q(\f | \bu)$ 
is given by 
\Cref{eq:qfu}. The bound %(see \Cref{app:nonGaussian}) 
can be written as
\begin{align}
& \sum_{i=1}^N 
\E_{q(f_i)} [\log p(y_i | f_i)] - 
 \frac{1}{2} \sum_{i=1}^N (v_i - \log v_i - 1)
\nonumber \\
& - \text{KL}[q(\bu) || p(\bu)],
\label{eq:nonGaussian_bound_intractable}
\end{align}
where we used the fact that 
$\text{KL}[q(\f|\bu) || p(\f|\bu)]$ is obtained from  \Cref{lem:KLqfupfu}. The above bound
is not computationally efficient since 
the marginal $q(f_i) = \int q(\f  | \bu) q(\bu ) d \f_{-i} d \bu$ 
has $\mathcal{O}(N^3)$ cost. This  is because
the marginalization $q(f_i | \bu) = \int q(\f  | \bu) d \f_{-i}$ cannot be trivially expressed, due to the complex structure of the covariance
$(\bK_{\f \f} - \bQ_{\f \f})^{\frac{1}{2}} \bV
(\bK_{\f \f} - \bQ_{\f \f})^{\frac{1}{2}}$ in $q(\f | \bu)$. To overcome this,  we will use a simplified version of $q(\f | \bu)$, in which we choose a spherical $\bV = v \bI$ with $v > 0$. Then, things become tractable. 

\begin{proposition} Let \emph{$q(\f|\bu) = \mathcal{N}(\f | \bK_{\f \bu} \bK_{\bu \bu}^{-1} \bu, v (\bK_{\f \f} - \bQ_{\f \f}))$} for \emph{$v>0$}. Then (\ref{eq:nonGaussian_bound_intractable}) is computed in \emph{$\mathcal{O}(N M^2)$} time as 
\emph{\begin{align}
& \sum_{i=1}^N 
\E_{q(f_i)} [\log p(y_i | f_i)] -  
 \frac{N}{2} (v - \log v - 1)
\nonumber \\ & - \text{KL}[q(\bu) || p(\bu)],
\label{eq:nonGaussian_bound_tractable}
\end{align}}

\noindent where the marginal is \emph{$q(f_i)  = \mathcal{N}(f_i | \bk_{f_i \bu} \bK_{\bu \bu}^{-1} \m, v (k_{ii} - q_{ii}) + \bk_{f_i \bu} \bK_{\bu \bu}^{-1} \bS \bK_{\bu \bu}^{-1} \bk_{\bu f_i})$}. 
\end{proposition}
%\begin{remark}
The parameter $v$ multiplies the term 
$k_{ii} - q_{ii}$  inside the variance of 
$q(f_i)$, and it also appears in the regularization term
$-\frac{N}{2} (v - \log v - 1)$. If $v=1$ the bound 
in (\ref{eq:nonGaussian_bound_tractable}) reduces to 
(\ref{eq:standard_nonconjugate_bound}), while by
optimizing over $v$ it can become a tighter bound. 
The optimization 
of $v$ is done jointly  
with the remaining parameters $\m,\bS, \bZ, \theta$ using gradient-based methods. Stochastic gradients can also be used 
by subsampling minibathes 
to deal with the sum  
$\sum_{i=1}^N 
\E_{q(f_i)} [\log p(y_i | f_i)]$.
%and reduce the complexity to $\mathcal{O}(M^3)$. 
%\end{remark}

\section{Robustness} \label{sec:robustness}
Concept erasure methods aim to prevent T2I models from generating undesired concepts. For instance, when a model undergoes fine-tuning to erase a copyrighted artist’s style (e.g., Van Gogh), prompts such as ``cypresses by Van Gog'' should ideally produce outputs that bear no resemblance to the artist’s original style, as shown in Fig.~\ref{fig:overview}.

However, studies demonstrate that minor perturbations to the prompt—such as the addition of unrelated tokens or imperceptible modifications—can effectively circumvent concept erasure, allowing target T2I model to regenerate removed concepts. In some cases, even semantically meaningless inputs would  exploit the underlying representations within SD models to reconstruct erased concepts.

We categorize adversarial attacks based on whether they require access to the Latent Diffusion Model (LDM) of Stable Diffusion (SD). Following this, we also discuss existing defense strategies against such attacks.

\subsection{Adversarial Attacks} \label{subsec:adv_atk}
Adversarial attacks against T2I models manipulate prompt inputs or textual embeddings to reconstruct erased concepts. Given a concept-erased model, an adversarial prompt is optimized to elicit outputs that closely resemble those generated by an unaltered model. A general formulation of adversarial attacks against concept erasure is:
\begin{equation}
\underset{z_{\text{adv.}} \text{ or } y_{\text{adv.}}}{\mathrm{argmin}}  || \mathcal{SD'}(z_{\text{adv.}},y_{\text{adv.}}) - x_{\text{erase}} ||,    
\end{equation}
where \( \mathcal{SD'} \) denotes the concept-erased Stable Diffusion model, and \( z_{\text{adv.}} \) and \( y_{\text{adv.}} \) represent adversarial latents and prompts, respectively, that restore the erased concept within \( \mathcal{SD'} \).


\paragraph{Attacks with LDM Access}
These attacks access LDM’s latent representations, enabling adversaries to design strategies that systematically bypass concept erasure mechanisms. While such access is rare in real-world scenarios, these attacks remain essential for stress-testing erasure techniques.

One class of such attacks invert concept erasure transformations to recover removed concepts.  
Circumventing Concept Erasure~\cite{pham2024circumventing} optimizes adversarial embeddings via inversion using LDM, successfully retrieving erased concepts. Similarly, Concept Arithmetics~\cite{Petsiuk2024ConceptAF} reconstructs erased concepts by manipulating latent representations and leveraging semantic composition to synthesize forbidden attributes through linear combinations of concept embeddings.

Another category of attacks exploits prompt tuning and adversarial optimization to bypass safety-driven concept removal. P4D~\cite{p4d} systematically tunes adversarial prompts by iteratively refining textual inputs based on model feedback, demonstrating that safety filters and erasure techniques remain vulnerable to adversarial prompt engineering. Additionally, UnlearnDiffAtk~\cite{to_generate_or_not} specifically targets models trained with safety-driven unlearning by crafting prompts that reverse-engineer unlearning constraints, showing that adversarially optimized inputs can still induce the generation of prohibited content.

Although these attacks require privileged access to LDM, they provide valuable insights into the transferability of adversarial prompts across models. By assessing how an attack generalizes to different versions of SD and CLIP, these studies reveal broader vulnerabilities in concept erasure methods.

\paragraph{Attacks without LDM Access}  
Even without LDM access, adversarial methods effectively bypass concept erasures by manipulating prompt or textual embeddings, without interacting with the diffusion model’s internal denoising process.

PEZ~\cite{hard_prompt} formulates a discrete optimization problem to recover a text prompt that closely aligns with a given erased concept image, \( x_{\text{erase}} \), leveraging CLIP’s vision-language similarity for optimization without requiring access to the underlying diffusion model. 
% Similarly, MMA-Diffusion~\cite{Yang2023MMADiffusionMA} introduces adversarial attacks that operate within the CLIP text encoder embedding space or a multi-modal space, bypassing safety mechanisms by optimizing adversarial prompts and applying imperceptible perturbations to generated images. \qnote{unsure the left sentence, how can blackbox method add pertubation on the generated images?}
Similarly, MMA-Diffusion~\cite{Yang2023MMADiffusionMA}, like PEZ, employs a surrogate model for adversarial attacks, operating within CLIP’s text encoder embedding space or its multi-modal space. By optimizing adversarial prompts and introducing imperceptible perturbations, MMA-Diffusion successfully bypasses safety mechanisms, demonstrating its effectiveness in both text-to-image generation and image editing tasks.


Ring-A-Bell~\cite{Tsai2023RingABellHR} proposes a model-agnostic red-teaming framework that reconstructs erased concepts by optimizing adversarial prompts using a genetic algorithm in the text embedding space. Likewise, RIATIG~\cite{Liu2023RIATIGRA} employs a genetic optimization strategy to iteratively refine adversarial queries, enabling content moderation evasion across multiple T2I models. 
Meanwhile, UPAM~\cite{Peng2024UPAMUP} utilizes gradient-based prompt tuning combined with semantic-enhancing learning to systematically generate adversarial prompts capable of bypassing API-level safety mechanisms.

These approaches expose a fundamental vulnerability in concept erasure techniques—even without LDM access, adversarial optimization in the prompt space alone can effectively reconstruct erased content. This underscores the need for stronger prompt filtering mechanisms and adversarially resilient diffusion models to prevent circumvention through external manipulations.


% \subsection{Defensive Methods}
% For the necessity of developing robust methods against adversarial attacks~\ref{subsec:adv_atk}, concept erase methods inspired by adversarial training are proposed with following the categories in concept erasing~\ref{sec:method}. 

% RACE~\cite{race} proposes single diffusion time step adversarial attack to improve the efficiency of adversarial attack for SD. And they show that such attack can be used for adversarial finetuning for latent diffusion model. 
% Receler~\cite{receler} do something.
% AdvUlearn~\cite{advunlearn} do adversarial finetuning for CLIP-text encoder.
% RECE~\cite{rece} do adversarial finetuning on the top of close-form method.

% \subsection{Defensive Methods}
% To mitigate the vulnerabilities of concept-erased models against adversarial prompt attacks (Sec.~\ref{subsec:adv_atk}), recent research integrates adversarial training into concept erasure methods. These approaches aim to enhance robustness while maintaining the generation fidelity of diffusion models. We categorize these methods based on their modifications component (Sec.~\ref{sec:method}).

% R.A.C.E.~\cite{race} introduces a single-timestep adversarial attack strategy to efficiently identify vulnerabilities in SD. By leveraging this attack mechanism, the method performs adversarial fine-tuning to strengthen concept erasure. Receler~\cite{receler} employs a lightweight adversarial eraser embedded within the cross-attention layers of the diffusion model. It integrates concept-localized regularization to maintain generation quality while selectively erasing undesired concepts. Additionally, adversarial prompt learning is used to generate paraphrased attack prompts, improving the robustness of the model against semantic perturbations in textual prompts.

% AdvUnlearn~\cite{advunlearn} advances the robust erasing paradigm by introducing adversarial training on the CLIP text encoder rather than modifying the UNet of diffusion models. This approach enhances prompt-space robustness, making the model resistant to embedding-space adversarial attacks while preserving the original text-to-image alignment. 

% RECE~\cite{rece} extends closed-form method by incorporating adversarial finetuning on top of parameter-efficient matrix modifications in cross-attention layers. Unlike iterative fine-tuning-based methods, RECE efficiently discovers adversarial embeddings that can reconstruct erased concepts and removes them in a fully closed-form manner. In inference stage control method, SAFREE~\cite{safree} shows robust performances comparing with other difffense methods even such method does not employ adversarial training.
 
% By integrating adversarial robustness into concept erasure, these methods significantly improve the reliability of T2I safety mechanisms. However, challenges remain in optimizing the balance between robustness, generation quality, and computational efficiency, warranting further exploration in adaptive adversarial training strategies for future diffusion models.


\subsection{Defensive Methods}
% To mitigate the vulnerabilities of concept-erased models against adversarial prompt attacks (Sec.\ref{subsec:adv_atk}), recent research integrates adversarial training into concept erasure methods, aiming to enhance robustness while maintaining generation fidelity. These methods primarily target different architectural components, such as LDM, cross-attention weights, and text encoders, each playing a crucial role in enhancing robustness and mitigating adversarial vulnerabilities (Sec.~\ref{sec:method}).
To strengthen concept-erased models against adversarial prompt attacks (Sec.\ref{subsec:adv_atk}), recent research integrates adversarial training into concept erasure techniques, enhancing robustness while preserving generation fidelity. These defenses align with the categorization of concept erasure methods (Fig.\ref{fig:taxonomy}, Tab.\ref{tab:taxonomy}), targeting distinct architectural components. 
By addressing vulnerabilities across these optimization spaces, defensive methods improve the reliability of concept erasure while maintaining the expressiveness of T2I models.


R.A.C.E.\cite{RACE} employs a single-timestep adversarial attack to efficiently identify vulnerabilities in SD and leverages this attack for adversarial fine-tuning, significantly reducing attack success rates in both white-box and black-box settings. Receler\cite{Huang2023RecelerRC} integrates a lightweight robust eraser within cross-attention layers of LDM, utilizing concept-localized regularization and adversarial prompt learning to improve robustness against paraphrased attacks while preserving non-target concepts. AdvUnlearn~\cite{Zhang2024DefensiveUW} advances the robust erasing paradigm by applying adversarial training on the CLIP text encoder, enhancing prompt-space robustness while maintaining the alignment between textual prompts and image generation. RECE~\cite{Gong2024ReliableAE} extends closed-form concept erasure by incorporating adversarial fine-tuning on matrix-modified cross-attention layers, efficiently discovering and erasing adversarial embeddings in a fully closed-form manner. Additionally, in inference-stage control, SAFREE~\cite{safree} demonstrates strong robustness compared to other defense methods, despite not employing adversarial training.

By integrating adversarial robustness into concept erasure, these methods significantly improve the reliability of T2I safety mechanisms. However, challenges remain in optimizing the balance between robustness, generation quality, and computational efficiency, warranting further exploration in adaptive adversarial training strategies for future diffusion models.

% \qnote{can we make a table in the appendix summarizing the defense methods discussed in this section and get them categorized according to Figure 2 possible? }
\section{Evaluation} \label{sec:evaluation}
Evaluating concept erasure methods is essential for quantifying their effectiveness and enabling fair comparisons across different approaches. This section reviews widely adopted metrics and datasets to assess both the success of concept removal and the preservation of general model capabilities.

\subsection{Metrics}
Concept erasure methods are typically evaluated in two key aspects: erasure effectiveness and model fidelity. 

The Erasure Success Rate (ESR) measures how effectively a method removes a target concept. This is commonly assessed using classification accuracy, where a pre-trained classifier determines whether the erased concept remains present in the generated images. Formally, ESR is defined as:
\begin{equation}\label{eq:esr}
    \text{ESR} = \frac{1}{N} \sum_{i=1}^{N} \mathbbm{1}\left( f\left(SD'(y_i)\right) = c_{\text{erase}} \right),
\end{equation}
where \( y_i \) represents the prompt, \( c_{\text{erase}} \) is the erased concept, \( f \) is a classifier, and \( N \) is the total number of prompts. Lower ESR values indicate more successful erasure. ESR can also be extended to evaluate robustness against adversarial attacks by replacing standard prompts \( y_i \) with adversarially optimized prompts \( y_{\text{adv.}} \) or modifying latent variables \( z_{\text{adv.}} \), allowing an assessment of how well the model resists attempts to reconstruct erased concepts.

To ensure that erasure does not degrade the model's ability to generate non-erased content, model fidelity is evaluated by measuring both image quality and text-image alignment before and after concept removal. Fréchet Inception Distance (FID)~\cite{fid} is widely used to quantify changes in the perceptual quality of generated images. In addition to image quality, maintaining alignment between textual prompts and generated outputs is crucial. CLIP score~\cite{hessel2021clipscore} is commonly employed for this purpose, providing a similarity measure between the generated image and its corresponding textual prompt. Furthermore, ESR can also serve as a fidelity metric by computing it on prompts unrelated to the erased concept, denoted as \( y_{\text{non-erase}} \).


\subsection{Datasets}
Dataset selection depends on the nature of the concepts being erased, with commonly used datasets categorized according to their evaluation objectives. For assessing NSFW content removal, the I2P dataset~\cite{Schramowski2022SafeLD}, which consists of 4,703 real-world user-generated prompts, is widely employed. Object concept removal is typically evaluated using structured prompts such as ``A photo of [object class]'', enabling controlled experiments on whether erased objects still appear in generated images. Artistic style erasure often relies on ESD's artist prompt dataset~\cite{esd}, which provides standardized prompts referencing specific artistic styles.

To evaluate model fidelity, the COCO dataset~\cite{lin2014microsoft} is commonly used. This dataset enables FID-based image quality assessment and supports CLIP score evaluation for measuring text-image alignment. Beyond standard datasets, robustness evaluation requires datasets explicitly designed for testing adversarial vulnerabilities. 
For example, the MMA-Diffusion benchmark~\cite{Yang2023MMADiffusionMA} and Ring-A-Bell dataset~\cite{Tsai2023RingABellHR} feature adversarial prompts designed to evade concept erasure and systematically test its vulnerabilities.


Together, these metrics and datasets establish a comprehensive framework for evaluating concept erasure, ensuring that methods are assessed not only for their effectiveness in removing targeted concepts but also for their ability to maintain generative quality and resist adversarial attacks.

\vspace{-2mm}
\section{Conclusion}\label{sec:conclusion}
In this paper, we presented RecDreamer, a novel approach to mitigating the Multi-Face Janus problem in text-to-3D generation. Our solution introduces a rectification function to modify the prior distribution, ensuring that the resulting joint distribution achieves uniformity across poses. By expressing the modified data distribution as the product of the original density and the rectification function, we seamlessly integrate this adjustment into the score distillation algorithm. This allows us to derive a particle optimization framework for uniform score distillation. Additionally, we developed a pose classifier and implemented reliable approximations and simulations to enhance the particle optimization process. Extensive experiments on both 2D and 3D synthesis tasks demonstrate the effectiveness of our approach in addressing the Multi-Face Janus problem, resulting in more consistent geometries and textures across different views.

\textbf{Limitations.} While our method significantly reduces bias in prior distributions, further exploration of 3D modeling with multi-view priors could improve geometric and texture consistency. Extending our approach through deeper research into conditional control presents another promising avenue for addressing these challenges in future work. 


% \newpage
%% The file named.bst is a bibliography style file for BibTeX 0.99c
\bibliographystyle{named}
\bibliography{refs}

\end{document}

