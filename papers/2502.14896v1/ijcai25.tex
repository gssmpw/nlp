%%%% ijcai25.tex

\typeout{IJCAI--25 Instructions for Authors}

% These are the instructions for authors for IJCAI-25.

\documentclass{article}
\pdfpagewidth=8.5in
\pdfpageheight=11in

% The file ijcai25.sty is a copy from ijcai22.sty
% The file ijcai22.sty is NOT the same as previous years'
\usepackage{ijcai25}

% Use the postscript times font!
\usepackage{times}
\usepackage{soul}
\usepackage{url}
\usepackage[hidelinks]{hyperref}
\usepackage[utf8]{inputenc}
\usepackage[small]{caption}
\usepackage{graphicx}
\usepackage{amsmath}
\usepackage{amsthm}
\usepackage{booktabs}
\usepackage{algorithm}
\usepackage{algorithmic}
\usepackage[switch]{lineno}
\usepackage{amsfonts}
\usepackage{bbm}

\usepackage{xcolor}


\usepackage{pifont}  % Required for checkmark and cross
\usepackage{adjustbox}
\usepackage{booktabs} % Better table formatting
\usepackage{multirow}


% Define checkmark and crossmark symbols
\newcommand{\cmark}{\ding{51}} % ✔
\newcommand{\xmark}{\ding{55}} % ✘


\newcommand{\qnote}[1]{[\textcolor{red}{Q-note: #1}]}

% Comment out this line in the camera-ready submission
% \linenumbers

\urlstyle{same}

% the following package is optional:
%\usepackage{latexsym}

% See https://www.overleaf.com/learn/latex/theorems_and_proofs
% for a nice explanation of how to define new theorems, but keep
% in mind that the amsthm package is already included in this
% template and that you must *not* alter the styling.
\newtheorem{example}{Example}
\newtheorem{theorem}{Theorem}

% Following comment is from ijcai97-submit.tex:
% The preparation of these files was supported by Schlumberger Palo Alto
% Research, AT\&T Bell Laboratories, and Morgan Kaufmann Publishers.
% Shirley Jowell, of Morgan Kaufmann Publishers, and Peter F.
% Patel-Schneider, of AT\&T Bell Laboratories collaborated on their
% preparation.

% These instructions can be modified and used in other conferences as long
% as credit to the authors and supporting agencies is retained, this notice
% is not changed, and further modification or reuse is not restricted.
% Neither Shirley Jowell nor Peter F. Patel-Schneider can be listed as
% contacts for providing assistance without their prior permission.

% To use for other conferences, change references to files and the
% conference appropriate and use other authors, contacts, publishers, and
% organizations.
% Also change the deadline and address for returning papers and the length and
% page charge instructions.
% Put where the files are available in the appropriate places.


% PDF Info Is REQUIRED.

% Please leave this \pdfinfo block untouched both for the submission and
% Camera Ready Copy. Do not include Title and Author information in the pdfinfo section
\pdfinfo{
/TemplateVersion (IJCAI.2025.0)
}

\title{A Comprehensive Survey on Concept Erasure in \\ Text-to-Image Diffusion Models}


% Multiple author syntax (remove the single-author syntax above and the \iffalse ... \fi here)
% \author{
% Changhoon Kim\thanks{Work done while the author was a Postdoctoral Scientist at Amazon.}$^{1,2}$ 
% \And
% Yanjun Qi$^1$
% \\
% \affiliations
% $^1$Amazon Bedrock Science\\
% $^2$Soongsil University\\
% \emails
% \{chkimm, yanjunqi\}@amazon.com,
% changhooon.kim@gmail.com
% }


\author{
Changhoon Kim 
\And
Yanjun Qi
\\
\affiliations
Amazon Bedrock Science\\
\emails
\{chkimm, yanjunqi\}@amazon.com,
changhooon.kim@gmail.com
}


\begin{document}

\maketitle

\begin{abstract}
Text-to-Image (T2I) models have made remarkable progress in generating high-quality, diverse visual content from natural language prompts. However, their ability to reproduce copyrighted styles, sensitive imagery, and harmful content raises significant ethical and legal concerns. Concept erasure offers a proactive alternative to external filtering by modifying T2I models to prevent the generation of undesired content. In this survey, we provide a structured overview of concept erasure, categorizing existing methods based on their optimization strategies and the architectural components they modify. We categorize concept erasure methods into fine-tuning for parameter updates, closed-form solutions for efficient edits, and inference-time interventions for content restriction without weight modification. Additionally, we explore adversarial attacks that bypass erasure techniques and discuss emerging defenses. To support further research, we consolidate key datasets, evaluation metrics, and benchmarks for assessing erasure effectiveness and model robustness. This survey serves as a comprehensive resource, offering insights into the evolving landscape of concept erasure, its challenges, and future directions.

\end{abstract}


\section{Introduction}
Safety is a critical consideration in various applications, including robots, autonomous vehicles, smart grids, and transportation control systems~\cite{wolf2017safety}. These safety-critical scenarios demand formal guarantees to ensure that systems operate as expected, as failures may result in severe consequences, such as harm to humans or significant financial costs. Safety verification refers to the task of determining whether a system satisfies a given safety specification over a specified period~\cite{guiochet2017safety, vicentini2019safety}. 
Conventional safe set specifications primarily focus on spatial requirements, ensuring that the system state never enters an unsafe region~\cite{prajna2004safety}. However, as the complexity of autonomous systems increases, many real-world tasks require specifications that are not only spatial but also temporal in nature. For instance, a mobile robot needs to pass Area A before entering Area B. In this paper, we focus on safety verification under Signal Temporal Logic (STL) specification, which uses both boolean and temporal logic operators to formulate constraints for continuous-valued systems~\cite{maler2004monitoring}. 

Real-world systems are subject to various types of uncertainty. It is essential for safety verification algorithms to account for disturbances. Many existing approaches model these uncertainties as bounded disturbances and employ worst-case analysis to guarantee the satisfaction of safety specifications. Examples of such approaches for safe set specification include Hamilton–Jacobi Reachability (HJ Reachability)~\cite{bansal2017hamilton}, reachability analysis, and barrier certificates~\cite{prajna2004safety}. For STL specifications, methods such as HJ Reachability~\cite{chen2018signal} and reachability analysis~\cite{roehm2016stl, lercher2024using, kochdumper2024fully} have been employed to formally verify STL satisfaction under bounded disturbance inputs.


In many practical situations, disturbances are better modeled as stochastic noise, which provides a more realistic representation, as in the case of sensor noise. When considering stochastic disturbances, the aforementioned deterministic methods are not applicable or tend to be overly conservative, as they focus on worst-case scenarios that rarely occur in practice. To better account for stochastic disturbances, we adopt a probabilistic setting, where the goal is to ensure the safety specification is satisfied with high probability, e.g., greater than 99.9\%. 
For safe set specifications, several methods have been proposed to verify stochastic systems, including martingale-based approaches~\cite{steinhardt2012finite, santoyo2021barrier}, risk estimation~\cite{frey2020collision}, and sampling-based methods~\cite{janson2017monte}. Our recent paper significantly reduces the conservativeness of the verification algorithms for safe set specifications~\cite{liu2024safety}.
For STL specifications, most existing approaches are limited to handling the probability constraint for a single trajectory satisfying the STL specification~\cite{sadigh2016safe, farahani2018shrinking, yang2023distributed, vlahakis2024probabilistic, kordabad2024control}. Very few studies have focused on STL verification under both bounded and stochastic disturbances. In \cite{salamati2021data}, a method is proposed to address this problem for linear systems under Gaussian noise. In this work, we focus on the problem of STL verification for nonlinear systems under both bounded and stochastic disturbances. 

In this work, we present a novel framework for verifying the probabilistic STL satisfaction of discrete-time nonlinear stochastic systems. To the best of our knowledge, this is the first approach capable of addressing this problem for nonlinear systems under both bounded and stochastic disturbances. Given a desired probability requirement, our method first erodes the superlevel set of the predicates in an STL formula to get a tighter STL formula. If the deterministic system is verified to satisfy the tighter STL formula, then the stochastic system is guaranteed to satisfy the original STL formula with the specified probability constraint. As a result, the stochastic verification problem is transformed into a deterministic one. The depth of erosion is determined by the sharp probabilistic bound proposed in our previous work~\cite{liu2024probabilistic}, which helps reduce the conservativeness of the verification result, especially when the probability tolerance is low and the time horizon is long. Our method does not rely on restrictive assumptions, such as linear system dynamics or affine predicates, which is common in previous work~\cite{vlahakis2024probabilistic}. This broader applicability makes our approach suitable for real-world applications.



\textit{Notations.}
% \textit{Vectors, matrices, and probability.} 
Denote by $\real$ and $\n$ the sets of real numbers and nonnegative integers, and define $\n_{[a,b]}=\setb{a, a+1, \dots, b}$ where $a,b\in \n$ and $a<b$. Given a vector sequence $\{x_t\}$, define $\boldsymbol{x}_{[t_1,t_2]} = (x_{t_1}, x_{t_1+1}, \dots, x_{t_2}) = [x\tran_{t_1}, x\tran_{t_1+1}, \dots, x\tran_{t_2}]\tran$ , $t\in\n_{[t_1,t_2]}$. When $X_t$ are random vectors, $\boldsymbol{X}_{[t_1,t_2]}$ is a random process. We use $\bP$ to denote probability. A random vector $X \sim \mathcal{N}(\mu, \Sigma)$ follows a multivariate Gaussian distribution with mean $\mu$ and covariance $\Sigma$.
Given a vector $x\in \real^n$, $\|x\|$ denotes the euclidean norm and $\|x\|_P = \sqrt{x\tran P x}$, where $P\in\real^{n\times n}$ is a positive definite matrix.
% \textit{Sets.} 
The $n$ dimensional ball with radius $r$ and center $y$ is denoted by $\BB^n(r, y)=\setb{x\in \real^n : \|x-y\| \leq r}$. Denote the complement of set $A$ as $\setcomp{A}$ and $-B = \setb{-y: \forall y\in B}$. Given sets $A$ and $B$, define the Minkowski sum of $A$ and $B$ by $A\oplus B = \setb{x+y: x\in A,~ y\in B}$, and the Minkowski difference or Pontryargin difference of $A$ and $B$ by $A\ominus B=\setb{x:x+y\in A, \forall y\in B}$ \cite{kolmanovsky1998theory}. The Minkowski sum and difference satisfy the relation $(A\ominus B)\oplus B \subseteq A$.






\begin{figure}[t]
    \centering
    \includegraphics[width=\linewidth]{figs/taxonomy_v3.pdf}
    \caption{
    Taxonomy of Concept Erasers. Concept erasure methods are categorized based on their optimization strategy (first level) and the model components they modify (second level). A detailed discussion is provided in Sec.~\ref{sec:method}.
    }
    \label{fig:taxonomy}
\end{figure}

\section{Backgrounds} \label{sec:preliminaries}

This section presents an overview of the Text-to-Image (T2I) diffusion model with a particular focus on Stable Diffusion (SD)~\cite{stable_diffusion}. %, which serves as the foundational framework for evaluating concept erasure methods. 
As shown in Fig.~\ref{fig:overview}, SD comprises three main components: a vision decoder for reconstructing images from latent representations, a latent diffusion model for iterative denoising, and a conditional text encoder that transforms textual prompts into conditioning vectors.
We outline both the training and inference mechanisms of SD, which are essential for understanding how concept erasure techniques modify key model components or inference steps to suppress undesired concepts.

% \begin{table*}[t!]
%   \caption{Taxonomy of Concept Erasure Methods in T2I Models. Methods are categorized based on their optimization strategies and the specific model components they modify. CA denotes the cross-attention layers within the U-Net, while CFG refers to Classifier-Free Guidance adjustments. A comprehensive discussion of these methods is provided in Sec.~\ref{sec:method}.
% }
%   \centering
%   % \footnotesize
%   % \small
%   \scriptsize
%   \setlength{\tabcolsep}{3pt} % Reduce spacing for the first three columns
%   \begin{adjustbox}{width=\textwidth,center}
%     \begin{tabular}{p{1.8cm}p{3.6cm}cccccp{7.0cm}} 
%       \toprule
%       \textbf{Category} & \textbf{Representative Works} & \multicolumn{5}{c}{\textbf{Optimization Space}} & \textbf{Optimization Strategy} \\ 
%       \cmidrule(lr){3-7} 
%        & & \textbf{U-Net} & \textbf{CA} & \textbf{CLIP} & \textbf{LLM} & \textbf{CFG} & \\ 
%       \midrule
%       \multirow{13}{*}{\textbf{Fine-tuning}}  
%       & FMN~\cite{Zhang2023ForgetMeNotLT} &  & \cmark &  &  &  & Attention reweighting \\ 
%       & AC~\cite{Ablating_Concept} &  & \cmark &  &  &  & Remapping erased concepts \\ 
%       & SALUN~\cite{fan2024salun} & \cmark &  &  &  &  & Saliency-guided tuning \\ 
%       & ESD~\cite{esd} & \cmark & \cmark &  &  &  & Concept removal in generative noise process \\ 
%       & DT~\cite{Ni2023DegenerationTuningUS} & \cmark &  &  &  &  & Targeted concept degradation \\ 
%       & Geom-Erasing~\cite{Liu2023ImplicitCR} & \cmark &  &  &  &  & Targeted concept degradation \\ 
%       & SA~\cite{Heng2023SelectiveAA} & \cmark &  &  &  &  & Targeted concept degradation \\ 
%       & IMMA~\cite{Zheng2023IMMAIT} & \cmark &  &  &  &  & Prevents unauthorized fine-tuning \\ 
%       & SAFE-CLIP~\cite{safe_clip} &  &  & \cmark &  &  & Adversarial robustness for CLIP \\ 
%       & Latent Guard~\cite{Liu2024LatentGA} &  &  & \cmark &  &  & Targeted feature suppression \\ 
%       & AdvUnlearn~\cite{Zhang2024DefensiveUW} &  &  & \cmark &  &  & Adversarial fine-tuning for CLIP \\
%       & Receler~\cite{Huang2023RecelerRC} & \cmark  &  &  &  &  & Introduces adapter for robustness in U-Net \\
%       & R.A.C.E~\cite{RACE} & \cmark  &  &  &  &  & Adversarial fine-tuning for U-Net \\
%       \midrule
%       \multirow{7}{*}{\textbf{Closed-form}}  
%       & ReFACT~\cite{Arad2023ReFACTUT} &  &  & \cmark &  &  & Low-rank memory update in CLIP MLP layers \\ 
%       & TIME~\cite{Orgad2023EditingIA} &  & \cmark &  &  &  & Projection matrix updates in cross-attention \\ 
%       & UCE~\cite{Gandikota2023UnifiedCE} &  & \cmark &  &  &  & Multi-concept projection learning \\ 
%       & MACE~\cite{Lu2024MACEMC} &  & \cmark &  &  &  & LoRA-based parameter refinement for erasure \\ 
%       & EMCID~\cite{Xiong2024EditingMC} & \cmark & \cmark &  &  &  & Two-stage closed-form editing (self-distillation + projection) \\ 
%       & MUNBa~\cite{Wu2024MUNBaMU} &  & \cmark & \cmark &  &  & Nash bargaining-based concept unlearning \\ 
%       & RECE~\cite{Gong2024ReliableAE} & \cmark  &  &  &  &  & Adversarial fine-tuning \\
%       \midrule
%       \multirow{6}{*}{\textbf{Inference-Time}}  
%       & SLD~\cite{sld} &  & \cmark &  &  & \cmark & Adjusts latent denoising dynamics \\ 
%       & AMG~\cite{Chen2024TowardsMD} &  & \cmark &  &  & \cmark & Prevents overfitting to erased concepts \\ 
%       & SAFREE~\cite{safree} &  &  & \cmark &  &  & Prevents undesired text-image associations \\ 
%       & Content Suppression~\cite{Li2024GetWY} &  &  & \cmark &  &  & Enforces embedding constraints \\ 
%       & ORES~\cite{ores} &  &  &  & \cmark &  & LLM-based adversarial filtering \\ 
%       & GuardT2I~\cite{Yang2024GuardT2IDT} &  &  &  & \cmark &  & Detects circumvention prompts \\ 
%       \bottomrule
%     \end{tabular}
%   \end{adjustbox}
%   \label{tab:taxonomy}
% \end{table*}

\begin{table*}[t!]
  \caption{Taxonomy of Concept Erasure Methods in T2I Models. Methods are categorized based on their optimization strategies and the specific model components they modify. In the third column, "CA" denotes the cross-attention layers within the latent diffusion model, while "CFG" refers to Classifier-Free Guidance adjustments. A comprehensive discussion of these methods is provided in Sec.~\ref{sec:method}.}
  \centering
  \scriptsize
  \setlength{\tabcolsep}{3pt} % Reduce spacing for the first three columns
  \begin{adjustbox}{width=\textwidth,center}
    \begin{tabular}{p{1.8cm}p{3.6cm}cccccp{6.5cm}} 
      \toprule
      \textbf{Category} & \textbf{Representative Works} & \multicolumn{5}{c}{\textbf{Optimization Space}} & \hspace{5mm} \textbf{Description} \\ 
      \cmidrule(lr){3-7} 
       & & \textbf{U-Net} & \textbf{CA} & \textbf{CLIP} & \textbf{LLM} & \textbf{CFG} & \\ 
      \midrule
      \multirow{13}{1.8cm}{\textbf{Fine-Tuning}}  
      & FMN~\cite{Zhang2023ForgetMeNotLT} &  & \cmark &  &  &  & Minimize attention activation to erase concepts. \\ 
      & AC~\cite{Ablating_Concept} &  & \cmark &  &  &  & Remaps erased concepts to general concepts. \\ 
      & SALUN~\cite{fan2024salun} & \cmark &  &  &  &  & Modifies influential weights to remove concepts. \\ 
      & ESD~\cite{esd} & \cmark & \cmark &  &  &  & Edits noise prediction to remove concepts. \\ 
      & DT~\cite{Ni2023DegenerationTuningUS} & \cmark &  &  &  &  & Degrades model’s ability to reconstruct erased concepts. \\ 
      & Geom-Erasing~\cite{Liu2023ImplicitCR} & \cmark &  &  &  &  & Uses geometric constraints for concept removal. \\ 
      & SA~\cite{Heng2023SelectiveAA} & \cmark &  &  &  &  & Continual learning-based forgetting approach. \\ 
      & IMMA~\cite{Zheng2023IMMAIT} & \cmark &  &  &  &  & Enhances robustness against unauthorized fine-tuning. \\ 
      & SAFE-CLIP~\cite{safe_clip} &  &  & \cmark &  &  & Fine-tunes CLIP with safe and unsafe text-image quadruplets. \\ 
      & Latent Guard~\cite{Liu2024LatentGA} &  &  & \cmark &  &  & Fine-tunes the CLIP text encoder with safe and unsafe pairs. \\ 
      & AdvUnlearn~\cite{Zhang2024DefensiveUW} &  &  & \cmark &  &  & Adversarial finetuning for CLIP text encoder. \\
      & Receler~\cite{Huang2023RecelerRC} &  & \cmark &  &  &  & Uses adapters to enhance robustness. \\
      & R.A.C.E~\cite{RACE} & \cmark  & \cmark &  &  &  & Adversarially fine-tunes U-Net for resilience. \\
      \midrule
      \multirow{7}{1.8cm}{\textbf{Closed-form Model Editing}}  
      & ReFACT~\cite{Arad2023ReFACTUT} &  &  & \cmark &  &  & Updates CLIP’s memory via low-rank edits. \\ 
      & TIME~\cite{Orgad2023EditingIA} &  & \cmark &  &  &  & Modifies CA projection matrices for concept editing. \\ 
      & UCE~\cite{Gandikota2023UnifiedCE} &  & \cmark &  &  &  & Simultaneously erases multiple concepts. \\ 
      & MACE~\cite{Lu2024MACEMC} &  & \cmark &  &  &  & Utilize adapters for large-scale concept erasure. \\ 
      & EMCID~\cite{Xiong2024EditingMC} &  &  & \cmark  &  &  & Large-scale concept erasure via two-stage closed-form editing \\ 
      & MUNBa~\cite{Wu2024MUNBaMU} &  &  & \cmark &  &  & Uses Nash bargaining for controlled concept removal. \\ 
      & RECE~\cite{Gong2024ReliableAE} &  & \cmark &  &  &  & Integrates adversarial fine-tuning with closed-form editing. \\
      \midrule
      \multirow{6}{1.8cm}{\textbf{Inference-Time Intervention}}  
      & SLD~\cite{sld} &  &  &  &  & \cmark & Incorporates safety guidance to mitigate undesired concepts. \\ 
      & AMG~\cite{Chen2024TowardsMD} &  & &  &  & \cmark & Introduces three guidance strategies to prevent memorization. \\ 
      & SAFREE~\cite{safree} &  &  & \cmark &  &  & Self-validating filtering and re-attention for safe generation.\\ 
      & Content Suppression~\cite{Li2024GetWY} &  &  & \cmark &  &  & Adjusts embeddings to suppress concept generation. \\ 
      & ORES~\cite{ores} &  &  &  & \cmark & \cmark & Utilizes LLMs to filter and rewrite prompts for safer generation. \\ 
      & GuardT2I~\cite{Yang2024GuardT2IDT} &  &  &  & \cmark &  & Propose conditional LLM to detect adversarial prompts. \\ 
      \bottomrule
    \end{tabular}
  \end{adjustbox}
  \label{tab:taxonomy}
\end{table*}

% \subsection{Comparison with Related Work}  
% Concept erasure differs from machine unlearning, image editing. While machine unlearning focuses on removing specific data points from a model to comply with privacy regulations, concept erasure targets entire content categories, such as explicit or copyrighted styles, preventing their regeneration. Unlike image editing, which modifies specific attributes of an input image based on auxiliary inputs, concept erasure alters a model’s ability to generate certain concepts across all inputs.

\subsection{Three Components of Stable Diffusion }

Stable Diffusion comprises three primary components:

%~\cite{esser2021taming}
\paragraph{(1) Image Autoencoder.} The model leverages a pre-trained autoencoder to compress high-dimensional image data into a low-dimensional latent representation. The encoding network $\mathcal{E}(\cdot)$ maps an image $x$ to a latent variable $z = \mathcal{E}(x)$, and the decoding network $\mathcal{D}(\cdot)$ reconstructs the image from the latent space such that $\mathcal{D}(z) = \hat{x} \approx x$. This design ensures effective data compression while minimizing reconstruction error, preserving essential image features critical for generative tasks.



\paragraph{(2) Latent Diffusion Model.} The core generative process in SD is governed by a U-Net-based Latent Diffusion Model (LDM) that progressively refines noisy latent representations toward high-fidelity outputs. The training objective is formulated as:
\begin{equation}
    L_{\text{SD}} = \mathbb{E}_{n \sim \mathcal{N}(0,1), z, c, t} \left[
    \| n - \Phi_{\theta}(z_t, c) \|_2^2
    \right],
\end{equation}
where $c$ is the text embedding derived from the input prompt and integrated via cross-attention, $t$ denotes the diffusion timestep, $n$ is a noise vector sampled from a standard Gaussian distribution $\mathcal{N}(0,1)$, and $z_t$ is the noisy latent variable at timestep $t$. The LDM \( \Phi_{\theta} \), parameterized by \( \theta \), is trained to predict and remove noise at each step, progressively refining the latent variable along the diffusion trajectory.


\paragraph{(3) Conditional Text Encoding.} The model employs a text encoder to transform user-provided text prompts into conditioning vectors, enabling fine-grained control over the generation process. Specifically, the textual prompt $y$ is embedded as $c = \mathcal{E}_{\text{txt}}(y)$, where $\mathcal{E}_{\text{txt}}$ typically textual encoder of CLIP~\cite{CLIP}. These text embeddings are integrated through the cross-attention layers within the latent diffusion model~\cite{stable_diffusion}, allowing the textual context to dynamically influence each denoising step.

\subsection{Inference in Stable Diffusion}
Classifier-free guidance~\cite{ho2022classifier} enhances the conditionality of the image synthesis process during the inference phase of SD. The process starts with initializing latent representations $z_T$ sampled from a Gaussian distribution. The denoising trajectory is steered by classifier-free guidance, which modifies the denoising function as follows:
\begin{equation}\label{eq:classifier-free-guidance}
    \Tilde{\Phi}_{\theta}(z_t, c) = \Phi_{\theta}(z_t, \phi) + \alpha \left( \Phi_{\theta}(z_t, c) - \Phi_{\theta}(z_t, \phi) \right),
\end{equation}
where $\Phi_{\theta}(z_t, c)$ and $\Phi_{\theta}(z_t, \phi)$ represent the conditioned and unconditioned latent noises, respectively. The guidance scale $\alpha > 1$ amplifies the influence of the conditioned path, embedding the textual information into the generative process. 
Iterative refinement reduces noise through sequential calculations of $z_{t-1} = \Tilde{\Phi}_{\theta}(z_t, c)$, progressing until $t=0$. The final coherent image representation $z_0$ is transformed into the output image $\hat{x}$ by the decoder, $\hat{x} = \mathcal{D}(z_0)$. The T2I generation process can be succinctly expressed as $SD(y) = \mathcal{D}(\Tilde{\Phi}_{\theta}(z_T, \mathcal{E}_{\text{txt}}(y)))$.


\section{Methods} \label{sec:method}

\subsection{Concept Erase}
Concept erasure in T2I models, particularly SD, involves modifying model parameters or adjusting inference procedures to selectively suppress or eliminate the generation of specific, unwanted concepts. This technique is crucial for addressing the risks associated with generating potentially harmful or copyrighted content in the model's outputs. The primary goal of concept erasure is to condition the model so that it does not produce images corresponding to undesired prompts. For instance, to erase the influence of a copyrighted artist's style, the model is adjusted such that a prompt like ``A painting in the style of [artist]'' results in outputs that bear no resemblance to that artist's work. This objective can be succinctly stated as \(SD(y_{\text{erase}}) \not\in  \{ x_{\text{erase}}\}\), where \(y_{\text{erase}}\) is the prompt that includes the concept to be erased, and \(x_{\text{erase}}\) denotes any image typically representative of that concept.



Concept erasure can be achieved through various optimization methods. And in each optimization group, methods can get classified by which components are modified to achieve the goal. This section,  therefore, categorizes existing methods by their optimization strategies and components they modify.
A comprehensive taxonomy with detailed explanations is provided in Tab.~\ref{tab:taxonomy}.
For a detailed explanation of each component and inference stage, please refer to Sec.~\ref{sec:preliminaries}.


\subsection{Fine-tuning Methods} \label{subsec:finetuning}
Fine-tuning is one of the most intuitive methods to erase undesired concepts from the T2I models. These methods iteratively optimize weights of component of Stable Diffusion (SD) to match erasing concept to its designed corresponding concept. For example, match representation of erasing concept $c_{erase}$, ``Van Gogh'' to $c_{target}$, ``Artist''. We categorize this by which component is updated to erase concept. 

% \paragraph{Fine-tuning Latent Diffusion Model.}
% Fine-tuning the latent diffusion model (LDM) is a widely used approach for concept erasure, as it allows precise modification of generative capabilities at the image synthesis level. Unlike fine-tuning the text encoder, which modifies how prompts are interpreted, LDM fine-tuning directly alters how concepts are manifested in generated images. These methods primarily target U-Net's cross-attention layers, denoising network parameters, or classifier-free guidance terms to remove or suppress undesired content.

% Several works have explored targeted concept removal through fine-tuning U-Net's attention layers. Forget-Me-Not~\cite{forgetmenot} selectively fine-tunes cross-attention projection matrices to steer attention away from erased concepts while preserving generalization. ESD~\cite{esd} fine-tunes the denoising process of diffusion models to prevent the generation of specific objects, styles, or NSFW content. Selective Amnesia~\cite{selectiveamnesia} introduces a regularization-based continual learning approach, modifying the model incrementally to unlearn specific visual patterns while maintaining general generative ability.

% Overall, fine-tuning methods in LDMs enable direct control over concept erasure at the generative process level, ensuring that removed concepts are not reconstructable through adversarial prompting. However, these methods often require computationally expensive retraining and may introduce unintended degradation in model diversity and quality. Future work could explore more efficient fine-tuning strategies, including hybrid approaches that combine fine-tuning with closed-form updates for faster adaptation.

\paragraph{Fine-tuning  Latent Diffusion Model.}  To edit SD's Latent Diffusion Models (LDM) component, 
fine-tuning-based concept erasure methods selectively update model parameters to remove undesired concepts while preserving overall generative capabilities. These approaches can get categorized further based on the specific LDM's model components they modify, as different architectural elements govern distinct aspects of the image synthesis process.

A general formulation of fine-tuning for concept erasure in LDMs is as follows: \begin{equation}
    \min_{\theta} \left\| \Phi_{\theta}(z_t, c_{\text{erase}}) - \Phi_{0}(z_t, c_{\text{target}}) \right\|^{2}_2,
\end{equation}
where \( \Phi_{0} \) represents the pretrained LDM model, and \( \Phi_{\theta} \) denotes the fine-tuned LDM with updated parameters \( \theta \). The terms \( c_{\text{erase}} \) and \( c_{\text{target}} \) correspond to the text embeddings \( \mathcal{E}_{\text{txt}}(y_{\text{erase}}) \) and \( \mathcal{E}_{\text{txt}}(y_{\text{target}}) \), respectively. The objective enforces alignment between the erased concept \( c_{\text{erase}} \) and the target concept \( c_{\text{target}} \), ensuring that the model learns to replace undesired representations in the latent space.

One class of methods targets to update the \textbf{cross-attention layers in LDM}, which determine how textual prompts influence the generated visual output. FMN~\cite{Zhang2023ForgetMeNotLT} fine-tunes cross-attention module to re-steer attention mechanisms to eliminate certain concepts while maintaining generative quality. 
% All but One~\cite{Hong2023AllBO} modifies cross-attention layers by fine-tuning classifier guidance, ensuring that erased concepts are suppressed while maintaining the model’s ability to generate diverse and high-quality images. 
 AC~\cite{Ablating_Concept} introduces an anchor-based fine-tuning strategy, aligning erased concepts with broader semantic categories to suppress their stylistic or object-based representations.

Another set of methods fine-tunes the \textbf{LDM backbone}, directly modifying the denoising process to eliminate specific concepts from the model’s latent representations. 
ESD~\cite{esd} fine-tunes the LDM to match the noise prediction of \( c_{\text{erase}} \) to that of \( c_{\text{target}} \), ensuring erased concepts remain irrecoverable. This optimization is guided by classifier-free guidance (Eq.~\eqref{eq:classifier-free-guidance}), which directs the model's learning signal. During fine-tuning, ESD modifies either cross-attention or non-attention modules to reinforce robustness against adversarial prompts.
SALUN~\cite{fan2024salun} applies saliency-guided erasing, selectively updating high-impact weights to maximize forgetting while minimizing unintended side effects. 
DT~\cite{Ni2023DegenerationTuningUS} conditions the model to generate structurally degraded outputs when prompted with erased concepts, effectively neutralizing their representation in the latent space.

Several approaches incorporate geometric constraints, continual learning, or robustness against personalization to enhance fine-tuning methods. Geom-Erasing~\cite{Liu2023ImplicitCR} removes implicit visual concepts, such as watermarks and hidden signals, by introducing geometric constraints that disrupt structured artifacts without degrading unrelated content. SA~\cite{Heng2023SelectiveAA} leverages continual learning techniques, employing regularization-based forgetting to erase targeted concepts while preserving generalization and mitigating catastrophic forgetting. Lastly, IMMA~\cite{Zheng2023IMMAIT} adopts a preventive fine-tuning approach, modifying model weights preemptively to resist unauthorized adaptation via fine-tuning techniques, thereby preventing the downstream personalization of diffusion models for unethical or restricted purposes. Additionally, SPM~\cite{Lyu2023OnedimensionalAT} introduces a lightweight, one-dimensional adapter that enables precise and transferable concept erasure across different diffusion models.



\paragraph{Fine-tuning CLIP.}
Fine-tuning latent diffusion model showed great success for concept erasure, plus their interpretability to understand erasing concepts. However, to apply these methods to the updated LDM or other structured LDM, they have to get changed or redesigned since these methods are designed only to specific LDM. As one of the strength of fine-tuning CLIP model for concept erase, the CLIP text encoder whose concepts get erased, $\mathcal{E}_{\text{txt}}$, can transfer to the other structure LDM as long as they still depends on CLIP model. To fine-tune the CLIP model, \cite{safe_clip} generates a dataset composed of quadruplets of safe and unsafe text-image pairs (ViSU dataset). 
Similarly, \cite{Liu2024LatentGA} generates a dataset composed of safe and unsafe text pairs (CoPro dataset), where unsafe prompts are synthesized using a large language model, and safe counterparts are created by removing harmful concepts while preserving context.
After generating datasets, this study finetunes CLIP based on designed loss inspired by contrastive loss. This finetuned CLIP text encoder, $\mathcal{E'_{\text{txt}}}$, leads $\mathcal{E'}_{\text{txt}}(y_{erase}) \approx \mathcal{E}_{\text{txt}}(y_{target})$. Even if $y_{erase}$ is given to the SD model, it will generate images that are not aligned with concept to erase, $SD'(y_{erase}) \approx x_{target}$, where $SD'(y) = \mathcal{D}(\Tilde{\Phi}_{\theta}(z_T, \mathcal{E'}_{\text{txt}}(y)))$.
These methods offer better adaptability than LDM fine-tuning approaches. However, this approach requires a carefully curated and extensive dataset to fine-tune the existing text encoder, unlike other concept erasure methods.


\subsection{Closed-form Model Editing Methods} \label{subsec:closed}
Fine-tuning methods are intuitive and effective for modifying SD. However, they require iterative optimization through gradient descent, making them computationally expensive and time-consuming. Moreover, fine-tuning introduces risks of overfitting and unintended degradation of the model’s capabilities, necessitating careful hyperparameter tuning. In contrast, closed-form solutions provide a direct mathematical update to model parameters without iterative training. This enables faster application of model modifications while eliminating the need for extensive hyperparameter tuning. 

A general formulation of closed-form model editing follows a least squares based optimization:
\begin{equation} \label{eq:closed_form}
\min_{W} \left\| W c_{\text{erase}} - W_0 c_{\text{target}} \right\|_2^2,
\end{equation}
where \( W \) denotes the editable parameters of the model, primarily the key and value projection matrices in the cross-attention module, while \( W_0 \) represents the pre-trained weights. Closed-form solutions directly compute the optimal update for \( W \) directly, enabling efficient modification while preserving overall model coherence. To enhance stability and prevent unintended interference, regularization terms are incorporated into Eq.~\eqref{eq:closed_form}, balancing alignment with the target concept while minimizing deviations from the original model.

Notable closed-form methods include ReFACT~\cite{Arad2023ReFACTUT}, which applies a low-rank memory update to the CLIP text encoder, ensuring persistent factual knowledge updates while minimizing interference by unrelated concepts. TIME~\cite{Orgad2023EditingIA} modifies the LDM’s cross-attention projection matrices, aligning implicit assumptions in generated images with desired attributes. Unified Concept Editing (UCE)~\cite{Gandikota2023UnifiedCE} introduces a closed-form method for simultaneous multi-concept editing in T2I models, enabling scalable erasure, moderation, and debiasing by modifying cross-attention projections while minimizing interference against unedited concepts. MACE~\cite{Lu2024MACEMC} further refines cross-attention weights by integrating adapter-based concept erasure, achieving precise removal of up to 100 concepts in a more memory-efficient manner.

Other recent works have explored additional extensions of closed-form model editing. EMCID~\cite{Xiong2024EditingMC} introduces a two-stage framework combining self-distillation and closed-form updates, scaling to over 1,000 concurrent modifications. MUNBa~\cite{Wu2024MUNBaMU} formulates concept erasure as a Nash bargaining problem, deriving an equilibrium update that balances forgetting and preservation objectives.

Overall, closed-form methods offer a computationally efficient alternative to fine-tuning by providing direct parameter updates. These methods ensure fast and stable modifications. 





% \subsection{Inference Stage Control Methods} \label{subsec:inference}
% Both of fine-tuning methods and closed-form methods shows their intuitive and effective performance to erase concepts. Fine-tuning LDM and closed-form methods shows it is possible to edit diffusion process for mapping random noise to visual latent. And CLIP finetuning achieves its flexibility to plug-and-play to the other T2I models, which are using CLIP text encoder. 

% However, previous three categories of concept erasing methods commonly requires weight updates of SD's component. Inference stage control methods are developed for erasing concepts without modification of components of SD. Instead, edit diffusion process~\eqref{eq:classifier-free-guidance} or textual embedding $c$ or sanitizing prompt $y$ using LLM, such as LLaMA\cite{Touvron2023LLaMAOA}.



\subsection{Inference-time Intervention Methods} \label{subsec:inference}
Both fine-tuning and closed-form methods demonstrate intuitive and effective performance in concept erasure. Fine-tuning LDM and closed-form parameter updates enable direct control over the generative process.
%, modifying the mapping from random noise to visual latents. 
Among them,  CLIP fine-tuning provides a flexible plug-and-play concept erasure solution for T2I models that rely on CLIP encoders.

However, these approaches require weight modifications to SD components, limiting their adaptability and deployment efficiency. In contrast, inference-stage control methods enable concept erasure without modifying SD’s parameters. These methods instead intervene at the inference stage by modifying classifier-free guidance (Eq.~\eqref{eq:classifier-free-guidance}), editing textual embeddings \( c \), or sanitizing input prompts \( y \) using large language models.


\paragraph{Modifying Classifier-Free Guidance.}  
A core approach for inference-stage concept erasure is adjusting classifier-free guidance (Eq.~\eqref{eq:classifier-free-guidance}) to steer the generative process away from undesired content. Safe Latent Diffusion~\cite{sld} modifies the classifier-free guidance signal in SD’s denoising process, redirecting latent activations to prevent the generation of unsafe concepts. Anti-Memorization Guidance~\cite{Chen2024TowardsMD} introduces despecification and dissimilarity constraints that adjust classifier-free guidance dynamically, ensuring that models do not overfit to specific training instances or regenerate memorized images. Both methods leverage guidance re-weighting strategies to suppress undesired features while maintaining high image quality.

\paragraph{Editing Textual Embeddings.}  
Rather than modifying the diffusion process, another class of inference-stage methods operates on text embeddings to enforce concept erasure. SAFREE~\cite{safree} applies subspace projection and adaptive re-attention to detect and suppress undesirable content within CLIP text embeddings before they get used for image synthesis. Similarly, Content Suppression in T2I Models~\cite{Li2024GetWY} employs soft-weighted regularization to refine textual embeddings during sampling, ensuring that forbidden concepts do not appear in generated outputs. These methods enable fine-grained, token-level control over the generation process while preserving overall model flexibility.

\paragraph{Sanitizing Input Prompts Using LLMs.}  
A third category of inference-stage control leverages Large Language Models (LLM) to preprocess prompts, ensuring that user inputs do not contain prohibited content before the diffusion process begins. ORES ~\cite{ores} employs LLM-based query rewriting to automatically sanitize user prompts, replacing restricted terms with conceptually aligned yet safe alternatives. On the other hand, GuardT2I~\cite{Yang2024GuardT2IDT} detects adversarial prompts that attempt to bypass safety mechanisms, leveraging a fine-tuned LLM to analyze and reject unsafe queries before image generation. 

By operating externally to the SD components and modifying only the diffusion process at inference, these methods remain model-agnostic and scalable across different T2I model architectures.



% \subsection{Inference Stage Control Methods} \label{subsec:inference}
% Both fine-tuning and closed-form methods demonstrate intuitive and effective performance in concept erasure. Fine-tuning latent diffusion models (LDMs) and closed-form parameter updates enable direct control over the generative process, modifying the mapping from random noise to visual latents. Additionally, CLIP fine-tuning provides a flexible plug-and-play solution for text-to-image (T2I) models that rely on CLIP encoders.

% However, all three of these approaches require weight modifications to Stable Diffusion (SD) components. In contrast, inference-stage control methods have been developed to erase concepts without altering SD’s parameters. These methods instead manipulate the diffusion process during sampling by modifying classifier-free guidance Eq.~\eqref{eq:classifier-free-guidance}, editing the textual embedding \( c \), or sanitizing the input prompt \( y \) using large language models (LLMs) such as LLaMA~\cite{Touvron2023LLaMAOA}.

% One class of inference-stage methods directly modifies the diffusion process to steer the generation away from undesired content. SLD~\cite{sld} introduces classifier-free guidance modification to suppress unsafe content in latent space, ensuring that harmful concepts are naturally omitted from generated images. Similarly, SAFREE~\cite{safree} utilizes subspace projection and adaptive re-attention to detect and suppress unsafe features in both text embeddings and denoising steps, preventing undesired generations.

% Another approach operates by modifying the input prompt or its textual embedding to erase concepts. ORES~\cite{ores} leverages LLM-based query rewriting to sanitize unsafe user prompts, replacing prohibited content with more appropriate alternatives during diffusion sampling. Similarly, GuardT2I~\cite{guardt2i} detects adversarial prompts that attempt to bypass safety mechanisms by translating latent embeddings into natural language using a fine-tuned LLM, filtering out manipulative queries before image generation.

% Beyond safety-oriented methods, inference-stage control is also applied to prevent model memorization and enforce negative constraints. AMG (Anti-Memorization Guidance)~\cite{amg} introduces despecification and dissimilarity constraints during inference to prevent models from reproducing copyrighted or sensitive training data. Content Suppression in T2I Models~\cite{content_suppression} refines textual embedding optimization by applying soft-weighted regularization to ensure that forbidden content remains absent, even under adversarial prompting.

% Inference-stage control methods provide a training-free, model-agnostic alternative to fine-tuning-based concept erasure. These techniques enable dynamic intervention during generation, offering scalability and adaptability across different T2I architectures.


\section{Robustness} \label{sec:robustness}
Concept erasure methods aim to prevent T2I models from generating undesired concepts. For instance, when a model undergoes fine-tuning to erase a copyrighted artist’s style (e.g., Van Gogh), prompts such as ``cypresses by Van Gog'' should ideally produce outputs that bear no resemblance to the artist’s original style, as shown in Fig.~\ref{fig:overview}.

However, studies demonstrate that minor perturbations to the prompt—such as the addition of unrelated tokens or imperceptible modifications—can effectively circumvent concept erasure, allowing target T2I model to regenerate removed concepts. In some cases, even semantically meaningless inputs would  exploit the underlying representations within SD models to reconstruct erased concepts.

We categorize adversarial attacks based on whether they require access to the Latent Diffusion Model (LDM) of Stable Diffusion (SD). Following this, we also discuss existing defense strategies against such attacks.

\subsection{Adversarial Attacks} \label{subsec:adv_atk}
Adversarial attacks against T2I models manipulate prompt inputs or textual embeddings to reconstruct erased concepts. Given a concept-erased model, an adversarial prompt is optimized to elicit outputs that closely resemble those generated by an unaltered model. A general formulation of adversarial attacks against concept erasure is:
\begin{equation}
\underset{z_{\text{adv.}} \text{ or } y_{\text{adv.}}}{\mathrm{argmin}}  || \mathcal{SD'}(z_{\text{adv.}},y_{\text{adv.}}) - x_{\text{erase}} ||,    
\end{equation}
where \( \mathcal{SD'} \) denotes the concept-erased Stable Diffusion model, and \( z_{\text{adv.}} \) and \( y_{\text{adv.}} \) represent adversarial latents and prompts, respectively, that restore the erased concept within \( \mathcal{SD'} \).


\paragraph{Attacks with LDM Access}
These attacks access LDM’s latent representations, enabling adversaries to design strategies that systematically bypass concept erasure mechanisms. While such access is rare in real-world scenarios, these attacks remain essential for stress-testing erasure techniques.

One class of such attacks invert concept erasure transformations to recover removed concepts.  
Circumventing Concept Erasure~\cite{pham2024circumventing} optimizes adversarial embeddings via inversion using LDM, successfully retrieving erased concepts. Similarly, Concept Arithmetics~\cite{Petsiuk2024ConceptAF} reconstructs erased concepts by manipulating latent representations and leveraging semantic composition to synthesize forbidden attributes through linear combinations of concept embeddings.

Another category of attacks exploits prompt tuning and adversarial optimization to bypass safety-driven concept removal. P4D~\cite{p4d} systematically tunes adversarial prompts by iteratively refining textual inputs based on model feedback, demonstrating that safety filters and erasure techniques remain vulnerable to adversarial prompt engineering. Additionally, UnlearnDiffAtk~\cite{to_generate_or_not} specifically targets models trained with safety-driven unlearning by crafting prompts that reverse-engineer unlearning constraints, showing that adversarially optimized inputs can still induce the generation of prohibited content.

Although these attacks require privileged access to LDM, they provide valuable insights into the transferability of adversarial prompts across models. By assessing how an attack generalizes to different versions of SD and CLIP, these studies reveal broader vulnerabilities in concept erasure methods.

\paragraph{Attacks without LDM Access}  
Even without LDM access, adversarial methods effectively bypass concept erasures by manipulating prompt or textual embeddings, without interacting with the diffusion model’s internal denoising process.

PEZ~\cite{hard_prompt} formulates a discrete optimization problem to recover a text prompt that closely aligns with a given erased concept image, \( x_{\text{erase}} \), leveraging CLIP’s vision-language similarity for optimization without requiring access to the underlying diffusion model. 
% Similarly, MMA-Diffusion~\cite{Yang2023MMADiffusionMA} introduces adversarial attacks that operate within the CLIP text encoder embedding space or a multi-modal space, bypassing safety mechanisms by optimizing adversarial prompts and applying imperceptible perturbations to generated images. \qnote{unsure the left sentence, how can blackbox method add pertubation on the generated images?}
Similarly, MMA-Diffusion~\cite{Yang2023MMADiffusionMA}, like PEZ, employs a surrogate model for adversarial attacks, operating within CLIP’s text encoder embedding space or its multi-modal space. By optimizing adversarial prompts and introducing imperceptible perturbations, MMA-Diffusion successfully bypasses safety mechanisms, demonstrating its effectiveness in both text-to-image generation and image editing tasks.


Ring-A-Bell~\cite{Tsai2023RingABellHR} proposes a model-agnostic red-teaming framework that reconstructs erased concepts by optimizing adversarial prompts using a genetic algorithm in the text embedding space. Likewise, RIATIG~\cite{Liu2023RIATIGRA} employs a genetic optimization strategy to iteratively refine adversarial queries, enabling content moderation evasion across multiple T2I models. 
Meanwhile, UPAM~\cite{Peng2024UPAMUP} utilizes gradient-based prompt tuning combined with semantic-enhancing learning to systematically generate adversarial prompts capable of bypassing API-level safety mechanisms.

These approaches expose a fundamental vulnerability in concept erasure techniques—even without LDM access, adversarial optimization in the prompt space alone can effectively reconstruct erased content. This underscores the need for stronger prompt filtering mechanisms and adversarially resilient diffusion models to prevent circumvention through external manipulations.


% \subsection{Defensive Methods}
% For the necessity of developing robust methods against adversarial attacks~\ref{subsec:adv_atk}, concept erase methods inspired by adversarial training are proposed with following the categories in concept erasing~\ref{sec:method}. 

% RACE~\cite{race} proposes single diffusion time step adversarial attack to improve the efficiency of adversarial attack for SD. And they show that such attack can be used for adversarial finetuning for latent diffusion model. 
% Receler~\cite{receler} do something.
% AdvUlearn~\cite{advunlearn} do adversarial finetuning for CLIP-text encoder.
% RECE~\cite{rece} do adversarial finetuning on the top of close-form method.

% \subsection{Defensive Methods}
% To mitigate the vulnerabilities of concept-erased models against adversarial prompt attacks (Sec.~\ref{subsec:adv_atk}), recent research integrates adversarial training into concept erasure methods. These approaches aim to enhance robustness while maintaining the generation fidelity of diffusion models. We categorize these methods based on their modifications component (Sec.~\ref{sec:method}).

% R.A.C.E.~\cite{race} introduces a single-timestep adversarial attack strategy to efficiently identify vulnerabilities in SD. By leveraging this attack mechanism, the method performs adversarial fine-tuning to strengthen concept erasure. Receler~\cite{receler} employs a lightweight adversarial eraser embedded within the cross-attention layers of the diffusion model. It integrates concept-localized regularization to maintain generation quality while selectively erasing undesired concepts. Additionally, adversarial prompt learning is used to generate paraphrased attack prompts, improving the robustness of the model against semantic perturbations in textual prompts.

% AdvUnlearn~\cite{advunlearn} advances the robust erasing paradigm by introducing adversarial training on the CLIP text encoder rather than modifying the UNet of diffusion models. This approach enhances prompt-space robustness, making the model resistant to embedding-space adversarial attacks while preserving the original text-to-image alignment. 

% RECE~\cite{rece} extends closed-form method by incorporating adversarial finetuning on top of parameter-efficient matrix modifications in cross-attention layers. Unlike iterative fine-tuning-based methods, RECE efficiently discovers adversarial embeddings that can reconstruct erased concepts and removes them in a fully closed-form manner. In inference stage control method, SAFREE~\cite{safree} shows robust performances comparing with other difffense methods even such method does not employ adversarial training.
 
% By integrating adversarial robustness into concept erasure, these methods significantly improve the reliability of T2I safety mechanisms. However, challenges remain in optimizing the balance between robustness, generation quality, and computational efficiency, warranting further exploration in adaptive adversarial training strategies for future diffusion models.


\subsection{Defensive Methods}
% To mitigate the vulnerabilities of concept-erased models against adversarial prompt attacks (Sec.\ref{subsec:adv_atk}), recent research integrates adversarial training into concept erasure methods, aiming to enhance robustness while maintaining generation fidelity. These methods primarily target different architectural components, such as LDM, cross-attention weights, and text encoders, each playing a crucial role in enhancing robustness and mitigating adversarial vulnerabilities (Sec.~\ref{sec:method}).
To strengthen concept-erased models against adversarial prompt attacks (Sec.\ref{subsec:adv_atk}), recent research integrates adversarial training into concept erasure techniques, enhancing robustness while preserving generation fidelity. These defenses align with the categorization of concept erasure methods (Fig.\ref{fig:taxonomy}, Tab.\ref{tab:taxonomy}), targeting distinct architectural components. 
By addressing vulnerabilities across these optimization spaces, defensive methods improve the reliability of concept erasure while maintaining the expressiveness of T2I models.


R.A.C.E.\cite{RACE} employs a single-timestep adversarial attack to efficiently identify vulnerabilities in SD and leverages this attack for adversarial fine-tuning, significantly reducing attack success rates in both white-box and black-box settings. Receler\cite{Huang2023RecelerRC} integrates a lightweight robust eraser within cross-attention layers of LDM, utilizing concept-localized regularization and adversarial prompt learning to improve robustness against paraphrased attacks while preserving non-target concepts. AdvUnlearn~\cite{Zhang2024DefensiveUW} advances the robust erasing paradigm by applying adversarial training on the CLIP text encoder, enhancing prompt-space robustness while maintaining the alignment between textual prompts and image generation. RECE~\cite{Gong2024ReliableAE} extends closed-form concept erasure by incorporating adversarial fine-tuning on matrix-modified cross-attention layers, efficiently discovering and erasing adversarial embeddings in a fully closed-form manner. Additionally, in inference-stage control, SAFREE~\cite{safree} demonstrates strong robustness compared to other defense methods, despite not employing adversarial training.

By integrating adversarial robustness into concept erasure, these methods significantly improve the reliability of T2I safety mechanisms. However, challenges remain in optimizing the balance between robustness, generation quality, and computational efficiency, warranting further exploration in adaptive adversarial training strategies for future diffusion models.

% \qnote{can we make a table in the appendix summarizing the defense methods discussed in this section and get them categorized according to Figure 2 possible? }
\section{Evaluation} \label{sec:evaluation}
Evaluating concept erasure methods is essential for quantifying their effectiveness and enabling fair comparisons across different approaches. This section reviews widely adopted metrics and datasets to assess both the success of concept removal and the preservation of general model capabilities.

\subsection{Metrics}
Concept erasure methods are typically evaluated in two key aspects: erasure effectiveness and model fidelity. 

The Erasure Success Rate (ESR) measures how effectively a method removes a target concept. This is commonly assessed using classification accuracy, where a pre-trained classifier determines whether the erased concept remains present in the generated images. Formally, ESR is defined as:
\begin{equation}\label{eq:esr}
    \text{ESR} = \frac{1}{N} \sum_{i=1}^{N} \mathbbm{1}\left( f\left(SD'(y_i)\right) = c_{\text{erase}} \right),
\end{equation}
where \( y_i \) represents the prompt, \( c_{\text{erase}} \) is the erased concept, \( f \) is a classifier, and \( N \) is the total number of prompts. Lower ESR values indicate more successful erasure. ESR can also be extended to evaluate robustness against adversarial attacks by replacing standard prompts \( y_i \) with adversarially optimized prompts \( y_{\text{adv.}} \) or modifying latent variables \( z_{\text{adv.}} \), allowing an assessment of how well the model resists attempts to reconstruct erased concepts.

To ensure that erasure does not degrade the model's ability to generate non-erased content, model fidelity is evaluated by measuring both image quality and text-image alignment before and after concept removal. Fréchet Inception Distance (FID)~\cite{fid} is widely used to quantify changes in the perceptual quality of generated images. In addition to image quality, maintaining alignment between textual prompts and generated outputs is crucial. CLIP score~\cite{hessel2021clipscore} is commonly employed for this purpose, providing a similarity measure between the generated image and its corresponding textual prompt. Furthermore, ESR can also serve as a fidelity metric by computing it on prompts unrelated to the erased concept, denoted as \( y_{\text{non-erase}} \).


\subsection{Datasets}
Dataset selection depends on the nature of the concepts being erased, with commonly used datasets categorized according to their evaluation objectives. For assessing NSFW content removal, the I2P dataset~\cite{Schramowski2022SafeLD}, which consists of 4,703 real-world user-generated prompts, is widely employed. Object concept removal is typically evaluated using structured prompts such as ``A photo of [object class]'', enabling controlled experiments on whether erased objects still appear in generated images. Artistic style erasure often relies on ESD's artist prompt dataset~\cite{esd}, which provides standardized prompts referencing specific artistic styles.

To evaluate model fidelity, the COCO dataset~\cite{lin2014microsoft} is commonly used. This dataset enables FID-based image quality assessment and supports CLIP score evaluation for measuring text-image alignment. Beyond standard datasets, robustness evaluation requires datasets explicitly designed for testing adversarial vulnerabilities. 
For example, the MMA-Diffusion benchmark~\cite{Yang2023MMADiffusionMA} and Ring-A-Bell dataset~\cite{Tsai2023RingABellHR} feature adversarial prompts designed to evade concept erasure and systematically test its vulnerabilities.


Together, these metrics and datasets establish a comprehensive framework for evaluating concept erasure, ensuring that methods are assessed not only for their effectiveness in removing targeted concepts but also for their ability to maintain generative quality and resist adversarial attacks.

\section{Conclusions and Future Works}
\label{sec:Conclusions and Future Works}
In this work, we developed a framework for the runtime enforcement against STL formula. This framework inputs a signal and outputs a minimally modified signal that satisfy the formula. Specially, given an STL formula, we derive timed transducers for the atomic components, compose them according to the formula, and apply them to the input timed words, which are obtained by encoding the signal. We present detail procedure for signal encoding, translating STL temporal operators into timed transducers, and an enforcement algorithm. Our approach effectively enforces a signal against an STL property on CPS.

As in \cite{10.1145/3126500,10.1145/3092282.3092291,10.1109/TII.2019.2945520}, we plan to extend the work to accommodate bidirectionality and also extend the framework for more general STL formulas.


%\noindent \textit{Future Works.}  
%As in \cite{10.1145/3126500,10.1145/3092282.3092291,10.1109/TII.2019.2945520},  in a bidirectional framework involving an environment and a program, we require two enforcers—one for monitoring inputs to the controller from the environment and the other for monitoring outputs from the controller to the environment. These enforcers will (minimally) correct any erroneous inputs or outputs to ensure that a specified property is maintained. Therefore, we plan to extend the work to accommodate bidirectionality.


%Also, the translation from STL to timed transducer that we demonstrate is specifically designed for enforcement. However, a more general translation approach, such as from STL to hybrid automata, could also be explored for enforcement and broader applications. Therefore, a broader question we aim to address in the future is enforcement based on hybrid automata specifications, with the current STL to timed transducer translation serving as a foundational step.




% \newpage
%% The file named.bst is a bibliography style file for BibTeX 0.99c
\bibliographystyle{named}
\bibliography{refs}

\end{document}

