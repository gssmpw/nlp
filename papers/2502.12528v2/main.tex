\documentclass{article}

\usepackage{microtype}
\usepackage{graphicx}
\usepackage{subfigure}
\usepackage{booktabs}
\usepackage{hyperref}


\usepackage[accepted]{icml2025}

\usepackage{amsmath}
\usepackage{amssymb}
\usepackage{mathtools}
\usepackage{amsthm}
\usepackage{thmtools}

\usepackage[capitalize,noabbrev]{cleveref}

\theoremstyle{plain}
\newtheorem{theorem}{Theorem}[section]
\newtheorem{proposition}[theorem]{Proposition}
\newtheorem{lemma}[theorem]{Lemma}
\newtheorem{corollary}[theorem]{Corollary}
\theoremstyle{definition}
\newtheorem{definition}[theorem]{Definition}
\newtheorem{assumption}[theorem]{Assumption}
\theoremstyle{remark}
\newtheorem{remark}[theorem]{Remark}


\usepackage[textsize=tiny]{todonotes}



\icmltitlerunning{Contextual Linear Bandits with Delay as Payoff}
% \usepackage{xargs} % Used for new commands with optional arguments
\usepackage{soul}  % Used for custom comments
\usepackage{color} % Used for custom colors in comments
% \usepackage{comment}
\usepackage{xspace}
\usepackage{listings}
\usepackage{enumitem}
% \usepackage{ulem}
%% Note: Some commands for spacing Latin letters/abbreviations
\newcommand{\eg}{{\it e.g.,\ }}
\newcommand{\etal}{{\it et al.\ }}
\newcommand{\etc}{{\it etc.}}
\newcommand{\ie}{{\it i.e.,\ }}
\newcommand{\cf}{{c.f.}\xspace}
\newcommand{\aka}{{a.k.a.}\xspace}

%%%%%%%%%%%%%%%%%%%%%%%%%%%%%%%%%%%%%%%%%%%%%%%%
%% Commands for adding comments to the paper. %%
%%%%%%%%%%%%%%%%%%%%%%%%%%%%%%%%%%%%%%%%%%%%%%%%

\usepackage{booktabs}
\definecolor{oxfordblue}{rgb}{0.0, 0.13, 0.28}
\definecolor{harvardcrimson}{rgb}{0.79, 0.0, 0.09}
\definecolor{dartmouthgreen}{rgb}{0.05, 0.5, 0.06}
\definecolor{princetonorange}{rgb}{1.0, 0.56, 0.0}
\definecolor{yaleblue}{rgb}{0.06, 0.3, 0.57}
\definecolor{usccardinal}{rgb}{0.6, 0.0, 0.0}
\definecolor{uclablue}{rgb}{0.33, 0.41, 0.58}
\definecolor{msugreen}{rgb}{0.09, 0.27, 0.23}
\definecolor{cornellred}{rgb}{0.7, 0.11, 0.11}
\definecolor{pomegranate}{RGB}{192, 57, 43}
\definecolor{anti-pomegranate}{RGB}{43,178,192}
\definecolor{alizarin}{RGB}{231, 76, 60}
\definecolor{anti-belize}{RGB}{185, 41, 56}
\definecolor{belize}{RGB}{41, 128, 185}
\definecolor{peter}{RGB}{52, 152, 219}
\definecolor{green}{RGB}{22, 160, 133}
\definecolor{anti-green}{RGB}{160,22,118}
\definecolor{turquoise}{RGB}{26, 188, 156}
\definecolor{pumpkin}{RGB}{211, 84, 0}
\definecolor{anti-pumpkin}{RGB}{0,22,211}
\definecolor{carrot}{RGB}{230, 126, 34}
\definecolor{wisteria}{RGB}{142, 68, 173}
\definecolor{anti-wisteria}{RGB}{99,173,68}
\definecolor{amethyst}{RGB}{155, 89, 182}
\definecolor{nephritis}{RGB}{39, 174, 96}
\definecolor{anti-nephritis}{RGB}{174,39,117}

% \newcommand{\pzh}[1]{{#1}}
% \newcommand{\peng}[1]{{\color{red} #1}}
% \newcommand{\xingbo}[1]{{\textcolor{black}{#1}}}
% \newcommand{\wxb}[1]{{\textcolor{orange}{#1}}}
% \newcommand{\chen}[1]{{\color{green} #1}}

\newcommand{\penguin}[1]{{#1}}
\newcommand{\pzh}[1]{{#1}}
\newcommand{\peng}[1]{{#1}}
\newcommand{\zhenhui}[1]{{#1}}
\newcommand{\haoxiang}[1]{{#1}}
\newcommand{\yh}[1]{{#1}}
\newcommand{\fhx}[1]{{{#1}}}

% \newcommand{\penguin}[1]{{\color{red} #1}}
% \newcommand{\fhx}[1]{{\textcolor{orange}{#1}}}
% \newcommand{\zhenhui}[1]{{\color{carrot} #1}}

% \newcommand{\penguin}[1]{{\color{blue} #1}}
% \newcommand{\fhx}[1]{{\color{blue}{#1}}}
% \newcommand{\zhenhui}[1]{{\color{blue} #1}}

% \newcommand{\fanhx}[1]{{\color{blue}{#1}}}
\newcommand{\fanhx}[1]{{{#1}}}

%% Note: Comment this in to see all comments and unfinished text.
\newcommand{\todo}[1]{\textcolor{red}{[TODO] \emph{#1}}}
\newcommand{\cut}[1]{\textcolor{red}{\st{#1}}}
\newcommand{\sout}[1]{\cut{#1}}
\newcommand{\gray}[1]{\textcolor{gray}{#1}}


\newcommand{\systemname}{{\textit{SystemName}}}
\newcommand{\name}{{\textit{LitLinker}}}
    
% Capitalizing the first letter for section autorefs
\renewcommand{\sectionautorefname}{Section}
\renewcommand{\subsectionautorefname}{Section}
\renewcommand{\subsubsectionautorefname}{Section}


\begin{document}

\twocolumn[
\icmltitle{Contextual Linear Bandits with Delay as Payoff}
\icmlsetsymbol{equal}{*}

\begin{icmlauthorlist}
\icmlauthor{Mengxiao Zhang}{a}
\icmlauthor{Yingfei Wang}{b}
\icmlauthor{Haipeng Luo}{c}
\end{icmlauthorlist}

\icmlaffiliation{a}{University of Iowa}
\icmlaffiliation{b}{University of Washington}
\icmlaffiliation{c}{University of Southern California}
%\icmlaffiliation{comp}{Company Name, Location, Country}
%\icmlaffiliation{sch}{School of ZZZ, Institute of WWW, Location, Country}

%\icmlcorrespondingauthor{Firstname2 Lastname2}{first2.last2@www.uk}

% You may provide any keywords that you
% find helpful for describing your paper; these are used to populate
% the "keywords" metadata in the PDF but will not be shown in the document
\icmlkeywords{Machine Learning, ICML}

\vskip 0.3in
]
\printAffiliationsAndNotice{}


 

\begin{abstract}
A recent work by \citet{schlisselberg2024delay} studies a delay-as-payoff model for stochastic multi-armed bandits, where the payoff (either loss or reward) is delayed for a period that is proportional to the payoff itself.
While this captures many real-world applications, the simple multi-armed bandit setting limits the practicality of their results.
In this paper, we address this limitation by studying the delay-as-payoff model for contextual linear bandits.
Specifically, we start from the case with a fixed action set and propose an efficient algorithm whose regret overhead compared to the standard no-delay case is at most
$D\Delta_{\max}\log T$, where $T$ is the total horizon, $D$ is the maximum delay, and $\Delta_{\max}$ is the maximum suboptimality gap. 
When payoff is loss, we also show further improvement of the bound, demonstrating a separation between reward and loss similar to \citet{schlisselberg2024delay}.
Contrary to standard linear bandit algorithms that construct least squares estimator and confidence ellipsoid, the main novelty of our algorithm is to apply a phased arm elimination procedure by only picking actions in a \emph{volumetric spanner} of the action set, which addresses challenges arising from both payoff-dependent delays and large action sets.
We further extend our results to the case with varying action sets by adopting the reduction from~\citet{hanna2023contexts}.  
Finally, we implement our algorithm and showcase its effectiveness and superior performance in experiments.
\end{abstract}




\section{Introduction}
\label{sec:intro}

\begin{figure*}[tb]
    \centering
    \includegraphics[width=0.848\linewidth]{figs/circuitnn.pdf} 
    \caption{Illustration of differentiable CircuitNN. CircuitNN is designed based on differentiable NAND gates. After DAS is guided by PI and PO pairs of the truth table, CircuitNN can get the precise circuit architecture logic equivalent to the truth table.}
    \label{fig:circuitnn}
\end{figure*}

% 1. Describe the importance of logic synthesis
% 2. Existing Problems
% (a) Neural Architecture Search: Unstable, Predefined Setting, etc.
% (b) Circuit Generation: Probabilistic Model, Logic Equivalence

With the rapid advancement of technology, the scale of integrated circuits (ICs) has expanded exponentially. 
This expansion has introduced significant challenges in chip manufacturing, particularly concerning power and area metrics.
A primary objective in IC design is achieving the same circuit function with fewer transistors, thereby reducing power usage and area occupancy.

Logic synthesis~\cite{hachtel2005logicsynth}, a critical step in electronic design automation (EDA), transforms behavioral-level circuit designs into optimized gate-level circuits, ultimately yielding the final IC layout. 
The primary goal of logic synthesis is to identify the physical implementation with the fewest gates for a given circuit function. 
This task constitutes a challenging NP-hard combinatorial optimization problem. 
Current logic synthesis tools~\cite{brayton2010abc, wolf2013yosys} rely on human-designed heuristics, often leading to sub-optimal outcomes.

Differentiable architecture search (DAS) techniques~\cite{liu2018darts, chu2020darts} offer novel perspectives on addressing challenges in this problem.
Circuit functions can be represented through truth tables, which map binary inputs to their corresponding outputs. 
Truth tables provide a precise representation of input-output relationships, ensuring the design of functionally equivalent circuits.
Inspired by this, researchers~\cite{deepmind2024ai4sys, wang2024tnet} have begun exploring the application of DAS to synthesize circuits directly from truth tables.
Specifically, \citet{deepmind2024ai4sys} proposed CircuitNN, a framework that learns differentiable connection structures with logic gates, enabling the automatic generation of logic circuits from truth tables.
This approach significantly reduces the complexity of traditional circuit generation. 
Building on this, \citet{wang2024tnet} introduced T-Net, a triangle-shaped variant of CircuitNN, incorporating regularization techniques to enhance the efficiency of DAS.

Despite these advancements, several challenges remain. 
The computational complexity of DAS grows quadratically with the number of gates, posing scalability issues.
Although triangle-shaped architecture~\cite{wang2024tnet} partially mitigates this problem, redundancy persists. 
%Additionally, DAS is susceptible to converging to local optima, limiting the ability to search architectures that satisfy the given truth tables~\cite{liu2018darts}. 
%Furthermore, hyperparameters (network depth and layer width) require extensive searches, introducing complexity and prolonging the synthesis process. 
Additionally, DAS is susceptible to converging to local optima~\cite{liu2018darts} and hyperparameters (network depth and layer width) require extensive searches. 
The challenges arise from the vast search space in DAS. 
% Even with predefined settings for CircuitNN, finding a configuration that meets the truth table requires extensive trial and error during the DAS process. 
Intuitively, limiting the search space through predefined parameters (network depth, gates per layer, and connection probabilities) can significantly reduce the complexity.

Recent advances~\cite{openai2023gpt4, abramson2024alphafold3, esser2024sd3, li2024mar} in conditional generative models have demonstrated remarkable performance across language, vision, and graph generation tasks. 
Motivated by these developments, we propose a novel approach to circuit generation that generates preliminary circuit structures to guide DAS in generating refined circuits matching specified truth tables. 
Firstly, we introduce CircuitVQ, a tokenizer with a discrete codebook for circuit tokenization. 
Built upon our Circuit AutoEncoder framework~\cite{hou2022graphmae,li2023maskgae,wu2025mgvga}, CircuitVQ is trained through a circuit reconstruction task. 
Specifically, the CircuitVQ encoder encodes input circuits into discrete tokens using a learnable codebook, while the decoder reconstructs the circuit adjacency matrix based on these tokens.
Subsequently, the CircuitVQ encoder serves as a circuit tokenizer for CircuitAR pretraining, which employs a masked autoregressive modeling paradigm~\cite{chang2022maskgit, li2023mage}. 
In this process, the discrete codes function as supervision signals. 
After training, CircuitAR can generate discrete tokens progressively, which can be decoded into initial circuit structures by the decoder of the CircuitVQ. 
These prior insights can guide DAS in producing refined circuits that match the target truth tables precisely.

Our key contributions can be summarized as follows:
\begin{itemize}
\item We introduce CircuitVQ, a circuit tokenizer that facilitates graph autoregressive modeling for circuit generation, based on our Circuit AutoEncoder framework;
\item Develop CircuitAR, a model trained using masked autoregressive modeling, which generates initial circuit structures conditioned on given truth tables;
\item Propose a refinement framework that integrates differentiable architecture search to produce functionally equivalent circuits guided by target truth tables;
\item Comprehensive experiments demonstrating the scalability and capability emergence of our CircuitAR and the superior performance of the proposed circuit generation approach.
\end{itemize}

% Motivation
% (a) Diffusion (Vision, Graph), Autoregressive (Language, Vision)
% (b) Circuit Generation for Predefined Setting
% (c) Neural Architecture Search for Strict Logic Equivalence

% Contribution
% (a) Circuit Tokenizer (new transformer arch, training strategy)
% (b) CircuitAR (train and gen strategies, post-ar strategy)
% (c) Extensive Evaluation including BitD (Bit Distance) for Scalability

\section{Preliminaries}
\label{sec:preliminaries}
We first set up notations and mathematically formulate tasks.

\noindent\textbf{Language-Conditioned Imitation Learning (LC-IL)}. The task of LC-IL aims to train an agent to mimic expert behaviors from a given demonstration set $\mathcal{D}_d = \{(\mathbf{\tau}_i,l_i)\}_{i=1}^N$, where $l_i \in \mathcal{L} $ represents a task-specific language instruction. Each trajectory $\mathbf{\tau}_i\in\mathcal{T}$ consists of a sequence of state-action pairs $\mathbf{\tau}_i = \{(\mathbf{s}_j, \mathbf{a}_j)\}_{j=1}^T$ of the horizon length $T$. In robot manipulation tasks, action $\mathbf{a}_j\in\mathcal{A}$ corresponds to the control commands executed by the agent and state $\mathbf{s}_j = [\mathbf{p}_j; \mathbf{v}_j] \in\mathcal{S}$ records proprioceptive data $\mathbf{p}_j$ (\textit{e.g.,} joint positions, velocities) and visual inputs $\mathbf{o}_j\in\mathcal{O}$ (\textit{e.g.,} camera images) at the time step $j$. The objective of LC-IL is to find an optimal language-conditioned policy $\pi^*(\mathbf{a}|\mathbf{s},l): \mathcal{S}\times\mathcal{L}\mapsto\mathcal{A}$ via solving the supervised optimization as follows,
\begin{equation}\nonumber
    \pi^* \in \arg\min_{\pi} \mathbb{E}_{(\tau_i, l_i)\sim \mathcal{T}} \left[ \frac{1}{T} \sum_{(\mathbf{s}_j, \mathbf{a}_j) \sim \tau_i} \ell(\pi(\hat{\mathbf{a}}_j, \mathbf{s}_j|l_i),  \mathbf{a}_j)\right],
\end{equation}
where \(\ell(\cdot, \cdot)\) is a task-specific loss, such as mean squared error or cross-entropy. Training the policy \(\pi_\theta\) in an end-to-end fashion may require \textit{hundreds} of high-quality expert demonstrations to converge, primarily due to the high variance of visual inputs $\mathbf{o}$ and language instructions $l$.

% We study the problem of Language-Conditioned Imitation Learning ~\cite{rss21-gcil}, where the goal is to train an agent to perform tasks by conditioning its policy on both the state of the environment and language instruction. Formally, let \(\mathcal{O}\) be the observation space, \(\mathcal{A}\) the action space, and \(\mathcal{L}\) the language instruction space. The observation space \(\mathcal{O}\) typically includes visual or sensor data, such as images, that represent the partial observation of state \(\mathcal{S}\). The objective is to learn a policy \(\pi_\theta : \mathcal{O} \times \mathcal{L} \to \mathcal{A}\), parameterized by \(\theta\), that maps an observation \(o \in \mathcal{O}\) and a language instruction \(L \in \mathcal{L}\) to an action \(a \in \mathcal{A}\). We assume access to a dataset of expert demonstrations \(\mathcal{D}_{\operatorname{demo}} = \{(\{o_k^i, a_k^i\}_{i=1}^T, L_k)\}_{k=1}^N\), where each sample consists of a $T$-step observation-action trajectory and a corresponding language instruction \(L_k \in \mathcal{L}\). The goal is to train the policy \(\pi_\theta\) by minimizing the following loss function:
% \[
% \mathcal{L}(\theta) = \frac{1}{N} \sum_{k=1}^N \sum_{i=1}^T \ell(a_k^i, \pi_\theta(o_k^i, L_k)),
% \]
% where \(\ell(\cdot, \cdot)\) is a task-specific loss function, such as mean squared error or cross-entropy. 
\begin{table}
\centering
\caption{Comparison of different component designs in time contrast learning across mainstream vision-language pre-training. \vspace{1ex}
% The goal frame $o_g$ is typically set as the last frame $o_{T}$.
 }
\label{tab:comp}
\Large
\resizebox{\linewidth}{!}{ 
\begin{tabular}{llll}
\toprule
$\operatorname{Method}$      & \textcolor{black}{$\mathcal{P}(\mathcal{O}_{i})$}  & \textcolor{black}{$\mathcal{N}(\mathcal{O}_{i})$} & $\mathfrak{R}(\mathbf{v},\mathbf{l}_i)$  \\ \hline
$\operatorname{R3M}$         & $(o_0, o_{j>i})$      &  $(o_0,o_i,o_j^{\notin O_i})$   & $\operatorname{reward}(\mathbf{v},\mathbf{l}_i)$   \\    
$\operatorname{LIV}$         & $(o_T)$    &  $(o_T^{\notin O_i})$    & $\operatorname{cos}(\mathbf{v},\mathbf{l}_i)$  \\    
$\operatorname{DecisionNCE}$ & $(o_i,o_{j>i})$     &     $(o_i^{\notin O_i},o_{j>i}^{\notin O_i})$  & $\operatorname{cos}(\mathbf{v}_j-\mathbf{v}_i, \mathbf{l}_i)$  \\          
$\operatorname{AcTOL}$        & $(o_i,o_{j \in [T] \setminus \{i\}})$ & $(o_i,o_k: d_{i, k}>d_{i, j})$  & $-\Vert \operatorname{cos}(\mathbf{v}_i, \mathbf{l}_i)-\operatorname{cos}(\mathbf{v}_j, \mathbf{l}_i) \Vert_2 $     \\  \bottomrule                                                              
\end{tabular}
}
\end{table}

\paragraph{Vision-language Pre-training.}  Address such scalability issues can be achieved by leveraging large-scale, easily accessible human action video datasets $\mathcal{D}_p = \{(\mathcal{O}_i, l_i)\}_{i=1}^M$ \cite{corr18-epickitchen,cvpr22-ego4d}, where $\mathcal{O}_i=\{o_j\}_{j=1}^T$ represents a video clip with $T$ frames and $l_i$ the corresponding description. Pretraining on such datasets enables policies to rapidly learn visual-language correspondences with minimal expert demonstrations. Mainstream pretraining methods employ time contrastive learning \cite{icra18-tcn} to fine-tune a visual encoder $\mathcal{\phi}$ and a text encoder $\mathcal{\varphi}$, which project frames and descriptions into a shared $d$-dimensional embedding space, \textit{i.e.}, $\mathbf{v}_j = \phi(o_j)\in\mathbb{R}^d$ and $\mathbf{l}_i = \varphi(l_i)\in\mathbb{R}^d$. To provide a unified perspective on various pretraining approaches, we formulate them within the objective $\mathcal{L}_{\operatorname{tNCE}}(\phi, \varphi)$: \vspace{-2ex}
\begin{align}\nonumber\small
\mathcal{L}_{\operatorname{tNCE}}&=
-\mathbb{E}_{\substack{\scriptstyle o^+\sim\textcolor{black}{\mathcal{P}(\mathcal{O}_i)}}}
    \log  
    \frac{
        \exp(\mathfrak{R}(\mathbf{v}^+, \mathbf{l}_i))
    }{
        \mathbb{E}_{\scriptstyle o^- \sim \textcolor{black}{\mathcal{N}(\mathcal{O}_i)}}
        \exp(\mathfrak{R}(\mathbf{v}^-, \mathbf{l}_i))
    },
\end{align}

% \begin{align}\nonumber\small
% \mathcal{L}_{\operatorname{tNCE}}&=
% -\mathbb{E}_{\substack{\scriptstyle o\sim O_i \\ \scriptstyle o^+\sim\textcolor{black}{\mathcal{P}(o)}}}
%     \log  
%     \frac{
%         \exp(\mathfrak{R}(\mathbf{v}^+, \mathbf{v}, \mathbf{l}_i))
%     }{
%         \mathbb{E}_{\scriptstyle o^- \sim \textcolor{black}{\mathcal{N}(o)}}
%         \exp(\mathfrak{R}(\mathbf{v}, \mathbf{v}^-, \mathbf{l}_i))
%     },\vspace{-2ex}
% \end{align}
% where $\mathbf{v} = \phi(o)$, and 
where $\mathbf{v}^{+/-} = \phi(o^{+/-})$. Different pretraining strategies differ in their selection of (1) the positive frame set $\mathcal{P}(\mathcal{O}_i)$, (2) negative frame set $\mathcal{N}(\mathcal{O}_i)$; and (3) the semantic alignment scoring function $\mathfrak{R}(\mathbf{v}, \mathbf{l}_i)$ measuring the gap of VL similarities as detailed in Table \ref{tab:comp}. 

\noindent\textbf{Discussion.} As motivated by goal-conditioned RL \cite{nips17-her}, current approaches \textit{explicitly} select future frames (\textit{e.g.}, DecisionNCE) or the last frame (\textit{e.g.}, LIV) as the goal within the positive set, enforcing their visual embedding to align with the semantics. Likewise, the scoring functions $\mathfrak{R}$ are often designed to maximize this transition direction. However, the pretraining action videos are \textit{noisy} as actions may terminate early or include irrelevant subsequent actions, which may mislead the encoders and result in inaccurate vision-language association. As detecting precise action boundaries is non-trivial, we argue for a more flexible approach that leverages \textit{intrinsic} characteristics of actions to guide pretraining.



% we first pre-train a visual encoder \(\mathcal{\phi}: \mathcal{O} \to \mathbb{R}^d\) and a text encoder \(\mathcal{\varphi}: \mathcal{L} \to \mathbb{R}^d\) to learn mappings from the observation and the language instruction space to $d-$dimensional feature spaces. This pre-training can be done using large, less-expensive data without action annotation, such as human action videos . Then, with the frozen learned features \(\boldsymbol{v}\) and \(\boldsymbol{l}\) as input, we can only fine-tune a simple Multi-Layer Perceptron (MLP) with a few demonstrations to learn the map from the feature space \(\mathbb{R}^d \times \mathbb{R}^d\) to the action space \(\mathcal{A}\). Since both the observation space \(\mathcal{O}\) and the action space \(\mathcal{A}\) are continuous and ordered over time, we expect the representations learned through pre-training to also exhibit continuity and orderliness. This property in the representations allows for better learning of the continuous mapping between observations and actions. This property offers three significant benefits: First, the orderliness of the representation ensures that different states of the task, such as the start and end of an action, can be better captured and distinguished. Second, the continuity of the representation allows it to evolve smoothly as the task progresses, enabling the model to output stable actions based on the current state. Finally, we can demonstrate that even under small perturbations to the language instruction, these properties ensure the robustness of the learned representation. This robustness is crucial for maintaining performance in real-world scenarios where language instructions might contain minor ambiguities or variations.





% We consider a partially observable Markov Decision Process (POMDP) with language conditions, which models the interaction between an agent and an environment where observations are incomplete and actions are guided by natural language instructions. Formally, a POMDP is defined as a tuple $\langle \mathcal{S}, \mathcal{A}, \mathcal{O}, \mathcal{T}, \mathcal{R}, \mathcal{Z}, \gamma \rangle$, where $\mathcal{S}$ is the state space, $\mathcal{A}$ is the action space available to the agent. $\mathcal{O}$ is the observation space, which provides partial information about the environment. $\mathcal{T}(s' \mid s, a)$ is the state transition function. $\mathcal{R}(s, a)$ is the reward function. $\mathcal{Z}(o \mid s, a)$ is the observation function. $\gamma \in [0, 1)$ is the discount factor.

% To incorporate language instructions, we introduce a task description $L$, which specifies the agent's goal in natural language. The task description conditions the agent's policy $\pi(a \mid o, L)$, where $o$ is the agent's current observation. The agent aims to maximize the expected cumulative reward while adhering to the task described by $L$.

% Further, we assume the availability of a large-scale human action video dataset including $N$ video-instruction pairs, $\{(\{o_k^i\}_{i=1}^{t_k}, L_k)\}_{k=1}^N$, where each pair representing an action video with $t_k$ frames and its corresponding language description $L_k$. We pre-train the visual and language encoders on this dataset, with the visual features $\boldsymbol{v} = \operatorname{Enc}_v(o)$ and the language features $\boldsymbol{l} = \operatorname{Enc}_l(L)$. These pre-trained representations are then frozen and applied to train the policy $\pi$ in the aforementioned decision-making process, enabling the agent to better interpret and act upon language-conditioned tasks.
\section{First Step: Non-Contextual Linear Bandits}\label{sec: linear}

In this section, we focus on the non-contextual case, which serves as a building block for eventually solving the contextual case. Before introducing our algorithm, we first briefly introduce the successive arm elimination algorithm for the simpler MAB setting proposed by \citet{schlisselberg2024delay} and their ideas of handling payoff-dependent delay. Specifically, their algorithm starts with a guess $B=1/D$ on the optimal action's loss, and maintains an active set of arms. The algorithm pulls each arm in the active set once, and constructs two LCB's (lower confidence bound) and one UCB (upper confidence bound) for each action in the active set as follows (supposing the current round being $t$):
\begin{align}
        \LCB_{t,1}(a) &= \frac{1}{\cnt_t(a)}\sum_{\tau\in\obs_t(a)}u_\tau - \sqrt{\frac{2\log T}{\cnt_t(a)}}, \label{eqn:lcb-1-mab}\\
        \LCB_{t,2}(a) &= \frac{1}{\unbiasSize_t(a)}\sum_{\tau\in\unbias_t(a)}u_{\tau} - \sqrt{\frac{2\log T}{\unbiasSize_t(a)\vee 1}}, \label{eqn:lcb-2-mab}\\
        \UCB_{t}(a) &= \frac{1}{\unbiasSize_t(a)}\sum_{\tau\in\unbias_t(a)}u_{\tau} + \sqrt{\frac{2\log T}{\unbiasSize_t(a)\vee 1}},\label{eqn:ucb-mab}
\end{align}
where $\cnt_t(a) = \sum_{\tau=1}^t\mathbbm{1}\{a_t=a\}$ is the total number of pulls of action $a$ till round $t$, $\obs_t(a) = \{\tau: \tau+d_{\tau}\leq t \text{~and~} a_{\tau}=a\}$ is the set of rounds where action $a$ is chosen and its loss has been received by the end of round $t$, $\unbias_t(a) = \{\tau \leq t-D: a_\tau = a\}$ is the set of rounds up to $t-D$ where action $a$ is chosen (so its loss has for sure been received by the end of round $t$), 
and $\unbiasSize_t(a)=|\unbias_t(a)|$. Specifically, \pref{eqn:lcb-1-mab} constructs an LCB of action $a$ assuming all the action's unobserved loss to be $0$ (the smallest possible), while \pref{eqn:lcb-2-mab} and \pref{eqn:ucb-mab} construct an LCB and a UCB using only the losses no later than round $t-D$ (which must have been received by round $t$), making the empirical average a better estimate of the expected loss. With $\UCB_t(a)$ and $\LCB_t(a) = \max\{\LCB_{t,1}(a), \LCB_{t,2}(a)\}$ constructed, the algorithm eliminates an action $a$ if its $\LCB_t(a)$ is larger than $\min\{\UCB_t(a'),B\}$ for some $a'$ in the active set. If all the actions are eliminated, this means that the guess $B$ on the optimal loss is too small, and the algorithm starts a new epoch with $B$ doubled.\footnote{In fact, \citet{schlisselberg2024delay} construct yet another LCB based on the number of unobserved losses. We omit this detail since we are not able to use this to further improve our bounds for linear bandits.}

\paragraph{Challenges} However, this approach cannot be directly applied to linear bandits. Specifically, standard algorithms for stochastic linear bandits without delay (e.g., \citet{li2010contextual,abbasi2011improved}) all construct  UCB/LCB for each action by constructing an ellipsoidal confidence set for $\theta$. In the delay-as-payoff model, while it is still viable to construct UCB/LCB similar to \pref{eqn:lcb-2-mab} and \pref{eqn:ucb-mab} via a standard confidence set of $\theta$, it is difficult to construct an LCB counterpart similar to \pref{eqn:lcb-1-mab}.
This is because one action's loss is estimated using observations of all other actions in linear bandits, and naively treating the unobserved loss of one action as zero might not necessarily lead to an underestimation of another action. 

\paragraph{Our ideas} To bypass this barrier, we give up on estimating $\theta$ itself and propose to construct UCB/LCB for each action using the observed losses of the \emph{volumatric spanner} of the action set. A volumetric spanner of an action set $\calA$ is defined such that every action in $\calA$ can be expressed as a linear combination of the spanner. 

\begin{definition}[Volumetric Spanner~\citep{hazan2016volumetric}]\label{def:volume}
Suppose that $\calA = \{a_1, a_2, \dots , a_N\}$ is a set of vectors in $\R^n$. We say $\calS\subseteq \calA$ is a \emph{volumetric spanner} of $\calA$ if for any $a\in \calA$, we can write it as $a=\sum_{b\in \calS}\lambda_b\cdot b$ for some $\lambda\in \R^{|\calS|}$ with $\|\lambda\|_2\leq 1$. 
\end{definition}

Due to the linear structure, it is clear that the loss $\mu_a$ of action $a$ can be decomposed in a similar way as $\sum_{b\in \calS}\lambda_b \mu_b$,
making it possible to estimate every action's loss by only estimating the loss of the spanner.
Moreover, such a spanner can be efficiently computed:
\begin{proposition}[\citet{bhaskara2023tight}]\label{prop:volume}
Given a finite set $\calA$ of size $K$, there exists an efficient algorithm finding a volumetric spanner $\calS$ of $\calA$ with $|\calS|=3n$ within $\order(Kn^3\log n)$ runtime.
\end{proposition}

\setcounter{AlgoLine}{0}
\begin{algorithm}
\caption{Phased Elimination via Volumetric Spanner for Linear Bandits with Delay-as-Loss}\label{alg:lossLB}

\nl Input: maximum possible delay $D$, action set $\calA$, $\beta>0$. 

\nl Initialization: optimal loss guess $B=1/D$.

\nl Initialization: active action set $\calA_1=\calA$. \label{line: restart}

 \For{$m=1,2,\dots,$}{
    \nl Find $\calS_m=\{a_{m,1},\dots,a_{m,|\calS_m|}\}$, a volumetric spanner of $\calA_m$ with $|\calS_m|= 3n$. \label{line:volume}
    
    \nl Pick each $a\in \calS_m$ $2^m$ times in a round-robin way. \label{line:round-robin}

    \nl Let $\calI_m$ contain all the rounds in this epoch.
    
    \nl For each $a\in \calS_m$, calculate the following quantities: \label{line:spanner-ucb-lcb}
    {\small
    \begin{align}
        &\hat{\mu}_{m}^+(a)=\frac{1}{2^m}\Big(\sum_{\tau\in \obs_m(a)}u_{\tau} + \sum_{\tau\in \unobs_m(a)}1\Big), \label{eqn:mean-up}\\
        &\hat{\mu}_{m}^-(a)=\frac{1}{2^m}\sum_{\tau\in \obs_m(a)}u_{\tau}, \label{eqn:mean-low}\\
        &\hat{\mu}_{m,1}^{+}(a)=\hat{\mu}^{+}_{m}(a)+\frac{\beta}{2^{m/2}}\|a\|_2, \label{eqn:loss-ucb-linear-1}\\
        &\hat{\mu}_{m,1}^{-}(a)=\hat{\mu}^{-}_{m}(a)-\frac{\beta}{2^{m/2}}\|a\|_2,\label{eqn:loss-lcb-linear-1}\\
        &\hat{\mu}_m^{F}(a)=\frac{1}{\unbiasSize_m(a)}\sum_{\tau\in \unbias_m(a)}u_{\tau}, \label{eqn:mean_unbiased}\\
        &\hat{\mu}_{m,2}^{+}(a)=\hat{\mu}_m^F(a)+\frac{\beta}{\sqrt{\unbiasSize_m(a)}}\|a\|_2, \label{eqn:loss-ucb-linear-2}\\
        &\hat{\mu}_{m,2}^{-}(a)=\hat{\mu}_m^F(a)-\frac{\beta}{\sqrt{\unbiasSize_m(a)}}\|a\|_2, \label{eqn:loss-lcb-linear-2}
    \end{align}
    }
    where $\unbiasSize_m(a) = |\unbias_m(a)|$, $\unbias_m(a) = \{\tau\in \calI_m: \tau+D\in\calI_m, a_{\tau}=a\}$, $\obs_m(a) = \{\tau\in \calI_m: \tau+d_{\tau}\in\calI_m, a_{\tau}=a\}$, and
    $\unobs_m(a)= \{\tau\in \calI_m: a_{\tau}=a\}\setminus\obs_m(a)$.

    \For{each $a\in \calA_m$}{
        \nl \label{line: decompose}
        Decompose $a$ as $a=\sum_{i=1}^{|S_m|}\lambda_{m,i}^{(a)}a_{m,i}$ with $\|\lambda_{m}^{(a)}\|_2\leq 1$ and calculate 
        {\small
        \begin{align}
            &\UCB_{m}(a)=\sum_{i=1}^{|\calS_m|}\lambda_{m,i}^{(a)}\cdot\hat{\mu}_{m,2}^{\sgn(\lambda_{m,i}^{(a)})}(a_{m,i}), \label{eqn:loss-ucb-f-all-action} \\
            &\LCB_m(a) = \max_{j\in \{1,2\}}\{\LCB_{m,j}(a)\} \;\;\text{where} \nonumber  \\
            & \LCB_{m,j}(a)=\sum_{i=1}^{|\calS_m|}\lambda_{m,i}^{(a)}\cdot\hat{\mu}_{m,j}^{\sgn(-\lambda_{m,i}^{(a)})}(a_{m,i}),\label{eqn:loss-lcb-all-action}
        \end{align}
        }
    }
    
    \nl Set $\calA_{m+1} = \calA_m$.
    
    \For{$a\in \calA_m$}{
        \nl \label{line:eliminate}  
        \If{$\exists a'\in \calA_m$, s.t. $\LCB_m(a) \geq \min\{\UCB_m(a'),B\} $}
        {
          Eliminate $a$ from $\calA_{m+1}$.
        }
    }
    \nl \If{$\calA_{m+1}=\emptyset$}{
        Set $B\leftarrow 2B$ and go to \pref{line: restart}.
    }
}
\end{algorithm}

Equipped with the concept of volumetric spanner, we are now ready to introduce our algorithm (see \pref{alg:lossLB}). 
Specifically, our algorithm also makes a guess $B$ on the loss of the optimal action. 
With this guess, it proceeds to multiple epochs of arm elimination procedures, with the active action set initialized as $\calA_1 = \calA$.
In each epoch $m$, instead of picking every action in the active set $\calA_m$, we first compute a volumetric spanner $\calS_m$ of $\calA_m$ with $|\calS_m|=3n$ (\pref{line:volume}), which can be done efficiently according to \pref{prop:volume}, 
and then pick each action in the spanner set $\calS_m$ for $2^m$ rounds in a round-robin way (\pref{line:round-robin}).

After that, we calculate two UCBs and two LCBs for actions in the spanner, in a way similar to the simpler MAB setting discussed earlier (\pref{line:spanner-ucb-lcb}).
Specifically, 
the first one is in the same spirit of \pref{eqn:lcb-1-mab}:
we calculate $\hat{\mu}_m^+(a)$ ( $\hat{\mu}_m^-(a)$) as an overestimation (underestimation) of the expected loss of action $a$ by averaging over all observed losses from the rounds in $\obs_m(a)$ as well as the maximum (minimum) possible value of unobserved losses from the rounds in $\unobs_m(a)$; see \pref{eqn:mean-up} and \pref{eqn:mean-low}.
The first UCB (LCB) $\hat{\mu}_{m,1}^+(a)$ ($\hat{\mu}_{m,1}^-(a)$) is then computed based on $\hat{\mu}_m^+(a)$ ($\hat{\mu}_m^-(a)$) by incorporating a standard confidence width $\frac{\beta}{\sqrt{2^m}}\|a\|_2$ for some coefficient $\beta$; see \pref{eqn:loss-ucb-linear-1} and \pref{eqn:loss-lcb-linear-1}.
Then, to compute the second UCB/LCB, which is in the same spirit as \pref{eqn:lcb-2-mab} and \pref{eqn:ucb-mab}, we calculate an unbiased estimation $\hat{\mu}_m^F(a)$ of the expected loss of $a$ by averaging losses from the rounds in $\unbias_m(a)$, that is, all the rounds where the observation must have been revealed; see \pref{eqn:mean_unbiased}.
Note that the number of such rounds, $\unbiasSize_m(a) = |\unbias_m(a)|$, is a fixed number, and thus $\hat{\mu}_m^F(a)$ is indeed unbiased.
Similarly, we incorporate a standard confidence width $\frac{\beta}{\sqrt{c_m(a)}}\|a\|_2$ to arrive at the second UCB $\hat{\mu}_{m,2}^+(a)$ and LCB $\hat{\mu}_{m,2}^-(a)$; see \pref{eqn:loss-ucb-linear-2} and \pref{eqn:loss-lcb-linear-2}.

The next step of our algorithm is to use these UCBs/LCBs for the spanner to compute corresponding UCBs/LCBs for every active action in $\calA_m$ (\pref{line: decompose}). Specifically, for each action $a\in \calA_m$, according to the definition of a volumetric spanner (\pref{def:volume}), we can write $a$ as a linear combination of actions in $\calS_m$: $\sum_{i=1}^{|S_m|}\lambda_{m,i}^{(a)}a_{m,i}$. As mentioned, due to the linear structure of losses, we also have $\mu_a = \sum_{i=1}^{|S_m|}\lambda_{m,i}^{(a)}\mu_{a_{m,i}}$.
Thus, when constructing a UCB (or similarly LCB) for $a$, based on whether $\lambda_{m,i}^{(a)}$ is positive or not, we decide whether to use the UCB or LCB of $a_{m,i}$; see \pref{eqn:loss-ucb-f-all-action}, a counterpart of \pref{eqn:ucb-mab}, and \pref{eqn:loss-lcb-all-action}, a counterpart of \pref{eqn:lcb-1-mab} and \pref{eqn:lcb-2-mab}.\footnote{This also explains why we need $\hat{\mu}_m^+(a)$, a quantity not used in~\citet{schlisselberg2024delay}.}

At the end of an epoch, we eliminate all actions from the active action set if their LCB is either larger than the UCB of certain action or the guess $B$ on the optimal loss  (\pref{line:eliminate}). 
If the active set becomes empty, this means that the guess $B$ is too small, and the algorithm restarts with the guess doubled; 
otherwise, we continue to the next epoch.

\paragraph{Theoretical performance}
We prove the following regret bound for our algorithm. 
\begin{restatable}{theorem}{lossLB}
\label{thm:main-non-contextual}
    \pref{alg:lossLB} with $\beta=\sqrt{2\log(KT^3)}$ guarantees: 
\begin{align*}
        \Reg &\leq \order\left(\min\left\{V_1,V_2\right\}\right) + \log(d^\star)\cdot \order\left( \min\left\{W_1,W_2\right\}\right),
    \end{align*}
    where $V_1=\frac{n^2\log(KT)\log(T/n)\log(d^\star)}{\Delta_{\min}}$, $V_2=n\sqrt{T\log(d^\star)\log(KT)}$, $W_1=nd^\star\log (T/n)+D\Delta_{\max}$, and $W_2=D\Delta_{\max}\log (T/n)$. 
\end{restatable}
The first term in the regret bound $\order\left(\min\left\{V_1,V_2\right\}\right)$ is of order $\otil(\min\{\frac{n^2}{\Delta_{\min}}, n\sqrt{T}\})$, which matches the standard regret bound of LinUCB in the case without delay~\citep{abbasi2011improved}.
The second term is the overhead caused by delay and is in the same spirit as the result of~\citet{schlisselberg2024delay}:
focusing only on the part that grows in $T$, 
we see that $W_1$ only depends on $d^\star$, the expected delay of the optimal action (and hence the smallest delay among all actions),
while $W_2$ depends on the maximum possible delay $D$ but scaled by $\Delta_{\max}$, the largest sub-optimality gap.
Therefore, the delay overhead of our algorithm is small when either the shortest delay is small or all actions have similar losses.
We remark again that in the delay-as-reward setting, we obtain similar results; see \pref{app: reward} for details.

\subsection{Analysis}\label{sec: alg}
In this section, we provide a proof sketch of \pref{thm:main-non-contextual}. Detailed proofs are deferred to \pref{app:loss}.

The proof starts by proving that $\UCB_m(a)$ and $\LCB_m(a)$ are indeed valid UCB and LCB respectively for all actions in $\calA_m$. 
This follows from first using standard concentration inequalities to show that $\hat{\mu}_{m,1}^+(a)$ and $\hat{\mu}_{m,2}^+(a)$ ($\hat{\mu}_{m,1}^-(a)$ and $\hat{\mu}_{m,2}^-(a)$) are valid UCBs (LCBs) for each action in the spanner, 
and then generalizing it to every action $a \in \calA_m$ according to its decomposition over the actions in the spanner.

With this property, our analysis then proceeds to control the regret of \pref{alg:lossLB} for each guess of $B$ separately. Let $\calT_B$ be the set of rounds when \pref{alg:lossLB} runs with guess $B$. 
In \textbf{Step 1}, we first show that the use of $\LCB_{m,2}(a)$ and $\UCB_m(a)$ ensures a regret bound of $\order\left(\min\{R_1,R_2\}+D\Delta_{\max}\log(T/n)\right)$ where $R_1=\frac{n^2\log(KT)\log(T/n)}{\Delta_{\min}}$ and $R_2=n\sqrt{|\calT_B|\log(KT)}$,
and then in \textbf{Step 2}, we show that the use of $\LCB_{m,1}(a)$ and $\UCB_m(a)$ ensures a regret bound of
$\order(\min\{R_1,R_2\}+(nd^\star+DB)\log(T/n)+D\Delta_{\max})$.

\paragraph{Step 1}
For notational convenience, we define 
\begin{align*}
    \rad_{m,a}^F=\beta\sum_{i=1}^{|\calS_m|}|\lambda_{m,i}^{(a)}|\cdot\frac{\|a\|_2}{\sqrt{\unbiasSize_m(a_{m,i})}}
\end{align*}
to be the total confidence radius of action $a$ coming from the definition of $\LCB_{m,2}(a)$ and $\UCB_m(a)$. 
Via a standard analysis of arm elimination, 
we show that that if an action $a$ is not eliminated at the end of epoch $m$, we have
\begin{align*}
    \Delta_a \leq 4\max_{a\in\calA_m}\rad_{m,a}^F \leq \frac{4\sqrt{3n}\beta}{\min_{a_m\in\calS_m}\sqrt{\unbiasSize_m(a_m)}},
\end{align*}
where the second inequality uses Cauchy-Schwarz inequality and the properties of volumetric spanners, specifically that $\|\lambda_{m}^{(a)}\|_2\leq 1$ and $|\calS_m|=3n$. To provide a lower bound on $c_m(a')$ for any $a'\in\calS_m$, note that we pick each action $a'\in \calS_m$ $2^m$ times in a round-robin manner, and thus
\begin{align*}
    c_m(a') \geq 2^m - \frac{D}{|\calS_m|}-1 = 2^m - \frac{D}{3n}-1.
\end{align*}
Rearranging the terms, we then obtain
\begin{align}\label{eqn:epoch_bound_1}
    2^m\Delta_a \leq \frac{48n\beta^2}{\Delta_a} + \frac{D\Delta_a}{3n} + \Delta_a.
\end{align}
Taking summation over all $a\in\calS_m$ and $m$, and noticing that the total number of epochs is bounded by $M=\lceil\log_2(|\calT_B|/3n)\rceil$, we arrive at the following $\order(R_1+D\Delta_{\max}\log(T/n))$ regret guarantee:
\begin{align}
&\sum_{m=1}^{M}\sum_{a\in\calS_m}2^m\Delta_a \nonumber\\
    &\leq \sum_{m=1}^{M}\sum_{a\in\calS_m,\Delta_a>0}2\cdot\left(\frac{48n\beta^2}{\Delta_a}+\frac{D\Delta_a}{3n}+\Delta_a\right) \nonumber\\
    &\leq \sum_{m=1}^{M}\sum_{a\in\calS_m,\Delta_a>0}\order\left(\frac{n\log (KT)}{\Delta_a}\right) + \order\left(D\Delta_{\max}\log(T/n)\right),\nonumber \\
    &\leq \order\left(\frac{n^2\log(T/n)\log (KT)}{\Delta_{\min}}\right) + \order\left(D\Delta_{\max}\log(T/n)\right),\nonumber
\end{align}
where the first inequality is because $a\in\calS_m$ is not eliminated in epoch $m-1$ and the last inequality is by lower bounding $\Delta_a$ by $\Delta_{\min}$.

To obtain the other instance-independent regret bound $\order(R_2+D\Delta_{\max}\log(T/n))$, we bound the regret differently by considering $\Delta_a\geq \beta\sqrt{n/2^m}$ and $\Delta_a\leq \beta\sqrt{n/2^m}$ separately:
\begin{align}
    &\sum_{m=1}^{M}\sum_{a\in\calS_m}2^m\Delta_a \nonumber \\
    &\leq \sum_{m=1}^{M}\sum_{a\in\calS_m,\Delta_a\geq\beta\sqrt{n/2^m}}\left(\frac{512n\beta^2}{\Delta_a}+\frac{2D\Delta_a}{3n}+2\Delta_a\right) \nonumber\\
    &\qquad +\sum_{m=1}^{M}\sum_{a\in\calS_m,\Delta_a\leq\beta\sqrt{n/2^m}}\left(2^m\Delta_a\right) \nonumber \\
    &\leq \order(n\sqrt{|\calT_B|\log(KT)} + D\Delta_{\max}\log(T/n))\nonumber.
\end{align}

\paragraph{Step 2}
To obtain the other regret bound $\order(\min\{R_1,R_2\}+(nd^\star+DB)\log(T/n)+D\Delta_{\max})$ with a different delay overhead, we similarly define
\begin{align*}
    \rad_{m,a}^{N} &= \beta\sum_{i=1}^{|\calS_m|}|\lambda_{m,i}^{(a)}|\cdot \frac{\|a\|_2}{\sqrt{2^m}}
\end{align*}
as the total confidence radius of action $a$ coming from the definition of $\LCB_{m,1}(a)$. 
Further let $\wh{\mu}_m(a) = \frac{1}{2^m}\left(\sum_{\tau\in \obs_m(a)\cup\unobs_m(a)}u_{\tau}\right)$ be the empirical mean of action $a$'s loss within epoch $m$ (which is generally not available to the algorithm due to delay). According to the construction of $\wh{\mu}_m^{+}(a)$ and $\wh{\mu}_m^{-}(a)$, we know that for all $a\in\calS_m$,
\begin{align*}
    \wh{\mu}_m^{+}(a)\leq \wh{\mu}_m(a) + \frac{|\unobs_m(a)|}{2^m},~~\wh{\mu}_m^{-}(a)\geq \wh{\mu}_m(a) - \frac{|\unobs_m(a)|}{2^m}.
\end{align*}
Then, for any action $a\in\calA_m$ that is not eliminated at the end of epoch $m$, using the fact that $a=\sum_{i=1}^{|\calS_m|}\lambda_{m,i}^{(a)}a_{m,i}$, we obtain with high probability:
\begin{align}
    \mu_a &\leq \sum_{i=1}^{|\calS_m|}\lambda_{m,i}^{(a)}\cdot \hat{\mu}_{m}(a_{m,i}) + \rad_{m,a}^{N} \nonumber\\
    &\leq \LCB_{m,1}(a) + \rad_{m,a}^{N}+\sum_{i=1}^{|\calS_m|}|\lambda_{m,i}^{(a)}|\cdot \frac{|\unobs_m(a_{m,i})|}{2^m} \nonumber\\
    &\leq \LCB_{m,1}(a) +\rad_{m,a}^{N} \nonumber\\
    &\qquad + \sum_{i=1}^{|\calS_m|}|\lambda_{m,i}^{(a)}|\cdot\left(\frac{2D\mu_{a_{m,i}}}{2^m|\calS_m|}+\frac{16\log KT +2}{2^m}\right) \\
    &\leq B +\rad_{m,a}^{N} \nonumber\\
    &\qquad + \sum_{i=1}^{|\calS_m|}|\lambda_{m,i}^{(a)}|\cdot\left(\frac{2D\mu_{a_{m,i}}}{2^m|\calS_m|}+\frac{16\log KT +2}{2^m}\right),\label{eqn:small-loss}
\end{align}
where the first inequality is by standard Azuma-Hoeffding's inequality, the third inequality is by Lemma C.2 of \citet{schlisselberg2024delay} (included as \pref{lem:high-prob-event} in the appendix for completeness), and the last inequality is because $a$ is not eliminated at the end of epoch $m$.

\setcounter{AlgoLine}{0}
\begin{algorithm*}[htbp]
\caption{Reduction from Contextual Linear Bandits to Non-Contextual Linear Bandits~\citep{hanna2023contexts}}\label{alg:reduction}
Input: confidence level $\delta$, an instance $\Alg_{\nctx}$ of \pref{alg:lossLBmis} with $\beta=\sqrt{2\log(KT^3)}$. 

Let $\Theta'$ be a $\frac{1}{T}$-cover of $\Theta$ with size $\order(T^n)$.

\For{$m=1,2,\dots$}{
    \nl Construct action set $\calX_{m}=\{\gup{m}(\theta)~\vert~\theta\in \Theta'\}$ where   $\gup{m}(\theta)=\frac{1}{2^{m-1}}\sum_{\tau=1}^{2^{m-1}}\argmin_{a\in \calA_\tau}\inner{a,\theta}$.
    

    \nl Initiate $\Alg_{\nctx}$ with action set $\calX_m$ and misspecification level $\epsilon_m=\min\{1,2\sqrt{\log(T|\Theta'|/\delta)/2^m}\}$. \label{line:misspecific_level}
    
    \nl \For{$t=2^{m-1}+1,\dots,2^m$}{
        \nl $\Alg_{\nctx}$ outputs action $\gup{m}(\theta_t)$.

        \nl Observe $\calA_t$ and select $a_t=\argmin_{a\in \calA_t}\inner{a,\theta_t}$.

        \nl Observe the loss $u_\tau$ for all $\tau$ such that $\tau+d_{\tau}\in (t-1,t]$ and send them to $\Alg_{\nctx}$.
    }
    
}
\end{algorithm*}

Now consider two cases. When $B\geq \frac{\mu_a}{2}$, we know that $\Delta_a\leq \mu_a - \mu^\star\leq 2B$. Using the previous \pref{eqn:epoch_bound_1}, we know that
\begin{align}\label{eqn:small-loss-1}
    2^m\Delta_a\leq \order\left(\frac{n\beta^2}{\Delta_a}+\frac{DB}{n}\right).
\end{align}
Otherwise, when $B < \frac{\mu_a}{2}$, with some manipulation on \pref{eqn:small-loss}, we show that
\begin{align}\label{eqn:small-loss-2}
    2^m\Delta_a\leq \order\left(\frac{n\beta^2}{\Delta_a}+\frac{\sum_{i=1}^{|\calS_m|}D\mu_{a_{m,i}}}{n}\right).
\end{align}
Combining \pref{eqn:small-loss-1} and \pref{eqn:small-loss-2}, we then obtain that within epoch $m$, the regret is bounded by
\begin{align}\label{eqn:small-loss-3}
\order\left(\sum_{a\in\calS_m}\frac{n\beta^2}{\Delta_a}+DB+D\sum_{i=1}^{|\calS_{m-1}|}\mu_{a_{m-1,i}}\right),
\end{align} 
since all active actions in epoch $m$ are not eliminated in epoch $m-1$.
The first term $\sum_{a\in\calS_m}\frac{n\beta^2}{\Delta_a}$ in \pref{eqn:small-loss-3} eventually leads to the $\min\{R_1,R_2\}$ term in the claimed regret bound, by the exact same reasoning as in \textbf{Step 1}.
The second term explains the final $DB\log(T/n)$ term in the regret bound (recall that number of epoch is of order $\order(\log(T/n))$).
Finally, the last term in \pref{eqn:small-loss-3} can be written as
$D\sum_{i=1}^{|\calS_{m-1}|} \Delta_{a_{m-1,i}} + 3n\cdot d^\star$,
and the term $D\sum_{i=1}^{|\calS_{m-1}|} \Delta_{a_{m-1,i}}$ is one half of the regret incurred in epoch $m-1$ as long as $2^{m-1}>2D$ (otherwise, the epoch length is smaller than $D$, and we bound the regret trivially by $D\Delta_{\max}$).
Summing over all epochs and rearranging the terms thus leads to the a term $nd^\star\log(T/n)$ in the regret.
This proves the goal of the second step.

\paragraph{Combining everything} 
Finally, note that the number of different values of $B$ \pref{alg:lossLB} uses is upper bounded by $\lceil\log_2(d^\star)\rceil=\lceil\log_2(D\mu^\star)\rceil$ since the optimal action $a^\star$ will never be eliminated when $B\geq \mu^\star$. Summing up the regret over these different values of $B$ arrives at the the final bound $\order(\min\{V_1,V_2\},\log(d^\star)\min\{W_1,W_2\})$.











\section{Extension to Contextual Linear Bandits}\label{sec: contextual}


In this section, we extend our results to the stochastic contextual setting where the action set at each round is drawn i.i.d. from a distribution $\dist$. 
While the arm elimination procedure is critical in solving our problem in the non-contextual case with a fixed action set, it is not clear (if possible at all) to directly generalize it to the contextual setting due to the dynamic nature of the action set.


Fortunately, a recent work by \citet{hanna2023contexts} proposes a reduction from contextual linear bandits to non-contextual linear bandits (both without delay).
At a high level, this reduction utilizes a subroutine of a non-contextual linear bandits algorithm by constructing a fixed action for each possible parameter $\theta$ of the contextual bandit instance. 
Importantly, the subroutine needs to be able to deal with an $\epsilon$-misspecified model, where the loss of each $a\in \calA$ is almost linear: $\mu_a=\inner{a,\theta}+\epsilon_a\in [0,1]$, with $\epsilon \geq \max_{a\in\calA}|\epsilon_a|$ indicating the misspecification level. 
It turns out that, a simple modification of our \pref{alg:lossLB} can address such misspecification --- it only requires incorporating the misspecification level $\epsilon$ into the criteria of arm elimination;
see \pref{alg:lossLBmis} and specifically its \pref{line:eliminate-mis} for details.


We then plugin this subroutine, denoted as $\Alg_{\nctx}$, into their reduction, as shown in \pref{alg:reduction}.
Specifically, the algorithm first constructs a $\frac{1}{T}$-cover $\Theta'$ of the parameter space $\Theta=\R_+^n\cap\mathbb{B}_2^n(1)$ with size $|\Theta'| = \mathcal{O}(T^n)$. 
It then proceeds in epochs with doubling length. 
At the start of epoch $m$, 
a new \emph{fixed} action set $\mathcal{X}_m = \{g^{(m)}(\theta) : \theta \in \Theta'\}$ is constructed, where $g^{(m)}(\theta)$ is the averaged optimal action over the previous $m-1$ epochs, assuming the model parameter being $\theta$.
Then, a new instance of $\Alg_{\nctx}$ with action set $\mathcal{X}_m$ and some 
misspecification level $\epsilon_m$ is initiated and run for the entire epoch.
At each round $t$ of this epoch, $\Alg_{\nctx}$ outputs an action $g^{(m)}(\theta_t) \in \mathcal{X}_m$, and the algorithm's final decision upon receiving the true action set $\calA_t$ is $a_t=\argmin_{a\in \calA_t}\inner{a,\theta_t}$.
Finally, at the end of this round, all newly observed losses are sent to $\Alg_{\nctx}$.


\begin{figure*}[t]
\centering

\includegraphics[width=0.33\textwidth]{Figure/cost_dimension_6.pdf}
\includegraphics[width=0.33\textwidth]{Figure/cost_dimension_8.pdf}
\includegraphics[width=0.33\textwidth]{Figure/cost_dimension_10.pdf}

\includegraphics[width=0.33\textwidth]{Figure/reward_dimension_6.pdf}
\includegraphics[width=0.33\textwidth]{Figure/reward_dimension_8.pdf}
\includegraphics[width=0.33\textwidth]{Figure/reward_dimension_10.pdf}
\caption{Comparison of the empirical results of our algorithm and \texttt{LinUCB}. The top row is the delay-as-loss setting and the bottom row is the delay-as-reward setting. The left, middle, and right column correspond to $n=6,8,10$ respectively.}
\label{fig:synthetic_dataset}
\end{figure*}

\paragraph{Guarantees and Analysis}
Even though our algorithm is a direct application of the reduction of~\citet{hanna2023contexts}, it is a priori unclear whether it enjoys any favorable regret guarantee in the delay-as-loss setting.
By adopting and generalizing their analysis, we show that this is indeed the case.
Before introducing our results, we define the following quantities:
    \begin{align*}
        g(\theta) &\triangleq \E_{\calA\sim \dist}\left[\argmin_{a\in\calA}\inner{a,\theta}\right],\\
        \Delta_{\min}^{\nctx} &\triangleq\min_{\theta'\in \Theta', \inner{g(\theta'),\theta}\neq \inner{ g(\theta),\theta}}\E\left[\inner{g(\theta)-g(\theta'),\theta}\right],\\
        \Delta_{\max}^{\nctx}  &\triangleq\max_{\theta'\in \Theta'}\E\left[\inner{g(\theta)-g(\theta'),\theta}\right],\\
        \overline{d}^{\star} &\triangleq D\cdot \inner{g(\theta),\theta} = D\cdot \E_{\calA\sim \dist}\left[\min_{a\in \calA}\inner{a,\theta}\right],
    \end{align*}
    where $g(\theta)$ denotes the optimal action in expectation, $\Delta_{\min}^{\nctx}$ ($\Delta_{\max}^{\nctx}$) denotes the minimum (maximum) suboptimality gap for the reduced non-contextual linear bandit instance, and $\overline{d}^\star$ denotes the expected delay of the optimal action.

\begin{theorem}\label{thm:reduction}
    \pref{alg:reduction} with %$t^{(m)}=2^{m-1}$ and 
    $\delta = 1/T^2$ guarantees
    \begin{align*}
        &\Reg =\order\big(n\sqrt{T\log T}+\min\{V_1,V_2\} \\
        &\quad\quad\quad\quad +\log(\overline{d}^\star)\min\{W_1,W_2\}\big),
    \end{align*}
     where $V_1=\frac{n^3\log^2(T)\log(T/n)\log(\overline{d}^\star)}{\Delta_{\min}^{\nctx}}$, $V_2=n^{1.5}\sqrt{T\log(\overline{d}^\star)\log(T)}$, $W_1=\log T(n\overline{d}^\star\log(T/n)+D\Delta_{\max}^{\nctx})$, and $W_2=D\Delta_{\max}^{\nctx}\log T\log(T/n)$.
\end{theorem}
The proof is deferred to \pref{app: contextual}. 
The regret bound is in the same spirit as the one for the non-contextual case (\pref{thm:main-non-contextual}) and consists of a term for standard regret and a term for delay overhead.
The standard part unfortunately suffers higher dependence on the dimension $n$, while the delay overhead is in a similar problem-dependent form.
We remark that this is the first regret guarantee for contextual linear bandits with delay-as-payoff, resolving an open problem asked by \citep{schlisselberg2024delay}.
\section{Experiments: Planning outperforms Heuristics}
\label{sec:experiment}

We begin our empirical demonstrations by showcasing the effectiveness of our planning framework on both synthetic and real datasets. We focus on the simplest planning algorithm, 1-step lookaheads (Algorithm~\ref{alg:complete}), and show that even basic planning can hold great promise. 
We illustrate our framework using two uncertainty quantification modules---GPs and 
\ensembles/ \ensembleplus. 

Throughout this section, we focus on evaluating the mean squared error of 
a regression model $\model$,  and develop adaptive policies that minimize uncertainty on $g(f)$ defined in~\eqref{eqn:l2-g-f}.
When GPs provide a valid model of uncertainty, 
our experiments show that our planning framework significantly outperforms other baselines. 
We further demonstrate that our conceptual framework extends to deep learning-based uncertainty quantification methods such as  \ensembleplus while highlighting computational challenges that need to be resolved in order to scale our ideas. 
For simplicity, we assume a naive predictor, i.e., $\psi(\cdot) \equiv 0$. However, we emphasize that this problem is just as complex as if we were using a sophisticated model $\psi(.)$. The performance gap between the algorithms 
primarily depends
on the level  of uncertainty in our prior beliefs.

To evaluate the performance of our algorithm, we benchmark it against several baselines. 
%Active learning baselines use an acquisition function $\ac$ to select points that have the highest   function value: $X\opt_t \in \argmax_{X \in \xpoolj{t}} \ac({X})$ at every step $t$. These methods may also need an UQ module, which we simply use the same UQ module as in our algorithm, and it  outputs $V(X)$ that measures the the uncertainty of each point $X \in \xpoolj{t}$.
Our first set of baselines are from active learning~\citep{AggarwalKoGuHaPh14}:
\\ % \noindent\textbf{Active Learning Heuristics:} 
\textbf{(1)} 
\textsf{Uncertainty Sampling (Static):}  In this approach, we query the samples for which the model is least certain about. Specifically, we estimate the variance of the latent output $f(X)$ for each $X \in \xpool$ using the UQ module and select the top-$K$ points with the highest uncertainty. \\
\textbf{(2)} \textsf{Uncertainty Sampling (Sequential):} This is a greedy heuristic that sequentially selects the points with the highest uncertainty within a batch, while updating the posterior beliefs using pseudo labels from the current posterior state. Unlike \textsf{Uncertainty Sampling (Static)}, this method takes into account the information gained from each point within batch, and hence tries to diversify the selected points within a batch. 

 
We also compare our approach to the  \textbf{(3)} \textsf{Random Sampling}, which selects each batch uniformly at random from the pool. Additionally, we compare solving the planning problem using  \textsf{REINFORCE}-based policy gradients with   $\mathsf{Smoothed\text{-}Autodiff}$ policy gradients.\footnote{Our code repository is available at
  \url{https://github.com/namkoong-lab/adaptive-labeling}.}
%Detailed experimental setups are provided in Section \ref{sec:details-experiments}.

%We repeat all experiments with 10 random seeds.




\begin{figure}[t]
\centering
\begin{minipage}[b]{0.49\textwidth}
\centering
\includegraphics[width=\textwidth, height=5cm]{figures/original_scale/Var_of_l_2_loss.pdf}
\caption{(Synthetic data) Variance of mean squared loss evaluated through the posterior belief $\mu_t$ at each horizon $t$. This is the objective that policy gradient methods like \textsf{REINFORCE} and $\ouralgo$ optimizes. 1-step lookaheads are surprisingly effective even in long horizons.}
\label{fig:var-l2-sim}
\end{minipage}
\hfill
\begin{minipage}[b]{0.49\textwidth}
\centering \includegraphics[width=\textwidth, height=5cm]{figures/original_scale/Error_of_estimated_model_l_2_loss.pdf}
\caption{(Synthetic data) Error between MSE calculated based on collected data $\mc{D}^{0:T}$ vs. population oracle MSE over $\mc{D}_{\rm eval} \sim P_X$. Reducing uncertainty over posteriors directly leads to better OOD evaluations. 1-step lookaheads significantly outperform active learning heuristics in small horizons.}
\label{fig:mean-l2-sim}
\end{minipage}
%\caption{Simulated data for GPs}
%\label{fig:both_plots}
\end{figure}

\subsection{Planning with Gaussian processes}
\label{sec:experiment-plan-GP}
We now briefly describe the data generation process for the GP experiments,  deferring a more detailed discussion of the dataset generation to Section~\ref{sec:details-experiments}. 
We use both the synthetic data and the real data to test our methodology.
For the \emph{simulated data},  we construct a setting where the general population is distributed across \emph{51 non-overlapping clusters} while the initial labeled data $\dtrain$ just comes from one cluster. In contrast, both $\dpool \defeq (\xpool,\ypool),\deval \defeq (\xeval,\yeval)$ are generated   from all the clusters. 
We begin with a low-dimensional scenario, generating a one-dimensional regression setting using a GP. %Gaussian Process (GP).
Although the data-generating process is not known to the algorithms,  we assume that the GP hyperparameters are known to all the algorithms
to ensure fair comparisons. This can be viewed as a setting where our prior is well-specified, allowing us to isolate the effects
of different policy optimization approaches
 without any concerns about the misspecified priors. We select $10$ batches, each of size $K=5$ across $T = 10$ time horizons.

To examine the robustness of our method against the distributional assumptions made  in the simulated case, we then move to a real dataset where the correct prior is not known. We simulate selection bias from the eICU dataset~\citep{PollardJoRaCeMaBa18}, which contains real-world patient data with in-hospital mortality outcomes. 
We conduct a $k$-means clustering to generate 51 clusters and then select data from those clusters. We view this to be a credible replication of practice, as severe distribution shifts are common due to selection bias in clinical labels.  To convert the binary mortality labels into a regression setting, we train a  random forest classifier and fit a GP on predicted scores, which serves as the UQ module for all the algorithms. As before, the task is to select 10 batches, each consisting of 5 samples, across 10 time horizons.

 In Figures~\ref{fig:var-l2-sim} and~\ref{fig:mean-l2-sim}, we present results for the simulated data. 
Figure~\ref{fig:var-l2-sim} shows the variance of $\ell_2$ loss, and Figure~\ref{fig:mean-l2-sim} presents the error in the estimated $\ell_2$ loss using $\mu_t$ (relative to true $\ell_2$ loss, that is unknown to the algorithm). 
As we can see from these plots, our method one-step lookahead  gives substantial improvements  over active learning baselines and random sampling. In addition,
compared to the one-step lookahead planning approach using \textsf{REINFORCE}-based policy gradients, 
we observe that $\mathsf{Smoothed\text{-}Autodiff}$-based policy gradients provide significantly more robust performance over all horizons.

In Figures~\ref{fig:var-l2-real}~and~\ref{fig:mean-l2-real}, we observe similar findings on the eICU data. We see that planning policies (\textsf{REINFORCE} and $\mathsf{Smoothed\text{-}Autodiff}$) consistently outperform other heuristics by a large margin.  Active learning baselines perform poorly in these small-horizon batched problems and can sometimes be even worse than the random search baselines.  Overall, our results show the importance of careful planning in adaptive labeling for reliable model evaluation. 

We offer some intuition as to why one-step lookahead planning may outperform other heuristic algorithms. 
 First,  \textsf{Uncertainty sampling (Static)} while myopically selects the
 top-$K$ inputs with the highest uncertainty, it fails to consider 
the overlap in information content among the ``best” instances; see \citep{AggarwalKoGuHaPh14} for more details. 
In other words,  it might acquire points from the same region with high uncertainty while failing to induce diversity among the batch.
Although \textsf{Uncertainty Sampling (Sequential)} somewhat addresses the issue of information overlap, a significant drawback of 
this algorithm
is the disconnect between the objective we aim to optimize and the algorithm. For example, it might sample from a region with high uncertainty but very low density. 

\begin{figure}[t]
\centering
\begin{minipage}[b]{0.48\textwidth}
\centering
\includegraphics[width=\textwidth, height=5cm]{figures/original_scale/Var_of_l_2_loss_real.pdf}
\caption{(Real-world eICU data) Variance of mean squared loss evaluated through the posterior belief $\mu_t$ at each horizon $t$. Even 1-step lookaheads are extremely effective planners, and auto-differentiation-based pathwise policy gradients provide a reliable optimization algorithm based on low-variance gradient estimates.}
\label{fig:var-l2-real}
\end{minipage}
\hfill
\begin{minipage}[b]{0.48\textwidth}
\centering \includegraphics[width=\textwidth, height=5cm]{figures/original_scale/Error_of_estimated_model_l_2_loss_real.pdf}
\caption{(Real-world eICU data) Error between MSE calculated based on collected data $\mc{D}^{0:T}$ vs. population oracle MSE over $\mc{D}_{\rm eval} \sim P_X$. Reducing uncertainty over posteriors directly leads to better OOD evaluations. Our method significantly outperforms active learning-based heuristics, and random sampling.}
\label{fig:mean-l2-real}
\end{minipage}
%\caption{Real data for GPs}
\end{figure}
 
%\vspace{-1.5cm}
% \begin{wrapfigure}{r}{.32\columnwidth}
%   \vspace{-.5cm} 
%   \centering
% \includegraphics[scale=.29]{figures/Var of l2l_2 loss.pdf}
%   \vspace{-0.2cm}
%   \caption{Results of GP}
% \label{fig:var-l2-gp}
%   \vspace{-0.1cm}
% \end{wrapfigure}


% Attempts have been made  in the past to address these  drawbacks heuristically  (see \citep{AggarwalKoGuHaPh14}). We give a unified computational framework while approaching the problem in a more principled manner and solving it more optimally.




\subsection{Planning with  neural network-based uncertainty quantification methods ($\ensembleplus$)}


We now provide a proof-of-concept that shows the generalizability of our conceptual framework  to the deep learning-based UQ modules, specifically focusing on $\ensembleplus$ due to their previously observed superior performance~\citep{OsbandWenAsDwIbLuRo23}. Recall that implementing our framework with deep learning-based UQ modules  requires us to retrain the model across multiple possible random actions $\bm{a}(\theta)$ sampled from the current policy $\pi_\theta$.
This requires significant computational resources, in sharp contrast to the GPs where the posteriors are in closed form and can be readily updated and differentiated. 

Due to the computational constraints, we test $\ensembleplus$ on a toy setting to demonstrate the generalizability of our framework. We consider a setting where the general population consists of four clusters, while the initial labeled data only comes from one cluster. Again we generate data using GPs.  The task is to select a batch of 2 points in one horizon. We detail the $\ensembleplus$ architecture in Section \ref{sec:details-experiments}, and we assume prior uncertainty to be large (depends on the scaling of the prior generating functions). 
The results are summarized in the Table~\ref{tab:UQ_ensemble}.

% \begin{table}[H]
% \vspace{-10pt}
% \caption{Performance under \ensembleplus as UQ module}
%     \centering
%     \begin{tabular}{|m{3cm}|m{2.5cm}|m{2cm}|} 
%     \hline
%       Algorithm   & Variance of $\loss_2$ loss estimate & Error of $\loss_2$ loss estimate  \\ \hline Random Sampling 
%          & $1710.9 \pm 1352.1$ & $8.67\pm6.62$ 
%       \\ \hline \ouralgo & $1.30 \pm 0.68$ & $0.91\pm0.25$ \\ \hline
%     \end{tabular}
%     \label{tab:UQ_ensemble}
%     %\vspace{-10pt}
% \end{table}




\begin{table}[h]
\vspace{-10pt}
\caption{Performance under \ensembleplus as the UQ module}
\centering
\begin{tabular}{|l|l|l|}
\hline
Algorithm   & Variance of $\loss_2$ loss estimate & Error of $\loss_2$ loss estimate  \\
\hline
\textsf{Random sampling} & 7129.8 $\pm$ 1027.0 & 136.2 $\pm$ 8.28 \\ \hline
\textsf{Uncertainty sampling (Static)} & 10852 $\pm$ 0.0 & 162.156 $\pm$ 0.0 \\ \hline
\textsf{Uncertainty sampling (Sequential)} & 8585.5 $\pm$ 898.9 & 144 $\pm$ 6.93 \\ \hline
\textsf{REINFORCE} & 1697.1 $\pm$ 0.0 & 45.27 $\pm$ 0.0 \\ \hline
\ouralgo & 1697.1 $\pm$ 0.0 & 45.27 $\pm$ 0.0 \\ \hline
\end{tabular}
%\caption{Comparison of different algorithms based on variance   and   error in $\ell_2$ loss estimation with Ensemble $+$ as the UQ module. Our results demonstrate that {\ouralgo} and REINFORCE outperformthe other active learning based heuristics, confirming the benefits of our MDP formulation for the adaptive labeling problem, as also demonstrated in Section 4.\\
%\footnotesize{Experimental details: We use Gaussian Processes as our data generating process, GP parameters are the same as in Section D.3.  The task is to select a batch of 2 points along one horizon.The marginal distribution $p_X$ has 4 \textit{non-overlapping} clusters. Initial data comes from one cluster, while pool and evaluation points comes from all the clusters. We have $20$ initial labeled data points, $10$ pool points, and $252$ evaluation points.  Training procedures are similar to the one in Section D.3.} }
\label{tab:UQ_ensemble}
\end{table}



% We faced  issues in scaling up these experiments which will be our focus in the future. 





% \begin{itemize}
%     \item Posteriors should be consistent. Two dimensions: even with less training,  
%     \item the inference should be  fast enough
% \end{itemize}


% Potential research directions for uncertainty quantification

% In this section we consider a simple setting We consider a simpler setting and 


% For synthetic dataset generation, we use ...... For real datasets, we use ...... We compare our methodolgy to several baselines ()    This Section is structured as follows:
% \begin{itemize}
%     \item \textbf{GPs, square loss objective} (Section \ref{}): 
%     %the broad aim of the experiments  in this section is to isolate the performance of our methodology without any concerns for the inefficiencies induced due to a mis-specified prior or imperfect posterior inference. To accomplish this we generate synthetic datasets using GPs (detailed later). We use the well specified prior (GPs - with same hyperparameter setting) as our UQ module.   
%      As GPs provide differentaible posterior inference - any errors induced due to imperfect posterior updates are also isolated. We note that under this setting
%      \item In Section\ref{} we demonstrate why our methodology performs better than other baselines - by devising various synthetic experiments ()
%     \item  \textbf{UQ Benchmarking }(Section \ref{}): Before diving into the experiments using $\ensembleplus$ and ENNs,  we showcase our benchmarking experiments in Section \ref{}. We use real datasets We observe that ENNs perform better
%      \item \textbf{Ensemble $+$}, objective: recall, accuracy
%     \item \textbf{ENN}, objective: recall, accuracy
% \end{itemize}




% In Section {}, we test 
% \subsection{Experimental details}

% \begin{itemize}
%     \item UQ methodologies - GPs, ENNs
%     \item Objectives - Recall,  ATE
%     \item Datasets - ATE-synthetic datasets, Recall-synthetic, real datasets
%     \item Baselines - 
%     \begin{itemize}
%         \item Random sampling
%         \item Active learning - Uncertainty based sampling - In regression setting almost all of the 
%         \item Myopic greedy - Greedy Batch based sampling
%         \item Policy Gradient
%     \end{itemize}
    
% \end{itemize}

% \subsection{Experiments}
%     \begin{itemize}
%     \item GPs with square loss
%     \item Benchmarking ENN
%         \item ENNs with ATE
%         \item ENNs with Recall
%     \end{itemize}

% \subsection{Benefits over other algorithms - intuition and experiments}

%Active learning - Myopic greedy / Don't rely on the objective rather some entropy version.


%%% Local Variables:
%%% mode: latex
%%% TeX-master: "main"
%%% End:


\section{Conclusion}
In this work, we initiate the study of the delay-as-payoff model for contextual linear bandits and develop provable algorithms that require novel ideas compared to standard linear bandits.
Interesting future directions include proving matching regret lower bounds and extending our results to general payoff-dependent delays~\citep{lancewicki2021stochastic} and other even more challenging settings, such as those with intermediate observations~\citep{esposito2023delayed} or
evolving observations~\citep{bar2024non}.


\bibliography{ref}
\bibliographystyle{icml2025}


\newpage
\appendix
\onecolumn
\section{Omitted Details in \pref{sec: linear}}\label{app:loss}

\setcounter{AlgoLine}{0}
\begin{algorithm}
\caption{Phased Elimination via Volumetric Spanner for Linear Bandits with Delay-as-Loss with misspecification}\label{alg:lossLBmis}

\nl Input: maximum possible delay $D$, action set $\calA$, $\beta>0$, a misspecification level $\epsilon$. 

\nl Initialization: optimal loss guess $B=1/D$.

\nl Initialization: active action set $\calA_1=\calA$. \label{line: restart-mis}

 \For{$m=1,2,\dots,$}{
    \nl Find $\calS_m=\{a_{m,1},\dots,a_{m,|\calS_m|}\}$, a volumetric spanner of $\calA_m$ with $|\calS_m|= 3n$. \label{line:volume-mis}
    
    \nl Pick each $a\in \calS_m$ $2^m$ times in a round-robin way. \label{line:round-robin-mis}

    \nl Let $\calI_m$ contain all the rounds in this epoch.
    
    \nl For each $a\in \calS_m$, calculate the following quantities: \label{line:spanner-ucb-lcb-mis}
    {\small
    \begin{align}
        &\hat{\mu}_{m}^+(a)=\frac{1}{2^m}\Big(\sum_{\tau\in \obs_m(a)}u_{\tau} + \sum_{\tau\in \unobs_m(a)}1\Big), \label{eqn:mean-up-mis}\\
        &\hat{\mu}_{m}^-(a)=\frac{1}{2^m}\sum_{\tau\in \obs_m(a)}u_{\tau}, \label{eqn:mean-low-mis}\\
        &\hat{\mu}_{m,1}^{+}(a)=\hat{\mu}^{+}_{m}(a)+\frac{\beta}{2^{m/2}}\|a\|_2, \label{eqn:loss-ucb-linear-1-mis}\\
        &\hat{\mu}_{m,1}^{-}(a)=\hat{\mu}^{-}_{m}(a)-\frac{\beta}{2^{m/2}}\|a\|_2,\label{eqn:loss-lcb-linear-1-mis}\\
        &\hat{\mu}_m^{F}(a)=\frac{1}{\unbiasSize_m(a)}\sum_{\tau\in \unbias_m(a)}u_{\tau}, \label{eqn:mean_unbiased-mis}\\
        &\hat{\mu}_{m,2}^{+}(a)=\hat{\mu}_m^F(a)+\frac{\beta}{\sqrt{\unbiasSize_m(a)}}\|a\|_2, \label{eqn:loss-ucb-linear-2-mis}\\
        &\hat{\mu}_{m,2}^{-}(a)=\hat{\mu}_m^F(a)-\frac{\beta}{\sqrt{\unbiasSize_m(a)}}\|a\|_2, \label{eqn:loss-lcb-linear-2-mis}
    \end{align}
    }
    where $\unbiasSize_m(a) = |\unbias_m(a)|$, $\unbias_m(a) = \{\tau\in \calI_m: \tau+D\in\calI_m, a_{\tau}=a\}$, $\obs_m(a) = \{\tau\in \calI_m: \tau+d_{\tau}\in\calI_m, a_{\tau}=a\}$, and
    $\unobs_m(a)= \{\tau\in \calI_m: a_{\tau}=a\}\setminus\obs_m(a)$.

    \For{each $a\in \calA_m$}{
        \nl \label{line: decompose-mis}
        Decompose $a$ as $a=\sum_{i=1}^{|S_m|}\lambda_{m,i}^{(a)}a_{m,i}$ with $\|\lambda_{m}^{(a)}\|_2\leq 1$ and calculate 
        {\small
        \begin{align}
            &\UCB_{m}(a)=\sum_{i=1}^{|\calS_m|}\lambda_{m,i}^{(a)}\cdot\hat{\mu}_{m,2}^{\sgn(\lambda_{m,i}^{(a)})}(a_{m,i}), \label{eqn:loss-ucb-f-all-action-mis} \\
            &\LCB_m(a) = \max_{j\in \{1,2\}}\{\LCB_{m,j}(a)\} \;\;\text{where} \nonumber  \\
            & \LCB_{m,j}(a)=\sum_{i=1}^{|\calS_m|}\lambda_{m,i}^{(a)}\cdot\hat{\mu}_{m,j}^{\sgn(-\lambda_{m,i}^{(a)})}(a_{m,i}),\label{eqn:loss-lcb-all-action-mis}
        \end{align}
        }
    }
    
    \nl Set $\calA_{m+1} = \calA_m$.
    
    \For{$a\in \calA_m$}{
        \nl \label{line:eliminate-mis}  
        \If{$\exists a'\in \calA_m$, s.t. $\LCB_m(a) \geq \min\{\UCB_m(a'),B\} + 4\sqrt{3n}\epsilon$}
        {
          Eliminate $a$ from $\calA_{m+1}$.
        }
    }
    \nl \If{$\calA_{m+1}=\emptyset$}{
        Set $B\leftarrow 2B$ and go to \pref{line: restart-mis}.
    }
}
\end{algorithm}

In this section, we provide the detailed proof for \pref{thm:main-non-contextual}. Specifically, as mentioned in \pref{sec: contextual}, we prove the guarantee of a modified algorithm (\pref{alg:lossLBmis}) for the more general $\epsilon$-misspecified linear bandits. 

Recall that in misspecified linear bandits, $\mu_a = \inner{a,\theta}+\epsilon_a\in[0,1]$ with $|\epsilon_a|\leq\epsilon$ for all $a\in\calA$. Due to this difference, we clarify on the definitions of $\Delta_a$, $a^\star$, $\mu^\star$, $\Delta_{\min}$, $\Delta_{\max}$, and $d^\star$ in misspecified linear bandits as follows. We still define $\Delta_a = \inner{a^\star-a,\theta}$ as the suboptimality gap of action $a$, where $a^\star \in \argmin_{a\in\calA}\inner{a, \theta}$, but $\mu^\star \triangleq \min_{a\in\calA}\mu_a$ as the loss of the optimal action. Note that due to the misspecification, $\mu^\star$ may not necessarily be $\mu_{a^\star}$. Define $\Delta_{\min} = \min_{a\in \calA, \Delta_a>0}\Delta_a$ and $\Delta_{\max} = \max_{a\in \calA}\Delta_a$ to be the minimum non-zero, and maximum sub-optimality gap. The delay at round $t$ is still defined as $d_t=D\cdot u_t$ and $d^\star = D\cdot \mu^\star$ is the expected delay of the optimal action.

As for the algorithm, \pref{alg:lossLBmis} differs from \pref{alg:lossLB} only in \pref{line:eliminate-mis} where we add one misspecification term $4\sqrt{3n}\epsilon$ in the criteria of eliminating an action. 

The following theorem shows the guarantee of our algorithm in the misspecified linear bandits.

\begin{theorem}\label{thm:lossLBmis}
    \pref{alg:lossLBmis} with $\beta = \sqrt{2\log(KT^3)}$ guarantees that
    \begin{align*}
        \Reg &\leq \order\left(\min\left\{\frac{n^2\log(KT)\log(T/n)\log(d^\star)}{\Delta_{\min}},n\sqrt{T\log(d^\star)\log(KT)}\right\}+\epsilon\sqrt{n}T\right) \\
        &\qquad + \log(d^\star)\cdot \order\left( \min\left\{nd^\star\log (T/n)+D\Delta_{\max},D\Delta_{\max}\log (T/n)\right\}\right).
    \end{align*}
\end{theorem}

To prove \pref{thm:lossLBmis}, recall the following quantities
\begin{align}
    \wh{\mu}_{m}(a) &= \frac{1}{2^m}\sum_{\tau\in\obs_m(a)\cup\unobs_m(a)}u_{\tau},~~~\forall a\in \calS_m,\label{eqn:loss-all-mean-app}\\
    \hat{\mu}_{m,1}(a)&=\sum_{i=1}^{|\calS_m|}\lambda_{m,i}^{(a)}\cdot\hat{\mu}_{m}(a_{m,i}),~~~\forall a\in \calA_m,\\
    \hat{\mu}_{m,2}(a)&=\sum_{i=1}^{|\calS_m|}\lambda_{m,i}^{(a)}\cdot\hat{\mu}_{m}^{F}(a_{m,i}),~~~\forall a\in \calA_m.
\end{align}
We then define the following event and show that the event holds with high probability.

\begin{event}\label{event:misLoss}
    For all action $a\in \calA_m$, $m\in[T]$,
    \begin{align}
        \left|\inner{a,\theta}-\hat{\mu}_{m,1}(a)\right|&\leq \sqrt{|\calS_m|}\epsilon + \beta\sum_{i=1}^{|\calS_m|}\left|\lambda_{m,i}^{(a)}\right|\sqrt{\frac{1}{2^{m}}}, \label{eqn:concentr-1}\\
        \left|\inner{a,\theta}-\hat{\mu}_{m,2}(a)\right|&\leq \sqrt{|\calS_m|}\epsilon +\beta \sum_{i=1}^{|\calS_m|}\left|\lambda_{m,i}^{(a)}\right|\sqrt{\frac{1}{\unbiasSize_m(a_{m,i})}}, \label{eqn:concentr-2}\\
        |\unobs_m(a)| &\leq \frac{2D\mu_a}{|\calS_m|}+16\log KT+2,\label{eqn:concentr-3}
    \end{align}
    where $\beta = \sqrt{2\log KT^3}$.
\end{event}
\begin{lemma}\label{lem:high-prob-event}
    \pref{alg:lossLBmis} guarantees that \pref{event:misLoss} holds with probability at least $1-\frac{2}{T^2}$.
\end{lemma}
\begin{proof}
    Fix an action $a\in \calS_m$ in epoch $m\in[T]$. According to standard Azuma's inequality, we know that with probability at least $1-\delta$,
    \begin{align*}
        \left|\mu_a-\hat{\mu}_{m,1}(a)\right|&\leq \sqrt{\frac{2\log(2/\delta)}{2^m}}\|a\|_2,\\
        \left|\mu_a-\hat{\mu}_{m,2}(a)\right|&\leq \sqrt{\frac{2\log(2/\delta)}{\unbiasSize_m(a)}}\|a\|_2.
    \end{align*}
    Taking union bound over all possible $a\in \calA$ and all $m\in[T]$, we know that with probability at least $1-\delta$, for all $a\in \calS_m$ and all $m\in [T]$,
    \begin{align*}
        \left|\mu_a-\hat{\mu}_{m,1}(a)\right|&\leq \sqrt{\frac{2\log(2TK/\delta)}{n_t(a)}}\|a\|_2,\\
        \left|\mu_a-\hat{\mu}_{m,2}(a)\right|&\leq \sqrt{\frac{2\log(2TK/\delta)}{\unbiasSize_m(a)}}\|a\|_2.
    \end{align*}
    Then, given that the above equation holds, for $a\in \calA_m$, due to the property of volumetric spanners, we have $\mu_a = \inner{a,\theta^\star}+\epsilon_a =  \sum_{i=1}^{|\calS_m|}\lambda_{m,i}^{(a)}\inner{a_{m,i},\theta^\star}+ \epsilon_a$. Therefore, we can obtain that
    \begin{align*}
        \left|\inner{a,\theta}-\hat{\mu}_{m,1}(a)\right|
        &\leq \left|\sum_{i=1}^{|\calS_m|}\lambda_{m,i}^{(a)}(\inner{a_{m,i},\theta^\star}-\mu_{a_{m,i}})\right| + \sum_{i=1}^{|\calS_m|}\left|\lambda_{m,i}^{(a)}\right|\cdot\left|\mu_{a_{m,i}}-\hat{\mu}_{m}(a_{m,i})\right| \\
        &\leq \sum_{i=1}^{|\calS_m|}\left|\lambda_{m,i}^{(a)}\right|\left(\epsilon_{a_{m,i}}+\sqrt{\frac{2\log(2TK/\delta)}{2^{m}}}\right) \\
        &\leq \sqrt{|\calS_m|}\epsilon + \sum_{i=1}^{|\calS_m|}\left|\lambda_{m,i}^{(a)}\right|\sqrt{\frac{2\log(2TK/\delta)}{2^{m}}},
    \end{align*}
    where the last inequality uses $\|\lambda_{m}^{(a)}\|_1\leq \sqrt{|\calS_m|}\cdot \|\lambda_{m}^{(a)}\|_2\leq \sqrt{|\calS_m|}$. A similar analysis proves \pref{eqn:concentr-2}. \pref{eqn:concentr-3} holds with probability at least $1-\frac{1}{T^2}$ according to Lemma 4.1 of \citep{schlisselberg2024delay}. Picking $\delta = \frac{1}{T^2}$ finishes the proof.
\end{proof}

The next lemma shows that if $B\geq \mu^\star$, then \pref{alg:lossLBmis} will not reach an empty active set.
\begin{lemma}\label{lem:end-of-B}
    Suppose that \pref{event:misLoss} holds. If $B\geq \mu^\star$, then $a^\star\in \calA_m$ for all $m$.
\end{lemma}
\begin{proof}
    Since \pref{event:misLoss} holds, we have, we know that for all $a\in\calA_m$, $\LCB_m(a)\leq \inner{a,\theta} + \sqrt{|\calS_m|}\epsilon $ and $\UCB_m(a)\geq \inner{a,\theta} - \sqrt{|\calS_m|}\epsilon$. If $B\geq \mu^\star$, then we have $a^\star$ never eliminated since for any $a\in \calA_m$
    \begin{align*}
        \LCB_{m}(a^\star) &\leq \inner{a^\star,\theta} + \epsilon\sqrt{|\calS_m|} \leq \mu^\star + \epsilon + \epsilon\sqrt{|\calS_m|} \leq \mu^\star + 2\epsilon\sqrt{|\calS_m|},\\
        \LCB_{m}(a^\star) &\leq \inner{a^\star,\theta} + \epsilon\sqrt{|\calS_m|} \leq \inner{a,\theta} + 2\epsilon\sqrt{|\calS_m|} \leq \UCB_m(a) + 4\epsilon\sqrt{|\calS_m|}.
    \end{align*}
    Therefore, $a^\star$ never satisfy the elimination condition.
\end{proof}

The following lemma shows that the regret within epoch $m$ can be well-controlled.

\begin{lemma}\label{lem:delta_1_loss_miss}
    Suppose that \pref{event:misLoss} holds. \pref{alg:lossLBmis} guarantees that if $a\in\calA$ is not eliminated at the end of epoch $m$ (meaning that $a\in \calA_{m+1}$), then 
    \begin{align*}
        2^m\cdot \Delta_a\leq 2^m\cdot 24\sqrt{n}\epsilon+\frac{256n\beta^2}{\Delta_a} + \frac{2D\Delta_a}{|\calS_m|}.
    \end{align*}
\end{lemma}
\begin{proof}
    For notational convenience, define $\rad_{m,a}^{N} = \frac{\beta}{\sqrt{2^m}}\|a\|_2$ and $\rad_{m,a}^{F} = \frac{\beta}{\sqrt{\unbiasSize_m(a)}}\|a\|_2$ for all $a\in \calS_m$. In addition, we also define $\rad_{m,a}^{N}$ and $\rad_{m,a}^{F}$ for $a\notin \calS_m$ as follows:
    \begin{align*}
        \rad_{m,a}^{N} &= \sum_{i=1}^{|\calS_m|}|\lambda_{m,i}^{(a)}|\cdot \rad_{m,a_{m,i}}^{N}, \\
        \rad_{m,a}^{F} &= \sum_{i=1}^{|\calS_m|}|\lambda_{m,i}^{(a)}|\cdot \rad_{m,a_{m,i}}^{F}.
    \end{align*}
    Since \pref{event:misLoss} holds, we know that for all $a\in\calA_m$, $\LCB_m(a)\leq \inner{a,\theta} + \sqrt{|\calS_m|}\epsilon$, $\UCB_m(a)\geq \inner{a,\theta} - \sqrt{|\calS_m|}\epsilon$. Moreover, as $\LCB_m(a)=\max\{\LCB_{m,1}(a),\LCB_{m,2}(a)\}$, we know that for all $a\in \calA_m$
    \begin{align*}
        \LCB_{m,1}(a) + 2\rad_{m,a}^{N} + 2\epsilon\sqrt{|\calS_m|}\geq  \hat{\mu}_{m,1}(a) + \rad_{m,a}^{N} + 2\epsilon\sqrt{|\calS_m|}\geq \inner{a,\theta},\\
        \LCB_{m,2}(a) + 2\rad_{m,a}^{F} + 2\epsilon\sqrt{|\calS_m|}\geq  \hat{\mu}_{m,2}(a) + \rad_{m,a}^{F} + 2\epsilon\sqrt{|\calS_m|}\geq \inner{a,\theta},\\
        \UCB_{m}(a) - 2\rad_{m,a}^{F} - 2\epsilon\sqrt{|\calS_m|} = \hat{\mu}_{m,2}(a) - \rad_{m,a}^{F}-2\epsilon\sqrt{|\calS_m|}\leq \inner{a,\theta}.        
    \end{align*}
    If $B\geq \mu^\star$, then $a^\star\in \calA_m$ according to \pref{lem:end-of-B}.
    Moreover, if $a$ is not eliminated in epoch $m$, we have $\LCB(a)\leq \min\{\UCB_m(a^\star),B\}+4\sqrt{|S_m|}\epsilon$, meaning that
    \begin{align*}
        &\inner{a,\theta} - 2\rad_{m,a}^{F} - 2\epsilon\sqrt{|\calS_m|} \\
        &\leq \wh{\mu}_{m,2}(a) - \rad_{m,a}^{F} \\
        &\leq \LCB_m(a) \\
        &\leq \min\{\UCB_m(a^\star),B\}+4\sqrt{|S_m|}\epsilon \\
        &\leq \UCB_m(a^\star) + 4\sqrt{|S_m|}\epsilon \\
        &= \wh{\mu}_{m,2}(a^\star) + \rad_{m,a^\star}^{F}+ 4\sqrt{|S_m|}\epsilon \\
        &\leq \inner{a^\star,\theta} + 2\rad_{m,a^\star}^{F} + 6\sqrt{|S_m|}\epsilon.
    \end{align*}
    Since $\rad_{m,a}^F = \sum_{i=1}^{|\calS_m|}|\lambda_{m,i}^{(a)}|\cdot \rad_{m,a_{m,i}}^{F}$ with $\|\lambda_{m}^{(a)}\|_2\leq 1$, we have that $\|\lambda_{m}^{(a)}\|_1\leq \sqrt{|\calS_m|}$ and
    \begin{align*}
        &\Delta_a\leq 4\sqrt{|\calS_m|}\left(\max_{a\in \calS_m}\rad_{m,a}^{F}+2\epsilon\right)= 4\sqrt{3n}\max_{a\in \calS_m}\rad_{m,a}^{F}+8\sqrt{3n}\epsilon \leq \frac{8\sqrt{n}\beta}{\min_{a'\in \calS_m}\sqrt{\unbiasSize_m(a')}}+16\sqrt{n}\epsilon.
    \end{align*}
    
    If $B\leq \mu^\star$, then we have
    \begin{align*}
        \inner{a^\star,\theta}+\epsilon\geq \mu^\star\geq B \geq \LCB_{m}(a) - 4\sqrt{|\calS_m|}\epsilon \geq \inner{a,\theta} - 2\rad_{m,a}^{F} - 5\sqrt{|\calS_m|}\epsilon,
    \end{align*}
    where the second inequality is because $a$ is not eliminated in epoch $m$. Therefore, we always have
    \begin{align*}
        \Delta_a &\leq 2\rad_{m,a}^{F} + 6\sqrt{|\calS_m|}\epsilon \leq \frac{8\sqrt{n}\beta}{\min_{a'\in \calS_m}\sqrt{\unbiasSize_m(a')}} + 12\sqrt{n}\epsilon.
    \end{align*}
    In addition, we know that for all $a\in \calS_m$,
    \begin{align*}
        2^m \leq \unbiasSize_m(a) + \frac{D}{|\calS_m|} + 1 \leq \unbiasSize_m(a) + \frac{2D}{|\calS_m|}.
    \end{align*}
    Therefore, if $12\sqrt{n}\epsilon\geq \frac{\Delta_a}{2}$, then we have
    \begin{align*}
        2^m\Delta_a\leq 2^m\cdot 24\sqrt{n}\epsilon;
    \end{align*}
    otherwise, we have $\Delta_a \leq \frac{8\sqrt{n}\beta}{\min_{a\in \calS_m}\sqrt{\unbiasSize_m(a)}} + 12\sqrt{n}\epsilon \leq \frac{8\sqrt{n}\beta}{\min_{a\in \calS_m}\sqrt{\unbiasSize_m(a)}}  + \frac{\Delta_a}{2}$ and
    \begin{align*}
        \Delta_a \leq \frac{16\sqrt{n}\beta}{\min_{a'\in \calS_m}\sqrt{\unbiasSize_m(a')}},
    \end{align*}
    and we can obtain that
    \begin{align*}
        \min_{a'\in \calS_m}{\unbiasSize_m(a')}\cdot \Delta_a\leq \frac{256d\beta^2}{\Delta_a}.
    \end{align*}
    Combining the above two cases, we know that for all $a\in\calA_m$, $$2^m\cdot \Delta_a\leq 2^m\cdot 24\sqrt{n}\epsilon+ \min_{a'\in \calS_m}\unbiasSize_m(a')\cdot \Delta_a + \frac{2D\Delta_a}{|\calS_m|} \leq  2^m\cdot 24\sqrt{n}\epsilon+\frac{256n\beta^2}{\Delta_a} + \frac{2D\Delta_a}{|\calS_m|}.$$
\end{proof}

In fact, the bound above can be obtained by only using $\LCB_{m,1}$. Next, we provide yet-another regret bound within epoch $m$, which utilizes $\LCB_{m,2}$.

\begin{lemma}\label{lem:epoch_B_with_mis}
    \pref{alg:lossLBmis} guarantees that under \pref{event:misLoss}, if action $a$ is not eliminated at the end of epoch $m$ (meaning that $a\in \calA_{m+1}$), then
    \begin{align*}
    \inner{a,\theta}\leq B +\rad_{m,a}^{N}+ \sum_{i=1}^{|\calS_m|}|\lambda_{m,i}^{(a)}|\cdot\left(\frac{2D\mu_{a_{m,i}}}{2^m|\calS_m|}+\frac{16\log T +2}{2^m}\right) + 8\sqrt{|\calS_m|}\epsilon.
\end{align*}
\end{lemma}
\begin{proof}
For all $a\in \calS_m$, since $u_{t}\in[0,1]$, we know that
\begin{align}
    \hat{\mu}_{m}^+(a) &=\frac{1}{2^m}\left(\sum_{\tau\in \obs_m(a)}u_{\tau} + \sum_{\tau\in \unobs_m(a)}1\right) \leq \hat{\mu}_{m,a} + \frac{|\unobs_m(a)|}{2^m}, \label{eqn:pos-bias}\\
    \hat{\mu}_{m}^-(a) &=\frac{1}{2^m}\left(\sum_{\tau\in \obs_m(a)}u_{\tau} \right) \geq \hat{\mu}_{m,a} - \frac{|\unobs_m(a)|}{2^m} \label{eqn:neg-bias}.
\end{align}
Then, under \pref{event:misLoss}, we know that for all $a\in \calA_m$,
\begin{align*}
    \inner{a,\theta} &= \sum_{i=1}^{|\calS_m|}\lambda_{m,i}^{(a)}\inner{a_{m,i},\theta^\star}\\
    &= \sum_{i=1}^{|\calS_m|}\lambda_{m,i}^{(a)}(\mu_{a_{m,i}}-\epsilon_{a_{m,i}}) \tag{since $\mu_a = \inner{a,\theta^\star}+\epsilon_a$}\\
    &\leq \sum_{i=1}^{|\calS_m|}\lambda_{m,i}^{(a)}\cdot \mu_{a_{m,i}} + \sqrt{|\calS_m|}\epsilon \tag{since $\|\lambda_{m}^{(a)}\|_1\leq \sqrt{|\calS_m|}$} \\
    &\leq \sum_{i=1}^{|\calS_m|}\lambda_{m,i}^{(a)}\cdot \hat{\mu}_{m}(a_{m,i}) + \rad_{m,a}^{N} + 3\sqrt{|\calS_m|}\epsilon \tag{since \pref{event:misLoss} holds}\\
    &\leq \sum_{i=1}^{|\calS_m|}\lambda_{m,i}^{(a)}\cdot\hat{\mu}_{m}^{sgn(-\lambda_{m,i}^{(a)})}(a_{m,i}) + \rad_{m,a}^{N}+\sum_{i=1}^{|\calS_m|}|\lambda_{m,i}^{(a)}|\cdot \frac{|\unobs_m(a_{m,i})|}{2^m} + 3\sqrt{|\calS_m|}\epsilon \tag{using \pref{eqn:pos-bias} and \pref{eqn:neg-bias}}\\
    &= \LCB_{m,1}(a) + \rad_{m,a}^{N}+\sum_{i=1}^{|\calS_m|}|\lambda_{m,i}^{(a)}|\cdot \frac{|\unobs_m(a_{m,i})|}{2^m} + 3\sqrt{|\calS_m|}\epsilon\\
    &\leq \LCB_{m,1}(a) +\rad_{m,a}^{N}+ \sum_{i=1}^{|\calS_m|}|\lambda_{m,i}^{(a)}|\cdot\left(\frac{2D\mu_{a_{m,i}}}{2^m|\calS_m|}+\frac{16\log KT +2}{2^m}\right) + 3\sqrt{|\calS_m|}\epsilon. \tag{since \pref{event:misLoss} holds}
\end{align*}
Since $\LCB_{m,1}(a)\leq B+4\sqrt{|\calS_m|}\epsilon$ (as $a$ is not eliminated at the end of epoch $m$), we have
\begin{align*}
    \inner{a,\theta}\leq B +\rad_{m,a}^{N}+ \sum_{i=1}^{|\calS_m|}|\lambda_{m,i}^{(a)}|\cdot\left(\frac{2D\mu_{a_{m,i}}}{2^m|\calS_m|}+\frac{16\log T +2}{2^m}\right) + 8\sqrt{|\calS_m|}\epsilon.
\end{align*}
\end{proof}

\begin{lemma}\label{lem:bound_2_mis}
    If \pref{event:misLoss} holds, \pref{alg:lossLBmis} guarantees that if $a$ is not eliminated at the end of epoch $m$, then we also have
    \begin{align*}
        2^m\Delta_a\leq \frac{256n\beta^2}{\Delta_a} +\frac{4DB + 12\sum_{i=1}^{|\calS_m|}|\lambda_{m,i}^{(a)}|\cdot D\mu_{a_{m,i}}}{|\calS_m|}+(128\log T +16)\sqrt{n}+2^m\cdot 64\sqrt{n}\epsilon.
    \end{align*}
\end{lemma}
\begin{proof}
    If $\inner{a,\theta}\leq 2B$, we know that $\Delta_a = \inner{a-a^\star,\theta} \leq 2B$. Using \pref{lem:delta_1_loss_miss}, we can obtain that
    \begin{align*}
        2^m\cdot \Delta_a &\leq 2^m\cdot 24\sqrt{n}\epsilon+\frac{256n\beta^2}{\Delta_a} + \frac{2D\Delta_a}{|\calS_m|} \\
        &\leq 2^m\cdot 24\sqrt{n}\epsilon+\frac{256n\beta^2}{\Delta_a} + \frac{4DB}{|\calS_m|}
    \end{align*}
    If $\inner{a,\theta}\geq 2B$, we have $B\leq \frac{\inner{a,\theta}}{2}$. Using \pref{lem:epoch_B_with_mis}, we know that
    \begin{align*}
        \Delta_a &\leq \inner{a,\theta} \leq \underbrace{2\cdot \rad_{m,a}^{N}}_{\term{1}}+ \underbrace{2\sum_{i=1}^{|\calS_m|}|\lambda_{m,i}^{(a)}|\cdot\left(\frac{2D\mu_{a_{m,i}}}{2^m|\calS_m|}+\frac{16\log T +2}{2^m}\right) + 16\sqrt{|\calS_m|}\epsilon}_{\term{2}}.
    \end{align*}

    If $\term{1}\geq \term{2}$, we have
    \begin{align*}
        \Delta_a &\leq 4\rad_{m,a}^{N} \epsilon \leq 4\sqrt{|\calS_m|}\max_{a_m\in\calS_m}\rad_{m,a_m}^N \leq \frac{8\beta\sqrt{n}}{2^{m/2}},
    \end{align*}
    meaning that $2^m\Delta_a \leq \frac{64n\beta^2}{\Delta_a}$.
    Otherwise, we have
    \begin{align*}
        \Delta_a\leq 4\sum_{i=1}^{|\calS_m|}|\lambda_{m,i}^{(a)}|\cdot \left(\frac{2D\mu_{a_{m,i}}}{2^m|\calS_m|}+\frac{16\log T +2}{2^m}\right) + 64\sqrt{n}\epsilon,
    \end{align*}
    meaning that
    \begin{align*}
        2^m\Delta_a\leq \frac{8\sum_{i=1}^{|\calS_m|}|\lambda_{m,i}^{(a)}|\cdot D\mu_{a_{m,i}}}{|\calS_m|}+(128\log T +16)\sqrt{n}+2^m\cdot 64\sqrt{n}\epsilon.
    \end{align*}
    Combining both cases, we know that
    \begin{align*}
        2^m\Delta_a\leq \frac{256n\beta^2}{\Delta_a} +\frac{4DB + 12\sum_{i=1}^{|\calS_m|}|\lambda_{m,i}^{(a)}|\cdot D\mu_{a_{m,i}}}{|\calS_m|}+(128\log T +16)\sqrt{n}+2^m\cdot 64\sqrt{n}\epsilon.
    \end{align*}
\end{proof}

Now we are ready to prove our main result \pref{thm:lossLBmis}.
\begin{proof}[Proof of Theorem~\ref{thm:lossLBmis}]
    We analyze the regret when \pref{event:misLoss} holds, which happens with probability at least $1-\frac{2}{T^2}$. When \pref{event:misLoss} does not hold, the expected regret is bounded by $\frac{2}{T}$.
    
    We then bound the regret with a fixed choice of $B$. Combining \pref{lem:delta_1_loss_miss} and \pref{lem:epoch_B_with_mis}, if action $a$ is not eliminated at the end of epoch $m$, we have
    \begin{align*}
        2^{m}\cdot \Delta_a&\leq \frac{256n\beta^2}{\Delta_a} +\frac{4DB + 12\sum_{i=1}^{|\calS_{m}|}|\lambda_{m,i}^{(a)}|\cdot D\mu_{a_{m,i}}}{|\calS_m|}+(128\log T +16)\sqrt{n}+2^m\cdot 64\sqrt{n}\epsilon, \\
        2^m\cdot \Delta_a&\leq 2^m\cdot 24\sqrt{n}\epsilon+\frac{256n\beta^2}{\Delta_a} + \frac{2D\Delta_a}{|\calS_m|}.
    \end{align*}
    Therefore, we have
    \begin{align*}
        \Delta_a \leq \order\left(\frac{n\beta^2}{2^m\cdot \Delta_a} + \sqrt{n}\epsilon + \frac{\sqrt{n}\log T}{2^m}\right) +  \frac{1}{2^m}\min\left\{\frac{4DB+12\sum_{i=1}^{|\calS_m|}|\lambda_{m,i}^{(a)}|\cdot D\mu_{a_{m,i}}}{n}, \frac{D\Delta_a}{n}\right\}.
    \end{align*}
    Denote $\calT_B$ to be the number of rounds \pref{alg:lossLBmis} proceeds with $B$ and define $\Reg_B$ be the expected regret within $\calT_B$ rounds.
    Then, for any $\alpha_m\geq 0$,  the overall regret is then upper bounded as follows:
    \begin{align*}
        \Reg_B &\triangleq \sum_{m= 1}^{\lceil\log_2(|\calT_B|/3n\rceil}\sum_{a\in \calS_m}2^{m}\cdot\Delta_a \\
        &\leq \sum_{m=1}^{\lceil\log_2(|\calT_B|/3n\rceil}\sum_{a\in \calS_m}\mathbbm{1}\{\Delta_a> \alpha_m\}\left(\order\left(\frac{n\beta^2}{\Delta_a} + 2^m\sqrt{n}\epsilon + \sqrt{n}\log T\right) \right.\\
        &\qquad +\left.\min\left\{\frac{4DB+12\sum_{i=1}^{|\calS_{m-1}|}|\lambda_{m-1,i}^{(a)}|\cdot d(a_{m-1,i})}{n}, \frac{2D\Delta_a}{n}\right\}\right) \tag{since $a$ is not eliminated in epoch $m-1$ for all $a\in\calS_m$} \\
        &\qquad + \sum_{m\geq 1}\sum_{a\in \calS_m}\mathbbm{1}\{\Delta_a\leq \alpha_m\}2^m\Delta_a.
    \end{align*}
    Picking $\alpha_m = \beta\sqrt{\frac{n}{2^m}}$, we can obtain that
    \begin{align*}
        \Reg_B &=\sum_{m= 1}^{\lceil\log_2(|\calT_B|/3n\rceil}\sum_{a\in \calS_m}\left(\order\left(\beta\sqrt{n\cdot 2^m} +2^m\sqrt{n}\epsilon+ \sqrt{n}\log T \right) \right.\\
        &\qquad +\left.\min\left\{\frac{4DB+12\sum_{i=1}^{|\calS_{m-1}|}|\lambda_{m-1,i}^{(a)}|\cdot d(a_{m-1,i})}{n}, \frac{2D\Delta_a}{n}\right\}\right) \\
        &\leq \order\left(|\calT_B|\sqrt{n}\epsilon + \beta n\sqrt{|\calT_B|}+\sqrt{n}\log T\log(T/n)\right) \\
        &\qquad + \sum_{m= 1}^{\lceil\log_2(|\calT_B|/3n)\rceil}\sum_{a\in \calS_m}\min\left\{\frac{4DB+12\sum_{i=1}^{|\calS_{m-1}|}|\lambda_{m-1,i}^{(a)}|\cdot d(a_{m-1,i})}{n}, \frac{2D\Delta_a}{n}\right\}.
    \end{align*}
    
    On the other hand, picking $\alpha_m=0$, we have
    \begin{align*}
        \Reg_B &\leq \sum_{m=1}^{\lceil\log_2(|\calT_B|/3n\rceil}\sum_{a\in \calS_m}\left(\order\left(\frac{n\beta^2}{\Delta_{\min}} + 2^m\sqrt{n}\epsilon + \sqrt{n}\log T\right) \right.\\
        &\qquad +\left.\min\left\{\frac{4DB+12\sum_{i=1}^{|\calS_{m-1}|}|\lambda_{m-1,i}^{(a)}|\cdot d(a_{m-1,i})}{n}, \frac{2D\Delta_a}{n}\right\}\right) \\
        &\leq \order\left(\frac{n^2\beta^2\log(T/n)}{\Delta_{\min}}+\epsilon\sqrt{n}|\calT_B|+\sqrt{n}\log T\log(T/n)\right)\\
        &\qquad + \sum_{m= 1}^{\lceil\log_2(|\calT_B|/3n\rceil}\sum_{a\in \calS_m} \min\left\{\frac{4DB+12\sum_{i=1}^{|\calS_{m-1}|}|\lambda_{m-1,i}^{(a)}|\cdot d(a_{m-1,i})}{n}, \frac{2D\Delta_a}{n}\right\}.
    \end{align*}

    Using the fact that $\beta=\sqrt{2\log(KT^3)}$ and combining both bounds, we can obtain that
    \begin{align}
        \Reg_B &\leq \order\left(\min\left\{\frac{n^2\log(KT)\log(T/n)}{\Delta_{\min}}, n\sqrt{|\calT_B|\log(KT)}\right\}+\epsilon\sqrt{n}|\calT_B|\right)\nonumber\\
        &\qquad + \sum_{m= 1}^{\lceil\log_2(|\calT_B|/3n\rceil}\sum_{a\in \calS_m}\min\left\{\frac{4DB+12\sum_{i=1}^{|\calS_{m-1}|}|\lambda_{m-1,i}^{(a)}|\cdot d(a_{m-1,i})}{n}, \frac{2D\Delta_a}{n}\right\}. \label{eqn:reg_b}%
    \end{align}
    For notational convenience, let $R_B=\order\left(\min\left\{\frac{n^2\log(KT)\log(T/n)}{\Delta_{\min}},n\sqrt{|\calT_B|\log(KT)}\right\}+\epsilon\sqrt{n}|\calT_B|\right)$. To further analyze this bound, we first upper bound $\min\left\{\frac{4DB+12\sum_{i=1}^{|\calS_{m-1}|}|\lambda_{m-1,i}^{(a)}|\cdot d(a_{m-1,i})}{n}, \frac{2D\Delta_a}{n}\right\}$ by $\frac{2D\Delta_a}{n}$ and obtain that
    \begin{align}\label{eqn:reg_B_1}
        \Reg_B \leq R_B + \order\left(D\Delta_{\max}\log(T/n)\right).
    \end{align}

    On the other hand, we can also upper bound $\min\left\{\frac{4DB+12\sum_{i=1}^{|\calS_{m-1}|}|\lambda_{m-1,i}^{(a)}|\cdot d(a_{m-1,i})}{n}, \frac{2D\Delta_a}{n}\right\}$ by $\frac{12\sum_{i=1}^{|\calS_{m-1}|}|\lambda_{m-1,i}^{(a)}|\cdot d(a_{m-1,i})}{n}$ and obtain that
    \begin{align*}
        \Reg_B \leq R_B + \left(\sum_{m=1}^{\lceil\log_2(|\calT_B|/3n\rceil}\sum_{a\in\calS_m}\frac{4DB+12\sum_{i=1}^{|\calS_{m-1}|}|\lambda_{m-1,i}^{(a)}|\cdot d(a_{m-1,i})}{n}\right).
    \end{align*} 

    Let $L_{\Alg}^m=\sum_{a\in \calS_m}2^m\mu_a$ be the total expected loss within epoch $m$ and $L_{\star}^m=|\calS_m|\cdot 2^m\cdot\mu^\star$ be the total expected loss for the optimal action. Define $\Reg_m=L_{\Alg}^m-L_{\star}^m$. Direct calculation shows that
    \begin{align*}
        &\sum_{a\in \calS_m}\frac{\sum_{i=1}^{|\calS_{m-1}|}|\lambda_{m-1,i}^{(a)}|\cdot d(a_{m-1,i})}{n} \\
        &\leq \frac{3D}{2^{m-1}}\cdot 2^{m-1}\sum_{i=1}^{|\calS_{m-1}|}\mu_{a_{m-1,i}}\tag{since $|\lambda_{m-1,i}^{(a)}|\leq 1$ and $|\calS_m|=3n$}\\
        &= \frac{3D}{2^{m-1}}L_{\Alg}^{m-1}.
    \end{align*}
    Using the fact that $\Reg_B = \sum_{m=1}^{\lceil\log(|\calT_B|/3n)\rceil}\Reg_m$, we know that
    \begin{align*}
        &\sum_{m=1}^{\lceil\log(|\calT_B|/3n)\rceil} (L_{\Alg}^m-L^m_{\star})\\
        &\leq \sum_{m=1}^{\lceil\log(|\calT_B|/3n)\rceil}\Reg_m + 2\epsilon\cdot |\calT_B|\\
        &\leq R_B + \sum_{m=\lceil\log_2(72D)\rceil+1}^{\lceil\log(|\calT_B|/3n)\rceil}\frac{36D}{2^{m-1}}\cdot L_{\Alg}^{m-1} + \sum_{m=1}^{\lceil\log_2(72D)\rceil}2^m\Delta_{\max} + 12DB\log(T/n) \tag{$2\epsilon\cdot |\calT_B|$ is subsumed in $R_B$}\\
        &\leq R_B + \sum_{m=\lceil\log_2(72D)\rceil+1}^{\lceil\log(|\calT_B|/3n)\rceil}\frac{36D}{2^{m-1}}\cdot \left(L_{\Alg}^{m-1}-L_{\star}^{m-1}\right) + \sum_{m=\lceil\log_2(72D)\rceil+1}^{\lceil\log(|\calT_B|/3n)\rceil}\frac{36D}{2^{m-1}}\cdot L_{\star}^{m-1} + \sum_{m=1}^{\lceil\log_2(72D)\rceil}2^m\Delta_{\max} +12DB\log(T/n)\\
        &\leq R_B + \frac{1}{2}\sum_{m=\lceil\log_2(72D)\rceil+1}^{\lceil\log(|\calT_B|/3n)\rceil} \left(L_{\Alg}^{m-1}-L_{\star}^{m-1}\right) + 36nD\mu^\star\log(T/(216nD))+144D\Delta_{\max} +12DB\log(T/n)\\
        &= R_B + \frac{1}{2}\sum_{m=\lceil\log_2(72D)\rceil+1}^{\lceil\log(|\calT_B|/3n)\rceil} \left(L_{\Alg}^{m-1}-L_{\star}^{m-1}\right) + 36nd^\star\log(T/(216nD))+144D\Delta_{\max}+12DB\log(T/n).
    \end{align*}
    Rearranging the terms, we can obtain that
    \begin{align}\label{eqn:reg_B_2}
        \Reg_B \leq R_B + 72nd^\star\log(T/(216nD))+288D\Delta_{\max}+12DB\log(T/n).
    \end{align}
    Combining \pref{eqn:reg_B_1} and \pref{eqn:reg_B_2}, we know that
    \begin{align}
        \Reg_B &\leq \order\left(\min\left\{\frac{n^2\log(KT)\log(T/n)}{\Delta_{\min}},n\sqrt{|\calT_B|\log(KT)}\right\}+\epsilon\sqrt{n}|\calT_B|\right) \nonumber \\
        &\qquad +\order\left( \min\left\{nd^\star\log (T/nD)+D\Delta_{\max}+DB\log(T/n),D\Delta_{\max}\log (T/n)\right\}\right).\label{eqn:reg_B_final}
    \end{align}
    Finally, according to \pref{lem:end-of-B}, \pref{alg:lossLBmis} fails at most $\lceil\log_2(D\mu^\star))\rceil = \lceil\log_2(d^\star))\rceil$ times. Summing up the regret over all rounds, we know that the overall regret is bounded as follows
    \begin{align*}
        \Reg \leq \sum_{r=0}^{\lceil\log_2(d^\star))\rceil}\Reg_{2^r/D} &\leq \order\left(\min\left\{\frac{n^2\log(KT)\log(T/n)\log(d^\star)}{\Delta_{\min}},n\sqrt{T\log(d^\star)\log(KT)}\right\}+\epsilon\sqrt{n}T\right) \\
        &\qquad + \log(d^\star)\cdot \order\left( \min\left\{nd^\star\log (T/n)+D\Delta_{\max},D\Delta_{\max}\log (T/n)\right\}\right),
    \end{align*}
	which finishes the proof.
\end{proof}
\section{Omitted Details for Delay-as-Reward}\label{app: reward}
In this section, we show our results for the delay-as-reward setting. The difference compared with the delay-as-loss setting is that now, $\mu_a=\inner{a,\theta}+\epsilon_a\in[0,1]$ represents the expected reward of picking action $a$, where $|\epsilon_a|\leq \epsilon$ for all $a\in\calA$. The learner's goal is to minimize the pseudo regret defined as follows:
\begin{align}\label{eqn:reward-regret}
    \Reg\triangleq T\max_{a\in\calA}\inner{a,\theta} - \E\left[\sum_{t=1}^T\inner{a_t,\theta}\right].
\end{align}
 Define $\Delta_a = \inner{a^\star-a,\theta}$ as the suboptimality gap of action $a$, where $a^\star \in \argmax_{a\in\calA}\inner{a, \theta}$, and $\mu^\star \triangleq \max_{a\in\calA}\mu_a$ as the reward of the optimal action. Again, note that due to the misspecification, $\mu^\star$ may not necessarily be $\mu_{a^\star}$. Define $\Delta_{\min} = \min_{a\in \calA, \Delta_a>0}\Delta_a$ to be the minimum non-zero sub-optimality gap. The delay at round $t$ is still defined as $d_t=D\cdot u_t$, and $d^\star = D\cdot \mu^\star$ is  the expected delay of the optimal action. We also define $d(a)=D\mu_a$ to be the expected delay for action $a$.

\newpage
\subsection{Algorithm for Linear Bandits with Delay-as-Reward}
We list our algorithm for the reward case in \pref{alg:rewardLBmis} for completeness. The algorithm shares the same idea as \pref{alg:lossLBmis}.

\setcounter{AlgoLine}{0}
\begin{algorithm}[H]
\caption{Phased Elimination for Linear Bandits with Delay-as-Reward}\label{alg:rewardLBmis}

\nl Input: maximum possible delay $D$, action set $\calA$, $\beta>0$, a misspecification level $\epsilon$. 

\nl Initialize optimal reward guess $B=1$.

\nl Initialize active action set $\calA_1=\calA$.  \label{line:reward-restart} 

\nl \For{$m=1,2,\dots,$}{
    \nl Find $\calS_m=\{a_{m,1},\dots,a_{m,|\calS_m|}\}$ to be the volumetric spanner of $\calA_m$, where $|\calS_m|= 3n$. \label{line:volume-reward}
    
    \nl Pick each $a\in \calS_m$ $2^m$ times in a round-robin way. \label{line:round-robin-reward}

    \nl Let $\calI_m$ contain all the rounds in this epoch.
    
    \nl For all $a\in \calS_m$, calculate the following quantities
    \begin{align}
        &\hat{\mu}_{m}^+(a)=\frac{1}{2^m}\Big(\sum_{\tau\in \obs_m(a)}u_{\tau} + \sum_{\tau\in \unobs_m(a)}1\Big), \\
        &\hat{\mu}_{m}^-(a)=\frac{1}{2^m}\sum_{\tau\in \obs_m(a)}u_{\tau}, \\
        &\hat{\mu}_{m,1}^{+}(a)=\hat{\mu}^{+}_{m}(a)+\frac{\beta}{2^{m/2}}\|a\|_2, \label{eqn:reward-ucb-linear-1-mis}\\
        &\hat{\mu}_{m,1}^{-}(a)=\hat{\mu}^{-}_{m}(a)-\frac{\beta}{2^{m/2}}\|a\|_2,\label{eqn:reward-lcb-linear-1-mis}\\
        &\hat{\mu}_m^{F}(a)=\frac{1}{\unbiasSize_m(a)}\sum_{\tau\in \unbias_m(a)}u_{\tau},\\
        &\hat{\mu}_{m,2}^{+}(a)=\hat{\mu}_m^F(a)+\frac{\beta}{\sqrt{\unbiasSize_m(a)}}\|a\|_2, \label{eqn:reward-ucb-linear-2-mis}\\
        &\hat{\mu}_{m,2}^{-}(a)=\hat{\mu}_m^F(a)-\frac{\beta}{\sqrt{\unbiasSize_m(a)}}\|a\|_2, \label{eqn:reward-lcb-linear-2-mis}
    \end{align}
    where $\unbiasSize_m(a) = |\unbias_m(a)|$, $\unbias_m(a) = \{\tau\in \calI_m: \tau+D\in\calI_m, a_{\tau}=a\}$, $\obs_m(a) = \{\tau\in \calI_m: \tau+d_{\tau}\in\calI_m, a_{\tau}=a\}$, and
    $\unobs_m(a)= \{\tau\in \calI_m: a_{\tau}=a\}\setminus\obs_m(a)$.

    \nl \For{each $a\in \calA_m$}{
        \nl \label{line: decompose-rewward}
        Decompose $a$ as $a=\sum_{i=1}^{|S_m|}\lambda_{m,i}^{(a)}a_{m,i}$ with $\|\lambda_{m}^{(a)}\|_2\leq 1$ and calculate 
        {\small
        \begin{align}
            &\LCB_{m}(a)=\sum_{i=1}^{|\calS_m|}\lambda_{m,i}^{(a)}\cdot\hat{\mu}_{m,2}^{-\sgn(\lambda_{m,i}^{(a)})}(a_{m,i}), \label{eqn:reward-ucb-f-all-action-mis} \\
            &\UCB_m(a) = \max_{j\in \{1,2\}}\{\UCB_{m,j}(a)\} \;\;\text{where} \nonumber  \\
            & \UCB_{m,j}(a)=\sum_{i=1}^{|\calS_m|}\lambda_{m,i}^{(a)}\cdot\hat{\mu}_{m,j}^{\sgn(\lambda_{m,i}^{(a)})}(a_{m,i}),\label{eqn:reward-lcb-f-all-action-mis}
        \end{align}
        }
    }
    
    \nl Set $\calA_{m+1} = \calA_m$.
    
    \nl \For{$a_1\in \calA_m$}{
        \nl \label{line:reward_eliminate_miss}\If{$\exists a_2\in \calA_m$, such that $\max\{\LCB_m(a_2),B\} \geq \UCB_m(a_1)+4\sqrt{3n}\epsilon $}
        {
         \nl Eliminate $a_1$ from $\calA_{m+1}$.
        }
    }
    \nl \If{$\calA_{m+1}$ is empty}{
        Set $B\leftarrow B/2$ and go to \pref{line:reward-restart}.
    }
}
\end{algorithm}

\subsection{Regret Guarantees}\label{app: reward-regret}
In this section, we show the theoretical guarantees for our algorithm in the delay-as-reward setting.

\begin{theorem}\label{thm:reward-main}
    \pref{alg:rewardLBmis} with $\beta = \sqrt{2\log(KT^3)}$ guarantees that
    \begin{align*}
        \Reg &\leq \order\left(\min\left\{\frac{n^2\log(KT)\log(T/n)\log(1/\mu^\star)}{\Delta_{\min}}, n\sqrt{T\log(KT)\log(1/\mu^\star)}\right\}+\epsilon\sqrt{n}T\right)\\
        &\qquad+\order\left(\min\left\{\sum_{j=0}^{\lceil\log_2(1/\mu^\star)\rceil}\sum_{m=1}^{\lceil\log_2(|\calT_{2^{-j}}|/3n\rceil}\sum_{i=1}^{3n} d(a_{m-1,i}^{(2^{-j})}), D\Delta_{\max}\log(1/\mu^\star)\log(T/n)\right\}\right),
    \end{align*}
    where $\{a_{m,i}^{(B)}\}_{i=1}^{3n}$ represents the set of volumetric spanner at epoch $m$ with the optimal reward guess $B$. 
\end{theorem}

Similar to the analysis in \pref{app:loss}, our analysis is based on the condition that \pref{event:misLoss} holds, which happens with probability $1-\frac{2}{T^2}$ according to \pref{lem:high-prob-event}. The following lemma is a counterpart of \pref{lem:end-of-B}, providing an upper bound of the number of guesses on the optimal reward $B$.

\begin{lemma}\label{lem:end-of-B-reward}
    Suppose that \pref{event:misLoss} holds. If $B\leq \mu^\star$, then $a^\star\in \calA_m$ for all $m$.
\end{lemma}
\begin{proof}
    Since \pref{event:misLoss} holds, we have, we know that for all $a\in\calA_m$, $\UCB_m(a)+\sqrt{|\calS_m|}\epsilon\geq \inner{a,\theta}$, $\LCB_m(a)+\sqrt{|\calS_m|}\epsilon\leq \inner{a,\theta}$
    If $B\leq \mu^\star$, then we have $a^\star$ never eliminated since for any $a\in\calA_m$,
    \begin{align*}
         \UCB_{m}(a^\star) +2\epsilon\sqrt{|\calS_m|} &\geq \max_{a\in\calA}\{\inner{a,\theta}+\epsilon_a\} \geq \mu^\star \geq B,\\
         \UCB_{m}(a^\star) +4\epsilon\sqrt{|\calS_m|} &\geq \mu^\star + 2\epsilon\sqrt{|\calS_m|} \geq \inner{a,\theta}  + \epsilon\sqrt{|\calS_m|}\geq \LCB_m(a).
    \end{align*}
    Therefore, $a^\star$ never satisfy the elimination condition.
\end{proof}

The following lemma is a counterpart of \pref{lem:delta_1_loss_miss}.

\begin{lemma}\label{lem:delta_1_reward_miss}
    Suppose that \pref{event:misLoss} holds. \pref{alg:rewardLBmis} guarantees that if $a\in\calA$ is not eliminated at the end of epoch $m$ (meaning that $a\in \calA_{m+1}$), then 
    \begin{align*}
        2^m\cdot \Delta_a\leq 2^m\cdot 24\sqrt{n}\epsilon+\frac{256n\beta^2}{\Delta_a} + \frac{2D\Delta_a}{|\calS_m|}.
    \end{align*}
\end{lemma}
\begin{proof}
    Since \pref{event:misLoss} holds, we know that for all $a\in\calA_m$, $\LCB_m(a)\leq \mu_a + \sqrt{|\calS_m|}\epsilon$, $\UCB_m(a)\geq \mu_a - \sqrt{|\calS_m|}\epsilon$. Moreover, as $\UCB_m(a)=\min\{\UCB_{m,1}(a),\UCB_{m,2}(a)\}$, we know that for all $a\in \calA_m$
    \begin{align*}
        \UCB_{m,1}(a) - 2\rad_{m,a}^{N} - 2\epsilon\sqrt{|\calS_m|}=  \hat{\mu}_{m,1}(a) - \rad_{m,a}^{N} - 2\epsilon\sqrt{|\calS_m|}\leq \inner{a,\theta},\\
        \UCB_{m,2}(a) - 2\rad_{m,a}^{F} - 2\epsilon\sqrt{|\calS_m|}=  \hat{\mu}_{m,2}(a) - \rad_{m,a}^{F} - 2\epsilon\sqrt{|\calS_m|}\leq \inner{a,\theta},\\
        \LCB_{m}(a) + 2\rad_{m,a}^{F} + 2\epsilon\sqrt{|\calS_m|} = \hat{\mu}_{m,2}(a) + \rad_{m,a}^{F}+2\epsilon\sqrt{|\calS_m|}\geq \inner{a,\theta}.        
    \end{align*}
    If $B\leq \mu^\star$, then $a^\star\in \calA_m$ according to \pref{lem:end-of-B-reward}.
    Moreover, if $a$ is not eliminated in epoch $m$, we have $\UCB_m(a)+4\sqrt{|S_m|}\epsilon\geq \max\{\LCB_m(a^\star),B\}$, meaning that
    \begin{align*}
        &\inner{a,\theta} + 2\rad_{m,a}^{F} + 2\epsilon\sqrt{|\calS_m|} \\
        &\geq \wh{\mu}_{m,2}(a) + \rad_{m,a}^{F} \\
        &\geq \UCB_m(a) \\
        &\geq \max\{\LCB_m(a^\star),B\}-4\sqrt{|S_m|}\epsilon \\
        &\geq \LCB_m(a^\star) - 4\sqrt{|S_m|}\epsilon \\
        &= \wh{\mu}_{m,2}(a^\star) - \rad_{m,a^\star}^{F} - 4\sqrt{|S_m|}\epsilon \\
        &\geq \inner{a^\star,\theta}  - 2\rad_{m,a^\star}^{F} - 6\sqrt{|S_m|}\epsilon.
    \end{align*}
    Since $\rad_{m,a}^F = \sum_{i=1}^{|\calS_m|}|\lambda_{m,i}^{(a)}|\cdot \rad_{m,a_{m,i}}^{F}$ with $\|\lambda_{m}^{(a)}\|_2\leq 1$, we have that $\|\lambda_{m}^{(a)}\|_1\leq \sqrt{|\calS_m|}$ and
    \begin{align*}
        &\Delta_a\leq 4\sqrt{|\calS_m|}\left(\max_{a\in S_m}\rad_{m,a}^{F}+2\epsilon\right)= 4\sqrt{3n}\max_{a\in S_m}\rad_{m,a}^{F}+8\sqrt{3n}\epsilon \leq \frac{8\sqrt{n}\beta}{\min_{a'\in \calS_m}\sqrt{\unbiasSize_m(a')}}+16\sqrt{n}\epsilon.
    \end{align*}
    
    If $B\geq \mu^\star$, then we have
    \begin{align*}
        \mu^\star\leq B \leq \UCB_{m}(a) + 4\sqrt{|\calS_m|}\epsilon \leq \mu_a + 2\rad_{m,a}^{F} + 6\sqrt{|\calS_m|}\epsilon,
    \end{align*}
    where the second inequality is because $a$ is not eliminated in epoch $m$. Therefore, we always have
    \begin{align*}
        \Delta_a &\leq 2\rad_{m,a}^{F} + 6\sqrt{|\calS_m|}\epsilon \leq \frac{8\sqrt{n}\beta}{\min_{a'\in \calS_m}\sqrt{\unbiasSize_m(a')}} + 12\sqrt{n}\epsilon.
    \end{align*} 
    In addition, we know that for all $a\in \calS_m$,
    \begin{align*}
        2^m &= |\calS_m| \leq \unbiasSize_m(a) + \frac{D}{|S_m|} + 1 \leq \unbiasSize_m(a) + \frac{2D}{|S_m|}.
    \end{align*}
    Therefore, if $12\sqrt{n}\epsilon\geq \frac{\Delta_a}{2}$, then we have
    \begin{align*}
        2^m\Delta_a\leq 2^m\cdot 24\sqrt{n}\epsilon;
    \end{align*}
    otherwise, we have $\Delta_a \leq \frac{8\sqrt{n}\beta}{\min_{a\in \calS_m}\sqrt{\unbiasSize_m(a)}} + 12\sqrt{n}\epsilon \leq \frac{8\sqrt{n}\beta}{\min_{a\in S_m}\sqrt{\unbiasSize_m(a)}}  + \frac{\Delta_a}{2}$ and
    \begin{align*}
        \Delta_a \leq \frac{16\sqrt{n}\beta}{\min_{a'\in \calS_m}\sqrt{\unbiasSize_m(a')}},
    \end{align*}
    and we can obtain that
    \begin{align*}
        \min_{a'\in S_m}{\unbiasSize_m(a')}\cdot \Delta_a\leq \frac{256d\beta^2}{\Delta_a}.
    \end{align*}
    Combining the above two cases, we know that for all $a\in\calA_m$, $$2^m\cdot \Delta_a\leq 2^m\cdot 24\sqrt{n}\epsilon+ \min_{a'\in \calS_m}\unbiasSize_m(a')\cdot \Delta_a + \frac{2D\Delta_a}{|\calS_m|} \leq  2^m\cdot 24\sqrt{n}\epsilon+\frac{256n\beta^2}{\Delta_a} + \frac{2D\Delta_a}{|\calS_m|}.$$
\end{proof}

The following lemma is a counterpart of \pref{lem:epoch_B_with_mis}. 

\begin{lemma}\label{lem:epoch_B_with_mis-reward}
    \pref{alg:rewardLBmis} guarantees that under \pref{event:misLoss}, if an action $a$ is eliminated at the end of epoch $m$ (meaning that $a\in \calA_m$), then
    \begin{align*}
    B\leq \inner{a,\theta} +\rad_{m,a}^{N}+ \sum_{i=1}^{|\calS_m|}|\lambda_{m,i}^{(a)}|\cdot\left(\frac{2d(a_{m,i})}{2^m|\calS_m|}+\frac{16\log T +2}{2^m}\right) + 8\sqrt{|\calS_m|}\epsilon,
\end{align*}
where $d(a)=D\mu_a$.
\end{lemma}
\begin{proof}
Under \pref{event:misLoss}, we know that for all $a\in \calA_m$,
\begin{align*}
    \inner{a,\theta} &= \sum_{i=1}^{|\calS_m|}\lambda_{m,i}^{(a)}\inner{a_{m,i},\theta^\star} \\
    &= \sum_{i=1}^{|\calS_m|}\lambda_{m,i}^{(a)}(\mu_{a_{m,i}}-\epsilon_{a_{m,i}}) \tag{since $\mu_a = \inner{a,\theta^\star}+\epsilon_a$}\\
    &\geq \sum_{i=1}^{|\calS_m|}\lambda_{m,i}^{(a)}\cdot \mu_{a_{m,i}} - \sqrt{|\calS_m|}\epsilon \tag{since $\|\lambda_{m}^{(a)}\|_1\leq \sqrt{|\calS_m|}$} \\
    &\geq \sum_{i=1}^{|\calS_m|}\lambda_{m,i}^{(a)}\cdot \hat{\mu}_{m}(a_{m,i}) - \rad_{m,a}^{N} - 3\sqrt{|\calS_m|}\epsilon \tag{since \pref{event:misLoss} holds}\\
    &\geq \sum_{i=1}^{|\calS_m|}\lambda_{m,i}^{(a)}\cdot\hat{\mu}_{m}^{sgn(\lambda_{m,i}^{(a)})}(a_{m,i}) - \rad_{m,a}^{N} -\sum_{i=1}^{|\calS_m|}|\lambda_{m,i}^{(a)}|\cdot \frac{|\unobs_m(a_{m,i})|}{2^m} - 3\sqrt{|\calS_m|}\epsilon \tag{using \pref{eqn:pos-bias} and \pref{eqn:neg-bias}}\\
    &= \UCB_{m,1}(a) - \rad_{m,a}^{N} -\sum_{i=1}^{|\calS_m|}|\lambda_{m,i}^{(a)}|\cdot \frac{|\unobs_m(a_{m,i})|}{2^m} - 3\sqrt{|\calS_m|}\epsilon\\
    &\geq \UCB_{m,1}(a) - \rad_{m,a}^{N} - \sum_{i=1}^{|\calS_m|}|\lambda_{m,i}^{(a)}|\cdot\left(\frac{2d(a_{m,i})}{2^m|\calS_m|}+\frac{16\log KT +2}{2^m}\right) - 4\sqrt{|\calS_m|}\epsilon. \tag{since \pref{event:misLoss} holds}
\end{align*}
Since $\UCB_{m,1}(a)\geq B - 4\sqrt{|\calS_m|}\epsilon$ (as $a$ is not eliminated at the end of epoch $m$), we have
\begin{align*}
    B\leq \inner{a,\theta} +\rad_{m,a}^{N}+ \sum_{i=1}^{|\calS_m|}|\lambda_{m,i}^{(a)}|\cdot\left(\frac{2d(a_{m,i})}{2^m|\calS_m|}+\frac{16\log T +2}{2^m}\right) + 8\sqrt{|\calS_m|}\epsilon.
\end{align*}
\end{proof}

The following lemma is a counterpart of \pref{lem:bound_2_mis}.

\begin{lemma}\label{lem:bound_2_mis_reward}
    If $B\geq \frac{\mu^\star}{2}$ and \pref{event:misLoss} holds, \pref{alg:rewardLBmis} guarantees that if $a$ is not eliminated at the end of epoch $m$, then we also have
    \begin{align*}
        2^m\Delta_a\leq \frac{256n\beta^2}{\Delta_a} +\frac{12\sum_{i=1}^{|\calS_m|}|\lambda_{m,i}^{(a)}|\cdot d(a_{m,i})}{|\calS_m|}+(128\log T +16)\sqrt{n}+2^m\cdot 64\sqrt{n}\epsilon,
    \end{align*}
    where $d(a)=D\mu_a$.
\end{lemma}
\begin{proof}
    If $\inner{a,\theta}\geq \frac{B}{2}$, we know that $\Delta_a = \inner{a^\star-a,\theta} \leq 3\inner{a,\theta}$. Using \pref{lem:delta_1_reward_miss}, we can obtain that
    \begin{align*}
        2^m\cdot \Delta_a &\leq 2^m\cdot 24\sqrt{n}\epsilon+\frac{256n\beta^2}{\Delta_a} + \frac{2D\inner{a,\theta}}{|\calS_m|} \\
        &\leq 2^m\cdot 24\sqrt{n}\epsilon+\frac{256n\beta^2}{\Delta_a} + \frac{2\sum_{i=1}^{|\calS_m|}|\lambda_{m,i}^{(a)}|\cdot d(a_{m,i})}{|\calS_m|}.
    \end{align*}
    If $\inner{a,\theta} \leq \frac{B}{2}$, we have $3(B-\inner{a,\theta} ) \geq \frac{3B}{2} \geq \inner{a^\star-a,\theta}$. Using \pref{lem:epoch_B_with_mis}, we know that
    \begin{align*}
        \Delta_a &\leq \mu_a \leq \underbrace{3\cdot \rad_{m,a}^{N}}_{\term{1}}+ \underbrace{3\sum_{i=1}^{|\calS_m|}|\lambda_{m,i}^{(a)}|\cdot\left(\frac{2d(a_{m,i})}{2^m|\calS_m|}+\frac{16\log T +2}{2^m}\right) + 24\sqrt{|\calS_m|}\epsilon}_{\term{2}}.
    \end{align*}

    If $\term{1}\geq \term{2}$, we have
    \begin{align*}
        \Delta_a &\leq \mu_a \leq 6\rad_{m,a}^{N} \epsilon \leq 6\sqrt{|\calS_m|}\max_{a_m\in\calS_m}\rad_{m,a_m}^N \leq \frac{12\beta\sqrt{n}}{2^{m/2}},
    \end{align*}
    meaning that $2^m\Delta_a \leq \frac{144n\beta^2}{\Delta_a}$.
    Otherwise, we have
    \begin{align*}
        \Delta_a\leq 6\sum_{i=1}^{|\calS_m|}|\lambda_{m,i}^{(a)}|\cdot \left(\frac{2d(a_{m,i})}{2^m|\calS_m|}+\frac{16\log T +2}{2^m}\right) + 96\sqrt{n}\epsilon,
    \end{align*}
    meaning that
    \begin{align*}
        2^m\Delta_a\leq \frac{12\sum_{i=1}^{|\calS_m|}|\lambda_{m,i}^{(a)}|\cdot d(a_{m,i})}{|\calS_m|}+(96\log T +12)\sqrt{n}+2^m\cdot 96\sqrt{n}\epsilon.
    \end{align*}
    Combining both cases, we know that
    \begin{align*}
        2^m\Delta_a\leq \frac{256n\beta^2}{\Delta_a} +\frac{12\sum_{i=1}^{|\calS_m|}|\lambda_{m,i}^{(a)}|\cdot d(a_{m,i})}{|\calS_m|}+(96\log T +12)\sqrt{n}+2^m\cdot 96\sqrt{n}\epsilon.
    \end{align*}
\end{proof}

Now we are ready to prove our main result \pref{thm:reward-main}.
\begin{proof}[Proof of Theorem~\ref{thm:reward-main}]
Combining \pref{lem:delta_1_reward_miss} and \pref{lem:bound_2_mis_reward} and following the exact same process of obtaining \pref{eqn:reg_b} in \pref{thm:lossLBmis}, we can obtain that for a fixed value of $B$, \pref{alg:rewardLBmis} guarantees that
    \begin{align*}
        \Reg_B &\leq \order\left(\min\left\{\frac{n^2\log(KT)\log(T/n)}{\Delta_{\min}}, n\sqrt{|\calT_B|\log(KT)}\right\}+\epsilon\sqrt{n}|\calT_B|\right)\\
        &\qquad + \sum_{m= 1}^{\lceil\log_2(|\calT_B|/3n\rceil}\sum_{a\in \calS_m}\order\left(\min\left\{\frac{\sum_{i=1}^{|\calS_{m-1}|}|\lambda_{m-1,i}^{(a)}|\cdot d(a_{m-1,i})}{n}, \frac{D\Delta_a}{n}\right\}\right) \\
        &\leq \order\Bigg(\min\left\{\frac{n^2\log(KT)\log(T/n)}{\Delta_{\min}}, n\sqrt{|\calT_B|\log(KT)}\right\}+\epsilon\sqrt{n}|\calT_B|\\
        &\qquad \left.+\min\left\{\sum_{m= 1}^{\lceil\log_2(|\calT_B|/3n\rceil}\sum_{i=1}^{|\calS_{m-1}|} d(a_{m-1,i}), D\Delta_{\max}\log(T/n)\right\}\right).
    \end{align*}
    According to \pref{lem:end-of-B-reward}, there are at most $\lceil\log_2(1/\mu^\star)\rceil$ different values of $B$. With an abuse of notation, we define $\calS_{m}^{(B)}=\{a_{m,i}^{(B)}\}_{i\in [3n]}$ to be the volumetric spanner at epoch $m$ with the reward guess $B$.
    Taking summation over all these values, we can obtain that
    \begin{align*}
        \Reg &\leq \order\left(\min\left\{\frac{n^2\log(KT)\log(T/n)\log(1/\mu^\star)}{\Delta_{\min}}, n\sqrt{T\log(KT)\log(1/\mu^\star)}\right\}+\epsilon\sqrt{n}T\right)\\
        &\qquad+\order\left(\min\left\{\sum_{j=0}^{\lceil\log_2(1/\mu^\star)\rceil}\sum_{m=1}^{\lceil\log_2(|\calT_{2^{-j}}|/3n\rceil}\sum_{i=1}^{3n} d(a_{m-1,i}^{(2^{-j})}), D\Delta_{\max}\log(1/\mu^\star)\log(T/n)\right\}\right),
    \end{align*}
    completing the proof.
\end{proof}    

    While we can further apply a similar analysis to the one in \pref{thm:lossLBmis} to bound the term $\sum_{j=0}^{\lceil\log_2(1/\mu^\star)\rceil}\sum_{m=1}^{\lceil\log_2(|\calT_{2^{-j}}|/3n\rceil}\sum_{i=1}^{3n} d(a_{m-1,i}^{(2^{-j})})$ and obtain a bound with respect to $d^\star$, since $d^\star\geq D\Delta_{\max}+\epsilon$, this $d^\star$ dependent bound does not provide a significantly better regret guarantee in the worst case. This  difference in loss versus reward is also pointed out in \citep{schlisselberg2024delay} in the MAB setting. We keep this term in the upper bound since this quantity can still be potentially smaller than $D\Delta_{\max}\log(1/\mu^\star)\log(T/n)$.




\section{Omitted Details in \pref{sec: contextual}}\label{app: contextual}
In this section, we provide the omitted details in \pref{sec: contextual}.
We start with the following lemma that is a standard application of the Azuma-Hoeffding's inequality.
\begin{lemma}[Proposition 2 in \citep{hanna2023contexts}]\label{lem:prop_two}
    For each epoch $m$, \pref{alg:reduction} guarantees that with probability at least $1-\frac{\delta}{T}$, the following holds:
    \begin{align*}
        \left|\inner{g(\theta),\theta'} - \inner{g^{(m)}(\theta),\theta'}\right| \leq 2\sqrt{\frac{\log(2T|\Theta'|/\delta)}{2^{m-1}}},~~\forall \theta,\theta'\in\Theta'.
    \end{align*}
\end{lemma}

Next, we provide the proof for \pref{thm:reduction}.
\begin{proof}[Proof of Theorem~\ref{thm:reduction}]
    Define $\theta_0 = \argmin_{\theta'\in \Theta'}\|\theta'-\theta\|_2$. 
    Following the analysis of \citet{hanna2023contexts}, we decompose the regret $\Reg_m$ within epoch $m$ as follows:
    \begin{align*}
        \Reg_m &=\E\left[\sum_{\tau=2^{m-1}+1}^{2^m}\left(\inner{\argmin_{a\in\calA_t}\inner{a,\theta_t},\theta} - \min_{a_\tau^\star\in\calA_\tau}\inner{a_{\tau}^\star,\theta}\right)\right] \\
        &\leq \E\left[\sum_{\tau=2^{m-1}+1}^{2^m}\left(\inner{\argmin_{a\in\calA_t}\inner{a,\theta_t},\theta_0} - \min_{a_\tau^\star\in\calA_\tau}\inner{a_{\tau}^\star,\theta_0}\right)\right] + \order\left(\frac{2^{m-1}}{T}\right)\\
        &= \E\left[\sum_{\tau=2^{m-1}+1}^{2^m}\inner{g(\theta_t)-g(\theta_0),\theta_0}\right] + \order\left(\frac{2^{m-1}}{T}\right)\\
        &= \E\underbrace{\left[\sum_{\tau=2^{m-1}+1}^{2^{m}}\inner{g(\theta_t)-g^{(m)}(\theta_t),\theta_0}\right]}_{\Err{1}}  + \E\underbrace{\left[\sum_{\tau=2^{m-1}+1}^{2^m}\inner{g^{(m)}(\theta_t)-g^{(m)}(\theta_0),\theta_0}\right]}_{\regnctx} \\
        &\qquad + \underbrace{\E\left[\sum_{\tau=2^{m-1}+1}^{2^m}\inner{g^{(m)}(\theta_0)-g(\theta_0),\theta_0}\right]}_{\Err{2}} +\order\left(\frac{2^{m-1}}{T}\right),
    \end{align*}
    where the second equality is because
    $\E\left[\min_{a\in\calA_t}\inner{a,\theta_0}\right] = \E\left[\inner{\argmin_{a\in\calA_t}\inner{a,\theta_0},\theta_0}\right] = \inner{g(\theta_0), \theta
    _0}$.
    
    For $\Err{1}$ and $\Err{2}$, we apply \pref{lem:prop_two} to bound both terms  by $\order\left(\sqrt{2^m\log(T|\Theta'|)}\right)$.
    As for $\regnctx$, this is in fact the regret of misspecified non-contextual linear bandits with action set $\calX_m$ and misspecification level $\max_{\theta'\in \Theta'}\left|\inner{g^{(m)}(\theta')-g(\theta'),\theta}\right|$, since $\E[u_t]=\inner{g(\theta_t),\theta}$ for all $t$. Applying \pref{lem:prop_two} again, we know that the misspecification is of order $\epsilon_m=\order(\sqrt{\log(T|\Theta'|)/2^m})$ with probability at least $1-\frac{1}{T^2}$. 
    Then, applying the regret guarantee of \pref{alg:lossLBmis} proven in \pref{thm:lossLBmis}, we know that
    \begin{align*}
        &\regnctx 
        \leq \order\left(\sqrt{n2^m\log(T|\Theta'|)}\right)
        +\order\left(\min\{V_{m,1},V_{m,2}\},\log(\overline{d}^\star)\min\{W_{m,1},W_{m,2}\}\right),
    \end{align*}
    where $V_{m,1}=\frac{n^2\log(T|\Theta'|)\log(T/n)\log(\overline{d}^\star)}{\Delta_{\min}^{\nctx}}$, $V_{m,2}=n\sqrt{2^m\log(\overline{d}^\star)\log(T|\Theta'|)}$, 
    $W_{m,1}=n\overline{d}^\star\log(T/n)+D\Delta_{\max}^{\nctx}$, and $W_{m,2}=D\Delta_{\max}^{\nctx}\log(T/n)$.
    Taking a summation over all $m\in[\lceil\log_2(T)\rceil]$ epochs and using the fact that $|\Theta'|\leq \order(T^n)$ finishes the proof.
\end{proof}
\newpage
\section{Omitted Details in \pref{sec: experiment}}\label{app: experiment}
For completeness, we include the pseudo code for the benchmark used in our experiment, that is, \texttt{LinUCB} using only the observed feedback;
see \pref{alg:linUCB}.

\begin{algorithm}[h]
\caption{LinUCB with Delayed Feedback}\label{alg:linUCB}
Input: action set $\calA$, a parameter $\lambda>0$.

Initialize: $\wh{\theta}_1$ arbitrarily, $\beta_t = \sqrt{\lambda} + \sqrt{2\log T+n\log(1+\frac{t}{n\lambda})}$ for all $t\in[T]$, $H_1 = \lambda I$.

\For{$t=1,2,\dots,T$}{
     Pick 
     \begin{align*}
         a_t=
         \begin{cases}
            \argmin_{a\in\calA} \inner{a,\wh{\theta}_t} - \beta \|a\|_{H_t}^{-1}, &\mbox{in the loss case,} \\
            \argmax_{a\in\calA} \inner{a,\wh{\theta}_t} + \beta \|a\|_{H_t}^{-1}, &\mbox{in the reward case.}
        \end{cases}
     \end{align*}
     
     Observe the payoff $u_\tau$ for all $\tau$ such that $\tau+d_{\tau}\in (t-1,t]$.

     Update $H_{t+1} = H_t + \sum_{\tau:\tau+d_{\tau}\in(t-1,t]}a_{\tau}a_{\tau}^\top$ and $\wh{\theta}_{t+1}=H_{t+1}^{-1}\sum_{\tau:\tau+d_{\tau}\leq t}a_{\tau}u_{\tau}$.
     
}
\end{algorithm}



\end{document}

