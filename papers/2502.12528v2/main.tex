\documentclass{article}

\usepackage{microtype}
\usepackage{graphicx}
\usepackage{subfigure}
\usepackage{booktabs}
\usepackage{hyperref}


\usepackage[accepted]{icml2025}

\usepackage{amsmath}
\usepackage{amssymb}
\usepackage{mathtools}
\usepackage{amsthm}
\usepackage{thmtools}

\usepackage[capitalize,noabbrev]{cleveref}

\theoremstyle{plain}
\newtheorem{theorem}{Theorem}[section]
\newtheorem{proposition}[theorem]{Proposition}
\newtheorem{lemma}[theorem]{Lemma}
\newtheorem{corollary}[theorem]{Corollary}
\theoremstyle{definition}
\newtheorem{definition}[theorem]{Definition}
\newtheorem{assumption}[theorem]{Assumption}
\theoremstyle{remark}
\newtheorem{remark}[theorem]{Remark}


\usepackage[textsize=tiny]{todonotes}



\icmltitlerunning{Contextual Linear Bandits with Delay as Payoff}
%!TEX root=main.tex
\newif\ifspacehack
%\spacehacktrue
\usepackage{natbib}
\hypersetup{
    colorlinks = blue,
    breaklinks,
    linkcolor = blue,
    citecolor = blue,
    urlcolor  = blue,
}
\usepackage{url} 
\usepackage{graphicx}
\usepackage{mathtools}
\usepackage{footnote}
\usepackage{float}
\usepackage{xspace}
\usepackage{multirow}
\usepackage{xcolor}
\usepackage{wrapfig}
\usepackage{framed}
\usepackage{bbm}
\usepackage[most]{tcolorbox}

\usepackage{footnote}
\usepackage{nicefrac}
\usepackage{makecell}
\usepackage[ruled,vlined]{algorithm2e}
\usepackage{amssymb}
\usepackage{bm}
\makesavenoteenv{tabular}
\makesavenoteenv{table}

\newcommand{\hl}[1]{{\color{red}[HL: #1]}}
\newcommand{\mznote}[1]{{\color{blue}[MZ: #1]}}

% macros@Peng
\newcommand\innerp[2]{\langle #1, #2 \rangle}
\renewcommand{\tilde}{\widetilde}
\renewcommand{\hat}{\widehat}


\newcommand{\TVD}[1]{\norm{#1}_\text{TV}}
\newcommand{\corral}{\textsc{Corral}\xspace}
\newcommand{\expthree}{\textsc{Exp3}\xspace}
\newcommand{\expfour}{\ensuremath{\mathsf{Exp4}}\xspace}
\newcommand{\expthreeP}{\textsc{Exp3.P}\xspace}
\newcommand{\scrible}{\textsc{SCRiBLe}\xspace}

\def \R {\mathbb{R}}
\newcommand{\eps}{\epsilon}
\newcommand{\vecc}{\mathrm{vec}}
\newcommand{\LS}{\mathrm{LS}}
\newcommand{\FG}{\mathrm{FG}}
\newcommand{\DL}{\Delta \ellhat}
\newcommand{\calA}{{\mathcal{A}}}
\newcommand{\smax}{{\mathrm{smax}}}
\newcommand{\calB}{{\mathcal{B}}}
\newcommand{\calX}{{\mathcal{X}}}
\newcommand{\calS}{{\mathcal{S}}}
\newcommand{\calF}{{\mathcal{F}}}
\newcommand{\calI}{{\mathcal{I}}}
\newcommand{\calJ}{{\mathcal{J}}}
\newcommand{\calK}{{\mathcal{K}}}
\newcommand{\calH}{{\mathcal{H}}}
\newcommand{\calD}{{\mathcal{D}}}
\newcommand{\calE}{{\mathcal{E}}}
\newcommand{\calG}{{\mathcal{G}}}
\newcommand{\calU}{{\mathcal{U}}}
\newcommand{\calR}{{\mathcal{R}}}
\newcommand{\calT}{{\mathcal{T}}}
\newcommand{\calP}{{\mathcal{P}}}
\newcommand{\calQ}{{\mathcal{Q}}}
\newcommand{\calZ}{{\mathcal{Z}}}
\newcommand{\calM}{{\mathcal{M}}}
\newcommand{\calN}{{\mathcal{N}}}
\newcommand{\calW}{{\mathcal{W}}}
\newcommand{\calY}{{\mathcal{Y}}}
\newcommand{\cD}{{\mathcal{D}_{\mathcal{X}}}}
\newcommand{\mcD}{{\mathcal{D}}}
\newcommand{\cF}{{\mathcal{F}}}
\newcommand{\cA}{{\mathcal{A}}}
\newcommand{\cX}{{\mathcal{X}}}
\newcommand{\cE}{{\mathcal{E}}}
\newcommand{\cV}{{\mathcal{V}}}
\newcommand{\cR}{{\mathcal{R}}}
\newcommand{\wcR}{\widehat{\mathcal{R}}}
\newcommand{\Reg}{{\mathrm{Reg}}}
\newcommand{\Alg}{{\mathsf{Alg}}}
\newcommand{\wReg}{\widehat{\mathrm{Reg}}}
\newcommand{\cB}{\mathcal{B}}
\newcommand{\cP}{\mathcal{P}}
\newcommand{\nctx}{\text{n-ctx}}
\newcommand{\ctx}{\text{ctx}}
\newcommand{\E}{{\mathbb{E}}}
\newcommand{\V}{\mathbb{V}}
\newcommand{\Prob}{\mathbb{P}}
\newcommand{\1}{\mathbb{I}}
\newcommand{\N}{\mathbb{N}}
\newcommand{\tup}[1]{t^{(#1)}}
\newcommand{\gup}[1]{g^{(#1)}}
\newcommand{\hatfm}{\widehat{f}_m}
\newcommand{\haty}{\widehat{y}}
\newcommand{\hatx}{\widehat{x}}
\newcommand{\yhat}{\widehat{y}}
\newcommand{\xhat}{\widehat{x}}
\newcommand{\fhat}{\widehat{f}}
\newcommand{\ghat}{\widehat{g}}

\newcommand{\inner}[1]{ \left\langle {#1} \right\rangle }
\newcommand{\ind}{\mathbb{I}}
\newcommand{\diag}{\textrm{diag}}
\newcommand{\Nout}{N_{\textrm{out}}}
\newcommand{\nout}{N_{\textrm{out}}}
\newcommand{\Nin}{{\textrm{Nin}}}
\newcommand{\nin}{{\textrm{Nin}}}
\newcommand{\order}{\mathcal{O}}


\newcommand{\Acal}{\mathcal{A}}
\newcommand{\Bcal}{\mathcal{B}}
\newcommand{\Ccal}{\mathcal{C}}
\newcommand{\Dcal}{\mathcal{D}}
\newcommand{\Ecal}{\mathcal{E}}
\newcommand{\Fcal}{\mathcal{F}}
\newcommand{\Gcal}{\mathcal{G}}
\newcommand{\Hcal}{\mathcal{H}}
\newcommand{\Ical}{\mathcal{I}}
\newcommand{\Jcal}{\mathcal{J}}
\newcommand{\Kcal}{\mathcal{K}}
\newcommand{\Lcal}{\mathcal{L}}
\newcommand{\Mcal}{\mathcal{M}}
\newcommand{\Ncal}{\mathcal{N}}
\newcommand{\Ocal}{\mathcal{O}}
\newcommand{\Pcal}{\mathcal{P}}
\newcommand{\Qcal}{\mathcal{Q}}
\newcommand{\Rcal}{\mathcal{R}}
\newcommand{\Scal}{\mathcal{S}}
\newcommand{\Tcal}{\mathcal{T}}
\newcommand{\Ucal}{\mathcal{U}}
\newcommand{\Vcal}{\mathcal{V}}
\newcommand{\Wcal}{\mathcal{W}}
\newcommand{\Xcal}{\mathcal{X}}
\newcommand{\Ycal}{\mathcal{Y}}
\newcommand{\Zcal}{\mathcal{Z}}
\newcommand{\wkdn}{d}


\newcommand{\avgR}{\wh{\cal{R}}}
%\newcommand{\ips}{\wh{r}}
\newcommand{\whpi}{\wh{\pi}}
\newcommand{\whE}{\wh{\E}}
\newcommand{\whV}{\wh{V}}

\newcommand{\whReg}{\wh{\text{\rm Reg}}}
\newcommand{\flg}{\text{\rm flag}}
\newcommand{\one}{\boldsymbol{1}}
\newcommand{\var}{\Delta}
\newcommand{\Var}{\mathrm{Var}}
\newcommand{\bvar}{\bar{\Delta}}
\newcommand{\p}{\prime}
\newcommand{\evt}{\textsc{Event}}
\newcommand{\unif}{\text{\rm Unif}}
\newcommand{\KL}{\text{\rm KL}}
\newcommand{\Lstar}{{L^\star}}
\newcommand{\istar}{{i^\star}}
\newcommand{\dynreg}{\text{Dyn-Reg}}
\newcommand{\tildedynreg}{\widetilde{\text{Dyn-Reg}}}
\newcommand{\Bstar}{{B^\star}}
\newcommand{\Ustar}{\rho}
\newcommand{\Aconst}{a}
\newcommand{\dplus}[1]{\bm{#1}}
\newcommand{\lambdamax}{\lambda_\text{\rm max}}
\newcommand{\biasone}{\textsc{Deviation}\xspace}
\newcommand{\bias}{\textsc{Bias-1}\xspace}
\newcommand{\biastwo}{\textsc{Bias-2}\xspace}
\newcommand{\biasthree}{\textsc{Bias-3}\xspace}
\newcommand{\term}[1]{\texttt{Term} ~(#1)\xspace}
\newcommand{\x}{\mathbf{x}}
\newcommand{\errorterm}{\textsc{Error}\xspace}
\newcommand{\Err}[1]{\textsc{Err-Term}(#1)\xspace}
\newcommand{\regnctx}{\textsc{Reg-NCTX}\xspace}
\newcommand{\regterm}{\textsc{Reg-Term}\xspace}
\newcommand{\LTtilde}{\wt{L}_T}
\newcommand{\Bomega}{B_{\Omega}}
\newcommand{\UOB}{UOB-REPS\xspace}
\newcommand{\Holder}{{H{\"o}lder}\xspace}
\newcommand{\dpg}{\dplus{g}}
\newcommand{\dpM}{\dplus{M}}
\newcommand{\dpf}{\dplus{f}}
\newcommand{\dpX}{\dplus{\calX}}
\newcommand{\dpw}{\dplus{w}}
\newcommand{\dpF}{\dplus{F}}
\newcommand{\dpu}{\dplus{u}}
\newcommand{\dpwtilde}{\dplus{\wtilde}}
\newcommand{\dps}{\dplus{s}}
\newcommand{\dpe}{\dplus{e}}
\newcommand{\dpx}{\dplus{x}}
\newcommand{\dpy}{\dplus{y}}
\newcommand{\dpH}{\dplus{H}}
\newcommand{\dpOmega}{\dplus{\Omega}}
\newcommand{\dpellhat}{\dplus{\ellhat}}
\newcommand{\dpell}{\dplus{\ell}}
\newcommand{\dpr}{\dplus{r}}
\newcommand{\dpxi}{\dplus{\xi}}
\newcommand{\dpv}{\dplus{v}}
\newcommand{\dpI}{\dplus{I}}
\newcommand{\dpA}{\dplus{A}}
\newcommand{\dph}{\dplus{h}}
\newcommand{\cprob}{6}
\newcommand{\sigmoid}{\ensuremath{\mathsf{Sigmoid}}\xspace}
\newcommand{\relu}{\ensuremath{\mathsf{ReLU}}\xspace}

\DeclareMathOperator*{\argmin}{argmin}
\DeclareMathOperator*{\argmax}{argmax}
\DeclareMathOperator*{\argsmax}{argsmax}
%\DeclareMathOperator*{\range}{range}
%\DeclareMathOperator*{\mydet}{det_{+}}
%\DeclarePairedDelimiter\abs{\lvert}{\rvert}
%\DeclarePairedDelimiter\bigabs{\big\lvert}{\big\rvert}
\DeclarePairedDelimiter\ceil{\lceil}{\rceil}
%\DeclarePairedDelimiter\floor{\lfloor}{\rfloor}
%\DeclarePairedDelimiter\bigceil{\big\lceil}{\big\rceil}
%\DeclarePairedDelimiter\bigfloor{\big\lfloor}{\big\rfloor}

\newcommand{\field}[1]{\mathbb{#1}}
\newcommand{\fY}{\field{Y}}
\newcommand{\fX}{\field{X}}
\newcommand{\fH}{\field{H}}
\newcommand{\fR}{\field{R}}
\newcommand{\fN}{\field{N}}
\newcommand{\fS}{\field{S}}
\newcommand{\UCB}{{\operatorname{UCB}}}
\newcommand{\LCB}{{\operatorname{LCB}}}
\newcommand{\testblock}{\textsc{EndofBlockTest}\xspace}
\newcommand{\testreplay}{\textsc{EndofReplayTest}\xspace}

\newcommand{\theset}[2]{ \left\{ {#1} \,:\, {#2} \right\} }
% \newcommand{\inner}[1]{ \langle {#1} \rangle }
\newcommand{\inn}[1]{ \langle {#1} \rangle }
\newcommand{\Ind}[1]{ \field{I}_{\{{#1}\}} }
\newcommand{\eye}[1]{ \boldsymbol{I}_{#1} }
\newcommand{\norm}[1]{\left\|{#1}\right\|}
%\newcommand{\trace}[1]{\text{tr}\left({#1}\right)}
\newcommand{\trace}[1]{\textsc{tr}({#1})}


\newcommand{\defeq}{\stackrel{\rm def}{=}}
\newcommand{\sgn}{\mbox{\sc sgn}}
\newcommand{\scI}{\mathcal{I}}
\newcommand{\scO}{\mathcal{O}}
\newcommand{\scN}{\mathcal{N}}
\newcommand{\msmwc}{\textsc{MsMwC}}

\newcommand{\dt}{\displaystyle}
\renewcommand{\ss}{\subseteq}
\newcommand{\wh}{\widehat}
\newcommand{\wt}{\widetilde}
\newcommand{\wb}{\overline}
\newcommand{\ve}{\varepsilon}
\newcommand{\hlambda}{\wh{\lambda}}

\newcommand{\Jd}{J}
\newcommand{\ellhat}{\wh{\ell}}
\newcommand{\rhat}{\wh{r}}
\newcommand{\elltilde}{\wt{\ell}}
\newcommand{\wtilde}{\wt{w}}
\newcommand{\what}{\wh{w}}

\DeclareMathOperator{\conv}{conv}
\newcommand{\ellprime}{\ellhat^\prime}

\newcommand{\upconf}{\phi}

%\newcommand{\Ltilde}{\wt{L}}

\newcommand{\hDelta}{\wh{\Delta}}
\newcommand{\hdelta}{\wh{\delta}}
\newcommand{\spin}{\{-1,+1\}}

\newcommand{\ep}[1]{\E\!\left[#1\right]}
\newcommand{\LT}{L_T}
\newcommand{\LTbar}{\overline{L}_T}
\newcommand{\LTbarep}{\mathring{L}_T}
\newcommand{\circxhat}{\mathring{\xh}}
\newcommand{\circx}{\mathring{x}}
\newcommand{\circu}{\mathring{u}}
\newcommand{\circcalX}{\mathring{\calX}}
\newcommand{\circg}{\mathring{g}}
\newcommand{\Lubar}{\overline{L}_{u}}
%\newcommand{\Lustarbar}{\overline{L}_{u^\star}}

\newcommand{\Lyr}{J}
\newcommand{\QQ}{{w}}
\newcommand{\Qt}{{\QQ_t}}
\newcommand{\Qstar}{{u}}
\newcommand{\Qpistar}{{\Qstar^{\star}}}
\newcommand{\Qhat}{\wh{\QQ}}
\newcommand{\Ut}{{\upconf_t}}
\newcommand{\intO}{\mathrm{int}(\Omega)}
\newcommand{\intK}{\mathrm{int}(K)}

\newcommand{\squareCB}{\ensuremath{\mathsf{SquareCB}}\xspace}
\newcommand{\feelgood}{\ensuremath{\mathsf{FGTS}}\xspace}
\newcommand{\graphCB}{\ensuremath{\mathsf{SquareCB.G}}\xspace}
\newcommand{\squareCBAuc}{\ensuremath{\mathsf{SquareCB.A}}\xspace}
\newcommand{\AlgSq}{\ensuremath{\mathsf{AlgSq}}\xspace}
\newcommand{\AlgLog}{\ensuremath{\mathsf{AlgLog}}\xspace}
\newcommand{\ips}{\ensuremath{\mathsf{(IPS)}}\xspace}
\newcommand{\optsq}{\ensuremath{\mathsf{(OptSq)}}\xspace}
\newcommand{\sq}{\ensuremath{\mathsf{(Sq)}}\xspace}
\newcommand{\dec}{\ensuremath{\mathsf{dec}_\gamma}\xspace}
\newcommand{\dectwo}{\ensuremath{\mathsf{dec}_{\gamma_1,\gamma_2}}\xspace}
%\newcommand{\theHalgorithm}{\arabic{algorithm}}
\newtheorem{cor}[theorem]{Corollary}
\newcommand{\context}{\text{ctx}}
\newcommand{\noncontext}{\text{n-ctx}}
%\newtheorem{remark}{Remark}
%\newtheorem{prop}{Proposition}
%\newtheorem{definition}{Definition}
%\newtheorem{assumption}{Assumption}
\newtheorem{event}{Event}
%\newtheorem*{main}{Main Result}
%\newtheorem{fact}[theorem]{Fact}

\newcommand{\paren}[1]{\left({#1}\right)}
\newcommand{\brackets}[1]{\left[{#1}\right]}
\newcommand{\braces}[1]{\left\{{#1}\right\}}

\newcommand{\normt}[1]{\norm{#1}_{t}}
\newcommand{\dualnormt}[1]{\norm{#1}_{t,*}}

\newcommand{\otil}{\ensuremath{\tilde{\mathcal{O}}}}

\newcommand{\dist}{\calP}

%%%%  brackets
\newcommand{\rbr}[1]{\left(#1\right)}
\newcommand{\sbr}[1]{\left[#1\right]}
\newcommand{\cbr}[1]{\left\{#1\right\}}
\newcommand{\nbr}[1]{\left\|#1\right\|}
\newcommand{\abr}[1]{\left|#1\right|}

\usepackage{lipsum,booktabs}
\usepackage{amsmath,mathrsfs,amssymb,amsfonts,bm,enumitem}
\usepackage{rotating}
\usepackage{pdflscape}
\usepackage{hyperref,url}
\hypersetup{
    colorlinks,
    breaklinks,
    linkcolor = blue,
    citecolor = blue,
    urlcolor  = blue,
}
\allowdisplaybreaks
\usepackage{appendix}
\usepackage{multirow,makecell}

%\usepackage{algorithmic,algorithm}
%\renewcommand{\algorithmicrequire}{ \textbf{Input:}}
%\renewcommand{\algorithmicensure}{ \textbf{Output:}}

\renewcommand{\tilde}{\widetilde}
\renewcommand{\hat}{\widehat}
\newcommand{\obs}{O}
\newcommand{\unobs}{E}
\newcommand{\unbiasSize}{c}
\newcommand{\unbias}{C}
\newcommand{\cnt}{k}

% define some macros
\def \A {\mathcal{A}}

\def \B {\mathbb{B}}
\def \B {\mathcal{B}}
\def \C {\mathcal{C}}
\def \D {\mathcal{D}}
\def \E {\mathbb{E}}
\def \F {\mathcal{F}}
\def \G {\mathcal{G}}
\def \H {\mathcal{H}}
\def \I {\mathcal{I}}
\def \J {\mathcal{J}}
\def \K {\mathcal{K}}
\def \L {\mathcal{L}}
\def \M {\mathcal{M}}
\def \N {\mathcal{N}}
\def \O {\mathcal{O}}
\def \P {\mathcal{P}}
\def \Q {\mathcal{Q}}
\def \R {\mathbb{R}}
\def \S {\mathcal{S}}
% \def \T {\mathrm{T}}
\def \T {\top}
\def \U {\mathcal{U}}
\def \V {\mathcal{V}}
\def \W {\mathcal{W}}
\def \X {\mathcal{X}}
\def \Y {\mathcal{Y}}
\def \Z {\mathcal{Z}}

\def \a {\mathbf{a}}
\def \b {\mathbf{b}}
\def \c {\mathbf{c}}
\def \d {\mathbf{d}}
\def \e {\mathbf{e}}
\def \f {\mathbf{f}}
\def \g {\mathbf{g}}
\def \h {\mathbf{h}}
\def \m {\mathbf{m}}
\def \p {\mathbf{p}}
\def \q {\mathbf{q}}
\def \u {\mathbf{u}}
\def \w {\mathbf{w}}
\def \s {\mathbf{s}}
\def \t {\mathbf{t}}
\def \v {\mathbf{v}}
\def \x {\mathbf{x}}
\def \y {y}
\def \z {\mathbf{z}}

\def \ph {\hat{p}}

\def \fh {\hat{f}}
\def \fb {\bar{f}}
\def \ft{\tilde{f}}

\def \gh {\hat{\g}}
\def \gb {\bar{\g}}
\def \gt {\tilde{g}}

\def \uh {\hat{\u}}
\def \ub {\bar{\u}}
\def \ut{\tilde{\u}}

\def \vh {\hat{\v}}
\def \vb {\bar{\v}}
\def \vt{\tilde{\v}}

\def \xh {\hat{x}}
\def \xb {\bar{\x}}
\def \xt {\tilde{\x}}

\def \zh {\hat{\z}}
\def \zb {\bar{\z}}
\def \zt {\tilde{\z}}

\def \Ecal {\mathcal{E}}
\def \Rcal {\mathcal{R}}
\def \Ot {\tilde{\O}}
\def \indicator {\mathds{1}}
\def \regret {\mbox{Regret}}
\def \proj {\mbox{Proj}}
\def \Pr {\mathsf{Pr}}
\def \ellb {\boldsymbol{\ell}}
\def \thetah {\hat{\theta}}

\newcommand{\RegSq}{\ensuremath{\mathrm{\mathbf{Reg}}_{\mathsf{Sq}}}\xspace}
\newcommand{\RegCB}{\ensuremath{\mathrm{\mathbf{Reg}}_{\mathsf{CB}}}\xspace}
\newcommand{\RegDyn}{\ensuremath{\mathrm{\mathbf{Reg}}_{\mathsf{Dyn}}}\xspace}
\usepackage{mathtools}
\let\oldnorm\norm   % <-- Store original \norm as \oldnorm
\let\norm\undefined % <-- "Undefine" \norm
\DeclarePairedDelimiter\norm{\lVert}{\rVert}
\DeclarePairedDelimiter\abs{\lvert}{\rvert}
%\newcommand\inner[2]{\langle #1, #2 \rangle}
\newcommand*\diff{\mathop{}\!\mathrm{d}}
\newcommand*\Diff[1]{\mathop{}\!\mathrm{d^#1}}

%\DeclareMathOperator*{\Reg}{Regret}
\DeclareMathOperator*{\AReg}{A-Regret}
\DeclareMathOperator*{\WAReg}{WA-Regret}
\DeclareMathOperator*{\SAReg}{SA-Regret}
\DeclareMathOperator*{\DReg}{\mbox{D-Regret}}
\DeclareMathOperator*{\poly}{poly}
%\DeclareMathOperator*{\argmax}{arg\,max}
%\DeclareMathOperator*{\argmin}{arg\,min}

% define new theorem environments
% \let\proof\relax
% \let\endproof\relax
% \newenvironment{proof}{\par\noindent{\bf Proof\ }}{\hfill\BlackBox\\[2mm]}
% \renewcommand\qedsymbol{$\blacksquare$}
\newtheorem{myThm}{Theorem}
\newtheorem{myFact}{Fact}
\newtheorem{myClaim}{Claim}
\newtheorem{myLemma}[myThm]{Lemma}
\newtheorem{myObservation}{Observation}
\newtheorem{myProp}[myThm]{Proposition}
\newtheorem{myProperty}{Property}

% Define a custom environment for prompts
\newtcolorbox{promptbox}[1][]{
  colback=blue!5!white, colframe=blue!75!black,
  fonttitle=\bfseries, title=Prompt,
  left=2mm, right=2mm, top=2mm, bottom=2mm,
  boxrule=0.5mm,  % Thickness of the frame
  coltitle=black, % Color of the title text
  colbacktitle=blue!15!white, % Background color of the title
  breakable,      % Allows the box to break across pages
  #1
}
\newtheorem{myAssum}{Assumption}
\newtheorem{myConj}{Conjecture}
\newtheorem{myCor}{Corollary}
\newtheorem{myDef}{Definition}
\newtheorem{myExample}{Example}
\newtheorem{myNote}{Note}
\newtheorem{myProblem}{Problem}

\newtheorem{myRemark}{Remark}

% add comments
\usepackage{graphicx,color} % more modern
\newcommand{\red}{\color{red}}
\newcommand{\blue}{\color{blue}}
\definecolor{wine_red}{RGB}{228,48,64}
\definecolor{DSgray}{cmyk}{0,1,0,0}
%\newcommand{\Authornote}[2]{{\small\textcolor{NavyBlue}{\sf$<<<${  #1: #2 }$>>>$}}}
% \newcommand{\Authormarginnote}[2]{\marginpar{\parbox{2cm}{\raggedright\tiny \textcolor{DSgray}{#1: #2}}}}
% \newcommand{\pnote}[1]{{\Authornote{Peng}{#1}}}
% \newcommand{\pmarginnote}[1]{{\Authormarginnote{Peng}{#1}}}

\usepackage{prettyref}
\newcommand{\pref}[1]{\prettyref{#1}}
\newcommand{\pfref}[1]{Proof of \prettyref{#1}}
\newcommand{\savehyperref}[2]{\texorpdfstring{\hyperref[#1]{#2}}{#2}}
\newrefformat{eq}{\savehyperref{#1}{Eq. \textup{(\ref*{#1})}}}
\newrefformat{eqn}{\savehyperref{#1}{Eq.~(\ref*{#1})}}
\newrefformat{lem}{\savehyperref{#1}{Lemma~\ref*{#1}}}
\newrefformat{event}{\savehyperref{#1}{Event~\ref*{#1}}}
\newrefformat{def}{\savehyperref{#1}{Definition~\ref*{#1}}}
\newrefformat{line}{\savehyperref{#1}{Line~\ref*{#1}}}
\newrefformat{thm}{\savehyperref{#1}{Theorem~\ref*{#1}}}
\newrefformat{tab}{\savehyperref{#1}{Table~\ref*{#1}}}
\newrefformat{corr}{\savehyperref{#1}{Corollary~\ref*{#1}}}
\newrefformat{cor}{\savehyperref{#1}{Corollary~\ref*{#1}}}
\newrefformat{sec}{\savehyperref{#1}{Section~\ref*{#1}}}
\newrefformat{app}{\savehyperref{#1}{Appendix~\ref*{#1}}}
\newrefformat{assum}{\savehyperref{#1}{Assumption~\ref*{#1}}}
\newrefformat{asm}{\savehyperref{#1}{Assumption~\ref*{#1}}}
\newrefformat{ex}{\savehyperref{#1}{Example~\ref*{#1}}}
\newrefformat{fig}{\savehyperref{#1}{Figure~\ref*{#1}}}
\newrefformat{alg}{\savehyperref{#1}{Algorithm~\ref*{#1}}}
\newrefformat{rem}{\savehyperref{#1}{Remark~\ref*{#1}}}
\newrefformat{conj}{\savehyperref{#1}{Conjecture~\ref*{#1}}}
\newrefformat{prop}{\savehyperref{#1}{Proposition~\ref*{#1}}}
\newrefformat{proto}{\savehyperref{#1}{Protocol~\ref*{#1}}}
\newrefformat{prob}{\savehyperref{#1}{Problem~\ref*{#1}}}
\newrefformat{claim}{\savehyperref{#1}{Claim~\ref*{#1}}}
\newrefformat{que}{\savehyperref{#1}{Question~\ref*{#1}}}
\newrefformat{op}{\savehyperref{#1}{Open Problem~\ref*{#1}}}
\newrefformat{fn}{\savehyperref{#1}{Footnote~\ref*{#1}}}

\def \p {\boldsymbol{p}}
\def \s {\boldsymbol{s}}
\def \m {\boldsymbol{m}}
\def \epsilon {\varepsilon}

% \def \base {\mathtt{base}\mbox{-}\mathtt{regret}}
% \def \meta {\mathtt{meta}\mbox{-}\mathtt{regret}}
\def \base {\textsc{base-regret}}
\def \meta {\textsc{meta-regret}}
\def \xref {\x_{\text{ref}}}
\def \fb {\bar{f}}
\def \interior {\text{int}}
\def \yh {\hat{\y}}
\def \RegLog {\Reg_{\log}^G}
\newcommand{\bra}[1]{\left[#1\right]}
\newcommand{\pa}[1]{\left(#1\right)}
\newcommand{\hhat}{\wh{h}}
\newcommand{\epsn}{\epsilon_N}
\newcommand{\rad}{\mathsf{rad}}
\newcommand{\hatr}{\wh{r}}
\newcommand{\fl}{\underline{f}^\star}
\begin{document}

\twocolumn[
\icmltitle{Contextual Linear Bandits with Delay as Payoff}
\icmlsetsymbol{equal}{*}

\begin{icmlauthorlist}
\icmlauthor{Mengxiao Zhang}{a}
\icmlauthor{Yingfei Wang}{b}
\icmlauthor{Haipeng Luo}{c}
\end{icmlauthorlist}

\icmlaffiliation{a}{University of Iowa}
\icmlaffiliation{b}{University of Washington}
\icmlaffiliation{c}{University of Southern California}
%\icmlaffiliation{comp}{Company Name, Location, Country}
%\icmlaffiliation{sch}{School of ZZZ, Institute of WWW, Location, Country}

%\icmlcorrespondingauthor{Firstname2 Lastname2}{first2.last2@www.uk}

% You may provide any keywords that you
% find helpful for describing your paper; these are used to populate
% the "keywords" metadata in the PDF but will not be shown in the document
\icmlkeywords{Machine Learning, ICML}

\vskip 0.3in
]
\printAffiliationsAndNotice{}


 

\begin{abstract}
A recent work by \citet{schlisselberg2024delay} studies a delay-as-payoff model for stochastic multi-armed bandits, where the payoff (either loss or reward) is delayed for a period that is proportional to the payoff itself.
While this captures many real-world applications, the simple multi-armed bandit setting limits the practicality of their results.
In this paper, we address this limitation by studying the delay-as-payoff model for contextual linear bandits.
Specifically, we start from the case with a fixed action set and propose an efficient algorithm whose regret overhead compared to the standard no-delay case is at most
$D\Delta_{\max}\log T$, where $T$ is the total horizon, $D$ is the maximum delay, and $\Delta_{\max}$ is the maximum suboptimality gap. 
When payoff is loss, we also show further improvement of the bound, demonstrating a separation between reward and loss similar to \citet{schlisselberg2024delay}.
Contrary to standard linear bandit algorithms that construct least squares estimator and confidence ellipsoid, the main novelty of our algorithm is to apply a phased arm elimination procedure by only picking actions in a \emph{volumetric spanner} of the action set, which addresses challenges arising from both payoff-dependent delays and large action sets.
We further extend our results to the case with varying action sets by adopting the reduction from~\citet{hanna2023contexts}.  
Finally, we implement our algorithm and showcase its effectiveness and superior performance in experiments.
\end{abstract}




\section{Introduction}


\begin{figure}[t]
\centering
\includegraphics[width=0.6\columnwidth]{figures/evaluation_desiderata_V5.pdf}
\vspace{-0.5cm}
\caption{\systemName is a platform for conducting realistic evaluations of code LLMs, collecting human preferences of coding models with real users, real tasks, and in realistic environments, aimed at addressing the limitations of existing evaluations.
}
\label{fig:motivation}
\end{figure}

\begin{figure*}[t]
\centering
\includegraphics[width=\textwidth]{figures/system_design_v2.png}
\caption{We introduce \systemName, a VSCode extension to collect human preferences of code directly in a developer's IDE. \systemName enables developers to use code completions from various models. The system comprises a) the interface in the user's IDE which presents paired completions to users (left), b) a sampling strategy that picks model pairs to reduce latency (right, top), and c) a prompting scheme that allows diverse LLMs to perform code completions with high fidelity.
Users can select between the top completion (green box) using \texttt{tab} or the bottom completion (blue box) using \texttt{shift+tab}.}
\label{fig:overview}
\end{figure*}

As model capabilities improve, large language models (LLMs) are increasingly integrated into user environments and workflows.
For example, software developers code with AI in integrated developer environments (IDEs)~\citep{peng2023impact}, doctors rely on notes generated through ambient listening~\citep{oberst2024science}, and lawyers consider case evidence identified by electronic discovery systems~\citep{yang2024beyond}.
Increasing deployment of models in productivity tools demands evaluation that more closely reflects real-world circumstances~\citep{hutchinson2022evaluation, saxon2024benchmarks, kapoor2024ai}.
While newer benchmarks and live platforms incorporate human feedback to capture real-world usage, they almost exclusively focus on evaluating LLMs in chat conversations~\citep{zheng2023judging,dubois2023alpacafarm,chiang2024chatbot, kirk2024the}.
Model evaluation must move beyond chat-based interactions and into specialized user environments.



 

In this work, we focus on evaluating LLM-based coding assistants. 
Despite the popularity of these tools---millions of developers use Github Copilot~\citep{Copilot}---existing
evaluations of the coding capabilities of new models exhibit multiple limitations (Figure~\ref{fig:motivation}, bottom).
Traditional ML benchmarks evaluate LLM capabilities by measuring how well a model can complete static, interview-style coding tasks~\citep{chen2021evaluating,austin2021program,jain2024livecodebench, white2024livebench} and lack \emph{real users}. 
User studies recruit real users to evaluate the effectiveness of LLMs as coding assistants, but are often limited to simple programming tasks as opposed to \emph{real tasks}~\citep{vaithilingam2022expectation,ross2023programmer, mozannar2024realhumaneval}.
Recent efforts to collect human feedback such as Chatbot Arena~\citep{chiang2024chatbot} are still removed from a \emph{realistic environment}, resulting in users and data that deviate from typical software development processes.
We introduce \systemName to address these limitations (Figure~\ref{fig:motivation}, top), and we describe our three main contributions below.


\textbf{We deploy \systemName in-the-wild to collect human preferences on code.} 
\systemName is a Visual Studio Code extension, collecting preferences directly in a developer's IDE within their actual workflow (Figure~\ref{fig:overview}).
\systemName provides developers with code completions, akin to the type of support provided by Github Copilot~\citep{Copilot}. 
Over the past 3 months, \systemName has served over~\completions suggestions from 10 state-of-the-art LLMs, 
gathering \sampleCount~votes from \userCount~users.
To collect user preferences,
\systemName presents a novel interface that shows users paired code completions from two different LLMs, which are determined based on a sampling strategy that aims to 
mitigate latency while preserving coverage across model comparisons.
Additionally, we devise a prompting scheme that allows a diverse set of models to perform code completions with high fidelity.
See Section~\ref{sec:system} and Section~\ref{sec:deployment} for details about system design and deployment respectively.



\textbf{We construct a leaderboard of user preferences and find notable differences from existing static benchmarks and human preference leaderboards.}
In general, we observe that smaller models seem to overperform in static benchmarks compared to our leaderboard, while performance among larger models is mixed (Section~\ref{sec:leaderboard_calculation}).
We attribute these differences to the fact that \systemName is exposed to users and tasks that differ drastically from code evaluations in the past. 
Our data spans 103 programming languages and 24 natural languages as well as a variety of real-world applications and code structures, while static benchmarks tend to focus on a specific programming and natural language and task (e.g. coding competition problems).
Additionally, while all of \systemName interactions contain code contexts and the majority involve infilling tasks, a much smaller fraction of Chatbot Arena's coding tasks contain code context, with infilling tasks appearing even more rarely. 
We analyze our data in depth in Section~\ref{subsec:comparison}.



\textbf{We derive new insights into user preferences of code by analyzing \systemName's diverse and distinct data distribution.}
We compare user preferences across different stratifications of input data (e.g., common versus rare languages) and observe which affect observed preferences most (Section~\ref{sec:analysis}).
For example, while user preferences stay relatively consistent across various programming languages, they differ drastically between different task categories (e.g. frontend/backend versus algorithm design).
We also observe variations in user preference due to different features related to code structure 
(e.g., context length and completion patterns).
We open-source \systemName and release a curated subset of code contexts.
Altogether, our results highlight the necessity of model evaluation in realistic and domain-specific settings.





% Consider a lasso optimization procedure with potentially distinct regularization penalties:
% \begin{align}
%     \hat{\beta} = \arg\min_{\beta}\{\|y-X\beta\|^2_2+\sum_{i=1}^{N}\lambda_i|\beta_i|\}.
% \end{align}
\subsection{Supervised Data-Driven Learning}\label{subsec:supervised}
We consider a generic data-driven supervised learning procedure. Given a dataset \( \mathcal{D} \) consisting of \( n \) data points \( (x_i, y_i) \in \mathcal{X} \times \mathcal{Y} \) drawn from an underlying distribution \( p(\cdot|\theta) \), our goal is to estimate parameters \( \theta \in \Theta \) through a learning procedure, defined as \( f: (\mathcal{X} \times \mathcal{Y})^n \rightarrow \Theta \) 
that minimizes the predictive error on observed data. 
Specifically, the learning objective is defined as follows:
\begin{align}
\hat{\theta}_f := f(\mathcal{D}) = \arg\min_{\theta} \mathcal{L}(\theta, \mathcal{D}),
\end{align}
where \( \mathcal{L}(\cdot,\mathcal{D}) := \sum_{i=1}^{n} \mathcal{L}(\cdot, (x_i, y_i))\), and $\mathcal{L}$ is a loss function quantifying the error between predictions and true outcomes. 
Here, $\hat{\theta}_f$ is the parameter that best explains the observed data pairs \( (x_i, y_i) \) according to the chosen loss function \( \mathcal{L} (\cdot) \).

\paragraph{Feature Selection.}
Feature selection aims to improve model \( f \)'s predictive performance while minimizing redundancy. 
%Formally, given data \( X \), response \( y \), feature set \( \mathcal{F} \), loss function \( \mathcal{L}(\cdot) \), and a feature limit \( k \), the objective is:
% \begin{align}
% \mathcal{S}^* = \arg \min_{\mathcal{S} \subseteq \mathcal{F}, |\mathcal{S}| \leq k} \mathcal{L}(y, f(X_\mathcal{S})) + \lambda R(\mathcal{S}),
% \end{align}
% where \( X_\mathcal{S} \) is the submatrix of \( X \) for selected features \( \mathcal{S} \), \( \lambda \) is a regularization parameter, and \( R(\mathcal{S}) \) penalizes feature redundancy.
 State-of-the-art techniques fall into four categories: (i) filter methods, which rank features based on statistical properties like Fisher score \citep{duda2001pattern,song2012feature}; (ii) wrapper methods, which evaluate model performance on different feature subsets \citep{kohavi1997wrappers}; (iii) embedded methods, which integrate feature selection into the learning process using techniques like regularization \citep{tibshirani1996LASSO,lemhadri2021lassonet}; and (iv) hybrid methods, which combine elements of (i)-(iii) \citep{SINGH2021104396,li2022micq}. This paper focuses on embedded methods via Lasso, benchmarking against approaches from (i)-(iii).

\subsection{Language Modeling}
% The objective of language modeling is to learn a probability distribution \( p_{LM}(x) \) over sequences of text \( x = (X_1, \ldots, X_{|x|}) \), such that \( p_{LM}(x) \approx p_{text}(x) \), where \( p_{text}(x) \) represents the true distribution of natural language. This process involves estimating the likelihood of token sequences across variable lengths and diverse linguistic structures.
% Modern large language models (LLMs) are trained on vast datasets spanning encyclopedias, news, social media, books, and scientific papers \cite{gao2020pile}. This broad training enables them to generalize across domains, learn contextual knowledge, and perform zero-shot learning—tackling new tasks using only task descriptions without fine-tuning \cite{brown2020gpt3}.
Language modeling aims to approximate the true distribution of natural language \( p_{\text{text}}(x) \) by learning \( p_{\text{LM}}(x) \), a probability distribution over text sequences \( x = (X_1, \ldots, X_{|x|}) \). Modern large language models, trained on diverse datasets \citep{gao2020pile}, exhibit strong generalization across domains, acquire contextual knowledge, and perform zero-shot learning—solving new tasks using only task descriptions—or few-shot learning by leveraging a small number of demonstrations \citep{brown2020gpt3}.
\paragraph{Retrieval-Augmented Generation (RAG).} Retrieval-Augmented Generation (RAG) enhances the performance of generative language models by  integrating a domain-specific information retrieval process  \citep{lewis2020retrieval}. The RAG framework comprises two main components: \textit{retrieval}, which extracts relevant information from external knowledge sources, and \textit{generation}, where an LLM generates context-aware responses using the prompt combined with the retrieved context. Documents are indexed through various databases, such as relational, graph, or vector databases \citep{khattab2020colbert, douze2024faiss, peng2024graphretrievalaugmentedgenerationsurvey}, enabling efficient organization and retrieval via algorithms like semantic similarity search to match the prompt with relevant documents in the knowledge base. RAG has gained much traction recently due to its demonstrated ability to reduce incidence of hallucinations and boost LLMs' reliability as well as performance \citep{huang2023hallucination, zhang2023merging}. 
 
% image source: https://medium.com/@bindurani_22/retrieval-augmented-generation-815c1ae438d8
\begin{figure}
    \centering
\includegraphics[width=1.03\linewidth]{fig/fig1.pdf}
\vspace{-0.6cm}
\scriptsize 
    \caption{Retrieval Augmented Generation (RAG) based $\ell_1$-norm weights (penalty factors) for Lasso. Only feature names---no training data--- are included in LLM prompt.} 
    \label{fig:rag}
\end{figure}
% However, for the RAG model to be effective given the input token constraints of the LLM model used, we need to effectively process the retrieval documents through a procedure known as \textit{chunking}.

\subsection{Task-Specific Data-Driven Learning}
LLM-Lasso aims to bridge the gap between data-driven supervised learning and the predictive capabilities of LLMs trained on rich metadata. This fusion not only enhances traditional data-driven methods by incorporating key task-relevant contextual information often overlooked by such models, but can also be especially valuable in low-data regimes, where the learning algorithm $f:\mathcal{D}\rightarrow\Theta$ (seen as a map from datasets $\mathcal{D}$ to the space of decisions $\Theta$) is susceptible to overfitting.

The task-specific data-driven learning model $\tilde{f}:\mathcal{D}\times\mathcal{D}_\text{meta}\rightarrow\Theta$ can be described as a metadata-augmented version of $f$, where a link function $h(\cdot)$ integrates metadata (i.e. $\mathcal{D}_\text{meta}$) to refine the original learning process. This can be expressed as:
\[
\tilde{f}(\mathcal{D}, \mathcal{D}_\text{meta}) := \mathcal{T}(f(\mathcal{D}),  h(\mathcal{D}_{\text{meta}})),
\]
where the functional $\mathcal{T}$ takes the original learning algorithm $f(\mathcal{D})$ and transforms it into a task-specific learning algorithm $\tilde{f}(\mathcal{D}, \mathcal{D}_\text{meta})$ by incorporating the metadata $\mathcal{D}_\text{meta}$. 
% In particular, the link function $h(\mathcal{D}_{\text{meta}})$ provides a structured mechanism summarizing the contextual knowledge.

There are multiple approaches to formulate $\mathcal{T}$ and $h$.
%to ``inform" the data-driven model $f$ of %meta knowledge. 
For instance, LMPriors \citep{choi2022lmpriorspretrainedlanguagemodels} designed $h$ and $\mathcal{T}$ such that $h(\mathcal{D}_{\text{meta}})$ first specifies which features to retain (based on a probabilistic prior framework), and then $\mathcal{T}$ keeps the selected features and removes all the others from the original learning objective of $f$. 
Note that this approach inherently is restricted as it selects important features solely based on $\mathcal{D}_\text{meta}$ without seeing $\mathcal{D}$.

In contrast, we directly embed task-specific knowledge into the optimization landscape through regularization by introducing a structured inductive bias. This bias guides the learning process toward solutions that are consistent with metadata-informed insights, without relying on explicit probabilistic modeling. Abstractly, this can be expressed as:
\begin{align}
    \!\!\!\!\!\hat{\theta}_{\tilde{f}} := \tilde{f}(\mathcal{D},\mathcal{D}
    _\text{meta})= \arg\min_{\theta} \mathcal{L}(\theta, \mathcal{D}) + \lambda R(\theta, \mathcal{D}_{\text{meta}}),
\end{align}
where \( \lambda \) is a regularization parameter, \( R(\cdot) \) is a regularizer, and $\theta$ is the prediction parameter.
%We explain our framework with more details in the following section.


% Our research diverges from both aforementioned approaches by positioning the LLM not as a standalone feature selector but as an enhancement to data-driven models through an embedded feature selection method, L-LASSO. L-LASSO incorporates domain expertise—auxiliary natural language metadata about the task—via the LLM-informed LASSO penalty, which is then used in statistical models to enhance predictive performance. This method integrates the rich, context-sensitive insights of LLMs with the rigor and transparency of statistical modeling, bridging the gap between data-driven and knowledge-driven feature selection approaches. To approach this task, we need to tackle two key components: (i). train an LLM that is expert in the task-specific knowledge; (ii). inform data-driven feature selector LASSO with LLM knowledge.

% In practice, this involves combining techniques like prompt engineering and data engineering to develop an effective framework for integrating metadata into existing data-driven models. We will go through this in detail in Section \ref{mthd} and \ref{experiment}.


\section{First Step: Non-Contextual Linear Bandits}\label{sec: linear}

In this section, we focus on the non-contextual case, which serves as a building block for eventually solving the contextual case. Before introducing our algorithm, we first briefly introduce the successive arm elimination algorithm for the simpler MAB setting proposed by \citet{schlisselberg2024delay} and their ideas of handling payoff-dependent delay. Specifically, their algorithm starts with a guess $B=1/D$ on the optimal action's loss, and maintains an active set of arms. The algorithm pulls each arm in the active set once, and constructs two LCB's (lower confidence bound) and one UCB (upper confidence bound) for each action in the active set as follows (supposing the current round being $t$):
\begin{align}
        \LCB_{t,1}(a) &= \frac{1}{\cnt_t(a)}\sum_{\tau\in\obs_t(a)}u_\tau - \sqrt{\frac{2\log T}{\cnt_t(a)}}, \label{eqn:lcb-1-mab}\\
        \LCB_{t,2}(a) &= \frac{1}{\unbiasSize_t(a)}\sum_{\tau\in\unbias_t(a)}u_{\tau} - \sqrt{\frac{2\log T}{\unbiasSize_t(a)\vee 1}}, \label{eqn:lcb-2-mab}\\
        \UCB_{t}(a) &= \frac{1}{\unbiasSize_t(a)}\sum_{\tau\in\unbias_t(a)}u_{\tau} + \sqrt{\frac{2\log T}{\unbiasSize_t(a)\vee 1}},\label{eqn:ucb-mab}
\end{align}
where $\cnt_t(a) = \sum_{\tau=1}^t\mathbbm{1}\{a_t=a\}$ is the total number of pulls of action $a$ till round $t$, $\obs_t(a) = \{\tau: \tau+d_{\tau}\leq t \text{~and~} a_{\tau}=a\}$ is the set of rounds where action $a$ is chosen and its loss has been received by the end of round $t$, $\unbias_t(a) = \{\tau \leq t-D: a_\tau = a\}$ is the set of rounds up to $t-D$ where action $a$ is chosen (so its loss has for sure been received by the end of round $t$), 
and $\unbiasSize_t(a)=|\unbias_t(a)|$. Specifically, \pref{eqn:lcb-1-mab} constructs an LCB of action $a$ assuming all the action's unobserved loss to be $0$ (the smallest possible), while \pref{eqn:lcb-2-mab} and \pref{eqn:ucb-mab} construct an LCB and a UCB using only the losses no later than round $t-D$ (which must have been received by round $t$), making the empirical average a better estimate of the expected loss. With $\UCB_t(a)$ and $\LCB_t(a) = \max\{\LCB_{t,1}(a), \LCB_{t,2}(a)\}$ constructed, the algorithm eliminates an action $a$ if its $\LCB_t(a)$ is larger than $\min\{\UCB_t(a'),B\}$ for some $a'$ in the active set. If all the actions are eliminated, this means that the guess $B$ on the optimal loss is too small, and the algorithm starts a new epoch with $B$ doubled.\footnote{In fact, \citet{schlisselberg2024delay} construct yet another LCB based on the number of unobserved losses. We omit this detail since we are not able to use this to further improve our bounds for linear bandits.}

\paragraph{Challenges} However, this approach cannot be directly applied to linear bandits. Specifically, standard algorithms for stochastic linear bandits without delay (e.g., \citet{li2010contextual,abbasi2011improved}) all construct  UCB/LCB for each action by constructing an ellipsoidal confidence set for $\theta$. In the delay-as-payoff model, while it is still viable to construct UCB/LCB similar to \pref{eqn:lcb-2-mab} and \pref{eqn:ucb-mab} via a standard confidence set of $\theta$, it is difficult to construct an LCB counterpart similar to \pref{eqn:lcb-1-mab}.
This is because one action's loss is estimated using observations of all other actions in linear bandits, and naively treating the unobserved loss of one action as zero might not necessarily lead to an underestimation of another action. 

\paragraph{Our ideas} To bypass this barrier, we give up on estimating $\theta$ itself and propose to construct UCB/LCB for each action using the observed losses of the \emph{volumatric spanner} of the action set. A volumetric spanner of an action set $\calA$ is defined such that every action in $\calA$ can be expressed as a linear combination of the spanner. 

\begin{definition}[Volumetric Spanner~\citep{hazan2016volumetric}]\label{def:volume}
Suppose that $\calA = \{a_1, a_2, \dots , a_N\}$ is a set of vectors in $\R^n$. We say $\calS\subseteq \calA$ is a \emph{volumetric spanner} of $\calA$ if for any $a\in \calA$, we can write it as $a=\sum_{b\in \calS}\lambda_b\cdot b$ for some $\lambda\in \R^{|\calS|}$ with $\|\lambda\|_2\leq 1$. 
\end{definition}

Due to the linear structure, it is clear that the loss $\mu_a$ of action $a$ can be decomposed in a similar way as $\sum_{b\in \calS}\lambda_b \mu_b$,
making it possible to estimate every action's loss by only estimating the loss of the spanner.
Moreover, such a spanner can be efficiently computed:
\begin{proposition}[\citet{bhaskara2023tight}]\label{prop:volume}
Given a finite set $\calA$ of size $K$, there exists an efficient algorithm finding a volumetric spanner $\calS$ of $\calA$ with $|\calS|=3n$ within $\order(Kn^3\log n)$ runtime.
\end{proposition}

\setcounter{AlgoLine}{0}
\begin{algorithm}
\caption{Phased Elimination via Volumetric Spanner for Linear Bandits with Delay-as-Loss}\label{alg:lossLB}

\nl Input: maximum possible delay $D$, action set $\calA$, $\beta>0$. 

\nl Initialization: optimal loss guess $B=1/D$.

\nl Initialization: active action set $\calA_1=\calA$. \label{line: restart}

 \For{$m=1,2,\dots,$}{
    \nl Find $\calS_m=\{a_{m,1},\dots,a_{m,|\calS_m|}\}$, a volumetric spanner of $\calA_m$ with $|\calS_m|= 3n$. \label{line:volume}
    
    \nl Pick each $a\in \calS_m$ $2^m$ times in a round-robin way. \label{line:round-robin}

    \nl Let $\calI_m$ contain all the rounds in this epoch.
    
    \nl For each $a\in \calS_m$, calculate the following quantities: \label{line:spanner-ucb-lcb}
    {\small
    \begin{align}
        &\hat{\mu}_{m}^+(a)=\frac{1}{2^m}\Big(\sum_{\tau\in \obs_m(a)}u_{\tau} + \sum_{\tau\in \unobs_m(a)}1\Big), \label{eqn:mean-up}\\
        &\hat{\mu}_{m}^-(a)=\frac{1}{2^m}\sum_{\tau\in \obs_m(a)}u_{\tau}, \label{eqn:mean-low}\\
        &\hat{\mu}_{m,1}^{+}(a)=\hat{\mu}^{+}_{m}(a)+\frac{\beta}{2^{m/2}}\|a\|_2, \label{eqn:loss-ucb-linear-1}\\
        &\hat{\mu}_{m,1}^{-}(a)=\hat{\mu}^{-}_{m}(a)-\frac{\beta}{2^{m/2}}\|a\|_2,\label{eqn:loss-lcb-linear-1}\\
        &\hat{\mu}_m^{F}(a)=\frac{1}{\unbiasSize_m(a)}\sum_{\tau\in \unbias_m(a)}u_{\tau}, \label{eqn:mean_unbiased}\\
        &\hat{\mu}_{m,2}^{+}(a)=\hat{\mu}_m^F(a)+\frac{\beta}{\sqrt{\unbiasSize_m(a)}}\|a\|_2, \label{eqn:loss-ucb-linear-2}\\
        &\hat{\mu}_{m,2}^{-}(a)=\hat{\mu}_m^F(a)-\frac{\beta}{\sqrt{\unbiasSize_m(a)}}\|a\|_2, \label{eqn:loss-lcb-linear-2}
    \end{align}
    }
    where $\unbiasSize_m(a) = |\unbias_m(a)|$, $\unbias_m(a) = \{\tau\in \calI_m: \tau+D\in\calI_m, a_{\tau}=a\}$, $\obs_m(a) = \{\tau\in \calI_m: \tau+d_{\tau}\in\calI_m, a_{\tau}=a\}$, and
    $\unobs_m(a)= \{\tau\in \calI_m: a_{\tau}=a\}\setminus\obs_m(a)$.

    \For{each $a\in \calA_m$}{
        \nl \label{line: decompose}
        Decompose $a$ as $a=\sum_{i=1}^{|S_m|}\lambda_{m,i}^{(a)}a_{m,i}$ with $\|\lambda_{m}^{(a)}\|_2\leq 1$ and calculate 
        {\small
        \begin{align}
            &\UCB_{m}(a)=\sum_{i=1}^{|\calS_m|}\lambda_{m,i}^{(a)}\cdot\hat{\mu}_{m,2}^{\sgn(\lambda_{m,i}^{(a)})}(a_{m,i}), \label{eqn:loss-ucb-f-all-action} \\
            &\LCB_m(a) = \max_{j\in \{1,2\}}\{\LCB_{m,j}(a)\} \;\;\text{where} \nonumber  \\
            & \LCB_{m,j}(a)=\sum_{i=1}^{|\calS_m|}\lambda_{m,i}^{(a)}\cdot\hat{\mu}_{m,j}^{\sgn(-\lambda_{m,i}^{(a)})}(a_{m,i}),\label{eqn:loss-lcb-all-action}
        \end{align}
        }
    }
    
    \nl Set $\calA_{m+1} = \calA_m$.
    
    \For{$a\in \calA_m$}{
        \nl \label{line:eliminate}  
        \If{$\exists a'\in \calA_m$, s.t. $\LCB_m(a) \geq \min\{\UCB_m(a'),B\} $}
        {
          Eliminate $a$ from $\calA_{m+1}$.
        }
    }
    \nl \If{$\calA_{m+1}=\emptyset$}{
        Set $B\leftarrow 2B$ and go to \pref{line: restart}.
    }
}
\end{algorithm}

Equipped with the concept of volumetric spanner, we are now ready to introduce our algorithm (see \pref{alg:lossLB}). 
Specifically, our algorithm also makes a guess $B$ on the loss of the optimal action. 
With this guess, it proceeds to multiple epochs of arm elimination procedures, with the active action set initialized as $\calA_1 = \calA$.
In each epoch $m$, instead of picking every action in the active set $\calA_m$, we first compute a volumetric spanner $\calS_m$ of $\calA_m$ with $|\calS_m|=3n$ (\pref{line:volume}), which can be done efficiently according to \pref{prop:volume}, 
and then pick each action in the spanner set $\calS_m$ for $2^m$ rounds in a round-robin way (\pref{line:round-robin}).

After that, we calculate two UCBs and two LCBs for actions in the spanner, in a way similar to the simpler MAB setting discussed earlier (\pref{line:spanner-ucb-lcb}).
Specifically, 
the first one is in the same spirit of \pref{eqn:lcb-1-mab}:
we calculate $\hat{\mu}_m^+(a)$ ( $\hat{\mu}_m^-(a)$) as an overestimation (underestimation) of the expected loss of action $a$ by averaging over all observed losses from the rounds in $\obs_m(a)$ as well as the maximum (minimum) possible value of unobserved losses from the rounds in $\unobs_m(a)$; see \pref{eqn:mean-up} and \pref{eqn:mean-low}.
The first UCB (LCB) $\hat{\mu}_{m,1}^+(a)$ ($\hat{\mu}_{m,1}^-(a)$) is then computed based on $\hat{\mu}_m^+(a)$ ($\hat{\mu}_m^-(a)$) by incorporating a standard confidence width $\frac{\beta}{\sqrt{2^m}}\|a\|_2$ for some coefficient $\beta$; see \pref{eqn:loss-ucb-linear-1} and \pref{eqn:loss-lcb-linear-1}.
Then, to compute the second UCB/LCB, which is in the same spirit as \pref{eqn:lcb-2-mab} and \pref{eqn:ucb-mab}, we calculate an unbiased estimation $\hat{\mu}_m^F(a)$ of the expected loss of $a$ by averaging losses from the rounds in $\unbias_m(a)$, that is, all the rounds where the observation must have been revealed; see \pref{eqn:mean_unbiased}.
Note that the number of such rounds, $\unbiasSize_m(a) = |\unbias_m(a)|$, is a fixed number, and thus $\hat{\mu}_m^F(a)$ is indeed unbiased.
Similarly, we incorporate a standard confidence width $\frac{\beta}{\sqrt{c_m(a)}}\|a\|_2$ to arrive at the second UCB $\hat{\mu}_{m,2}^+(a)$ and LCB $\hat{\mu}_{m,2}^-(a)$; see \pref{eqn:loss-ucb-linear-2} and \pref{eqn:loss-lcb-linear-2}.

The next step of our algorithm is to use these UCBs/LCBs for the spanner to compute corresponding UCBs/LCBs for every active action in $\calA_m$ (\pref{line: decompose}). Specifically, for each action $a\in \calA_m$, according to the definition of a volumetric spanner (\pref{def:volume}), we can write $a$ as a linear combination of actions in $\calS_m$: $\sum_{i=1}^{|S_m|}\lambda_{m,i}^{(a)}a_{m,i}$. As mentioned, due to the linear structure of losses, we also have $\mu_a = \sum_{i=1}^{|S_m|}\lambda_{m,i}^{(a)}\mu_{a_{m,i}}$.
Thus, when constructing a UCB (or similarly LCB) for $a$, based on whether $\lambda_{m,i}^{(a)}$ is positive or not, we decide whether to use the UCB or LCB of $a_{m,i}$; see \pref{eqn:loss-ucb-f-all-action}, a counterpart of \pref{eqn:ucb-mab}, and \pref{eqn:loss-lcb-all-action}, a counterpart of \pref{eqn:lcb-1-mab} and \pref{eqn:lcb-2-mab}.\footnote{This also explains why we need $\hat{\mu}_m^+(a)$, a quantity not used in~\citet{schlisselberg2024delay}.}

At the end of an epoch, we eliminate all actions from the active action set if their LCB is either larger than the UCB of certain action or the guess $B$ on the optimal loss  (\pref{line:eliminate}). 
If the active set becomes empty, this means that the guess $B$ is too small, and the algorithm restarts with the guess doubled; 
otherwise, we continue to the next epoch.

\paragraph{Theoretical performance}
We prove the following regret bound for our algorithm. 
\begin{restatable}{theorem}{lossLB}
\label{thm:main-non-contextual}
    \pref{alg:lossLB} with $\beta=\sqrt{2\log(KT^3)}$ guarantees: 
\begin{align*}
        \Reg &\leq \order\left(\min\left\{V_1,V_2\right\}\right) + \log(d^\star)\cdot \order\left( \min\left\{W_1,W_2\right\}\right),
    \end{align*}
    where $V_1=\frac{n^2\log(KT)\log(T/n)\log(d^\star)}{\Delta_{\min}}$, $V_2=n\sqrt{T\log(d^\star)\log(KT)}$, $W_1=nd^\star\log (T/n)+D\Delta_{\max}$, and $W_2=D\Delta_{\max}\log (T/n)$. 
\end{restatable}
The first term in the regret bound $\order\left(\min\left\{V_1,V_2\right\}\right)$ is of order $\otil(\min\{\frac{n^2}{\Delta_{\min}}, n\sqrt{T}\})$, which matches the standard regret bound of LinUCB in the case without delay~\citep{abbasi2011improved}.
The second term is the overhead caused by delay and is in the same spirit as the result of~\citet{schlisselberg2024delay}:
focusing only on the part that grows in $T$, 
we see that $W_1$ only depends on $d^\star$, the expected delay of the optimal action (and hence the smallest delay among all actions),
while $W_2$ depends on the maximum possible delay $D$ but scaled by $\Delta_{\max}$, the largest sub-optimality gap.
Therefore, the delay overhead of our algorithm is small when either the shortest delay is small or all actions have similar losses.
We remark again that in the delay-as-reward setting, we obtain similar results; see \pref{app: reward} for details.

\subsection{Analysis}\label{sec: alg}
In this section, we provide a proof sketch of \pref{thm:main-non-contextual}. Detailed proofs are deferred to \pref{app:loss}.

The proof starts by proving that $\UCB_m(a)$ and $\LCB_m(a)$ are indeed valid UCB and LCB respectively for all actions in $\calA_m$. 
This follows from first using standard concentration inequalities to show that $\hat{\mu}_{m,1}^+(a)$ and $\hat{\mu}_{m,2}^+(a)$ ($\hat{\mu}_{m,1}^-(a)$ and $\hat{\mu}_{m,2}^-(a)$) are valid UCBs (LCBs) for each action in the spanner, 
and then generalizing it to every action $a \in \calA_m$ according to its decomposition over the actions in the spanner.

With this property, our analysis then proceeds to control the regret of \pref{alg:lossLB} for each guess of $B$ separately. Let $\calT_B$ be the set of rounds when \pref{alg:lossLB} runs with guess $B$. 
In \textbf{Step 1}, we first show that the use of $\LCB_{m,2}(a)$ and $\UCB_m(a)$ ensures a regret bound of $\order\left(\min\{R_1,R_2\}+D\Delta_{\max}\log(T/n)\right)$ where $R_1=\frac{n^2\log(KT)\log(T/n)}{\Delta_{\min}}$ and $R_2=n\sqrt{|\calT_B|\log(KT)}$,
and then in \textbf{Step 2}, we show that the use of $\LCB_{m,1}(a)$ and $\UCB_m(a)$ ensures a regret bound of
$\order(\min\{R_1,R_2\}+(nd^\star+DB)\log(T/n)+D\Delta_{\max})$.

\paragraph{Step 1}
For notational convenience, we define 
\begin{align*}
    \rad_{m,a}^F=\beta\sum_{i=1}^{|\calS_m|}|\lambda_{m,i}^{(a)}|\cdot\frac{\|a\|_2}{\sqrt{\unbiasSize_m(a_{m,i})}}
\end{align*}
to be the total confidence radius of action $a$ coming from the definition of $\LCB_{m,2}(a)$ and $\UCB_m(a)$. 
Via a standard analysis of arm elimination, 
we show that that if an action $a$ is not eliminated at the end of epoch $m$, we have
\begin{align*}
    \Delta_a \leq 4\max_{a\in\calA_m}\rad_{m,a}^F \leq \frac{4\sqrt{3n}\beta}{\min_{a_m\in\calS_m}\sqrt{\unbiasSize_m(a_m)}},
\end{align*}
where the second inequality uses Cauchy-Schwarz inequality and the properties of volumetric spanners, specifically that $\|\lambda_{m}^{(a)}\|_2\leq 1$ and $|\calS_m|=3n$. To provide a lower bound on $c_m(a')$ for any $a'\in\calS_m$, note that we pick each action $a'\in \calS_m$ $2^m$ times in a round-robin manner, and thus
\begin{align*}
    c_m(a') \geq 2^m - \frac{D}{|\calS_m|}-1 = 2^m - \frac{D}{3n}-1.
\end{align*}
Rearranging the terms, we then obtain
\begin{align}\label{eqn:epoch_bound_1}
    2^m\Delta_a \leq \frac{48n\beta^2}{\Delta_a} + \frac{D\Delta_a}{3n} + \Delta_a.
\end{align}
Taking summation over all $a\in\calS_m$ and $m$, and noticing that the total number of epochs is bounded by $M=\lceil\log_2(|\calT_B|/3n)\rceil$, we arrive at the following $\order(R_1+D\Delta_{\max}\log(T/n))$ regret guarantee:
\begin{align}
&\sum_{m=1}^{M}\sum_{a\in\calS_m}2^m\Delta_a \nonumber\\
    &\leq \sum_{m=1}^{M}\sum_{a\in\calS_m,\Delta_a>0}2\cdot\left(\frac{48n\beta^2}{\Delta_a}+\frac{D\Delta_a}{3n}+\Delta_a\right) \nonumber\\
    &\leq \sum_{m=1}^{M}\sum_{a\in\calS_m,\Delta_a>0}\order\left(\frac{n\log (KT)}{\Delta_a}\right) + \order\left(D\Delta_{\max}\log(T/n)\right),\nonumber \\
    &\leq \order\left(\frac{n^2\log(T/n)\log (KT)}{\Delta_{\min}}\right) + \order\left(D\Delta_{\max}\log(T/n)\right),\nonumber
\end{align}
where the first inequality is because $a\in\calS_m$ is not eliminated in epoch $m-1$ and the last inequality is by lower bounding $\Delta_a$ by $\Delta_{\min}$.

To obtain the other instance-independent regret bound $\order(R_2+D\Delta_{\max}\log(T/n))$, we bound the regret differently by considering $\Delta_a\geq \beta\sqrt{n/2^m}$ and $\Delta_a\leq \beta\sqrt{n/2^m}$ separately:
\begin{align}
    &\sum_{m=1}^{M}\sum_{a\in\calS_m}2^m\Delta_a \nonumber \\
    &\leq \sum_{m=1}^{M}\sum_{a\in\calS_m,\Delta_a\geq\beta\sqrt{n/2^m}}\left(\frac{512n\beta^2}{\Delta_a}+\frac{2D\Delta_a}{3n}+2\Delta_a\right) \nonumber\\
    &\qquad +\sum_{m=1}^{M}\sum_{a\in\calS_m,\Delta_a\leq\beta\sqrt{n/2^m}}\left(2^m\Delta_a\right) \nonumber \\
    &\leq \order(n\sqrt{|\calT_B|\log(KT)} + D\Delta_{\max}\log(T/n))\nonumber.
\end{align}

\paragraph{Step 2}
To obtain the other regret bound $\order(\min\{R_1,R_2\}+(nd^\star+DB)\log(T/n)+D\Delta_{\max})$ with a different delay overhead, we similarly define
\begin{align*}
    \rad_{m,a}^{N} &= \beta\sum_{i=1}^{|\calS_m|}|\lambda_{m,i}^{(a)}|\cdot \frac{\|a\|_2}{\sqrt{2^m}}
\end{align*}
as the total confidence radius of action $a$ coming from the definition of $\LCB_{m,1}(a)$. 
Further let $\wh{\mu}_m(a) = \frac{1}{2^m}\left(\sum_{\tau\in \obs_m(a)\cup\unobs_m(a)}u_{\tau}\right)$ be the empirical mean of action $a$'s loss within epoch $m$ (which is generally not available to the algorithm due to delay). According to the construction of $\wh{\mu}_m^{+}(a)$ and $\wh{\mu}_m^{-}(a)$, we know that for all $a\in\calS_m$,
\begin{align*}
    \wh{\mu}_m^{+}(a)\leq \wh{\mu}_m(a) + \frac{|\unobs_m(a)|}{2^m},~~\wh{\mu}_m^{-}(a)\geq \wh{\mu}_m(a) - \frac{|\unobs_m(a)|}{2^m}.
\end{align*}
Then, for any action $a\in\calA_m$ that is not eliminated at the end of epoch $m$, using the fact that $a=\sum_{i=1}^{|\calS_m|}\lambda_{m,i}^{(a)}a_{m,i}$, we obtain with high probability:
\begin{align}
    \mu_a &\leq \sum_{i=1}^{|\calS_m|}\lambda_{m,i}^{(a)}\cdot \hat{\mu}_{m}(a_{m,i}) + \rad_{m,a}^{N} \nonumber\\
    &\leq \LCB_{m,1}(a) + \rad_{m,a}^{N}+\sum_{i=1}^{|\calS_m|}|\lambda_{m,i}^{(a)}|\cdot \frac{|\unobs_m(a_{m,i})|}{2^m} \nonumber\\
    &\leq \LCB_{m,1}(a) +\rad_{m,a}^{N} \nonumber\\
    &\qquad + \sum_{i=1}^{|\calS_m|}|\lambda_{m,i}^{(a)}|\cdot\left(\frac{2D\mu_{a_{m,i}}}{2^m|\calS_m|}+\frac{16\log KT +2}{2^m}\right) \\
    &\leq B +\rad_{m,a}^{N} \nonumber\\
    &\qquad + \sum_{i=1}^{|\calS_m|}|\lambda_{m,i}^{(a)}|\cdot\left(\frac{2D\mu_{a_{m,i}}}{2^m|\calS_m|}+\frac{16\log KT +2}{2^m}\right),\label{eqn:small-loss}
\end{align}
where the first inequality is by standard Azuma-Hoeffding's inequality, the third inequality is by Lemma C.2 of \citet{schlisselberg2024delay} (included as \pref{lem:high-prob-event} in the appendix for completeness), and the last inequality is because $a$ is not eliminated at the end of epoch $m$.

\setcounter{AlgoLine}{0}
\begin{algorithm*}[htbp]
\caption{Reduction from Contextual Linear Bandits to Non-Contextual Linear Bandits~\citep{hanna2023contexts}}\label{alg:reduction}
Input: confidence level $\delta$, an instance $\Alg_{\nctx}$ of \pref{alg:lossLBmis} with $\beta=\sqrt{2\log(KT^3)}$. 

Let $\Theta'$ be a $\frac{1}{T}$-cover of $\Theta$ with size $\order(T^n)$.

\For{$m=1,2,\dots$}{
    \nl Construct action set $\calX_{m}=\{\gup{m}(\theta)~\vert~\theta\in \Theta'\}$ where   $\gup{m}(\theta)=\frac{1}{2^{m-1}}\sum_{\tau=1}^{2^{m-1}}\argmin_{a\in \calA_\tau}\inner{a,\theta}$.
    

    \nl Initiate $\Alg_{\nctx}$ with action set $\calX_m$ and misspecification level $\epsilon_m=\min\{1,2\sqrt{\log(T|\Theta'|/\delta)/2^m}\}$. \label{line:misspecific_level}
    
    \nl \For{$t=2^{m-1}+1,\dots,2^m$}{
        \nl $\Alg_{\nctx}$ outputs action $\gup{m}(\theta_t)$.

        \nl Observe $\calA_t$ and select $a_t=\argmin_{a\in \calA_t}\inner{a,\theta_t}$.

        \nl Observe the loss $u_\tau$ for all $\tau$ such that $\tau+d_{\tau}\in (t-1,t]$ and send them to $\Alg_{\nctx}$.
    }
    
}
\end{algorithm*}

Now consider two cases. When $B\geq \frac{\mu_a}{2}$, we know that $\Delta_a\leq \mu_a - \mu^\star\leq 2B$. Using the previous \pref{eqn:epoch_bound_1}, we know that
\begin{align}\label{eqn:small-loss-1}
    2^m\Delta_a\leq \order\left(\frac{n\beta^2}{\Delta_a}+\frac{DB}{n}\right).
\end{align}
Otherwise, when $B < \frac{\mu_a}{2}$, with some manipulation on \pref{eqn:small-loss}, we show that
\begin{align}\label{eqn:small-loss-2}
    2^m\Delta_a\leq \order\left(\frac{n\beta^2}{\Delta_a}+\frac{\sum_{i=1}^{|\calS_m|}D\mu_{a_{m,i}}}{n}\right).
\end{align}
Combining \pref{eqn:small-loss-1} and \pref{eqn:small-loss-2}, we then obtain that within epoch $m$, the regret is bounded by
\begin{align}\label{eqn:small-loss-3}
\order\left(\sum_{a\in\calS_m}\frac{n\beta^2}{\Delta_a}+DB+D\sum_{i=1}^{|\calS_{m-1}|}\mu_{a_{m-1,i}}\right),
\end{align} 
since all active actions in epoch $m$ are not eliminated in epoch $m-1$.
The first term $\sum_{a\in\calS_m}\frac{n\beta^2}{\Delta_a}$ in \pref{eqn:small-loss-3} eventually leads to the $\min\{R_1,R_2\}$ term in the claimed regret bound, by the exact same reasoning as in \textbf{Step 1}.
The second term explains the final $DB\log(T/n)$ term in the regret bound (recall that number of epoch is of order $\order(\log(T/n))$).
Finally, the last term in \pref{eqn:small-loss-3} can be written as
$D\sum_{i=1}^{|\calS_{m-1}|} \Delta_{a_{m-1,i}} + 3n\cdot d^\star$,
and the term $D\sum_{i=1}^{|\calS_{m-1}|} \Delta_{a_{m-1,i}}$ is one half of the regret incurred in epoch $m-1$ as long as $2^{m-1}>2D$ (otherwise, the epoch length is smaller than $D$, and we bound the regret trivially by $D\Delta_{\max}$).
Summing over all epochs and rearranging the terms thus leads to the a term $nd^\star\log(T/n)$ in the regret.
This proves the goal of the second step.

\paragraph{Combining everything} 
Finally, note that the number of different values of $B$ \pref{alg:lossLB} uses is upper bounded by $\lceil\log_2(d^\star)\rceil=\lceil\log_2(D\mu^\star)\rceil$ since the optimal action $a^\star$ will never be eliminated when $B\geq \mu^\star$. Summing up the regret over these different values of $B$ arrives at the the final bound $\order(\min\{V_1,V_2\},\log(d^\star)\min\{W_1,W_2\})$.











\section{Extension to Contextual Linear Bandits}\label{sec: contextual}


In this section, we extend our results to the stochastic contextual setting where the action set at each round is drawn i.i.d. from a distribution $\dist$. 
While the arm elimination procedure is critical in solving our problem in the non-contextual case with a fixed action set, it is not clear (if possible at all) to directly generalize it to the contextual setting due to the dynamic nature of the action set.


Fortunately, a recent work by \citet{hanna2023contexts} proposes a reduction from contextual linear bandits to non-contextual linear bandits (both without delay).
At a high level, this reduction utilizes a subroutine of a non-contextual linear bandits algorithm by constructing a fixed action for each possible parameter $\theta$ of the contextual bandit instance. 
Importantly, the subroutine needs to be able to deal with an $\epsilon$-misspecified model, where the loss of each $a\in \calA$ is almost linear: $\mu_a=\inner{a,\theta}+\epsilon_a\in [0,1]$, with $\epsilon \geq \max_{a\in\calA}|\epsilon_a|$ indicating the misspecification level. 
It turns out that, a simple modification of our \pref{alg:lossLB} can address such misspecification --- it only requires incorporating the misspecification level $\epsilon$ into the criteria of arm elimination;
see \pref{alg:lossLBmis} and specifically its \pref{line:eliminate-mis} for details.


We then plugin this subroutine, denoted as $\Alg_{\nctx}$, into their reduction, as shown in \pref{alg:reduction}.
Specifically, the algorithm first constructs a $\frac{1}{T}$-cover $\Theta'$ of the parameter space $\Theta=\R_+^n\cap\mathbb{B}_2^n(1)$ with size $|\Theta'| = \mathcal{O}(T^n)$. 
It then proceeds in epochs with doubling length. 
At the start of epoch $m$, 
a new \emph{fixed} action set $\mathcal{X}_m = \{g^{(m)}(\theta) : \theta \in \Theta'\}$ is constructed, where $g^{(m)}(\theta)$ is the averaged optimal action over the previous $m-1$ epochs, assuming the model parameter being $\theta$.
Then, a new instance of $\Alg_{\nctx}$ with action set $\mathcal{X}_m$ and some 
misspecification level $\epsilon_m$ is initiated and run for the entire epoch.
At each round $t$ of this epoch, $\Alg_{\nctx}$ outputs an action $g^{(m)}(\theta_t) \in \mathcal{X}_m$, and the algorithm's final decision upon receiving the true action set $\calA_t$ is $a_t=\argmin_{a\in \calA_t}\inner{a,\theta_t}$.
Finally, at the end of this round, all newly observed losses are sent to $\Alg_{\nctx}$.


\begin{figure*}[t]
\centering

\includegraphics[width=0.33\textwidth]{Figure/cost_dimension_6.pdf}
\includegraphics[width=0.33\textwidth]{Figure/cost_dimension_8.pdf}
\includegraphics[width=0.33\textwidth]{Figure/cost_dimension_10.pdf}

\includegraphics[width=0.33\textwidth]{Figure/reward_dimension_6.pdf}
\includegraphics[width=0.33\textwidth]{Figure/reward_dimension_8.pdf}
\includegraphics[width=0.33\textwidth]{Figure/reward_dimension_10.pdf}
\caption{Comparison of the empirical results of our algorithm and \texttt{LinUCB}. The top row is the delay-as-loss setting and the bottom row is the delay-as-reward setting. The left, middle, and right column correspond to $n=6,8,10$ respectively.}
\label{fig:synthetic_dataset}
\end{figure*}

\paragraph{Guarantees and Analysis}
Even though our algorithm is a direct application of the reduction of~\citet{hanna2023contexts}, it is a priori unclear whether it enjoys any favorable regret guarantee in the delay-as-loss setting.
By adopting and generalizing their analysis, we show that this is indeed the case.
Before introducing our results, we define the following quantities:
    \begin{align*}
        g(\theta) &\triangleq \E_{\calA\sim \dist}\left[\argmin_{a\in\calA}\inner{a,\theta}\right],\\
        \Delta_{\min}^{\nctx} &\triangleq\min_{\theta'\in \Theta', \inner{g(\theta'),\theta}\neq \inner{ g(\theta),\theta}}\E\left[\inner{g(\theta)-g(\theta'),\theta}\right],\\
        \Delta_{\max}^{\nctx}  &\triangleq\max_{\theta'\in \Theta'}\E\left[\inner{g(\theta)-g(\theta'),\theta}\right],\\
        \overline{d}^{\star} &\triangleq D\cdot \inner{g(\theta),\theta} = D\cdot \E_{\calA\sim \dist}\left[\min_{a\in \calA}\inner{a,\theta}\right],
    \end{align*}
    where $g(\theta)$ denotes the optimal action in expectation, $\Delta_{\min}^{\nctx}$ ($\Delta_{\max}^{\nctx}$) denotes the minimum (maximum) suboptimality gap for the reduced non-contextual linear bandit instance, and $\overline{d}^\star$ denotes the expected delay of the optimal action.

\begin{theorem}\label{thm:reduction}
    \pref{alg:reduction} with %$t^{(m)}=2^{m-1}$ and 
    $\delta = 1/T^2$ guarantees
    \begin{align*}
        &\Reg =\order\big(n\sqrt{T\log T}+\min\{V_1,V_2\} \\
        &\quad\quad\quad\quad +\log(\overline{d}^\star)\min\{W_1,W_2\}\big),
    \end{align*}
     where $V_1=\frac{n^3\log^2(T)\log(T/n)\log(\overline{d}^\star)}{\Delta_{\min}^{\nctx}}$, $V_2=n^{1.5}\sqrt{T\log(\overline{d}^\star)\log(T)}$, $W_1=\log T(n\overline{d}^\star\log(T/n)+D\Delta_{\max}^{\nctx})$, and $W_2=D\Delta_{\max}^{\nctx}\log T\log(T/n)$.
\end{theorem}
The proof is deferred to \pref{app: contextual}. 
The regret bound is in the same spirit as the one for the non-contextual case (\pref{thm:main-non-contextual}) and consists of a term for standard regret and a term for delay overhead.
The standard part unfortunately suffers higher dependence on the dimension $n$, while the delay overhead is in a similar problem-dependent form.
We remark that this is the first regret guarantee for contextual linear bandits with delay-as-payoff, resolving an open problem asked by \citep{schlisselberg2024delay}.
\section{Experiments}
\label{sec:experiments}
The experiments are designed to address two key research questions.
First, \textbf{RQ1} evaluates whether the average $L_2$-norm of the counterfactual perturbation vectors ($\overline{||\perturb||}$) decreases as the model overfits the data, thereby providing further empirical validation for our hypothesis.
Second, \textbf{RQ2} evaluates the ability of the proposed counterfactual regularized loss, as defined in (\ref{eq:regularized_loss2}), to mitigate overfitting when compared to existing regularization techniques.

% The experiments are designed to address three key research questions. First, \textbf{RQ1} investigates whether the mean perturbation vector norm decreases as the model overfits the data, aiming to further validate our intuition. Second, \textbf{RQ2} explores whether the mean perturbation vector norm can be effectively leveraged as a regularization term during training, offering insights into its potential role in mitigating overfitting. Finally, \textbf{RQ3} examines whether our counterfactual regularizer enables the model to achieve superior performance compared to existing regularization methods, thus highlighting its practical advantage.

\subsection{Experimental Setup}
\textbf{\textit{Datasets, Models, and Tasks.}}
The experiments are conducted on three datasets: \textit{Water Potability}~\cite{kadiwal2020waterpotability}, \textit{Phomene}~\cite{phomene}, and \textit{CIFAR-10}~\cite{krizhevsky2009learning}. For \textit{Water Potability} and \textit{Phomene}, we randomly select $80\%$ of the samples for the training set, and the remaining $20\%$ for the test set, \textit{CIFAR-10} comes already split. Furthermore, we consider the following models: Logistic Regression, Multi-Layer Perceptron (MLP) with 100 and 30 neurons on each hidden layer, and PreactResNet-18~\cite{he2016cvecvv} as a Convolutional Neural Network (CNN) architecture.
We focus on binary classification tasks and leave the extension to multiclass scenarios for future work. However, for datasets that are inherently multiclass, we transform the problem into a binary classification task by selecting two classes, aligning with our assumption.

\smallskip
\noindent\textbf{\textit{Evaluation Measures.}} To characterize the degree of overfitting, we use the test loss, as it serves as a reliable indicator of the model's generalization capability to unseen data. Additionally, we evaluate the predictive performance of each model using the test accuracy.

\smallskip
\noindent\textbf{\textit{Baselines.}} We compare CF-Reg with the following regularization techniques: L1 (``Lasso''), L2 (``Ridge''), and Dropout.

\smallskip
\noindent\textbf{\textit{Configurations.}}
For each model, we adopt specific configurations as follows.
\begin{itemize}
\item \textit{Logistic Regression:} To induce overfitting in the model, we artificially increase the dimensionality of the data beyond the number of training samples by applying a polynomial feature expansion. This approach ensures that the model has enough capacity to overfit the training data, allowing us to analyze the impact of our counterfactual regularizer. The degree of the polynomial is chosen as the smallest degree that makes the number of features greater than the number of data.
\item \textit{Neural Networks (MLP and CNN):} To take advantage of the closed-form solution for computing the optimal perturbation vector as defined in (\ref{eq:opt-delta}), we use a local linear approximation of the neural network models. Hence, given an instance $\inst_i$, we consider the (optimal) counterfactual not with respect to $\model$ but with respect to:
\begin{equation}
\label{eq:taylor}
    \model^{lin}(\inst) = \model(\inst_i) + \nabla_{\inst}\model(\inst_i)(\inst - \inst_i),
\end{equation}
where $\model^{lin}$ represents the first-order Taylor approximation of $\model$ at $\inst_i$.
Note that this step is unnecessary for Logistic Regression, as it is inherently a linear model.
\end{itemize}

\smallskip
\noindent \textbf{\textit{Implementation Details.}} We run all experiments on a machine equipped with an AMD Ryzen 9 7900 12-Core Processor and an NVIDIA GeForce RTX 4090 GPU. Our implementation is based on the PyTorch Lightning framework. We use stochastic gradient descent as the optimizer with a learning rate of $\eta = 0.001$ and no weight decay. We use a batch size of $128$. The training and test steps are conducted for $6000$ epochs on the \textit{Water Potability} and \textit{Phoneme} datasets, while for the \textit{CIFAR-10} dataset, they are performed for $200$ epochs.
Finally, the contribution $w_i^{\varepsilon}$ of each training point $\inst_i$ is uniformly set as $w_i^{\varepsilon} = 1~\forall i\in \{1,\ldots,m\}$.

The source code implementation for our experiments is available at the following GitHub repository: \url{https://anonymous.4open.science/r/COCE-80B4/README.md} 

\subsection{RQ1: Counterfactual Perturbation vs. Overfitting}
To address \textbf{RQ1}, we analyze the relationship between the test loss and the average $L_2$-norm of the counterfactual perturbation vectors ($\overline{||\perturb||}$) over training epochs.

In particular, Figure~\ref{fig:delta_loss_epochs} depicts the evolution of $\overline{||\perturb||}$ alongside the test loss for an MLP trained \textit{without} regularization on the \textit{Water Potability} dataset. 
\begin{figure}[ht]
    \centering
    \includegraphics[width=0.85\linewidth]{img/delta_loss_epochs.png}
    \caption{The average counterfactual perturbation vector $\overline{||\perturb||}$ (left $y$-axis) and the cross-entropy test loss (right $y$-axis) over training epochs ($x$-axis) for an MLP trained on the \textit{Water Potability} dataset \textit{without} regularization.}
    \label{fig:delta_loss_epochs}
\end{figure}

The plot shows a clear trend as the model starts to overfit the data (evidenced by an increase in test loss). 
Notably, $\overline{||\perturb||}$ begins to decrease, which aligns with the hypothesis that the average distance to the optimal counterfactual example gets smaller as the model's decision boundary becomes increasingly adherent to the training data.

It is worth noting that this trend is heavily influenced by the choice of the counterfactual generator model. In particular, the relationship between $\overline{||\perturb||}$ and the degree of overfitting may become even more pronounced when leveraging more accurate counterfactual generators. However, these models often come at the cost of higher computational complexity, and their exploration is left to future work.

Nonetheless, we expect that $\overline{||\perturb||}$ will eventually stabilize at a plateau, as the average $L_2$-norm of the optimal counterfactual perturbations cannot vanish to zero.

% Additionally, the choice of employing the score-based counterfactual explanation framework to generate counterfactuals was driven to promote computational efficiency.

% Future enhancements to the framework may involve adopting models capable of generating more precise counterfactuals. While such approaches may yield to performance improvements, they are likely to come at the cost of increased computational complexity.


\subsection{RQ2: Counterfactual Regularization Performance}
To answer \textbf{RQ2}, we evaluate the effectiveness of the proposed counterfactual regularization (CF-Reg) by comparing its performance against existing baselines: unregularized training loss (No-Reg), L1 regularization (L1-Reg), L2 regularization (L2-Reg), and Dropout.
Specifically, for each model and dataset combination, Table~\ref{tab:regularization_comparison} presents the mean value and standard deviation of test accuracy achieved by each method across 5 random initialization. 

The table illustrates that our regularization technique consistently delivers better results than existing methods across all evaluated scenarios, except for one case -- i.e., Logistic Regression on the \textit{Phomene} dataset. 
However, this setting exhibits an unusual pattern, as the highest model accuracy is achieved without any regularization. Even in this case, CF-Reg still surpasses other regularization baselines.

From the results above, we derive the following key insights. First, CF-Reg proves to be effective across various model types, ranging from simple linear models (Logistic Regression) to deep architectures like MLPs and CNNs, and across diverse datasets, including both tabular and image data. 
Second, CF-Reg's strong performance on the \textit{Water} dataset with Logistic Regression suggests that its benefits may be more pronounced when applied to simpler models. However, the unexpected outcome on the \textit{Phoneme} dataset calls for further investigation into this phenomenon.


\begin{table*}[h!]
    \centering
    \caption{Mean value and standard deviation of test accuracy across 5 random initializations for different model, dataset, and regularization method. The best results are highlighted in \textbf{bold}.}
    \label{tab:regularization_comparison}
    \begin{tabular}{|c|c|c|c|c|c|c|}
        \hline
        \textbf{Model} & \textbf{Dataset} & \textbf{No-Reg} & \textbf{L1-Reg} & \textbf{L2-Reg} & \textbf{Dropout} & \textbf{CF-Reg (ours)} \\ \hline
        Logistic Regression   & \textit{Water}   & $0.6595 \pm 0.0038$   & $0.6729 \pm 0.0056$   & $0.6756 \pm 0.0046$  & N/A    & $\mathbf{0.6918 \pm 0.0036}$                     \\ \hline
        MLP   & \textit{Water}   & $0.6756 \pm 0.0042$   & $0.6790 \pm 0.0058$   & $0.6790 \pm 0.0023$  & $0.6750 \pm 0.0036$    & $\mathbf{0.6802 \pm 0.0046}$                    \\ \hline
%        MLP   & \textit{Adult}   & $0.8404 \pm 0.0010$   & $\mathbf{0.8495 \pm 0.0007}$   & $0.8489 \pm 0.0014$  & $\mathbf{0.8495 \pm 0.0016}$     & $0.8449 \pm 0.0019$                    \\ \hline
        Logistic Regression   & \textit{Phomene}   & $\mathbf{0.8148 \pm 0.0020}$   & $0.8041 \pm 0.0028$   & $0.7835 \pm 0.0176$  & N/A    & $0.8098 \pm 0.0055$                     \\ \hline
        MLP   & \textit{Phomene}   & $0.8677 \pm 0.0033$   & $0.8374 \pm 0.0080$   & $0.8673 \pm 0.0045$  & $0.8672 \pm 0.0042$     & $\mathbf{0.8718 \pm 0.0040}$                    \\ \hline
        CNN   & \textit{CIFAR-10} & $0.6670 \pm 0.0233$   & $0.6229 \pm 0.0850$   & $0.7348 \pm 0.0365$   & N/A    & $\mathbf{0.7427 \pm 0.0571}$                     \\ \hline
    \end{tabular}
\end{table*}

\begin{table*}[htb!]
    \centering
    \caption{Hyperparameter configurations utilized for the generation of Table \ref{tab:regularization_comparison}. For our regularization the hyperparameters are reported as $\mathbf{\alpha/\beta}$.}
    \label{tab:performance_parameters}
    \begin{tabular}{|c|c|c|c|c|c|c|}
        \hline
        \textbf{Model} & \textbf{Dataset} & \textbf{No-Reg} & \textbf{L1-Reg} & \textbf{L2-Reg} & \textbf{Dropout} & \textbf{CF-Reg (ours)} \\ \hline
        Logistic Regression   & \textit{Water}   & N/A   & $0.0093$   & $0.6927$  & N/A    & $0.3791/1.0355$                     \\ \hline
        MLP   & \textit{Water}   & N/A   & $0.0007$   & $0.0022$  & $0.0002$    & $0.2567/1.9775$                    \\ \hline
        Logistic Regression   &
        \textit{Phomene}   & N/A   & $0.0097$   & $0.7979$  & N/A    & $0.0571/1.8516$                     \\ \hline
        MLP   & \textit{Phomene}   & N/A   & $0.0007$   & $4.24\cdot10^{-5}$  & $0.0015$    & $0.0516/2.2700$                    \\ \hline
       % MLP   & \textit{Adult}   & N/A   & $0.0018$   & $0.0018$  & $0.0601$     & $0.0764/2.2068$                    \\ \hline
        CNN   & \textit{CIFAR-10} & N/A   & $0.0050$   & $0.0864$ & N/A    & $0.3018/
        2.1502$                     \\ \hline
    \end{tabular}
\end{table*}

\begin{table*}[htb!]
    \centering
    \caption{Mean value and standard deviation of training time across 5 different runs. The reported time (in seconds) corresponds to the generation of each entry in Table \ref{tab:regularization_comparison}. Times are }
    \label{tab:times}
    \begin{tabular}{|c|c|c|c|c|c|c|}
        \hline
        \textbf{Model} & \textbf{Dataset} & \textbf{No-Reg} & \textbf{L1-Reg} & \textbf{L2-Reg} & \textbf{Dropout} & \textbf{CF-Reg (ours)} \\ \hline
        Logistic Regression   & \textit{Water}   & $222.98 \pm 1.07$   & $239.94 \pm 2.59$   & $241.60 \pm 1.88$  & N/A    & $251.50 \pm 1.93$                     \\ \hline
        MLP   & \textit{Water}   & $225.71 \pm 3.85$   & $250.13 \pm 4.44$   & $255.78 \pm 2.38$  & $237.83 \pm 3.45$    & $266.48 \pm 3.46$                    \\ \hline
        Logistic Regression   & \textit{Phomene}   & $266.39 \pm 0.82$ & $367.52 \pm 6.85$   & $361.69 \pm 4.04$  & N/A   & $310.48 \pm 0.76$                    \\ \hline
        MLP   &
        \textit{Phomene} & $335.62 \pm 1.77$   & $390.86 \pm 2.11$   & $393.96 \pm 1.95$ & $363.51 \pm 5.07$    & $403.14 \pm 1.92$                     \\ \hline
       % MLP   & \textit{Adult}   & N/A   & $0.0018$   & $0.0018$  & $0.0601$     & $0.0764/2.2068$                    \\ \hline
        CNN   & \textit{CIFAR-10} & $370.09 \pm 0.18$   & $395.71 \pm 0.55$   & $401.38 \pm 0.16$ & N/A    & $1287.8 \pm 0.26$                     \\ \hline
    \end{tabular}
\end{table*}

\subsection{Feasibility of our Method}
A crucial requirement for any regularization technique is that it should impose minimal impact on the overall training process.
In this respect, CF-Reg introduces an overhead that depends on the time required to find the optimal counterfactual example for each training instance. 
As such, the more sophisticated the counterfactual generator model probed during training the higher would be the time required. However, a more advanced counterfactual generator might provide a more effective regularization. We discuss this trade-off in more details in Section~\ref{sec:discussion}.

Table~\ref{tab:times} presents the average training time ($\pm$ standard deviation) for each model and dataset combination listed in Table~\ref{tab:regularization_comparison}.
We can observe that the higher accuracy achieved by CF-Reg using the score-based counterfactual generator comes with only minimal overhead. However, when applied to deep neural networks with many hidden layers, such as \textit{PreactResNet-18}, the forward derivative computation required for the linearization of the network introduces a more noticeable computational cost, explaining the longer training times in the table.

\subsection{Hyperparameter Sensitivity Analysis}
The proposed counterfactual regularization technique relies on two key hyperparameters: $\alpha$ and $\beta$. The former is intrinsic to the loss formulation defined in (\ref{eq:cf-train}), while the latter is closely tied to the choice of the score-based counterfactual explanation method used.

Figure~\ref{fig:test_alpha_beta} illustrates how the test accuracy of an MLP trained on the \textit{Water Potability} dataset changes for different combinations of $\alpha$ and $\beta$.

\begin{figure}[ht]
    \centering
    \includegraphics[width=0.85\linewidth]{img/test_acc_alpha_beta.png}
    \caption{The test accuracy of an MLP trained on the \textit{Water Potability} dataset, evaluated while varying the weight of our counterfactual regularizer ($\alpha$) for different values of $\beta$.}
    \label{fig:test_alpha_beta}
\end{figure}

We observe that, for a fixed $\beta$, increasing the weight of our counterfactual regularizer ($\alpha$) can slightly improve test accuracy until a sudden drop is noticed for $\alpha > 0.1$.
This behavior was expected, as the impact of our penalty, like any regularization term, can be disruptive if not properly controlled.

Moreover, this finding further demonstrates that our regularization method, CF-Reg, is inherently data-driven. Therefore, it requires specific fine-tuning based on the combination of the model and dataset at hand.

\section{Conclusion}
In this work, we initiate the study of the delay-as-payoff model for contextual linear bandits and develop provable algorithms that require novel ideas compared to standard linear bandits.
Interesting future directions include proving matching regret lower bounds and extending our results to general payoff-dependent delays~\citep{lancewicki2021stochastic} and other even more challenging settings, such as those with intermediate observations~\citep{esposito2023delayed} or
evolving observations~\citep{bar2024non}.


\bibliography{ref}
\bibliographystyle{icml2025}


\newpage
\appendix
\onecolumn
\section{Omitted Details in \pref{sec: linear}}\label{app:loss}

\setcounter{AlgoLine}{0}
\begin{algorithm}
\caption{Phased Elimination via Volumetric Spanner for Linear Bandits with Delay-as-Loss with misspecification}\label{alg:lossLBmis}

\nl Input: maximum possible delay $D$, action set $\calA$, $\beta>0$, a misspecification level $\epsilon$. 

\nl Initialization: optimal loss guess $B=1/D$.

\nl Initialization: active action set $\calA_1=\calA$. \label{line: restart-mis}

 \For{$m=1,2,\dots,$}{
    \nl Find $\calS_m=\{a_{m,1},\dots,a_{m,|\calS_m|}\}$, a volumetric spanner of $\calA_m$ with $|\calS_m|= 3n$. \label{line:volume-mis}
    
    \nl Pick each $a\in \calS_m$ $2^m$ times in a round-robin way. \label{line:round-robin-mis}

    \nl Let $\calI_m$ contain all the rounds in this epoch.
    
    \nl For each $a\in \calS_m$, calculate the following quantities: \label{line:spanner-ucb-lcb-mis}
    {\small
    \begin{align}
        &\hat{\mu}_{m}^+(a)=\frac{1}{2^m}\Big(\sum_{\tau\in \obs_m(a)}u_{\tau} + \sum_{\tau\in \unobs_m(a)}1\Big), \label{eqn:mean-up-mis}\\
        &\hat{\mu}_{m}^-(a)=\frac{1}{2^m}\sum_{\tau\in \obs_m(a)}u_{\tau}, \label{eqn:mean-low-mis}\\
        &\hat{\mu}_{m,1}^{+}(a)=\hat{\mu}^{+}_{m}(a)+\frac{\beta}{2^{m/2}}\|a\|_2, \label{eqn:loss-ucb-linear-1-mis}\\
        &\hat{\mu}_{m,1}^{-}(a)=\hat{\mu}^{-}_{m}(a)-\frac{\beta}{2^{m/2}}\|a\|_2,\label{eqn:loss-lcb-linear-1-mis}\\
        &\hat{\mu}_m^{F}(a)=\frac{1}{\unbiasSize_m(a)}\sum_{\tau\in \unbias_m(a)}u_{\tau}, \label{eqn:mean_unbiased-mis}\\
        &\hat{\mu}_{m,2}^{+}(a)=\hat{\mu}_m^F(a)+\frac{\beta}{\sqrt{\unbiasSize_m(a)}}\|a\|_2, \label{eqn:loss-ucb-linear-2-mis}\\
        &\hat{\mu}_{m,2}^{-}(a)=\hat{\mu}_m^F(a)-\frac{\beta}{\sqrt{\unbiasSize_m(a)}}\|a\|_2, \label{eqn:loss-lcb-linear-2-mis}
    \end{align}
    }
    where $\unbiasSize_m(a) = |\unbias_m(a)|$, $\unbias_m(a) = \{\tau\in \calI_m: \tau+D\in\calI_m, a_{\tau}=a\}$, $\obs_m(a) = \{\tau\in \calI_m: \tau+d_{\tau}\in\calI_m, a_{\tau}=a\}$, and
    $\unobs_m(a)= \{\tau\in \calI_m: a_{\tau}=a\}\setminus\obs_m(a)$.

    \For{each $a\in \calA_m$}{
        \nl \label{line: decompose-mis}
        Decompose $a$ as $a=\sum_{i=1}^{|S_m|}\lambda_{m,i}^{(a)}a_{m,i}$ with $\|\lambda_{m}^{(a)}\|_2\leq 1$ and calculate 
        {\small
        \begin{align}
            &\UCB_{m}(a)=\sum_{i=1}^{|\calS_m|}\lambda_{m,i}^{(a)}\cdot\hat{\mu}_{m,2}^{\sgn(\lambda_{m,i}^{(a)})}(a_{m,i}), \label{eqn:loss-ucb-f-all-action-mis} \\
            &\LCB_m(a) = \max_{j\in \{1,2\}}\{\LCB_{m,j}(a)\} \;\;\text{where} \nonumber  \\
            & \LCB_{m,j}(a)=\sum_{i=1}^{|\calS_m|}\lambda_{m,i}^{(a)}\cdot\hat{\mu}_{m,j}^{\sgn(-\lambda_{m,i}^{(a)})}(a_{m,i}),\label{eqn:loss-lcb-all-action-mis}
        \end{align}
        }
    }
    
    \nl Set $\calA_{m+1} = \calA_m$.
    
    \For{$a\in \calA_m$}{
        \nl \label{line:eliminate-mis}  
        \If{$\exists a'\in \calA_m$, s.t. $\LCB_m(a) \geq \min\{\UCB_m(a'),B\} + 4\sqrt{3n}\epsilon$}
        {
          Eliminate $a$ from $\calA_{m+1}$.
        }
    }
    \nl \If{$\calA_{m+1}=\emptyset$}{
        Set $B\leftarrow 2B$ and go to \pref{line: restart-mis}.
    }
}
\end{algorithm}

In this section, we provide the detailed proof for \pref{thm:main-non-contextual}. Specifically, as mentioned in \pref{sec: contextual}, we prove the guarantee of a modified algorithm (\pref{alg:lossLBmis}) for the more general $\epsilon$-misspecified linear bandits. 

Recall that in misspecified linear bandits, $\mu_a = \inner{a,\theta}+\epsilon_a\in[0,1]$ with $|\epsilon_a|\leq\epsilon$ for all $a\in\calA$. Due to this difference, we clarify on the definitions of $\Delta_a$, $a^\star$, $\mu^\star$, $\Delta_{\min}$, $\Delta_{\max}$, and $d^\star$ in misspecified linear bandits as follows. We still define $\Delta_a = \inner{a^\star-a,\theta}$ as the suboptimality gap of action $a$, where $a^\star \in \argmin_{a\in\calA}\inner{a, \theta}$, but $\mu^\star \triangleq \min_{a\in\calA}\mu_a$ as the loss of the optimal action. Note that due to the misspecification, $\mu^\star$ may not necessarily be $\mu_{a^\star}$. Define $\Delta_{\min} = \min_{a\in \calA, \Delta_a>0}\Delta_a$ and $\Delta_{\max} = \max_{a\in \calA}\Delta_a$ to be the minimum non-zero, and maximum sub-optimality gap. The delay at round $t$ is still defined as $d_t=D\cdot u_t$ and $d^\star = D\cdot \mu^\star$ is the expected delay of the optimal action.

As for the algorithm, \pref{alg:lossLBmis} differs from \pref{alg:lossLB} only in \pref{line:eliminate-mis} where we add one misspecification term $4\sqrt{3n}\epsilon$ in the criteria of eliminating an action. 

The following theorem shows the guarantee of our algorithm in the misspecified linear bandits.

\begin{theorem}\label{thm:lossLBmis}
    \pref{alg:lossLBmis} with $\beta = \sqrt{2\log(KT^3)}$ guarantees that
    \begin{align*}
        \Reg &\leq \order\left(\min\left\{\frac{n^2\log(KT)\log(T/n)\log(d^\star)}{\Delta_{\min}},n\sqrt{T\log(d^\star)\log(KT)}\right\}+\epsilon\sqrt{n}T\right) \\
        &\qquad + \log(d^\star)\cdot \order\left( \min\left\{nd^\star\log (T/n)+D\Delta_{\max},D\Delta_{\max}\log (T/n)\right\}\right).
    \end{align*}
\end{theorem}

To prove \pref{thm:lossLBmis}, recall the following quantities
\begin{align}
    \wh{\mu}_{m}(a) &= \frac{1}{2^m}\sum_{\tau\in\obs_m(a)\cup\unobs_m(a)}u_{\tau},~~~\forall a\in \calS_m,\label{eqn:loss-all-mean-app}\\
    \hat{\mu}_{m,1}(a)&=\sum_{i=1}^{|\calS_m|}\lambda_{m,i}^{(a)}\cdot\hat{\mu}_{m}(a_{m,i}),~~~\forall a\in \calA_m,\\
    \hat{\mu}_{m,2}(a)&=\sum_{i=1}^{|\calS_m|}\lambda_{m,i}^{(a)}\cdot\hat{\mu}_{m}^{F}(a_{m,i}),~~~\forall a\in \calA_m.
\end{align}
We then define the following event and show that the event holds with high probability.

\begin{event}\label{event:misLoss}
    For all action $a\in \calA_m$, $m\in[T]$,
    \begin{align}
        \left|\inner{a,\theta}-\hat{\mu}_{m,1}(a)\right|&\leq \sqrt{|\calS_m|}\epsilon + \beta\sum_{i=1}^{|\calS_m|}\left|\lambda_{m,i}^{(a)}\right|\sqrt{\frac{1}{2^{m}}}, \label{eqn:concentr-1}\\
        \left|\inner{a,\theta}-\hat{\mu}_{m,2}(a)\right|&\leq \sqrt{|\calS_m|}\epsilon +\beta \sum_{i=1}^{|\calS_m|}\left|\lambda_{m,i}^{(a)}\right|\sqrt{\frac{1}{\unbiasSize_m(a_{m,i})}}, \label{eqn:concentr-2}\\
        |\unobs_m(a)| &\leq \frac{2D\mu_a}{|\calS_m|}+16\log KT+2,\label{eqn:concentr-3}
    \end{align}
    where $\beta = \sqrt{2\log KT^3}$.
\end{event}
\begin{lemma}\label{lem:high-prob-event}
    \pref{alg:lossLBmis} guarantees that \pref{event:misLoss} holds with probability at least $1-\frac{2}{T^2}$.
\end{lemma}
\begin{proof}
    Fix an action $a\in \calS_m$ in epoch $m\in[T]$. According to standard Azuma's inequality, we know that with probability at least $1-\delta$,
    \begin{align*}
        \left|\mu_a-\hat{\mu}_{m,1}(a)\right|&\leq \sqrt{\frac{2\log(2/\delta)}{2^m}}\|a\|_2,\\
        \left|\mu_a-\hat{\mu}_{m,2}(a)\right|&\leq \sqrt{\frac{2\log(2/\delta)}{\unbiasSize_m(a)}}\|a\|_2.
    \end{align*}
    Taking union bound over all possible $a\in \calA$ and all $m\in[T]$, we know that with probability at least $1-\delta$, for all $a\in \calS_m$ and all $m\in [T]$,
    \begin{align*}
        \left|\mu_a-\hat{\mu}_{m,1}(a)\right|&\leq \sqrt{\frac{2\log(2TK/\delta)}{n_t(a)}}\|a\|_2,\\
        \left|\mu_a-\hat{\mu}_{m,2}(a)\right|&\leq \sqrt{\frac{2\log(2TK/\delta)}{\unbiasSize_m(a)}}\|a\|_2.
    \end{align*}
    Then, given that the above equation holds, for $a\in \calA_m$, due to the property of volumetric spanners, we have $\mu_a = \inner{a,\theta^\star}+\epsilon_a =  \sum_{i=1}^{|\calS_m|}\lambda_{m,i}^{(a)}\inner{a_{m,i},\theta^\star}+ \epsilon_a$. Therefore, we can obtain that
    \begin{align*}
        \left|\inner{a,\theta}-\hat{\mu}_{m,1}(a)\right|
        &\leq \left|\sum_{i=1}^{|\calS_m|}\lambda_{m,i}^{(a)}(\inner{a_{m,i},\theta^\star}-\mu_{a_{m,i}})\right| + \sum_{i=1}^{|\calS_m|}\left|\lambda_{m,i}^{(a)}\right|\cdot\left|\mu_{a_{m,i}}-\hat{\mu}_{m}(a_{m,i})\right| \\
        &\leq \sum_{i=1}^{|\calS_m|}\left|\lambda_{m,i}^{(a)}\right|\left(\epsilon_{a_{m,i}}+\sqrt{\frac{2\log(2TK/\delta)}{2^{m}}}\right) \\
        &\leq \sqrt{|\calS_m|}\epsilon + \sum_{i=1}^{|\calS_m|}\left|\lambda_{m,i}^{(a)}\right|\sqrt{\frac{2\log(2TK/\delta)}{2^{m}}},
    \end{align*}
    where the last inequality uses $\|\lambda_{m}^{(a)}\|_1\leq \sqrt{|\calS_m|}\cdot \|\lambda_{m}^{(a)}\|_2\leq \sqrt{|\calS_m|}$. A similar analysis proves \pref{eqn:concentr-2}. \pref{eqn:concentr-3} holds with probability at least $1-\frac{1}{T^2}$ according to Lemma 4.1 of \citep{schlisselberg2024delay}. Picking $\delta = \frac{1}{T^2}$ finishes the proof.
\end{proof}

The next lemma shows that if $B\geq \mu^\star$, then \pref{alg:lossLBmis} will not reach an empty active set.
\begin{lemma}\label{lem:end-of-B}
    Suppose that \pref{event:misLoss} holds. If $B\geq \mu^\star$, then $a^\star\in \calA_m$ for all $m$.
\end{lemma}
\begin{proof}
    Since \pref{event:misLoss} holds, we have, we know that for all $a\in\calA_m$, $\LCB_m(a)\leq \inner{a,\theta} + \sqrt{|\calS_m|}\epsilon $ and $\UCB_m(a)\geq \inner{a,\theta} - \sqrt{|\calS_m|}\epsilon$. If $B\geq \mu^\star$, then we have $a^\star$ never eliminated since for any $a\in \calA_m$
    \begin{align*}
        \LCB_{m}(a^\star) &\leq \inner{a^\star,\theta} + \epsilon\sqrt{|\calS_m|} \leq \mu^\star + \epsilon + \epsilon\sqrt{|\calS_m|} \leq \mu^\star + 2\epsilon\sqrt{|\calS_m|},\\
        \LCB_{m}(a^\star) &\leq \inner{a^\star,\theta} + \epsilon\sqrt{|\calS_m|} \leq \inner{a,\theta} + 2\epsilon\sqrt{|\calS_m|} \leq \UCB_m(a) + 4\epsilon\sqrt{|\calS_m|}.
    \end{align*}
    Therefore, $a^\star$ never satisfy the elimination condition.
\end{proof}

The following lemma shows that the regret within epoch $m$ can be well-controlled.

\begin{lemma}\label{lem:delta_1_loss_miss}
    Suppose that \pref{event:misLoss} holds. \pref{alg:lossLBmis} guarantees that if $a\in\calA$ is not eliminated at the end of epoch $m$ (meaning that $a\in \calA_{m+1}$), then 
    \begin{align*}
        2^m\cdot \Delta_a\leq 2^m\cdot 24\sqrt{n}\epsilon+\frac{256n\beta^2}{\Delta_a} + \frac{2D\Delta_a}{|\calS_m|}.
    \end{align*}
\end{lemma}
\begin{proof}
    For notational convenience, define $\rad_{m,a}^{N} = \frac{\beta}{\sqrt{2^m}}\|a\|_2$ and $\rad_{m,a}^{F} = \frac{\beta}{\sqrt{\unbiasSize_m(a)}}\|a\|_2$ for all $a\in \calS_m$. In addition, we also define $\rad_{m,a}^{N}$ and $\rad_{m,a}^{F}$ for $a\notin \calS_m$ as follows:
    \begin{align*}
        \rad_{m,a}^{N} &= \sum_{i=1}^{|\calS_m|}|\lambda_{m,i}^{(a)}|\cdot \rad_{m,a_{m,i}}^{N}, \\
        \rad_{m,a}^{F} &= \sum_{i=1}^{|\calS_m|}|\lambda_{m,i}^{(a)}|\cdot \rad_{m,a_{m,i}}^{F}.
    \end{align*}
    Since \pref{event:misLoss} holds, we know that for all $a\in\calA_m$, $\LCB_m(a)\leq \inner{a,\theta} + \sqrt{|\calS_m|}\epsilon$, $\UCB_m(a)\geq \inner{a,\theta} - \sqrt{|\calS_m|}\epsilon$. Moreover, as $\LCB_m(a)=\max\{\LCB_{m,1}(a),\LCB_{m,2}(a)\}$, we know that for all $a\in \calA_m$
    \begin{align*}
        \LCB_{m,1}(a) + 2\rad_{m,a}^{N} + 2\epsilon\sqrt{|\calS_m|}\geq  \hat{\mu}_{m,1}(a) + \rad_{m,a}^{N} + 2\epsilon\sqrt{|\calS_m|}\geq \inner{a,\theta},\\
        \LCB_{m,2}(a) + 2\rad_{m,a}^{F} + 2\epsilon\sqrt{|\calS_m|}\geq  \hat{\mu}_{m,2}(a) + \rad_{m,a}^{F} + 2\epsilon\sqrt{|\calS_m|}\geq \inner{a,\theta},\\
        \UCB_{m}(a) - 2\rad_{m,a}^{F} - 2\epsilon\sqrt{|\calS_m|} = \hat{\mu}_{m,2}(a) - \rad_{m,a}^{F}-2\epsilon\sqrt{|\calS_m|}\leq \inner{a,\theta}.        
    \end{align*}
    If $B\geq \mu^\star$, then $a^\star\in \calA_m$ according to \pref{lem:end-of-B}.
    Moreover, if $a$ is not eliminated in epoch $m$, we have $\LCB(a)\leq \min\{\UCB_m(a^\star),B\}+4\sqrt{|S_m|}\epsilon$, meaning that
    \begin{align*}
        &\inner{a,\theta} - 2\rad_{m,a}^{F} - 2\epsilon\sqrt{|\calS_m|} \\
        &\leq \wh{\mu}_{m,2}(a) - \rad_{m,a}^{F} \\
        &\leq \LCB_m(a) \\
        &\leq \min\{\UCB_m(a^\star),B\}+4\sqrt{|S_m|}\epsilon \\
        &\leq \UCB_m(a^\star) + 4\sqrt{|S_m|}\epsilon \\
        &= \wh{\mu}_{m,2}(a^\star) + \rad_{m,a^\star}^{F}+ 4\sqrt{|S_m|}\epsilon \\
        &\leq \inner{a^\star,\theta} + 2\rad_{m,a^\star}^{F} + 6\sqrt{|S_m|}\epsilon.
    \end{align*}
    Since $\rad_{m,a}^F = \sum_{i=1}^{|\calS_m|}|\lambda_{m,i}^{(a)}|\cdot \rad_{m,a_{m,i}}^{F}$ with $\|\lambda_{m}^{(a)}\|_2\leq 1$, we have that $\|\lambda_{m}^{(a)}\|_1\leq \sqrt{|\calS_m|}$ and
    \begin{align*}
        &\Delta_a\leq 4\sqrt{|\calS_m|}\left(\max_{a\in \calS_m}\rad_{m,a}^{F}+2\epsilon\right)= 4\sqrt{3n}\max_{a\in \calS_m}\rad_{m,a}^{F}+8\sqrt{3n}\epsilon \leq \frac{8\sqrt{n}\beta}{\min_{a'\in \calS_m}\sqrt{\unbiasSize_m(a')}}+16\sqrt{n}\epsilon.
    \end{align*}
    
    If $B\leq \mu^\star$, then we have
    \begin{align*}
        \inner{a^\star,\theta}+\epsilon\geq \mu^\star\geq B \geq \LCB_{m}(a) - 4\sqrt{|\calS_m|}\epsilon \geq \inner{a,\theta} - 2\rad_{m,a}^{F} - 5\sqrt{|\calS_m|}\epsilon,
    \end{align*}
    where the second inequality is because $a$ is not eliminated in epoch $m$. Therefore, we always have
    \begin{align*}
        \Delta_a &\leq 2\rad_{m,a}^{F} + 6\sqrt{|\calS_m|}\epsilon \leq \frac{8\sqrt{n}\beta}{\min_{a'\in \calS_m}\sqrt{\unbiasSize_m(a')}} + 12\sqrt{n}\epsilon.
    \end{align*}
    In addition, we know that for all $a\in \calS_m$,
    \begin{align*}
        2^m \leq \unbiasSize_m(a) + \frac{D}{|\calS_m|} + 1 \leq \unbiasSize_m(a) + \frac{2D}{|\calS_m|}.
    \end{align*}
    Therefore, if $12\sqrt{n}\epsilon\geq \frac{\Delta_a}{2}$, then we have
    \begin{align*}
        2^m\Delta_a\leq 2^m\cdot 24\sqrt{n}\epsilon;
    \end{align*}
    otherwise, we have $\Delta_a \leq \frac{8\sqrt{n}\beta}{\min_{a\in \calS_m}\sqrt{\unbiasSize_m(a)}} + 12\sqrt{n}\epsilon \leq \frac{8\sqrt{n}\beta}{\min_{a\in \calS_m}\sqrt{\unbiasSize_m(a)}}  + \frac{\Delta_a}{2}$ and
    \begin{align*}
        \Delta_a \leq \frac{16\sqrt{n}\beta}{\min_{a'\in \calS_m}\sqrt{\unbiasSize_m(a')}},
    \end{align*}
    and we can obtain that
    \begin{align*}
        \min_{a'\in \calS_m}{\unbiasSize_m(a')}\cdot \Delta_a\leq \frac{256d\beta^2}{\Delta_a}.
    \end{align*}
    Combining the above two cases, we know that for all $a\in\calA_m$, $$2^m\cdot \Delta_a\leq 2^m\cdot 24\sqrt{n}\epsilon+ \min_{a'\in \calS_m}\unbiasSize_m(a')\cdot \Delta_a + \frac{2D\Delta_a}{|\calS_m|} \leq  2^m\cdot 24\sqrt{n}\epsilon+\frac{256n\beta^2}{\Delta_a} + \frac{2D\Delta_a}{|\calS_m|}.$$
\end{proof}

In fact, the bound above can be obtained by only using $\LCB_{m,1}$. Next, we provide yet-another regret bound within epoch $m$, which utilizes $\LCB_{m,2}$.

\begin{lemma}\label{lem:epoch_B_with_mis}
    \pref{alg:lossLBmis} guarantees that under \pref{event:misLoss}, if action $a$ is not eliminated at the end of epoch $m$ (meaning that $a\in \calA_{m+1}$), then
    \begin{align*}
    \inner{a,\theta}\leq B +\rad_{m,a}^{N}+ \sum_{i=1}^{|\calS_m|}|\lambda_{m,i}^{(a)}|\cdot\left(\frac{2D\mu_{a_{m,i}}}{2^m|\calS_m|}+\frac{16\log T +2}{2^m}\right) + 8\sqrt{|\calS_m|}\epsilon.
\end{align*}
\end{lemma}
\begin{proof}
For all $a\in \calS_m$, since $u_{t}\in[0,1]$, we know that
\begin{align}
    \hat{\mu}_{m}^+(a) &=\frac{1}{2^m}\left(\sum_{\tau\in \obs_m(a)}u_{\tau} + \sum_{\tau\in \unobs_m(a)}1\right) \leq \hat{\mu}_{m,a} + \frac{|\unobs_m(a)|}{2^m}, \label{eqn:pos-bias}\\
    \hat{\mu}_{m}^-(a) &=\frac{1}{2^m}\left(\sum_{\tau\in \obs_m(a)}u_{\tau} \right) \geq \hat{\mu}_{m,a} - \frac{|\unobs_m(a)|}{2^m} \label{eqn:neg-bias}.
\end{align}
Then, under \pref{event:misLoss}, we know that for all $a\in \calA_m$,
\begin{align*}
    \inner{a,\theta} &= \sum_{i=1}^{|\calS_m|}\lambda_{m,i}^{(a)}\inner{a_{m,i},\theta^\star}\\
    &= \sum_{i=1}^{|\calS_m|}\lambda_{m,i}^{(a)}(\mu_{a_{m,i}}-\epsilon_{a_{m,i}}) \tag{since $\mu_a = \inner{a,\theta^\star}+\epsilon_a$}\\
    &\leq \sum_{i=1}^{|\calS_m|}\lambda_{m,i}^{(a)}\cdot \mu_{a_{m,i}} + \sqrt{|\calS_m|}\epsilon \tag{since $\|\lambda_{m}^{(a)}\|_1\leq \sqrt{|\calS_m|}$} \\
    &\leq \sum_{i=1}^{|\calS_m|}\lambda_{m,i}^{(a)}\cdot \hat{\mu}_{m}(a_{m,i}) + \rad_{m,a}^{N} + 3\sqrt{|\calS_m|}\epsilon \tag{since \pref{event:misLoss} holds}\\
    &\leq \sum_{i=1}^{|\calS_m|}\lambda_{m,i}^{(a)}\cdot\hat{\mu}_{m}^{sgn(-\lambda_{m,i}^{(a)})}(a_{m,i}) + \rad_{m,a}^{N}+\sum_{i=1}^{|\calS_m|}|\lambda_{m,i}^{(a)}|\cdot \frac{|\unobs_m(a_{m,i})|}{2^m} + 3\sqrt{|\calS_m|}\epsilon \tag{using \pref{eqn:pos-bias} and \pref{eqn:neg-bias}}\\
    &= \LCB_{m,1}(a) + \rad_{m,a}^{N}+\sum_{i=1}^{|\calS_m|}|\lambda_{m,i}^{(a)}|\cdot \frac{|\unobs_m(a_{m,i})|}{2^m} + 3\sqrt{|\calS_m|}\epsilon\\
    &\leq \LCB_{m,1}(a) +\rad_{m,a}^{N}+ \sum_{i=1}^{|\calS_m|}|\lambda_{m,i}^{(a)}|\cdot\left(\frac{2D\mu_{a_{m,i}}}{2^m|\calS_m|}+\frac{16\log KT +2}{2^m}\right) + 3\sqrt{|\calS_m|}\epsilon. \tag{since \pref{event:misLoss} holds}
\end{align*}
Since $\LCB_{m,1}(a)\leq B+4\sqrt{|\calS_m|}\epsilon$ (as $a$ is not eliminated at the end of epoch $m$), we have
\begin{align*}
    \inner{a,\theta}\leq B +\rad_{m,a}^{N}+ \sum_{i=1}^{|\calS_m|}|\lambda_{m,i}^{(a)}|\cdot\left(\frac{2D\mu_{a_{m,i}}}{2^m|\calS_m|}+\frac{16\log T +2}{2^m}\right) + 8\sqrt{|\calS_m|}\epsilon.
\end{align*}
\end{proof}

\begin{lemma}\label{lem:bound_2_mis}
    If \pref{event:misLoss} holds, \pref{alg:lossLBmis} guarantees that if $a$ is not eliminated at the end of epoch $m$, then we also have
    \begin{align*}
        2^m\Delta_a\leq \frac{256n\beta^2}{\Delta_a} +\frac{4DB + 12\sum_{i=1}^{|\calS_m|}|\lambda_{m,i}^{(a)}|\cdot D\mu_{a_{m,i}}}{|\calS_m|}+(128\log T +16)\sqrt{n}+2^m\cdot 64\sqrt{n}\epsilon.
    \end{align*}
\end{lemma}
\begin{proof}
    If $\inner{a,\theta}\leq 2B$, we know that $\Delta_a = \inner{a-a^\star,\theta} \leq 2B$. Using \pref{lem:delta_1_loss_miss}, we can obtain that
    \begin{align*}
        2^m\cdot \Delta_a &\leq 2^m\cdot 24\sqrt{n}\epsilon+\frac{256n\beta^2}{\Delta_a} + \frac{2D\Delta_a}{|\calS_m|} \\
        &\leq 2^m\cdot 24\sqrt{n}\epsilon+\frac{256n\beta^2}{\Delta_a} + \frac{4DB}{|\calS_m|}
    \end{align*}
    If $\inner{a,\theta}\geq 2B$, we have $B\leq \frac{\inner{a,\theta}}{2}$. Using \pref{lem:epoch_B_with_mis}, we know that
    \begin{align*}
        \Delta_a &\leq \inner{a,\theta} \leq \underbrace{2\cdot \rad_{m,a}^{N}}_{\term{1}}+ \underbrace{2\sum_{i=1}^{|\calS_m|}|\lambda_{m,i}^{(a)}|\cdot\left(\frac{2D\mu_{a_{m,i}}}{2^m|\calS_m|}+\frac{16\log T +2}{2^m}\right) + 16\sqrt{|\calS_m|}\epsilon}_{\term{2}}.
    \end{align*}

    If $\term{1}\geq \term{2}$, we have
    \begin{align*}
        \Delta_a &\leq 4\rad_{m,a}^{N} \epsilon \leq 4\sqrt{|\calS_m|}\max_{a_m\in\calS_m}\rad_{m,a_m}^N \leq \frac{8\beta\sqrt{n}}{2^{m/2}},
    \end{align*}
    meaning that $2^m\Delta_a \leq \frac{64n\beta^2}{\Delta_a}$.
    Otherwise, we have
    \begin{align*}
        \Delta_a\leq 4\sum_{i=1}^{|\calS_m|}|\lambda_{m,i}^{(a)}|\cdot \left(\frac{2D\mu_{a_{m,i}}}{2^m|\calS_m|}+\frac{16\log T +2}{2^m}\right) + 64\sqrt{n}\epsilon,
    \end{align*}
    meaning that
    \begin{align*}
        2^m\Delta_a\leq \frac{8\sum_{i=1}^{|\calS_m|}|\lambda_{m,i}^{(a)}|\cdot D\mu_{a_{m,i}}}{|\calS_m|}+(128\log T +16)\sqrt{n}+2^m\cdot 64\sqrt{n}\epsilon.
    \end{align*}
    Combining both cases, we know that
    \begin{align*}
        2^m\Delta_a\leq \frac{256n\beta^2}{\Delta_a} +\frac{4DB + 12\sum_{i=1}^{|\calS_m|}|\lambda_{m,i}^{(a)}|\cdot D\mu_{a_{m,i}}}{|\calS_m|}+(128\log T +16)\sqrt{n}+2^m\cdot 64\sqrt{n}\epsilon.
    \end{align*}
\end{proof}

Now we are ready to prove our main result \pref{thm:lossLBmis}.
\begin{proof}[Proof of Theorem~\ref{thm:lossLBmis}]
    We analyze the regret when \pref{event:misLoss} holds, which happens with probability at least $1-\frac{2}{T^2}$. When \pref{event:misLoss} does not hold, the expected regret is bounded by $\frac{2}{T}$.
    
    We then bound the regret with a fixed choice of $B$. Combining \pref{lem:delta_1_loss_miss} and \pref{lem:epoch_B_with_mis}, if action $a$ is not eliminated at the end of epoch $m$, we have
    \begin{align*}
        2^{m}\cdot \Delta_a&\leq \frac{256n\beta^2}{\Delta_a} +\frac{4DB + 12\sum_{i=1}^{|\calS_{m}|}|\lambda_{m,i}^{(a)}|\cdot D\mu_{a_{m,i}}}{|\calS_m|}+(128\log T +16)\sqrt{n}+2^m\cdot 64\sqrt{n}\epsilon, \\
        2^m\cdot \Delta_a&\leq 2^m\cdot 24\sqrt{n}\epsilon+\frac{256n\beta^2}{\Delta_a} + \frac{2D\Delta_a}{|\calS_m|}.
    \end{align*}
    Therefore, we have
    \begin{align*}
        \Delta_a \leq \order\left(\frac{n\beta^2}{2^m\cdot \Delta_a} + \sqrt{n}\epsilon + \frac{\sqrt{n}\log T}{2^m}\right) +  \frac{1}{2^m}\min\left\{\frac{4DB+12\sum_{i=1}^{|\calS_m|}|\lambda_{m,i}^{(a)}|\cdot D\mu_{a_{m,i}}}{n}, \frac{D\Delta_a}{n}\right\}.
    \end{align*}
    Denote $\calT_B$ to be the number of rounds \pref{alg:lossLBmis} proceeds with $B$ and define $\Reg_B$ be the expected regret within $\calT_B$ rounds.
    Then, for any $\alpha_m\geq 0$,  the overall regret is then upper bounded as follows:
    \begin{align*}
        \Reg_B &\triangleq \sum_{m= 1}^{\lceil\log_2(|\calT_B|/3n\rceil}\sum_{a\in \calS_m}2^{m}\cdot\Delta_a \\
        &\leq \sum_{m=1}^{\lceil\log_2(|\calT_B|/3n\rceil}\sum_{a\in \calS_m}\mathbbm{1}\{\Delta_a> \alpha_m\}\left(\order\left(\frac{n\beta^2}{\Delta_a} + 2^m\sqrt{n}\epsilon + \sqrt{n}\log T\right) \right.\\
        &\qquad +\left.\min\left\{\frac{4DB+12\sum_{i=1}^{|\calS_{m-1}|}|\lambda_{m-1,i}^{(a)}|\cdot d(a_{m-1,i})}{n}, \frac{2D\Delta_a}{n}\right\}\right) \tag{since $a$ is not eliminated in epoch $m-1$ for all $a\in\calS_m$} \\
        &\qquad + \sum_{m\geq 1}\sum_{a\in \calS_m}\mathbbm{1}\{\Delta_a\leq \alpha_m\}2^m\Delta_a.
    \end{align*}
    Picking $\alpha_m = \beta\sqrt{\frac{n}{2^m}}$, we can obtain that
    \begin{align*}
        \Reg_B &=\sum_{m= 1}^{\lceil\log_2(|\calT_B|/3n\rceil}\sum_{a\in \calS_m}\left(\order\left(\beta\sqrt{n\cdot 2^m} +2^m\sqrt{n}\epsilon+ \sqrt{n}\log T \right) \right.\\
        &\qquad +\left.\min\left\{\frac{4DB+12\sum_{i=1}^{|\calS_{m-1}|}|\lambda_{m-1,i}^{(a)}|\cdot d(a_{m-1,i})}{n}, \frac{2D\Delta_a}{n}\right\}\right) \\
        &\leq \order\left(|\calT_B|\sqrt{n}\epsilon + \beta n\sqrt{|\calT_B|}+\sqrt{n}\log T\log(T/n)\right) \\
        &\qquad + \sum_{m= 1}^{\lceil\log_2(|\calT_B|/3n)\rceil}\sum_{a\in \calS_m}\min\left\{\frac{4DB+12\sum_{i=1}^{|\calS_{m-1}|}|\lambda_{m-1,i}^{(a)}|\cdot d(a_{m-1,i})}{n}, \frac{2D\Delta_a}{n}\right\}.
    \end{align*}
    
    On the other hand, picking $\alpha_m=0$, we have
    \begin{align*}
        \Reg_B &\leq \sum_{m=1}^{\lceil\log_2(|\calT_B|/3n\rceil}\sum_{a\in \calS_m}\left(\order\left(\frac{n\beta^2}{\Delta_{\min}} + 2^m\sqrt{n}\epsilon + \sqrt{n}\log T\right) \right.\\
        &\qquad +\left.\min\left\{\frac{4DB+12\sum_{i=1}^{|\calS_{m-1}|}|\lambda_{m-1,i}^{(a)}|\cdot d(a_{m-1,i})}{n}, \frac{2D\Delta_a}{n}\right\}\right) \\
        &\leq \order\left(\frac{n^2\beta^2\log(T/n)}{\Delta_{\min}}+\epsilon\sqrt{n}|\calT_B|+\sqrt{n}\log T\log(T/n)\right)\\
        &\qquad + \sum_{m= 1}^{\lceil\log_2(|\calT_B|/3n\rceil}\sum_{a\in \calS_m} \min\left\{\frac{4DB+12\sum_{i=1}^{|\calS_{m-1}|}|\lambda_{m-1,i}^{(a)}|\cdot d(a_{m-1,i})}{n}, \frac{2D\Delta_a}{n}\right\}.
    \end{align*}

    Using the fact that $\beta=\sqrt{2\log(KT^3)}$ and combining both bounds, we can obtain that
    \begin{align}
        \Reg_B &\leq \order\left(\min\left\{\frac{n^2\log(KT)\log(T/n)}{\Delta_{\min}}, n\sqrt{|\calT_B|\log(KT)}\right\}+\epsilon\sqrt{n}|\calT_B|\right)\nonumber\\
        &\qquad + \sum_{m= 1}^{\lceil\log_2(|\calT_B|/3n\rceil}\sum_{a\in \calS_m}\min\left\{\frac{4DB+12\sum_{i=1}^{|\calS_{m-1}|}|\lambda_{m-1,i}^{(a)}|\cdot d(a_{m-1,i})}{n}, \frac{2D\Delta_a}{n}\right\}. \label{eqn:reg_b}%
    \end{align}
    For notational convenience, let $R_B=\order\left(\min\left\{\frac{n^2\log(KT)\log(T/n)}{\Delta_{\min}},n\sqrt{|\calT_B|\log(KT)}\right\}+\epsilon\sqrt{n}|\calT_B|\right)$. To further analyze this bound, we first upper bound $\min\left\{\frac{4DB+12\sum_{i=1}^{|\calS_{m-1}|}|\lambda_{m-1,i}^{(a)}|\cdot d(a_{m-1,i})}{n}, \frac{2D\Delta_a}{n}\right\}$ by $\frac{2D\Delta_a}{n}$ and obtain that
    \begin{align}\label{eqn:reg_B_1}
        \Reg_B \leq R_B + \order\left(D\Delta_{\max}\log(T/n)\right).
    \end{align}

    On the other hand, we can also upper bound $\min\left\{\frac{4DB+12\sum_{i=1}^{|\calS_{m-1}|}|\lambda_{m-1,i}^{(a)}|\cdot d(a_{m-1,i})}{n}, \frac{2D\Delta_a}{n}\right\}$ by $\frac{12\sum_{i=1}^{|\calS_{m-1}|}|\lambda_{m-1,i}^{(a)}|\cdot d(a_{m-1,i})}{n}$ and obtain that
    \begin{align*}
        \Reg_B \leq R_B + \left(\sum_{m=1}^{\lceil\log_2(|\calT_B|/3n\rceil}\sum_{a\in\calS_m}\frac{4DB+12\sum_{i=1}^{|\calS_{m-1}|}|\lambda_{m-1,i}^{(a)}|\cdot d(a_{m-1,i})}{n}\right).
    \end{align*} 

    Let $L_{\Alg}^m=\sum_{a\in \calS_m}2^m\mu_a$ be the total expected loss within epoch $m$ and $L_{\star}^m=|\calS_m|\cdot 2^m\cdot\mu^\star$ be the total expected loss for the optimal action. Define $\Reg_m=L_{\Alg}^m-L_{\star}^m$. Direct calculation shows that
    \begin{align*}
        &\sum_{a\in \calS_m}\frac{\sum_{i=1}^{|\calS_{m-1}|}|\lambda_{m-1,i}^{(a)}|\cdot d(a_{m-1,i})}{n} \\
        &\leq \frac{3D}{2^{m-1}}\cdot 2^{m-1}\sum_{i=1}^{|\calS_{m-1}|}\mu_{a_{m-1,i}}\tag{since $|\lambda_{m-1,i}^{(a)}|\leq 1$ and $|\calS_m|=3n$}\\
        &= \frac{3D}{2^{m-1}}L_{\Alg}^{m-1}.
    \end{align*}
    Using the fact that $\Reg_B = \sum_{m=1}^{\lceil\log(|\calT_B|/3n)\rceil}\Reg_m$, we know that
    \begin{align*}
        &\sum_{m=1}^{\lceil\log(|\calT_B|/3n)\rceil} (L_{\Alg}^m-L^m_{\star})\\
        &\leq \sum_{m=1}^{\lceil\log(|\calT_B|/3n)\rceil}\Reg_m + 2\epsilon\cdot |\calT_B|\\
        &\leq R_B + \sum_{m=\lceil\log_2(72D)\rceil+1}^{\lceil\log(|\calT_B|/3n)\rceil}\frac{36D}{2^{m-1}}\cdot L_{\Alg}^{m-1} + \sum_{m=1}^{\lceil\log_2(72D)\rceil}2^m\Delta_{\max} + 12DB\log(T/n) \tag{$2\epsilon\cdot |\calT_B|$ is subsumed in $R_B$}\\
        &\leq R_B + \sum_{m=\lceil\log_2(72D)\rceil+1}^{\lceil\log(|\calT_B|/3n)\rceil}\frac{36D}{2^{m-1}}\cdot \left(L_{\Alg}^{m-1}-L_{\star}^{m-1}\right) + \sum_{m=\lceil\log_2(72D)\rceil+1}^{\lceil\log(|\calT_B|/3n)\rceil}\frac{36D}{2^{m-1}}\cdot L_{\star}^{m-1} + \sum_{m=1}^{\lceil\log_2(72D)\rceil}2^m\Delta_{\max} +12DB\log(T/n)\\
        &\leq R_B + \frac{1}{2}\sum_{m=\lceil\log_2(72D)\rceil+1}^{\lceil\log(|\calT_B|/3n)\rceil} \left(L_{\Alg}^{m-1}-L_{\star}^{m-1}\right) + 36nD\mu^\star\log(T/(216nD))+144D\Delta_{\max} +12DB\log(T/n)\\
        &= R_B + \frac{1}{2}\sum_{m=\lceil\log_2(72D)\rceil+1}^{\lceil\log(|\calT_B|/3n)\rceil} \left(L_{\Alg}^{m-1}-L_{\star}^{m-1}\right) + 36nd^\star\log(T/(216nD))+144D\Delta_{\max}+12DB\log(T/n).
    \end{align*}
    Rearranging the terms, we can obtain that
    \begin{align}\label{eqn:reg_B_2}
        \Reg_B \leq R_B + 72nd^\star\log(T/(216nD))+288D\Delta_{\max}+12DB\log(T/n).
    \end{align}
    Combining \pref{eqn:reg_B_1} and \pref{eqn:reg_B_2}, we know that
    \begin{align}
        \Reg_B &\leq \order\left(\min\left\{\frac{n^2\log(KT)\log(T/n)}{\Delta_{\min}},n\sqrt{|\calT_B|\log(KT)}\right\}+\epsilon\sqrt{n}|\calT_B|\right) \nonumber \\
        &\qquad +\order\left( \min\left\{nd^\star\log (T/nD)+D\Delta_{\max}+DB\log(T/n),D\Delta_{\max}\log (T/n)\right\}\right).\label{eqn:reg_B_final}
    \end{align}
    Finally, according to \pref{lem:end-of-B}, \pref{alg:lossLBmis} fails at most $\lceil\log_2(D\mu^\star))\rceil = \lceil\log_2(d^\star))\rceil$ times. Summing up the regret over all rounds, we know that the overall regret is bounded as follows
    \begin{align*}
        \Reg \leq \sum_{r=0}^{\lceil\log_2(d^\star))\rceil}\Reg_{2^r/D} &\leq \order\left(\min\left\{\frac{n^2\log(KT)\log(T/n)\log(d^\star)}{\Delta_{\min}},n\sqrt{T\log(d^\star)\log(KT)}\right\}+\epsilon\sqrt{n}T\right) \\
        &\qquad + \log(d^\star)\cdot \order\left( \min\left\{nd^\star\log (T/n)+D\Delta_{\max},D\Delta_{\max}\log (T/n)\right\}\right),
    \end{align*}
	which finishes the proof.
\end{proof}
\section{Omitted Details for Delay-as-Reward}\label{app: reward}
In this section, we show our results for the delay-as-reward setting. The difference compared with the delay-as-loss setting is that now, $\mu_a=\inner{a,\theta}+\epsilon_a\in[0,1]$ represents the expected reward of picking action $a$, where $|\epsilon_a|\leq \epsilon$ for all $a\in\calA$. The learner's goal is to minimize the pseudo regret defined as follows:
\begin{align}\label{eqn:reward-regret}
    \Reg\triangleq T\max_{a\in\calA}\inner{a,\theta} - \E\left[\sum_{t=1}^T\inner{a_t,\theta}\right].
\end{align}
 Define $\Delta_a = \inner{a^\star-a,\theta}$ as the suboptimality gap of action $a$, where $a^\star \in \argmax_{a\in\calA}\inner{a, \theta}$, and $\mu^\star \triangleq \max_{a\in\calA}\mu_a$ as the reward of the optimal action. Again, note that due to the misspecification, $\mu^\star$ may not necessarily be $\mu_{a^\star}$. Define $\Delta_{\min} = \min_{a\in \calA, \Delta_a>0}\Delta_a$ to be the minimum non-zero sub-optimality gap. The delay at round $t$ is still defined as $d_t=D\cdot u_t$, and $d^\star = D\cdot \mu^\star$ is  the expected delay of the optimal action. We also define $d(a)=D\mu_a$ to be the expected delay for action $a$.

\newpage
\subsection{Algorithm for Linear Bandits with Delay-as-Reward}
We list our algorithm for the reward case in \pref{alg:rewardLBmis} for completeness. The algorithm shares the same idea as \pref{alg:lossLBmis}.

\setcounter{AlgoLine}{0}
\begin{algorithm}[H]
\caption{Phased Elimination for Linear Bandits with Delay-as-Reward}\label{alg:rewardLBmis}

\nl Input: maximum possible delay $D$, action set $\calA$, $\beta>0$, a misspecification level $\epsilon$. 

\nl Initialize optimal reward guess $B=1$.

\nl Initialize active action set $\calA_1=\calA$.  \label{line:reward-restart} 

\nl \For{$m=1,2,\dots,$}{
    \nl Find $\calS_m=\{a_{m,1},\dots,a_{m,|\calS_m|}\}$ to be the volumetric spanner of $\calA_m$, where $|\calS_m|= 3n$. \label{line:volume-reward}
    
    \nl Pick each $a\in \calS_m$ $2^m$ times in a round-robin way. \label{line:round-robin-reward}

    \nl Let $\calI_m$ contain all the rounds in this epoch.
    
    \nl For all $a\in \calS_m$, calculate the following quantities
    \begin{align}
        &\hat{\mu}_{m}^+(a)=\frac{1}{2^m}\Big(\sum_{\tau\in \obs_m(a)}u_{\tau} + \sum_{\tau\in \unobs_m(a)}1\Big), \\
        &\hat{\mu}_{m}^-(a)=\frac{1}{2^m}\sum_{\tau\in \obs_m(a)}u_{\tau}, \\
        &\hat{\mu}_{m,1}^{+}(a)=\hat{\mu}^{+}_{m}(a)+\frac{\beta}{2^{m/2}}\|a\|_2, \label{eqn:reward-ucb-linear-1-mis}\\
        &\hat{\mu}_{m,1}^{-}(a)=\hat{\mu}^{-}_{m}(a)-\frac{\beta}{2^{m/2}}\|a\|_2,\label{eqn:reward-lcb-linear-1-mis}\\
        &\hat{\mu}_m^{F}(a)=\frac{1}{\unbiasSize_m(a)}\sum_{\tau\in \unbias_m(a)}u_{\tau},\\
        &\hat{\mu}_{m,2}^{+}(a)=\hat{\mu}_m^F(a)+\frac{\beta}{\sqrt{\unbiasSize_m(a)}}\|a\|_2, \label{eqn:reward-ucb-linear-2-mis}\\
        &\hat{\mu}_{m,2}^{-}(a)=\hat{\mu}_m^F(a)-\frac{\beta}{\sqrt{\unbiasSize_m(a)}}\|a\|_2, \label{eqn:reward-lcb-linear-2-mis}
    \end{align}
    where $\unbiasSize_m(a) = |\unbias_m(a)|$, $\unbias_m(a) = \{\tau\in \calI_m: \tau+D\in\calI_m, a_{\tau}=a\}$, $\obs_m(a) = \{\tau\in \calI_m: \tau+d_{\tau}\in\calI_m, a_{\tau}=a\}$, and
    $\unobs_m(a)= \{\tau\in \calI_m: a_{\tau}=a\}\setminus\obs_m(a)$.

    \nl \For{each $a\in \calA_m$}{
        \nl \label{line: decompose-rewward}
        Decompose $a$ as $a=\sum_{i=1}^{|S_m|}\lambda_{m,i}^{(a)}a_{m,i}$ with $\|\lambda_{m}^{(a)}\|_2\leq 1$ and calculate 
        {\small
        \begin{align}
            &\LCB_{m}(a)=\sum_{i=1}^{|\calS_m|}\lambda_{m,i}^{(a)}\cdot\hat{\mu}_{m,2}^{-\sgn(\lambda_{m,i}^{(a)})}(a_{m,i}), \label{eqn:reward-ucb-f-all-action-mis} \\
            &\UCB_m(a) = \max_{j\in \{1,2\}}\{\UCB_{m,j}(a)\} \;\;\text{where} \nonumber  \\
            & \UCB_{m,j}(a)=\sum_{i=1}^{|\calS_m|}\lambda_{m,i}^{(a)}\cdot\hat{\mu}_{m,j}^{\sgn(\lambda_{m,i}^{(a)})}(a_{m,i}),\label{eqn:reward-lcb-f-all-action-mis}
        \end{align}
        }
    }
    
    \nl Set $\calA_{m+1} = \calA_m$.
    
    \nl \For{$a_1\in \calA_m$}{
        \nl \label{line:reward_eliminate_miss}\If{$\exists a_2\in \calA_m$, such that $\max\{\LCB_m(a_2),B\} \geq \UCB_m(a_1)+4\sqrt{3n}\epsilon $}
        {
         \nl Eliminate $a_1$ from $\calA_{m+1}$.
        }
    }
    \nl \If{$\calA_{m+1}$ is empty}{
        Set $B\leftarrow B/2$ and go to \pref{line:reward-restart}.
    }
}
\end{algorithm}

\subsection{Regret Guarantees}\label{app: reward-regret}
In this section, we show the theoretical guarantees for our algorithm in the delay-as-reward setting.

\begin{theorem}\label{thm:reward-main}
    \pref{alg:rewardLBmis} with $\beta = \sqrt{2\log(KT^3)}$ guarantees that
    \begin{align*}
        \Reg &\leq \order\left(\min\left\{\frac{n^2\log(KT)\log(T/n)\log(1/\mu^\star)}{\Delta_{\min}}, n\sqrt{T\log(KT)\log(1/\mu^\star)}\right\}+\epsilon\sqrt{n}T\right)\\
        &\qquad+\order\left(\min\left\{\sum_{j=0}^{\lceil\log_2(1/\mu^\star)\rceil}\sum_{m=1}^{\lceil\log_2(|\calT_{2^{-j}}|/3n\rceil}\sum_{i=1}^{3n} d(a_{m-1,i}^{(2^{-j})}), D\Delta_{\max}\log(1/\mu^\star)\log(T/n)\right\}\right),
    \end{align*}
    where $\{a_{m,i}^{(B)}\}_{i=1}^{3n}$ represents the set of volumetric spanner at epoch $m$ with the optimal reward guess $B$. 
\end{theorem}

Similar to the analysis in \pref{app:loss}, our analysis is based on the condition that \pref{event:misLoss} holds, which happens with probability $1-\frac{2}{T^2}$ according to \pref{lem:high-prob-event}. The following lemma is a counterpart of \pref{lem:end-of-B}, providing an upper bound of the number of guesses on the optimal reward $B$.

\begin{lemma}\label{lem:end-of-B-reward}
    Suppose that \pref{event:misLoss} holds. If $B\leq \mu^\star$, then $a^\star\in \calA_m$ for all $m$.
\end{lemma}
\begin{proof}
    Since \pref{event:misLoss} holds, we have, we know that for all $a\in\calA_m$, $\UCB_m(a)+\sqrt{|\calS_m|}\epsilon\geq \inner{a,\theta}$, $\LCB_m(a)+\sqrt{|\calS_m|}\epsilon\leq \inner{a,\theta}$
    If $B\leq \mu^\star$, then we have $a^\star$ never eliminated since for any $a\in\calA_m$,
    \begin{align*}
         \UCB_{m}(a^\star) +2\epsilon\sqrt{|\calS_m|} &\geq \max_{a\in\calA}\{\inner{a,\theta}+\epsilon_a\} \geq \mu^\star \geq B,\\
         \UCB_{m}(a^\star) +4\epsilon\sqrt{|\calS_m|} &\geq \mu^\star + 2\epsilon\sqrt{|\calS_m|} \geq \inner{a,\theta}  + \epsilon\sqrt{|\calS_m|}\geq \LCB_m(a).
    \end{align*}
    Therefore, $a^\star$ never satisfy the elimination condition.
\end{proof}

The following lemma is a counterpart of \pref{lem:delta_1_loss_miss}.

\begin{lemma}\label{lem:delta_1_reward_miss}
    Suppose that \pref{event:misLoss} holds. \pref{alg:rewardLBmis} guarantees that if $a\in\calA$ is not eliminated at the end of epoch $m$ (meaning that $a\in \calA_{m+1}$), then 
    \begin{align*}
        2^m\cdot \Delta_a\leq 2^m\cdot 24\sqrt{n}\epsilon+\frac{256n\beta^2}{\Delta_a} + \frac{2D\Delta_a}{|\calS_m|}.
    \end{align*}
\end{lemma}
\begin{proof}
    Since \pref{event:misLoss} holds, we know that for all $a\in\calA_m$, $\LCB_m(a)\leq \mu_a + \sqrt{|\calS_m|}\epsilon$, $\UCB_m(a)\geq \mu_a - \sqrt{|\calS_m|}\epsilon$. Moreover, as $\UCB_m(a)=\min\{\UCB_{m,1}(a),\UCB_{m,2}(a)\}$, we know that for all $a\in \calA_m$
    \begin{align*}
        \UCB_{m,1}(a) - 2\rad_{m,a}^{N} - 2\epsilon\sqrt{|\calS_m|}=  \hat{\mu}_{m,1}(a) - \rad_{m,a}^{N} - 2\epsilon\sqrt{|\calS_m|}\leq \inner{a,\theta},\\
        \UCB_{m,2}(a) - 2\rad_{m,a}^{F} - 2\epsilon\sqrt{|\calS_m|}=  \hat{\mu}_{m,2}(a) - \rad_{m,a}^{F} - 2\epsilon\sqrt{|\calS_m|}\leq \inner{a,\theta},\\
        \LCB_{m}(a) + 2\rad_{m,a}^{F} + 2\epsilon\sqrt{|\calS_m|} = \hat{\mu}_{m,2}(a) + \rad_{m,a}^{F}+2\epsilon\sqrt{|\calS_m|}\geq \inner{a,\theta}.        
    \end{align*}
    If $B\leq \mu^\star$, then $a^\star\in \calA_m$ according to \pref{lem:end-of-B-reward}.
    Moreover, if $a$ is not eliminated in epoch $m$, we have $\UCB_m(a)+4\sqrt{|S_m|}\epsilon\geq \max\{\LCB_m(a^\star),B\}$, meaning that
    \begin{align*}
        &\inner{a,\theta} + 2\rad_{m,a}^{F} + 2\epsilon\sqrt{|\calS_m|} \\
        &\geq \wh{\mu}_{m,2}(a) + \rad_{m,a}^{F} \\
        &\geq \UCB_m(a) \\
        &\geq \max\{\LCB_m(a^\star),B\}-4\sqrt{|S_m|}\epsilon \\
        &\geq \LCB_m(a^\star) - 4\sqrt{|S_m|}\epsilon \\
        &= \wh{\mu}_{m,2}(a^\star) - \rad_{m,a^\star}^{F} - 4\sqrt{|S_m|}\epsilon \\
        &\geq \inner{a^\star,\theta}  - 2\rad_{m,a^\star}^{F} - 6\sqrt{|S_m|}\epsilon.
    \end{align*}
    Since $\rad_{m,a}^F = \sum_{i=1}^{|\calS_m|}|\lambda_{m,i}^{(a)}|\cdot \rad_{m,a_{m,i}}^{F}$ with $\|\lambda_{m}^{(a)}\|_2\leq 1$, we have that $\|\lambda_{m}^{(a)}\|_1\leq \sqrt{|\calS_m|}$ and
    \begin{align*}
        &\Delta_a\leq 4\sqrt{|\calS_m|}\left(\max_{a\in S_m}\rad_{m,a}^{F}+2\epsilon\right)= 4\sqrt{3n}\max_{a\in S_m}\rad_{m,a}^{F}+8\sqrt{3n}\epsilon \leq \frac{8\sqrt{n}\beta}{\min_{a'\in \calS_m}\sqrt{\unbiasSize_m(a')}}+16\sqrt{n}\epsilon.
    \end{align*}
    
    If $B\geq \mu^\star$, then we have
    \begin{align*}
        \mu^\star\leq B \leq \UCB_{m}(a) + 4\sqrt{|\calS_m|}\epsilon \leq \mu_a + 2\rad_{m,a}^{F} + 6\sqrt{|\calS_m|}\epsilon,
    \end{align*}
    where the second inequality is because $a$ is not eliminated in epoch $m$. Therefore, we always have
    \begin{align*}
        \Delta_a &\leq 2\rad_{m,a}^{F} + 6\sqrt{|\calS_m|}\epsilon \leq \frac{8\sqrt{n}\beta}{\min_{a'\in \calS_m}\sqrt{\unbiasSize_m(a')}} + 12\sqrt{n}\epsilon.
    \end{align*} 
    In addition, we know that for all $a\in \calS_m$,
    \begin{align*}
        2^m &= |\calS_m| \leq \unbiasSize_m(a) + \frac{D}{|S_m|} + 1 \leq \unbiasSize_m(a) + \frac{2D}{|S_m|}.
    \end{align*}
    Therefore, if $12\sqrt{n}\epsilon\geq \frac{\Delta_a}{2}$, then we have
    \begin{align*}
        2^m\Delta_a\leq 2^m\cdot 24\sqrt{n}\epsilon;
    \end{align*}
    otherwise, we have $\Delta_a \leq \frac{8\sqrt{n}\beta}{\min_{a\in \calS_m}\sqrt{\unbiasSize_m(a)}} + 12\sqrt{n}\epsilon \leq \frac{8\sqrt{n}\beta}{\min_{a\in S_m}\sqrt{\unbiasSize_m(a)}}  + \frac{\Delta_a}{2}$ and
    \begin{align*}
        \Delta_a \leq \frac{16\sqrt{n}\beta}{\min_{a'\in \calS_m}\sqrt{\unbiasSize_m(a')}},
    \end{align*}
    and we can obtain that
    \begin{align*}
        \min_{a'\in S_m}{\unbiasSize_m(a')}\cdot \Delta_a\leq \frac{256d\beta^2}{\Delta_a}.
    \end{align*}
    Combining the above two cases, we know that for all $a\in\calA_m$, $$2^m\cdot \Delta_a\leq 2^m\cdot 24\sqrt{n}\epsilon+ \min_{a'\in \calS_m}\unbiasSize_m(a')\cdot \Delta_a + \frac{2D\Delta_a}{|\calS_m|} \leq  2^m\cdot 24\sqrt{n}\epsilon+\frac{256n\beta^2}{\Delta_a} + \frac{2D\Delta_a}{|\calS_m|}.$$
\end{proof}

The following lemma is a counterpart of \pref{lem:epoch_B_with_mis}. 

\begin{lemma}\label{lem:epoch_B_with_mis-reward}
    \pref{alg:rewardLBmis} guarantees that under \pref{event:misLoss}, if an action $a$ is eliminated at the end of epoch $m$ (meaning that $a\in \calA_m$), then
    \begin{align*}
    B\leq \inner{a,\theta} +\rad_{m,a}^{N}+ \sum_{i=1}^{|\calS_m|}|\lambda_{m,i}^{(a)}|\cdot\left(\frac{2d(a_{m,i})}{2^m|\calS_m|}+\frac{16\log T +2}{2^m}\right) + 8\sqrt{|\calS_m|}\epsilon,
\end{align*}
where $d(a)=D\mu_a$.
\end{lemma}
\begin{proof}
Under \pref{event:misLoss}, we know that for all $a\in \calA_m$,
\begin{align*}
    \inner{a,\theta} &= \sum_{i=1}^{|\calS_m|}\lambda_{m,i}^{(a)}\inner{a_{m,i},\theta^\star} \\
    &= \sum_{i=1}^{|\calS_m|}\lambda_{m,i}^{(a)}(\mu_{a_{m,i}}-\epsilon_{a_{m,i}}) \tag{since $\mu_a = \inner{a,\theta^\star}+\epsilon_a$}\\
    &\geq \sum_{i=1}^{|\calS_m|}\lambda_{m,i}^{(a)}\cdot \mu_{a_{m,i}} - \sqrt{|\calS_m|}\epsilon \tag{since $\|\lambda_{m}^{(a)}\|_1\leq \sqrt{|\calS_m|}$} \\
    &\geq \sum_{i=1}^{|\calS_m|}\lambda_{m,i}^{(a)}\cdot \hat{\mu}_{m}(a_{m,i}) - \rad_{m,a}^{N} - 3\sqrt{|\calS_m|}\epsilon \tag{since \pref{event:misLoss} holds}\\
    &\geq \sum_{i=1}^{|\calS_m|}\lambda_{m,i}^{(a)}\cdot\hat{\mu}_{m}^{sgn(\lambda_{m,i}^{(a)})}(a_{m,i}) - \rad_{m,a}^{N} -\sum_{i=1}^{|\calS_m|}|\lambda_{m,i}^{(a)}|\cdot \frac{|\unobs_m(a_{m,i})|}{2^m} - 3\sqrt{|\calS_m|}\epsilon \tag{using \pref{eqn:pos-bias} and \pref{eqn:neg-bias}}\\
    &= \UCB_{m,1}(a) - \rad_{m,a}^{N} -\sum_{i=1}^{|\calS_m|}|\lambda_{m,i}^{(a)}|\cdot \frac{|\unobs_m(a_{m,i})|}{2^m} - 3\sqrt{|\calS_m|}\epsilon\\
    &\geq \UCB_{m,1}(a) - \rad_{m,a}^{N} - \sum_{i=1}^{|\calS_m|}|\lambda_{m,i}^{(a)}|\cdot\left(\frac{2d(a_{m,i})}{2^m|\calS_m|}+\frac{16\log KT +2}{2^m}\right) - 4\sqrt{|\calS_m|}\epsilon. \tag{since \pref{event:misLoss} holds}
\end{align*}
Since $\UCB_{m,1}(a)\geq B - 4\sqrt{|\calS_m|}\epsilon$ (as $a$ is not eliminated at the end of epoch $m$), we have
\begin{align*}
    B\leq \inner{a,\theta} +\rad_{m,a}^{N}+ \sum_{i=1}^{|\calS_m|}|\lambda_{m,i}^{(a)}|\cdot\left(\frac{2d(a_{m,i})}{2^m|\calS_m|}+\frac{16\log T +2}{2^m}\right) + 8\sqrt{|\calS_m|}\epsilon.
\end{align*}
\end{proof}

The following lemma is a counterpart of \pref{lem:bound_2_mis}.

\begin{lemma}\label{lem:bound_2_mis_reward}
    If $B\geq \frac{\mu^\star}{2}$ and \pref{event:misLoss} holds, \pref{alg:rewardLBmis} guarantees that if $a$ is not eliminated at the end of epoch $m$, then we also have
    \begin{align*}
        2^m\Delta_a\leq \frac{256n\beta^2}{\Delta_a} +\frac{12\sum_{i=1}^{|\calS_m|}|\lambda_{m,i}^{(a)}|\cdot d(a_{m,i})}{|\calS_m|}+(128\log T +16)\sqrt{n}+2^m\cdot 64\sqrt{n}\epsilon,
    \end{align*}
    where $d(a)=D\mu_a$.
\end{lemma}
\begin{proof}
    If $\inner{a,\theta}\geq \frac{B}{2}$, we know that $\Delta_a = \inner{a^\star-a,\theta} \leq 3\inner{a,\theta}$. Using \pref{lem:delta_1_reward_miss}, we can obtain that
    \begin{align*}
        2^m\cdot \Delta_a &\leq 2^m\cdot 24\sqrt{n}\epsilon+\frac{256n\beta^2}{\Delta_a} + \frac{2D\inner{a,\theta}}{|\calS_m|} \\
        &\leq 2^m\cdot 24\sqrt{n}\epsilon+\frac{256n\beta^2}{\Delta_a} + \frac{2\sum_{i=1}^{|\calS_m|}|\lambda_{m,i}^{(a)}|\cdot d(a_{m,i})}{|\calS_m|}.
    \end{align*}
    If $\inner{a,\theta} \leq \frac{B}{2}$, we have $3(B-\inner{a,\theta} ) \geq \frac{3B}{2} \geq \inner{a^\star-a,\theta}$. Using \pref{lem:epoch_B_with_mis}, we know that
    \begin{align*}
        \Delta_a &\leq \mu_a \leq \underbrace{3\cdot \rad_{m,a}^{N}}_{\term{1}}+ \underbrace{3\sum_{i=1}^{|\calS_m|}|\lambda_{m,i}^{(a)}|\cdot\left(\frac{2d(a_{m,i})}{2^m|\calS_m|}+\frac{16\log T +2}{2^m}\right) + 24\sqrt{|\calS_m|}\epsilon}_{\term{2}}.
    \end{align*}

    If $\term{1}\geq \term{2}$, we have
    \begin{align*}
        \Delta_a &\leq \mu_a \leq 6\rad_{m,a}^{N} \epsilon \leq 6\sqrt{|\calS_m|}\max_{a_m\in\calS_m}\rad_{m,a_m}^N \leq \frac{12\beta\sqrt{n}}{2^{m/2}},
    \end{align*}
    meaning that $2^m\Delta_a \leq \frac{144n\beta^2}{\Delta_a}$.
    Otherwise, we have
    \begin{align*}
        \Delta_a\leq 6\sum_{i=1}^{|\calS_m|}|\lambda_{m,i}^{(a)}|\cdot \left(\frac{2d(a_{m,i})}{2^m|\calS_m|}+\frac{16\log T +2}{2^m}\right) + 96\sqrt{n}\epsilon,
    \end{align*}
    meaning that
    \begin{align*}
        2^m\Delta_a\leq \frac{12\sum_{i=1}^{|\calS_m|}|\lambda_{m,i}^{(a)}|\cdot d(a_{m,i})}{|\calS_m|}+(96\log T +12)\sqrt{n}+2^m\cdot 96\sqrt{n}\epsilon.
    \end{align*}
    Combining both cases, we know that
    \begin{align*}
        2^m\Delta_a\leq \frac{256n\beta^2}{\Delta_a} +\frac{12\sum_{i=1}^{|\calS_m|}|\lambda_{m,i}^{(a)}|\cdot d(a_{m,i})}{|\calS_m|}+(96\log T +12)\sqrt{n}+2^m\cdot 96\sqrt{n}\epsilon.
    \end{align*}
\end{proof}

Now we are ready to prove our main result \pref{thm:reward-main}.
\begin{proof}[Proof of Theorem~\ref{thm:reward-main}]
Combining \pref{lem:delta_1_reward_miss} and \pref{lem:bound_2_mis_reward} and following the exact same process of obtaining \pref{eqn:reg_b} in \pref{thm:lossLBmis}, we can obtain that for a fixed value of $B$, \pref{alg:rewardLBmis} guarantees that
    \begin{align*}
        \Reg_B &\leq \order\left(\min\left\{\frac{n^2\log(KT)\log(T/n)}{\Delta_{\min}}, n\sqrt{|\calT_B|\log(KT)}\right\}+\epsilon\sqrt{n}|\calT_B|\right)\\
        &\qquad + \sum_{m= 1}^{\lceil\log_2(|\calT_B|/3n\rceil}\sum_{a\in \calS_m}\order\left(\min\left\{\frac{\sum_{i=1}^{|\calS_{m-1}|}|\lambda_{m-1,i}^{(a)}|\cdot d(a_{m-1,i})}{n}, \frac{D\Delta_a}{n}\right\}\right) \\
        &\leq \order\Bigg(\min\left\{\frac{n^2\log(KT)\log(T/n)}{\Delta_{\min}}, n\sqrt{|\calT_B|\log(KT)}\right\}+\epsilon\sqrt{n}|\calT_B|\\
        &\qquad \left.+\min\left\{\sum_{m= 1}^{\lceil\log_2(|\calT_B|/3n\rceil}\sum_{i=1}^{|\calS_{m-1}|} d(a_{m-1,i}), D\Delta_{\max}\log(T/n)\right\}\right).
    \end{align*}
    According to \pref{lem:end-of-B-reward}, there are at most $\lceil\log_2(1/\mu^\star)\rceil$ different values of $B$. With an abuse of notation, we define $\calS_{m}^{(B)}=\{a_{m,i}^{(B)}\}_{i\in [3n]}$ to be the volumetric spanner at epoch $m$ with the reward guess $B$.
    Taking summation over all these values, we can obtain that
    \begin{align*}
        \Reg &\leq \order\left(\min\left\{\frac{n^2\log(KT)\log(T/n)\log(1/\mu^\star)}{\Delta_{\min}}, n\sqrt{T\log(KT)\log(1/\mu^\star)}\right\}+\epsilon\sqrt{n}T\right)\\
        &\qquad+\order\left(\min\left\{\sum_{j=0}^{\lceil\log_2(1/\mu^\star)\rceil}\sum_{m=1}^{\lceil\log_2(|\calT_{2^{-j}}|/3n\rceil}\sum_{i=1}^{3n} d(a_{m-1,i}^{(2^{-j})}), D\Delta_{\max}\log(1/\mu^\star)\log(T/n)\right\}\right),
    \end{align*}
    completing the proof.
\end{proof}    

    While we can further apply a similar analysis to the one in \pref{thm:lossLBmis} to bound the term $\sum_{j=0}^{\lceil\log_2(1/\mu^\star)\rceil}\sum_{m=1}^{\lceil\log_2(|\calT_{2^{-j}}|/3n\rceil}\sum_{i=1}^{3n} d(a_{m-1,i}^{(2^{-j})})$ and obtain a bound with respect to $d^\star$, since $d^\star\geq D\Delta_{\max}+\epsilon$, this $d^\star$ dependent bound does not provide a significantly better regret guarantee in the worst case. This  difference in loss versus reward is also pointed out in \citep{schlisselberg2024delay} in the MAB setting. We keep this term in the upper bound since this quantity can still be potentially smaller than $D\Delta_{\max}\log(1/\mu^\star)\log(T/n)$.




\section{Omitted Details in \pref{sec: contextual}}\label{app: contextual}
In this section, we provide the omitted details in \pref{sec: contextual}.
We start with the following lemma that is a standard application of the Azuma-Hoeffding's inequality.
\begin{lemma}[Proposition 2 in \citep{hanna2023contexts}]\label{lem:prop_two}
    For each epoch $m$, \pref{alg:reduction} guarantees that with probability at least $1-\frac{\delta}{T}$, the following holds:
    \begin{align*}
        \left|\inner{g(\theta),\theta'} - \inner{g^{(m)}(\theta),\theta'}\right| \leq 2\sqrt{\frac{\log(2T|\Theta'|/\delta)}{2^{m-1}}},~~\forall \theta,\theta'\in\Theta'.
    \end{align*}
\end{lemma}

Next, we provide the proof for \pref{thm:reduction}.
\begin{proof}[Proof of Theorem~\ref{thm:reduction}]
    Define $\theta_0 = \argmin_{\theta'\in \Theta'}\|\theta'-\theta\|_2$. 
    Following the analysis of \citet{hanna2023contexts}, we decompose the regret $\Reg_m$ within epoch $m$ as follows:
    \begin{align*}
        \Reg_m &=\E\left[\sum_{\tau=2^{m-1}+1}^{2^m}\left(\inner{\argmin_{a\in\calA_t}\inner{a,\theta_t},\theta} - \min_{a_\tau^\star\in\calA_\tau}\inner{a_{\tau}^\star,\theta}\right)\right] \\
        &\leq \E\left[\sum_{\tau=2^{m-1}+1}^{2^m}\left(\inner{\argmin_{a\in\calA_t}\inner{a,\theta_t},\theta_0} - \min_{a_\tau^\star\in\calA_\tau}\inner{a_{\tau}^\star,\theta_0}\right)\right] + \order\left(\frac{2^{m-1}}{T}\right)\\
        &= \E\left[\sum_{\tau=2^{m-1}+1}^{2^m}\inner{g(\theta_t)-g(\theta_0),\theta_0}\right] + \order\left(\frac{2^{m-1}}{T}\right)\\
        &= \E\underbrace{\left[\sum_{\tau=2^{m-1}+1}^{2^{m}}\inner{g(\theta_t)-g^{(m)}(\theta_t),\theta_0}\right]}_{\Err{1}}  + \E\underbrace{\left[\sum_{\tau=2^{m-1}+1}^{2^m}\inner{g^{(m)}(\theta_t)-g^{(m)}(\theta_0),\theta_0}\right]}_{\regnctx} \\
        &\qquad + \underbrace{\E\left[\sum_{\tau=2^{m-1}+1}^{2^m}\inner{g^{(m)}(\theta_0)-g(\theta_0),\theta_0}\right]}_{\Err{2}} +\order\left(\frac{2^{m-1}}{T}\right),
    \end{align*}
    where the second equality is because
    $\E\left[\min_{a\in\calA_t}\inner{a,\theta_0}\right] = \E\left[\inner{\argmin_{a\in\calA_t}\inner{a,\theta_0},\theta_0}\right] = \inner{g(\theta_0), \theta
    _0}$.
    
    For $\Err{1}$ and $\Err{2}$, we apply \pref{lem:prop_two} to bound both terms  by $\order\left(\sqrt{2^m\log(T|\Theta'|)}\right)$.
    As for $\regnctx$, this is in fact the regret of misspecified non-contextual linear bandits with action set $\calX_m$ and misspecification level $\max_{\theta'\in \Theta'}\left|\inner{g^{(m)}(\theta')-g(\theta'),\theta}\right|$, since $\E[u_t]=\inner{g(\theta_t),\theta}$ for all $t$. Applying \pref{lem:prop_two} again, we know that the misspecification is of order $\epsilon_m=\order(\sqrt{\log(T|\Theta'|)/2^m})$ with probability at least $1-\frac{1}{T^2}$. 
    Then, applying the regret guarantee of \pref{alg:lossLBmis} proven in \pref{thm:lossLBmis}, we know that
    \begin{align*}
        &\regnctx 
        \leq \order\left(\sqrt{n2^m\log(T|\Theta'|)}\right)
        +\order\left(\min\{V_{m,1},V_{m,2}\},\log(\overline{d}^\star)\min\{W_{m,1},W_{m,2}\}\right),
    \end{align*}
    where $V_{m,1}=\frac{n^2\log(T|\Theta'|)\log(T/n)\log(\overline{d}^\star)}{\Delta_{\min}^{\nctx}}$, $V_{m,2}=n\sqrt{2^m\log(\overline{d}^\star)\log(T|\Theta'|)}$, 
    $W_{m,1}=n\overline{d}^\star\log(T/n)+D\Delta_{\max}^{\nctx}$, and $W_{m,2}=D\Delta_{\max}^{\nctx}\log(T/n)$.
    Taking a summation over all $m\in[\lceil\log_2(T)\rceil]$ epochs and using the fact that $|\Theta'|\leq \order(T^n)$ finishes the proof.
\end{proof}

\section{Overhead of \ourSystem.}
We report the size and inference time of the model for NeRF and \ourSystem in Table~\ref{table_overhead}.  
The results indicate that \ourSystem has a larger model size than NeRF, \ie 27.1 \vs 8.0\,MB.  
Correspondingly, \ourSystem exhibits a longer inference time, \ie 1.79 \vs 0.43\,s.  
Unlike NeRF, \ourSystem requires neighboring spectra as input. 
During inference, the target transmitter's neighbors are extracted from the training dataset, so \ourSystem does not incur additional data burdens.  
Moreover, since \ourSystem can operate in unseen scenes, it significantly reduces the requirement for a time-consuming training process.



\begin{table}[h]
\centering
\caption{Comparison of model size and inference time.}

\begin{tabular}{lC{0.8in}C{0.8in}}
\toprule
     & \nerft    & \ourSystem    
     \\ \midrule
Model size (MB) & 8.0  & 27.1   \\
Inference time (s) & 0.43    & 1.79  \\
\bottomrule
\end{tabular}
\label{table_overhead}
\end{table}







\end{document}

