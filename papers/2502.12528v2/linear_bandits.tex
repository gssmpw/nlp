\section{First Step: Non-Contextual Linear Bandits}\label{sec: linear}

In this section, we focus on the non-contextual case, which serves as a building block for eventually solving the contextual case. Before introducing our algorithm, we first briefly introduce the successive arm elimination algorithm for the simpler MAB setting proposed by \citet{schlisselberg2024delay} and their ideas of handling payoff-dependent delay. Specifically, their algorithm starts with a guess $B=1/D$ on the optimal action's loss, and maintains an active set of arms. The algorithm pulls each arm in the active set once, and constructs two LCB's (lower confidence bound) and one UCB (upper confidence bound) for each action in the active set as follows (supposing the current round being $t$):
\begin{align}
        \LCB_{t,1}(a) &= \frac{1}{\cnt_t(a)}\sum_{\tau\in\obs_t(a)}u_\tau - \sqrt{\frac{2\log T}{\cnt_t(a)}}, \label{eqn:lcb-1-mab}\\
        \LCB_{t,2}(a) &= \frac{1}{\unbiasSize_t(a)}\sum_{\tau\in\unbias_t(a)}u_{\tau} - \sqrt{\frac{2\log T}{\unbiasSize_t(a)\vee 1}}, \label{eqn:lcb-2-mab}\\
        \UCB_{t}(a) &= \frac{1}{\unbiasSize_t(a)}\sum_{\tau\in\unbias_t(a)}u_{\tau} + \sqrt{\frac{2\log T}{\unbiasSize_t(a)\vee 1}},\label{eqn:ucb-mab}
\end{align}
where $\cnt_t(a) = \sum_{\tau=1}^t\mathbbm{1}\{a_t=a\}$ is the total number of pulls of action $a$ till round $t$, $\obs_t(a) = \{\tau: \tau+d_{\tau}\leq t \text{~and~} a_{\tau}=a\}$ is the set of rounds where action $a$ is chosen and its loss has been received by the end of round $t$, $\unbias_t(a) = \{\tau \leq t-D: a_\tau = a\}$ is the set of rounds up to $t-D$ where action $a$ is chosen (so its loss has for sure been received by the end of round $t$), 
and $\unbiasSize_t(a)=|\unbias_t(a)|$. Specifically, \pref{eqn:lcb-1-mab} constructs an LCB of action $a$ assuming all the action's unobserved loss to be $0$ (the smallest possible), while \pref{eqn:lcb-2-mab} and \pref{eqn:ucb-mab} construct an LCB and a UCB using only the losses no later than round $t-D$ (which must have been received by round $t$), making the empirical average a better estimate of the expected loss. With $\UCB_t(a)$ and $\LCB_t(a) = \max\{\LCB_{t,1}(a), \LCB_{t,2}(a)\}$ constructed, the algorithm eliminates an action $a$ if its $\LCB_t(a)$ is larger than $\min\{\UCB_t(a'),B\}$ for some $a'$ in the active set. If all the actions are eliminated, this means that the guess $B$ on the optimal loss is too small, and the algorithm starts a new epoch with $B$ doubled.\footnote{In fact, \citet{schlisselberg2024delay} construct yet another LCB based on the number of unobserved losses. We omit this detail since we are not able to use this to further improve our bounds for linear bandits.}

\paragraph{Challenges} However, this approach cannot be directly applied to linear bandits. Specifically, standard algorithms for stochastic linear bandits without delay (e.g., \citet{li2010contextual,abbasi2011improved}) all construct  UCB/LCB for each action by constructing an ellipsoidal confidence set for $\theta$. In the delay-as-payoff model, while it is still viable to construct UCB/LCB similar to \pref{eqn:lcb-2-mab} and \pref{eqn:ucb-mab} via a standard confidence set of $\theta$, it is difficult to construct an LCB counterpart similar to \pref{eqn:lcb-1-mab}.
This is because one action's loss is estimated using observations of all other actions in linear bandits, and naively treating the unobserved loss of one action as zero might not necessarily lead to an underestimation of another action. 

\paragraph{Our ideas} To bypass this barrier, we give up on estimating $\theta$ itself and propose to construct UCB/LCB for each action using the observed losses of the \emph{volumatric spanner} of the action set. A volumetric spanner of an action set $\calA$ is defined such that every action in $\calA$ can be expressed as a linear combination of the spanner. 

\begin{definition}[Volumetric Spanner~\citep{hazan2016volumetric}]\label{def:volume}
Suppose that $\calA = \{a_1, a_2, \dots , a_N\}$ is a set of vectors in $\R^n$. We say $\calS\subseteq \calA$ is a \emph{volumetric spanner} of $\calA$ if for any $a\in \calA$, we can write it as $a=\sum_{b\in \calS}\lambda_b\cdot b$ for some $\lambda\in \R^{|\calS|}$ with $\|\lambda\|_2\leq 1$. 
\end{definition}

Due to the linear structure, it is clear that the loss $\mu_a$ of action $a$ can be decomposed in a similar way as $\sum_{b\in \calS}\lambda_b \mu_b$,
making it possible to estimate every action's loss by only estimating the loss of the spanner.
Moreover, such a spanner can be efficiently computed:
\begin{proposition}[\citet{bhaskara2023tight}]\label{prop:volume}
Given a finite set $\calA$ of size $K$, there exists an efficient algorithm finding a volumetric spanner $\calS$ of $\calA$ with $|\calS|=3n$ within $\order(Kn^3\log n)$ runtime.
\end{proposition}

\setcounter{AlgoLine}{0}
\begin{algorithm}
\caption{Phased Elimination via Volumetric Spanner for Linear Bandits with Delay-as-Loss}\label{alg:lossLB}

\nl Input: maximum possible delay $D$, action set $\calA$, $\beta>0$. 

\nl Initialization: optimal loss guess $B=1/D$.

\nl Initialization: active action set $\calA_1=\calA$. \label{line: restart}

 \For{$m=1,2,\dots,$}{
    \nl Find $\calS_m=\{a_{m,1},\dots,a_{m,|\calS_m|}\}$, a volumetric spanner of $\calA_m$ with $|\calS_m|= 3n$. \label{line:volume}
    
    \nl Pick each $a\in \calS_m$ $2^m$ times in a round-robin way. \label{line:round-robin}

    \nl Let $\calI_m$ contain all the rounds in this epoch.
    
    \nl For each $a\in \calS_m$, calculate the following quantities: \label{line:spanner-ucb-lcb}
    {\small
    \begin{align}
        &\hat{\mu}_{m}^+(a)=\frac{1}{2^m}\Big(\sum_{\tau\in \obs_m(a)}u_{\tau} + \sum_{\tau\in \unobs_m(a)}1\Big), \label{eqn:mean-up}\\
        &\hat{\mu}_{m}^-(a)=\frac{1}{2^m}\sum_{\tau\in \obs_m(a)}u_{\tau}, \label{eqn:mean-low}\\
        &\hat{\mu}_{m,1}^{+}(a)=\hat{\mu}^{+}_{m}(a)+\frac{\beta}{2^{m/2}}\|a\|_2, \label{eqn:loss-ucb-linear-1}\\
        &\hat{\mu}_{m,1}^{-}(a)=\hat{\mu}^{-}_{m}(a)-\frac{\beta}{2^{m/2}}\|a\|_2,\label{eqn:loss-lcb-linear-1}\\
        &\hat{\mu}_m^{F}(a)=\frac{1}{\unbiasSize_m(a)}\sum_{\tau\in \unbias_m(a)}u_{\tau}, \label{eqn:mean_unbiased}\\
        &\hat{\mu}_{m,2}^{+}(a)=\hat{\mu}_m^F(a)+\frac{\beta}{\sqrt{\unbiasSize_m(a)}}\|a\|_2, \label{eqn:loss-ucb-linear-2}\\
        &\hat{\mu}_{m,2}^{-}(a)=\hat{\mu}_m^F(a)-\frac{\beta}{\sqrt{\unbiasSize_m(a)}}\|a\|_2, \label{eqn:loss-lcb-linear-2}
    \end{align}
    }
    where $\unbiasSize_m(a) = |\unbias_m(a)|$, $\unbias_m(a) = \{\tau\in \calI_m: \tau+D\in\calI_m, a_{\tau}=a\}$, $\obs_m(a) = \{\tau\in \calI_m: \tau+d_{\tau}\in\calI_m, a_{\tau}=a\}$, and
    $\unobs_m(a)= \{\tau\in \calI_m: a_{\tau}=a\}\setminus\obs_m(a)$.

    \For{each $a\in \calA_m$}{
        \nl \label{line: decompose}
        Decompose $a$ as $a=\sum_{i=1}^{|S_m|}\lambda_{m,i}^{(a)}a_{m,i}$ with $\|\lambda_{m}^{(a)}\|_2\leq 1$ and calculate 
        {\small
        \begin{align}
            &\UCB_{m}(a)=\sum_{i=1}^{|\calS_m|}\lambda_{m,i}^{(a)}\cdot\hat{\mu}_{m,2}^{\sgn(\lambda_{m,i}^{(a)})}(a_{m,i}), \label{eqn:loss-ucb-f-all-action} \\
            &\LCB_m(a) = \max_{j\in \{1,2\}}\{\LCB_{m,j}(a)\} \;\;\text{where} \nonumber  \\
            & \LCB_{m,j}(a)=\sum_{i=1}^{|\calS_m|}\lambda_{m,i}^{(a)}\cdot\hat{\mu}_{m,j}^{\sgn(-\lambda_{m,i}^{(a)})}(a_{m,i}),\label{eqn:loss-lcb-all-action}
        \end{align}
        }
    }
    
    \nl Set $\calA_{m+1} = \calA_m$.
    
    \For{$a\in \calA_m$}{
        \nl \label{line:eliminate}  
        \If{$\exists a'\in \calA_m$, s.t. $\LCB_m(a) \geq \min\{\UCB_m(a'),B\} $}
        {
          Eliminate $a$ from $\calA_{m+1}$.
        }
    }
    \nl \If{$\calA_{m+1}=\emptyset$}{
        Set $B\leftarrow 2B$ and go to \pref{line: restart}.
    }
}
\end{algorithm}

Equipped with the concept of volumetric spanner, we are now ready to introduce our algorithm (see \pref{alg:lossLB}). 
Specifically, our algorithm also makes a guess $B$ on the loss of the optimal action. 
With this guess, it proceeds to multiple epochs of arm elimination procedures, with the active action set initialized as $\calA_1 = \calA$.
In each epoch $m$, instead of picking every action in the active set $\calA_m$, we first compute a volumetric spanner $\calS_m$ of $\calA_m$ with $|\calS_m|=3n$ (\pref{line:volume}), which can be done efficiently according to \pref{prop:volume}, 
and then pick each action in the spanner set $\calS_m$ for $2^m$ rounds in a round-robin way (\pref{line:round-robin}).

After that, we calculate two UCBs and two LCBs for actions in the spanner, in a way similar to the simpler MAB setting discussed earlier (\pref{line:spanner-ucb-lcb}).
Specifically, 
the first one is in the same spirit of \pref{eqn:lcb-1-mab}:
we calculate $\hat{\mu}_m^+(a)$ ( $\hat{\mu}_m^-(a)$) as an overestimation (underestimation) of the expected loss of action $a$ by averaging over all observed losses from the rounds in $\obs_m(a)$ as well as the maximum (minimum) possible value of unobserved losses from the rounds in $\unobs_m(a)$; see \pref{eqn:mean-up} and \pref{eqn:mean-low}.
The first UCB (LCB) $\hat{\mu}_{m,1}^+(a)$ ($\hat{\mu}_{m,1}^-(a)$) is then computed based on $\hat{\mu}_m^+(a)$ ($\hat{\mu}_m^-(a)$) by incorporating a standard confidence width $\frac{\beta}{\sqrt{2^m}}\|a\|_2$ for some coefficient $\beta$; see \pref{eqn:loss-ucb-linear-1} and \pref{eqn:loss-lcb-linear-1}.
Then, to compute the second UCB/LCB, which is in the same spirit as \pref{eqn:lcb-2-mab} and \pref{eqn:ucb-mab}, we calculate an unbiased estimation $\hat{\mu}_m^F(a)$ of the expected loss of $a$ by averaging losses from the rounds in $\unbias_m(a)$, that is, all the rounds where the observation must have been revealed; see \pref{eqn:mean_unbiased}.
Note that the number of such rounds, $\unbiasSize_m(a) = |\unbias_m(a)|$, is a fixed number, and thus $\hat{\mu}_m^F(a)$ is indeed unbiased.
Similarly, we incorporate a standard confidence width $\frac{\beta}{\sqrt{c_m(a)}}\|a\|_2$ to arrive at the second UCB $\hat{\mu}_{m,2}^+(a)$ and LCB $\hat{\mu}_{m,2}^-(a)$; see \pref{eqn:loss-ucb-linear-2} and \pref{eqn:loss-lcb-linear-2}.

The next step of our algorithm is to use these UCBs/LCBs for the spanner to compute corresponding UCBs/LCBs for every active action in $\calA_m$ (\pref{line: decompose}). Specifically, for each action $a\in \calA_m$, according to the definition of a volumetric spanner (\pref{def:volume}), we can write $a$ as a linear combination of actions in $\calS_m$: $\sum_{i=1}^{|S_m|}\lambda_{m,i}^{(a)}a_{m,i}$. As mentioned, due to the linear structure of losses, we also have $\mu_a = \sum_{i=1}^{|S_m|}\lambda_{m,i}^{(a)}\mu_{a_{m,i}}$.
Thus, when constructing a UCB (or similarly LCB) for $a$, based on whether $\lambda_{m,i}^{(a)}$ is positive or not, we decide whether to use the UCB or LCB of $a_{m,i}$; see \pref{eqn:loss-ucb-f-all-action}, a counterpart of \pref{eqn:ucb-mab}, and \pref{eqn:loss-lcb-all-action}, a counterpart of \pref{eqn:lcb-1-mab} and \pref{eqn:lcb-2-mab}.\footnote{This also explains why we need $\hat{\mu}_m^+(a)$, a quantity not used in~\citet{schlisselberg2024delay}.}

At the end of an epoch, we eliminate all actions from the active action set if their LCB is either larger than the UCB of certain action or the guess $B$ on the optimal loss  (\pref{line:eliminate}). 
If the active set becomes empty, this means that the guess $B$ is too small, and the algorithm restarts with the guess doubled; 
otherwise, we continue to the next epoch.

\paragraph{Theoretical performance}
We prove the following regret bound for our algorithm. 
\begin{restatable}{theorem}{lossLB}
\label{thm:main-non-contextual}
    \pref{alg:lossLB} with $\beta=\sqrt{2\log(KT^3)}$ guarantees: 
\begin{align*}
        \Reg &\leq \order\left(\min\left\{V_1,V_2\right\}\right) + \log(d^\star)\cdot \order\left( \min\left\{W_1,W_2\right\}\right),
    \end{align*}
    where $V_1=\frac{n^2\log(KT)\log(T/n)\log(d^\star)}{\Delta_{\min}}$, $V_2=n\sqrt{T\log(d^\star)\log(KT)}$, $W_1=nd^\star\log (T/n)+D\Delta_{\max}$, and $W_2=D\Delta_{\max}\log (T/n)$. 
\end{restatable}
The first term in the regret bound $\order\left(\min\left\{V_1,V_2\right\}\right)$ is of order $\otil(\min\{\frac{n^2}{\Delta_{\min}}, n\sqrt{T}\})$, which matches the standard regret bound of LinUCB in the case without delay~\citep{abbasi2011improved}.
The second term is the overhead caused by delay and is in the same spirit as the result of~\citet{schlisselberg2024delay}:
focusing only on the part that grows in $T$, 
we see that $W_1$ only depends on $d^\star$, the expected delay of the optimal action (and hence the smallest delay among all actions),
while $W_2$ depends on the maximum possible delay $D$ but scaled by $\Delta_{\max}$, the largest sub-optimality gap.
Therefore, the delay overhead of our algorithm is small when either the shortest delay is small or all actions have similar losses.
We remark again that in the delay-as-reward setting, we obtain similar results; see \pref{app: reward} for details.

\subsection{Analysis}\label{sec: alg}
In this section, we provide a proof sketch of \pref{thm:main-non-contextual}. Detailed proofs are deferred to \pref{app:loss}.

The proof starts by proving that $\UCB_m(a)$ and $\LCB_m(a)$ are indeed valid UCB and LCB respectively for all actions in $\calA_m$. 
This follows from first using standard concentration inequalities to show that $\hat{\mu}_{m,1}^+(a)$ and $\hat{\mu}_{m,2}^+(a)$ ($\hat{\mu}_{m,1}^-(a)$ and $\hat{\mu}_{m,2}^-(a)$) are valid UCBs (LCBs) for each action in the spanner, 
and then generalizing it to every action $a \in \calA_m$ according to its decomposition over the actions in the spanner.

With this property, our analysis then proceeds to control the regret of \pref{alg:lossLB} for each guess of $B$ separately. Let $\calT_B$ be the set of rounds when \pref{alg:lossLB} runs with guess $B$. 
In \textbf{Step 1}, we first show that the use of $\LCB_{m,2}(a)$ and $\UCB_m(a)$ ensures a regret bound of $\order\left(\min\{R_1,R_2\}+D\Delta_{\max}\log(T/n)\right)$ where $R_1=\frac{n^2\log(KT)\log(T/n)}{\Delta_{\min}}$ and $R_2=n\sqrt{|\calT_B|\log(KT)}$,
and then in \textbf{Step 2}, we show that the use of $\LCB_{m,1}(a)$ and $\UCB_m(a)$ ensures a regret bound of
$\order(\min\{R_1,R_2\}+(nd^\star+DB)\log(T/n)+D\Delta_{\max})$.

\paragraph{Step 1}
For notational convenience, we define 
\begin{align*}
    \rad_{m,a}^F=\beta\sum_{i=1}^{|\calS_m|}|\lambda_{m,i}^{(a)}|\cdot\frac{\|a\|_2}{\sqrt{\unbiasSize_m(a_{m,i})}}
\end{align*}
to be the total confidence radius of action $a$ coming from the definition of $\LCB_{m,2}(a)$ and $\UCB_m(a)$. 
Via a standard analysis of arm elimination, 
we show that that if an action $a$ is not eliminated at the end of epoch $m$, we have
\begin{align*}
    \Delta_a \leq 4\max_{a\in\calA_m}\rad_{m,a}^F \leq \frac{4\sqrt{3n}\beta}{\min_{a_m\in\calS_m}\sqrt{\unbiasSize_m(a_m)}},
\end{align*}
where the second inequality uses Cauchy-Schwarz inequality and the properties of volumetric spanners, specifically that $\|\lambda_{m}^{(a)}\|_2\leq 1$ and $|\calS_m|=3n$. To provide a lower bound on $c_m(a')$ for any $a'\in\calS_m$, note that we pick each action $a'\in \calS_m$ $2^m$ times in a round-robin manner, and thus
\begin{align*}
    c_m(a') \geq 2^m - \frac{D}{|\calS_m|}-1 = 2^m - \frac{D}{3n}-1.
\end{align*}
Rearranging the terms, we then obtain
\begin{align}\label{eqn:epoch_bound_1}
    2^m\Delta_a \leq \frac{48n\beta^2}{\Delta_a} + \frac{D\Delta_a}{3n} + \Delta_a.
\end{align}
Taking summation over all $a\in\calS_m$ and $m$, and noticing that the total number of epochs is bounded by $M=\lceil\log_2(|\calT_B|/3n)\rceil$, we arrive at the following $\order(R_1+D\Delta_{\max}\log(T/n))$ regret guarantee:
\begin{align}
&\sum_{m=1}^{M}\sum_{a\in\calS_m}2^m\Delta_a \nonumber\\
    &\leq \sum_{m=1}^{M}\sum_{a\in\calS_m,\Delta_a>0}2\cdot\left(\frac{48n\beta^2}{\Delta_a}+\frac{D\Delta_a}{3n}+\Delta_a\right) \nonumber\\
    &\leq \sum_{m=1}^{M}\sum_{a\in\calS_m,\Delta_a>0}\order\left(\frac{n\log (KT)}{\Delta_a}\right) + \order\left(D\Delta_{\max}\log(T/n)\right),\nonumber \\
    &\leq \order\left(\frac{n^2\log(T/n)\log (KT)}{\Delta_{\min}}\right) + \order\left(D\Delta_{\max}\log(T/n)\right),\nonumber
\end{align}
where the first inequality is because $a\in\calS_m$ is not eliminated in epoch $m-1$ and the last inequality is by lower bounding $\Delta_a$ by $\Delta_{\min}$.

To obtain the other instance-independent regret bound $\order(R_2+D\Delta_{\max}\log(T/n))$, we bound the regret differently by considering $\Delta_a\geq \beta\sqrt{n/2^m}$ and $\Delta_a\leq \beta\sqrt{n/2^m}$ separately:
\begin{align}
    &\sum_{m=1}^{M}\sum_{a\in\calS_m}2^m\Delta_a \nonumber \\
    &\leq \sum_{m=1}^{M}\sum_{a\in\calS_m,\Delta_a\geq\beta\sqrt{n/2^m}}\left(\frac{512n\beta^2}{\Delta_a}+\frac{2D\Delta_a}{3n}+2\Delta_a\right) \nonumber\\
    &\qquad +\sum_{m=1}^{M}\sum_{a\in\calS_m,\Delta_a\leq\beta\sqrt{n/2^m}}\left(2^m\Delta_a\right) \nonumber \\
    &\leq \order(n\sqrt{|\calT_B|\log(KT)} + D\Delta_{\max}\log(T/n))\nonumber.
\end{align}

\paragraph{Step 2}
To obtain the other regret bound $\order(\min\{R_1,R_2\}+(nd^\star+DB)\log(T/n)+D\Delta_{\max})$ with a different delay overhead, we similarly define
\begin{align*}
    \rad_{m,a}^{N} &= \beta\sum_{i=1}^{|\calS_m|}|\lambda_{m,i}^{(a)}|\cdot \frac{\|a\|_2}{\sqrt{2^m}}
\end{align*}
as the total confidence radius of action $a$ coming from the definition of $\LCB_{m,1}(a)$. 
Further let $\wh{\mu}_m(a) = \frac{1}{2^m}\left(\sum_{\tau\in \obs_m(a)\cup\unobs_m(a)}u_{\tau}\right)$ be the empirical mean of action $a$'s loss within epoch $m$ (which is generally not available to the algorithm due to delay). According to the construction of $\wh{\mu}_m^{+}(a)$ and $\wh{\mu}_m^{-}(a)$, we know that for all $a\in\calS_m$,
\begin{align*}
    \wh{\mu}_m^{+}(a)\leq \wh{\mu}_m(a) + \frac{|\unobs_m(a)|}{2^m},~~\wh{\mu}_m^{-}(a)\geq \wh{\mu}_m(a) - \frac{|\unobs_m(a)|}{2^m}.
\end{align*}
Then, for any action $a\in\calA_m$ that is not eliminated at the end of epoch $m$, using the fact that $a=\sum_{i=1}^{|\calS_m|}\lambda_{m,i}^{(a)}a_{m,i}$, we obtain with high probability:
\begin{align}
    \mu_a &\leq \sum_{i=1}^{|\calS_m|}\lambda_{m,i}^{(a)}\cdot \hat{\mu}_{m}(a_{m,i}) + \rad_{m,a}^{N} \nonumber\\
    &\leq \LCB_{m,1}(a) + \rad_{m,a}^{N}+\sum_{i=1}^{|\calS_m|}|\lambda_{m,i}^{(a)}|\cdot \frac{|\unobs_m(a_{m,i})|}{2^m} \nonumber\\
    &\leq \LCB_{m,1}(a) +\rad_{m,a}^{N} \nonumber\\
    &\qquad + \sum_{i=1}^{|\calS_m|}|\lambda_{m,i}^{(a)}|\cdot\left(\frac{2D\mu_{a_{m,i}}}{2^m|\calS_m|}+\frac{16\log KT +2}{2^m}\right) \\
    &\leq B +\rad_{m,a}^{N} \nonumber\\
    &\qquad + \sum_{i=1}^{|\calS_m|}|\lambda_{m,i}^{(a)}|\cdot\left(\frac{2D\mu_{a_{m,i}}}{2^m|\calS_m|}+\frac{16\log KT +2}{2^m}\right),\label{eqn:small-loss}
\end{align}
where the first inequality is by standard Azuma-Hoeffding's inequality, the third inequality is by Lemma C.2 of \citet{schlisselberg2024delay} (included as \pref{lem:high-prob-event} in the appendix for completeness), and the last inequality is because $a$ is not eliminated at the end of epoch $m$.

\setcounter{AlgoLine}{0}
\begin{algorithm*}[htbp]
\caption{Reduction from Contextual Linear Bandits to Non-Contextual Linear Bandits~\citep{hanna2023contexts}}\label{alg:reduction}
Input: confidence level $\delta$, an instance $\Alg_{\nctx}$ of \pref{alg:lossLBmis} with $\beta=\sqrt{2\log(KT^3)}$. 

Let $\Theta'$ be a $\frac{1}{T}$-cover of $\Theta$ with size $\order(T^n)$.

\For{$m=1,2,\dots$}{
    \nl Construct action set $\calX_{m}=\{\gup{m}(\theta)~\vert~\theta\in \Theta'\}$ where   $\gup{m}(\theta)=\frac{1}{2^{m-1}}\sum_{\tau=1}^{2^{m-1}}\argmin_{a\in \calA_\tau}\inner{a,\theta}$.
    

    \nl Initiate $\Alg_{\nctx}$ with action set $\calX_m$ and misspecification level $\epsilon_m=\min\{1,2\sqrt{\log(T|\Theta'|/\delta)/2^m}\}$. \label{line:misspecific_level}
    
    \nl \For{$t=2^{m-1}+1,\dots,2^m$}{
        \nl $\Alg_{\nctx}$ outputs action $\gup{m}(\theta_t)$.

        \nl Observe $\calA_t$ and select $a_t=\argmin_{a\in \calA_t}\inner{a,\theta_t}$.

        \nl Observe the loss $u_\tau$ for all $\tau$ such that $\tau+d_{\tau}\in (t-1,t]$ and send them to $\Alg_{\nctx}$.
    }
    
}
\end{algorithm*}

Now consider two cases. When $B\geq \frac{\mu_a}{2}$, we know that $\Delta_a\leq \mu_a - \mu^\star\leq 2B$. Using the previous \pref{eqn:epoch_bound_1}, we know that
\begin{align}\label{eqn:small-loss-1}
    2^m\Delta_a\leq \order\left(\frac{n\beta^2}{\Delta_a}+\frac{DB}{n}\right).
\end{align}
Otherwise, when $B < \frac{\mu_a}{2}$, with some manipulation on \pref{eqn:small-loss}, we show that
\begin{align}\label{eqn:small-loss-2}
    2^m\Delta_a\leq \order\left(\frac{n\beta^2}{\Delta_a}+\frac{\sum_{i=1}^{|\calS_m|}D\mu_{a_{m,i}}}{n}\right).
\end{align}
Combining \pref{eqn:small-loss-1} and \pref{eqn:small-loss-2}, we then obtain that within epoch $m$, the regret is bounded by
\begin{align}\label{eqn:small-loss-3}
\order\left(\sum_{a\in\calS_m}\frac{n\beta^2}{\Delta_a}+DB+D\sum_{i=1}^{|\calS_{m-1}|}\mu_{a_{m-1,i}}\right),
\end{align} 
since all active actions in epoch $m$ are not eliminated in epoch $m-1$.
The first term $\sum_{a\in\calS_m}\frac{n\beta^2}{\Delta_a}$ in \pref{eqn:small-loss-3} eventually leads to the $\min\{R_1,R_2\}$ term in the claimed regret bound, by the exact same reasoning as in \textbf{Step 1}.
The second term explains the final $DB\log(T/n)$ term in the regret bound (recall that number of epoch is of order $\order(\log(T/n))$).
Finally, the last term in \pref{eqn:small-loss-3} can be written as
$D\sum_{i=1}^{|\calS_{m-1}|} \Delta_{a_{m-1,i}} + 3n\cdot d^\star$,
and the term $D\sum_{i=1}^{|\calS_{m-1}|} \Delta_{a_{m-1,i}}$ is one half of the regret incurred in epoch $m-1$ as long as $2^{m-1}>2D$ (otherwise, the epoch length is smaller than $D$, and we bound the regret trivially by $D\Delta_{\max}$).
Summing over all epochs and rearranging the terms thus leads to the a term $nd^\star\log(T/n)$ in the regret.
This proves the goal of the second step.

\paragraph{Combining everything} 
Finally, note that the number of different values of $B$ \pref{alg:lossLB} uses is upper bounded by $\lceil\log_2(d^\star)\rceil=\lceil\log_2(D\mu^\star)\rceil$ since the optimal action $a^\star$ will never be eliminated when $B\geq \mu^\star$. Summing up the regret over these different values of $B$ arrives at the the final bound $\order(\min\{V_1,V_2\},\log(d^\star)\min\{W_1,W_2\})$.








