\clearpage
\section{Error Analysis}
\label{sec:error-analysis}
In this section, we look into the outputs of each model and their errors given different data pairs. More specifically, we look into errors made by \geminiPro{}, \gptFouroEleven{}, \internvlTwoFiveEightB{}, \pixtral{}, \qwenTwoVLSevenB{}, and \phiThreeFive{}.
\subsection{Gemini Models}
Below is an error example of \geminiPro{}{} on a data-pair from \mmscorecoco with color jittering (CJ).
% \begin{tcolorbox}[enhanced,attach boxed title to top center={yshift=-3mm,yshifttext=-1mm},
%   colback=red!5!white,colframe=red!20!gray,colbacktitle=red!20!gray,
%   title=\geminiPro{},fonttitle=\bfseries,
%   boxed title style={size=small,colframe=red!20!gray} ]

%         \textbf{User prompt}: Assess the degree to which a text description corresponds to an image of two objects, taking into account the following conditions: \\\emph{- Be invariant to color jittering while evaluating images. For example, even if one image has been modified with random color changes such as brightness or contrast adjustments, the similarity score should remain high. If the images are different, the similarity score should decrease regardless of the jittering.}\\The scoring should reflect:
%         - A perfect alignment with the image that satisfies all conditions merits a score of 10.
%         - Descriptions that partially match the image but fail to meet certain conditions should still receive a higher score than entirely unrelated descriptions.
%         - A score of 0 should be reserved for cases where no correlation exists between the text and the image.
%         The text and image pair will be provided below. Provide your evaluation using the following format:
        
%         \begin{verbatim}
%         Score: <1-10>
%         Reason: <reason for score>
%         \end{verbatim}
        
%         What score would you assign?
%         \vspace{-8mm}
%         \begin{center}
%         \includegraphics[width=0.3\linewidth]{imgs/pair-burger-color-rev.png}
%         \end{center}

%         \textbf{Model response}:
%         \begin{tcolorbox}[colback=gray!10!white, colframe=gray!50, sharp corners=south, boxrule=0.8pt]
        
%         \textit{Score: 1\\Reason: The first image appears to be a mold of some sort, possibly for dental work or other medical purposes.  
%         The second image is a Caprese sandwich. They are entirely different objects. While the layered structure might be vaguely similar,  
%         the content, context, and purpose are completely dissimilar.}
%         \end{tcolorbox}
        
%         % \textbf{Score: 1\\Reason: The first image appears to be a mold of some sort, possibly for dental work or other medical purposes.  The second image is a Caprese sandwich.  They are entirely different objects.  While the layered structure might be vaguely similar, the content, context, and purpose are completely dissimilar.}

% \end{tcolorbox}

% \begin{tcolorbox}[enhanced,attach boxed title to top center={yshift=-3mm,yshifttext=-1mm},
%   colback=red!5!white,colframe=red!20!gray,colbacktitle=red!20!gray,
%   title=\geminiPro{},fonttitle=\bfseries,
%   boxed title style={size=small,colframe=red!20!gray} ]

%         \textbf{User prompt}: Assess the degree to which a text description corresponds to an image of two objects, taking into account the following conditions: \texttt{\{conditions\}}. The scoring should reflect:
%         - A perfect alignment with the image that satisfies all conditions merits a score of 10.
%         - Descriptions that partially match the image but fail to meet certain conditions should still receive a higher score than entirely unrelated descriptions.
%         - A score of 0 should be reserved for cases where no correlation exists between the text and the image.
%         The text and image pair will be provided below. Provide your evaluation using the following format:
        
%         \begin{verbatim}
%         Score: <1-10>
%         Reason: <reason for score>
%         \end{verbatim}
        
%         What score would you assign?
%         \vspace{-8mm}
%         \begin{center}
%         \includegraphics[width=0.3\linewidth]{imgs/pair-burger-color.png}
%         \end{center}

%         \textbf{Model response}:
%         \begin{tcolorbox}[colback=gray!10!white, colframe=gray!50, sharp corners=south, boxrule=0.8pt]
        
%         \textit{Score: 10\\Reason: The images are of the same sandwich, one in normal color and the other with a strong purple color filter applied.  Despite the heavy color distortion, the content is identical, fulfilling the color invariance requirement.}
%         \end{tcolorbox}
        
%         % \textbf{Score: 1\\Reason: The first image appears to be a mold of some sort, possibly for dental work or other medical purposes.  The second image is a Caprese sandwich.  They are entirely different objects.  While the layered structure might be vaguely similar, the content, context, and purpose are completely dissimilar.}

% \end{tcolorbox}

\begin{tcolorbox}[enhanced,attach boxed title to top center={yshift=-3mm,yshifttext=-1mm},
  colback=red!5!white,colframe=red!20!gray,colbacktitle=red!20!gray,
  title=\geminiPro{} Evaluation Instructions,fonttitle=\bfseries,
  boxed title style={size=small,colframe=red!20!gray} ]

\textbf{User prompt}: Evaluate the similarity of the images based on the following conditions:
\emph{- Be invariant to color jittering while evaluating images. Even if one image has been modified with random color changes (e.g., brightness or contrast adjustments), the similarity score should remain high. If the images are different, the similarity score should decrease regardless of the jittering.}

The score should take into account how well the images align with these conditions, as well as their overall resemblance. Even if the images are identical but fail to meet the conditions, they should still receive a higher score than completely different images. Provide a score from 1 to 10, where 1 indicates no similarity and 10 indicates identical images that fully satisfy the conditions. Respond only in this format:

\begin{verbatim}
Score: <1-10>
Reason: <reason for score>
\end{verbatim}
Nothing else should be included. What score would you give?  ``\texttt{data1}'' ``\texttt{data2}'' 
\end{tcolorbox}

\noindent
\begin{minipage}{0.48\textwidth}
\begin{tcolorbox}[enhanced,attach boxed title to top center={yshift=-3mm,yshifttext=-1mm},
  colback=red!5!white,colframe=red!20!gray,colbacktitle=red!20!gray,
  title=\geminiPro{} Response - Original Order,fonttitle=\bfseries,
  boxed title style={size=small,colframe=red!20!gray},
  equal height group=examplegroup]

\begin{center}
    \includegraphics[width=0.9\linewidth]{imgs/pair-burger-color1.png}
\end{center}

\textbf{Model response:}
\begin{tcolorbox}[colback=gray!10!white, colframe=gray!50, sharp corners=south, boxrule=0.8pt]
\textit{Score: 10\\Reason: The images are of the same sandwich, one in normal color and the other with a strong purple color filter applied.  
Despite the heavy color distortion, the content is identical, fulfilling the color invariance requirement.}
\end{tcolorbox}

\end{tcolorbox}
\end{minipage}
\hfill
\begin{minipage}{0.48\textwidth}
\begin{tcolorbox}[enhanced,attach boxed title to top center={yshift=-3mm,yshifttext=-1mm},
  colback=red!5!white,colframe=red!20!gray,colbacktitle=red!20!gray,
  title=\geminiPro{} Response - Reversed Order,fonttitle=\bfseries,
  boxed title style={size=small,colframe=red!20!gray},
  equal height group=examplegroup]

\begin{center}
    \includegraphics[width=0.9\linewidth]{imgs/pair-burger-color2.png}
\end{center}

\textbf{Model response:}
\begin{tcolorbox}[colback=gray!10!white, colframe=gray!50, sharp corners=south, boxrule=0.8pt]
\textit{Score: 1\\Reason: The first image appears to be a mold of some sort, possibly for dental work or other medical purposes.  
The second image is a Caprese sandwich. They are entirely different objects. While the layered structure might be vaguely similar,  
the content, context, and purpose are completely dissimilar.}
\end{tcolorbox}

\end{tcolorbox}
\end{minipage}


\subsection{GPT4o}
Below is an error example of \gptFouroEleven{} on a data-pair from \mmscorecoco with color jittering (CJ).
\begin{tcolorbox}[enhanced,attach boxed title to top center={yshift=-3mm,yshifttext=-1mm},
  colback=red!5!white,colframe=red!20!gray,colbacktitle=red!20!gray,
  title=\gptFouroEleven{} Evaluation Instructions,fonttitle=\bfseries,
  boxed title style={size=small,colframe=red!20!gray} ]

\textbf{User prompt}: Evaluate the similarity of the images based on the following conditions:

\emph{- Be invariant to color jittering while evaluating images. Even if one image has been modified with random color changes (e.g., brightness or contrast adjustments), the similarity score should remain high. If the images are different, the similarity score should decrease regardless of the jittering.}

The score should take into account how well the images align with these conditions, as well as their overall resemblance. Even if the images are identical but fail to meet the conditions, they should still receive a higher score than completely different images. 

Provide a score from 1 to 10, where 1 indicates no similarity and 10 indicates identical images that fully satisfy the conditions.

Respond only in this format:
\begin{verbatim}
Score: <1-10>
Reason: <reason for score>
\end{verbatim}

Nothing else should be included. What score would you give? \texttt{"data1"} \texttt{"data2"} 

\end{tcolorbox}

\noindent
\begin{minipage}{0.48\textwidth}
\begin{tcolorbox}[enhanced,attach boxed title to top center={yshift=-3mm,yshifttext=-1mm},
  colback=red!5!white,colframe=red!20!gray,colbacktitle=red!20!gray,
  title=\gptFouroEleven{} Response - Original Order,fonttitle=\bfseries,
  boxed title style={size=small,colframe=red!20!gray},
  equal height group=examplegroup]

\begin{center}
\includegraphics[width=0.9\linewidth]{imgs/pair-burger-color1.png}
\end{center}

\textbf{Model response:}
\begin{tcolorbox}[colback=gray!10!white, colframe=gray!50, sharp corners=south, boxrule=0.8pt]
\textit{Score: 8\\Reason: The images are of the same sandwich, with similar shapes and arrangements of the bread, tomato, mozzarella, and basil.  
The second image has substantial color jittering, but since the structural alignment and composition remain consistent, the similarity remains high.}
\end{tcolorbox}

\end{tcolorbox}
\end{minipage}
\hfill
\begin{minipage}{0.48\textwidth}
\begin{tcolorbox}[enhanced,attach boxed title to top center={yshift=-3mm,yshifttext=-1mm},
  colback=red!5!white,colframe=red!20!gray,colbacktitle=red!20!gray,
  title=\gptFouroEleven{} Response - Reversed Order,fonttitle=\bfseries,
  boxed title style={size=small,colframe=red!20!gray},
  equal height group=examplegroup]

\begin{center}
    \includegraphics[width=0.9\linewidth]{imgs/pair-burger-color2.png}
\end{center}

\textbf{Model response:}
\begin{tcolorbox}[colback=gray!10!white, colframe=gray!50, sharp corners=south, boxrule=0.8pt]
\textit{Score: 3\\Reason: The images depict similar objects, presumably a sandwich with layers, but there are notable differences.  
The first image appears distorted with altered coloration, and while it shares structural similarities with the second image,  
the overall resemblance is weak.}
\end{tcolorbox}

\end{tcolorbox}
\end{minipage}


\subsection{InternVL2.5}
Below is an error example of \internvlTwoFiveEightB{} on a data-pair from \mmscorecoco with that are irrelevant to each other.

\begin{tcolorbox}[enhanced,attach boxed title to top center={yshift=-3mm,yshifttext=-1mm},
  colback=red!5!white,colframe=red!20!gray,colbacktitle=red!20!gray,
  title=\internvlTwoFiveEightB{} Evaluation Instructions,fonttitle=\bfseries,
  boxed title style={size=small,colframe=red!20!gray} ]

\textbf{User prompt}: Evaluate the similarity of the images based on the following conditions: 
\emph{- Be invariant to Gaussian blurs while evaluating images. For example, even if both images are the exact same, even if one is more blurred compared to the other, the similarity score should remain high. If the images are different, the similarity score should decrease regardless of how blurred they are.}  
The score should take into account how well the images align with these conditions, as well as their overall resemblance.  
Even if the images are identical but fail to meet the conditions, they should still receive a higher score than completely different images.  

Provide a score from 1 to 10, where 1 indicates no similarity and 10 indicates identical images that fully satisfy the conditions.  
Respond only in this format:

\begin{verbatim}
Score: <1-10>
Reason: <reason for score>
\end{verbatim}

Nothing else should be included. What score would you give? ``\texttt{data1}'' ``\texttt{data2}'' 
\end{tcolorbox}

\noindent
\begin{minipage}{0.48\textwidth}
\begin{tcolorbox}[enhanced,attach boxed title to top center={yshift=-3mm,yshifttext=-1mm},
  colback=red!5!white,colframe=red!20!gray,colbacktitle=red!20!gray,
  title=\internvlTwoFiveEightB{} - Original Order,fonttitle=\bfseries,
  boxed title style={size=small,colframe=red!20!gray},
  equal height group=examplegroup]

\begin{center}
    \includegraphics[width=0.9\linewidth]{imgs/parrots-image1.png}
\end{center}

\textbf{Model response:}
\begin{tcolorbox}[colback=gray!10!white, colframe=gray!50, sharp corners=south, boxrule=0.8pt]
\textit{Score: 7\\Reason: The images are similar in content, featuring parrots, but they are not identical.  
The first image is a close-up of a spider web on a wooden surface, while the second image shows two parrots on a ledge.  
The parrots are the main subjects in the second image, and the first image has a different focus and subject matter.  
The images do not meet the condition of being identical, but they share some visual elements like color and subject type.}
\end{tcolorbox}

\end{tcolorbox}
\end{minipage}
\hfill
\begin{minipage}{0.48\textwidth}
\begin{tcolorbox}[enhanced,attach boxed title to top center={yshift=-3mm,yshifttext=-1mm},
  colback=red!5!white,colframe=red!20!gray,colbacktitle=red!20!gray,
  title=\internvlTwoFiveEightB{} - Reverse Order,fonttitle=\bfseries,
  boxed title style={size=small,colframe=red!20!gray},
  equal height group=examplegroup]

\begin{center}
    \includegraphics[width=0.9\linewidth]{imgs/parrots-image2.png}
\end{center}

\textbf{Model response:}
\begin{tcolorbox}[colback=gray!10!white, colframe=gray!50, sharp corners=south, boxrule=0.8pt]
\textit{Score: 1\\Reason: The images are completely different in terms of content, subject matter, and context.  
The first image features colorful parrots, while the second image shows a wooden door with a spider web.  
There is no similarity in terms of subject, color, or composition.}
\end{tcolorbox}

\end{tcolorbox}
\end{minipage}

\subsection{\pixtral{}}

Below is an error example of \pixtral{} on a data-pair from \mmscorecoco with rotation (R).


\begin{tcolorbox}[enhanced,attach boxed title to top center={yshift=-3mm,yshifttext=-1mm},
  colback=red!5!white,colframe=red!20!gray,colbacktitle=red!20!gray,
  title=\pixtral{} Evaluation Instructions,fonttitle=\bfseries,
  boxed title style={size=small,colframe=red!20!gray} ]

\textbf{User prompt}: Rate the similarity of these images on a scale of 1 to 10 while considering the following condition:  

\emph{- Be invariant to rotations while evaluating images. For example, if one image is a rotated version of the other, the similarity score should remain high. If the images are different, the similarity score should decrease regardless of their orientation.}  

The score should account for both how well the images satisfy the conditions and their overall resemblance. If the images are identical but do not meet the conditions, they should still score higher than unrelated images.  

Your response must follow this exact format:
\begin{verbatim}
Score: <1-10>
Reason: <reason for score>
\end{verbatim}

Provide nothing else. What’s your score? ``\texttt{data1}'' ``\texttt{data2}'' 
\end{tcolorbox}

\noindent
\begin{minipage}{0.48\textwidth}
\begin{tcolorbox}[enhanced,attach boxed title to top center={yshift=-3mm,yshifttext=-1mm},
  colback=red!5!white,colframe=red!20!gray,colbacktitle=red!20!gray,
  title=\pixtral{} Response - Original Order, fonttitle=\bfseries,
  boxed title style={size=small,colframe=red!20!gray},
  equal height group=examplegroup]

\begin{center}
    \includegraphics[width=0.9\linewidth]{imgs/birds-interaction1.png}
\end{center}

\textbf{Model response:}
\begin{tcolorbox}[colback=gray!10!white, colframe=gray!50, sharp corners=south, boxrule=0.8pt]
\textit{Score: 8\\Reason: The images depict similar scenes of people interacting with birds in natural settings.  
The first image features a group of people sitting on rocks with birds nearby, while the second image has a person extending their hand  
towards a bird in a mountainous area. Despite slight differences in the environment and specific actions,  
the overall theme and presence of birds and human interaction score high.}
\end{tcolorbox}

\end{tcolorbox}
\end{minipage}
\hfill
\begin{minipage}{0.48\textwidth}
\begin{tcolorbox}[enhanced,attach boxed title to top center={yshift=-3mm,yshifttext=-1mm},
  colback=red!5!white,colframe=red!20!gray,colbacktitle=red!20!gray,
  title=\pixtral{} Response - Reversed Order,fonttitle=\bfseries,
  boxed title style={size=small,colframe=red!20!gray},
  equal height group=examplegroup]

\begin{center}
    \includegraphics[width=0.9\linewidth]{imgs/birds-interaction2.png}
\end{center}

\textbf{Model response:}
\begin{tcolorbox}[colback=gray!10!white, colframe=gray!50, sharp corners=south, boxrule=0.8pt]
\textit{Score: 2\\Reason: The images depict completely different scenes and subjects, with no recognizable similarity between them.  
One image shows a bird and people in a rocky area, while the other does not bear any resemblance to it.}
\end{tcolorbox}

\end{tcolorbox}
\end{minipage}


\subsection{Qwen2-VL}
Below is an error example of \qwenTwoVLSevenB{} on a data-pair from \mmscorecoco with gaussian blur (GB).

\begin{tcolorbox}[enhanced,attach boxed title to top center={yshift=-3mm,yshifttext=-1mm},
  colback=red!5!white,colframe=red!20!gray,colbacktitle=red!20!gray,
  title=\qwenTwoVLSevenB{} Evaluation Instructions,fonttitle=\bfseries,
  boxed title style={size=small,colframe=red!20!gray} ]

\textbf{User prompt}: As a similarity evaluator, your responsibility is to assess the similarity of the given images while considering these conditions:  
\emph{- Be sensitive to Gaussian blurs while evaluating images. For example, if both images are the same except one is a blurred version of the other, the similarity score should decrease. If the images are the same and have the same exact sharpness, the similarity score should remain high.}  

The similarity score should reflect both how well the images align with the conditions and their overall resemblance.  
Images that are identical but do not meet the conditions should receive a moderate score, while completely unrelated images should receive the lowest score.  

Provide your score on a scale of 1 to 10, with 10 being identical images that fully meet the conditions.  
Ensure your response is in the following format:
\begin{verbatim}
Score: <1-10>
Reason: <reason for score>
\end{verbatim}

Provide nothing else. What is your score? ``\texttt{data1}'' ``\texttt{data2}'' 
\end{tcolorbox}

\noindent
\begin{minipage}{0.48\textwidth}
\begin{tcolorbox}[enhanced,attach boxed title to top center={yshift=-3mm,yshifttext=-1mm},
  colback=red!5!white,colframe=red!20!gray,colbacktitle=red!20!gray,
  title=\qwenTwoVLSevenB{} - Original Order,fonttitle=\bfseries,
  boxed title style={size=small,colframe=red!20!gray},
  equal height group=examplegroup]

\begin{center}
    \includegraphics[width=0.9\linewidth]{imgs/pizza-image1.png}
\end{center}

\textbf{Model response:}
\begin{tcolorbox}[colback=gray!10!white, colframe=gray!50, sharp corners=south, boxrule=0.8pt]
\textit{Score: 6\\Reason: The images are very similar, with the second image being a slightly blurred version of the first.  
However, they do not fully meet the conditions as the images are not identical and the blurring is not perfect.}
\end{tcolorbox}

\end{tcolorbox}
\end{minipage}
\hfill
\begin{minipage}{0.48\textwidth}
\begin{tcolorbox}[enhanced,attach boxed title to top center={yshift=-3mm,yshifttext=-1mm},
  colback=red!5!white,colframe=red!20!gray,colbacktitle=red!20!gray,
  title=\qwenTwoVLSevenB{} - Reverse Order,fonttitle=\bfseries,
  boxed title style={size=small,colframe=red!20!gray},
  equal height group=examplegroup]

\begin{center}
    \includegraphics[width=0.9\linewidth]{imgs/pizza-image2.png}
\end{center}

\textbf{Model response:}
\begin{tcolorbox}[colback=gray!10!white, colframe=gray!50, sharp corners=south, boxrule=0.8pt]
\textit{Score: 9\\Reason: The images are nearly identical, with only a slight difference in the sharpness of the objects.  
The overall resemblance is high, and the conditions are met.}
\end{tcolorbox}

\end{tcolorbox}
\end{minipage}


\subsection{Phi3.5}
Below is an error example of \phiThreeFive{} on a data-pair from \mmscorecoco with elastic transform (ET).

\begin{tcolorbox}[enhanced,attach boxed title to top center={yshift=-3mm,yshifttext=-1mm},
  colback=red!5!white,colframe=red!20!gray,colbacktitle=red!20!gray,
  title=\phiThreeFive{} Evaluation Instructions,fonttitle=\bfseries,
  boxed title style={size=small,colframe=red!20!gray} ]

\textbf{User prompt}: Judge the similarity of these images based on:  
\emph{- Be sensitive to elastic transformations while evaluating images. For example, if the images are the same but have differing distortions due to elastic transformations, the similarity score should decrease. If the images are identical and have matching elastic transformations, the similarity score should remain high.}  

The similarity score should reflect both the overall resemblance of the images and how well they satisfy the conditions.  
Identical images that do not meet the conditions should still score higher than completely unrelated images.  

Provide a score on a scale of 1 to 10, with 1 being no similarity and 10 being identical images that satisfy all conditions.  
Respond strictly in this format:
\begin{verbatim}
Score: <1-10>
Reason: <reason for score>
\end{verbatim}

Do not include additional text. What’s your rating? ``\texttt{data1}'' ``\texttt{data2}'' 
\end{tcolorbox}

\noindent
\begin{minipage}{0.48\textwidth}
\begin{tcolorbox}[enhanced,attach boxed title to top center={yshift=-3mm,yshifttext=-1mm},
  colback=red!5!white,colframe=red!20!gray,colbacktitle=red!20!gray,
  title=\phiThreeFive{} - Original Order,fonttitle=\bfseries,
  boxed title style={size=small,colframe=red!20!gray},
  equal height group=examplegroup]

\begin{center}
    \includegraphics[width=0.9\linewidth]{imgs/motorcycle-image1.png}
\end{center}

\textbf{Model response:}
\begin{tcolorbox}[colback=gray!10!white, colframe=gray!50, sharp corners=south, boxrule=0.8pt]
\textit{Score: 1\\Reason: The images are highly distorted with elastic transformations, making it impossible to accurately compare the content.}
\end{tcolorbox}

\end{tcolorbox}
\end{minipage}
\hfill
\begin{minipage}{0.48\textwidth}
\begin{tcolorbox}[enhanced,attach boxed title to top center={yshift=-3mm,yshifttext=-1mm},
  colback=red!5!white,colframe=red!20!gray,colbacktitle=red!20!gray,
  title=\phiThreeFive{} - Reverse Order,fonttitle=\bfseries,
  boxed title style={size=small,colframe=red!20!gray},
  equal height group=examplegroup]

\begin{center}
    \includegraphics[width=0.9\linewidth]{imgs/motorcycle-image2.png}
\end{center}

\textbf{Model response:}
\begin{tcolorbox}[colback=gray!10!white, colframe=gray!50, sharp corners=south, boxrule=0.8pt]
\textit{Score: 10\\Reason: The images are identical with no differences in content, color, or composition.  
They both depict the same scene with motorcycles and people in a bar setting, and the elastic transformations do not alter the image in any way.}
\end{tcolorbox}

\end{tcolorbox}
\end{minipage}
