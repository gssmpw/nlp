\FloatBarrier
\section{Full Results}
\label{sec:full-results}

In this section, we provide the \nmi of all models on all the different splits of \mmscorecoco, \mmscorein, \mmscorewuimgimg, and \mmscorewuimgtext{} in Tables \ref{tab:mi-coco-in100-var}, \ref{tab:mi-coco-in100-invar}, \ref{tab:mi-coco-in100-var}, \ref{tab:mi-coco-in100-invar}, \ref{tab:mi-wu-imgimg-var}, \ref{tab:mi-wu-imgimg-invar}, and \ref{tab:mi-wu-imgtext}.

\begin{table*}[ht]
\centering
\caption{Comparison of the \nmi{} metric ($\times 100$) of \modelss{} on \mmscorecoco{} and \mmscorein{} benchmarks in the \textit{sensitive} setting. Models are evaluated across multiple criteria:  color jitter (CJ), elastic transform (ET), gaussian blur (GB), perspective shift (PS), and rotation (R). Higher scores indicate better performance.}
\begin{tabular}{l*{5}{c}*{5}{c}}
\toprule
\multirow{2}{*}{\textbf{Model}} & \multicolumn{5}{c}{\textbf{\mmscorecoco}} & \multicolumn{5}{c}{\textbf{\mmscorein}} \\
        \cmidrule(lr){2-6} \cmidrule(lr){7-11} & \textbf{CJ} & \textbf{ET} & \textbf{GB} & \textbf{PS} & \textbf{R} &  \textbf{CJ} & \textbf{ET} & \textbf{GB} & \textbf{PS} & \textbf{R} \\
\midrule
% model & coco &  &  &  &  & in100 &  &  &  &  \\
%  & CJ & ET & GB & PS & R & CJ & ET & GB & PS & R \\
\chameleon & 00.37 & 00.34 & 00.19 & 00.31 & 00.60 & 00.38 & 00.26 & 00.31 & 00.50 & 00.52 \\
\llavaonevision & 36.51 & 44.05 & 38.57 & 43.80 & 41.41 & 37.05 & 49.89 & 40.00 & 46.01 & 49.30 \\
\phiThreeFive & 38.21 & 51.61 & 61.94 & 47.33 & 34.56 & 25.74 & 43.03 & 51.40 & 32.51 & 23.61 \\
\pixtral & 37.67 & 56.25 & 54.32 & 49.53 & 36.80 & 30.75 & 52.30 & 51.94 & 46.04 & 40.76 \\
\rowcolor{blue!15}
\internvlTwoOneB & 03.23 & 03.47 & 03.27 & 03.63 & 03.51 & 02.59 & 02.38 & 01.70 & 02.02 & 02.23 \\
\rowcolor{blue!15}
\internvlTwoTwoB & 23.89 & 32.76 & 34.32 & 31.53 & 24.76 & 18.32 & 34.02 & 33.35 & 28.17 & 23.35 \\
\rowcolor{blue!15}
\internvlTwoFourB & 52.13 & 69.43 & 62.46 & 63.77 & 52.68 & 45.25 & 65.90 & 59.90 & 60.28 & 51.04 \\
\rowcolor{blue!15}
\internvlTwoEightB & 51.58 & 62.80 & 62.35 & 60.27 & 54.80 & 47.94 & 60.18 & 58.60 & 56.66 & 53.00 \\
\rowcolor{purple!15}
\internvlTwoFiveOneB & 16.74 & 25.38 & 27.67 & 24.83 & 16.54 & 15.63 & 33.67 & 39.23 & 37.97 & 22.53 \\
\rowcolor{purple!15}
\internvlTwoFiveTwoB & 12.48 & 19.58 & 25.26 & 18.33 & 13.84 & 17.27 & 38.28 & 39.21 & 31.23 & 21.45 \\
\rowcolor{purple!15}
\internvlTwoFiveFourB & 42.61 & 59.78 & 54.33 & 55.34 & 49.47 & 41.35 & 62.35 & 54.21 & 56.18 & 49.90 \\
\rowcolor{purple!15}
\internvlTwoFiveEightB & 54.51 & 73.37 & 78.31 & 63.17 & 60.71 & 51.76 & 77.10 & 76.40 & 60.40 & 55.30 \\
\rowcolor{orange!15}
\molmoEOneB & 00.40 & 00.09 & 01.20 & 00.03 & 00.05 & 00.41 & 00.01 & 00.45 & 00.01 & 00.01 \\
\rowcolor{orange!15}
\molmoOSevenB & 14.32 & 16.02 & 48.93 & 16.12 & 15.40 & 12.91 & 14.20 & 48.43 & 13.83 & 12.16 \\
\rowcolor{orange!15}
\molmoDSevenB & 27.06 & 45.28 & 34.46 & 49.60 & 30.39 & 22.88 & 41.06 & 35.83 & 44.49 & 32.22 \\
\rowcolor{yellow!15}
\qwenTwoVLTwoB & 09.91 & 11.82 & 09.01 & 13.13 & 11.95 & 10.63 & 13.69 & 10.41 & 13.21 & 12.23 \\
\rowcolor{yellow!15}
\qwenTwoVLSevenB & 42.58 & 61.90 & 50.22 & 55.81 & 51.10 & 38.24 & 61.73 & 50.23 & 53.07 & 52.29 \\
\midrule
\rowcolor{green!15}
\gptFouroMini & 49.98 & 65.97 & 58.29 & 53.23 & 53.60 & 47.06 & 67.06 & 56.43 & 49.97 & 52.59 \\
\rowcolor{green!15}
\gptFouroFive & 50.96 & 65.54 & 61.67 & 56.69 & 56.71 & 48.55 & 65.68 & 57.48 & 54.11 & 55.00 \\
\rowcolor{green!15}
\gptFouroEight & 42.26 & 60.58 & 56.62 & 50.13 & 53.63 & 40.35 & 60.66 & 52.65 & 49.62 & 49.77 \\
\rowcolor{green!15}
\gptFouroEleven & 51.31 & 63.50 & 61.35 & 57.84 & 57.16 & 50.88 & 66.55 & 58.14 & 56.25 & 55.52 \\
\rowcolor{green!30}
\geminiFlash & \textbf{58.26} & 82.64 & 87.41 & 65.92 & 61.08 & \textbf{56.25} & 79.69 & 85.21 & 62.07 & 61.15 \\
\rowcolor{green!30}
\geminiPro & 53.33 & \textbf{87.86} & \textbf{89.56} & \textbf{74.92} & \textbf{71.04} & 51.19 & \textbf{91.36} & \textbf{92.98} & \textbf{71.56} & \textbf{74.22} \\

\bottomrule
\end{tabular}
\label{tab:mi-coco-in100-var}
\end{table*}

\begin{table*}[ht]
\centering
\caption{Comparison of the \nmi{} metric ($\times 100$) of \modelss{} on \mmscorecoco{} and \mmscorein{} benchmarks in the \textit{invariant} setting. Models are evaluated across multiple criteria: color jitter (CJ), elastic transform (ET), gaussian blur (GB), perspective shift (PS), and rotation (R). Higher scores indicate better performance.}
\begin{tabular}{l*{5}{c}*{5}{c}}
\toprule
\multirow{2}{*}{\textbf{Model}} & \multicolumn{5}{c}{\textbf{\mmscorecoco}} & \multicolumn{5}{c}{\textbf{\mmscorein}} \\
        \cmidrule(lr){2-6} \cmidrule(lr){7-11} & \textbf{CJ} & \textbf{ET} & \textbf{GB} & \textbf{PS} & \textbf{R} &  \textbf{CJ} & \textbf{ET} & \textbf{GB} & \textbf{PS} & \textbf{R} \\
\midrule
% model & coco &  &  &  &  & in100 &  &  &  &  \\
%  & CJ & ET & GB & PS & R & CJ & ET & GB & PS & R \\
\chameleon & 00.89 & 00.34 & 00.44 & 00.51 & 00.38 & 00.57 & 00.35 & 00.53 & 00.58 & 00.45 \\
\llavaonevision & 35.13 & 37.26 & 39.22 & 40.29 & 38.29 & 38.09 & 43.04 & 41.83 & 40.86 & 42.24 \\
\phiThreeFive & 49.41 & 40.19 & 42.93 & 55.03 & 47.90 & 45.88 & 33.79 & 39.72 & 50.41 & 39.46 \\
\pixtral  & 48.26 & 47.34 & 45.35 & 60.20 & 55.65 & 41.53 & 45.30 & 42.84 & 52.63 & 52.65 \\
\rowcolor{blue!15}
\internvlTwoOneB & 02.69 & 01.76 & 02.71 & 02.00 & 02.69 & 01.39 & 00.82 & 01.22 & 00.90 & 01.40 \\
\rowcolor{blue!15}
\internvlTwoTwoB & 36.38 & 31.55 & 31.99 & 39.18 & 37.28 & 32.68 & 31.40 & 30.13 & 35.98 & 34.70 \\
\rowcolor{blue!15}
\internvlTwoFourB & 59.44 & 55.47 & 51.35 & 59.61 & 59.02 & 51.74 & 52.77 & 49.60 & 54.63 & 53.11 \\
\rowcolor{blue!15}
\internvlTwoEightB & 58.69 & 58.56 & 53.60 & 61.91 & 64.22 & 58.44 & 54.48 & 51.78 & 61.97 & 62.90 \\
\rowcolor{purple!15}
\internvlTwoFiveOneB  & 21.39 & 18.59 & 21.65 & 23.19 & 22.86 & 22.52 & 14.63 & 24.34 & 22.76 & 19.24 \\
\rowcolor{purple!15}
\internvlTwoFiveTwoB & 22.85 & 19.05 & 21.46 & 27.62 & 25.99 & 32.09 & 33.03 & 37.34 & 34.65 & 34.75 \\
\rowcolor{purple!15}
\internvlTwoFiveFourB & 56.24 & 47.41 & 43.93 & 53.71 & 55.28 & 61.80 & 50.50 & 47.33 & 51.58 & 58.56 \\
\rowcolor{purple!15}
\internvlTwoFiveEightB & \textbf{75.11} & 65.18 & 66.32 & \textbf{78.56} & \textbf{81.77} & \textbf{72.53} & 61.61 & 62.23 & 65.18 & 74.27 \\
\rowcolor{orange!15}
\molmoEOneB & 00.10 & 00.11 & 00.06 & 00.02 & 00.00 & 00.02 & 00.11 & 00.10 & 00.07 & 00.25 \\
\rowcolor{orange!15}
\molmoOSevenB & 26.86 & 34.58 & 33.46 & 34.70 & 24.55 & 25.04 & 30.81 & 38.52 & 32.79 & 27.65 \\
\rowcolor{orange!15}
\molmoDSevenB & 47.20 & 45.02 & 43.02 & 50.54 & 48.64 & 45.01 & 45.83 & 45.47 & 49.25 & 40.87 \\
\rowcolor{yellow!15}
\qwenTwoVLTwoB & 09.55 & 09.10 & 10.21 & 12.65 & 08.83 & 09.02 & 09.61 & 10.01 & 14.97 & 09.33 \\
\rowcolor{yellow!15}
\qwenTwoVLSevenB & 50.52 & 51.80 & 52.70 & 54.50 & 53.29 & 47.86 & 49.73 & 51.18 & 51.55 & 50.67 \\
\midrule
\rowcolor{green!15}
\gptFouroMini & 59.76 & 57.94 & 56.55 & 61.31 & 58.17 & 56.33 & 55.56 & 55.35 & 60.99 & 60.83 \\
\rowcolor{green!15}
\gptFouroFive & 70.83 & 61.70 & 59.40 & 61.13 & 62.10 & 68.82 & 56.16 & 56.70 & 57.79 & 59.80 \\
\rowcolor{green!15}
\gptFouroEight & 55.14 & 50.31 & 46.00 & 52.15 & 52.45 & 54.13 & 45.43 & 44.25 & 48.26 & 52.18 \\
\rowcolor{green!15}
\gptFouroEleven & 73.48 & 69.06 & 61.51 & 67.60 & 63.99 & 70.16 & 61.33 & 58.89 & 65.06 & 60.84 \\
\rowcolor{green!30}
\geminiFlash & 72.11 & 67.81 & 68.17 & 71.88 & 78.31 & 70.32 & 65.94 & 66.58 & 69.10 & 74.77 \\
\rowcolor{green!30}
\geminiPro & 68.93 & \textbf{69.64} & \textbf{71.50} & 72.06 & 68.42 & 66.31 & \textbf{70.03} & \textbf{72.17} & \textbf{70.13} & \textbf{69.32} \\

\bottomrule
\end{tabular}
\label{tab:mi-coco-in100-invar}
\end{table*}

\begin{table*}[ht]
\centering
\caption{Comparison of the \nmi{} metric ($\times 100$) of \modelss{} on \mmscorewuimgimg{} (subset A and B) benchmark in the \textit{sensitive} setting. Models are evaluated across multiple criteria: spatial position (SP), spatial position and color jitter (SP-CJ), spatial position and elastic transform (SP-ET), spatial position and gaussian blur (SP-GB), spatial position and perspective shift (SP-PS), and spatial position and rotation (SP-R). Higher scores indicate better performance.}
\resizebox{0.95\textwidth}{!}{%
\begin{tabular}{l*{6}{c}*{6}{c}}
\toprule
\multirow{2}{*}{\textbf{Model}} & \multicolumn{6}{c}{\textbf{\mmscore$_{WU_a}$}} & \multicolumn{6}{c}{\textbf{\mmscore$_{WU_b}$}} \\
        \cmidrule(lr){2-7} \cmidrule(lr){8-13} & \textbf{SP} & \textbf{SP-CJ} & \textbf{SP-ET} & \textbf{SP-GB} & \textbf{SP-PS} & \textbf{SP-R} & \textbf{SP} &  \textbf{SP-CJ} & \textbf{SP-ET} & \textbf{SP-GB} & \textbf{SP-PS} & \textbf{SP-R}\\
\midrule
% model & coco &  &  &  &  & in100 &  &  &  &  \\
%  & CJ & ET & GB & PS & R & CJ & ET & GB & PS & R \\
\chameleon & 00.28 & 00.47 & 00.23 & 00.52 & 0.21 & 00.20 & 00.34 & 00.38 & 00.35 & 00.26 & 00.31 & 00.33 \\
\llavaonevision & 38.95 & 18.83 & 24.03 & 26.78 & 29.46 & 24.63 & 19.70 & 14.03 & 16.51 & 16.78 & 17.76 & 17.02 \\
\phiThreeFive & 23.44 & 08.46 & 15.70 & 19.41 & 13.34 & 10.83 & 15.38 & 12.98 & 18.91 & 20.19 & 11.69 & 17.06 \\
\pixtral & 37.91 & 26.09 & 32.05 & 33.52 & 32.47 & 25.00 & 28.02 & 19.58 & 22.32 & 22.31 & 23.46 & 24.50 \\
\rowcolor{blue!15}
\internvlTwoOneB & 00.44 & 00.98 & 00.79 & 00.65 & 00.30 & 00.28 & 00.20 & - & - & 00.41 & 01.18 & 00.90 \\
\rowcolor{blue!15}
\internvlTwoTwoB & 22.85 & 12.03 & 14.37 & 17.84 & 18.66 & 15.50 & 20.72 & 10.89 & 11.22 & 15.74 & 17.74 & 13.58 \\
\rowcolor{blue!15}
\internvlTwoFourB & 46.89 & 27.91 & 36.67 & 43.03 & 44.27 & 27.76 & 44.89 & 27.77 & 33.35 & 38.12 & 42.23 & 36.16 \\
\rowcolor{blue!15}
\internvlTwoEightB & 41.99 & 32.06 & 35.71 & 41.02 & 40.12 & 29.11 & 46.36 & 32.17 & 39.24 & 41.90 & 45.59 & 40.30 \\
\rowcolor{purple!15}
\internvlTwoFiveOneB & 25.50 & 14.16 & 21.32 & 15.69 & 21.49 & 16.30 & 24.77 & 16.16 & 21.10 & 19.95 & 27.89 & 21.47 \\
\rowcolor{purple!15}
\internvlTwoFiveTwoB & 20.63 & 11.76 & 16.75 & 15.21 & 18.03 & 13.79 & 23.44 & 09.33 & 15.90 & 17.64 & 18.17 & 17.56 \\
\rowcolor{purple!15}
\internvlTwoFiveFourB & 46.15 & 32.74 & 39.05 & 39.24 & 42.28 & 32.94 & 47.93 & 33.75 & 40.23 & 39.82 & 44.07 & 42.57 \\
\rowcolor{purple!15}
\internvlTwoFiveEightB & 44.27 & 36.99 & 41.49 & 42.60 & 43.65 & 33.24 & 41.32 & 31.69 & 40.10 & 39.73 & 44.03 & 42.99 \\
\rowcolor{orange!15}
\molmoEOneB & 00.47 & 01.03 & 00.00 & 00.03 & 00.14 & 00.01 & 00.32 & 00.36 & 00.01 & 00.04 & 00.04 & 00.09 \\
\rowcolor{orange!15}
\molmoOSevenB & 15.94 & 09.90 & 11.32 & 15.38 & 12.92 & 12.01 & 15.15 & 08.40 & 11.39 & 11.33 & 13.60 & 12.50 \\
\rowcolor{orange!15}
\molmoDSevenB & 23.82 & 17.75 & 20.41 & 18.40 & 22.21 & 17.81 & 26.74 & 18.37 & 19.55 & 18.77 & 18.19 & 22.21 \\
\rowcolor{yellow!15}
\qwenTwoVLTwoB & 02.26 & 01.76 & 02.58 & 02.15 & 03.17 & 01.68 & 00.88 & 00.44 & 00.73 & 00.37 & 00.72 & 00.82 \\
\rowcolor{yellow!15}
\qwenTwoVLSevenB & 41.95 & 29.47 & 36.32 & 39.93 & 40.33 & 34.11 & 42.80 & 28.75 & 31.42 & 37.27 & 39.76 & 36.25 \\
\midrule
\rowcolor{green!15}
\gptFouroMini & 42.55 & 37.21 & 39.50 & 40.44 & 38.83 & 41.05 & 48.86 & 38.38 & 43.82 & 45.42 & 46.32 & 46.66 \\
\rowcolor{green!15}
\gptFouroFive & 40.27 & 37.83 & 36.79 & 38.52 & 38.84 & 38.07 & 44.13 & 39.46 & 39.46 & 43.58 & 43.49 & 46.25 \\
\rowcolor{green!15}
\gptFouroEight & 37.58 & 33.72 & 34.24 & 33.36 & 34.80 & 33.17 & 40.11 & 33.36 & 32.36 & 34.32 & 39.91 & 38.67 \\
\rowcolor{green!15}
\gptFouroEleven & 40.68 & 39.06 & 40.10 & 40.35 & 40.96 & 40.40 & 47.34 & 40.91 & 43.07 & 47.18 & 50.22 & 50.68 \\
\rowcolor{green!30}
\geminiFlash & 44.63 & 38.85 & 37.19 & 39.11 & 35.76 & 34.57 & 49.91 & 40.29 & 42.92 & 46.34 & 47.01 & 46.40 \\
\rowcolor{green!30}
\geminiPro & 40.38 & 36.07 & 31.52 & 37.85 & 29.92 & 30.37 & 49.20 & 38.26 & 39.16 & 44.98 & 41.70 & 40.72 \\
\bottomrule
\end{tabular}
}
\label{tab:mi-wu-imgimg-var}
\end{table*}


\begin{table*}[ht]
\centering
\caption{Comparison of the \nmi{} metric ($\times 100$) of \modelss{} on \mmscorewuimgimg{} (subset A and B) benchmark in the \textit{invariant} setting. Models are evaluated across multiple criteria:spatial position (SP), spatial position and color jitter (SP-CJ), spatial position and elastic transform (SP-ET), spatial position and gaussian blur (SP-GB), spatial position and perspective shift (SP-PS), and spatial position and rotation (SP-R). Higher scores indicate better performance.}
\resizebox{0.95\textwidth}{!}{%
\begin{tabular}{l*{6}{c}*{6}{c}}
\toprule
\multirow{2}{*}{\textbf{Model}} & \multicolumn{6}{c}{\textbf{\mmscore$_{WU_a}$}} & \multicolumn{6}{c}{\textbf{\mmscore$_{WU_b}$}} \\
        \cmidrule(lr){2-7} \cmidrule(lr){8-13} & \textbf{SP} & \textbf{SP-CJ} & \textbf{SP-ET} & \textbf{SP-GB} & \textbf{SP-PS} & \textbf{SP-R} & \textbf{SP} &  \textbf{SP-CJ} & \textbf{SP-ET} & \textbf{SP-GB} & \textbf{SP-PS} & \textbf{SP-R} \\
\midrule
% model & coco &  &  &  &  & in100 &  &  &  &  \\
%  & CJ & ET & GB & PS & R & CJ & ET & GB & PS & R \\
\chameleon & 00.34 & 00.39 & 00.76 & 00.47 & 00.43 & 00.41 & 00.47 & 00.34 & 00.56 & 00.24 & 00.62 & 00.34 \\
\llavaonevision & 34.79 & 31.56 & 30.23 & 34.14 & 32.61 & 28.69 & 13.12 & 18.41 & 16.21 & 22.69 & 15.34 & 17.91 \\
\phiThreeFive & 23.66 & 32.84 & 18.90 & 21.36 & 30.14 & 19.10 & 19.88 & 36.74 & 22.40 & 23.47 & 30.04 & 26.06 \\
\pixtral & 36.93 & 37.32 & 41.17 & 35.31 & 38.52 & 36.05 & 36.03 & 30.44 & 33.32 & 29.84 & 35.48 & 33.32 \\
\rowcolor{blue!15}
\internvlTwoOneB & 00.57 & 01.08 & 02.02 & 01.02 & 00.89 & 00.37 & 00.65 & 00.81 & 00.96 & 00.50 & 00.56 & 00.54 \\
\rowcolor{blue!15}
\internvlTwoTwoB & 26.25 & 25.53 & 25.76 & 21.12 & 26.57 & 26.98 & 26.03 & 24.52 & 26.49 & 25.81 & 31.01 & 29.33 \\
\rowcolor{blue!15}
\internvlTwoFourB & 39.33 & 40.23 & 37.80 & 42.25 & 43.10 & 34.57 & 51.43 & 41.55 & 45.96 & 50.20 & 54.94 & 50.34 \\
\rowcolor{blue!15}
\internvlTwoEightB & 43.80 & 44.31 & 44.53 & 43.99 & 46.02 & 40.43 & 60.92 & 46.63 & 54.53 & 51.31 & 56.94 & 53.88 \\
\rowcolor{purple!15}
\internvlTwoFiveOneB & 12.82 & 13.84 & 09.34 & 07.24 & 12.91 & 16.93 & 19.87 & 24.92 & 19.36 & 17.94 & 22.66 & 30.60 \\
\rowcolor{purple!15}
\internvlTwoFiveTwoB & 31.38 & 29.79 & 30.53 & 23.16 & 31.75 & 24.69 & 36.01 & 30.13 & 35.52 & 27.07 & 37.01 & 31.18 \\
\rowcolor{purple!15}
\internvlTwoFiveFourB & 48.79 & 53.58 & 54.52 & 48.09 & 52.78 & 46.46 & 50.51 & 48.71 & 53.45 & 52.03 & 53.77 & 50.12 \\
\rowcolor{purple!15}
\internvlTwoFiveEightB & 59.03 & 55.57 & 59.70 & 57.16 & 58.01 & 50.84 & 65.21 & 51.31 & 61.10 & 63.54 & 62.38 & 60.83 \\
\rowcolor{orange!15}
\molmoEOneB & 03.83 & 00.09 & 00.02 & 00.02 & 00.10 & 00.17 & 04.22 & 00.07 & 00.02 & 00.07 & 00.12 & 00.00 \\
\rowcolor{orange!15}
\molmoOSevenB & 18.63 & 17.50 & 19.68 & 16.42 & 19.58 & 14.99 & 15.94 & 19.46 & 20.93 & 17.98 & 24.21 & 21.68 \\
\rowcolor{orange!15}
\molmoDSevenB & 28.21 & 36.47 & 31.95 & 26.89 & 35.57 & 33.58 & 37.50 & 35.90 & 34.70 & 33.51 & 33.04 & 34.35 \\
\rowcolor{yellow!15}
\qwenTwoVLTwoB & 02.63 & 02.88 & 03.58 & 03.53 & 03.34 & 02.97 & 00.79 & 00.73 & 00.99 & 00.88 & 00.71 & 00.82 \\
\rowcolor{yellow!15}
\qwenTwoVLSevenB & 40.21 & 38.96 & 39.94 & 46.88 & 40.11 & 39.55 & 47.65 & 39.51 & 40.94 & 48.63 & 44.68 & 41.88 \\
\midrule
\rowcolor{green!15}
\gptFouroMini & 47.60 & 48.33 & 51.04 & 46.15 & 48.86 & 43.75 & 57.50 & 49.19 & 51.38 & 53.76 & 55.82 & 54.07 \\
\rowcolor{green!15}
\gptFouroFive & 52.39 & 51.58 & 48.78 & 47.11 & 47.50 & 52.68 & 61.59 & 59.77 & 58.08 & 60.95 & 61.53 & 63.74 \\
\rowcolor{green!15}
\gptFouroEight & 50.94 & 47.21 & 46.52 & 42.90 & 45.84 & 52.50 & 62.75 & 54.23 & 53.20 & 51.19 & 58.50 & 57.21 \\
\rowcolor{green!15}
\gptFouroEleven & 57.47 & 56.25 & 54.40 & 56.11 & 54.40 & 57.93 & 65.91 & 62.22 & 63.93 & 67.96 & 66.86 & 68.10 \\
\rowcolor{green!30}
\geminiFlash & 46.62 & 55.28 & 54.31 & 57.98 & 57.01 & 58.74 & 62.04 & 54.43 & 56.89 & 62.24 & 66.88 & 60.72 \\
\rowcolor{green!30}
\geminiPro & 38.07 & 35.08 & 35.05 & 36.11 & 33.21 & 33.23 & 56.43 & 42.24 & 43.74 & 48.41 & 50.40 & 45.83 \\
\bottomrule
\end{tabular}
}
\label{tab:mi-wu-imgimg-invar}
\end{table*}


\begin{table*}[ht]
\centering
\caption{Comparison of the \nmi{} metric ($\times 100$) of \modelss{} on the \mmscorewuimgtext{} (Subset A and B) benchmark in the \textit{sensitive} and \textit{invariant} settings. Models are evaluated across the spatial position (SP) criterion. Higher scores indicate better performance.}
\begin{tabular}{l*{2}{c}*{2}{c}}
\toprule
\multirow{2}{*}{\textbf{Model}} & \multicolumn{2}{c}{\textbf{\mmscore$_{WU_a}$}} & \multicolumn{2}{c}{\textbf{\mmscore$_{WU_b}$}} \\
        \cmidrule(lr){2-3} \cmidrule(lr){4-5} & \textbf{Sens.} & \textbf{Invar.} & \textbf{Sens.} & \textbf{Invar.} \\
\midrule
% model & coco &  &  &  &  & in100 &  &  &  &  \\
%  & CJ & ET & GB & PS & R & CJ & ET & GB & PS & R \\
\chameleon & 00.25 & 00.34 & 00.23 & 00.47 \\
\llavaonevision & 23.35 & 22.78 & 27.38 & 25.98  \\
\phiThreeFive & 13.86 & 12.30 & 25.67 & 24.74  \\
\pixtral & 05.14 & 05.04 & 03.27 & 04.58  \\
\rowcolor{blue!15}
\internvlTwoOneB & 06.29 & 03.75 & 15.90 & 08.31  \\
\rowcolor{blue!15}
\internvlTwoTwoB & 17.07 & 14.26 & 24.46 & 16.49  \\
\rowcolor{blue!15}
\internvlTwoFourB & 15.69 & 15.69 & 24.27 & 22.96  \\
\rowcolor{blue!15}
\internvlTwoEightB & 22.40 & 19.27 & 29.45 & 31.46  \\
\rowcolor{purple!15}
\internvlTwoFiveOneB & 20.80 & 09.49 & 16.86 & 13.23  \\
\rowcolor{purple!15}
\internvlTwoFiveTwoB & 15.36 & 11.15 & 19.69 & 18.42  \\
\rowcolor{purple!15}
\internvlTwoFiveFourB & 23.90 & 23.85 & 29.75 & 32.45  \\
\rowcolor{purple!15}
\internvlTwoFiveEightB & 24.16 & 25.55 & 24.00 & 28.22  \\
\rowcolor{orange!15}
\molmoEOneB & 00.12 & 00.04 & 00.02 & 00.21  \\
\rowcolor{orange!15}
\molmoOSevenB & 07.53 & 07.45 & 07.18 & 08.29  \\
\rowcolor{orange!15}
\molmoDSevenB & 09.45 & 12.26 & 08.34 & 11.26  \\
\rowcolor{yellow!15}
\qwenTwoVLTwoB & 02.65 & 03.09 & 05.09 & 05.86  \\
\rowcolor{yellow!15}
\qwenTwoVLSevenB & 09.43 & 09.19 & 15.99 & 16.13  \\
\midrule 
\rowcolor{green!15}
\gptFouroMini & 16.18 & 16.14 & 16.18 & 15.30  \\
\rowcolor{green!15}
\gptFouroFive & 11.49 & 20.48 & 12.63 & 20.98  \\
\rowcolor{green!15}
\gptFouroEight & 20.27 & 31.80 & 22.97 & 36.56  \\
\rowcolor{green!15}
\gptFouroEleven & 18.97 & 31.91 & 20.57 & 34.99  \\
\rowcolor{green!30}
\geminiFlash & 27.46 & 26.54 & 26.53 & 32.07  \\
\rowcolor{green!30}
\geminiPro & 26.89 & 27.16 & 28.57 & 29.23  \\

\bottomrule
\end{tabular}
\label{tab:mi-wu-imgtext}
\end{table*}
\begin{figure*}[ht]
    \centering
    \includegraphics[width=0.95\linewidth,trim={.1cm .2cm .2cm .2cm},clip]{imgs/all-metric-mi-distinct-plt.pdf}
    \includegraphics[width=0.95\linewidth,trim={.1cm .2cm .2cm .2cm},clip]{imgs/all-metric-entropy-distinct-plt.pdf}
    \includegraphics[width=0.95\linewidth,trim={.1cm .2cm .2cm .2cm},clip]{imgs/all-metric-relaxed1_sym-distinct-plt.pdf}
    \caption{\nmi, \textbf{Smoothness}, and \textbf{Controllability} for the best performing models in both \texttt{sens} and \texttt{invar} settings.}
    \label{fig:best-models-mmscore-smoothness-sym}
\end{figure*}


\subsubsection{All \relaxsym for different $\epsilon$s}
To show the \relaxsym{} for different values of $\varepsilon$, we plot Figure \ref{fig:diff-relax-sym-eps} and show as $\varepsilon$ gets higher, the values go higher. However, some models such as the GPT4o models struggle with symmetry. Please note that if $\varepsilon = 0$, it is the same as not having a threshold and hence calculating exact symmetry rather than a relaxed version.


\begin{figure*}[ht]
    \centering
    \includegraphics[width=0.95\linewidth,trim={.1cm .2cm .2cm .2cm},clip]{imgs/all-metric-exact_sym-relaxed1_sym-relaxed2_sym-relaxed3_sym-distinct-plt.pdf}
    \caption{\relaxsym{} for different $\varepsilon$s.}
    \label{fig:diff-relax-sym-eps}
\end{figure*}


\subsubsection{Different versions of same model}
\label{sec:model-versions}
% We further look into the affect of capacity on the different metrics of \mmscore. As seen in Figure \ref{fig:mi-model-versions} and \ref{fig:all-model-versions}, the larger capacity models tend to better across \nmi, \relaxsym, and \control. However, we observe there are exceptions, e.g., \internvlTwoFourB{} being more controllable in rotation (R) and perspective shift (PS), compared to \internvlTwoEightB. Also, we see \smoothness{} is not monotonically increasing as the model capacity increases. This shows that the stronger models may tend to be more certain about their responses, hence not generating similarity scores as diverse as the lower capacity ones.

% On the other hand, we saw in Table \ref{tab:benchmark_comparison} and Figure \ref{fig:control-vs-bm} that Smoothness also has a positive correlation with model performance and other benchmarks, showing that better models tend to be more smoother and create more diverse outputs compared to the weaker ones. Ultimately, we conclude that \smoothness{} is not a property of performance; however, it is a characteristic of a \model{} as a judge model which could be desirable depending on the use-case.

We further examine the effect of model capacity on the different metrics of \mmscore. As seen in Figures \ref{fig:mi-model-versions} and \ref{fig:all-model-versions}, larger-capacity models tend to perform better across \nmi, \relaxsym, and \control. However, there are exceptions—for example, \internvlTwoFourB{} demonstrates greater controllability in rotation (R) and perspective shift (PS) compared to \internvlTwoEightB. Additionally, smoothness (\smoothness{}) does not increase monotonically with model capacity. This suggests that stronger models may be more confident in their responses, leading to less diversity in their similarity scores compared to lower-capacity models.

On the other hand, Table \ref{tab:benchmark_comparison} and Figure \ref{fig:control-vs-bm} show that \smoothness{} correlates positively with model performance and other benchmarks, indicating that better models tend to produce smoother and more diverse outputs than weaker ones. Ultimately, we conclude that \smoothness{} is not strictly a property of model performance but rather a characteristic of a \model{} as a judge model that may be desirable (or not) depending on the use case. 


\begin{figure*}[ht]
    \centering
    % \includegraphics[width=0.95\linewidth,trim={.1cm .2cm .2cm .2cm},clip]{imgs/GPT-4o-Models-metric-mi-distinct-plt.pdf}
    \includegraphics[width=0.95\linewidth,trim={.1cm .2cm .2cm .2cm},clip]{imgs/invl-metric-entropy-distinct-plt.pdf}
    % \caption{Caption for First PDF.}
    % \label{fig:first-plot}

    \vspace{1em} % Add vertical spacing between figures if needed

    \includegraphics[width=0.95\linewidth,trim={.1cm .2cm .2cm .2cm},clip]{imgs/invl-metric-relaxed1_sym-distinct-plt.pdf}
    % \caption{Caption for Second PDF.}
    % \label{fig:second-plot}

    \vspace{1em} % Add vertical spacing between figures if needed

    \includegraphics[width=0.95\linewidth,trim={.1cm .2cm .2cm .2cm},clip]{imgs/invl-metric-entropy-distinct-plt.pdf}
    % \caption{Caption for Third PDF.}
    % \label{fig:third-plot}

    \caption{InternVL2.5 models with different capacities aggregated on \mmscorecoco{} and \mmscorein.}
    \label{fig:mi-model-versions}
\end{figure*}
\begin{figure*}[ht]
    \centering

    \includegraphics[width=0.95\linewidth,trim={.1cm .2cm .2cm .2cm},clip]{imgs/invl-metric-mi-relaxed1_sym-entropy-control_mult-distinct-plt.pdf}
    % \caption{Caption for Third PDF.}
    % \label{fig:third-plot}

    \caption{Aggregated \mmscore{} metrics across different versions of InternVL2.5 models.}
    \label{fig:all-model-versions}
\end{figure*}


\subsection{Correlations}
\label{sec:bm-correlations}

In this section, we plot the further correlations of the different metrics and show them in Figures \ref{fig:sym-vs-bm}, \ref{fig:control-vs-bm}, \ref{fig:control-vs-bm}. As seen, all these metrics have positive correlations as seen in the scatter plots.


\begin{figure*}[ht]
    \centering
    % \includegraphics[width=0.95\linewidth,trim={.1cm .2cm .2cm .2cm},clip]{imgs/sym-vs-bm-scatter-plot-correlation.pdf}
    \includegraphics[width=0.95\linewidth,trim={.1cm .2cm .2cm .2cm},clip]{imgs/mmscore_n_relxsym-v2.pdf}
    \caption{Other benchmarks versus \mmscore{} on \relaxsymone{}.}
    \label{fig:sym-vs-bm}
\end{figure*}
\begin{figure*}[ht]
    \centering
    \includegraphics[width=0.95\linewidth,trim={.1cm .2cm .2cm .2cm},clip]{imgs/mmscore_control.pdf}
    \caption{Other benchmarks versus \control{} on \mmscore{}.}
    \label{fig:control-vs-bm}
\end{figure*}
\begin{figure*}[ht]
    \centering
    \includegraphics[width=0.95\linewidth,trim={.1cm .2cm .2cm .2cm},clip]{imgs/mmscore_ent-v2.pdf}
    \caption{Other benchmarks versus Smoothness (\smoothness).}
    \label{fig:smoothness-vs-bm}
\end{figure*}

% \input{imgs/sym-plot}
