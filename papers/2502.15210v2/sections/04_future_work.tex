\vspace{-5mm}
\section{Conclusion and Future Work}
\vspace{-1mm}
% \vspace{-1ex}
We introduced \mmscore{}, a comprehensive framework for evaluating the reliability of \modelss{} when used to define similarity kernels. \mmscore{} enables assessing how different models will behave when acting as evaluators by measuring kernel properties such as alignment with ground truth relevance, symmetry, smoothness, and controllability. Interestingly, by leveraging controlled data transformations, we found that \mmscore{} not only allows for fine-grained analysis of model biases and strengths, but it also offers a cost-effective alternative to large-scale benchmarks.

\vspace{-0.5mm}

We carried out a large-scale benchmarking covering several \modelss{} and demonstrated that no single model excels across all four metrics or dataset configurations. While commercial-grade models generally performed better on image-image comparisons, openly available models such as \internvlTwoFiveEightB{} showed competitive results, particularly in \nmi. Furthermore, our findings indicate that commonly used judge models exhibit limitations. For instance, \gptFouroEleven{} lacks in terms of symmetry and smoothness, highlighting the necessity of careful selection based on specific evaluation needs.

\vspace{-0.5mm}

From a more practical perspective, we established that \mmscore{} metrics, particularly \nmi{}, correlate strongly with model performance on well-known benchmarks, reinforcing its utility as a low-cost surrogate for ranking models or guiding cross-validation during training. As the field progresses, we anticipate that \mmscore{} will serve as a valuable tool for improving model evaluation practices. 

% As future work, \mmscore{} could be extended by incorporating additional modalities, refining transformation techniques, and further exploring the impact of architectural choices on similarity estimation. By advancing our understanding of \modelss{} as similarity kernels, we move closer to more robust and interpretable model-based evaluation methods.