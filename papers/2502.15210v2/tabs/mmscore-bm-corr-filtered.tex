
\begin{table}[ht]
\centering
\tabcolsep 2pt
\resizebox{0.48\textwidth}{!}{%
\begin{tabular}{lcccccc}
\toprule
Metric       & \nmi       & \textbf{1-RS}
%\relaxsymone  
& \smoothness  & \control   \\ \midrule
AI2D \small{\cite{kembhavi2016diagram}}    & 76\%  & 28\%      & 30\%        & 54\%      \\
HallusionBench \small{\cite{guan2024hallusionbench}} & 75\%      & \textbf{43\%}      & 31\%        & 39\%      \\
MMBench \small{\cite{liu2025mmbench}}     & 81\%      & 25\%      & \textbf{44\%}        & 63\%      \\
MMMU \small{\cite{yue2024mmmu}}         & \textbf{89\%}      & 35\%      & 31\%        & 60\%      \\
MMStar \small{\cite{chen2024we}}      & 81\%      & 20\%      & 42\%        & 58\%      \\
MMVet \small{\cite{yu2023mm} }    & 79\%      & 34\%      & \textbf{44\%}        & 51\%      \\
MathVista \small{\cite{lu2023mathvista} }   & 73\%      & 11\%      & 41\%        & \textbf{68\%}      \\
OCRBench \small{\cite{liu2024ocrbench}}    & 50\%      & 10\%      & 41\%        & 35\%      \\ 
\bottomrule
\end{tabular}
}
\caption{Spearman correlation of different metrics of \mmscore{} with performance on other benchmarks for 23 models. All metrics correlate with benchmarks; since \nmi{} has the highest correlation, it is chosen as the main metric of \mmscore.}
% \vspace{-1mm}
\label{tab:benchmark_comparison}
\end{table}
%}
