%%%%%%%% ICML 2025 EXAMPLE LATEX SUBMISSION FILE %%%%%%%%%%%%%%%%%

\documentclass{article}
\pdfobjcompresslevel=0

% Recommended, but optional, packages for figures and better typesetting:
\usepackage{microtype}
\usepackage{graphicx}
\usepackage{subfigure}
\usepackage{booktabs} % for professional tables
\usepackage{mathrsfs}

% hyperref makes hyperlinks in the resulting PDF.
% If your build breaks (sometimes temporarily if a hyperlink spans a page)
% please comment out the following usepackage line and replace
% \usepackage{icml2025} with \usepackage[nohyperref]{icml2025} above.
% \usepackage[nohyperref]{icml2025} 
\usepackage{hyperref}


\usepackage{url}            % simple URL typesetting
\usepackage{amsfonts}       % blackboard math symbols
\usepackage{nicefrac}       % compact symbols for 1/2, etc.
\usepackage{microtype}      % microtypography
\usepackage{xcolor}         % colors
\usepackage{multirow}
\usepackage{colortbl}
\usepackage{arydshln}

\usepackage{graphicx}
\usepackage{fancyvrb} 
% \usepackage{tcolorbox}
\usepackage{listings}
\usepackage{xcolor}
\usepackage{placeins}
\usepackage[most]{tcolorbox}
\usepackage{bbm}

% \usepackage{colortbl}
\newcommand{\red}[1]{{\color{red}#1}}

\lstdefinestyle{prompt}{
    basicstyle=\ttfamily\small,
    backgroundcolor=\color{gray!10},
    breaklines=true,
    frame=single,
    framerule=0.5pt,
    rulecolor=\color{gray!80},
    columns=flexible,
    keepspaces=true,
    xleftmargin=0pt,
    xrightmargin=0pt,
    breakindent=5pt,
}



% Attempt to make hyperref and algorithmic work together better:
\newcommand{\theHalgorithm}{\arabic{algorithm}}

% Use the following line for the initial blind version submitted for review:
% \usepackage{icml2025}

% If accepted, instead use the following line for the camera-ready submission:
\usepackage[accepted]{icml2025}

% For theorems and such
\usepackage{amsmath}
\usepackage{amssymb}
\usepackage{mathtools}
\usepackage{amsthm}

% if you use cleveref..
\usepackage[capitalize,noabbrev]{cleveref}

%%%%%%%%%%%%%%%%%%%%%%%%%%%%%%%%
% THEOREMS
%%%%%%%%%%%%%%%%%%%%%%%%%%%%%%%%
\theoremstyle{plain}
\newtheorem{theorem}{Theorem}[section]
\newtheorem{proposition}[theorem]{Proposition}
\newtheorem{lemma}[theorem]{Lemma}
\newtheorem{corollary}[theorem]{Corollary}
\theoremstyle{definition}
\newtheorem{definition}[theorem]{Definition}
\newtheorem{assumption}[theorem]{Assumption}
\theoremstyle{remark}
\newtheorem{remark}[theorem]{Remark}




% Todonotes is useful during development; simply uncomment the next line
%    and comment out the line below the next line to turn off comments
%\usepackage[disable,textsize=tiny]{todonotes}
\usepackage[textsize=tiny]{todonotes}


% The \icmltitle you define below is probably too long as a header.r
% Therefore, a short form for the running title is supplied here:
\icmltitlerunning{\mmscore: A Systematic Framework for Selecting Reliable Judge VLMs}

%
\setlength\unitlength{1mm}
\newcommand{\twodots}{\mathinner {\ldotp \ldotp}}
% bb font symbols
\newcommand{\Rho}{\mathrm{P}}
\newcommand{\Tau}{\mathrm{T}}

\newfont{\bbb}{msbm10 scaled 700}
\newcommand{\CCC}{\mbox{\bbb C}}

\newfont{\bb}{msbm10 scaled 1100}
\newcommand{\CC}{\mbox{\bb C}}
\newcommand{\PP}{\mbox{\bb P}}
\newcommand{\RR}{\mbox{\bb R}}
\newcommand{\QQ}{\mbox{\bb Q}}
\newcommand{\ZZ}{\mbox{\bb Z}}
\newcommand{\FF}{\mbox{\bb F}}
\newcommand{\GG}{\mbox{\bb G}}
\newcommand{\EE}{\mbox{\bb E}}
\newcommand{\NN}{\mbox{\bb N}}
\newcommand{\KK}{\mbox{\bb K}}
\newcommand{\HH}{\mbox{\bb H}}
\newcommand{\SSS}{\mbox{\bb S}}
\newcommand{\UU}{\mbox{\bb U}}
\newcommand{\VV}{\mbox{\bb V}}


\newcommand{\yy}{\mathbbm{y}}
\newcommand{\xx}{\mathbbm{x}}
\newcommand{\zz}{\mathbbm{z}}
\newcommand{\sss}{\mathbbm{s}}
\newcommand{\rr}{\mathbbm{r}}
\newcommand{\pp}{\mathbbm{p}}
\newcommand{\qq}{\mathbbm{q}}
\newcommand{\ww}{\mathbbm{w}}
\newcommand{\hh}{\mathbbm{h}}
\newcommand{\vvv}{\mathbbm{v}}

% Vectors

\newcommand{\av}{{\bf a}}
\newcommand{\bv}{{\bf b}}
\newcommand{\cv}{{\bf c}}
\newcommand{\dv}{{\bf d}}
\newcommand{\ev}{{\bf e}}
\newcommand{\fv}{{\bf f}}
\newcommand{\gv}{{\bf g}}
\newcommand{\hv}{{\bf h}}
\newcommand{\iv}{{\bf i}}
\newcommand{\jv}{{\bf j}}
\newcommand{\kv}{{\bf k}}
\newcommand{\lv}{{\bf l}}
\newcommand{\mv}{{\bf m}}
\newcommand{\nv}{{\bf n}}
\newcommand{\ov}{{\bf o}}
\newcommand{\pv}{{\bf p}}
\newcommand{\qv}{{\bf q}}
\newcommand{\rv}{{\bf r}}
\newcommand{\sv}{{\bf s}}
\newcommand{\tv}{{\bf t}}
\newcommand{\uv}{{\bf u}}
\newcommand{\wv}{{\bf w}}
\newcommand{\vv}{{\bf v}}
\newcommand{\xv}{{\bf x}}
\newcommand{\yv}{{\bf y}}
\newcommand{\zv}{{\bf z}}
\newcommand{\zerov}{{\bf 0}}
\newcommand{\onev}{{\bf 1}}

% Matrices

\newcommand{\Am}{{\bf A}}
\newcommand{\Bm}{{\bf B}}
\newcommand{\Cm}{{\bf C}}
\newcommand{\Dm}{{\bf D}}
\newcommand{\Em}{{\bf E}}
\newcommand{\Fm}{{\bf F}}
\newcommand{\Gm}{{\bf G}}
\newcommand{\Hm}{{\bf H}}
\newcommand{\Id}{{\bf I}}
\newcommand{\Jm}{{\bf J}}
\newcommand{\Km}{{\bf K}}
\newcommand{\Lm}{{\bf L}}
\newcommand{\Mm}{{\bf M}}
\newcommand{\Nm}{{\bf N}}
\newcommand{\Om}{{\bf O}}
\newcommand{\Pm}{{\bf P}}
\newcommand{\Qm}{{\bf Q}}
\newcommand{\Rm}{{\bf R}}
\newcommand{\Sm}{{\bf S}}
\newcommand{\Tm}{{\bf T}}
\newcommand{\Um}{{\bf U}}
\newcommand{\Wm}{{\bf W}}
\newcommand{\Vm}{{\bf V}}
\newcommand{\Xm}{{\bf X}}
\newcommand{\Ym}{{\bf Y}}
\newcommand{\Zm}{{\bf Z}}

% Calligraphic

\newcommand{\Ac}{{\cal A}}
\newcommand{\Bc}{{\cal B}}
\newcommand{\Cc}{{\cal C}}
\newcommand{\Dc}{{\cal D}}
\newcommand{\Ec}{{\cal E}}
\newcommand{\Fc}{{\cal F}}
\newcommand{\Gc}{{\cal G}}
\newcommand{\Hc}{{\cal H}}
\newcommand{\Ic}{{\cal I}}
\newcommand{\Jc}{{\cal J}}
\newcommand{\Kc}{{\cal K}}
\newcommand{\Lc}{{\cal L}}
\newcommand{\Mc}{{\cal M}}
\newcommand{\Nc}{{\cal N}}
\newcommand{\nc}{{\cal n}}
\newcommand{\Oc}{{\cal O}}
\newcommand{\Pc}{{\cal P}}
\newcommand{\Qc}{{\cal Q}}
\newcommand{\Rc}{{\cal R}}
\newcommand{\Sc}{{\cal S}}
\newcommand{\Tc}{{\cal T}}
\newcommand{\Uc}{{\cal U}}
\newcommand{\Wc}{{\cal W}}
\newcommand{\Vc}{{\cal V}}
\newcommand{\Xc}{{\cal X}}
\newcommand{\Yc}{{\cal Y}}
\newcommand{\Zc}{{\cal Z}}

% Bold greek letters

\newcommand{\alphav}{\hbox{\boldmath$\alpha$}}
\newcommand{\betav}{\hbox{\boldmath$\beta$}}
\newcommand{\gammav}{\hbox{\boldmath$\gamma$}}
\newcommand{\deltav}{\hbox{\boldmath$\delta$}}
\newcommand{\etav}{\hbox{\boldmath$\eta$}}
\newcommand{\lambdav}{\hbox{\boldmath$\lambda$}}
\newcommand{\epsilonv}{\hbox{\boldmath$\epsilon$}}
\newcommand{\nuv}{\hbox{\boldmath$\nu$}}
\newcommand{\muv}{\hbox{\boldmath$\mu$}}
\newcommand{\zetav}{\hbox{\boldmath$\zeta$}}
\newcommand{\phiv}{\hbox{\boldmath$\phi$}}
\newcommand{\psiv}{\hbox{\boldmath$\psi$}}
\newcommand{\thetav}{\hbox{\boldmath$\theta$}}
\newcommand{\tauv}{\hbox{\boldmath$\tau$}}
\newcommand{\omegav}{\hbox{\boldmath$\omega$}}
\newcommand{\xiv}{\hbox{\boldmath$\xi$}}
\newcommand{\sigmav}{\hbox{\boldmath$\sigma$}}
\newcommand{\piv}{\hbox{\boldmath$\pi$}}
\newcommand{\rhov}{\hbox{\boldmath$\rho$}}
\newcommand{\upsilonv}{\hbox{\boldmath$\upsilon$}}

\newcommand{\Gammam}{\hbox{\boldmath$\Gamma$}}
\newcommand{\Lambdam}{\hbox{\boldmath$\Lambda$}}
\newcommand{\Deltam}{\hbox{\boldmath$\Delta$}}
\newcommand{\Sigmam}{\hbox{\boldmath$\Sigma$}}
\newcommand{\Phim}{\hbox{\boldmath$\Phi$}}
\newcommand{\Pim}{\hbox{\boldmath$\Pi$}}
\newcommand{\Psim}{\hbox{\boldmath$\Psi$}}
\newcommand{\Thetam}{\hbox{\boldmath$\Theta$}}
\newcommand{\Omegam}{\hbox{\boldmath$\Omega$}}
\newcommand{\Xim}{\hbox{\boldmath$\Xi$}}


% Sans Serif small case

\newcommand{\Gsf}{{\sf G}}

\newcommand{\asf}{{\sf a}}
\newcommand{\bsf}{{\sf b}}
\newcommand{\csf}{{\sf c}}
\newcommand{\dsf}{{\sf d}}
\newcommand{\esf}{{\sf e}}
\newcommand{\fsf}{{\sf f}}
\newcommand{\gsf}{{\sf g}}
\newcommand{\hsf}{{\sf h}}
\newcommand{\isf}{{\sf i}}
\newcommand{\jsf}{{\sf j}}
\newcommand{\ksf}{{\sf k}}
\newcommand{\lsf}{{\sf l}}
\newcommand{\msf}{{\sf m}}
\newcommand{\nsf}{{\sf n}}
\newcommand{\osf}{{\sf o}}
\newcommand{\psf}{{\sf p}}
\newcommand{\qsf}{{\sf q}}
\newcommand{\rsf}{{\sf r}}
\newcommand{\ssf}{{\sf s}}
\newcommand{\tsf}{{\sf t}}
\newcommand{\usf}{{\sf u}}
\newcommand{\wsf}{{\sf w}}
\newcommand{\vsf}{{\sf v}}
\newcommand{\xsf}{{\sf x}}
\newcommand{\ysf}{{\sf y}}
\newcommand{\zsf}{{\sf z}}


% mixed symbols

\newcommand{\sinc}{{\hbox{sinc}}}
\newcommand{\diag}{{\hbox{diag}}}
\renewcommand{\det}{{\hbox{det}}}
\newcommand{\trace}{{\hbox{tr}}}
\newcommand{\sign}{{\hbox{sign}}}
\renewcommand{\arg}{{\hbox{arg}}}
\newcommand{\var}{{\hbox{var}}}
\newcommand{\cov}{{\hbox{cov}}}
\newcommand{\Ei}{{\rm E}_{\rm i}}
\renewcommand{\Re}{{\rm Re}}
\renewcommand{\Im}{{\rm Im}}
\newcommand{\eqdef}{\stackrel{\Delta}{=}}
\newcommand{\defines}{{\,\,\stackrel{\scriptscriptstyle \bigtriangleup}{=}\,\,}}
\newcommand{\<}{\left\langle}
\renewcommand{\>}{\right\rangle}
\newcommand{\herm}{{\sf H}}
\newcommand{\trasp}{{\sf T}}
\newcommand{\transp}{{\sf T}}
\renewcommand{\vec}{{\rm vec}}
\newcommand{\Psf}{{\sf P}}
\newcommand{\SINR}{{\sf SINR}}
\newcommand{\SNR}{{\sf SNR}}
\newcommand{\MMSE}{{\sf MMSE}}
\newcommand{\REF}{{\RED [REF]}}

% Markov chain
\usepackage{stmaryrd} % for \mkv 
\newcommand{\mkv}{-\!\!\!\!\minuso\!\!\!\!-}

% Colors

\newcommand{\RED}{\color[rgb]{1.00,0.10,0.10}}
\newcommand{\BLUE}{\color[rgb]{0,0,0.90}}
\newcommand{\GREEN}{\color[rgb]{0,0.80,0.20}}

%%%%%%%%%%%%%%%%%%%%%%%%%%%%%%%%%%%%%%%%%%
\usepackage{hyperref}
\hypersetup{
    bookmarks=true,         % show bookmarks bar?
    unicode=false,          % non-Latin characters in AcrobatÕs bookmarks
    pdftoolbar=true,        % show AcrobatÕs toolbar?
    pdfmenubar=true,        % show AcrobatÕs menu?
    pdffitwindow=false,     % window fit to page when opened
    pdfstartview={FitH},    % fits the width of the page to the window
%    pdftitle={My title},    % title
%    pdfauthor={Author},     % author
%    pdfsubject={Subject},   % subject of the document
%    pdfcreator={Creator},   % creator of the document
%    pdfproducer={Producer}, % producer of the document
%    pdfkeywords={keyword1} {key2} {key3}, % list of keywords
    pdfnewwindow=true,      % links in new window
    colorlinks=true,       % false: boxed links; true: colored links
    linkcolor=red,          % color of internal links (change box color with linkbordercolor)
    citecolor=green,        % color of links to bibliography
    filecolor=blue,      % color of file links
    urlcolor=blue           % color of external links
}
%%%%%%%%%%%%%%%%%%%%%%%%%%%%%%%%%%%%%%%%%%%



\begin{document}


\twocolumn[
% \icmltitle{Submission and Formatting Instructions for \\
           % International Conference on Machine Learning (ICML 2025)}
% \icmltitle{\mmscore: Can Large Multimodal Models Score the Similarity of Data Pairs?}
% \icmltitle{\mmscore: Systematic Framework to Choose Your Judge VLM}
\icmltitle{\mmscore: A Systematic Framework for Selecting Reliable Judge VLMs}
% \icmltitle{\mmscore: How to choose your Judge VLM Wisely}
% \icmltitle{\mmscore: Choose your Judge VLM Wisely}
% \icmltitle{\mmscore: Systematic Framework to Choose Your Judge VLM}
% \icmltitle{\mmscore: Choose your Judge VLM Wisely}
% \icmltitle{Assessing the Effectiveness of \\Large Multimodal Models as Similarity Kernels}
% \icmltitle{MMScore: Evaluating Multi-Modal Language Models as Customizable Similarity Kernels}
% \icmltitle{Benchmarking VLMs with simple data pairs}

% It is OKAY to include author information, even for blind
% submissions: the style file will automatically remove it for you
% unless you've provided the [accepted] option to the icml2025
% package.

% List of affiliations: The first argument should be a (short)
% identifier you will use later to specify author affiliations
% Academic affiliations should list Department, University, City, Region, Country
% Industry affiliations should list Company, City, Region, Country

% You can specify symbols, otherwise they are numbered in order.
% Ideally, you should not use this facility. Affiliations will be numbered
% in order of appearance and this is the preferred way.
\icmlsetsymbol{equal}{*}

\begin{icmlauthorlist}
\icmlauthor{Aarash Feizi}{ServiceNow Research,McGill,Mila}
\icmlauthor{Sai Rajeswar}{ServiceNow Research,Mila}
\icmlauthor{Adriana Romero-Soriano}{McGill,Mila}
\icmlauthor{Reihaneh Rabbany}{McGill,Mila}
\\
\icmlauthor{Spandana Gella}{ServiceNow Research}
\icmlauthor{Valentina Zantedeschi}{ServiceNow Research}
\icmlauthor{João Monteiro}{AutoDesk}
%\icmlauthor{}{sch}
%\icmlauthor{}{sch}
\end{icmlauthorlist}

% \icmlaffiliation{McGill}{Department of XXX, University of YYY, Location, Country}
% \icmlaffiliation{Mila}{Department of XXX, University of YYY, Location, Country}
% \icmlaffiliation{ServiceNow}{Company Name, Location, Country}
% \icmlaffiliation{AutoDesk}{School of ZZZ, Institute of WWW, Location, Country}
\icmlaffiliation{McGill}{McGill University}
\icmlaffiliation{Mila}{Mila}
\icmlaffiliation{ServiceNow Research}{ServiceNow Research}
\icmlaffiliation{AutoDesk}{AutoDesk}

\icmlcorrespondingauthor{Aarash Feizi}{aarash.feizi@servicenow.com}
\icmlcorrespondingauthor{João Monteiro}{joao.monteiro@autodesk.com}
% \icmlcorrespondingauthor{Firstname2 Lastname2}{first2.last2@www.uk}

% You may provide any keywords that you
% find helpful for describing your paper; these are used to populate
% the "keywords" metadata in the PDF but will not be shown in the document
\icmlkeywords{Machine Learning, ICML}

\vskip 0.3in
]

% this must go after the closing bracket ] following \twocolumn[ ...

% This command actually creates the footnote in the first column
% listing the affiliations and the copyright notice.
% The command takes one argument, which is text to display at the start of the footnote.
% The \icmlEqualContribution command is standard text for equal contribution.
% Remove it (just {}) if you do not need this facility.

\printAffiliationsAndNotice{}  % leave blank if no need to mention equal contribution
%\printAffiliationsAndNotice{\icmlEqualContribution} % otherwise use the standard text.

\begin{abstract}
  In this work, we present a novel technique for GPU-accelerated Boolean satisfiability (SAT) sampling. Unlike conventional sampling algorithms that directly operate on conjunctive normal form (CNF), our method transforms the logical constraints of SAT problems by factoring their CNF representations into simplified multi-level, multi-output Boolean functions. It then leverages gradient-based optimization to guide the search for a diverse set of valid solutions. Our method operates directly on the circuit structure of refactored SAT instances, reinterpreting the SAT problem as a supervised multi-output regression task. This differentiable technique enables independent bit-wise operations on each tensor element, allowing parallel execution of learning processes. As a result, we achieve GPU-accelerated sampling with significant runtime improvements ranging from $33.6\times$ to $523.6\times$ over state-of-the-art heuristic samplers. We demonstrate the superior performance of our sampling method through an extensive evaluation on $60$ instances from a public domain benchmark suite utilized in previous studies. 


  
  % Generating a wide range of diverse solutions to logical constraints is crucial in software and hardware testing, verification, and synthesis. These solutions can serve as inputs to test specific functionalities of a software program or as random stimuli in hardware modules. In software verification, techniques like fuzz testing and symbolic execution use this approach to identify bugs and vulnerabilities. In hardware verification, stimulus generation is particularly vital, forming the basis of constrained-random verification. While generating multiple solutions improves coverage and increases the chances of finding bugs, high-throughput sampling remains challenging, especially with complex constraints and refined coverage criteria. In this work, we present a novel technique that enables GPU-accelerated sampling, resulting in high-throughput generation of satisfying solutions to Boolean satisfiability (SAT) problems. Unlike conventional sampling algorithms that directly operate on conjunctive normal form (CNF), our method refines the logical constraints of SAT problems by transforming their CNF into simplified multi-level Boolean expressions. It then leverages gradient-based optimization to guide the search for a diverse set of valid solutions.
  % Our method specifically takes advantage of the circuit structure of refined SAT instances by using GD to learn valid solutions, reinterpreting the SAT problem as a supervised multi-output regression task. This differentiable technique enables independent bit-wise operations on each tensor element, allowing parallel execution of learning processes. As a result, we achieve GPU-accelerated sampling with significant runtime improvements ranging from $10\times$ to $1000\times$ over state-of-the-art heuristic samplers. Specifically, we demonstrate the superior performance of our sampling method through an extensive evaluation on $60$ instances from a public domain benchmark suite utilized in previous studies.

\end{abstract}

\begin{IEEEkeywords}
Boolean Satisfiability, Gradient Descent, Multi-level Circuits, Verification, and Testing.
\end{IEEEkeywords}
\section{Introduction}
\label{section:introduction}

% redirection is unique and important in VR
Virtual Reality (VR) systems enable users to embody virtual avatars by mirroring their physical movements and aligning their perspective with virtual avatars' in real time. 
As the head-mounted displays (HMDs) block direct visual access to the physical world, users primarily rely on visual feedback from the virtual environment and integrate it with proprioceptive cues to control the avatar’s movements and interact within the VR space.
Since human perception is heavily influenced by visual input~\cite{gibson1933adaptation}, 
VR systems have the unique capability to control users' perception of the virtual environment and avatars by manipulating the visual information presented to them.
Leveraging this, various redirection techniques have been proposed to enable novel VR interactions, 
such as redirecting users' walking paths~\cite{razzaque2005redirected, suma2012impossible, steinicke2009estimation},
modifying reaching movements~\cite{gonzalez2022model, azmandian2016haptic, cheng2017sparse, feick2021visuo},
and conveying haptic information through visual feedback to create pseudo-haptic effects~\cite{samad2019pseudo, dominjon2005influence, lecuyer2009simulating}.
Such redirection techniques enable these interactions by manipulating the alignment between users' physical movements and their virtual avatar's actions.

% % what is hand/arm redirection, motivation of study arm-offset
% \change{\yj{i don't understand the purpose of this paragraph}
% These illusion-based techniques provide users with unique experiences in virtual environments that differ from the physical world yet maintain an immersive experience. 
% A key example is hand redirection, which shifts the virtual hand’s position away from the real hand as the user moves to enhance ergonomics during interaction~\cite{feuchtner2018ownershift, wentzel2020improving} and improve interaction performance~\cite{montano2017erg, poupyrev1996go}. 
% To increase the realism of virtual movements and strengthen the user’s sense of embodiment, hand redirection techniques often incorporate a complete virtual arm or full body alongside the redirected virtual hand, using inverse kinematics~\cite{hartfill2021analysis, ponton2024stretch} or adjustments to the virtual arm's movement as well~\cite{li2022modeling, feick2024impact}.
% }

% noticeability, motivation of predicting a probability, not a classification
However, these redirection techniques are most effective when the manipulation remains undetected~\cite{gonzalez2017model, li2022modeling}. 
If the redirection becomes too large, the user may not mitigate the conflict between the visual sensory input (redirected virtual movement) and their proprioception (actual physical movement), potentially leading to a loss of embodiment with the virtual avatar and making it difficult for the user to accurately control virtual movements to complete interaction tasks~\cite{li2022modeling, wentzel2020improving, feuchtner2018ownershift}. 
While proprioception is not absolute, users only have a general sense of their physical movements and the likelihood that they notice the redirection is probabilistic. 
This probability of detecting the redirection is referred to as \textbf{noticeability}~\cite{li2022modeling, zenner2024beyond, zenner2023detectability} and is typically estimated based on the frequency with which users detect the manipulation across multiple trials.

% version B
% Prior research has explored factors influencing the noticeability of redirected motion, including the redirection's magnitude~\cite{wentzel2020improving, poupyrev1996go}, direction~\cite{li2022modeling, feuchtner2018ownershift}, and the visual characteristics of the virtual avatar~\cite{ogawa2020effect, feick2024impact}.
% While these factors focus on the avatars, the surrounding virtual environment can also influence the users' behavior and in turn affect the noticeability of redirection.
% One such prominent external influence is through the visual channel - the users' visual attention is constantly distracted by complex visual effects and events in practical VR scenarios.
% Although some prior studies have explored how to leverage user blindness caused by visual distractions to redirect users' virtual hand~\cite{zenner2023detectability}, there remains a gap in understanding how to quantify the noticeability of redirection under visual distractions.

% visual stimuli and gaze behavior
Prior research has explored factors influencing the noticeability of redirected motion, including the redirection's magnitude~\cite{wentzel2020improving, poupyrev1996go}, direction~\cite{li2022modeling, feuchtner2018ownershift}, and the visual characteristics of the virtual avatar~\cite{ogawa2020effect, feick2024impact}.
While these factors focus on the avatars, the surrounding virtual environment can also influence the users' behavior and in turn affect the noticeability of redirection.
This, however, remains underexplored.
One such prominent external influence is through the visual channel - the users' visual attention is constantly distracted by complex visual effects and events in practical VR scenarios.
We thus want to investigate how \textbf{visual stimuli in the virtual environment} affect the noticeability of redirection.
With this, we hope to complement existing works that focus on avatars by incorporating environmental visual influences to enable more accurate control over the noticeability of redirected motions in practical VR scenarios.
% However, in realistic VR applications, the virtual environment often contains complex visual effects beyond the virtual avatar itself. 
% We argue that these visual effects can \textbf{distract users’ visual attention and thus affect the noticeability of redirection offsets}, while current research has yet taken into account.
% For instance, in a VR boxing scenario, a user’s visual attention is likely focused on their opponent rather than on their virtual body, leading to a lower noticeability of redirection offsets on their virtual movements. 
% Conversely, when reaching for an object in the center of their field of view, the user’s attention is more concentrated on the virtual hand’s movement and position to ensure successful interaction, resulting in a higher noticeability of offsets.

Since each visual event is a complex choreography of many underlying factors (type of visual effect, location, duration, etc.), it is extremely difficult to quantify or parameterize visual stimuli.
Furthermore, individuals respond differently to even the same visual events.
Prior neuroscience studies revealed that factors like age, gender, and personality can influence how quickly someone reacts to visual events~\cite{gillon2024responses, gale1997human}. 
Therefore, aiming to model visual stimuli in a way that is generalizable and applicable to different stimuli and users, we propose to use users' \textbf{gaze behavior} as an indicator of how they respond to visual stimuli.
In this paper, we used various gaze behaviors, including gaze location, saccades~\cite{krejtz2018eye}, fixations~\cite{perkhofer2019using}, and the Index of Pupil Activity (IPA)~\cite{duchowski2018index}.
These behaviors indicate both where users are looking and their cognitive activity, as looking at something does not necessarily mean they are attending to it.
Our goal is to investigate how these gaze behaviors stimulated by various visual stimuli relate to the noticeability of redirection.
With this, we contribute a model that allows designers and content creators to adjust the redirection in real-time responding to dynamic visual events in VR.

To achieve this, we conducted user studies to collect users' noticeability of redirection under various visual stimuli.
To simulate realistic VR scenarios, we adopted a dual-task design in which the participants performed redirected movements while monitoring the visual stimuli.
Specifically, participants' primary task was to report if they noticed an offset between the avatar's movement and their own, while their secondary task was to monitor and report the visual stimuli.
As realistic virtual environments often contain complex visual effects, we started with simple and controlled visual stimulus to manage the influencing factors.

% first user study, confirmation study
% collect data under no visual stimuli, different basic visual stimuli
We first conducted a confirmation study (N=16) to test whether applying visual stimuli (opacity-based) actually affects their noticeability of redirection. 
The results showed that participants were significantly less likely to detect the redirection when visual stimuli was presented $(F_{(1,15)}=5.90,~p=0.03)$.
Furthermore, by analyzing the collected gaze data, results revealed a correlation between the proposed gaze behaviors and the noticeability results $(r=-0.43)$, confirming that the gaze behaviors could be leveraged to compute the noticeability.

% data collection study
We then conducted a data collection study to obtain more accurate noticeability results through repeated measurements to better model the relationship between visual stimuli-triggered gaze behaviors and noticeability of redirection.
With the collected data, we analyzed various numerical features from the gaze behaviors to identify the most effective ones. 
We tested combinations of these features to determine the most effective one for predicting noticeability under visual stimuli.
Using the selected features, our regression model achieved a mean squared error (MSE) of 0.011 through leave-one-user-out cross-validation. 
Furthermore, we developed both a binary and a three-class classification model to categorize noticeability, which achieved an accuracy of 91.74\% and 85.62\%, respectively.

% evaluation study
To evaluate the generalizability of the regression model, we conducted an evaluation study (N=24) to test whether the model could accurately predict noticeability with new visual stimuli (color- and scale-based animations).
Specifically, we evaluated whether the model's predictions aligned with participants' responses under these unseen stimuli.
The results showed that our model accurately estimated the noticeability, achieving mean squared errors (MSE) of 0.014 and 0.012 for the color- and scale-based visual stimili, respectively, compared to participants' responses.
Since the tested visual stimuli data were not included in the training, the results suggested that the extracted gaze behavior features capture a generalizable pattern and can effectively indicate the corresponding impact on the noticeability of redirection.

% application
Based on our model, we implemented an adaptive redirection technique and demonstrated it through two applications: adaptive VR action game and opportunistic rendering.
We conducted a proof-of-concept user study (N=8) to compare our adaptive redirection technique with a static redirection, evaluating the usability and benefits of our adaptive redirection technique.
The results indicated that participants experienced less physical demand and stronger sense of embodiment and agency when using the adaptive redirection technique. 
These results demonstrated the effectiveness and usability of our model.

In summary, we make the following contributions.
% 
\begin{itemize}
    \item 
    We propose to use users' gaze behavior as a medium to quantify how visual stimuli influences the noticebility of redirection. 
    Through two user studies, we confirm that visual stimuli significantly influences noticeability and identify key gaze behavior features that are closely related to this impact.
    \item 
    We build a regression model that takes the user's gaze behavioral data as input, then computes the noticeability of redirection.
    Through an evaluation study, we verify that our model can estimate the noticeability with new participants under unseen visual stimuli.
    These findings suggest that the extracted gaze behavior features effectively capture the influence of visual stimuli on noticeability and can generalize across different users and visual stimuli.
    \item 
    We develop an adaptive redirection technique based on our regression model and implement two applications with it.
    With a proof-of-concept study, we demonstrate the effectiveness and potential usability of our regression model on real-world use cases.

\end{itemize}

% \delete{
% Virtual Reality (VR) allows the user to embody a virtual avatar by mirroring their physical movements through the avatar.
% As the user's visual access to the physical world is blocked in tasks involving motion control, they heavily rely on the visual representation of the avatar's motions to guide their proprioception.
% Similar to real-world experiences, the user is able to resolve conflicts between different sensory inputs (e.g., vision and motor control) through multisensory integration, which is essential for mitigating the sensory noise that commonly arises.
% However, it also enables unique manipulations in VR, as the system can intentionally modify the avatar's movements in relation to the user's motions to achieve specific functional outcomes,
% for example, 
% % the manipulations on the avatar's movements can 
% enabling novel interaction techniques of redirected walking~\cite{razzaque2005redirected}, redirected reaching~\cite{gonzalez2022model}, and pseudo haptics~\cite{samad2019pseudo}.
% With small adjustments to the avatar's movements, the user can maintain their sense of embodiment, due to their ability to resolve the perceptual differences.
% % However, a large mismatch between the user and avatar's movements can result in the user losing their sense of embodiment, due to an inability to resolve the perceptual differences.
% }

% \delete{
% However, multisensory integration can break when the manipulation is so intense that the user is aware of the existence of the motion offset and no longer maintains the sense of embodiment.
% Prior research studied the intensity threshold of the offset applied on the avatar's hand, beyond which the embodiment will break~\cite{li2022modeling}. 
% Studies also investigated the user's sensitivity to the offsets over time~\cite{kohm2022sensitivity}.
% Based on the findings, we argue that one crucial factor that affects to what extent the user notices the offset (i.e., \textit{noticeability}) that remains under-explored is whether the user directs their visual attention towards or away from the virtual avatar.
% Related work (e.g., Mise-unseen~\cite{marwecki2019mise}) has showcased applications where adjustments in the environment can be made in an unnoticeable manner when they happen in the area out of the user's visual field.
% We hypothesize that directing the user's visual attention away from the avatar's body, while still partially keeping the avatar within the user's field-of-view, can reduce the noticeability of the offset.
% Therefore, we conduct two user studies and implement a regression model to systematically investigate this effect.
% }

% \delete{
% In the first user study (N = 16), we test whether drawing the user's visual attention away from their body impacts the possibility of them noticing an offset that we apply to their arm motion in VR.
% We adopt a dual-task design to enable the alteration of the user's visual attention and a yes/no paradigm to measure the noticeability of motion offset. 
% The primary task for the user is to perform an arm motion and report when they perceive an offset between the avatar's virtual arm and their real arm.
% In the secondary task, we randomly render a visual animation of a ball turning from transparent to red and becoming transparent again and ask them to monitor and report when it appears.
% We control the strength of the visual stimuli by changing the duration and location of the animation.
% % By changing the time duration and location of the visual animation, we control the strengths of attraction to the users.
% As a result, we found significant differences in the noticeability of the offsets $(F_{(1,15)}=5.90,~p=0.03)$ between conditions with and without visual stimuli.
% Based on further analysis, we also identified the behavioral patterns of the user's gaze (including pupil dilation, fixations, and saccades) to be correlated with the noticeability results $(r=-0.43)$ and they may potentially serve as indicators of noticeability.
% }

% \delete{
% To further investigate how visual attention influences the noticeability, we conduct a data collection study (N = 12) and build a regression model based on the data.
% The regression model is able to calculate the noticeability of the offset applied on the user's arm under various visual stimuli based on their gaze behaviors.
% Our leave-one-out cross-validation results show that the proposed method was able to achieve a mean-squared error (MSE) of 0.012 in the probability regression task.
% }

% \delete{
% To verify the feasibility and extendability of the regression model, we conduct an evaluation study where we test new visual animations based on adjustments on scale and color and invite 24 new participants to attend the study.
% Results show that the proposed method can accurately estimate the noticeability with an MSE of 0.014 and 0.012 in the conditions of the color- and scale-based visual effects.
% Since these animations were not included in the dataset that the regression model was built on, the study demonstrates that the gaze behavioral features we extracted from the data capture a generalizable pattern of the user's visual attention and can indicate the corresponding impact on the noticeability of the offset.
% }

% \delete{
% Finally, we demonstrate applications that can benefit from the noticeability prediction model, including adaptive motion offsets and opportunistic rendering, considering the user's visual attention. 
% We conclude with discussions of our work's limitations and future research directions.
% }

% \delete{
% In summary, we make the following contributions.
% }
% % 
% \begin{itemize}
%     \item 
%     \delete{
%     We quantify the effects of the user's visual attention directed away by stimuli on their noticeability of an offset applied to the avatar's arm motion with respect to the user's physical arm. 
%     Through two user studies, we identified gaze behavioral features that are indicative of the changes in noticeability.
%     }
%     \item 
%     \delete{We build a regression model that takes the user's gaze behavioral data and the offset applied to the arm motion as input, then computes the probability of the user noticing the offset.
%     Through an evaluation study, we verified that the model needs no information about the source attracting the user's visual attention and can be generalizable in different scenarios.
%     }
%     \item 
%     \delete{We demonstrate two applications that potentially benefit from the regression model, including adaptive motion offsets and opportunistic rendering.
%     }

% \end{itemize}

\begin{comment}
However, users will lose the sense of embodiment to the virtual avatars if they notice the offset between the virtual and physical movements.
To address this, researchers have been exploring the noticing threshold of offsets with various magnitudes and proposing various redirection techniques that maintain the sense of embodiment~\cite{}.

However, when users embody virtual avatars to explore virtual environments, they encounter various visual effects and content that can attract their attention~\cite{}.
During this, the user may notice an offset when he observes the virtual movement carefully while ignoring it when the virtual contents attract his attention from the movements.
Therefore, static offset thresholds are not appropriate in dynamic scenarios.

Past research has proposed dynamic mapping techniques that adapted to users' state, such as hand moving speed~\cite{frees2007prism} or ergonomically comfortable poses~\cite{montano2017erg}, but not considering the influence of virtual content.
More specifically, PRISM~\cite{frees2007prism} proposed adjusting the C/D ratio with a non-linear mapping according to users' hand moving speed, but it might not be optimal for various virtual scenarios.
While Erg-O~\cite{montano2017erg} redirected users' virtual hands according to the virtual target's relative position to reduce physical fatigue, neglecting the change of virtual environments. 

Therefore, how to design redirection techniques in various scenarios with different visual attractions remains unknown.
To address this, we investigate how visual attention affects the noticing probability of movement offsets.
Based on our experiments, we implement a computational model that automatically computes the noticing probability of offsets under certain visual attractions.
VR application designers and developers can easily leverage our model to design redirection techniques maintaining the sense of embodiment adapt to the user's visual attention.
We implement a dynamic redirection technique with our model and demonstrate that it effectively reduces the target reaching time without reducing the sense of embodiment compared to static redirection techniques.

% Need to be refined
This paper offers the following contributions.
\begin{itemize}
    \item We investigate how visual attractions affect the noticing probability of redirection offsets.
    \item We construct a computational model to predict the noticing probability of an offset with a given visual background.
    \item We implement a dynamic redirection technique adapting to the visual background. We evaluate the technique and develop three applications to demonstrate the benefits. 
\end{itemize}



First, we conducted a controlled experiment to understand how users perceived the movement offset while subjected to various distractions.
Since hand redirection is one of the most frequently used redirections in VR interactions, we focused on the dynamic arm movements and manually added angular offsets to the' elbow joint~\cite{li2022modeling, gonzalez2022model, zenner2019estimating}. 
We employed flashing spheres in the user's field of view as distractions to attract users' visual attention.
Participants were instructed to report the appearing location of the spheres while simultaneously performing the arm movements and reporting if they perceived an offset during the movement. 
(\zhipeng{Add the results of data collection. Analyze the influence of the distance between the gaze map and the offset.}
We measured the visual attraction's magnitude with the gaze distribution on it.
Results showed that stronger distractions made it harder for users to notice the offset.)
\zhipeng{Need to rewrite. Not sure to use gaze distribution or a metric obtained from the visual content.}
Secondly, we constructed a computational model to predict the noticing probability of offsets with given visual content.
We analyzed the data from the user studies to measure the influence of visual attractions on the noticing probability of offsets.
We built a statistical model to predict the offset's noticing probability with a given visual content.
Based on the model, we implement a dynamic redirection technique to adjust the redirection offset adapted to the user's current field of view.
We evaluated the technique in a target selection task compared to no hand redirection and static hand redirection.
\zhipeng{Add the results of the evaluation.}
Results showed that the dynamic hand redirection technique significantly reduced the target selection time with similar accuracy and a comparable sense of embodiment.
Finally, we implemented three applications to demonstrate the potential benefits of the visual attention adapted dynamic redirection technique.
\end{comment}

% This one modifies arm length, not redirection
% \citeauthor{mcintosh2020iteratively} proposed an adaptation method to iteratively change the virtual avatar arm's length based on the primary tasks' performance~\cite{mcintosh2020iteratively}.



% \zhipeng{TO ADD: what is redirection}
% Redirection enables novel interactions in Virtual Reality, including redirected walking, haptic redirection, and pseudo haptics by introducing an offset to users' movement.
% \zhipeng{TO ADD: extend this sentence}
% The price of this is that users' immersiveness and embodiment in VR can be compromised when they notice the offset and perceive the virtual movement not as theirs~\cite{}.
% \zhipeng{TO ADD: extend this sentence, elaborate how the virtual environment attracts users' attention}
% Meanwhile, the visual content in the virtual environment is abundant and consistently captures users' attention, making it harder to notice the offset~\cite{}.
% While previous studies explored the noticing threshold of the offsets and optimized the redirection techniques to maintain the sense of embodiment~\cite{}, the influence of visual content on the probability of perceiving offsets remains unknown.  
% Therefore, we propose to investigate how users perceive the redirection offset when they are facing various visual attractions.


% We conducted a user study to understand how users notice the shift with visual attractions.
% We used a color-changing ball to attract the user's attention while instructing users to perform different poses with their arms and observe it meanwhile.
% \zhipeng{(Which one should be the primary task? Observe the ball should be the primary one, but if the primary task is too simple, users might allocate more attention on the secondary task and this makes the secondary task primary.)}
% \zhipeng{(We need a good and reasonable dual-task design in which users care about both their pose and the visual content, at least in the evaluation study. And we need to be able to control the visual content's magnitude and saliency maybe?)}
% We controlled the shift magnitude and direction, the user's pose, the ball's size, and the color range.
% We set the ball's color-changing interval as the independent factor.
% We collect the user's response to each shift and the color-changing times.
% Based on the collected data, we constructed a statistical model to describe the influence of visual attraction on the noticing probability.
% \zhipeng{(Are we actually controlling the attention allocation? How do we measure the attracting effect? We need uniform metrics, otherwise it is also hard for others to use our knowledge.)}
% \zhipeng{(Try to use eye gaze? The eye gaze distribution in the last five seconds to decide the attention allocation? Basically constructing a model with eye gaze distribution and noticing probability. But the user's head is moving, so the eye gaze distribution is not aligned well with the current view.)}

% \zhipeng{Saliency and EMD}
% \zhipeng{Gaze is more than just a point: Rethinking visual attention
% analysis using peripheral vision-based gaze mapping}

% Evaluation study(ideal case): based on the visual content, adjusting the redirection magnitude dynamically.

% \zhipeng{(The risk is our model's effect is trivial.)}

% Applications:
% Playing Lego while watching demo videos, we can accelerate the reaching process of bricks, and forbid the redirection during the manipulation.

% Beat saber again: but not make a lot of sense? Difficult game has complicated visual effects, while allows larger shift, but do not need large shift with high difficulty




\section{Related Work}

% [what are some different benchmarks]

% [how are people evaluating vlms]

% [encoders are also being used as judges (text-image and image-image)]

% [what benchmarks are using vlms as judges]

% [literature on looking into invariance in LMMs]


% [downsides of using clip and other models (do not get negation), look at it as bag of words, spatially flawed (whatsup), looking into propperties that are not captures in benchmarks]
Using language models as automatic evaluators has become a somewhat common practice with popular approaches such as \textsc{GPTScore} and G-eval~\citep{fu2023gptscore,liu2023geval} being used to rank responses in the NLP domain. 
Due to that, there has been a significant amount of recent work that has investigated the capabilities and limitations of using LLMs as judges~\citep{thakur2024judging, chiang2023can, murugadoss2024evaluating, shankar2024validates}. 
\citet{chiang2023can} have shown that LLM evaluations are consistent and reproducible, making them suitable alternatives for human evaluation, they argue that these models inherent biases should prevent them using independently rather than \textit{alongside} human experts. 
Furthermore, \citet{zheng2023judging} reveal that large \modelss{{}, e.g., GPT-4 Turbo, align well with human judgments and \citet{thakur2024judging} further states that simpler models may still outperform GPT-4 Turbo in ranking tasks due to superior alignment metrics. Also, recent work assessed how humans can help LLMs evaluate better by testing different instruction types or designing tools that result in more balanced evaluations~\citep{murugadoss2024evaluating, shankar2024validates}.

It is worth noting that known limitations of LLMs such as their lack of invariance to the order of examples given in a prompt, which is a well studied issue of natural language models~\citep{fang2024rethinking}, 
% and also observed in the multimodal case by the lack of symmetry we reported,
may render auto evaluation unreliable. Similarly, \citet{berglund2023reversal} show failure cases where models trained on unidirectional relationships do not infer the reverse, indicating systemic limitations even in state-of-the-art LLMs such as \textsc{GPT-4} (as seen in Figure~\ref{fig:fig1} and in Appendix \ref{sec:error-analysis} for \modelss{}). Our main goal is
to investigate the reliability of automated evaluation in the multimodal context, by probing the models to compare data pairs. 
%thus to assess to what extent auto-evaluation can be done reliably, focusing in the multimodal case, by probing models on their ability to compare.

Namely, the evaluations we carry out focus on testing in multiple different ways how good \modelss{} are when it comes to comparing data instances, such as whether \modelss{} prompted to compare are symmetric or smooth for instance, and to what extent they can be controlled, i.e., instructed to pay attention to or ignore certain features of the inputs. While the literature is more sparse regarding testing \modelss{} in this setting, recent work has tested for something along those lines. \citet{chen2024mllm} for instance propose a benchmark for evaluating \modelss{} in multiple different scenarios, including checking whether pairwise comparisons of responses to a query correlated with human judgments. They concluded that although correlations are relatively high on comparison tasks, biases and inconsistencies affect performance on pair scoring and batch ranking. Similarly, \citet{awal2024vismin} introduced a synthetic dataset containing paired images that differ only along one feature (e.g., the color of an object). We seek to add to this branch of the literature by introducing a framework where controlled experiments can be carried out to anticipate the performance of models when being used as judges, and various different characteristics of automatic judges can be identified (e.g., how smooth they are).

Unlike the case of generative \modelss{} discussed above, discriminative visual language models such as CLIP~\citep{radford2021learning} are covered by a significant amount of recent work, and several failure modes are well reported, mostly deriving from the fact this class of \modelss{} tends to behave as bag-of-words models, focusing on nouns and ignoring relationships and semantics in their input data~\citep{yuksekgonul2023when}. For instance, CLIP was observed to struggle with spatial reasoning~\citep{kamath2023s} and ignore negation~\citep{alhamoud2025vision}. On the other hand, fine-tuning CLIP to reason about pairwise differences \citet{sam2024finetuning} showed that discriminative \modelss{} can improve on how well they manage to reason about pairwise differences if training is tailored for enabling so, highlighting the benefits that being able to measure these skills may inform training and improve models as a consequence. \citet{ouali2024discriminative} showed that fine-tuning generative \modelss{} to turn them into discriminative models results in improved image-retrieval from text, which aligns with results we reported in Section~\ref{sec:ecoders_vs_decoders} showing a gap between open-sources \modelss{} and CLIP-style encoders.

\begin{figure*}[ht]
    \centering
    % \includegraphics[width=0.95\linewidth,trim={.1cm .2cm .2cm .2cm},clip]{imgs/mmscore-examples2.pdf}
    \includegraphics[width=0.95\linewidth,trim={.1cm .2cm .2cm .2cm},clip]{imgs/fig2-v2.pdf}
    \caption{Examples of transformations (green boxes) applied to the original data points (gray boxes) of each subset instantiated with \mmscore. The first row shows the different splits of \mmscorecoco{} and \mmscorein, the second row for \mmscorewuimgimg, and the third for \mmscorewuimgtext.}
    \label{fig:mmscore-examples}
\end{figure*}
\vspace{-2mm}
\section{\mmscore{}}
\subsection{Dataset Creation}
The \mmscore{} framework takes in a source dataset and creates augmented versions of the data to obtain data pairs to probe the evaluation skills of a model. 
In our instances, we use \coco{} \cite{lin2014microsoft}, \imagenet{} \cite{deng2009imagenet} and \wu{} \cite{kamath2023s} datasets as the source for the original data points. We utilize \coco{} and \imagenet{} as image-only datasets and \wu{} as an image-text dataset. We select 500 random images from each of \coco{} and \imagenet{} and all the image-text pairs from both subsets provided by the \wu{} dataset to be used in our instantiation of \mmscore. Full details of our released datasets are given in Table \ref{tab:mmscore-info}.

% Considering that the \mmscore{} framework aims to measure how well \modelss{} can detect different features and score them between data points, 
To isolate the effect of different data characteristics on model performance,
\mmscore{} creates pairs of image-image and image-text data that are identical except for one or a few controlled features. The generated data consists of points from the original dataset paired with their transformed version.
% these image-image pairs by using all three source datasets and construct the image-text pairs using the two subsets of \wu{}. 
For \coco{} and \imagenet, we create a different control sample for each one of the transformations in $\{$color jitter, rotation, gaussian blur, perspective shift, elastic transformation$\}$, which defines the characteristic that differs between images. For the data from \wu{}, we construct the data pairs by either only using the `spatial position' transform, or `spatial position' transform in addition to one of the previous five characteristics to additionally assess coupling effects. 
However, since transforms are not well-defined for texts, only `spatial position' transform is applied for the image-text pairs. Note that the image-image pairs from \wu{} are the most challenging since they all have at least the `spatial position' transform, which is a well-known blind-spot of \modelss{} as shown by previous literature \cite{kamath2023s, wang2024picture}. As a result, we end up creating five image-image sub-datasets for each of \coco{} and \imagenet, six subsets for each of the two subsets of \wu{}, using each of the transformations, and one image-text sub-dataset for each of the subsets of \wu{}. The details of the transforms applied to each category are shown in Figure \ref{fig:mmscore-examples}.

Next, for each original image, we construct three types of pairs: an identical, a transformed, and an irrelevant pair. In all three versions of these pairs, the first data point is the original (non-transformed) image. For the `identical' pair, the second data point is another version of the image with $95\%$ of its original size for the image-image pair and the correct caption for the image-text pair. The second data point in the `transformed' pair is the original image (caption) with the transformation applied to it for the image-image (image-text) pair. Finally, the `irrelevant' pair's second data point is a transformed version of a random image (caption) from the rest of the dataset. 

Equipped with the constructed control samples, \mmscore{} prompts the \model{} to score the similarity of each data pair based on a set of criteria. 
The criteria consists of the conditions indicating whether the model under examination should be `sensitive' or `invariant' to the transformations applied for that specific sub-dataset. These two settings (sensitive or invariant) measure how well each model can recognize the differences between the data pair and follow the prompt's criteria. If a model can successfully capture a specific feature, it will have no problem being variant or invariant to it; however, if it cannot detect it or has a bias towards a feature, it will favor being sensitive or invariant to that feature over its opposite. On a scale of 1 to 10, we consider the ground-truth score of the `identical' and `irrelevant' pair 10 and 1, respectively, in both `sensitive' and `invariant' settings. However, for the `transformed' pair, we consider the score 10 in the `invariant' version, and `8' in the `sensitive' version of the prompt. 
To make sure the performance gap between models is not merely a consequence of biased prompt wording, \mmscore{} comes with five template prompts with different lengths and wordings but with the same semantic meaning, that are randomly selected for each data pair, to make sure the prompting does not affect the model's performance. These prompt templates are reported in Appendix \ref{sec:prompt-templates}.

Ultimately, we end up with 4 different datasets created by \mmscore: \mmscorecoco{}, \mmscorein{}, \mmscorewuimgimg{}, and \mmscorewuimgtext{}. \mmscorecoco{} and \mmscorein{} compare and score image-pairs and have 5 splits (Color Jitter (CJ), Rotation (R), Gaussian Blur (GB), Perspective Shift (PS), and Elastic Transformation (ET). \mmscorewuimgimg{} consists of 2 subsets, each with 6 splits; one split with only the Spatial Position transform (SP), and the rest with SP combined with one of the previous five transformations (CJ, R, GB, PS, and ET). \mmscorewuimgtext{} consists of only the SP split for each of the two subsets in the \wu{} dataset. Details of each split in Appendix \ref{sec:mmscore-info}.

\subsection{Metrics}
To measure the reliability of \modelss{} in scoring data pairs, we define four metrics that we measure across datasets and models: \nmi, \relaxsym, Smoothness (\smoothness), and Controllability (\control). 

% We follow this notation to formulate the metrics: we denote the \model{} being evaluated as $\mathcal{M}$ and the prompt as $P_{C}$ where $C \in \left\{\texttt{sens}, \texttt{inv} \right\}$, as sensitive or invariant. Finally, given a dataset $\mathcal{D}_N = \{(d_1, d_2), (d_3, d_4), \dots, (d_{2N-1}, d_{2N}))\}$, we denote the similarity score of a data pair $(d_i, d_j) \in \mathcal{D}_N$ returned by an \model{} ($\mathcal{M}$) and for a given prompt ($C$) as:
% % $$s_{\mathcal{M}}^{C}(d_i, d_j) \vcentcolon= \mathcal{M}(C, d_i, d_j),$$
% $$s_{\mathcal{M}}^{C}(d_i, d_j) \vcentcolon= \mathcal{M}(C, d_i, d_j),$$

We adopt the following notation to formulate the metrics: we denote the \model{} being evaluated as $\mathcal{M}$ and the condition, which determines if the prompt instructs the model to be sensitive or invariant to a visual feature, as $C \in \left\{\texttt{sens}, \texttt{inv} \right\}$. Finally, given a dataset $\mathcal{D}_N = \{(d_1, d_2), (d_3, d_4), \dots, (d_{2N-1}, d_{2N}))\}$, we denote the similarity score of a data pair $(d_i, d_j) \in \mathcal{D}_N$ returned by an \model{} ($\mathcal{M}$) for a given condition ($C$) as:
% $$s_{\mathcal{M}}^{C}(d_i, d_j) \vcentcolon= \mathcal{M}(C, d_i, d_j),$$
$$s_{\mathcal{M}}^{C}(d_i, d_j) \vcentcolon= \mathcal{M}(C, d_i, d_j),$$
where $(d_i, d_j)$ could be an image-image or image-text pair. Note that we instruct the model to generate the output in a structured format to make sure the predicted score is parsable from the model output. If $s_{\mathcal{M}}^{C}(d_i, d_j)$ is valid, it would fall in the set
% $\mathcal{V} = [1, 2, 3, \dots, 10\}$. 
$\mathcal{V} = [1, 10]$. 
However, models often do not consistently follow the details of the prompt and may produce scores not in $\mathcal{V}$ or outputs that do not satisfy the output format, in which case we set $s_{\mathcal{M}}^{C}(d_i, d_j) = -1$. 

Finally, to evaluate a model $\mathcal{M}$ on $\mathcal{D}_N$ given condition $C$, we create and annotate the set of all its outputs as:
$$S_{\mathcal{M}}^{C}(\mathcal{D}_N) = \left\{ s_{\mathcal{M}}^{C}(d_i, d_j) \,\middle|\, (d_i, d_j) \in \mathcal{D}_N \cup \text{rev}(\mathcal{D}_N) \right\},
$$
where $\text{rev}(\mathcal{D}_N) = \{(d_2, d_1), (d_4, d_3), \dots, (d_{2N}, d_{2N-1}))\}$ are the data pairs in reverse order.



\subsubsection{\nmi{}}
We consider the normalized mutual information (\nmi{}) between the predicted scores and the ground-truth ones as the main metric of \mmscore{}. Instead of accuracy or squared error metric, we consider \nmi{} since we do not explicitly prompt the \model{} with examples of the correct scores and hence, cannot expect it to predict them directly. 
\nmi{} is better suited for \mmscore{} as it focuses on whether the \model{}'s scores are predictive of the ground-truth ones without penalizing outputs that do not exactly match them. 
The better a model can reproduce the variance in the ground-truth score, the better it is able to recognize that characteristic. Hence we write,
% \vspace{-1mm}
\[
\nmi(\mathcal{M}, C, \mathcal{D}_N) = \text{NMI}(S_{\mathcal{M}}^{C}(\mathcal{D}_N), GT_C(\mathcal{D}_N)),
\]

where $\text{NMI}(.,.)$ is the normalized mutual information and $GT_C(.)$ is the ground truth of the input dataset considering the condition of $C$.



% \subsubsection{\softsymmetrytitle (\normsym)}
\vspace{-1mm}
\subsubsection{\relaxsym{}}

When leveraging \modelss{} as similarity kernels or auto evaluators, a fundamental characteristic one would expect is their symmetry as a kernel.  Surprisingly, however, we found that most models do not satisfy exact symmetry, i.e., the equality of $sim(a, b)$ and $sim(b, a)$. We thus introduce \relaxsym, which tolerates a difference of $\varepsilon$ between the scores that should be equal. More specifically, to analyze the symmetry of \modelss{} on a dataset $\mathcal{D}_N$, we compute the \relaxsym{} of $\mathcal(M)$ on $\mathcal{D_N}$:

% $$\text{\normsym}(\mathcal{M}, \mathcal{D}_N) = 1 - \frac{1}{N}\sum_{(d_i, d_j) \in \mathcal{D}_N}\text{Diff}(\mathcal{M}, d_i, d_j),$$
\resizebox{\linewidth}{!}{$
\text{\relaxsym}(\mathcal{M}, \mathcal{D}_N) = \frac{1}{N}\sum_{(d_i, d_j) \in \mathcal{D}_N}\text{SoftEq}_\varepsilon(\mathcal{M}, d_i, d_j),
$}

where $\text{SoftEq}_\varepsilon(\mathcal{M}, d_i, d_j)$ is defined as:
$$
% \text{Diff}(\mathcal{M}, d_i, d_j)=
\text{SoftEq}_\varepsilon(\mathcal{M}, d_i, d_j)=
$$
\resizebox{\linewidth}{!}{$
% Diff(\mathcal{M}, d_i, d_j) &=
\begin{cases} 
    % \frac{\lvert s_{\mathcal{M}}^{P_.}(d_i, d_j) - s_{\mathcal{M}}^{P_.}(d_j, d_i)\rvert}{9}, & s_{\mathcal{M}}^{P_.}(d_i, d_j), s_{\mathcal{M}}^{P_.}(d_j, d_i) \in \mathcal{V}, \\
    % 1, & \text{otherwise}.
    \mathbbm{1}(\lvert s_{\mathcal{M}}^{C}(d_i, d_j)-s_{\mathcal{M}}^{C}(d_j, d_i) \le \varepsilon \rvert), & s_{\mathcal{M}}^{C}(d_i, d_j), s_{\mathcal{M}}^{C}(d_j, d_i) \in \mathcal{V}, \\
    0, & \text{otherwise}.
\end{cases}
$}


In the continuation of this paper, we set $\varepsilon = 1$ and provide ablation studies in Figure \ref{fig:diff-relax-sym-eps} in the Appendix.

\vspace{-2mm}
\subsubsection{\textbf{Smoothness}}

We aim to measure how smooth kernels induced by \modelss{} are. For instance, a non-smooth kernel would be such that pairs are either exactly the same or completely different, while a smoother kernel is more nuanced. We measure for smoothness via the diversity of the predicted scores. Given $S^{C}_{\mathcal{M}}$, smoothness (\smoothness) is computed as:
% \begin{align}
$$\smoothness(\mathcal{M}, \mathcal{D}_N, C) = Ent(\left\{s \,\middle|\, s \in S_{\mathcal{M}}^{C}(\mathcal{D}_N)  \,\text{and}\, s \in \mathcal{V}\right\}),$$
% \end{align}

where $Ent(.)$ is the entropy of a set relative to its support, i.e., the set of candidate inputs.

\subsubsection{\textbf{Controllability}}

To measure how responsive a model is to the given prompt, we define a metric based on the difference of its \nmi{} in the sensitive and invariant settings. The more controllable a model is, the less discrepancy is observed between the \texttt{sens} and \texttt{invar} settings. Hence, when measuring the controllability on $\mathcal{D}_N$ for a model $\mathcal{M}$ is defined as
\vspace{-5mm}

$$ \control({\mathcal{M}, \mathcal{D}_N}) = $$
\resizebox{\linewidth}{!}{$
\frac{|\nmi(\mathcal{M}, \texttt{sens}, \mathcal{D}_N) - \nmi(\mathcal{M}, \texttt{inv}, \mathcal{D}_N)|}{\sqrt{(\nmi(\mathcal{M}, \texttt{sens}, \mathcal{D}_N) \times \nmi(\mathcal{M}, \texttt{inv}, \mathcal{D}_N))}}$}.

% $$ C_{\mathcal{M}, f} = 1 - \frac{|\mathcal{M}_{sens}(f) - \mathcal{M}_{invar}(f)|}{max(\mathcal{M}_{sens}(f), \mathcal{M}_{invar}(f))}$$






\definecolor{light blue}{RGB}{215, 242, 252}
\definecolor{light purple}{RGB}{247, 215, 252}
\definecolor{light orange}{rgb}{0.9961, 0.875, 0.7188}



\section{Results}
\label{4 results}

We first extensively compare rewrite strategies focusing on the overall translation quality achieved by MT-Agnostic rewrites (\S \ref{simplification best}) and Translatability-Aware rewrites (\S \ref{input selection}). To understand how rewrites change translations, we then analyze the trade-offs between translatability and meaning preservation (\S \ref{pareto optimality}). Finally, we test whether the best-performing methods identified so far generalize to new language pairs (\S \ref{sec:newlanguages}).


\subsection{Simplifying Inputs Works Best}
\label{simplification best}
We first compare the \hl{MT Agnostic} rewriting methods: simplification, paraphrasing, and stylistic edits. Due to space limits, we show the best and worst performing variations for each input rewriting method based on the overall translation quality metric \textsc{xCOMET}$(s,t,r)$ for each language pair in Table~\ref{tab:main_results}. Full results are available in Appendix \ref{appendix:detailed results}.

Results show that all rewriting strategies improve translatability, but only \textbf{simplification} also improves the overall translation quality. Even the lowest performing rewrites reach higher translatability than the original baseline. Each method surpasses the baseline by up to 0.056 and 0.027 \textsc{xCOMET}$(s,t)$ average scores for \textsc{En-De}, up to 0.058 and 0.036 average scores for \textsc{En-Ru}, and up to 0.054 and 0.028 average scores for \textsc{En-Zh} pair. Trends are consistent with \textsc{MetricX}$(s,t)$. However, making inputs easier to translate often degrades quality when comparing against references $r$. Simplification with \textsc{Tower-Instruct} distinguishes itself by improving translation quality based on \textsc{xCOMET}$(s,t,r)$ scores and maintaining it according to the \textsc{MetricX}$(t,r)$ scores \---\ a harder metric to improve since the reference might be biased toward the original wording of the source.

%However, as detailed in Appendix Table~\ref{tab:detailed_results_ende} to \ref{tab:detailed_results_enzh}, this improvement comes at the cost of lower meaning preservation score, resulting in lower overall translation quality \textsc{xCOMET}$(s,t,r)$ scores for most methods. A notable exception is simplification, which outperforms the the original MT in the \textsc{xCOMET}$(s,t,r)$ metric.

Among the three LLMs used for simplification, \textsc{Tower-Instruct} achieves the best translation quality, while \textsc{LLaMA-3} excels in translatability at the expense of meaning preservation. Interestingly, there is no benefit to using a separate LLM, even one fine-tuned specifically on paraphrasing or style edits such as \textsc{DIPPER} or \textsc{CoEdIT}. Overall, the best performing method for MT-agnostic rewrites is simplification with \textsc{Tower-Instruct}, the same model we use as our MT system. We attribute this to \textsc{Tower-Instruct} being instruction fine-tuned on translation related tasks (but not simplification) and having more domain knowledge of the WMT dataset used in our evaluation.\footnote{\url{https://huggingface.co/datasets/Unbabel/TowerBlocks-v0.1}} 


% Further details are provided in Appendix \ref{appendix:impact of llm} and \ref{appendix:same llm}.

As shown in Table~\ref{tab:main_results}, simplifying with \textsc{Tower-Instruct} still holds the top spot when compared to \hlpurple{Task-Aware} rewriting methods, as indicated by higher \textsc{xCOMET}$(s,t,r)$ scores. This suggests that injecting knowledge about the end-task (MT) to LLMs is less effective than simplifying inputs to improve translation quality.

Overall, these results confirm the intuition that simpler text is easier to translate, but establish that rewrites are not uniformly helpful for translation quality, motivating the need for more selective input rewriting strategies.

%We observe similar trends with our secondary evaluation metric, \textsc{MetricX}. Simplification with \textsc{Tower-Instruct} consistently improves both \textsc{MetricX}$(s,t)$ and \textsc{MetricX}$(t,r)$ scores over the baseline; -0.534 and -0.015 for \textsc{En-De}, -1.4 and -0.107 for \textsc{En-Ru}, and -1.924 and -0.055 for \textsc{En-Zh}. However, note that the latter score may be inherently biased toward the original MT outputs since it does not use the source as part of input.

\begin{figure*}
    \centering
    \includegraphics[width=\linewidth]{figures/pareto_vis.pdf}
    \caption{Pareto frontier per language pair. For each subplot, the $x$-axis is the translatability and $y$-axis is the meaning preservation scores. Pareto frontier (\textbf{dashed} line) visualizes the optimal solutions that take into account the trade-off between the two metrics. Each shape represents different rewriting methods and each color represent specific prompt or model variation.}
    \label{fig:pareto_frontier}
\end{figure*}


\subsection{Selection via Translatability Improves MT}
\label{input selection}

We evaluate the impact of inference-time selection based on \hlorange{translatability} scores (\textit{Selection} in Table~\ref{tab:main_results}), and compare it further with the more expensive supervised fine-tuning strategy (\textit{Fine-tune}). 

All language pairs consistently benefit from selection. Translation quality improves significantly, with average \textsc{xCOMET}$(s,t,r)$ gains of 0.024 for \textsc{En-De}, 0.031 for \textsc{En-Ru}, and 0.025 for \textsc{En-Zh}, marking the best performance among all variants. \textsc{MetricX}$(t,r)$ scores confirm this trend, showing average improvements of 0.073 for \textsc{En-De}, 0.198 for \textsc{En-Ru}, and 0.076 for \textsc{En-Zh}. At the segment level, rewrites are preferred to original inputs in 1197/1557 cases for \textsc{En-De}, 1610/2074 cases for \textsc{En-Ru}, and 2163/3074 cases for \textsc{En-Zh}. Fine-tuning shows smaller gains compared to MT-Agnostic or Task-Aware methods, both in terms of translatability and translation quality, despite being more resource-intensive.

In summary, the results suggest that inference-time selection of inputs based on translatability scores is a promising strategy, outperforming MT-agnostic rewrites and rewrites obtained via a more expensive fine-tuning process.
% \footnote{We also found no benefit from adding preference learning with Direct Preference Optimization \citep{rafailov2023direct}. More details are provided in Appendix \ref{appendix:dpo}.}

\subsection{Input Rewriting Trades Off Translatability and Meaning Preservation}
\label{pareto optimality}

We observe a moderate negative correlation between translatability and meaning preservation scores, with Pearson coefficients of -0.48, -0.66, and -0.52 for \textsc{En-De}, \textsc{En-Ru}, and \textsc{En-Zh}, respectively. This trade-off between the two metrics poses a Pareto optimization challenge: when a rewrite is easier to translate, it often results in lower meaning preservation. Therefore, we aim to find Pareto optimal solutions, which balance these trade-offs on a Pareto frontier \citep{huang-etal-2023-towards}.\footnote{In Pareto optimization, Pareto optimal solutions are those where no single solution outperforms another in all tasks \citep{pareto}. The set of Pareto optimal solutions forms the Pareto frontier.}

In Figure~\ref{fig:pareto_frontier}, we visualize our two objectives, translatability and meaning preservation, on each axis and identify the Pareto frontier. The results are consistent with the overall translation quality metric, \textsc{xCOMET}$(s,t,r)$, where the scores for rewriting methods on the Pareto frontier are consistently the same as or on par with the original baseline. This also aligns with our earlier findings from comparing MT-Agnostic and Task-Aware rewrites (\S \ref{simplification best}), where simplification with \textsc{Tower-Instruct} lies on the Pareto frontier for \textsc{En-De} and \textsc{En-Ru}. Even for \textsc{En-Zh}, although this does not lie on the frontier, it has a higher \textsc{xCOMET}$(s,t,r)$ score (0.802) than the original baseline (0.794). Furthermore, the best rewriting method according to \textsc{xCOMET}$(s,t,r)$, translatability-based selection (\S \ref{input selection}), always lies on the Pareto frontier across all language pairs.




\begin{table}[!htp]
\centering
\resizebox{\linewidth}{!}{%
    \begin{tabular}{l l l l l l l}
    \specialrule{1.3pt}{0pt}{0pt}
    \textbf{Language} & \textbf{Type} & \textbf{\textsc{x}$(s,t)$} & \textbf{\textsc{x}$(s,t,r)$} & \textbf{\textsc{M}$(s,t)$} & \textbf{\textsc{M}$(t,r)$} \\
    \toprule
        
    \multirow{3}{*}{\textbf{\large{\textsc{en-cs}}}} & Original & 0.646 & 0.655 & 5.376 & 4.493 \\
    & Simplification & 0.691 & 0.675 & 4.684 & 4.333\\
    & Selection & \textbf{0.736} & \textbf{0.718} & \textbf{4.152} & \textbf{3.663}\\ \midrule

    \multirow{3}{*}{\textbf{\large{\textsc{en-he}}}} & Original & 0.327 & 0.320 & 16.66 & 15.48 \\
    & Simplification & 0.351 & 0.332 & 15.97 & 15.43 \\
    & Selection & \textbf{0.389} & \textbf{0.363} & \textbf{15.39} & \textbf{14.51} \\ \midrule

    \multirow{3}{*}{\textbf{\large{\textsc{en-ja}}}} & Original & 0.746 & 0.718 & 3.514 & 2.688 \\
    & Simplification & 0.789 & 0.738 & 2.957 & 2.508 \\
    & Selection & \textbf{0.826} & \textbf{0.769} & \textbf{2.781} & \textbf{2.273} \\

    % \multirow{3}{*}{\textbf{\large{\textsc{en-ru}}}} & Original & 0.876 & 0.872 & 2.504 & 1.947 \\
    % & Simplification & 0.911 & 0.891 & 2.046 & 1.840 \\
    % & Selection & \textbf{0.924} & \textbf{0.901} & \textbf{2.015} & \textbf{1.765} \\ \midrule

    % \multirow{3}{*}{\textbf{\large{\textsc{en-zh}}}} & Original & 0.811 & 0.837 & 3.003 & 1.743 \\
    % & Simplification & 0.840 & 0.842 & 2.649 & 1.817 \\
    % & Selection & \textbf{0.857} & \textbf{0.868} & \textbf{2.612} & \textbf{1.689} \\

    \specialrule{1.3pt}{0pt}{0pt}
    \end{tabular}
}
\caption{Results of simplification and translatability-based selection for held-out test sets. We abbreviate \textsc{xCOMET} to \textbf{\textsc{x}} and \textsc{MetricX} to \textbf{\textsc{M}} due to space constraints. Best scores for each metric is \textbf{bold}.} 
\label{tab:heldout}
\end{table}

\subsection{Best Input Rewriting Strategy Improves MT on Held-out Test sets}
\label{sec:newlanguages}

We evaluate whether the top methods that have emerged from the controlled empirical comparison conducted so far generalize to further test settings. As shown in Table~\ref{tab:heldout}, we test both simplification with \textsc{Tower-Instruct} (\textit{Simplification}) and translatability-based input selection (\textit{Selection}) on new test sets from the  WMT-23 General MT task, English-Czech (\textsc{En-Cs}), English-Hebrew (\textsc{En-He}), and English-Japanese (\textsc{En-Ja}) to assess generalization to lower-resource target languages.

Both simplification and translatability-based selection lead to progressive improvements in translation quality, as measured by \textsc{xCOMET}$(s,t,r)$. Notably, the selection strategy tends to excel in language pairs with lower-resource target languages, showing translation quality gains of 0.064, 0.043, 0.051 scores for \textsc{En-Cs}, \textsc{En-He}, \textsc{En-Ja}, respectively, compared to increases of 0.017, 0.031, and 0.025 for \textsc{En-De}, \textsc{En-Ru} and \textsc{En-Zh}. At the segment level, rewrites are also more preferred over original inputs, selected in 1395/2074 cases for \textsc{En-Cs}, 1309/2074 for \textsc{En-He}, and 1411/2074 for \textsc{En-Ja}. \textsc{MetricX} trends are consistent.

In sum, our findings generalize well to held-out test sets, further validating the effectiveness of the translatability-based selection strategy. This approach offers a practical and scalable solution for input rewriting across a broader range of domains and language pairs, though there are many other dimensions that remain unexplored. We have conducted initial experiments with additional LLMs and source languages, shown in Appendix \ref{appendix:more_llms} and \ref{appendix:more_lang_pairs}, which confirms our previous findings that simplification rewriting enhances translation quality. We leave a more comprehensive exploration of this direction for future work.

% \clearpage
\section{Error Analysis}
\label{sec:error-analysis}
In this section, we look into the outputs of each model and their errors given different data pairs. More specifically, we look into errors made by \geminiPro{}, \gptFouroEleven{}, \internvlTwoFiveEightB{}, \pixtral{}, \qwenTwoVLSevenB{}, and \phiThreeFive{}.
\subsection{Gemini Models}
Below is an error example of \geminiPro{}{} on a data-pair from \mmscorecoco with color jittering (CJ).
% \begin{tcolorbox}[enhanced,attach boxed title to top center={yshift=-3mm,yshifttext=-1mm},
%   colback=red!5!white,colframe=red!20!gray,colbacktitle=red!20!gray,
%   title=\geminiPro{},fonttitle=\bfseries,
%   boxed title style={size=small,colframe=red!20!gray} ]

%         \textbf{User prompt}: Assess the degree to which a text description corresponds to an image of two objects, taking into account the following conditions: \\\emph{- Be invariant to color jittering while evaluating images. For example, even if one image has been modified with random color changes such as brightness or contrast adjustments, the similarity score should remain high. If the images are different, the similarity score should decrease regardless of the jittering.}\\The scoring should reflect:
%         - A perfect alignment with the image that satisfies all conditions merits a score of 10.
%         - Descriptions that partially match the image but fail to meet certain conditions should still receive a higher score than entirely unrelated descriptions.
%         - A score of 0 should be reserved for cases where no correlation exists between the text and the image.
%         The text and image pair will be provided below. Provide your evaluation using the following format:
        
%         \begin{verbatim}
%         Score: <1-10>
%         Reason: <reason for score>
%         \end{verbatim}
        
%         What score would you assign?
%         \vspace{-8mm}
%         \begin{center}
%         \includegraphics[width=0.3\linewidth]{imgs/pair-burger-color-rev.png}
%         \end{center}

%         \textbf{Model response}:
%         \begin{tcolorbox}[colback=gray!10!white, colframe=gray!50, sharp corners=south, boxrule=0.8pt]
        
%         \textit{Score: 1\\Reason: The first image appears to be a mold of some sort, possibly for dental work or other medical purposes.  
%         The second image is a Caprese sandwich. They are entirely different objects. While the layered structure might be vaguely similar,  
%         the content, context, and purpose are completely dissimilar.}
%         \end{tcolorbox}
        
%         % \textbf{Score: 1\\Reason: The first image appears to be a mold of some sort, possibly for dental work or other medical purposes.  The second image is a Caprese sandwich.  They are entirely different objects.  While the layered structure might be vaguely similar, the content, context, and purpose are completely dissimilar.}

% \end{tcolorbox}

% \begin{tcolorbox}[enhanced,attach boxed title to top center={yshift=-3mm,yshifttext=-1mm},
%   colback=red!5!white,colframe=red!20!gray,colbacktitle=red!20!gray,
%   title=\geminiPro{},fonttitle=\bfseries,
%   boxed title style={size=small,colframe=red!20!gray} ]

%         \textbf{User prompt}: Assess the degree to which a text description corresponds to an image of two objects, taking into account the following conditions: \texttt{\{conditions\}}. The scoring should reflect:
%         - A perfect alignment with the image that satisfies all conditions merits a score of 10.
%         - Descriptions that partially match the image but fail to meet certain conditions should still receive a higher score than entirely unrelated descriptions.
%         - A score of 0 should be reserved for cases where no correlation exists between the text and the image.
%         The text and image pair will be provided below. Provide your evaluation using the following format:
        
%         \begin{verbatim}
%         Score: <1-10>
%         Reason: <reason for score>
%         \end{verbatim}
        
%         What score would you assign?
%         \vspace{-8mm}
%         \begin{center}
%         \includegraphics[width=0.3\linewidth]{imgs/pair-burger-color.png}
%         \end{center}

%         \textbf{Model response}:
%         \begin{tcolorbox}[colback=gray!10!white, colframe=gray!50, sharp corners=south, boxrule=0.8pt]
        
%         \textit{Score: 10\\Reason: The images are of the same sandwich, one in normal color and the other with a strong purple color filter applied.  Despite the heavy color distortion, the content is identical, fulfilling the color invariance requirement.}
%         \end{tcolorbox}
        
%         % \textbf{Score: 1\\Reason: The first image appears to be a mold of some sort, possibly for dental work or other medical purposes.  The second image is a Caprese sandwich.  They are entirely different objects.  While the layered structure might be vaguely similar, the content, context, and purpose are completely dissimilar.}

% \end{tcolorbox}

\begin{tcolorbox}[enhanced,attach boxed title to top center={yshift=-3mm,yshifttext=-1mm},
  colback=red!5!white,colframe=red!20!gray,colbacktitle=red!20!gray,
  title=\geminiPro{} Evaluation Instructions,fonttitle=\bfseries,
  boxed title style={size=small,colframe=red!20!gray} ]

\textbf{User prompt}: Evaluate the similarity of the images based on the following conditions:
\emph{- Be invariant to color jittering while evaluating images. Even if one image has been modified with random color changes (e.g., brightness or contrast adjustments), the similarity score should remain high. If the images are different, the similarity score should decrease regardless of the jittering.}

The score should take into account how well the images align with these conditions, as well as their overall resemblance. Even if the images are identical but fail to meet the conditions, they should still receive a higher score than completely different images. Provide a score from 1 to 10, where 1 indicates no similarity and 10 indicates identical images that fully satisfy the conditions. Respond only in this format:

\begin{verbatim}
Score: <1-10>
Reason: <reason for score>
\end{verbatim}
Nothing else should be included. What score would you give?  ``\texttt{data1}'' ``\texttt{data2}'' 
\end{tcolorbox}

\noindent
\begin{minipage}{0.48\textwidth}
\begin{tcolorbox}[enhanced,attach boxed title to top center={yshift=-3mm,yshifttext=-1mm},
  colback=red!5!white,colframe=red!20!gray,colbacktitle=red!20!gray,
  title=\geminiPro{} Response - Original Order,fonttitle=\bfseries,
  boxed title style={size=small,colframe=red!20!gray},
  equal height group=examplegroup]

\begin{center}
    \includegraphics[width=0.9\linewidth]{imgs/pair-burger-color1.png}
\end{center}

\textbf{Model response:}
\begin{tcolorbox}[colback=gray!10!white, colframe=gray!50, sharp corners=south, boxrule=0.8pt]
\textit{Score: 10\\Reason: The images are of the same sandwich, one in normal color and the other with a strong purple color filter applied.  
Despite the heavy color distortion, the content is identical, fulfilling the color invariance requirement.}
\end{tcolorbox}

\end{tcolorbox}
\end{minipage}
\hfill
\begin{minipage}{0.48\textwidth}
\begin{tcolorbox}[enhanced,attach boxed title to top center={yshift=-3mm,yshifttext=-1mm},
  colback=red!5!white,colframe=red!20!gray,colbacktitle=red!20!gray,
  title=\geminiPro{} Response - Reversed Order,fonttitle=\bfseries,
  boxed title style={size=small,colframe=red!20!gray},
  equal height group=examplegroup]

\begin{center}
    \includegraphics[width=0.9\linewidth]{imgs/pair-burger-color2.png}
\end{center}

\textbf{Model response:}
\begin{tcolorbox}[colback=gray!10!white, colframe=gray!50, sharp corners=south, boxrule=0.8pt]
\textit{Score: 1\\Reason: The first image appears to be a mold of some sort, possibly for dental work or other medical purposes.  
The second image is a Caprese sandwich. They are entirely different objects. While the layered structure might be vaguely similar,  
the content, context, and purpose are completely dissimilar.}
\end{tcolorbox}

\end{tcolorbox}
\end{minipage}


\subsection{GPT4o}
Below is an error example of \gptFouroEleven{} on a data-pair from \mmscorecoco with color jittering (CJ).
\begin{tcolorbox}[enhanced,attach boxed title to top center={yshift=-3mm,yshifttext=-1mm},
  colback=red!5!white,colframe=red!20!gray,colbacktitle=red!20!gray,
  title=\gptFouroEleven{} Evaluation Instructions,fonttitle=\bfseries,
  boxed title style={size=small,colframe=red!20!gray} ]

\textbf{User prompt}: Evaluate the similarity of the images based on the following conditions:

\emph{- Be invariant to color jittering while evaluating images. Even if one image has been modified with random color changes (e.g., brightness or contrast adjustments), the similarity score should remain high. If the images are different, the similarity score should decrease regardless of the jittering.}

The score should take into account how well the images align with these conditions, as well as their overall resemblance. Even if the images are identical but fail to meet the conditions, they should still receive a higher score than completely different images. 

Provide a score from 1 to 10, where 1 indicates no similarity and 10 indicates identical images that fully satisfy the conditions.

Respond only in this format:
\begin{verbatim}
Score: <1-10>
Reason: <reason for score>
\end{verbatim}

Nothing else should be included. What score would you give? \texttt{"data1"} \texttt{"data2"} 

\end{tcolorbox}

\noindent
\begin{minipage}{0.48\textwidth}
\begin{tcolorbox}[enhanced,attach boxed title to top center={yshift=-3mm,yshifttext=-1mm},
  colback=red!5!white,colframe=red!20!gray,colbacktitle=red!20!gray,
  title=\gptFouroEleven{} Response - Original Order,fonttitle=\bfseries,
  boxed title style={size=small,colframe=red!20!gray},
  equal height group=examplegroup]

\begin{center}
\includegraphics[width=0.9\linewidth]{imgs/pair-burger-color1.png}
\end{center}

\textbf{Model response:}
\begin{tcolorbox}[colback=gray!10!white, colframe=gray!50, sharp corners=south, boxrule=0.8pt]
\textit{Score: 8\\Reason: The images are of the same sandwich, with similar shapes and arrangements of the bread, tomato, mozzarella, and basil.  
The second image has substantial color jittering, but since the structural alignment and composition remain consistent, the similarity remains high.}
\end{tcolorbox}

\end{tcolorbox}
\end{minipage}
\hfill
\begin{minipage}{0.48\textwidth}
\begin{tcolorbox}[enhanced,attach boxed title to top center={yshift=-3mm,yshifttext=-1mm},
  colback=red!5!white,colframe=red!20!gray,colbacktitle=red!20!gray,
  title=\gptFouroEleven{} Response - Reversed Order,fonttitle=\bfseries,
  boxed title style={size=small,colframe=red!20!gray},
  equal height group=examplegroup]

\begin{center}
    \includegraphics[width=0.9\linewidth]{imgs/pair-burger-color2.png}
\end{center}

\textbf{Model response:}
\begin{tcolorbox}[colback=gray!10!white, colframe=gray!50, sharp corners=south, boxrule=0.8pt]
\textit{Score: 3\\Reason: The images depict similar objects, presumably a sandwich with layers, but there are notable differences.  
The first image appears distorted with altered coloration, and while it shares structural similarities with the second image,  
the overall resemblance is weak.}
\end{tcolorbox}

\end{tcolorbox}
\end{minipage}


\subsection{InternVL2.5}
Below is an error example of \internvlTwoFiveEightB{} on a data-pair from \mmscorecoco with that are irrelevant to each other.

\begin{tcolorbox}[enhanced,attach boxed title to top center={yshift=-3mm,yshifttext=-1mm},
  colback=red!5!white,colframe=red!20!gray,colbacktitle=red!20!gray,
  title=\internvlTwoFiveEightB{} Evaluation Instructions,fonttitle=\bfseries,
  boxed title style={size=small,colframe=red!20!gray} ]

\textbf{User prompt}: Evaluate the similarity of the images based on the following conditions: 
\emph{- Be invariant to Gaussian blurs while evaluating images. For example, even if both images are the exact same, even if one is more blurred compared to the other, the similarity score should remain high. If the images are different, the similarity score should decrease regardless of how blurred they are.}  
The score should take into account how well the images align with these conditions, as well as their overall resemblance.  
Even if the images are identical but fail to meet the conditions, they should still receive a higher score than completely different images.  

Provide a score from 1 to 10, where 1 indicates no similarity and 10 indicates identical images that fully satisfy the conditions.  
Respond only in this format:

\begin{verbatim}
Score: <1-10>
Reason: <reason for score>
\end{verbatim}

Nothing else should be included. What score would you give? ``\texttt{data1}'' ``\texttt{data2}'' 
\end{tcolorbox}

\noindent
\begin{minipage}{0.48\textwidth}
\begin{tcolorbox}[enhanced,attach boxed title to top center={yshift=-3mm,yshifttext=-1mm},
  colback=red!5!white,colframe=red!20!gray,colbacktitle=red!20!gray,
  title=\internvlTwoFiveEightB{} - Original Order,fonttitle=\bfseries,
  boxed title style={size=small,colframe=red!20!gray},
  equal height group=examplegroup]

\begin{center}
    \includegraphics[width=0.9\linewidth]{imgs/parrots-image1.png}
\end{center}

\textbf{Model response:}
\begin{tcolorbox}[colback=gray!10!white, colframe=gray!50, sharp corners=south, boxrule=0.8pt]
\textit{Score: 7\\Reason: The images are similar in content, featuring parrots, but they are not identical.  
The first image is a close-up of a spider web on a wooden surface, while the second image shows two parrots on a ledge.  
The parrots are the main subjects in the second image, and the first image has a different focus and subject matter.  
The images do not meet the condition of being identical, but they share some visual elements like color and subject type.}
\end{tcolorbox}

\end{tcolorbox}
\end{minipage}
\hfill
\begin{minipage}{0.48\textwidth}
\begin{tcolorbox}[enhanced,attach boxed title to top center={yshift=-3mm,yshifttext=-1mm},
  colback=red!5!white,colframe=red!20!gray,colbacktitle=red!20!gray,
  title=\internvlTwoFiveEightB{} - Reverse Order,fonttitle=\bfseries,
  boxed title style={size=small,colframe=red!20!gray},
  equal height group=examplegroup]

\begin{center}
    \includegraphics[width=0.9\linewidth]{imgs/parrots-image2.png}
\end{center}

\textbf{Model response:}
\begin{tcolorbox}[colback=gray!10!white, colframe=gray!50, sharp corners=south, boxrule=0.8pt]
\textit{Score: 1\\Reason: The images are completely different in terms of content, subject matter, and context.  
The first image features colorful parrots, while the second image shows a wooden door with a spider web.  
There is no similarity in terms of subject, color, or composition.}
\end{tcolorbox}

\end{tcolorbox}
\end{minipage}

\subsection{\pixtral{}}

Below is an error example of \pixtral{} on a data-pair from \mmscorecoco with rotation (R).


\begin{tcolorbox}[enhanced,attach boxed title to top center={yshift=-3mm,yshifttext=-1mm},
  colback=red!5!white,colframe=red!20!gray,colbacktitle=red!20!gray,
  title=\pixtral{} Evaluation Instructions,fonttitle=\bfseries,
  boxed title style={size=small,colframe=red!20!gray} ]

\textbf{User prompt}: Rate the similarity of these images on a scale of 1 to 10 while considering the following condition:  

\emph{- Be invariant to rotations while evaluating images. For example, if one image is a rotated version of the other, the similarity score should remain high. If the images are different, the similarity score should decrease regardless of their orientation.}  

The score should account for both how well the images satisfy the conditions and their overall resemblance. If the images are identical but do not meet the conditions, they should still score higher than unrelated images.  

Your response must follow this exact format:
\begin{verbatim}
Score: <1-10>
Reason: <reason for score>
\end{verbatim}

Provide nothing else. What’s your score? ``\texttt{data1}'' ``\texttt{data2}'' 
\end{tcolorbox}

\noindent
\begin{minipage}{0.48\textwidth}
\begin{tcolorbox}[enhanced,attach boxed title to top center={yshift=-3mm,yshifttext=-1mm},
  colback=red!5!white,colframe=red!20!gray,colbacktitle=red!20!gray,
  title=\pixtral{} Response - Original Order, fonttitle=\bfseries,
  boxed title style={size=small,colframe=red!20!gray},
  equal height group=examplegroup]

\begin{center}
    \includegraphics[width=0.9\linewidth]{imgs/birds-interaction1.png}
\end{center}

\textbf{Model response:}
\begin{tcolorbox}[colback=gray!10!white, colframe=gray!50, sharp corners=south, boxrule=0.8pt]
\textit{Score: 8\\Reason: The images depict similar scenes of people interacting with birds in natural settings.  
The first image features a group of people sitting on rocks with birds nearby, while the second image has a person extending their hand  
towards a bird in a mountainous area. Despite slight differences in the environment and specific actions,  
the overall theme and presence of birds and human interaction score high.}
\end{tcolorbox}

\end{tcolorbox}
\end{minipage}
\hfill
\begin{minipage}{0.48\textwidth}
\begin{tcolorbox}[enhanced,attach boxed title to top center={yshift=-3mm,yshifttext=-1mm},
  colback=red!5!white,colframe=red!20!gray,colbacktitle=red!20!gray,
  title=\pixtral{} Response - Reversed Order,fonttitle=\bfseries,
  boxed title style={size=small,colframe=red!20!gray},
  equal height group=examplegroup]

\begin{center}
    \includegraphics[width=0.9\linewidth]{imgs/birds-interaction2.png}
\end{center}

\textbf{Model response:}
\begin{tcolorbox}[colback=gray!10!white, colframe=gray!50, sharp corners=south, boxrule=0.8pt]
\textit{Score: 2\\Reason: The images depict completely different scenes and subjects, with no recognizable similarity between them.  
One image shows a bird and people in a rocky area, while the other does not bear any resemblance to it.}
\end{tcolorbox}

\end{tcolorbox}
\end{minipage}


\subsection{Qwen2-VL}
Below is an error example of \qwenTwoVLSevenB{} on a data-pair from \mmscorecoco with gaussian blur (GB).

\begin{tcolorbox}[enhanced,attach boxed title to top center={yshift=-3mm,yshifttext=-1mm},
  colback=red!5!white,colframe=red!20!gray,colbacktitle=red!20!gray,
  title=\qwenTwoVLSevenB{} Evaluation Instructions,fonttitle=\bfseries,
  boxed title style={size=small,colframe=red!20!gray} ]

\textbf{User prompt}: As a similarity evaluator, your responsibility is to assess the similarity of the given images while considering these conditions:  
\emph{- Be sensitive to Gaussian blurs while evaluating images. For example, if both images are the same except one is a blurred version of the other, the similarity score should decrease. If the images are the same and have the same exact sharpness, the similarity score should remain high.}  

The similarity score should reflect both how well the images align with the conditions and their overall resemblance.  
Images that are identical but do not meet the conditions should receive a moderate score, while completely unrelated images should receive the lowest score.  

Provide your score on a scale of 1 to 10, with 10 being identical images that fully meet the conditions.  
Ensure your response is in the following format:
\begin{verbatim}
Score: <1-10>
Reason: <reason for score>
\end{verbatim}

Provide nothing else. What is your score? ``\texttt{data1}'' ``\texttt{data2}'' 
\end{tcolorbox}

\noindent
\begin{minipage}{0.48\textwidth}
\begin{tcolorbox}[enhanced,attach boxed title to top center={yshift=-3mm,yshifttext=-1mm},
  colback=red!5!white,colframe=red!20!gray,colbacktitle=red!20!gray,
  title=\qwenTwoVLSevenB{} - Original Order,fonttitle=\bfseries,
  boxed title style={size=small,colframe=red!20!gray},
  equal height group=examplegroup]

\begin{center}
    \includegraphics[width=0.9\linewidth]{imgs/pizza-image1.png}
\end{center}

\textbf{Model response:}
\begin{tcolorbox}[colback=gray!10!white, colframe=gray!50, sharp corners=south, boxrule=0.8pt]
\textit{Score: 6\\Reason: The images are very similar, with the second image being a slightly blurred version of the first.  
However, they do not fully meet the conditions as the images are not identical and the blurring is not perfect.}
\end{tcolorbox}

\end{tcolorbox}
\end{minipage}
\hfill
\begin{minipage}{0.48\textwidth}
\begin{tcolorbox}[enhanced,attach boxed title to top center={yshift=-3mm,yshifttext=-1mm},
  colback=red!5!white,colframe=red!20!gray,colbacktitle=red!20!gray,
  title=\qwenTwoVLSevenB{} - Reverse Order,fonttitle=\bfseries,
  boxed title style={size=small,colframe=red!20!gray},
  equal height group=examplegroup]

\begin{center}
    \includegraphics[width=0.9\linewidth]{imgs/pizza-image2.png}
\end{center}

\textbf{Model response:}
\begin{tcolorbox}[colback=gray!10!white, colframe=gray!50, sharp corners=south, boxrule=0.8pt]
\textit{Score: 9\\Reason: The images are nearly identical, with only a slight difference in the sharpness of the objects.  
The overall resemblance is high, and the conditions are met.}
\end{tcolorbox}

\end{tcolorbox}
\end{minipage}


\subsection{Phi3.5}
Below is an error example of \phiThreeFive{} on a data-pair from \mmscorecoco with elastic transform (ET).

\begin{tcolorbox}[enhanced,attach boxed title to top center={yshift=-3mm,yshifttext=-1mm},
  colback=red!5!white,colframe=red!20!gray,colbacktitle=red!20!gray,
  title=\phiThreeFive{} Evaluation Instructions,fonttitle=\bfseries,
  boxed title style={size=small,colframe=red!20!gray} ]

\textbf{User prompt}: Judge the similarity of these images based on:  
\emph{- Be sensitive to elastic transformations while evaluating images. For example, if the images are the same but have differing distortions due to elastic transformations, the similarity score should decrease. If the images are identical and have matching elastic transformations, the similarity score should remain high.}  

The similarity score should reflect both the overall resemblance of the images and how well they satisfy the conditions.  
Identical images that do not meet the conditions should still score higher than completely unrelated images.  

Provide a score on a scale of 1 to 10, with 1 being no similarity and 10 being identical images that satisfy all conditions.  
Respond strictly in this format:
\begin{verbatim}
Score: <1-10>
Reason: <reason for score>
\end{verbatim}

Do not include additional text. What’s your rating? ``\texttt{data1}'' ``\texttt{data2}'' 
\end{tcolorbox}

\noindent
\begin{minipage}{0.48\textwidth}
\begin{tcolorbox}[enhanced,attach boxed title to top center={yshift=-3mm,yshifttext=-1mm},
  colback=red!5!white,colframe=red!20!gray,colbacktitle=red!20!gray,
  title=\phiThreeFive{} - Original Order,fonttitle=\bfseries,
  boxed title style={size=small,colframe=red!20!gray},
  equal height group=examplegroup]

\begin{center}
    \includegraphics[width=0.9\linewidth]{imgs/motorcycle-image1.png}
\end{center}

\textbf{Model response:}
\begin{tcolorbox}[colback=gray!10!white, colframe=gray!50, sharp corners=south, boxrule=0.8pt]
\textit{Score: 1\\Reason: The images are highly distorted with elastic transformations, making it impossible to accurately compare the content.}
\end{tcolorbox}

\end{tcolorbox}
\end{minipage}
\hfill
\begin{minipage}{0.48\textwidth}
\begin{tcolorbox}[enhanced,attach boxed title to top center={yshift=-3mm,yshifttext=-1mm},
  colback=red!5!white,colframe=red!20!gray,colbacktitle=red!20!gray,
  title=\phiThreeFive{} - Reverse Order,fonttitle=\bfseries,
  boxed title style={size=small,colframe=red!20!gray},
  equal height group=examplegroup]

\begin{center}
    \includegraphics[width=0.9\linewidth]{imgs/motorcycle-image2.png}
\end{center}

\textbf{Model response:}
\begin{tcolorbox}[colback=gray!10!white, colframe=gray!50, sharp corners=south, boxrule=0.8pt]
\textit{Score: 10\\Reason: The images are identical with no differences in content, color, or composition.  
They both depict the same scene with motorcycles and people in a bar setting, and the elastic transformations do not alter the image in any way.}
\end{tcolorbox}

\end{tcolorbox}
\end{minipage}

\vspace{-5mm}
\section{Conclusion and Future Work}
\vspace{-1mm}
% \vspace{-1ex}
We introduced \mmscore{}, a comprehensive framework for evaluating the reliability of \modelss{} when used to define similarity kernels. \mmscore{} enables assessing how different models will behave when acting as evaluators by measuring kernel properties such as alignment with ground truth relevance, symmetry, smoothness, and controllability. Interestingly, by leveraging controlled data transformations, we found that \mmscore{} not only allows for fine-grained analysis of model biases and strengths, but it also offers a cost-effective alternative to large-scale benchmarks.

\vspace{-0.5mm}

We carried out a large-scale benchmarking covering several \modelss{} and demonstrated that no single model excels across all four metrics or dataset configurations. While commercial-grade models generally performed better on image-image comparisons, openly available models such as \internvlTwoFiveEightB{} showed competitive results, particularly in \nmi. Furthermore, our findings indicate that commonly used judge models exhibit limitations. For instance, \gptFouroEleven{} lacks in terms of symmetry and smoothness, highlighting the necessity of careful selection based on specific evaluation needs.

\vspace{-0.5mm}

From a more practical perspective, we established that \mmscore{} metrics, particularly \nmi{}, correlate strongly with model performance on well-known benchmarks, reinforcing its utility as a low-cost surrogate for ranking models or guiding cross-validation during training. As the field progresses, we anticipate that \mmscore{} will serve as a valuable tool for improving model evaluation practices. 

% As future work, \mmscore{} could be extended by incorporating additional modalities, refining transformation techniques, and further exploring the impact of architectural choices on similarity estimation. By advancing our understanding of \modelss{} as similarity kernels, we move closer to more robust and interpretable model-based evaluation methods.
\section*{Impact Statement}
This paper presents work whose goal is to advance the field of Machine Learning. There are many potential societal consequences of our work, none which we feel must be specifically highlighted here.

% In the unusual situation where you want a paper to appear in the
% references without citing it in the main text, use \nocite
\nocite{langley00}

\bibliography{main}
\bibliographystyle{icml2025}


%%%%%%%%%%%%%%%%%%%%%%%%%%%%%%%%%%%%%%%%%%%%%%%%%%%%%%%%%%%%%%%%%%%%%%%%%%%%%%%
%%%%%%%%%%%%%%%%%%%%%%%%%%%%%%%%%%%%%%%%%%%%%%%%%%%%%%%%%%%%%%%%%%%%%%%%%%%%%%%
% APPENDIX
%%%%%%%%%%%%%%%%%%%%%%%%%%%%%%%%%%%%%%%%%%%%%%%%%%%%%%%%%%%%%%%%%%%%%%%%%%%%%%%
%%%%%%%%%%%%%%%%%%%%%%%%%%%%%%%%%%%%%%%%%%%%%%%%%%%%%%%%%%%%%%%%%%%%%%%%%%%%%%%
\newpage
\appendix
\onecolumn
%\section{Related Work}

% [what are some different benchmarks]

% [how are people evaluating vlms]

% [encoders are also being used as judges (text-image and image-image)]

% [what benchmarks are using vlms as judges]

% [literature on looking into invariance in LMMs]


% [downsides of using clip and other models (do not get negation), look at it as bag of words, spatially flawed (whatsup), looking into propperties that are not captures in benchmarks]
Using language models as automatic evaluators has become a somewhat common practice with popular approaches such as \textsc{GPTScore} and G-eval~\citep{fu2023gptscore,liu2023geval} being used to rank responses in the NLP domain. 
Due to that, there has been a significant amount of recent work that has investigated the capabilities and limitations of using LLMs as judges~\citep{thakur2024judging, chiang2023can, murugadoss2024evaluating, shankar2024validates}. 
\citet{chiang2023can} have shown that LLM evaluations are consistent and reproducible, making them suitable alternatives for human evaluation, they argue that these models inherent biases should prevent them using independently rather than \textit{alongside} human experts. 
Furthermore, \citet{zheng2023judging} reveal that large \modelss{{}, e.g., GPT-4 Turbo, align well with human judgments and \citet{thakur2024judging} further states that simpler models may still outperform GPT-4 Turbo in ranking tasks due to superior alignment metrics. Also, recent work assessed how humans can help LLMs evaluate better by testing different instruction types or designing tools that result in more balanced evaluations~\citep{murugadoss2024evaluating, shankar2024validates}.

It is worth noting that known limitations of LLMs such as their lack of invariance to the order of examples given in a prompt, which is a well studied issue of natural language models~\citep{fang2024rethinking}, 
% and also observed in the multimodal case by the lack of symmetry we reported,
may render auto evaluation unreliable. Similarly, \citet{berglund2023reversal} show failure cases where models trained on unidirectional relationships do not infer the reverse, indicating systemic limitations even in state-of-the-art LLMs such as \textsc{GPT-4} (as seen in Figure~\ref{fig:fig1} and in Appendix \ref{sec:error-analysis} for \modelss{}). Our main goal is
to investigate the reliability of automated evaluation in the multimodal context, by probing the models to compare data pairs. 
%thus to assess to what extent auto-evaluation can be done reliably, focusing in the multimodal case, by probing models on their ability to compare.

Namely, the evaluations we carry out focus on testing in multiple different ways how good \modelss{} are when it comes to comparing data instances, such as whether \modelss{} prompted to compare are symmetric or smooth for instance, and to what extent they can be controlled, i.e., instructed to pay attention to or ignore certain features of the inputs. While the literature is more sparse regarding testing \modelss{} in this setting, recent work has tested for something along those lines. \citet{chen2024mllm} for instance propose a benchmark for evaluating \modelss{} in multiple different scenarios, including checking whether pairwise comparisons of responses to a query correlated with human judgments. They concluded that although correlations are relatively high on comparison tasks, biases and inconsistencies affect performance on pair scoring and batch ranking. Similarly, \citet{awal2024vismin} introduced a synthetic dataset containing paired images that differ only along one feature (e.g., the color of an object). We seek to add to this branch of the literature by introducing a framework where controlled experiments can be carried out to anticipate the performance of models when being used as judges, and various different characteristics of automatic judges can be identified (e.g., how smooth they are).

Unlike the case of generative \modelss{} discussed above, discriminative visual language models such as CLIP~\citep{radford2021learning} are covered by a significant amount of recent work, and several failure modes are well reported, mostly deriving from the fact this class of \modelss{} tends to behave as bag-of-words models, focusing on nouns and ignoring relationships and semantics in their input data~\citep{yuksekgonul2023when}. For instance, CLIP was observed to struggle with spatial reasoning~\citep{kamath2023s} and ignore negation~\citep{alhamoud2025vision}. On the other hand, fine-tuning CLIP to reason about pairwise differences \citet{sam2024finetuning} showed that discriminative \modelss{} can improve on how well they manage to reason about pairwise differences if training is tailored for enabling so, highlighting the benefits that being able to measure these skills may inform training and improve models as a consequence. \citet{ouali2024discriminative} showed that fine-tuning generative \modelss{} to turn them into discriminative models results in improved image-retrieval from text, which aligns with results we reported in Section~\ref{sec:ecoders_vs_decoders} showing a gap between open-sources \modelss{} and CLIP-style encoders.

%\section{Related Work}

% [what are some different benchmarks]

% [how are people evaluating vlms]

% [encoders are also being used as judges (text-image and image-image)]

% [what benchmarks are using vlms as judges]

% [literature on looking into invariance in LMMs]


% [downsides of using clip and other models (do not get negation), look at it as bag of words, spatially flawed (whatsup), looking into propperties that are not captures in benchmarks]
Using language models as automatic evaluators has become a somewhat common practice with popular approaches such as \textsc{GPTScore} and G-eval~\citep{fu2023gptscore,liu2023geval} being used to rank responses in the NLP domain. 
Due to that, there has been a significant amount of recent work that has investigated the capabilities and limitations of using LLMs as judges~\citep{thakur2024judging, chiang2023can, murugadoss2024evaluating, shankar2024validates}. 
\citet{chiang2023can} have shown that LLM evaluations are consistent and reproducible, making them suitable alternatives for human evaluation, they argue that these models inherent biases should prevent them using independently rather than \textit{alongside} human experts. 
Furthermore, \citet{zheng2023judging} reveal that large \modelss{{}, e.g., GPT-4 Turbo, align well with human judgments and \citet{thakur2024judging} further states that simpler models may still outperform GPT-4 Turbo in ranking tasks due to superior alignment metrics. Also, recent work assessed how humans can help LLMs evaluate better by testing different instruction types or designing tools that result in more balanced evaluations~\citep{murugadoss2024evaluating, shankar2024validates}.

It is worth noting that known limitations of LLMs such as their lack of invariance to the order of examples given in a prompt, which is a well studied issue of natural language models~\citep{fang2024rethinking}, 
% and also observed in the multimodal case by the lack of symmetry we reported,
may render auto evaluation unreliable. Similarly, \citet{berglund2023reversal} show failure cases where models trained on unidirectional relationships do not infer the reverse, indicating systemic limitations even in state-of-the-art LLMs such as \textsc{GPT-4} (as seen in Figure~\ref{fig:fig1} and in Appendix \ref{sec:error-analysis} for \modelss{}). Our main goal is
to investigate the reliability of automated evaluation in the multimodal context, by probing the models to compare data pairs. 
%thus to assess to what extent auto-evaluation can be done reliably, focusing in the multimodal case, by probing models on their ability to compare.

Namely, the evaluations we carry out focus on testing in multiple different ways how good \modelss{} are when it comes to comparing data instances, such as whether \modelss{} prompted to compare are symmetric or smooth for instance, and to what extent they can be controlled, i.e., instructed to pay attention to or ignore certain features of the inputs. While the literature is more sparse regarding testing \modelss{} in this setting, recent work has tested for something along those lines. \citet{chen2024mllm} for instance propose a benchmark for evaluating \modelss{} in multiple different scenarios, including checking whether pairwise comparisons of responses to a query correlated with human judgments. They concluded that although correlations are relatively high on comparison tasks, biases and inconsistencies affect performance on pair scoring and batch ranking. Similarly, \citet{awal2024vismin} introduced a synthetic dataset containing paired images that differ only along one feature (e.g., the color of an object). We seek to add to this branch of the literature by introducing a framework where controlled experiments can be carried out to anticipate the performance of models when being used as judges, and various different characteristics of automatic judges can be identified (e.g., how smooth they are).

Unlike the case of generative \modelss{} discussed above, discriminative visual language models such as CLIP~\citep{radford2021learning} are covered by a significant amount of recent work, and several failure modes are well reported, mostly deriving from the fact this class of \modelss{} tends to behave as bag-of-words models, focusing on nouns and ignoring relationships and semantics in their input data~\citep{yuksekgonul2023when}. For instance, CLIP was observed to struggle with spatial reasoning~\citep{kamath2023s} and ignore negation~\citep{alhamoud2025vision}. On the other hand, fine-tuning CLIP to reason about pairwise differences \citet{sam2024finetuning} showed that discriminative \modelss{} can improve on how well they manage to reason about pairwise differences if training is tailored for enabling so, highlighting the benefits that being able to measure these skills may inform training and improve models as a consequence. \citet{ouali2024discriminative} showed that fine-tuning generative \modelss{} to turn them into discriminative models results in improved image-retrieval from text, which aligns with results we reported in Section~\ref{sec:ecoders_vs_decoders} showing a gap between open-sources \modelss{} and CLIP-style encoders.

\clearpage
\section{Error Analysis}
\label{sec:error-analysis}
In this section, we look into the outputs of each model and their errors given different data pairs. More specifically, we look into errors made by \geminiPro{}, \gptFouroEleven{}, \internvlTwoFiveEightB{}, \pixtral{}, \qwenTwoVLSevenB{}, and \phiThreeFive{}.
\subsection{Gemini Models}
Below is an error example of \geminiPro{}{} on a data-pair from \mmscorecoco with color jittering (CJ).
% \begin{tcolorbox}[enhanced,attach boxed title to top center={yshift=-3mm,yshifttext=-1mm},
%   colback=red!5!white,colframe=red!20!gray,colbacktitle=red!20!gray,
%   title=\geminiPro{},fonttitle=\bfseries,
%   boxed title style={size=small,colframe=red!20!gray} ]

%         \textbf{User prompt}: Assess the degree to which a text description corresponds to an image of two objects, taking into account the following conditions: \\\emph{- Be invariant to color jittering while evaluating images. For example, even if one image has been modified with random color changes such as brightness or contrast adjustments, the similarity score should remain high. If the images are different, the similarity score should decrease regardless of the jittering.}\\The scoring should reflect:
%         - A perfect alignment with the image that satisfies all conditions merits a score of 10.
%         - Descriptions that partially match the image but fail to meet certain conditions should still receive a higher score than entirely unrelated descriptions.
%         - A score of 0 should be reserved for cases where no correlation exists between the text and the image.
%         The text and image pair will be provided below. Provide your evaluation using the following format:
        
%         \begin{verbatim}
%         Score: <1-10>
%         Reason: <reason for score>
%         \end{verbatim}
        
%         What score would you assign?
%         \vspace{-8mm}
%         \begin{center}
%         \includegraphics[width=0.3\linewidth]{imgs/pair-burger-color-rev.png}
%         \end{center}

%         \textbf{Model response}:
%         \begin{tcolorbox}[colback=gray!10!white, colframe=gray!50, sharp corners=south, boxrule=0.8pt]
        
%         \textit{Score: 1\\Reason: The first image appears to be a mold of some sort, possibly for dental work or other medical purposes.  
%         The second image is a Caprese sandwich. They are entirely different objects. While the layered structure might be vaguely similar,  
%         the content, context, and purpose are completely dissimilar.}
%         \end{tcolorbox}
        
%         % \textbf{Score: 1\\Reason: The first image appears to be a mold of some sort, possibly for dental work or other medical purposes.  The second image is a Caprese sandwich.  They are entirely different objects.  While the layered structure might be vaguely similar, the content, context, and purpose are completely dissimilar.}

% \end{tcolorbox}

% \begin{tcolorbox}[enhanced,attach boxed title to top center={yshift=-3mm,yshifttext=-1mm},
%   colback=red!5!white,colframe=red!20!gray,colbacktitle=red!20!gray,
%   title=\geminiPro{},fonttitle=\bfseries,
%   boxed title style={size=small,colframe=red!20!gray} ]

%         \textbf{User prompt}: Assess the degree to which a text description corresponds to an image of two objects, taking into account the following conditions: \texttt{\{conditions\}}. The scoring should reflect:
%         - A perfect alignment with the image that satisfies all conditions merits a score of 10.
%         - Descriptions that partially match the image but fail to meet certain conditions should still receive a higher score than entirely unrelated descriptions.
%         - A score of 0 should be reserved for cases where no correlation exists between the text and the image.
%         The text and image pair will be provided below. Provide your evaluation using the following format:
        
%         \begin{verbatim}
%         Score: <1-10>
%         Reason: <reason for score>
%         \end{verbatim}
        
%         What score would you assign?
%         \vspace{-8mm}
%         \begin{center}
%         \includegraphics[width=0.3\linewidth]{imgs/pair-burger-color.png}
%         \end{center}

%         \textbf{Model response}:
%         \begin{tcolorbox}[colback=gray!10!white, colframe=gray!50, sharp corners=south, boxrule=0.8pt]
        
%         \textit{Score: 10\\Reason: The images are of the same sandwich, one in normal color and the other with a strong purple color filter applied.  Despite the heavy color distortion, the content is identical, fulfilling the color invariance requirement.}
%         \end{tcolorbox}
        
%         % \textbf{Score: 1\\Reason: The first image appears to be a mold of some sort, possibly for dental work or other medical purposes.  The second image is a Caprese sandwich.  They are entirely different objects.  While the layered structure might be vaguely similar, the content, context, and purpose are completely dissimilar.}

% \end{tcolorbox}

\begin{tcolorbox}[enhanced,attach boxed title to top center={yshift=-3mm,yshifttext=-1mm},
  colback=red!5!white,colframe=red!20!gray,colbacktitle=red!20!gray,
  title=\geminiPro{} Evaluation Instructions,fonttitle=\bfseries,
  boxed title style={size=small,colframe=red!20!gray} ]

\textbf{User prompt}: Evaluate the similarity of the images based on the following conditions:
\emph{- Be invariant to color jittering while evaluating images. Even if one image has been modified with random color changes (e.g., brightness or contrast adjustments), the similarity score should remain high. If the images are different, the similarity score should decrease regardless of the jittering.}

The score should take into account how well the images align with these conditions, as well as their overall resemblance. Even if the images are identical but fail to meet the conditions, they should still receive a higher score than completely different images. Provide a score from 1 to 10, where 1 indicates no similarity and 10 indicates identical images that fully satisfy the conditions. Respond only in this format:

\begin{verbatim}
Score: <1-10>
Reason: <reason for score>
\end{verbatim}
Nothing else should be included. What score would you give?  ``\texttt{data1}'' ``\texttt{data2}'' 
\end{tcolorbox}

\noindent
\begin{minipage}{0.48\textwidth}
\begin{tcolorbox}[enhanced,attach boxed title to top center={yshift=-3mm,yshifttext=-1mm},
  colback=red!5!white,colframe=red!20!gray,colbacktitle=red!20!gray,
  title=\geminiPro{} Response - Original Order,fonttitle=\bfseries,
  boxed title style={size=small,colframe=red!20!gray},
  equal height group=examplegroup]

\begin{center}
    \includegraphics[width=0.9\linewidth]{imgs/pair-burger-color1.png}
\end{center}

\textbf{Model response:}
\begin{tcolorbox}[colback=gray!10!white, colframe=gray!50, sharp corners=south, boxrule=0.8pt]
\textit{Score: 10\\Reason: The images are of the same sandwich, one in normal color and the other with a strong purple color filter applied.  
Despite the heavy color distortion, the content is identical, fulfilling the color invariance requirement.}
\end{tcolorbox}

\end{tcolorbox}
\end{minipage}
\hfill
\begin{minipage}{0.48\textwidth}
\begin{tcolorbox}[enhanced,attach boxed title to top center={yshift=-3mm,yshifttext=-1mm},
  colback=red!5!white,colframe=red!20!gray,colbacktitle=red!20!gray,
  title=\geminiPro{} Response - Reversed Order,fonttitle=\bfseries,
  boxed title style={size=small,colframe=red!20!gray},
  equal height group=examplegroup]

\begin{center}
    \includegraphics[width=0.9\linewidth]{imgs/pair-burger-color2.png}
\end{center}

\textbf{Model response:}
\begin{tcolorbox}[colback=gray!10!white, colframe=gray!50, sharp corners=south, boxrule=0.8pt]
\textit{Score: 1\\Reason: The first image appears to be a mold of some sort, possibly for dental work or other medical purposes.  
The second image is a Caprese sandwich. They are entirely different objects. While the layered structure might be vaguely similar,  
the content, context, and purpose are completely dissimilar.}
\end{tcolorbox}

\end{tcolorbox}
\end{minipage}


\subsection{GPT4o}
Below is an error example of \gptFouroEleven{} on a data-pair from \mmscorecoco with color jittering (CJ).
\begin{tcolorbox}[enhanced,attach boxed title to top center={yshift=-3mm,yshifttext=-1mm},
  colback=red!5!white,colframe=red!20!gray,colbacktitle=red!20!gray,
  title=\gptFouroEleven{} Evaluation Instructions,fonttitle=\bfseries,
  boxed title style={size=small,colframe=red!20!gray} ]

\textbf{User prompt}: Evaluate the similarity of the images based on the following conditions:

\emph{- Be invariant to color jittering while evaluating images. Even if one image has been modified with random color changes (e.g., brightness or contrast adjustments), the similarity score should remain high. If the images are different, the similarity score should decrease regardless of the jittering.}

The score should take into account how well the images align with these conditions, as well as their overall resemblance. Even if the images are identical but fail to meet the conditions, they should still receive a higher score than completely different images. 

Provide a score from 1 to 10, where 1 indicates no similarity and 10 indicates identical images that fully satisfy the conditions.

Respond only in this format:
\begin{verbatim}
Score: <1-10>
Reason: <reason for score>
\end{verbatim}

Nothing else should be included. What score would you give? \texttt{"data1"} \texttt{"data2"} 

\end{tcolorbox}

\noindent
\begin{minipage}{0.48\textwidth}
\begin{tcolorbox}[enhanced,attach boxed title to top center={yshift=-3mm,yshifttext=-1mm},
  colback=red!5!white,colframe=red!20!gray,colbacktitle=red!20!gray,
  title=\gptFouroEleven{} Response - Original Order,fonttitle=\bfseries,
  boxed title style={size=small,colframe=red!20!gray},
  equal height group=examplegroup]

\begin{center}
\includegraphics[width=0.9\linewidth]{imgs/pair-burger-color1.png}
\end{center}

\textbf{Model response:}
\begin{tcolorbox}[colback=gray!10!white, colframe=gray!50, sharp corners=south, boxrule=0.8pt]
\textit{Score: 8\\Reason: The images are of the same sandwich, with similar shapes and arrangements of the bread, tomato, mozzarella, and basil.  
The second image has substantial color jittering, but since the structural alignment and composition remain consistent, the similarity remains high.}
\end{tcolorbox}

\end{tcolorbox}
\end{minipage}
\hfill
\begin{minipage}{0.48\textwidth}
\begin{tcolorbox}[enhanced,attach boxed title to top center={yshift=-3mm,yshifttext=-1mm},
  colback=red!5!white,colframe=red!20!gray,colbacktitle=red!20!gray,
  title=\gptFouroEleven{} Response - Reversed Order,fonttitle=\bfseries,
  boxed title style={size=small,colframe=red!20!gray},
  equal height group=examplegroup]

\begin{center}
    \includegraphics[width=0.9\linewidth]{imgs/pair-burger-color2.png}
\end{center}

\textbf{Model response:}
\begin{tcolorbox}[colback=gray!10!white, colframe=gray!50, sharp corners=south, boxrule=0.8pt]
\textit{Score: 3\\Reason: The images depict similar objects, presumably a sandwich with layers, but there are notable differences.  
The first image appears distorted with altered coloration, and while it shares structural similarities with the second image,  
the overall resemblance is weak.}
\end{tcolorbox}

\end{tcolorbox}
\end{minipage}


\subsection{InternVL2.5}
Below is an error example of \internvlTwoFiveEightB{} on a data-pair from \mmscorecoco with that are irrelevant to each other.

\begin{tcolorbox}[enhanced,attach boxed title to top center={yshift=-3mm,yshifttext=-1mm},
  colback=red!5!white,colframe=red!20!gray,colbacktitle=red!20!gray,
  title=\internvlTwoFiveEightB{} Evaluation Instructions,fonttitle=\bfseries,
  boxed title style={size=small,colframe=red!20!gray} ]

\textbf{User prompt}: Evaluate the similarity of the images based on the following conditions: 
\emph{- Be invariant to Gaussian blurs while evaluating images. For example, even if both images are the exact same, even if one is more blurred compared to the other, the similarity score should remain high. If the images are different, the similarity score should decrease regardless of how blurred they are.}  
The score should take into account how well the images align with these conditions, as well as their overall resemblance.  
Even if the images are identical but fail to meet the conditions, they should still receive a higher score than completely different images.  

Provide a score from 1 to 10, where 1 indicates no similarity and 10 indicates identical images that fully satisfy the conditions.  
Respond only in this format:

\begin{verbatim}
Score: <1-10>
Reason: <reason for score>
\end{verbatim}

Nothing else should be included. What score would you give? ``\texttt{data1}'' ``\texttt{data2}'' 
\end{tcolorbox}

\noindent
\begin{minipage}{0.48\textwidth}
\begin{tcolorbox}[enhanced,attach boxed title to top center={yshift=-3mm,yshifttext=-1mm},
  colback=red!5!white,colframe=red!20!gray,colbacktitle=red!20!gray,
  title=\internvlTwoFiveEightB{} - Original Order,fonttitle=\bfseries,
  boxed title style={size=small,colframe=red!20!gray},
  equal height group=examplegroup]

\begin{center}
    \includegraphics[width=0.9\linewidth]{imgs/parrots-image1.png}
\end{center}

\textbf{Model response:}
\begin{tcolorbox}[colback=gray!10!white, colframe=gray!50, sharp corners=south, boxrule=0.8pt]
\textit{Score: 7\\Reason: The images are similar in content, featuring parrots, but they are not identical.  
The first image is a close-up of a spider web on a wooden surface, while the second image shows two parrots on a ledge.  
The parrots are the main subjects in the second image, and the first image has a different focus and subject matter.  
The images do not meet the condition of being identical, but they share some visual elements like color and subject type.}
\end{tcolorbox}

\end{tcolorbox}
\end{minipage}
\hfill
\begin{minipage}{0.48\textwidth}
\begin{tcolorbox}[enhanced,attach boxed title to top center={yshift=-3mm,yshifttext=-1mm},
  colback=red!5!white,colframe=red!20!gray,colbacktitle=red!20!gray,
  title=\internvlTwoFiveEightB{} - Reverse Order,fonttitle=\bfseries,
  boxed title style={size=small,colframe=red!20!gray},
  equal height group=examplegroup]

\begin{center}
    \includegraphics[width=0.9\linewidth]{imgs/parrots-image2.png}
\end{center}

\textbf{Model response:}
\begin{tcolorbox}[colback=gray!10!white, colframe=gray!50, sharp corners=south, boxrule=0.8pt]
\textit{Score: 1\\Reason: The images are completely different in terms of content, subject matter, and context.  
The first image features colorful parrots, while the second image shows a wooden door with a spider web.  
There is no similarity in terms of subject, color, or composition.}
\end{tcolorbox}

\end{tcolorbox}
\end{minipage}

\subsection{\pixtral{}}

Below is an error example of \pixtral{} on a data-pair from \mmscorecoco with rotation (R).


\begin{tcolorbox}[enhanced,attach boxed title to top center={yshift=-3mm,yshifttext=-1mm},
  colback=red!5!white,colframe=red!20!gray,colbacktitle=red!20!gray,
  title=\pixtral{} Evaluation Instructions,fonttitle=\bfseries,
  boxed title style={size=small,colframe=red!20!gray} ]

\textbf{User prompt}: Rate the similarity of these images on a scale of 1 to 10 while considering the following condition:  

\emph{- Be invariant to rotations while evaluating images. For example, if one image is a rotated version of the other, the similarity score should remain high. If the images are different, the similarity score should decrease regardless of their orientation.}  

The score should account for both how well the images satisfy the conditions and their overall resemblance. If the images are identical but do not meet the conditions, they should still score higher than unrelated images.  

Your response must follow this exact format:
\begin{verbatim}
Score: <1-10>
Reason: <reason for score>
\end{verbatim}

Provide nothing else. What’s your score? ``\texttt{data1}'' ``\texttt{data2}'' 
\end{tcolorbox}

\noindent
\begin{minipage}{0.48\textwidth}
\begin{tcolorbox}[enhanced,attach boxed title to top center={yshift=-3mm,yshifttext=-1mm},
  colback=red!5!white,colframe=red!20!gray,colbacktitle=red!20!gray,
  title=\pixtral{} Response - Original Order, fonttitle=\bfseries,
  boxed title style={size=small,colframe=red!20!gray},
  equal height group=examplegroup]

\begin{center}
    \includegraphics[width=0.9\linewidth]{imgs/birds-interaction1.png}
\end{center}

\textbf{Model response:}
\begin{tcolorbox}[colback=gray!10!white, colframe=gray!50, sharp corners=south, boxrule=0.8pt]
\textit{Score: 8\\Reason: The images depict similar scenes of people interacting with birds in natural settings.  
The first image features a group of people sitting on rocks with birds nearby, while the second image has a person extending their hand  
towards a bird in a mountainous area. Despite slight differences in the environment and specific actions,  
the overall theme and presence of birds and human interaction score high.}
\end{tcolorbox}

\end{tcolorbox}
\end{minipage}
\hfill
\begin{minipage}{0.48\textwidth}
\begin{tcolorbox}[enhanced,attach boxed title to top center={yshift=-3mm,yshifttext=-1mm},
  colback=red!5!white,colframe=red!20!gray,colbacktitle=red!20!gray,
  title=\pixtral{} Response - Reversed Order,fonttitle=\bfseries,
  boxed title style={size=small,colframe=red!20!gray},
  equal height group=examplegroup]

\begin{center}
    \includegraphics[width=0.9\linewidth]{imgs/birds-interaction2.png}
\end{center}

\textbf{Model response:}
\begin{tcolorbox}[colback=gray!10!white, colframe=gray!50, sharp corners=south, boxrule=0.8pt]
\textit{Score: 2\\Reason: The images depict completely different scenes and subjects, with no recognizable similarity between them.  
One image shows a bird and people in a rocky area, while the other does not bear any resemblance to it.}
\end{tcolorbox}

\end{tcolorbox}
\end{minipage}


\subsection{Qwen2-VL}
Below is an error example of \qwenTwoVLSevenB{} on a data-pair from \mmscorecoco with gaussian blur (GB).

\begin{tcolorbox}[enhanced,attach boxed title to top center={yshift=-3mm,yshifttext=-1mm},
  colback=red!5!white,colframe=red!20!gray,colbacktitle=red!20!gray,
  title=\qwenTwoVLSevenB{} Evaluation Instructions,fonttitle=\bfseries,
  boxed title style={size=small,colframe=red!20!gray} ]

\textbf{User prompt}: As a similarity evaluator, your responsibility is to assess the similarity of the given images while considering these conditions:  
\emph{- Be sensitive to Gaussian blurs while evaluating images. For example, if both images are the same except one is a blurred version of the other, the similarity score should decrease. If the images are the same and have the same exact sharpness, the similarity score should remain high.}  

The similarity score should reflect both how well the images align with the conditions and their overall resemblance.  
Images that are identical but do not meet the conditions should receive a moderate score, while completely unrelated images should receive the lowest score.  

Provide your score on a scale of 1 to 10, with 10 being identical images that fully meet the conditions.  
Ensure your response is in the following format:
\begin{verbatim}
Score: <1-10>
Reason: <reason for score>
\end{verbatim}

Provide nothing else. What is your score? ``\texttt{data1}'' ``\texttt{data2}'' 
\end{tcolorbox}

\noindent
\begin{minipage}{0.48\textwidth}
\begin{tcolorbox}[enhanced,attach boxed title to top center={yshift=-3mm,yshifttext=-1mm},
  colback=red!5!white,colframe=red!20!gray,colbacktitle=red!20!gray,
  title=\qwenTwoVLSevenB{} - Original Order,fonttitle=\bfseries,
  boxed title style={size=small,colframe=red!20!gray},
  equal height group=examplegroup]

\begin{center}
    \includegraphics[width=0.9\linewidth]{imgs/pizza-image1.png}
\end{center}

\textbf{Model response:}
\begin{tcolorbox}[colback=gray!10!white, colframe=gray!50, sharp corners=south, boxrule=0.8pt]
\textit{Score: 6\\Reason: The images are very similar, with the second image being a slightly blurred version of the first.  
However, they do not fully meet the conditions as the images are not identical and the blurring is not perfect.}
\end{tcolorbox}

\end{tcolorbox}
\end{minipage}
\hfill
\begin{minipage}{0.48\textwidth}
\begin{tcolorbox}[enhanced,attach boxed title to top center={yshift=-3mm,yshifttext=-1mm},
  colback=red!5!white,colframe=red!20!gray,colbacktitle=red!20!gray,
  title=\qwenTwoVLSevenB{} - Reverse Order,fonttitle=\bfseries,
  boxed title style={size=small,colframe=red!20!gray},
  equal height group=examplegroup]

\begin{center}
    \includegraphics[width=0.9\linewidth]{imgs/pizza-image2.png}
\end{center}

\textbf{Model response:}
\begin{tcolorbox}[colback=gray!10!white, colframe=gray!50, sharp corners=south, boxrule=0.8pt]
\textit{Score: 9\\Reason: The images are nearly identical, with only a slight difference in the sharpness of the objects.  
The overall resemblance is high, and the conditions are met.}
\end{tcolorbox}

\end{tcolorbox}
\end{minipage}


\subsection{Phi3.5}
Below is an error example of \phiThreeFive{} on a data-pair from \mmscorecoco with elastic transform (ET).

\begin{tcolorbox}[enhanced,attach boxed title to top center={yshift=-3mm,yshifttext=-1mm},
  colback=red!5!white,colframe=red!20!gray,colbacktitle=red!20!gray,
  title=\phiThreeFive{} Evaluation Instructions,fonttitle=\bfseries,
  boxed title style={size=small,colframe=red!20!gray} ]

\textbf{User prompt}: Judge the similarity of these images based on:  
\emph{- Be sensitive to elastic transformations while evaluating images. For example, if the images are the same but have differing distortions due to elastic transformations, the similarity score should decrease. If the images are identical and have matching elastic transformations, the similarity score should remain high.}  

The similarity score should reflect both the overall resemblance of the images and how well they satisfy the conditions.  
Identical images that do not meet the conditions should still score higher than completely unrelated images.  

Provide a score on a scale of 1 to 10, with 1 being no similarity and 10 being identical images that satisfy all conditions.  
Respond strictly in this format:
\begin{verbatim}
Score: <1-10>
Reason: <reason for score>
\end{verbatim}

Do not include additional text. What’s your rating? ``\texttt{data1}'' ``\texttt{data2}'' 
\end{tcolorbox}

\noindent
\begin{minipage}{0.48\textwidth}
\begin{tcolorbox}[enhanced,attach boxed title to top center={yshift=-3mm,yshifttext=-1mm},
  colback=red!5!white,colframe=red!20!gray,colbacktitle=red!20!gray,
  title=\phiThreeFive{} - Original Order,fonttitle=\bfseries,
  boxed title style={size=small,colframe=red!20!gray},
  equal height group=examplegroup]

\begin{center}
    \includegraphics[width=0.9\linewidth]{imgs/motorcycle-image1.png}
\end{center}

\textbf{Model response:}
\begin{tcolorbox}[colback=gray!10!white, colframe=gray!50, sharp corners=south, boxrule=0.8pt]
\textit{Score: 1\\Reason: The images are highly distorted with elastic transformations, making it impossible to accurately compare the content.}
\end{tcolorbox}

\end{tcolorbox}
\end{minipage}
\hfill
\begin{minipage}{0.48\textwidth}
\begin{tcolorbox}[enhanced,attach boxed title to top center={yshift=-3mm,yshifttext=-1mm},
  colback=red!5!white,colframe=red!20!gray,colbacktitle=red!20!gray,
  title=\phiThreeFive{} - Reverse Order,fonttitle=\bfseries,
  boxed title style={size=small,colframe=red!20!gray},
  equal height group=examplegroup]

\begin{center}
    \includegraphics[width=0.9\linewidth]{imgs/motorcycle-image2.png}
\end{center}

\textbf{Model response:}
\begin{tcolorbox}[colback=gray!10!white, colframe=gray!50, sharp corners=south, boxrule=0.8pt]
\textit{Score: 10\\Reason: The images are identical with no differences in content, color, or composition.  
They both depict the same scene with motorcycles and people in a bar setting, and the elastic transformations do not alter the image in any way.}
\end{tcolorbox}

\end{tcolorbox}
\end{minipage}

\FloatBarrier
\section{Full Results}
\label{sec:full-results}

In this section, we provide the \nmi of all models on all the different splits of \mmscorecoco, \mmscorein, \mmscorewuimgimg, and \mmscorewuimgtext{} in Tables \ref{tab:mi-coco-in100-var}, \ref{tab:mi-coco-in100-invar}, \ref{tab:mi-coco-in100-var}, \ref{tab:mi-coco-in100-invar}, \ref{tab:mi-wu-imgimg-var}, \ref{tab:mi-wu-imgimg-invar}, and \ref{tab:mi-wu-imgtext}.

\begin{table*}[ht]
\centering
\caption{Comparison of the \nmi{} metric ($\times 100$) of \modelss{} on \mmscorecoco{} and \mmscorein{} benchmarks in the \textit{sensitive} setting. Models are evaluated across multiple criteria:  color jitter (CJ), elastic transform (ET), gaussian blur (GB), perspective shift (PS), and rotation (R). Higher scores indicate better performance.}
\begin{tabular}{l*{5}{c}*{5}{c}}
\toprule
\multirow{2}{*}{\textbf{Model}} & \multicolumn{5}{c}{\textbf{\mmscorecoco}} & \multicolumn{5}{c}{\textbf{\mmscorein}} \\
        \cmidrule(lr){2-6} \cmidrule(lr){7-11} & \textbf{CJ} & \textbf{ET} & \textbf{GB} & \textbf{PS} & \textbf{R} &  \textbf{CJ} & \textbf{ET} & \textbf{GB} & \textbf{PS} & \textbf{R} \\
\midrule
% model & coco &  &  &  &  & in100 &  &  &  &  \\
%  & CJ & ET & GB & PS & R & CJ & ET & GB & PS & R \\
\chameleon & 00.37 & 00.34 & 00.19 & 00.31 & 00.60 & 00.38 & 00.26 & 00.31 & 00.50 & 00.52 \\
\llavaonevision & 36.51 & 44.05 & 38.57 & 43.80 & 41.41 & 37.05 & 49.89 & 40.00 & 46.01 & 49.30 \\
\phiThreeFive & 38.21 & 51.61 & 61.94 & 47.33 & 34.56 & 25.74 & 43.03 & 51.40 & 32.51 & 23.61 \\
\pixtral & 37.67 & 56.25 & 54.32 & 49.53 & 36.80 & 30.75 & 52.30 & 51.94 & 46.04 & 40.76 \\
\rowcolor{blue!15}
\internvlTwoOneB & 03.23 & 03.47 & 03.27 & 03.63 & 03.51 & 02.59 & 02.38 & 01.70 & 02.02 & 02.23 \\
\rowcolor{blue!15}
\internvlTwoTwoB & 23.89 & 32.76 & 34.32 & 31.53 & 24.76 & 18.32 & 34.02 & 33.35 & 28.17 & 23.35 \\
\rowcolor{blue!15}
\internvlTwoFourB & 52.13 & 69.43 & 62.46 & 63.77 & 52.68 & 45.25 & 65.90 & 59.90 & 60.28 & 51.04 \\
\rowcolor{blue!15}
\internvlTwoEightB & 51.58 & 62.80 & 62.35 & 60.27 & 54.80 & 47.94 & 60.18 & 58.60 & 56.66 & 53.00 \\
\rowcolor{purple!15}
\internvlTwoFiveOneB & 16.74 & 25.38 & 27.67 & 24.83 & 16.54 & 15.63 & 33.67 & 39.23 & 37.97 & 22.53 \\
\rowcolor{purple!15}
\internvlTwoFiveTwoB & 12.48 & 19.58 & 25.26 & 18.33 & 13.84 & 17.27 & 38.28 & 39.21 & 31.23 & 21.45 \\
\rowcolor{purple!15}
\internvlTwoFiveFourB & 42.61 & 59.78 & 54.33 & 55.34 & 49.47 & 41.35 & 62.35 & 54.21 & 56.18 & 49.90 \\
\rowcolor{purple!15}
\internvlTwoFiveEightB & 54.51 & 73.37 & 78.31 & 63.17 & 60.71 & 51.76 & 77.10 & 76.40 & 60.40 & 55.30 \\
\rowcolor{orange!15}
\molmoEOneB & 00.40 & 00.09 & 01.20 & 00.03 & 00.05 & 00.41 & 00.01 & 00.45 & 00.01 & 00.01 \\
\rowcolor{orange!15}
\molmoOSevenB & 14.32 & 16.02 & 48.93 & 16.12 & 15.40 & 12.91 & 14.20 & 48.43 & 13.83 & 12.16 \\
\rowcolor{orange!15}
\molmoDSevenB & 27.06 & 45.28 & 34.46 & 49.60 & 30.39 & 22.88 & 41.06 & 35.83 & 44.49 & 32.22 \\
\rowcolor{yellow!15}
\qwenTwoVLTwoB & 09.91 & 11.82 & 09.01 & 13.13 & 11.95 & 10.63 & 13.69 & 10.41 & 13.21 & 12.23 \\
\rowcolor{yellow!15}
\qwenTwoVLSevenB & 42.58 & 61.90 & 50.22 & 55.81 & 51.10 & 38.24 & 61.73 & 50.23 & 53.07 & 52.29 \\
\midrule
\rowcolor{green!15}
\gptFouroMini & 49.98 & 65.97 & 58.29 & 53.23 & 53.60 & 47.06 & 67.06 & 56.43 & 49.97 & 52.59 \\
\rowcolor{green!15}
\gptFouroFive & 50.96 & 65.54 & 61.67 & 56.69 & 56.71 & 48.55 & 65.68 & 57.48 & 54.11 & 55.00 \\
\rowcolor{green!15}
\gptFouroEight & 42.26 & 60.58 & 56.62 & 50.13 & 53.63 & 40.35 & 60.66 & 52.65 & 49.62 & 49.77 \\
\rowcolor{green!15}
\gptFouroEleven & 51.31 & 63.50 & 61.35 & 57.84 & 57.16 & 50.88 & 66.55 & 58.14 & 56.25 & 55.52 \\
\rowcolor{green!30}
\geminiFlash & \textbf{58.26} & 82.64 & 87.41 & 65.92 & 61.08 & \textbf{56.25} & 79.69 & 85.21 & 62.07 & 61.15 \\
\rowcolor{green!30}
\geminiPro & 53.33 & \textbf{87.86} & \textbf{89.56} & \textbf{74.92} & \textbf{71.04} & 51.19 & \textbf{91.36} & \textbf{92.98} & \textbf{71.56} & \textbf{74.22} \\

\bottomrule
\end{tabular}
\label{tab:mi-coco-in100-var}
\end{table*}

\begin{table*}[ht]
\centering
\caption{Comparison of the \nmi{} metric ($\times 100$) of \modelss{} on \mmscorecoco{} and \mmscorein{} benchmarks in the \textit{invariant} setting. Models are evaluated across multiple criteria: color jitter (CJ), elastic transform (ET), gaussian blur (GB), perspective shift (PS), and rotation (R). Higher scores indicate better performance.}
\begin{tabular}{l*{5}{c}*{5}{c}}
\toprule
\multirow{2}{*}{\textbf{Model}} & \multicolumn{5}{c}{\textbf{\mmscorecoco}} & \multicolumn{5}{c}{\textbf{\mmscorein}} \\
        \cmidrule(lr){2-6} \cmidrule(lr){7-11} & \textbf{CJ} & \textbf{ET} & \textbf{GB} & \textbf{PS} & \textbf{R} &  \textbf{CJ} & \textbf{ET} & \textbf{GB} & \textbf{PS} & \textbf{R} \\
\midrule
% model & coco &  &  &  &  & in100 &  &  &  &  \\
%  & CJ & ET & GB & PS & R & CJ & ET & GB & PS & R \\
\chameleon & 00.89 & 00.34 & 00.44 & 00.51 & 00.38 & 00.57 & 00.35 & 00.53 & 00.58 & 00.45 \\
\llavaonevision & 35.13 & 37.26 & 39.22 & 40.29 & 38.29 & 38.09 & 43.04 & 41.83 & 40.86 & 42.24 \\
\phiThreeFive & 49.41 & 40.19 & 42.93 & 55.03 & 47.90 & 45.88 & 33.79 & 39.72 & 50.41 & 39.46 \\
\pixtral  & 48.26 & 47.34 & 45.35 & 60.20 & 55.65 & 41.53 & 45.30 & 42.84 & 52.63 & 52.65 \\
\rowcolor{blue!15}
\internvlTwoOneB & 02.69 & 01.76 & 02.71 & 02.00 & 02.69 & 01.39 & 00.82 & 01.22 & 00.90 & 01.40 \\
\rowcolor{blue!15}
\internvlTwoTwoB & 36.38 & 31.55 & 31.99 & 39.18 & 37.28 & 32.68 & 31.40 & 30.13 & 35.98 & 34.70 \\
\rowcolor{blue!15}
\internvlTwoFourB & 59.44 & 55.47 & 51.35 & 59.61 & 59.02 & 51.74 & 52.77 & 49.60 & 54.63 & 53.11 \\
\rowcolor{blue!15}
\internvlTwoEightB & 58.69 & 58.56 & 53.60 & 61.91 & 64.22 & 58.44 & 54.48 & 51.78 & 61.97 & 62.90 \\
\rowcolor{purple!15}
\internvlTwoFiveOneB  & 21.39 & 18.59 & 21.65 & 23.19 & 22.86 & 22.52 & 14.63 & 24.34 & 22.76 & 19.24 \\
\rowcolor{purple!15}
\internvlTwoFiveTwoB & 22.85 & 19.05 & 21.46 & 27.62 & 25.99 & 32.09 & 33.03 & 37.34 & 34.65 & 34.75 \\
\rowcolor{purple!15}
\internvlTwoFiveFourB & 56.24 & 47.41 & 43.93 & 53.71 & 55.28 & 61.80 & 50.50 & 47.33 & 51.58 & 58.56 \\
\rowcolor{purple!15}
\internvlTwoFiveEightB & \textbf{75.11} & 65.18 & 66.32 & \textbf{78.56} & \textbf{81.77} & \textbf{72.53} & 61.61 & 62.23 & 65.18 & 74.27 \\
\rowcolor{orange!15}
\molmoEOneB & 00.10 & 00.11 & 00.06 & 00.02 & 00.00 & 00.02 & 00.11 & 00.10 & 00.07 & 00.25 \\
\rowcolor{orange!15}
\molmoOSevenB & 26.86 & 34.58 & 33.46 & 34.70 & 24.55 & 25.04 & 30.81 & 38.52 & 32.79 & 27.65 \\
\rowcolor{orange!15}
\molmoDSevenB & 47.20 & 45.02 & 43.02 & 50.54 & 48.64 & 45.01 & 45.83 & 45.47 & 49.25 & 40.87 \\
\rowcolor{yellow!15}
\qwenTwoVLTwoB & 09.55 & 09.10 & 10.21 & 12.65 & 08.83 & 09.02 & 09.61 & 10.01 & 14.97 & 09.33 \\
\rowcolor{yellow!15}
\qwenTwoVLSevenB & 50.52 & 51.80 & 52.70 & 54.50 & 53.29 & 47.86 & 49.73 & 51.18 & 51.55 & 50.67 \\
\midrule
\rowcolor{green!15}
\gptFouroMini & 59.76 & 57.94 & 56.55 & 61.31 & 58.17 & 56.33 & 55.56 & 55.35 & 60.99 & 60.83 \\
\rowcolor{green!15}
\gptFouroFive & 70.83 & 61.70 & 59.40 & 61.13 & 62.10 & 68.82 & 56.16 & 56.70 & 57.79 & 59.80 \\
\rowcolor{green!15}
\gptFouroEight & 55.14 & 50.31 & 46.00 & 52.15 & 52.45 & 54.13 & 45.43 & 44.25 & 48.26 & 52.18 \\
\rowcolor{green!15}
\gptFouroEleven & 73.48 & 69.06 & 61.51 & 67.60 & 63.99 & 70.16 & 61.33 & 58.89 & 65.06 & 60.84 \\
\rowcolor{green!30}
\geminiFlash & 72.11 & 67.81 & 68.17 & 71.88 & 78.31 & 70.32 & 65.94 & 66.58 & 69.10 & 74.77 \\
\rowcolor{green!30}
\geminiPro & 68.93 & \textbf{69.64} & \textbf{71.50} & 72.06 & 68.42 & 66.31 & \textbf{70.03} & \textbf{72.17} & \textbf{70.13} & \textbf{69.32} \\

\bottomrule
\end{tabular}
\label{tab:mi-coco-in100-invar}
\end{table*}

\begin{table*}[ht]
\centering
\caption{Comparison of the \nmi{} metric ($\times 100$) of \modelss{} on \mmscorewuimgimg{} (subset A and B) benchmark in the \textit{sensitive} setting. Models are evaluated across multiple criteria: spatial position (SP), spatial position and color jitter (SP-CJ), spatial position and elastic transform (SP-ET), spatial position and gaussian blur (SP-GB), spatial position and perspective shift (SP-PS), and spatial position and rotation (SP-R). Higher scores indicate better performance.}
\resizebox{0.95\textwidth}{!}{%
\begin{tabular}{l*{6}{c}*{6}{c}}
\toprule
\multirow{2}{*}{\textbf{Model}} & \multicolumn{6}{c}{\textbf{\mmscore$_{WU_a}$}} & \multicolumn{6}{c}{\textbf{\mmscore$_{WU_b}$}} \\
        \cmidrule(lr){2-7} \cmidrule(lr){8-13} & \textbf{SP} & \textbf{SP-CJ} & \textbf{SP-ET} & \textbf{SP-GB} & \textbf{SP-PS} & \textbf{SP-R} & \textbf{SP} &  \textbf{SP-CJ} & \textbf{SP-ET} & \textbf{SP-GB} & \textbf{SP-PS} & \textbf{SP-R}\\
\midrule
% model & coco &  &  &  &  & in100 &  &  &  &  \\
%  & CJ & ET & GB & PS & R & CJ & ET & GB & PS & R \\
\chameleon & 00.28 & 00.47 & 00.23 & 00.52 & 0.21 & 00.20 & 00.34 & 00.38 & 00.35 & 00.26 & 00.31 & 00.33 \\
\llavaonevision & 38.95 & 18.83 & 24.03 & 26.78 & 29.46 & 24.63 & 19.70 & 14.03 & 16.51 & 16.78 & 17.76 & 17.02 \\
\phiThreeFive & 23.44 & 08.46 & 15.70 & 19.41 & 13.34 & 10.83 & 15.38 & 12.98 & 18.91 & 20.19 & 11.69 & 17.06 \\
\pixtral & 37.91 & 26.09 & 32.05 & 33.52 & 32.47 & 25.00 & 28.02 & 19.58 & 22.32 & 22.31 & 23.46 & 24.50 \\
\rowcolor{blue!15}
\internvlTwoOneB & 00.44 & 00.98 & 00.79 & 00.65 & 00.30 & 00.28 & 00.20 & - & - & 00.41 & 01.18 & 00.90 \\
\rowcolor{blue!15}
\internvlTwoTwoB & 22.85 & 12.03 & 14.37 & 17.84 & 18.66 & 15.50 & 20.72 & 10.89 & 11.22 & 15.74 & 17.74 & 13.58 \\
\rowcolor{blue!15}
\internvlTwoFourB & 46.89 & 27.91 & 36.67 & 43.03 & 44.27 & 27.76 & 44.89 & 27.77 & 33.35 & 38.12 & 42.23 & 36.16 \\
\rowcolor{blue!15}
\internvlTwoEightB & 41.99 & 32.06 & 35.71 & 41.02 & 40.12 & 29.11 & 46.36 & 32.17 & 39.24 & 41.90 & 45.59 & 40.30 \\
\rowcolor{purple!15}
\internvlTwoFiveOneB & 25.50 & 14.16 & 21.32 & 15.69 & 21.49 & 16.30 & 24.77 & 16.16 & 21.10 & 19.95 & 27.89 & 21.47 \\
\rowcolor{purple!15}
\internvlTwoFiveTwoB & 20.63 & 11.76 & 16.75 & 15.21 & 18.03 & 13.79 & 23.44 & 09.33 & 15.90 & 17.64 & 18.17 & 17.56 \\
\rowcolor{purple!15}
\internvlTwoFiveFourB & 46.15 & 32.74 & 39.05 & 39.24 & 42.28 & 32.94 & 47.93 & 33.75 & 40.23 & 39.82 & 44.07 & 42.57 \\
\rowcolor{purple!15}
\internvlTwoFiveEightB & 44.27 & 36.99 & 41.49 & 42.60 & 43.65 & 33.24 & 41.32 & 31.69 & 40.10 & 39.73 & 44.03 & 42.99 \\
\rowcolor{orange!15}
\molmoEOneB & 00.47 & 01.03 & 00.00 & 00.03 & 00.14 & 00.01 & 00.32 & 00.36 & 00.01 & 00.04 & 00.04 & 00.09 \\
\rowcolor{orange!15}
\molmoOSevenB & 15.94 & 09.90 & 11.32 & 15.38 & 12.92 & 12.01 & 15.15 & 08.40 & 11.39 & 11.33 & 13.60 & 12.50 \\
\rowcolor{orange!15}
\molmoDSevenB & 23.82 & 17.75 & 20.41 & 18.40 & 22.21 & 17.81 & 26.74 & 18.37 & 19.55 & 18.77 & 18.19 & 22.21 \\
\rowcolor{yellow!15}
\qwenTwoVLTwoB & 02.26 & 01.76 & 02.58 & 02.15 & 03.17 & 01.68 & 00.88 & 00.44 & 00.73 & 00.37 & 00.72 & 00.82 \\
\rowcolor{yellow!15}
\qwenTwoVLSevenB & 41.95 & 29.47 & 36.32 & 39.93 & 40.33 & 34.11 & 42.80 & 28.75 & 31.42 & 37.27 & 39.76 & 36.25 \\
\midrule
\rowcolor{green!15}
\gptFouroMini & 42.55 & 37.21 & 39.50 & 40.44 & 38.83 & 41.05 & 48.86 & 38.38 & 43.82 & 45.42 & 46.32 & 46.66 \\
\rowcolor{green!15}
\gptFouroFive & 40.27 & 37.83 & 36.79 & 38.52 & 38.84 & 38.07 & 44.13 & 39.46 & 39.46 & 43.58 & 43.49 & 46.25 \\
\rowcolor{green!15}
\gptFouroEight & 37.58 & 33.72 & 34.24 & 33.36 & 34.80 & 33.17 & 40.11 & 33.36 & 32.36 & 34.32 & 39.91 & 38.67 \\
\rowcolor{green!15}
\gptFouroEleven & 40.68 & 39.06 & 40.10 & 40.35 & 40.96 & 40.40 & 47.34 & 40.91 & 43.07 & 47.18 & 50.22 & 50.68 \\
\rowcolor{green!30}
\geminiFlash & 44.63 & 38.85 & 37.19 & 39.11 & 35.76 & 34.57 & 49.91 & 40.29 & 42.92 & 46.34 & 47.01 & 46.40 \\
\rowcolor{green!30}
\geminiPro & 40.38 & 36.07 & 31.52 & 37.85 & 29.92 & 30.37 & 49.20 & 38.26 & 39.16 & 44.98 & 41.70 & 40.72 \\
\bottomrule
\end{tabular}
}
\label{tab:mi-wu-imgimg-var}
\end{table*}


\begin{table*}[ht]
\centering
\caption{Comparison of the \nmi{} metric ($\times 100$) of \modelss{} on \mmscorewuimgimg{} (subset A and B) benchmark in the \textit{invariant} setting. Models are evaluated across multiple criteria:spatial position (SP), spatial position and color jitter (SP-CJ), spatial position and elastic transform (SP-ET), spatial position and gaussian blur (SP-GB), spatial position and perspective shift (SP-PS), and spatial position and rotation (SP-R). Higher scores indicate better performance.}
\resizebox{0.95\textwidth}{!}{%
\begin{tabular}{l*{6}{c}*{6}{c}}
\toprule
\multirow{2}{*}{\textbf{Model}} & \multicolumn{6}{c}{\textbf{\mmscore$_{WU_a}$}} & \multicolumn{6}{c}{\textbf{\mmscore$_{WU_b}$}} \\
        \cmidrule(lr){2-7} \cmidrule(lr){8-13} & \textbf{SP} & \textbf{SP-CJ} & \textbf{SP-ET} & \textbf{SP-GB} & \textbf{SP-PS} & \textbf{SP-R} & \textbf{SP} &  \textbf{SP-CJ} & \textbf{SP-ET} & \textbf{SP-GB} & \textbf{SP-PS} & \textbf{SP-R} \\
\midrule
% model & coco &  &  &  &  & in100 &  &  &  &  \\
%  & CJ & ET & GB & PS & R & CJ & ET & GB & PS & R \\
\chameleon & 00.34 & 00.39 & 00.76 & 00.47 & 00.43 & 00.41 & 00.47 & 00.34 & 00.56 & 00.24 & 00.62 & 00.34 \\
\llavaonevision & 34.79 & 31.56 & 30.23 & 34.14 & 32.61 & 28.69 & 13.12 & 18.41 & 16.21 & 22.69 & 15.34 & 17.91 \\
\phiThreeFive & 23.66 & 32.84 & 18.90 & 21.36 & 30.14 & 19.10 & 19.88 & 36.74 & 22.40 & 23.47 & 30.04 & 26.06 \\
\pixtral & 36.93 & 37.32 & 41.17 & 35.31 & 38.52 & 36.05 & 36.03 & 30.44 & 33.32 & 29.84 & 35.48 & 33.32 \\
\rowcolor{blue!15}
\internvlTwoOneB & 00.57 & 01.08 & 02.02 & 01.02 & 00.89 & 00.37 & 00.65 & 00.81 & 00.96 & 00.50 & 00.56 & 00.54 \\
\rowcolor{blue!15}
\internvlTwoTwoB & 26.25 & 25.53 & 25.76 & 21.12 & 26.57 & 26.98 & 26.03 & 24.52 & 26.49 & 25.81 & 31.01 & 29.33 \\
\rowcolor{blue!15}
\internvlTwoFourB & 39.33 & 40.23 & 37.80 & 42.25 & 43.10 & 34.57 & 51.43 & 41.55 & 45.96 & 50.20 & 54.94 & 50.34 \\
\rowcolor{blue!15}
\internvlTwoEightB & 43.80 & 44.31 & 44.53 & 43.99 & 46.02 & 40.43 & 60.92 & 46.63 & 54.53 & 51.31 & 56.94 & 53.88 \\
\rowcolor{purple!15}
\internvlTwoFiveOneB & 12.82 & 13.84 & 09.34 & 07.24 & 12.91 & 16.93 & 19.87 & 24.92 & 19.36 & 17.94 & 22.66 & 30.60 \\
\rowcolor{purple!15}
\internvlTwoFiveTwoB & 31.38 & 29.79 & 30.53 & 23.16 & 31.75 & 24.69 & 36.01 & 30.13 & 35.52 & 27.07 & 37.01 & 31.18 \\
\rowcolor{purple!15}
\internvlTwoFiveFourB & 48.79 & 53.58 & 54.52 & 48.09 & 52.78 & 46.46 & 50.51 & 48.71 & 53.45 & 52.03 & 53.77 & 50.12 \\
\rowcolor{purple!15}
\internvlTwoFiveEightB & 59.03 & 55.57 & 59.70 & 57.16 & 58.01 & 50.84 & 65.21 & 51.31 & 61.10 & 63.54 & 62.38 & 60.83 \\
\rowcolor{orange!15}
\molmoEOneB & 03.83 & 00.09 & 00.02 & 00.02 & 00.10 & 00.17 & 04.22 & 00.07 & 00.02 & 00.07 & 00.12 & 00.00 \\
\rowcolor{orange!15}
\molmoOSevenB & 18.63 & 17.50 & 19.68 & 16.42 & 19.58 & 14.99 & 15.94 & 19.46 & 20.93 & 17.98 & 24.21 & 21.68 \\
\rowcolor{orange!15}
\molmoDSevenB & 28.21 & 36.47 & 31.95 & 26.89 & 35.57 & 33.58 & 37.50 & 35.90 & 34.70 & 33.51 & 33.04 & 34.35 \\
\rowcolor{yellow!15}
\qwenTwoVLTwoB & 02.63 & 02.88 & 03.58 & 03.53 & 03.34 & 02.97 & 00.79 & 00.73 & 00.99 & 00.88 & 00.71 & 00.82 \\
\rowcolor{yellow!15}
\qwenTwoVLSevenB & 40.21 & 38.96 & 39.94 & 46.88 & 40.11 & 39.55 & 47.65 & 39.51 & 40.94 & 48.63 & 44.68 & 41.88 \\
\midrule
\rowcolor{green!15}
\gptFouroMini & 47.60 & 48.33 & 51.04 & 46.15 & 48.86 & 43.75 & 57.50 & 49.19 & 51.38 & 53.76 & 55.82 & 54.07 \\
\rowcolor{green!15}
\gptFouroFive & 52.39 & 51.58 & 48.78 & 47.11 & 47.50 & 52.68 & 61.59 & 59.77 & 58.08 & 60.95 & 61.53 & 63.74 \\
\rowcolor{green!15}
\gptFouroEight & 50.94 & 47.21 & 46.52 & 42.90 & 45.84 & 52.50 & 62.75 & 54.23 & 53.20 & 51.19 & 58.50 & 57.21 \\
\rowcolor{green!15}
\gptFouroEleven & 57.47 & 56.25 & 54.40 & 56.11 & 54.40 & 57.93 & 65.91 & 62.22 & 63.93 & 67.96 & 66.86 & 68.10 \\
\rowcolor{green!30}
\geminiFlash & 46.62 & 55.28 & 54.31 & 57.98 & 57.01 & 58.74 & 62.04 & 54.43 & 56.89 & 62.24 & 66.88 & 60.72 \\
\rowcolor{green!30}
\geminiPro & 38.07 & 35.08 & 35.05 & 36.11 & 33.21 & 33.23 & 56.43 & 42.24 & 43.74 & 48.41 & 50.40 & 45.83 \\
\bottomrule
\end{tabular}
}
\label{tab:mi-wu-imgimg-invar}
\end{table*}


\begin{table*}[ht]
\centering
\caption{Comparison of the \nmi{} metric ($\times 100$) of \modelss{} on the \mmscorewuimgtext{} (Subset A and B) benchmark in the \textit{sensitive} and \textit{invariant} settings. Models are evaluated across the spatial position (SP) criterion. Higher scores indicate better performance.}
\begin{tabular}{l*{2}{c}*{2}{c}}
\toprule
\multirow{2}{*}{\textbf{Model}} & \multicolumn{2}{c}{\textbf{\mmscore$_{WU_a}$}} & \multicolumn{2}{c}{\textbf{\mmscore$_{WU_b}$}} \\
        \cmidrule(lr){2-3} \cmidrule(lr){4-5} & \textbf{Sens.} & \textbf{Invar.} & \textbf{Sens.} & \textbf{Invar.} \\
\midrule
% model & coco &  &  &  &  & in100 &  &  &  &  \\
%  & CJ & ET & GB & PS & R & CJ & ET & GB & PS & R \\
\chameleon & 00.25 & 00.34 & 00.23 & 00.47 \\
\llavaonevision & 23.35 & 22.78 & 27.38 & 25.98  \\
\phiThreeFive & 13.86 & 12.30 & 25.67 & 24.74  \\
\pixtral & 05.14 & 05.04 & 03.27 & 04.58  \\
\rowcolor{blue!15}
\internvlTwoOneB & 06.29 & 03.75 & 15.90 & 08.31  \\
\rowcolor{blue!15}
\internvlTwoTwoB & 17.07 & 14.26 & 24.46 & 16.49  \\
\rowcolor{blue!15}
\internvlTwoFourB & 15.69 & 15.69 & 24.27 & 22.96  \\
\rowcolor{blue!15}
\internvlTwoEightB & 22.40 & 19.27 & 29.45 & 31.46  \\
\rowcolor{purple!15}
\internvlTwoFiveOneB & 20.80 & 09.49 & 16.86 & 13.23  \\
\rowcolor{purple!15}
\internvlTwoFiveTwoB & 15.36 & 11.15 & 19.69 & 18.42  \\
\rowcolor{purple!15}
\internvlTwoFiveFourB & 23.90 & 23.85 & 29.75 & 32.45  \\
\rowcolor{purple!15}
\internvlTwoFiveEightB & 24.16 & 25.55 & 24.00 & 28.22  \\
\rowcolor{orange!15}
\molmoEOneB & 00.12 & 00.04 & 00.02 & 00.21  \\
\rowcolor{orange!15}
\molmoOSevenB & 07.53 & 07.45 & 07.18 & 08.29  \\
\rowcolor{orange!15}
\molmoDSevenB & 09.45 & 12.26 & 08.34 & 11.26  \\
\rowcolor{yellow!15}
\qwenTwoVLTwoB & 02.65 & 03.09 & 05.09 & 05.86  \\
\rowcolor{yellow!15}
\qwenTwoVLSevenB & 09.43 & 09.19 & 15.99 & 16.13  \\
\midrule 
\rowcolor{green!15}
\gptFouroMini & 16.18 & 16.14 & 16.18 & 15.30  \\
\rowcolor{green!15}
\gptFouroFive & 11.49 & 20.48 & 12.63 & 20.98  \\
\rowcolor{green!15}
\gptFouroEight & 20.27 & 31.80 & 22.97 & 36.56  \\
\rowcolor{green!15}
\gptFouroEleven & 18.97 & 31.91 & 20.57 & 34.99  \\
\rowcolor{green!30}
\geminiFlash & 27.46 & 26.54 & 26.53 & 32.07  \\
\rowcolor{green!30}
\geminiPro & 26.89 & 27.16 & 28.57 & 29.23  \\

\bottomrule
\end{tabular}
\label{tab:mi-wu-imgtext}
\end{table*}
\begin{figure*}[ht]
    \centering
    \includegraphics[width=0.95\linewidth,trim={.1cm .2cm .2cm .2cm},clip]{imgs/all-metric-mi-distinct-plt.pdf}
    \includegraphics[width=0.95\linewidth,trim={.1cm .2cm .2cm .2cm},clip]{imgs/all-metric-entropy-distinct-plt.pdf}
    \includegraphics[width=0.95\linewidth,trim={.1cm .2cm .2cm .2cm},clip]{imgs/all-metric-relaxed1_sym-distinct-plt.pdf}
    \caption{\nmi, \textbf{Smoothness}, and \textbf{Controllability} for the best performing models in both \texttt{sens} and \texttt{invar} settings.}
    \label{fig:best-models-mmscore-smoothness-sym}
\end{figure*}


\subsubsection{All \relaxsym for different $\epsilon$s}
To show the \relaxsym{} for different values of $\varepsilon$, we plot Figure \ref{fig:diff-relax-sym-eps} and show as $\varepsilon$ gets higher, the values go higher. However, some models such as the GPT4o models struggle with symmetry. Please note that if $\varepsilon = 0$, it is the same as not having a threshold and hence calculating exact symmetry rather than a relaxed version.


\begin{figure*}[ht]
    \centering
    \includegraphics[width=0.95\linewidth,trim={.1cm .2cm .2cm .2cm},clip]{imgs/all-metric-exact_sym-relaxed1_sym-relaxed2_sym-relaxed3_sym-distinct-plt.pdf}
    \caption{\relaxsym{} for different $\varepsilon$s.}
    \label{fig:diff-relax-sym-eps}
\end{figure*}


\subsubsection{Different versions of same model}
\label{sec:model-versions}
% We further look into the affect of capacity on the different metrics of \mmscore. As seen in Figure \ref{fig:mi-model-versions} and \ref{fig:all-model-versions}, the larger capacity models tend to better across \nmi, \relaxsym, and \control. However, we observe there are exceptions, e.g., \internvlTwoFourB{} being more controllable in rotation (R) and perspective shift (PS), compared to \internvlTwoEightB. Also, we see \smoothness{} is not monotonically increasing as the model capacity increases. This shows that the stronger models may tend to be more certain about their responses, hence not generating similarity scores as diverse as the lower capacity ones.

% On the other hand, we saw in Table \ref{tab:benchmark_comparison} and Figure \ref{fig:control-vs-bm} that Smoothness also has a positive correlation with model performance and other benchmarks, showing that better models tend to be more smoother and create more diverse outputs compared to the weaker ones. Ultimately, we conclude that \smoothness{} is not a property of performance; however, it is a characteristic of a \model{} as a judge model which could be desirable depending on the use-case.

We further examine the effect of model capacity on the different metrics of \mmscore. As seen in Figures \ref{fig:mi-model-versions} and \ref{fig:all-model-versions}, larger-capacity models tend to perform better across \nmi, \relaxsym, and \control. However, there are exceptions—for example, \internvlTwoFourB{} demonstrates greater controllability in rotation (R) and perspective shift (PS) compared to \internvlTwoEightB. Additionally, smoothness (\smoothness{}) does not increase monotonically with model capacity. This suggests that stronger models may be more confident in their responses, leading to less diversity in their similarity scores compared to lower-capacity models.

On the other hand, Table \ref{tab:benchmark_comparison} and Figure \ref{fig:control-vs-bm} show that \smoothness{} correlates positively with model performance and other benchmarks, indicating that better models tend to produce smoother and more diverse outputs than weaker ones. Ultimately, we conclude that \smoothness{} is not strictly a property of model performance but rather a characteristic of a \model{} as a judge model that may be desirable (or not) depending on the use case. 


\begin{figure*}[ht]
    \centering
    % \includegraphics[width=0.95\linewidth,trim={.1cm .2cm .2cm .2cm},clip]{imgs/GPT-4o-Models-metric-mi-distinct-plt.pdf}
    \includegraphics[width=0.95\linewidth,trim={.1cm .2cm .2cm .2cm},clip]{imgs/invl-metric-entropy-distinct-plt.pdf}
    % \caption{Caption for First PDF.}
    % \label{fig:first-plot}

    \vspace{1em} % Add vertical spacing between figures if needed

    \includegraphics[width=0.95\linewidth,trim={.1cm .2cm .2cm .2cm},clip]{imgs/invl-metric-relaxed1_sym-distinct-plt.pdf}
    % \caption{Caption for Second PDF.}
    % \label{fig:second-plot}

    \vspace{1em} % Add vertical spacing between figures if needed

    \includegraphics[width=0.95\linewidth,trim={.1cm .2cm .2cm .2cm},clip]{imgs/invl-metric-entropy-distinct-plt.pdf}
    % \caption{Caption for Third PDF.}
    % \label{fig:third-plot}

    \caption{InternVL2.5 models with different capacities aggregated on \mmscorecoco{} and \mmscorein.}
    \label{fig:mi-model-versions}
\end{figure*}
\begin{figure*}[ht]
    \centering

    \includegraphics[width=0.95\linewidth,trim={.1cm .2cm .2cm .2cm},clip]{imgs/invl-metric-mi-relaxed1_sym-entropy-control_mult-distinct-plt.pdf}
    % \caption{Caption for Third PDF.}
    % \label{fig:third-plot}

    \caption{Aggregated \mmscore{} metrics across different versions of InternVL2.5 models.}
    \label{fig:all-model-versions}
\end{figure*}


\subsection{Correlations}
\label{sec:bm-correlations}

In this section, we plot the further correlations of the different metrics and show them in Figures \ref{fig:sym-vs-bm}, \ref{fig:control-vs-bm}, \ref{fig:control-vs-bm}. As seen, all these metrics have positive correlations as seen in the scatter plots.


\begin{figure*}[ht]
    \centering
    % \includegraphics[width=0.95\linewidth,trim={.1cm .2cm .2cm .2cm},clip]{imgs/sym-vs-bm-scatter-plot-correlation.pdf}
    \includegraphics[width=0.95\linewidth,trim={.1cm .2cm .2cm .2cm},clip]{imgs/mmscore_n_relxsym-v2.pdf}
    \caption{Other benchmarks versus \mmscore{} on \relaxsymone{}.}
    \label{fig:sym-vs-bm}
\end{figure*}
\begin{figure*}[ht]
    \centering
    \includegraphics[width=0.95\linewidth,trim={.1cm .2cm .2cm .2cm},clip]{imgs/mmscore_control.pdf}
    \caption{Other benchmarks versus \control{} on \mmscore{}.}
    \label{fig:control-vs-bm}
\end{figure*}
\begin{figure*}[ht]
    \centering
    \includegraphics[width=0.95\linewidth,trim={.1cm .2cm .2cm .2cm},clip]{imgs/mmscore_ent-v2.pdf}
    \caption{Other benchmarks versus Smoothness (\smoothness).}
    \label{fig:smoothness-vs-bm}
\end{figure*}

% \input{imgs/sym-plot}

% \newpage
\section{\mmscore{} Details}
\label{sec:mmscore-info}
We provide all the information regarding the released datasets \mmscorecoco, \mmscorein, \mmscorewuimgimg, and \mmscorewuimgtext{} in Table \ref{tab:mmscore-info}.
\begin{table*}[ht]
\centering
\caption{Information of different splits in \mmscorecoco, \mmscorein, \mmscorewuimgimg, and \mmscorewuimgtext.}
\label{tab:mmscore-info}
\resizebox{\textwidth}{!}{%
\begin{tabular}{llccc}
\toprule
\textbf{Modality} & \textbf{Source} & \textbf{Number of Selected} & \textbf{Splits} & \textbf{New Data Points / Total Data-Pair Comparisons} \\
\midrule
% \multirow{11}{*}{MMSCore (Image-Image)} & \multirow{6}{*}{COCO} & \multirow{6}{*}{500} & CJ & 1000 / 3000 \\ 
%  & & & R & 1000 / 3000 \\ 
%  & & & ET & 1000 / 3000 \\ 
%  & & & PS & 1000 / 3000 \\ 
%  & & & GB & 1000 / 3000 \\ 
% \cmidrule{2-5}
%  & \multirow{6}{*}{IN100} & \multirow{6}{*}{500} & CJ & 1000 / 3000 \\ 
%  & & & R & 1000 / 3000 \\ 
%  & & & ET & 1000 / 3000 \\ 
%  & & & PS & 1000 / 3000 \\ 
%  & & & GB & 1000 / 3000 \\ 
% \cmidrule{2-5}
\multirow{22}{*}{\mmscoreimgimg} & \multirow{6}{*}{\coco} & \multirow{6}{*}{500} & CJ & 1000 / 3000 \\ 
 & & & R & 1000 / 3000 \\ 
 & & & ET & 1000 / 3000 \\ 
 & & & PS & 1000 / 3000 \\ 
 & & & GB & 1000 / 3000 \\ 
\cmidrule{2-5}
 & \multirow{6}{*}{\imagenet} & \multirow{6}{*}{500} & CJ & 1000 / 3000 \\ 
 & & & R & 1000 / 3000 \\ 
 & & & ET & 1000 / 3000 \\ 
 & & & PS & 1000 / 3000 \\ 
 & & & GB & 1000 / 3000 \\ 
\cmidrule{2-5}
 & \multirow{6}{*}{\wu{} (subset A)} & \multirow{6}{*}{418} & SP & 0 / 3344 \\ 
 & & & SP \& CJ & 1254 / 3344 \\ 
 & & & SP \& R & 1254 / 3344 \\ 
 & & & SP \& ET & 1254 / 3344 \\ 
 & & & SP \& PS & 1254 / 3344 \\ 
 & & & SP \& GB & 1254 / 3344 \\ 
\cmidrule{2-5}
 & \multirow{6}{*}{\wu{} (subset B)} & \multirow{6}{*}{408} & SP & 0 / 3264 \\ 
 & & & SP \& CJ & 1224 / 3264 \\ 
 & & & SP \& R & 1224 / 3264 \\ 
 & & & SP \& ET & 1224 / 3264 \\ 
 & & & SP \& PS & 1224 / 3264 \\ 
 & & & SP \& GB & 1224 / 3264 \\ 
\midrule
\multirow{2}{*}{\mmscoreimgtext} & \wu{} (subset A) & 418 & SP & 1254 / 3344 \\ 
 & \wu{} (Subset B) & 408 & SP & 1224 / 3264 \\ 
\midrule
\textbf{In total} & - & 1826 & all splits & \textbf{22390 / 69648} \\ 
\bottomrule
\end{tabular}
}
\end{table*}


\subsection{Prompt Templates for Different \mmscore{} datasets}
\label{sec:prompt-templates}
We provide the 5 different templates that we choose at random for each data pair for the image-image and image-text prompts.

The following are the tempaltes we utilize for \mmscorecoco and \mmscorein, and \mmscorewuimgimg, i.e., the image-image pairs.

\begin{tcolorbox}[enhanced,attach boxed title to top center={yshift=-3mm,yshifttext=-1mm},
  colback=blue!5!white,colframe=blue!20!gray,colbacktitle=blue!20!gray,
  title=Image-Image Prompt Template V1,fonttitle=\bfseries,
  boxed title style={size=small,colframe=blue!20!gray} ]

        \emph{User prompt}: You are tasked with evaluating the similarity between two images while paying attention to the following conditions: \texttt{\{conditions\}}. Your goal is to judge the similarity of the images overall, where satisfying the conditions increases the similarity score. If the images are identical but fail to meet any of the conditions, they should still receive a higher score than completely unrelated images. Provide a similarity score on a scale from 1 to 10, where 1 represents entirely dissimilar images and 10 represents identical images that satisfy all conditions. Ensure your response is strictly in the following format:
        
        \begin{verbatim}
        Score: <1-10>
        Reason: <reason for score>
        \end{verbatim}
                
        Do not include anything else in your response. What score would you assign to this pair of images? \texttt{"data1"} \texttt{"data2"}.

\end{tcolorbox}

\begin{tcolorbox}[enhanced,attach boxed title to top center={yshift=-3mm,yshifttext=-1mm},
  colback=blue!5!white,colframe=blue!20!gray,colbacktitle=blue!20!gray,
  title=Image-Image Prompt Template V2,fonttitle=\bfseries,
  boxed title style={size=small,colframe=blue!20!gray} ]

        \emph{User prompt}: As a similarity evaluator, your responsibility is to assess the similarity of the given images while considering these conditions: \texttt{\{conditions\}}. The similarity score should reflect both how well the images align with the conditions and their overall resemblance. Images that are identical but do not meet the conditions should receive a moderate score, while completely unrelated images should receive the lowest score. Provide your score on a scale of 1 to 10, with 10 being identical images that fully meet the conditions. Ensure your response is in the following format:
        
        \begin{verbatim}
        Score: <1-10>
        Reason: <reason for score>
        \end{verbatim}
                
        Provide nothing else. What is your score? \texttt{"data1"} \texttt{"data2"}

\end{tcolorbox}

\begin{tcolorbox}[enhanced,attach boxed title to top center={yshift=-3mm,yshifttext=-1mm},
  colback=blue!5!white,colframe=blue!20!gray,colbacktitle=blue!20!gray,
  title=Image-Image Prompt Template V3,fonttitle=\bfseries,
  boxed title style={size=small,colframe=blue!20!gray} ]

        \emph{User prompt}: Evaluate the similarity of the images based on the following conditions: \texttt{\{conditions\}}. The score should take into account how well the images align with these conditions, as well as their overall resemblance. Even if the images are identical but fail to meet the conditions, they should still receive a higher score than completely different images. Provide a score from 1 to 10, where 1 indicates no similarity and 10 indicates identical images that fully satisfy the conditions. Respond only in this format:
        
        \begin{verbatim}
        Score: <1-10>
        Reason: <reason for score>
        \end{verbatim}
                
        Nothing else should be included. What score would you give? \texttt{"data1"} \texttt{"data2"}

\end{tcolorbox}

\begin{tcolorbox}[enhanced,attach boxed title to top center={yshift=-3mm,yshifttext=-1mm},
  colback=blue!5!white,colframe=blue!20!gray,colbacktitle=blue!20!gray,
  title=Image-Image Prompt Template V4,fonttitle=\bfseries,
  boxed title style={size=small,colframe=blue!20!gray} ]

        \emph{User prompt}: Judge the similarity of these images based on: \texttt{\{conditions\}}. The similarity score should reflect both the overall resemblance of the images and how well they satisfy the conditions. Identical images that do not meet the conditions should still score higher than completely unrelated images. Provide a score on a scale of 1 to 10, with 1 being no similarity and 10 being identical images that satisfy all conditions. Respond strictly in this format:
        
        \begin{verbatim}
        Score: <1-10>
        Reason: <reason for score>
        \end{verbatim}
                
        Do not include additional text. What's your rating? \texttt{"data1"} \texttt{"data2"}

\end{tcolorbox}

\begin{tcolorbox}[enhanced,attach boxed title to top center={yshift=-3mm,yshifttext=-1mm},
  colback=blue!5!white,colframe=blue!20!gray,colbacktitle=blue!20!gray,
  title=Image-Image Prompt Template V5,fonttitle=\bfseries,
  boxed title style={size=small,colframe=blue!20!gray} ]

        \emph{User prompt}: Rate the similarity of these images on a scale of 1 to 10 while considering \texttt{\{conditions\}}. The score should account for both how well the images satisfy the conditions and their overall resemblance. If the images are identical but do not meet the conditions, they should still score higher than unrelated images. Your response must follow this exact format:
        
        \begin{verbatim}
        Score: <1-10>
        Reason: <reason for score>
        \end{verbatim}
                
        Provide nothing else. What's your score? \texttt{"data1"} \texttt{"data2"}

\end{tcolorbox}


And these are the image-text versions for \mmscorewuimgtext:

\begin{tcolorbox}[enhanced,attach boxed title to top center={yshift=-3mm,yshifttext=-1mm},
  colback=green!5!white,colframe=green!20!gray,colbacktitle=green!20!gray,
  title=Image-Text Prompt Template V1,fonttitle=\bfseries,
  boxed title style={size=small,colframe=green!20!gray} ]

        \emph{User prompt}: You are assigned the task of determining how accurately a text description represents an image of two objects. Pay special attention to the following conditions: \texttt{\{conditions\}}. Your goal is to assess the alignment between the image and the text, where adherence to the conditions increases the similarity score. If the text partially describes the image but fails to meet specific conditions, it should still score higher than a text entirely unrelated to the image. Assign a similarity score on a scale from 0 to 10, where 0 signifies no correlation, and 10 denotes perfect alignment satisfying all conditions. The data pair (text and image) will follow this prompt. Respond strictly in the following format:
        
        \begin{verbatim}
        Score: <1-10>
        Reason: <reason for score>
        \end{verbatim}
        
        What score would you assign to this text-image pair?\
        \texttt{"data1"}\
        \texttt{"data2"}.\
        
\end{tcolorbox}

\begin{tcolorbox}[enhanced,attach boxed title to top center={yshift=-3mm,yshifttext=-1mm},
  colback=green!5!white,colframe=green!20!gray,colbacktitle=green!20!gray,
  title=Image-Text Prompt Template V2,fonttitle=\bfseries,
  boxed title style={size=small,colframe=green!20!gray} ]

        \emph{User prompt}: Figure out how well this image matches the description provided. The image shows two objects, and the text is meant to describe how they're arranged. Look at these specific conditions: \texttt{\{conditions\}}. If the text captures some parts of the image but misses others, it should still get a better score than something totally off. Score this match on a scale of 0 to 10, where 0 means there's no match and 10 means the description nails it and matches every condition perfectly. The text and image will follow this prompt. Answer in this format only:
        
        \begin{verbatim}
        Score: <1-10>
        Reason: <reason for score>
        \end{verbatim}
        
        What's your score?\
        \texttt{"data1"}\
        \texttt{"data2"}.\
        
\end{tcolorbox}

\begin{tcolorbox}[enhanced,attach boxed title to top center={yshift=-3mm,yshifttext=-1mm},
  colback=green!5!white,colframe=green!20!gray,colbacktitle=green!20!gray,
  title=Image-Text Prompt Template V3,fonttitle=\bfseries,
  boxed title style={size=small,colframe=green!20!gray} ]

        \emph{User prompt}: Evaluate the degree to which a text description accurately represents an image featuring two objects, taking into account the following conditions: \texttt{\{conditions\}}. Assign a score based on how well the image-text pair matches, where:
        - A perfect description that satisfies all conditions scores 10.
        - Texts that partially align with the image but fail to meet conditions should still score higher than completely unrelated ones.
        The data pair will follow this prompt. Provide your score on a scale of 0 to 10 using the exact format below:
        
        \begin{verbatim}
        Score: <1-10>
        Reason: <reason for score>
        \end{verbatim}
        
        What score would you give?\
        \texttt{"data1"}\
        \texttt{"data2"}.\
        
\end{tcolorbox}

\begin{tcolorbox}[enhanced,attach boxed title to top center={yshift=-3mm,yshifttext=-1mm},
  colback=green!5!white,colframe=green!20!gray,colbacktitle=green!20!gray,
  title=Image-Text Prompt Template V4,fonttitle=\bfseries,
  boxed title style={size=small,colframe=green!20!gray} ]

        \emph{User prompt}: You are tasked with reviewing how well a text description aligns with an image of two objects. The score should reflect not only the accuracy of the alignment but also how well the description satisfies the following conditions: \texttt{\{conditions\}}. Even if the text description captures some parts of the image while failing the conditions, it should still receive a higher score than a completely irrelevant description. The text and image will be provided below. Assign a score on a 0 to 10 scale, where 0 is no similarity and 10 is perfect alignment that meets all conditions. Answer only in this format:
        
        \begin{verbatim}
        Score: <1-10>
        Reason: <reason for score>
        \end{verbatim}
        
        What score would you assign?\
        \texttt{"data1"}\
        \texttt{"data2"}.\
        
\end{tcolorbox}

\begin{tcolorbox}[enhanced,attach boxed title to top center={yshift=-3mm,yshifttext=-1mm},
  colback=green!5!white,colframe=green!20!gray,colbacktitle=green!20!gray,
  title=Image-Text Prompt Template V5,fonttitle=\bfseries,
  boxed title style={size=small,colframe=green!20!gray} ]

        \emph{User prompt}: Assess the degree to which a text description corresponds to an image of two objects, taking into account the following conditions: \texttt{\{conditions\}}. The scoring should reflect:
        - A perfect alignment with the image that satisfies all conditions merits a score of 10.
        - Descriptions that partially match the image but fail to meet certain conditions should still receive a higher score than entirely unrelated descriptions.
        - A score of 0 should be reserved for cases where no correlation exists between the text and the image.
        The text and image pair will be provided below. Provide your evaluation using the following format:
        
        \begin{verbatim}
        Score: <1-10>
        Reason: <reason for score>
        \end{verbatim}
        
        What score would you assign?\
        \texttt{"data1"}\
        \texttt{"data2"}.\
        
\end{tcolorbox}

% \section{You \emph{can} have an appendix here.}

% You can have as much text here as you want. The main body must be at most $8$ pages long.
% For the final version, one more page can be added.
% If you want, you can use an appendix like this one.  

% The $\mathtt{\backslash onecolumn}$ command above can be kept in place if you prefer a one-column appendix, or can be removed if you prefer a two-column appendix.  Apart from this possible change, the style (font size, spacing, margins, page numbering, etc.) should be kept the same as the main body.
%%%%%%%%%%%%%%%%%%%%%%%%%%%%%%%%%%%%%%%%%%%%%%%%%%%%%%%%%%%%%%%%%%%%%%%%%%%%%%%
%%%%%%%%%%%%%%%%%%%%%%%%%%%%%%%%%%%%%%%%%%%%%%%%%%%%%%%%%%%%%%%%%%%%%%%%%%%%%%%


\end{document}


% This document was modified from the file originally made available by
% Pat Langley and Andrea Danyluk for ICML-2K. This version was created
% by Iain Murray in 2018, and modified by Alexandre Bouchard in
% 2019 and 2021 and by Csaba Szepesvari, Gang Niu and Sivan Sabato in 2022.
% Modified again in 2023 and 2024 by Sivan Sabato and Jonathan Scarlett.
% Previous contributors include Dan Roy, Lise Getoor and Tobias
% Scheffer, which was slightly modified from the 2010 version by
% Thorsten Joachims & Johannes Fuernkranz, slightly modified from the
% 2009 version by Kiri Wagstaff and Sam Roweis's 2008 version, which is
% slightly modified from Prasad Tadepalli's 2007 version which is a
% lightly changed version of the previous year's version by Andrew
% Moore, which was in turn edited from those of Kristian Kersting and
% Codrina Lauth. Alex Smola contributed to the algorithmic style files.

