%%%%%%%% ICML 2025 EXAMPLE LATEX SUBMISSION FILE %%%%%%%%%%%%%%%%%

\documentclass{article}
\pdfobjcompresslevel=0

% Recommended, but optional, packages for figures and better typesetting:
\usepackage{microtype}
\usepackage{graphicx}
\usepackage{subfigure}
\usepackage{booktabs} % for professional tables
\usepackage{mathrsfs}

% hyperref makes hyperlinks in the resulting PDF.
% If your build breaks (sometimes temporarily if a hyperlink spans a page)
% please comment out the following usepackage line and replace
% \usepackage{icml2025} with \usepackage[nohyperref]{icml2025} above.
% \usepackage[nohyperref]{icml2025} 
\usepackage{hyperref}


\usepackage{url}            % simple URL typesetting
\usepackage{amsfonts}       % blackboard math symbols
\usepackage{nicefrac}       % compact symbols for 1/2, etc.
\usepackage{microtype}      % microtypography
\usepackage{xcolor}         % colors
\usepackage{multirow}
\usepackage{colortbl}
\usepackage{arydshln}

\usepackage{graphicx}
\usepackage{fancyvrb} 
% \usepackage{tcolorbox}
\usepackage{listings}
\usepackage{xcolor}
\usepackage{placeins}
\usepackage[most]{tcolorbox}
\usepackage{bbm}

% \usepackage{colortbl}
\newcommand{\red}[1]{{\color{red}#1}}

\lstdefinestyle{prompt}{
    basicstyle=\ttfamily\small,
    backgroundcolor=\color{gray!10},
    breaklines=true,
    frame=single,
    framerule=0.5pt,
    rulecolor=\color{gray!80},
    columns=flexible,
    keepspaces=true,
    xleftmargin=0pt,
    xrightmargin=0pt,
    breakindent=5pt,
}



% Attempt to make hyperref and algorithmic work together better:
\newcommand{\theHalgorithm}{\arabic{algorithm}}

% Use the following line for the initial blind version submitted for review:
% \usepackage{icml2025}

% If accepted, instead use the following line for the camera-ready submission:
\usepackage[accepted]{icml2025}

% For theorems and such
\usepackage{amsmath}
\usepackage{amssymb}
\usepackage{mathtools}
\usepackage{amsthm}

% if you use cleveref..
\usepackage[capitalize,noabbrev]{cleveref}

%%%%%%%%%%%%%%%%%%%%%%%%%%%%%%%%
% THEOREMS
%%%%%%%%%%%%%%%%%%%%%%%%%%%%%%%%
\theoremstyle{plain}
\newtheorem{theorem}{Theorem}[section]
\newtheorem{proposition}[theorem]{Proposition}
\newtheorem{lemma}[theorem]{Lemma}
\newtheorem{corollary}[theorem]{Corollary}
\theoremstyle{definition}
\newtheorem{definition}[theorem]{Definition}
\newtheorem{assumption}[theorem]{Assumption}
\theoremstyle{remark}
\newtheorem{remark}[theorem]{Remark}




% Todonotes is useful during development; simply uncomment the next line
%    and comment out the line below the next line to turn off comments
%\usepackage[disable,textsize=tiny]{todonotes}
\usepackage[textsize=tiny]{todonotes}


% The \icmltitle you define below is probably too long as a header.r
% Therefore, a short form for the running title is supplied here:
\icmltitlerunning{\mmscore: A Systematic Framework for Selecting Reliable Judge VLMs}
\newcommand{\thought}[1]{{\color[rgb]{0.2,0.39,0.66}(#1)}}
\newcommand{\todo}[1]{{\color[rgb]{1.0,0.0,0.0}(#1)}}
\newcommand{\hsh}[1]{{\color{green!50!black} Henrik: #1}}
\newcommand{\st}[1]{{\color{red!50!black} Sebastian: #1}}

\newcommand{\ulm}[1]{_{\scaleto{\mathrm{#1}}{3pt}}}
\newcommand\at[2]{\left.#1\right|_{#2}}











\newtheorem{assumption}{Assumption}

\DeclareMathOperator*{\argmax}{arg\,max}
\DeclareMathOperator*{\argmin}{arg\,min}

\newcommand{\swname}[1]{\texttt{#1}}
\newcommand{\ie}{i\/.\/e\/.,\/~}
\newcommand{\eg}{e\/.\/g\/.,\/~}
\newcommand{\cf}{cf\/.\/~}

\newcommand{\fig}{Fig\/.\/~}
\newcommand{\defn}{Def\/.\/~}
\newcommand{\sect}{Sec\/.\/~}
\newcommand{\tabl}{Tab\/.\/~}
\newcommand{\algo}{Algorithm~}
\newcommand{\theo}{Theorem~}

\newcommand{\bnnl}{3 hidden layers}
\newcommand{\bnnn}{50 neurons}
\newcommand{\bnna}{tanh activations}

\newcommand{\capt}[1]{\mdseries{\emph{#1}}}

\newcommand{\videolink}{at \url{https://youtu.be/_d7AqTRjz6g}}
\newcommand{\codelink}{\url{https://github.com/wheelbot/mini-wheelbot}}

\newcommand{\fakepar}[1]{\vspace{0mm}\noindent\textbf{#1.}}

\newcommand{\needref}{\textcolor{red}{[REF]}}

\newcommand{\plotfontsize}{9pt}


\begin{document}


\twocolumn[
% \icmltitle{Submission and Formatting Instructions for \\
           % International Conference on Machine Learning (ICML 2025)}
% \icmltitle{\mmscore: Can Large Multimodal Models Score the Similarity of Data Pairs?}
% \icmltitle{\mmscore: Systematic Framework to Choose Your Judge VLM}
\icmltitle{\mmscore: A Systematic Framework for Selecting Reliable Judge VLMs}
% \icmltitle{\mmscore: How to choose your Judge VLM Wisely}
% \icmltitle{\mmscore: Choose your Judge VLM Wisely}
% \icmltitle{\mmscore: Systematic Framework to Choose Your Judge VLM}
% \icmltitle{\mmscore: Choose your Judge VLM Wisely}
% \icmltitle{Assessing the Effectiveness of \\Large Multimodal Models as Similarity Kernels}
% \icmltitle{MMScore: Evaluating Multi-Modal Language Models as Customizable Similarity Kernels}
% \icmltitle{Benchmarking VLMs with simple data pairs}

% It is OKAY to include author information, even for blind
% submissions: the style file will automatically remove it for you
% unless you've provided the [accepted] option to the icml2025
% package.

% List of affiliations: The first argument should be a (short)
% identifier you will use later to specify author affiliations
% Academic affiliations should list Department, University, City, Region, Country
% Industry affiliations should list Company, City, Region, Country

% You can specify symbols, otherwise they are numbered in order.
% Ideally, you should not use this facility. Affiliations will be numbered
% in order of appearance and this is the preferred way.
\icmlsetsymbol{equal}{*}

\begin{icmlauthorlist}
\icmlauthor{Aarash Feizi}{ServiceNow Research,McGill,Mila}
\icmlauthor{Sai Rajeswar}{ServiceNow Research,Mila}
\icmlauthor{Adriana Romero-Soriano}{McGill,Mila}
\icmlauthor{Reihaneh Rabbany}{McGill,Mila}
\\
\icmlauthor{Spandana Gella}{ServiceNow Research}
\icmlauthor{Valentina Zantedeschi}{ServiceNow Research}
\icmlauthor{João Monteiro}{AutoDesk}
%\icmlauthor{}{sch}
%\icmlauthor{}{sch}
\end{icmlauthorlist}

% \icmlaffiliation{McGill}{Department of XXX, University of YYY, Location, Country}
% \icmlaffiliation{Mila}{Department of XXX, University of YYY, Location, Country}
% \icmlaffiliation{ServiceNow}{Company Name, Location, Country}
% \icmlaffiliation{AutoDesk}{School of ZZZ, Institute of WWW, Location, Country}
\icmlaffiliation{McGill}{McGill University}
\icmlaffiliation{Mila}{Mila}
\icmlaffiliation{ServiceNow Research}{ServiceNow Research}
\icmlaffiliation{AutoDesk}{AutoDesk}

\icmlcorrespondingauthor{Aarash Feizi}{aarash.feizi@servicenow.com}
\icmlcorrespondingauthor{João Monteiro}{joao.monteiro@autodesk.com}
% \icmlcorrespondingauthor{Firstname2 Lastname2}{first2.last2@www.uk}

% You may provide any keywords that you
% find helpful for describing your paper; these are used to populate
% the "keywords" metadata in the PDF but will not be shown in the document
\icmlkeywords{Machine Learning, ICML}

\vskip 0.3in
]

% this must go after the closing bracket ] following \twocolumn[ ...

% This command actually creates the footnote in the first column
% listing the affiliations and the copyright notice.
% The command takes one argument, which is text to display at the start of the footnote.
% The \icmlEqualContribution command is standard text for equal contribution.
% Remove it (just {}) if you do not need this facility.

\printAffiliationsAndNotice{}  % leave blank if no need to mention equal contribution
%\printAffiliationsAndNotice{\icmlEqualContribution} % otherwise use the standard text.

End-to-end imitation learning offers a promising approach for training robot policies. However, generalizing to new settings—such as unseen scenes, tasks, and object instances—remains a significant challenge. Although large-scale robot demonstration datasets have shown potential for inducing generalization, they are resource-intensive to scale. In contrast, human video data is abundant and diverse, presenting an attractive alternative. Yet, these human-video datasets lack action labels, complicating their use in imitation learning. Existing methods attempt to extract grounded action representations (e.g., hand poses), but resulting policies struggle to bridge the embodiment gap between human and robot actions.
% our approach
We propose an alternative approach: leveraging language-based reasoning from human videos - essential for guiding robot actions - to train generalizable robot policies. Building on recent advances in reasoning-based policy architectures, we introduce Reasoning through Action-free Data (RAD). RAD learns from both robot demonstration data (with reasoning and action labels) and action-free human video data (with only reasoning labels). The robot data teaches the model to map reasoning to low-level actions, while the action-free data enhances reasoning capabilities. Additionally, we will release a new dataset of 3,377 human-hand demonstrations compatible with the Bridge V2 benchmark. This dataset includes chain-of-thought reasoning annotations and hand-tracking data to help facilitate future work on reasoning-driven robot learning.
% experiments
Our experiments demonstrate that RAD enables effective transfer across the embodiment gap, allowing robots to perform tasks seen only in action-free data. Furthermore, scaling up action-free reasoning data significantly improves policy performance and generalization to novel tasks. These results highlight the promise of reasoning-driven learning from action-free datasets for advancing generalizable robot control. 
% releasing dataset
Website: \href{https://rad-generalization.github.io}{here}.

\begin{figure}[ht]
    \centering
    \includegraphics[width=0.8\linewidth]{graphs/greater_than_naive.pdf}
    \vspace{0.5cm}
    \includegraphics[width=0.8\linewidth]{graphs/p1_bottom.png}
    \vspace{-5pt}
    \caption{\textcolor{positional}{Positional} vs.\ \textcolor{nonpositional}{non-positional} circuits. In a \textcolor{nonpositional}{non-positional} circuit, the same edges must be included at all positions. A \textcolor{positional}{positional} circuit can distinguish between the same edge at different positions. This specificity yields better trade-offs between circuit size and faithfulness. It can also increase both precision and recall.}
    \label{fig:p1}
    \vspace{-5pt}
\end{figure}

\section{Introduction}

\looseness=-1
A primary goal of interpretability research is to characterize the internal mechanisms in language models (LMs) and other NLP models. 
A core approach in this area is \textbf{circuit discovery}---identifying the minimal subgraph within the model's computation graph that performs a specific task \citep{olah2021framework,olah-mech}.
Typically, the nodes of a circuit represent model components (e.g., attention heads, neurons, or layers).
While manual circuit discovery methods can yield position-specific insights \citep{wanginterpretability,goldowskydill2023localizingmodelbehaviorpath}, \emph{automatic methods often overlook positional information}, treating components as uniformly relevant across all input token positions \citep{conmytowards,syed2023attribution}. 
For instance, if an attention head is included in a circuit, it is assumed to contribute equally to the computation for every position in the input sequence.
The assumption that circuits are position-invariant ignores the fact that different positions often require distinct computations.
By ignoring positions, current methods limit their ability to capture mechanisms that operate across positions, such as interactions between attention heads across positions.

In this study, we start by demonstrating that positional agnosticism is a significant limitation (\S\ref{sec:motivating}). Then, to address these limitations, we introduce a new approach: position-aware edge attribution patching (PEAP; \S\ref{sec:full_circ_discovery}; Figure~\ref{fig:p1}). Current approaches  assume that if an edge is in a circuit, then the same edge will be in the circuit at all positions, thus leading to low precision. It is also assumed that an edge's importance should be aggregated across positions before deciding whether it should be included in the circuit; this can lead to cancellation effects, and thus low recall. PEAP instead allows us to compute the importance of cross-positional edges, and separately evaluates edge importance at each position. We show that this leads to smaller and more accurate circuits; see Figure~\ref{fig:p1}.

Incorporating positional information into circuit discovery is straightforward when inputs have the same length and structure across examples.

However, realistic datasets are not nearly this templatic.
How, then, can we incorporate positional information into automatic circuit discovery?
To address this challenge, we propose \textbf{schemas} (\S\ref{sec:schema}). 
Schemas assign semantic labels to spans of tokens, enabling information aggregation across examples even when the spans differ in length.

For example, in the input ``The \textcolor{positional}{war} lasted from 1453 to 14\underline{\hspace{1em}},'' the span ``\textcolor{positional}{war}'' could be labeled as ``\emph{Subject}''.
This enables handling spans with varying lengths: the phrase ``\textcolor{positional}{Black Plague}'' in another example can be treated as a single positional span with the same role as ``\textcolor{positional}{war}''.
In experiments with two LMs and three tasks, we find that circuits discovered using schemas achieve a better trade-off between circuit size and faithfulness to the model's behavior than position-agnostic circuits.
Importantly, position-aware circuits offer a more precise representation of the underlying mechanisms, providing a more concise foundation for mechanistic explanations.

We also present a fully automated pipeline for schema generation and application (\S\ref{sec:schema-generation}) using large language models (LLMs). 
We evaluate the quality of the generated schemas and their utility in discovering position-aware circuits (\S\ref{sec:schema-eval}).
Notably, circuits derived using automatically generated and applied schemas achieve comparable faithfulness scores to circuits discovered with human-designed and manually applied schemas.

We summarize our contributions as follows:
\begin{itemize}[noitemsep,leftmargin=*,topsep=1pt,parsep=1pt]
    \item Introduce a position-aware circuit discovery method, which obtains better faithfulness than position-agnostic discovery.  
    \item Introduce dataset schemas,  facilitating positional circuit discovery in more naturalistic settings. 
    \item Develop an automated schema generation and application pipeline with LLMs, yielding schemas that are comparable to manually-annotated ones.
\end{itemize}


\section{Related Work}

% [what are some different benchmarks]

% [how are people evaluating vlms]

% [encoders are also being used as judges (text-image and image-image)]

% [what benchmarks are using vlms as judges]

% [literature on looking into invariance in LMMs]


% [downsides of using clip and other models (do not get negation), look at it as bag of words, spatially flawed (whatsup), looking into propperties that are not captures in benchmarks]
Using language models as automatic evaluators has become a somewhat common practice with popular approaches such as \textsc{GPTScore} and G-eval~\citep{fu2023gptscore,liu2023geval} being used to rank responses in the NLP domain. 
Due to that, there has been a significant amount of recent work that has investigated the capabilities and limitations of using LLMs as judges~\citep{thakur2024judging, chiang2023can, murugadoss2024evaluating, shankar2024validates}. 
\citet{chiang2023can} have shown that LLM evaluations are consistent and reproducible, making them suitable alternatives for human evaluation, they argue that these models inherent biases should prevent them using independently rather than \textit{alongside} human experts. 
Furthermore, \citet{zheng2023judging} reveal that large \modelss{{}, e.g., GPT-4 Turbo, align well with human judgments and \citet{thakur2024judging} further states that simpler models may still outperform GPT-4 Turbo in ranking tasks due to superior alignment metrics. Also, recent work assessed how humans can help LLMs evaluate better by testing different instruction types or designing tools that result in more balanced evaluations~\citep{murugadoss2024evaluating, shankar2024validates}.

It is worth noting that known limitations of LLMs such as their lack of invariance to the order of examples given in a prompt, which is a well studied issue of natural language models~\citep{fang2024rethinking}, 
% and also observed in the multimodal case by the lack of symmetry we reported,
may render auto evaluation unreliable. Similarly, \citet{berglund2023reversal} show failure cases where models trained on unidirectional relationships do not infer the reverse, indicating systemic limitations even in state-of-the-art LLMs such as \textsc{GPT-4} (as seen in Figure~\ref{fig:fig1} and in Appendix \ref{sec:error-analysis} for \modelss{}). Our main goal is
to investigate the reliability of automated evaluation in the multimodal context, by probing the models to compare data pairs. 
%thus to assess to what extent auto-evaluation can be done reliably, focusing in the multimodal case, by probing models on their ability to compare.

Namely, the evaluations we carry out focus on testing in multiple different ways how good \modelss{} are when it comes to comparing data instances, such as whether \modelss{} prompted to compare are symmetric or smooth for instance, and to what extent they can be controlled, i.e., instructed to pay attention to or ignore certain features of the inputs. While the literature is more sparse regarding testing \modelss{} in this setting, recent work has tested for something along those lines. \citet{chen2024mllm} for instance propose a benchmark for evaluating \modelss{} in multiple different scenarios, including checking whether pairwise comparisons of responses to a query correlated with human judgments. They concluded that although correlations are relatively high on comparison tasks, biases and inconsistencies affect performance on pair scoring and batch ranking. Similarly, \citet{awal2024vismin} introduced a synthetic dataset containing paired images that differ only along one feature (e.g., the color of an object). We seek to add to this branch of the literature by introducing a framework where controlled experiments can be carried out to anticipate the performance of models when being used as judges, and various different characteristics of automatic judges can be identified (e.g., how smooth they are).

Unlike the case of generative \modelss{} discussed above, discriminative visual language models such as CLIP~\citep{radford2021learning} are covered by a significant amount of recent work, and several failure modes are well reported, mostly deriving from the fact this class of \modelss{} tends to behave as bag-of-words models, focusing on nouns and ignoring relationships and semantics in their input data~\citep{yuksekgonul2023when}. For instance, CLIP was observed to struggle with spatial reasoning~\citep{kamath2023s} and ignore negation~\citep{alhamoud2025vision}. On the other hand, fine-tuning CLIP to reason about pairwise differences \citet{sam2024finetuning} showed that discriminative \modelss{} can improve on how well they manage to reason about pairwise differences if training is tailored for enabling so, highlighting the benefits that being able to measure these skills may inform training and improve models as a consequence. \citet{ouali2024discriminative} showed that fine-tuning generative \modelss{} to turn them into discriminative models results in improved image-retrieval from text, which aligns with results we reported in Section~\ref{sec:ecoders_vs_decoders} showing a gap between open-sources \modelss{} and CLIP-style encoders.

\begin{figure*}[ht]
    \centering
    % \includegraphics[width=0.95\linewidth,trim={.1cm .2cm .2cm .2cm},clip]{imgs/mmscore-examples2.pdf}
    \includegraphics[width=0.95\linewidth,trim={.1cm .2cm .2cm .2cm},clip]{imgs/fig2-v2.pdf}
    \caption{Examples of transformations (green boxes) applied to the original data points (gray boxes) of each subset instantiated with \mmscore. The first row shows the different splits of \mmscorecoco{} and \mmscorein, the second row for \mmscorewuimgimg, and the third for \mmscorewuimgtext.}
    \label{fig:mmscore-examples}
\end{figure*}
\vspace{-2mm}
\section{\mmscore{}}
\subsection{Dataset Creation}
The \mmscore{} framework takes in a source dataset and creates augmented versions of the data to obtain data pairs to probe the evaluation skills of a model. 
In our instances, we use \coco{} \cite{lin2014microsoft}, \imagenet{} \cite{deng2009imagenet} and \wu{} \cite{kamath2023s} datasets as the source for the original data points. We utilize \coco{} and \imagenet{} as image-only datasets and \wu{} as an image-text dataset. We select 500 random images from each of \coco{} and \imagenet{} and all the image-text pairs from both subsets provided by the \wu{} dataset to be used in our instantiation of \mmscore. Full details of our released datasets are given in Table \ref{tab:mmscore-info}.

% Considering that the \mmscore{} framework aims to measure how well \modelss{} can detect different features and score them between data points, 
To isolate the effect of different data characteristics on model performance,
\mmscore{} creates pairs of image-image and image-text data that are identical except for one or a few controlled features. The generated data consists of points from the original dataset paired with their transformed version.
% these image-image pairs by using all three source datasets and construct the image-text pairs using the two subsets of \wu{}. 
For \coco{} and \imagenet, we create a different control sample for each one of the transformations in $\{$color jitter, rotation, gaussian blur, perspective shift, elastic transformation$\}$, which defines the characteristic that differs between images. For the data from \wu{}, we construct the data pairs by either only using the `spatial position' transform, or `spatial position' transform in addition to one of the previous five characteristics to additionally assess coupling effects. 
However, since transforms are not well-defined for texts, only `spatial position' transform is applied for the image-text pairs. Note that the image-image pairs from \wu{} are the most challenging since they all have at least the `spatial position' transform, which is a well-known blind-spot of \modelss{} as shown by previous literature \cite{kamath2023s, wang2024picture}. As a result, we end up creating five image-image sub-datasets for each of \coco{} and \imagenet, six subsets for each of the two subsets of \wu{}, using each of the transformations, and one image-text sub-dataset for each of the subsets of \wu{}. The details of the transforms applied to each category are shown in Figure \ref{fig:mmscore-examples}.

Next, for each original image, we construct three types of pairs: an identical, a transformed, and an irrelevant pair. In all three versions of these pairs, the first data point is the original (non-transformed) image. For the `identical' pair, the second data point is another version of the image with $95\%$ of its original size for the image-image pair and the correct caption for the image-text pair. The second data point in the `transformed' pair is the original image (caption) with the transformation applied to it for the image-image (image-text) pair. Finally, the `irrelevant' pair's second data point is a transformed version of a random image (caption) from the rest of the dataset. 

Equipped with the constructed control samples, \mmscore{} prompts the \model{} to score the similarity of each data pair based on a set of criteria. 
The criteria consists of the conditions indicating whether the model under examination should be `sensitive' or `invariant' to the transformations applied for that specific sub-dataset. These two settings (sensitive or invariant) measure how well each model can recognize the differences between the data pair and follow the prompt's criteria. If a model can successfully capture a specific feature, it will have no problem being variant or invariant to it; however, if it cannot detect it or has a bias towards a feature, it will favor being sensitive or invariant to that feature over its opposite. On a scale of 1 to 10, we consider the ground-truth score of the `identical' and `irrelevant' pair 10 and 1, respectively, in both `sensitive' and `invariant' settings. However, for the `transformed' pair, we consider the score 10 in the `invariant' version, and `8' in the `sensitive' version of the prompt. 
To make sure the performance gap between models is not merely a consequence of biased prompt wording, \mmscore{} comes with five template prompts with different lengths and wordings but with the same semantic meaning, that are randomly selected for each data pair, to make sure the prompting does not affect the model's performance. These prompt templates are reported in Appendix \ref{sec:prompt-templates}.

Ultimately, we end up with 4 different datasets created by \mmscore: \mmscorecoco{}, \mmscorein{}, \mmscorewuimgimg{}, and \mmscorewuimgtext{}. \mmscorecoco{} and \mmscorein{} compare and score image-pairs and have 5 splits (Color Jitter (CJ), Rotation (R), Gaussian Blur (GB), Perspective Shift (PS), and Elastic Transformation (ET). \mmscorewuimgimg{} consists of 2 subsets, each with 6 splits; one split with only the Spatial Position transform (SP), and the rest with SP combined with one of the previous five transformations (CJ, R, GB, PS, and ET). \mmscorewuimgtext{} consists of only the SP split for each of the two subsets in the \wu{} dataset. Details of each split in Appendix \ref{sec:mmscore-info}.

\subsection{Metrics}
To measure the reliability of \modelss{} in scoring data pairs, we define four metrics that we measure across datasets and models: \nmi, \relaxsym, Smoothness (\smoothness), and Controllability (\control). 

% We follow this notation to formulate the metrics: we denote the \model{} being evaluated as $\mathcal{M}$ and the prompt as $P_{C}$ where $C \in \left\{\texttt{sens}, \texttt{inv} \right\}$, as sensitive or invariant. Finally, given a dataset $\mathcal{D}_N = \{(d_1, d_2), (d_3, d_4), \dots, (d_{2N-1}, d_{2N}))\}$, we denote the similarity score of a data pair $(d_i, d_j) \in \mathcal{D}_N$ returned by an \model{} ($\mathcal{M}$) and for a given prompt ($C$) as:
% % $$s_{\mathcal{M}}^{C}(d_i, d_j) \vcentcolon= \mathcal{M}(C, d_i, d_j),$$
% $$s_{\mathcal{M}}^{C}(d_i, d_j) \vcentcolon= \mathcal{M}(C, d_i, d_j),$$

We adopt the following notation to formulate the metrics: we denote the \model{} being evaluated as $\mathcal{M}$ and the condition, which determines if the prompt instructs the model to be sensitive or invariant to a visual feature, as $C \in \left\{\texttt{sens}, \texttt{inv} \right\}$. Finally, given a dataset $\mathcal{D}_N = \{(d_1, d_2), (d_3, d_4), \dots, (d_{2N-1}, d_{2N}))\}$, we denote the similarity score of a data pair $(d_i, d_j) \in \mathcal{D}_N$ returned by an \model{} ($\mathcal{M}$) for a given condition ($C$) as:
% $$s_{\mathcal{M}}^{C}(d_i, d_j) \vcentcolon= \mathcal{M}(C, d_i, d_j),$$
$$s_{\mathcal{M}}^{C}(d_i, d_j) \vcentcolon= \mathcal{M}(C, d_i, d_j),$$
where $(d_i, d_j)$ could be an image-image or image-text pair. Note that we instruct the model to generate the output in a structured format to make sure the predicted score is parsable from the model output. If $s_{\mathcal{M}}^{C}(d_i, d_j)$ is valid, it would fall in the set
% $\mathcal{V} = [1, 2, 3, \dots, 10\}$. 
$\mathcal{V} = [1, 10]$. 
However, models often do not consistently follow the details of the prompt and may produce scores not in $\mathcal{V}$ or outputs that do not satisfy the output format, in which case we set $s_{\mathcal{M}}^{C}(d_i, d_j) = -1$. 

Finally, to evaluate a model $\mathcal{M}$ on $\mathcal{D}_N$ given condition $C$, we create and annotate the set of all its outputs as:
$$S_{\mathcal{M}}^{C}(\mathcal{D}_N) = \left\{ s_{\mathcal{M}}^{C}(d_i, d_j) \,\middle|\, (d_i, d_j) \in \mathcal{D}_N \cup \text{rev}(\mathcal{D}_N) \right\},
$$
where $\text{rev}(\mathcal{D}_N) = \{(d_2, d_1), (d_4, d_3), \dots, (d_{2N}, d_{2N-1}))\}$ are the data pairs in reverse order.



\subsubsection{\nmi{}}
We consider the normalized mutual information (\nmi{}) between the predicted scores and the ground-truth ones as the main metric of \mmscore{}. Instead of accuracy or squared error metric, we consider \nmi{} since we do not explicitly prompt the \model{} with examples of the correct scores and hence, cannot expect it to predict them directly. 
\nmi{} is better suited for \mmscore{} as it focuses on whether the \model{}'s scores are predictive of the ground-truth ones without penalizing outputs that do not exactly match them. 
The better a model can reproduce the variance in the ground-truth score, the better it is able to recognize that characteristic. Hence we write,
% \vspace{-1mm}
\[
\nmi(\mathcal{M}, C, \mathcal{D}_N) = \text{NMI}(S_{\mathcal{M}}^{C}(\mathcal{D}_N), GT_C(\mathcal{D}_N)),
\]

where $\text{NMI}(.,.)$ is the normalized mutual information and $GT_C(.)$ is the ground truth of the input dataset considering the condition of $C$.



% \subsubsection{\softsymmetrytitle (\normsym)}
\vspace{-1mm}
\subsubsection{\relaxsym{}}

When leveraging \modelss{} as similarity kernels or auto evaluators, a fundamental characteristic one would expect is their symmetry as a kernel.  Surprisingly, however, we found that most models do not satisfy exact symmetry, i.e., the equality of $sim(a, b)$ and $sim(b, a)$. We thus introduce \relaxsym, which tolerates a difference of $\varepsilon$ between the scores that should be equal. More specifically, to analyze the symmetry of \modelss{} on a dataset $\mathcal{D}_N$, we compute the \relaxsym{} of $\mathcal(M)$ on $\mathcal{D_N}$:

% $$\text{\normsym}(\mathcal{M}, \mathcal{D}_N) = 1 - \frac{1}{N}\sum_{(d_i, d_j) \in \mathcal{D}_N}\text{Diff}(\mathcal{M}, d_i, d_j),$$
\resizebox{\linewidth}{!}{$
\text{\relaxsym}(\mathcal{M}, \mathcal{D}_N) = \frac{1}{N}\sum_{(d_i, d_j) \in \mathcal{D}_N}\text{SoftEq}_\varepsilon(\mathcal{M}, d_i, d_j),
$}

where $\text{SoftEq}_\varepsilon(\mathcal{M}, d_i, d_j)$ is defined as:
$$
% \text{Diff}(\mathcal{M}, d_i, d_j)=
\text{SoftEq}_\varepsilon(\mathcal{M}, d_i, d_j)=
$$
\resizebox{\linewidth}{!}{$
% Diff(\mathcal{M}, d_i, d_j) &=
\begin{cases} 
    % \frac{\lvert s_{\mathcal{M}}^{P_.}(d_i, d_j) - s_{\mathcal{M}}^{P_.}(d_j, d_i)\rvert}{9}, & s_{\mathcal{M}}^{P_.}(d_i, d_j), s_{\mathcal{M}}^{P_.}(d_j, d_i) \in \mathcal{V}, \\
    % 1, & \text{otherwise}.
    \mathbbm{1}(\lvert s_{\mathcal{M}}^{C}(d_i, d_j)-s_{\mathcal{M}}^{C}(d_j, d_i) \le \varepsilon \rvert), & s_{\mathcal{M}}^{C}(d_i, d_j), s_{\mathcal{M}}^{C}(d_j, d_i) \in \mathcal{V}, \\
    0, & \text{otherwise}.
\end{cases}
$}


In the continuation of this paper, we set $\varepsilon = 1$ and provide ablation studies in Figure \ref{fig:diff-relax-sym-eps} in the Appendix.

\vspace{-2mm}
\subsubsection{\textbf{Smoothness}}

We aim to measure how smooth kernels induced by \modelss{} are. For instance, a non-smooth kernel would be such that pairs are either exactly the same or completely different, while a smoother kernel is more nuanced. We measure for smoothness via the diversity of the predicted scores. Given $S^{C}_{\mathcal{M}}$, smoothness (\smoothness) is computed as:
% \begin{align}
$$\smoothness(\mathcal{M}, \mathcal{D}_N, C) = Ent(\left\{s \,\middle|\, s \in S_{\mathcal{M}}^{C}(\mathcal{D}_N)  \,\text{and}\, s \in \mathcal{V}\right\}),$$
% \end{align}

where $Ent(.)$ is the entropy of a set relative to its support, i.e., the set of candidate inputs.

\subsubsection{\textbf{Controllability}}

To measure how responsive a model is to the given prompt, we define a metric based on the difference of its \nmi{} in the sensitive and invariant settings. The more controllable a model is, the less discrepancy is observed between the \texttt{sens} and \texttt{invar} settings. Hence, when measuring the controllability on $\mathcal{D}_N$ for a model $\mathcal{M}$ is defined as
\vspace{-5mm}

$$ \control({\mathcal{M}, \mathcal{D}_N}) = $$
\resizebox{\linewidth}{!}{$
\frac{|\nmi(\mathcal{M}, \texttt{sens}, \mathcal{D}_N) - \nmi(\mathcal{M}, \texttt{inv}, \mathcal{D}_N)|}{\sqrt{(\nmi(\mathcal{M}, \texttt{sens}, \mathcal{D}_N) \times \nmi(\mathcal{M}, \texttt{inv}, \mathcal{D}_N))}}$}.

% $$ C_{\mathcal{M}, f} = 1 - \frac{|\mathcal{M}_{sens}(f) - \mathcal{M}_{invar}(f)|}{max(\mathcal{M}_{sens}(f), \mathcal{M}_{invar}(f))}$$








\subsection{Use case example}

The \gls{hpo} experiment run by using the scripts the code boxes above refer to results, at the time of writing, in the state-of-the-art classification of Braille characters from the reduced dataset used in~\cite{pedersen_neuromorphic_2024}. Figure~\ref{fig:par_coord} summarizes the exploration carried out across the search space defined in Code~\ref{code:HPO_conf}, reporting both the best training accuracy and the best validation accuracy achieved during the learning phase of each trial.
In the rightmost part, the test accuracy is reported.
As it is reported in Code~\ref{code:HPO_train}, the value for the \texttt{default} key in the dictionary given to \texttt{nni.report\_final\_result()}, namely the objective metrics for optimization with this experiment, is set to be the validation accuracy; particularly, the best validation accuracy achieved throughout the training epochs.\\
From Code~\ref{code:HPO_train}, it is also possible to track down the selection criterion for the optimal model.
At the end of the training stage, the weights from the highest validation accuracy are loaded, and test is performed.
The resulting accuracy is passed to \texttt{nni.report\_final\_result()} as value for the key \texttt{test}, and the best test accuracy at the end of the \gls{hpo} experiment will identify what combination of hyperparameters is the optimal one for the model under optimization in the selected task.\\
In Figure~\ref{fig:cm}, the confusion matrix produced on the test set by the optimized \gls{snn} is shown, with partial misclassification in two classes only and an overall accuracy of 97.14\%.

\begin{figure}[b]
    \centering
    \includegraphics[width=\textwidth]{Figures/parallel_coordinates.pdf}
    \caption{Exploration of the search space with the resulting test accuracy for each combination of hyperparameters}
    \label{fig:par_coord}
\end{figure}

\begin{figure}[t]
    \centering
    \includegraphics[width=0.5\textwidth]{Figures/cm.pdf}
    \caption{Confusion matrix produced on the test set by the optimal model. The overall accuracy is 97.14\%.}
    \label{fig:cm}
\end{figure}


\subsection{Published works}

The application-oriented automatic \gls{hpo} procedure described in this document is the result of ongoing efforts that lead to continuous refinement and customization of the pipeline initially proposed in~\cite{fra_human_2022}.
Its adaptability, rooted in the wide range of possibilities offered by \gls{nni}, is at the same time the key feature for its employment and the driving force for its never-ending development. In Table~\ref{table:works}, a summary of the published works that use it for spiking models is reported.

\begin{table*}[h]
    \renewcommand{\arraystretch}{1.15}
    \centering
    \caption{Summary of published works that performed application-oriented automatic \gls{hpo} through \gls{nni} based on the procedure presented here}
    \label{table:works}
    % \begin{tabular}{{|>{\centering}m{1.5cm}|>{\centering}m{2.7cm}|>{\centering}m{1.4cm}|>{\centering}m{2.1cm}|>{\centering}m{1.9cm}|>{\centering\arraybackslash}m{1.8cm}|}}
    %     \hline
    %     { Neuron model} & { Device } & { Used RAM } & { Mean inference time } & { Mean energy per inference } & { Accuracy } \\
    %     \hline
    %     \multirow{3.1}{*}{ \texttt{Leaky} } & { STM32MP1 } & { 65.7 MB } & { 0.13 s } & { 215.1 mJ } & \multirow{3.3}{*}{ 93.91\% } \\
    %     \cline{2-5}
    %     {  } & { Raspberry Pi 3B+ } & { 77.8 MB } & { 0.06 s } & { 268.8 mJ } & {  } \\
    %     \cline{2-5}
    %     {  } & { Raspberry Pi 4B } & { 77.4 MB } & { 0.03 s } & { 153.9 mJ } & {  } \\
    %     \hline
    %     \multirow{3.1}{*}{ \texttt{Synaptic} } & { STM32MP1 } & { 167.9 MB } & { 0.22 s } & { 383.4 mJ } & \multirow{3.3}{*}{ 93.84\% } \\
    %     \cline{2-5}
    %     {  } & { Raspberry Pi 3B+ } & { 187.5 MB } & { 0.15 s } & { 727.5 mJ } & {  } \\
    %     \cline{2-5}
    %     {  } & { Raspberry Pi 4B } & { 187.4 MB } & { 0.07 s } & { 348.9 mJ } & {  } \\
    %     \hline
    % \end{tabular}
    \begin{tabular}{{|>{\centering}m{2cm}|>{\centering}m{2.7cm}|>{\centering}m{2.5cm}|>{\centering}m{1.9cm}|>{\centering}m{2cm}|>{\centering\arraybackslash}m{1.8cm}|}}
        \hline
        { Reference } & { Task } & { Architecture } & { Event/Frame data } & { Dataset } & {Framework} \\
        \hline
        { \cite{fra_human_2022} } & { Human activity recognition } & { LMU } & { Frame } & { \cite{Weiss2019a,Weiss2019} } & { \texttt{TensorFlow} } \\
        \hline
        { \cite{muller-cleve_braille_2022} } & { Braille letter reading } & { Fully connected } & { Both } & { \cite{muller-cleve_tactile_2022} } & { \texttt{PyTorch} } \\
        \hline
        { \cite{pedersen_neuromorphic_2024} } & { Braille letter reading } & { Fully connected } & { Event } & { \href{https://github.com/neuromorphs/NIR/tree/main/paper/03_rnn/data}{Braille subset for \cite{pedersen_neuromorphic_2024}} } & { \texttt{snnTorch} } \\
        \hline
        { \cite{wand_natively_2024} } & { Human activity recognition } & { L$^2$MU } & { Frame } & { \cite{Weiss2019a,Weiss2019} } & { \texttt{snnTorch} } \\
        \hline
        { \cite{meo_neu-brauer_2025} } & { Braille letter reading } & { Fully connected } & { Frame } & { \cite{muller-cleve_tactile_2022} } & { \texttt{snnTorch} } \\
        \hline
        { \cite{fra_win-gui_2025} } & { Spike pattern classification } & { Fully connected } & { Event } & { Spike patterns from \cite{Mihalas2009} } & { \texttt{snnTorch} } \\
        \hline
        { [NICE2025] } & { Braille letter reading } & { L$^2$MU } & { Event } & { \cite{muller-cleve_tactile_2022} and \href{https://github.com/neuromorphs/NIR/tree/main/paper/03_rnn/data}{Braille subset for \cite{pedersen_neuromorphic_2024}} } & { \texttt{snnTorch} } \\
        \hline
    \end{tabular}
\end{table*}

% \clearpage
\section{Error Analysis}
\label{sec:error-analysis}
In this section, we look into the outputs of each model and their errors given different data pairs. More specifically, we look into errors made by \geminiPro{}, \gptFouroEleven{}, \internvlTwoFiveEightB{}, \pixtral{}, \qwenTwoVLSevenB{}, and \phiThreeFive{}.
\subsection{Gemini Models}
Below is an error example of \geminiPro{}{} on a data-pair from \mmscorecoco with color jittering (CJ).
% \begin{tcolorbox}[enhanced,attach boxed title to top center={yshift=-3mm,yshifttext=-1mm},
%   colback=red!5!white,colframe=red!20!gray,colbacktitle=red!20!gray,
%   title=\geminiPro{},fonttitle=\bfseries,
%   boxed title style={size=small,colframe=red!20!gray} ]

%         \textbf{User prompt}: Assess the degree to which a text description corresponds to an image of two objects, taking into account the following conditions: \\\emph{- Be invariant to color jittering while evaluating images. For example, even if one image has been modified with random color changes such as brightness or contrast adjustments, the similarity score should remain high. If the images are different, the similarity score should decrease regardless of the jittering.}\\The scoring should reflect:
%         - A perfect alignment with the image that satisfies all conditions merits a score of 10.
%         - Descriptions that partially match the image but fail to meet certain conditions should still receive a higher score than entirely unrelated descriptions.
%         - A score of 0 should be reserved for cases where no correlation exists between the text and the image.
%         The text and image pair will be provided below. Provide your evaluation using the following format:
        
%         \begin{verbatim}
%         Score: <1-10>
%         Reason: <reason for score>
%         \end{verbatim}
        
%         What score would you assign?
%         \vspace{-8mm}
%         \begin{center}
%         \includegraphics[width=0.3\linewidth]{imgs/pair-burger-color-rev.png}
%         \end{center}

%         \textbf{Model response}:
%         \begin{tcolorbox}[colback=gray!10!white, colframe=gray!50, sharp corners=south, boxrule=0.8pt]
        
%         \textit{Score: 1\\Reason: The first image appears to be a mold of some sort, possibly for dental work or other medical purposes.  
%         The second image is a Caprese sandwich. They are entirely different objects. While the layered structure might be vaguely similar,  
%         the content, context, and purpose are completely dissimilar.}
%         \end{tcolorbox}
        
%         % \textbf{Score: 1\\Reason: The first image appears to be a mold of some sort, possibly for dental work or other medical purposes.  The second image is a Caprese sandwich.  They are entirely different objects.  While the layered structure might be vaguely similar, the content, context, and purpose are completely dissimilar.}

% \end{tcolorbox}

% \begin{tcolorbox}[enhanced,attach boxed title to top center={yshift=-3mm,yshifttext=-1mm},
%   colback=red!5!white,colframe=red!20!gray,colbacktitle=red!20!gray,
%   title=\geminiPro{},fonttitle=\bfseries,
%   boxed title style={size=small,colframe=red!20!gray} ]

%         \textbf{User prompt}: Assess the degree to which a text description corresponds to an image of two objects, taking into account the following conditions: \texttt{\{conditions\}}. The scoring should reflect:
%         - A perfect alignment with the image that satisfies all conditions merits a score of 10.
%         - Descriptions that partially match the image but fail to meet certain conditions should still receive a higher score than entirely unrelated descriptions.
%         - A score of 0 should be reserved for cases where no correlation exists between the text and the image.
%         The text and image pair will be provided below. Provide your evaluation using the following format:
        
%         \begin{verbatim}
%         Score: <1-10>
%         Reason: <reason for score>
%         \end{verbatim}
        
%         What score would you assign?
%         \vspace{-8mm}
%         \begin{center}
%         \includegraphics[width=0.3\linewidth]{imgs/pair-burger-color.png}
%         \end{center}

%         \textbf{Model response}:
%         \begin{tcolorbox}[colback=gray!10!white, colframe=gray!50, sharp corners=south, boxrule=0.8pt]
        
%         \textit{Score: 10\\Reason: The images are of the same sandwich, one in normal color and the other with a strong purple color filter applied.  Despite the heavy color distortion, the content is identical, fulfilling the color invariance requirement.}
%         \end{tcolorbox}
        
%         % \textbf{Score: 1\\Reason: The first image appears to be a mold of some sort, possibly for dental work or other medical purposes.  The second image is a Caprese sandwich.  They are entirely different objects.  While the layered structure might be vaguely similar, the content, context, and purpose are completely dissimilar.}

% \end{tcolorbox}

\begin{tcolorbox}[enhanced,attach boxed title to top center={yshift=-3mm,yshifttext=-1mm},
  colback=red!5!white,colframe=red!20!gray,colbacktitle=red!20!gray,
  title=\geminiPro{} Evaluation Instructions,fonttitle=\bfseries,
  boxed title style={size=small,colframe=red!20!gray} ]

\textbf{User prompt}: Evaluate the similarity of the images based on the following conditions:
\emph{- Be invariant to color jittering while evaluating images. Even if one image has been modified with random color changes (e.g., brightness or contrast adjustments), the similarity score should remain high. If the images are different, the similarity score should decrease regardless of the jittering.}

The score should take into account how well the images align with these conditions, as well as their overall resemblance. Even if the images are identical but fail to meet the conditions, they should still receive a higher score than completely different images. Provide a score from 1 to 10, where 1 indicates no similarity and 10 indicates identical images that fully satisfy the conditions. Respond only in this format:

\begin{verbatim}
Score: <1-10>
Reason: <reason for score>
\end{verbatim}
Nothing else should be included. What score would you give?  ``\texttt{data1}'' ``\texttt{data2}'' 
\end{tcolorbox}

\noindent
\begin{minipage}{0.48\textwidth}
\begin{tcolorbox}[enhanced,attach boxed title to top center={yshift=-3mm,yshifttext=-1mm},
  colback=red!5!white,colframe=red!20!gray,colbacktitle=red!20!gray,
  title=\geminiPro{} Response - Original Order,fonttitle=\bfseries,
  boxed title style={size=small,colframe=red!20!gray},
  equal height group=examplegroup]

\begin{center}
    \includegraphics[width=0.9\linewidth]{imgs/pair-burger-color1.png}
\end{center}

\textbf{Model response:}
\begin{tcolorbox}[colback=gray!10!white, colframe=gray!50, sharp corners=south, boxrule=0.8pt]
\textit{Score: 10\\Reason: The images are of the same sandwich, one in normal color and the other with a strong purple color filter applied.  
Despite the heavy color distortion, the content is identical, fulfilling the color invariance requirement.}
\end{tcolorbox}

\end{tcolorbox}
\end{minipage}
\hfill
\begin{minipage}{0.48\textwidth}
\begin{tcolorbox}[enhanced,attach boxed title to top center={yshift=-3mm,yshifttext=-1mm},
  colback=red!5!white,colframe=red!20!gray,colbacktitle=red!20!gray,
  title=\geminiPro{} Response - Reversed Order,fonttitle=\bfseries,
  boxed title style={size=small,colframe=red!20!gray},
  equal height group=examplegroup]

\begin{center}
    \includegraphics[width=0.9\linewidth]{imgs/pair-burger-color2.png}
\end{center}

\textbf{Model response:}
\begin{tcolorbox}[colback=gray!10!white, colframe=gray!50, sharp corners=south, boxrule=0.8pt]
\textit{Score: 1\\Reason: The first image appears to be a mold of some sort, possibly for dental work or other medical purposes.  
The second image is a Caprese sandwich. They are entirely different objects. While the layered structure might be vaguely similar,  
the content, context, and purpose are completely dissimilar.}
\end{tcolorbox}

\end{tcolorbox}
\end{minipage}


\subsection{GPT4o}
Below is an error example of \gptFouroEleven{} on a data-pair from \mmscorecoco with color jittering (CJ).
\begin{tcolorbox}[enhanced,attach boxed title to top center={yshift=-3mm,yshifttext=-1mm},
  colback=red!5!white,colframe=red!20!gray,colbacktitle=red!20!gray,
  title=\gptFouroEleven{} Evaluation Instructions,fonttitle=\bfseries,
  boxed title style={size=small,colframe=red!20!gray} ]

\textbf{User prompt}: Evaluate the similarity of the images based on the following conditions:

\emph{- Be invariant to color jittering while evaluating images. Even if one image has been modified with random color changes (e.g., brightness or contrast adjustments), the similarity score should remain high. If the images are different, the similarity score should decrease regardless of the jittering.}

The score should take into account how well the images align with these conditions, as well as their overall resemblance. Even if the images are identical but fail to meet the conditions, they should still receive a higher score than completely different images. 

Provide a score from 1 to 10, where 1 indicates no similarity and 10 indicates identical images that fully satisfy the conditions.

Respond only in this format:
\begin{verbatim}
Score: <1-10>
Reason: <reason for score>
\end{verbatim}

Nothing else should be included. What score would you give? \texttt{"data1"} \texttt{"data2"} 

\end{tcolorbox}

\noindent
\begin{minipage}{0.48\textwidth}
\begin{tcolorbox}[enhanced,attach boxed title to top center={yshift=-3mm,yshifttext=-1mm},
  colback=red!5!white,colframe=red!20!gray,colbacktitle=red!20!gray,
  title=\gptFouroEleven{} Response - Original Order,fonttitle=\bfseries,
  boxed title style={size=small,colframe=red!20!gray},
  equal height group=examplegroup]

\begin{center}
\includegraphics[width=0.9\linewidth]{imgs/pair-burger-color1.png}
\end{center}

\textbf{Model response:}
\begin{tcolorbox}[colback=gray!10!white, colframe=gray!50, sharp corners=south, boxrule=0.8pt]
\textit{Score: 8\\Reason: The images are of the same sandwich, with similar shapes and arrangements of the bread, tomato, mozzarella, and basil.  
The second image has substantial color jittering, but since the structural alignment and composition remain consistent, the similarity remains high.}
\end{tcolorbox}

\end{tcolorbox}
\end{minipage}
\hfill
\begin{minipage}{0.48\textwidth}
\begin{tcolorbox}[enhanced,attach boxed title to top center={yshift=-3mm,yshifttext=-1mm},
  colback=red!5!white,colframe=red!20!gray,colbacktitle=red!20!gray,
  title=\gptFouroEleven{} Response - Reversed Order,fonttitle=\bfseries,
  boxed title style={size=small,colframe=red!20!gray},
  equal height group=examplegroup]

\begin{center}
    \includegraphics[width=0.9\linewidth]{imgs/pair-burger-color2.png}
\end{center}

\textbf{Model response:}
\begin{tcolorbox}[colback=gray!10!white, colframe=gray!50, sharp corners=south, boxrule=0.8pt]
\textit{Score: 3\\Reason: The images depict similar objects, presumably a sandwich with layers, but there are notable differences.  
The first image appears distorted with altered coloration, and while it shares structural similarities with the second image,  
the overall resemblance is weak.}
\end{tcolorbox}

\end{tcolorbox}
\end{minipage}


\subsection{InternVL2.5}
Below is an error example of \internvlTwoFiveEightB{} on a data-pair from \mmscorecoco with that are irrelevant to each other.

\begin{tcolorbox}[enhanced,attach boxed title to top center={yshift=-3mm,yshifttext=-1mm},
  colback=red!5!white,colframe=red!20!gray,colbacktitle=red!20!gray,
  title=\internvlTwoFiveEightB{} Evaluation Instructions,fonttitle=\bfseries,
  boxed title style={size=small,colframe=red!20!gray} ]

\textbf{User prompt}: Evaluate the similarity of the images based on the following conditions: 
\emph{- Be invariant to Gaussian blurs while evaluating images. For example, even if both images are the exact same, even if one is more blurred compared to the other, the similarity score should remain high. If the images are different, the similarity score should decrease regardless of how blurred they are.}  
The score should take into account how well the images align with these conditions, as well as their overall resemblance.  
Even if the images are identical but fail to meet the conditions, they should still receive a higher score than completely different images.  

Provide a score from 1 to 10, where 1 indicates no similarity and 10 indicates identical images that fully satisfy the conditions.  
Respond only in this format:

\begin{verbatim}
Score: <1-10>
Reason: <reason for score>
\end{verbatim}

Nothing else should be included. What score would you give? ``\texttt{data1}'' ``\texttt{data2}'' 
\end{tcolorbox}

\noindent
\begin{minipage}{0.48\textwidth}
\begin{tcolorbox}[enhanced,attach boxed title to top center={yshift=-3mm,yshifttext=-1mm},
  colback=red!5!white,colframe=red!20!gray,colbacktitle=red!20!gray,
  title=\internvlTwoFiveEightB{} - Original Order,fonttitle=\bfseries,
  boxed title style={size=small,colframe=red!20!gray},
  equal height group=examplegroup]

\begin{center}
    \includegraphics[width=0.9\linewidth]{imgs/parrots-image1.png}
\end{center}

\textbf{Model response:}
\begin{tcolorbox}[colback=gray!10!white, colframe=gray!50, sharp corners=south, boxrule=0.8pt]
\textit{Score: 7\\Reason: The images are similar in content, featuring parrots, but they are not identical.  
The first image is a close-up of a spider web on a wooden surface, while the second image shows two parrots on a ledge.  
The parrots are the main subjects in the second image, and the first image has a different focus and subject matter.  
The images do not meet the condition of being identical, but they share some visual elements like color and subject type.}
\end{tcolorbox}

\end{tcolorbox}
\end{minipage}
\hfill
\begin{minipage}{0.48\textwidth}
\begin{tcolorbox}[enhanced,attach boxed title to top center={yshift=-3mm,yshifttext=-1mm},
  colback=red!5!white,colframe=red!20!gray,colbacktitle=red!20!gray,
  title=\internvlTwoFiveEightB{} - Reverse Order,fonttitle=\bfseries,
  boxed title style={size=small,colframe=red!20!gray},
  equal height group=examplegroup]

\begin{center}
    \includegraphics[width=0.9\linewidth]{imgs/parrots-image2.png}
\end{center}

\textbf{Model response:}
\begin{tcolorbox}[colback=gray!10!white, colframe=gray!50, sharp corners=south, boxrule=0.8pt]
\textit{Score: 1\\Reason: The images are completely different in terms of content, subject matter, and context.  
The first image features colorful parrots, while the second image shows a wooden door with a spider web.  
There is no similarity in terms of subject, color, or composition.}
\end{tcolorbox}

\end{tcolorbox}
\end{minipage}

\subsection{\pixtral{}}

Below is an error example of \pixtral{} on a data-pair from \mmscorecoco with rotation (R).


\begin{tcolorbox}[enhanced,attach boxed title to top center={yshift=-3mm,yshifttext=-1mm},
  colback=red!5!white,colframe=red!20!gray,colbacktitle=red!20!gray,
  title=\pixtral{} Evaluation Instructions,fonttitle=\bfseries,
  boxed title style={size=small,colframe=red!20!gray} ]

\textbf{User prompt}: Rate the similarity of these images on a scale of 1 to 10 while considering the following condition:  

\emph{- Be invariant to rotations while evaluating images. For example, if one image is a rotated version of the other, the similarity score should remain high. If the images are different, the similarity score should decrease regardless of their orientation.}  

The score should account for both how well the images satisfy the conditions and their overall resemblance. If the images are identical but do not meet the conditions, they should still score higher than unrelated images.  

Your response must follow this exact format:
\begin{verbatim}
Score: <1-10>
Reason: <reason for score>
\end{verbatim}

Provide nothing else. What’s your score? ``\texttt{data1}'' ``\texttt{data2}'' 
\end{tcolorbox}

\noindent
\begin{minipage}{0.48\textwidth}
\begin{tcolorbox}[enhanced,attach boxed title to top center={yshift=-3mm,yshifttext=-1mm},
  colback=red!5!white,colframe=red!20!gray,colbacktitle=red!20!gray,
  title=\pixtral{} Response - Original Order, fonttitle=\bfseries,
  boxed title style={size=small,colframe=red!20!gray},
  equal height group=examplegroup]

\begin{center}
    \includegraphics[width=0.9\linewidth]{imgs/birds-interaction1.png}
\end{center}

\textbf{Model response:}
\begin{tcolorbox}[colback=gray!10!white, colframe=gray!50, sharp corners=south, boxrule=0.8pt]
\textit{Score: 8\\Reason: The images depict similar scenes of people interacting with birds in natural settings.  
The first image features a group of people sitting on rocks with birds nearby, while the second image has a person extending their hand  
towards a bird in a mountainous area. Despite slight differences in the environment and specific actions,  
the overall theme and presence of birds and human interaction score high.}
\end{tcolorbox}

\end{tcolorbox}
\end{minipage}
\hfill
\begin{minipage}{0.48\textwidth}
\begin{tcolorbox}[enhanced,attach boxed title to top center={yshift=-3mm,yshifttext=-1mm},
  colback=red!5!white,colframe=red!20!gray,colbacktitle=red!20!gray,
  title=\pixtral{} Response - Reversed Order,fonttitle=\bfseries,
  boxed title style={size=small,colframe=red!20!gray},
  equal height group=examplegroup]

\begin{center}
    \includegraphics[width=0.9\linewidth]{imgs/birds-interaction2.png}
\end{center}

\textbf{Model response:}
\begin{tcolorbox}[colback=gray!10!white, colframe=gray!50, sharp corners=south, boxrule=0.8pt]
\textit{Score: 2\\Reason: The images depict completely different scenes and subjects, with no recognizable similarity between them.  
One image shows a bird and people in a rocky area, while the other does not bear any resemblance to it.}
\end{tcolorbox}

\end{tcolorbox}
\end{minipage}


\subsection{Qwen2-VL}
Below is an error example of \qwenTwoVLSevenB{} on a data-pair from \mmscorecoco with gaussian blur (GB).

\begin{tcolorbox}[enhanced,attach boxed title to top center={yshift=-3mm,yshifttext=-1mm},
  colback=red!5!white,colframe=red!20!gray,colbacktitle=red!20!gray,
  title=\qwenTwoVLSevenB{} Evaluation Instructions,fonttitle=\bfseries,
  boxed title style={size=small,colframe=red!20!gray} ]

\textbf{User prompt}: As a similarity evaluator, your responsibility is to assess the similarity of the given images while considering these conditions:  
\emph{- Be sensitive to Gaussian blurs while evaluating images. For example, if both images are the same except one is a blurred version of the other, the similarity score should decrease. If the images are the same and have the same exact sharpness, the similarity score should remain high.}  

The similarity score should reflect both how well the images align with the conditions and their overall resemblance.  
Images that are identical but do not meet the conditions should receive a moderate score, while completely unrelated images should receive the lowest score.  

Provide your score on a scale of 1 to 10, with 10 being identical images that fully meet the conditions.  
Ensure your response is in the following format:
\begin{verbatim}
Score: <1-10>
Reason: <reason for score>
\end{verbatim}

Provide nothing else. What is your score? ``\texttt{data1}'' ``\texttt{data2}'' 
\end{tcolorbox}

\noindent
\begin{minipage}{0.48\textwidth}
\begin{tcolorbox}[enhanced,attach boxed title to top center={yshift=-3mm,yshifttext=-1mm},
  colback=red!5!white,colframe=red!20!gray,colbacktitle=red!20!gray,
  title=\qwenTwoVLSevenB{} - Original Order,fonttitle=\bfseries,
  boxed title style={size=small,colframe=red!20!gray},
  equal height group=examplegroup]

\begin{center}
    \includegraphics[width=0.9\linewidth]{imgs/pizza-image1.png}
\end{center}

\textbf{Model response:}
\begin{tcolorbox}[colback=gray!10!white, colframe=gray!50, sharp corners=south, boxrule=0.8pt]
\textit{Score: 6\\Reason: The images are very similar, with the second image being a slightly blurred version of the first.  
However, they do not fully meet the conditions as the images are not identical and the blurring is not perfect.}
\end{tcolorbox}

\end{tcolorbox}
\end{minipage}
\hfill
\begin{minipage}{0.48\textwidth}
\begin{tcolorbox}[enhanced,attach boxed title to top center={yshift=-3mm,yshifttext=-1mm},
  colback=red!5!white,colframe=red!20!gray,colbacktitle=red!20!gray,
  title=\qwenTwoVLSevenB{} - Reverse Order,fonttitle=\bfseries,
  boxed title style={size=small,colframe=red!20!gray},
  equal height group=examplegroup]

\begin{center}
    \includegraphics[width=0.9\linewidth]{imgs/pizza-image2.png}
\end{center}

\textbf{Model response:}
\begin{tcolorbox}[colback=gray!10!white, colframe=gray!50, sharp corners=south, boxrule=0.8pt]
\textit{Score: 9\\Reason: The images are nearly identical, with only a slight difference in the sharpness of the objects.  
The overall resemblance is high, and the conditions are met.}
\end{tcolorbox}

\end{tcolorbox}
\end{minipage}


\subsection{Phi3.5}
Below is an error example of \phiThreeFive{} on a data-pair from \mmscorecoco with elastic transform (ET).

\begin{tcolorbox}[enhanced,attach boxed title to top center={yshift=-3mm,yshifttext=-1mm},
  colback=red!5!white,colframe=red!20!gray,colbacktitle=red!20!gray,
  title=\phiThreeFive{} Evaluation Instructions,fonttitle=\bfseries,
  boxed title style={size=small,colframe=red!20!gray} ]

\textbf{User prompt}: Judge the similarity of these images based on:  
\emph{- Be sensitive to elastic transformations while evaluating images. For example, if the images are the same but have differing distortions due to elastic transformations, the similarity score should decrease. If the images are identical and have matching elastic transformations, the similarity score should remain high.}  

The similarity score should reflect both the overall resemblance of the images and how well they satisfy the conditions.  
Identical images that do not meet the conditions should still score higher than completely unrelated images.  

Provide a score on a scale of 1 to 10, with 1 being no similarity and 10 being identical images that satisfy all conditions.  
Respond strictly in this format:
\begin{verbatim}
Score: <1-10>
Reason: <reason for score>
\end{verbatim}

Do not include additional text. What’s your rating? ``\texttt{data1}'' ``\texttt{data2}'' 
\end{tcolorbox}

\noindent
\begin{minipage}{0.48\textwidth}
\begin{tcolorbox}[enhanced,attach boxed title to top center={yshift=-3mm,yshifttext=-1mm},
  colback=red!5!white,colframe=red!20!gray,colbacktitle=red!20!gray,
  title=\phiThreeFive{} - Original Order,fonttitle=\bfseries,
  boxed title style={size=small,colframe=red!20!gray},
  equal height group=examplegroup]

\begin{center}
    \includegraphics[width=0.9\linewidth]{imgs/motorcycle-image1.png}
\end{center}

\textbf{Model response:}
\begin{tcolorbox}[colback=gray!10!white, colframe=gray!50, sharp corners=south, boxrule=0.8pt]
\textit{Score: 1\\Reason: The images are highly distorted with elastic transformations, making it impossible to accurately compare the content.}
\end{tcolorbox}

\end{tcolorbox}
\end{minipage}
\hfill
\begin{minipage}{0.48\textwidth}
\begin{tcolorbox}[enhanced,attach boxed title to top center={yshift=-3mm,yshifttext=-1mm},
  colback=red!5!white,colframe=red!20!gray,colbacktitle=red!20!gray,
  title=\phiThreeFive{} - Reverse Order,fonttitle=\bfseries,
  boxed title style={size=small,colframe=red!20!gray},
  equal height group=examplegroup]

\begin{center}
    \includegraphics[width=0.9\linewidth]{imgs/motorcycle-image2.png}
\end{center}

\textbf{Model response:}
\begin{tcolorbox}[colback=gray!10!white, colframe=gray!50, sharp corners=south, boxrule=0.8pt]
\textit{Score: 10\\Reason: The images are identical with no differences in content, color, or composition.  
They both depict the same scene with motorcycles and people in a bar setting, and the elastic transformations do not alter the image in any way.}
\end{tcolorbox}

\end{tcolorbox}
\end{minipage}

\vspace{-5mm}
\section{Conclusion and Future Work}
\vspace{-1mm}
% \vspace{-1ex}
We introduced \mmscore{}, a comprehensive framework for evaluating the reliability of \modelss{} when used to define similarity kernels. \mmscore{} enables assessing how different models will behave when acting as evaluators by measuring kernel properties such as alignment with ground truth relevance, symmetry, smoothness, and controllability. Interestingly, by leveraging controlled data transformations, we found that \mmscore{} not only allows for fine-grained analysis of model biases and strengths, but it also offers a cost-effective alternative to large-scale benchmarks.

\vspace{-0.5mm}

We carried out a large-scale benchmarking covering several \modelss{} and demonstrated that no single model excels across all four metrics or dataset configurations. While commercial-grade models generally performed better on image-image comparisons, openly available models such as \internvlTwoFiveEightB{} showed competitive results, particularly in \nmi. Furthermore, our findings indicate that commonly used judge models exhibit limitations. For instance, \gptFouroEleven{} lacks in terms of symmetry and smoothness, highlighting the necessity of careful selection based on specific evaluation needs.

\vspace{-0.5mm}

From a more practical perspective, we established that \mmscore{} metrics, particularly \nmi{}, correlate strongly with model performance on well-known benchmarks, reinforcing its utility as a low-cost surrogate for ranking models or guiding cross-validation during training. As the field progresses, we anticipate that \mmscore{} will serve as a valuable tool for improving model evaluation practices. 

% As future work, \mmscore{} could be extended by incorporating additional modalities, refining transformation techniques, and further exploring the impact of architectural choices on similarity estimation. By advancing our understanding of \modelss{} as similarity kernels, we move closer to more robust and interpretable model-based evaluation methods.
\section*{Impact Statement}
% \color{blue}{Authors are \textbf{required} to include a statement of the potential broader impact of their work, including its ethical aspects and future societal consequences. This statement should be in an unnumbered section at the end of the paper (co-located with Acknowledgements -- the two may appear in either order, but both must be before References), and does not count toward the paper page limit. In many cases, where the ethical impacts and expected societal implications are those that are well established when advancing the field of Machine Learning, substantial discussion is not required, and a simple statement such as the following will suffice:}

%This paper presents work whose goal is to advance the field of Machine Learning. There are many potential societal consequences of our work, none which we feel must be specifically highlighted here.

This work contributes to the multimodal AI research community by introducing a systematic evaluation framework for assessing vision-language models capabilities as automated evaluators. By providing a comprehensive benchmarking methodology, we examine these models capabilities in comparing multimodal data pairs and generating meaningful similarity scores. Our analysis reveals critical limitations in employing VLMs as evaluators  including asymmetric scoring patterns and inconsistent alignment with human judgments. These insights are particularly important as VLMs are increasingly adopted as automated evaluation tools across various domains. While our framework helps surface these limitations, addressing the underlying challenges will require continued research effort from the broader AI community. 

% The above statement can be used verbatim in such cases, but we 
% encourage authors to think about whether there is content which does 
% warrant further discussion, as this statement will be apparent if the 
% paper is later flagged for ethics review.


% In the unusual situation where you want a paper to appear in the
% references without citing it in the main text, use \nocite
\nocite{langley00}

\bibliography{main}
\bibliographystyle{icml2025}


%%%%%%%%%%%%%%%%%%%%%%%%%%%%%%%%%%%%%%%%%%%%%%%%%%%%%%%%%%%%%%%%%%%%%%%%%%%%%%%
%%%%%%%%%%%%%%%%%%%%%%%%%%%%%%%%%%%%%%%%%%%%%%%%%%%%%%%%%%%%%%%%%%%%%%%%%%%%%%%
% APPENDIX
%%%%%%%%%%%%%%%%%%%%%%%%%%%%%%%%%%%%%%%%%%%%%%%%%%%%%%%%%%%%%%%%%%%%%%%%%%%%%%%
%%%%%%%%%%%%%%%%%%%%%%%%%%%%%%%%%%%%%%%%%%%%%%%%%%%%%%%%%%%%%%%%%%%%%%%%%%%%%%%
\newpage
\appendix
\onecolumn
%\section{Related Work}

% [what are some different benchmarks]

% [how are people evaluating vlms]

% [encoders are also being used as judges (text-image and image-image)]

% [what benchmarks are using vlms as judges]

% [literature on looking into invariance in LMMs]


% [downsides of using clip and other models (do not get negation), look at it as bag of words, spatially flawed (whatsup), looking into propperties that are not captures in benchmarks]
Using language models as automatic evaluators has become a somewhat common practice with popular approaches such as \textsc{GPTScore} and G-eval~\citep{fu2023gptscore,liu2023geval} being used to rank responses in the NLP domain. 
Due to that, there has been a significant amount of recent work that has investigated the capabilities and limitations of using LLMs as judges~\citep{thakur2024judging, chiang2023can, murugadoss2024evaluating, shankar2024validates}. 
\citet{chiang2023can} have shown that LLM evaluations are consistent and reproducible, making them suitable alternatives for human evaluation, they argue that these models inherent biases should prevent them using independently rather than \textit{alongside} human experts. 
Furthermore, \citet{zheng2023judging} reveal that large \modelss{{}, e.g., GPT-4 Turbo, align well with human judgments and \citet{thakur2024judging} further states that simpler models may still outperform GPT-4 Turbo in ranking tasks due to superior alignment metrics. Also, recent work assessed how humans can help LLMs evaluate better by testing different instruction types or designing tools that result in more balanced evaluations~\citep{murugadoss2024evaluating, shankar2024validates}.

It is worth noting that known limitations of LLMs such as their lack of invariance to the order of examples given in a prompt, which is a well studied issue of natural language models~\citep{fang2024rethinking}, 
% and also observed in the multimodal case by the lack of symmetry we reported,
may render auto evaluation unreliable. Similarly, \citet{berglund2023reversal} show failure cases where models trained on unidirectional relationships do not infer the reverse, indicating systemic limitations even in state-of-the-art LLMs such as \textsc{GPT-4} (as seen in Figure~\ref{fig:fig1} and in Appendix \ref{sec:error-analysis} for \modelss{}). Our main goal is
to investigate the reliability of automated evaluation in the multimodal context, by probing the models to compare data pairs. 
%thus to assess to what extent auto-evaluation can be done reliably, focusing in the multimodal case, by probing models on their ability to compare.

Namely, the evaluations we carry out focus on testing in multiple different ways how good \modelss{} are when it comes to comparing data instances, such as whether \modelss{} prompted to compare are symmetric or smooth for instance, and to what extent they can be controlled, i.e., instructed to pay attention to or ignore certain features of the inputs. While the literature is more sparse regarding testing \modelss{} in this setting, recent work has tested for something along those lines. \citet{chen2024mllm} for instance propose a benchmark for evaluating \modelss{} in multiple different scenarios, including checking whether pairwise comparisons of responses to a query correlated with human judgments. They concluded that although correlations are relatively high on comparison tasks, biases and inconsistencies affect performance on pair scoring and batch ranking. Similarly, \citet{awal2024vismin} introduced a synthetic dataset containing paired images that differ only along one feature (e.g., the color of an object). We seek to add to this branch of the literature by introducing a framework where controlled experiments can be carried out to anticipate the performance of models when being used as judges, and various different characteristics of automatic judges can be identified (e.g., how smooth they are).

Unlike the case of generative \modelss{} discussed above, discriminative visual language models such as CLIP~\citep{radford2021learning} are covered by a significant amount of recent work, and several failure modes are well reported, mostly deriving from the fact this class of \modelss{} tends to behave as bag-of-words models, focusing on nouns and ignoring relationships and semantics in their input data~\citep{yuksekgonul2023when}. For instance, CLIP was observed to struggle with spatial reasoning~\citep{kamath2023s} and ignore negation~\citep{alhamoud2025vision}. On the other hand, fine-tuning CLIP to reason about pairwise differences \citet{sam2024finetuning} showed that discriminative \modelss{} can improve on how well they manage to reason about pairwise differences if training is tailored for enabling so, highlighting the benefits that being able to measure these skills may inform training and improve models as a consequence. \citet{ouali2024discriminative} showed that fine-tuning generative \modelss{} to turn them into discriminative models results in improved image-retrieval from text, which aligns with results we reported in Section~\ref{sec:ecoders_vs_decoders} showing a gap between open-sources \modelss{} and CLIP-style encoders.

%\section{Related Work}

% [what are some different benchmarks]

% [how are people evaluating vlms]

% [encoders are also being used as judges (text-image and image-image)]

% [what benchmarks are using vlms as judges]

% [literature on looking into invariance in LMMs]


% [downsides of using clip and other models (do not get negation), look at it as bag of words, spatially flawed (whatsup), looking into propperties that are not captures in benchmarks]
Using language models as automatic evaluators has become a somewhat common practice with popular approaches such as \textsc{GPTScore} and G-eval~\citep{fu2023gptscore,liu2023geval} being used to rank responses in the NLP domain. 
Due to that, there has been a significant amount of recent work that has investigated the capabilities and limitations of using LLMs as judges~\citep{thakur2024judging, chiang2023can, murugadoss2024evaluating, shankar2024validates}. 
\citet{chiang2023can} have shown that LLM evaluations are consistent and reproducible, making them suitable alternatives for human evaluation, they argue that these models inherent biases should prevent them using independently rather than \textit{alongside} human experts. 
Furthermore, \citet{zheng2023judging} reveal that large \modelss{{}, e.g., GPT-4 Turbo, align well with human judgments and \citet{thakur2024judging} further states that simpler models may still outperform GPT-4 Turbo in ranking tasks due to superior alignment metrics. Also, recent work assessed how humans can help LLMs evaluate better by testing different instruction types or designing tools that result in more balanced evaluations~\citep{murugadoss2024evaluating, shankar2024validates}.

It is worth noting that known limitations of LLMs such as their lack of invariance to the order of examples given in a prompt, which is a well studied issue of natural language models~\citep{fang2024rethinking}, 
% and also observed in the multimodal case by the lack of symmetry we reported,
may render auto evaluation unreliable. Similarly, \citet{berglund2023reversal} show failure cases where models trained on unidirectional relationships do not infer the reverse, indicating systemic limitations even in state-of-the-art LLMs such as \textsc{GPT-4} (as seen in Figure~\ref{fig:fig1} and in Appendix \ref{sec:error-analysis} for \modelss{}). Our main goal is
to investigate the reliability of automated evaluation in the multimodal context, by probing the models to compare data pairs. 
%thus to assess to what extent auto-evaluation can be done reliably, focusing in the multimodal case, by probing models on their ability to compare.

Namely, the evaluations we carry out focus on testing in multiple different ways how good \modelss{} are when it comes to comparing data instances, such as whether \modelss{} prompted to compare are symmetric or smooth for instance, and to what extent they can be controlled, i.e., instructed to pay attention to or ignore certain features of the inputs. While the literature is more sparse regarding testing \modelss{} in this setting, recent work has tested for something along those lines. \citet{chen2024mllm} for instance propose a benchmark for evaluating \modelss{} in multiple different scenarios, including checking whether pairwise comparisons of responses to a query correlated with human judgments. They concluded that although correlations are relatively high on comparison tasks, biases and inconsistencies affect performance on pair scoring and batch ranking. Similarly, \citet{awal2024vismin} introduced a synthetic dataset containing paired images that differ only along one feature (e.g., the color of an object). We seek to add to this branch of the literature by introducing a framework where controlled experiments can be carried out to anticipate the performance of models when being used as judges, and various different characteristics of automatic judges can be identified (e.g., how smooth they are).

Unlike the case of generative \modelss{} discussed above, discriminative visual language models such as CLIP~\citep{radford2021learning} are covered by a significant amount of recent work, and several failure modes are well reported, mostly deriving from the fact this class of \modelss{} tends to behave as bag-of-words models, focusing on nouns and ignoring relationships and semantics in their input data~\citep{yuksekgonul2023when}. For instance, CLIP was observed to struggle with spatial reasoning~\citep{kamath2023s} and ignore negation~\citep{alhamoud2025vision}. On the other hand, fine-tuning CLIP to reason about pairwise differences \citet{sam2024finetuning} showed that discriminative \modelss{} can improve on how well they manage to reason about pairwise differences if training is tailored for enabling so, highlighting the benefits that being able to measure these skills may inform training and improve models as a consequence. \citet{ouali2024discriminative} showed that fine-tuning generative \modelss{} to turn them into discriminative models results in improved image-retrieval from text, which aligns with results we reported in Section~\ref{sec:ecoders_vs_decoders} showing a gap between open-sources \modelss{} and CLIP-style encoders.

\clearpage
\section{Error Analysis}
\label{sec:error-analysis}
In this section, we look into the outputs of each model and their errors given different data pairs. More specifically, we look into errors made by \geminiPro{}, \gptFouroEleven{}, \internvlTwoFiveEightB{}, \pixtral{}, \qwenTwoVLSevenB{}, and \phiThreeFive{}.
\subsection{Gemini Models}
Below is an error example of \geminiPro{}{} on a data-pair from \mmscorecoco with color jittering (CJ).
% \begin{tcolorbox}[enhanced,attach boxed title to top center={yshift=-3mm,yshifttext=-1mm},
%   colback=red!5!white,colframe=red!20!gray,colbacktitle=red!20!gray,
%   title=\geminiPro{},fonttitle=\bfseries,
%   boxed title style={size=small,colframe=red!20!gray} ]

%         \textbf{User prompt}: Assess the degree to which a text description corresponds to an image of two objects, taking into account the following conditions: \\\emph{- Be invariant to color jittering while evaluating images. For example, even if one image has been modified with random color changes such as brightness or contrast adjustments, the similarity score should remain high. If the images are different, the similarity score should decrease regardless of the jittering.}\\The scoring should reflect:
%         - A perfect alignment with the image that satisfies all conditions merits a score of 10.
%         - Descriptions that partially match the image but fail to meet certain conditions should still receive a higher score than entirely unrelated descriptions.
%         - A score of 0 should be reserved for cases where no correlation exists between the text and the image.
%         The text and image pair will be provided below. Provide your evaluation using the following format:
        
%         \begin{verbatim}
%         Score: <1-10>
%         Reason: <reason for score>
%         \end{verbatim}
        
%         What score would you assign?
%         \vspace{-8mm}
%         \begin{center}
%         \includegraphics[width=0.3\linewidth]{imgs/pair-burger-color-rev.png}
%         \end{center}

%         \textbf{Model response}:
%         \begin{tcolorbox}[colback=gray!10!white, colframe=gray!50, sharp corners=south, boxrule=0.8pt]
        
%         \textit{Score: 1\\Reason: The first image appears to be a mold of some sort, possibly for dental work or other medical purposes.  
%         The second image is a Caprese sandwich. They are entirely different objects. While the layered structure might be vaguely similar,  
%         the content, context, and purpose are completely dissimilar.}
%         \end{tcolorbox}
        
%         % \textbf{Score: 1\\Reason: The first image appears to be a mold of some sort, possibly for dental work or other medical purposes.  The second image is a Caprese sandwich.  They are entirely different objects.  While the layered structure might be vaguely similar, the content, context, and purpose are completely dissimilar.}

% \end{tcolorbox}

% \begin{tcolorbox}[enhanced,attach boxed title to top center={yshift=-3mm,yshifttext=-1mm},
%   colback=red!5!white,colframe=red!20!gray,colbacktitle=red!20!gray,
%   title=\geminiPro{},fonttitle=\bfseries,
%   boxed title style={size=small,colframe=red!20!gray} ]

%         \textbf{User prompt}: Assess the degree to which a text description corresponds to an image of two objects, taking into account the following conditions: \texttt{\{conditions\}}. The scoring should reflect:
%         - A perfect alignment with the image that satisfies all conditions merits a score of 10.
%         - Descriptions that partially match the image but fail to meet certain conditions should still receive a higher score than entirely unrelated descriptions.
%         - A score of 0 should be reserved for cases where no correlation exists between the text and the image.
%         The text and image pair will be provided below. Provide your evaluation using the following format:
        
%         \begin{verbatim}
%         Score: <1-10>
%         Reason: <reason for score>
%         \end{verbatim}
        
%         What score would you assign?
%         \vspace{-8mm}
%         \begin{center}
%         \includegraphics[width=0.3\linewidth]{imgs/pair-burger-color.png}
%         \end{center}

%         \textbf{Model response}:
%         \begin{tcolorbox}[colback=gray!10!white, colframe=gray!50, sharp corners=south, boxrule=0.8pt]
        
%         \textit{Score: 10\\Reason: The images are of the same sandwich, one in normal color and the other with a strong purple color filter applied.  Despite the heavy color distortion, the content is identical, fulfilling the color invariance requirement.}
%         \end{tcolorbox}
        
%         % \textbf{Score: 1\\Reason: The first image appears to be a mold of some sort, possibly for dental work or other medical purposes.  The second image is a Caprese sandwich.  They are entirely different objects.  While the layered structure might be vaguely similar, the content, context, and purpose are completely dissimilar.}

% \end{tcolorbox}

\begin{tcolorbox}[enhanced,attach boxed title to top center={yshift=-3mm,yshifttext=-1mm},
  colback=red!5!white,colframe=red!20!gray,colbacktitle=red!20!gray,
  title=\geminiPro{} Evaluation Instructions,fonttitle=\bfseries,
  boxed title style={size=small,colframe=red!20!gray} ]

\textbf{User prompt}: Evaluate the similarity of the images based on the following conditions:
\emph{- Be invariant to color jittering while evaluating images. Even if one image has been modified with random color changes (e.g., brightness or contrast adjustments), the similarity score should remain high. If the images are different, the similarity score should decrease regardless of the jittering.}

The score should take into account how well the images align with these conditions, as well as their overall resemblance. Even if the images are identical but fail to meet the conditions, they should still receive a higher score than completely different images. Provide a score from 1 to 10, where 1 indicates no similarity and 10 indicates identical images that fully satisfy the conditions. Respond only in this format:

\begin{verbatim}
Score: <1-10>
Reason: <reason for score>
\end{verbatim}
Nothing else should be included. What score would you give?  ``\texttt{data1}'' ``\texttt{data2}'' 
\end{tcolorbox}

\noindent
\begin{minipage}{0.48\textwidth}
\begin{tcolorbox}[enhanced,attach boxed title to top center={yshift=-3mm,yshifttext=-1mm},
  colback=red!5!white,colframe=red!20!gray,colbacktitle=red!20!gray,
  title=\geminiPro{} Response - Original Order,fonttitle=\bfseries,
  boxed title style={size=small,colframe=red!20!gray},
  equal height group=examplegroup]

\begin{center}
    \includegraphics[width=0.9\linewidth]{imgs/pair-burger-color1.png}
\end{center}

\textbf{Model response:}
\begin{tcolorbox}[colback=gray!10!white, colframe=gray!50, sharp corners=south, boxrule=0.8pt]
\textit{Score: 10\\Reason: The images are of the same sandwich, one in normal color and the other with a strong purple color filter applied.  
Despite the heavy color distortion, the content is identical, fulfilling the color invariance requirement.}
\end{tcolorbox}

\end{tcolorbox}
\end{minipage}
\hfill
\begin{minipage}{0.48\textwidth}
\begin{tcolorbox}[enhanced,attach boxed title to top center={yshift=-3mm,yshifttext=-1mm},
  colback=red!5!white,colframe=red!20!gray,colbacktitle=red!20!gray,
  title=\geminiPro{} Response - Reversed Order,fonttitle=\bfseries,
  boxed title style={size=small,colframe=red!20!gray},
  equal height group=examplegroup]

\begin{center}
    \includegraphics[width=0.9\linewidth]{imgs/pair-burger-color2.png}
\end{center}

\textbf{Model response:}
\begin{tcolorbox}[colback=gray!10!white, colframe=gray!50, sharp corners=south, boxrule=0.8pt]
\textit{Score: 1\\Reason: The first image appears to be a mold of some sort, possibly for dental work or other medical purposes.  
The second image is a Caprese sandwich. They are entirely different objects. While the layered structure might be vaguely similar,  
the content, context, and purpose are completely dissimilar.}
\end{tcolorbox}

\end{tcolorbox}
\end{minipage}


\subsection{GPT4o}
Below is an error example of \gptFouroEleven{} on a data-pair from \mmscorecoco with color jittering (CJ).
\begin{tcolorbox}[enhanced,attach boxed title to top center={yshift=-3mm,yshifttext=-1mm},
  colback=red!5!white,colframe=red!20!gray,colbacktitle=red!20!gray,
  title=\gptFouroEleven{} Evaluation Instructions,fonttitle=\bfseries,
  boxed title style={size=small,colframe=red!20!gray} ]

\textbf{User prompt}: Evaluate the similarity of the images based on the following conditions:

\emph{- Be invariant to color jittering while evaluating images. Even if one image has been modified with random color changes (e.g., brightness or contrast adjustments), the similarity score should remain high. If the images are different, the similarity score should decrease regardless of the jittering.}

The score should take into account how well the images align with these conditions, as well as their overall resemblance. Even if the images are identical but fail to meet the conditions, they should still receive a higher score than completely different images. 

Provide a score from 1 to 10, where 1 indicates no similarity and 10 indicates identical images that fully satisfy the conditions.

Respond only in this format:
\begin{verbatim}
Score: <1-10>
Reason: <reason for score>
\end{verbatim}

Nothing else should be included. What score would you give? \texttt{"data1"} \texttt{"data2"} 

\end{tcolorbox}

\noindent
\begin{minipage}{0.48\textwidth}
\begin{tcolorbox}[enhanced,attach boxed title to top center={yshift=-3mm,yshifttext=-1mm},
  colback=red!5!white,colframe=red!20!gray,colbacktitle=red!20!gray,
  title=\gptFouroEleven{} Response - Original Order,fonttitle=\bfseries,
  boxed title style={size=small,colframe=red!20!gray},
  equal height group=examplegroup]

\begin{center}
\includegraphics[width=0.9\linewidth]{imgs/pair-burger-color1.png}
\end{center}

\textbf{Model response:}
\begin{tcolorbox}[colback=gray!10!white, colframe=gray!50, sharp corners=south, boxrule=0.8pt]
\textit{Score: 8\\Reason: The images are of the same sandwich, with similar shapes and arrangements of the bread, tomato, mozzarella, and basil.  
The second image has substantial color jittering, but since the structural alignment and composition remain consistent, the similarity remains high.}
\end{tcolorbox}

\end{tcolorbox}
\end{minipage}
\hfill
\begin{minipage}{0.48\textwidth}
\begin{tcolorbox}[enhanced,attach boxed title to top center={yshift=-3mm,yshifttext=-1mm},
  colback=red!5!white,colframe=red!20!gray,colbacktitle=red!20!gray,
  title=\gptFouroEleven{} Response - Reversed Order,fonttitle=\bfseries,
  boxed title style={size=small,colframe=red!20!gray},
  equal height group=examplegroup]

\begin{center}
    \includegraphics[width=0.9\linewidth]{imgs/pair-burger-color2.png}
\end{center}

\textbf{Model response:}
\begin{tcolorbox}[colback=gray!10!white, colframe=gray!50, sharp corners=south, boxrule=0.8pt]
\textit{Score: 3\\Reason: The images depict similar objects, presumably a sandwich with layers, but there are notable differences.  
The first image appears distorted with altered coloration, and while it shares structural similarities with the second image,  
the overall resemblance is weak.}
\end{tcolorbox}

\end{tcolorbox}
\end{minipage}


\subsection{InternVL2.5}
Below is an error example of \internvlTwoFiveEightB{} on a data-pair from \mmscorecoco with that are irrelevant to each other.

\begin{tcolorbox}[enhanced,attach boxed title to top center={yshift=-3mm,yshifttext=-1mm},
  colback=red!5!white,colframe=red!20!gray,colbacktitle=red!20!gray,
  title=\internvlTwoFiveEightB{} Evaluation Instructions,fonttitle=\bfseries,
  boxed title style={size=small,colframe=red!20!gray} ]

\textbf{User prompt}: Evaluate the similarity of the images based on the following conditions: 
\emph{- Be invariant to Gaussian blurs while evaluating images. For example, even if both images are the exact same, even if one is more blurred compared to the other, the similarity score should remain high. If the images are different, the similarity score should decrease regardless of how blurred they are.}  
The score should take into account how well the images align with these conditions, as well as their overall resemblance.  
Even if the images are identical but fail to meet the conditions, they should still receive a higher score than completely different images.  

Provide a score from 1 to 10, where 1 indicates no similarity and 10 indicates identical images that fully satisfy the conditions.  
Respond only in this format:

\begin{verbatim}
Score: <1-10>
Reason: <reason for score>
\end{verbatim}

Nothing else should be included. What score would you give? ``\texttt{data1}'' ``\texttt{data2}'' 
\end{tcolorbox}

\noindent
\begin{minipage}{0.48\textwidth}
\begin{tcolorbox}[enhanced,attach boxed title to top center={yshift=-3mm,yshifttext=-1mm},
  colback=red!5!white,colframe=red!20!gray,colbacktitle=red!20!gray,
  title=\internvlTwoFiveEightB{} - Original Order,fonttitle=\bfseries,
  boxed title style={size=small,colframe=red!20!gray},
  equal height group=examplegroup]

\begin{center}
    \includegraphics[width=0.9\linewidth]{imgs/parrots-image1.png}
\end{center}

\textbf{Model response:}
\begin{tcolorbox}[colback=gray!10!white, colframe=gray!50, sharp corners=south, boxrule=0.8pt]
\textit{Score: 7\\Reason: The images are similar in content, featuring parrots, but they are not identical.  
The first image is a close-up of a spider web on a wooden surface, while the second image shows two parrots on a ledge.  
The parrots are the main subjects in the second image, and the first image has a different focus and subject matter.  
The images do not meet the condition of being identical, but they share some visual elements like color and subject type.}
\end{tcolorbox}

\end{tcolorbox}
\end{minipage}
\hfill
\begin{minipage}{0.48\textwidth}
\begin{tcolorbox}[enhanced,attach boxed title to top center={yshift=-3mm,yshifttext=-1mm},
  colback=red!5!white,colframe=red!20!gray,colbacktitle=red!20!gray,
  title=\internvlTwoFiveEightB{} - Reverse Order,fonttitle=\bfseries,
  boxed title style={size=small,colframe=red!20!gray},
  equal height group=examplegroup]

\begin{center}
    \includegraphics[width=0.9\linewidth]{imgs/parrots-image2.png}
\end{center}

\textbf{Model response:}
\begin{tcolorbox}[colback=gray!10!white, colframe=gray!50, sharp corners=south, boxrule=0.8pt]
\textit{Score: 1\\Reason: The images are completely different in terms of content, subject matter, and context.  
The first image features colorful parrots, while the second image shows a wooden door with a spider web.  
There is no similarity in terms of subject, color, or composition.}
\end{tcolorbox}

\end{tcolorbox}
\end{minipage}

\subsection{\pixtral{}}

Below is an error example of \pixtral{} on a data-pair from \mmscorecoco with rotation (R).


\begin{tcolorbox}[enhanced,attach boxed title to top center={yshift=-3mm,yshifttext=-1mm},
  colback=red!5!white,colframe=red!20!gray,colbacktitle=red!20!gray,
  title=\pixtral{} Evaluation Instructions,fonttitle=\bfseries,
  boxed title style={size=small,colframe=red!20!gray} ]

\textbf{User prompt}: Rate the similarity of these images on a scale of 1 to 10 while considering the following condition:  

\emph{- Be invariant to rotations while evaluating images. For example, if one image is a rotated version of the other, the similarity score should remain high. If the images are different, the similarity score should decrease regardless of their orientation.}  

The score should account for both how well the images satisfy the conditions and their overall resemblance. If the images are identical but do not meet the conditions, they should still score higher than unrelated images.  

Your response must follow this exact format:
\begin{verbatim}
Score: <1-10>
Reason: <reason for score>
\end{verbatim}

Provide nothing else. What’s your score? ``\texttt{data1}'' ``\texttt{data2}'' 
\end{tcolorbox}

\noindent
\begin{minipage}{0.48\textwidth}
\begin{tcolorbox}[enhanced,attach boxed title to top center={yshift=-3mm,yshifttext=-1mm},
  colback=red!5!white,colframe=red!20!gray,colbacktitle=red!20!gray,
  title=\pixtral{} Response - Original Order, fonttitle=\bfseries,
  boxed title style={size=small,colframe=red!20!gray},
  equal height group=examplegroup]

\begin{center}
    \includegraphics[width=0.9\linewidth]{imgs/birds-interaction1.png}
\end{center}

\textbf{Model response:}
\begin{tcolorbox}[colback=gray!10!white, colframe=gray!50, sharp corners=south, boxrule=0.8pt]
\textit{Score: 8\\Reason: The images depict similar scenes of people interacting with birds in natural settings.  
The first image features a group of people sitting on rocks with birds nearby, while the second image has a person extending their hand  
towards a bird in a mountainous area. Despite slight differences in the environment and specific actions,  
the overall theme and presence of birds and human interaction score high.}
\end{tcolorbox}

\end{tcolorbox}
\end{minipage}
\hfill
\begin{minipage}{0.48\textwidth}
\begin{tcolorbox}[enhanced,attach boxed title to top center={yshift=-3mm,yshifttext=-1mm},
  colback=red!5!white,colframe=red!20!gray,colbacktitle=red!20!gray,
  title=\pixtral{} Response - Reversed Order,fonttitle=\bfseries,
  boxed title style={size=small,colframe=red!20!gray},
  equal height group=examplegroup]

\begin{center}
    \includegraphics[width=0.9\linewidth]{imgs/birds-interaction2.png}
\end{center}

\textbf{Model response:}
\begin{tcolorbox}[colback=gray!10!white, colframe=gray!50, sharp corners=south, boxrule=0.8pt]
\textit{Score: 2\\Reason: The images depict completely different scenes and subjects, with no recognizable similarity between them.  
One image shows a bird and people in a rocky area, while the other does not bear any resemblance to it.}
\end{tcolorbox}

\end{tcolorbox}
\end{minipage}


\subsection{Qwen2-VL}
Below is an error example of \qwenTwoVLSevenB{} on a data-pair from \mmscorecoco with gaussian blur (GB).

\begin{tcolorbox}[enhanced,attach boxed title to top center={yshift=-3mm,yshifttext=-1mm},
  colback=red!5!white,colframe=red!20!gray,colbacktitle=red!20!gray,
  title=\qwenTwoVLSevenB{} Evaluation Instructions,fonttitle=\bfseries,
  boxed title style={size=small,colframe=red!20!gray} ]

\textbf{User prompt}: As a similarity evaluator, your responsibility is to assess the similarity of the given images while considering these conditions:  
\emph{- Be sensitive to Gaussian blurs while evaluating images. For example, if both images are the same except one is a blurred version of the other, the similarity score should decrease. If the images are the same and have the same exact sharpness, the similarity score should remain high.}  

The similarity score should reflect both how well the images align with the conditions and their overall resemblance.  
Images that are identical but do not meet the conditions should receive a moderate score, while completely unrelated images should receive the lowest score.  

Provide your score on a scale of 1 to 10, with 10 being identical images that fully meet the conditions.  
Ensure your response is in the following format:
\begin{verbatim}
Score: <1-10>
Reason: <reason for score>
\end{verbatim}

Provide nothing else. What is your score? ``\texttt{data1}'' ``\texttt{data2}'' 
\end{tcolorbox}

\noindent
\begin{minipage}{0.48\textwidth}
\begin{tcolorbox}[enhanced,attach boxed title to top center={yshift=-3mm,yshifttext=-1mm},
  colback=red!5!white,colframe=red!20!gray,colbacktitle=red!20!gray,
  title=\qwenTwoVLSevenB{} - Original Order,fonttitle=\bfseries,
  boxed title style={size=small,colframe=red!20!gray},
  equal height group=examplegroup]

\begin{center}
    \includegraphics[width=0.9\linewidth]{imgs/pizza-image1.png}
\end{center}

\textbf{Model response:}
\begin{tcolorbox}[colback=gray!10!white, colframe=gray!50, sharp corners=south, boxrule=0.8pt]
\textit{Score: 6\\Reason: The images are very similar, with the second image being a slightly blurred version of the first.  
However, they do not fully meet the conditions as the images are not identical and the blurring is not perfect.}
\end{tcolorbox}

\end{tcolorbox}
\end{minipage}
\hfill
\begin{minipage}{0.48\textwidth}
\begin{tcolorbox}[enhanced,attach boxed title to top center={yshift=-3mm,yshifttext=-1mm},
  colback=red!5!white,colframe=red!20!gray,colbacktitle=red!20!gray,
  title=\qwenTwoVLSevenB{} - Reverse Order,fonttitle=\bfseries,
  boxed title style={size=small,colframe=red!20!gray},
  equal height group=examplegroup]

\begin{center}
    \includegraphics[width=0.9\linewidth]{imgs/pizza-image2.png}
\end{center}

\textbf{Model response:}
\begin{tcolorbox}[colback=gray!10!white, colframe=gray!50, sharp corners=south, boxrule=0.8pt]
\textit{Score: 9\\Reason: The images are nearly identical, with only a slight difference in the sharpness of the objects.  
The overall resemblance is high, and the conditions are met.}
\end{tcolorbox}

\end{tcolorbox}
\end{minipage}


\subsection{Phi3.5}
Below is an error example of \phiThreeFive{} on a data-pair from \mmscorecoco with elastic transform (ET).

\begin{tcolorbox}[enhanced,attach boxed title to top center={yshift=-3mm,yshifttext=-1mm},
  colback=red!5!white,colframe=red!20!gray,colbacktitle=red!20!gray,
  title=\phiThreeFive{} Evaluation Instructions,fonttitle=\bfseries,
  boxed title style={size=small,colframe=red!20!gray} ]

\textbf{User prompt}: Judge the similarity of these images based on:  
\emph{- Be sensitive to elastic transformations while evaluating images. For example, if the images are the same but have differing distortions due to elastic transformations, the similarity score should decrease. If the images are identical and have matching elastic transformations, the similarity score should remain high.}  

The similarity score should reflect both the overall resemblance of the images and how well they satisfy the conditions.  
Identical images that do not meet the conditions should still score higher than completely unrelated images.  

Provide a score on a scale of 1 to 10, with 1 being no similarity and 10 being identical images that satisfy all conditions.  
Respond strictly in this format:
\begin{verbatim}
Score: <1-10>
Reason: <reason for score>
\end{verbatim}

Do not include additional text. What’s your rating? ``\texttt{data1}'' ``\texttt{data2}'' 
\end{tcolorbox}

\noindent
\begin{minipage}{0.48\textwidth}
\begin{tcolorbox}[enhanced,attach boxed title to top center={yshift=-3mm,yshifttext=-1mm},
  colback=red!5!white,colframe=red!20!gray,colbacktitle=red!20!gray,
  title=\phiThreeFive{} - Original Order,fonttitle=\bfseries,
  boxed title style={size=small,colframe=red!20!gray},
  equal height group=examplegroup]

\begin{center}
    \includegraphics[width=0.9\linewidth]{imgs/motorcycle-image1.png}
\end{center}

\textbf{Model response:}
\begin{tcolorbox}[colback=gray!10!white, colframe=gray!50, sharp corners=south, boxrule=0.8pt]
\textit{Score: 1\\Reason: The images are highly distorted with elastic transformations, making it impossible to accurately compare the content.}
\end{tcolorbox}

\end{tcolorbox}
\end{minipage}
\hfill
\begin{minipage}{0.48\textwidth}
\begin{tcolorbox}[enhanced,attach boxed title to top center={yshift=-3mm,yshifttext=-1mm},
  colback=red!5!white,colframe=red!20!gray,colbacktitle=red!20!gray,
  title=\phiThreeFive{} - Reverse Order,fonttitle=\bfseries,
  boxed title style={size=small,colframe=red!20!gray},
  equal height group=examplegroup]

\begin{center}
    \includegraphics[width=0.9\linewidth]{imgs/motorcycle-image2.png}
\end{center}

\textbf{Model response:}
\begin{tcolorbox}[colback=gray!10!white, colframe=gray!50, sharp corners=south, boxrule=0.8pt]
\textit{Score: 10\\Reason: The images are identical with no differences in content, color, or composition.  
They both depict the same scene with motorcycles and people in a bar setting, and the elastic transformations do not alter the image in any way.}
\end{tcolorbox}

\end{tcolorbox}
\end{minipage}

\FloatBarrier
\section{Full Results}
\label{sec:full-results}

In this section, we provide the \nmi of all models on all the different splits of \mmscorecoco, \mmscorein, \mmscorewuimgimg, and \mmscorewuimgtext{} in Tables \ref{tab:mi-coco-in100-var}, \ref{tab:mi-coco-in100-invar}, \ref{tab:mi-coco-in100-var}, \ref{tab:mi-coco-in100-invar}, \ref{tab:mi-wu-imgimg-var}, \ref{tab:mi-wu-imgimg-invar}, and \ref{tab:mi-wu-imgtext}.

\begin{table*}[ht]
\centering
\caption{Comparison of the \nmi{} metric ($\times 100$) of \modelss{} on \mmscorecoco{} and \mmscorein{} benchmarks in the \textit{sensitive} setting. Models are evaluated across multiple criteria:  color jitter (CJ), elastic transform (ET), gaussian blur (GB), perspective shift (PS), and rotation (R). Higher scores indicate better performance.}
\begin{tabular}{l*{5}{c}*{5}{c}}
\toprule
\multirow{2}{*}{\textbf{Model}} & \multicolumn{5}{c}{\textbf{\mmscorecoco}} & \multicolumn{5}{c}{\textbf{\mmscorein}} \\
        \cmidrule(lr){2-6} \cmidrule(lr){7-11} & \textbf{CJ} & \textbf{ET} & \textbf{GB} & \textbf{PS} & \textbf{R} &  \textbf{CJ} & \textbf{ET} & \textbf{GB} & \textbf{PS} & \textbf{R} \\
\midrule
% model & coco &  &  &  &  & in100 &  &  &  &  \\
%  & CJ & ET & GB & PS & R & CJ & ET & GB & PS & R \\
\chameleon & 00.37 & 00.34 & 00.19 & 00.31 & 00.60 & 00.38 & 00.26 & 00.31 & 00.50 & 00.52 \\
\llavaonevision & 36.51 & 44.05 & 38.57 & 43.80 & 41.41 & 37.05 & 49.89 & 40.00 & 46.01 & 49.30 \\
\phiThreeFive & 38.21 & 51.61 & 61.94 & 47.33 & 34.56 & 25.74 & 43.03 & 51.40 & 32.51 & 23.61 \\
\pixtral & 37.67 & 56.25 & 54.32 & 49.53 & 36.80 & 30.75 & 52.30 & 51.94 & 46.04 & 40.76 \\
\rowcolor{blue!15}
\internvlTwoOneB & 03.23 & 03.47 & 03.27 & 03.63 & 03.51 & 02.59 & 02.38 & 01.70 & 02.02 & 02.23 \\
\rowcolor{blue!15}
\internvlTwoTwoB & 23.89 & 32.76 & 34.32 & 31.53 & 24.76 & 18.32 & 34.02 & 33.35 & 28.17 & 23.35 \\
\rowcolor{blue!15}
\internvlTwoFourB & 52.13 & 69.43 & 62.46 & 63.77 & 52.68 & 45.25 & 65.90 & 59.90 & 60.28 & 51.04 \\
\rowcolor{blue!15}
\internvlTwoEightB & 51.58 & 62.80 & 62.35 & 60.27 & 54.80 & 47.94 & 60.18 & 58.60 & 56.66 & 53.00 \\
\rowcolor{purple!15}
\internvlTwoFiveOneB & 16.74 & 25.38 & 27.67 & 24.83 & 16.54 & 15.63 & 33.67 & 39.23 & 37.97 & 22.53 \\
\rowcolor{purple!15}
\internvlTwoFiveTwoB & 12.48 & 19.58 & 25.26 & 18.33 & 13.84 & 17.27 & 38.28 & 39.21 & 31.23 & 21.45 \\
\rowcolor{purple!15}
\internvlTwoFiveFourB & 42.61 & 59.78 & 54.33 & 55.34 & 49.47 & 41.35 & 62.35 & 54.21 & 56.18 & 49.90 \\
\rowcolor{purple!15}
\internvlTwoFiveEightB & 54.51 & 73.37 & 78.31 & 63.17 & 60.71 & 51.76 & 77.10 & 76.40 & 60.40 & 55.30 \\
\rowcolor{orange!15}
\molmoEOneB & 00.40 & 00.09 & 01.20 & 00.03 & 00.05 & 00.41 & 00.01 & 00.45 & 00.01 & 00.01 \\
\rowcolor{orange!15}
\molmoOSevenB & 14.32 & 16.02 & 48.93 & 16.12 & 15.40 & 12.91 & 14.20 & 48.43 & 13.83 & 12.16 \\
\rowcolor{orange!15}
\molmoDSevenB & 27.06 & 45.28 & 34.46 & 49.60 & 30.39 & 22.88 & 41.06 & 35.83 & 44.49 & 32.22 \\
\rowcolor{yellow!15}
\qwenTwoVLTwoB & 09.91 & 11.82 & 09.01 & 13.13 & 11.95 & 10.63 & 13.69 & 10.41 & 13.21 & 12.23 \\
\rowcolor{yellow!15}
\qwenTwoVLSevenB & 42.58 & 61.90 & 50.22 & 55.81 & 51.10 & 38.24 & 61.73 & 50.23 & 53.07 & 52.29 \\
\midrule
\rowcolor{green!15}
\gptFouroMini & 49.98 & 65.97 & 58.29 & 53.23 & 53.60 & 47.06 & 67.06 & 56.43 & 49.97 & 52.59 \\
\rowcolor{green!15}
\gptFouroFive & 50.96 & 65.54 & 61.67 & 56.69 & 56.71 & 48.55 & 65.68 & 57.48 & 54.11 & 55.00 \\
\rowcolor{green!15}
\gptFouroEight & 42.26 & 60.58 & 56.62 & 50.13 & 53.63 & 40.35 & 60.66 & 52.65 & 49.62 & 49.77 \\
\rowcolor{green!15}
\gptFouroEleven & 51.31 & 63.50 & 61.35 & 57.84 & 57.16 & 50.88 & 66.55 & 58.14 & 56.25 & 55.52 \\
\rowcolor{green!30}
\geminiFlash & \textbf{58.26} & 82.64 & 87.41 & 65.92 & 61.08 & \textbf{56.25} & 79.69 & 85.21 & 62.07 & 61.15 \\
\rowcolor{green!30}
\geminiPro & 53.33 & \textbf{87.86} & \textbf{89.56} & \textbf{74.92} & \textbf{71.04} & 51.19 & \textbf{91.36} & \textbf{92.98} & \textbf{71.56} & \textbf{74.22} \\

\bottomrule
\end{tabular}
\label{tab:mi-coco-in100-var}
\end{table*}

\begin{table*}[ht]
\centering
\caption{Comparison of the \nmi{} metric ($\times 100$) of \modelss{} on \mmscorecoco{} and \mmscorein{} benchmarks in the \textit{invariant} setting. Models are evaluated across multiple criteria: color jitter (CJ), elastic transform (ET), gaussian blur (GB), perspective shift (PS), and rotation (R). Higher scores indicate better performance.}
\begin{tabular}{l*{5}{c}*{5}{c}}
\toprule
\multirow{2}{*}{\textbf{Model}} & \multicolumn{5}{c}{\textbf{\mmscorecoco}} & \multicolumn{5}{c}{\textbf{\mmscorein}} \\
        \cmidrule(lr){2-6} \cmidrule(lr){7-11} & \textbf{CJ} & \textbf{ET} & \textbf{GB} & \textbf{PS} & \textbf{R} &  \textbf{CJ} & \textbf{ET} & \textbf{GB} & \textbf{PS} & \textbf{R} \\
\midrule
% model & coco &  &  &  &  & in100 &  &  &  &  \\
%  & CJ & ET & GB & PS & R & CJ & ET & GB & PS & R \\
\chameleon & 00.89 & 00.34 & 00.44 & 00.51 & 00.38 & 00.57 & 00.35 & 00.53 & 00.58 & 00.45 \\
\llavaonevision & 35.13 & 37.26 & 39.22 & 40.29 & 38.29 & 38.09 & 43.04 & 41.83 & 40.86 & 42.24 \\
\phiThreeFive & 49.41 & 40.19 & 42.93 & 55.03 & 47.90 & 45.88 & 33.79 & 39.72 & 50.41 & 39.46 \\
\pixtral  & 48.26 & 47.34 & 45.35 & 60.20 & 55.65 & 41.53 & 45.30 & 42.84 & 52.63 & 52.65 \\
\rowcolor{blue!15}
\internvlTwoOneB & 02.69 & 01.76 & 02.71 & 02.00 & 02.69 & 01.39 & 00.82 & 01.22 & 00.90 & 01.40 \\
\rowcolor{blue!15}
\internvlTwoTwoB & 36.38 & 31.55 & 31.99 & 39.18 & 37.28 & 32.68 & 31.40 & 30.13 & 35.98 & 34.70 \\
\rowcolor{blue!15}
\internvlTwoFourB & 59.44 & 55.47 & 51.35 & 59.61 & 59.02 & 51.74 & 52.77 & 49.60 & 54.63 & 53.11 \\
\rowcolor{blue!15}
\internvlTwoEightB & 58.69 & 58.56 & 53.60 & 61.91 & 64.22 & 58.44 & 54.48 & 51.78 & 61.97 & 62.90 \\
\rowcolor{purple!15}
\internvlTwoFiveOneB  & 21.39 & 18.59 & 21.65 & 23.19 & 22.86 & 22.52 & 14.63 & 24.34 & 22.76 & 19.24 \\
\rowcolor{purple!15}
\internvlTwoFiveTwoB & 22.85 & 19.05 & 21.46 & 27.62 & 25.99 & 32.09 & 33.03 & 37.34 & 34.65 & 34.75 \\
\rowcolor{purple!15}
\internvlTwoFiveFourB & 56.24 & 47.41 & 43.93 & 53.71 & 55.28 & 61.80 & 50.50 & 47.33 & 51.58 & 58.56 \\
\rowcolor{purple!15}
\internvlTwoFiveEightB & \textbf{75.11} & 65.18 & 66.32 & \textbf{78.56} & \textbf{81.77} & \textbf{72.53} & 61.61 & 62.23 & 65.18 & 74.27 \\
\rowcolor{orange!15}
\molmoEOneB & 00.10 & 00.11 & 00.06 & 00.02 & 00.00 & 00.02 & 00.11 & 00.10 & 00.07 & 00.25 \\
\rowcolor{orange!15}
\molmoOSevenB & 26.86 & 34.58 & 33.46 & 34.70 & 24.55 & 25.04 & 30.81 & 38.52 & 32.79 & 27.65 \\
\rowcolor{orange!15}
\molmoDSevenB & 47.20 & 45.02 & 43.02 & 50.54 & 48.64 & 45.01 & 45.83 & 45.47 & 49.25 & 40.87 \\
\rowcolor{yellow!15}
\qwenTwoVLTwoB & 09.55 & 09.10 & 10.21 & 12.65 & 08.83 & 09.02 & 09.61 & 10.01 & 14.97 & 09.33 \\
\rowcolor{yellow!15}
\qwenTwoVLSevenB & 50.52 & 51.80 & 52.70 & 54.50 & 53.29 & 47.86 & 49.73 & 51.18 & 51.55 & 50.67 \\
\midrule
\rowcolor{green!15}
\gptFouroMini & 59.76 & 57.94 & 56.55 & 61.31 & 58.17 & 56.33 & 55.56 & 55.35 & 60.99 & 60.83 \\
\rowcolor{green!15}
\gptFouroFive & 70.83 & 61.70 & 59.40 & 61.13 & 62.10 & 68.82 & 56.16 & 56.70 & 57.79 & 59.80 \\
\rowcolor{green!15}
\gptFouroEight & 55.14 & 50.31 & 46.00 & 52.15 & 52.45 & 54.13 & 45.43 & 44.25 & 48.26 & 52.18 \\
\rowcolor{green!15}
\gptFouroEleven & 73.48 & 69.06 & 61.51 & 67.60 & 63.99 & 70.16 & 61.33 & 58.89 & 65.06 & 60.84 \\
\rowcolor{green!30}
\geminiFlash & 72.11 & 67.81 & 68.17 & 71.88 & 78.31 & 70.32 & 65.94 & 66.58 & 69.10 & 74.77 \\
\rowcolor{green!30}
\geminiPro & 68.93 & \textbf{69.64} & \textbf{71.50} & 72.06 & 68.42 & 66.31 & \textbf{70.03} & \textbf{72.17} & \textbf{70.13} & \textbf{69.32} \\

\bottomrule
\end{tabular}
\label{tab:mi-coco-in100-invar}
\end{table*}

\begin{table*}[ht]
\centering
\caption{Comparison of the \nmi{} metric ($\times 100$) of \modelss{} on \mmscorewuimgimg{} (subset A and B) benchmark in the \textit{sensitive} setting. Models are evaluated across multiple criteria: spatial position (SP), spatial position and color jitter (SP-CJ), spatial position and elastic transform (SP-ET), spatial position and gaussian blur (SP-GB), spatial position and perspective shift (SP-PS), and spatial position and rotation (SP-R). Higher scores indicate better performance.}
\resizebox{0.95\textwidth}{!}{%
\begin{tabular}{l*{6}{c}*{6}{c}}
\toprule
\multirow{2}{*}{\textbf{Model}} & \multicolumn{6}{c}{\textbf{\mmscore$_{WU_a}$}} & \multicolumn{6}{c}{\textbf{\mmscore$_{WU_b}$}} \\
        \cmidrule(lr){2-7} \cmidrule(lr){8-13} & \textbf{SP} & \textbf{SP-CJ} & \textbf{SP-ET} & \textbf{SP-GB} & \textbf{SP-PS} & \textbf{SP-R} & \textbf{SP} &  \textbf{SP-CJ} & \textbf{SP-ET} & \textbf{SP-GB} & \textbf{SP-PS} & \textbf{SP-R}\\
\midrule
% model & coco &  &  &  &  & in100 &  &  &  &  \\
%  & CJ & ET & GB & PS & R & CJ & ET & GB & PS & R \\
\chameleon & 00.28 & 00.47 & 00.23 & 00.52 & 0.21 & 00.20 & 00.34 & 00.38 & 00.35 & 00.26 & 00.31 & 00.33 \\
\llavaonevision & 38.95 & 18.83 & 24.03 & 26.78 & 29.46 & 24.63 & 19.70 & 14.03 & 16.51 & 16.78 & 17.76 & 17.02 \\
\phiThreeFive & 23.44 & 08.46 & 15.70 & 19.41 & 13.34 & 10.83 & 15.38 & 12.98 & 18.91 & 20.19 & 11.69 & 17.06 \\
\pixtral & 37.91 & 26.09 & 32.05 & 33.52 & 32.47 & 25.00 & 28.02 & 19.58 & 22.32 & 22.31 & 23.46 & 24.50 \\
\rowcolor{blue!15}
\internvlTwoOneB & 00.44 & 00.98 & 00.79 & 00.65 & 00.30 & 00.28 & 00.20 & - & - & 00.41 & 01.18 & 00.90 \\
\rowcolor{blue!15}
\internvlTwoTwoB & 22.85 & 12.03 & 14.37 & 17.84 & 18.66 & 15.50 & 20.72 & 10.89 & 11.22 & 15.74 & 17.74 & 13.58 \\
\rowcolor{blue!15}
\internvlTwoFourB & 46.89 & 27.91 & 36.67 & 43.03 & 44.27 & 27.76 & 44.89 & 27.77 & 33.35 & 38.12 & 42.23 & 36.16 \\
\rowcolor{blue!15}
\internvlTwoEightB & 41.99 & 32.06 & 35.71 & 41.02 & 40.12 & 29.11 & 46.36 & 32.17 & 39.24 & 41.90 & 45.59 & 40.30 \\
\rowcolor{purple!15}
\internvlTwoFiveOneB & 25.50 & 14.16 & 21.32 & 15.69 & 21.49 & 16.30 & 24.77 & 16.16 & 21.10 & 19.95 & 27.89 & 21.47 \\
\rowcolor{purple!15}
\internvlTwoFiveTwoB & 20.63 & 11.76 & 16.75 & 15.21 & 18.03 & 13.79 & 23.44 & 09.33 & 15.90 & 17.64 & 18.17 & 17.56 \\
\rowcolor{purple!15}
\internvlTwoFiveFourB & 46.15 & 32.74 & 39.05 & 39.24 & 42.28 & 32.94 & 47.93 & 33.75 & 40.23 & 39.82 & 44.07 & 42.57 \\
\rowcolor{purple!15}
\internvlTwoFiveEightB & 44.27 & 36.99 & 41.49 & 42.60 & 43.65 & 33.24 & 41.32 & 31.69 & 40.10 & 39.73 & 44.03 & 42.99 \\
\rowcolor{orange!15}
\molmoEOneB & 00.47 & 01.03 & 00.00 & 00.03 & 00.14 & 00.01 & 00.32 & 00.36 & 00.01 & 00.04 & 00.04 & 00.09 \\
\rowcolor{orange!15}
\molmoOSevenB & 15.94 & 09.90 & 11.32 & 15.38 & 12.92 & 12.01 & 15.15 & 08.40 & 11.39 & 11.33 & 13.60 & 12.50 \\
\rowcolor{orange!15}
\molmoDSevenB & 23.82 & 17.75 & 20.41 & 18.40 & 22.21 & 17.81 & 26.74 & 18.37 & 19.55 & 18.77 & 18.19 & 22.21 \\
\rowcolor{yellow!15}
\qwenTwoVLTwoB & 02.26 & 01.76 & 02.58 & 02.15 & 03.17 & 01.68 & 00.88 & 00.44 & 00.73 & 00.37 & 00.72 & 00.82 \\
\rowcolor{yellow!15}
\qwenTwoVLSevenB & 41.95 & 29.47 & 36.32 & 39.93 & 40.33 & 34.11 & 42.80 & 28.75 & 31.42 & 37.27 & 39.76 & 36.25 \\
\midrule
\rowcolor{green!15}
\gptFouroMini & 42.55 & 37.21 & 39.50 & 40.44 & 38.83 & 41.05 & 48.86 & 38.38 & 43.82 & 45.42 & 46.32 & 46.66 \\
\rowcolor{green!15}
\gptFouroFive & 40.27 & 37.83 & 36.79 & 38.52 & 38.84 & 38.07 & 44.13 & 39.46 & 39.46 & 43.58 & 43.49 & 46.25 \\
\rowcolor{green!15}
\gptFouroEight & 37.58 & 33.72 & 34.24 & 33.36 & 34.80 & 33.17 & 40.11 & 33.36 & 32.36 & 34.32 & 39.91 & 38.67 \\
\rowcolor{green!15}
\gptFouroEleven & 40.68 & 39.06 & 40.10 & 40.35 & 40.96 & 40.40 & 47.34 & 40.91 & 43.07 & 47.18 & 50.22 & 50.68 \\
\rowcolor{green!30}
\geminiFlash & 44.63 & 38.85 & 37.19 & 39.11 & 35.76 & 34.57 & 49.91 & 40.29 & 42.92 & 46.34 & 47.01 & 46.40 \\
\rowcolor{green!30}
\geminiPro & 40.38 & 36.07 & 31.52 & 37.85 & 29.92 & 30.37 & 49.20 & 38.26 & 39.16 & 44.98 & 41.70 & 40.72 \\
\bottomrule
\end{tabular}
}
\label{tab:mi-wu-imgimg-var}
\end{table*}


\begin{table*}[ht]
\centering
\caption{Comparison of the \nmi{} metric ($\times 100$) of \modelss{} on \mmscorewuimgimg{} (subset A and B) benchmark in the \textit{invariant} setting. Models are evaluated across multiple criteria:spatial position (SP), spatial position and color jitter (SP-CJ), spatial position and elastic transform (SP-ET), spatial position and gaussian blur (SP-GB), spatial position and perspective shift (SP-PS), and spatial position and rotation (SP-R). Higher scores indicate better performance.}
\resizebox{0.95\textwidth}{!}{%
\begin{tabular}{l*{6}{c}*{6}{c}}
\toprule
\multirow{2}{*}{\textbf{Model}} & \multicolumn{6}{c}{\textbf{\mmscore$_{WU_a}$}} & \multicolumn{6}{c}{\textbf{\mmscore$_{WU_b}$}} \\
        \cmidrule(lr){2-7} \cmidrule(lr){8-13} & \textbf{SP} & \textbf{SP-CJ} & \textbf{SP-ET} & \textbf{SP-GB} & \textbf{SP-PS} & \textbf{SP-R} & \textbf{SP} &  \textbf{SP-CJ} & \textbf{SP-ET} & \textbf{SP-GB} & \textbf{SP-PS} & \textbf{SP-R} \\
\midrule
% model & coco &  &  &  &  & in100 &  &  &  &  \\
%  & CJ & ET & GB & PS & R & CJ & ET & GB & PS & R \\
\chameleon & 00.34 & 00.39 & 00.76 & 00.47 & 00.43 & 00.41 & 00.47 & 00.34 & 00.56 & 00.24 & 00.62 & 00.34 \\
\llavaonevision & 34.79 & 31.56 & 30.23 & 34.14 & 32.61 & 28.69 & 13.12 & 18.41 & 16.21 & 22.69 & 15.34 & 17.91 \\
\phiThreeFive & 23.66 & 32.84 & 18.90 & 21.36 & 30.14 & 19.10 & 19.88 & 36.74 & 22.40 & 23.47 & 30.04 & 26.06 \\
\pixtral & 36.93 & 37.32 & 41.17 & 35.31 & 38.52 & 36.05 & 36.03 & 30.44 & 33.32 & 29.84 & 35.48 & 33.32 \\
\rowcolor{blue!15}
\internvlTwoOneB & 00.57 & 01.08 & 02.02 & 01.02 & 00.89 & 00.37 & 00.65 & 00.81 & 00.96 & 00.50 & 00.56 & 00.54 \\
\rowcolor{blue!15}
\internvlTwoTwoB & 26.25 & 25.53 & 25.76 & 21.12 & 26.57 & 26.98 & 26.03 & 24.52 & 26.49 & 25.81 & 31.01 & 29.33 \\
\rowcolor{blue!15}
\internvlTwoFourB & 39.33 & 40.23 & 37.80 & 42.25 & 43.10 & 34.57 & 51.43 & 41.55 & 45.96 & 50.20 & 54.94 & 50.34 \\
\rowcolor{blue!15}
\internvlTwoEightB & 43.80 & 44.31 & 44.53 & 43.99 & 46.02 & 40.43 & 60.92 & 46.63 & 54.53 & 51.31 & 56.94 & 53.88 \\
\rowcolor{purple!15}
\internvlTwoFiveOneB & 12.82 & 13.84 & 09.34 & 07.24 & 12.91 & 16.93 & 19.87 & 24.92 & 19.36 & 17.94 & 22.66 & 30.60 \\
\rowcolor{purple!15}
\internvlTwoFiveTwoB & 31.38 & 29.79 & 30.53 & 23.16 & 31.75 & 24.69 & 36.01 & 30.13 & 35.52 & 27.07 & 37.01 & 31.18 \\
\rowcolor{purple!15}
\internvlTwoFiveFourB & 48.79 & 53.58 & 54.52 & 48.09 & 52.78 & 46.46 & 50.51 & 48.71 & 53.45 & 52.03 & 53.77 & 50.12 \\
\rowcolor{purple!15}
\internvlTwoFiveEightB & 59.03 & 55.57 & 59.70 & 57.16 & 58.01 & 50.84 & 65.21 & 51.31 & 61.10 & 63.54 & 62.38 & 60.83 \\
\rowcolor{orange!15}
\molmoEOneB & 03.83 & 00.09 & 00.02 & 00.02 & 00.10 & 00.17 & 04.22 & 00.07 & 00.02 & 00.07 & 00.12 & 00.00 \\
\rowcolor{orange!15}
\molmoOSevenB & 18.63 & 17.50 & 19.68 & 16.42 & 19.58 & 14.99 & 15.94 & 19.46 & 20.93 & 17.98 & 24.21 & 21.68 \\
\rowcolor{orange!15}
\molmoDSevenB & 28.21 & 36.47 & 31.95 & 26.89 & 35.57 & 33.58 & 37.50 & 35.90 & 34.70 & 33.51 & 33.04 & 34.35 \\
\rowcolor{yellow!15}
\qwenTwoVLTwoB & 02.63 & 02.88 & 03.58 & 03.53 & 03.34 & 02.97 & 00.79 & 00.73 & 00.99 & 00.88 & 00.71 & 00.82 \\
\rowcolor{yellow!15}
\qwenTwoVLSevenB & 40.21 & 38.96 & 39.94 & 46.88 & 40.11 & 39.55 & 47.65 & 39.51 & 40.94 & 48.63 & 44.68 & 41.88 \\
\midrule
\rowcolor{green!15}
\gptFouroMini & 47.60 & 48.33 & 51.04 & 46.15 & 48.86 & 43.75 & 57.50 & 49.19 & 51.38 & 53.76 & 55.82 & 54.07 \\
\rowcolor{green!15}
\gptFouroFive & 52.39 & 51.58 & 48.78 & 47.11 & 47.50 & 52.68 & 61.59 & 59.77 & 58.08 & 60.95 & 61.53 & 63.74 \\
\rowcolor{green!15}
\gptFouroEight & 50.94 & 47.21 & 46.52 & 42.90 & 45.84 & 52.50 & 62.75 & 54.23 & 53.20 & 51.19 & 58.50 & 57.21 \\
\rowcolor{green!15}
\gptFouroEleven & 57.47 & 56.25 & 54.40 & 56.11 & 54.40 & 57.93 & 65.91 & 62.22 & 63.93 & 67.96 & 66.86 & 68.10 \\
\rowcolor{green!30}
\geminiFlash & 46.62 & 55.28 & 54.31 & 57.98 & 57.01 & 58.74 & 62.04 & 54.43 & 56.89 & 62.24 & 66.88 & 60.72 \\
\rowcolor{green!30}
\geminiPro & 38.07 & 35.08 & 35.05 & 36.11 & 33.21 & 33.23 & 56.43 & 42.24 & 43.74 & 48.41 & 50.40 & 45.83 \\
\bottomrule
\end{tabular}
}
\label{tab:mi-wu-imgimg-invar}
\end{table*}


\begin{table*}[ht]
\centering
\caption{Comparison of the \nmi{} metric ($\times 100$) of \modelss{} on the \mmscorewuimgtext{} (Subset A and B) benchmark in the \textit{sensitive} and \textit{invariant} settings. Models are evaluated across the spatial position (SP) criterion. Higher scores indicate better performance.}
\begin{tabular}{l*{2}{c}*{2}{c}}
\toprule
\multirow{2}{*}{\textbf{Model}} & \multicolumn{2}{c}{\textbf{\mmscore$_{WU_a}$}} & \multicolumn{2}{c}{\textbf{\mmscore$_{WU_b}$}} \\
        \cmidrule(lr){2-3} \cmidrule(lr){4-5} & \textbf{Sens.} & \textbf{Invar.} & \textbf{Sens.} & \textbf{Invar.} \\
\midrule
% model & coco &  &  &  &  & in100 &  &  &  &  \\
%  & CJ & ET & GB & PS & R & CJ & ET & GB & PS & R \\
\chameleon & 00.25 & 00.34 & 00.23 & 00.47 \\
\llavaonevision & 23.35 & 22.78 & 27.38 & 25.98  \\
\phiThreeFive & 13.86 & 12.30 & 25.67 & 24.74  \\
\pixtral & 05.14 & 05.04 & 03.27 & 04.58  \\
\rowcolor{blue!15}
\internvlTwoOneB & 06.29 & 03.75 & 15.90 & 08.31  \\
\rowcolor{blue!15}
\internvlTwoTwoB & 17.07 & 14.26 & 24.46 & 16.49  \\
\rowcolor{blue!15}
\internvlTwoFourB & 15.69 & 15.69 & 24.27 & 22.96  \\
\rowcolor{blue!15}
\internvlTwoEightB & 22.40 & 19.27 & 29.45 & 31.46  \\
\rowcolor{purple!15}
\internvlTwoFiveOneB & 20.80 & 09.49 & 16.86 & 13.23  \\
\rowcolor{purple!15}
\internvlTwoFiveTwoB & 15.36 & 11.15 & 19.69 & 18.42  \\
\rowcolor{purple!15}
\internvlTwoFiveFourB & 23.90 & 23.85 & 29.75 & 32.45  \\
\rowcolor{purple!15}
\internvlTwoFiveEightB & 24.16 & 25.55 & 24.00 & 28.22  \\
\rowcolor{orange!15}
\molmoEOneB & 00.12 & 00.04 & 00.02 & 00.21  \\
\rowcolor{orange!15}
\molmoOSevenB & 07.53 & 07.45 & 07.18 & 08.29  \\
\rowcolor{orange!15}
\molmoDSevenB & 09.45 & 12.26 & 08.34 & 11.26  \\
\rowcolor{yellow!15}
\qwenTwoVLTwoB & 02.65 & 03.09 & 05.09 & 05.86  \\
\rowcolor{yellow!15}
\qwenTwoVLSevenB & 09.43 & 09.19 & 15.99 & 16.13  \\
\midrule 
\rowcolor{green!15}
\gptFouroMini & 16.18 & 16.14 & 16.18 & 15.30  \\
\rowcolor{green!15}
\gptFouroFive & 11.49 & 20.48 & 12.63 & 20.98  \\
\rowcolor{green!15}
\gptFouroEight & 20.27 & 31.80 & 22.97 & 36.56  \\
\rowcolor{green!15}
\gptFouroEleven & 18.97 & 31.91 & 20.57 & 34.99  \\
\rowcolor{green!30}
\geminiFlash & 27.46 & 26.54 & 26.53 & 32.07  \\
\rowcolor{green!30}
\geminiPro & 26.89 & 27.16 & 28.57 & 29.23  \\

\bottomrule
\end{tabular}
\label{tab:mi-wu-imgtext}
\end{table*}
\begin{figure*}[ht]
    \centering
    \includegraphics[width=0.95\linewidth,trim={.1cm .2cm .2cm .2cm},clip]{imgs/all-metric-mi-distinct-plt.pdf}
    \includegraphics[width=0.95\linewidth,trim={.1cm .2cm .2cm .2cm},clip]{imgs/all-metric-entropy-distinct-plt.pdf}
    \includegraphics[width=0.95\linewidth,trim={.1cm .2cm .2cm .2cm},clip]{imgs/all-metric-relaxed1_sym-distinct-plt.pdf}
    \caption{\nmi, \textbf{Smoothness}, and \textbf{Controllability} for the best performing models in both \texttt{sens} and \texttt{invar} settings.}
    \label{fig:best-models-mmscore-smoothness-sym}
\end{figure*}


\subsubsection{All \relaxsym for different $\epsilon$s}
To show the \relaxsym{} for different values of $\varepsilon$, we plot Figure \ref{fig:diff-relax-sym-eps} and show as $\varepsilon$ gets higher, the values go higher. However, some models such as the GPT4o models struggle with symmetry. Please note that if $\varepsilon = 0$, it is the same as not having a threshold and hence calculating exact symmetry rather than a relaxed version.


\begin{figure*}[ht]
    \centering
    \includegraphics[width=0.95\linewidth,trim={.1cm .2cm .2cm .2cm},clip]{imgs/all-metric-exact_sym-relaxed1_sym-relaxed2_sym-relaxed3_sym-distinct-plt.pdf}
    \caption{\relaxsym{} for different $\varepsilon$s.}
    \label{fig:diff-relax-sym-eps}
\end{figure*}


\subsubsection{Different versions of same model}
\label{sec:model-versions}
% We further look into the affect of capacity on the different metrics of \mmscore. As seen in Figure \ref{fig:mi-model-versions} and \ref{fig:all-model-versions}, the larger capacity models tend to better across \nmi, \relaxsym, and \control. However, we observe there are exceptions, e.g., \internvlTwoFourB{} being more controllable in rotation (R) and perspective shift (PS), compared to \internvlTwoEightB. Also, we see \smoothness{} is not monotonically increasing as the model capacity increases. This shows that the stronger models may tend to be more certain about their responses, hence not generating similarity scores as diverse as the lower capacity ones.

% On the other hand, we saw in Table \ref{tab:benchmark_comparison} and Figure \ref{fig:control-vs-bm} that Smoothness also has a positive correlation with model performance and other benchmarks, showing that better models tend to be more smoother and create more diverse outputs compared to the weaker ones. Ultimately, we conclude that \smoothness{} is not a property of performance; however, it is a characteristic of a \model{} as a judge model which could be desirable depending on the use-case.

We further examine the effect of model capacity on the different metrics of \mmscore. As seen in Figures \ref{fig:mi-model-versions} and \ref{fig:all-model-versions}, larger-capacity models tend to perform better across \nmi, \relaxsym, and \control. However, there are exceptions—for example, \internvlTwoFourB{} demonstrates greater controllability in rotation (R) and perspective shift (PS) compared to \internvlTwoEightB. Additionally, smoothness (\smoothness{}) does not increase monotonically with model capacity. This suggests that stronger models may be more confident in their responses, leading to less diversity in their similarity scores compared to lower-capacity models.

On the other hand, Table \ref{tab:benchmark_comparison} and Figure \ref{fig:control-vs-bm} show that \smoothness{} correlates positively with model performance and other benchmarks, indicating that better models tend to produce smoother and more diverse outputs than weaker ones. Ultimately, we conclude that \smoothness{} is not strictly a property of model performance but rather a characteristic of a \model{} as a judge model that may be desirable (or not) depending on the use case. 


\begin{figure*}[ht]
    \centering
    % \includegraphics[width=0.95\linewidth,trim={.1cm .2cm .2cm .2cm},clip]{imgs/GPT-4o-Models-metric-mi-distinct-plt.pdf}
    \includegraphics[width=0.95\linewidth,trim={.1cm .2cm .2cm .2cm},clip]{imgs/invl-metric-entropy-distinct-plt.pdf}
    % \caption{Caption for First PDF.}
    % \label{fig:first-plot}

    \vspace{1em} % Add vertical spacing between figures if needed

    \includegraphics[width=0.95\linewidth,trim={.1cm .2cm .2cm .2cm},clip]{imgs/invl-metric-relaxed1_sym-distinct-plt.pdf}
    % \caption{Caption for Second PDF.}
    % \label{fig:second-plot}

    \vspace{1em} % Add vertical spacing between figures if needed

    \includegraphics[width=0.95\linewidth,trim={.1cm .2cm .2cm .2cm},clip]{imgs/invl-metric-entropy-distinct-plt.pdf}
    % \caption{Caption for Third PDF.}
    % \label{fig:third-plot}

    \caption{InternVL2.5 models with different capacities aggregated on \mmscorecoco{} and \mmscorein.}
    \label{fig:mi-model-versions}
\end{figure*}
\begin{figure*}[ht]
    \centering

    \includegraphics[width=0.95\linewidth,trim={.1cm .2cm .2cm .2cm},clip]{imgs/invl-metric-mi-relaxed1_sym-entropy-control_mult-distinct-plt.pdf}
    % \caption{Caption for Third PDF.}
    % \label{fig:third-plot}

    \caption{Aggregated \mmscore{} metrics across different versions of InternVL2.5 models.}
    \label{fig:all-model-versions}
\end{figure*}


\subsection{Correlations}
\label{sec:bm-correlations}

In this section, we plot the further correlations of the different metrics and show them in Figures \ref{fig:sym-vs-bm}, \ref{fig:control-vs-bm}, \ref{fig:control-vs-bm}. As seen, all these metrics have positive correlations as seen in the scatter plots.


\begin{figure*}[ht]
    \centering
    % \includegraphics[width=0.95\linewidth,trim={.1cm .2cm .2cm .2cm},clip]{imgs/sym-vs-bm-scatter-plot-correlation.pdf}
    \includegraphics[width=0.95\linewidth,trim={.1cm .2cm .2cm .2cm},clip]{imgs/mmscore_n_relxsym-v2.pdf}
    \caption{Other benchmarks versus \mmscore{} on \relaxsymone{}.}
    \label{fig:sym-vs-bm}
\end{figure*}
\begin{figure*}[ht]
    \centering
    \includegraphics[width=0.95\linewidth,trim={.1cm .2cm .2cm .2cm},clip]{imgs/mmscore_control.pdf}
    \caption{Other benchmarks versus \control{} on \mmscore{}.}
    \label{fig:control-vs-bm}
\end{figure*}
\begin{figure*}[ht]
    \centering
    \includegraphics[width=0.95\linewidth,trim={.1cm .2cm .2cm .2cm},clip]{imgs/mmscore_ent-v2.pdf}
    \caption{Other benchmarks versus Smoothness (\smoothness).}
    \label{fig:smoothness-vs-bm}
\end{figure*}

% \input{imgs/sym-plot}

% \newpage
\section{\mmscore{} Details}
\label{sec:mmscore-info}
We provide all the information regarding the released datasets \mmscorecoco, \mmscorein, \mmscorewuimgimg, and \mmscorewuimgtext{} in Table \ref{tab:mmscore-info}.
\begin{table*}[ht]
\centering
\caption{Information of different splits in \mmscorecoco, \mmscorein, \mmscorewuimgimg, and \mmscorewuimgtext.}
\label{tab:mmscore-info}
\resizebox{\textwidth}{!}{%
\begin{tabular}{llccc}
\toprule
\textbf{Modality} & \textbf{Source} & \textbf{Number of Selected} & \textbf{Splits} & \textbf{New Data Points / Total Data-Pair Comparisons} \\
\midrule
% \multirow{11}{*}{MMSCore (Image-Image)} & \multirow{6}{*}{COCO} & \multirow{6}{*}{500} & CJ & 1000 / 3000 \\ 
%  & & & R & 1000 / 3000 \\ 
%  & & & ET & 1000 / 3000 \\ 
%  & & & PS & 1000 / 3000 \\ 
%  & & & GB & 1000 / 3000 \\ 
% \cmidrule{2-5}
%  & \multirow{6}{*}{IN100} & \multirow{6}{*}{500} & CJ & 1000 / 3000 \\ 
%  & & & R & 1000 / 3000 \\ 
%  & & & ET & 1000 / 3000 \\ 
%  & & & PS & 1000 / 3000 \\ 
%  & & & GB & 1000 / 3000 \\ 
% \cmidrule{2-5}
\multirow{22}{*}{\mmscoreimgimg} & \multirow{6}{*}{\coco} & \multirow{6}{*}{500} & CJ & 1000 / 3000 \\ 
 & & & R & 1000 / 3000 \\ 
 & & & ET & 1000 / 3000 \\ 
 & & & PS & 1000 / 3000 \\ 
 & & & GB & 1000 / 3000 \\ 
\cmidrule{2-5}
 & \multirow{6}{*}{\imagenet} & \multirow{6}{*}{500} & CJ & 1000 / 3000 \\ 
 & & & R & 1000 / 3000 \\ 
 & & & ET & 1000 / 3000 \\ 
 & & & PS & 1000 / 3000 \\ 
 & & & GB & 1000 / 3000 \\ 
\cmidrule{2-5}
 & \multirow{6}{*}{\wu{} (subset A)} & \multirow{6}{*}{418} & SP & 0 / 3344 \\ 
 & & & SP \& CJ & 1254 / 3344 \\ 
 & & & SP \& R & 1254 / 3344 \\ 
 & & & SP \& ET & 1254 / 3344 \\ 
 & & & SP \& PS & 1254 / 3344 \\ 
 & & & SP \& GB & 1254 / 3344 \\ 
\cmidrule{2-5}
 & \multirow{6}{*}{\wu{} (subset B)} & \multirow{6}{*}{408} & SP & 0 / 3264 \\ 
 & & & SP \& CJ & 1224 / 3264 \\ 
 & & & SP \& R & 1224 / 3264 \\ 
 & & & SP \& ET & 1224 / 3264 \\ 
 & & & SP \& PS & 1224 / 3264 \\ 
 & & & SP \& GB & 1224 / 3264 \\ 
\midrule
\multirow{2}{*}{\mmscoreimgtext} & \wu{} (subset A) & 418 & SP & 1254 / 3344 \\ 
 & \wu{} (Subset B) & 408 & SP & 1224 / 3264 \\ 
\midrule
\textbf{In total} & - & 1826 & all splits & \textbf{22390 / 69648} \\ 
\bottomrule
\end{tabular}
}
\end{table*}


\subsection{Prompt Templates for Different \mmscore{} datasets}
\label{sec:prompt-templates}
We provide the 5 different templates that we choose at random for each data pair for the image-image and image-text prompts.

The following are the tempaltes we utilize for \mmscorecoco and \mmscorein, and \mmscorewuimgimg, i.e., the image-image pairs.

\begin{tcolorbox}[enhanced,attach boxed title to top center={yshift=-3mm,yshifttext=-1mm},
  colback=blue!5!white,colframe=blue!20!gray,colbacktitle=blue!20!gray,
  title=Image-Image Prompt Template V1,fonttitle=\bfseries,
  boxed title style={size=small,colframe=blue!20!gray} ]

        \emph{User prompt}: You are tasked with evaluating the similarity between two images while paying attention to the following conditions: \texttt{\{conditions\}}. Your goal is to judge the similarity of the images overall, where satisfying the conditions increases the similarity score. If the images are identical but fail to meet any of the conditions, they should still receive a higher score than completely unrelated images. Provide a similarity score on a scale from 1 to 10, where 1 represents entirely dissimilar images and 10 represents identical images that satisfy all conditions. Ensure your response is strictly in the following format:
        
        \begin{verbatim}
        Score: <1-10>
        Reason: <reason for score>
        \end{verbatim}
                
        Do not include anything else in your response. What score would you assign to this pair of images? \texttt{"data1"} \texttt{"data2"}.

\end{tcolorbox}

\begin{tcolorbox}[enhanced,attach boxed title to top center={yshift=-3mm,yshifttext=-1mm},
  colback=blue!5!white,colframe=blue!20!gray,colbacktitle=blue!20!gray,
  title=Image-Image Prompt Template V2,fonttitle=\bfseries,
  boxed title style={size=small,colframe=blue!20!gray} ]

        \emph{User prompt}: As a similarity evaluator, your responsibility is to assess the similarity of the given images while considering these conditions: \texttt{\{conditions\}}. The similarity score should reflect both how well the images align with the conditions and their overall resemblance. Images that are identical but do not meet the conditions should receive a moderate score, while completely unrelated images should receive the lowest score. Provide your score on a scale of 1 to 10, with 10 being identical images that fully meet the conditions. Ensure your response is in the following format:
        
        \begin{verbatim}
        Score: <1-10>
        Reason: <reason for score>
        \end{verbatim}
                
        Provide nothing else. What is your score? \texttt{"data1"} \texttt{"data2"}

\end{tcolorbox}

\begin{tcolorbox}[enhanced,attach boxed title to top center={yshift=-3mm,yshifttext=-1mm},
  colback=blue!5!white,colframe=blue!20!gray,colbacktitle=blue!20!gray,
  title=Image-Image Prompt Template V3,fonttitle=\bfseries,
  boxed title style={size=small,colframe=blue!20!gray} ]

        \emph{User prompt}: Evaluate the similarity of the images based on the following conditions: \texttt{\{conditions\}}. The score should take into account how well the images align with these conditions, as well as their overall resemblance. Even if the images are identical but fail to meet the conditions, they should still receive a higher score than completely different images. Provide a score from 1 to 10, where 1 indicates no similarity and 10 indicates identical images that fully satisfy the conditions. Respond only in this format:
        
        \begin{verbatim}
        Score: <1-10>
        Reason: <reason for score>
        \end{verbatim}
                
        Nothing else should be included. What score would you give? \texttt{"data1"} \texttt{"data2"}

\end{tcolorbox}

\begin{tcolorbox}[enhanced,attach boxed title to top center={yshift=-3mm,yshifttext=-1mm},
  colback=blue!5!white,colframe=blue!20!gray,colbacktitle=blue!20!gray,
  title=Image-Image Prompt Template V4,fonttitle=\bfseries,
  boxed title style={size=small,colframe=blue!20!gray} ]

        \emph{User prompt}: Judge the similarity of these images based on: \texttt{\{conditions\}}. The similarity score should reflect both the overall resemblance of the images and how well they satisfy the conditions. Identical images that do not meet the conditions should still score higher than completely unrelated images. Provide a score on a scale of 1 to 10, with 1 being no similarity and 10 being identical images that satisfy all conditions. Respond strictly in this format:
        
        \begin{verbatim}
        Score: <1-10>
        Reason: <reason for score>
        \end{verbatim}
                
        Do not include additional text. What's your rating? \texttt{"data1"} \texttt{"data2"}

\end{tcolorbox}

\begin{tcolorbox}[enhanced,attach boxed title to top center={yshift=-3mm,yshifttext=-1mm},
  colback=blue!5!white,colframe=blue!20!gray,colbacktitle=blue!20!gray,
  title=Image-Image Prompt Template V5,fonttitle=\bfseries,
  boxed title style={size=small,colframe=blue!20!gray} ]

        \emph{User prompt}: Rate the similarity of these images on a scale of 1 to 10 while considering \texttt{\{conditions\}}. The score should account for both how well the images satisfy the conditions and their overall resemblance. If the images are identical but do not meet the conditions, they should still score higher than unrelated images. Your response must follow this exact format:
        
        \begin{verbatim}
        Score: <1-10>
        Reason: <reason for score>
        \end{verbatim}
                
        Provide nothing else. What's your score? \texttt{"data1"} \texttt{"data2"}

\end{tcolorbox}


And these are the image-text versions for \mmscorewuimgtext:

\begin{tcolorbox}[enhanced,attach boxed title to top center={yshift=-3mm,yshifttext=-1mm},
  colback=green!5!white,colframe=green!20!gray,colbacktitle=green!20!gray,
  title=Image-Text Prompt Template V1,fonttitle=\bfseries,
  boxed title style={size=small,colframe=green!20!gray} ]

        \emph{User prompt}: You are assigned the task of determining how accurately a text description represents an image of two objects. Pay special attention to the following conditions: \texttt{\{conditions\}}. Your goal is to assess the alignment between the image and the text, where adherence to the conditions increases the similarity score. If the text partially describes the image but fails to meet specific conditions, it should still score higher than a text entirely unrelated to the image. Assign a similarity score on a scale from 0 to 10, where 0 signifies no correlation, and 10 denotes perfect alignment satisfying all conditions. The data pair (text and image) will follow this prompt. Respond strictly in the following format:
        
        \begin{verbatim}
        Score: <1-10>
        Reason: <reason for score>
        \end{verbatim}
        
        What score would you assign to this text-image pair?\
        \texttt{"data1"}\
        \texttt{"data2"}.\
        
\end{tcolorbox}

\begin{tcolorbox}[enhanced,attach boxed title to top center={yshift=-3mm,yshifttext=-1mm},
  colback=green!5!white,colframe=green!20!gray,colbacktitle=green!20!gray,
  title=Image-Text Prompt Template V2,fonttitle=\bfseries,
  boxed title style={size=small,colframe=green!20!gray} ]

        \emph{User prompt}: Figure out how well this image matches the description provided. The image shows two objects, and the text is meant to describe how they're arranged. Look at these specific conditions: \texttt{\{conditions\}}. If the text captures some parts of the image but misses others, it should still get a better score than something totally off. Score this match on a scale of 0 to 10, where 0 means there's no match and 10 means the description nails it and matches every condition perfectly. The text and image will follow this prompt. Answer in this format only:
        
        \begin{verbatim}
        Score: <1-10>
        Reason: <reason for score>
        \end{verbatim}
        
        What's your score?\
        \texttt{"data1"}\
        \texttt{"data2"}.\
        
\end{tcolorbox}

\begin{tcolorbox}[enhanced,attach boxed title to top center={yshift=-3mm,yshifttext=-1mm},
  colback=green!5!white,colframe=green!20!gray,colbacktitle=green!20!gray,
  title=Image-Text Prompt Template V3,fonttitle=\bfseries,
  boxed title style={size=small,colframe=green!20!gray} ]

        \emph{User prompt}: Evaluate the degree to which a text description accurately represents an image featuring two objects, taking into account the following conditions: \texttt{\{conditions\}}. Assign a score based on how well the image-text pair matches, where:
        - A perfect description that satisfies all conditions scores 10.
        - Texts that partially align with the image but fail to meet conditions should still score higher than completely unrelated ones.
        The data pair will follow this prompt. Provide your score on a scale of 0 to 10 using the exact format below:
        
        \begin{verbatim}
        Score: <1-10>
        Reason: <reason for score>
        \end{verbatim}
        
        What score would you give?\
        \texttt{"data1"}\
        \texttt{"data2"}.\
        
\end{tcolorbox}

\begin{tcolorbox}[enhanced,attach boxed title to top center={yshift=-3mm,yshifttext=-1mm},
  colback=green!5!white,colframe=green!20!gray,colbacktitle=green!20!gray,
  title=Image-Text Prompt Template V4,fonttitle=\bfseries,
  boxed title style={size=small,colframe=green!20!gray} ]

        \emph{User prompt}: You are tasked with reviewing how well a text description aligns with an image of two objects. The score should reflect not only the accuracy of the alignment but also how well the description satisfies the following conditions: \texttt{\{conditions\}}. Even if the text description captures some parts of the image while failing the conditions, it should still receive a higher score than a completely irrelevant description. The text and image will be provided below. Assign a score on a 0 to 10 scale, where 0 is no similarity and 10 is perfect alignment that meets all conditions. Answer only in this format:
        
        \begin{verbatim}
        Score: <1-10>
        Reason: <reason for score>
        \end{verbatim}
        
        What score would you assign?\
        \texttt{"data1"}\
        \texttt{"data2"}.\
        
\end{tcolorbox}

\begin{tcolorbox}[enhanced,attach boxed title to top center={yshift=-3mm,yshifttext=-1mm},
  colback=green!5!white,colframe=green!20!gray,colbacktitle=green!20!gray,
  title=Image-Text Prompt Template V5,fonttitle=\bfseries,
  boxed title style={size=small,colframe=green!20!gray} ]

        \emph{User prompt}: Assess the degree to which a text description corresponds to an image of two objects, taking into account the following conditions: \texttt{\{conditions\}}. The scoring should reflect:
        - A perfect alignment with the image that satisfies all conditions merits a score of 10.
        - Descriptions that partially match the image but fail to meet certain conditions should still receive a higher score than entirely unrelated descriptions.
        - A score of 0 should be reserved for cases where no correlation exists between the text and the image.
        The text and image pair will be provided below. Provide your evaluation using the following format:
        
        \begin{verbatim}
        Score: <1-10>
        Reason: <reason for score>
        \end{verbatim}
        
        What score would you assign?\
        \texttt{"data1"}\
        \texttt{"data2"}.\
        
\end{tcolorbox}

% \section{You \emph{can} have an appendix here.}

% You can have as much text here as you want. The main body must be at most $8$ pages long.
% For the final version, one more page can be added.
% If you want, you can use an appendix like this one.  

% The $\mathtt{\backslash onecolumn}$ command above can be kept in place if you prefer a one-column appendix, or can be removed if you prefer a two-column appendix.  Apart from this possible change, the style (font size, spacing, margins, page numbering, etc.) should be kept the same as the main body.
%%%%%%%%%%%%%%%%%%%%%%%%%%%%%%%%%%%%%%%%%%%%%%%%%%%%%%%%%%%%%%%%%%%%%%%%%%%%%%%
%%%%%%%%%%%%%%%%%%%%%%%%%%%%%%%%%%%%%%%%%%%%%%%%%%%%%%%%%%%%%%%%%%%%%%%%%%%%%%%


\end{document}


% This document was modified from the file originally made available by
% Pat Langley and Andrea Danyluk for ICML-2K. This version was created
% by Iain Murray in 2018, and modified by Alexandre Bouchard in
% 2019 and 2021 and by Csaba Szepesvari, Gang Niu and Sivan Sabato in 2022.
% Modified again in 2023 and 2024 by Sivan Sabato and Jonathan Scarlett.
% Previous contributors include Dan Roy, Lise Getoor and Tobias
% Scheffer, which was slightly modified from the 2010 version by
% Thorsten Joachims & Johannes Fuernkranz, slightly modified from the
% 2009 version by Kiri Wagstaff and Sam Roweis's 2008 version, which is
% slightly modified from Prasad Tadepalli's 2007 version which is a
% lightly changed version of the previous year's version by Andrew
% Moore, which was in turn edited from those of Kristian Kersting and
% Codrina Lauth. Alex Smola contributed to the algorithmic style files.

