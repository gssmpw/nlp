\documentclass[conference,compsoc]{IEEEtran}
\IEEEoverridecommandlockouts
% The preceding line is only needed to identify funding in the first footnote. If that is unneeded, please comment it out.
\usepackage{cite}
\usepackage{amsmath,amssymb,amsfonts}
\usepackage{algorithmic}
\usepackage{graphicx}
\usepackage{textcomp}
\usepackage{xcolor}
\usepackage{multirow}
\usepackage{xspace}
\usepackage{hyperref}
\usepackage{url}
\hypersetup{hidelinks}

%%% REVIEW
\newcommand{\tocite}{{\color{red}CITE} }
\newcommand{\toref}{{\color{red}REF} }

%%% LOGO
\newcommand{\usc}{\raisebox{-1pt}{\includegraphics[height=0.8em]{figures/usc_logo.png}}}
\newcommand{\vuam}{\raisebox{-1pt}{\includegraphics[height=0.8em]{figures/vu_logo.png}}}

%%% SIGNS and SYMBOLS
\newcommand{\grad}{\texttt{grad-CROP}}
\newcommand{\att}{\texttt{att-CROP}}
\newcommand{\seg}{\texttt{seg}}
\newcommand{\clip}{\texttt{clip-CROP}}
\newcommand{\sam}{\texttt{sam-CROP}}
\newcommand{\yolo}{\texttt{yolo-CROP}}
\newcommand{\hc}{\texttt{human-CROP}}
\newcommand{\zsvqa}{\texttt{ZSVQA}}
\newcommand{\vic}{\textbf{ViCrop}}
\newcommand{\xmark}{\text{\ding{55}}}
\newcommand{\cmark}{\text{\ding{51}}}
\newcommand{\success}{\texttt{\color{green} \cmark}}
\newcommand{\failure}{\texttt{\color{red} \xmark}}
\newcommand{\rel}{\texttt{rel-att}}
\newcommand{\gra}{\texttt{grad-att}}
\newcommand{\pgra}{\texttt{pure-grad}}
\newcommand{\relh}{\texttt{rel-att$^h$}}
\newcommand{\grah}{\texttt{grad-att$^h$}}
\newcommand{\pgrah}{\texttt{pure-grad$^h$}}


%%% Text Abb.
\makeatletter
\DeclareRobustCommand\onedot{\futurelet\@let@token\@onedot}
\def\@onedot{\ifx\@let@token.\else.\null\fi\xspace}

\def\aka{\emph{a.k.a}\onedot} \def\Eg{\emph{E.g}\onedot}
\def\eg{\emph{e.g}\onedot} \def\Eg{\emph{E.g}\onedot}
\def\ie{\emph{i.e}\onedot} \def\Ie{\emph{I.e}\onedot}
\def\cf{\emph{c.f}\onedot} \def\Cf{\emph{C.f}\onedot}
\def\etc{\emph{etc}\onedot} \def\vs{\emph{vs}\onedot}
\def\wrt{w.r.t\onedot} \def\dof{d.o.f\onedot}
\def\etal{\emph{et al}\onedot}
\makeatletter



\definecolor{myred}{HTML}{FF8577}
\definecolor{mygreen}{HTML}{0FA958}
\definecolor{myblue}{HTML}{1982C4}
\definecolor{codegreen}{rgb}{0,0.5,0}
\definecolor{codegray}{rgb}{0.5,0.5,0.5}
\definecolor{codepurple}{rgb}{0.07,0,0.53}
\definecolor{codered}{RGB}{189,41,0}
\definecolor{codecomment}{RGB}{153,153,153}
\definecolor{backcolour}{rgb}{0.96,0.96,0.96}
\definecolor{royalblue}{rgb}{0.0, 0.14, 0.4}
\definecolor{egyptianblue}{rgb}{0.06, 0.2, 0.65}
\definecolor{royalazure}{rgb}{0.0, 0.22, 0.66}
\definecolor{portlandorange}{rgb}{1.0, 0.35, 0.21}
\definecolor{sienna}{RGB}{183,105,68}
\definecolor{saddlebrown}{RGB}{139,69,19}
\definecolor{mediumbrown}{RGB}{83,41,11}
\definecolor{darkbrown}{RGB}{58,28,7}
\hypersetup{
    colorlinks=true,
    linkcolor=sienna,
    urlcolor=royalblue,
    citecolor=royalblue,
}
\def\BibTeX{{\rm B\kern-.05em{\sc i\kern-.025em b}\kern-.08em
    T\kern-.1667em\lower.7ex\hbox{E}\kern-.125emX}}
\begin{document}



\title{An Interactive Framework for Implementing Privacy-Preserving Federated Learning: Experiments on Large Language Models}

\makeatletter
\newcommand{\linebreakand}{%
  \end{@IEEEauthorhalign}
  \hfill\mbox{}\par
  \mbox{}\hfill\begin{@IEEEauthorhalign}
}
\makeatother
    \author{
    \IEEEauthorblockN{Kasra Ahmadi}
    \IEEEauthorblockA{ 
    \textit{University of South Florida}\\
    Tampa, FL 33620 \\
    ahmadi1@usf.edu}
    \and
    \IEEEauthorblockN{Rouzbeh Behnia}
    \IEEEauthorblockA{ 
    \textit{University of South Florida}\\
    Tampa, FL 33620 \\
    behnia@usf.edu}
    \and
    \IEEEauthorblockN{Reza Ebrahimi}
    \IEEEauthorblockA{ 
    \textit{University of South Florida}\\
    Tampa, FL 33620 \\
    ebrahimim@usf.edu}
    \linebreakand
    \IEEEauthorblockN{Mehran Mozaffari Kermani}
    \IEEEauthorblockA{ 
    \textit{University of South Florida}\\
    Tampa, FL 33620 \\
    mehran2@usf.edu}
    \and
    \IEEEauthorblockN{Jeremiah Birrell}
    \IEEEauthorblockA{ 
    \textit{Texas State University}\\
    San Marcos, TX 78666 \\
    jbirrell@txstate.edu}
    \and
    \IEEEauthorblockN{Jason Pacheco}
    \IEEEauthorblockA{
    \textit{University of Arizona}\\
    Tucson, AZ 85721 \\
    pachecoj@cs.arizona.edu}
    \and
    \IEEEauthorblockN{Attila A. Yavuz}
    \IEEEauthorblockA{
    \textit{University of South Florida}\\
    Tampa, FL 33620 \\
    attilaayavuz@usf.edu}
    }
    
\maketitle
\begin{abstract}

Federated learning (FL) enhances privacy by keeping user data on local devices. 
%
However, emerging attacks have demonstrated that the updates shared by users during training can reveal significant information about their data.
%
% This has greatly thwart the adoption of FL methods for training robust AI models in sensitive applications. 
%
Differential Privacy (DP) is considered the gold standard for safeguarding user data.
%
However, DP  guarantees are highly conservative, providing worst-case privacy guarantees. 
%
This can result in overestimating privacy needs, which may compromise the model's accuracy. 
%
Additionally, interpretations of these privacy guarantees have proven to be challenging in different contexts.
%
This is further exacerbated when other factors, such as the number of training iterations, data distribution, and specific application requirements, can add further complexity to this problem.
In this work, we proposed a framework that integrates a human entity as a privacy practitioner to determine an optimal trade-off between the model's privacy and utility.
%
Our framework is the first to address the variable memory requirement of existing DP methods in FL settings, where resource-limited devices (e.g., cell phones) can participate. To support such settings, we adopt a recent DP method with fixed memory usage to ensure scalable private FL. 

%
 We evaluated our proposed framework by fine-tuning a BERT-based LLM model using the GLUE dataset (a common approach in literature), leveraging the new accountant, and employing diverse data partitioning strategies to mimic real-world conditions.
%
As a result, we achieved stable memory usage, with an average accuracy reduction of 1.33\% for \(\epsilon = 10\) and 1.9\% for \(\epsilon = 6\), when compared to the state-of-the-art DP~accountant which does not support fixed memory usage.
\end{abstract}

\begin{IEEEkeywords}
Differential Privacy, Federated Learning, Privacy Cost, Fine-Tuning, LLM
\end{IEEEkeywords}
\section{Introduction}
\label{sec:intro}

\begin{figure*}[tb]
    \centering
    \includegraphics[width=0.848\linewidth]{figs/circuitnn.pdf} 
    \caption{Illustration of differentiable CircuitNN. CircuitNN is designed based on differentiable NAND gates. After DAS is guided by PI and PO pairs of the truth table, CircuitNN can get the precise circuit architecture logic equivalent to the truth table.}
    \label{fig:circuitnn}
\end{figure*}

% 1. Describe the importance of logic synthesis
% 2. Existing Problems
% (a) Neural Architecture Search: Unstable, Predefined Setting, etc.
% (b) Circuit Generation: Probabilistic Model, Logic Equivalence

With the rapid advancement of technology, the scale of integrated circuits (ICs) has expanded exponentially. 
This expansion has introduced significant challenges in chip manufacturing, particularly concerning power and area metrics.
A primary objective in IC design is achieving the same circuit function with fewer transistors, thereby reducing power usage and area occupancy.

Logic synthesis~\cite{hachtel2005logicsynth}, a critical step in electronic design automation (EDA), transforms behavioral-level circuit designs into optimized gate-level circuits, ultimately yielding the final IC layout. 
The primary goal of logic synthesis is to identify the physical implementation with the fewest gates for a given circuit function. 
This task constitutes a challenging NP-hard combinatorial optimization problem. 
Current logic synthesis tools~\cite{brayton2010abc, wolf2013yosys} rely on human-designed heuristics, often leading to sub-optimal outcomes.

Differentiable architecture search (DAS) techniques~\cite{liu2018darts, chu2020darts} offer novel perspectives on addressing challenges in this problem.
Circuit functions can be represented through truth tables, which map binary inputs to their corresponding outputs. 
Truth tables provide a precise representation of input-output relationships, ensuring the design of functionally equivalent circuits.
Inspired by this, researchers~\cite{deepmind2024ai4sys, wang2024tnet} have begun exploring the application of DAS to synthesize circuits directly from truth tables.
Specifically, \citet{deepmind2024ai4sys} proposed CircuitNN, a framework that learns differentiable connection structures with logic gates, enabling the automatic generation of logic circuits from truth tables.
This approach significantly reduces the complexity of traditional circuit generation. 
Building on this, \citet{wang2024tnet} introduced T-Net, a triangle-shaped variant of CircuitNN, incorporating regularization techniques to enhance the efficiency of DAS.

Despite these advancements, several challenges remain. 
The computational complexity of DAS grows quadratically with the number of gates, posing scalability issues.
Although triangle-shaped architecture~\cite{wang2024tnet} partially mitigates this problem, redundancy persists. 
%Additionally, DAS is susceptible to converging to local optima, limiting the ability to search architectures that satisfy the given truth tables~\cite{liu2018darts}. 
%Furthermore, hyperparameters (network depth and layer width) require extensive searches, introducing complexity and prolonging the synthesis process. 
Additionally, DAS is susceptible to converging to local optima~\cite{liu2018darts} and hyperparameters (network depth and layer width) require extensive searches. 
The challenges arise from the vast search space in DAS. 
% Even with predefined settings for CircuitNN, finding a configuration that meets the truth table requires extensive trial and error during the DAS process. 
Intuitively, limiting the search space through predefined parameters (network depth, gates per layer, and connection probabilities) can significantly reduce the complexity.

Recent advances~\cite{openai2023gpt4, abramson2024alphafold3, esser2024sd3, li2024mar} in conditional generative models have demonstrated remarkable performance across language, vision, and graph generation tasks. 
Motivated by these developments, we propose a novel approach to circuit generation that generates preliminary circuit structures to guide DAS in generating refined circuits matching specified truth tables. 
Firstly, we introduce CircuitVQ, a tokenizer with a discrete codebook for circuit tokenization. 
Built upon our Circuit AutoEncoder framework~\cite{hou2022graphmae,li2023maskgae,wu2025mgvga}, CircuitVQ is trained through a circuit reconstruction task. 
Specifically, the CircuitVQ encoder encodes input circuits into discrete tokens using a learnable codebook, while the decoder reconstructs the circuit adjacency matrix based on these tokens.
Subsequently, the CircuitVQ encoder serves as a circuit tokenizer for CircuitAR pretraining, which employs a masked autoregressive modeling paradigm~\cite{chang2022maskgit, li2023mage}. 
In this process, the discrete codes function as supervision signals. 
After training, CircuitAR can generate discrete tokens progressively, which can be decoded into initial circuit structures by the decoder of the CircuitVQ. 
These prior insights can guide DAS in producing refined circuits that match the target truth tables precisely.

Our key contributions can be summarized as follows:
\begin{itemize}
\item We introduce CircuitVQ, a circuit tokenizer that facilitates graph autoregressive modeling for circuit generation, based on our Circuit AutoEncoder framework;
\item Develop CircuitAR, a model trained using masked autoregressive modeling, which generates initial circuit structures conditioned on given truth tables;
\item Propose a refinement framework that integrates differentiable architecture search to produce functionally equivalent circuits guided by target truth tables;
\item Comprehensive experiments demonstrating the scalability and capability emergence of our CircuitAR and the superior performance of the proposed circuit generation approach.
\end{itemize}

% Motivation
% (a) Diffusion (Vision, Graph), Autoregressive (Language, Vision)
% (b) Circuit Generation for Predefined Setting
% (c) Neural Architecture Search for Strict Logic Equivalence

% Contribution
% (a) Circuit Tokenizer (new transformer arch, training strategy)
% (b) CircuitAR (train and gen strategies, post-ar strategy)
% (c) Extensive Evaluation including BitD (Bit Distance) for Scalability

\section{Preliminary}

\paragraph{Notation} Consider a sentence of $T$ tokens $\vx=\{\vx_1,\ldots, \vx_T\}\in\gX$, and let $P$ be the unknown target language distribution, $\tilde P(\vx)$ be the empirical distribution of the training data (which is an approximation of $P$), and $Q$ be the distribution of our model at hand. Since our paper is also closely related to RLHF, we will also use $\pi$ to represent the distributions. In particular, we sometimes write $\pi_\theta$ for a distribution that is parameterized by $\theta$, where $\theta$ is usually the set of trainable parameters of the LLM; we write $\pr$ for a reference distribution that should be clear given the context. The next token prediction loss is minimizing the forward-KL between $P$ and $Q$. 




% TODO more details in methodology and data processing
% merge methodology and 
\section{Framework for Analyzing Emotion}
In this section, we present our framework for analyzing emotion. We first establish a basic understanding of emotion polarity by determining the sentiment valence of each root tweet and comment. We then use multi-label emotion detection to predict the emotion categories associated with each post. Based on this data, we explore the interactive nature of emotions, by identifying common patterns in emotion transition pairs between temporally-adjacent posts. Finally we investigate the emotional trajectory within threads to understand how emotional intensity and type shift over time, by aggregating the predicted labels for posts at each time stamp in a given thread. As part of this, we contrast rumour with non-rumour threads, to gain a holistic understanding of emotional expression in rumours and non-rumours on Twitter.

% elaborate a bit on why we choose EmoLLM, compared with other automatic emotion detection methods
\paragraph{Affective Computing: Automatic Emotion Detection}
Manually annotating emotions is both costly and time-consuming, so we use an LLM-based emotion detection model, EmoLLM~\citep{liu2024emollms}, which is specifically designed for sentiment analysis and emotion detection. The model was instruction-tuned on SemEval 2018 Task1 using a comprehensive emotion labeling scheme grounded in established theoretical frameworks. We prompt the model to perform Valence Ordinal Classification (V-oc), Emotion Classification (E-c), and Emotion Intensity regression (E-i). Detailed prompts are shown in \Cref{tab:emollm_ins}.

\paragraph{Categorical Emotion Labeling Scheme} \label{para:emotion_label}
Numerous emotion label sets  have been proposed~\citep{Ekman1992AnAF, Plutchik1980AGP, Russell1980ACM}. According to \citet{Ekman1992AnAF, Plutchik1980AGP}, certain emotions, such as joy, fear, and sadness, are considered more fundamental than others, both physiologically and cognitively. The Valence-Arousal-Dominance (VAD) model \citep{Russell1980ACM} categorizes emotions within a three-dimensional space of valence (positivity-negativity), arousal (active-passive), and dominance (dominant-submissive). Inspired by \citet{mohammad-etal-2018-semeval}, we incorporate elements from both basic emotion theories and the VAD model, and further ground EmoLLM emotion predictions to develop the following emotion label schemes: (1) \textit{neutral or no emotion}; (2) \textit{negative emotions}: anger (also includes annoyance and rage),  disgust (also includes disinterest, dislike, and loathing), fear (also includes apprehension, anxiety, and terror), pessimism (also includes cynicism, and no confidence), sadness (also includes pensiveness and grief); 3) \textit{positive emotions}: joy (also includes serenity and ecstasy), love (also includes affection), optimism (also includes hopefulness and confidence), anticipation (also includes interest and vigilance), surprise (also includes distraction and amazement) and trust (also includes acceptance, liking, and admiration). 


\paragraph{Emotion Polarity: Sentiment Valence} 
To understand the basic emotion polarity expressed in rumour and non-rumour content, we begin with sentiment valence analysis. Sentiment valence aims to capture the overall emotional tone conveyed by a post, in terms of how positive or negative it is~\citep{liu2024emosurvey}. We frame the sentiment valence task as ordinal regression~\citep{mohammad-etal-2018-semeval}. As shown in \Cref{tab:emollm_ins}, for a given tweet post, we classify it into one of seven ordinal levels of sentiment intensity, spanning varying degrees of positive and negative valence, that best represents the tweeter's mental state. The tweet posts within a thread can be divided into two categories: root tweets, which are posted by the publisher, and follow posts, which include all subsequent replies under the root post. We begin by conducting sentiment valence analysis on each post within the thread conversation. 
% TJB: confused by how comments can include all subsequent replies; we seem to be overloading the terminology, for comments to be both individual posts and series of posts
% RX: yes, I am unifiying all terms.
For each category, we compute the mean sentiment valence to enable further investigation into the specific emotions associated with different sentiment valences over a thread.
% TJB: clarify for comments whether the classification is done over the combined meta-document (i.e. the root + all comments to that point) or individually over the separate documents and then combined ... or over individual documents, in which case the statement about "all replies" needs clarification
% RX: we separate root and comments for each tweet conversation, the former is the root tweet posted by the publisher while the rest are comments. "all replies" mean all comments under root tweet, we aggregate them by computing the mean sentiment, and then average over each part.

\paragraph{Emotion Distribution} 
Following sentiment valence analysis, we then examine specific emotions and their distribution in rumour and non-rumour tweet posts.
Motivated by the fact that a certain tweet might exhibit more than one emotion, we frame the task as multi-label emotion detection problem. As shown as V-oc in \Cref{tab:emollm_ins}, given a tweet, we classify it into one of seven ordinal classes, corresponding to various levels of positive and negative sentiment intensity. To reduce noise from automatic emotion detectors, we take the top-three predicted emotions for each tweet. We then aggregate and plot the emotion distribution to provide an overview of dominant emotional trends across the rumour and non-rumour posts. Given that the follow posts make up the majority of the data compared to the root posts, we will focus on using follow posts in our next analysis.
% TJB: what is the basis of saying that the signal is richer? simply that there are more reply posts than root posts? clarify
% RX: yes, and we are more interested in interaction in comments.

\paragraph{Emotion Transitions} 
Emotions are contagious and highly interactive~\citep{Ferrara_2015}. When publishers write tweets that convey their emotions, readers are likely to respond with emotional reactions of their own~\citep{Ferrara_2015,emotion_dynamics}. In this part, we model this interactive nature of emotions in the form of emotion transition pairs, which are built from two chronologically-adjacent tweets. In each pair, the first element represents the emotion inferred from the initial content published at a given time, and the second element represents the emotion inferred from the reply content published immediately after. For example, if the first tweet exhibits \textit{joy} \textit{trust} and \textit{anticipation}, and the second tweet shows \textit{anger}, \textit{disgust} and \textit{surprise}, we form the pairs (\textit{joy}, \textit{anger}), (\textit{joy}, \textit{disgust}), (\textit{joy}, \textit{surprise}), (\textit{trust}, \textit{surprise}), (\textit{trust}, \textit{surprise}), (\textit{trust}, \textit{disgust}), (\textit{anticipation}, \textit{anger}), (\textit{anticipation}, \textit{surprise}) and (\textit{anticipation}, \textit{disgust}). We create transitions for all combinations of emotion pairs and explore the likelihood of emotion transition pairs occurring in rumour and non-rumour content. Exploring emotion transitions allows us to understand the emotional flow in social media conversations and uncover typical patterns of rumour and non-rumour content, and any differences between the two.

\paragraph{Emotion Trajectories} 
We explore the cumulative trajectory of emotion over time to observe how emotions evolve during the conversational thread. We collect all detected emotion labels for each tweet from both rumour and non-rumour content, then track cumulative emotion counts at each chronological step. Finally, we visualize these trends and apply regression models to analyze the growth of emotions over time. This temporal analysis reveals how emotions accumulate or intensify across time, offering insight into the trajectory of emotions in rumour and non-rumour content.

\begin{table*}[!h]
    \centering
    \small
    \begin{tabular}{cccccccccccc}
        \toprule
        \textbf{Setting} & \textbf{Ru} & \textbf{Non} & \textbf{p} & \textbf{\#Ru/Non} & \textbf{T} & \textbf{F} & \textbf{U} & \textbf{$p$ (U vs T)} & \textbf{$p$ (U vs F)} & \textbf{\#T/\#F/\#U} \\
        \midrule
        \textbf{PHEME root} & \textbf{$-$0.25} & $-$0.17 & 0.00 & 2602/2602 & $-$0.21 & $-$0.11 & \textbf{$-$0.39} & 7.75e-11 & 4.41e-11 & 629/629/629 \\
        \textbf{PHEME follow} & \textbf{$-$0.33} & $-$0.26 & 6.47e-09 & & $-$0.35 & $-$0.20 & \textbf{$-$0.39} & 0.03 & 8.38e-15 & \\
        \textbf{Twitter15 root} & \textbf{$-$0.26} & $-$0.01 & 3.51e-05 & 372/372 & $-$0.21 & $-$0.20 & \textbf{$-$0.34} & 0.01 & 0.01 & 359/359/359 \\
        \textbf{Twitter15 follow} & \textbf{$-$0.27} & $-$0.06 & 1.65e-09 & & $-$0.24 & $-$0.25 & \textbf{$-$0.30} & 0.16 & 0.21 & \\
        \textbf{Twitter16 root} & \textbf{$-$0.18} & \z0.07 & 0.00 & 205/205 & \z0.11 & $-$0.22 & \textbf{$-$0.30} & 1.35e-06 & 0.18 & 63/63/63 \\
        \textbf{Twitter16 follow} & \textbf{$-$0.31} & $-$0.12 & 9.19e-06 & & $-$0.30 & \textbf{$-$0.36} & $-$0.27 & 0.67 & 0.90 & \\
        % \textbf{CoAID root} & \textbf{$-$0.34} & $-$0.16 & 0.01 & 167/167 & - & - & - & - & - & - \\
        % \textbf{CoAID follow} & \textbf{$-$0.24} & $-$0.13 & 0.01 & & - & - & - & - & - & \\
        \bottomrule
    \end{tabular}
    \caption{Valence Ordinal Regression results for all datasets. root = root posts, follow = follow posts, Ru = rumour, Non = Non-rumour, T = True rumour, F = False rumour, U = Unverified rumour; $p$ values indicates significance of a one-tailed t-test.}
\label{tab:voc_results}
\end{table*}

\begin{algorithm}[ht!] 
\caption{PC Algorithm}
\label{pc}
\begin{algorithmic}[1] 
\State \textbf{Input:} Data $\mathbf{X}$, significance level $\alpha$
\State \textbf{Output:} Completed Partially Directed Acyclic Graph (CPDAG)

\State Initialize a complete undirected graph $G$ with all variables as nodes.

\State \textbf{Step 1: Skeleton Identification}
\For{each pair of variables $(X, Y)$ in $G$}
    \State Find the subset $S \subseteq \text{Adj}(X, G) \setminus \{Y\}$ such that 
    $X \indep Y \mid S$ with significance $\alpha$.
    \If{such a subset $S$ exists}
        \State Remove the edge $X - Y$ from $G$.
    \EndIf
\EndFor

\State \textbf{Step 2: Edge Orientation}
\For{each triple of variables $(X, Y, Z)$ in $G$ where $X - Z - Y$ and $X, Y$ are not adjacent}
    \If{$Z \notin S$ for all separating sets $S$ for $X$ and $Y$}
        \State Orient as $X \to Z \leftarrow Y$ (identify a collider).
    \EndIf
\EndFor

\While{possible}
    \For{each edge $(X - Y)$ in $G$}
        \If{there exists a directed path $X \to \dots \to Z$ such that $Z - Y$}
            \State Orient as $X \to Y$ (acyclicity rule).
        \ElsIf{orienting $X - Y$ as $X \to Y$ creates a new v-structure}
            \State Orient as $X \to Y$ (v-structure rule).
        \EndIf
    \EndFor
\EndWhile

\State \textbf{return} the CPDAG representing the equivalence class of causal graphs.

% how we frame the task, compute the emotion intensity, how to aggregate on conversation level

\end{algorithmic}
\end{algorithm}


\paragraph{Causal Relationship of Emotions in Rumour \& Non-Rumour Threads}
To gain a deeper insight into the relationship between rumours and the emotions underlying them, we extend our analysis beyond statistical correlation by conducting a causal analysis. Specifically, we apply the Peter-Clark (PC) algorithm \cite{Spirtes2000}, a classical constraint-based causal discovery algorithm on the three merged datasets. 

Uncovering causal relations between variables of interest is never an easy problem. Under the fundamental assumption of \textit{causal Markov condition} that a variable is conditionally independent of all its non-effects given its direct cause, \textit{faithfulness} ensures that the casual graph exactly encodes the independence and conditional independence relations among variables. These two assumptions allow us to infer causal relationships from observed statistical independencies, forming the cornerstone of constraint-based causal discovery methods. 

The PC algorithm identifies causal relationships among the variables of interest, represented as a directed acyclic graph (DAG), by numerating the independence and conditional independence relationships. The algorithm consists of two main steps: 
\begin{enumerate}
    \item \textbf{Skeleton Identification}: Starting with a complete undirected graph where all variables are connected, edges are iteratively removed based on conditional independence and independence relationships among variables, inferred by a conditional independence test. This step returns an undirected graph, which we call a skeleton. 
    \item \textbf{Edge Orientation}: After constructing the skeleton, edges are oriented by a set of predefined rules (Meek's Rule \cite{meek1997graphical}) to avoid cycles and orient collider structures.
\end{enumerate}

The complete PC algorithm is provided in algorithm \ref{pc}. It returns a  completed partially directed acyclic graph (CPDAG), which represents an equivalence class of causal graphs that are consistent with the observed data’s independence and conditional independence relations. In our implementation, we adopt the  Fisher-z test \cite{fisher_probable_1921} to infer the conditional independence relations.

%%% Local Variables:
%%% mode: latex
%%% TeX-master: "../main_anonymous"
%%% End:

 \begin{figure*}[!h]
    \centering
    \begin{subfigure}[b]{0.8\linewidth}
        \centering
        \includegraphics[width=0.45\linewidth]{images/residual/text/CIReVL_Recall5.png}
        \hfil
        \includegraphics[width=0.45\linewidth]{images/residual/text/pic2word_recall5.png}
        \caption{\textbf{PDV-T}: Impact of $\alpha$ scaling on composed text embeddings}
        \label{fig:residual_text_sub}
    \end{subfigure}
    
    \begin{subfigure}[b]{0.8\linewidth}
        \centering
        \includegraphics[width=0.45\linewidth]{images/residual/image/CIReVL_Recall5.png}
        \hfil
        \includegraphics[width=0.45\linewidth]{images/residual/image/pic2word_recall5.png}
        \caption{\textbf{PDV-I}: Impact of $\alpha$ scaling on composed image embeddings}
        \label{fig:residual_image_sub}
    \end{subfigure}
    
    \begin{subfigure}[b]{0.8\linewidth}
        \centering
        \includegraphics[width=0.45\linewidth]{images/residual/fusion/CIReVL_Recall5.png}
        \hfil
        \includegraphics[width=0.45\linewidth]{images/residual/fusion/pic2word_recall5.png}
        \caption{\textbf{PDV-F}: Impact of varying $\beta$ with on composed fused embeddings}
        \label{fig:residual_fusion_sub}
    \end{subfigure}
    \caption{Impact of changing $\alpha$/$\beta$ on Recall@5 performance across different PDV applications. For each row, results are shown for the CIReVL (left) and Pic2Word (right) baseline methods.}
    \label{fig:residual_all}
\end{figure*}

\section{Experiments} 
\label{sec:exp}
\noindent\textbf{Implementation Details.} We utilize the official implementations of four ZS-CIR baseline methods: CIReVL\footnote{https://github.com/ExplainableML/Vision\_by\_Language} and LDRE \footnote{https://github.com/yzy-bupt/LDRE} as representative caption-based feature extraction approaches and Pic2Word\footnote{https://github.com/google-research/composed\_image\_retrieval} and SEARLE\footnote{https://github.com/miccunifi/SEARLE} as representative pseudo tokenization-based methods. All feature extraction processes follow the original implementations provided by these baseline methods. However, to calculate $\Delta_{PDV}$, we need text embeddings without prompts, which are not provided in the original implementations. For CIReVL and LDRE, we obtain these embeddings by passing the generated image captions directly to CLIP. For Pic2Word and SEARL, we construct the base text embedding by passing the phrase ``a photo of $\langle$token$\rangle$" to CLIP, where $\langle$token$\rangle$ represents the extracted image token obtained via text inversion.

\noindent\textbf{Datasets and Base Vision-Language Models.} Following previous work, we evaluated our method on a suite of datasets including Fashion-IQ \cite{wu2021fashion}, CIRR \cite{liu2021image} and CIRCO \cite{baldrati2023zero}. Our proposed method is a plug-and-play approach requiring no additional training, leveraging only pre-trained models. For feature extraction, we use three CLIP variants: ViT-B/32, ViT-L/14, and ViT-G/14, and used the same pre-trained weights as used by the baseline methods. For image tokenization, we employ the pre-trained Pic2Word model. 

\subsection{Effect of Using the PDV}
We now explore the impact of the three proposed uses of the PDV: Using the PDV to augment text queries (PDV-T, see Sec. \ref{sec:exp1}), using the PDV to augment image queries (PDV-I, see Sec. \ref{sec:exp2}), and using the PDV in queries that fuse image and text data (PDV-F, see Sec. \ref{sec:exp3}).

\begin{table*}
	\footnotesize
	\centering
	\begin{tabular}{l|l|c|c|c|cccccccc}
		\hline
		\textbf{Fashion-IQ} & & & & & \multicolumn{2}{c}{\textbf{Shirt}} & \multicolumn{2}{c}{\textbf{Dress}} & \multicolumn{2}{c}{\textbf{Toptee}} & \multicolumn{2}{c}{\textbf{Average}} \\ \hline
		Backbone & Method& $\beta$ & $\alpha_{I}$& $\alpha_{T}$ & R@10 & R@50 & R@10 & R@50 & R@10 & R@50 & R@10 & R@50 \\
		\hline
		\multirow{6}{*}{ViT-B/32} %
		& SEARLE & - & - & - & 24.14 & 41.81 & 18.39 & 38.08 & 25.91 & 47.02 & 22.81 & 42.30 \\
		& SEARLE + \textbf{PDV-F} & 0.9 & 1.1 & 0.9 & \hli{24.83} & 41.71 & \hli{20.13} & \hli{41.40} & \hli{25.96} & \hli{47.17}  & \hli{23.64} & \hli{43.43} \\
		& CIReVL \textdagger &- & -& -& 28.36 & 47.84 & 25.29 & 46.36 & 31.21 & 53.85 & 28.29 & 49.35 \\
		& CIReVL + \textbf{PDV-F} & 0.75 & 1.4 & 1.4 & \hlb{32.88} & \hlb{52.80} & \hlb{32.67} & \hlb{54.49} & \hlb{38.91} & \hlb{61.81} & \hlb{34.82} & \hlb{56.37} \\
		& LDRE \textdagger & - & - & - & 27.38 & 46.27 & 19.97 & 41.84 & 27.07 & 48.78 & 24.81 & 45.63 \\
		& SEIZE \textdagger & - & - & - & \underline{29.38} & \underline{47.97} & \underline{25.37} & \underline{46.84} & \underline{32.07} & \underline{54.78} & \underline{28.94} & \underline{49.86} \\
		\hline
		\multirow{8}{*}{ViT-L/14} & Pic2Word & & & & 25.96 & 43.52 & 19.63 & 40.90 & 27.28 & 47.83 & 24.29 & 44.08 \\
		& Pic2Word + \textbf{PV-F} & 0.8 & 1.0 & 1.0 & \hli{28.21} & \hli{44.55} & \hli{20.92} & \hli{42.24} & \hli{29.02} & \hli{48.90}& \hli{26.05} & \hli{45.23}\\
		& SEARLE & - & - & - & 26.84 & 45.19 & 20.08 & 42.19 & 28.40 & 49.62 & 25.11 & 45.67 \\
		& SEARLE +\textbf{PDV-F} & 0.8 & 1.2 & 1.0 & \hli{28.66} & \hli{46.76} & \hli{23.60} & \hli{46.41} & \hli{31.00} & \hli{52.32} & \hli{27.75} & \hli{48.50} \\
		& CIReVL \textdagger & & & & 29.49 & 47.40 & 24.79 & 44.76 & 31.36 & 53.65 & 28.55 & 48.57 \\
		
		& CIReVL + \textbf{PDV-F} & 0.55 & 1 & 1.3 & \hlb{37.78} & \hlb{54.22} & \hlb{33.61} & \hlb{56.07} & \hlb{41.61} & \hlb{62.16} & \hlb{37.67} & \hlb{57.48} \\
		& LinCIR & - & - & - & 29.10 & 46.81 & 20.92 & 42.44 & 28.81 & 50.18 & 26.82 & 46.49 \\
        & SEIZE & -& -& -& \underline{33.04} & \underline{53.22} & \underline{30.93} & \underline{50.76} & \underline{35.57} & \underline{58.64} & \underline{33.18} & \underline{54.21} \\
		\hline
        \multirow{6}{*}{ViT-G/14} & Pic2Word  & - & - & - & 33.17 & 50.39 & 25.43 & 47.65 & 35.24 & 57.62 & 31.28 & 51.89\\
         & SEARLE  & - & - & - & 36.46 & 55.35 & 28.16 & 50.32 & 39.83 & 61.45 & 34.81 & 55.71\\
		  & CIReVL \textdagger & -& -& -& 33.71 & 51.42 & 27.07 & 49.53 & 35.80 & 56.14 & 32.19 & 52.36 \\
		& CIReVL + \textbf{PV-F} & 0.6 & 1.4 & 1.4 & \hli{41.90} & \hli{58.19} & \hlb{40.70} & \hlb{62.82} & \underline{\hli{48.09}}& \hli{67.77}& \underline{\hli{43.56}}& \hli{62.93}\\
        & LinCIR & - & - & - & \textbf{46.76} & \underline{65.11} & 38.08& 60.88& \textbf{50.48}& \underline{71.09}& \textbf{45.11} & \underline{65.69}\\
        & SEIZE & - & - & - & \underline{43.60} & \textbf{65.42}& \underline{39.61} & \underline{61.02} & 45.94& \textbf{71.12}& 43.05& \textbf{65.85}\\
		\hline
	\end{tabular}
	\caption{Average recall for different methods on Fashion-IQ validation dataset. \textdagger~denotes that numbers are taken from the original paper.}
	\label{tab:fashion_iq_results}
\end{table*}


\begin{table*}
	\centering
	\footnotesize
	\setlength{\tabcolsep}{4pt}
	\begin{tabular}{ll|c|c|c|cccc|cccc|ccc}
		\hline
		\multicolumn{2}{c|}{\textbf{Dataset}} & & & &  \multicolumn{4}{c|}{\textbf{CIRCO}} & \multicolumn{7}{c}{\textbf{CIRR}} \\
		\hline
		\multicolumn{2}{c|}{Metric} & & & & \multicolumn{4}{c|}{mAP@k} & \multicolumn{4}{c|}{Recall@k} &\multicolumn{3}{c}{$R_s$@k} \\
		\cline{3-16}
		Arch & Method & $\beta$ & $\alpha_I$ & $\alpha_T$ & k=5 & k=10 & k=25 & k=50 & k=1 & k=5 & k=10 & k=50 & k=1 & k=2 & k=3 \\
		\hline
		\multirow{8}{*}{ViT-B/32} 
		& PALAVRA\cite{cohen2022my} \textdagger & -& -& -& 4.61 & 5.32 & 6.33 & 6.80 & 16.62 & 43.49 & 58.51 & 83.95 & 41.61 & 65.30 & 80.94 \\
		& SEARLE \textdagger & -& -&- & 9.35 & 9.94 & 11.13 & 11.84 & 24.00 & 53.42 & 66.82 
		& 89.78 & 54.89 & 76.60 & 88.19 \\
		& SEARLE + \textbf{PDV-F} & 0.9 & 1.4 & 1.2 & \hli{9.99} & \hli{10.50}  & \hli{11.70} & \hli{12.40} & \hli{24.53} & \hli{53.71} & \hli{67.33} & \hli{89.81} & \hli{56.94} & \hli{78.05} & \hli{88.99} \\
		&CIReVL \textdagger & - & - & -& 14.94 & 15.42 & 17.00 & 17.82 & 23.94 & 52.51 & 66.00 & 86.95 & 60.17 & 80.05 & 90.19 \\
		& CIReVL + \textbf{PDV-F} & 0.75 & 1.4 & 1.2 & \hlb{19.90} & \hlb{20.61} & \hlb{22.64} & \hlb{23.52} & \hlb{33.25} & \hlb{64.15} & \hlb{75.23} & \hlb{92.43} & \hlb{65.81} &\underline{\hli{83.76}} &\underline{\hli{92.10}} \\
		& LDRE & -& -& -& 17.81 & 18.04 & 19.73 & 20.67 & 25.69 & 55.52 & 68.77 & 89.86 & 60.10 & 80.58 & 91.04 \\
		& LDRE + \textbf{PDV-F} & 0.75 & 1.4 & 1.4 & \hli{17.80} & \hli{18.78} & \hli{20.61} & \hli{21.56} & \underline{\hli{29.30}} & \underline{\hli{60.39}} & \underline{\hli{72.51}} & \underline{\hli{91.42}} & \hli{63.06} & \hli{82.36} & \hli{91.54} \\
        & SEIZE & -&- &- & \underline{19.04} & \underline{19.64} & \underline{21.55}& \underline{22.49}& 27.47 & 57.42& 70.17 & - & \underline{65.59} & \textbf{84.48}& \textbf{92.77} \\
 		\hline
		\multirow{10}{*}{ViT-L/14}
		& Pic2Word & -& -& -& 6.81 & 7.49 & 8.51 & 9.07 & 23.69 & 51.32 & 63.66 & 86.21 & 53.61 & 74.34 & 87.28 \\
		& Pic2Word + \textbf{PDV-F} & 0.85 & 1.2 & 1.0 & \hli{7.74} &  \hli{8.67} & \hli{9.77} & \hli{10.37} & \hli{23.90} & \hli{51.95} & \hli{64.63} & \hli{87.04} & \hli{53.16}  & \hli{74.07} & \hli{87.08}\\
		& SEARLE \textdagger & - & - & - & 11.68 & 12.73 & 14.33 & 15.12 & 24.24 & 52.48 & 66.29 & 88.84 & 53.76 & 75.01 & 88.19 \\
		& SEARLE + \textbf{PDV-F} & 0.85 & 1.4 & 1.2 & \hli{12.58} & \hli{13.57} & \hli{15.30} & \hli{16.07} & \hli{25.64} & \hli{53.61} & \hli{66.58} & \hli{88.55} & \hli{55.83} & \hli{76.48} & \hli{88.53} \\
		& CIReVL \textdagger & -& -& -& 18.57 & 19.01 & 20.89 & 21.80 & 24.55 & 52.31 & 64.92 & 86.34 & 59.54 & 79.88 & 89.69 \\
		& CIReVL + \textbf{PDV-F} & 0.75 & 1.4 & 1.2 & \hlb{25.67} & \hlb{26.61} & \underline{\hli{28.81}} & \hlb{29.95} & \hlb{36.24} & \hlb{66.17} & \hlb{76.96} & \hlb{92.29} & \hlb{68.07} & \hlb{85.35} & \hlb{93.47} \\
		& LDRE & -& -& -& 22.32 & 23.75 & 25.97 & 27.03 & 26.68 &55.45  & 67.49 & 88.65 & 60.39 & 80.53 & 90.15 \\
		& LDRE + \textbf{PDV-F} & 0.75 & 1.4 & 1.4 & \hli{25.23} & \hli{26.52} & \hlb{28.94} & \hlb{29.95} & \underline{\hli{30.16}} & \underline{\hli{59.98}} & \underline{\hli{71.90}} & \underline{\hli{90.87}} & \hli{63.66} & \hli{82.87} & \hli{91.57} \\

        & LinCIR & - & - & - &12.59 &13.58 &15.00 &15.85 &25.04 &53.25 &66.68 & - &57.11 &77.37 &88.89\\
        & SEIZE & -& -& -& 24.98 & 25.82 &28.24 &\underline{29.35}& 28.65 &57.16& 69.23& - &\underline{66.22} &\underline{84.05} &\underline{92.34} \\
        

        
		\hline
		\multirow{7}{*}{ViT-G/14} & CIReVL \textdagger & -& -& -& 26.77 & 27.59 & 29.96 & 31.03 & 34.65 & 64.29 & 75.06 & 91.66 & 67.95 & 84.87 & 93.21 \\

		& CIReVL + \textbf{PDV-F} & 0.75 & 1.4 & 1.2 & \hli{30.02} & \hli{31.46} & \hli{34.01} & \hli{35.08} & \hli{38.15} &\hli{67.93} & \hli{77.90} & \hli{92.77} & \hli{69.37} & \hli{85.37} & \hli{93.45}  \\
		
		& LDRE & -& -& -& \underline{33.30} & \underline{34.32} & \underline{37.17} & \underline{38.27} & 37.40 & 66.96 & 78.17 & 93.66 & 68.84 & 85.64 & 93.90 \\
		& LDRE + \textbf{PDV-F} & 0.75 & 1.4 & 1.4 & \hlb{34.88} & \hlb{36.41} & \hlb{39.12} & \hlb{40.23} & \hlb{42.51} & \hlb{72.22} & \hlb{81.71} & \hlb{94.94} & \underline{\hli{72.39}} & \underline{\hli{88.34}} & \underline{\hli{94.80}} \\
        & SEARLE & - & - & - & 13.20 &13.85 &15.32 &16.04 & 34.80 & 64.07 & 75.11 &-&68.72 &84.70 &93.23 \\
        & LinCIR & - & - & - & 19.71 &21.01 &23.13 &24.18 &35.25 &64.72 &76.05 & - &63.35 &82.22 &91.98 \\
        & SEIZE & -& -& -& 32.46 & 33.77 &36.46 &37.55 &\underline{38.87} & \underline{69.42} & \underline{79.42} & -&\textbf{74.15} & \textbf{89.23} & \textbf{95.71} \\
		\hline
	\end{tabular}
	\caption{Performance comparison on CIRCO and CIRR test datasets. As in previous works, for CIRCO, mAP@k is reported, while for CIRR both Recall@k and $R_s$@k metrics are used. \textdagger~denotes that numbers are taken from the original paper.}
	\label{tab:circo_cirr_results}
\end{table*}

\noindent{\textbf{Analysing the PDV for Text (PDV-T)}}
\label{sec:exp1}
To investigate how scaling the prompt vector, $\Delta_{PDV}$, affects retrieval performance with composed text embeddings, we conducted experiments using two zero-shot approaches (CIReVL and Pic2Word) with different backbone networks across three datasets. We evaluated the performance by varying the scaling parameter, $\alpha$ (Eq. \ref{eqn:text_embedding}), from -0.5 to 3 by an interval of 0.1.

The results are presented in Figure \ref{fig:residual_text_sub}. To account for scale variations across different experiments, we report relative recall values, where a baseline of zero is established at $\alpha=1$. As shown in Figure \ref{fig:residual_text_sub}, varying $\alpha$ leads to significant changes in relative recall performance\footnote{See supplementary material for Recall@10 and Recall@50 figures}. Our analysis reveals method-specific patterns across datasets. With CIReVL, increasing $\alpha$ improves relative recall on both FashionIQ and CIRCO datasets. In contrast, Pic2Word shows no significant improvement on FashionIQ and CIRR when varying $\alpha$, while CIRCO's performance improves when $\alpha$ is reduced to 0.8-1.0. This divergent behavior is fundamentally linked to each method's ability to generate an accurate $\Delta_{PDV}$. As demonstrated in Tables \ref{tab:fashion_iq_results} and \ref{tab:circo_cirr_results}, CIReVL consistently outperforms Pic2Word across various benchmarks, indicating its superior ability to generate a more accuraute composed query, and thus a more accurate $\Delta_{PDV}$. Consequently, increasing $\alpha$ yields greater benefits for CIReVL compared to Pic2Word.

We visualize the top-5 retrieval results using CIReVL with a ViT-B-32 backbone across three datasets (one reference image from each) under varying $\alpha$ values, as shown in Figure \ref{fig:residual_qual}\red{a}. As $\alpha$ increases, the retrieved results show stronger alignment with the prompt. Conversely, when $\alpha$ exceeds 1, the results include semantically related but unseen variations, while $\alpha$ values below 0.5 yields results opposite to the prompt's intent. For instance, ``brighter blue and sleeveless" retrieves ``dark blue with sleeves," ``plain background" yields ``natural/dark background," and ``young boy" returns ``adult" images.





\noindent{\textbf{Analysing the PDV for Image (PDV-I)}}
\label{sec:exp2}
To evaluate whether $\Delta_{PDV}$ enhances the retrieval performance of image embeddings, we conducted experiments following the protocol described in Section~\ref{sec:exp1}. We modified image embeddings by adding $\Delta_{PDV}$ scaled with $\alpha$ values ranging from -0.5 to 2.0, where $\alpha=0$ represents the original image-only embeddings. As shown in Figure \ref{fig:residual_image_sub}, Recall@K exhibits a positive correlation with $\alpha$ for values below 1. This upward trend continues until $\alpha=2.0$ for CIReVL, while Pic2Word's performance peaks when $\alpha$ reaches 1.4.  The performance of PDV-I was evaluated on the CIRR and CIRCO datasets by comparing it with other visual embedding-based methods, as detailed in Table \ref{tab:circo_cirr_results_pdv-I}. The results reveal that PDV-I achieved marginal improvements over existing approaches.

Following the methodology in Section~\ref{sec:exp1}, we conduct similar visualizations, with results shown in Figure \ref{fig:residual_qual}\red{b}. As with PDV-T, increasing $\alpha$ leads to stronger alignment between retrieved results and the prompt. When $\alpha$ exceeds 0.5, the results exhibit semantic relationships to the query, while $\alpha$ values below 0.5 yield results opposing the prompt's intent.
Notably, PDV-I's top retrievals demonstrate higher visual similarity to reference images compared to PDV-F, as evidenced by the preserved design elements in the clothing item (left) and laptop (middle). This characteristic is particularly valuable for applications include fashion search \cite{wu2021fashion} and logo retrieval \cite{tursun2019component}, where visual similarity plays a crucial role.



\begin{figure*}[!tbh]
	\centering
	\includegraphics[width=0.825\linewidth]{images/qualitative/PV_qual_all_mini.pdf}
	\caption{Visualisation of the impact of $\alpha$/$\beta$ scaling on top-5 retrieval results. CIReVL with ViT-B-32 Clip model is the baseline method used. Representative examples with prompts from three datasets: FashionIQ (left), CIRR (middle), and CIRCO (right) are shown at the top. \textbf{\textcolor{boxgreen}{Green}} and \textbf{\textcolor{boxblue}{blue}} bounding boxes indicate true positives and near-true positives, respectively.}
	\label{fig:residual_qual}
	
\end{figure*}

\noindent{\textbf{Analysing PDV Fusion (PDV-F)}}
\label{sec:exp3}
Finally, we evaluate the effectiveness of fusing image and text-composed embeddings by varying the fusion parameter, $\beta$, from 0 to 1 while maintaining $\alpha=1$
for both PDV-I and PDV-F. At $\beta=0$, the model relies solely on composed image embeddings, while at $\beta=1$, it uses only composed text embeddings. As shown in Figure \ref{fig:residual_fusion_sub}, the fusion of both embeddings consistently outperforms using either embedding type alone. Optimal retrieval performance is typically achieved when $\beta$ is between 0.4 and 0.8.

We similarly visualize the top-5 retrieved results across different $\beta$ values. As shown in Figure \ref{fig:residual_qual}\red{c}, when $\beta$ is small, the retrieved results maintain high visual similarity to the reference image. Conversely, as $\beta$ exceeds 0.5, the results demonstrate stronger semantic alignment with the prompt.



\subsection{ZS-CIR Benchmark Comparison}






\begin{table*}
	\centering
	\footnotesize
	\setlength{\tabcolsep}{4pt}
	\begin{tabular}{l|l|c|cccc|cccc|ccc}
		\hline
		\multicolumn{2}{c|}{\textbf{Dataset}} & & \multicolumn{4}{c|}{\textbf{CIRCO}} & \multicolumn{7}{c}{\textbf{CIRR}} \\
		\hline
		& Metric & & \multicolumn{4}{c|}{mAP@k} & \multicolumn{4}{c|}{Recall@k} & \multicolumn{3}{c}{$R_s$@k} \\
		\cline{2-14}
		Arch & Method & $\alpha_I$ & k=5 & k=10 & k=25 & k=50 & k=1 & k=5 & k=10 & k=50 & k=1 & k=2 & k=3 \\
		\hline
		\multirow{6}{*}{ViT-B/32} 
		& Image-only \textdagger & - & 1.34 & 1.60 & 2.12 & 2.41 & 6.89 & 22.99 & 33.68 & 59.23 & 21.04 & 41.04 & 60.31 \\
		& Text-only \textdagger & - & 2.56 & 2.67 & 2.98 & 3.18 & 21.81 & 45.22 & 57.42 & 81.01 & 62.24 & 81.13 & 90.70 \\
		& Image + Text \textdagger & - & 2.65 & 3.25 & 4.14 & 4.54 & 11.71 & 35.06 & 48.94 & 77.49 & 32.77 & 56.89 & 74.96 \\
		& SEARLE + \textbf{PDV-I} & 1.5 & 4.77 & 5.23  & 6.31 & 6.82 & 16.65 & 42.53 & 55.16 & 81.42 & 44.68 & 67.78 & 82.94\\
		& CIReVL + \textbf{PDV-I} & 2.0 & \textbf{10.29 }& \textbf{10.80} & \textbf{12.23} & \textbf{12.93} & \textbf{27.18} & \textbf{56.53} & \textbf{67.76} & \textbf{87.64} & \textbf{59.81} & \textbf{79.59} & \textbf{90.15}\\
		& LDRE + \textbf{PDV-I} & 2.0 & 8.00 & 8.88 & 10.06 & 10.72 & 23.37 & 51.21 & 63.69 & 85.57 & 55.57 & 76.63 & 88.15\\
		\hline
	\end{tabular}
	\caption{PDV-I performance on CIRCO and CIRR test datasets. Note that the image-only approach utilizes the visual embedding of the reference image, whereas the text-only approach employs the text embedding of the prompt.}
	\label{tab:circo_cirr_results_pdv-I}
\end{table*}

We evaluated PDV-F alongside four baseline approaches (CIReVL, LDRE, Pic2Word, and SEARLE) across three benchmarks. Notably, CIReVL was tested with three different backbones on three datasets, as its models and intermediate results are publicly available. However, for the remaining methods, we conducted partial evaluations due to limited open-source availability or restricted support.

The numerical results are presented in Tables \ref{tab:fashion_iq_results} and \ref{tab:circo_cirr_results}.
On the FashionIQ benchmark, PDV-F yields substantial improvements for all baseline approaches, with CIReVL showing particularly strong gains that scale with backbone size. Similarly, all methods demonstrate significant performance improvements on CIRCO and CIRR datasets. Notably, CIReVL achieves larger improvements compared to other methods, with the most substantial gains observed when using small and medium backbone architectures. Our PDV-F implementation within the CIReVL framework consistently outperformed other state-of-the-art methods, including LinCIR and SEIZE, across most evaluation metrics. Similar to SEIZE, PDV-F offers the advantage of being entirely training-free; however, unlike SEIZE, it does not significantly increase feature extraction computational costs. While LinCIR demonstrates exceptional inference speed, it lacks the training-free nature of our approach, requiring dedicated model training before deployment.  





This work identifies signal collapse as a critical bottleneck in one-shot neural network pruning. Performance loss in pruned networks is due to \textbf{signal collapse} in addition to the removal of critical parameters. We propose \textbf{REFLOW} (\textbf{Re}storing \textbf{F}low of \textbf{Low}-variance signals), a simple yet effective method that mitigates signal collapse without computationally expensive weight updates. By focusing on signal preservation, REFLOW highlights the importance of mitigating signal collapse in sparse networks and enables magnitude pruning to match or surpass state-of-the-art one-shot pruning methods such as CHITA, CBS, and WF.

REFLOW consistently achieves state-of-the-art accuracy across diverse architectures, restoring ResNeXt-101 from under 4.1\% to 78.9\% top-1 accuracy at 80\% sparsity on ImageNet. Its lightweight design makes it a practical solution for both research and deployment, delivering high-quality sparse models without the overhead of traditional approaches. These findings challenge the traditional emphasis on weight selection strategies and underscore the critical role of signal propagation for achieving high-quality sparse networks in the context of one-shot pruning.



\section*{Conclusion}
This paper aims to enhance our understanding of the computational complexity of computing various Shapley value variants. We found that for various ML models --- including decision trees, regression tree ensembles, weighted automata, and linear regression --- both local and global interventional and baseline SHAP can be computed in polynomial time under HMM modeled distributions. This extends popular algorithms, such as TreeSHAP, beyond their empirical distributional scope. We also establish strict complexity gaps between the various SHAP variants (baseline, interventional, and conditional) and prove the intractability of computing SHAP for tree ensembles and neural networks in simplified scenarios. Overall, we present SHAP as a versatile framework whose complexity depends on four key factors: \begin{inparaenum}[(i)] \item model type, \item SHAP variant, \item distribution modeling approach, \item and local vs. global explanations\end{inparaenum}. We believe this perspective provides deeper insight into the computational complexity of SHAP, paving the way for future work.




%We believe that our framework provides a more intricate understanding of SHAP computation complexity across different models, distributions, and variants, paving the way for further research.

Our work opens promising directions for future research. First, expanding our computational analysis to other SHAP-related metrics, such as asymmetric SHAP~\citep{frye20} and SAGE~\citep{covert2020understanding}, would be valuable. Additionally, we aim to explore more expressive distribution classes and relaxed assumptions beyond those in Section \ref{sec:tractable} while maintaining tractable SHAP computation. Finally, when exact computation is intractable (Section \ref{sec:intractable}), investigating the approximability of SHAP metrics through approximation and parameterized complexity theory~\citep{downey2012parameterized} is an important direction.

%Our work opens several promising avenues for future research on the computational properties of explainable AI methods, with a particular focus on SHAP. First, it would be interesting to broaden the computational analysis conducted in this work to include other popular SHAP-related metrics in the literature, such as asymmetric SHAP \cite{frye20} and SAGE \cite{covert2020understanding}. Also, in the future, we aim to explore more expressive distribution classes and relaxed distributional assumptions—extending beyond those examined in Section \ref{sec:tractable} —that still yield tractable SHAP computation. Finally, when exact computation proves intractable (Section \ref{sec:intractable}), it is worthwhile to theoretically investigate the question of the approximability of computing the SHAP metrics across various configurations, through the lens of approximation and parametrized complexity theory \cite{arora2009computational}.

%This paper aims to deepen our understanding of the computational complexity involved in obtaining different Shapley value variants. We found that for a variety of ML models, including decision trees, tree ensembles for regression, weighted automata, and linear regression models — computing both local and global interventional and baseline SHAP can be done in polynomial time when distributions are modeled by HMMs. This extends the distributional scope of popular algorithms like TreeSHAP, which is limited to empirical distributions. Additionally, we demonstrate a strict complexity gap between SHAP variants, showing that interventional and baseline SHAP can be strictly easier to compute than conditional SHAP. Despite these positive results, we uncovered intractability for various SHAP variants in neural networks and tree ensembles. Finally, we provided generalized complexity relations across SHAP variants. We believe that our framework offers a deeper understanding of the complexity involved in computing SHAP across various variants, models, distributions, as well as in both local and global computations, laying the groundwork for future research.



\bibliographystyle{ieeetr}
\bibliography{reference}


\end{document}
