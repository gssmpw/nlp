\RequirePackage{xcolor}
\documentclass[journal,twoside,web]{ieeecolor}
% \documentclass[]{article}
\makeatletter
\let\NAT@parse\undefined
\makeatother
\usepackage[colorlinks=true]{hyperref} % Load hyperref BEFORE cite (important!)
\usepackage{cite}     % Load cite AFTER hyperref

\usepackage[pdftex]{graphicx}
\usepackage{generic}
% \usepackage{cite}
\usepackage{amssymb,amsfonts}
\usepackage{amsthm}
\setlength{\marginparwidth}{1.3cm}
% \usepackage{hyperref}
% \usepackage[colorlinks]{hyperref}

\usepackage{todonotes} % [disable]
\usepackage[inkscapearea=page]{svg}

\usepackage{mathtools}
\usepackage{hyperref}
\usepackage{amssymb}
\usepackage{bm}
\usepackage{graphicx}
\usepackage{float}
\usepackage{amsmath}
% \usepackage{algpseudocode}
% \usepackage{algpseudocode}
\usepackage{braket}
\usepackage{tikz}
\usepackage{bbm}
\usepackage{multirow}
\usetikzlibrary{positioning, backgrounds, fit, shapes.arrows}
\usetikzlibrary{decorations.markings}
\usepackage{amsthm}
\usepackage{cuted}
\usepackage{enumitem}
\usepackage{caption}
\usepackage{subcaption}
\newtheorem{definition}{Definition}[section]
\newtheorem{theorem}{Theorem}[section]
\newtheorem{corollary}{Corollary}[theorem]
\newtheorem{lemma}[theorem]{Lemma}
\newtheorem{prop}{Proposition}
\newtheorem*{problem}{Problem Statement}
\newtheorem{assumption}{Assumption}[section]
% \usepackage[table]{xcolor}
% \usepackage{booktabs}
% \usepackage{multirow}
% \usepackage{tcolorbox}
% \usepackage{bbding}
% \usepackage{amssymb}% http://ctan.org/pkg/amssymb
\usepackage{pifont}% http://ctan.org/pkg/pifont
%%%%%%%%%%%---SETME-----%%%%%%%%%%%%%
%replace @@ with the submission number submission site.
\newcommand{\thiswork}{INF$^2$\xspace}
%%%%%%%%%%%%%%%%%%%%%%%%%%%%%%%%%%%%


%\newcommand{\rev}[1]{{\color{olivegreen}#1}}
\newcommand{\rev}[1]{{#1}}


\newcommand{\JL}[1]{{\color{cyan}[\textbf{\sc JLee}: \textit{#1}]}}
\newcommand{\JW}[1]{{\color{orange}[\textbf{\sc JJung}: \textit{#1}]}}
\newcommand{\JY}[1]{{\color{blue(ncs)}[\textbf{\sc JSong}: \textit{#1}]}}
\newcommand{\HS}[1]{{\color{magenta}[\textbf{\sc HJang}: \textit{#1}]}}
\newcommand{\CS}[1]{{\color{navy}[\textbf{\sc CShin}: \textit{#1}]}}
\newcommand{\SN}[1]{{\color{olive}[\textbf{\sc SNoh}: \textit{#1}]}}

%\def\final{}   % uncomment this for the submission version
\ifdefined\final
\renewcommand{\JL}[1]{}
\renewcommand{\JW}[1]{}
\renewcommand{\JY}[1]{}
\renewcommand{\HS}[1]{}
\renewcommand{\CS}[1]{}
\renewcommand{\SN}[1]{}
\fi

%%% Notion for baseline approaches %%% 
\newcommand{\baseline}{offloading-based batched inference\xspace}
\newcommand{\Baseline}{Offloading-based batched inference\xspace}


\newcommand{\ans}{attention-near storage\xspace}
\newcommand{\Ans}{Attention-near storage\xspace}
\newcommand{\ANS}{Attention-Near Storage\xspace}

\newcommand{\wb}{delayed KV cache writeback\xspace}
\newcommand{\Wb}{Delayed KV cache writeback\xspace}
\newcommand{\WB}{Delayed KV Cache Writeback\xspace}

\newcommand{\xcache}{X-cache\xspace}
\newcommand{\XCACHE}{X-Cache\xspace}


%%% Notions for our methods %%%
\newcommand{\schemea}{\textbf{Expanding supported maximum sequence length with optimized performance}\xspace}
\newcommand{\Schemea}{\textbf{Expanding supported maximum sequence length with optimized performance}\xspace}

\newcommand{\schemeb}{\textbf{Optimizing the storage device performance}\xspace}
\newcommand{\Schemeb}{\textbf{Optimizing the storage device performance}\xspace}

\newcommand{\schemec}{\textbf{Orthogonally supporting Compression Techniques}\xspace}
\newcommand{\Schemec}{\textbf{Orthogonally supporting Compression Techniques}\xspace}



% Circular numbers
\usepackage{tikz}
\newcommand*\circled[1]{\tikz[baseline=(char.base)]{
            \node[shape=circle,draw,inner sep=0.4pt] (char) {#1};}}

\newcommand*\bcircled[1]{\tikz[baseline=(char.base)]{
            \node[shape=circle,draw,inner sep=0.4pt, fill=black, text=white] (char) {#1};}}

% referencing objects
\newcommand{\figref}[1]{fig.\,\ref{fig:#1}}
\newcommand{\Figref}[1]{Fig.\,\ref{fig:#1}}
\newcommand{\tabref}[1]{tab.\,\ref{tab:#1}}
\newcommand{\Tabref}[1]{Tab.\,\ref{tab:#1}}
\newcommand{\secref}[1]{sec.\,\ref{sec:#1}}
\newcommand{\Secref}[1]{Sec.\,\ref{sec:#1}}

% complexity classes
\newcommand{\nc}[0]{NC$^1$\xspace}
\newcommand{\tc}[0]{TC$^0$\xspace}
\newcommand{\ac}[0]{AC$^0$\xspace}
\newcommand{\AND}[0]{\texttt{AND}}
\newcommand{\OR}[0]{\texttt{OR}}
\newcommand{\NOT}[0]{\texttt{NOT}}
\newcommand{\MAJ}[0]{\texttt{MAJORITY}}

% math
\newcommand{\expect}[1]{\mathbb{E}\left[ #1 \right]}
\newcommand{\transpose}{^\intercal}
\newcommand{\func}[1]{\left( #1 \right)}
\newcommand{\step}[1]{^{(#1)}}
\newcommand{\R}[1]{\mathbb{R}^{#1}}
\newcommand{\C}[1]{\mathbb{C}^{#1}}
\newcommand{\Order}[1]{\mathcal{O}\left(#1\right)}
\newcommand{\fp}{^{*}}
\newcommand{\iter}[1]{^{\left( #1 \right)}}
\newcommand{\loss}{\mathcal{L}}
\newcommand{\partialdiff}[3]{\left. \frac{\partial #1}{\partial #2}\right|_{#3}}
\newcommand{\distribution}[1]{$\mathcal{D}_{#1}$}

% variables
\newcommand{\bfM}{\mathbf{M}}
\newcommand{\bfX}{\mathbf{X}}
\newcommand{\bfZ}{\mathbf{Z}}
\usepackage{nameref}
% \usepackage{color-edits} % [suppress]
% \addauthor[Yoram]{yb}{red}
% \addauthor[Hai]{hvh}{blue}
% \addauthor[Hagit]{hmy}{green}




\begin{document}
\bstctlcite{bstctl:etal, bstctl:nodash, bstctl:simpurl}

\pgfdeclarelayer{background}
\pgfdeclarelayer{background1}
\pgfsetlayers{background1,background,main}

\title{Learned Bayesian Cram\'er-Rao Bound \\
for Unknown Measurement Models \\ {Using} Score Neural Networks}
\author{Hai Victor Habi, Hagit~Messer,~Life~Fellow,~IEEE and Yoram~Bresler,~Life~Fellow,~IEEE
\thanks{H.V. Habi and H. Meseer are with the School of Electrical Engineering, Tel Aviv University, Tel Aviv 6139001, Israel (e-mail: haivictorh@mail.tau.ac.il; messer@eng.tau.ac.il).}
\thanks{Y.Bresler is with the Department of Electrical and Computer Engineering and the Coordinated Science Lab, University of Illinois Urbana-Champaign, USA (e-mail: ybresler@illinois.edu).
}}
\date{}

\markboth{February 2025}%
{Habi, Messer, and Bresler: Learned Bayesian Cram\'er-Rao Bound for Unknown Measurement Models  using Score Neural Networks}

\maketitle

\begin{abstract}
    The Bayesian Cramér-Rao bound (BCRB) is a crucial tool in signal processing for assessing the fundamental limitations of any estimation problem as well as benchmarking within a Bayesian frameworks. However, the BCRB cannot be computed without full knowledge of the prior and the measurement distributions. In this work, we propose a fully learned Bayesian Cramér-Rao bound (LBCRB) that learns both the prior and the measurement distributions. Specifically, we suggest two approaches to obtain the LBCRB: the Posterior Approach and the Measurement-Prior Approach. The Posterior Approach provides a simple method to obtain the LBCRB, whereas the Measurement-Prior Approach enables us to incorporate domain knowledge to improve the sample complexity and {interpretability}. To achieve this, we introduce a Physics-encoded score neural network which enables us to easily incorporate such domain knowledge into a neural network. We {study the learning} errors of the two suggested approaches theoretically, and  validate them numerically. We demonstrate the two approaches on several signal processing examples, including a linear measurement problem with unknown mixing and Gaussian noise covariance matrices, frequency estimation, and quantized measurement. In addition, we test our approach on a nonlinear signal processing problem of frequency estimation with real-world underwater ambient noise.
\end{abstract}
\raggedbottom

\begin{IEEEkeywords}
 Score Matching, Bayesian-CRB, Parameter Estimation, Bayesian Fisher information, \pe{}.
\end{IEEEkeywords}

\section{Introduction}
The Bayesian Cramér-Rao Bound (BCRB) \cite{van2004detection} is a crucial tool in signal processing for {assessing} the fundamental limitations of any estimation problem within a Bayesian framework. For instance, the BCRB has been employed to elucidate various signal processing applications such as localization \cite{10184105,10140073} and MIMO systems \cite{nasir2013phase}, among others \cite{xu2004bayesian,rosentha2024asymptotically,mazor2024limitations}. Besides understanding the intrinsic limits of problems, the BCRB has also been used for system design, including waveform design \cite{huleihel2013optimal,turlapaty2014bayesian,zuo2010conditional,sun2024optimal}. 

However, to obtain the BCRB requires \emph{complete knowledge} of both the prior and the measurement distributions. In addition, in  some cases, even when both prior and measurement distributions are known, the BCRB cannot be computed, because the required integration over the parameter distribution does not have a close-form solution.  

%\todo[inline,color=green]{Break into paragraphs, for readability}
Various methods have been proposed %methods 
to derive a bound from data, {rather than analytically.} Some leverage prior knowledge of the problem combined with a learnable component. For instance, \cite{lutati22_interspeech} presents a non-Bayesian bound for single-channel speech separation. Another category of methods, as those described in \cite{duy2022fisher,6975144}, proposes bounds based on f-divergences. However, these methods require access to a {special} dataset {containing observations where for each parameter value there are also observations with a slightly perturbed parameter value.} 
% \todo[color=green]{What's that? HVH: It should be pertubation} 
This condition is generally viable only when one can generate an observation vector {for any desired value of the parameter vector that one aims to estimate}. 


Recently, thanks to the success of generative models in modeling complex, high-dimensional data distributions\cite{song2019generative,kobyzev2020normalizing}, a new approach has been introduced that suggested using a learned Generative Cram\'er Rao bound \cite{habi2023learning} when the measurement distribution is \emph{completely unknown,} but a dataset of independent and identically distributed  (i.i.d)  measurement-parameter pairs is available. The  Generative Cram\'er Rao bound \cite{habi2023learning} achieves this by first learning the measurement distribution using a generative model, and then utilizing it to obtain the learned Generative Cram\'er Rao bound. The approach of using a generative model to learn a performance estimation bound from data has been extended to several other non-Bayesian bounds, such as the misspecified CRB\cite{habi2023learned} and the Barankin bound \cite{habi2024learning}.  

Those approaches utilized normalizing flows\cite{kobyzev2020normalizing,papamakarios2021normalizing} which enable the computation of the probability density function of the measurements. This comes with a major limitation: that there exist an invertible mapping between the measurements and some base distribution that is analytically tractable (usually standard Gaussian). An example in which such mappings do not exist is when the measurements are quantized\cite{habi2022generative}. 

In %case of a 
the Bayesian setting, recent work \cite{crafts2023bayesian} suggested to learn the prior distribution using score matching \cite{hyvarinen2005estimation}. %\hvhedit
{After learning the prior score ($\nabla_{\p}\log\probt{\p}{\pr}$), obtaining the BCRB using this method requires complete {knowledge} of the Fisher score function ($\nabla_{\p}\log\probt{\x|\p}{\X|\pr}$) and its computation. {Given the known Fisher score function and the learned prior score function,} the BCRB is then approximately calculated using empirical means. A {limitation} of this approach is the need for  full knowledge of the Fisher score and ability to compute it, which can be challenging. For instance, calculating the Fisher score for quantized measurement with correlated noise involves integrating a multidimensional Gaussian with nondiagonal covariance, which must be done numerically. Another example is an application with incomplete knowledge of the Fisher score, such as noise from a physical image sensor \cite{abdelhamed2019noise}, or underwater noise \cite{weiss2023towards,msg0-ag12-22}.}



In this work, we %suggest
introduce the \emph{learned Bayesian Cram'er Rao bound (\name{})}, which learns both the prior and the measurement distributions. Specifically, we suggest two approaches to learn the BCRB. The first, the \emph{Posterior Approach,} is based on the score of the posterior distribution ($\nabla_{\p}\log\probt{\p|\x}{\pr|\X}$). The second, the \emph{Measurement-Prior Approach,} is based on the Fisher score function ($\nabla_{\p}\log\probt{\x|\p}{\X|\pr}$) and the prior score function ($\nabla_{\p}\log\probt{\p}{\pr}$). 

%\todo[inline,color=green]{Break into paragraphs, for readability}
The \emph{Posterior Approach} emphasizes learning a conditional score function of the parameter based on the given observation. This process requires 
{to learn} the posterior score function from a dataset consisting of observation-parameter pairs. Following this, the trained posterior score and the dataset are employed to evaluate the BCRB. The \emph{Posterior} Approach offers the advantage of simplicity in both the learning process and the design of the neural network. However, it is restricted in its capacity to integrate domain knowledge regarding the relationship between the parameter and the observations.  

To utilize such domain knowledge and obtain the benefits of combining with a learnable score (similar to model-based deep learning \cite{shlezinger2022model,shlezinger2023model}, physics-informed neural network \cite{banerjee2024physics}, {physics-encoded neural networks \cite{faroughi2024physics,meinders2024application}}), we propose the \emph{Measurement-Prior} Approach. This technique involves  learning the Fisher and prior score functions independently, and subsequently using them to calculate the learned BCRB. 

%Initially, 
We {learn} the Fisher and prior score functions from a dataset consisting of observation-parameter pairs. Although it is possible to learn the prior score function using traditional score matching \cite{hyvarinen2005estimation}, the same approach, {or existing conditional score matching \cite{hyvarinen2005estimation,liu2022estimating,yu2019generalized,yu2022generalized}},
%\todoing{add citation}, 
do not work for learning the Fisher score $\nabla_{\p}\log\probt{\x|\p}{\X|\pr}$ since the derivative is w.r.t. {the conditioning variable,} the parameter $\p$, {rather than w.r.t the conditioned variable $\x$, as in conditional  score matching.} To address this issue, we introduce a variation of score matching that enables to learn the Fisher score, which we call \emph{Fisher Score Matching (FSM)}. This {new} score matching process involves first learning the prior score function via standard score matching \cite{hyvarinen2005estimation} and then using the obtained prior score function to learn the Fisher score function. 

The primary advantage of employing the {Measurement-Prior} Approach is the integration of domain {knowledge}. {Inspired by \cite{faroughi2024physics,meinders2024application}, we suggest a \emph{\pe{} Score Neural Network,} a new type of score neural network that encodes the physicals of the problem in the structure of the network. This improves the accuracy of the approximation and {reduces} the sample complexity of learning it.} Moreover, this method enables to evaluate the learned BCRB in scenarios with {any desired number of} i.i.d. measurements, without {having to learn} a new score function.

% \todo[inline]{Inspired by \cite{}, we suggest a \pe{} score neural network a new type of score neural network that encodes the physicals of the problem in the structure of the network. This improves the accuracy of the approximation and simplifies the sample complexity.}


% \ybdelete{\hvhedit{In the experiment, neural networks were trained with condition variables\cite{},
% \todoin{missing citation} such as the signal-to-noise ratio, allowing us to obtain a single neural network for all SNR values. }}

We apply the \name{} %across multiple 
to several examples to show its benefits and {study} its behavior. {As a technical aspect in the experiments, neural networks were trained with conditioning variables\cite{mirza2014conditional,abdelhamed2019noise,liu2019conditional,ho2021classifier}, 
%\todoing{Missing citation} 
such as the signal-to-noise ratio, allowing us to obtain a single neural network for all SNR values.}


Our findings on a linear mixing measurement model with Gaussian noise,  and on its {1-bit} quantized variant illustrate the advantages of directly learning the Fisher score function from data. In these scenarios, the BCRB is known analytically or can be calculated numerically,
{enabling us to assess} the  {deviation of the \name{} from the exact BCRB due to learning error.} {Next, testing} 
 our approach on a non-linear signal processing problem of frequency estimation, highlights the advantages of a \pe{} score neural network {for reducing}
 sample complexity, {and enabling highly-accurate approximation to the BCRB with limited data.}
Lastly, we {study} two signal processing problems in which the BCRB cannot be computed with previous methods: %which include 
measurement with correlated noise and quantization; and frequency estimation with underwater ambient noise. {While the first problem is important in many sensor systems, including sensor arrays, the second is important for vessel identification \cite{erbe2019effects}.}  


The {main} contributions of this paper are %summarized 
as follows:
\begin{itemize}
    \item We introduce a fully learned Bayesian Cram'er-Rao bound that learns both prior and measurement distributions using score matching.
    
    \item  We introduce Fisher score matching (FSM) that enables to learn the Fisher score function. %\hvhedit{
    The Fisher score {%has additional 
    may be of independent}
    interest, 
    %usage 
    e.g., for the computation of {a learned} non-Bayesian CRB.
    \item  %\hvhedit
    We theoretically explore the FSM and demonstrate that {it produces a strongly consistent estimate of the true score.}
    \item  We {quantify theoretically the non-asymptotic, finite sample learning} error of the two suggested approaches, and validate {the theoretical predictions.}  numerically. %\hvhedit{
    In addition, we show that the LBCRB is a strongly consistent {estimator of} the true BCRB. 

    % HVH: Done and yes they have this in different way.}
    \item  {We demonstrate the two new approaches} on several signal processing examples, including linear and non-linear estimation {and} a real-world application  with underwater ambient noise. 
    %\todoing{Listing the examples one by one does not read well in a list of main contributions. Say something more general?}
\end{itemize}

In the spirit of reproducible research, we make the code of the Learned Bayesian Cram\'er-Rao bound available online \cite{lbcrb_repo}.

The paper is organized as follows. The background and notation are  in Section~\ref{sec:background}. 
The Learned Bayesian Cram'er-Rao bound method is presented in Sec.~\ref{sec:lbcrb_method}. A detailed derivation of Learned Bayesian Cram'er-Rao bound is present in Sec.~\ref{sec:lbcrb} followed by a {study} of its theoretical properties in Sec.~\ref{sec:theory}.   In Sec.~\ref{sec:example_models} we present a set of parameter estimation examples, including linear and nonlinear estimation problems, and frequency estimation with underwater ambient noise. The experimental results for the \name{} are described in Sec.~\ref{sec:experimental}, with conclusions in
%and 
Sec.~\ref{sec:conclusions}.
%provides discussion and conclusions. 
Proofs of the theoretical results of this paper are  in the appendices, which are included in the online Supplementary Material.



\section{Background}\label{sec:backgrnd}

\subsection{Cold Start Latency and Mitigation Techniques}

Traditional FaaS platforms mitigate cold starts through snapshotting, lightweight virtualization, and warm-state management. Snapshot-based methods like \textbf{REAP} and \textbf{Catalyzer} reduce initialization time by preloading or restoring container states but require significant memory and I/O resources, limiting scalability~\cite{dong_catalyzer_2020, ustiugov_benchmarking_2021}. Lightweight virtualization solutions, such as \textbf{Firecracker} microVMs, achieve fast startup times with strong isolation but depend on robust infrastructure, making them less adaptable to fluctuating workloads~\cite{agache_firecracker_2020}. Warm-state management techniques like \textbf{Faa\$T}~\cite{romero_faa_2021} and \textbf{Kraken}~\cite{vivek_kraken_2021} keep frequently invoked containers ready, balancing readiness and cost efficiency under predictable workloads but incurring overhead when demand is erratic~\cite{romero_faa_2021, vivek_kraken_2021}. While these methods perform well in resource-rich cloud environments, their resource intensity challenges applicability in edge settings.

\subsubsection{Edge FaaS Perspective}

In edge environments, cold start mitigation emphasizes lightweight designs, resource sharing, and hybrid task distribution. Lightweight execution environments like unikernels~\cite{edward_sock_2018} and \textbf{Firecracker}~\cite{agache_firecracker_2020}, as used by \textbf{TinyFaaS}~\cite{pfandzelter_tinyfaas_2020}, minimize resource usage and initialization delays but require careful orchestration to avoid resource contention. Function co-location, demonstrated by \textbf{Photons}~\cite{v_dukic_photons_2020}, reduces redundant initializations by sharing runtime resources among related functions, though this complicates isolation in multi-tenant setups~\cite{v_dukic_photons_2020}. Hybrid offloading frameworks like \textbf{GeoFaaS}~\cite{malekabbasi_geofaas_2024} balance edge-cloud workloads by offloading latency-tolerant tasks to the cloud and reserving edge resources for real-time operations, requiring reliable connectivity and efficient task management. These edge-specific strategies address cold starts effectively but introduce challenges in scalability and orchestration.

\subsection{Predictive Scaling and Caching Techniques}

Efficient resource allocation is vital for maintaining low latency and high availability in serverless platforms. Predictive scaling and caching techniques dynamically provision resources and reduce cold start latency by leveraging workload prediction and state retention.
Traditional FaaS platforms use predictive scaling and caching to optimize resources, employing techniques (OFC, FaasCache) to reduce cold starts. However, these methods rely on centralized orchestration and workload predictability, limiting their effectiveness in dynamic, resource-constrained edge environments.



\subsubsection{Edge FaaS Perspective}

Edge FaaS platforms adapt predictive scaling and caching techniques to constrain resources and heterogeneous environments. \textbf{EDGE-Cache}~\cite{kim_delay-aware_2022} uses traffic profiling to selectively retain high-priority functions, reducing memory overhead while maintaining readiness for frequent requests. Hybrid frameworks like \textbf{GeoFaaS}~\cite{malekabbasi_geofaas_2024} implement distributed caching to balance resources between edge and cloud nodes, enabling low-latency processing for critical tasks while offloading less critical workloads. Machine learning methods, such as clustering-based workload predictors~\cite{gao_machine_2020} and GRU-based models~\cite{guo_applying_2018}, enhance resource provisioning in edge systems by efficiently forecasting workload spikes. These innovations effectively address cold start challenges in edge environments, though their dependency on accurate predictions and robust orchestration poses scalability challenges.

\subsection{Decentralized Orchestration, Function Placement, and Scheduling}

Efficient orchestration in serverless platforms involves workload distribution, resource optimization, and performance assurance. While traditional FaaS platforms rely on centralized control, edge environments require decentralized and adaptive strategies to address unique challenges such as resource constraints and heterogeneous hardware.



\subsubsection{Edge FaaS Perspective}

Edge FaaS platforms adopt decentralized and adaptive orchestration frameworks to meet the demands of resource-constrained environments. Systems like \textbf{Wukong} distribute scheduling across edge nodes, enhancing data locality and scalability while reducing network latency. Lightweight frameworks such as \textbf{OpenWhisk Lite}~\cite{kravchenko_kpavelopenwhisk-light_2024} optimize resource allocation by decentralizing scheduling policies, minimizing cold starts and latency in edge setups~\cite{benjamin_wukong_2020}. Hybrid solutions like \textbf{OpenFaaS}~\cite{noauthor_openfaasfaas_2024} and \textbf{EdgeMatrix}~\cite{shen_edgematrix_2023} combine edge-cloud orchestration to balance resource utilization, retaining latency-sensitive functions at the edge while offloading non-critical workloads to the cloud. While these approaches improve flexibility, they face challenges in maintaining coordination and ensuring consistent performance across distributed nodes.


\section{Learned Bayesian Cram\'er Rao Bound:Overview}\label{sec:lbcrb_method}
%Here, 
We briefly %provide a short 
overview the problem that the LBCRB address, and the methods to compute it. {The  overview in this section covers all the information that a practitioner would need to apply the proposed techniques. Detailed formulation, derivations and theoretical analysis are in later sections.}
%that address it.
Our goal is to determine the Bayesian Cram\'er-Rao bound \eqref{eq:bcrb}
in scenarios where the prior $\probt{\p}{\pr}$, the measurement $\probt{\x|\p}{\X|\pr}$, or both are either
unknown or partially known. However, a data set 
\begin{equation}\label{eq:dataset_rel}
 \ds=\set{\p_n, \xsetr_n=\set{{\x}_{n,j}}_{j=1}^{\niiddata}   }_{%\ybedit
 {n=1}}^{\nds},
\end{equation}
of $\nds$ parameter-measurement sets pairs is given,
{where  $\p_n$ drawn from $\probt{\pr}{\pr}$ and  each $\x_{n,j}$ drawn from $\probt{\x|\p_n}{\X|\pr}$.}. 
%\hvhedit
{The measurement set in $\ds$ contains $\niiddata$ i.i.d samples %\ybreplace{given}
{for} the same value of $\p_i$.} 
% \ybreplace{which may differ from $\niideval$, $\niiddata\neq \niideval$.}
{It is acceptable that $\niiddata\neq \niideval$, that is, 
$\niiddata$  may differ from $\niideval$ defined earlier in the context of \eqref{eq:bfim_decomposition_base}}.

We %suggest 
propose two approaches %to obtain \eqref{eq:bcrb}
%\ybedit
{to learn the \name{} from $\ds$:}
 the \emph{Posterior Approach}; and the \emph{Measurement-Prior Approach}.
 
\subsection{Posterior Approach}
%\noindent
\subsubsection{\textbf{Learning Step}} Define a neural network $\postscore{\p}{\xsetr;\Omega}$ 
 parameterized by $\Omega$ 
 %\ybdelete{\footnote{\label{fn:model}
%For conciseness, we drop $\Omega$ or $\Omega^*$ from the notation, whenever the NN parameters are fixed (e.g. after training). }  } %\ybedit
{to {model} $\nabla_{\pr}\log\probt{\xsetr,\p}{\xset,\pr}$.} Then minimize the following objective %\ybedit
 {with respect to $\Omega$}:
\begin{align}\label{eq:score_post_mean}
     &\lossbsm\brackets{\Omega}=\frac{1}{\nds}\sum_{\p,\xsetr\in\ds}\ell_B\brackets{\p,\xsetr;\Omega},\\
     &\ell_B\brackets{\p,\xsetr;\Omega}\triangleq\norm{\postscore{\p}{\xsetr;\Omega}}_2^2+2\trace{{\frac{\partial \postscore{\p}{\xsetr;\Omega}}{\partial\p}}}.\nonumber
 \end{align}
%\ybedit
{Denote the minimizer determined in this step by $\Omega^*$.}

%\noindent
\subsubsection{\textbf{Evaluation Step}} %\ybedit
{Using} %\ybedit
{
$\postscores{\p}{\xsetr} \triangleq \postscore{\p}{\xsetr;\Omega^*}$
} \footnote{\label{fn:model}%
 %\hvhedit
 %{
%Note that throughout the text, 
For conciseness, we drop $\Omega$ or $\Omega^*$ from the notation, whenever the NN parameters are fixed (e.g. after training). }  
compute the Learned Bayesian Fisher Information Matrix % given by,
\begin{equation}\label{eq:mean_fully}
    \lbfimbs\triangleq \frac{1}{\nds}\sum_{\p,\xsetr\in\mathcal{D}}\postscores{\p}{\xsetr}\postscores{\p}{\xsetr}^T.
\end{equation}
Finally, to obtain the LBCRB, invert $\lbfimbs$, which results in ${\bcrb}\approx\lbcrbbs\triangleq\lbfimbs^{-1}$.  
% \end{tcolorbox}
The Posterior Approach is illustrated in Figure~\ref{fig:main_post}. 

%\hvhedit
{We emphasize that the LBCRB in \eqref{eq:mean_fully} is calculated  {for a measurement that contains  $\niideval$ i.i.d samples $\x_i, i=1, \ldots, \niideval$, where $\niideval= \niiddata$, that is, $\niideval$ coincides with the number $\niiddata = |\xsetr|$ of i.i.d samples available in the training set $\ds$ for each  value of $\p$.  %Given a dataset $\ds$ with $\niiddata$, 
}
The LBCRB can be determined for $\niideval\leq \niiddata$ {using such a data set,}  but this requires learning a different score function for each desired $\niideval$. {On the other hand, computing 
the LBCRB for $\niideval > \niiddata$ is not possible using the same $\ds$, and would require a data set with a larger $\niiddata$.
 %In contrast, the LBCRB computation for $\niideval>\niiddata$ is feasible with the upcoming 
 Both of these limitations are overcome by the Measurement-Prior Approach of the next section.
}

% \todo[inline,color=green]{YB: Need to specify that this BCRB is computed for $\niideval=\niiddata$. Is it possible to use the same $\ds$ to compute the BCRB for $\niideval <\niiddata$? What about $\niideval >\niiddata$?  Explain.\\
% HVH: Please see update text above.
% How was the vaccination ? 
% }
We %have shown 
will {quantify the approximation of $\bcrb$ by $\lbcrbbs$ and also}
show that if  {the model neural net $\postscore{\p}{\xsetr;\Omega}$} has sufficient capacity, then
{
%\begin{equation*}
$
   % \norm{
    \lbcrbbs %-\bcrb
    %}_2
    \xrightarrow{\nds\rightarrow \infty} \bcrb \quad\text{%\hvhedit
    {a.s}} 
%\end{equation*}
$
(almost surely, i.e., with probability 1). In other words, the approximation $\lbcrbbs$ enjoys the important statistical 
 property of \emph{strong consistency}.
}



\subsection{Measurement-Prior Approach}
\label{subsec:MP}
To improve the sample complexity of the Posterior Approach, we suggest to decompose, {similar to \eqref{eq:bfim_decomposition_base},} the LBCRB into two parts: Prior; and Measurement. As an additional advantage, this enables to introduce domain knowledge. We begin with the prior term.

%\noindent
\subsubsection{\textbf{Learning Prior Step}}  Define a neural network $\priorscore{\p;\paramp}$  parameterized by $\paramp$ %\ybedit
{to {model} $\nabla_{\p}\log\probt{\p}{\pr}$.} Then
%\ybedit
{find a minimizer $\Omega_P^*$ %minimize 
of the following objective
%with respect to $\Omega_P$.
}
\begin{align}\label{eq:score_prior_mean}
     &\losspsm\brackets{\paramp}=\frac{1}{\nds}\sum_{\p\in\ds}\ell_P\brackets{\p;\paramp},\\
     &\ell_P\brackets{\p;\paramp}\triangleq\norm{\priorscore{\p;\paramp}}_2^2+2\trace{\frac{\partial \priorscore{\p;\paramp}}{\partial\p}}.\nonumber
 \end{align}
%\noindent
\subsubsection{\textbf{Construct \pe{} Score Neural Network (\peac{})}} \label{subsubsec:MoISNN}
Assume knowledge of a model function $\mathcal{M}\brackets{\p}$ such that {the PDF of a single measurement sample can be expressed as} $\probt{\x|\p}{\X|\pr}=\probt{{\x}|\mathcal{M}\brackets{\p}}{\X|\vectorsym{\tau}}$, and define a neural network $\iscore{\x}{\vectorsym{\tau}; \paramf}$ that is parameterized by same parameter $\paramf$ of $\lscore{\x}{\p;\paramf}$. 
% \todo[inline,color=green]{YB:I suggest to avoid defining $\Omega_I$, as we don't need it anywhere. Just use  $\Omega_F$? }
Then, a \emph{\pe{} Score Neural Network} (\peac{}) %\ybedit
{to {model} the Fisher score $\nabla_{\p}\log\probt{{\x}|\p}{{\X}|\p}$} is given by:
\begin{equation}\label{eq:model_base_score}
\lscore{{\x}}{\p;\paramf}=\divc{\mathcal{M}\brackets{\p}}{\p}^T\at{\iscore{{\x}}{\vectorsym{\tau};\paramf}}{\vectorsym{\tau}=\mathcal{M}\brackets{\p}}.
\end{equation}
The \pe{} Score 
Neural Network is illustrated %shown 
in Figure~\ref{fig:model_inforamed}.

As an example of the application of \peac{}, consider %\ybedit
{a frequency estimation problem} . 
Let $\squareb{\X}_n=%\ybreplace{\exp\brackets{j\omega n}}
{\cos\brackets{ \theta n }}+\squareb{\randomvec{W}}_n$,
%\ybedit
{$n= 1, \ldots, N$} 
% \ybreplace{represent an observation vector in a frequency estimation context, where $\randomvec{W}$ denotes unknown random noise and $\omega$ is the frequency.}
{be the observation, with $\theta$ the frequency to be estimated, and $\randomvec{W}$   random noise with unknown distribution.} Here, $\squareb{\mathcal{M}\brackets{\p}}_n=
%\ybreplace{\exp\brackets{j\omega n}}
{\cos\brackets{\theta n}}$. Using \peac{}, we only need to learn the %\ybedit
{the score for the PDF of the} noise component, $\randomvec{W}$, {eliminating the need to learn and represent the cosine function}.
%\noindent 
\definecolor{mydarkblue}{RGB}{0,0,139} % This defines a dark blue color
\begin{figure}
    \centering
     \begin{tikzpicture}[every label/.append style={align=center}]
        
        \node[startstop,fill=green!15, text width=2.5cm,align=center] (model_learn) {Neural Network\\   $ \iscore{{\x}}{\vectorsym{\tau};\paramf}$ };
        \node[below=0.5cm of model_learn,startstop,fill=white, text width=3cm,align=center] (model_base) {\pe{}\\   $\vectorsym{\tau}=\mathcal{M}\brackets{\p}$ };
        \node[right=0.5cm of model_learn,startstop,fill=white, text width=2.5cm,align=center] (model_base_porj) {Projection\\   $\divc{\mathcal{M}\brackets{\p}}{\p}^T$ };

        \node[left=0.7 of model_learn ] (x) {${\x}$};
        \node[left=0.7 of model_base,yshift=-0.7cm] (theta) {$\p$};



        
        \begin{pgfonlayer}{background}
            \node[draw,rounded corners,fit= (model_base) (model_learn) (model_base_porj), fill=yellow!30,inner sep=14pt,label={\pe{} Score Neural Network}](fit2){};

            
        \end{pgfonlayer}

        \draw[->,line width=0.25mm] (x)  -- ( model_learn);
        \draw[->,line width=0.25mm] (theta)  -| (model_base);
        \draw[->,line width=0.25mm] (theta)  -| (model_base_porj);
        \draw[->,line width=0.25mm] (model_base)  -- (model_learn);
        \draw[->,line width=0.25mm] (model_learn)  -- (model_base_porj);
 %        \draw[->,dotted,line width=0.25mm] (eval_prior)  -| (eval_bfim);
\end{tikzpicture} 

    \caption{\pe{} Score Neural Network: The white blocks indicate the usage of $\mathcal{M}\brackets{\p}$ (a known function), whereas the green block represents a neural network that is trained during the learning process. 
    % \todo[inline,color=green]{YB: Update the terminology in the figure and caption once we decide on it. Also increase the fonts, per the comment on Fig.1\\
    % VHV:Update using a macro should be easy to change if needed.}  
    }
    \label{fig:model_inforamed}
\end{figure}


\subsubsection{\textbf{Learning Fisher Score Step}} 
%\hvhedit
{Use the neural network $\lscore{{\x}}{\p;\paramf}$ parameterized by $\paramf$ to %represent 
model {the Fisher score} $\nabla_{\p}\log\probt{\x|\p}{\X|\pr}$ {for a single measurement sample.}
Then %\ybedit
{find a mininimizer $\paramf^*$}
%minimize 
of the following objective 
%by optimizing $\paramf$ 
($\paramp%\ybedit
{^*}$ is known from the step of learning the Prior).
\begin{align}\label{eq:score_lik_mean}
    &\lossfsm\brackets{\paramf; \paramp%\ybedit
    {^*}}=\frac{1}{\nds\cdot\niiddata}\sum_{\xsetr,\p\in\mathcal{D}}\sum_{{\x}\in\xsetr}{\ell_{F}\brackets{\x,\p;\paramf,\paramp%\ybedit
    {^*}}}{} \nonumber\\
    % &+2\trace{
    % \overline{\matsym{J}}_{F}\ybedit{(\paramf)}}    +\sum_{\xsetr,\p\in\mathcal{D}}\sum_{\x\in\xsetr}\frac{\norm{\lscore{{\x}}{\p;\paramf}}_2^2}{\nds\cdot\niiddata}.\\
    &\ell_{F}\brackets{\x,\p;\paramf,\paramp%\ybedit
    {^*}}\triangleq \norm{\lscore{{\x}}{\p;\paramf}}_2^2\\
    &+2\lscore{{\x}}{\p;\paramf}^T\priorscore{\p;\paramp%\ybedit
    {^*}}+2\trace{\frac{\partial \lscore{{\x}}{\p;\paramf}}{\partial\p}}\nonumber
\end{align}
\subsubsection{\textbf{Evaluation Step}}  %\ybedit
{Using $\priorscores{\p}=\priorscore{\p; \Omega^*_P}$ and $\lscores{\x}{\p}$ = $\lscore{\x}{\p;\Omega^*_F}$ } compute %\ybedit
{the learned Measurement {FIM for a single measurement sample} and the Prior FIM}
\begin{equation} \label{eq:mean_efim_likd}
    \lmfim=\frac{1}{\nds\cdot \niiddata}\sum_{\p,\xsetr\in\ds} \sum_{\x\in\xsetr}\lscores{\x}{\p}\lscores{\x}{\p}^T,
\end{equation}
\begin{equation}\label{eq:bprior_mean}
    \lpfim=\frac{1}{\nds} \sum_{\p\in\ds} \priorscores{\p}\priorscores{\p}^T,
\end{equation}
%\ybedit
{Then}
the Learned Bayesian FIM for any desired number $\niideval$ of i.i.d samples is given by
\begin{equation}\label{eq:bfim_apx_final}
     \lbfimlps{(\niideval)} =\niideval\cdot  \lmfim+ \lpfim.
\end{equation}
Finally, %\ybedit
{%to obtain 
the Learned BCRB 
%we invert \eqref{eq:bfim_apx_final} which results in:
is obtained as }
% \begin{equation}
    $\lbcrblps=\lbfimlps^{-1}.$   
    
    %An illustration of 
    The Measurement-Prior Approach is %shown 
    illustrated in Figure~\ref{fig:main_lik_prior}. 
    
    %\ybedit
    {This method has important advantages over the Posterior Approach. First,} 
    %Using this method, we achieve 
    it provides the same guarantees as those %acquired through 
    of the Posterior Approach but with 
    %improved 
    lower sample complexity, and with a more interpretable model. %\ybedit
    {
   % \hvhedit{
   Second, because 
  %  We would like to emphasis that 
    $\niideval$ %is 
    can be chosen different to $\niiddata$,} %\hvhedit{
   % This enable an 
   the  Measurement-Prior Approach 
   can %to obtain 
   provide the LBCRB
   %} 
   for \emph{any} %\ybedit
   {desired} number of $\niideval$ i.i.d. samples without any additional effort %\ybedit
   {or additional training data, which the Posterior Approach cannot}.
   
   % \hvhedit
    {We will prove
    %provide theoretical derivation 
    that minimizing that objective in \eqref{eq:score_lik_mean} is equivalent to learning the true Fisher score $\nabla_{\p}\log\probt{\x|\p}{\X|\pr}$, and that the Fisher score neural network is consistent estimator of the true Fisher score. Moreover, we will quantify the approximation of $\bcrb$ by $\lbcrblps$ and also show that if the score neural networks $\priorscores{\p}$ and $\lscores{\x}{\p}$ have sufficient capacity, then {
    $\lbcrblps\xrightarrow{\nds\rightarrow\infty} \bcrb$} a.s. (almost surely, i.e., with probability 1). In other words, the approximation $\lbcrblps$ enjoys the important statistical property of \emph{strong consistency}. }
   % \todo[inline,color=green]{Add a summary of the theoretical analysis  and convergence result similar to that at the end of Sec. III-A?}

\section{{Derivation of the \name{}}}\label{sec:lbcrb}
{We turn now to a precise problem statement and derivation of the two proposed methods for the \name{}.}
\begin{tcolorbox}
\begin{problem}\label{problem_one}
 %Assume 
 Suppose that $\probt{\x|\p}{\X|\pr}$ and $\probt{\p}{\pr}$ satisfy Assumptions \ref{assum:bcrb_reg} and %some 
 score matching Assumptions \ref{ass:score_cond_reg} or \ref{ass:score_reg} and \ref{ass:score_reg_prior}. 
 {Let $\bcrb$ be the}  %a
Bayesian Cram\'er-Rao lower bound (BCRB) \eqref{eq:bcrb} on the estimation error of parameter $\pr\in\Ps$ from a %the
measurement {$\xset = \set{{\X}_{i}}_{i=1}^{\niideval}$ containing $\niideval$ i.i.d. measurements ${\X}_i \sim  \probt{\X|\pr_n}{\X|\pr}$.}
{Assume that $\probt{\x|\p}{\X|\pr}$ and $\probt{\p}{\pr}$ are completely or partially unknown.}
Given a data set 
        %
        {\begin{equation}\label{eq:dataset}
          \ds=\set{\pr_n, \mathcal{\X}_n=\set{{\X}_{n,j}}_{j=1}^{\niiddata}   }_{n=1}^{\nds} 
        \end{equation}}
        % \todo[inline,color=green]{Same comments as before about $\ds$.}
\noindent 
of  {parameter-measurement} samples that are independent and identically-distributed (i.i.d) 
as    %
{${\X}_{n,j} \sim  \probt{\X|\pr_n}{\X|\pr}, \pr_n\sim \probt{\pr}{\pr}$, }
obtain {a learned approximation  \name{} $\hat{\bcrb}(\ds)$ to  $\bcrb$} satisfying:
\begin{equation}
     {\hat{\bcrb}}\brackets{\ds}
     \xrightarrow{\nds \rightarrow \infty}\bcrb\quad\text{%
     {a.s}}.
\end{equation}

\end{problem}
\end{tcolorbox}
% \todo[inline,color=green]{YB: Note the technical and language changes in the problem statement. Pls check that you agree.\\
% \textbf{HVH}: What do you think about emphasizing that the PDF are unknown ?, please suggest changes in the text. }
\begin{remark}
    The data set in \eqref{eq:dataset_rel} represents an instance {(realization)} of the data set defined in Problem~\ref{problem_one} (\eqref{eq:dataset}). For theoretical analysis concerning the stochastic nature of the \name{}, we employ the random data set specified in \eqref{eq:dataset}.
\end{remark}

To address Problem~\ref{problem_one}, we propose two approaches: 
(A) the Posterior Approach, which 
%utilized a 
uses conditional score matching to learn the posterior score;  %Second, 
and (B) the Measurement-Prior Approach, which learns two score functions -- one for the prior, and another for the measurement distribution.
 The Measurement-Prior Approach facilitates the incorporation of domain knowledge into the score neural network {improving the representation and learning of the true score}. 
Similar to previous works \cite{habi2023learned,habi2023learning, crafts2023bayesian}, both approaches comprise two stages - of learning, and evaluation. {However, as discussed in the Introduction, they differ from the previous works in most other key aspects.}   
%An overview of 
The two approaches %is shown 
are illustrated in Figure~\ref{fig:main}.

\subsection{Posterior Approach}\label{sec:post_learn}
In the Posterior Approach {( Fig.~\ref{fig:main_post}),} we use the BFIM in \eqref{eq:bcrb}, which only requires the posterior score $\nabla_{\p}\log\probt{\p|\xsetr}{\p|\xset}$, %which is 
a conditional score of $\p$ given a measurement $\X$. Then we use the learned conditional score to evaluate the \name{} by replacing the expectation with an empirical mean. 
\definecolor{mydarkblue}{RGB}{0,0,139} % This defines a dark blue color




\begin{figure*}[ht]
    \centering
    \begin{subfigure}[t]{1.0\textwidth}
        \centering
             \begin{tikzpicture}[font=\small,every label/.append style={font=\small,align=center}]
        


        
        
        
        
        % \node[startstop,fill=white, text width=3.1cm,align=center,yshift=-0.0cm,fill=yellow!15] (loss_prior) { \tiny Prior Score Matching \eqref{eq:score_prior_mean}\\ 
        % $\Omega_p =\arg\min\limits_{\widetilde{\Omega_{p}}}\overline{\mathcal{L}_{p}}\brackets{\widetilde{\Omega_{p}}}$};

        % \node[below=0.4cm of loss_prior,startstop,fill=cyan, text width=3.1cm,align=center,yshift=0cm] (loss_dmle) {\tiny Likelihood Score Matching \eqref{eq:score_lik_mean} \\
        % $\Omega_h =\arg\min\limits_{\widetilde{\Omega_{h}}}\overline{\mathcal{L}_{d}}\brackets{\widetilde{\Omega_{h}};\Omega_{p}}$};

        \node[startstop,fill=yellow!15, text width=5.0cm,align=center,yshift=0cm] (loss_post) { Conditional Score Matching \eqref{eq:score_post_mean} \\
       $\Omega^* =\arg\min\limits_{{\Omega_{}}}\lossbsm\brackets{{\Omega}}$};

  %       \node[right=0.3 cm of loss_prior, startstop,fill=green!15, text width=4cm,align=center,yshift=+3pt] (prior_score) {\tiny Prior Score \\$\nabla_{\p}\log\probt{\p}{\p}\approx \vectorsym{p}_{\Omega_p}\brackets{\p}$};
		% \node[ right=0.3cm of loss_dmle,startstop,fill=green!15, text width=4cm,align=center,yshift=+3pt] (dmle) {\tiny Likelihood Score\\ $\nabla_{\p}\log\probt{\X|\p}{\X|\p}\approx$ $\vectorsym{h}_{\Omega_d}\brackets{\randomvec{X}|\p}$};
  

        \node[right=0.3 cm of loss_post, startstop,fill=green!15, text width=5.5cm,align=center,yshift=+3pt] (post_score) { Posterior Score \\$\nabla_{\p}\log\probt{\p|\x}{\pr|\X} \approx \postscore{\p}{\x;\Omega^*}$};
        
        % \node[right=0.3cm of prior_score,startstop,fill=white, text width=1.7cm,align=center,fill=red!15] (eval_prior) { \tiny Compute $\overline{\mathrm{LPFIM}}$\\ using \eqref{eq:bprior_mean} };

        



        % \node[right=0.3cm of dmle,startstop,fill=white,fill=red!15 , text width=1.7cm,align=center] (eval_fim) { \tiny  Compute $\overline{\mathrm{LEFIM}}$\\ using \eqref{eq:mean_efim_likd}};

        %         \node[right=0.3cm of eval_prior,startstop,fill=white, text width=4cm,align=center,yshift=-4pt,fill=red!15] (eval_bfim) { \tiny Compute the LBFIM \\
        %         $\overline{\mathrm{LBFIM}}_{lp} =\niideval\cdot  \overline{\mathrm{LEFIM}}+ \overline{\mathrm{LPFIM}}$\\
        %         and Invert LBFIM to obtain LBCRB\\
        %         $\overline{\mathrm{LBCRB}}_{lp}=\overline{\mathrm{LBFIM}}^{-1}_{lp}$
        %         };
        
        \node[right=0.3cm of post_score,startstop,fill=white, text width=2cm,align=center,fill=red!15] (eval_bfim_post) {   Compute $\lbfimbs$\\ using \eqref{eq:mean_fully}};

        \node[right=0.3cm of eval_bfim_post,startstop,fill=white, text width=2.5cm,align=center,fill=red!15] (inv_post) {   Invert $\lbfimbs$\\ $\lbcrbbs\triangleq\lbfimbs^{-1}$ };
        
        \begin{pgfonlayer}{background}
            % \node[draw,rounded corners,fit= (prior_score) (dmle), fill=white,inner sep=2pt,label={Score Models}](fit1){};

            % \node[draw,rounded corners,fit= (loss_dmle) (loss_prior), fill=white,inner sep=2pt,label={Learning Stage}](fit2){};

            % \node[draw,rounded corners,fit= (eval_bfim) (eval_fim)  (eval_prior), fill=white,inner sep=2pt,label=Evaluation Stage](fit_eval){};



            \node[draw,rounded corners,fit= (post_score), fill=white,inner sep=2pt,label={Score Model}](fit11){};

            \node[draw,rounded corners,fit= (loss_post), fill=white,inner sep=2pt,label={Learning Stage}](fit12){};

                        \node[draw,rounded corners,fit= (inv_post)  (eval_bfim_post), fill=white,inner sep=2pt,label=Evaluation Stage](fit_eval2){};
            
        \end{pgfonlayer}
        % \begin{pgfonlayer}{background1}
        %                 \node[draw,rounded corners,fit= (fit1) (fit2)  (fit_eval), fill=teal!20,inner sep=13pt,label=Likelihood-Prior Approach](main_lik_post){};

        %                 \node[draw,rounded corners,fit= (fit11) (fit12)  (fit_eval2), fill=violet!20,inner sep=13pt,label=Posterior Approach](main_post){};
        % \end{pgfonlayer}
        
        % \node[right=0.5cm of main_post,inner sep=0pt,yshift=0cm] (no)  {\includesvg[width=0.14\textwidth]{files/results/freq_est_v2/phase_error.svg}};

        % \begin{pgfonlayer}{background}
        %     \node[draw,rounded corners,fit= (no) , fill=white!30,inner sep=2pt](fit2_res){};
        % \end{pgfonlayer}
        % \draw[->,line width=0.25mm] (loss_prior)  -- (prior_score);
        % \draw[->,dashed,line width=0.25mm,color=red,label=$\Omega_p$] (loss_prior)  -- node[left] {$\Omega_p$} (loss_dmle);
        % \draw[->,line width=0.25mm] (loss_dmle)  -- (dmle);
        % \draw[->,line width=0.25mm] (dmle)  -- (eval_fim);
        % \draw[->,line width=0.25mm] (prior_score)  -- (eval_prior);
        % \draw[->,line width=0.25mm] (eval_prior)  -- (eval_bfim);
        % \draw[->,line width=0.25mm] (eval_fim)  -| (eval_bfim);
        % \draw[->,line width=0.25mm,color=blue] (eval_bfim)  -| (no);
        % \draw[-,dashed,line width=0.25mm,color=red] (prior_score)  -- (loss_prior);


        \draw[->,line width=0.25mm] (loss_post)  -- ( post_score);
        \draw[->,line width=0.25mm] (post_score)  -- (eval_bfim_post);
        \draw[->,line width=0.25mm] (eval_bfim_post)  -- (inv_post);
        % \draw[->,line width=0.25mm,color=blue] (inv_post)  -- (no);
        % \draw[->,line width=0.25mm,color=blue] (eval_bfim)  -| (no);
\end{tikzpicture} 
        \caption{Posterior Approach}
        \label{fig:main_post}
    \end{subfigure}%
    
    \begin{subfigure}[t]{1.0\textwidth}
        \centering
             \begin{tikzpicture}[font=\small,every label/.append style={font=\small,align=center}]
        


        
        
        
        
        \node[startstop,fill=white, text width=5cm,align=center,yshift=-0.0cm,fill=yellow!15] (loss_prior) {  Prior Score Matching \eqref{eq:score_prior_mean}\\ 
        $\paramp^* =\arg\min\limits_{{\paramp}}\losspsm\brackets{{\paramp}}$};

        \node[below=0.4cm of loss_prior,startstop,fill=cyan, text width=5cm,align=center,yshift=0cm] (loss_dmle) { Fisher Score Matching \eqref{eq:score_lik_mean} \\
        $\paramf^* =\arg\min\limits_{{\paramf}}\lossfsm\brackets{{\paramf};\paramp^*}$};

       %  \node[below=2cm of loss_dmle,startstop,fill=yellow!15, text width=3.1cm,align=center,yshift=0cm] (loss_post) {\tiny Conditional Score Matching \eqref{eq:score_post_mean} \\
       % $\Omega =\arg\min\limits_{\widetilde{\Omega_{}}}\overline{\mathcal{L}_{b}}\brackets{\widetilde{\Omega}}$};

        \node[right=0.3 cm of loss_prior, startstop,fill=green!15, text width=5.5cm,align=center,yshift=+3pt] (prior_score) { Prior Score \\$\nabla_{\p}\log\probt{\p}{\pr}\approx \priorscore{\p;\paramp^*}$};
		\node[ right=0.3cm of loss_dmle,startstop,fill=green!15, text width=5.5cm,align=center,yshift=+3pt] (dmle) { Fisher Score\\ $\nabla_{\p}\log\probt{\x|\p}{\X|\pr}\approx$ $\lscore{\x}{\p;\paramf^*}$};
  

        % \node[right=0.3 cm of loss_post, startstop,fill=green!15, text width=3.5cm,align=center,yshift=+3pt] (post_score) {\tiny Posterior Score \\$\nabla_{\p}\log\probt{\p|\X}{\p|\X} \approx \vectorsym{s}_{\Omega}\brackets{\p|\X}$};
        
        \node[right=0.3cm of prior_score,startstop,fill=white, text width=2cm,align=center,fill=red!15] (eval_prior) {  Compute $\lpfim$\\ using \eqref{eq:bprior_mean} };

        



        \node[right=0.3cm of dmle,startstop,fill=white,fill=red!15 , text width=2cm,align=center] (eval_fim) {   Compute $\lmfim$\\ using \eqref{eq:mean_efim_likd}};

                \node[right=0.3cm of eval_prior,startstop,fill=white, text width=3.5cm,align=center,yshift=-4pt,fill=red!15] (eval_bfim) {  Compute the LBFIM \\
                $\lbfimlps =\niideval\cdot  \lmfim+ \lpfim$\\
                and Invert LBFIM to obtain LBCRB\\
                $\lbcrblps=\lbfimlps^{-1}$
                };
        
        % \node[right=0.3cm of post_score,startstop,fill=white, text width=2cm,align=center,fill=red!15] (eval_bfim_post) { \tiny  Compute $\overline{\mathrm{LBFIM}}_p$\\ using \eqref{eq:mean_fully}};

        % \node[right=0.3cm of eval_bfim_post,startstop,fill=white, text width=2.5cm,align=center,fill=red!15] (inv_post) { \tiny  Invert $\overline{\mathrm{LBFIM}}_p$\\ $\overline{\mathrm{LBCRB}}_{p}\triangleq\overline{\mathrm{LBFIM}}_{p}^{-1}$ };
        
        \begin{pgfonlayer}{background}
            \node[draw,rounded corners,fit= (prior_score) (dmle), fill=white,inner sep=2pt,label={Score Models}](fit1){};

            \node[draw,rounded corners,fit= (loss_dmle) (loss_prior), fill=white,inner sep=2pt,label={Learning Stage}](fit2){};

            \node[draw,rounded corners,fit= (eval_bfim) (eval_fim)  (eval_prior), fill=white,inner sep=2pt,label=Evaluation Stage](fit_eval){};



            % \node[draw,rounded corners,fit= (post_score), fill=white,inner sep=2pt,label={Score Model}](fit11){};

            % \node[draw,rounded corners,fit= (loss_post), fill=white,inner sep=2pt,label={Learning Stage}](fit12){};

            %             \node[draw,rounded corners,fit= (inv_post)  (eval_bfim_post), fill=white,inner sep=2pt,label=Evaluation Stage](fit_eval2){};
            
        \end{pgfonlayer}
        % \begin{pgfonlayer}{background1}
        %                 \node[draw,rounded corners,fit= (fit1) (fit2)  (fit_eval), fill=teal!20,inner sep=13pt,label=Likelihood-Prior Approach](main_lik_post){};

        %                 \node[draw,rounded corners,fit= (fit11) (fit12)  (fit_eval2), fill=violet!20,inner sep=13pt,label=Posterior Approach](main_post){};
        % \end{pgfonlayer}
        
        % \node[right=0.5cm of main_post,inner sep=0pt,yshift=0cm] (no)  {\includesvg[width=0.14\textwidth]{files/results/freq_est_v2/phase_error.svg}};

        % \begin{pgfonlayer}{background}
        %     \node[draw,rounded corners,fit= (no) , fill=white!30,inner sep=2pt](fit2_res){};
        % \end{pgfonlayer}
        \draw[->,line width=0.25mm] (loss_prior)  -- (prior_score);
        \draw[->,dashed,line width=0.25mm,color=red,label=$\Omega_p^*$] (loss_prior)  -- node[left] {$\Omega_p^*$} (loss_dmle);
        \draw[->,line width=0.25mm] (loss_dmle)  -- (dmle);
        \draw[->,line width=0.25mm] (dmle)  -- (eval_fim);
        \draw[->,line width=0.25mm] (prior_score)  -- (eval_prior);
        \draw[->,line width=0.25mm] (eval_prior)  -- (eval_bfim);
        \draw[->,line width=0.25mm] (eval_fim)  -| (eval_bfim);
        % \draw[->,line width=0.25mm,color=blue] (eval_bfim)  -| (no);
        % \draw[-,dashed,line width=0.25mm,color=red] (prior_score)  -- (loss_prior);


        % \draw[->,line width=0.25mm] (loss_post)  -- ( post_score);
        % \draw[->,line width=0.25mm] (post_score)  -- (eval_bfim_post);
        % \draw[->,line width=0.25mm] (eval_bfim_post)  -- (inv_post);
        % \draw[->,line width=0.25mm,color=blue] (inv_post)  -- (no);
        % \draw[->,line width=0.25mm,color=blue] (eval_bfim)  -| (no);
\end{tikzpicture}     
        \caption{Measurement-Prior Approach}
        \label{fig:main_lik_prior}
    \end{subfigure}
    
    \caption{ {%An overview of 
    \name{}: Overview. 
    %that compares the measurement prior approach with the posterior approach. 
    %In the  
    Fig.~\ref{fig:main_post}: %the 
    Posterior Approach. % is described: 
    The learning phase consists of a single training step for the posterior score; this neural net model is subsequently used in the evaluation phase. %In 
    Fig.~\ref{fig:main_lik_prior}: Measurement-Prior Approach. % is detailed: 
    The learning phase comprises two consecutive training steps: %—first 
    (i) training the prior score; and (ii)} training the measurement score—yielding two distinct NN models: the Prior and the Fisher scores. These models are then applied in the evaluation phase. }
    \label{fig:main}
\end{figure*}

 
\subsubsection{Score Learning}\label{sec:score_learning_post}
 To learn $\nabla_{\p}\log\probt{\p|\xsetr}{\pr|\xset}$, we suggest to use conditional score matching \cite{hyvarinen2005estimation,song2019generative,song2020score}. The goal is to {to fit} a model $\postscore{\p}{\xsetr; \Omega}$ parameterized by $\Omega$ (usually implemented as a neural network) to %obtain 
 the true score function $\nabla_{\p}\log\probt{\p|\xsetr}{\pr|\xset}$ {by minimizing w.r.t. $\Omega$ the discrepancy between the two expressed by} the objective
\begin{equation}
\label{eq:score_matching_objective}
    \lossbs\brackets{\Omega}=%\expectation{
    \mathbb{E}_{\pr, \xset} \norm{\postscore{\pr}{\xset;\Omega}-\nabla_{\pr}\log\probt{\pr|\xset}{\pr|\xset}}_2^2
    %}{\pr,\xset}
\end{equation}
Since we do not have direct access to $\nabla_{\p}\log\probt{\p|\xsetr}{\pr|\xset}$, only a set of i.i.d. samples $\ds$, we cannot directly minimize the objective %in
\eqref{eq:score_matching_objective}. 
%In a 
Similar %way as 
to standard score matching, an %alternative 
equivalent objective function is used {that does not require direct access to $\nabla_{\p}\log\probt{\p|\xsetr}{\pr|\xset}$:
\begin{align}\label{eq:conditional_score_matching}
&\lossbst\brackets{\Omega}=\expectation{\ell_B\brackets{\X,\pr;\Omega}}{\X,\pr}=\\
&\expectation{\norm{\postscore{\pr}{\xset;\Omega}}_2^2}{\pr,\xset}+2\trace{\expectation{\frac{\partial \postscore{\pr}{\xset;\Omega}}{\partial\pr}}{\xset,\pr}}.\nonumber
\end{align}

Objective \eqref{eq:conditional_score_matching} is equivalent to \eqref{eq:score_matching_objective} for the purpose of finding the minimizer $\Omega^*$,
in the sense that the two only differ by a constant $C$ that is independent of $\Omega$, i.e., $\lossbs\brackets{\Omega}=\lossbst\brackets{\Omega}+C$, provided that the following conditions hold.
%utilized that requires the  following 
First,} the boundary condition
\begin{equation}\label{eq:boundary_conditions_post}
    \lim\limits_{\p\rightarrow \Psb}\postscore{\p}{\xsetr}\probt{\p|\xsetr}{\pr|\xset}=0,\quad\forall\xsetr,
\end{equation}
% \end{assumption}
{%If the boundary conditions in~\eqref{eq:boundary_conditions_post} hold along some 
Second, the following} regularity conditions. 
\begin{assumption}\label{ass:score_cond_reg}$\,$

\begin{enumerate}[label={\ref*{ass:score_cond_reg}}.\arabic*,labelsep=*, leftmargin=*]
    % \setcounter{enumi}{9}
\item \label{assum:diff_prob_post} The {log-posterior} $\log\probt{\p|\xsetr}{\pr|\xset}$ is differentiable w.r.t. $\p$ {at all $\xsetr$ and $\p\in\Ps$  where $\probt{\p,\xsetr}{\pr,\xset} > 0$}. 
% \todo[inline,color=green]{Shouldn't the condition be
% $\probt{\p,\xsetr}{\pr,\xset} > 0$ (joint PDF, rather conditional), because the expectations are wrt the joint PDF - so this is where we need the gradient to exist?\\
% HVH: Right and done.}

\item The expectation $\expectation{\norm{\nabla\log\probt{\pr|\xset}{\pr|\xset}}_2^2}{\pr,\xset}$ is finite. \label{assum:bound_expection_post}
\item The score neural network $\postscore{\p}{\xsetr}$ is differentiable w.r.t. $\p$. \label{assum:net_cond_score}
%\todo[inline,color=green]{What score network?}
\item The expectation $\expectation{\norm{\postscore{\pr}{\xset}}_2^2}{\pr,\xset}$  is finite. \label{assum:expected_cond_score}

\end{enumerate}
\end{assumption}
% \ybdelete{Then an alternative loss function can be derived that does not require direct access to $\nabla_{\p}\log\probt{\p|\xsetr}{\pr|\xset}${, by using the following relation:}
% % \begin{equation}
%     $\lossbs\brackets{\Omega}=\lossbst\brackets{\Omega}+C,$
% % \end{equation}
% where $C$ is a constant independent of $\Omega$,
% \begin{align}\label{eq:conditional_score_matching}
% &\lossbst\brackets{\Omega}=\expectation{\ell_B\brackets{\X,\pr;\Omega}}{\X,\pr}=\\
% &\expectation{\norm{\postscore{\pr}{\xset;\Omega}}_2^2}{\pr,\xset}+2\trace{\expectation{\frac{\partial \postscore{\pr}{\xset;\Omega}}{\partial\pr}}{\xset,\pr}},\nonumber
% \end{align}
% is the score matching alternative objective} \hvhdelete{and $\matsym{J}_{B}\brackets{\Omega}\triangleq\expectation{\frac{\partial \postscore{\pr}{\xset;\Omega}}{\partial\pr}}{\xset,\pr}$ is the expected Jacobian matrix of $\postscore{\p}{\xset;\Omega}$ w.r.t. $\pr$.}
\begin{remark}\label{remark:condtion_hold_post}
    %
    {Assumptions~\ref{assum:diff_prob_post} and \ref{assum:bound_expection_post} 
%is 
are implied by the regularity Assumptions~\ref{sas:derivative} and \ref{assume:non_singular} of the BCRB, %Assumptions~\ref{sas:derivative} and~\ref{assume:non_singular}, 
respectively.}
{On the other hand,}
Assumptions~\ref{assum:net_cond_score}, \ref{assum:expected_cond_score} and the boundary condition \eqref{eq:boundary_conditions_post} can be inherently satisfied by selecting an appropriate neural network architecture and non-linear activation function.
\end{remark}
The objective \eqref{eq:conditional_score_matching} enables 
{ to determine the optimum parameters $\Omega^*$ by minimizing a sample average version \eqref{eq:score_post_mean} using} 
a dataset $\ds$ of i.i.d. measurements.
{ 
This yields} an approximation of the Bayesian score $\postscores{\p}{\xsetr} \triangleq \postscore{\p}{\xsetr;\Omega^*}\approx\nabla_{\p}\log\probt{\p|\xsetr}{\pr|\xset}$. 
\subsubsection{Evaluation of the LBCRB}\label{sec:lbcrb_eval_post}
We replace the Bayesian score in \eqref{eq:bfim} with the learned one.%This results learned Bayesian Fisher Information Matrix
\footnote{%
{ By Assumption~\ref{ass:score_expection} we have $\expectation{\nabla_{\pr}\log\probt{{\X},\pr}{{\X},\pr}}{\X,\pr}=0$.  This suggests that the mean of the learned score function can be subtracted to eliminate any bias introduced by it. However, in our numerical experiments this mean subtraction had a negligible effect 
%on the results 
for moderate-size training sets (e.g., $\nds=600$.), and was not used. %
{%However, 
It may prove useful though} for situations with small $\nds$.     }} 
This results  in the learned Bayesian Fisher Information Matrix% which is given by:
\begin{equation}\label{eq:f_score}
\lbfimb\triangleq\expectation{\postscores{\pr}{\xset}\postscores{\pr}{\xset}^T}{\xset,\pr}.
\end{equation}
Since we cannot evaluate the expectation {in \eqref{eq:f_score}}, we replace the expectation with an empirical mean over the entire dataset $\mathcal{D}$ as shown in \eqref{eq:mean_fully}, %that results in 
producing $\lbfimbs$.
Finally, to obtain the LBCRB, we invert $\lbfimbs$, which results in ${\bcrb}\approx\lbcrbbs\triangleq\lbfimbs^{-1}$.

{The Posterior Approach addresses Problem~\ref{problem_one} in a simple way, requiring the training of only one model network for the posterior conditional score. However, this simplicity comes at the cost of two main drawbacks.} 
{First, because
%Given that
}
the \emph{Posterior} Approach involves directly learning of $\nabla_{\p}\log\probt{\p|\xsetr}{\pr|\xset}$, it focuses only on the score of $\xsetr$ without considering that $\xsetr$ may consist of a set of i.i.d. measurements.  This %oversight necessitates 
requires learning  {a score model with a  conditioning input $\xsetr$ of high dimension $\niiddata$, increasing the complexity of the neural network and the sample complexity for training it}. {Second}, if one has {some domain}  knowledge about the problem, it is unclear how to incorporate it into the score function model. In Section~\ref{sec:lik_learn}, we propose the Measurement-Prior Approach, which
{ overcomes both of these limitations, while providing additional advantages.}
\subsection{Measurement-Prior Approach}\label{sec:lik_learn}
% \ybdelete{Now, we present the \emph{Measurement-Prior} Approach which allows incorporating additional domain knowledge.}

% \hvhdelete{
% Here we represent the Bayesian score as a decomposition of the prior score $\nabla_{\p}\log\probt{\p}{\pr}$ and the fisher score $\nabla_{\p}\log\probt{\x|\p}{\x|\pr}$ which is given by:
% \begin{align}\label{eq:lik_prior_decompistion}
%     \nabla_{\p}\log\probt{\hvhreplace{\x}{\mathcal{\X}},\p}{\X,\pr}&=\nabla_{\p}\log\probt{\x|\p}{\X|\pr}\nonumber\\
%     &+\nabla_{\p}\log\probt{\p}{\pr}.
% \end{align}
% }

%
{%Here, 
In this approach, illustrated
%presented 
in Fig.~\ref{fig:main_lik_prior},
}
we employ the decomposition of the BFIM 
%as given 
in \eqref{eq:bfim_decomposition_base} %, dividing it 
into two components: the prior Fisher Information Matrix (FIM), which requires only the score $\nabla_{\p}\log\probt{\p}{\pr}$, and the measurement FIM, which relies on the Fisher score $\nabla_{\p}\log\probt{{\x}|\p}{{\X}|\pr}$.
To learn of the prior score we use the standard score matching presented in Section~\ref{sec:background}, whereas for learning the measurement score, we derive a new Fisher Score Matching (FSM) objective.
% \ybdelete{The entire score learning process is described in Section~\ref{sec:score_learning_lik}.}

{%By dividing 
The separation of the Bayesian score into two learned components %and allowing their learning,
provides several important} 
%we outline and highlight the 
advantages, which we highlight here, {and discuss in detail later in this section.

}
%The first advantage is that 
First, it enables to determine
%the establishment of 
bounds 
{for a measurement with} 
an arbitrary number of 
independently and identically distributed 
i.i.d samples without requiring additional data or training.

{Second, it enables to learn and use a model for the Fisher score for only a \emph{single measurement sample}, reducing the complexity of the model and facilitating its training. 
%The second advantage is the ability 
Third, it enables to}
to incorporate domain knowledge about the estimation problem %within 
into the score neural network{, further reducing its complexity and facilitating its training.
%, which is discussed inSection~\ref{sec:model_informed}. 
}


\subsubsection{Score learning}\label{sec:score_learning_lik}
We  learn two distinct score functions, parametrized by $\paramf$ and $\paramp$, respectively: the Fisher score function $\lscore{\x}{\p; \paramf}$ for a single measurement sample;  and the prior score function $\priorscore{\p;\paramp}$.
%each associated with their respective set of parameters $\paramf$ and $\paramp$.
% \hvhdelete{Substituting them into \eqref{eq:lik_prior_decompistion} yields the following:
% % \begin{equation}
% $\postscore{\p}{\hvhreplace{\x}{\mathcal{X}}}={\sum_{\tilde{\x}\in\mathcal{X}}}\lscore{\hvhreplace{\x}{\tilde{\x}}}{\p}+\priorscore{\p},$
% % \end{equation}
% where now $\Omega=\{\paramf,\paramp\}$ is the combine set of learnable weights.} 
% \ybdelete{To learn an informative fisher and prior score function, we divided the learning into two parts.}  
We begin by learning the prior score $\priorscore{\p {;\paramp}}\approx\nabla_{\p}\log\probt{\p}{\pr}$ using the conventional score matching method detailed in Section~\ref{sec:score_over_view}. To apply the objective %function 
in \eqref{eq:loss_prior_sm}, 
% {note that Assumption~\ref{ass:score_reg_prior} is satisfied by the regularity Assumption~\ref{sas:derivative_prior} of the BCRB.} 
we assume that 
the boundary condition %described in
\eqref{eq:boundary_conditions} 
and Assumptions~\ref{ass:score_reg_prior} hold  
for %similar 
reasons analogous to {those given in Remark~\ref{remark:condtion_hold_post}} for the {posterior} score matching. %assumption mentioned 
% \ybdelete{in Remark\hvhreplace{~\ref{remark:fisher_score}}{~\ref{remark:condtion_hold_post}} .}
% \todo[inline,color=green]{YB: 
% 1. The forward pointer to Remark~\ref{remark:fisher_score}, which in turn points back to Remark~\ref{remark:condtion_hold_post} is confusing. I think that it will be better to justify the assumptions here explicitly, perhaps pointing back to Remark~\ref{remark:condtion_hold_post}.
% \\
% 2. Remark~\ref{remark:fisher_score} is currently incorrect, as commented there. Also,  differentiability of $\log\probt{\p}{\pr}$ does not follow from \ref{sas:derivative}, as commented after \eqref{eq:PriorFim}. We need to add an extra assumption there.\\
% \textbf{HVH:} 1. I think it is more correcrt just point to remark 5.\\
% 2.  Once we clear the base assumption of the BCRB I will correct remake 6 -> Cleared.

% }

Next we address the learning of the Fisher score function. We wish to minimize the mismatch between the true and model scores in the objective
% \hvhreplace{
% \begin{align}
%         &\lossfs\brackets{\paramf}\nonumber\\
%         &=\expectation{\norm{\lscore{\X}{\p;\paramf}-\nabla_{\pr}\log\probt{\X|\pr}{\X|\pr}}_2^2}{\X,\pr}.
% \end{align}}
{
\begin{align}
        &\lossfs\brackets{\paramf}\nonumber\\
        &=\expectation{\norm{\lscore{{\X}}{\p;\paramf}-\nabla_{\pr}\log\probt{{\X}|\pr}{{\X}|\pr}}_2^2}{{\X},\pr}.
\end{align}
}
However, in this case, we cannot use the standard {(conditional)} score matching {technique, because unlike the latter, where the gradient is with respect to the conditioned variable, the gradient in }
%on %\hvhreplace{$\nabla_{\p}\log\probt{\x|\p}{\X|\pr}$}
{$\nabla_{\p}\log\probt{{\x}|\p}{{\X}|\pr}$} {is with respect to the conditioning variable $\p$.} 
To {handle this fundamentally different scenario}, we 
{derive a new Fisher score matching objective. It requires the boundary condition \begin{equation}\label{eq:boundary_condtions_gen_direct}
    \lim\limits_{\p\rightarrow\Psb}  \lscore{{\x}}{\p}\probt{{\x},\p}{{\X},\p} =0,\quad \forall {\x}\in\widetilde{\Upsilon}
\end{equation} and the following regularity conditions.
}
\begin{assumption}[Fisher Score Matching Regularity]\label{ass:score_reg}$\,$
    \begin{enumerate}[label=\text{\ref{ass:score_reg}}.\arabic*,labelsep=*, leftmargin=*]
    \item The %\ybreplace{p.d.f.}
    {measurement {log-likelihood}} %\hvhreplace{$\probt{\x|\p}{\X|\pr}$}
    {$\log\probt{{\x}|\p}{{\X}|\pr}$} is differentiable w.r.t. $\p$ {at all $\x\in\Upsilon$ and $\p\in\Ps$  where $\probt{{\x},\p}{{\X},\pr} > 0$} \label{assum:diff_prob}. 

    \item\label{ass:bound_expection} The expectation %\hvhreplace{$\expectation{\norm{\nabla_{\pr}\log\probt{\x|\pr}{\X|\pr}}_2^2}{\X,\pr}$}
    {$\expectation{\norm{\nabla_{\pr}\log\probt{{\X}|\pr}{{\X}|\pr}}_2^2}{{\X},\pr}$} is finite. 
    \item The score neural network %\hvhreplace{$\lscore{\x}{\p}$}
    {$\lscore{{\x}}{\p}$} is differentiable w.r.t. $\p$. \label{assum:diff_net} 
    \item\label{ass:bound_expection_v2} The expectation   %\hvhreplace{$\expectation{\norm{\lscore{\X}{\pr}}_2^2}{\X,\pr}$}
    {$\expectation{\norm{\lscore{{\X}}{\pr}}_2^2}{{\X},\pr}$} is finite. 
\end{enumerate}
\end{assumption}

\begin{remark}\label{remark:fisher_score}
Assumptions  \ref{assum:diff_net}, \ref{ass:bound_expection_v2}, and the boundary conditions  \eqref{eq:boundary_condtions_gen_direct} hold by the arguments %as present 
in Remark~\ref{remark:condtion_hold_post}.  {As for Assumption~\ref{assum:diff_prob} it implied by Assumptions~\ref{sas:derivative} and \ref{sas:derivative_prior}.}

Similarly, by \eqref{eq:bfim_decomposition_base} and \eqref{eq:non-BayesFim}, 
Assumption~\ref{ass:bound_expection} is implied by the non-degeneracy BCRB regularity Assumption~\ref{assume:non_singular}.

\end{remark}
\begin{theorem}[Fisher Score Matching]\label{thm:liklihood} 
{Suppose that the boundary condition \eqref{eq:boundary_condtions_gen_direct} and the
%Assume that some 
}
regularity conditions Assumptions~\ref{ass:score_reg}) hold.
% \ybdelete{Also, assume that the following boundary condition holds:
% % \hvhreplace{
% % \begin{equation}\label{eq:boundary_condtions_gen_direct}
% %     \lim\limits_{\p\rightarrow\Psb}  \lscore{\x}{\p}\probt{\vectorsym{x},\p}{\X,\p} =0,\quad \forall \x\in\Upsilon
% % \end{equation}}
% {\begin{equation}\label{eq:boundary_condtions_gen_direct}
%     \lim\limits_{\p\rightarrow\Psb}  \lscore{{\x}}{\p}\probt{{\x},\p}{{\X},\p} =0,\quad \forall {\x}\in\widetilde{\Upsilon}
% \end{equation}}
% Then $\lossfs=\lossfst+C$, where $C$ is a constant independent of $\paramf$,
% }
% \hvhreplace{\begin{align}\label{eq:score_match_param_gen_direct}
%     \lossfst\brackets{\paramf}&=\expectation{\norm{\lscore{\X}{\pr;\paramf}}_2^2}{\randomvec{X},\pr}+2\trace{\matsym{J}_{F}\brackets{\paramf}}\nonumber \\ &+2\expectation{\lscore{\X}{\pr;\paramf}^T\nabla_{\p}\log\probt{\pr}{\pr}}{\randomvec{X},\pr},
% \end{align}}
{Define}\begin{align}\label{eq:score_match_param_gen_direct}
    &\lossfst\brackets{\paramf}=2\expectation{\lscore{{\X}}{\pr;\paramf}^T\nabla_{\p}\log\probt{\pr}{\pr}}{{\X},\pr} +\\
    &2\trace{\expectation{\frac{\partial \lscore{\x}{\pr;\paramf}}{\partial\pr}}{\X,\pr}}+\expectation{\norm{\lscore{{\X}}{\pr;\paramf}}_2^2}{{\X},\pr}.\nonumber 
\end{align}
{Then $\lossfs=\lossfst+C$, where $C$ is a constant independent of $\paramf$.}
% \ybdelete{
% is the alternative objective function }\hvhdelete{and  \hvhreplace{$\matsym{J}_{F}\brackets{\paramf}\triangleq\expectation{\frac{\partial \lscore{\x}{\pr;\paramf}}{\partial\pr}}{\X,\pr}$}{$\matsym{J}_{F}\brackets{\paramf}\triangleq\expectation{\frac{\partial \lscore{{\X}}{\pr;\paramf}}{\partial\pr}}{{\X},\pr}$} is the expected Jacobian matrix of \hvhreplace{$\lscore{\x}{\p;\paramf}$}{$\lscore{{\x}}{\p;\paramf}$} w.r.t. $\p$}.
\end{theorem}
%The proof of 
Theorem~\ref{thm:liklihood} is proved %shown 
in Appendix~\ref{sec:lik_score_proof}. {Comparing the new Fisher Score Matching objective \eqref{eq:score_match_param_gen_direct}
with the standard conditional score matching objective (e.g., \eqref{eq:conditional_score_matching}), we notice that the difference is the added new first term in \eqref{eq:score_match_param_gen_direct}. This term involves the expectation of the inner product of the model conditional score of the conditioned variable $\X$ (i.e., the Fisher score $\lscore{{\X}}{\pr}$) with the true score of the conditioning variable $\pr$ (i.e, the true score of the prior, $\nabla_{\p}\log\probt{\p}{\pr}$).}
%{
% \ybdelete{\begin{remark}\label{remark:fisher_score}
% %Regarding the satisfaction of the assumptions outlined in Assumption~\ref{ass:score_reg}: 
% %(i) ~
% \hvhdelete{
% \ref{assum:diff_prob} 
% is implied by Assumption~\ref{sas:derivative} 
% -- one of the BCRB's regularity conditions; 
% %(ii) 
% Assumptions ~\ref{assum:diff_net} and ~\ref{ass:bound_expection_v2} can be inherently satisfied by selecting an appropriate neural network architecture and non-linear activation function;
% } %(iii) 
% %{
% Assumptions \ref{assum:diff_prob}, \ref{assum:diff_net}, \ref{ass:bound_expection_v2}, and the boundary conditions  \eqref{eq:boundary_condtions_gen_direct} hold by the arguments %as present 
% in Remark ~\ref{remark:condtion_hold_post}.  
% \todoin{YB: A noted in the comment after \eqref{eq:PriorFim}, it is not true that Assumption~\ref{assum:diff_prob} follows from Assumption~\ref{sas:derivative}. It will, if we assume differentiability of the log prior. }
% %
% {%While 
% Similarly, by \eqref{eq:bfim_decomposition_base} and \eqref{eq:non-BayesFim}, 
% Assumption ~\ref{ass:bound_expection} is implied by the non-degeneracy BCRB regularity Assumption~\ref{assume:non_singular}.
% }
% \end{remark}
% }

% \hvhdelete{To apply the objective function in \eqref{eq:score_match_param_gen_direct}, we assume that the boundary condition described in \eqref{eq:boundary_condtions_gen_direct} is satisfied. To ensure this, we adopt the assumption~\ref{assume:boundary_prior_lim} and ~\ref{assume:boundary_post_lim} together with the assumption that $\norm{\lscore{\x}{\p}}_2\leq \infty\quad \forall\p\in\Ps\quad\mathrm{and}\quad \x\in\Upsilon$.}

To employ Theorem~\ref{thm:liklihood}, %it is necessary to know 
requires the true prior score function. %However, 
Instead, we substitute the prior score function with the learned version $\priorscore{\p}\approx \nabla_{\p}\log\probt{\p}{\pr}$. This allows us to leverage a dataset of i.i.d. measurements and parameter pairs to learn the Fisher score %through 
using the objective %shown in 
\eqref{eq:score_lik_mean}.
\subsubsection{\pe{} Score Neural Network}\label{sec:model_informed}
{The separation of the modeling and learning of the prior score from that of the Fisher (measurement) score in the Measurement-Prior Approach opens a new opportunity to} integrate domain-knowledge (e.g., knowledge of the physics of the measurement process) into the {Fisher score model. This reduces the NN model complexity, and improves sample complexity.}  




{Specifically, } let us assume that we know some deterministic physical model $\mathcal{M}\brackets{\p}$ that relates the parameter vector $\p$ to the measurement $\X$ %using %\hvhreplace{second order statistics}
{in the following manner}:
\begin{equation}\label{eq:physical_model_score}
    \probt{{\x}|\p}{{\X}|\pr}=\probt{{\x}|\mathcal{M}\brackets{\p}}{{\X}|\vectorsym{\tau}},
\end{equation}
where $\probt{{\x}|\vectorsym{\tau}}{{\X}|\vectorsym{\tau}}$ is an unknown PDF %distribution 
parameterized by the known physical model $\mathcal{M}$. %
{%Note that 
Such a representation can fit several signal processing problems as demonstrated in {Sec.~\ref{subsubsec:MoISNN} and further} in Sec.~\ref{sec:example_models}. }
The score function of the model presented in \eqref{eq:physical_model_score} is given by:
% \begin{equation*}
    $\nabla_{\p}\log\probt{{\x}|\pr}{{\X}|\p}=\divc{\mathcal{M}\brackets{\p}}{\p}^T\at{\nabla_{\vectorsym{\tau}}\log\probt{{\x}|\vectorsym{\tau}}{{\X}|\vectorsym{\tau}}}{\vectorsym{\tau}=\mathcal{M}\brackets{\p}}.$
% \end{equation*}
Since $\probt{{\x}|\vectorsym{\tau}}{{\X}|\vectorsym{\tau}}$ is unknown we replace it %via 
by a neural network model $\iscore{\x}{\vectorsym{\tau}}$.  {The combined model for the Fisher score then becomes} \eqref{eq:model_base_score}.
% \hvhdelete{
% \begin{equation}\label{eq:model_base_score_without_iid}
% \vectorsym{s}_{F}\brackets{\tilde{\x}|\p;\paramf}=\divc{\mathcal{M}\brackets{\p}}{\p}^T\at{\iscore{\tilde{x}}{\vectorsym{\tau};\paramf}}{\vectorsym{\tau}=\mathcal{M}\brackets{\p}}.
% \end{equation}
% Finally, combining \eqref{eq:model_base_score_without_iid} and \eqref{eq:iid_score} we obtain \eqref{eq:model_base_score}.}
%An overview of 
The \pe{} Score 
Neural Network is illustrated %shown 
in Figure~\ref{fig:model_inforamed}.
%{
\begin{remark}\label{remark:opt_mi}
    To optimize the parameters %
    {$\paramf$ defining} $\vectorsym{s}_{I}$, we substitute \eqref{eq:model_base_score} into \eqref{eq:score_lik_mean}, implying that the optimization process relies on a dataset consisting of %\ybreplace{measurement ($\x$)-parameter ($\p$) pairs rather than $\vectorsym{\tau}$}
    {measurement-parameter ($\x,\p$) rather than $(\x,\vectorsym{\tau})$ pairs}. 
    %It is important to note that
    Importantly, $\vectorsym{\tau}=\mathcal{M}\brackets{\p}$ is calculated during the optimization for every given value of $\p$.
\end{remark}
%}
\subsubsection{Evaluation of the LBCRB}\label{sec:lbcrb_eval_lik}
%Now we will present the use of 
The Prior Score and Fisher Score {models} learned in Section~\ref{sec:lbcrb_eval_lik} 
%by showing 
are used in the evaluation phase of the \emph{Measurement-Prior} Approach. %
{Specifically, let $\lscores{\x}{\p}=\lscore{\x}{\p;\paramf^*}\approx\nabla_{\p}\log\probt{\x|\p}{\X|\pr}$ and $\priorscores{\x}=\priorscore{\p;\paramp^*}\approx\nabla_{\p}\log\probt{\p}{\pr}$ be the learned Fisher and prior scores, where $\paramf^*$ and $\paramp$ denote the minimizer's of \eqref{eq:score_lik_mean} and \eqref{eq:score_prior_mean}, respectively.  }
We use the learned Prior and Fisher scores to {express} their FIM's 
\begin{subequations}\label{eq:split_fim_exp}
\begin{equation}
    \tlmfim\triangleq\expectation{\lscores{{\X}}{\pr}\lscores{{\X}}{\pr}^T}{{\X},\pr},
\end{equation}
\begin{equation}
\tlpfim\triangleq\expectation{\priorscores{\pr}\priorscores{\pr}^T}{\pr}.
\end{equation}
\end{subequations}
%Then we utilized 
Substituting into the decomposition of the Bayesian FIM in \eqref{eq:bfim_decomposition_base} 
%to obtain:
yields {an approximation for the Bayesian FIM for a measurement contatining $\niideval$ i.i.d samples,}
\begin{equation}\label{eq:f_score_decomposition}
    \fb\approx\lbfimlp\brackets{\niideval}\triangleq\niideval\cdot\tlmfim+\tlpfim.
\end{equation}
To compute \eqref{eq:split_fim_exp}, we replace the expectation with an empirical mean over the training dataset  $\ds$, %which results 
in \eqref{eq:mean_efim_likd} and \eqref{eq:bprior_mean}. 
{Combining the two learned FIMs as in \eqref{eq:f_score_decomposition} yields
the final learned Bayesian FIM (LBFIM) \eqref{eq:bfim_apx_final} of the Measurement-Prior Approach,
with the corresponding LBCRB $\lbcrblps(\niideval)=\lbfimlps^{-1}(\niideval).$}
 
We would like to emphasize that \eqref{eq:f_score_decomposition} and \eqref{eq:bfim_apx_final} enable us to compute a {learned} BCRB for a measurement containing any desired number $\niideval$ of i.i.d. samples, {regardless of the number $\niiddata$ of i.i.d measurement samples available in the training data set $\ds$ for each value of $\p$.} 

{The explicit use of the statistical independence of the $\niideval$ measurement samples in the Measurement-Prior Approach provides two additional advantages:} (1) 
%we can generalize 
the bound generalizes to any number $\niideval$ of i.i.d. samples {in the measurement,} without  training a new model; and (2) {
the model $\lscore{\x}{\p;\paramf}$ for the score function for a single measurement sample is far simpler than that for multiple samples, making training easier and reducing the number of training samples required.}
 
% \ybdelete{To compute \eqref{eq:split_fim_exp}, we replace the expectation with an empirical mean over the training dataset  $\ds$, which results in \eqref{eq:mean_efim_likd} and \eqref{eq:bprior_mean}. To provide a bound for an arbitrary number of i.i.d. samples $\niideval$, we utilized the FIM decomposition in \eqref{eq:bfim_decomposition_base} and computed the LBFIM (denoted by $\lbfimlps$) as shown in \eqref{eq:bfim_apx_final}. Finally, to obtain the LBCRB we invert \eqref{eq:bfim_apx_final} which results in $\lbcrblps=\lbfimlps^{-1}.$ .}

% \todo[inline,color=green]{Broken sentence and missing equation?}
\section{Theoretical Results}\label{sec:theory}
Here, we %theoretically 
investigate the errors of the \name{} due to learning error. %
{Following the standard approach in } learning theory \cite[Chapter 5]{shalev2014understanding}, we divide the learning error into two components: the \emph{approximation error} and the \emph{empirical-mean error}\footnote{In {learning theory} \cite[Chapter 5]{shalev2014understanding}, the empirical mean error is %termed 
known as the "estimation error". However, to avoid confusion with the estimator error %as defined 
in \eqref{eq:bcrb}, we %denote 
refer to it here as the "empirical-mean error".} . We then examine each of these errors individually and their collective impact.
 % \ybdelete{Initially, we evaluate the approximation error of both approaches, demonstrating that the relative score matching objective provides an upper bound on the Fisher Information Matrix (FIM) errors.  Subsequently, we explore the effect of employing the empirical mean in place of the expected value and the constraints imposed by a finite dataset.} %
 % \ybdelete{Afterwards, we present the combined error and the condition when \name{} is invertible and has relation to the score matching objective.}
 In the last part, we demonstrate that the {learned score neural network models are strongly consistent approximations of}  %with 
 the true scores, and that \name{} converges {with probability 1 to the BCRB as the size of the training data set $\ds$ increases,} providing conditions to achieve an accurate approximation. 



% \hvhdelete{Finally, we present the combined error and show the condition under which we obtain a good approximation. Following the %
% {approximation error}, we present the condition when \name{} provides an invertible and there is a relation to the score matching objective. }

{The results} in this section
%, we utilized 
use the  concept of \emph{intrinsic dimension of a matrix}\cite{tropp2015introduction,ipsen2024stable} as a measure of its {\emph{effective}} dimension: %here we prove its determination: 
\begin{definition}[Intrinsic Dimension of a Matrix]
    Let $\matsym{A}\in\mathbb{R}^{d\times d}$ be a non-zero positive semi-definite square matrix. Then its intrinsic dimension is defined as:
    \begin{equation*}
        \mathrm{intdim}\brackets{\matsym{A}}\triangleq\frac{\trace{\matsym{A}}}{\norm{\matsym{A}}_2}.
    \end{equation*}
\end{definition}
The intrinsic dimension of a matrix can be understood as quantifying the number of dimensions
%. Importantly, %The Intrinsic Dimension Of a Matrix 
%it considers 
{by accounting for} the spectral intensity over all dimensions. 
% \ybdelete{%To further explore its relationship with various quantities, refer to the following inequality:
% % \begin{equation*}
%     $1\leq \mathrm{intdim}\brackets{\matsym{A}} \leq \mathrm{rank}\brackets{\matsym{A}}\leq d.$
% % \end{equation*}
% }
From 
%the inequality and 
the definition of intrinsic dimension and the inequality $1\leq \mathrm{intdim}\brackets{\matsym{A}} \leq \mathrm{rank}\brackets{\matsym{A}}\leq d,$
we observe that $\mathrm{intdim}\brackets{\matsym{A}}=d$ when all %signaler 
eigenvalues are identical and nonzero, whereas $\mathrm{intdim}\brackets{\matsym{A}}=1$ when $\matsym{A}$ has rank one. In addition, the intrinsic dimension is more robust to small perturbations than the matrix rank \cite{ipsen2024stable}. 

\subsection{Approximation error}
{We evaluate the approximation error of both approaches, demonstrating that the %relative 
}
score matching objective {that expresses the score mismatch} provides an upper bound on the {learned} Fisher Information Matrix (FIM) errors.
\begin{theorem}[\name{} %
{Approximation Error:} %\ybreplace{Direct}
{Posterior} Approach]\label{thm:lrn:direct}
    Let $\mathrm{sRE}_B\triangleq\frac{\lossbs}{\trace{\fb}}$ be the %
    {relative %
    {approximation error} of the} Bayesian score 
    %
    {and let $\db=\mathrm{intdim}\brackets{\F_B}$.} Suppose that Assumptions~\ref{assum:bcrb_reg}, \ref{ass:score_cond_reg} hold and $\db{\cdot \mathrm{sRE}_B}\leq {0.16}$. Then:
    % \begin{subequations}
    \begin{align}\label{eq:error_learn_post_approch}
        %
        {\mathrm{RE}_{B}^{(a)}}=\frac{\norm{\lbfimb-\fb}_2}{\norm{\fb}_2} &\leq\bfle \triangleq {2.4}\sqrt{\db\cdot\mathrm{sRE}_B}.
                                                % &=\lossbs+2\sqrt{\norm{\widetilde{\mathrm{LBFIM}}}_2\cdot \lossbs}
    \end{align}
   % \hvhdelete{ where $\mathrm{sRE}_B\triangleq\frac{\lossbs}{\trace{\fb}}$ is the {relative {approximation error} of the} Bayesian score.}
    % \end{subequations}
\end{theorem}

\begin{theorem}[\name{} %
{Approximation Error}: Measurement-Prior Approach]\label{thm:lrn:lik_prior}
    Let $\mathrm{sRE}_M\triangleq\frac{\lossfs}{\trace{\fm}}$ and $\mathrm{sRE}_P\triangleq\frac{\lossps}{\trace{\fp}}$ be the {relative errors in learning the Fisher and prior scores}, respectively and let $\dm=\mathrm{intdim}\brackets{\fm}$ and $\intdp=\mathrm{intdim}\brackets{\fm}$. %\ybdelete{be the intrinsic dimension of measurement FIM $\fm$ and Prior FIM $\fp$ , respectively.}
    %Assume 
    Suppose that Assumptions~\ref{assum:bcrb_reg}, \ref{ass:score_reg_prior}, and \ref{ass:score_reg} hold and ${\dm\cdot \mathrm{sRE}_M}\leq {0.16}$, ${\intdp\cdot \mathrm{sRE}_P}\leq {0.16}$. 
    Then:
    \begin{align}\label{eq:error_learn_lik_prior_approch}
        &%
        {\mathrm{RE}_{MP}^{(a)}}=\frac{\norm{\lbfimlp-\fb}_2}{\norm{\fb}_2} \leq  \mpfle\\
    &\mpfle\triangleq2.4\brackets{\frac{\norm{%
    {\niideval}\fm}_2}{\norm{\fb}_2}\sqrt{\dm\cdot\mathrm{sRE}_M}+\frac{\norm{\fp}_2}{\norm{\fb}_2}\sqrt{ \intdp\cdot\mathrm{sRE}_P}}\nonumber
    \end{align}
    % \hvhdelete{where
    % % \begin{equation*}
    %     $\mfle=2.4\sqrt{\dm\cdot\mathrm{sRE}_M}$
    % % \end{equation*}
    % % \begin{equation*}
    % and 
    %     $\pfle=2.4\sqrt{ \intdp\cdot\mathrm{sRE}_P}$
    % % \end{equation*}
    % % \begin{equation}
    % % \end{equation}
    % are the upper bounds on {approximation error} of the measurement FIM and the prior FIM, respectively. }
\end{theorem}

Theorems~\ref{thm:lrn:direct} and \ref{thm:lrn:lik_prior} %
{(proved in Appendix C)} %\ybreplace{provide a useful}
{quantify the} relationship between the %
{score matching} optimization objective, %
{(or equivalently, the relative error in %
{approximating} the score vector),} and the  %
{approximation error} %ybedit
{of the corresponding FIM}. They %\ybreplace{showing that}
{show how a} reduction in the objective (smaller score %
{approximation error}) %\ybreplace{will provide}
{translates to} a smaller learning error %
{for the FIM}.  %{
Moreover, the tightest bound for a selected neural network architecture is achieved when %
{$\Omega
%$ is equal to $
=\Omega^*$ --  the NN parameters that minimize the objective $\mathcal{L}$.} 
 %{
Finally, the assumptions $\mathrm{intdim}\brackets{\fb}\frac{\lossbs}{\norm{\fb}_2}\leq 0.16$, $\mathrm{intdim}\brackets{\fm}\frac{\lossfs}{\norm{\fm}_2}\leq 0.16$ and $\mathrm{intdim}\brackets{\fp}\frac{\lossps}{\trace{\fp}}\leq 0.16$ 
% \todo[inline,color=green]{YB: update to match the actual assumptions in the Thms.}
introduced in Theorems~\ref{thm:lrn:direct} and \ref{thm:lrn:lik_prior} can be satisfied  by minimizing the score %\ybreplace{loss functions}
{objectives} $\lossbs, \lossfs, \text{and } \lossps$.


%
{Comparing between the theoretical results on the {approximation error} of the two approaches reveals 
%we observe 
that the Measurement-Prior Approach will %benefit 
have an advantage in cases where the measurement FIM is the dominant part, and some additional information can be used in the Fisher (Measurement) score model,
%-- the neural network has 
%some additional information, 
e.g. the reuse of i.i.d. samples,  or a model-informed score neural network. This is also demonstrated numerically in Fig.~\ref{fig:k_study} and Fig.~\ref{fig:analysis_mi}.    }

\subsection{Empirical Mean Error}\label{sec:empirical_mean_error}
In this section, we discuss the error introduced by using a finite empirical mean in \eqref{eq:mean_fully}, \eqref{eq:mean_efim_likd}, and \eqref{eq:bprior_mean} instead of the true expectation. We 
%start by 
assume that the %{output of the }
learned score function is bounded for any sample in the dataset $\ds$, namely:
\begin{assumption}[%\ybreplace{Bounded Outer Product Assumption}
{Score Function Bounds}]\label{assume:bounded_score}
For $\Omega^*$, $\paramp^*$ and $\paramf^*$ we assume that:
    % The FIM are bounded for every sample in $\ds$:
\begin{enumerate}[label=\text{\ref{assume:bounded_score}}.\arabic*,labelsep=*, leftmargin=*]
    \item\label{assume:bound_post} $\norm{\postscores{\p_n}{\xsetr_n}}_2^2\leq \cb \quad \forall \xsetr_n,\p_n \in \ds$.
    \item \label{assume:bound_fisher}$\norm{\frac{1}{\niiddata}\sum_{\x\in\xsetr_n}\lscores{\x}{\p_n}\lscores{\x}{\p_n}^T}_2\leq \cm \quad \forall  \xsetr_n,\p_n  \in \ds $ .
\item\label{assume:bound_prior} $\norm{\priorscores{\p_n}}_2^2\leq \cp \quad \forall  \p_n  \in \ds$.
\end{enumerate}
\end{assumption}

{\begin{remark}
The bounds $\cb, \cm$, and $\cp$ are determined for the specific realization in $\ds$, \emph{after it has been observed.} {Hence these bounds are functions of $\ds$, $\cb=\cb(\ds), \cm=\cm(\ds), \cp=\cp(\ds)$, but this dependence will not be shown explicitly.}
\end{remark}
}

Assumptions~\ref{assume:bounded_score} can be inherently satisfied by selecting an appropriate neural network architecture and nonlinear activation function and by the assumption that all samples in finite size set $\ds$ are bounded as well. {Subject to these assumption, we obtain the following results.}

% \todo[inline,color=green]{YB: I think that the results of the paper are correct and complete at this point, and I started a new pass on the technical parts. \\
% 1. I accepted all the previous edits from the beginning through Thm. V.3, so that new edits are clearer.\\
% 2. Edited Thm. V.3 for a cleaner presentation. \\
% 3. @Hai, can you make the corresponding adjustments in the proof of Thm V.3 in the Appendix? Also, can you make similar edits in Thm. V.4?\\
% HVH?: Yes and done. Also note that 
% }
\begin{theorem}[LBCRB Empirical Mean Error: Posterior Approach]\label{thm:sampling_post}
%
{Suppose that Assumptions~\ref{assum:bcrb_reg}, \ref{ass:score_cond_reg}, and the score bound %score  
condition~\ref{assume:bound_post} hold.} 
    Let $\dbb=\mathrm{intdim}\brackets{\lbfimb}$,
    % \ybdelete{be the intrinsic dimension of matrix $\lbfimb$. Assume that assumptions~\ref{assum:bcrb_reg}, ~\ref{ass:score_cond_reg}, the bound score  condition ~\ref{assume:bound_post}} \hvhdelete{and $\nds>>\dbb\frac{C_B^2}{\trace{\lbfimb}}$} 
    %
    {and define 
    $\nbe\triangleq\frac{4}{3}\brackets{u+\log\brackets{8 \dbb}}\brackets{\frac{\cb}{\trace{\lbfimb}}\cdot \dbb+1}$. %hold, 
    }
    Then for any $u>0$ 
    % \ybreplace{
    % {and $\nds\geq \frac{4}{3}\nbe\triangleq\frac{4}{3}\brackets{u+\log\brackets{8 \dbb}}\brackets{\frac{\cb}{\trace{\lbfimb}}\cdot \dbb+1}$}}
    {and $\nds \geq \nbe$} the following bound holds with probability %
    {of at least } $1-\exp{\brackets{-u}}$:
    \begin{align}\label{eq:re_error_post_mean}
        %
        {\mathrm{RE}_{B}^{(e)}}=&\tfrac{\norm{\lbfimbs-\lbfimb}_2}{\norm{\lbfimb}_2}\leq \bfse\triangleq 
        % \ybreplace{\sqrt{3\frac{\nbe}{\nds}}}
        {1.5\sqrt{\frac{\nbe}{\nds}}}.
    \end{align}
    % \hvhdelete{where,
    % % \begin{align}
    % $\mathrm{Q}\brackets{u,d,r}={u+\log\brackets{8d}}\brackets{r\cdot d+1}$
    % {is the function of empirical mean error bound, $\dbb=\mathrm{intdim}\brackets{\lbfimb}$ is the intrinsic dimension of matrix $\lbfimb$. }} 
    Furthermore, %
    {define
    $\nbet=\brackets{\log\brackets{1+2\dbb} +{0.52}} \cdot\brackets{\dbb\frac{\cb}{\trace{\lbfimb}}+1}$. Then for any $\nds\geq\nbet$ the relative empirical mean error is bounded in expectation by
    }
     \begin{align}\label{eq:emprical_mean_error_post_expection}
        \expectation{\mathrm{RE}_{B}^{(e)}}{\ds}\leq
        {\frac{6+1.5\sqrt{2\log\brackets{1+2\dbb}}}{\sqrt{\log\brackets{1+2\dbb} +{0.52}}
        }}\sqrt{\frac{\nbet}{\nds}}.
    \end{align}
    %{ where $\nbet=\frac{\cb}{\norm{\lbfimb}_2}+1$.}
\end{theorem}

\begin{theorem}[LBCRB Empirical Mean Error: Measurement-Prior Approach]\label{thm:sampling_mp}
    Suppose that  Assumptions~\ref{assum:bcrb_reg}, \ref{ass:score_reg_prior}, \ref{ass:score_reg} and the bounded score conditions~\ref{assume:bound_fisher} and \ref{assume:bound_prior} hold.
    Let %
    { $\dmpb=\mathrm{intdim}\brackets{\lbfimlp}$, and define $\nmpe\triangleq \frac{4}{3}\brackets{u+\log\brackets{8 \dmpb}}\brackets{\frac{ \cm+\frac{\cp}{\niideval}}{\trace{\tlmfim}+\frac{\trace{\tlpfim}}{\niideval}}\cdot \dmpb+1}$. }
     Then for any $u>0$ and $\nds\geq \nmpe$ the following bound holds with probability {of at least } $1-\exp{\brackets{-u}}$:
    \begin{align}\label{eq:re_error_mp_mean}
        %
        {\mathrm{RE}_{MP}^{(e)}}=&\frac{\norm{\lbfimlps-\lbfimlp}_2}{\norm{\lbfimlp}_2}\leq \mpfse\triangleq {1.5}\sqrt{\frac{\nmpe}{\nds}}. 
    \end{align}
    Furthermore, define $\nmpet\triangleq\brackets{\log\brackets{1+2\dmpb} +{0.52}}\cdot\brackets{\dmpb\frac{ \cm+\frac{\cp}{\niideval}}{\trace{\tlmfim+\frac{\tlpfim}{\niideval}}}+1}$. Then %
    {for any $\nds\geq \nmpet$ the relative empirical mean error is bounded in expectation by}
     \begin{align}\label{eq:emprical_mean_error_mp_expection}
        &\expectation{\mathrm{RE}_{MP}^{(e)}}{\ds}\leq \frac{6+1.5\sqrt{2\log\brackets{1+2\dmpb}}}{\sqrt{\log\brackets{1+2\dmpb} +{0.52}}}\sqrt{\frac{\nmpet}{\nds}}.
    \end{align}
\end{theorem}
% \todo[color=green,inline]{YB: Moved here (and elaborated) the interpretation of Theorems \ref{thm:sampling_post} and \ref{thm:sampling_mp}, because this is about the finite $\nds$ case, not the case addressed by the corollaries. }
{Theorems \ref{thm:sampling_post} and \ref{thm:sampling_mp}, proved in Appendix~\ref{apx:proof_sample_error}, quantify, with explicit constants (and for sufficiently large $u$, with overwhelming probability) the effect of the size $\nds$ of the training set $\ds$ on the error in the estimated FIMs due to replacing expected values by finite means over $\ds$.  
We also observe
the effect of the intrinsic dimension of the respective FIM (which is upper bounded by $\np$):  as the effective number of 
%we have more
parameters to estimate grows, %we require 
more training samples $\nds$ are required 
to obtain the same FIM error, with the dependence being (up to log factors ) roughly linear in $\dbb$ or $\dmpb$. 


%{
{
%When examining the sampling error, 
Recall from Sec.~\ref{subsec:MP} that an advantage of the Measurement-Prior Approach 
over the Posterior Approach is that it is able to determine the LBCRB for any number $\niideval$ of i.i.d. samples using the same $\ds$, without any additional effort or data.
Another advantage of the Measurement-Prior Approach is revealed by a close comparison
of Thm.~\ref{thm:sampling_post} with Thm.~\ref{thm:sampling_mp}:
% (i) it allows  determination of the LBCRB for any number of i.i.d. samples, denoted as $\niideval$; (ii) 
 % it exhibits 
the Measurement-Prior Approach enjoys
a reduced 
%sampling 
finite mean}
error for datasets $\ds$ consisting of measurements with multiple $\niiddata > 1$ i.i.d. samples per parameter value.
%such that $\niiddata > 1$ . 
}

{The reason for this behavior is that, as we argue below, with high probability, $\cb\geq \niiddata \cm +\cp$. Because $\lbfimlp\approx \fb \approx \lbfimb $, we also have $\dbb \approx \dmpb$. Toghether, this implies that $\nbe \geq \niiddata \nmpe$, and in turn
$\bfse \geq \sqrt{\niiddata}\mpfse$, i.e, the upper bound in \eqref{eq:re_error_mp_mean} on the empirical-mean error for the Measurement-Posterior Approach, is, with high probability, smaller by a factor of $\sqrt{\niiddata}$ than the corresponding bound in \eqref{eq:re_error_post_mean} for the Posterior Approach. 

To justify the claim that with high probability, $\cb\geq \niiddata \cm +\cp$, 
recall the definitions of $\cb, \cm$, and $\cp$ in Assumption~\ref{assume:bounded_score} as the maxima over $\ds$ of the expressions on the left hand sides of the respective definitions. To compare these quantities, we first compare their expected values by} proving in Appendix~\ref{apx:remark_c_relation_proof} the following relation between the true score functions.
\begin{prop} \label{prop:expected_score_ineq}
    \begin{align}
&\expectation{\norm{\nabla_{\p}\log\probt{\p|\xsetr}{\p|\xset}}_2^2}{}\geq\expectation{\norm{\nabla_{\p}\log\probt{\p}{\pr}}_2^2}{}+\nonumber\\
    &\niiddata \expectation{\norm{\frac{1}{\niiddata}\sum_{i=1}^{\niiddata}\nabla_{\p}\log\probt{\x_i|\p}{\X|\pr}\nabla_{\p}\log\probt{\x_i|\p}{\X|\pr}^T}_2}{}.\nonumber
\end{align}
\end{prop}
{Now, assuming that the learning of the score functions was successful and the score mismatches $\mathrm{sRE}_B, \mathrm{sRE}_M$ and $\mathrm{sRE}_P$ defined in Theorem~\ref{thm:lrn:direct} and Theorem~\ref{thm:lrn:lik_prior} are small, the same inequality as in Proposition~\ref{prop:expected_score_ineq} will apply to the respective learned scores $\postscores{\p_n}{\xsetr_n},\lscores{\x}{\p_n}$ and $\priorscores{\p_n}$. For $\niideval>1$, the large gap between the expected values suggests also a similar large gap between the maxima, implying the claim that $\cb\geq \niiddata \cm +\cp$ with high probability. 

This analysis is supported by the empirical results in Fig.~\ref{fig:sample_error_vs_n_samples}, which show a consistently lower empirical-mean errors for the Measurement-Prior Approach than for the Posterior Approach, but in terms of the bounds, and the actual errors.
 }


{In %the consistency 
Section~\ref{subsec:consistency} we %present 
show, using Theorems~\ref{thm:sampling_post} and \ref{thm:sampling_mp},
that the empirical mean error vanishes as the size $\nds$ of the training data set grows.
%convergence  to zero base on  Theorems ~\ref{thm:sampling_post} and ~\ref{thm:sampling_mp}. 



\subsection{{\name{}} Learning Error% and Invertibility
}
\label{subsec:Conv_Invert}
{
}
% \ybreplace{Following Theorems ~\ref{thm:sampling_post} and ~\ref{thm:sampling_mp} we provide two theoretical insights into the \ybreplace{\name}{learned BCRBs using the two approaches}: 1)  a condition under which the learned BFIM is invertible {(i.e., the \name \   exists) }; and 2) a bound on the {relative error in} the \name \  due to the {Approximation error} and  empirical mean error.}
{
To obtain bounds on the deviation $\hat{\bcrb} -\bcrb $ of the \name{} from the BCRB due to learning error, we combine the effects of the approximation error (Theorems \ref{thm:lrn:direct} and \ref{thm:lrn:lik_prior}) and the empirical-mean error (Theorems \ref{thm:sampling_post} and \ref{thm:sampling_mp}, respectively) on the learned FIM, and propagate them through the inverse relation $\lbcrbbs = \lbfimbs^{-1} $ and $\lbcrblps = \lbfimlps^{-1}$. For the latter, we include conditions for the learned FIMs $\lbfimbs$ and $\lbfimlps$ to remain positive definite in the presence of learning errors, to ensure that its inverse, the \name, exists. This is expressed in the following corollaries, proved in Appendix~\ref{apx:proof_inv_re}.
}
\begin{corollary}\label{corr:bound_inv_post}
Suppose that the assumptions of Theorems  {\ref{thm:lrn:direct} and} \ref{thm:sampling_post}  hold {and 
\begin{equation}\label{eq:inv_cond_post}
    \frac{\norm{\lbfimb}_2}{\norm{\fb}_2}\bfse+\bfle <\frac{1}{\kappa\brackets{\fb}}.
\end{equation}
%if 
Then }
{$\lbfimbs\succ 0$} {(i.e., %the learned BCRB,
$\lbcrbbs$ exists)}, 
{and} the following holds with probability {of at least} $1-\exp\brackets{-u}$:
\begin{align}\label{eq:inv_bound_post}
    \mathrm{RE}_{B} {\triangleq
    \frac{\norm{\lbcrbbs-\bcrb}_2}{\norm{
    \bcrb
    }_2}}
\leq\kappa\brackets{\lbfimbs}\brackets{\frac{\norm{\lbfimb}_2}{\norm{\lbfimbs}_2}\bfse+\frac{\norm{\fb}_2}{\norm{\lbfimbs}_2}\bfle}
\end{align}
\end{corollary}

% \todo[inline]{Write proof of the corollary below.}
\begin{corollary}\label{corr:bound_inv_split}
Suppose that %all previous 
the assumptions of Theorems {\ref{thm:lrn:lik_prior} and} \ref{thm:sampling_mp} hold,
{and 
\begin{equation}\label{eq:inv_cond_mp}
    \frac{\norm{\lbfimlp}_2}{\norm{\fb}_2} \mpfse +\mpfle<\frac{1}{\kappa\brackets{\fb}}.
\end{equation}
Then,  
%if 
}
{$\lmfim\succ 0$ and $\lpfim\succ 0$} {(i.e., %the learned BCRB, 
$\lbcrblps$ exists)}, {and} the following
holds with probability {of at least} $1-\exp\brackets{-u}$:
\begin{align}\label{eq:inv_bound_mp}
\mathrm{RE}_{MP} & \triangleq\frac{\norm{\lbcrblps-\bcrb}_2}{\norm{\bcrb}_2} \nonumber \\
&\leq\kappa\brackets{\lbfimlps}\brackets{\frac{\norm{\lbfimlp}_2}{\norm{\lbfimlps}_2} \mpfse+\frac{\norm{\fb}_2}{\norm{\lbfimlps}_2}\mpfle}.
\end{align}
\end{corollary}
% \ybdelete{The proofs of 
% Corollaries ~\ref{corr:bound_inv_post} and ~\ref{corr:bound_inv_split} 
% can be found 
% in Appendix~\ref{apx:proof_inv_re}. }
%\todo[inline,color=green]{
% YB: what principle are you using to number corollaries? why not use the same numbering scheme as for Theorems, Lemmas, etc, so that the above corollaries will be numbered V.5 and V.6? The general recommendation is to use a common running counter for all results, so that seeing a result referenced somewhere in the paper, it is easy to know where to look for it. \\
% HVH: Done, change to running index in the main paper.\\
%
% \todo[inline,color=green]{YB: what do you think, Hai, about the simplified interpretation below?\\
% HVH:It is nice and make the results clearer. By a question? What does the symbol $\tau$ mean or should it be $\eta$?
% }
{Corollaries \ref{corr:bound_inv_post} and \ref{corr:bound_inv_split} admit a simplified interpretation to first order in $\eta^{(e)}$ and $\eta^{(a)}$. Assuming small learning errors, we have $\lbfimlps \approx \lbfimlp \approx\fb \approx\lbfimb  \approx  \lbfimbs$. The bounds in \eqref{eq:inv_bound_post} and \eqref{eq:inv_bound_mp} then simplify, respectively, to
\begin{equation}
   \mathrm{RE}_{B} \leq\kappa\brackets{\bcrb}\brackets{
    \bfse+\bfle} 
\end{equation}
\begin{equation}
   \mathrm{RE}_{MP} \leq\kappa\brackets{\bcrb}\brackets{
    \mpfse+\mpfle}, 
\end{equation}
showing that for both methods, the respective relative empirical-mean and approximation errors essentially add up, with scaling by the condition number $\kappa\brackets{\bcrb} = \kappa\brackets{\fb}$ of the BCRB. 
}

{Conditions \eqref{eq:inv_cond_post} and \eqref{eq:inv_cond_mp} for positive-definiteness of the 
learned FIMs in Corollaries~\ref{corr:bound_inv_post} and \ref{corr:bound_inv_split} are sufficient conditions, but they are the weakest such conditions: if violated, then there may exist learning errors such that the learned FIM is not invertible. In this sense, they are necessary to be able to guarantee the existence of the \name{}.}

%
{These conditions %for invertibility 
can be further detailed into %more 
specific criteria concerning the empirical-mean error and the 
%
approximation error}.
{Assuming small approximation errors, $\lbfimb \approx \fb \approx \lbfimlp$}
%Based on 
\eqref{eq:inv_cond_post} and \eqref{eq:inv_cond_mp} imply that the minimum required sample size for the Posterior Approach is :
{
\begin{equation*}
   % \sqrt
    {\nds}>  2.25\kappa^2\brackets{\fb}
    %
% \frac{\norm{\lbfimb}_2}{\norm{\fb}_2}
    %
    %\sqrt
    %3
    \nbe,
\end{equation*}
}
}
and for the Measurement-Prior Approach:
\begin{equation*}
    \nds>   2.25\kappa^2\brackets{\fb}\nmpe.
\end{equation*}
% \todo[inline,color=green]{YB: Check my change in the condition on $\nds$ above, and make a similar change below.\\
% HVH:Done}
{
%These expression offer a necessary condition regarding the minimal sample size; 
}
Note that both requirements on $\nds$ are quite sensitive to the condition number of the BCRB $\bcrb$: larger condition number will require more training samples.
If these conditions on the training set size are not met, $\lbfimb$ and $\lbfimlp$ 
{may} 
be non-invertible. 
{Furthermore, by the discussion comparing Theorems Theorems \ref{thm:sampling_post} and \ref{thm:sampling_mp} in Section~\ref{sec:empirical_mean_error}, we  typically have $\nbe \geq \niiddata \nmpe$. Combining with the conditions above implies that the {Measurement-Prior} Approach will require smaller $\nds$ when $\niiddata>1$.}

{Turning to the approximation error, Conditions \eqref{eq:inv_cond_post} and \eqref{eq:inv_cond_mp} imply
$\bfle, \mpfle\leq \frac{1}{\kappa\brackets{\fb}} = \frac{1}{\kappa\brackets{\bcrb}}$,
i.e., that the relative approximation errors must be kept below a threshold determined by $\kappa\brackets{\bcrb}$. 
}
% \ybdelete{Moreover, if the minimal sample size condition is hold, we still require an additional condition over the {approximation error} which is given by:
% \begin{equation*}
%     \bfle\leq {\frac{1}{\kappa\brackets{\fb}}-\frac{\norm{\lbfimb}_2}{\norm{\fb}_2}\bfse},
% \end{equation*}
% and for the Measurement-Prior Approach:
% \begin{equation*}
%     \mpfle\leq {\frac{1}{\kappa\brackets{\fb}}-\frac{\norm{\lbfimlp}_2}{\norm{\fb}_2} \mpfse}.
% \end{equation*}


% {From the result of observe that the minimal number of sample scale with $\fb$ condition number. Moreover, to ensure invertibility, the \name{} {approximation error} should be smaller than of both approaches should be at less small than $\frac{1}{\kappa\brackets{\fb}}$.  }}

% \hvhdelete{Specifically, when the true BFIM has large eigenvalues, it is more easily invertible, implying that at higher loss values, the learned FIMs will also be invertible. Moreover, the \name{} error scales with the loss function in both methods, indicating that smaller loss values result in better trained FIMs.}






\subsection{ Consistency \& Convergence}
\label{subsec:consistency}
%
{We consider}  
the convergence of \name{} to the BCRB as number {$\nds$ of training} samples goes to infinity. 
%
{%To show the convergence of \name{}, we 
First we}
show that the empirical-mean error converges to zero as {$\nds \rightarrow \infty$.} Then we prove that a neural network trained using score matching is a consistent estimator of %with 
the true score function. Finally, 
%\ybreplace{based on the consistency and}
{combining these results with} Corollaries \ref{corr:bound_inv_post} and \ref{corr:bound_inv_split}, we show the convergence of the \name{} . 


% \todo[inline,color=green]{Say something about the proof of these corollaries? Do they follow directly from something, or is their proof presented somewhere?\\
% HVH: Done\\
% YB: I think that I was not clear enough. Looks like you thought I was asking about the proof of Corollaries \ref{corr:bound_inv_post} and \ref{corr:bound_inv_split}. I now  deleted your explanations here about the proof of those corollaries, because the proofs are actually available, pointed to after these corollaries.\\
% Instead, I am asking about the proof of Corollaries \ref{corr:sampling_post} and \ref{corr:sampling_mp}. Where is it?\\
% HVH: We have sad it is easy to proof so we obit it from the text. Now I have added a text to say what the proof is base one.}

%
{The convergence of the empirical-mean error follows} directly from
Theorems~\ref{thm:sampling_post} and \ref{thm:sampling_mp} subject to the additional assumption $\frac{\cb}{\nds}, \frac{\cm}{\nds}, \frac{\cp}{\nds}\xrightarrow{\nds\rightarrow\infty}0
{\quad \text{a.s.}}$, as summarized in the following results.

\begin{corollary}[LBCRB Empirical-Mean Error Convergence: Posterior Approach]\label{corr:sampling_post} 
Suppose
%Assume 
that the assumptions of Theorem~\ref{thm:sampling_post} hold and that $\frac{\cb}{\nds}\xrightarrow{\nds\rightarrow\infty}0 
{\quad \text{a.s.}}$, then:
\begin{equation}
\mathrm{RE}_B^{(e)}\xrightarrow{\nds\rightarrow\infty}0 \quad\text{a.s.}
\end{equation}
    
\end{corollary}

\begin{corollary}[LBCRB Empirical-Mean Error Convergence: Measurement-Prior Approach]\label{corr:sampling_mp} %Assume 
Suppose that the assumptions of Theorem~\ref{thm:sampling_mp} hold and that
$%\ybreplace{\frac{\cm+\frac{\cp}{\niideval}}{\nds}}
{\frac{\cm}{\nds}, \frac{\cp}{\nds}}\xrightarrow{\nds\rightarrow\infty}0 %
{\quad \text{a.s.}}$, then:
\begin{equation}
\mathrm{RE}_{MP}^{(e)}\xrightarrow{\nds\rightarrow\infty}0 \quad\text{a.s.}
\end{equation}
    
\end{corollary}





From Corollaries~\ref{corr:sampling_post} and \ref{corr:sampling_mp}, we observe that as the number of samples $\nds$
increases, the %
{empirical}-mean error is reduced%
{, vanishing with probability 1 in the limit of infinite training data}.
%{
 \begin{remark}\label{remark:mean_error_convergance}
    The assumption  $\frac{\cb}{\nds}\xrightarrow{\nds\rightarrow\infty}0 {\quad \text{a.s.}}$ in Corrolary~\ref{corr:sampling_post} is 
    {rather mild. 
    It means that the maximum over the data set $\ds$ of the squared norm of the learned score grows at a slower rate than $|\ds|=\nds$. For example, 
    {as we show in Appendix~\ref{apx:remark_convergance_proof}}, it holds if $\postscores{\pr_n}{\xsetr_n}$ has finite moments up to 6th order. In particular, this requirement is less demanding than the assumption made in \cite{crafts2023bayesian} that the score vector is
        Sub-Gaussian. Indeed, a Sub-Gaussian r.v. has finite moments of any order, rather than just up to 6th order.} Similar comments apply to the assumption about $\cp$ and $\cm$ in Corollary~\ref{corr:sampling_mp}.
\end{remark}

{Next we address the approximation error. We will need the following assumptions about the score neural networks. The first set of assumptions, of realizability of the true score functions by the chosen NN model architectures, is the more fundamental.}
\begin{assumption}[{NN-}Realizable Score Function]\label{assume:realizable}
Let {$\mathcal{S}_B$, $\mathcal{S}_F$ and $\mathcal{S}_P$} be the sets of all score functions representable by the  architectures of the neural networks chosen for the Bayesian, Fisher and prior scores, respectively. {Define the shorthand notation for the true score functions $\vectorsym{s}^{0}_{B}\brackets{\p|\xsetr} \triangleq\nabla_{\p}\log\probt{\p|\xsetr}{ \p | \xset}$,}  $\vectorsym{s}^{0}_{F}\brackets{\x|\p} \triangleq\nabla_{\p}\log\probt{\x|\p}{\X|\p}$ and $\vectorsym{s}^{0}_{P}\brackets{\p}\triangleq\nabla_{\p}\log\probt{\p}{\p}$. Then assume that {all three true scores are realizable {by the chosen NN architectures,} i.e.}:
{$\vectorsym{s}^{0}_\alpha \in \mathcal{S}_\alpha$, $\alpha= B, F, P$.}
\end{assumption}

We argue that Assumption~\ref{assume:realizable} holds {(or can be approximated arbitrarily well)} in the regime of a neural networks with sufficient capacity. {In terms of our parametrized score neural network notation, this means, e.g., for the Bayesian score, that $\vectorsym{s}_{B}\brackets{\p|\xsetr; \Omega} = \vectorsym{s}^{0}_{B}\brackets{\p|\xsetr}$ for some $\Omega=\Omega^{*}$. (Unlike in \cite{hyvarinen2005estimation}, the presence of multiple such alternative $\Omega^*$ is not a concern in our application, and may in fact facilitate the optimization step of score matching.)} 

{The second set of assumptions, of Lipschitz continuity, are more technical regularity assumptions.}
They can be readily satisfied by choice of neural network architecture and training strategy (e.g. $\ell_2$ regularization).

\begin{assumption}[Lipschitz Continuity]\label{assume:lip_cont}
%
For each of the following three alternative settings, 
\begin{enumerate}[label=\text{\ref{assume:lip_cont}}.\arabic*,labelsep=*, leftmargin=*]
    \item $\alpha=B$, $\, f\brackets{\z;\Omega}=\postscore{\p}{\x;\Omega}$, $\, \z=[\p,\x]$, %and% 
    $\, \randomvec{Z}=[\pr,\X]$, \label{assume:lip_cont_post}
    
    \item  $\alpha=P$, $\, f\brackets{\z;\Omega}=\priorscore{\p;\Omega}$, $\, \z=\p$, %and 
    $\, \randomvec{Z}=\pr$, \label{assume:lip_cont_prior}
    
    \item  $\alpha=F$, $f\brackets{\z;\Omega}=\lscore{\x}{\p;\Omega}$, $\z=[\p,\x]$, %and
    $\randomvec{Z}=[\pr,\X]$.\label{assume:lip_cont_fisher}
    
\end{enumerate}

assume the following holds.\\
% \todo[inline,color=orange]{YB: $\nabla_{\p}$ below is applied to vector function $f_\alpha$, so it should be replaced by the Jacobian matrix notation.}
% }
Functions $f_{\alpha}\brackets{\z;\Omega_{}}$, $\divc{f_{\alpha}\brackets{\z;\Omega_{}}}{\p}$ and $f_{\alpha}\brackets{\z;\Omega_{}}f_{\alpha}\brackets{\z;\Omega_{}}^T$ are Lipschitz continuous {with Lipschitz constants finite in expectation,} 
% \todo[inline,color=green]{In 1) and 3) the norms are applied to vectors, so shouldn't the norm be just the 2-norm? The Frobenius norm is only relevant in 2), so why not use the appropriat subscripts on $\norm{\cdot}$?}
namely, for any {$\Omega_{1}, \Omega_{2} \in \mathcal{S}_{\alpha}$ {with $\Delta\Omega=\norm{\Omega_{1}-\Omega_{2}}_2$}:} \\
% \todo[inline,color=green]{\textbf{YB:} Why superscripts on $\Omega$ rather than simple subscripts?}
   1) 
  $\norm{f_{\alpha}\brackets{\z;\Omega_{1}}-f_{\alpha}\brackets{\z;\Omega_{2}}}_2\leq \xi_{\alpha}\brackets{\z}\Delta\Omega$,\\
    2) $\norm{\divc{f_{\alpha}\brackets{\z;\Omega_{1}}}{\p}-\divc{f_{\alpha}\brackets{\z;\Omega_{2}}}{\p}}_2\leq \tau_{\alpha}\brackets{\z}\Delta\Omega$,\\
    3) $\norm{f_{\alpha}\brackets{\z;\Omega_{1}}f_{\alpha}\brackets{\z;\Omega_{1}}^T-f_{\alpha}\brackets{\z;\Omega_{2}}f_{\alpha}\brackets{\z;\Omega_{2}}^T}_F\leq \zeta_{\alpha}\brackets{\z}\Delta\Omega$.\\
    % \todo[inline,color=green]{Typo in the first outer product: should have $\Omega^{(1)}$ in both factors.}
{with} $\expectation{\xi_\alpha\brackets{\randomvec{Z}}{^2}}{\randomvec{Z}},
%<\infty$, 
\expectation{\tau_\alpha\brackets{\randomvec{Z}}}{\randomvec{Z}},
%<\infty$ and $
\expectation{ \zeta_{\alpha}\brackets{\randomvec{Z}}}{\randomvec{Z}}<\infty$. 
% \todo[inline,color=green]{Writing $\leq$ above is meaningless. Did you mean $<$?}


\end{assumption}

% \todo[inline,color=green]{YB: Say that these Lipschitz assumptions are reasonable - can be satisfied by the design of the NN?}

Then, {we obtain the following results.  }

\begin{theorem}[Posterior Score Consistency] \label{thm:post_consist}
    %Assume 
    Suppose that Assumptions~\ref{assum:bcrb_reg}, \ref{ass:score_cond_reg}, {\ref{assume:lip_cont},} and {\ref{assume:realizable}}~ hold, {and that the minimization algorithm applied to the objective in \eqref{eq:score_post_mean} succeeds in finding a global minimizer $\Omega^*={\arg\min\limits_{\Omega}\lossbsm\brackets{\Omega}}$.
    }  Then,
    \begin{equation*}
        \lossbs\brackets{\Omega^{*}}\xrightarrow{a.s} 0\quad\text{as} \quad\abs{\ds}=\nds\rightarrow \infty.
    \end{equation*}
\end{theorem}

\begin{theorem}[Measurement and Prior Score Consistency]\label{thm:mp_consist}
    %Assume 
    Suppose that Assumptions~\ref{assum:bcrb_reg}, \ref{ass:score_reg_prior}, \ref{ass:score_reg}, {\ref{assume:lip_cont},} {and \ref{assume:realizable}} hold, 
    {and that the minimization algorithms applied to the objectives in \eqref{eq:score_prior_mean} and \eqref{eq:score_lik_mean} succeed in finding respective global minimizers {$\paramf^{*}=\arg\min\limits_{\Omega}\lossfsm\brackets{\Omega}$ and $\paramp^{*}=\arg\min\limits_{\Omega}\losspsm\brackets{\Omega}$.}
    }
    Then,
        \begin{equation*}
        \lossfs\brackets{\paramf^{*}}\xrightarrow{a.s} 0\mkern9mu \text{and}\mkern9mu\lossps\brackets{\paramp^{*}}\xrightarrow{a.s} 0, \mkern9mu\text{as} \mkern9mu \abs{\ds}=\nds\rightarrow \infty. 
    \end{equation*}
\end{theorem}


Theorems~\ref{thm:post_consist} and \ref{thm:mp_consist} (proved in Appendix~\ref{apx:proof:consistency_post}) demonstrate that {the score neural networks optimized via score matching are strongly consistent estimators of the true score. }

% \ybdelete{In cases of neural networks, they are realizable when the architecture is sufficiently complex with a sufficiently large dataset.}    

% \ybdelete{Moreover, these consistency results are crucial for ensuring the convergence of LBCRB, as shown in the following Corollaries.}

%hvhedit
{From a practical point of view, {the dependence of}
Theorems~\ref{thm:post_consist} and \ref{thm:mp_consist} {on the  realizability Assumption~\ref{assume:realizable}}
highlights the advantages of \peac{} that uses domain knowledge {in the Measurement-Prior Approach.}  \peac{} enables {the realization and learning of accurate score functions models}: (i) using less complex neural networks (fewer parameters, nonlinearity, etc.); (ii)  {using} less training data; and (iii) {with more reliable training %identifying its parameters
(%optimization 
minimization of the loss function).% can be done more reliably.
}



% \hvhdelete{score matching optimization
% losses are consistent when the score function is realizable with a sufficiently large dataset.}
% \todo[inline,color=green]{YB: I don't think that one can say that "losses are consistent" in our context. Usually one talks about consistency of an estimator.} 
%  \hvhdelete{This also highlight 
% the advantages of MoISNN, which enables a realizable neural network with domain knowledge more easily (see the frequency estimation example in Section~\ref{sec:freq_est}).}

% \todo[inline,color=green]{How does one see this in these corollaries?\\

% The issue is "realizing the score function by a combination of a known function and a neural network of sufficiently low complexity that (1) it can be learned from the available training data; and (2) identifying its parameters (optimizing the loss function) can be done more reliably. Then appeal to the frequency estimation example.\\
% % Is this what you meant? This can be explained along these lines, but this explanation should be separated from the precise statements, so that it is clear that it is a somewhat loose qualitative statement.\\
% HVH: Please see updated text.
% } 



%\todo[inline,color=green]{Interpret these theorems?}
 }

 

% \todo[inline,color=green]{YB: state the converge results in the corollaries for the \name{} error itself, as done at the end of Sec.III-A, rather than the relative error. \\
% HVH: What do you think about the following way I wrote it? I think it address the problem statment?}

{Having established the a.s convergence of the empirical-mean  and the strong consistency of score matching, we are ready for our final results for the \name{} for the two methods.}

\begin{corollary}[Posterior Approach Convergence] \label{corr:re_zero}
Suppose that {the assumptions in Theorem~\ref{thm:post_consist}} {and Corollary~\ref{corr:sampling_post}} hold. Then,
%\begin{equation*}
    $\lim\limits_{\nds\rightarrow \infty } {\lbcrbbs =\bcrb}. \quad\text{a.s}$
%\end{equation*}
\end{corollary}

\begin{corollary}[Measurement-Prior Approach Convergence] \label{corr:re_zero_mp}
Suppose that {the assumptions in Theorem~\ref{thm:mp_consist}} {and Corollary~\ref{corr:sampling_mp}} hold. Then,
%\begin{equation*}
  $  \lim\limits_{\nds\rightarrow \infty } {\lbcrblps=\bcrb}. \quad\text{a.s}$
%\end{equation*}
\end{corollary}
% \todo[inline,color=green]{YB: Good, these are the results we want. But these corollaries need an argument/proof to justify them.}
{
We prove Corollary~\ref{corr:re_zero_mp}. The proof of Corollary~\ref{corr:re_zero} is exactly analogous, with  Corollary~\ref{corr:sampling_post} and Theorems~\ref{thm:lrn:direct} and \ref{thm:post_consist}  playing the role of  Corollary~\ref{corr:sampling_mp} and Theorems
\ref{thm:lrn:lik_prior}, and \ref{thm:mp_consist}, respectively. 
}

{\begin{proof}
   Observe that $\expectation{\lbfimlps}{\X,\pr}=\lbfimlp$.
   %and $\expectation{\lbfimbs}{\X,\pr}=\lbfimb$. 
   Using  Corollary~\ref{corr:sampling_mp} %~\ref{corr:sampling_post} and
   we have that, $\lbfimlps$ 
   %and $\lbfimbs$ 
   from %\eqref{eq:mean_fully} and
   \eqref{eq:bfim_apx_final} converges almost surely to 
   its %their 
   expected value,%s, such that
   i.e., $\lim\limits_{\nds\rightarrow\infty}\lbfimlps=\lbfimlp$ a.s.  
   % and $\lim\limits_{\nds\rightarrow\infty}\lbfimbs=\lbfimb$ a.s. 
   Furthermore, combining
 %  by integrating 
   Theorem~\ref{thm:lrn:lik_prior} 
   %and \ref{thm:lrn:direct} 
   with Theorem~\ref{thm:mp_consist}
   %and \ref{thm:post_consist}, 
   , it follows that $\lim\limits_{\nds\rightarrow\infty} \lbfimlp=\fb$ a.s.
   %and $\lim\limits_{\nds\rightarrow\infty} \lbfimb=\fb$ a.s. 
 Hence, $\lim\limits_{\nds\rightarrow\infty}\lbfimlps=\fb$ a.s.  
 Inverting yields the result. 
\end{proof}
}

{In summary, } as stated in Corollarys~\ref{corr:re_zero} and \ref{corr:re_zero_mp}, with sufficient training data and a properly trained neural networks, the\name{} {will provide the BCRB accurately}. 
\section{MEASUREMENTS MODEL EXAMPLES}\label{sec:example_models}
We present three measurement models to illustrate the benefits of the \name.

\subsection{Linear Observation Model}\label{sec:linear_model} 
Let $\pr\in\mathbb{R}^{\np}$ be the parameter {we wish to estimate} and $\matsym{A}\in\mathbb{R}^{{\nx}\times \np}$  a fixed mixing matrix, %then 
and define the  ${n}^{th}$ snapshot measurement ${\X}_{n}\in\mathbb{R}^{{\nx}}$ %is defined as follows:
\begin{equation}\label{eq:linear-model}
    {\randomvec{X}}_{{n}}=\matsym{A}\pr+\randomvec{W}_{n},
\end{equation}
% \todo[inline,color=green]{Remove the tildes above - they are no longer relevant?}
where $\randomvec{W}_{{n}}\sim\normaldis{0}{\matsym{\Sigma}}$ is  additive Gaussian noise with zero mean and covariance matrix $\matsym{\Sigma}$. 
We are given a measurement set $\xset=\set{{\X}_{n}}_{{n}=1}^{\niideval}$,
that consists of $\niideval$ snapshots of the measurement in \eqref{eq:linear-model}.  %{
For the prior of $\p$, we assume a Gaussian distribution,  $\pr\sim\normaldis{0}{\matsym{I}\sigma_P^2}.$  The linear model in \eqref{eq:linear-model}  with the Gaussian prior enables a detailed {study} of the suggested methods and also ensures that Assumption ~\ref{assume:realizable} %
{is satisfied by simple neural network realizations of the score functions}. 

The BCRB for %
the estimation of $\p$ from the measurement {set $\xset$} in this example is given by:
\begin{align}\label{eq:bcrb_linear}
\bcrb=\brackets{{\niideval} \cdot\matsym{A}^T\Sigma^{-1}\matsym{A}+\frac{1}{\sigma_P^2}}^{-1},
\end{align}
and %
{(as expected for the linear Gaussian observation model)} coincides with the covariance of LMMSE. In addition, the {\emph{true}} score vectors of the measurement model  \eqref{eq:linear-model} are given by:
\begin{equation}\label{eq:score_opt_linear}
\vectorsym{s}^{0}_F\brackets{{\x}|\p}=  \matsym{A}^T\Sigma^{-1}\brackets{{\x}-\matsym{A}\p}\textbf{ and } \vectorsym{s}^{0}_P\brackets{\p}=-\frac{\p}{\sigma_p^2}.
\end{equation}
% \todo[inline,color=green]{Say something about $\vectorsym{s}^{0}_B \brackets{{\p}|\x}$, which I assume you use in computing the \name{} using the Posterior Approach? \\
% HVH: Yes and Done.}
{The true posterior score is given by $\vectorsym{s}^{0}_B\brackets{\p|\xsetr}=\sum_{\x\in\xsetr}\vectorsym{s}^{0}_F\brackets{{\x}|\p}+\vectorsym{s}^{0}_P$, where $\xsetr$ represents the set of $\niideval$ i.i.d. measurements. }

{The goal is to compute the LBCRB (a close approximation to the BCRB), when the matrices $\matsym{A}$ and $\matsym{\Sigma}$ are \emph{unknown}, by using a training data set 
$\ds=\set{\xset_n=\set{{\x}_m}_{m=1}^{\niiddata},\p_n}_{n=1}^{\nds}$. 
 It is important to observe that the number $\niiddata$ of i.i.d. samples present in the dataset for each value of $\p$ may differ from the desired number $\niideval$ for obtaining the LBCRB,
 i.e., $\niideval \neq \niiddata$.



\subsection{Frequency Estimation }\label{sec:freq_est}
A simple frequency estimation problem  illustrates the benefits of the \name{}. 
The observation model is  
\begin{equation}\label{eq:freq_est_model}
    \squareb{\X}_n=\cos{\brackets{\pr n}}+\squareb{\randomvec{W}}_n \quad\forall n \in [0,N-1],
\end{equation}
% \todo[inline,color=green]{Why not simply $\cos (\theta n)$?}
where $\p$ is the digital  frequency  parameter to be estimated, and $\randomvec{W}$ is an additive noise. 
%In this example, 
We consider two types of noise: 1)  Gaussian noise $\randomvec{W}\sim \mathcal{N}\brackets{0,\Sigma}$ to validate the \name{} using closed-form results from \cite{van2007bayesian} (Example 3); and 2) a real-world underwater ambient noise taken from \cite{msg0-ag12-22} to simulate frequency estimation with  non-Gaussian noise. 
We consider a four-parameter Beta prior  $\p\sim\mathrm{Beta}\brackets{\alpha_{\p},\alpha_{\p},0,\pi}$. 
% \todo[color=green]{Only one?}
Recall that a random variable $\randomvec{z}\sim\mathrm{Beta}\brackets{\alpha,\beta,l,u}$ has PDF
\begin{equation}
    \probt{z}{\randomvec{z}}\triangleq \frac{\brackets{z-l}^{\alpha-1}\brackets{u-z}^{\beta-1}}{\brackets{u-l}^{\alpha+\beta+1}B\brackets{\alpha,\beta}},
\end{equation}
where  $l<u$ are the lower and upper support parameters, respectively, $\alpha, \beta >0$ are the shape parameters, and $B\brackets{\alpha,\beta}=\frac{\Gamma\brackets{\alpha}\Gamma\brackets{\beta}}{\Gamma\brackets{\beta+\alpha}}$ is the normalization constant, where $\Gamma\brackets{\alpha}$ is the Gamma function. 


In the case of Gaussian noise, the observation model in \eqref{eq:freq_est_model} has the following true score functions:
\begin{subequations}
\begin{equation}\label{eq:score_opt_freq_lik}
\vectorsym{s}^{0}_F\brackets{\x|\p}=\matsym{\frac{\partial \mathcal{M}\brackets{\p} }{\partial \p}}^{T}\Sigma^{-1}\brackets{\x-\mathcal{M}\brackets{\p}},
\end{equation}
\begin{equation}
\vectorsym{s}^{0}_{P}\brackets{\p}=\frac{\alpha_{{\p}}-1}{{\p}}-\frac{\alpha_{{\p}}-1}{\pi-{\p}}
\end{equation}
\end{subequations}
where $\squareb{\mathcal{M}\brackets{\p}}_n=\cos{\brackets{\p n}}$. 
%is sinusoidal function of the parameter vector $\p$.
Here, we show the use of a \pe{} score neural network and its effect on the structure of the neural network. 
%In case of frequency estimation, 
We assume that $\squareb{\mathcal{M}\brackets{\p}}_n=\cos{\brackets{\p n}}$ 
%
{has the known form of a sinusoidal function of the unknown parameter $\p$,}
% \todo[inline,color=green]{So what is there to estimate?\\
% HVH: The noise distribution or more specifically the score of $\randomvec{W}$. } 
whereas the distribution of $\matsym{W}$ is unknown. 

Applying $\squareb{\mathcal{M}\brackets{\p}}_n=\cos{\brackets{\p n}}$ to \eqref{eq:model_base_score} 
% with $\squareb{\mathcal{M}\brackets{\p}}_n=\cos{\brackets{\p n}}$ 
and comparing with \eqref{eq:score_opt_freq_lik}, we find that the  
 $\vectorsym{s}_I$ %
 {that provides the true score}
% \todo[inline,color=green]{Replace the term "optimal score" by "true score throughout, and also use the notation with supersrict $0$ rather than $*$ - check throughout.\\
% HVH:Done} 
is given by $\Sigma^{-1}\brackets{\x-\vectorsym{\tau}}$. %
{In this example, the score neural network only 
%require to obtain 
needs to learn (using $\ds$ and score matching) the score function of the noise distribution, 
%using score matching and $\ds$, 
since the modeling of the dependence of the measurement on the frequency parameter is known. }
{Furthermore, $\vectorsym{s}_I$ } can be represented by a single layer of a fully connected neural network that needs to learn only the inverse of the covariance matrix. %
{%It is important to highlight that, 
Importantly, as explained in Remark~\ref{remark:opt_mi}, $\vectorsym{s}_{I}$ is only indirectly {optimized} through the application of $\mathcal{M}$ using \eqref{eq:model_base_score}.%, rather than being optimized directly. 
}
% \todo[inline,color=green]{1. See comments on linear model.\\
% 2. In this case the derivative  $\partial^T M(\p)/\partial \p$ depends on $\p$, which is the parameter to be estimated, and therefore assumed unknown. Also, it is random with some distribution - so what value of $\p$ should be used to define $S_F$ for the actual computation of the LBCRB? Is this value needed for the training of $S_I$? Elaborate (also in the introduction of this approach in Section IV-B) to avoid confusion.\\
% HVH: 
% 1.  Done.
% 2. Add remark ~\ref{remark:opt_mi} and refernced here.
% } 

In contrast, without leveraging the model information, we %would 
have to learn the cosine function for the score -- a rather complicated nonlinear function. This 
%would 
leads to a significantly more complex neural network that %results with necessitating 
requires more training samples  to train properly, %
{as demonstrated later in Figure~\ref{fig:analysis_mi}}. 
% \todo[inline,color=green]{Is this something you will demonstrate in experiments, or just making an unsubstantiated claim here?\\
% HVH: shown in the experiment in Figure~\ref{fig:analysis_m} and comment to the figure in the text.}

\subsection{%
{One Bit }Quantization}\label{sec:quantization_example}
% \todo[inline]{Maybe change to generic function.}
%Here, we present a 
Consider the linear observation model with quantization. %
{Specifically, let Let $\pr\in\mathbb{R}^{\np}$ be the parameter %
{we wish to estimate} and $\matsym{A}\in\mathbb{R}^{{\nx}\times \np}$  a fixed mixing matrix, %then 
and define the  ${n}^{th}$ snapshot measurement ${\X}_{n}\in\mathbb{R}^{{\nx}}$:} % which is given by:
\begin{equation}\label{eq:lin_q}
{\X}_{n}=\mathrm{sign}\brackets{\matsym{A}\pr+\randomvec{W}_{n}+\boldsymbol{1}_{\nx}\cdot b },
\end{equation}
where %
{$$\squareb{\mathrm{sign}\brackets{\vectorsym{x}}}_i=\begin{cases}
    1 & \squareb{\vectorsym{x}}_i\geq 0\\
    -1 & \squareb{\vectorsym{x}}_i< 0
\end{cases}$$ is an element-wise sign function,} 
%
{ $\pr \in \mathbb{R}^{\np}$,}
$\randomvec{W}_{n}\sim\normaldis{0}{\matsym{\Sigma}}$ is additive zero-mean Gaussian noise  with covariance %matrix
$\matsym{\Sigma}$, $\boldsymbol{1}_{\nx} \in \mathbb{R}^{\nx}$ is a vector of ones, % of size $n$, 
and $b$ is a constant shift. Finally, for the prior
%we assume a  prior Gaussian distribution of 
we assume $\pr\sim\normaldis{0}{\matsym{I}\sigma_p^2}$. 
%It is important to highlight that 
Importantly, even for this simple observation model
%, as depicted in 
\eqref{eq:lin_q}, there is no closed-form solution for the BCRB. However, in 
%simpler 
 the special case %where 
of a diagonal 
$\matsym{\Sigma}$, 
%is a diagonal matrix, 
the %\hvhreplace{Bayesian}
{true} {Fisher} score 
% \todo[inline,color=green]{Do you also need the two other score functions for the experiments?}
has a closed-form solution (derived in Appendix~\ref{apx:derivation_q_score})
%{
\begin{align}\label{eq:score_quantization}
    \vectorsym{s}^{0}_{F}\brackets{\x|\p}&=\frac{\partial\vectorsym{u}\brackets{\p}}{\partial\p}^T\vectorsym{\rho}\brackets{\x,\p}, \\
    \text{where } \squareb{\vectorsym{\rho}\brackets{\vectorsym{x},\p}}_i&=
    \textstyle{\frac{\squareb{\x}_i\exp\brackets{-\frac{\squareb{\vectorsym{u}\brackets{\p}}_i^2}{2\sigma_i^2}}}{\sqrt{2\pi\sigma^2_i}\Phi\brackets{\squareb{\x}_i\frac{\squareb{\vectorsym{u}\brackets{\p}}_i}{\sigma_i} }} } , \nonumber
\end{align}
$\vectorsym{u}\brackets{\p}=\matsym{A}\p+\boldsymbol{1}_{\nx}\cdot b$ is the un-quantized observation function, $\sigma_i^2=\squareb{\Sigma}_{i,i}$ is the $i^{th}$ element in the diagonal matrix $\Sigma$, and
$\Phi$ is the 
%cumulative distribution function (CDF) of
standard Gaussian CDF. %
{Note that the true prior and posterior scores are 
given by the same expressions as for the linear observation model in section~\ref{sec:linear_model}.}

Using \eqref{eq:score_quantization} in \eqref{eq:mean_fully}, we can determine, {for known $\matsym{A}, \matsym{\Sigma}$ and $b$}, the BCRB numerically.
However, this is a very limited special case, and a more general bound cannot be obtained since the CDF of the multivariate Gaussian distribution does not have a close form\cite{genz2009computation}, preventing  the computation of the PMF
%probability mass function (PMF) 
$\pmft{\X|\p}{\x|\pr}$. %
{Needless to say, for unknown $\matsym{\Sigma}$, the BCRB cannot be determined conventionally at all.}

 %
 {Similar to the linear problem of Section~\ref{sec:linear_model}, our goal is to compute the LBCRB (a close approximation to the BCRB), when $\matsym{A}$, $\matsym{\Sigma}$ and $b$ are \emph{unknown}, by using a training data set $\ds=\set{\xset_n=\set{{\x}_m}_{m=1}^{\niiddata},\p_n}_{n=1}^{\nds}$. 



\begin{table*}[t]
\centering
\fontsize{11pt}{11pt}\selectfont
\begin{tabular}{lllllllllllll}
\toprule
\multicolumn{1}{c}{\textbf{task}} & \multicolumn{2}{c}{\textbf{Mir}} & \multicolumn{2}{c}{\textbf{Lai}} & \multicolumn{2}{c}{\textbf{Ziegen.}} & \multicolumn{2}{c}{\textbf{Cao}} & \multicolumn{2}{c}{\textbf{Alva-Man.}} & \multicolumn{1}{c}{\textbf{avg.}} & \textbf{\begin{tabular}[c]{@{}l@{}}avg.\\ rank\end{tabular}} \\
\multicolumn{1}{c}{\textbf{metrics}} & \multicolumn{1}{c}{\textbf{cor.}} & \multicolumn{1}{c}{\textbf{p-v.}} & \multicolumn{1}{c}{\textbf{cor.}} & \multicolumn{1}{c}{\textbf{p-v.}} & \multicolumn{1}{c}{\textbf{cor.}} & \multicolumn{1}{c}{\textbf{p-v.}} & \multicolumn{1}{c}{\textbf{cor.}} & \multicolumn{1}{c}{\textbf{p-v.}} & \multicolumn{1}{c}{\textbf{cor.}} & \multicolumn{1}{c}{\textbf{p-v.}} &  &  \\ \midrule
\textbf{S-Bleu} & 0.50 & 0.0 & 0.47 & 0.0 & 0.59 & 0.0 & 0.58 & 0.0 & 0.68 & 0.0 & 0.57 & 5.8 \\
\textbf{R-Bleu} & -- & -- & 0.27 & 0.0 & 0.30 & 0.0 & -- & -- & -- & -- & - &  \\
\textbf{S-Meteor} & 0.49 & 0.0 & 0.48 & 0.0 & 0.61 & 0.0 & 0.57 & 0.0 & 0.64 & 0.0 & 0.56 & 6.1 \\
\textbf{R-Meteor} & -- & -- & 0.34 & 0.0 & 0.26 & 0.0 & -- & -- & -- & -- & - &  \\
\textbf{S-Bertscore} & \textbf{0.53} & 0.0 & {\ul 0.80} & 0.0 & \textbf{0.70} & 0.0 & {\ul 0.66} & 0.0 & {\ul0.78} & 0.0 & \textbf{0.69} & \textbf{1.7} \\
\textbf{R-Bertscore} & -- & -- & 0.51 & 0.0 & 0.38 & 0.0 & -- & -- & -- & -- & - &  \\
\textbf{S-Bleurt} & {\ul 0.52} & 0.0 & {\ul 0.80} & 0.0 & 0.60 & 0.0 & \textbf{0.70} & 0.0 & \textbf{0.80} & 0.0 & {\ul 0.68} & {\ul 2.3} \\
\textbf{R-Bleurt} & -- & -- & 0.59 & 0.0 & -0.05 & 0.13 & -- & -- & -- & -- & - &  \\
\textbf{S-Cosine} & 0.51 & 0.0 & 0.69 & 0.0 & {\ul 0.62} & 0.0 & 0.61 & 0.0 & 0.65 & 0.0 & 0.62 & 4.4 \\
\textbf{R-Cosine} & -- & -- & 0.40 & 0.0 & 0.29 & 0.0 & -- & -- & -- & -- & - & \\ \midrule
\textbf{QuestEval} & 0.23 & 0.0 & 0.25 & 0.0 & 0.49 & 0.0 & 0.47 & 0.0 & 0.62 & 0.0 & 0.41 & 9.0 \\
\textbf{LLaMa3} & 0.36 & 0.0 & \textbf{0.84} & 0.0 & {\ul{0.62}} & 0.0 & 0.61 & 0.0 &  0.76 & 0.0 & 0.64 & 3.6 \\
\textbf{our (3b)} & 0.49 & 0.0 & 0.73 & 0.0 & 0.54 & 0.0 & 0.53 & 0.0 & 0.7 & 0.0 & 0.60 & 5.8 \\
\textbf{our (8b)} & 0.48 & 0.0 & 0.73 & 0.0 & 0.52 & 0.0 & 0.53 & 0.0 & 0.7 & 0.0 & 0.59 & 6.3 \\  \bottomrule
\end{tabular}
\caption{Pearson correlation on human evaluation on system output. `R-': reference-based. `S-': source-based.}
\label{tab:sys}
\end{table*}



\begin{table}%[]
\centering
\fontsize{11pt}{11pt}\selectfont
\begin{tabular}{llllll}
\toprule
\multicolumn{1}{c}{\textbf{task}} & \multicolumn{1}{c}{\textbf{Lai}} & \multicolumn{1}{c}{\textbf{Zei.}} & \multicolumn{1}{c}{\textbf{Scia.}} & \textbf{} & \textbf{} \\ 
\multicolumn{1}{c}{\textbf{metrics}} & \multicolumn{1}{c}{\textbf{cor.}} & \multicolumn{1}{c}{\textbf{cor.}} & \multicolumn{1}{c}{\textbf{cor.}} & \textbf{avg.} & \textbf{\begin{tabular}[c]{@{}l@{}}avg.\\ rank\end{tabular}} \\ \midrule
\textbf{S-Bleu} & 0.40 & 0.40 & 0.19* & 0.33 & 7.67 \\
\textbf{S-Meteor} & 0.41 & 0.42 & 0.16* & 0.33 & 7.33 \\
\textbf{S-BertS.} & {\ul0.58} & 0.47 & 0.31 & 0.45 & 3.67 \\
\textbf{S-Bleurt} & 0.45 & {\ul 0.54} & {\ul 0.37} & 0.45 & {\ul 3.33} \\
\textbf{S-Cosine} & 0.56 & 0.52 & 0.3 & {\ul 0.46} & {\ul 3.33} \\ \midrule
\textbf{QuestE.} & 0.27 & 0.35 & 0.06* & 0.23 & 9.00 \\
\textbf{LlaMA3} & \textbf{0.6} & \textbf{0.67} & \textbf{0.51} & \textbf{0.59} & \textbf{1.0} \\
\textbf{Our (3b)} & 0.51 & 0.49 & 0.23* & 0.39 & 4.83 \\
\textbf{Our (8b)} & 0.52 & 0.49 & 0.22* & 0.43 & 4.83 \\ \bottomrule
\end{tabular}
\caption{Pearson correlation on human ratings on reference output. *not significant; we cannot reject the null hypothesis of zero correlation}
\label{tab:ref}
\end{table}


\begin{table*}%[]
\centering
\fontsize{11pt}{11pt}\selectfont
\begin{tabular}{lllllllll}
\toprule
\textbf{task} & \multicolumn{1}{c}{\textbf{ALL}} & \multicolumn{1}{c}{\textbf{sentiment}} & \multicolumn{1}{c}{\textbf{detoxify}} & \multicolumn{1}{c}{\textbf{catchy}} & \multicolumn{1}{c}{\textbf{polite}} & \multicolumn{1}{c}{\textbf{persuasive}} & \multicolumn{1}{c}{\textbf{formal}} & \textbf{\begin{tabular}[c]{@{}l@{}}avg. \\ rank\end{tabular}} \\
\textbf{metrics} & \multicolumn{1}{c}{\textbf{cor.}} & \multicolumn{1}{c}{\textbf{cor.}} & \multicolumn{1}{c}{\textbf{cor.}} & \multicolumn{1}{c}{\textbf{cor.}} & \multicolumn{1}{c}{\textbf{cor.}} & \multicolumn{1}{c}{\textbf{cor.}} & \multicolumn{1}{c}{\textbf{cor.}} &  \\ \midrule
\textbf{S-Bleu} & -0.17 & -0.82 & -0.45 & -0.12* & -0.1* & -0.05 & -0.21 & 8.42 \\
\textbf{R-Bleu} & - & -0.5 & -0.45 &  &  &  &  &  \\
\textbf{S-Meteor} & -0.07* & -0.55 & -0.4 & -0.01* & 0.1* & -0.16 & -0.04* & 7.67 \\
\textbf{R-Meteor} & - & -0.17* & -0.39 & - & - & - & - & - \\
\textbf{S-BertScore} & 0.11 & -0.38 & -0.07* & -0.17* & 0.28 & 0.12 & 0.25 & 6.0 \\
\textbf{R-BertScore} & - & -0.02* & -0.21* & - & - & - & - & - \\
\textbf{S-Bleurt} & 0.29 & 0.05* & 0.45 & 0.06* & 0.29 & 0.23 & 0.46 & 4.2 \\
\textbf{R-Bleurt} & - &  0.21 & 0.38 & - & - & - & - & - \\
\textbf{S-Cosine} & 0.01* & -0.5 & -0.13* & -0.19* & 0.05* & -0.05* & 0.15* & 7.42 \\
\textbf{R-Cosine} & - & -0.11* & -0.16* & - & - & - & - & - \\ \midrule
\textbf{QuestEval} & 0.21 & {\ul{0.29}} & 0.23 & 0.37 & 0.19* & 0.35 & 0.14* & 4.67 \\
\textbf{LlaMA3} & \textbf{0.82} & \textbf{0.80} & \textbf{0.72} & \textbf{0.84} & \textbf{0.84} & \textbf{0.90} & \textbf{0.88} & \textbf{1.00} \\
\textbf{Our (3b)} & 0.47 & -0.11* & 0.37 & 0.61 & 0.53 & 0.54 & 0.66 & 3.5 \\
\textbf{Our (8b)} & {\ul{0.57}} & 0.09* & {\ul 0.49} & {\ul 0.72} & {\ul 0.64} & {\ul 0.62} & {\ul 0.67} & {\ul 2.17} \\ \bottomrule
\end{tabular}
\caption{Pearson correlation on human ratings on our constructed test set. 'R-': reference-based. 'S-': source-based. *not significant; we cannot reject the null hypothesis of zero correlation}
\label{tab:con}
\end{table*}

\section{Results}
We benchmark the different metrics on the different datasets using correlation to human judgement. For content preservation, we show results split on data with system output, reference output and our constructed test set: we show that the data source for evaluation leads to different conclusions on the metrics. In addition, we examine whether the metrics can rank style transfer systems similar to humans. On style strength, we likewise show correlations between human judgment and zero-shot evaluation approaches. When applicable, we summarize results by reporting the average correlation. And the average ranking of the metric per dataset (by ranking which metric obtains the highest correlation to human judgement per dataset). 

\subsection{Content preservation}
\paragraph{How do data sources affect the conclusion on best metric?}
The conclusions about the metrics' performance change radically depending on whether we use system output data, reference output, or our constructed test set. Ideally, a good metric correlates highly with humans on any data source. Ideally, for meta-evaluation, a metric should correlate consistently across all data sources, but the following shows that the correlations indicate different things, and the conclusion on the best metric should be drawn carefully.

Looking at the metrics correlations with humans on the data source with system output (Table~\ref{tab:sys}), we see a relatively high correlation for many of the metrics on many tasks. The overall best metrics are S-BertScore and S-BLEURT (avg+avg rank). We see no notable difference in our method of using the 3B or 8B model as the backbone.

Examining the average correlations based on data with reference output (Table~\ref{tab:ref}), now the zero-shoot prompting with LlaMA3 70B is the best-performing approach ($0.59$ avg). Tied for second place are source-based cosine embedding ($0.46$ avg), BLEURT ($0.45$ avg) and BertScore ($0.45$ avg). Our method follows on a 5. place: here, the 8b version (($0.43$ avg)) shows a bit stronger results than 3b ($0.39$ avg). The fact that the conclusions change, whether looking at reference or system output, confirms the observations made by \citet{scialom-etal-2021-questeval} on simplicity transfer.   

Now consider the results on our test set (Table~\ref{tab:con}): Several metrics show low or no correlation; we even see a significantly negative correlation for some metrics on ALL (BLEU) and for specific subparts of our test set for BLEU, Meteor, BertScore, Cosine. On the other end, LlaMA3 70B is again performing best, showing strong results ($0.82$ in ALL). The runner-up is now our 8B method, with a gap to the 3B version ($0.57$ vs $0.47$ in ALL). Note our method still shows zero correlation for the sentiment task. After, ranks BLEURT ($0.29$), QuestEval ($0.21$), BertScore ($0.11$), Cosine ($0.01$).  

On our test set, we find that some metrics that correlate relatively well on the other datasets, now exhibit low correlation. Hence, with our test set, we can now support the logical reasoning with data evidence: Evaluation of content preservation for style transfer needs to take the style shift into account. This conclusion could not be drawn using the existing data sources: We hypothesise that for the data with system-based output, successful output happens to be very similar to the source sentence and vice versa, and reference-based output might not contain server mistakes as they are gold references. Thus, none of the existing data sources tests the limits of the metrics.  


\paragraph{How do reference-based metrics compare to source-based ones?} Reference-based metrics show a lower correlation than the source-based counterpart for all metrics on both datasets with ratings on references (Table~\ref{tab:sys}). As discussed previously, reference-based metrics for style transfer have the drawback that many different good solutions on a rewrite might exist and not only one similar to a reference.


\paragraph{How well can the metrics rank the performance of style transfer methods?}
We compare the metrics' ability to judge the best style transfer methods w.r.t. the human annotations: Several of the data sources contain samples from different style transfer systems. In order to use metrics to assess the quality of the style transfer system, metrics should correctly find the best-performing system. Hence, we evaluate whether the metrics for content preservation provide the same system ranking as human evaluators. We take the mean of the score for every output on each system and the mean of the human annotations; we compare the systems using the Kendall's Tau correlation. 

We find only the evaluation using the dataset Mir, Lai, and Ziegen to result in significant correlations, probably because of sparsity in a number of system tests (App.~\ref{app:dataset}). Our method (8b) is the only metric providing a perfect ranking of the style transfer system on the Lai data, and Llama3 70B the only one on the Ziegen data. Results in App.~\ref{app:results}. 


\subsection{Style strength results}
%Evaluating style strengths is a challenging task. 
Llama3 70B shows better overall results than our method. However, our method scores higher than Llama3 70B on 2 out of 6 datasets, but it also exhibits zero correlation on one task (Table~\ref{tab:styleresults}).%More work i s needed on evaluating style strengths. 
 
\begin{table}%[]
\fontsize{11pt}{11pt}\selectfont
\begin{tabular}{lccc}
\toprule
\multicolumn{1}{c}{\textbf{}} & \textbf{LlaMA3} & \textbf{Our (3b)} & \textbf{Our (8b)} \\ \midrule
\textbf{Mir} & 0.46 & 0.54 & \textbf{0.57} \\
\textbf{Lai} & \textbf{0.57} & 0.18 & 0.19 \\
\textbf{Ziegen.} & 0.25 & 0.27 & \textbf{0.32} \\
\textbf{Alva-M.} & \textbf{0.59} & 0.03* & 0.02* \\
\textbf{Scialom} & \textbf{0.62} & 0.45 & 0.44 \\
\textbf{\begin{tabular}[c]{@{}l@{}}Our Test\end{tabular}} & \textbf{0.63} & 0.46 & 0.48 \\ \bottomrule
\end{tabular}
\caption{Style strength: Pearson correlation to human ratings. *not significant; we cannot reject the null hypothesis of zero corelation}
\label{tab:styleresults}
\end{table}

\subsection{Ablation}
We conduct several runs of the methods using LLMs with variations in instructions/prompts (App.~\ref{app:method}). We observe that the lower the correlation on a task, the higher the variation between the different runs. For our method, we only observe low variance between the runs.
None of the variations leads to different conclusions of the meta-evaluation. Results in App.~\ref{app:results}.

\section{Conclusions}\label{sec:conclusions}
%\todo[inline,color=green]{Break into paragraphs, for readability}
This paper proposes two approaches to determine a learned Bayesian Cramér-Rao Bound (LBCRB): the Posterior Approach, and the Measurement-Prior Approach. Both approaches derive the LBCRB using only a dataset of independent and identically distributed (i.i.d.) measurement-parameter pairs. In the Posterior Approach we learn the score of the parameter conditioned on the measurements and use it to derive the BCRB. For the Measurement-Prior Approach, we introduce a variant of score matching that enables  to learn the Fisher score from i.i.d. samples. In addition, we propose a \pe{} score neural network that integrates domain knowledge into the network. 

We  {studied} both approaches theoreticallyand demonstrated the superiority of the Measurement-Prior Approach  over the Posterior Approach when domain knowledge is available, {or when it is desired to compute the \name{} for a measurement comprising multiple i.i.d. samples}. Furthermore, we showed theoretically that in both approaches the LBCRB converges almost surely to the true BCRB . 

We conduct several numerical experiments that validate our theoretical findings and highlight the advantages of the Measurement-Prior Approach when domain knowledge is present. Finally, we present two scenarios where the BCRB cannot be evaluated but the LBCRB can: one involving correlated quantization measurement and the other involving frequency estimation with real-world underwater noise. These examples highlight the usefulness of LBCRB in real life problems. 

% Questions for future research include quantifying the
% impacts of limited representation power of the score model and a limited training data set on the accuracy of
% the LBCRB.  
% \todoin{This can be confusing for the reader: we did study the effect of limited training data, and the effect of finite representation power - as expressed in the score mismatch. Can you provide a tighter description of this proposed direction to eliminate such confusion - or just drop this suggested direction?}

A future research direction is to ensure that LBCRB is
a valid lower bound (rather than a good appropriation to
it) by utilizing methods for error estimation and model
selection \cite{shalev2014understanding}. On the practical side, it will be interesting
to study some of the many real-world applications that
can benefit from this approach, such as inverse problems with poorly characterized sensors. Finally, an addition direction is to use the Fisher score to obtain a learned maximum likelihood estimator. 




\bibliographystyle{IEEEtran}
\bibliography{ref}

\clearpage
\appendix
\subsection{Implementation Details}
{%Here, we provide details of the implantation. 
We 
%begin with the description of the 
use a conditional NN for the score NN
%neural network and the utilization of 
and (similar to \cite{habi2024learning}) train a single NN  on multiple S-s, which are used as a conditioning variable. % (this is similar to \cite{habi2024learning})
}
% \subsubsection{Conditional Score Neural Network}\label{apx:cond_score}
% In many cases, we have a known condition variable, such as the signal-to-noise ratio, sensor parameter, etc. that affects the measurement distribution. Instead of training a multiple score that matches one for each condition, we suggest learning a single score model. Specifically,  we denote this condition by $\vectorsym{c}$, then the score model is given by:
% \begin{equation}\label{eq:model_cond}
% \postscore{\p}{\x,\vectorsym{c}}=\lscore{\x}{\p,\vectorsym{c}}+\priorscore{\p}.
% \end{equation}
% Then to optimize the model in \eqref{eq:model_cond} we use \eqref{eq:score_lik_mean} while the data set also includes the condition vector, namely $\ds=\{\x_i,\p_i,\vectorsym{c}_i\}_{i=1}^{\nds}$ and the objectives change into:
% \begin{subequations}
% \begin{equation}
%     \lossbsm\brackets{\Omega}=\frac{1}{2\nds}\sum_{\p,\x,\vectorsym{c}\in\ds}\norm{\postscore{\p}{\x,\vectorsym{c};\Omega}}_2^2+\trace{\overline{\matsym{J}}_B},
% \end{equation}
% \begin{align}
%     \lossfsm\brackets{\paramf; \paramp}&=\frac{1}{2\nds}\sum_{\x,\p,\vectorsym{c}\in\mathcal{D}}\norm{\lscore{\x}{\p,\vectorsym{c};\paramf}}_2^2+\trace{\overline{\matsym{J}}_{F}}\nonumber \\
%     &+\frac{1}{\nds}\sum_{\x,\p,\vectorsym{c}\in\mathcal{D}}\lscore{\x}{\p,\vectorsym{c};\paramf}^T\priorscore{\p;\paramp},
% \end{align}
% \end{subequations}
% where $\overline{\matsym{J}}_{F}\brackets{\paramf}\triangleq\frac{1}{\nds}\sum_{\x,\p,\vectorsym{c}\in\mathcal{D}}{\frac{\partial \lscore{\x}{\p,\vectorsym{c};\paramf}}{\partial\p}}$ and $\overline{\matsym{J}}_B\brackets{\Omega}\triangleq\frac{1}{\nds}\sum_{\p,\x,\vectorsym{c}\in\ds}{\frac{\partial \postscore{\p}{\x,\vectorsym{c};\Omega}}{\partial\p}}$ are the average Jacobin matrix. 
\subsubsection{Neural Network Structure}
% \ybdelete{In this paper} We use a neural network architecture similar to a multilayer perceptron that incorporates conditional variables \cite{ho2021classifier,karras2020analyzing}. The MLP consists of a sequence of $N$ \todo[inline,color=green]{What $L$ is used for each example? \\
% HVH:L and N are the same this is a mistake on my part, fixed.} 
basic blocks {%that operate as
illustrated in Figure~\ref{fig:basic_block}, each performing the following sequence of operations:}
(i) parameter injection, %is performed,
where the input is scaled and shifted based on the parameter vector $\p$; 
(ii) a fully connected operation; 
%is executed; 
(iii) condition injection (not used in the prior score model $\priorscore{\p}$ ) involving the application of a scale factor to the features; and finally (iv) %a non-linear operation is applied, 
point-wise Swish \cite{ramachandran2017searching} non-linearity.
%for which we utilize Swish \cite{ramachandran2017searching}.
{The parameters of the NN structures used for the various examples are listed in Table~\ref{tab:net-param}.}
\begin{figure}
    \centering
    \begin{tikzpicture}[node distance=2cm]
    \node (ry) {$\vectorsym{z}_\ell$};
    \node[below=0.25 cm of ry] (rc) {$\vectorsym{c}$};
    \node[above=0.25 cm of ry] (rp) {$\p$};
    \node[fc,text width=1.4cm,minimum width=1.4cm,right=0.2cm of ry,align=center] (fc2)  {\small Parameter \\
    Injection};
    \node[fc,text width=1.4cm,minimum width=1.5cm,right =0.2cm of fc2,align=center] (fc1)  {\small $\mathrm{FC}(n_{i},n_{h})$};


    \node[fc,text width=1.4cm,minimum width=1.4cm,right=0.2cm of fc1,align=center] (fc_n)  {\small Condition \\
    Injection};
        \node[pool,text width=0.8cm,minimum width=0.5cm,right=0.2cm of fc_n,align=center] (nl)  {\small NL};
    \node[right =0.2cm of nl] (out) {$\vectorsym{z}_{\ell+1}$};
    \draw[->] (ry) -- (fc2);
    \draw[->] (fc2) -- (fc1);
    \draw[->] (fc1) -- (fc_n);
    \draw[->] (fc_n) -- (nl);
    \draw[->] (nl) -- (out);
    \draw[->] (rc) -| (fc_n);
    \draw[->] (rp) -| (fc2);
  \end{tikzpicture}
    \caption{Basic NN Block}
    \label{fig:basic_block}
\end{figure}
\subsubsection{Hyper-parameters {and Training}}\label{apx:param}
In all experiments, we use the following hyperparameters. We train the score neural network for  200 epochs using the AdamW \cite{loshchilov2018decoupled} optimizer with learning rate 4e-4 and weight decay $1e-4$, %. Using 
and a data set of $60k$ samples for each S- condition.% (S-). 
If a smaller dataset is used, then the number of epochs is adjusted to keep the same
%such that the 
number of gradient updates.
%will remain the same. 
The batch size is set to 512. We employ exponentially decayed averages (EMA) of the weights with a decay rate of 0.999 as suggested in \cite{song2020improved}. At the end of the training we take the weights from the last EMA update. 
\begin{table}[]
\caption{Neural Network Configuration and Parameters for Each Example. The Score Model specifies the type of score function approximated by the NN.
%the neural network approximates. 
"N Blocks" indicates the number of basic blocks. 
% \todo[inline,color=green]{In the text you say "$L$ basic blocks". What is the connection between blocks and layers?} 
"Non-linearity" {indicates the presence/absence of a nonlinearity in the NN.} {"True Score Realizable" indicates whether the true score function} can be represented by the NN. %neural network.
{{"\pe{}"} {indicates whether the NN is \pe{}.} }. 
}
\label{tab:net-param}
\resizebox{\columnwidth}{!}{%
\begin{tabular}{ccccccc}
\hline
Example                                                                                          & \begin{tabular}[c]{@{}c@{}}Score\\ Model\end{tabular} & \begin{tabular}[c]{@{}c@{}}N\\ Blocks\end{tabular} & Width & \begin{tabular}[c]{@{}c@{}}Non\\ Linearity\end{tabular} & \begin{tabular}[c]{@{}c@{}}True \\ Score \\ Realizable\end{tabular} & \begin{tabular}[c]{@{}c@{}}Physics \\ Encoded\end{tabular} \\ \hline
Linear                                                                                           & Posterior                                             & 1                                                  & -    & -                                                      & $\cmark$                                                           & -                                                \\
\multirow{2}{*}{Linear}                                                                          & Fisher                                            & 1                                                  & -    & -                                                      & $\cmark$                                                           & -                                                 \\
                                                                                                 & Prior                                                 & 1                                                  & -    & -                                                      & $\cmark$                                                           & -                                                 \\
\multirow{2}{*}{Quantization}                                                                    & Fisher                                            & 3                                                  & 96    & $\cmark$                                                        & -                                                         & -                                              \\
                                                                                                 & Prior                                                 & 1                                                  & -    & -                                                      & $\cmark$                                                           & -                                                 \\
\multirow{2}{*}{\begin{tabular}[c]{@{}c@{}}Frequency Estimation\\ Gaussian Noise \end{tabular}}   & Fisher                                            & 3                                                  & 96    & $\cmark$                                                        & -                                                          & -                                                 \\
                                                                                                 & Prior                                                 & 3                                                  & 96    &$\cmark$                                                        & -                                                          & -                                                  \\
\multirow{2}{*}{\begin{tabular}[c]{@{}c@{}}Frequency Estimation\\ Gaussian Noise\end{tabular}}   & Fisher                                            & 1                                                  & -    & -                                                      & $\cmark$                                                           & $\cmark$                                                  \\
                                                                                                 & Prior                                                 & 3                                                  & 96    & $\cmark$                                                        & -                                                          &-                                                \\
\multirow{2}{*}{\begin{tabular}[c]{@{}c@{}}Frequency Estimation\\ Underwater Noise\end{tabular}} & Fisher                                            & 2                                                  & 96    & $\cmark$                                                        & -                                                          & $\cmark$                                                  \\
                                                                                                 & Prior                                                 & 3                                                  & 96    & $\cmark$                                                       & -                                                          & -                                              \\ \hline
\end{tabular}%
}
\end{table}
\subsubsection{Measurement Models {Setup}}\label{apx:init_sp}
%In the cases of a 
The linear measurement model \eqref{eq:linear-model} or the quantized linear measurement model \eqref{eq:lin_q} are {set up as follows.} Each element in $\matsym{A}$ is generated by $\squareb{\matsym{A}}_{ij}\sim\normaldis{0}{1}$. For the covariance matrix, we first generate $\matsym{U}\in\mathbb{R}^{d_x\times d_x}$   as %in the following meaner
$\squareb{\matsym{U}}_{ij}\sim\normaldis{0}{1}$ and then $\matsym{\Sigma}=\frac{\matsym{U}\matsym{U}^T}{\trace{\matsym{U}\matsym{U}^T}}$. 
\section{Proofs}
\label{sec:appendix}



\begin{figure*}[ht!]
    \centering
    \includegraphics[width=0.99\linewidth]{plots/squares_p4_and_z1z2_d5000_10seeds_overlaps.pdf}
    \caption{Overlap $\normf{\bM}^2 / \Tr(\bQ)$ as a function of the sample complexity $\alpha$. The dots represent numerical simulation results, computed for $n = 5000$ (for the asymmetric method) or $d = 5000$ (for the symmetric method) and averaging over $10$ instances. (\textbf{Left}) Link function $g(z_1,z_2) = z_1z_2$. Solid lines are obtained from state evolution predictions eq. (\ref{eq:overlap_prod_zk},\ref{eq:examples_symmetric_general}). Dashed line at $\alpha_c \approx 0.59375$. (\textbf{Right}) Link function $g(\bz) = p^{-1}\norm{\bz}^2$, $p = 4$. Solid lines are obtained from state evolution predictions eq. (\ref{eq:overlap_asymmetric_squares},\ref{eq:examples_symmetric_general}). Dashed line at $\alpha_c = 2$.}
    \label{fig:z1z2_overlap}
\end{figure*}
In this section we illustrate the framework introduced in Section \ref{sec:main_results} to predict the asymptotic performance of the spectral estimators (\ref{eq:def:spectral_asymmetric},\ref{eq:def:spectral_symmetric}) for specific examples of link functions, providing a comparison between our asymptotic analytical results and finite size numerical simulations for the overlap between the spectral estimators and the weights $\mat{W}_\star$, defined as $m \coloneqq \nicefrac{\normf{\bM}}{\sqrt{\Tr(\bQ)}}$, where $\bM$ and $\bQ$ are the overlap matrices defined in eq. (\ref{eq:def:overlaps_amp}) correspondent to the fixed points in Lemmas \ref{result:1}, \ref{result:2}, \ref{result:3}, \ref{result:4}. In Figure \ref{fig:z1z2_overlap} we compare these theoretical predictions to numerical simulations at finite dimensions, respectively for the link functions $g(z_1,z_2) = z_1z_2$ and $g(\bz) = p^{-1}\norm{\bz}^2$. Additional numerical experiments are presented in Appendix \ref{app:example_details}.
\subsection{Asymmetric spectral method}\label{sec:examples_asymmetric}
%\subsection{$\dgout(y)$ jointly diagonalizable $\forall y$}
We provide closed-form expressions for the overlap parameter $m \coloneqq \nicefrac{\normf{\bM}}{\sqrt{\Tr(\bQ)}}$ of the spectral estimator $\matwhat_{\tens{L}}$ (\ref{eq:def:spectral_asymmetric}), for a selection of examples of link functions. The details of the derivation are given in Appendix \ref{app:example_details}.
\begin{itemize}[leftmargin=2em,wide=1pt]
    \item $g(z\in\R)$ (single-index model):
    \begin{align}
        \alpha_c &= \left(\E_{\rdm{y}\sim\Zout}\left[\left(\Var[z\big|\rdm{y}] - 1\right)^2\right]\right)^{-1},\\
        m^2 &= \Bigg(\frac{\alpha - \alpha_c}{\alpha + \alpha_c^2\E_{\rdm{y}\sim\Zout}\left[\left(\Var[z\big|\rdm{y}] - 1\right)^3\right]}\Bigg)_+
    \end{align}
    \item $g(\bz) = p^{-1}||\bz||^2$:
    \begin{equation}\label{eq:overlap_asymmetric_squares}
        \alpha_c = \frac{p}{2},\quad m^2 = \left(\frac{(2\alpha-p)}{2(\alpha+2)}\right)_+
    \end{equation}
    \item $g(\bz) = \operatorname{sign}(z_1z_2)$: 
    \begin{equation}
        \alpha_c = \frac{\pi^2}{4},\quad m^2 = \left(1-\frac{\pi^2}{4\alpha}\right)_+
    \end{equation}
    \item $g(\bz) = \prod_{k=1}^pz_k$:
   \begin{align}\label{eq:overlap_prod_zk}
        \alpha_c &= \left(\E_{\rdm{y}\sim\Zout}\left[\lambda(\rdm{y})^2\right]\right)^{-1},\\
        m^2 &= \left(\frac{\alpha - \alpha_c}{\alpha + \alpha_c^2\E_{\rdm{y}\sim\Zout}\left[\lambda(\rdm{y})^3\right]}\right)_+,
    \end{align}
    where
    \begin{equation}\label{eq:lambda_prod_zk}
        \lambda(y) = \begin{cases}
            \begin{array}{ll}
              |y|\frac{K_1(|y|)}{K_0(|y|)} +\rdm{y}- 1,   & p = 2 \\
              \frac{2G^{p, 0}_{0, p} \left( y^22^{-p}  \, \bigg| \, \begin{array}{c}
0 \\
\vect{e}_p
\end{array} \right)}{ G^{p, 0}_{0, p} \left( y^22^{-p} \, \bigg| \, \begin{array}{c}
0 \\
\bzero_p
\end{array} \right)} - 1,   & p\geq 3
            \end{array}
        \end{cases}
    \end{equation}
    and the previous expression are written in terms of the modified Bessel function of the second kind and Meijer $G$-function, with the notations $\bzero_p\in\R^p = (0,\ldots,0)^T$ and $\vect{e}_p\in\R^p=(0,\ldots,0,1)^T$.
    \item $g(z_1,z_2) = z_1z_2^{-1}$: $
        \alpha_c = 1,\quad m^2 = (1 -\alpha^{-1})_+$
\end{itemize}

\subsection{Symmetric spectral method}
We provide expressions for the overlap parameter $m \coloneqq \nicefrac{\normf{\bM}}{\sqrt{\Tr(\bQ)}}$ of the spectral estimator $\matwhat_{\tens{T}}$ (\ref{eq:def:spectral_symmetric}), for a selection of examples of link functions.
In all the following cases, the state evolution equations simplify, allowing to write the results as functionals of $\lambda:\R\to\R$, which is a specific to each problem:
\begin{itemize}
    \item $g(z\in\R)$ (single-index model): $\lambda(y) = \Var[z\big|y] - 1$;
    
    \item $g(\bz) = p^{-1}||\bz||^2$: $\lambda(y) = y - 1$;
    \item $g(\bz) = \operatorname{sign}(z_1z_2)$: $\lambda(y) = 2\pi^{-1}y$;
    \item $g(\bz) = \prod_{k=1}^pz_k$: $\lambda(y)$ defined in eq. (\ref{eq:lambda_prod_zk}).
\end{itemize}
For all these example, the value $\alpha_c$ can be found in Section \ref{sec:examples_asymmetric}.
For $\alpha>\alpha_c$, consider $a$ and $\gamma$ solutions of
\begin{align}\label{eq:examples_symmetric_a}
&\E_{\rdm{y}\sim\Zout}\left[\frac{\lambda(\rdm{y})^2}{a(1 + \lambda(\rdm{y})) - \lambda(\rdm{y}) }\right] = \frac{1}{\alpha}\\
\label{eq:examples_symmetric_gamma}
&\gamma = 1 + \alpha\E_{\rdm{y}\sim\Zout}\left[\frac{\lambda(\rdm{y})}{a(1 + \lambda(\rdm{y})) - \lambda(\rdm{y}) }\right].
\end{align}
Then, for any $\alpha$, the overlap $m \coloneqq \nicefrac{\normf{\bM}}{\sqrt{\Tr(\bQ)}} $ is given by
\begin{equation}\label{eq:examples_symmetric_general}
    m^2 = \Bigg(\frac{1 - \alpha\E_{\rdm{y}\sim\Zout}[\lambda^2(\rdm{y})\left(a(1 + \lambda(\rdm{y})) - \lambda(\rdm{y})\right)^{-2}]}{1 + \alpha\E_{\rdm{y}\sim\Zout}[\lambda^3(\rdm{y})\left(a(1+ \lambda(\rdm{y})) - \lambda(\rdm{y})\right)^{-2}] } \Bigg)_+,
\end{equation}
which is strictly positive $\forall \alpha > \alpha_c$. Additional details on the derivation of this result can be found in Appendix \ref{app:details_examples_symmetric}.

\subsection{Prior FIMs}\label{apx:prior_fim}
% \subsubsection{Prior FIM of Gamma Distribution}\label{apx:prior_gamma}
% \begin{equation}
%     f_{\rho}\brackets{\rho}=\frac{\beta_{\rho}^{\alpha_{\rho}}}{\Gamma\brackets{\alpha_{\rho}}}\rho^{\alpha_{\rho}-1}\exp\brackets{-\beta_{\rho}\rho}
% \end{equation}
% \begin{align}
%     \nabla_{\rho}\log\brackets{ f_{\rho}\brackets{\rho}}=\frac{\alpha_{\rho}-1}{\rho}-\beta_{\rho}
% \end{align}
% \begin{align}\label{eq:base_rho_prior}
%     \F_{\rho}=&\expectation{\brackets{\frac{\alpha_{\rho}-1}{\rho}-\beta_{\rho}}^2}{\rho}\nonumber\\
%     &=\expectation{\frac{\brackets{\alpha_{\rho}-1}^2}{\rho^2}-2\beta_{\rho}\frac{\alpha_{\rho}-1}{\rho}+\beta_{\rho}^2}{\rho}
% \end{align}
% Assume that $\alpha-1>0$ we have that:
% \begin{align}\label{eq:exp_gamma_prior_one}
%     \expectation{\frac{1}{\rho}}{\rho}=\frac{\beta_{\rho}^{\alpha_{\rho}}\Gamma\brackets{\alpha_{\rho}-1}}{\beta_{\rho}^{\alpha_{\rho}-1}\Gamma\brackets{\alpha_{\rho}}}=\frac{\beta_{\rho}}{\alpha_{\rho}-1}
% \end{align}
% Now assume that $\alpha-2>0$ we have that:
% \begin{align}\label{eq:exp_gamma_prior_two}
%     \expectation{\frac{1}{\rho^2}}{\rho}=\frac{\beta_{\rho}^{\alpha_{\rho}}\Gamma\brackets{\alpha_{\rho}-2}}{\beta_{\rho}^{\alpha_{\rho}-2}\Gamma\brackets{\alpha_{\rho}}}=\frac{\beta_{\rho}^2}{\brackets{\alpha_{\rho}-2}\brackets{\alpha_{\rho}-1}}
% \end{align}
% Plugging \eqref{eq:exp_gamma_prior_one} and \eqref{eq:exp_gamma_prior_two} into \eqref{eq:base_rho_prior} results in:
% \begin{equation}
%     \F_{\rho}=\beta^2_{\rho}\brackets{\frac{\alpha_{\rho}-1}{\alpha_{\rho}-2}-1}=\frac{\beta_{\rho}^2}{\alpha_{\rho}-2}.
% \end{equation}
% Note that we assume that $\rho$ is independent of any other parameter and 
% \begin{equation}
%     \expectation{\nabla_{\rho}\log\brackets{ f_{\rho}\brackets{\rho}}}{\rho}=0,
% \end{equation}
% we have that there off diagonal elements in the Prior FIM are equal to zero.
In this section we derive the prior Fisher information matrix of Beta and Gaussian distributions.
\subsubsection{Prior FIM of Beta Distribution}
Let $\kappa\sim\mathrm{Beta}\brackets{\alpha_{\kappa},\beta_{\kappa},a_l,a_h}$ be a random variable that is distribution according to the four-parameter beta distribution. Then, the PDF of $\kappa$ is given by:
\begin{equation}
    \probt{\kappa}{\kappa}=\frac{\brackets{\kappa-a_l}^{\alpha_{\kappa}-1}\brackets{a_u-\kappa}^{\beta_{\kappa}-1}}{\brackets{a_h-a_l}^{\alpha_{\kappa}+\beta_{\kappa}-1}\mathrm{B}\brackets{\alpha_{\kappa},\beta_{\kappa}}}
\end{equation}
where  $a_l,a_u$ are the lower and upper support parameters ($a_u>a_l$), respectively, $\alpha_{\kappa}>0$, $\beta_{\kappa}>0$ are the shape parameters, $B\brackets{\alpha,\beta}=\frac{\Gamma\brackets{\alpha}\Gamma\brackets{\beta}}{\Gamma\brackets{\beta+\alpha}}$ is the normalization constant (Beta function) and $\Gamma\brackets{\alpha}$ is the Gamma function.
As a direct result, the score of $\kappa$ is given by:
%\todoing{Use $d/d\kappa$ for derivative wrt to scalar $\kappa$?}
\begin{equation}
    \frac{d \log\probt{\kappa}{\kappa}}{d\kappa}=\frac{\alpha_{\kappa}-1}{\kappa-a_l}-\frac{\beta_{\kappa}-1}{a_h-\kappa}
\end{equation}
Assuming that $\alpha_{\kappa}-2>0$
\begin{equation}
    \expectation{\frac{\brackets{\alpha_{\kappa}-1}^2}{\brackets{\kappa-a_l}^2}}{\kappa}=\frac{ \brackets{\alpha_{\kappa}-1}^2B\brackets{\alpha_{\kappa}-2,\beta_{\kappa}}}{\brackets{a_h-a_l}^{2} B\brackets{\alpha_{\kappa},\beta_{\kappa}}}
\end{equation}
Next, using the definition of Beta function we have:
\begin{equation}
    \expectation{\frac{\brackets{\alpha_{\kappa}-1}^2}{\brackets{\kappa-a_l}^2}}{\kappa}=\frac{ \brackets{\alpha_{\kappa}-1}^2\Gamma\brackets{\alpha_{\kappa}-2}\Gamma\brackets{\alpha_{\kappa}+\beta_{\kappa}}}{\brackets{a_h-a_l}^{2} \Gamma\brackets{\alpha_{\kappa}}\Gamma\brackets{\alpha_{\kappa}-2+\beta_{\kappa}}}
\end{equation}
Using the Gamma function properties we have:
\begin{equation}\label{eq:beta_prior_p1}
\expectation{\frac{\brackets{\alpha_{\kappa}-1}^2}{\brackets{\kappa-a_l}^2}}{\kappa}=\frac{ \brackets{\alpha_{\kappa}-1+\beta_{\kappa}}\brackets{\alpha_{\kappa}-2+\beta_{\kappa}}\brackets{\alpha_{\kappa}-1}}{\brackets{a_h-a_l}^{2} \brackets{\alpha_{\kappa}-2}}
\end{equation}
In a similar fashion we have that:
\begin{align}\label{eq:beta_prior_p2}
    &\expectation{\frac{\brackets{\beta_{\kappa}-1}^2}{\brackets{a_h-\kappa}^2}}{\kappa}=\frac{ \brackets{\beta_{\kappa}-1}^2B\brackets{\alpha_{\kappa},\beta_{\kappa}-2}}{\brackets{a_h-a_l}^{2} B\brackets{\alpha_{\kappa},\beta_{\kappa}}}\nonumber\\
    &=\frac{\brackets{\beta_{\kappa}-1}^2\Gamma\brackets{\beta_{\kappa}-2}\Gamma\brackets{\alpha_{\kappa}+\beta_{\kappa}}}{\brackets{a_h-a_l}^{2} \Gamma\brackets{\alpha_{\kappa}+\beta_{\kappa}-2} \Gamma\brackets{\beta_{\kappa}}}\nonumber\\
    &=\frac{ \brackets{\alpha_{\kappa}-1+\beta_{\kappa}}\brackets{\alpha_{\kappa}-2+\beta_{\kappa}}\brackets{\beta_{\kappa}-1}}{\brackets{a_h-a_l}^{2} \brackets{\beta_{\kappa}-2}}
\end{align}
\begin{align}\label{eq:beta_prior_p3}
    &\expectation{\frac{\brackets{\alpha_{\kappa}-1}\brackets{\beta_{\kappa}-1}}{\brackets{\kappa-a_l}\brackets{a_h-\kappa}}}{\kappa}=\frac{ \brackets{\alpha_{\kappa}-1}\brackets{\beta_{\kappa}-1}B\brackets{\alpha_{\kappa}-1,\beta_{\kappa}-1}}{\brackets{a_h-a_l}^{2} B\brackets{\alpha_{\kappa},\beta_{\kappa}}}\nonumber\\
    &=\frac{\brackets{\alpha_{\kappa}-1}\brackets{\beta_{\kappa}-1}\Gamma\brackets{\alpha_{\kappa}-1}\Gamma\brackets{\beta_{\kappa}-1}\Gamma\brackets{\beta_{\kappa}+\alpha_{\kappa}}}{\brackets{a_h-a_l}^{2}\Gamma\brackets{\alpha_{\kappa}}\Gamma\brackets{\beta_{\kappa}}\Gamma\brackets{\beta_{\kappa}+\alpha_{\kappa}-2}}\nonumber\\
    &=\frac{\brackets{\beta_{\kappa}+\alpha_{\kappa}-1}\brackets{\beta_{\kappa}+\alpha_{\kappa}-2}}{\brackets{a_h-a_l}^{2}}
\end{align}
Finally combining \eqref{eq:beta_prior_p1}, \eqref{eq:beta_prior_p2} and \eqref{eq:beta_prior_p3}  results in:
\begin{align}\label{eq:beta_prior_fim}
&\expectation{\brackets{\frac{d \log\probt{\kappa}{\kappa}}{d\kappa}}^2}{\kappa}\nonumber \\
&=\frac{\brackets{\alpha_{\kappa}-1+\beta_{\kappa}}\brackets{\alpha_{\kappa}-2+\beta_{\kappa}}}{\brackets{a_h-a_l}^{2}}\brackets{\frac{1}{\alpha_{\kappa}-2}+\frac{1}{\beta_{\kappa}-2}}
\end{align}

% Now let's assume that we wise to have mean $m=\expectation{\kappa}{\kappa}$ and variance $v$, we would like to have the $\alpha_{\kappa}$ and $\beta_{\kappa}$ values. From the mean condition we have:
% \begin{equation}
%     m=\frac{\alpha_{\kappa}}{\alpha_{\kappa}+\beta_{\kappa}}\brackets{a_h-a_l}+a_l,
% \end{equation}
% which results in:
% \begin{equation}\label{eq:beta_scale}
%     \beta_{\kappa}=\alpha_{\kappa}\frac{a_h-a_l}{m-a_l}=\alpha_{\kappa}\Delta_{\alpha}.
% \end{equation}
% Next we plunging \eqref{eq:beta_scale} into variance  equation results in:
% \begin{align}
%     v&=\frac{\alpha^2\Delta_{\alpha}\brackets{a_h-a_l}^2}{\brackets{\alpha+\alpha\Delta_{\alpha}}^2\brackets{\alpha+\alpha\Delta_{\alpha}+1}}\\
%     &=\frac{\Delta_{\alpha}\brackets{a_h-a_l}^2}{\brackets{1+\Delta_{\alpha}}^2\brackets{\alpha\brackets{1+\Delta_{\alpha}}+1}}
% \end{align}
% which results in:

% \begin{equation}
%     \alpha=
% \end{equation}





\subsubsection{Prior FIM of Gaussian Distribution}\label{apx:prior_fim_gaussian}
Let $\vectorsym{\kappa}\sim\normaldis{\vectorsym{\mu}}{\matsym{\Sigma}}$ be a random vector of size $k$ that distributed according to Gaussian distribution with mean $\vectorsym{\mu}$ and covariance $\matsym{\Sigma}$.
\begin{equation}
    \probt{\vectorsym{\kappa}}{\vectorsym{\kappa}}=\frac{1}{\sqrt{\brackets{2\pi}^k\det{\matsym{\Sigma}}}}\exp\brackets{-\frac{1}{2}\brackets{\vectorsym{\kappa}-\vectorsym{\mu}}^T\Sigma^{-1}\brackets{\vectorsym{\kappa}-\vectorsym{\mu}}}
\end{equation}
Then the score vector of $\vectorsym{\kappa}$ is given by:
\begin{equation}
    \nabla_{\vectorsym{\kappa}}\log\brackets{\probt{\vectorsym{\kappa}}{\vectorsym{\kappa}}}=-\matsym{\Sigma}^{-1}\brackets{\vectorsym{\kappa}-\vectorsym{\mu}}
\end{equation}
Finally the Prior FIM of Gaussian random vector is given by:
\begin{equation}
    \expectation{\nabla_{\vectorsym{\kappa}}\log\brackets{\probt{\vectorsym{\kappa}}{\vectorsym{\kappa}}}\nabla_{\vectorsym{\kappa}}\log\brackets{\probt{\vectorsym{\kappa}}{\vectorsym{\kappa}}}^T}{\vectorsym{\kappa}}=\matsym{\Sigma}^{-1}.
\end{equation}

% \todo[inline,color=pink]{
% Let $\randomvec{S}\in\mathcal{C}^{m}$ be  a signal  and $\matsym{A}\brackets{\p}\in\mathcal{C}^{n\times m}$ is a sterring matrix then:
% \begin{equation}
%     \X=A(\p)\randomvec{S}+\randomvec{W},
% \end{equation}


% \begin{equation}
%     A
% \end{equation}

% TODO:\\
% 1. Start with a simple ULA. \\
% 2. Add to it the weather acoustic noise.\\
% 3. Expand to three coordinate systems. \\

% Let $\randomvec{S}$ a random signal from $K$ sources emitting on array with $N$ sensors  such that we obtain the following measurements:
% \begin{equation}
%     \X=\matsym{A}\brackets{\p}\randomvec{S}+\randomvec{W},
% \end{equation}
% where $\matsym{A}\brackets{\p}$ is known string matrix and $\randomvec{W}$ is a non-Gaussian additive complex noise. The PDF of $\X$ given $\p$ is:
% \begin{equation}
%     \probt{\x|\p}{\X|\p}=\int \probt{\X-\matsym{A}\brackets{\p}\vectorsym{s}}{\randomvec{W}}\probt{\vectorsym{s}}{\randomvec{S}}d\vectorsym{s},
% \end{equation}
% and its corresponding score function:
% \begin{equation}
%     \divc{\log\probt{\x|\p}{\X|\p}}{\p_i}=-\frac{\int\divc{\matsym{A}\brackets{\p}\vectorsym{s}}{\p_i}^{\mathrm{H}} \at{\nabla_{\vectorsym{w}}\probt{\vectorsym{w}}{\randomvec{W}}}{\vectorsym{w}=\X-\matsym{A}\brackets{\p}\vectorsym{s}}\probt{\vectorsym{s}}{\randomvec{S}}d\vectorsym{s}}{\int \probt{\X-\matsym{A}\brackets{\p}\vectorsym{s}}{\randomvec{W}}\probt{\vectorsym{s}}{\randomvec{S}}d\vectorsym{s}}
% \end{equation}
% \todo[inline]{
% 1) The gradient in the numerator should be $\nabla_w f_W(w) |_{w=X-A(\theta)s}$.\\
% 2) The derivative in the numerator is not quite right. Left out the effect of $s$ in the expression $w=X-A(\theta)s$ on the derivative of $w$ wrt $\theta_i$.}
% In these cases, one can assume that $\mathcal{M}\brackets{\p}=\matsym{A}\brackets{\p}$, then the optimal neural network is given by:
% $$\vectorsym{g}_{\Omega_g}^*\brackets{\x,\vectorsym{\tau}}=-\frac{\int \at{\nabla_{\vectorsym{w}}\probt{\vectorsym{w}}{\randomvec{W}}}{\vectorsym{s}=\X-\vectorsym{\tau}\vectorsym{s}}\probt{\vectorsym{s}}{\randomvec{S}}d\vectorsym{s}}{\int \probt{\X-\vectorsym{\tau}\vectorsym{s}}{\randomvec{W}}\probt{\vectorsym{s}}{\randomvec{S}}d\vectorsym{s}}.$$
% }
% \subsection{Uniform Rolling Off}
Here we suggest a uniform rolling off distribution which has the following probability density function, when $\epsilon>0$, $\alpha>1$ and $b>a$

\begin{align}
    \probt{x}{X}=c\begin{cases}
                        \brackets{\frac{b-x}{\epsilon}}^{\alpha}\quad x\in (b-\epsilon,b)\\
                        1\quad x\in [a+\epsilon,b-\epsilon] \\
                        \brackets{\frac{x-a}{\epsilon}}^{\alpha}\quad x \in (a,a+\epsilon)
    \end{cases},
\end{align}
where $c$ is the normalization constant. 
\begin{align}
    u&=\int_{b-\epsilon}^{b}\brackets{\frac{b-x}{\epsilon}}^{\alpha}dx+\int_{a+\epsilon}^{b-\epsilon}dx +\int_{a}^{a+\epsilon}\brackets{\frac{x-a}{\epsilon}}^{\alpha}dx,\nonumber\\
     &=\frac{\epsilon}{\alpha+1}+b-a-2\epsilon+\frac{\epsilon}{\alpha+1}.
\end{align}
then $c=\frac{1}{u}$ and it score function is given by:
\begin{align}
    \nabla_{x}\log\probt{x}{X}=\begin{cases}
                        -\alpha\frac{\epsilon}{b-x}\quad x\in (b-\epsilon,b)\\
                        0\quad x\in [a+\epsilon,b-\epsilon] \\
                        \alpha\frac{\epsilon }{x-a}\quad x \in (a,a+\epsilon)
    \end{cases},
\end{align}
Next, we compute the prior FIM:
\begin{align}
    &\expectation{\brackets{\nabla_{x}\log\probt{x}{X}}^2}{X} =\\
    &c\alpha^{2}\int_{b-\epsilon}^{b}\brackets{\frac{b-x}{\epsilon}}^{\alpha-2}dx+c\int_{a+\epsilon}^{b-\epsilon}1\cdot 0 dx +c\alpha^{2}\int_{a}^{a+\epsilon}\brackets{\frac{x-a}{\epsilon}}^{\alpha-2}dx\\
    &=2c\alpha^{2}\frac{\epsilon}{\alpha-1} .
\end{align}
Finally, to enable sampling we compute the inverse CDF, to do so, we start by computing the CDF which is given by:
\begin{align}
    F_X\brackets{x}= c\begin{cases}
                        \frac{2\epsilon}{\alpha+1}+b-a-2\epsilon-\frac{\epsilon}{\alpha+1}\brackets{\frac{b-x}{\epsilon}}^{\alpha+1}\quad x\in (b-\epsilon,b)\\
                        \frac{\epsilon}{\alpha+1}+x-a-\epsilon\quad x\in [a+\epsilon,b-\epsilon] \\
                        \frac{\epsilon}{\alpha+1}\brackets{\frac{x-a}{\epsilon}}^{\alpha+1}\quad x \in (a,a+\epsilon)
    \end{cases},
\end{align}
and it inverse is given by:

\begin{align}
    F_X^{-1}\brackets{p}=\begin{cases}
                        \frac{2\epsilon}{\alpha+1}+b-a-2\epsilon-\frac{\epsilon}{\alpha+1}\brackets{\frac{b-x}{\epsilon}}^{\alpha+1}\quad x\in (b-\epsilon,b)\\
                        \frac{p}{c}+a+\epsilon-\frac{\epsilon}{\alpha+1}\quad x\in [\frac{c\epsilon}{\alpha+1},1-\frac{c\epsilon}{\alpha+1}] \\
                        a+\epsilon\brackets{\frac{p\brackets{\alpha+1}}{c\epsilon}}^{\frac{1}{\alpha+1}}\quad x \in \brackets{0,\frac{c\epsilon}{\alpha+1}}
    \end{cases},
\end{align}


\end{document}
