\subsection{Proof Fisher Score Matching Theorem~\ref{thm:liklihood}}\label{sec:lik_score_proof}
We will need the following minor extension of
Lemma 4 from \cite{hyvarinen2005estimation} to include, in addition to infinite limits, also the case of a finite boundary as in \cite{liu2022estimating}: 
\begin{lemma}\label{lemma:int_by_parts} Let $\z\in\mathcal{Z}\subset\mathbb{R}^n$ {
%be a vector,
and assume that 
${h}:\mathcal{Z}\rightarrow\mathbb{R}$ and $g:\mathcal{Z}\rightarrow\mathbb{R}$ are differentiable functions. T}hen: 

% \todo[inline,color=green]{$f$ is used for various PDFs in the paper. To avoid awkward  expressions such as " set $f(x, \theta) = f_{X|\Theta}(x,\theta)$", replace $f$ in the lemma by some other symbol, e.g., $h$.\\
% HVH:Done.}
    \begin{align}
        &\lim\limits_{a\rightarrow \partial z_1^{-},b\rightarrow \partial z_1^{+}} {h}\brackets{a,z_2,\hdots,z_n}g\brackets{a,z_2,\hdots,z_n}\\
        &-{h}\brackets{b,z_2,\hdots,z_n}g\brackets{b,z_2,\hdots,z_n}\\
        &=\int_{\zeta_1^{-}}^{z_1^{+}}{h}\brackets{\z}\frac{\partial g\brackets{\z}}{\partial z_1}dz_1+\int_{z_1^{-}}^{z_1^{+}}g\brackets{\z}\frac{\partial {h}\brackets{\z}}{\partial z_1}dz_1,\nonumber
    \end{align}
    where $\partial z_1^{-}$ and $\partial z_1^{+}$ represent the lower and upper limits of $z_1$, respectively{, which may be finite or infinite.}
\end{lemma}
\begin{proof}
    \begin{equation}
        \frac{\partial {h}\brackets{\z}g\brackets{\z}}{\partial z_1}={h}\brackets{\z}\frac{\partial g\brackets{\z}}{\partial z_1}+g\brackets{\z}\frac{\partial {h}\brackets{\z}}{\partial z_1}.
    \end{equation}
    Now integrate w.r.t. $z_1$, 
    keeping %while 
    the rest of the elements in $\z$ %are 
    fixed.
\end{proof}
Note that {although we only wrote the case of $i=1$,}
Lemma~\ref{lemma:int_by_parts} can be applied to any component $\z_i$ of $\z$. 

Now we proceed to prove the {Fisher Score Matching} Theorem.  
\begin{proof}
% \todo[inline,color=green]{1. Why $\vectorsym{u}$ rather than $\vectorsym{h}$ as in the text?\\
% 2. Why lower case $\ell_d$ rather than upper case as in the text?\\
% HVH: Corrected.}  
{%First we assume that
Subject to Assumption~\ref{ass:bound_expection} %holds and result in:
}
\begin{align}\label{eq:step_one_proof_alt}
        \lossfs\brackets{\paramf}&=\expectation{\norm{\lscore{\X}{\p;\paramf}-\nabla_{\p}\log\probt{\randomvec{X}|\p}{\X|\p}}_2^2}{\randomvec{X},\p},\nonumber\\
        &=\expectation{\norm{\lscore{\X}{\p;\paramf}}_2^2}{\randomvec{X},\p}\nonumber\\
        &-2\expectation{\lscore{\X}{\p;\paramf}^T\nabla_{\p}\log\probt{\randomvec{X}|\p}{\X|\p}}{\randomvec{X},\p}+C.
    \end{align}
    where $C=\expectation{\norm{\nabla_{\p}\log\probt{\randomvec{X}|\p}{\X|\p}}_2^2}{\randomvec{X},\p}$  is constant independent of $\paramf$. %
     {%Note that we omit the notation of 
    For convenience, we drop $\paramf$ 
    in the sequel, 
    denoting
    %for clarity of the text, meaning that
    $\lscore{\X}{\p}=\lscore{\X}{\p;\paramf}$. }
    Define
\begin{align}\label{eq:direct_r_define_v2}
    r&\triangleq2\expectation{\nabla_{\p}\log\probt{\vectorsym{x}|\p}{\X|\p}^T\lscore{\X}{\p}}{\randomvec{X},\p}\nonumber\\
    &=2\sum_i\expectation{\squareb{\nabla_{\p}\log\probt{\vectorsym{x}|\p}{\X|\p}}_i\squareb{\lscore{\X}{\p}}_i}{\randomvec{X},\p}.
\end{align}
%We define
$w_i\triangleq\expectation{\squareb{\nabla_{\p}\log\probt{\vectorsym{x}|\p}{\X|\p}}_i\squareb{\lscore{\X}{\p}}_i}{\randomvec{X},\p}$, then:
\begin{align}\label{eq:w_before_egt}
    &w_i\\
    &=\int_{\x\in\Upsilon}\int_{\p\in\Theta} \squareb{\lscore{\X}{\p}}_i\divc{\log\probt{\vectorsym{x}|\p}{\X|\p}}{\squareb{\p}_i} \probt{\x,\p}{\X,\p}d\x d\p\nonumber\\
    &=\int_{\x\in\Upsilon}\int_{\p\in\Theta} \squareb{\lscore{\X}{\p}}_i\divc{\probt{\vectorsym{x}|\p}{\X|\p}}{\squareb{\p}_i} \probt{\p}{\p}d\x d\p.\nonumber
\end{align}
     {%Assuming the 
    Thanks to the boundary 
     %assumption 
     condition \eqref{eq:boundary_condtions_gen_direct} %holds along side 
     and Assumptions~\ref{assum:diff_prob} and ~\ref{assum:diff_net} we can %Then using 
     apply} Lemma ~\ref{lemma:int_by_parts} by setting $g\brackets{\x,\p}=\divc{\probt{\vectorsym{x}|\p}{\X|\p}}{\squareb{\p}_i}$ and ${h}\brackets{\x,\p}=\squareb{\lscore{\X}{\p}}_i\probt{\p}{\p}$. %We have that: 
     This yields
\begin{align}\label{eq:finaly_step_direct}
    &w_i\nonumber\\
    &=-\int_{\x\in\Upsilon}\int_{\p\in\Theta}\probt{\vectorsym{x}|\p}{\X|\p} \divc{\squareb{\lscore{\X}{\p}}_i\probt{\p}{\p}}{\squareb{\p}_i}d\x d\p\nonumber\\
        &=-\expectation{\divc{\squareb{\lscore{\X}{\p}}_i}{\squareb{\p}_i}}{\X,\p}
        \nonumber\\
        &-\expectation{\squareb{\lscore{\X}{\p}}_i\squareb{\nabla_{\p}\log\probt{\p}{\p}}_i}{\X,\p}.
\end{align}
Finally,  combining \eqref{eq:step_one_proof_alt}, \eqref{eq:direct_r_define_v2} and \eqref{eq:finaly_step_direct} we obtain %the results in
\eqref{eq:score_match_param_gen_direct}.
\end{proof}
% \todo[inline,color=green]{You have shown that  $r=\mathbb{E}_{X,\p}\trace{\matsym{J}_{u}}$, but how do you get the third term in \eqref{eq:score_match_param_gen_direct}?\\
% HVH: It derivative of multiplication of functions for some reason the text has comment out.}