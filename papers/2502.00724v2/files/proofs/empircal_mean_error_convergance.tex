\subsection{%Proofs of 
%Sampling 
Empirical-Mean Error Convergence Condition (Remark~\ref{remark:mean_error_convergance})}\label{apx:remark_convergance_proof}


\begin{prop}
    \label{prop:nonneg_z}
    Let $\randomvec{Z}_n\sim\probt{\z}{\Z}$ be a sequence of $N$ nonegative {i.i.d} random variables. Assume that the 3rd moment  {of $\probt{\z}{\Z}$} is finite, i.e.,
    ${\expectation{\randomvec{Z}^3}{}}\leq \infty$.  Then
     {
     \begin{equation*}
       \lim\limits_{N \rightarrow \infty } 
       \tfrac{1}{N} \max\limits_{1\leq n\leq N} \randomvec{Z}_n
       %\rightarrow 
       = 0 \quad \text{a.s.}
       %as} \quad N\rightarrow \infty.
    \end{equation*}
    }
\end{prop}
\begin{proof}
By 
the Markov inequality, for any nonegative r.v.
$Z$,
%, let $Z \geq 0$ be a nonegative r.v. then for any $q>0$
\begin{equation} \label{eq:Markov3}
\probP(Z > t) = \probP(Z^3 > t^3) \leq \frac{\expectation{Z^3}{}}{t^3}.
\end{equation}
Now, {using first} the union bound {and then \eqref{eq:Markov3} we have}
%the probability of the event $\frac{\max\limits_{1\leq n\leq N} \randomvec{Z}_n}{N}>s$ is bounded by:
{
\begin{align}\label{eq:prob_max_union_N}
    \probP\brackets{
   \tfrac{1}{N} \max\limits_{1\leq n\leq N} \randomvec{Z}_n
    >N^{-\tfrac{1}{4}}}
    &\leq N \probP\brackets{\randomvec{Z}>N \cdot N^{-\tfrac{1}{4}}
    } 
     \nonumber\\
      &\leq N \frac{\expectation{Z^3}{}}{(N^{\tfrac{3}{4}})^3 }
          = \frac{\expectation{Z^3}{}}{N^{1.25}}.
\end{align}
}
% \ybdelete{Where the first inequality is using the union bound and second inequality using Markov
% inequality using $q=3$. Setting $s= N^{-1/4}$. Then
% \begin{align}
% \label{eq:Max_Nbnd}
%    \probP\brackets{
%     \frac{\max\limits_{1\leq n\leq N} \randomvec{Z}_n}{N}
%     >N^{-1/4}} \leq \frac{\expectation{Z^3}{}}{N^{1.25}}
% \end{align}
% }
Next, it is easily established that 
$$ \sum_{N=1}^\infty \frac{\expectation{Z^3}{}}{N^{1.25}} < \infty .$$
Therefore, by the Borel-Cantelli Lemma, the event on the LHS of \eqref{eq:prob_max_union_N} cannot happen infinitely often as $N\rightarrow \infty$.
\end{proof}
% \todo[color=green,inline]{YB: changed to agree with standard language  that the distribution, rather than the seqeuence of r.v.s has finite moments. Also,  to say that all moments \emph{up to} 6th order must exist, not just the 6th. } 
\begin{corollary}
    Let $\randomvec{S}_n \in \mathbb{R}^k$, {$\randomvec{S}_n \sim \probt{\boldsymbol{s}}{\randomvec{s}} $} be a sequence of $\nds$ {i.i.d} random vectors. Assume that 
    {
    $\probt{\boldsymbol{s}}{\randomvec{s}}$ has a finite 6$^{th}$ moment
   % finite moments up to 6$^{th}$ order, 
    i.e., $\expectation{\squareb{\randomvec{S}}_i^l\squareb{\randomvec{S}}_j^m}{}<\infty$ 
    for all  $i,j\in[1,k]$, and $1 \leq l+m\leq 6$. 
    }  Then,
    \begin{equation*}
      \lim\limits_{N \rightarrow \infty }   \tfrac{1}{N}\max\limits_{1\leq n\leq N} \norm{\randomvec{S}_n}_2^2
        %}{\nds}
        =0 \quad \text{a.s.}
       % \rightarrow 0 \quad \text{a.s. as} \quad \nds\rightarrow \infty.
    \end{equation*}
\end{corollary}
\begin{proof}
   Consider the sequence of nonnegative random variables $\randomvec{Z}_n=\norm{\randomvec{S}_n}^2_2=\sum_i\squareb{\randomvec{S}_n}_i^2$. 
   %be a sequence of nonnegative random variable. 
   %Using that 
   Because $\randomvec{S}_n$ has  finite moments up to 6$^{th}$ order,
   %moment then 
   it follows that $\randomvec{Z}_n$ has a %3$^{th}$ 
   finite 
   3rd moment. Applying Proposition ~\ref{prop:nonneg_z} yields desired result.  
    % \begin{align}
    %     \randomvec{Z}_n=\norm{\randomvec{S}_n}^2_2=
    % \end{align}
\end{proof}

% \todo[inline,color=cyan]{
% Let $\randomvec{S}_n\sim \mathrm{subG}\brackets{\sigma^2}$ be a sequence of i.i.d  random score variable which are the result of the postier score network.  Our goal is to show that:
% $$\frac{\max\brackets{\randomvec{S}_1^2,\hdots,\randomvec{S}_{\nds}^2}}{\nds}\xrightarrow{\nds\rightarrow\infty} 0 \quad\text{a.s}.$$

% \textbf{Proof:}
% \begin{align}\label{eq:base_c}
%     c&=\max\limits_{1\leq n \leq\nds} \randomvec{S}^2_n =\max\limits_{1\leq n \leq\nds} \randomvec{S}^2_n-\expectation{\randomvec{S}^2_n}{\randomvec{S}_n}+\expectation{\randomvec{S}^2_n}{\randomvec{S}_n}\\
%      &\leq \expectation{\randomvec{S}^2_1}{\randomvec{S}_1} +\max\limits_{1\leq n\leq\nds} \abs{\randomvec{S}^2_b-\expectation{\randomvec{S}^2_n}{\randomvec{S}_n}}
% \end{align}
% Note that  $\expectation{\randomvec{S}^2_1}{\randomvec{S}_1}=\max\limits_{1\leq n\leq\nds}\expectation{\randomvec{S}^2_n}{\randomvec{S}_n}$ since the sequence is of i.i.d random variables. 

% Next, let $\randomvec{Z}_n=\randomvec{S}^2_b-\expectation{\randomvec{S}^2_n}{\randomvec{S}_n}$ be the center random variable, we derive that:
% \begin{align}\label{eq:prob_max_union}
%     &\probP\brackets{\max\limits_{1\leq n\leq\nds}\abs{\randomvec{Z}_n}>t}=\probP\brackets{ \bigcup\limits_{1\leq n\leq\nds}\abs{\randomvec{Z}_n}>t}\\
%     &\leq \sum_{n=1}^{\nds} \probP\brackets{\abs{\randomvec{Z}_n}>t}
% \end{align}
% From Lemma 1.12 in \cite{rigollet2023high} we have if $\randomvec{S}_n\sim\mathrm{subG}\brackets{\sigma^2}$ then the random variable $\randomvec{Z}_n=\randomvec{S}_n^2-\expectation{\randomvec{S}_n^2}{\randomvec{S}_n}$ is sub-exponential, namely $\randomvec{Z}_n\sim\mathrm{subE}\brackets{16\sigma^2}$.
% Using that \eqref{eq:prob_max_union}, with that $\randomvec{Z}_n\sim\mathrm{subE}\brackets{16\sigma^2}$ is sub-exponential and $\randomvec{Z}_n$ is a sequence of i.i.d random variables we have: 
% \begin{equation}\label{eq:prob_max}
%     \probP\brackets{\max\limits_{1\leq n\leq\nds}\abs{\randomvec{Z}_n}>t}\leq \nds \exp{\brackets{-\frac{1}{2}\min\brackets{\frac{t^2}{16\sigma^2},t}}}
% \end{equation}
% From \eqref{eq:prob_max} we have that given $u>0$ with probability as least $1-\exp{\brackets{-u}}$:
% \begin{equation}
%     \max\limits_{1\leq n\leq\nds}\abs{\randomvec{Z}_n}\leq 2u +2\log\nds
% \end{equation}
% By setting $u=\log\nds$ and combing with \eqref{eq:base_c} we have that:
% \begin{equation}
%     \label{eq:CB_convergence}\lim\limits_{\nds\rightarrow\infty}\frac{c}{\nds}\leq\lim\limits_{\nds\rightarrow\infty}\frac{\expectation{\randomvec{S}^2_1}{\randomvec{S}_1}}{\nds} +\frac{4\log\nds}{\nds}=0
% \end{equation}
% \\
% \textbf{YB:} Good, I think that your argument can be made into a rigorous proof with some extra detail, to consider subG vector score, and to argue that \eqref{eq:CB_convergence} hold a.s. But consider the somewhat modified (and simpler) argument below, which shows convergence for a much weaker requirement than subG. 
% }
% \todo[inline,color=green]{
% \textbf{YB:}
% Then, by the Markov inequality, 

%  Let $Z \geq 0$ be a nonegative r.v. then for any $q>0$
% \begin{equation}
% \probP(Z > t) = \probP(Z^q > t^q) \leq \frac{\expectation{Z^q}{}}{t^q}
% \end{equation}
% Consider now a sequence of nonnegative iid r.v.s
% $Z_n, n= 1, \hdots, N$, $Z_n \sim p_Z$, where $Z\sim p_Z$ has finite $3$rd moment, i.e, $\expectation{Z^3}{} < \infty $. Then by the union bound
% \begin{align}
%     \probP\brackets{
%     \frac{\max\limits_{1\leq n\leq N} \randomvec{Z}_n}{N}
%     >s}
%     &\leq N \probP\brackets{{\randomvec{Z}}>Ns}\\
%     &\leq N \frac{\expectation{Z^3}{}}{(Ns)^3} = \frac{\expectation{Z^3}{}}{N^{2}s^3}
% \end{align}
% Let $s= N^{-1/4}$,  Then
% \begin{align}
%    \probP\brackets{
%     \frac{\max\limits_{1\leq n\leq N} \randomvec{Z}_n}{N}
%     >N^{-1/4}} \leq \frac{\expectation{Z^3}{}}{N^{1.25}}
% \end{align}
% It is easily established that 
% $$ \sum_{N=1}^\infty \frac{\expectation{Z^3}{}}{N^{1.25}} < \infty .$$
% Therefore, by the Borel-Cantelli Lemma, the event on the LHS of \eqref{eq:Max_Nbnd} cannot happen infinitely often as $N\rightarrow \infty$. Hence
% $$ \frac{\max\limits_{1\leq n\leq N} \randomvec{Z}_n}{N} \rightarrow 0 \quad \text{a.s. as} N\rightarrow \infty$$\\
% (Note: the argument can be made a bit more precise using the $\limsup$ notation, as in standard proofs of almost sure convergence using the Borel-Cantelli Lemma.)\\ 
% Now applying the above result to $Z_n = S_n^2$, shows that existence of up to the $6$th moment of $S_n$ suffices for a.s convergence. This is a much weaker requirement than $S_n \sim SubG$, because the latter has  finite  moments of {all} orders $q\geq1$. \\
% What do you think ?\\
% \textbf{HVH: } I like the new result, please see the rewriting in the text and generalization to vector case. 


% }
