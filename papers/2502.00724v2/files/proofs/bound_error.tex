

\subsection{Proofs of 
%Theoretical Analysis of 
the \name{} {Approximation Error}}
\newcommand{\vDelta}{\vectorsym{\Delta}}
\newcommand{\deltasvec}[0]{\vectorsym{\Delta}_{s}}
\newcommand{\shat}[0]{\hat{\vectorsym{s}}}
\newcommand{\svec}[0]{\vectorsym{s}}
{We will need the following technical result.}
\begin{lemma}[{From vector Difference to Covariance Difference}]\label{lemma:cov_vec_error}
    Let {$\Xr,\Yr \in \mathbb{R}^n$ be random vectors on a common probability space, and define \begin{equation*}
        \mathcal{J}=\expectation{\norm{\Xr-\Yr}_2^2}{}, \quad \quad  \matsym{F}=\expectation{\Yr\Yr^T}{}.
    \end{equation*}
    % \begin{equation*}
    %     \matsym{F}=\expectation{\Yr\Yr^T}{}.
    % \end{equation*}
    Then:}
    {
\begin{align}\label{eq:vector_error}
\frac{\norm{\expectation{\Xr\Xr^T -\Yr\Yr^T}{}}_2}{\norm{\matsym{F}}_2} \leq \frac{\mathcal{J}}{\norm{\matsym{F}}_2}+2\sqrt{\frac{\mathcal{J}}{\norm{\matsym{F}}_2}}.
\end{align}
Furthermore, if $\frac{\mathcal{J}}{\expectation{\norm{\Yr}_2^2}{}} < {0.16}$, then the RHS of \eqref{eq:vector_error} is upper bounded by
\begin{equation} \label{eq:vector_error-simplified}
%3
2.4\sqrt{\frac{\mathcal{J}}{\norm{\matsym{F}}_2}} =
%3
2.4\sqrt{\intdim\brackets{\F}\frac{\mathcal{J}}{\expectation{\norm{\Yr}_2^2}{}}}.
\end{equation}
    }    
\end{lemma}
\begin{proof}
    {Let $\vDelta \triangleq \Xr-\Yr$. Then
    \begin{align}\label{eq:bound_proof_eq_delta_cov}
 \matsym{P}&=\norm{\expectation{\Xr\Xr^T -\Yr\Yr^T}{}}_2 \nonumber\\
 &=
 \norm{\expectation{\vDelta\vDelta^T +\vDelta\Yr^T + \Yr \vDelta^T}{}}_2
    \end{align}
    }
    %\ybreplace{
    Considering
    %}{Using the convexity of the norm and applying the Jensen inequality to} 
    the first term in \eqref{eq:bound_proof_eq_delta_cov} yields
    {\begin{equation}\label{eq:part_one}
        \norm{\expectation{\vDelta\vDelta^T}{}}_2 \leq 
        \trace{\expectation{{\vDelta\vDelta^T}}{}} =  \expectation{\norm{\vDelta}_2^2}{}=\mathcal{J}
        \end{equation}
    }
    Next, {considering}  the second term in \eqref{eq:bound_proof_eq_delta_cov}, yields:
    \begin{align}\label{eq:part_two}
        &\norm{\expectation{\vDelta\Yr^T+\Yr\vDelta^T}{}}_2 
        {=} 2 \norm{\expectation{\vDelta\Yr^T}{}}_2\nonumber\\
        &\leq 2\sqrt{\norm{\expectation{\vDelta\vDelta^T}{}}_2\norm{\expectation{\Yr\Yr^T}{\z}}_2}\nonumber\\
        &\leq 2\sqrt{\expectation{\norm{\vDelta}_2^2}{}\norm{\expectation{\Yr\Yr^T}{\z}}_2}
    \end{align}
    where the first equality holds by the identity $\norm{\matsym{A} + \matsym{A}^T}_2 = 2\norm{\matsym{A}}_2$, the inequality on the second line follows by Lemma XIII.1 from \cite{habi2023learning}, and the third line by \eqref{eq:part_one}.
    Combining \eqref{eq:part_one}, \eqref{eq:part_two} with \eqref{eq:bound_proof_eq_delta_cov}, {applying the triangle inequality,} and dividing by $\norm{\matsym{F}}_2$ {establishes \eqref{eq:vector_error}}:
    \begin{equation}\label{eq:covar_error_full}
        \frac{\matsym{P}}{\norm{\F}_2}\leq\frac{\mathcal{J}}{\norm{\matsym{F}}_2}+2\sqrt{\frac{\mathcal{J}}{\norm{\matsym{F}}_2}}.
    \end{equation}
   {Now, it is easily verified that $x + 2 \sqrt{x} \leq 2.4 \sqrt{x} $ holds for any $0 \leq x \leq 0.16$. 
   % \todo[inline,color=green]{YB:The calculations leading to finding a tight simplified bound are usually tedious, and not shown in papers. What is shown is a demonstration of the correctness of the simplified form, which in this case is the one liner that I included above.\\
   % HVH: I wrote the the new ones in the simplifed form. }
This leads immediately to the upper bound on the LHS of \eqref{eq:vector_error-simplified}.  
   } 
    Finally, replacing 
    \begin{align}
        \frac{\mathcal{J}}{\norm{\matsym{F}}_2}&=\frac{\trace{\matsym{\F}}\mathcal{J}}{\trace{\matsym{\F}}\norm{\matsym{F}}_2}=\mathrm{intdim}\brackets{\F}\frac{\mathcal{J}}{\trace{\matsym{\F}}}\nonumber\\
                                              &=\mathrm{intdim}\brackets{\F}\frac{\mathcal{J}}{\expectation{\norm{\Yr}_2^2}{}}
    \end{align}
    {yields the alternative form of the upper bound in \eqref{eq:vector_error-simplified}.}
\end{proof}
{With} Lemma~\ref{lemma:cov_vec_error} {in hand,} we {proceed to prove the various bounds on the {Approximation error}}.
\subsubsection{Posterior {FIM } {Approximation Error} {Thm.~\ref{thm:lrn:direct}}}\label{proof:post:lrn}
% \todo[inline,color=green]{YB: Writing style: Having a descriptive subtitle for a proof of a Thm or lemma is fine, but the exact reference to the result that is proven should be included in the subtitle, so that a reader looking for a proof of say Thm.~\ref{thm:lrn:direct} can find it easily, without having to search through the entire text of the appendix. I made this change in a few places already, but there remain a few more instances where this needs to be done.}
\begin{proof}
    We set $\Xr=\postscores{\p}{\xset}$ and $\Yr=\nabla_{\p}\log\probt{\p|\xset}{\p|\xset}$, then we apply Lemma~\ref{lemma:cov_vec_error}, 
    {establishing \eqref{eq:error_learn_post_approch}}. 
\end{proof}

\subsubsection{Measurement-Prior {FIM } {Approximation error} {Thm.~\ref{thm:lrn:lik_prior}}}\label{proof:lik_prior:lrn}
% \hvhreplace{Starting with two Lemma's that analyze the individual learning error of each term: }
\begin{proof}
{
%First, 
We apply Lemma~\ref{lemma:cov_vec_error} twice.
%two times: once for the measurement Fisher Information Matrix (FIM) and once for the prior FIM. First, In the context of 
First, for} the measurement FIM  
we set $\Xr=\lscores{\x}{\p}$ and $\Yr=\nabla_{\p}\log\probt{\X|\p}{\x|\p}$
{in} Lemma~\ref{lemma:cov_vec_error}, which results in
\begin{equation}\label{eq:mfim_error}
    \frac{\norm{\tlmfim-\fm}_2}{\norm{\fm}_2}\leq \mfle\triangleq {2.4}\sqrt{\mathrm{intdim}\brackets{\fm}\cdot \frac{\lossfs}{\trace{\fm}}}.
\end{equation}
{Second, for} the 
prior FIM, we set $\Xr=\priorscores{\p}$ and $\Yr=\nabla_{\p}\log\probt{\p}{\p}$
{in} Lemma~\ref{lemma:cov_vec_error}, yielding
\begin{equation}\label{eq:pfim_error}
    \frac{\norm{\tlpfim-\fp}_2}{\norm{\fp}_2}  \leq\eta_{p}^{(l)}\triangleq {2.4}\sqrt{\mathrm{intdim}\brackets{\fp}\cdot \frac{\lossps}{\trace{\fp}}}.
\end{equation}
% \hvhdelete{\begin{lemma}[Prior FIM Error]\label{lem:prior_error_lern}
%     Assume that assumptions ~\ref{assum:bcrb_reg} and ~\ref{ass:score_reg_prior} hold, then:
%     \begin{align}
%         \frac{\norm{\widetilde{\mathrm{LPFIM}}-\fp}_2}{\norm{\fp}_2}  \leq\eta_{p}^{(l)}\triangleq 3\sqrt{\mathrm{intdim}\brackets{\fp}\cdot \frac{\lossps}{\trace{\fp}}}
%     \end{align}
% \end{lemma}
% \begin{lemma}[Measurement FIM Error]\label{lem:lik_error_lern}
%     Assume that assumptions ~\ref{assum:bcrb_reg} and ~\ref{ass:score_reg} hold, then:
%     \begin{align}
%         \frac{\norm{\tlmfim-\fm}_2}{\norm{\fm}_2} \leq \mfle,
%     \end{align}
%     where $\mfle\triangleq 3\sqrt{\mathrm{intdim}\brackets{\fm}_2\cdot \frac{\lossfs}{\trace{\fm}}}$ is the measurement FIM learning error bound.
% \end{lemma}
% Detailed proofs of Lemma ~\ref{lem:prior_error_lern} and ~\ref{lem:lik_error_lern} are presented in Sections ~\ref{proof:prior:lrn} and ~\ref{proof:lik:lrn}, respectively.}
Finally, we use the BFIM decomposition $\fb=\niideval\cdot\fm+\fp$ with the triangular inequality. Then, we employ equations \eqref{eq:mfim_error} and \eqref{eq:pfim_error}, to obtain Theorem~\ref{thm:lrn:lik_prior}.
\end{proof}


% \hvhdelete{
% By employing equations \eqref{eq:mfim_error} and \eqref{eq:pfim_error}, we establish a bound on the learning error of the measurement prior approach as detailed in Theorem~\ref{thm:lrn:lik_prior}. The proofs of Theorem~\ref{thm:lrn:lik_prior} are constructed on the basis of the BFIM decomposition $\fb=\fm+\fp$ in conjunction with the triangular inequality. We subsequently utilized results from \eqref{eq:mfim_error} and \eqref{eq:pfim_error} result in Theorem~\ref{thm:lrn:lik_prior}. }
% \hvhdelete{
% \subsubsection{Proof Prior Error}\label{proof:prior:lrn}
% \begin{proof}
%     We set $\z=\p$, $\hat{f}\brackets{\z}=\priorscore{\p}$ and $f\brackets{\z}=\nabla_{\p}\log\probt{\p}{\p}$, then we apply Lemma~\ref{lemma:cov_vec_error}, which results in Lemma~\ref{lem:prior_error_lern}.
% \end{proof}

% \subsubsection{Proof measurement Error}\label{proof:lik:lrn}
% \begin{proof}
%     We set $\z=[\x,\p]$, $\hat{f}\brackets{\z}=\lscore{\x}{\p}$ and $f\brackets{\z}=\nabla_{\p}\log\probt{\X|\p}{\x|\p}$, then we apply Lemma~\ref{lemma:cov_vec_error}, which results in Lemma~\ref{lem:lik_error_lern}.
% \end{proof}}


\subsection{Proofs of 
%Sampling 
Empirical-Mean Error}\label{apx:proof_sample_error}
{We will use the following result, which is an easy corollary of} "Matrix Bernstein: Hermitian Case with Intrinsic Dimension" \cite[Chapter 7]{tropp2015introduction}.
% \todo[inline,color=green]{YB: Made this a Proposition - so that it can have a corollary.}
\begin{prop}
    [Matrix Bernstein {for Spectral Norm}: Symmetric Case with Intrinsic Dimension]\label{prop:bernstein}
    Let $\matsym{X}_k$ be a finite set of random symmetric matrices, $\matsym{Y}=\sum_{k}\matsym{X}_k$ be a sum over the finite set,  $$\matsym{V}\succeq \expectation{\matsym{Y}^2}{}=\sum_k\expectation{\matsym{X}_k^2}{},$$ 
    be a semi-definite upper bound, and $\nu=\norm{\matsym{V}}_2$ its {spectral} norm.     Assume that:
    $$\expectation{\matsym{X}_k}{\matsym{X}_k}=0\quad\mathrm{and} \quad \norm{\matsym{X}_k}_2\leq L \quad\forall k,$$
     then for $t\geq \sqrt{\nu}+\frac{L}{3}$,
   \begin{equation}\label{eq:lemma_bernstein_prob}
        \mathbb{P}\brackets{\norm{\matsym{Y}}\geq t}\leq 8 d_i
        \cdot\exp{\brackets{-\frac{0.5t^2}{\nu+Lt/3}}},
    \end{equation}
    where $d_i=\mathrm{intdim}\brackets{\matsym{V}}\triangleq\frac{\trace{\matsym{V}}}{\norm{\matsym{V}}_2}$ is the Intrinsic Dimension of matrix $\matsym{V}$. In addition the Expectation Bound is given by:
    \begin{align}\label{eq:lemma_bound_expection}
        \expectation{\norm{\matsym{Y}}_2}{\ds}&\leq \sqrt{\nu}\brackets{4+\sqrt{2\log\brackets{1+2d_i}}} \nonumber\\ 
        &+\frac{2L}{3}\brackets{4+\log\brackets{
        1+2d_i}}
    \end{align}
    
\end{prop}
\begin{proof}
    {The relation} 
    $\norm{\matsym{Y}}_2=\max\brackets{\lambda_{max}\brackets{\matsym{Y}},\lambda_{max}\brackets{-\matsym{Y}}}$ %and 
    together with the union bound {yield}
    \begin{align*}
        \mathbb{P}\brackets{\norm{\matsym{Y}}_2\geq t}\leq\mathbb{P}\brackets{\lambda_{max}\brackets{\matsym{Y}}\geq t}+\mathbb{P}\brackets{\lambda_{max}\brackets{-\matsym{Y}}\geq t}.
    \end{align*}
    {Applying} Matrix Bernstein: Hermitian Case with Intrinsic Dimension from \cite[Chapter 7]{tropp2015introduction} twice {establishes}  Proposition~\ref{prop:bernstein}. 
\end{proof}
{Given some $u>0$,} we wish to bound  $\norm{\matsym{Y}}<t$ with probability at least $1-\exp\brackets{-u}$. {This is provided by the following corollary of Proposition~\ref{prop:bernstein}}.
\begin{corollary} \label{cor:bernstein}
Let
\begin{flalign} %\label{eq:t0_bound}
t_u \triangleq &{\frac{bL}{3}}\brackets{1+\sqrt{\frac{{18}\nu}{bL^2}+1}}, 
\text{ where }  b= {u+\log\brackets{8d_i}}. \nonumber
\end{flalign}
Then, subject to the assumptions of Proposition~\ref{prop:bernstein},
    \begin{equation} \label{eq:t_bound}
    \mathbb{P}\brackets{\norm{\matsym{Y}}\leq t_u} \geq 1-\exp\brackets{-u} %\quad \forall t\geq t_0 
    .
    \end{equation}
\end{corollary}
\begin{proof}
    Consider the event complementary to that in \eqref{eq:t_bound},
%
% \ybreplace{, for any $u>0$, meaning {$\mathbb{P}\brackets{\norm{\matsym{Y}}< t}\geq1-\exp\brackets{-u}$,} and  using the following relation 
% \begin{align}\label{eq:prob_exp_releation}
%     1-\exp\brackets{-u}\leq\mathbb{P}\brackets{\norm{\matsym{Y}}< t}=1-\mathbb{P}\brackets{\norm{\matsym{Y}}\geq t},
% \end{align}
% we have:}{or for the complementary event}
\begin{align}\label{eq:apply_bound_one}
   {\mathbb{P}\brackets{\norm{\matsym{Y}}\geq  t} \leq \exp\brackets{-u}} .
\end{align}
To ensue that \eqref{eq:apply_bound_one} holds, we 
{
%use \eqref{eq:lemma_bernstein_prob} from Lemma~\ref{lemma:bernstein} and 
require 
%the following:
}
% \todoin{How do you explain the following inequality? You need to explain why $\exp\brackets{-u}\geq 8 d_i
%         \cdot\exp{\brackets{-\frac{0.5t^2}{\nu+Lt/3}}}$ rather than $\exp\brackets{-u}\leq 8 d_i
%         \cdot\exp{\brackets{-\frac{0.5t^2}{\nu+Lt/3}}}$, which is the whole point of the discussion in the red box. }
\begin{align}\label{eq:apply_bound}
        {8 d_i
        \cdot\exp{\brackets{-\frac{0.5t^2}{\nu+Lt/3}}} \leq \exp\brackets{-u},}
\end{align}
which, {by \eqref{eq:lemma_bernstein_prob} from Proposition~\ref{prop:bernstein},} implies ~\eqref{eq:apply_bound_one}.
% $$\exp\brackets{-u}\leq 8 d_i
%         \cdot\exp{\brackets{-\frac{0.5t^2}{\nu+Lt/3}}}$$
%  
        % \todo[inline,color=red]{YB: 
        % The inequality that you wrote above can be satisfied e.g., for $u=100$, and $t=0$, which makes no sense.\\
        % Shouldn't you have the reverse inequality here? That is: 
        % $$\exp\brackets{-u}\geq 8 d_i
        % \cdot\exp{\brackets{-\frac{0.5t^2}{\nu+Lt/3}}}$$\\
        % This is also related to my next comment about reversing the inequality in \eqref{eq:t_bound}. \\
        % HVH: I think I am missing something, I rewrote the way to the inequality and I don't see an error. In \eqref{eq:prob_exp_releation} we have an exact relation and in \eqref{eq:apply_bound} I simply apply the bound. \\
        % YB: the error is not in the algebra, rather in the logic. You have an inequality pointing the wrong way.  In fact, the way you have rewritten the \eqref{eq:prob_exp_releation} and \eqref{eq:apply_bound} the error is more difficult to spot than in the previous version!\\
%         Here is another take on an explanation.\\
%         We want to guarantee that 
%         \begin{equation} \label{eq:exp_u_Guarantee}
%         \mathbb{P}\brackets{\norm{\matsym{Y}}\geq  t} \leq \exp(-u) %\quad \forall w \leq u
%         \end{equation}
%      (Note the direction of the inequality) Why? Because this is what we promise in the theorems. (To emphasize this, I edited (change in bold) to  "with probability \textbf{of at least} $1-\exp(-u)$" to the statement of the theorems.) \\
%      Now, by  Lemma~\ref{lemma:bernstein}, we have that  
%      \begin{align}\label{eq:Berenstein_bound}
%     \mathbb{P}\brackets{\norm{\matsym{Y}}\geq  t}\leq 8 d_i
%         \cdot\exp{\brackets{-\frac{0.5t^2}{\nu+Lt/3}}}.
% \end{align}
%      How should we combine \eqref{eq:exp_u_Guarantee} with \eqref{eq:Berenstein_bound}?
% I claim that the correct combination is the following:
% \begin{align}\label{eq:Berenstein_Exp_bound}
%     \mathbb{P}\brackets{\norm{\matsym{Y}}\geq  t}\leq 8 d_i
%         \cdot\exp{\brackets{-\frac{0.5t^2}{\nu+Lt/3}}} \leq \exp(-u)
% \end{align}
% Why? Because then \eqref{eq:Berenstein_bound} will imply \eqref{eq:exp_u_Guarantee}.
% Hence, we need to solve the inequality 
% \begin{align}
%     \exp\brackets{-u} \geq  8 d_i
%         \cdot\exp{\brackets{-\frac{0.5t^2} {\nu+Lt/3}}} 
% \end{align}
% which has the opposite direction of the inequality to that in \eqref{eq:apply_bound}.
% Do you agree?\\
% HVH: Now I understand... 
% Please, see the change in the text.\\
% YB: I don't think your change in the text explains  why the inequality points one way rather than the other way. See my edits and comment.
% }
{
Solving \eqref{eq:apply_bound} for $t\geq 0$ yields $t\geq t_u$.
%

}
Now, any $t\geq t_u$
also satisfies $t\geq \sqrt{\nu}+\frac{L}{3}$ as required by Proposition~\ref{prop:bernstein}, and therefore yields a valid bound in \eqref{eq:apply_bound_one}.
%Note that this ensue that $t\geq \sqrt{\nu}+\frac{L}{3}$.
%      
{Selecting $t=t_u$ yields 
 the tightest bound in \eqref{eq:t_bound}, and establishes the corollary. 
%the smallest %possible 
%value of $t$ { in \eqref{eq:t_bound}
}
\end{proof}

% \todo[inline,color=green]{YB: did you mean $t\geq ...$ rather than $t\leq ...$ in \eqref{eq:t_bound}? \\
% HVH: to solve the inequility, we can select any $t$ between the range of  \eqref{eq:t_bound_lower} and \eqref{eq:t_bound} and since we also required that $t\geq \sqrt{\nu}+\frac{L}{3}$, I have select $t=\frac{1}{3}\brackets{u+\log\brackets{8d_i}}L+\sqrt{\nu}$ which address. Maybe there is exist a tighter value. Do this make sense? \\
% YB: As pointed out in my previous comment, I think that you have an error. }
%
%



% \todo[color=green,inline]{Once you reverse the inequality in \eqref{eq:t_bound}, the condition $t\geq \sqrt{\nu}+\frac{L}{3}$ will be automatically satisfied}
\subsubsection{Posterior Approach Empirical Mean Error {Thm. ~\ref{thm:sampling_post}}}
\begin{proof}
Denote $\vectorsym{s}_k=\postscores{\p_k}{\x_k}$ {and} $\matsym{X}_k=\frac{1}{\nds}\brackets{\vectorsym{s}_k\vectorsym{s}_k^T-\lbfimb}$. 
 First we validate the assumptions of Proposition~\ref{prop:bernstein}.
    \begin{align*}
        \expectation{\matsym{X}_k}{\ds}=\frac{1}{\nds}\brackets{\expectation{\vectorsym{s}_k\vectorsym{s}_k^T}{\ds}-\lbfimb}=0,
    \end{align*}
    and 
    \begin{align}\label{eq:l_value_post}
        \norm{\matsym{X}_k}&\leq\frac{1}{\nds}\brackets{\norm{\vectorsym{s}_k\vectorsym{s}_k^T}+\norm{\lbfimb}}\nonumber\\
        &\leq \frac{1}{\nds}\brackets{\cb+\norm{\lbfimb}_2} \triangleq L.
    \end{align}
    Next, we calculate $\matsym{V}$:
    \begin{align*}
        \expectation{\matsym{X}_k^2}{\ds}&=\frac{1}{\nds^2}\expectation{{\vectorsym{s}_k\vectorsym{s}_k^T}\vectorsym{s}_k\vectorsym{s}_k^T -\lbfimb^2}{\ds}\\
        &\preceq \frac{1}{\nds^2}\brackets{\cb\expectation{\vectorsym{s}_k\vectorsym{s}_k^T}{\ds} -\lbfimb^2}\preceq  \frac{\cb\lbfimb}{\nds^2}.
    \end{align*}
    {The first inequality is obtained by $\vectorsym{s}_k\vectorsym{s}_k^T\preceq \norm{\vectorsym{s}_k\vectorsym{s}_k^T}_2\matsym{I}$}.
    Then
    \begin{equation}\label{eq:v_value_post}
    \matsym{V}=\frac{\cb \lbfimb}{\nds} \quad\mathrm{and} \quad \nu=\frac{\cb}{\nds}\norm{\lbfimb}_2.
    \end{equation}
           {Now, using} \eqref{eq:l_value_post}, \eqref{eq:v_value_post} in Corollary ~\ref{cor:bernstein} we obtain:
            \begin{align}
              &\frac{t_u}{\norm{\lbfimb}_2}=\frac{\nbe}{4\nds}\brackets{1+\sqrt{1+24\frac{\nds}{\nbe}\frac{\cb}{\cb+\norm{\fb}_2}}}\nonumber\\
              &=\frac{\nbe}{4\nds}\brackets{1+\sqrt{1+24\frac{\nds}{\nbe}\frac{1}{1+\frac{\norm{\fb}_2}{\cb}}}}\nonumber\\
              &\leq \frac{\nbe}{4\nds}\brackets{1+\sqrt{1+24\frac{\nds}{\nbe}}} \label{eq:tBe_bound}
            \end{align}
            where $\nbe {\triangleq}\frac{4}{3}\brackets{u+\log\brackets{8\dbb}}\brackets{\frac{\cb}{\norm{\lbfimb}_2}+1}$. Using the 
            %following 
            relation {$\trace{\lbfimb}=\dbb\norm{\lbfimb}_2$} 
provides the alternative form
            $\nbe=\frac{4}{3}\brackets{u+\log\brackets{8\dbb}}\brackets{\dbb\frac{\cb}{\trace{\lbfimb}}+1}$.  {Finally, using the easiliy verified inequality $1+\sqrt{1+x} \leq \sqrt{3x/2}$ {for $x\geq 24$,} in \eqref{eq:tBe_bound} yields }
            \begin{equation}\label{eq:bound_t_post_v2}
                \frac{t_u}{\norm{\lbfimb}_2}\leq { 1.5\sqrt{\frac{\nbe}{\nds}}},
            \end{equation}
            for any {$\nds \geq \nbe$}. {Applying the  upper bound of \eqref{eq:bound_t_post_v2}
 in Corollary~\ref{cor:bernstein} yields \eqref{eq:re_error_post_mean} of Theorem~\ref{thm:sampling_post}.}
 % \todo[inline]{HVH: 
 % $$\nbet=\brackets{\log\brackets{1+2d_B} +{0.52}} \cdot\brackets{\dbb\frac{\cb}{\trace{\lbfimb}}+1}$$
 
 % }
             
             Next, {defining $\phi_B{\triangleq}\frac{\cb}{\norm{\lbfimb}_2}+1$,
             we use \eqref{eq:l_value_post} and  \eqref{eq:v_value_post} in 
        \eqref{eq:lemma_bound_expection} %we 
        to obtain a bound on the normalized expected value 
        %which obtained using \eqref{eq:lemma_bound_expection} and given by:
             }
    \begin{align*}
        &\expectation{\frac{\norm{\lbfimbs-\lbfimb}_2}{\norm{\lbfimb}_2}}{\ds}\leq 
         \sqrt{\frac{\cb}{\norm{\lbfimb}_2\nds}}\brackets{4+\sqrt{2
         {\psi_B}
         }}\\
        &+\frac{2}{3\nds}\phi_B\brackets{4+
       {\psi_B}
        } \text{ where } \psi_B= \log\brackets{1+2\dbb} 
    \end{align*}
      Using  $\frac{\cb}{\norm{\lbfimb}_2}\leq \phi_B$ and defining $\alpha\triangleq \frac{2\brackets{4+\psi_B}}{3\brackets{4+\sqrt{2\psi_B}}}$ yields %results in
    {\begin{align*}
        \expectation{\frac{\norm{\lbfimbs-\lbfimb}_2}{\norm{\lbfimb}_2}}{\ds}\leq \brackets{4+\sqrt{2\psi_B}}\brackets{\alpha\frac{\phi_B}{\nds}+\sqrt{\frac{\phi_B}{\nds}} }.
    \end{align*}
    }
     {Now, it is easily verified that $\alpha x+\sqrt{x}\leq 1.5\sqrt{x}$ holds for any $0\leq x \leq \frac{1}{4\alpha^2}$. It  follows that:}
    {\begin{flalign}
      \label{eq:expection_bound_post}
        \textstyle{\expectation{\frac{\norm{\lbfimbs-\lbfimb}_2}{\norm{\lbfimb}_2}}{\ds}} \leq \textstyle{\brackets{6+1.5\sqrt{2\log\brackets{1+2\dbb}}}}\sqrt{\frac{\phi_B}{\nds}}
    \end{flalign}
    for any $\nds \geq 4\alpha^2 \phi_B$.  Since $\psi_B=\log\brackets{1+2\dbb}\geq 1$ it is easy to verify that $4\alpha^2\leq \psi_B +{0.52}$. Hence the error bound \eqref{eq:expection_bound_post} 
    holds for $\nds \geq \brackets{\psi_B +{0.52}}\phi_B$.  Using the relation $\trace{\lbfimb}=\dbb\norm{\lbfimb}_2$ we define $\nbet\triangleq\brackets{\log\brackets{1+2d_B} +{0.52}} \cdot\brackets{\dbb\frac{\cb}{\trace{\lbfimb}}+1}$, producing the form used in Thm. ~\ref{thm:sampling_post}.
    }
\end{proof}
\subsubsection{%Proof of 
Measurement-Prior Empirical Mean Error {Thm.~\ref{thm:sampling_mp}}}
\begin{proof}
 % For notation simplistic we 
  Denote {$\vectorsym{s}_{k,i}=\lscores{\tilde{\x}_{k,i}}{\p_k}$} as the Fisher score  vector  and $\vectorsym{p}_k=\priorscores{\p_k}$ as the prior score of the $k^{th}$ sample in $\ds$.
    Now, let $$\matsym{X}_k=\frac{1}{\nds}\brackets{\frac{\niideval}{\niiddata}\cdot\sum_{i=1}^{\niiddata} \vectorsym{s}_{k,i}\vectorsym{s}_{k,i}^T+ \vectorsym{p}_k\vectorsym{p}_k^T-\lbfimlp}=\frac{1}{\nds}\matsym{R}_k.$$
    First we validate the assumptions of {Proposition~\ref{prop:bernstein}}:
    \begin{align*}
        &\expectation{\matsym{X}_k}{\ds}\nonumber\\
        &=\frac{1}{\nds}\brackets{\niideval\expectation{\vectorsym{s}_{k,i}\vectorsym{s}_{k,i}^T}{\ds}+\expectation{\vectorsym{p}_k\vectorsym{p}_k^T}{\ds}-\lbfimlp}=0,
    \end{align*}
    {for every $i^{th}$ i.i.d sample} and 
    \begin{align}\label{eq:l_value_lik_prior}
        \norm{\matsym{X}_k}&\leq\frac{1}{\nds}\brackets{\frac{\niideval}{\niiddata}\cdot{\norm{\sum_{i=1}^{\niiddata}\vectorsym{s}_{k,i}\vectorsym{s}_{k,i}^T}}+\norm{\vectorsym{p}_k\vectorsym{p}_k^T}+\norm{\lbfimlp}}\nonumber\\
        &\leq \frac{1}{\nds}\brackets{\niideval \cm+\cp+\norm{\lbfimlp}_2}=L.
    \end{align}
   Next, we calculate $\matsym{V}$:
    \begin{align*}
        &\matsym{R}^2_k={\frac{\niideval}{\niiddata}\sum_{i=1}^{\niiddata}\vectorsym{s}_{k,i}\vectorsym{s}_{k,i}^T}{\frac{\niideval}{\niiddata}\sum_{i=1}^{\niiddata}\vectorsym{s}_{k,i}\vectorsym{s}_{k,i}^T}+\vectorsym{p}_k\vectorsym{p}_k^T\vectorsym{p}_k\vectorsym{p}_k^T\nonumber\\
        &+\lbfimlp^2+{\frac{\niideval}{\niiddata}\sum_{i=1}^{\niiddata}\vectorsym{s}_{k,i}\vectorsym{s}_{k,i}^T}\vectorsym{p}_k\vectorsym{p}_k^T+\vectorsym{p}_k\vectorsym{p}_k^T{\frac{\niideval}{\niiddata}\sum_{i=1}^{\niiddata}\vectorsym{s}_{k,i}\vectorsym{s}_{k,i}^T}\\
        &-\brackets{{\frac{\niideval}{\niiddata}\sum_{i=1}^{\niiddata}\vectorsym{s}_{k,i}\vectorsym{s}_{k,i}^T}+\vectorsym{p}_k\vectorsym{p}_k^T}\lbfimlp\\
            &-\lbfimlp \brackets{{\frac{\niideval}{\niiddata}\sum_{i=1}^{\niiddata}\vectorsym{s}_{k,i}\vectorsym{s}_{k,i}^T}+\vectorsym{p}_k\vectorsym{p}_k^T}
            % &\preceq C_F^2\vectorsym{s}_k\vectorsym{s}_k^T+\cp^2\vectorsym{p}_k\vectorsym{p}_k^T+\lbfimlp^2-\brackets{\vectorsym{s}_k\vectorsym{s}_k^T+\vectorsym{p}_k\vectorsym{p}_k^T}\lbfimlp\\
            % &-\lbfimlp \brackets{\vectorsym{s}_k\vectorsym{s}_k^T+\vectorsym{p}_k\vectorsym{p}_k^T}+\vectorsym{s}_k\vectorsym{s}_k^T \cp^2+\vectorsym{p}_k\vectorsym{p}_k^T C_F^2
    \end{align*}
    Taking the expectation and using that $\expectation{\frac{\niideval}{\niiddata}\sum_{i=1}^{\niiddata}\vectorsym{s}_{k,i}\vectorsym{s}_{k,i}^T+\vectorsym{p}_k\vectorsym{p}_k^T}{\ds}=\lbfimlp$ yields
    \begin{align*}
        &\expectation{\matsym{R}_k^2}{\ds}= \frac{\niideval^2}{\niiddata}\cm\expectation{\sum_{i=1}^{\niiddata}\vectorsym{s}_{k,i}\vectorsym{s}_{k,i}^T} {\ds}+\cp\expectation{\vectorsym{p}_k\vectorsym{p}_k^T}{\ds}\\
        &+\lbfimlp^2+ \frac{\niideval}{\niiddata}\cp\expectation{\sum_{i=1}^{\niiddata}\vectorsym{s}_{k,i}\vectorsym{s}_{k,i}^T} {\ds}+\cm\niideval\expectation{\vectorsym{p}_k\vectorsym{p}_k^T}{\ds}\\
        &-2\lbfimlp^2.
    \end{align*}
    Reordering and using $\expectation{\frac{\niideval}{\niiddata}\sum_{i=1}^{\niiddata}\vectorsym{s}_{k,i}\vectorsym{s}_{k,i}^T+\vectorsym{p}_k\vectorsym{p}_k^T}{\ds}=\lbfimlp$ we have:
    
    % \begin{align*}
    %     &\expectation{\matsym{R}_k^2}{\ds}=\brackets{C_F^2+\cp^2}\expectation{{\vectorsym{s}_k\vectorsym{s}_k^T+\vectorsym{p}_k\vectorsym{p}_k^T}}{\ds}\\
    %     &+\lbfimlp^2 -\expectation{\vectorsym{s}_k\vectorsym{s}_k^T+\vectorsym{p}_k\vectorsym{p}_k^T}{\ds}\lbfimlp\\
    %         &-\lbfimlp \expectation{\vectorsym{s}_k\vectorsym{s}_k^T+\vectorsym{p}_k\vectorsym{p}_k^T}{\ds}
    % \end{align*}
    % By using that $\expectation{\vectorsym{s}_k\vectorsym{s}_k^T+\vectorsym{p}_k\vectorsym{p}_k^T}{\ds}=\lbfimlp$ we have:
    \begin{align*}
        \expectation{\matsym{R}_k^2}{\ds}&=\brackets{\niideval \cm+\cp}\lbfimlp-\lbfimlp^2 \\
        &\preceq\brackets{\niideval \cm+\cp}\lbfimlp.
    \end{align*}
    Then,
    \begin{equation}\label{eq:v_value_mp}
\matsym{V}=\frac{\expectation{\matsym{R}_k^2}{\ds}}{\nds}\preceq\frac{\brackets{\frac{\niideval}{\niiddata}\cm+\cp}\lbfimlp}{\nds} 
    \end{equation}
    Finally, {using}  equations \eqref{eq:l_value_lik_prior}, \eqref{eq:v_value_mp} {and} { Corollary~\ref{cor:bernstein} 
    {and following the same steps as  in the proof of Theorem~\ref{thm:sampling_post}},
    we establish Theorem~\ref{thm:sampling_mp}.}

\end{proof}
\subsection{%Proofs of 
%Sampling 
  Score Norms Squared - Relation (Proposition~\ref{prop:expected_score_ineq})}\label{apx:remark_c_relation_proof}


\begin{proof}
   Define:
    \begin{align*}
        \matsym{A}&\triangleq \norm{\nabla_{\p}\log\probt{\p|\xsetr}{\p|\xset}}_2^2\\
        \matsym{B}&\triangleq\norm{\frac{1}{\niiddata}\sum_{i=1}^{\niiddata}\nabla_{\p}\log\probt{\x_i|\p}{\X|\pr}\nabla_{\p}\log\probt{\x_i|\p}{\X|\pr}^T}_2
    \end{align*}
Then, 
\begin{align}
    &\expectation{\matsym{A}}{\X,\pr}=\expectation{\norm{\sum_{i=1}^{\niiddata}\nabla_{\p}\log\probt{\x_i|\p}{\X|\pr}}_2^2}{\X,\pr}\nonumber\\
    &+\expectation{\norm{\nabla_{\p}\log\probt{\p}{\pr}}_2^2}{\X,\pr}\nonumber\\
    &+\expectation{2\nabla_{\p}\log\probt{\p}{\pr}^T\sum_{i=1}^{\niiddata}\nabla_{\p}\log\probt{\x_i|\p}{\X|\pr}}{\X,\pr} \label{eq:expect_post_score_squared}
\end{align}
Now, by  assumption ~\ref{assume:common_support} we have $\expectation{\sum_{i=1}^{\niiddata}\nabla_{\p}\log\probt{\x_i|\p}{\X|\pr}}{\xset|\pr}$ $=0$, {thus %which results in the 
eliminating the third term on the RHS of \eqref{eq:expect_post_score_squared}.} Turning to the first term on the RHS of \eqref{eq:expect_post_score_squared} and expanding yields
\begin{align*}
%\ybdelete
{
% &\expectation{\norm{\sum_{i=1}^{\niiddata}\nabla_{\p}\log\probt{\x_i|\p}{\X|\pr}}_2^2}{\X,\pr}\\
%     &=
    } 
    &\sum_{i=1}^{\niiddata}\expectation{\norm{\nabla_{\p}\log\probt{\x_i|\p}{\X|\pr}}_2^2}{\X,\pr}\nonumber\\
    &+\sum_{\substack{i=1\\
    j=1\\ 
    j\neq i} }\expectation{\nabla_{\p}\log\probt{\x_i|\p}{\X|\pr}^T\nabla_{\p}\log\probt{\x_{j}|\p}{\X|\pr}}{\X,\pr}\nonumber
\end{align*}
Using assumption ~\ref{assume:common_support} the cross-terms zero out resulting in:
\begin{align*}
    &\expectation{\norm{\sum_{i=1}^{\niiddata}\nabla_{\p}\log\probt{\x_i|\p}{\X|\pr}}_2^2}{\X,\pr}
    =\nonumber\\
    &\sum_{i=1}^{\niiddata}\expectation{\norm{\nabla_{\p}\log\probt{\x_i|\p}{\X|\pr}\nabla_{\p}\log\probt{\x_i|\p}{\X|\pr}^T}_2}{\X,\pr}\nonumber\\
    &\geq \niiddata \expectation{\matsym{B}}{\X,\pr}
\end{align*}
Where the inequality  follows by Jensen's inequality. {Combining with \eqref{eq:expect_post_score_squared} (without the third term) yields the result.}
\end{proof}


\subsection{%Proofs of 
%Sampling 
Empirical-Mean Error Convergence Condition (Remark~\ref{remark:mean_error_convergance})}\label{apx:remark_convergance_proof}


\begin{prop}
    \label{prop:nonneg_z}
    Let $\randomvec{Z}_n\sim\probt{\z}{\Z}$ be a sequence of $N$ nonegative {i.i.d} random variables. Assume that the 3rd moment  {of $\probt{\z}{\Z}$} is finite, i.e.,
    ${\expectation{\randomvec{Z}^3}{}}\leq \infty$.  Then
     {
     \begin{equation*}
       \lim\limits_{N \rightarrow \infty } 
       \tfrac{1}{N} \max\limits_{1\leq n\leq N} \randomvec{Z}_n
       %\rightarrow 
       = 0 \quad \text{a.s.}
       %as} \quad N\rightarrow \infty.
    \end{equation*}
    }
\end{prop}
\begin{proof}
By 
the Markov inequality, for any nonegative r.v.
$Z$,
%, let $Z \geq 0$ be a nonegative r.v. then for any $q>0$
\begin{equation} \label{eq:Markov3}
\probP(Z > t) = \probP(Z^3 > t^3) \leq \frac{\expectation{Z^3}{}}{t^3}.
\end{equation}
Now, {using first} the union bound {and then \eqref{eq:Markov3} we have}
%the probability of the event $\frac{\max\limits_{1\leq n\leq N} \randomvec{Z}_n}{N}>s$ is bounded by:
{
\begin{align}\label{eq:prob_max_union_N}
    \probP\brackets{
   \tfrac{1}{N} \max\limits_{1\leq n\leq N} \randomvec{Z}_n
    >N^{-\tfrac{1}{4}}}
    &\leq N \probP\brackets{\randomvec{Z}>N \cdot N^{-\tfrac{1}{4}}
    } 
     \nonumber\\
      &\leq N \frac{\expectation{Z^3}{}}{(N^{\tfrac{3}{4}})^3 }
          = \frac{\expectation{Z^3}{}}{N^{1.25}}.
\end{align}
}
% \ybdelete{Where the first inequality is using the union bound and second inequality using Markov
% inequality using $q=3$. Setting $s= N^{-1/4}$. Then
% \begin{align}
% \label{eq:Max_Nbnd}
%    \probP\brackets{
%     \frac{\max\limits_{1\leq n\leq N} \randomvec{Z}_n}{N}
%     >N^{-1/4}} \leq \frac{\expectation{Z^3}{}}{N^{1.25}}
% \end{align}
% }
Next, it is easily established that 
$$ \sum_{N=1}^\infty \frac{\expectation{Z^3}{}}{N^{1.25}} < \infty .$$
Therefore, by the Borel-Cantelli Lemma, the event on the LHS of \eqref{eq:prob_max_union_N} cannot happen infinitely often as $N\rightarrow \infty$.
\end{proof}
% \todo[color=green,inline]{YB: changed to agree with standard language  that the distribution, rather than the seqeuence of r.v.s has finite moments. Also,  to say that all moments \emph{up to} 6th order must exist, not just the 6th. } 
\begin{corollary}
    Let $\randomvec{S}_n \in \mathbb{R}^k$, {$\randomvec{S}_n \sim \probt{\boldsymbol{s}}{\randomvec{s}} $} be a sequence of $\nds$ {i.i.d} random vectors. Assume that 
    {
    $\probt{\boldsymbol{s}}{\randomvec{s}}$ has a finite 6$^{th}$ moment
   % finite moments up to 6$^{th}$ order, 
    i.e., $\expectation{\squareb{\randomvec{S}}_i^l\squareb{\randomvec{S}}_j^m}{}<\infty$ 
    for all  $i,j\in[1,k]$, and $1 \leq l+m\leq 6$. 
    }  Then,
    \begin{equation*}
      \lim\limits_{N \rightarrow \infty }   \tfrac{1}{N}\max\limits_{1\leq n\leq N} \norm{\randomvec{S}_n}_2^2
        %}{\nds}
        =0 \quad \text{a.s.}
       % \rightarrow 0 \quad \text{a.s. as} \quad \nds\rightarrow \infty.
    \end{equation*}
\end{corollary}
\begin{proof}
   Consider the sequence of nonnegative random variables $\randomvec{Z}_n=\norm{\randomvec{S}_n}^2_2=\sum_i\squareb{\randomvec{S}_n}_i^2$. 
   %be a sequence of nonnegative random variable. 
   %Using that 
   Because $\randomvec{S}_n$ has  finite moments up to 6$^{th}$ order,
   %moment then 
   it follows that $\randomvec{Z}_n$ has a %3$^{th}$ 
   finite 
   3rd moment. Applying Proposition ~\ref{prop:nonneg_z} yields desired result.  
    % \begin{align}
    %     \randomvec{Z}_n=\norm{\randomvec{S}_n}^2_2=
    % \end{align}
\end{proof}

% \todo[inline,color=cyan]{
% Let $\randomvec{S}_n\sim \mathrm{subG}\brackets{\sigma^2}$ be a sequence of i.i.d  random score variable which are the result of the postier score network.  Our goal is to show that:
% $$\frac{\max\brackets{\randomvec{S}_1^2,\hdots,\randomvec{S}_{\nds}^2}}{\nds}\xrightarrow{\nds\rightarrow\infty} 0 \quad\text{a.s}.$$

% \textbf{Proof:}
% \begin{align}\label{eq:base_c}
%     c&=\max\limits_{1\leq n \leq\nds} \randomvec{S}^2_n =\max\limits_{1\leq n \leq\nds} \randomvec{S}^2_n-\expectation{\randomvec{S}^2_n}{\randomvec{S}_n}+\expectation{\randomvec{S}^2_n}{\randomvec{S}_n}\\
%      &\leq \expectation{\randomvec{S}^2_1}{\randomvec{S}_1} +\max\limits_{1\leq n\leq\nds} \abs{\randomvec{S}^2_b-\expectation{\randomvec{S}^2_n}{\randomvec{S}_n}}
% \end{align}
% Note that  $\expectation{\randomvec{S}^2_1}{\randomvec{S}_1}=\max\limits_{1\leq n\leq\nds}\expectation{\randomvec{S}^2_n}{\randomvec{S}_n}$ since the sequence is of i.i.d random variables. 

% Next, let $\randomvec{Z}_n=\randomvec{S}^2_b-\expectation{\randomvec{S}^2_n}{\randomvec{S}_n}$ be the center random variable, we derive that:
% \begin{align}\label{eq:prob_max_union}
%     &\probP\brackets{\max\limits_{1\leq n\leq\nds}\abs{\randomvec{Z}_n}>t}=\probP\brackets{ \bigcup\limits_{1\leq n\leq\nds}\abs{\randomvec{Z}_n}>t}\\
%     &\leq \sum_{n=1}^{\nds} \probP\brackets{\abs{\randomvec{Z}_n}>t}
% \end{align}
% From Lemma 1.12 in \cite{rigollet2023high} we have if $\randomvec{S}_n\sim\mathrm{subG}\brackets{\sigma^2}$ then the random variable $\randomvec{Z}_n=\randomvec{S}_n^2-\expectation{\randomvec{S}_n^2}{\randomvec{S}_n}$ is sub-exponential, namely $\randomvec{Z}_n\sim\mathrm{subE}\brackets{16\sigma^2}$.
% Using that \eqref{eq:prob_max_union}, with that $\randomvec{Z}_n\sim\mathrm{subE}\brackets{16\sigma^2}$ is sub-exponential and $\randomvec{Z}_n$ is a sequence of i.i.d random variables we have: 
% \begin{equation}\label{eq:prob_max}
%     \probP\brackets{\max\limits_{1\leq n\leq\nds}\abs{\randomvec{Z}_n}>t}\leq \nds \exp{\brackets{-\frac{1}{2}\min\brackets{\frac{t^2}{16\sigma^2},t}}}
% \end{equation}
% From \eqref{eq:prob_max} we have that given $u>0$ with probability as least $1-\exp{\brackets{-u}}$:
% \begin{equation}
%     \max\limits_{1\leq n\leq\nds}\abs{\randomvec{Z}_n}\leq 2u +2\log\nds
% \end{equation}
% By setting $u=\log\nds$ and combing with \eqref{eq:base_c} we have that:
% \begin{equation}
%     \label{eq:CB_convergence}\lim\limits_{\nds\rightarrow\infty}\frac{c}{\nds}\leq\lim\limits_{\nds\rightarrow\infty}\frac{\expectation{\randomvec{S}^2_1}{\randomvec{S}_1}}{\nds} +\frac{4\log\nds}{\nds}=0
% \end{equation}
% \\
% \textbf{YB:} Good, I think that your argument can be made into a rigorous proof with some extra detail, to consider subG vector score, and to argue that \eqref{eq:CB_convergence} hold a.s. But consider the somewhat modified (and simpler) argument below, which shows convergence for a much weaker requirement than subG. 
% }
% \todo[inline,color=green]{
% \textbf{YB:}
% Then, by the Markov inequality, 

%  Let $Z \geq 0$ be a nonegative r.v. then for any $q>0$
% \begin{equation}
% \probP(Z > t) = \probP(Z^q > t^q) \leq \frac{\expectation{Z^q}{}}{t^q}
% \end{equation}
% Consider now a sequence of nonnegative iid r.v.s
% $Z_n, n= 1, \hdots, N$, $Z_n \sim p_Z$, where $Z\sim p_Z$ has finite $3$rd moment, i.e, $\expectation{Z^3}{} < \infty $. Then by the union bound
% \begin{align}
%     \probP\brackets{
%     \frac{\max\limits_{1\leq n\leq N} \randomvec{Z}_n}{N}
%     >s}
%     &\leq N \probP\brackets{{\randomvec{Z}}>Ns}\\
%     &\leq N \frac{\expectation{Z^3}{}}{(Ns)^3} = \frac{\expectation{Z^3}{}}{N^{2}s^3}
% \end{align}
% Let $s= N^{-1/4}$,  Then
% \begin{align}
%    \probP\brackets{
%     \frac{\max\limits_{1\leq n\leq N} \randomvec{Z}_n}{N}
%     >N^{-1/4}} \leq \frac{\expectation{Z^3}{}}{N^{1.25}}
% \end{align}
% It is easily established that 
% $$ \sum_{N=1}^\infty \frac{\expectation{Z^3}{}}{N^{1.25}} < \infty .$$
% Therefore, by the Borel-Cantelli Lemma, the event on the LHS of \eqref{eq:Max_Nbnd} cannot happen infinitely often as $N\rightarrow \infty$. Hence
% $$ \frac{\max\limits_{1\leq n\leq N} \randomvec{Z}_n}{N} \rightarrow 0 \quad \text{a.s. as} N\rightarrow \infty$$\\
% (Note: the argument can be made a bit more precise using the $\limsup$ notation, as in standard proofs of almost sure convergence using the Borel-Cantelli Lemma.)\\ 
% Now applying the above result to $Z_n = S_n^2$, shows that existence of up to the $6$th moment of $S_n$ suffices for a.s convergence. This is a much weaker requirement than $S_n \sim SubG$, because the latter has  finite  moments of {all} orders $q\geq1$. \\
% What do you think ?\\
% \textbf{HVH: } I like the new result, please see the rewriting in the text and generalization to vector case. 


% }

\subsection{
LBCRB Relative Error Bound}\label{apx:proof_inv_re}
\subsubsection{%Proof 
%Inversion Conditions 
Relative Error Posterior Approach {Corr.~\ref{corr:bound_inv_post}}}\label{sec:proof_post_inv}
\begin{proof}
    If $\lbfimbs \succ 0$ it is  invertible and
    \begin{align}\label{eq:mat_inv_post}
        \mathrm{RE}_{B}&=\frac{\norm{\lbcrbbs-\bcrb}_2}{\norm{\bcrb}_2}=\frac{\norm{\lbfimbs^{-1}\brackets{\fb-\lbfimbs}\fb^{-1}}_2}{\norm{\bcrb}_2}\nonumber\\
        &\leq\norm{\lbcrbbs}_2\norm{\fb-\lbfimbs}_2\nonumber\\
        &\leq\norm{\lbcrbbs}_2\norm{\lbfimb-\lbfimbs}_2+\norm{\lbcrbbs}_2\norm{\fb-\lbfimb}_2 .
    \end{align}
    By combining \eqref{eq:error_learn_post_approch} and \eqref{eq:re_error_post_mean}  with \eqref{eq:mat_inv_post} we have:
    \begin{align}\label{eq:mat_inv_post_step2}
        \mathrm{RE}_{B}\leq\norm{\lbcrbbs}_2\brackets{\norm{\lbfimb}_2\bfse+\norm{\fb}\bfle}.
    \end{align}    
    Multiplying and dividing\eqref{eq:mat_inv_post_step2}  by $\norm{\lbfimbs}_2$  results in \eqref{eq:inv_bound_post}. Next, we provide the {%invertibility 
    condition %of 
    for $\lbfimbs\succ 0$.} {From \eqref{eq:error_learn_post_approch}, \eqref{eq:re_error_post_mean} and using the triangular  inequality we have
    \begin{align}\label{eq:fimb_bound_joint}
        \norm{\lbfimbs-\fb}_2&\leq\norm{\lbfimbs-\lbfimb}_2+\norm{\lbfimb-\fb}_2\nonumber\\
                              &\leq \norm{\lbfimb}_2\bfse+\norm{\fb}\bfle 
    \end{align}
    
    }
    
    Using Assumption~\ref{assume:non_singular}  we have that $\fb\succ 0$, and by definition we have that $\lbfimb$ {is symmetric}. %and using 
    Now by \eqref{eq:fimb_bound_joint} we have with probability $1-\exp\brackets{-u}$, $\lbfimbs \succ0$ %is invertiable 
    if
        \begin{equation*}
       \norm{\lbfimb}_2\bfse+\norm{\fb}\bfle <\eigmin{\fb},
    \end{equation*}
    % \todoing{How does this follow? You are missing a step of bounding the perturbation  $\norm{\lbfimbs - \fb}_2$. Also, not clear where you are using $\lbfimb\succeq 0$.
    % The matrix perturbation result is simply the following:
    % Suppose $A \succ 0$, and $B$ is symmetric, i.e.,  $B=B^T$. Then $B \succ 0$ if $\norm{A-B}_2 < \lambda_{\min}(A)$. Furthermore, if $\norm{A-B}_2 \geq \lambda_{\min}(A)$, then $\exists B$ s.t. $0 \succeq B$.\\
    % HVH: I was think it is simple :), please see updated text. }
    {where the inequality follows from a standard} matrix perturbation result, cf. \cite[Theorem 2.2]{stewart1977perturbation}.
    Finally, dividing both sides by $\norm{\fb}_2$ %we have that 
    yields \eqref{eq:inv_cond_post}.
\end{proof}
\subsubsection{%Proof 
%Inversion Conditions 
Relative Error Measurement-Prior Approach {Corr.~\ref{corr:bound_inv_split}}}\label{sec:proof_lik_prior_inv}
\begin{proof}
If $\lbfimlps \succ 0$ then it is invertible, %we have that:
and
\begin{align}\label{eq:mat_inv_mp}
    \mathrm{RE}_{MP}&\triangleq\frac{\norm{\lbcrblps-\bcrb}_2}{\norm{\bcrb}_2}\nonumber\\
                    &=\frac{\norm{\lbcrblps\brackets{\fb-\lbfimlps}\bcrb}_2}{\norm{\bcrb}_2}\nonumber\\
                    &\leq \norm{\lbcrblps}_2\norm{\fb-\lbfimlps}_2\nonumber\\
                    &\leq  \norm{\lbcrblps}_2\norm{\lbfimlp-\lbfimlps}_2\nonumber\\
                    &+\norm{\lbcrblps}_2\norm{\lbfimlp-\fb}_2
\end{align}
    By combining \eqref{eq:error_learn_lik_prior_approch} and \eqref{eq:re_error_mp_mean}  with \eqref{eq:mat_inv_mp} we have:
    \begin{align}\label{eq:mat_inv_mp_step2}
        \mathrm{RE}_{MP}&= \norm{\lbcrblps}_2 \brackets{\norm{\fm}_2\mfle+\norm{\fp}_2\pfle}\nonumber\\
                        &+\norm{\lbcrblps}_2\norm{\lbfimlp}_2 \mpfse
    \end{align}
    Multiplying and dividing \eqref{eq:mat_inv_mp_step2}  by $\norm{\lbfimlps}_2$  results in \eqref{eq:inv_bound_mp}.
    Next, we provide the 
    %invertibility 
    condition for
    %of 
    {$\lbfimlps \succ 0$}. {From \eqref{eq:error_learn_lik_prior_approch}, \eqref{eq:re_error_mp_mean} and using the triangular  inequality we have
    \begin{align}\label{eq:fim_mp_bound_joint}
        \norm{\lbfimlps-\fb}_2&\leq\norm{\lbfimlps-\lbfimlp}_2+\norm{\lbfimlp-\fb}_2\nonumber\\
                              &\leq \norm{\lbfimb}_2\bfse+\norm{\fb}\bfle 
    \end{align}}

    
    By definition we have $\lbcrblps\succeq 0$ and $\fb\succ 0$, using \eqref{eq:error_learn_lik_prior_approch} and \eqref{eq:re_error_mp_mean} we have with probability $1-\exp\brackets{-u}$, $\lbfimbs \succ 0$ %is invertiable 
    if:
    \begin{equation*}
       \norm{\fm}_2\mfle+\norm{\fp}_2\pfle+\norm{\lbfimlp}_2 \mpfse <\eigmin{\fb},
    \end{equation*}
    %\todoin{Similar comment to that in the previous proof, in Subsection G.1. }
    {where the inequality is by the matrix perturbation result as above. }
    Dividing both sides by $\norm{\fb}_2$ 
    %we have that 
    yields\eqref{eq:inv_cond_post}.
    \end{proof}
    

    

% By definition we have $\lmfim\succeq 0$ and $\lpfim\succeq 0$,  using \eqref{eq:error_lik_prior_lrn_mean} we have that:
% \begin{align}\label{eq:proof_inv_lik_pos}
%     &\eigmin{\lbfimlps}\\
%     &\geq \eigmin{\fb} -\mfle-\pfle-\mfse-\pfse\nonumber\\
%     &\geq\eigmin{\fp} -\pfle-\pfse +\eigmin{\fm}-\mfle-\mfse.\nonumber
% \end{align}
% In \eqref{eq:proof_inv_lik_pos} we use the decomposition of the Bayesian FIM \eqref{eq:bfim_decomposition} and the lower bound on the sum of matrices. Next, we ensure that each part is related to the prior and the measurement FIM is positive, meaning:
% \begin{subequations}
% \begin{align}
%     \eigmin{\fp} -\pfle-\pfse>0\\
%     \eigmin{\fm}-\mfle-\mfse>0
% \end{align}
% \end{subequations}
% Then we follow the same steps as in the proof of the inversion conditions of the poster approach shown in \ref{sec:proof_post_inv} which results in conditions \eqref{eq:cond_inv_lik} and \eqref{eq:cond_inv_prior}.

% Now following the conditions in \eqref{eq:cond_inv_lik}, \eqref{eq:cond_inv_prior} and $\lmfim\succ 0$, $\overline{\mathrm{LPFIM}}\succ 0$, the sum of the prior and measurement FIMs is invertiable, resulting in:
%     \begin{align}\label{eq:mat_inv_lik_prior}
%         &\norm{\lbfimlps^{-1}-\fb^{-1}}_2\nonumber\\
%         &=\norm{\lbfimlps^{-1}\brackets{\fb-\lbfimlps}\fb^{-1}}_2\nonumber\\
%         &\leq\norm{\lbfimlps^{-1}}_2\norm{\fb^{-1}}_2\norm{\fb-\lbfimlps}_2
%     \end{align}
%     Finally, combining \eqref{eq:error_lik_prior_lrn_mean} with \eqref{eq:mat_inv_lik_prior} we obtain \eqref{eq:nv_bound_lik_prior}.



% \subsubsection{Analysis invertability condition}
% Now we wise that the following term will be strictly positive:
%     \begin{align}\label{eq:con2inv}
%         \eigmin{\fb}-\bfse -\lossbs-2\sqrt{\norm{\fb}_2\cdot \lossbs}>0
%     \end{align}
%     {First, divide both sides of \eqref{eq:con2inv} by $\norm{\fb}_2$, yielding:
    
%     $$\kappa\brackets{\fb^{-1}}-\frac{\bfse}{\norm{\fb}_2} -\frac{\lossbs}{\norm{\fb}_2}-2\sqrt{ \frac{\lossbs}{\norm{\fb}_2}}$$
    
%     }
%     By definition of matrix norm, we have $\norm{\fb}_2=\eigmax{\fb}$ and we set $x=\sqrt{ \frac{\lossbs}{\norm{\fb}_2}}$ and solve the following quadratic equation: 
%     % \begin{equation}
%         $-x^2-2x+\kappa\brackets{\fb^{-1}}-\frac{\bfse}{\norm{\fb}_2}$.
%     % \end{equation}
%     To ensue that a solution exist and since $x\geq 0$, we assume that $\kappa\brackets{\fb^{-1}}-\frac{\bfse}{\norm{\fb}_2}>0$, solutions are given by:
%     \begin{align}\label{eq:cor_p1_proof}
%         x&=-\frac{2\pm \sqrt{4+4\brackets{\kappa\brackets{\fb^{-1}}-\frac{\bfse}{\norm{\fb}_2}}}}{2}\nonumber\\
%         &=-1\pm \sqrt{1+\kappa\brackets{\fb^{-1}}-\frac{\bfse}{\norm{\fb}_2}}.
%     \end{align}
%     Then, since $x=\sqrt{ \frac{\lossbs}{\norm{\fb}_2}}$ is strictly positive, we have $$0\leq\sqrt{ \frac{\lossbs}{\norm{\fb}_2}}< \sqrt{1+\kappa\brackets{\fb^{-1}}-\frac{\bfse}{\norm{\fb}_2}}-1. $$ Now if  \eqref{eq:inv_loss_bound_post} inequality holds,
%     then $\eigmin{\lbfimbs}>0$ and $\lbfimbs$ is invertable matrix. 
%     In the next part, we assume that \eqref{eq:inv_loss_bound_post} inequality holds which results in:

