% \definecolor{mydarkblue}{RGB}{0,0,139} % This defines a dark blue color




% \begin{figure*}
%     \centering
% %     \begin{subfigure}[t]{0.5\textwidth}
% % \begin{tikzpicture}[
% %   block/.style={draw, rectangle, rounded corners, drop shadow, minimum width=2cm, minimum height=1cm, align=center},
% %   popblock/.style={draw, rectangle, rounded corners, line width=1.2pt, fill=mydarkblue, text=white, drop shadow={shadow xshift=0.8ex, shadow yshift=-0.8ex}, minimum width=2.2cm, minimum height=1.2cm, align=center},
% %   largeblock/.style={draw, rectangle, rounded corners, drop shadow, minimum width=4.5cm, minimum height=1.5cm, align=center}, % Style for a larger block
% %   dataset/.style={draw, cylinder, shape border rotate=90, aspect=0.25, minimum width=1.5cm, minimum height=1cm, drop shadow, align=center},
% %   largelabel/.style={font=\Large} % Style for large text
% % ]
% % % First row of blocks with colors
% % \node (block1) [popblock] {Learning\\Score};  % Enhanced appearance
% % \node (block2) [block, right=of block1, fill=green!70] {Score Model\\
% % $s_{\Omega}\brackets{\p|\X}$};  % Green color



% % \node (block3) [popblock, right=of block2] {Evaluation\\Stage};  % Enhanced appearance


% % \node (block4) [dataset, above=0.5cm of block2, fill=red!70] {Dataset\\
% % $\mathcal{D}$};  % Red color

% % % Larger Block 5 aligned with the start of Block 1
% % \node (block5) [largeblock, below=of block1,xshift=1.2cm] {Block 5};  % Aligned with Block 1

% % % Large text above Block 5
% % \node (largeText) [largelabel, above=0.0cm of block5] {Result};  % Large text node

% % % Draw the connections
% % \draw[-latex] (block1) -- (block2);
% % \draw[-latex] (block2) -- (block3);
% % % \draw[-latex] (block7) |- (block5);

% % % Straight line connections
% % \draw[-latex] (block4) -| (block1);
% % \draw[-latex] (block4) -| (block3);
% % \end{tikzpicture}
% % \caption{\name approximate flow. }
% % \end{subfigure}
% % \begin{subfigure}[t]{1.0\textwidth}
%      \begin{tikzpicture}[font=\small,every label/.append style={font=\small,align=center}]
%         \node[startstop,fill=cyan!15, text width=3.0cm,align=center] (prior_score) {\tiny Prior Score \\$\nabla_{\p}\log\probt{\p}{\pr}\approx \vectorsym{p}_{\Omega_p}\brackets{\p}$};
% 		\node[ below=0.4cm of prior_score,startstop,fill=orange!15, text width=3cm,align=center,yshift=0cm,xshift=+2.5cm] (dmle) {\tiny Likelihood Score\\ $\nabla_{\p}\log\probt{\X|\p}{\X|\pr}\approx$ $\vectorsym{h}_{\Omega_d}\brackets{\randomvec{X}|\p}$};
%         % \node [sum,above right=0.2cm of  dmle] (add) {$+$};
%         % \node[cylinder,above=0.3cm of prior_score,
% 	       %     draw = blue!15, 
% 	       %     text = black,
% 	       %     cylinder uses custom fill, 
% 	       %     cylinder body fill = blue!10, 
% 	       %     cylinder end fill = blue!40,
% 	       %     aspect = 0.2, 
% 	       %     shape border rotate = 90,font=\tiny] (net) {$\mathcal{D}_T$};
        

        
        
        
        
%         \node[left=3cm of prior_score,startstop,fill=white, text width=7cm,align=center,yshift=-0.0cm] (loss_prior) { \tiny Prior Score Matching\\ $\mathcal{L}_{p}\brackets{\Omega_p}=\frac{1}{2}\expectation{\norm{p_{\Omega_{p}}\brackets{\p}}_2^2}{\p}+\trace{\expectation{\frac{\partial p_{\Omega_{p}}\brackets{\p}}{\partial\p}}{\p}}$};

%         \node[below=0.2cm of loss_prior,startstop,fill=cyan, text width=7cm,align=center,yshift=0cm] (loss_dmle) {\tiny Likelihood Score Matching \\$\mathcal{L}_{d}\brackets{\Omega_d}=\frac{1}{2}\expectation{\norm{\vectorsym{h}_{\Omega_h}\brackets{\randomvec{X}|\p}}_2^2}{\randomvec{X},\p}+\expectation{\trace{\nabla_{\p}\brackets{\vectorsym{h}_{\Omega_h}\brackets{\randomvec{X}|\p}}}}{\randomvec{X},\p}+\expectation{\vectorsym{h}_{\Omega_h}\brackets{\randomvec{X};\p}^T\vectorsym{p}_{\Omega_{p}}\brackets{\p}}{\randomvec{X},\p} $};


%         \node[below=3cm of prior_score,startstop,fill=white, text width=4cm,align=center,xshift=-2.3cm] (eval_prior) { \tiny $\overline{\F}_{p}\approx\frac{1}{\nds} \sum_{\p\in\mathcal{D}} p_{\Omega_{p}}\brackets{\p}p_{\Omega_{p}}\brackets{\p}^T$};

        

%         \node[left=-0.6cm of eval_prior,startstop,fill=white, text width=3cm,align=center,yshift=-1cm] (eval_bfim) { \tiny $\overline{\matsym{F}}_{b}\brackets{m}=m\cdot \overline{\F}_{d} + \overline{\F}_{p}$};

%         \node[below=0.3cm of eval_bfim,startstop,fill=white, text width=6.1cm,align=center,xshift=3.5cm] (eval_fim) { \tiny $\overline{\F}_{d}\approx\frac{1}{\nds\cdot K}\sum_{\p,\x\in\ds,} \sum_{k=1}^K h_{\Omega_{h}}\brackets{\x_k|\p}h_{\Omega_{h}}\brackets{\x_k|\p}^T$};

        
%         \begin{pgfonlayer}{background}
%             \node[draw,rounded corners,fit= (prior_score) (dmle), fill=yellow!30,inner sep=10pt,label={Score Models}](fit2){};

%             \node[draw,rounded corners,fit= (loss_dmle) (loss_prior), fill=blue!30,inner sep=10pt,label={Learning Stage}](fit2){};

%             \node[draw,rounded corners,fit= (eval_bfim) (eval_fim)  (eval_prior), fill=green!30,inner sep=5pt,label=Evaluation Stage](fit_eval){};

            
%         \end{pgfonlayer}
%         \node[left=1.2cm of fit_eval,inner sep=0pt,yshift=0cm] (no)  {\includesvg[width=0.2\textwidth]{files/results/freq_est_v2/phase_error.svg}};

%         \begin{pgfonlayer}{background}
%             \node[draw,rounded corners,fit= (no) , fill=white!30,inner sep=5pt,label={Results}](fit2_res){};
%         \end{pgfonlayer}
%         % \draw[->,dashed,line width=0.25mm,color=red] (prior_score)  -| (loss_prior);
%         \draw[-,dashed,line width=0.25mm,color=red] (dmle)  -- ([yshift=+0.17cm] loss_dmle.east);
%         \draw[-,line width=0.25mm] (prior_score)  |- (loss_dmle);
%         \draw[-,dashed,line width=0.25mm,color=red] (prior_score)  -- (loss_prior);
%         % \draw[->,dots,line width=0.25mm] (prior_score)  -| (eval_prior);
%         % \draw[->,line width=0.25mm] (dmle)  -| (add);
%         % \draw[->,line width=0.25mm] (net)  |- (dmle);
%         % \draw[->,line width=0.25mm] (net)  |- (prior_score);

%         \draw[->,line width=0.25mm] (dmle.south)  |- ( eval_fim);
%         \draw[->,line width=0.25mm] (prior_score)  |- (eval_prior);

%         \draw[->,dotted,line width=0.25mm] (eval_fim)  -| (eval_bfim);
%         \draw[->,dotted,line width=0.25mm] (eval_prior)  -| (eval_bfim);
% \end{tikzpicture} 
% % \caption{\name Informed score functions. }
% % \end{subfigure}
%     \caption{\name{} likelihood prior approach overview: The learning stage, illustrated by the purple block, involves two training steps: initially, the prior score is trained, followed by the likelihood score; this process results in two models, the prior and the likelihood scores. These models are subsequently utilized in the evaluation stage, as shown in the green box.  }
%     % \label{fig:main}
% \end{figure*}

