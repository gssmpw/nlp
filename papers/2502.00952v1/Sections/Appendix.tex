\section{Additional Method Details}
\label{sec:app_details}
\subsection{Generating Controversial Viewpoints}
\label{sec:hai_design}
\subsubsection{Prompts}
Our first step is to generate a list of topics that are likely to lead to disagreement. We provide the subreddit name, description of the subreddit, and community rules retrieved using \texttt{PRAW}. For our study, we generate $20$ topics per subreddit using \texttt{GPT-4} with temperature set to 1. 
\newline
\begin{lstlisting}[style=gptprompt]
You are a member of r/${SUBREDDIT}. Your task is to generate controversial issues,
aka issues that will lead to disagreement between r/${SUBREDDIT} members.
<name>
${SUBREDDIT}
</name>
<description> 
${DESCRIPTION}
</description>
<rules>
${RULES}
</rules>
Provide ${NUMBER} issues that will be controversial in r/${SUBREDDIT}. 
Return just the issue, no justification.
\end{lstlisting}

We generate viewpoints representing different stances for each topic. Again, we provide the subreddit name with the description and community rules. We use the same set of hyperparameters from topic generation for this task. 
\newline
\begin{lstlisting}[style=gptprompt]
You are a member of r/${SUBREDDIT}. Your task is to generate viewpoints that
r/${SUBREDDIT} members would hold on a given topic.
<name>
${SUBREDDIT}
</name>
<description> 
${DESCRIPTION}
</description>
<rules>
${RULES}
</rules>
For each side of the issue, write a 50-word Reddit comment from the perspective of that side that 
follows the rules of r/${SUBREDDIT}. 
Issue: ${TOPIC}
Return only the comments as a list, no justification. Example output: ["I am pro-choice.", "I am pro-life."] 
\end{lstlisting}
 
Finally, we shorten each of the viewpoints that was generated using the following prompt. We use a temperature of $0$ for this task. 
\newline
\begin{lstlisting}[style=gptprompt]
Shorten a statement while retaining the same viewpoint. 
The statement should be phrased as an opinion.
Text: ${VIEWPOINT}
Return just the shortened statement, no justification.
\end{lstlisting}

\subsubsection{Generated Topics and Viewpoints}
In Table~\ref{tab:viewpoint_list}, we provide the list of all eleven topics and their corresponding viewpoints. The first five topics were generated using \texttt{r/politics} as the seed subreddit and the subsequent six using \texttt{r/worldnews}. We also provide examples of applying our topic generation method to communities outside the realm of politics or social issues. In Table~\ref{tab:viewpoint_other}, we list five topics each from \texttt{r/nba} and \texttt{r/gaming}.

\begin{table*}[ht!]
    \centering
    \footnotesize
    \caption{The eleven topics, and their respective viewpoints, used in the opinion expression survey}
    \begin{tabular}{p{1.5in}p{2.5in}p{2.55in}}
    \toprule
    \RaggedRight{\textbf{Topic}} &  \RaggedRight{\textbf{Viewpoint 1}} &  \RaggedRight{\textbf{Viewpoint 2}}\\
    \midrule
    Universal Healthcare & \RaggedRight{I believe in Universal Healthcare because everyone deserves access to good health, funded by the government.} & \RaggedRight{I believe market competition and individual insurance plans are superior to Universal Healthcare.}\\
    \addlinespace
    Abortion Rights  & \RaggedRight{As a pro-choice supporter, I believe women's bodily autonomy and reproductive choices are crucial for gender equality.} & \RaggedRight{As a pro-life advocate, I believe every life from conception deserves legal protection.}\\ 
    \addlinespace
   \RaggedRight{Election Reform and Voter ID Laws} & \RaggedRight{I believe in strict voter ID laws for a secure democracy and fewer fraud allegations.} &	\RaggedRight{Stringent voter ID laws marginalize minority and low-income communities. We should make voting more accessible, not harder, and aim for higher turnout, not suppression.}\\ 
   \addlinespace
    Military Spending & \RaggedRight{Increasing military spending is vital for national security, global presence, and economic growth.}	
    & \RaggedRight{We should cut military spending to fund education, healthcare, and infrastructure.}\\ 
    \addlinespace
    Affirmative Action & \RaggedRight{I believe affirmative action counters systemic biases and fosters a diverse, inclusive society.}	
    & \RaggedRight{I believe affirmative action could unintentionally cause reverse discrimination and undermine merit, potentially increasing societal division.}\\
    \addlinespace
    Impact of Brexit on the European Union & \RaggedRight{Brexit, economically, appears detrimental to the EU, potentially signaling a decline in internationalism.}	
    & \RaggedRight{Brexit could potentially boost EU cohesion as member states see the difficulties of exiting.}\\
    \addlinespace
    \RaggedRight{Israeli-Palestinian conflict and the recognition of Jerusalem as Israel's capital} & 
    \RaggedRight{Recognizing Jerusalem as Israel's capital disrupts the peace process by favoring Israel's contested claims over Palestinian rights.}
    & 
    \RaggedRight{Recognizing Jerusalem as Israel's capital acknowledges Jewish ties to the city, but doesn't negate the need for fair negotiations.}\\
    \addlinespace
    \RaggedRight{Role of NATO in maintaining global peace} & 
    \RaggedRight{NATO's collective defense and strategic alliances are crucial for global peacekeeping and international security.}	
    & 
    \RaggedRight{NATO's actions can sometimes escalate global tension by infringing on sovereignty and threatening peace.}\\
    \addlinespace
    \RaggedRight{Ethics of drone warfare in the Middle East} & 
    \RaggedRight{Drone warfare is a necessary evil for global security due to its precision, efficiency, and safety for soldiers.}
    & 
    \RaggedRight{Drone warfare inevitably causes collateral damage, violates human rights, and induces terror.}\\
    \addlinespace
    \RaggedRight{Role of social media platforms in spreading fake news} & 
    \RaggedRight{Social media's lax approach has led to a fake news epidemic, undermining informed decision-making and threatening societal stability.}	& 
    \RaggedRight{Blaming social media for fake news is misplaced; users should fact-check and stricter content control risks censorship.} \\
    \addlinespace
    \RaggedRight{Role of the United States in the Venezuelan political crisis} 
    & \RaggedRight{I believe US intervention is vital for Venezuela's democratic restoration and humanitarian aid.}	
    & \RaggedRight{I view US involvement in Venezuela as neo-imperialism; each country should independently handle its internal affairs.}\\
    \bottomrule
    \end{tabular}
    \label{tab:viewpoint_list}
\end{table*}
\begin{table*}[ht!]
    \centering
    \footnotesize
    \caption{Our viewpoint generation process can be used for non-political communities. We present a sample of topics, and their respective viewpoints, proposed for \texttt{r/NBA} and \texttt{r/gaming}.}
    \begin{tabular}{p{1.5in}p{2.5in}p{2.5in}}
    \toprule
    \RaggedRight{\textbf{Topic}} &  \RaggedRight{\textbf{Viewpoint 1}} &  \RaggedRight{\textbf{Viewpoint 2}}\\
    \midrule
    \textbf{r/NBA} \\
    \midrule
    LeBron James vs Michael Jordan & \RaggedRight{LeBron's consistent dominance and achievements arguably make him the GOAT, surpassing MJ.} & \RaggedRight{Despite LeBron's feats, MJ's NBA Finals record, killer instinct, and game-changing ability make him the GOAT in my opinion.}\\
    \addlinespace
    NBA Season Length & \RaggedRight{I believe the 82-game season should be shortened to maintain game quality and reduce player fatigue.} & \RaggedRight{The 82-game NBA season is crucial for testing team endurance, reducing fluke performances, and providing ample basketball for fans.}\\ 
    \addlinespace
   \RaggedRight{International NBA games} & \RaggedRight{I believe international NBA games disrupt the season's rhythm, fatigue players with travel, and unfairly risk their health and team performance for sport's geographical expansion.} &	\RaggedRight{I support international NBA games as they expand the audience, grow the brand, inspire youth, and boost the sport's global popularity, despite logistical challenges.}\\ 
   \addlinespace
    NBA Draft System & \RaggedRight{I strongly favor abolishing the draft system for a free market system, allowing players to choose their teams and potentially balance league competition.}	
    & \RaggedRight{The NBA draft is crucial for maintaining balance between small and large market teams and preventing the formation of super-teams.}\\ 
    \addlinespace
    Golden State Warriors 2015-2019 & \RaggedRight{The Warriors' 2015-2019 run boosted the league's competitiveness by forcing teams to adapt and improve}	
    & \RaggedRight{The Warriors' 2015-2019 dominance made the NBA predictable and potentially drove fans away.}\\
    \addlinespace  
    \hline 
    \textbf{r/gaming} \\ 
    \hline
     \addlinespace
    Console vs PC & \RaggedRight{As a console gamer, I value its simplicity, exclusives, and plug-and-play convenience for relaxation.}	
    & \RaggedRight{As a PC enthusiast, I believe PCs provide unmatched customization, performance, and game variety for in-depth gaming.}\\
     \addlinespace
    \RaggedRight{Working conditions in gaming} & 
    \RaggedRight{Crunch culture in game development is harmful and unsustainable; companies should prioritize employee wellbeing.}
    & 
    \RaggedRight{Crunch may not be ideal, but without it, could top games be made? Many vocations are stressful, not just gaming.}\\
     \addlinespace
    \RaggedRight{Large vs indie game companies} & 
    \RaggedRight{Major game companies' dominance stifles creativity; we need more space for indie developers to innovate.}	
    & 
    \RaggedRight{Big companies like Nintendo prove that dominance doesn't necessarily kill creativity, as they continually innovate and fund ambitious projects.}\\
    \addlinespace
    \RaggedRight{eSports as a real sport} & 
    \RaggedRight{eSports, despite lacking physical exertion, is a sport due to its demand for strategy, teamwork, and skill.}
    & 
    \RaggedRight{Esports should be classified as competitive gaming, not traditional sports due to the lack of physical exertion.}\\
    \addlinespace
    \RaggedRight{Objectification of female characters} & 
    \RaggedRight{I believe game developers overly sexualize female characters, neglecting character depth.}	& 
    \RaggedRight{Video games are fantasy and idealized characters reflect artistic vision, not objectification.} \\
     \addlinespace
    \RaggedRight{Copyright issues and modding} 
    & \RaggedRight{I support modders as they express themselves by reshaping games, while respecting authorship.}	
    & \RaggedRight{I support developers because unauthorized mods can threaten their control over their costly game creations.}\\
    \bottomrule
    \end{tabular}
    \label{tab:viewpoint_other}
\end{table*}


\subsection{Identifying Relevant Subreddits}
To find our list of political and social issues-oriented subreddits, we use the following process. Starting with 2,040 subreddits from \texttt{r/ListOfSubreddits}, we look at the 50 hottest submissions in each subreddit as of January 5, 2024. We select subreddits where the majority of submissions are in English -- classified using \texttt{langdetect}~\cite{nakatani2010langdetect} -- leaving us with 1,975 subreddits. Next, using \texttt{GPT-3.5-Turbo} we classify whether post titles are political (see below for prompt). We choose subreddits where more than $80\%$ of submissions are labelled as being political (N=48). From the remaining subreddits, we select our final list of 23 based on subreddit size. In our final survey, participants also had the option to list an additional subreddit they were an active member of, bringing us to 33 subreddits in total. 


\begin{lstlisting}[style=gptprompt]
Political content on Reddit is varied and can be about officials and activists, social issues, or news and current events. Looking at the following post title, would you categorize it as POLITICAL or NOT POLITICAL content? 

Answer 1 if it is POLITICAL, 0 otherwise.

Post: ${POST}
Answer:
\end{lstlisting}
\subsection{Knowledge Quiz}
\label{sec:app_kq}
To validate that participants were members of the subreddits they selected, they were given a knowledge screening quiz. In the quiz, the participants were shown three post titles, two of which belonged to top 50 hottest submissions of their selected subreddit and one that came from the top 50 hottest submissions of a different political subreddit. We manually checked that the title from the other subreddit was not relevant to the selected subreddit. The titles used in the knowledge quiz is listed in Table~\ref{tab:kq}.  
\begin{table*}[ht]
\footnotesize
    \centering
    \begin{tabular}{p{0.75in}p{1.7in}p{1.7in}p{1.7in}}
    \toprule
    \RaggedRight{\textbf{Subreddit}} & 
    \RaggedRight{\textbf{Title 1}} & 
    \RaggedRight{\textbf{Title 2}} & 
    \RaggedRight{\textbf{Title 3}}\\
    \midrule
    worldnews	& Mongolia Commits to Fighting Corruption With International Help	&  Finland votes: Stubb wins presidency & 	Already two ufo references in Super Bowl commercials lol \\
socialism & Stopping the Cop Cities Countrywide &  Historic Newton Teachers Strike Highlights Divided MA Democratic Party & Biden Docs Confirm Hunter's Pay-To-Play Was A Family Affair\\
politics & Trump Asks Supreme Court to Pause Ruling Denying Him Absolute Immunity& Right-wing judges flaunting their bias and conflicts threaten democracy& Finland votes: Stubb wins presidency\\
unitedkingdom & Sinn Féin politician secretly attended son's PSNI graduation & HMS Prince of Wales sailed today to participate in NATO's Exercise Steadfast Defender & Mongolia Commits to Fighting Corruption With International Help\\
\bottomrule
    \end{tabular}
    \caption{Examples of post titles used in the knowledge quizzes. Participants must correctly select the post titles that belong to the subreddit (i.e., those in columns ``Title 1'' and ``Title 2'').}
    \label{tab:kq}
\end{table*}

% worldnews	& Mongolia Commits to Fighting Corruption With International Help	&  Finland votes: Stubb wins presidency & 	Already two ufo references in Super Bowl commercials lol \\
% changemyview & CMV: Morality Can Only Be Objective If God Exists.	&  CMV: Deadpool 3 isn't going to save the MCU & Finland votes: Stubb wins presidency \\
% europe	& Europe must hurry to defend itself against Russia—and Donald Trump &  EU prepares sanctions on Chinese and Indian companies over Russia links & Deadpool 3 isn't going to save the MCU\\
% politics & Trump Asks Supreme Court to Pause Ruling Denying Him Absolute Immunity& Right-wing judges flaunting their bias and conflicts threaten democracy& Finland votes: Stubb wins presidency\\
% conspiracy& The Illuminati are also doing major layoffs, especially of the Black Illuminati& Already two ufo references in Super Bowl commercials lol& Housing glut leaves China with excess homes for 150m people
% \\ canada& Ontario brewery slammed with negative reviews, abusive calls after hosting Trudeau& Canadian Russian pleads guilty to exporting technology to Russia& Housing glut leaves China with excess homes for 150m people
% \\
% conservative& Biden Docs Confirm Hunter's Pay-To-Play Was A Family Affair& Team Trump has released a new ad attacking Nikki Haley& Bernie Sanders Backs UAW Call for a 4-Day Work Week\\
% unitedkingdom& Sinn Féin politician secretly attended son's PSNI graduation& HMS Prince of Wales sailed today to participate in NATO's Exercise Steadfast Defender& Mongolia Commits to Fighting Corruption With International Help
% \\
% syriancivilwar& Astana talks on Syria to commence in Kazakhstan on Jan 24& 3 armed drones shot down near U.S. military base in N. Iraq& Biden Docs Confirm Hunter's Pay-To-Play Was A Family Affair
% \\
% sandersforpresident& Sanders and AOC Unveil Resolution to Apologize for US Role in Chile Coup, Declassify Secret Docs	Bernie& Sanders Backs UAW Call for a 4-Day Work Week& Already two ufo references in Super Bowl commercials lol
% \\
% socialism& Stopping the Cop Cities Countrywide& Historic Newton Teachers Strike Highlights Divided MA Democratic Party& Biden Docs Confirm Hunter's Pay-To-Play Was A Family Affair
% \\
% mensrights& Why is it always men running after women? Why are there tips and tricks all over the internet to impress women?& 	ShoeOnHead: Are Women Obsolete? | The Rise Of Ai Girlfriends	& Deadpool 3 isn't going to save the MCU
% \\
% ukpolitics& Sinn Fein’s windy talk of a united Ireland must not wreck the new Ulster deal& Over 125,000 applicants rejected from British Army& Trump Asks Supreme Court to Pause Ruling Denying Him Absolute Immunity
% \\
% geopolitics& US strikes in Iraq and Syria reportedly killed nearly 40 people& Diplomatic tensions between Ecuador and Russia over military equipment threaten banana exports& Historic Newton Teachers Strike Highlights Divided MA Democratic Party
% \\
% combatfootage& Footage of an unknown Ukrainian unit targeting Russian infantry with FPV drones& Destruction of Russian BMP& 	Ontario brewery slammed with negative reviews, abusive calls after hosting Trudeau
% \\
% vaxxhappened& Holy hell! Florida Surgeon General goes "Antichrist" on COVID vaccines& Thanks, Antivaxxers: Measles Is Back& Ontario brewery slammed with negative reviews, abusive calls after hosting Trudeau
% \\
% law& Hawaii top court upholds gun laws, criticizes US Supreme Court& The Only Way Trump Stays on the Ballot Is if the Supreme Court Rejects the Constitution& Deadpool 3 isn't going to save the MCU
% \\
% feminism& Does “gender” mean anything at all when stripped of essentialist conceptions?& Seriously? The "pink tax" on puppy toys?!?& Mongolia Commits to Fighting Corruption With International Help
% \\
% subredditdrama& Users on r/TransRacial argue about racism	& r/geeksandgamerscommunity goes private after the ban of its head mod.& Seriously? The "pink tax" on puppy toys?!?
% \\
% economics& The Mind-Boggling Reach of Super Bowl Commercials: A Statistical Analysis& Housing glut leaves China with excess homes for 150m people& Ontario brewery slammed with negative reviews, abusive calls after hosting Trudeau
% \\
% againsthatesubreddits& r/TrueFascism post promotes National Bolshevism with open Nazi imagery (black sun, NSDAP eagle logo)	& r/TeslaModel3 comments blowing up with racism& Finland votes: Stubb wins presidency
% \\
% propagandaposters& Europe tries to tie the Russian Bear from attacking Constantinople. France, 1840-1850s& "Only army's and working nation's power protects the country." Turkish WW II Propaganda poster.& Canadian Russian pleads guilty to exporting technology to Russia
% \\
% latestagecapitalism& Robotaxi vandalized, set ablaze in San Francisco's Chinatown& "Innovation" brought to you by capitalism. Trying to profit off every last ounce of the human experience, including the painful ones...& 3 armed drones shot down near U.S. military base in N. Iraq
\subsection{Survey Details}
Participants were compensated a prorated \$15/hr for completing the survey with a total of \$1,070 spent (including pilot studies, screenings, etc.). 

We provide the survey instrument used in the study as follows:
\begin{enumerate}
    \item In what year did you create your Reddit account?
\item Typically, how often do you post or comment on Reddit versus browsing what others have submitted (lurking)? 
\item Select up to 3 subreddits that you are an active member of. 
\item Are you an active member of any other subreddits that focus on political or social issues outside of those you have already selected? 
\item Provide the name of a subreddit that you are an active member of which focuses on political or social issues that is not one of the subreddits you have already selected. 
\item How diverse are the people for each subreddit? 
\item How included and able to contribute do new and existing members feel for each subreddit? 
\item Select up to 5 topics that are relevant to the subreddit. 
\item Which viewpoint on the topic do you agree with more? 
\item Indicate your level of agreement with the viewpoint. 
\item Indicate the level of agreement you think the majority of subreddit members have for the viewpoint.
\item Disregarding your own stance on the topic, should members of the subreddit be able to share this viewpoint? 
\item Imagine that you are browsing posts, and you see a post related to the selected topic. You notice that in the comments the following viewpoint has not been raised yet. Rate the likelihood you would share this viewpoint on the subreddit under your main account. 
\item Rate the likelihood you would upvote someone else’s comment expressing this viewpoint, if the comment was present.
\end{enumerate}

\subsection{Descriptive Statistics}
Finally, we provide the descriptive statistics of the variables we use in our linear mixed-effect models in Table~\ref{tab:variables}.

\begin{table}[t]
    \centering
    \small
    \begin{tabularx}{\linewidth}{Xrrrr}
        \textbf{Feature} & \textbf{Unit} & $\mu$ & $\sigma$ & \textbf{Median} \\ \midrule
        Age (hours) & 2x & 8.738 & 1.644 & 9.067 \\
        Num. Comments & 2x & 745.569 & 3.130 & 864 \\
        Rec. Comments & 2x & 11.829 & 3.164 & 12 \\
        Prop. Undesired & 2x & 0.189 & 1.660 & 0.190 \\
        Rec. Prop. Undesired & 2x & 0.160 & 2.417 & 0.201 \\
        Score & 2x & 13.161K & 2.587 & 13.697K \\
        Rec. Votes & 2x & 357.792 & 3.021 & 396 \\
        Prop. Upvotes & 1.05x & 0.917 & 1.065 & 0.930 \\
        Num. Subscribers & 2x & 3.781M & 3.704 & 3.411M
    \end{tabularx}
    \caption{
        Descriptive statistics for the features used in the regression analyses where $\mu$ and $\sigma$ are the geometric mean and standard deviation, respectively. The regression results in the following sections are scaled using the ``Unit'' column informed by $\sigma$. `K' is for thousand,`M' is for million.
    }
    \label{tab:descriptive}
\end{table}

\section{Robustness Checks}
\label{sec:robustness}
In this section, we conduct three robustness checks on our analyses. First, we report results using an ordinal mixed effect regression model in Table~\ref{tab:model1_ordinal}. Second, we use an alternative measure (\textit{ModRatio}) for moderation. In our pre-registration, we planned on using both \textit{ModRatio} and \textit{Content Removal Rate} as measures of community moderation but ultimately removed \textit{ModRatio} as the two were correlated. Finally, we provide results for our analysis including a quadratic term for \textit{Content Removal Rate}.

\subsubsection{Ordinal mixed effect regression}
In Table~\ref{tab:model1_ordinal}, we use an ordinal mixed effect regression to predict \textit{Share Likelihood}, which is a Likert-scale with values ranging from $1$ to $7$. We find that this set of models yields qualitatively similar results to Models 1a-c reported in Sec~\ref{sec:results}. There is a statistically significant negative relationship between \textit{Incongruency} ($\beta=-0.66$, $p<0.001$) and the likelihood of sharing a viewpoint. Furthermore, there is a positive relationship between the dependent variable and \textit{Inclusion} ($\beta=0.28$, $p<0.01$) as well as a interaction effect between \text{Inclusion} and \text{Incongruency} ($\beta = 0.37$, $p < 0.05$).
\begin{table*}[]
    \footnotesize
    \centering
    \caption{An ordinal mixed effect model yields qualitatively similar results to the linear mixed effect model reported in Sec. \ref{sec:results}.}
    \begin{tabular}{lrrlrrlrrl}
    \toprule
         & \multicolumn{3}{c}{Model 3a} &  \multicolumn{3}{c}{Model 3b} &  \multicolumn{3}{c}{Model 3c}\\
    \cmidrule(lr){2-4}\cmidrule(lr){5-7}\cmidrule(lr){8-10}
     & Coef. & SE & $p$ & Coef. & SE & $p$ & Coef. & SE & $p$\\ \midrule
    \textbf{Dependent Variables (Share Likelihood)} \\ 
    Share Likelihood = 1 & $-1.83$ & $0.25$ & $0.00^{***}$& $-2.24$ & $0.29$ & $0.00^{***}$&$-2.29$ & $0.28$ & $0.00^{***}$ \\
    Share Likelihood = 2 & $-0.96$ & $0.24$ & $0.00^{***}$ & $-1.33$ & $0.28$ & $0.00^{***}$ & $-1.36$ & $0.27$ & $0.00^{***}$\\
    Share Likelihood = 3 & $-0.55$ & $0.24$ &$0.02^{*}$& $-0.91$ & $0.28$ & $0.00^{**}$ & $-0.94$ & $0.27$ & $0.00^{***}$\\
    Share Likelihood = 4 & $-0.11$ & $0.24$ &$0.63$& $-0.44$ & $0.27$ & $0.11$ & $-0.47$ & $0.27$ & $0.08$\\
    Share Likelihood = 5 & $0.84$ & $0.24$ &$0.00^{***}$& $0.58$ & $0.27$ & $0.03^{*}$ & $0.55$ & $0.26$ & $0.04^{*}$\\
    Share Likelihood = 6 & $1.98$ & $0.26$ &$0.00^{***}$& $1.77$ & $0.28$ & $0.00^{***}$ & $1.73$ & $0.28$ & $0.00^{***}$\\
    % (Intercept) & $4.27$ & $0.25$ & $0.00^{***}$ & $4.56$ & $0.27$ & $0.00^{***}$ & $4.58$ & $0.27$ & $0.00^{***}$\\ \addlinespace
    \textbf{Controls} \\
    WTSC & $-0.41$ & $0.08$ & $0.00^{***}$ &$-0.40$ &$0.08$ &$0.00^{***}$ &$-0.41$ & $0.09$ & $0.00^{***}$\\ 
    Agreement Intensity & $0.44$ & $0.08$ & $0.00^{***}$ & $0.36$ & $0.08$ & $0.00^{***}$ & $0.39$ & $0.08$ & $0.00^{***}$\\
    Male (0/1) & $-0.22$ & $0.18$ & $0.22$ & $-0.31$ & $0.18$ & $0.09$ & $-0.33$ & $0.18$ & $0.07$\\
    White (0/1) & $0.01$ & $0.17$ &$0.94$ & $0.03$ & $0.17$ & $0.85$ & $0.05$ & $0.17$ & $0.77$\\ 
    Democrat (0/1) & $-0.40$ & $0.17$ &$0.03^{*}$ & $-0.64$ & $0.18$ & $0.00^{***}$ & $-0.64$ & $0.17$ & $0.00^{***}$\\
    Account Tenure & $0.22$ & $0.08$ & $0.00^{***}$ & $0.17$ & $0.08$ & $0.03^{*}$ & $0.17$ & $0.08$ & $0.04^{*}$\\
    Posting & $0.46$ & $0.08$ & $0.00^{***}$& $0.37$ & $0.08$ & $0.00^{***}$ & $0.40$ & $0.08$ & $0.00^{***}$\\ \addlinespace
    \textbf{Independent Variables} \\
    (H1) Incongruency (0/1) & & & & $-0.73$ & $0.18$ & $0.00^{***}$ & $-0.66$ & $0.19$& $0.00^{***}$\\
    (H2a) Inclusion & & & & $0.41$ & $0.09$ & $0.00^{***}$ & $0.28$ & $0.11$ & $0.01^{**}$\\
    (H2b) Diversity & & & & $0.04$ & $0.09$ & $0.56$ & $0.04$ & $0.10$ & $0.68$\\
    (H3) Content Removal Rate & & & & $0.25$ & $0.15$ & $0.10$ & $0.27$ & $0.15$ &$0.07$\\ \addlinespace
    (H2a) Incongruency $\times$ Inclusion  & & & & & & & $0.37$ & $0.19$ & $0.05^{*}$\\
    (H2b) Incongruency $\times$ Diversity & & & & & & & $0.04$ & $0.18$ & $0.84$\\
    (H3) Incongruency $\times$ Content Removal Rate & & & & & & & $-0.22$ & $0.19$ & $0.24$\\
    [1.2ex]
    % \midrule
    % \textit{Random Effects} & Var. & SD & & Var. & SD & & Var. & SD\\ 
    % Subreddit & $0.48$ & $0.69$ & & $0.65$ & $0.81$ & & $0.57$ & $0.75$\\ 
    % [1.2ex]
    \hline \\[-1.8ex] 
    \textit{Note: $^{*}$p$<$0.05; $^{**}$p$<$0.01; $^{***}$p$<$0.001} \\ 
    \bottomrule
    \end{tabular}
    \label{tab:model1_ordinal}
\end{table*}
\subsubsection{Replacing measures for moderation} We examine changes to the model when replacing \textit{Content Removal Rate} with \textit{ModRatio} as our measure of moderation. \textit{ModRatio} is the ratio of the number of subscribers to number of moderators in a subreddit. In this case, a higher \textit{ModRatio} would mean less active moderation as there are fewer moderators per subscribers. We apply a logarithmic transformation with base 2 after adding a start-value of 1 to \textit{ModRatio}. As shown in Table~\ref{tab:modratio}, we find a significant negative relationship between \textit{ModRatio} and \textit{Share Likelihood} ($p=0.04$). However, the impact on the dependent variable is minimal. When we look at model coefficients using non-standardized variables, a $1\%$ increase in the ratio of subscribers to moderators only leads to a $0.003$ increase in the likelihood of sharing an opinion. The trends with our other independent variables are the same as when we used \textit{Content Removal Rate}.

\subsubsection{Examining quadratic terms} We also report results of our linear mixed-effects model after including a quadratic term for \textit{Content Removal Rate} in Table~\ref{tab:quadratic}. Again, we see similar trends between \textit{Inclusion} and \textit{Incongruency} on \textit{Share Likelihood}. There is no significant relationship between the quadratic term and our dependent variable. 

\begin{table*}[]
    \footnotesize
    \centering
    \caption{Including \emph{ModRatio} does not substantially alter relationships between our independent variables and the likelihood of sharing a viewpoint. The results of a linear mixed-effect model that predicts a participant's likelihood of sharing their opinion to a subreddit. We present results including \emph{ModRatio} as an independent variable and results after adding interaction effects.}
    \begin{tabular}{lrrlrrl}
    \toprule
         & \multicolumn{3}{c}{ModRatio} & \multicolumn{3}{c}{ModRatio w/ Interactions}\\
    \cmidrule(lr){2-4} \cmidrule(lr){5-7}
     & Coef. & SE & $p$ & Coef. & SE & $p$ \\ \midrule
    \textit{Fixed Effects} & & \\
    (Intercept) & $4.56$ & $0.27$ & $0.00^{***}$ & $4.67$ & $0.28$ & $0.00^{***}$\\  \addlinespace
    \textbf{Controls} \\
    WTSC & $-0.37$ & $0.08$ & $0.00^{***}$ & $-0.38$ & $0.08$ & $0.00^{***}$ \\ 
    Agreement Intensity & $0.35$ & $0.08$ & $0.00^{***}$ & $0.38$ & $0.08$ & $0.00^{***}$\\
    Male (0/1) & $-0.50$ & $0.18$ & $0.01^{**}$ & $-0.52$ & $0.18$ & $0.00^{**}$ \\
    White (0/1) & $0.03$ & $0.17$ & $0.85$ & $0.06$ & $0.17$ &$0.74$ \\ 
    Democrat (0/1) & $-0.64$ & $0.17$ &$0.00^{**}$ & $-0.62$ & $0.17$ &$0.00^{**}$ \\
    Account Tenure &  $0.20$ & $0.08$ & $0.01^{*}$ & $0.20$ & $0.08$ & $0.01^{*}$ \\
    Posting & $0.40$ & $0.08$ & $0.00^{***}$ & $0.41$ & $0.08$ & $0.00^{***}$\\ \addlinespace
    \textbf{Independent Variables} \\
    Incongruency (0/1) & $-0.73$ & $0.18$ & $0.00^{**}$ & $-0.61$ & $0.19$ & $0.00^{**}$ \\
    Inclusion & $0.38$ & $0.08$ & $0.00^{**}$ & $0.28$ & $0.10$ & $0.00^{**}$\\
    Diversity & $0.06$ & $0.08$ & $0.54$ & $0.05$ & $0.10$ & $0.60$\\
    ModRatio & $0.37$ & $0.18$ & $0.05^{*}$ & $0.37$ & $0.18$ & $0.04^{*}$ \\ \addlinespace
    Incongruency $\times$ Inclusion & & & & $0.36$ & $0.18$&$0.05^{*}$ \\
    Incongruency $\times$ Diversity & & & & $0.01$ & $0.18$ & $0.95$\\
    Incongruency $\times$ ModRatio & & & & $-0.15$ & $0.21$ & $0.47$\\
    [1.2ex]
    % \midrule
    % \textit{Random Effects} & Var. & SD & & Var. & SD\\ 
    % Subreddit & $0.64$ & $0.80$ & & $0.59$ & $0.77$  \\ 
    % [1.2ex]
    \midrule
    \textit{Model Fit} & & \\ 
    Marginal $R^2$ & $0.22$ & & & $0.23$ & \\
    Conditional $R^2$ & $0.35$ & & & $0.35$ & \\
    \hline \\[-1.8ex] 
    \textit{Note: $^{*}$p$<$0.05; $^{**}$p$<$0.01; $^{***}$p$<$0.001} \\ 
    \bottomrule
    \end{tabular}
    \label{tab:modratio}
\end{table*}
\begin{table}[]
    \footnotesize
    \centering
    \caption{The results of a linear mixed-effect model predicting a participant's likelihood of sharing their opinion in a subreddit. We present results including a quadratic term for \emph{Content Removal Rate} as an independent variable.}
    \begin{tabular}{lrrl}
    \toprule
     & Coef. & SE & $p$ \\ \midrule
    \textit{Fixed Effects} & & \\
    (Intercept) & $4.60$ & $0.26$ & $0.00^{***}$\\  \addlinespace
    \textbf{Controls} \\
    WTSC & $-0.37$ & $0.08$ & $0.00^{***}$\\ 
    Agreement Intensity & $0.35$ & $0.08$ & $0.00^{***}$\\
    Male (0/1) & $-0.49$ & $0.18$ & $0.01^{**}$ \\
    White (0/1) & $0.04$ & $0.17$ & $0.82$ \\ 
    Democrat (0/1) & $-0.65$ & $0.17$ &$0.00^{**}$ \\
    Account Tenure &  $0.20$ & $0.08$ & $0.01^{*}$ \\
    Posting & $0.40$ & $0.08$ & $0.00^{***}$ \\ \addlinespace
    \textbf{Independent Variables} \\
    Incongruency (0/1) & $-0.74$ & $0.18$ & $0.00^{**}$ \\
    Inclusion & $0.39$ & $0.08$ & $0.00^{**}$ \\
    Diversity & $0.06$ & $0.08$ & $0.51$ \\
    Content Removal Rate & $1.31$ & $0.69$ & $0.06$\\ 
    $\text{Content Removal Rate}^2$ & $-1.10$ & $0.72$ & $0.13$\\\addlinespace
    [1.2ex]
    % \midrule
    % \textit{Random Effects} & Var. & SD \\ 
    % Subreddit & $0.52$ & $0.72$ \\ 
    % [1.2ex]
    \midrule
    \textit{Model Fit} & & \\ 
    Marginal $R^2$ & $0.23$  \\
    Conditional $R^2$ & $0.33$ \\
    \hline \\[-1.8ex] 
    \textit{Note: $^{*}$p$<$0.05; $^{**}$p$<$0.01; $^{***}$p$<$0.001} \\ 
    \bottomrule
    \end{tabular}
    \label{tab:quadratic}
\end{table}


\section{Deviations from Pre-registration}
We pre-register our analyses on OSF.\footnote{Pre-registration materials at https://osf.io/gze65?view\_only=03f45025105a42b1b50cd6936ece06d0.} We also report the following deviations from our pre-registration. First, we split community values into two separate hypotheses as we found perceived inclusion and diversity were measuring distinct constructs. We also updated the wording on our hypotheses to mention incongruent viewpoints. Second, we changed how we measured \textit{Content Removal Rate}. In the pre-registration, we had planned to use 500 posts sampled from each subreddit; however, during analysis, we found this method underreported the amount of content removal compared to results from prior work~\cite{jhaver2019does}. Thus, we decided to rely on Pushshift data~\cite{pushshift2023dumps}, which gave us access to a larger corpus of posts. Third, we remove the variable \textit{Mod Ratio} from our models since it was correlated with \textit{Content Removal Rate}. For robustness, we replicate our results using \textit{Mod Ratio} as our measure for community moderation, finding similar results. We also added a random effect for subreddit. Finally, we conducted additional analyses using \textit{Upvote Likelihood} as our dependent variable.


\section{Other Techniques for Avoiding Self-Silencing}
In our survey we also ask participants what other actions they are likely to take for viewpoints they would not share under their main account. As shown in Table~\ref{tab:otheractions}, participants are most likely to ``lurk,'' or read discussion on a topic but not share their viewpoint. After lurking, the second most common option is to discuss the viewpoint offline with friends and family. Prior work~\cite{pewresearchcenter_2014_sos} found that people are more likely to engage in conversation on controversial political topics in face-to-face settings rather than on social media platforms. 

We also find that participants leverage the affordances on Reddit to find other means for expressing their viewpoint. For example, sharing the viewpoint using a ``throwaway account''~\cite{leavitt2015throwaway}, a temporary account created under a different pseudonym than the user's main account, is selected as a preferred option for 11 viewpoints. This suggests that in some cases, having a greater degree of anonymity, which a throwaway account offers, may help counteract silencing effects. Another option is to post the viewpoint to a different subreddit. When participants were asked why they decided to post to alternative subreddits, reasons included selecting subreddits that were smaller in size, were more related to the topic, or were more likely to agree with the viewpoint. 

\begin{table}[hb!]
    \centering
    \caption{Summary of actions that participants would take if they did not feel comfortable sharing a viewpoint, meaning they agreed with a viewpoint but reported that they were unlikely to share the opinion to a given subreddit.}
    \begin{tabular}{lr}
    \toprule
    \textbf{Action} &  \textbf{N}\\
    \midrule
    Read discussion on a topic but not share viewpoint & 68 \\
    Discuss with friends and family offline & 34 \\
    Choose to ignore discussion on topic & 19 \\
    Post viewpoint using a throwaway account & 11 \\
    Post to a different subreddit & 3 \\
    Other & 3 \\
    \bottomrule
    \end{tabular}

    \label{tab:otheractions}
\end{table}