\begin{figure*}[ht]
    \centering
    \includegraphics[width=0.75\textwidth]{Figures/viewpoint_breakdown.pdf}
    \caption{Across all topics, the average percentage of incongruent viewpoints likely to be shared is $28.3\%$. On the left, we report the percentage of incongruent viewpoints willing to be shared (i.e., \textit{Share Likelihood} $>$ 4) out of all incongruent viewpoints for the given topic. On the right, we do the same for congruent viewpoints.}
    \label{fig:viewpoint_breakdown}
\end{figure*}

In this section, we analyze the responses from our survey using linear mixed-effect models. We start by presenting the model we used for analysis. Next, we compare opinion expression in congruent versus incongruent opinion climates. Then, we explore the impact of community-level factors, such as inclusivity and moderation practices. We conclude by analyzing what factors modulate upvoting behavior.

\subsection{Method of Analysis}
To analyze the results of our survey, we use linear mixed-effect models, with subreddit as a random effect.\footnote{We report our analysis with ordinal regression models in Appendix~\ref{sec:robustness} and find qualitatively similar results.} We introduce a series of three nested models. As shown in Table~\ref{tab:model1}, we first have a model using only our control variables as predictors. We then add additional measures for viewpoint incongruency, community inclusivity, diversity, and moderation. Finally, we include interaction effects between incongruency and our measurements for community values and moderation to our model. We standardize all continuous independent variables and controls. For our first set of models (Models 1a-1c), the dependent variable is the self-reported likelihood of sharing the viewpoint on a main account (\textit{Share Likelihood}). In Models 2a-2c, we use the likelihood of upvoting the viewpoint (\textit{Upvote Likelihood}) as our dependent variable. The dependent variables are left in raw units. 







\begin{table*}[t!]
    \footnotesize
    \centering
    \caption{Although incongruent viewpoints are less likely to be shared than congruent viewpoints, having more inclusive communities can help abate self-silencing behavior. We present the results of a linear mixed-effect model that predicts a participant's likelihood of sharing their opinion as a hierarchical regression. Model 1a contains only our controls. Model 1b includes the controls and fixed effects for opinion incongruency, community values, and community moderation. Model 1c adds interaction effects. Note: $^{*}$p$<$0.05; $^{**}$p$<$0.01; $^{***}$p$<$0.001.}
    \begin{tabular}{lrrlrrlrrl}
    \toprule
         & \multicolumn{3}{c}{Model 1a} &  \multicolumn{3}{c}{Model 1b} &  \multicolumn{3}{c}{Model 1c}\\
    \cmidrule(lr){2-4}\cmidrule(lr){5-7}\cmidrule(lr){8-10}
     & Coef. & SE & $p$ & Coef. & SE & $p$ & Coef. & SE & $p$\\ \midrule
    % \textit{Fixed Effects} & & & & & & & & &\\
    (Intercept) & $4.27$ & $0.25$ & $0.00^{***}$ & $4.56$ & $0.27$ & $0.00^{***}$ & $4.58$ & $0.27$ & $0.00^{***}$\\
    \textbf{Controls} \\
    WTSC & $-0.38$ & $0.08$ & $0.00^{***}$ &$-0.37$ &$0.08$ &$0.00^{***}$ &$-0.38$ & $0.08$ & $0.00^{***}$\\ 
    Agreement Intensity & $0.44$ & $0.08$ & $0.00^{***}$ & $0.35$ & $0.08$ & $0.00^{***}$ & $0.37$ & $0.08$ & $0.00^{***}$\\
    Male (0/1) & $-0.41$ & $0.18$ & $0.03^{*}$ & $-0.47$ & $0.18$ & $0.01^{**}$ & $-0.48$ & $0.18$ & $0.01^{**}$\\
    White (0/1) & $-0.05$ & $0.17$ &$0.79$ & $0.03$ & $0.17$ & $0.85$ & $0.04$ & $0.17$ & $0.79$\\ 
    Democrat (0/1) & $-0.40$ & $0.17$ &$0.02^{*}$ & $-0.65$ & $0.17$ & $0.00^{***}$ & $-0.64$ & $0.17$ & $0.00^{***}$\\
    Account Tenure & $0.25$ & $0.08$ & $0.00^{***}$ & $0.20$ & $0.08$ & $0.01^{*}$ & $0.20$ & $0.08$ & $0.01^{*}$\\
    Posting & $0.51$ & $0.08$ & $0.00^{***}$& $0.39$ & $0.08$ & $0.00^{***}$ & $0.40$ & $0.08$ & $0.00^{***}$\\ \addlinespace
    \textbf{Independent Variables} \\
    (H1) Incongruency (0/1) & & & & $-0.73$ & $0.18$ & $0.00^{***}$ & $-0.63$ & $0.19$& $0.00^{***}$\\
    (H2a) Inclusion & & & & $0.38$ & $0.08$ & $0.00^{***}$ & $0.28$ & $0.10$ & $0.00^{**}$\\
    (H2b) Diversity & & & & $0.06$ & $0.08$ & $0.49$ & $0.06$ & $0.09$ & $0.51$\\
    (H3) Content Removal Rate & & & & $0.30$ & $0.17$ & $0.08$ & $0.31$ & $0.16$ &$0.07$\\ \addlinespace
    (H2a) Incongruency $\times$ Inclusion  & & & & & & & $0.35$ & $0.18$ & $0.05^{*}$\\
    (H2b) Incongruency $\times$ Diversity & & & & & & & $-0.01$ & $0.18$ & $0.97$\\
    (H3) Incongruency $\times$ Content Removal Rate & & & & & & & $-0.16$ & $0.19$ & $0.39$\\
    % \textit{Random Effects} & Var. & SD & & Var. & SD & & Var. & SD\\ 
    % Subreddit & $0.48$ & $0.69$ & & $0.65$ & $0.81$ & & $0.57$ & $0.75$\\ 
    % [1.2ex]
    \midrule
    % \textit{Model Fit} & & & & & & & & & \\ 
    Marginal $R^2$ & $0.17$ & & & $0.22$ & & & $0.22$\\
    Conditional $R^2$ & $0.27$ & & & $0.35$ & & & $0.34$\\
    % \hline \\[-1.8ex] 
    \bottomrule
    \end{tabular}
    \label{tab:model1}
\end{table*}


\begin{figure}[ht!]
    \centering
    \includegraphics[width=0.95\linewidth]{Figures/Marginal_Plot_Resized.pdf}
    \caption{An increase in community inclusion has a greater positive impact on the likelihood of sharing incongruent viewpoints compared to congruent ones. We visualize the marginal effects of community inclusion on opinion expression for incongruent and congruent viewpoints. For every one standard-deviation unit increase in perceived inclusion, the likelihood of sharing increases by $0.63$ for incongruent viewpoints and $0.28$ for congruent viewpoints.}
    \label{fig:marginal}
\end{figure}

\subsection{Analyzing Opinion Sharing Behavior}
\label{subsec:sharing}
\subsubsection{Participants are slightly likely to share congruent viewpoints.} We start by examining the relationship between the likelihood of sharing a viewpoint and our control variables. In Model 1a, the intercept is $4.27$, representing the likelihood of sharing a congruent viewpoint for a non-Male, non-White, and non-Democrat individual who posts on Reddit rarely to occasionally, has an account age of 5.8 years, and WTSC of 3.68 out of 8. The intercept value falls between ``neither likely nor unlikely'' and ``slightly likely'' on the 7-point scale from the survey. 

\subsubsection{Incongruent viewpoints are shared less often.}
Participants were less likely to voice incongruent viewpoints across all topics despite agreeing with the stance. Approximately half ($51.5\%$) of participants responded that they are likely to share their viewpoint (\textit{Share Likelihood} $> 4$) when they believe themselves to agree with the majority (see Fig.~\ref{fig:viewpoint_breakdown}). In contrast, the proportion of participants likely to share incongruent viewpoints is lower, ranging from $11\%$ to $42\%$. On average, across topics, participants are twice as likely to share a viewpoint they perceive as being in the majority compared to viewpoints they believe are in the minority.

We also examine Model 1b, which includes a fixed effect for viewpoint incongruency, allowing us to compare the likelihood of sharing a viewpoint that is perceived to be in the majority versus minority. In line with the spiral of silence theory~\cite{noelle1974spiral}, participants are less likely to comment their viewpoint when they believe their opinion differs from that of the majority ($p<0.001$). The likelihood of sharing decreases by $-0.73$ for an incongruent viewpoint compared to a congruent viewpoint. On average, when participants perceive themselves as being in the minority, they are less likely to share their viewpoint ($M = 3.01$) compared to those who view themselves as in line with the majority ($M = 4.05$). 
% Further, in our incongruency measure, we combine participants who are somewhat incongruent with those who are fully incongruent. We find that there is an even lower likelihood of sharing fully incongruent ($M = 2.70$) viewpoints compared to somewhat incongruent viewpoints ($M = 3.34$).
While participants view themselves as being congruent with the majority for most viewpoints, $72.2\%$ (N=78) of participants hold at least one incongruent viewpoint. Given that users must self-select into being members of different communities on Reddit, it is unlikely that they will be completely incongruent across all topics. Although community members are not typically in the opinion minority, they are likely to feel incongruent on at least a small selection of topics discussed on the subreddit. 



\subsubsection{Community inclusivity positively influences sharing behavior.}
Model 1b also includes measures for perceived community inclusion and diversity. \textit{Inclusion} has a positive relationship with \textit{Share Likelihood} ($p<0.001$) --- the likelihood of sharing increases by $0.38$ for each standard-deviation increase in perceived community inclusivity. In other words, when participants perceived a subreddit as being more open to contributions and felt more included in the community, they were more likely to participate. 

Furthermore, we find evidence supporting H2a: that the positive relationship between inclusivity and viewpoint sharing is more pronounced for incongruent opinions. Model 1c includes an interaction effect between \textit{Incongruency} and \textit{Inclusion}. The main effect of inclusion continues to be significant ($p<0.01$), and there is a modest but positive relationship between the interaction effect and the likelihood of sharing ($\beta=0.35$, $p=0.05$). For an incongruent viewpoint, a one-standard-deviation increase in inclusivity results in the sharing likelihood increasing by $0.63$; for congruent viewpoints, the sharing likelihood only increases by $0.28$ (see Fig.~\ref{fig:marginal}). This suggests that fostering more inclusive communities can help combat the spiral of silence.

However, we do not find evidence to support H2b, that perceived community diversity is more positively related to opinion sharing for incongruent viewpoints ($p=0.97$). It is possible that in politics-oriented subreddits, diversity is less of a concern to users. When comparing the relative importance of different values across subreddits,~\citet{weld2021making} found that communities focused on news, such as those included in our analysis, placed the lowest priority on diversity. 




\subsubsection{Content removal has no-to-weak association.}
We do not find evidence at our level of statistical power to support a claim that more moderation is associated with a greater likelihood of opinion-sharing. The percentage of content removed has a positive but non-significant trend with opinion expression ($\beta=0.30$, $p=0.08$). We also do not find any significant interaction between \emph{Incongruence} and \emph{Content Removal Rate} ($p=0.39$).

Finally, we explored whether there is a quadratic relationship between moderation and participation, since draconian moderation policies could stifle people's desire to speak up~\cite{kraut2012building} (see Appendix~\ref{sec:robustness} for full results). While there may be weak evidence of such trend ($\beta= -1.10$), the relationship is non-significant ($p=0.13$). It is possible that the subreddits in our sample are not overly moderated to the point that participation is diminished or subreddit members are unaware of the extent of content removal within a community~\cite{jhaver2019did}. 

\begin{table*}[t!]
    \footnotesize
    \centering
    \caption{Incongruent viewpoints are less likely to be upvoted compared to congruent viewpoints. We present the results of a linear mixed-effect model that predicts the likelihood of upvoting as a hierarchical regression. Model 2a only includes the controls; Model 2b has the controls and fixed effects for \textit{Incongruency}, community values, and community moderation; and Model 2c adds interaction effects. Note: $^{*}$p$<$0.05; $^{**}$p$<$0.01; $^{***}$p$<$0.001}
    \begin{tabular}{lrrlrrlrrl}
    \toprule
         & \multicolumn{3}{c}{Model 2a} &  \multicolumn{3}{c}{Model 2b} &  \multicolumn{3}{c}{Model 2c}\\
        \cmidrule(lr){2-4}\cmidrule(lr){5-7}\cmidrule(lr){8-10}
     & Coef. & SE & $p$ & Coef. & SE & $p$ & Coef. & SE & $p$\\ \midrule
    % \textit{Fixed Effects} & & & & & & & & &\\
    (Intercept) & $5.37$ & $0.19$ & $0.00^{***}$ & $5.39$ & $0.21$ & $0.00^{***}$ & $5.43$ & $0.21$ & $0.00^{***}$\\ 
    \textbf{Controls} \\
    WTSC & $0.01$ & $0.06$ & $0.93$ &$0.01$ &$0.06$ &$0.85$ &$-0.00$ & $0.06$ & $0.94$\\ 
    Agreement Intensity & $0.50$ & $0.06$ & $0.00^{***}$ & $0.46$ & $0.06$ & $0.00^{***}$ & $0.47$ & $0.06$ & $0.00^{***}$\\
    Male (0/1) & $0.01$ & $0.13$ & $0.96$ & $-0.03$ & $0.13$ & $0.83$ & $-0.05$ & $0.13$ & $0.67$\\
    White (0/1) & $-0.06$ & $0.13$ &$0.66$ & $-0.03$ & $0.13$ & $0.82$ & $-0.03$ & $0.13$ & $0.81$\\ 
    Democrat (0/1) & $0.29$ & $0.13$ &$0.03^{*}$ & $0.19$ & $0.13$ & $0.14$ & $0.19$ & $0.13$ & $0.14$\\
    Account Tenure & $-0.07$ & $0.06$ & $0.25$ & $-0.10$ & $0.06$ & $0.09$ & $-0.10$ & $0.06$ & $0.10$\\
    Posting & $0.49$ & $0.06$ & $0.00^{***}$& $0.42$ & $0.06$ & $0.00^{***}$ & $0.42$ & $0.06$ & $0.00^{***}$\\ \addlinespace
    \textbf{Independent Variables} \\
    Incongruency (0/1) & & & & $-0.27$ & $0.13$ & $0.05^{*}$ & $-0.20$ & $0.14$& $0.15$\\
    Inclusion & & & & $0.24$ & $0.06$ & $0.00^{***}$ & $0.20$ & $0.07$ & $0.01^{***}$\\
    Diversity & & & & $0.08$ & $0.06$ & $0.21$ & $0.02$ & $0.07$ & $0.74$\\
    Content Removal Rate & & & & $-0.07$ & $0.13$ & $0.57$ & $-0.04$ & $0.13$ &$0.74$\\ \addlinespace
    Incongruency $\times$ Inclusion  & & & & & & & $0.09$ & $0.13$ & $0.48$\\
    Incongruency $\times$ Diversity & & & & & & & $0.23$ & $0.13$ & $0.09$\\
    Incongruency $\times$ Content Removal Rate & & & & & & & $-0.20$ & $0.14$ & $0.15$\\
    % \textit{Random Effects} & Var. & SD & & Var. & SD & & Var. & SD\\ 
    % Subreddit & $0.36$ & $0.60$ & & $0.38$ & $0.62$ & & $0.40$ & $0.64$\\ 
    % [1.2ex]
    \midrule
    % \textit{Model Fit} & & & & & & & & & \\ 
    Marginal $R^2$ & $0.18$ & & & $0.21$ & & & $0.22$\\
    Conditional $R^2$ & $0.32$ & & & $0.35$ & & & $0.36$\\
    % \hline \\[-1.8ex] 
    % \textit{Note: $^{*}$p$<$0.05; $^{**}$p$<$0.01; $^{***}$p$<$0.001} \\ 
    \bottomrule
    \end{tabular}
    \label{tab:model2}
\end{table*}

\subsection{Analyzing Upvoting Behavior}
We also explore participants' likelihood of upvoting as another form of opinion expression. Notably, unlike posting or commenting, upvoting is anonymous. 

\subsubsection{Incongruent viewpoints are less likely to be upvoted, even when people agree with the stance.} We observe a negative relationship between \textit{Incongruency} and the likelihood of upvoting ($\beta = -0.27$, $p=0.05$ in Model 2b). Similar to sharing behavior, participants are more likely to upvote content in more inclusive communities: the likelihood of upvoting a comment increases by $0.20$ for each standard deviation increase in perceived community inclusion. However, we do not find any significant interaction effects, indicating the measures for community values and moderation do not impact the likelihood of upvoting a congruent viewpoint compared to an incongruent one. 



\subsubsection{Upvoting provides an alternative for sharing otherwise self-silenced opinions.} Overall, upvoting provides a valuable avenue for users to express their opinion. We compare the number of viewpoints that users report that they are not willing to comment on but are willing to upvote. In congruent conditions, $68.9\%$ of viewpoints that would not be posted (\textit{Share Likelihood} $ \leq 4$) would be upvoted (\textit{Upvote Likelihood} $ > 4$). While this percentage of viewpoints is lower in comparison ($52.6\%$) for incongruent viewpoints, for a majority of viewpoints, participants are still likely to use upvoting as a mechanism for expressing their opinions even when they are unwilling to post them. 
