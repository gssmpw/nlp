Social media is often viewed as a mirror of society, or at the very least, one of our best available instruments for understanding public opinion. Social media data has been used for a range of tasks from forecasting box-office performances~\cite{asur2010predicting} to understanding political support for candidates~\cite{mcgregor2020taking}. Furthermore, social media content can actively shape people's opinions and influence their actions~\cite{laranjo2015influence}.


However, what we see online may not be a complete or accurate picture of public opinion. This is partly because online content originates from a small fraction of platform users~\cite{ruths2014social}. The majority of online communities consist of ``lurkers'' or individuals who visit the community but choose to remain silent~\cite{sun2014understanding}. There are many reasons why people may choose to stay as lurkers. In some cases, users may just want to learn about a topic or are not motivated to post if they see their opinion reflected in existing content~\cite{nonnecke2001lurkers}. Other times, however, users may want to share their opinions but are uncomfortable doing so. The disproportionate silencing of certain viewpoints skews the content that we see online away from the true opinions of users.  

One explanation for this phenomenon is \citet{noelle1974spiral}'s spiral of silence theory. The theory posits that, when people perceive their viewpoints as being \textit{incongruent}, or different from the majority of the group, they are less likely to engage in \textit{opinion expression}, or share their viewpoint. Since fewer people express the opinion, the viewpoint appears to be even further in the minority, triggering a spiral effect as the growing silence of incongruent opinions compounds upon itself. 


The spiral of silence is an empirical effect, not an unavoidable constant applying to all people: there are ``avant-garde'' and ``hard core'' individuals who stand up for their viewpoints even when they are in the minority~\cite{noelle1977turbulences}. Through empirical studies,  work has proposed factors that modulate the spiral of silence. Individual-level antecedents, including personality traits~\cite{hayes2005willingness,neuwirth2007spiral}, opinion certainty~\cite{matthes2018spiral}, and issue type~\cite{gearhart2018same} can influence opinion expression. In social media settings, prior work has focused on identifying platform affordances that influence self-silencing behavior~\cite{wu2018comment,pang2016can}. For example, greater perceived anonymity can dampen the effects as the potential social consequences of speaking out are diluted~\cite{wu2018comment}. Even commenting versus liking content can impact the degree of self-silencing~\cite{pang2016can}. 
 
While we have insight into how platform-level design decisions influence self-silencing, platforms can house many different online communities each with their own rules, norms, and values. Given the unique configuration of each online community, it is not surprising that users' behaviors, such as conversation patterns~\cite{choi2015characterizing} or self-disclosure~\cite{yang2017self}, will differ across communities. At the moment, the impact of these community-level design decisions has not been factored into our understanding of how the spiral of silence manifests. This leads us to our research questions:
\begin{addmargin}[1.25em]{0em}
\textbf{RQ1.} How does users' opinion expression on Reddit differ when they perceive their viewpoint as being in the majority versus the minority?\\
\textbf{RQ2.} How do online community design factors influence opinion expression of incongruent viewpoints?
\end{addmargin}

\section{Factors Influencing Opinion Expression}
\label{sec:factors}
In this section, we explore how the spiral of silence and community design decisions can shape opinion expression on social media and introduce our hypotheses.

\subsection{Opinion Incongruency}
The spiral of silence theory provides a potential explanation for opinion expression on social media platforms. Previous studies~\cite{pewresearchcenter_2014_sos,kushin2019societal} on social media platforms found evidence that users are less likely to express incongruent viewpoints. We expect a similar phenomenon on Reddit. Since subreddits offer a sense of community and socializing opportunities~\cite{moore2017redditors}, community members may still experience a fear of social isolation. Repercussions, such as being downvoted or facing personal attacks, can trigger this fear~\cite{neubaum2018we,cheng2017anyone}. These consequences encourage users to self-silence when they have an opinion they believe is in the minority. Thus, we expect our study will yield results in accordance with the spiral of silence: 
\begin{addmargin}[1.25em]{0em}
\textit{Hypothesis 1: Likelihood of opinion expression is positively associated with viewpoint congruency.}
\end{addmargin}

\subsection{Community Inclusion and Diversity}
How online communities are designed will influence how users behave, especially regarding sharing their opinions publicly. A greater sense of belonging can lead users to be more active as they feel more included and trusting of others in the group. Similarly, feeling psychologically safe can foster confidence and empower users to voice their opinions~\cite{edmondson2014psychological}.

Feeling a sense of belonging is an essential human need that applies not only to offline but also online settings~\cite{maslow1958dynamic,lambert2013belong}. \citet{hagerty1992sense} proposed two key dimensions that define a sense of belonging: (1)~having involvement in a group be valued, and (2)~feeling a sense of fit or congruency with other group members. When these criteria are met, people are more likely to be involved in a group and find this involvement to be more meaningful~\cite{hagerty1996sense}. The same principles apply to belonging in online communities~\cite{lampe2010motivations}. Users with strong shared identities or bonds are more likely to be active in online groups~\cite{ren2007applying}. When people feel included, they are more likely to have a sense of familiarity and trust other community members~\cite{zhao2012cultivating}. This encourages users to share more freely as they feel less at risk of being judged when sharing their viewpoints. Thus, we expect that:

\vspace{0.25em}
\begin{addmargin}[1.25em]{0em}
\textit{Hypothesis 2a: The perceived inclusivity of a community is more positively associated with the likelihood of opinion expression when users hold incongruent viewpoints.}
\end{addmargin}
\vspace{0.25em}

Who is part of the community also matters. Users frequently mention the fear of being personally attacked as a primary reason for not posting minority opinions~\cite{neubaum2018we}. In both online and offline settings, existing research has found that people are more secure when sharing ideas in psychologically safe communities~\cite{newman2017psychological}. That is, when people foresee fewer negative interpersonal consequences, they are more willing to take potentially controversial actions. Community diversity is an important antecedent for psychological safety~\cite{newman2017psychological}. Emphasizing diversity can foster a positive climate in which community members, regardless of their background, more strongly identify as members of the group and feel psychologically safe~\cite{mckay2007racial}. In this setting, group members are more willing to engage in potentially risky behaviors. From this literature, we hypothesize the following:

\vspace{0.25em}
\begin{addmargin}[1.25em]{0em}
\textit{Hypothesis 2b: The diversity of community members is more positively associated with the likelihood of opinion expression when users hold incongruent viewpoints.}
\end{addmargin}
\vspace{0.25em}

To conceptualize inclusion and diversity, we draw on work that has sought to quantify the values of online communities~\cite{weld2022makes}. We utilize the instrument proposed in this work~\cite{weld2021making} to capture users' perceived sense of inclusion and diversity across different subreddits.


\subsection{Community Moderation}
Moderation practices frequently shape the dynamics of online communities. Moderators can set the tone of the online community by encouraging other users to imitate their actions and by curbing instances of behavior that they deem to fall outside community norms~\cite{seering2017shaping}. Removing or prescreening content that does not adhere to the community's norms is an effective way of reducing ``bad behavior'' and can positively impact community health~\cite{kraut2012building,seering2017shaping}. By filtering out potentially harassing or offensive submissions, moderators can assuage members' fear of being personally attacked and facilitate more open opinion-sharing~\cite{gibson2019free}. For example,~\citet{wadden2021effect} studied mental health discussions in moderated and unmoderated spaces. They found that users were not only more likely to post in moderated spaces but also that they were more willing to share vulnerable messages in these discussions. From these results, we expect that users feel more comfortable sharing minoritized opinions in communities with more active moderation. Thus, we hypothesize:

\vspace{0.25em}
\begin{addmargin}[1.25em]{0em}
\textit{Hypothesis 3: Moderation activity is more positively associated with the likelihood of opinion expression when users hold incongruent viewpoints.}
\end{addmargin}
\vspace{0.25em}

In this work, we focus on moderation through content removal. We acknowledge that content removal represents only a subset of moderation actions available on Reddit. Subreddit moderators can intervene at the content level, which, in addition to removing content, includes approving content, managing users, changing settings (e.g., rules), or editing flairs / labels~\cite{li2022all}. Moderation can occur at the user-level (e.g., deplatforming, shadow-banning)~\cite{rogers2020deplatforming, myers2018censored} or community-level ~\cite{trujillo2022make,chandrasekharan2022quarantined} (e.g., quarantining). We choose to focus on content removal because it is one of the most common moderation interventions and has been shown to influence community members' participation~\cite{li2022all}.

