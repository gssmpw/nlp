Content shared on social media is often treated as a reflection of broader public opinion~\cite{anstead2015social,mcgregor2020taking,asur2010predicting}. But is this view distorted? For example, in 2019, a racist photo of the governor of Virginia surfaced online. As reported in \textit{The~New~York~Times}, this event led to universal condemnation of the governor on social media, leading major politicians to demand that he resign~\cite{cohn2019dem}. However, a poll of Virginians showed that most people, and especially Black constituents, actually wanted the governor to remain in office~\cite{jamison2012wapo}. Such voices were less vocal and less visible on social media, despite their prevalence.


Divergence between what we see on social media and what the public believes is not uncommon. What causes this distortion? While this could be an artifact of users' preferences for posting online on certain topics, or perhaps a product of social loafing~\cite{karau1993social}, we explore an alternative: users may feel uncomfortable sharing their opinion if they feel they are in the minority. We ground this explanation in the spiral of silence theory~\cite{noelle1974spiral}. The theory posits that, due to a fear of social isolation, people may \textit{self-silence}, defined as deciding not to share their opinion, when they view themselves to be in the minority. Self-silencing leads to less of the opinion being expressed in the group, making the viewpoint appear even further minoritized, producing a downward spiral where people self-silence even further. Since being proposed, the spiral of silence has been empirically studied in face-to-face communication across multiple issue types and participant populations~\cite{scheufle2000twenty,matthes2018spiral}. As online communication forms like social media continue to proliferate, it raises the question how the spiral of silence might affect user behavior in this setting.

Measuring the spiral of silence on social media, especially across diverse online communities, is challenging. Not all topics are likely to trigger the spiral of silence~\cite{lee2004cross}, and identifying such topics may require insider knowledge of community norms and practices~\cite{chen2013and,hessel2019something}. Furthermore, even with a set of topics in hand, the process of measuring rates of self-silencing requires capturing user attitudes, beliefs, and behaviors that cannot be observed from online content itself. This traditional approach of analyzing existing posts online is self-defeating, since it excludes viewpoints that users feel uncomfortable sharing online. 

In our work, we develop a human-plus-algorithm pipeline to generate topics that are contextually relevant to a community but also likely to spark internal disagreement. Then, drawing on survey methodologies common in empirical studies of the spiral of silence~\cite{matthes2018spiral}, we measure whether the spiral of silence is occurring on those topics in those communities. Further, we evaluate whether online community factors such as moderation and inclusivity mitigate or amplify the spiral. We apply this method to study the spiral of silence on Reddit --- a popular social media platform in the United States --- across a number of subreddits focused on political and social issues but with varying norms and values.  

We find that those who believe their opinion is in the minority remain silent $72.6\%$ of the time, and these opinions are only half as likely to be posted compared to those in the majority. They are also less likely to upvote posts sharing their opinion. However, this self-silencing effect is mitigated by the community's perceived level of inclusivity. We test these results through a series of three mixed-effects models and find that higher inclusivity is not only positively associated with posting overall, but leads to $2.3$ times more of an increase in opinion expression for those who believe they are in the minority. 


Our findings reinforce how social media data can present a substantial distortion of broader public opinion, systematically under-representing minority viewpoints. There are certainly some topics where minority opinions are normatively undesirable to represent, even if a portion of the population agrees with them, such as, hate speech or misinformation. However, for many other topics where the minority voice instead reflects under-represented groups or moderate views from people who believe themselves to be in the minority, communities may want to mitigate the spiral and hear from a more representative set of voices~\cite{farjam2024social,bowen2003spirals}. To reduce rates of self-silencing, we discuss tangible actions that our results suggest online moderators and platform designers can take to increase perceived community inclusivity and mitigate the spiral of silence.