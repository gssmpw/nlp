In our work, we seek to measure the extent of self-silencing behavior across different political subreddits. Through our analysis, we find evidence that those who believe their opinion to be in the minority are half as likely to post their viewpoint on Reddit, confirming the spiral of silence theory and reinforcing the distortion that social media data may be applying to public opinion. We also discuss community design factors that can help counteract these silencing effects. 
A summary of our hypotheses and findings can be found in Table~\ref{tab:summary}. 

\subsubsection{Confirming and measuring the spiral of silence online.} 
We find robust evidence that participants are less likely to share their viewpoint when they perceive their opinion to be in the minority. Corroborating~\citet{gearhart2018same}'s findings, we observe that the rate of self-silencing differs across issue types, although overall participants are consistently less willing to express incongruent viewpoints. 

We also introduce a method for identifying controversial viewpoints within a community and then measuring the extent of the spiral of silence. This method provides a flexible tool for generating topics that can be tailored to a specified community. In this work, we focused on identifying political and social issues, but our method can be applied to other topic areas. For example, we can generate plausible viewpoints even when using communities that focus on video games and sports --- \texttt{r/gaming} and \texttt{r/nba}), respectively (see Appendix~\ref{sec:hai_design} for details). Future work can adapt this method to explore other online communities and topics. 

\begin{table}[t]
    \footnotesize
    \centering
    \caption{Summary of hypotheses and findings. $\checkmark$  
   indicates the hypothesis is supported. ``NS'' indicates the hypothesis is not supported.}
    \begin{tabular}{>{\raggedright}p{2.7in}>{\raggedright\arraybackslash}p{0.24in}}
    \toprule
    \textbf{Hypothesis} & \textbf{Result} \\
    \midrule
    \RaggedRight{\textbf{H1}: Likelihood of opinion expression is positively associated with viewpoint congruency.} & \large{\checkmark} \\
    \addlinespace
    \RaggedRight{\textbf{H2a}: The perceived inclusivity of a community is more positively associated with the likelihood of opinion expression when users hold incongruent viewpoints.} & \large{\checkmark} \\
    \addlinespace
    \textbf{H2b}: The diversity of community members is more positively associated with the likelihood of opinion expression when users hold incongruent viewpoints. & NS \\
    \addlinespace
    \textbf{H3}: Moderation activity is more positively associated with the likelihood of opinion expression when users hold incongruent viewpoints. & NS\\
    \bottomrule
    \end{tabular}
    \label{tab:summary}
\end{table}

\subsubsection{Community inclusivity helps counter the spiral of silence.}
We identify community design factors that help counteract silencing effects. Previous work~\cite{neubaum2022s,wu2018comment} identified platform-level designs that influence the spiral of silence, but it is difficult for users and moderators to intervene at the platform level compared to the community level. We found that perceived inclusivity has a significant positive impact on community member participation. Moreover, inclusivity plays an outsized role in promoting sharing when individuals consider their opinion to be in the minority. Since moderators have more power to influence the values and norms of online communities, this finding suggests that there are feasible interventions to mitigate self-silencing without having to change platform-level design. For example, although community safety is an important aspect of inclusion~\cite{weld2021making}, prior work found that most anti-social behavior (e.g., personal attacks, bigotry) remains unmoderated on subreddits~\cite{park2022measuring}. Measures such as automated toxicity detectors to help moderators remove harmful content more efficiently can be a fruitful way to reduce self-silencing.





\subsubsection{Limitations and future work.}
We use self-reported measures of opinion expression to capture the spiral of silence. This approach may lack ecological validity compared to studying sharing behavior directly on the platform. It is possible that participants' reported willingness to share may differ from their situated behaviors. However, our approach remains well validated in prior work~\cite{gearhart2014gay,gearhart2018same,chia2014authoritarian}.One avenue for future work is to build systems that allow us to measure self-silencing directly on the platform rather than using surveys. 

In addition, there are limitations with our methodology that can be explored in future studies. We take an automated approach to generating topics and viewpoints. {While we reap the benefits of easily scaling the number of topics we can measure, this method does not guarantee coverage of all controversial topics within a community. There may be niche topics to specific online communities that we are not currently measuring. An alternative tactic is to augment the generated topics with crowdsourced viewpoints from community members, although crowdsourced responses may be skewed by the spiral of silence already occurring in the community. Finally, our analyses are centered on politically-oriented communities on Reddit. Results may differ if we looked at other topic areas or other social media platforms, such as Facebook or Twitter. Further exploration on these other communities or platforms can help bolster the generalizability of our findings. 
