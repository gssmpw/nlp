\begin{tabular}{p{0.9\textwidth}}
\toprule
\textbf{The prompt for \ourmethod} \\
\midrule
Please think in English and locate the relevant information from the text and table according to the question.\\
Here are several examples:\\
---\\
7. Nombre et coûts des employés...\\
| —                      | 2019   | 2018   |\\
| ---------------------- | ------ | ------ |\\
| —                      | Nombre | Nombre |\\
...\\
Question: Quelles sont les catégories d'employés listées dans le tableau ?\\
"Catégories des employés" links to the rows of the table "Opérations clients", "Produit et technologie", "Corporate" and the columns of the table "2019", "2018".\\
---\\
Le tableau suivant présente la répartition des revenus par catégorie et segment. ...\\
| Année se terminant le 31 décembre,    |          |         |\\
| ------------------------------------- | -------- | ------- |\\
|                                       | 2019     | 2018    |\\
...\\
Question: En 2019, combien de régions géographiques ont des revenus totaux supérieurs à 20 000 milliers de dollars\\
"2019" links to the column of the table "2019". "total revenues of geographic regions" links to the rows of the table "Total des revenus de l'Asie-Pacifique", "Total des revenus en Europe", "Total des revenus en Amérique du Nord".\\
---\\
Taux d'imposition effectif...\\
| —                                          | 31 décembre 2019 | 31 décembre 2018 |\\
...\\
Question: Quel a été le pourcentage de variation des pertes avant impôts en 2019 ?\\
"pérdidas antes de impuestos de 2019" y "pérdidas antes de impuestos de 2018" se vinculan a la parte del texto "In 2019 and 2018 we had pre-tax losses of \$19,573 and \$25,403, respectively".\\
---\\
Based on the examples above, analyze the question.\\
Please note that you **only** need to locate the relevant information, without performing additional calculations.\\
\{Table\}\\
\{Paragraph\}\\
Question :\{Question\}\\
\\
According to the relevant information, you should also think in English and write a python code to answer the question.\\
Here are several examples:\\
---\\
...\\
```python\\
ans = ['Opérations clients', 'Produit et technologie', 'Corporate']\\
```\\
---\\
...\\
```python\\
total\_revenues\_in\_all\_regions = \{'Asie-Pacifique': 6490, 'Europe': 36898, 'Amérique du Nord': 68024\}\\
regions\_have\_more\_than\_20000\_thousand\_total\_revenues = [k for k, v in total\_revenues\_in\_all\_regions.items() if v > 20000]\\
ans = len(regions\_have\_more\_than\_20000\_thousand\_total\_revenues)\\
```\\
---\\
...\\
```python\\
pre\_tax\_losses\_2018 = 25403  pre\_tax\_losses\_2019 = 19573\\
net\_change = pre\_tax\_losses\_2019 - pre\_tax\_losses\_2018\\
ans = net\_change / pre\_tax\_losses\_2018 * 100\\
```\\
---\\
Based on the examples above, answer the question with a Python code.\\
Please note:\\
1. In addition to numbers, try to use fr as the answer.\\
2. Keep your answer **short** with fewer statements.\\
3. Note the possible minus sign.\\
4. You MUST generate a Python code instead of returning the answer directly.\\
Represent your answer with: "ans = <your answer>"\\
\{Table\}\\
\{Paragraph\}\\
Question :\{Question\}\\
\bottomrule
\end{tabular}