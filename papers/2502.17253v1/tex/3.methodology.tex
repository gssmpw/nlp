\subsection{Overview}
% 本方法用于解决不同语言设置下的TATQA任务
\ourmethod is designed to address the TATQA task under the multilingual setting. 
% 为了将模型在英语上的强大的TATQA能力对齐到非英语语言,尤其是低资源语言上,我们将我们的方法分为两部分:Linking和Reasoning
% To align the strong TATQA capabilities of models in English with non-English languages, particularly low-resource languages, we divide \ourmethod into two modules: Linking and Reasoning. 
% 如图所示,Linking负责根据问题从母语语言的表格和文本中链接到相关信息,而Reasoning则根据链接到的信息用英文推理
% As shown in Figure~\ref{fig:method}, Linking is responsible for locating relevant information from tables and text in the native language based on the question, and Reasoning performs reasoning in English based on the linked information.
% 为了将模型在英语上的强大的TATQA能力对齐到非英语语言,尤其是低资源语言上,我们的方法采用跨语言推理
To align the strong TATQA capabilities of models in English with non-English languages, particularly low-resource languages, \ourmethod employs cross-lingual reasoning. 
% 为了令模型面对非英语的问题和表格、文本能进行英文推理,我们的方法分为Linking和Reasoning
To enable the model to perform English reasoning with non-English questions, tables, and text, \ourmethod is divided into two modules: Linking and Reasoning. 
% 如图所示,Linking负责根据问题从母语语言的表格和文本中链接到相关信息,而Reasoning则可以根据链接到的信息用英文推理
As shown in Figure~\ref{fig:method}, Linking is responsible for locating relevant information from tables and text in the native language based on the question, and Reasoning performs reasoning in English based on the linked information.
% 我们将我们的方法分为两部分:Linking和Reasoning
% 考虑到LLM在非英语语言,尤其是低资源语言上的推理能力相比英语较为落后,所以我们方法采用跨语言推理,即只用英语推理无论问题的语言
% Considering that the reasoning ability of LLMs is generally weaker for non-English languages, particularly low-resource languages, compared to English, \ourmethod adopts cross-lingual reasoning, i.e., reasoning in English regardless of the native language \cite{qin-etal-2023-cross-lingual-CLP,liu2024translation-all-you-need}. 
% 考虑到TATQA任务中需要根据问题链接到表格或文本中的相关信息,在跨语言推理时链接会更为困难,所以我们提出一个强baseline来解决这一问题
% Given that the TATQA task requires linking to the relevant information in the hybrid context based on the question, and performing complex reasoning, such as numerical computation, based on the relevant information, we propose \ourmethod. 
% 如图中所示,我们的方法分为linking和reasoning两部分
% As shown in Figure~\ref{fig:method}, \ourmethod is divided into two parts: Linking and Reasoning. 
% linking负责将问题中的实体信息对应到文本或表格中的信息,而reasoning负责根据链接的信息生成代码求解
% Linking is responsible for mapping the entities in the question to the corresponding information in the table or text, and Reasoning generates programs to derive the answer based on the linked information. 
% 本方法使用的prompt在附录提供
The prompts used in \ourmethod are provided in Appendix~\ref{sec:prompt}.


\subsection{Linking}
% 为了令模型面对非英语的表格和文本,能够进行英文推理,我们首先根据问题链接到相关信息
% Linking负责将问题中的实体链接到输入的文本和表格中的相关信息,以便Reasoning在生成代码时直接使用相关的信息
Linking is used to map the entities in the question to the relevant information in the input text and tables so that Reasoning can directly utilize this information when generating the code. 
% 具体来说,我们首先在第一轮对话提示LLM用英文思考,并将问题中的有效实体逐步对应到表格或文字中的信息
Specifically, we prompt the LLM to think in English and gradually map the relevant entities in the question to the information in the tables or text.

\subsection{Reasoning}
% Reasoning负责在Linking结果的基础上,生成代码求解问题得到最终答案
Reasoning is responsible for generating Python programs to solve the question and obtain the final answer based on the results of Linking. 
% 具体来说,我们在第二轮对话提示LLM根据相关信息生成代码
% Specifically, we prompt the LLM to generate the Python program based on the relevant information.
% 由于我们数据集的答案中不只有数字,所以我们还提示LLM注意代码返回的答案除了阿拉伯数字都要用原语言表示
Considering that there are not only numbers in the answers, we also remind the LLM to note that the answers should be represented in the native language except for Arabic numerals.
% 由于相关信息在linking阶段被提取出来,所以Reasoning在生成代码时可以直接用英文变量名定义数字或表格信息
Since the relevant information is extracted during Linking, Reasoning can directly use English variable names to define the numerical or tabular data when generating the program.