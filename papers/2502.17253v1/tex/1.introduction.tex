% 介绍TATQA任务
% 在表格和文本的混合数据上问答是一个很重要的任务,我们简称为TATQA任务,被广泛用于金融和科研等数据密集的领域
Question answering over the hybrid context of tabular and textual data (TATQA) is an important task \cite{chen-etal-2020-hybridqa}, which is widely used in data-intensive fields, such as finance and science, gaining increasing attention \cite{chen-etal-2021-finqa,auer2023sciqa}. 
% 提升模型TATQA的能力能够更好地帮助人们从混合数据中获取有效信息
Enhancing the TATQA capabilities of models can significantly aid in extracting useful information from hybrid data.
% 其中,TATQA任务需要模型根据问题中的实体链接到表格或文本中的相关信息,异质的evidence为这个任务带来了挑战
The heterogeneous evidence brings challenges to the TATQA task since it requires the model to link the relevant information in the table or text according to the entities in the question \cite{feng-etal-2022-multi-hop,lei-etal-2022-Graph-based-Encoder,wang2022hybridqa-survey}.
% 然而,数据的异质性为模型带来了挑战
% However, the hybrid data presents a significant challenge for model performance \cite{wang2022hybridqa-survey}.

\begin{figure}[t]
    \centering
    \includegraphics[width=1.\linewidth]{fig/intro.pdf}
    \vspace{-0.5em}
    \caption{
    % 我们数据集中中文示例和英语示例的对比
    Comparison of the English and Chinese examples in \ourdataset.
    % 相同颜色标注的代表对应的实体信息
    Entities with the same color annotation represent corresponding entity information.
    % 中文由于丰富的词法表达使模型链接相关信息更困难导致预测的结果错误
    In Chinese, the richness of lexical expressions makes it more challenging for the model to link relevant information, leading to the incorrect predicted answer.
    }
    \label{fig:intro}
\end{figure}

% 介绍前人TATQA数据集的不足,以及本文要做的事
% 为了评测模型TATQA的能力,前人提出了很多数据集
To evaluate the model capabilities on the TATQA task, several datasets are proposed \cite{li2021tsqa,chen-etal-2021-finqa,zhao-etal-2024-docmath}. 
% 比如,HybridQA,TAT-QA和SciTAT分别构建了wikipedia,金融和科研领域的英文TATQA数据
For example, HybridQA~\cite{chen-etal-2020-hybridqa}, TAT-QA~\cite{zhu-etal-2021-tat}, and SciTAT~\cite{zhang2024scitat} respectively construct English TATQA datasets in the domains of Wikipedia, finance, and science. 
% 然而,这些数据集只关注到了单一语言,而忽略了以下多语言TATQA的挑战
% However, these datasets focus solely on English, overlooking the multilingual TATQA:
% 然而,这些数据集只关注到了单一语言,存在以下缺点
However, these datasets focus solely on English, having the following shortcomings:
% 1. 不同语言的特性为TATQA任务带来挑战,因为其需要根据问题链接到表格或文本中相关的信息,而词法丰富的语言链接会更加困难
% (\emph{i})~The linguistic characteristics of different languages pose challenges to TATQA, as TATQA requires linking the question-relevant information in the table or text \cite{wang2022hybridqa-survey}. 
% This process is more difficult for morphologically rich languages.
% 2. 模型在不同语言上的理解及推理能力不同,导致在不同语言上的TATQA的性能不同
% (\emph{ii})~Variations in model comprehension and reasoning abilities across languages lead to the performance gap in TATQA \cite{qin2024multilingual-survey}.
% 1. 不能很好的评测模型的TATQA的能力,在多语言setting下
(\emph{i})~They \textbf{\textit{cannot adequately assess the TATQA performance in the multilingual setting, overlooking the challenges of multilingual TATQA}}.
% 尤其是不同语言复杂的词法和句法会为模型链接从异质上下文中链接信息带来挑战 and syntactic structures
As shown in Figure~\ref{fig:intro}, the complex lexical expressions of different languages pose challenges for models to link information across hybrid contexts \cite{MultiSpider}.
% 2. 和实际场景有gap,因为金融、科研等领域也存在着大量非英语的表格和文本
(\emph{ii})~They \textbf{\textit{create a gap with real-world scenarios}}, as domains such as finance and science contain substantial amounts of non-English tables and text \cite{hamotskyi-etal-2024-fincorpus,angulo2021non-English-scientific,bhagavatula-etal-2012-language-wiki}.
% 为了解决这些限制,本文
To address the limitations, 
% 提出了首歌多语言TATQA数据集,集合前人TATQA数据集将英语翻译为10个类型多样的语言
% we propose the first multilingual TATQA benchmark, which combines three mainstream TATQA datasets and translates English data into $10$ diverse languages, to evaluate the TATQA capabilities under the multilingual setting.
% 提出了首个多语言TATQA数据集,共包含11个多样语言的平行数据
we propose the first multilingual TATQA benchmark, comprising parallel data in $11$ diverse languages.
% The first multilingual TATQA data set is proposed, which contains a total of 11 parallel data in various languages.
% 2. 为了提升模型在不同语言上的TATQA能力,提出一个强baseline,能够处理不同语言的混合信息
% (\emph{ii})~We introduce a baseline designed to enhance TATQA performance across different languages, capable of processing multilingual hybrid data.


% 介绍数据集的构造
% 首先,我们提出了多语言TATQA数据集,来源于HybridQA, TATQA和SciTaT的英文数据
First, we introduce the multilingual TATQA dataset (\ourdataset).
% , which comes from the English data of HybridQA, TAT-QA and SciTAT. 
% 为了确保我们数据集翻译的高质量,我们采用了自动翻译和人工修改相结合的方法,将表格、文本问题和答案翻译为10种不同的语言
% 为了确保我们数据集的高质量,我们从三个主流的英文TAT-QA数据集中sample数据,并采用了自动翻译和人工修改相结合的方法将它们翻译为10种语言
% To ensure the high quality of translation, we employ a combination of automatic translation and manual revision to translate tables, text, questions and answers into $10$ languages. 
To ensure the high quality of \ourdataset, we sample data from three mainstream English TATQA datasets and employ a combination of machine translation and manual revision to translate them into $10$ languages.
% 总共,我们的数据集包括了来自233个混合上下文的250个问题,囊括wikipedia,金融和科学三个领域
In total, \ourdataset consists of $250$ questions from $233$ hybrid contexts, covering three domains: Wikipedia, finance, and science. 
% 人工校验证明了我们数据集翻译的质量
% Manual verification confirms the high quality of the translations in \ourdataset.
% 图1列出了我们数据集中不同语言的一个示例,包括表格,文字和问题
% Figure~\ref{fig:intro} presents an example across languages in \ourdataset.

% 介绍方法
% 为了提升模型在不同语言上的TATQA的性能,我们提出了一个强baseline,弥补模型在非英语语言的TATQA上和英语TATQA任务之间性能的差距
To enhance the performance on \ourdataset of non-English languages, we propose a baseline to bridge the performance gap between English and non-English on TATQA (\ourmethod). 
% 考虑到TATQA任务需要模型根据问题从混合的上下文中定位相关信息的能力,以及数值推理的能力,而这些能力模型在非英语语言上相比英语较差,所以我们的方法分为两部分:Linking和Reasoning
% Considering that the TATQA task requires the model to link relevant information in the hybrid data based on the question, as well as to perform complex reasoning, both of which are more challenging for non-English languages compared to English \cite{MultiSpider,shi2023MGSM}, \ourmethod is divided into two parts: Linking and Reasoning. 
% 为了将模型在英语上的TATQA能力对齐到其他语言上,我们的baseline分为两部分:链接非英语信息和用英语推理
To align the model TATQA capabilities in English with other languages, especially low-resource languages, \ourmethod is divided into two modules: linking non-English information and reasoning in English. 
% 具体来说,模型首先通过Linking从其他语言的表格和文字中定位到和问题相关的信息,再根据这些相关信息用英文推理生成代码求解
% Specifically, \ourmethod first identifies relevant information from both tables and text in Linking and then generates programs to solve the question based on the information in Reasoning.
Specifically, \ourmethod first identifies relevant information from tables and text according to the entities in the question through linking and then uses this information to perform reasoning in English by generating programs.


% 实验结果
% 我们评估了一系列baseline在我们数据集上的性能
We evaluate the performance of \ourmethod, compared with a series of baselines on \ourdataset. 
% 实验结果表明,非英语相比较英语的性能平均下降%,证明了多语言TATQA数据集的必要性
Experimental results indicate that the performance of non-English languages drops by an average of $19.4\%$ compared to English on all baselines, highlighting the necessity of \ourdataset. 
% 我们的方法相比其他baseline平均提升%,弥补了%非英语和英语之间性能的差距,证明了我们方法的有效性
\ourmethod outperforms other baselines by an average of $3.3$, demonstrating the effectiveness. 
% , bridging $9.1\%$ of the performance gap between non-English and English, demonstrating the effectiveness. 
% 即使提升,所有baseline的EM均小于32,进一步证明了我们数据集的挑战性
% Despite the improvements, all baselines demonstrate suboptimal performance, further underscoring the challenges of \ourdataset. 
% 分析实验表明,模型在不同语言上的TATQA的能力不仅和语言的高低资源有关,还与语言的特性有关
Analysis experiments reveal that the TATQA capabilities across languages are not only influenced by resource availability but also by their specific linguistic characteristics. 
% 错误分析表明,模型在非英语TATQA上性能的下降主要是因为模型在其他语言上对相关信息的查找、公式的运用,以及instruction following的能力下降
Error analysis shows that the performance decline in non-English TATQA is primarily due to the reduced ability to link relevant information, apply formulas, and follow instructions.


% 贡献
% 本文的贡献如下
Our contributions are as follows:
\begin{enumerate}
    % 1. 据我们所知,我们提出了首个多语言TATQA数据集,共包含11种语言
    \item To the best of our knowledge, we introduce the first multilingual TATQA dataset \ourdataset, which includes $11$ diverse languages.
    % 2. 我们提出了我们的方法,一个baseline将模型在英语上的TATQA能力对齐到非英语语言上
    \item We propose \ourmethod, a baseline to align the model TATQA capabilities in English to non-English languages.
    % 3. 我们做了一系列系统的实验,以实验结果和错误分析证明了我们数据集的挑战,为后续方法提高指明方向
    \item We conduct a series of experiments, supported by empirical results and error analysis, to demonstrate the challenges of \ourdataset and provide insights for future improvements.
\end{enumerate}




