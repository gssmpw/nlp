\subsection{Multilingual Datasets}
% 为了评测模型在不同语言上的性能,有很多在不同任务上的多语言数据集被提出,比如,QA,自然语言推理,文字总结,数值推理,代码生成,可读性等
To evaluate the performance of models across different languages, several multilingual datasets have been proposed for different tasks, such as question answering \cite{liu-etal-2019-xqa,clark-etal-2020-tydiQA,longpre-etal-2021-mkqa}, natural language inference \cite{conneau-etal-2018-xnli}, text summarization \cite{giannakopoulos-etal-2015-multiling-Summarization,ladhak-etal-2020-wikilingua,scialom-etal-2020-mlsum}, numerical reasoning \cite{shi2023MGSM}, code generation \cite{peng-etal-2024-humanevalxl}, text-to-SQL \cite{MultiSpider}, and readability \cite{trokhymovych-etal-2024-open-Readability,naous-etal-2024-readme}, among others. 
% 还有很多多语言数据集收集了不同的任务
Additionally, numerous multilingual datasets have been collected for different tasks \cite{hu-2020-XTREME,ruder-etal-2021-xtremer,zhang2024pmmeval-multitask,singh-etal-2024-indicgenbench}. 
% 但是目前为止,仍然没有多语言TATQA数据集,导致缺乏关于模型多语言TATQA能力的评测与分析
However, to date, there is no multilingual TATQA dataset, resulting in a lack of evaluation and analysis of multilingual TATQA capabilities and a gap with real scenarios. 
% 所以,我们本文提出了多语言TATQA数据集,并详细分析了多语言TATQA的挑战
Therefore, we introduce \ourdataset, a multilingual TATQA dataset, and provide a detailed analysis of the challenges in multilingual TATQA.

\subsection{QA Datasets for the Table and Text}
% 目前,在表格和文本上的QA数据集主要集中于单一语言
Currently, QA datasets for the table and text primarily focus on a single language. 
% 比如,HybridQA从Wikipedia收集英文的表格和相关段落,共包含70K个人工标注的问题和答案
For instance, HybridQA~\cite{chen-etal-2020-hybridqa} collects English tables and associated text from Wikipedia.
% , containing $70$K manually annotated question-answer pairs. 
% TAT-QA和FinQA和DOCMATH-EVAL主要关注于金融领域的数值计算问题
TAT-QA~\cite{zhu-etal-2021-tat}, FinQA~\cite{chen-etal-2021-finqa}, DOCMATH-EVAL~\cite{zhao-etal-2024-docmath}, and FinanceMATH~\cite{zhao-etal-2024-FinanceMATH} focus on numerical computation in the financial domain, and SciTAT~\cite{zhang2024scitat} addresses questions based on tables and text from English scientific papers. 
% SciTAT则关注于根据英文论文中的表格和文本回答用户问题
% 然而,单一语言的数据集无法进一步探索TATQA的挑战,无法全面评测模型的多语言TATQA的能力,并且和现实场景存在较大差距
However, single-language datasets cannot evaluate the multilingual TATQA capabilities, and overlook the diverse languages in real scenarios. 
% 所以我们提出了我们的数据集:首个多语言TATQA数据集,涉及包括英语的11种语言,8个语系
So we propose \ourdataset: the first multilingual TATQA dataset, involving $11$ languages and $8$ language families.
% 我们的数据集和前人工作的对比表格如表所示
A comparison of \ourdataset and prior works is presented in Appendix~\ref{sec:comparison}.


% 目前关于增强TATQA性能的工作主要集中于检索和生成两阶段
The current works on enhancing TATQA performance primarily focus on retrieving relevant information from the context \cite{luo2023hrot,bardhan2024ttqars,glenn-etal-2024-blendsql} and generating programs, equations, or step-by-step reasoning process to derive the final answer \cite{tonglet-etal-2023-seer,TAT-LLM,fatemi2024three-agent}.
% 检索,即设计检索器或直接使用LLM从上下文中抽取相关信息
% Retrieval involves designing a retriever to extract relevant information from the context.
% 生成,即生成代码、等式或逐步推理得到最终答案
% Generation refers to generating programs, equations, or step-by-step reasoning to derive the final answer.
% 比如S3HQA关注于检索阶段,首先训练检索器初步从上下文中检索到相关的表格行和文本,再根据问题的分类进一步选取出相关上下文
For example, S3HQA~\cite{lei-etal-2023-s3hqa} emphasizes retrieving, where a retriever is initially trained, followed by further filtering based on the question type. 
% 而Hpropro关注于生成阶段,通过提供给LLM一些常用函数,方便LLM生成代码时直接调用,且提示LLM根据代码执行错误的信息修改
Hpropro~\cite{shi-etal-2024-hpropro} focuses on generating, providing LLMs with commonly used functions to facilitate direct invocation during code generation.
% 然而,这些方法都是针对单一语言设计的,直接应用于其他语言会导致性能下降
However, previous methods are designed for single-language scenarios, directly used to other languages could lead to performance degradation.
% 所以,我们提出了一个多语言的baseline:我们的方法,实现跨语言的链接和推理
To address this, we propose \ourmethod, a multilingual baseline that aligns the English TATQA capabilities to other languages. 
% 并且我们详细分析了语言对不同答案类型性能的影响,以及在多语言场景下,instruction, 示例,表格文本以及问题的语言对性能的影响,
% We also provide a detailed analysis of how different languages impact performance across various answer types. 
% Furthermore, we examine the effects of language for instructions, demonstrations, tables, text, and questions in multilingual settings. 
% 我们还分析了在我们数据集上非英语语言相比英语的性能下降的原因,为未来多语言TATQA的研究指明了方向
% Additionally, we analyze the reasons for the performance decline in non-English languages compared to English on \ourdataset, providing directions for future research in multilingual TATQA.
% We also provide a detailed analysis of how different languages impact performance, providing directions for future research in multilingual TATQA.