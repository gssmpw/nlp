% This must be in the first 5 lines to tell arXiv to use pdfLaTeX, which is strongly recommended.
\pdfoutput=1
% In particular, the hyperref package requires pdfLaTeX in order to break URLs across lines.

\documentclass[11pt]{article}

% Change "review" to "final" to generate the final (sometimes called camera-ready) version.
% Change to "preprint" to generate a non-anonymous version with page numbers.
% \usepackage[review]{acl}
\usepackage{acl}

% Standard package includes
\usepackage{times}
\usepackage{latexsym}

% For proper rendering and hyphenation of words containing Latin characters (including in bib files)
\usepackage[T1]{fontenc}
% For Vietnamese characters
% \usepackage[T5]{fontenc}
% See https://www.latex-project.org/help/documentation/encguide.pdf for other character sets

% This assumes your files are encoded as UTF8
\usepackage[utf8]{inputenc}

% This is not strictly necessary, and may be commented out,
% but it will improve the layout of the manuscript,
% and will typically save some space.
\usepackage{microtype}

% This is also not strictly necessary, and may be commented out.
% However, it will improve the aesthetics of text in
% the typewriter font.
\usepackage{inconsolata}

%Including images in your LaTeX document requires adding
%additional package(s)
\usepackage{graphicx}

% If the title and author information does not fit in the area allocated, uncomment the following
%
%\setlength\titlebox{<dim>}
%
% and set <dim> to something 5cm or larger.
\usepackage{inconsolata}
\usepackage{graphicx}

\usepackage{enumitem}
\usepackage{multirow}
\usepackage{booktabs}
\usepackage{algorithm}
\usepackage{algorithmic}
\usepackage{tikz}
\usepackage{pgfplots}
\usepackage{graphicx}
\usepackage{subcaption}
\usepackage{pgf-pie}
\usepackage{enumitem}
\usepackage{pifont}
\usepackage{amsmath}
\usepackage{supertabular}
\usepackage{booktabs}
\numberwithin{equation}{section}
\usepackage{dsfont}
\usepgfplotslibrary{polar}


% \usepackage{inconsolata}
% \usepackage{ctex}
% \usepackage{algcompatible}
\pgfplotsset{compat=1.18}
\usepackage{bm}
\usepackage{xspace}
\usepackage{tcolorbox}
\usepackage{bbding}
\usepackage{wasysym}
\usepackage{amssymb}
\usepackage{fontawesome5}
\usepackage{arydshln}  
\usepackage{dashrule}


\newcommand{\ourmethod}{\textsc{Ours}\xspace}
\newcommand{\ourdataset}{\textsc{MultiTAT}\xspace}
\definecolor{cpurple}{rgb}{0.675, 0.573, 0.922}
\definecolor{cblue}{rgb}{0.310, 0.757, 0.910}
\definecolor{cgreen}{rgb}{0.310, 0.563, 0.214}
\definecolor{corange}{rgb}{1, 0.808, 0.329}
\definecolor{cred}{rgb}{0.8, 0.3, 0.3}
\definecolor{data_blue}{rgb}{0.809, 0.883, 0.949}
\definecolor{data_blue_light}{rgb}{0.930, 0.965, 1}
\definecolor{data_blue_dark}{rgb}{0.027, 0.215, 0.387}
\definecolor{annotator_pink}{rgb}{0.914, 0.816, 0.859}
\definecolor{gray}{rgb}{0.715, 0.715, 0.715}
\definecolor{gray_light}{rgb}{0.949, 0.949, 0.949}
% \definecolor{annotator_pink}{rgb}{0.297, 0.066, 0.188}
\definecolor{reasoner_green}{rgb}{0.848, 0.914, 0.824}
\newcommand{\datatext}[1]{{\color{data_blue_dark}#1}}
\newcommand{\textgray}[1]{{\color{gray}#1}}
\newcommand{\annotatortext}[1]{{\color{annotator_pink}#1}}
\newcommand{\greentext}[1]{{\color{reasoner_green}#1}}
% \newcommand{\textblue}[1]{{\color{softblue}#1}}
\providecommand{\longxu}[1]{{\protect\color{magenta}{[longxu: #1]}}}

\newtcolorbox{myexample}[2][]{
  colback=data_blue!40,
  colframe=data_blue,         % 设置边框颜色
  coltitle=black,
  title=\textbf{#2},
  fonttitle=\bfseries,
  #1,
}


% If the title and author information does not fit in the area allocated, uncomment the following
%
%\setlength\titlebox{<dim>}
%
% and set <dim> to something 5cm or larger.

\title{\ourdataset: Benchmarking Multilingual Table-and-Text\\Question Answering}
% MultiTAT: A Multilingual Benchmark for Question Answering on Table and Text
% MultiTAT: Benchmarking Multilingual Question Answering on Table and Text

% Author information can be set in various styles:
% For several authors from the same institution:
% \author{Author 1 \and ... \and Author n \\
%         Address line \\ ... \\ Address line}
% if the names do not fit well on one line use
%         Author 1 \\ {\bf Author 2} \\ ... \\ {\bf Author n} \\
% For authors from different institutions:
% \author{Author 1 \\ Address line \\  ... \\ Address line
%         \And  ... \And
%         Author n \\ Address line \\ ... \\ Address line}
% To start a seperate ``row'' of authors use \AND, as in
% \author{Author 1 \\ Address line \\  ... \\ Address line
%         \AND
%         Author 2 \\ Address line \\ ... \\ Address line \And
%         Author 3 \\ Address line \\ ... \\ Address line}
\author{
    Xuanliang Zhang\footnotemark[2], Dingzirui Wang\footnotemark[2], Keyan Xu, Qingfu Zhu, Wanxiang Che\\ 
    \texttt{\{xuanliangzhang, dzrwang, kyxu, qfzhu, car\}@ir.hit.edu.cn}\\
    Harbin Institute of Technology 
}

% \iffalse
% \author{First Author \\
%   Affiliation / Address line 1 \\
%   Affiliation / Address line 2 \\
%   Affiliation / Address line 3 \\
%   \texttt{email@domain} \\\And
%   Second Author \\
%   Affiliation / Address line 1 \\
%   Affiliation / Address line 2 \\
%   Affiliation / Address line 3 \\
%   \texttt{email@domain} \\}

\begin{document}
% \nolinenumbers
    \maketitle

\renewcommand{\thefootnote}{\fnsymbol{footnote}}
% \footnotetext[2]{The two authors contribute equally to this work.}
\footnotetext[2]{Equal contribution.}
% \footnotetext[1]{Corresponding author.}
\renewcommand{\thefootnote}{\arabic{footnote}}
    \begin{abstract}
    % TATQA是一个很重要的任务,旨在根据表格和文本的混合数据回答问题,广泛用于数据密集的领域
    Question answering on the hybrid context of tables and text (TATQA) is a critical task, with broad applications in data-intensive domains.
    % 然而,现在的TATQA数据集只局限于英语,存在以下缺点:
    % 1. 忽略了多语言TATQA的挑战,不能评测模型多语言TATQA的性能
    % 2. 和存在大量非英文的表格和文本的实际场景存在较大差距
    % However, existing TATQA datasets are limited to English, overlooking the challenges of multilingual TATQA, such as linguistic characteristics and variations in model performance across languages.
    However, existing TATQA datasets are limited to English, leading to several drawbacks:
    (i)~They overlook the challenges of multilingual TAT-QA and cannot assess model performance in the multilingual setting.
    (ii)~They do not reflect real-world scenarios where tables and texts frequently appear in non-English languages.
    % 所以,本文提出首个多语言TATQA数据集,来评测模型在多语言TATQA上的能力
    To address the limitations, we propose the first multilingual TATQA dataset (\ourdataset).
    % to evaluate TATQA capabilities in the multilingual setting.
    % 首先,我们从主流TATQA数据集中sample数据,并将其翻译为10种多样的语言
    Specifically, we sample data from $3$ mainstream TATQA datasets and translate it into $10$ diverse languages.
    % 为了提升模型在多语言TATQA任务上的性能,我们提出一个强baseline,能够提升模型处理不同语言的表格和文本中的信息的能力
    % To enhance performance on the multilingual TATQA task, we develop a baseline, \ourmethod, to align the model TATQA capabilities in English with other languages.
    To align the model TATQA capabilities in English with other languages, we develop a baseline, \ourmethod.
    % 实验结果表明,模型在我们数据集上的非英语相比英语上的性能平均下降18.9%,我们对此分析并总结了下降的原因
    Experimental results reveal that the performance on non-English data in \ourdataset drops by an average of $19.4\%$ compared to English, proving the necessity of \ourdataset.
    We further analyze the reasons for this performance gap.
    % 而我们的方法相比其他baseline平均提升2.6%,证明了我们方法的有效性
    Furthermore, \ourmethod outperforms other baselines by an average of $3.3$, demonstrating its effectiveness\footnote{Our data is available at \href{https://github.com/zhxlia/MULTITAT}{github.com/zhxlia/MULTITAT}}.


    % Question answering on the hybrid context of tables and text (TATQA) is a critical task, with broad applications in data-intensive domains. However, existing TATQA datasets are limited to English, leading to several drawbacks: (i) They overlook the challenges of multilingual TAT-QA and cannot assess model performance in the multilingual setting. (ii) They do not reflect real-world scenarios where tables and texts frequently appear in non-English languages. To address the limitations, we propose the first multilingual TATQA dataset (MULTITAT). Specifically, we sample data from 3 mainstream TATQA datasets and translate it into 10 diverse languages. To align the model TATQA capabilities in English with other languages, we develop a baseline, Ours. Experimental results reveal that the performance on non-English data in MULTITAT drops by an average of 19.4% compared to English, proving the necessity of MULTITAT. We further analyze the reasons for this performance gap. Furthermore, Ours outperforms other baselines by an average of 3.3, demonstrating its effectiveness.


    \end{abstract}


    \section{Introduction}
        \section{Introduction}

\begin{figure}[h]
    \centering
    \begin{overpic}[trim=0cm 0cm 0cm 0cm,clip,angle=0,origin=c,width=.4\linewidth]{images/teaser_absolute.png}
        %  trim={<left> <lower> <right> <upper>}
        %  \put(horiz, vert)
        %  \put(horiz, vert){\rotatebox{90}{Text}}
        %
        \put(107, 32){$\mathbf{\to}$}
    \end{overpic}\hspace{1cm}
    \begin{overpic}[trim=0cm 0cm 0cm 0cm,clip,angle=0,origin=c,width=.4\linewidth]{images/teaser_translated_yellow.png}
        %  trim={<left> <lower> <right> <upper>}
        %  \put(horiz, vert)
        %  \put(horiz, vert){\rotatebox{90}{Text}}
        %
    \end{overpic}
    \caption{Using translation methods, a controller trained on an environment with a given visual variation \textit{(left)} can be reused without any training or fine-tuning on a different environment (\textit{right}) with comparable performance. In red we see the trajectory of a car driven by the same controller when connected to two different encoders, one for each visual variation.
    }
    \label{fig:teaser}
\end{figure}

Deep Reinforcement Learning (RL) has enabled agents to achieve remarkable performance in complex decision-making tasks, from robotic manipulation to high-dimensional games (Mnih et al., 2015; Silver et al., 2017). 
Although recent RL techniques achieved strong improvements over sample efficiency \citep{yarats2021drqv2, kostrikov2020image}, training new agents remains a costly process, both in computational and temporal terms.
Despite these advances, most methods still require at least partial retraining when dealing with domain shifts such as visual appearance, reward functions, or action spaces \citep{pmlr-v97-cobbe19a, zhang2020learning}. These domain changes typically require expensive retraining, which can be prohibitive for real-world settings that require millions of interactions.

A variety of approaches have been proposed to address these shifting conditions. Domain randomization \citep{tobin2017domain, sadeghi2016cad2rl} trains agents across diverse visual styles or physics settings, promoting invariant features but demanding broader coverage of possible variations. Multi-task RL \citep{parisotto2015actor, teh2017distral} attempts to learn shared representations across multiple tasks.

In the supervised setting, recent representation learning techniques \citep{Moschella2022-yf,maiorca2023latent, norelli2022b, cannistraci2023bricks}, show that it is possible to zero-shot recombine encoders and decoders to perform new tasks across different modalities (images, text..) and tasks (classification, reconstruction) and even architectures.
In RL, methods adopting the relative representation framework \citep{Moschella2022-yf} have shown promising results in adapting encoders to different controllers with zero or few-shots adaptation, for robotic control from proprioceptive states \citep{jian2021adversarial} or for playing games in the Gymnasium suite \citep{towers2024gymnasium} from pixels \citep{ricciardi2025r3lrelativerepresentationsreinforcement}.
These methods, however, still require training models to use the new relative representations.

By contrast, \cite{maiorca2023latent} suggest that modules from independently trained neural networks can be connected via a simple linear or affine transformation, with no training constraint or fine-tuning required, if such transformations can be reliably estimated from a small set of “anchor” samples, pairs of states or observations deemed semantically equivalent.

Our main contribution is the implementation of a RL method based on semantic alignment to map between latent spaces of different neural models, so that their encoders and controllers can be stitched with the goal of creating new agents that can act on visual-task combinations never seen together in training. This includes the use of the transformations to map modules from different networks, and the collection of anchor samples used to estimate these transformations. We call our method Semantic Alignment for Policy Stitching (\textbf{SAPS}).
We perform analyses and empirical tests on the CarRacing and LunarLander environments to show the performance of new agents created via zero-shot stitching of encoders and controllers trained on different visual-task variations, demonstrating significant gains compared to existing zero-shot methods.
    
    \section{\ourdataset}
        \label{sec:dataset}
        % 我们数据集的输入为表格、文本和相关的问题,输出为问题的回答
The input of \ourdataset consists of a question, the hybrid context including the table and text, and the output is the answer to the question. 
% 并且,我们为每个问题标注了rationale
Additionally, we annotate the rationale, which is the reasoning process of answering the question in \ourdataset. 
% 我们称每个问题,以及它对应的表格、文本、rationale和答案为一个instance
We refer to each question, along with its corresponding table, text, rationale, and answer, as an instance. 
% 我们为每个instance标注了10种非英语的多样的语言
For each instance, we annotate $11$ diverse languages.
% 首先,我们介绍我们数据集的构建过程,采用了自动翻译和人工纠错相结合的方式,following前人工作,如图所示
We first describe the construction process of \ourdataset, which combines automatic generation with manual error correction, following previous works \cite{peng-etal-2024-humanevalxl,singh-etal-2024-indicgenbench,MultiSpider}, as shown in Figure~\ref{fig:framework}.

\begin{figure*}
    \centering
    \includegraphics[width=.85\linewidth]{fig/framework.pdf}
    % \vspace{-0.5em}
    \caption{
    % 我们数据集构建的流程
    The process of constructing \ourdataset.
    % 蓝色框代表数据,白色实线框代表构造步骤
    The \colorbox{data_blue_light}{\datatext{blue}} boxes represent the data, and the white solid boxes represent the construction steps.
    }
    \label{fig:framework}
\end{figure*}

\begin{table*}[t]
    \centering
    \small
    \begin{tabular}{@{}l l l l c c c c@{}}
\toprule
\multirow{2}{*}{\textbf{Dataset}} & \multirow{2}{*}{\textbf{Domain}} & \multirow{2}{*}{\textbf{Scale}} & \multirow{2}{*}{\textbf{Answer Type}} & \multicolumn{3}{c}{\textbf{Answer Source}} & \multirow{2}{*}{\textbf{Total}}\\ 
\cmidrule(lr){5-7}
 & & & & \textbf{Text} & \textbf{Table} & \textbf{Hybrid} \\
\midrule
HybridQA~\cite{chen-etal-2020-hybridqa} & Wikipedia & $50$ & Span & $0$ & $0$ & $50$ & $50$ \\
\midrule
\multirow{3}{*}{TAT-QA~\cite{zhu-etal-2021-tat}} & \multirow{3}{*}{Finance} & \multirow{3}{*}{$100$} & Span & $10$ & $10$ & $20$ & $40$ \\ 
 & & & Arithmetic & $10$ & $10$ & $30$ & $50$ \\ 
 & & & Count & $2$ & $3$ & $5$ & $10$ \\
\midrule
\multirow{2}{*}{SciTAT~\cite{zhang2024scitat}} & \multirow{2}{*}{Science} & \multirow{2}{*}{$100$} & Span & $10$ & $20$ & $20$ & $50$ \\
 & & & Arithmetic & $10$ & $20$ & $20$ & $50$ \\
\midrule
Total & - & $250$ & - & $42$ & $63$ & $145$ & $250$ \\
\bottomrule
\end{tabular}
    % \vspace{-0.5em}
    \caption{
    % 我们数据集中数据的分布
    The distribution of English data, including answer types and answer sources in \ourdataset, sourced from three mainstream datasets.
    % 表中列出的是每个数据集拥有的所有答案类型
    The listed answer types are the all answer types corresponding to each dataset.
    }
    \label{tab:answer_statistics}
\end{table*}


\subsection{Data Preparation}
We first collect English data from existing datasets and select languages to translate them.

\subsubsection{Source Data Collection}
% 我们分别选取了wikipedia,金融和科学领域的HybridQA, TAT-QA和SciTAT数据集作为我们的数据来源,因为这三个领域是目前TATQA任务主要分布的领域
We select HybridQA~\cite{chen-etal-2020-hybridqa}, TAT-QA~\cite{zhu-etal-2021-tat}, and SciTAT~\cite{zhang2024scitat} datasets from the Wikipedia, finance, and science domains as our data sources, as these three domains are the primary areas where TATQA tasks are currently distributed (see Table~\ref{tab:comparison_tat}). 
% 为了令我们的数据集在不同的答案来源和答案类型上均匀分布,我们按照表格中的比例从三个数据集共采样了250条数据,如表所示
To ensure an even distribution of different answer types and answer sources in \ourdataset, we sample a total of $250$ instances from the three datasets according to the proportions shown in Table~\ref{tab:answer_statistics}. 
% 其中,HybridQA数据集只采样了50条是由于其答案来源和类型较为单一
Among them, only $50$ instances are sampled from HybridQA due to its relatively limited answer sources and types. 

\subsubsection{Target Language Selection}
% 我们选取了除英语之外的10种语言覆盖8个语系,分别是Bengali (BN), Chinese (ZH), French (FR), German (DE), Japanese (JA), Russian (RU), Spanish (ES), Swahili (SW), Telugu (TE), and Thai (TH),follow前人工作.
For \ourdataset, we select $11$ languages, covering $8$ language families: Bengali (bn), Chinese (zh), English (en), French (fr), German (de), Japanese (ja), Russian (ru), Spanish (es), Swahili (sw), Telugu (te), and Thai (th), following the previous benchmark \cite{shi2023MGSM}. 
% 并且,我们在所有语言中都保持了原数据集中答案的阿拉伯数字,以便评测(引)
Additionally, we preserve the Arabic numerals from the original datasets across all languages to facilitate evaluation \cite{shi2023MGSM}.

\subsection{Rationale Annotation}
\label{subsec:Rationale Annotation}
% \paragraph{Rationale Generation}
% 在本节,我们首先展示了如何使用LLM结合人工修改来标注rationale
We first demonstrate how to annotate English rationales by employing the large language model (LLM) in combination with manual refinement. 
% 我们首先使用gpt-4o完成英文rationale的标注,因为其强大的推理能力以及instruction-following的能力
We use \texttt{gpt-4o}~\cite{openai2024gpt4technicalreport} to complete \textbf{rationale generation} due to its strong reasoning and instruction-following capabilities. 
% 具体地,我们将问题,相关的表格和文本以及答案输入LLM,提示模型生成推理过程。
Specifically, we input the question, relevant tables and texts, and the answer into the LLM, prompting the LLM to generate the corresponding rationale.
% \paragraph{Human Refinement}
% 由于LLM不能保证推理的正确性,我们采用人工校验及修改
Since LLMs cannot guarantee the accuracy of reasoning, we employ \textbf{manual refinement}. 
% 对于模型生成的推理过程,我们提示标注者判断英语rationale的正确性,对于错误的进行修改
The annotators are instructed to evaluate the accuracy of the generated rationale and make corrections where necessary.

\subsection{Instance Translation}
\label{subsec:Instance Translation}
% 在本节,我们介绍使用LLM结合人工标注将英文instance翻译为10种语言
In this section, we describe the process of combining the LLM with human annotations to translate English instances into $10$ languages. 
% \paragraph{Machine Translation}
% 我们选择gpt-4o,因为其强大的翻译能力
For \textbf{machine translation}, we select \texttt{gpt-4o} because of its strong translation capabilities \cite{yan2024gpt4vshumantranslators,hu-etal-2024-gentranslate}. 
% 我们将instance输入LLM,分别提示模型翻译为10种目标语言
Specifically, we input each instance into the LLM, with prompts to translate it into the target languages, respectively. 
% \paragraph{Human Refinement}
% 为了评测翻译的正确性,我们follow前人工作使用gpt-4o将翻译成目标语言的instance翻译回英语,计算其和原本英语版本的F1
To assess the accuracy of the translations, we use \texttt{gpt-4o} to translate the target language instances back into English, and calculate the F1 score between the back-translated version and the original English instance following previous works \cite{peng-etal-2024-humanevalxl}. 
% 对于F1<0.6的instance,我们提示标注者使用谷歌翻译重新翻译并检查,直至回译和原始英文版本一致
For instances with an F1 score below $0.6$, we prompt annotators to complete \textbf{manual refinement} by using Google Translation for a new translation.
% and verification, iterating until the back-translated instances align with the original English version.

% \subsection{Initial Annotation}
% % 我们首先使用LLM完成英文rationale的标注,以及将英文instance翻译为10种语言
% We first employ large language models (LLMs) to annotate the English rationales and translate the English instances into ten languages. 
% % 具体地,我们使用的是gpt-4o,因为其强大的推理能力,instruction-following的能力,以及翻译能力
% Specifically, we use \texttt{gpt-4o}~\cite{openai2024gpt4technicalreport}, given its strong reasoning ability, proficiency in instruction-following, and translation capabilities.
% % 1. 对于rationale的标注,我们将问题,相关的表格和文本以及答案输入LLM,提示模型生成推理过程。
% (\emph{i})~For rationale annotation, we input the question, relevant tables and text, and the answer into the LLM, prompting the model to generate the reasoning process. 
% % 在经过人工校验及修改后(过程见S2.3),我们将每个instance翻译为目标语言。
% After manual verification and modification (as described in Section \S\ref{subsec:Human Annotation}), we translate each instance into the target languages.
% % 2. 对于instance的翻译,我们将表格,文本,问题,rationale和答案都输入LLM,分别提示模型翻译为10种目标语言
% (\emph{ii})~For instance translation, we input the tables, text, question, rationale, and answer into the LLM, prompting the model to translate them into the target languages, respectively.
% % 我们标注过程中使用的prompt在附录中提供
% % The prompts used during the annotation process are provided in Appendix.

% \subsection{Human Refinement}
% \label{subsec:Human Refinement}
% % 由于LLM不能保证推理以及翻译的正确性,我们采用人工校验及修改
% Since LLMs cannot guarantee the accuracy of reasoning and translation, we employ manual verification and correction. 
% % 对于推理过程,我们提示标注者判断英语rationale的正确性,对于错误的进行修改
% (\emph{i})~For the rationale, we instruct the annotators to assess the correctness of the English rationale, making corrections where necessary. 
% % 对于翻译过程,我们follow前人工作采用回译进行判断。
% (\emph{ii})~For the translation, we follow previous works by using back-translation for validation \cite{peng-etal-2024-humanevalxl}. 
% % 具体来说,我们使用gpt-4o将翻译成目标语言的instance翻译回英语,计算其和原本英语版本的F1
% Specifically, we utilize \texttt{gpt-4o} to translate target language instances back into English and compute the F1 score between the back-translated version and the original English version. 
% % 对于F1<0.6的instance,我们提示标注者使用谷歌翻译重新翻译并检查,直至回译和原始英文版本一致
% For instances with an F1 score below $0.6$, we instruct the annotators to use Google Translation for a new translation and review it until the back-translated version matches the original English text.

\subsection{Quality Control}
% 为了保证我们的数据集中数据的质量,我们采取了严格的质量控制策略
To ensure the quality of \ourdataset, we implement rigorous quality control strategies.
% 首先,回译是广泛被前人采用的评价翻译质量的方法。
% First, back-translation is a widely adopted method for evaluating translation quality. 
% 并且,我们的方法是结合人工校验的gpt-4o和谷歌翻译的融合学习的结果,有着优越的翻译性能
% Our translation process, which combines the results of human-validated \texttt{gpt-4o} and Google Translation through ensemble learning, demonstrates superior translation performance. 
% \paragraph{Competent Annotators}
% 我们雇佣的标注者均为研究生及以上学历,精通英语,并支付给他们每条数据$1的费用
The annotators we hire hold graduate-level degrees, are proficient in English and are compensated with $\$1$ per data instance. 
% 我们首先对标注者进行培训,使其了解标注的要求以及标注工具的使用(见附录)
We first train the annotators to familiarize them with the annotation requirements and the use of the annotation tool (see Appendix~\ref{subsec:Annotator Training Process}). 
% 然后,我们令其试标注20条数据,检查他们标注的结果并给出反馈以及修改意见
Then, they try to annotate $20$ instances, and we review their annotations, providing feedback and suggestions for revisions.

% \paragraph{Quality Inspection}
% 为了进一步确保翻译的质量,我们选择了低资源语言,包括孟加拉语、斯瓦希里语以及泰卢固语,对我们的数据集的翻译进行评测
% To further ensure the quality of the translations, we select low-resource languages, including Bengali, Swahili, and Telugu, to evaluate the translations in \ourdataset. 
% % 具体来说,我们分别请了母语为这三种语言且精通英语的标注者分别对这些语言在我们数据集中的问题,对照英文的原始问题对翻译的流畅性、充分性和一致性进行打分
% Specifically, native speakers of these languages, who are also proficient in English, are asked to rate the fluency, adequacy, and consistency of the translated questions by comparing them to the original English questions in \ourdataset. 
% % 最终平均得分分别为,证明了我们数据集翻译的质量
% The final average scores is , demonstrating the high quality of \ourdataset. 
% % 我们将具体分值放在了附录
% Detailed annotation process and scoring information are provided in Appendix.

% \begin{table}[t]
%     \centering
%     \small
%     \begin{tabular}{@{}lc@{}}
\toprule
%  & \# \\ 
% \midrule
Question       & $250$   \\
Hybrid Contexts & $233$   \\
Avg. Rows      & $10.2$  \\
Avg. Columns   & $4.7$   \\
Avg. Paragraphs & $5.3$   \\
\bottomrule
\end{tabular}
%     \vspace{-0.5em}
%     \caption{
%     % 我们数据集中数据的分布
%     The basic statistics of \ourdataset.
%     }
%     \label{tab:basic_statistics}
% \end{table}

\subsection{Data Analysis}
% 我们展示了我们数据集的基本数据的统计结果,如表1和表2所示
We show the data distribution of \ourdataset in Table~\ref{tab:answer_statistics}.
% 我们数据集中的250个问题涉及233个混合上下文,其中每个混合上下文包括1个表格和平均5.3个段落。其中每个表格平均有10.2行和4.7列
The $250$ questions in \ourdataset involve $233$ hybrid contexts, each of which includes $1$ table and an average of $5.3$ paragraphs. 
Each table has an average of $10.2$ rows and $4.7$ columns.

\begin{figure*}[t]
    \centering
    \includegraphics[width=.95\linewidth]{fig/method.pdf}
    % \vspace{-0.5em}
    \caption{
    % 我们方法的示意图,我们方法分为两步:
    The overview of \ourmethod, which includes two modules:
    % 1. Linking根据问题链接到表格或文本中相关的信息
    (\emph{i})~\textbf{Linking}: Mapping the entities in the question to the relevant information in tables or text, which are marked with \datatext{\textbf{blue}} in the left part.
    % 2. Reasoning根据相关信息生成代码求解
    (\emph{ii})~\textbf{Reasoning}: Generating programs to solve the question using the information.
    % 例子和图一中的例子一致
    % The example is consistent with the Chinese one shown in Figure~\ref{fig:intro}.
    % 我们以中文TATQA的输入为例,括号中的灰色文字为对应的英文
    We take the Chinese TATQA input as an example, with the corresponding English text provided in \colorbox{gray_light}{\textgray{(gray)}}.
    % within parentheses.
    }
    \label{fig:method}
\end{figure*}


    
    \section{\ourmethod}
        \label{sec:methodology}
        


\section{Methodology}\label{sec:method}

\begin{figure}[!h]
\vspace{-2mm}
\centering
\includegraphics[width=1\linewidth]{fig/overview_syh.jpg}
\vspace{-5mm}
\caption{\tool Overview }
\label{fig:tool_overview}
\vspace{-1mm}
\end{figure}

%\shil{shall we provide an overview figure of the proposed framework?}
%\syh{I will work on a workflow figure if possible}
\tool (\figref{fig:tool_overview}) is a general-purpose testing framework that evaluates the LLM outputs for automatically generated test cases. 
The inputs for the response evaluation
contain a natural language (NL) query for LLM and its ground truth answer obtained using logic programming (\secref{subsec3.1}).  
Based on voluminous knowledge database dumps, \tool extracts factual knowledge (\secref{knowledge}), which outputs a set of 
predicates
in the form of Prolog facts. 
Then, \tool deploys a set of pre-defined or automatically generated reasoning rules to
extend the database with a set of derived facts (\secref{sec:derive_more_facts}, \secref{sec:encoding_MTL}). 
These derived facts facilitate an automated test generation (\secref{prompt}), which outputs question-answer pairs (Q\&A) and concrete prompts for LLMs. 
Given the Q\&A pairs and the LLM outputs, \tool evaluates the responses from LLMs and detects factual inconsistency automatically (\secref{response}). 
To this end, it 
first parses LLM outputs semantic-aware structure, and evaluates their semantic similarity to the ground truth. %Subsequently, 
Lastly, it develops similarity-based oracles that apply metamorphic testing to assess consistency against the ground truth. 



%Therefore, the response evaluation for automatically generated tests is achieved given the ground truth Q\&A pairs and the LLM outputs. 


%Lastly, to evaluate the responses from LLMs and detect factual inconsistency automatically (\secref{response}), it first parses LLM outputs semantic-aware structure. Then, it evaluates their semantic similarity to the ground truth. Subsequently, it develops similarity-based oracles that apply metamorphic testing to assess consistency against the ground truth. 






\subsection{Preliminary}
\label{subsec3.1}

\begin{figure}[!b]
{
\vspace{-2mm}
\centering
\small
$
\arraycolsep=3pt\def\arraystretch{1}
\begin{array}{@{}lrcl}
\m{(Program)}&  \Prolog &{::=  } &
\widetilde{\relation} \,\plus\plus\,   \widetilde{\drule} 
\\
\m{(Rule)} &  \m{\drule} &{ ::=  } & 
\relation ~\hornarrow~ \widetilde{body}
\\[0.3em]
\m{(Body)} & \m{body} &{  ::=  } & 
{\tt{Pos}}~ \relation
\,\mid\, {\tt{Neg}}~ \relation 
\,\mid\, \pi
  \\
\m{(Predicate)} &  \relation &{  ::=  } &
 \m{\nm}\,(\widetilde{\entity}) 
 %\text{\syh{how to link entity and term?}}
\\[0.3em]
 \m{(Pure)}  &\pi &{::=}~&
{ T }
  \mid  F
 \mid  {\m{bop}(}{t_1, t_2}{)}
 %\mid \nm(\widetilde{t})
 \mid   {{\pi_1}}  {\wedge}  \pi_2
 \mid  {{\pi_1}} {\vee} \pi_2
 \mid  \neg\pi
\\[0.3em]
 \m{(Term)}  &t &{::=}~& c 
 \mid X 
 \mid t_1{\text{\ttfamily +}}t_2
 \mid t_1\text{-}t_2
\end{array}$
\caption{A Core Syntax of Prolog}
\label{fig:Syntax_of_Prolog}
}
\end{figure}

Logic programming allows the programmer to specify the rules and facts, enabling the Prolog interpreter to infer answers to the given queries automatically. 
We define a core syntax of Prolog in \figref{fig:Syntax_of_Prolog}. 
A Prolog program consists of two parts: a set of facts ($\widetilde{\relation}$) and a set of rules ($\widetilde{\drule}$). 
Throughout the paper, we use the over-tilde notation to denote a set of items. 
For example, $\widetilde{X}$ refers to a set of variables, i.e., $\{X_1, \dots, X_n\}$. 
A fact is represented as a relational predicate with a name and a set of entity arguments, where $\nm$ is an arbitrary distinct identifier drawn from a finite set of relations. 
Entities are drawn from the knowledge database, ranging from string types (for names or events) and integers (for time points).  
A Prolog rule is a Horn clause that comprises a head predicate and a set of body predicates placed on the left and right side of the arrow symbol ($\hornarrow$).

A rule means that the left-hand side is logically implied by the right-hand side. 
The rule bodies are either positive or negative relations, corresponding to the requirements upon the presence or absence of facts. 
We use ``$\relation$'' and ``$\shortNeg\,\relation$'' as abbreviations for
``${\tt{Pos}}~\relation$'' and ``${\tt{Neg}}~\relation$'', respectively. 
Rule bodies contain pure formulae and simplified and decidable sets of Presburger arithmetic predicates over local variables. 
The Boolean values of \emph{true} and \emph{false} are respectively indicated by $T$ and $F$. 
%Other logical relations are represented using general abstract predicates over the terms, which contain the 
The binary operators $bop$ are from $\{ {<}, {\leq}, {=}, {\geq}, {>} \}$. 
Terms consist of constants (denoted by $c$), program variables (denoted by $X$ %\shil{since we use lowercase $c$ to represent constant, can we use lowercase $x$ to represent variables?}
%\syh{Upper case for variable is the Prolog convention.}
), or simple computations of terms, such as $t_1{\plus} t_2$ and $t_1\text{-}t_2$. 
%A Prolog query is executed against a database of facts. 





\subsection{Factual Knowledge Extraction}
\label{knowledge} 
%While predicates can have an arbitrary number of arguments in general, 
To facilitate an automated reasoning system, we extract the \emph{ground facts} in the structure of three-element predicates, i.e., $\m{\nm}\,(\Subj,\Obj)$, where ``$\Subj$'' (stands for $\m{subject}$) and ``$\Obj$'' (stands for $\m{object}$) are entities, and ``$\nm$'' is the name of the predicate. 
Here, we follow the convention of Prolog, where variable names must start with an uppercase letter, and any name that begins with a lowercase letter is a constant. %\shil{whether this format applies to the examples in Fig 6?}
%\syh{Fig. 6 is revised to lowercase now} 

Existing knowledge databases~\cite{freebase, DBpedia, Yago, WordNet} not only encompass a vast array of documents but also provide structured data, facilitating an ideal source for constructing a rich factual knowledge base. 
Thus, the genesis of our test cases is exclusively rooted in the entities and structured relations sourced from existing knowledge databases, ensuring a sophisticated and well-informed foundation for our testing framework. 
Specifically, we follow the categorization for entities (\figref{table:categories}) and relations (\figref{table:relations}) used by WikiPedia~\cite{DBpedia} to perform a thorough facts extraction. 
In particular, the {\small\textbf{Prop.}} (stands for properties) entry for relations guides the automated generation of reasoning rules detailed in \secref{sec:derive_more_facts}.

%as shown in \figref{table:entity_relations}, 


\begin{figure}[!b]
\vspace{-3mm}
\renewcommand{\arraystretch}{1.0}
\setlength{\tabcolsep}{2pt}
\footnotesize 
%\resizebox{\linewidth}{!}{
\begin{tabular}{l | l }
\Xhline{1.0\arrayrulewidth}
\textbf{Entity Cat.} & \textbf{Description}\\
        \Xhline{\arrayrulewidth}
        {Culture and the Arts} & Famous films, books, etc.\\ 
        % & 10,537\\
        %\hline
        {Geography and Places} & Countries, cities and locations. \\
        % & 8,806\\
        %\hline
        {Health and Fitness} & Diseases and genes. \\
        % & 179\\
        %\hline
        {History and Events} & Famous historical events, etc. \\
        % & 5,561\\
       % \hline
        {People and Self} & Important figures. \\
        % & 21,720\\
        %\hline
        {Mathematics and Logic} & Formulas and theorems. \\
        % & 141\\
       % \hline
        {Natural and Physical Sciences} & Celestial bodies and astronomy. \\
        % & 904\\
       % \hline
        {Society and Social Sciences} & Major social institutions, etc.\\ 
        % & 3,862\\
       % \hline
        {Technology and Applied Sciences} & Computer science, etc. \\
        % & 2,773\\
        \Xhline{1.5\arrayrulewidth} %添加表格底部粗线
    \end{tabular}
    %}
\caption{{Entity Categorization.}}
\label{table:categories}
\end{figure}


\begin{figure}[!b]
%\vspace{1.5mm}
\centering
\def\arraystretch{1.1}
\setlength{\tabcolsep}{2pt}
\footnotesize
%\resizebox{\linewidth}{!}{
\begin{tabular}{l | l | l}
\Xhline{1.5\arrayrulewidth}
\textbf{Relation Cat.} & \textbf{Examples}
& 
\textbf{Prop.} 
%(\figref{fig:basic_op_for_predicates})
\\
\Xhline{1.5\arrayrulewidth}
        {Noun Phrase} & 
\begin{tabular}[l]{@{}l@{}} \textit{place\_of\_birth\,(barack\_obama, honolulu).}\\ \textit{genre\,(28\_days\_later, horror\_film).} \end{tabular}
        &
\begin{tabular}[l]{@{}l@{}} 
        $\RNeg$ \\
        $\RSym$ \\
        $\RTrans$
        \end{tabular}
        \\
        \hline
        \begin{tabular}[l]{@{}l@{}} Verb Phrase \\  
        (Passive Voice) \end{tabular}
        & \begin{tabular}[l]{@{}l@{}} \textit{killed\_by}\,\textit{(alexander\_pushkin}, \\ \quad  \textit{georges-charles\_de\_heeckeren\_d'anthès)}.\\ \textit{located\_in\_time\_zone\,(arizona, utc-07:00).}\\ 
        % \textit{(Bayes' theorem, named after, Thomas Bayes)}
        \end{tabular}
        &
\begin{tabular}[l]{@{}l@{}} 
        $\RNeg$ \\
        $\RInv$
        \end{tabular}
        \\
        \hline
        \begin{tabular}[l]{@{}l@{}} Verb Phrase \\  
        (Active Voice) \end{tabular}
        & \begin{tabular}[l]{@{}l@{}} \textit{follows\,(4769\_Castalia, 4768\_hartley).}\\ \textit{replaces\,(american\_broadcasting\_company,} \\ \qquad \quad\ \   \textit{nbc\_blue\_network).} \end{tabular}
        &
\begin{tabular}[l]{@{}l@{}}  
        $\RNeg$ \\
        $\RInv$
        \end{tabular}
        \\
        \Xhline{1.5\arrayrulewidth} %添加表格底部粗线
    \end{tabular}
    %}
\caption{Relation Categorization.}
\label{table:relations}
\end{figure}









The facts extraction is done per-category basis, implementing a divide-and-conquer strategy, which efficiently integrates all the facts from all the categories. 
As shown in \algoref{alg:ground_truth}, for any given entity category and relation category, the function $\textsc{ExtractGroundFacts}$ iterates through all possible entities and relations. 
For each combination ($\m{entity}, \nm$), it queries the database using the $\textsc{QueryDB}$ function, which retrieves all three-element facts established with the specific predicate $\nm$ and the argument $\m{entity}$ placed either in the subject or the object position. 

{
\begin{algorithm}[!h]
\caption{Facts Extraction}
\label{alg:ground_truth}
\small
\begin{algorithmic}[1]
\Require  
Entity Category (\entityCat), Relation Category (\relationCat)
\Ensure Ground Facts ($\groundTruthTriples$)
\Function{ExtractGroundFacts}{
\entityCat, \relationCat}
\State$\groundTruthTriples \gets []$ \Comment{\commentstyle{Initialization}}
\For{$\m{entity}$ $\in$ \entityCat~} \Comment{\commentstyle{Iterate over each entity}}
\For{~$\nm$ $\in$  \relationCat~} \Comment{\commentstyle{Iterate over each relation}}
\State $\widetilde{\relation} \gets$ \Call{QueryDB}{$\m{entity}$, $\nm$} 
\Comment{\commentstyle{Retrieve ground facts}}
\State $\groundTruthTriples.\m{append}(\widetilde{\relation})$ \Comment{\commentstyle{Extend the ground facts}}
\EndFor
\EndFor
\State \Return $\groundTruthTriples$ \Comment{\commentstyle{Return the ground facts}}
\EndFunction
\end{algorithmic}
\end{algorithm}
}



\begin{figure}[!b]
%\vspace{-2mm}
\centering
\small
\begin{gather*}
\frac{
\begin{matrix}
\RNeg\\
{\drule}{=}\,\nm_{\m{new}}\,(\SUBJ, \OBJ){\hornarrow} !\nm\,(\SUBJ, \OBJ)
\end{matrix}
}{
\deriveRules{\nm}{\nm_{\m{new}}}{{\drule}}}
\ \  
\frac{
\begin{matrix}
\RInv\\
{\drule}{=}\,\nm_{\m{new}}\,(\SUBJ, \OBJ) {\hornarrow} \nm\,(\OBJ, \SUBJ)
\end{matrix}
}{
\deriveRules{\nm}{\nm_{\m{new}}}{{\drule}}}
\\[0.4em]
\frac{
\begin{matrix}
\RSym\\
{\drule}{=}\,\nm\,(\SUBJ, \OBJ) {\hornarrow} \nm\,(\OBJ, \SUBJ)
\end{matrix}
}{
\deriveRules{\nm}{\nm}{{\drule}}}
\quad\   
\frac{
\begin{matrix}
{\drule}{=}\,\nm\,(\SUBJ, \OBJ') {\hornarrow} 
\\ 
\nm\,(\SUBJ, \OBJ), 
\nm\,(\OBJ, \OBJ')
\end{matrix}
}{
\deriveRules{\nm}{\nm}{{\drule}}}  \RTrans
\end{gather*}
%\vspace{-1mm}
\caption{Deriving New Facts From the Known Facts}
\label{fig:basic_op_for_predicates}
\end{figure}


\vspace{-2mm}
\subsection{Deriving Simple Facts via Logical Reasoning}
\label{sec:derive_more_facts}
Based on the relation category, each predicate is labeled with a different set of properties, shown in \figref{table:relations}, which are mapped to different derivation rules. 

Based on the ground facts extracted from the databases, \tool derives additional facts to enrich the knowledge and generates test cases from each derived fact. 
As shown in \figref{fig:basic_op_for_predicates}, it provides four basic derivation rules, 
providing sound strategies to generate mutated facts from the ground facts. %\shil{1. How to define the he new name for $nm_{new}$? better provide this. 2. For negation rule, subject and object shall not be changed the position, can refer to oopsla paper.}
%\syh{1. added by the end of next paragraph; 2. revised}
%\footnote{
Note that these rules are also prevalently adopted in several literature~\cite{zhou2019completing, ren2020beta, liang2022reasoning, TIAN2022100159, abboud2020boxe} in the context of knowledge reasoning.
Given a predicate name $\nm$, the derivation  ``$\deriveRules {\nm}{\nm_{\m{new}}}{{\drule}}$'' holds if $\nm$ can be applied into a Prolog rule ${\drule}$, and produces more facts upon a new predicate with the name  $\nm_{\m{new}}$. 
These new predicates are freshly generated based on predefined suffixes. 


As indicated in \figref{table:relations}, 
%In particular, 
all the predicates can be applied to the $\RNeg$ rule, which derives the negated relations, e.g., ``!$\m{was}$'' using its positive counterpart, e.g., ``$\m{wasn't}$''. 
For the \emph{noun phrase} relations, both $\RSym$ and $\RTrans$ rules can apply, which generate more facts without creating new predicates. 
For the \emph{verb phrase} relations, both passive voice and active voice predicates can be applied to the  $\RInv$ rule, which captures the inverse relations, where the subject and object can be reversely linked through a variant of the original relation.  An example of such a rule is: 
\[
\m{influence(\OBJ, \SUBJ)}\hornarrow\m{influence\_by(\SUBJ, \OBJ)}
.\] 


We summarize the fact derivation process using \algoref{alg:logic_reasoning}. Given any relation category, we iterate its predicates and generates the derivation rule $\drule$ (Line 4). 
For simplicity, we assume that the choice of which derivation rule to apply is predetermined. Based on this assumption, a new Prolog program is constructed, comprising ground facts and $\drule$. 
In particular, we use $\llbracket \relation \rrbracket_{\Prolog}$ to denote the query results of $\relation$ concerning the Prolog program $\Prolog$, and $\Prolog$ contains all the ground facts and the derivation rule. 
{Note that when $\relation$ contains no variables, it returns Boolean results indicating the presence of the fact; otherwise, it outputs all the possible instantiation of the variables. }
For each instantiation that contains one subject ``\Subj'' and one object ``\Obj'', we then compose them with the new predicate, which is taken as a  \emph{derived fact}.  
These derived facts are later used to generate NL test cases, detailed in \secref{prompt}. 


{
\begin{algorithm}[!h]
\caption{Deriving New Facts}
\label{alg:logic_reasoning}
\small
\begin{algorithmic}[1]
\Require Ground Facts ($\groundTruthTriples$), Relation Category (\relationCat)
\Ensure Derived Facts ($\derivedFacts$)
\Function{DerivingFacts}{$\groundTruthTriples$, \relationCat}
\State $\derivedFacts \gets []$ \Comment{\commentstyle{Initialization}}
\For{$\nm$ in \relationCat}
\Comment{\commentstyle{Iterate each predicate}}
\State $\deriveRules{\nm}{\nm_{\m{new}}}{{\drule}}$\Comment{\commentstyle{Obtain the new predicate}} %the reasoning rule, and 
\State $\Prolog \gets \groundTruthTriples\plus\plus{\drule}$ \Comment{\commentstyle{Construct the prolog program}} 
\State $\m{instantiations} \gets \llbracket \nm_{\m{new}}(\SUBJ, \OBJ)\rrbracket_{\Prolog}$ 
%\Comment{\commentstyle{Obtain concrete entities}}
\For{(\Subj, \Obj) in $\m{instantiations}$}
\Comment{\commentstyle{Iterate each entity tuple}}
\State $\relation_{\m{new}} \gets \nm_{\m{new}}(\Subj, \Obj)$ 
\Comment{\commentstyle{Construct the derived fact}}
\State $\derivedFacts.\m{append}(\relation_{\m{new}})$ \Comment{\commentstyle{Append the derived facts}}
\EndFor
\EndFor
\State \Return $\derivedFacts$ \Comment{\commentstyle{Return the derived facts}}
\EndFunction 
\end{algorithmic}
\end{algorithm}
}






%-logic-based
\vspace{-2mm}
\subsection{Deriving Facts via Temporal Reasoning}
\label{sec:encoding_MTL}




%as the query language, 
%We convert the temporal-logic-based test cases into 
%temporal-logic-based natural language query, we use NLP techniques \cite{icaps2023fc,aaai2023fc} %\syh{cite here?} %to convert it into a 

Apart from the basic derivation rules, \tool can also automatically derive complex composition rules based on \emph{Metric Temporal Logic} (MTL) \cite{DBLP:conf/lics/OuaknineW05}. 
Specifically, we generate temporal test cases  based on randomly generated MTL formulae over historical events. 
We define the syntax for MTL formula in \figref{fig:syntax_of_the_metric_temporal_logic}, which contains the temporal operators for ``finally ($\mathcal{F}$)'', 
``globally ($\mathcal{G}$)'', 
``until ($\mathcal{U}$)'', 
and ``next ($\mathcal{N}$)''. 
The atomic propositions here are basic event relations $\nm$. 
%are three-element relations associated with time stamps. 
The time intervals are pairs of natural numbers indicating the lower and upper bounds of the years%\shil{only year?} \syh{so far yes, but I added a sentence by the end to make it more generic}
; and the constraint $\Istart \,{\leq}\, \Iend$ is enforced implicitly for all time intervals. 
In this paper for simplicity, we use discrete time measured in years as the smallest time interval. However, the framework can be extended to accommodate any smaller discrete time intervals, such as days or seconds. 


\begin{figure}[!h] 
\vspace{-2.5mm}
\small
\centering
\begin{align*}
(\m{MTL})\quad & \mtl &{::=}\quad &
\nm %(\Subj, \Obj) %\ap  
\,{\mid}\, \mathcal{F}_\interval \,\mtl
\,{\mid}\, \mathcal{G}_\interval \,\mtl 
\,{\mid}\, \mtl_1  
\,\mathcal{U}_\interval \,  \mtl_2 
\,{\mid}\, \mathcal{N} \,\mtl
\,{\mid}\, 
\\
&&&
%\,{\mid}\, \mtl_1 \, {\rightarrow} \,\mtl_2
 \mtl_1  \,{\wedge}\, \mtl_2
\,{\mid}\, \mtl_1  \,{\vee}\, \mtl_2
\,{\mid}\, \neg \mtl 
\\[0.3em]
%&(\m{Atomic~Proposaition})~ \ap ~{::=}~ 
%\nm\_{\m{TS}}(\interval, \Subj, \Obj)
%\\[0.3em]
(\m{Time~Interval}) \quad& \interval &{::=}\quad & [\Istart, \Iend]
\end{align*}
\vspace{-4mm}
\caption{A Core Syntax of MTL}
\label{fig:syntax_of_the_metric_temporal_logic}
\vspace{-3mm}
\end{figure}


To facilitate the generation of temporal-based Q\&A pairs, we define the semantics model for the MTL formulae in \defref{def:semantics_MTL}, where the history is a set of facts. 
Here, we use $\history$ as a set of historical relations, 
e.g., ``$\nm\_{\m{TS}}(\interval, \Subj, \Obj)$'', which are the time-stamped version relations of the three-element relations ``$\nm(\Subj, \Obj)$'', derived by one of the following rules: \\[-0.5em]

\noindent 
{\small $\ \   
\nm\_{\m{TS}}(\interval, \Subj, \Obj) \hornarrow \nm(\Subj, \Obj), \m{start}(\Obj, n_1), \m{end}(\Obj, n_2), \interval{=}[n_1, n_2]. 
$}\\[-0.5em]

\noindent  {\small $\ \  \nm\_{\m{TS}}(\interval, \Subj, \Obj) \hornarrow \nm(\Subj, \Obj), \m{start}(\Subj, n_1), \m{end}(\Subj, n_2), \interval{=}[n_1, n_2].$}
\\

\noindent which construct the event intervals using the time stamps associated with the object or the subject, respectively. 
The ``$\m{start}$'' and ``$\m{end}$'' predicates are originally generated from the knowledge database and mark the starting and ending points of the duration of the object (or subject) event. 
For simplicity, we use ``$\nm\_{\m{TS}}(\interval)$'' to abbreviate ``$\nm\_{\m{TS}}(\interval, \Subj, \Obj)$'' when $\Subj$ and $\Obj$ are unambiguously unique from the context. 
We also use $\llbracket \nm\_{\m{TS}}(\interval) %, \Subj, \Obj
\rrbracket_{\history}$ to denote the validity of querying the presence of a fact $\nm\_{\m{TS}}(\interval)$ 
against the historical facts $\history$, which stores all the time-stamped three-element predicates. 


\begin{definition}[A Point-based Semantics for MTL]
\label{def:semantics_MTL}
Given a set of (historical) facts $\history$, recording all the events that happened in history, an MTL formula $\mtl$, and a concrete time point  $\timepoint$, the satisfaction relation $(\history, \timepoint) \models \mtl$  (read at the time point \timepoint, the history $\history$ satisfies $\mtl$) is recursively defined as follows: 

{
\small
\begin{align*}
%%%%%%%%%%%%%%%%%%%%%%%%%%%%%%
%%%%%%%%%%% AP  R   %%%%%%%%%%
%%%%%%%%%%%%%%%%%%%%%%%%%%%%%%
(\history, \timepoint) &\models 
\nm &\m{iff}&~ 
\m{\exists\,\interval}.~ 
\llbracket \nm\_{\m{TS}}(\interval) \text{$\rrbracket_{\history}$}{=}\m{true}
~\m{and}~
\timepoint\,{\in}\,\interval
\\[0.1em]
%, \Subj, \Obj
%%%%%%%%%%%%%%%%%%%%%%%%%%%%%%
%%%%%%%%%%% Finally %%%%%%%%%%
%%%%%%%%%%%%%%%%%%%%%%%%%%%%%%
(\history, \timepoint) &\models \mathcal{F}_\interval \,\mtl & 
\m{iff}&~ 
\m{\exists\,\distance}.~\distance\,{\in}\,I  ~ \m{and}
~ (\history, \timepoint\plus\distance)\models\mtl
\\[0.1em]
%%%%%%%%%%%%%%%%%%%%%%%%%%%%%%
%%%%%%%%%%% Globally %%%%%%%%%%
%%%%%%%%%%%%%%%%%%%%%%%%%%%%%%
(\history, \timepoint) &\models \mathcal{G}_\interval\,\mtl & 
\m{iff}&~ 
\m{\forall\,\distance}.~\distance\,{\in}\,I  ~ \m{and}
~ (\history, \timepoint\plus\distance)\models\mtl
\\[0.1em]
%%%%%%%%%%%%%%%%%%%%%%%%%%%%%%
%%%%%%%%%%% Next %%%%%%%%%%
%%%%%%%%%%%%%%%%%%%%%%%%%%%%%%
(\history, \timepoint) &\models \mathcal{N}\,\mtl & 
\m{iff}&~ 
(\history, \timepoint\plus 1)\models\mtl
\\[0.1em]
%%%%%%%%%%%%%%%%%%%%%%%%%%%%%%
%%%%%%%%%%% negation %%%%%%%%%%
%%%%%%%%%%%%%%%%%%%%%%%%%%%%%%
(\history, \timepoint) &\models\neg \mtl & \m{iff}&~
(\history, \timepoint)\not\models\mtl
\\[0.1em]
%%%%%%%%%%%%%%%%%%%%%%%%%%%%%%
%%%%%%%%%%% Unitl %%%%%%%%%%
%%%%%%%%%%%%%%%%%%%%%%%%%%%%%%
(\history, \timepoint) &\models \mtl_1 \, \mathcal{U}_\interval \,\mtl_2  & \m{iff}&~  \m{\exists\,\distance}.~ \distance\,{\in}\,\interval  ~ \m{and}~ (\history, \timepoint\plus\distance)\models\mtl_2 ~ \m{and}
\\[0.1em] 
&&& ~ 
\m{\forall}\, 
k~\m{with} ~\timepoint{<}k{<}(\timepoint\plus\distance), 
(\history, k)\models \mtl_1
\\[0.1em]
%%%%%%%%%%%%%%%%%%%%%%%%%%%%%%
%%%%%%%%%%% conjunction %%%%%%%%%%
%%%%%%%%%%%%%%%%%%%%%%%%%%%%%%
(\history, \timepoint) &\models\mtl_1 \, {\wedge} \,\mtl_2 & \m{iff}&~ (\history, \timepoint)\models\mtl_1 ~\m{and}~ (\history, \timepoint)\models\mtl_2
\\[0.1em]
%%%%%%%%%%%%%%%%%%%%%%%%%%%%%%
%%%%%%%%%%% disjunction %%%%%%%%%%
%%%%%%%%%%%%%%%%%%%%%%%%%%%%%%
(\history, \timepoint) &\models\mtl_1 \, {\vee} \,\mtl_2 & \m{iff}&~ (\history, \timepoint)\models\mtl_1 ~\m{or}~ (\history, \timepoint)\models\mtl_2 
\end{align*}}
\end{definition}
\vspace{2mm}



We randomly generate temporal test cases based on the rich set of historical events and the syntax templates defined in \figref{fig:syntax_of_the_metric_temporal_logic}. 
Each temporal question consists of a concrete MTL formula and a concrete time point, i.e., $(\phi, \timepoint)$. 
For example, the query ``\emph{At 1800, does Victorian era finally come within 40 years?}'' is represented as $(\mathcal{F}_{[0, 40]} \m{victorian\_era}, 1800)$. 
Next, we show how to obtain the expected answer by automatically generating Prolog reasoning rules. 

Given a query $\mtl$, the relation ``$\encoding{\mtl}{\nm}{\widetilde{\drule}}$'' holds if $\mtl$ can be translated into a set of Prolog rules, i.e., $\widetilde{\drule}$. 
Querying ``$\nm(\interval)$'' 
%with the set of Prolog rules 
$\widetilde{\drule}$, against the known database facts yields a set of instantiation of the interval $\interval$. 
The validity of $\mtl$ at any given time point $\timepoint$ is then indicated by the existence of a concrete interval  $\interval$ such that $\timepoint\,{\in}\,\interval$. 
We define the full set of encoding rules for MTL operators in \figref{fig:encoding_rules_mtl}. 

These encoding rules deploy several auxiliary predicates: the 
``$\m{findall}(\interval, \nm)$'' relation indicates that $\interval$ is a union of all the time intervals which satisfy $\nm$; 
the  ``$\m{compl}(\interval, \interval_1)$'' relation indicates that time intervals $\interval$ and $\interval_1$ complement each other, and their union encompasses the entire set of time points; the union and intersection operations, denoted by $\cup$ and $\cap$, are applied to two sets of time intervals. 

\begin{figure}[!h]
% \begin{minipage}[b]{1\linewidth}
\vspace{-2mm}
\vspace{0mm}
\begin{lstlisting}[xleftmargin=6em,numbersep=5pt,basicstyle=\footnotesize\ttfamily]
//nm1 = charles_dickens
charles_dickens_TS([1812, 1870]).
//nm2 = victorian_era
victorian_era_TS([1837, 1901]).
\end{lstlisting} 
\vspace{-1mm}
\caption{Database $\history_s$ Containing Two Time-stamped Events}
\label{fig:Prolog_encoding_Example}
\vspace{-2mm}
\end{figure}


Next, we illustrate the encoding rules for each MTL operator using a few examples. 
To facilitate the illustration, we use a small database $\history_s$ defined in \figref{fig:Prolog_encoding_Example}, which contains two facts: ``\emph{The author Charles Dickens was born in 1812 and he lived until 1870, which spanned a significant portion of the Victorian era}'' and ``\emph{The Victorian era started from 1837 until Queen Victoria died in 1901}'': 

\begin{figure}[!b]
\vspace{-3mm}
\centering\small
\begin{gather*} 
%%%%%%%%%%%%%%%%%%%%%%%%%%%%%%
%%%%%%%%%%% AP R %%%%%%%%%%%%%
%%%%%%%%%%%%%%%%%%%%%%%%%%%%%%
\frac{
\begin{matrix}
\widetilde{\drule} = [\nmNEW(\interval) \hornarrow \nm\_{\m{TS}}(\interval).]
\end{matrix}
}{\encoding {\nm}{\nmNEW}{\widetilde{\drule}}}\ [\trans\text{-}\m{AP}]
\\[0.6em]
%%%%%%%%%%%%%%%%%%%%%%%%%%%%%%
%%%%%%%%%% Finally %%%%%%%%%%%
%%%%%%%%%%%%%%%%%%%%%%%%%%%%%%
\frac{
\begin{matrix}
\widetilde{\drule} {=} 
[\nmNEW([\interval^\prime_\m{start}\text{-}\interval_{\m{end}}, \interval^\prime_\m{end}\text{-}\interval_{\m{start}}]) \hornarrow \nm(\interval^\prime).]
\end{matrix}
}{\encoding {\mathcal{F}_\interval\,\mtl}{\nmNEW}{\widetilde{\drule} }}\ [\trans\text{-}\m{Finally}]
\\[0.6em]
%%%%%%%%%%%%%%%%%%%%%%%%%%%%%%
%%%%%%%%% Globally %%%%%%%%%%%
%%%%%%%%%%%%%%%%%%%%%%%%%%%%%%
\frac{
\begin{matrix}
\widetilde{\drule} {=} 
[\nmNEW([\interval^\prime_\m{start}\text{-}\interval_{\m{start}}, \interval^\prime_\m{end}\text{-}\interval_{\m{end}}]) \hornarrow \nm(\interval^\prime).]
\end{matrix}
}{\encoding {\mathcal{G}_\interval\,\mtl}{\nmNEW}{\widetilde{\drule} }}\ [\trans\text{-}\m{Globally}]
\\[0.6em]
%%%%%%%%%%%%%%%%%%%%%%%%%%%%%%
%%%%%%%%% Next %%%%%%%%%%%
%%%%%%%%%%%%%%%%%%%%%%%%%%%%%%
\frac{
\begin{matrix}
\widetilde{\drule} {=} 
[\nmNEW([\interval^\prime_\m{start}\text{-}1, \interval^\prime_\m{end}\text{-}1]) \hornarrow \nm(\interval^\prime).]
\end{matrix}
}{\encoding {\mathcal{N}\,\mtl}{\nmNEW}{\widetilde{\drule} }}\ [\trans\text{-}\m{Next}]
\\[0.6em]
%%%%%%%%%%%%%%%%%%%%%%%%%%%%%%
%%%%%%%%% Until %%%%%%%%%%%
%%%%%%%%%%%%%%%%%%%%%%%%%%%%%%
\frac{
\begin{matrix}
\encoding{\mtl_1}{\nm_1}{\widetilde{\drule}_1}
\qquad 
\encoding{\mtl_2}{\nm_2}{\widetilde{\drule}_2}
\\[0.2em]
\widetilde{\drule}_3{=} [\m{helper1}([\interval^\prime_{\m{start}}\plus\interval_{\m{start}}, \interval^\prime_{\m{end}}\plus1]) \hornarrow 
\nm_1(\interval^\prime).]
\\[0.2em]
\widetilde{\drule}_4{=} [\m{helper2}(\interval_1\,{\cap}\,\interval_2) \hornarrow 
\m{helper1}(\interval_1), 
\nm_2(\interval_2).] 
\\[0.2em]
\encoding {\mathcal{F}_\interval\,(\m{helper2})}{\nm_f}{\widetilde{\drule}_5 }
\\[0.2em]
\widetilde{\drule}_6{=} [\nmNEW(\interval_1\cap \interval_2) \hornarrow 
\nm_1(\interval_1), 
\nm_
f(\interval_2). ] 
\end{matrix}
}{\encoding{\mtl_1\,\mathcal{U}_\interval\,\mtl_2}{\nmNEW}{\widetilde{\drule}_1\cup \widetilde{\drule}_2\cup
\widetilde{\drule}_3\cup
\widetilde{\drule}_4\cup
\widetilde{\drule}_5\cup
\widetilde{\drule}_6}}\ [\trans\text{-}\m{Until}]
%\shil{
%~2. ~what's ~the~ definition ~\interval ~in~ \mathcal{F}? }
%\\ \shil{~3. ~what's ~the ~meaning ~of ~;?}\text{\syh{to~construct~list~from~single~rules}}
\\[0.6em]
%%%%%%%%%%%%%%%%%%%%%%%%%%%%%%
%%%%%%%%% Negation %%%%%%%%%%%
%%%%%%%%%%%%%%%%%%%%%%%%%%%%%%
\frac{
\begin{matrix}
\encoding{\mtl}{\nm}{\widetilde{\drule}_1}
\\ 
\widetilde{\drule}{=}[\nmNEW(\interval) \hornarrow
\m{findall}(\interval_1, \nm), \m{compl}(\interval_1, \interval).]
\end{matrix}
}{
\encoding{\neg\mtl}{\nmNEW}{
\widetilde{\drule}_1\,{\cup}\,\widetilde{\drule}}
}\ [\trans\text{-}\m{Neg}]
\\[0.6em]
%%%%%%%%%%%%%%%%%%%%%%%%%%%%%%
%%%%%%%%% Conjunction %%%%%%%%%%%
%%%%%%%%%%%%%%%%%%%%%%%%%%%%%%
\frac{
\begin{matrix}
[\trans\text{-}\m{Conj}]\\
\encoding{\mtl_1}{\nm_1}{\widetilde{\drule}_1}
\qquad 
\encoding{\mtl_2}{\nm_1}{\widetilde{\drule}_2}
\\
\widetilde{\drule}{=}[\nmNEW(\interval_1\,{\cap}\,\interval_2) \hornarrow
\m{findall}(\interval_1, \nm_1), \m{findall}(\interval_2, \nm_2)]
\end{matrix}
}{
\encoding{\mtl_1{\wedge}\mtl_2}{\nmNEW}{ \widetilde{\drule}_1\,{\cup}\,\widetilde{\drule}_2\,{\cup}\,\widetilde{\drule}}
}
\\[0.6em]
%%%%%%%%%%%%%%%%%%%%%%%%%%%%%%
%%%%%%%%% Disjunction %%%%%%%%%%%
%%%%%%%%%%%%%%%%%%%%%%%%%%%%%%
\frac{
\begin{matrix}
[\trans\text{-}\m{Disj}]\\
\encoding{\mtl_1}{\nm_1}{\widetilde{\drule}_1}
\qquad 
\encoding{\mtl_2}{\nm_1}{\widetilde{\drule}_2}
\\
\widetilde{\drule}{=}[\nmNEW(\interval_1\,{\cup}\,\interval_2) \hornarrow
\m{findall}(\interval_1, \nm_1), \m{findall}(\interval_2, \nm_2)]
\end{matrix}
}{
\encoding{\mtl_1{\vee}\mtl_2}{\nmNEW}{ \widetilde{\drule}_1\,{\cup}\,\widetilde{\drule}_2\,{\cup}\,\widetilde{\drule}}
}
\end{gather*}
\caption{Encoding MTL Formula $\mtl$ using Prolog Rules}
\label{fig:encoding_rules_mtl}
\end{figure}




\begin{enumerate}[itemsep=0.7em,leftmargin=!,wide]
\item 
When 
$\mtl\,{=}\,\m{charles\_dickens}$ and $\timepoint\,{=}\,1800$: \\ 
According to the encoding rule $[\trans\text{-}\m{AP}]$, the generated Prolog rule is: $\m{\nm1(\interval)}\hornarrow\,\m{
charles\_dickens\_TS(\interval)}$.
Now, querying ``$\nm1(\interval)$'' against $\history_s$ yields $\interval\,{=}\,[1812, 1870]$. Since $1800\,{\not\in}\,\interval$, the expected result of this query is false.

Similarly, when 
$\mtl\,{=}\,\m{victorian\_era}$ and $\timepoint\,{=}\,1900$: \\ 
According to the encoding rule $[\trans\text{-}\m{AP}]$, the generated Prolog rule is: $\m{\nm2(\interval)}\hornarrow\,\m{
victorian\_era\_TS(\interval)}$.
Now, querying ``$\nm2(\interval)$'' against $\history_s$ yields $\interval\,{=}\,[1837, 1901]$. Since $1900\,{\in}\,\interval$, the expected result of this query is true. 

\item When $\mtl\,{=}\,\mathcal{F}_{[0, 40]}\,\m{victorian\_era}$ and $\timepoint\,{=}\,1800$: \\
According to the encoding rule $[\trans\text{-}\m{Finally}]$, the generated Prolog rule is: 
$\m{finally\_\nm2([n_1\text{-}40, n_2\text{-}0])}\hornarrow \m{
\nm2([n_1, n_2])}.$
Now, querying ``$\m{finally\_\nm2}(\interval)$'' against $\history_s$ yields $\interval\,{=}\,[1797, 1901]$. Since $1800\,{\in}\,\interval$, the expected result is true. 
Indeed, all the time points in $\interval$ satisfy the property: ``\emph{within 40 years, finally Victorian era came/still exist}''. 

\item When $\mtl\,{=}\,\mathcal{G}_{[30, 50]}\,\m{victorian\_era}$ and $\timepoint\,{=}\,1800$: \\
According to the rule $[\trans\text{-}\m{Globally}]$, the generated Prolog rule is: $\m{globally\_\nm2([n_1\text{-}30, n_2\text{-}50])}\hornarrow \m{
\nm2([n_1, n_2])}.$
Now, querying ``$globally\_\nm2(\interval)$'' against $\history_s$ yields $\interval\,{=}\,[1807, 1851]$. Since $1800\,{\not\in}\,\interval$, the expected result is false. 
Indeed, only all the time points in $\interval$ satisfy that ``\emph{Victorian era is globally true throughout the 30th to the 50th years in the future}''. 

\item When $\mtl\,{=}\,\mathcal{N}\,\m{victorian\_era}$ and $\timepoint\,{=}\,1836$: \\
According to the rule $[\trans\text{-}\m{Next}]$, the generated Prolog rule is: $\m{next\_\nm2([n_1\text{-}1, n_2\text{-}1])}\hornarrow \m{
\nm2([n_1, n_2])}.$
Now, querying ``$\m{next\_\nm2}(\interval)$'' against $\history_s$ yields $\interval\,{=}\,[1836, 1900]$. Since  $1836\,{\in}\,\interval$, the expected result is true. 
Indeed, all the time points in $\interval$ satisfy that ``\emph{next year Victorian era came/still exist}''. 

\item When $\mtl\,{=}\,\m{charles\_dickens}
~\mathcal{U}_{[10, 20]}\,\m{victorian\_era}$ and $\timepoint{=}1800$: This query aims to determine the time interval $\interval$ that encompasses all time points $\timepoint'$ for which there exists a future year ($\timepoint'\plus\distance$) when the Victorian era had begun; and during the time from $\timepoint'$ to $\timepoint'\plus\distance$, Charles Dickens must have been born and remained alive throughout. Lastly, check if $1800\,{\in}\,\interval$.

In $[\trans\text{-}\m{Until}]$, 
$\m{helper1}$ computes all the possible  $\timepoint'\plus d$  
%where \\  $d\,{\in}\,[10, 20]$, 
such that $(\history, k)\models \m{charles\_dickens} ~\m{forall}~ 
k~\m{with} ~\timepoint'{<}k{<}(\timepoint'\plus\distance)$. 
Next $\m{helper2}$ computes the overlapping  intervals of $\m{helper1}$ and the intervals that also satisfy $(\history, \timepoint'\plus\distance)\,{\models}\,\m{victorian\_era}$. 
Then $\nm_f$ computes the interval of $\timepoint'$ which finally reach $\m{helper2}$ within 10 to 20 years. 
Lastly, the final answer of the interval of $\timepoint'$~is the intersection of $\nm_f$ and $\nm_1$. 
Therefore, given the concrete query $(\phi, \timepoint)$, from $[\trans\text{-}\m{Until}]$, 
the generated rules are shown in \figref{fig:until10-20-encoding}. 




\begin{figure}[!h]
\vspace{-3mm}
{
\begin{align*}
&\m{helper1([n_1\plus10, n_2\plus1])}\hornarrow \m{\nm1([n_1, n_2])}.
& // [1822, 1871]
\\
&\m{helper2(\interval_1\cap\interval_2)}\hornarrow\m{helper1(\interval_1)}, \m{\nm2(\interval_2)}. 
& // [1837, 1871]
\\
& \m{\nm_f([n_1\text{-}20, n_2\text{-}10])}\hornarrow \m{
helper2([n_1, n_2])}.
& // [1817, 1861]
\end{align*}
\vspace{-4mm}
\[\m{charles}\_\m{until}\_\m{victorian\_era}(\interval_1\cap\interval_2) \hornarrow \m{\nm1(\interval_1)}, \m{\nm_f(\interval_2)}.\]}
\caption{Prolog Rules Generated for an "Until" Query}
\label{fig:until10-20-encoding}
\vspace{-1mm}
\end{figure}

Querying ``$\m{charles}\_\m{until}\_\m{victorian\_era}$'' against $\history_s$ yields $\interval\,{=}\,[1817, 1861]$. Since $1800\,{\not\in}\,\interval$, the expected result is false. 
Indeed, only all the time points in $\interval$ satisfy $\phi$ under the semantic definition of \emph{Until}, cf.  \defref{def:semantics_MTL}. For example when $\timepoint'{=}1817$, there exists $\distance{=}20$ such that $\phi$ holds; and when $\timepoint'{=}1861$ there exists $\distance{=}10$ such that $\phi$ holds. 

Note that, in this encoding, the interval of ``\emph{Until}'' operators does not include $[0, 0]$, as $\mtl_1  
\,\mathcal{U}_{[0, 0]} \,  \mtl_2$ essentially equals $\mtl_2$. Therefore, when the interval compasses $[0, 0]$, we use the following rule to decompose the encoding: (Note that when $\interval'\,{=}\,\interval{\setminus}[0, 0]$, it means $\interval'\cup[0, 0]\,{=}\,\interval$)
\begin{align*}
\frac{
[0, 0] \subseteq \interval 
\qquad 
\interval^\prime \,{=}\, \interval{\setminus}[0, 0]
}{
\mtl_1  
\,\mathcal{U}_\interval \,  \mtl_2 
\equiv  (\mtl_1  
\,\mathcal{U}_{\interval^\prime} \,  \mtl_2)  \vee \mtl_2 
} \ [\trans\text{-}\m{Until}\text{-}0]
\end{align*}

\vspace{2mm}
\item When $\mtl\,{=}\,\neg\,\m{victorian\_era}$ and $\timepoint\,{=}\,1800$: \\
By $[\trans\text{-}\m{Neg}]$, the generated Prolog rule is: 
$\m{neg\_\nm2(\interval)}\hornarrow$ 
$\m{findall}(\interval_1, \nm1), \m{compl}(\interval_1, \interval).$ 
Querying ``$\m{neg\_\nm2}(\interval)$'' against $\history_s$ yields $\interval\,{=}\,[1, 1836] \cup [1902, 2024].$
Here, we take all the after-century years to be the full set. 
Since $1800\,{\in}\,\interval$, the expected result is true. 
Indeed, all the time points in $\interval$ satisfy that ``\emph{Victorian era has not come/already passed}''.  


\item When $\mtl\,{=}\,\m{charles\_dickens}\,{\wedge}\,\m{victorian\_era}$ and $\timepoint{=}1900$: 
By $[\trans\text{-}\m{Conj}]$, the generated Prolog rule is: 
$\m{\nm1\_and\_\nm2(\interval_1{\cap}\interval_2)}\hornarrow 
\m{findall}(\interval_1, \nm1), \m{findall}(\interval_2, \nm2)$. 
Now, querying ``$\m{\nm1\_and\_\nm2}(\interval)$'' against $\history_s$ yields $\interval\,{=}\,[1837, 1870]$. Since $1900\,{\not\in}\,\interval$, the expected result is false. 
Indeed, only the time points in $\interval$ satisfy that ``\emph{Victorian era exists and Charles Dickens is alive}''. 


\item When $\mtl\,{=}\,\m{charles\_dickens}\,{\vee}\,\m{victorian\_era}$~and
$\timepoint{=}1900$: 
By $[\trans\text{-}\m{Disj}]$, the generated Prolog rule is as follows: 
$\m{\nm1\_or\_\nm2(\interval_1{\cup}\interval_2)}\hornarrow 
\m{findall}(\interval_1, \nm1), \m{findall}(\interval_2, \nm2)$. 
Now, querying ``$\m{\nm1\_or\_\nm2}(\interval)$'' against $\history_s$ yields $\interval\,{=}\,[1812, 1901]$. Since $1900\,{\in}\,\interval$, the expected result is true. 
Indeed, all the time points in $\interval$ satisfy that ``\emph{Victorian era exists or Charles Dickens is alive}''.  
%\shil{check implementation?}
\end{enumerate}
\vspace{3mm}






{\emph{\textbf{Remark.}}} 
While discrete-time MTL is commonly employed for model-checking timed verification~ \cite{DBLP:phd/us/Henzinger91}, utilizing Prolog to encode MTL for reasoning about the temporal relationships among events and detecting LLM hallucination is novel. 
These encoding rules can recursively accommodate the entire range of MTL formulas, including those with any level of nesting. 
We provide a formal definition for the correctness of these encoding rules in \theoref{ThemSoundAndComplete} and demonstrate that they are both sound and complete.



\begin{restatable}[Correctness of the encoding rules]{thm}{ThemSoundAndComplete}
\label{ThemSoundAndComplete}
~\\
Given any $\history$, 
$\mtl$, and 
$\encoding {\mtl}{\nm}{\widetilde{\drule}}$, let $\Prolog{=}\history \plus\plus \widetilde{\drule}$, we define,  \\
(1) Soundness: \\
$\forall\, \interval$.  
$\llbracket \nm(\interval) \rrbracket_{ \Prolog} {=} \m{true}$, then 
$\forall\, \timepoint\,{\in}\, \interval$, we have 
$(\history, \timepoint) \models \mtl$;  \\
(2) Completeness: \\ 
$\forall \,\timepoint\,$. $(\history, \timepoint) \models \mtl$, then $\exists\, \interval$. $\llbracket \nm(\interval) \rrbracket_{\Prolog} {=} \m{true}$  and $\timepoint\,{\in}\,\interval$. 
\end{restatable}

\begin{proof}
By structural induction over $\phi$. 
%a case analysis of the encoding rules.
The detailed proofs are given in the Appendix. %\appref{app:correctness}.  
\end{proof}




\begin{table*}[!t]
% \def\arraystretch{1.0}
\setlength{\tabcolsep}{1pt}
\centering
% \footnotesize
\caption{Relation-Template Mapping Patterns.}
 % \lnk{need to refine}}
\label{table:template}
\footnotesize
%\resizebox{\linewidth}{!}{
\begin{tabular}{l l}
\toprule 
\textbf{Relation} & \textbf{Template Examples}  \\
    \midrule
{Noun Phrase} & \begin{tabular}[l]{@{}l@{}} - Is it true that 
$\langle \m{Subject}\rangle$ and 
$\langle\m{Object}\rangle$ share 
$\langle\m{Relation}\rangle$? 
\\ - $\langle\m{Subject}\rangle$ and $\langle\m{Object}\rangle$ have/made/shared totally different $\langle\m{Relation}\rangle$. Please judge the truthfulness of this statement.
%\\Please judge the truthfulness of this statement. 
    \end{tabular}  \\
    \midrule
    \begin{tabular}[l]{@{}l@{}} Verb Phrase \\ (Passive Voice) \end{tabular} & \begin{tabular}[l]{@{}l@{}} - Is it true that $\langle \m{Subject}\rangle$ is/was/are/were $\langle \m{Relation}\rangle$ $\langle\m{Object}\rangle$? \\ - It is impossible for $\langle \m{Subject}\rangle$ to be $\langle\m{Relation}\rangle$ $\langle\m{Object}\rangle$. Am I right?%\\ Other formats... 
    \end{tabular}  \\
    \midrule
\begin{tabular}[l]{@{}l@{}} Verb Phrase \\ (Active Voice) \end{tabular}
 & \begin{tabular}[l]{@{}l@{}} - Is it true that 
 $\langle \m{Subject}\rangle$
 $\langle\m{Relation}\rangle$
 $\langle\m{Object}\rangle$?  \\ - $\langle \m{Subject}\rangle$ $\langle\m{Relation}\rangle$ $\langle\m{Object}\rangle$. 
 %\\ Please judge the truthfulness of this statement.
 %\\ Other formats... 
 \end{tabular}  \\

    \bottomrule %添加表格底部粗线
\end{tabular}
%}
\end{table*}

\begin{table*}[!t]
\setlength{\tabcolsep}{3pt}
\centering
% \footnotesize
\caption{Temporal-Template Mapping Patterns (implicitly querying upon year $\iyear$).}
\label{table:temporal_template}
\footnotesize
\begin{tabular}{l  l }
\toprule 
\textbf{MTL Formulae} & \textbf{Template Examples}  \\
\midrule 
\mtltoNL($\nm$) &  Did $\langle\nm\rangle$ happen at year $\langle \iyear \rangle$? 
\\  \midrule
\mtltoNL($\mathcal{F}_\interval \,\mtl$) & Did ``Event'' finally happen within the time frame of $\langle \interval \rangle$ after the year $\langle \iyear \rangle$, where ``Event'' is defined as $\langle \mtltoNL(\mtl)\rangle$? 
\\ 
\midrule 
\mtltoNL($\mathcal{G}_\interval \,\mtl$) &  Did ``Event'' globally happen within the time frame of $\langle \interval \rangle$ after the year $\langle \iyear \rangle$, where ``Event'' is defined as $\langle \mtltoNL(\mtl)\rangle$?
\\ 
\midrule 
\mtltoNL($\mathcal{N} \,\mtl$) & Did ``Event'' happen in the next year of $\langle \iyear \rangle$, where ``Event'' is defined as $\langle \mtltoNL(\mtl)\rangle$? 
\\ 
\midrule 
\mtltoNL($\mtl_1 \, \mathcal{U}_\interval \,\mtl_2$) &  
\begin{tabular}[l]{@{}l@{}} 
Did ``Event$_1$'' happen continuously until ``Event$_2$'' started, during the period $\langle \interval \rangle$ after the year $\langle \iyear \rangle$, \\ 
where ``Event$_1$'' is $\langle \mtltoNL(\mtl_1) \rangle$ and ``Event$_2$'' is $\langle \mtltoNL(\mtl_2) \rangle$?
\end{tabular}
\\ 
\midrule 
\mtltoNL($\mtl_1  \,{\wedge}\,  \mtl_2$) &  Did both ``Event$_1$'' and  ``Event$_2$'' happen at year $\langle \iyear \rangle$, where ``Event$_1$'' is $\langle \mtltoNL(\mtl_1) \rangle$ and ``Event$_2$'' is $\langle \mtltoNL(\mtl_2) \rangle$? 
\\ 
\midrule 
\mtltoNL($\mtl_1  \,{\vee}\,  \mtl_2$) &  Did either ``Event$_1$'' or ``Event$_2$''  happen at year $\langle \iyear \rangle$, where ``Event$_1$'' is $\langle \mtltoNL(\mtl_1) \rangle$ and ``Event$_2$'' is $\langle \mtltoNL(\mtl_2) \rangle$? 
\\ 
\midrule 
\mtltoNL($\neg \mtl $) &  Did ``Event'' not happen at year $\langle \iyear \rangle$, where ``Event'' is $\langle \mtltoNL(\mtl) \rangle$? 
\\ 
\bottomrule %添加表格底部粗线
\end{tabular}
\end{table*}




\subsection{Benchmark Construction}
\label{prompt}
%From the derived facts, 
\tool constructs question-answer~(Q\&A) pairs and prompts to facilitate the testing for FCH. 
To address the challenge of the high human efforts required in the test oracle generation, we design an automated approach based on mapping relations between various entities to problem templates, largely reducing reliance on manual efforts. 

\textbf{\emph{Question Generation.}}
%\wkl{rewritten, check}
%\syh{checked, ok}
To facilitate efficient and systematic generation of test cases and prompts, we have adopted a method that leverages entity relationships and mappings of temporal operators to predefined Q\&A templates. 

When constructing relation-based Q\&A templates (without temporal operators), one key aspect lies in aligning various types of relations with the corresponding question templates from the mutated triples, i.e., the predicate type in the triple. Different relation types possess unique characteristics and expressive requirements, leading to various predefined templates. 
As listed in \tabref{table:template}, we map the relation types to question templates based on speech and the grammatical tense of the predicate to guarantee comprehensive coverage. 

When constructing temporal-logic-related queries, we define a mapping pattern for each temporal operator, as outlined in \tabref{table:temporal_template}. For any query expressed as ``$\mtl$''  with any concrete year $\iyear$ in query, the $\mtltoNL(\mtl)$ function converts the MTL formula $\mtl$ into a natural language query. In this context, $\mtltoNL(\mtl')$ is called recursively to generate the natural language description for the CTL subformula $\mtl'$. 



In both mapping patterns, we enhance the construction of the naturally formatted questions by leveraging an LLM to reformulate the structure and grammar of the Q\&A pairs. 

\textbf{\emph{Answer Generation.}}
We note that the answer to the corresponding question is readily attainable from the factual knowledge and the Prolog reasoning rules, defined in both \figref{fig:basic_op_for_predicates} and \figref{fig:encoding_rules_mtl}. 
Primarily, it is easy to determine whether the answer is \emph{true} or \emph{false} based on the mutated triples and the ground-truth time intervals using temporal reasoning. 
Meanwhile, mutated templates with positive and negative semantics via the usage of synonyms or antonyms, which greatly enhance the diversity of the questions, can be treated similarly as the negation rule defined in \figref{fig:basic_op_for_predicates}. 
Specifically, if the answer to a question with original semantics is Yes/No, then for a question with mutated opposite semantics, the corresponding answer would %naturally 
be the opposite, i.e., No/Yes. For example, after obtaining the original Q\&A pair \textit{- ``Is it true that Crohn's disease and Huntington's disease could share similar symptoms and signs? - Yes.''}, we can use antonyms to mutate it into \textit{- ``Is it true that Crohn's disease and Huntington's disease have different symptoms and signs? - No.''}
%\syh{what does this mean?}







% In total, we have defined 60 templates according to the pre-defined rules.
% , from which we have generated 194,850 Q\&A pairs. To prevent the dataset scope from becoming overly extensive, we conduct an initial screening based on categories of reasoning rules, ultimately yielding a total of 14,228 question pairs.
\begin{comment}   
\begin{table*}[!t]
    % \def\arraystretch{1.0}
    \setlength{\tabcolsep}{1ex}
	\centering
        \Large
	% \footnotesize
	\caption{Relation-Template Mapping Pattern.}
 % \lnk{need to refine}}
        \label{table:template}
	\resizebox{\linewidth}{!}{
	\begin{tabular}{l l l}
    \toprule 
    \textbf{Relation} & \textbf{Template} &\textbf{Example} \\
    \midrule
    \textbf{Noun Phrase} & \begin{tabular}[l]{@{}l@{}} - Is it true that \textit{Subject} and \textit{Object} share \textit{Relation}? \\ - \textit{Subject} and \textit{Object} have/made/shared totally different [Relation]. \\Please judge the truthfulness of this statement. \end{tabular} & \begin{tabular}[l]{@{}l@{}} New Triple: \textit{[Crohn's disease, similar\_symptoms\_and\_signs, Huntington's disease]}\\ Template: - Is it true that \textit{Crohn's disease} and \textit{Huntington's disease} share \textit{similar symptoms and signs}? -Yes. \\- Does \textit{Crohn's disease} and \textit{Huntington's disease} have similarities on {symptoms and signs}? - Yes. \end{tabular} \\
    \midrule
    \textbf{Verb Phrase in Passive Voice} & \begin{tabular}[l]{@{}l@{}} - Is it true that \textit{Subject} is/was/are/were \textit{Relation} \textit{Object}? \\ - It is impossible for \textit{Subject} to be \textit{Relation} \textit{Object}. Am I right?\\ Other formats... \end{tabular} & \begin{tabular}[l]{@{}l@{}} New Triple: \textit{[Kuratowski's theorem, not\_proved\_by, Kurt Gödel]}\\ Template: - Is it true that Kuratowski's theorem was proved by Kurt Gödel? - No.\\- Someone else other than Kurt Gödel proved Kuratowski's theorem, am I right? - Yes. \end{tabular} \\
    \midrule
    \textbf{Verb Phrase in Active Voice} & \begin{tabular}[l]{@{}l@{}} - Is it true that \textit{Subject Relation Object}?  \\ - \textit{Subject} \textit{Relation} \textit{Object}. Please judge the truthfulness of this statement.\\ Other formats... \end{tabular} & \begin{tabular}[l]{@{}l@{}} New Triple: \textit{[Baby Don't Lie, appeared\_before, Spark the Fire]}\\ Template: - Is it true that \textit{Baby Don't Lie} appeared before \textit{Spark the Fire}? - Yes.\\- \textit{Baby Don't Lie} never appeared before \textit{Spark the Fire}. \\ Please judge the truthfulness of this statement. -No.   \end{tabular} \\
    % \midrule
    % \textbf{Custom-Designed} & \begin{tabular}[l]{@{}l@{}l@{}l@{}l@{}} Given the \textit{SubjectList}, is it true that \textit{SubjectSelected} is the {Relation} among them?\\ Other formats... \end{tabular} & \begin{tabular}[l]{@{}l@{}l@{}l@{}l@{}} New Triple: [\textit{Roman Holiday, Hindenburg disaster newsreel footage, Lassie Come Home, The White Parade}, \textit{descending\_duration\_order}, \\ \textit{Roman Holiday, Lassie Come Home, The White Parade}]\\ Template: - Given the list [Roman Holiday, Hindenburg disaster newsreel footage, Lassie Come Home, The White Parade],\\ is it true that Roman Holiday have the longest duration among them? - Yes. \end{tabular} \\
    \bottomrule %添加表格底部粗线
\end{tabular}}
\end{table*}
\end{comment}

\textbf{\emph{Prompt Construction.}}
% \wkl{here maybe give a sample prompt screenshot as a figure to illustrate?}
As illustrated in \tabref{table:prompt}, before initiating any interaction with LLMs, we predefine specific instructions and prompts, requesting the model to utilize its inherent knowledge and inferential capabilities to deliver explicit (Yes/No/I don't know) judgments on our queries. Additionally, we instruct the model to present its reasoning process in a template following the judgment. The main goal is to ensure that LLMs give easy-to-understand responses using standardized prompts and instructions. 
This approach also helps the model to effectively exercise its reasoning abilities based on the given instructions and examples.


\begin{table*}[!t]
%\vspace{-0.3cm}
    % \def\arraystretch{1.0}
    \setlength{\tabcolsep}{1ex}
	\centering
    %\large 
	\small
	\caption{Prompt Template. %\shil{Shall we restrict the answer for relation query begnin with Yes/No/I don't know?}
    }
        \label{table:prompt}
        \vspace{-0.1cm}
	%\resizebox{\linewidth}{!}{
	\begin{tabular}{l}
    \toprule 
    \rowcolor{mycolor}
    \textbf{\instruction:} Answer the question with your knowledge and reasoning power.\\
    \midrule
    \rowcolor{mycolor} \textbf{\query (Relation):}  Given the $\langle \textit{question} \rangle$: \textit{question}, please provide an answer with your knowledge and reasoning power.\\ 
    \rowcolor{mycolor} Think of it step by step with a human-like reasoning process. After giving the answer, list the knowledge used in your\\ 
    \rowcolor{mycolor} reasoning process in the form of declarative sentences and point by point. The answer must contain `Yes', `No' or `I \\
    \rowcolor{mycolor} don't know' at the beginning. \\
    \midrule
    \rowcolor{mycolor} \textbf{\query (Temporal):}  Given the question: 
    $\langle \textit{question} \rangle$, please provide an answer with your knowledge and reasoning power \\
    \rowcolor{mycolor}  upon metric temporal logic. Think it step by step with a human-like reasoning process. After giving the answer, list the \\
    \rowcolor{mycolor} evidence from your temporal reasoning  in the form of declarative sentences and point by point. The answer must contain   \\
\rowcolor{mycolor} `Yes', `No' or `I don't know' at the beginning.\\
    \bottomrule %添加表格底部粗线
\end{tabular}
\end{table*}




\vspace{-1mm}
\subsection{Response Evaluation}\label{response}
%
%\begin{lstlisting}
% \begin{align*}
%     <program>::=&<decoder>|<query>|<model>|<condition>|<distribution>\\&|<query*>|<parser> \\
%     <query*>::=&statement(transform(s,R,o))\\
%     <parser>::=&extract(statement)\\
%     <transform>::=&[Neg]|[Sym]|[Inverse]|[Trans]|[Comp]
% \end{align*}
% \begingroup\vspace*{-1cm}
% \captionof{figure}{Syntax of Extended LMQL.}\label{sec:syntax}
% \vspace*{\baselineskip}\endgroup
%\end{lstlisting}
%
% \begin{lstlisting}[language=Python, caption=LMQL Program Grammar]
% <decoder> ::= argmax | beam(n=<int>) | sample(n=<int>)
% <query> ::= (<python_statement>)+
% <cond> ::= <cond> and <cond> | <cond> or <cond> | not <cond> | <cond_term>
% <cond_term> ::= <python_expression>
% <cond_op> ::= < | > | = | in
% <dist> ::= <var> over <python_expression>
% \end{lstlisting}
% \begin{grammar}
% <LMQL Program> ::= <decoder> <query>

% <decoder> ::= `argmax' | `beam(n=\textit{int})' | `sample(n=\textit{int})'

% <query> ::= (<python\_statement>)+

% <cond> ::= <cond> `and' <cond> | <cond> `or' <cond> | `not' <cond> | <cond\_term>

% <cond\_term> ::= <python\_expression>

% <cond\_op> ::= `<` | `>' | `=' | `in'

% <dist> ::= <var> `over' <python\_expression>
% \end{grammar}

% To facilitate the automated the query and answer validation process, we extend the previously proposed LMQL~(language Model Query Language)~\cite{Beurer-Kellner-2023} designed for LLM programming. LMQL utilizes SQL-like elements and a imperative syntax for scripting, as shown in Figure~\ref{sec:syntax}. More specifically, LMQL defines the interactive process with an LLM as a python-like $<program>$, including a $<decoder>$ to the decoding procedure employed by the LMQL runtime when solving the
% query, a $<query>$ to model the interaction with the model, a $<model>$ to denote the LLM to interact with, a $<condition>$ to place constraints on the variables in the program, a $<distribution>$ to represent the probability for output predictions from the LLM. language is augmented with additional constructs to facilitate the interaction and generation capabilities of LLMs.

% Firstly, we introduce the <query*> element as an extension to the existing <query> element. The <query*> block models the interaction with the LLM, serving as the prompt that is fed into the model. These query strings allow for the use of specially escaped subfields, similar to Python's f-strings. These subfields, denoted by "[varname]", represent phrases that will be generated by the LLM, also known as holes.

% Furthermore, we introduce a new element named <parser>. The <parser> element is responsible for extracting triples (subject-predicate-object) from the LLM-generated answers for further semantic comparison. The <parser> is capable of processing one or multiple sentences, extracting triples from each sentence sequentially and recording the derived results for further use. The <parser> element plays a crucial role in the answer validation process by breaking down the LLM's generated responses into structured triples. These triples can then be compared against a knowledge base or a set of predefined rules to validate the semantic correctness and coherence of the generated content.


The objective of this module is to enhance the detection of FCH in LLM outputs, specifically focusing on identifying the discrepancies between LLM responses and the verified ground truths. Recognizing the inherent limitation in directly accepting ``Yes'' or ``No'' answers from LLMs, our approach underscores the automated detection of factual consistency during the reasoning process presented by LLMs. 
% This analysis is vital for accurately determining the factual consistency of LLM responses, thereby addressing the primary challenge in identifying FCH within LLM outputs. 
% To achieve automated detection of factual consistency, our methodology first incorporates a parsing step that leverages advanced NLP techniques. This step is designed to extract essential semantic elements from each sentence within LLM outputs, assembling these elements into a coherent, semantic-aware structure. 
% The foundational premise of our approach is predicated on evaluating the semantic similarity between these constructed structures, aiming to discern the degree of consistency in their underlying semantics.
% Subsequently, we propose the development of a set of similarity-based testing oracles. These oracles are instrumental in applying metamorphic testing principles, enabling us to systematically assess the consistency or inconsistency between LLM responses and the established ground truth. 
Our approach is structured around several critical steps, as listed in \algoref{alg:eval} and detailed below:
% \shil{(SL: to cite Algorithm 3 and when explanation, referring to line number? For lines 11 to 15, use ``or" to include three conditions?)}
% The key target of this module is to detect the FCH by identifying the inconsistency between answers from the LLMs and the groundtruth in Q\&A pairs. However, as we unable to directly trust the yes or no answers from LLM directly, we need to parse the reasoning process carefully before reaching the verdict on the correctness with accuracy, which is the aforementioned challenges for detecting FCH in LLMs.  
% %This module outlines our approach for detecting hallucinations in the responses of the target LLMs. A key insight is the premise that any response contradicting the answer of the factual Q\&A pairs we provide is inherently regarded as an occurrence of FCH.
% To facilitate the automatic detection, we first design a parsing step to utilize an NLP-based approach to extract the critical semantic component from each sentence from LLM outputs, and assemble them into a semantic-aware structure. The key insight is to examine the similarity between these structures to determine the consistency in their semantics. Then, by designing a set of similarity-based testing oracles, we are able to utilize metamorphic testing to determine the (in)consistency between LLM answers and groundtruth.  This method primarily comprises the following steps:
%During the aforementioned prompt design process, we have prepared question prompts and their corresponding answers, thereby establishing ground truth Q\&A pairs. Therefore, we can automatically compare the LLM output with the expected ground truth answer to detect the discrepancies. Moreover, we utilize an NLP-based approach to compare the semantics in the reasoning process to identify inconsistencies to assist in understanding the cause of FCH. %our similarity between the responses from LLMs with these ground truth Q\&A pairs to verify the consistency and logical soundness of the LLMs' responses. 
% Detailed conclusions are discussed in the following section.


%In summary, we propose a method based on semantic parsing and metamorphic relations to verify whether LLM responses contain FCHs. This method primarily comprises the following steps:

%labelwidth=!,
\begin{enumerate}[wide,  labelindent=9pt]
%Step 1. 
\item \textbf{Preliminary Screening.} Given the LLM response $\llmResponse$, we first eliminate scenarios when the LLM declines to provide an answer, indicated by the ``answer'' field of LLM's responses. 
Most of these responses arise because the LLM lacks the relevant knowledge for the reasoning process. As these responses adhere to the LLM's principle of honesty, we categorize them as correct and normal responses.
% (as described in Algorithm~\ref{alg:eval} Line 7-8). 

%Step 2. 
\item \textbf{Response Parsing and Semantic Structure Construction.} Provided with the remaining suspicious responses from $\llmResponse$ and ground truth facts $\groundTruthTriples$, we use \textsc{ExtractTriple} function to extract triples that follow the same structure as the fact defined in the \secref{subsec3.1}. For each LLM response, the extracted triples ($\widetilde{\m{Trpl}}$) are based on the statements contained in the \textit{reasoning process} part of the LLM's response, which is further utilized to construct a response semantic structure $\semantic_{\m{resp}}$ using the \textsc{BuildGraph} function. In this structure, the $\widetilde{\entity}$ are depicted as \emph{nodes} ($\m{N}$), and the relational predicate ($\nm$) between them are illustrated as \emph{edges} ($\m{E}$). Concurrently using the same approach, we construct another semantic structure $\semantic_{\m{ground}}$ using $\groundTruthTriples$.

%Step 3. 
\item \textbf{Similarity-based Metamorphic Testing and Oracles.} 
We apply metamorphic relations to identify hallucination answers from LLMs, i.e., comparing the similarity between semantic structures generated by LLMs and the ground truth counterparts. Note that we provide four classifications: correct responses ($\m{CO}$), hallucinations caused by error inference ($\m{EI}$), hallucinations caused by erroneous knowledge ($\m{EK}$), and hallucinations containing both issues ($\m{OL}$). 
Specifically, the oracles for metamorphic testing can be divided into the following types:
 
% We then apply metamorphic relations to detect and evaluate potential errors in LLM responses, based on the relationships between inputs and outputs, rather than relying on traditional labeled data. 
% In our context, metamorphic relations specifically refer to comparing the similarity between semantic structures generated by LLMs and the ground truth counterparts, to identify and classify hallucination answers from LLMs.
% (as mentioned in Algorithm~\ref{alg:eval} Line 12-18). 
\end{enumerate}

\begin{comment}

tree = Leaf() | Node ()

resp{
    answer = bool 
    steps = tree ??
}

ground_truth{
    answer = bool 
    reasoning = tree
}


evaluation (resp, ground_truth)
    if resp.answer = refusal then CO 
    else 
        s_e = SE(resp, R_derived)
        s_n = SN(resp, R_derived)
        if s_e < threadhold_e then EI 
        else if s_n < threadhold_n then EK 
        else CO 


    
\end{comment}


% \lnk{check this algo}
% \begin{algorithm}[!h]
% \caption{Response Evaluation}
% \label{alg:eval}
% \small
% \begin{algorithmic}[1]
% \Require LLM Response ($\llmResponse$), Ground Facts ($\groundTruthTriples$), Threshold ($\theta_{\m{e}}, \theta_{\m{n}}$)
% \Ensure Evaluation Result ($\eval$)
% \Function{EvaluateResponse}{$\llmResponse$, $\groundTruthTriples$, $\theta_{\m{e}}$, $\theta_{\m{n}}$}
%     % \State $hallu\_ei, hallu\_ek, hallu\_both \gets$ [] \Comment{\commentstyle{Initialization}}
%     % \State $\eval, \eval_{\m{ei}}, \eval_{\m{ek}}, \eval_{\m{co}} \gets$ [] \Comment{\commentstyle{Initialization}}
%     % \State $refuse\_to\_answer \gets$ \Call{FindRefuseToAnswer}{$\llmResponse$} \Comment{\commentstyle{Find `refuse to answer' responses}}
%     % \State $suspicious\_resps \gets$ \Call{FilterSuspiciousRes}{$\llmResponse$, $GT\_Answer$} \Comment{\commentstyle{Filter suspicious responses}}
%     % \For{$\m{resp}$ in $\llmResponse$} \Comment{\commentstyle{Iterate each response}}
%     \If{$\llmResponse.answer = \m{refusal}$}
%         \State
%         $\eval$ $\in$ $CO$ \Comment{\commentstyle{Preliminary Screening}}
%     % \State $\eval_{\m{co}}$.append($\m{resp_\m{refusal}}$) \Comment{\commentstyle{Preliminary Screening}}
%     \Else
%         \State $\deriveKG{\m{resp}}{\groundTruthTriples}{\semantic_{\m{resp}}}{\semantic_{\m{ground}}}$
%         %\Comment{\commentstyle{Extract Semantic Structure}}
%         \State $\m{s}_{\m{e}} \gets$ $\similarity_\m{e}${$(\semantic_{\m{resp}}$, $\semantic_{\m{ground}})$} \Comment{\commentstyle{Calculate edge similarity}}
%         \State $\m{s}_{\m{n}} \gets$ $\similarity_\m{n}${$(\semantic_{\m{resp}}$, $\semantic_{\m{ground}})$} \Comment{\commentstyle{Calculate node similarity}}
%         % \If{$edge\_sim < \theta\_e$ and $node\_sim < \theta\_n$}
%             % \State $\eval$.append($response$) \Comment{\commentstyle{Append mixed hallucination}}
%         \If{$\m{s}_{\m{e}} < \theta_{\m{e}}$}
%             \State 
%             $\eval$ $\in$ $EI$  \Comment{\commentstyle{Append error inference hallucination}}
%         \ElsIf{$\m{s}_{\m{n}} < \theta_{\m{n}}$}
%             \State 
%             $\eval$ $\in$ $EK$  \Comment{\commentstyle{Append error knowledge hallucination}}
%         \Else
%             \State
%             $\eval$ $\in$ $CO$  
%             \Comment{\commentstyle{Append correct response}}
%         \EndIf
%     \EndIf
%     % \EndFor
%     % \State $\eval$.extend($\eval_{\m{ei}}, \eval_{\m{ek}}, \eval_{\m{co}}$) \Comment{\commentstyle{Merge the result}}
%     % \State $evaluation\_result \gets$ \Call{GenerateResult}{$hallu\_both$, $hallu\_ei$, $hallu\_ek$}
%     % \shil{(we use three lists \_both, \_ei, and \_ek)} $contradictory\_answers$} 
%     % \Comment{Generate evaluation result}
%     \State \Return $\eval$ \Comment{\commentstyle{Return the evaluation result}}
% \EndFunction
% \end{algorithmic}
% \end{algorithm}
\begin{algorithm}[!b]
\caption{Response Evaluation}
\label{alg:eval}
\small
\begin{algorithmic}[1]
\Require LLM Response ($\llmResponse$), Ground Facts ($\groundTruthTriples$), Threshold ($\theta_{\m{e}}, \theta_{\m{n}}$)
\Ensure Evaluation Result Category~($CO, EK, EI, OL$)
\Function{EvaluateResponse}{$\llmResponse$, $\groundTruthTriples$, $\theta_{\m{e}}$, $\theta_{\m{n}}$}
    % \State $hallu\_ei, hallu\_ek, hallu\_both \gets$ [] \Comment{\commentstyle{Initialization}}
    % \State $\eval, \eval_{\m{ei}}, \eval_{\m{ek}}, \eval_{\m{co}} \gets$ [] \Comment{\commentstyle{Initialization}}
    % \State $refuse\_to\_answer \gets$ \Call{FindRefuseToAnswer}{$\llmResponse$} \Comment{\commentstyle{Find `refuse to answer' responses}}
    % \State $suspicious\_resps \gets$ \Call{FilterSuspiciousRes}{$\llmResponse$, $GT\_Answer$} \Comment{\commentstyle{Filter suspicious responses}}
    % \For{$\m{resp}$ in $\llmResponse$} \Comment{\commentstyle{Iterate each response}}
    \State $CO, EK, EI, OL \gets []$ \Comment{\commentstyle{Initialization}}
    \If{$\llmResponse.answer = refusal$}
        \State
        $CO.\m{append}(\llmResponse)$ \Comment{\commentstyle{Preliminary Screening}}
    % \State $\eval_{\m{co}}$.append($\m{resp_\m{refusal}}$) \Comment{\commentstyle{Preliminary Screening}}
    \Else
        \State $\widetilde{\m{Trpl}} \gets$ \Call{ExtractTriple}{$\m{Resp.reasoning}$} 
        % \Comment{\commentstyle{Extract useful triples}}
        \State $\semantic_{\m{resp}}, \semantic_{\m{ground}} \gets$ \Call{BuildGraph}{$\widetilde{\m{Trpl}}, \groundTruthTriples$} 
        % \Comment{\commentstyle{Build semantic structure}}
        % \State $\deriveKG{\m{Resp}}{\rall}{\semantic_{\m{resp}}}{\semantic_{\m{ground}}}$\lnk{More specific} \Comment{\commentstyle{Extract Semantic Structure}}
        \State $\m{s}_{\m{e}} \gets$ $\similarity_\m{e}${$(\semantic_{\m{resp}}$, $\semantic_{\m{ground}})$} \Comment{\commentstyle{Calculate edge similarity}}
        \State $\m{s}_{\m{n}} \gets$ $\similarity_\m{n}${$(\semantic_{\m{resp}}$, $\semantic_{\m{ground}})$} \Comment{\commentstyle{Calculate node similarity}}
        % \If{$edge\_sim < \theta\_e$ and $node\_sim < \theta\_n$}
            % \State $\eval$.append($response$) \Comment{\commentstyle{Append mixed hallucination}}
        \If {$s_e < \theta_{e}$ and $s_n < \theta_{n}$}
            \State
            $OL.\m{append}(\llmResponse)$  \Comment{\commentstyle{Append  overlapped cases}}
        \ElsIf{$\m{s}_{\m{e}} < \theta_{\m{e}}$}
            \State 
            $EI.\m{append}(\llmResponse)$  \Comment{\commentstyle{Append error inference}}
        \ElsIf{$\m{s}_{\m{n}} < \theta_{\m{n}}$}
            \State 
            $EK.\m{append}(\llmResponse)$  \Comment{\commentstyle{Append error knowledge}}
        \Else
            \State
            $CO.\m{append}(\llmResponse)$
            \Comment{\commentstyle{Append correct response}}
        \EndIf
    \EndIf
    % \EndFor
    % \State $\eval$.extend($\eval_{\m{ei}}, \eval_{\m{ek}}, \eval_{\m{co}}$) \Comment{\commentstyle{Merge the result}}
    % \State $evaluation\_result \gets$ \Call{GenerateResult}{$hallu\_both$, $hallu\_ei$, $hallu\_ek$}
    % \shil{(we use three lists \_both, \_ei, and \_ek)} $contradictory\_answers$} 
    % \Comment{Generate evaluation result}
    \State \Return $CO, EK, EI, OL$ 
    % \Comment{\commentstyle{Return the result}}
\EndFunction
\end{algorithmic}
\end{algorithm}

\textbf{Edge Vector Metamorphic Oracle ($MO_E$)}: This oracle is based on the similarity of edge vectors between $\semantic_{\m{resp}}$ and $\semantic_{\m{ground}}$. If the vector similarity ($\m{s}_{\m{e}}$) between the edges in the $\semantic_{\m{resp}}$ and those in $\semantic_{\m{ground}}$ falls below a predetermined threshold $\theta_{\m{e}}$, it indicates that the LLM's answer significantly diverges from the ground truth. This suggests the presence of an FCH, and vice versa. %Conversely, the LLM's answer is considered to {align with} the ground truth. % indicates the correct answer. Otherwise, it detects an occurrence of FCH.
More specifically, we utilize \emph{Jaccard Similarity}~\cite{J_S} to calculate the similarity score between edge vectors extracted from $\semantic_{\m{resp}}$ and  $\semantic_{\m{ground}}$. 
$$
\similarity_{\m{e}}(\semantic_{\m{resp}}, \semantic_{\m{ground}}) = \frac{|\widetilde{E}_{\m{resp}} \cap \widetilde{E}_{\m{ground}}|}{|\widetilde{E}_{\m{resp}} \cup \widetilde{E}_{\m{ground}}|}, $$check if $  \similarity_{\m{e}}(\semantic_{\m{resp}}, \semantic_{\m{ground}})  < \theta_{\m{e}} \ 
$~
where $\widetilde{E}_{\m{resp}}$ and $\widetilde{E}_{\m{ground}}$ denote the set of edges extracted from $\semantic_{\m{resp}}$ and $\semantic_{\m{ground}}$. 
% , and \( \theta_E \) is a predefined threshold~(to be detailed in Section~\ref{sec:ex_setup}). 
% Intuitively, the similarity score is calculated as the proportion of identical edges shared between the two sets against the total number of unique edges in both sets. If the similarity score is smaller than the threshold, then an FCH is detected. Note that when determining the joint and union of sets $E_{LLM}$ and $E_{GT}$, we consider two edges as identical if their corresponding relations are identical or represented by synonymous words, and vice versa.  

% Define a function \( \text{Sim}_E(KG_{LLM}, KG_{GT}) \) to calculate the similarity of edge vectors between the knowledge graph generated by the language model, \( KG_{LLM} \), and the ground truth knowledge graph, \( KG_{GT} \).
% $$\text{Sim}_E(KG_{LLM}, KG_{GT}) = \frac{|E_{\text{LLM}} \cap E_{\text{GT}}|}{|E_{\text{LLM}} \cup E_{\text{GT}}|}$$

% If \( \text{Sim}_E(KG_{LLM}, KG_{GT}) < \theta_E \), where \( \theta_E \) is a predefined threshold, then an error inference hallucination is identified.

   
\textbf{Node Vector Metamorphic Oracle ($MO_N$)}: This relation examines the similarity of node vectors between $\semantic_{\m{resp}}$ and $\semantic_{\m{ground}}$. 
Defined in a similar manner as $MO_E$, if the node similarity between the nodes ($\m{s}_{\m{n}}$) in the $\semantic_{\m{resp}}$ and those in $\semantic_{\m{ground}}$ falls below a predetermined threshold $\theta_{\m{n}}$, it indicates that the LLM's answer significantly diverges from the ground truth, and vice versa.
$MO_N$ can be captured by the Jaccard Similarity, defined as follows:
%When the similarity between the nodes in the $KG_{LLM}$ and those in $KG_{GT}$ is below a predetermined threshold, this metamorphic relation exposes an error knowledge hallucination.
%Define a function \( \text{Sim}_N(KG_{LLM}, KG_{GT}) \) to measure the similarity of node vectors between \( KG_{LLM} \) and \( KG_{GT} \).

$$\similarity_{\m{n}}(\semantic_{\m{resp}}, \semantic_{\m{ground}}) = \frac{|N_{\m{resp}} \cap N_{\m{ground}}|}{|N_{\m{resp}} \cup N_{\m{ground}}|}, $$check if $
\similarity_{\m{n}}(\semantic_{\m{resp}}, \semantic_{\m{ground}})  < \theta_{\m{n}}  
$~
where $N_{\m{resp}}$ and $N_{\m{ground}}$ denotes the set of nodes extracted from $\semantic_{\m{resp}}$ and $\semantic_{\m{ground}}$.
% , and \( \theta_N \) is a predefined threshold~(to be detailed in Section~\ref{sec:ex_setup}). 
% Intuitively, the similarity score is calculated as the proportion of identical edges/nodes shared between the two sets against the total number of unique edges/nodes in both sets. If the similarity score is smaller than the threshold, then an FCH is detected. 
Note that when determining the joint and union of the edges/nodes sets, we consider two edges/nodes as identical if their corresponding entities are identical or synonymous, and vice versa.
%If \( \text{Sim}_N(KG_{LLM}, KG_{GT}) < \theta_N \), where \( \theta_N \) is a predetermined threshold, then an error knowledge hallucination is recognized.


% \textbf{Answer Consistency Metamorphic Oracle ($MO_C$)}: This relation is distinct in that it focuses on the consistency or inconsistency of the model's final answer with the ground truth, regardless of whether the node or edge vector similarities meet the thresholds. This relation helps identify situations where, despite a seemingly reasonable reasoning process, the outcome contradicts the facts (or vice versa), indicating contradictory answers.

% Consider the final answer \( Ans_{LLM} \) provided by LLMs and the ground truth answer \( Ans_{GT} \).
% If the similarity between \( KG_{LLM} \) and \( KG_{GT} \) is above or below the threshold but there exists a contradiction or consistency between \( Ans_{LLM} \) and \( Ans_{GT} \), this scenario is considered as a contradictory answer.
% $$ \text{If similarity between } KG_{LLM} \text{ and } KG_{GT} \text{ is above or below threshold and } A_{LLM} \neq A_{GT} \text{, then identify a contradictory answer.} $$


% \subsection{Feedback Loop}
% Based on the evaluation results from the preceding section, this module is employed to select test case types that trigger higher levels of FCHs and to mutate them for more hallucination answers, thereby enhancing the ability of the testing process to expose LLM FCHs.


    
    \section{Experiments}
        \label{sec:experiments}
        \section{Experiments}\label{sec:exp}

\begin{table}[t]
\centering
\caption{\textbf{Quantitative results on OpenCompass~\cite{2023opencompass} multimodal leaderboard.}
$^{\ddag}$ denotes closed-source models. Hall denotes HallusionBench.
}
\label{tab:exp_it_oc}
\setlength{\tabcolsep}{1pt}
\begin{tabular}{l|c|c|cccccccc}
\toprule
Models   & Params & Avg. & MM- & MM- & MM- & Math- & Hall & AI2D  & OCR- & MMVet \\
   &  &  & Bench & Star & MU & Vista &  &  & Bench & \\
\midrule
Step-1o$^{\ddag}$   & N/A   & \textbf{77.7}  & 87.3  & 69.3  & 69.9 & 74.7  & 55.8 & 89.1 & 926 & \textbf{82.8}  \\
SenseNova$^{\ddag}$  & N/A   & 77.4  & 85.7  & \textbf{72.7}  & 69.6 & \textbf{78.4}  & 57.4 & 87.8 & 894 & 78.2  \\
InternVL2.5-78B-MPO~\cite{wang2024mpo}  & 78B  & 77.0   & 87.7  & 72.1  & 68.2  & 76.6  & 58.1  & 89.2 & 909 & 73.5  \\
Qwen2.5-VL-72B~\cite{bai2025qwen25vltechnicalreport}   & 73.4B  & 76.2  & \textbf{87.8}  & 71.1  & 67.9  & 70.8  & 58.8  & 88.2  & 881  & 76.7  \\
TeleMM$^{\ddag}$   & N/A   & 75.9  & 79.9 & 70.8 & 66.6 & 75.7  & \textbf{60.6}  & 88.5 & 891 & 75.7  \\
InternVL2.5-38B-MPO~\cite{wang2024mpo}  & 38B  & 75.3  & 85.4  & 70.1 & 63.8 & 73.6 & 59.7 & 87.9 & 894 & 72.6  \\
InternVL2.5-78B~\cite{chen2024expanding}  & 78B  & 75.2 & 87.5  & 69.5 & 70 & 71.4 & 57.4 & 89.1 & 853 & 71.8   \\
Qwen2-VL-72B~\cite{qwen2-vl_2024}   & 73.4B  & 74.8  & 85.9  & 68.6  & 64.3  & 69.7  & 58.7  & 88.3  & 888  & 73.9  \\
InternVL2.5-38B~\cite{chen2024expanding}  & 38B  & 73.5  & 85.4  & 68.5  & 64.6  & 72.4  & 57.9  & 87.6  & 841  & 67.2  \\
JT-VL-Chat-V3.0$^{\ddag}$  & N/A   & 73.4  & 81.7  & 67.5  & 59.3  & 71.9  & 53.9  & 87.2  & \textbf{967}  & 69.2  \\
Taiyi$^{\ddag}$  & N/A   & 73.0  & 84.8  & 69  & 60.4  & 72.3  & 56.8  & \textbf{90.8}  & 820  & 67.9  \\
Step-1.5V$^{\ddag}$  & N/A   & 72.5 & 82.0  & 65.1  & 61.2  & 69.7  & 54.3  & 87.5  & 886  & 71.3  \\
Gemini-1.5-Pro-002$^{\ddag}$~\cite{geminiteam2024gemini15unlockingmultimodal}   & N/A   & 72.1 & 82.8  & 67.1  & 68.6  & 67.8  & 55.9  & 83.3  & 770  & 74.6  \\
InternVL2.5-26B-MPO~\cite{wang2024mpo}  & 26B  & 72.1  & 84.2  & 67.7  & 56.4  & 71.5  & 52.4  & 86.2  & 905  & 68.1  \\
GPT-4o-20241120$^{\ddag}$~\cite{openai2024gpt4ocard}  & NA   & 72.0   & 84.3  & 65.1  & \textbf{70.7}  & 59.9  & 56.2  & 84.9  & 806  & 74.5  \\
LLaVA-OneVision-72B~\cite{li2024llavaonevision}  & 73B  & 68.0  & 84.5  & 65.8  & 56.6  & 68.4  & 47.9  & 86.2  & 741  & 60.6  \\
NVLM-D-72B~\cite{nvlm2024}   & 79.4B  & 67.6  & 78.5  & 63.7  & 60.8  & 63.9  & 49.7  & 80.1  & 849  & 58.9  \\
Molmo-72B~\cite{deitke2024molmo}  & 73.3B  & 64.1  & 79.5  & 63.3  & 52.8  & 55.8  & 46.6  & 83.4  & 701  & 61.1  \\
\rowcolor{Gray} \textbf{\method-72B}   & 71.8B  & 75.1  & 86.3  & 70.7  & 57.6  & 73.3  & 56.4  & 87.6  & 889   & 79.8  \\
\midrule
\multicolumn{11}{l}{\textit{Models smaller than 20B}} \\
\midrule
Ola-7b~\cite{ola_2025}   & 8.88B   & \textbf{72.6}  & \textbf{84.3}  & \textbf{70.8}  & \textbf{57.0}  & 68.4  & \textbf{53.5}  & \textbf{86.1}  & 822  & \textbf{78.6}  \\
Qwen2.5-VL-7B~\cite{bai2025qwen25vltechnicalreport}   & 8.29B   & 70.4  & 82.6  & 64.1  & 56.2  & 65.8  & 56.3  & 84.1  & 877  & 66.6  \\
InternVL2.5-8B-MPO~\cite{wang2024mpo}   & 8B   & 70.3  & 82  & 65.2  & 54.8  & 67.9  & 51.7  & 84.5  & \textbf{882}  & 68.1  \\
MiniCPM-o-2.6~\cite{yao2024minicpm}   & 8.67B   & 70.2  & 80.6  & 63.3  & 50.9  & \textbf{73.3}  & 51.1  & 86.1  & 889  & 67.2  \\
Ovis1.6-Gemma2-9B~\cite{lu2024ovis}  & 10.2B  & 68.8  & 80.5  & 62.9  & 55.0  & 67.2  & 52.2  & 84.4  & 830  & 65.0  \\
InternVL2.5-8B~\cite{chen2024expanding}   & 8B   & 68.1  & 82.5  & 63.2  & 56.2  & 64.5  & 49.0  & 84.6  & 821  & 62.8  \\
POINTS1.5-Qwen2.5-7B~\cite{points1.5_2024} & 8.3B   & 67.4  & 80.7  & 61.1  & 53.8  & 66.4  & 50.0  & 81.4  & 832  & 62.2  \\
Valley-Eagle$^{\ddag}$   & 8.9B   & 67.4  & 80.7  & 60.9  & \textbf{57.0}  & 64.6  & 48.0  & 82.5  & 842  & 61.3  \\
Qwen2-VL-7B~\cite{qwen2-vl_2024}  & 8B   & 67.0  & 81.0 & 60.7 & 53.7 & 61.4  & 50.4 & 83 & 843 & 61.8 \\
DeepSeek-VL2~\cite{wu2024deepseekvl2}   & 16.1B  & 66.4  & 81.2  & 61.0  & 50.7  & 59.4  & 51.5  & 84.5  & 825  & 60.0  \\
VITA-1.5~\cite{fu2025vita}   & 8.3B   & 63.3  & 76.8  & 60.2  & 52.6  & 66.2  & 44.6  & 79.2  & 741  & 52.7  \\
Baichuan-Omni~\cite{baichuan-omni}   & 7B   & -  & 75.6  & -  & 47.3  & 51.9  & 47.8  & -  & 700  & 65.4  \\
LLaVA-OneVision-7B~\cite{li2024llavaonevision}   & 8B   & 61.2  & 76.8  & 56.7  & 46.8  & 58.5  & 47.5  & 82.8  & 697  & 50.6  \\
Molmo-7B-D~\cite{deitke2024molmo}   & 8B   & 58.9  & 70.9  & 54.4  & 48.7  & 47.3  & 47.7  & 79.6  & 694  & 53.3  \\
% MiniCPM-o 2.6~\cite{yao2024minicpm}   & 8B   & 70.2  & 80.5  & 64.0  & 50.4  & 71.9  & 51.9  & 85.8  & 897  & 67.5  \\
\rowcolor{Gray} \textbf{\method-9B}  & 8.8B   & 69.7  & 80.7  & 60.5  & 51.2  & 68.3  & 51.8  & 84.5  & 883 & 72.3 \\
\bottomrule
\end{tabular}
\end{table}

\begin{table}[t]
  \caption{\textbf{Performance comparison on video and Interleave benchmarks} compared with existing approaches. $^*$ indicates officially released checkpoints evaluated by us. Best performance is marked \textbf{bold}. }
  \label{tab: video_n_interleave}
  \centering
  \setlength{\tabcolsep}{7.5pt}
  \begin{tabular}{lccccc}
    \toprule
       & \multicolumn{2}{c}{\textbf{VideoMME}} & \multicolumn{1}{c}{\textbf{MVBench}} & \multicolumn{2}{c}{\textbf{Llava-Interleave}}\\
    \cmidrule(r){2-3} \cmidrule(r){4-4} \cmidrule(r){5-6}
    Model & w/o subs & w subs & avg & in-domain & out-domain \\
    \midrule
     MiniCPM-V-2.6~\cite{yao2024minicpm} &  60.9 &  63.6 &  - &  - &  - \\
     LLaVA-OneVision-7B~\cite{li2024llavaonevision} &  58.2 &  - &  - &  - &  - \\
     Qwen2-VL-7B~\cite{qwen2-vl_2024} &  63.3 &  69.0 &  67.0 &  49.5$^*$ &  51.0$^*$ \\
     InternVL2-8B~\cite{chen2024far} &  56.3 & 59.3 &  65.8 &  - &  - \\
     VITA-1.5~\cite{fu2025vita} &  56.1 & 58.7 &  55.4 &  - &  - \\
     Baichuan-Omni~\cite{baichuan-omni} &  58.2 & - &  60.9 &  - &  - \\
     MiniCPM-o-2.6~\cite{yao2024minicpm} & 63.0$^*$ & 65.3$^*$ & 58.1$^*$ &  43.5$^*$ &  36.8$^*$ \\
     \rowcolor{Gray} \textbf{\method-9B} &  60.4  & 65.0 &  66.3 &  59.8 &  87.8 \\
    \midrule
    VideoLLaMA2-72B~\cite{cheng2024videollama2} & 61.4 & 63.1 & 62.0 & - & - \\
    LLaVA-OneVision-72B~\cite{li2024llavaonevision} &  66.2 &  69.5 &  59.4 &  - &  - \\
    Qwen2-VL-72B~\cite{qwen2-vl_2024} &  71.2 &  77.8 &  \textbf{73.6} &  - &  - \\
    InternVL2-Llama3-76B~\cite{chen2024far} &  64.7  & 67.8 &  69.6 &  - &  - \\
    \rowcolor{Gray} \textbf{\method-72B} &  65.2  & 67.7 &  69.6 &  \textbf{63.5} &  \textbf{89.9} \\
    \midrule
    GPT-4v~\cite{GPT4VisionSystemCard} & 59.9 & 63.3 & 43.7 & 39.2 & 57.78 \\
    GPT-4o-20240513~\cite{openai2024gpt4ocard} & 71.9 & 77.2 & - & - & - \\
    Gemini-1.5-Pro~\cite{geminiteam2024gemini15unlockingmultimodal} & \textbf{75.0} & \textbf{81.3} & - & - & - \\
    \bottomrule
\end{tabular}
\end{table}


In this section, we present a comprehensive evaluation of our \method model, comprising both quantitative and qualitative analyses of its performance. Furthermore, we conduct ablation studies to analyze the contributions of several key design components to the performance of our \method model, providing insights into their distinct impacts.

% In this section, we first evaluate the model’s performance on a variety of mainstream benchmarks, demonstrating the advantages of \method.
% Then, a series of qualitative results are presented to show the model’s specific capabilities, including multimodal understanding and free-form image generation.
% Finally, we conduct an ablation study to analyze several key components in \method.

\subsection{Quantitative Results}\label{subsec:exp_quantitative_results}

\subsubsection{Image-Text Understanding}
To evaluate the effectiveness of our \method in image-text understanding, we benchmark it against state-of-the-art MLLMs on the OpenCompass~\cite{2023opencompass} multimodal leaderboard, a widely recognized platform for multimodal evaluation. This leaderboard contains 8 different multimodal benchmarks, including complex VQA (MMBench~\cite{liu2025mmbench}, MMStar~\cite{chen2024we}, MMMU~\cite{yue2023mmmu}, AI2D~\cite{kembhavi2016diagram}, and MMVet~\cite{yu2024mm}), multimodal reasoning (MathVista~\cite{lu2024mathvista}), hallucination evaluation (Hallusionbench~\cite{Guan_2024_hallusionbench}), and OCR (OCRBench~\cite{Liu_2024}).
\cref{tab:exp_it_oc} shows the overall results. Our \method-72B model achieves top-tier performance on most benchmarks, surpassing closed-source models like GPT-4o and Gemini-1.5-Pro. Furthermore, our \method-9B model exhibits competitive performance among models of similar size, showcasing its robust capabilities in image-text understanding tasks.

% In this section, we compare our \method with leading MLLMs on the mainstream OpenCompass~\cite{2023opencompass} multimodal leaderboard to demonstrate its advancement on image-text understanding.
% This leaderboard contains 8 different multimodal benchmarks, including complex VQA (MMBench~\cite{liu2025mmbench}, MMStar~\cite{chen2024we}, MMMU~\cite{yue2023mmmu}, AI2D~\cite{kembhavi2016diagram}, and MMVet~\cite{yu2024mm}), multimodal reasoning (MathVista~\cite{lu2024mathvista}), hallucination evaluation (Hallusionbench~\cite{Guan_2024_hallusionbench}), and OCR (OCRBench~\cite{Liu_2024}).
% \cref{tab:exp_it_oc} shows the overall results.
% \method exhibits competitive performance compared with other MLLMs.
% Our \method-71B model achieves top-tier performance on most benchmarks.
% It outperforms closed-source models such as GPT-4o and Gemini-1.5-Pro.
% The \method-9B model also achieves competitive performance among other vision-language specific MLLM models smaller than 20B.
% Notably, it achieves excellent performance on MathVista, AI2D, and MMVet, demonstrating its comprehensive ability on multimodal reasoning and complex VQA.




\subsubsection{Video \& Interleaved Image-Text Understanding}

We evaluate our model's video and interleaved image-text understanding abilities on three mainstream benchmarks.

\textbf{Video-MME}~\cite{fu2024video}: Video-MME is a benchmark designed to evaluate MLLMs in full-spectrum video analysis. It encompasses a wide variety of video types across multiple domains and durations, featuring multimodal inputs such as video, subtitles, and audio. For this benchmark, testing is conducted with under 96 frames, and results are reported for both "with subtitles" and "without subtitles" settings.

\textbf{MVBench}~\cite{li2024mvbench}: MVBench serves as a video understanding benchmark aimed at thoroughly evaluating the temporal awareness of MLLMs in an open-world context. It includes 20 challenging video tasks that range from perception to cognition, which cannot be adequately addressed using a single frame. Testing for this benchmark utilizes dynamic sampling frames.

\textbf{LLaVA-Interleave Bench}~\cite{llava-next_2024}: LLaVA-Interleave Bench comprises a comprehensive suite of multi-image benchmarks collected from public datasets or generated via the GPT-4V API. It is created to assess the interleaved multi-image reasoning capabilities of MLLMs, with reported results for both "in-domain" and "out-domain" subsets.

As shown in Table~\ref{tab: video_n_interleave}, \method-9B achieves the second-best results across VideoMME and MVBench (outperformed only by Qwen2-VL-7B but requiring significantly fewer frames). However, the performance gains do not scale up to \method-72B due to limitations in the quantity of instruction-tuned video data. Moreover, both our \method-9B and \method-72B greatly surpass all other baselines in multi-image benchmarks, both in-domain and out-of-domain, highlighting their potential as strong competitors for complex tasks.


\subsubsection{Audio Understanding}

We evaluate our M2-omni model's audio understanding abilities on four mainstream benchmarks.

\textbf{Multilingual LibriSpeech (MLS)}~\cite{MLS_English}: The Multilingual LibriSpeech dataset is an extensive collection of read audiobooks sourced from Librivox, available in eight different languages. We utilize the English test set from this dataset to assess the model's speech comprehension capabilities. The latest version of this corpus comprises approximately 50,000 hours.

\textbf{Librispeech}~\cite{Librispeech}: The Librispeech corpus comprises approximately 1,000 hours of transcribed speech audio data derived from read English audiobooks. The entire dataset is categorized into three training sets (100 hours of clean, 360 hours of clean, and 500 hours of other), two validation sets (clean and other), and two test sets (clean and other). In this study, we assess our model's audio comprehension capabilities using both the clean and other testsets.

\textbf{Aishell1}~\cite{AISHELL1}:  The Aishell1 dataset comprises 178 hours of speech data, recorded by 400 speakers from various accent regions across China. It is organized into three subsets: a training set consisting of 340 speakers, a validation set with 40 speakers, and a test set featuring 20 speakers.

\textbf{AudioCaps}~\cite{AudioCaps}: AudioCaps is a comprehensive dataset featuring audio event descriptions specifically curated for the purpose of audio captioning. The sounds within this collection are derived from the AudioSet dataset. We utilize this dataset to assess the audio captioning capabilities of our \method.
 % To facilitate accurate captioning, annotators were supplied with audio tracks and corresponding categorical hints, with additional video hints provided as necessary.

The results are presented in Table~\ref{tab:exp_audio_understand}, and our \method-9B demonstrates competitive performance in speech recognition and audio captioning tasks. 
Specifically, our \method-9B is comparable to GPT-4o-Realtime~\cite{openai2024gpt4ocard}.
In addition, \method-9B significantly outperforms all other baselines on AudioCaps benchmarks, while achieving the second-best results for the MLS English, Librispeech other, Librispeech-clean and Aishell1 benchmarks.

\begin{table}[]
\centering
\caption{\textbf{Quantitative results on speech recognition and audio captioning.}
 $^*$ indicates results from \cite{yao2024minicpm}.
}
\label{tab:exp_audio_understand}
\setlength{\tabcolsep}{7pt}
\begin{tabular}{l|cccccccccc}
\toprule
Models   & MLS- & Librispeech- & Librispeech- & Aishell1 & AudioCaps \\
                & English & other & clean &  & \\
                & WER$\downarrow$ & WER$\downarrow$ & WER$\downarrow$ & WER$\downarrow$ & CIDER$\uparrow$ \\
\midrule
UIO2-L-1.1B~\cite{lu2023uio2}   & - & - & - & - & 45.7   \\
UIO2-XL-3.2B~\cite{lu2023uio2}  & - & - & - & - & 45.7   \\
UIO2-XXL-6.8B~\cite{lu2023uio2} & - & - & - & - & 48.9  \\
Whisper-large-v2~\cite{Whisper}  & \textbf{6.83} & \textbf{5.16} & 2.87 & - & - \\
Paraformer-cn~\cite{gao2022paraformer} & - & - & - & 2.12 & - \\
VITA-1.5~\cite{VITA_1.5} & - & 7.5 & 3.4 & 2.2 & - \\
Mini-Omini2~\cite{mini_omni2} & - & 9.8 & 4.8 & - & - \\
Freeze-Omini~\cite{Freeze_Omni} & - & 10.5 & 4.1 & 2.8 & - \\
MiniCPM-o-2.6~\cite{yao2024minicpm} & - & - & \textbf{1.7} & \textbf{1.6} & - \\
GPT-4o-Realtime~\cite{openai2024gpt4ocard} & - & - & 2.6$^*$ & 7.3$^*$ & - \\
\rowcolor{Gray} \textbf{\method-9B}   & 7.19 & 5.29 & 2.07 & 1.99 & \textbf{49.2} \\
\bottomrule
\end{tabular}
\end{table}

\begin{table}[t]
\centering
\caption{\textbf{Quantitative results on language benchmarks.} $^*$ indicates officially released checkpoints evaluated using the tools provided by OpenCompass~\cite{2023opencompass}.
}
\label{tab:exp_language}
\setlength{\tabcolsep}{5pt}
\begin{tabular}{cccccccc}
\hline
Tasks & MMLU & AGIEVAL & ARC-C & GPQA & MATH & HellaSwag & \begin{tabular}[c]{@{}l@{}}Avg.\\ Accuracy\end{tabular} \\ \hline
LLama3.1-8B & 69.4 & 41.2$^*$ & 83.4 & 30.4 & 51.9 & 75.1$^*$ & 58.6  \\
\rowcolor{Gray} \textbf{\method-9B} & 68.5 & 43.7 & 78.7 & 32.3 & 51.8 & 80.1 & 59.2  \\ \hline
\end{tabular}
\end{table}

\subsubsection{Audio Generation}
In this section, we also evaluated our model on the commonly-used test set: SEED-TTS test-zh. \textbf{SEED-TTS}~\cite{SEED_TTS} serves as an out-of-domain evaluation test set, comprising diverse input texts and reference speeches from various domains. We present the experimental results for \method-9B and the baseline models in Table~\ref{tab:exp_audio_generation}. As shown in Table~\ref{tab:exp_audio_generation}, our model outperforms MiniCPM-o-2.6~\cite{yao2024minicpm} in speech generation capability, achieving significant improvements in both evaluation metrics. However, our \method-9B still lags behind traditional vertical speech generation models, highlighting the need for further research and development to bridge this gap.


\subsubsection{Text-only Performance}
In this section, we assess the performance of our proposed \method-9B model and its initial counterpart, Llama3.1-8B~\cite{llama3_2024}. To evaluate the models' knowledge and examination capabilities, we employ a range of benchmarks, including AGIEVAL~\cite{zhong2023agievalhumancentricbenchmarkevaluating} and MMLU~\cite{hendrycks2021measuringmassivemultitasklanguage}. Furthermore, we utilize a diverse set of benchmarks to evaluate the models' multi-step problem-solving capabilities, including MATH~\cite{hendrycks2021measuringmathematicalproblemsolving} for mathematical derivation, HellaSwag~\cite{zellers2019hellaswagmachinereallyfinish} for commonsense reasoning in real-world contexts, ARC-C~\cite{allenai:arc} for scientific logical chains, and GPQA~\cite{rein2023gpqagraduatelevelgoogleproofqa} for critical analysis in expert-level domains. For all evaluation datasets, we adopt a generation-based assessment approach with greedy decoding.

Our experimental results, presented in \cref{tab:exp_language}, demonstrate that the performance of our proposed \method-9B model outperforms its initial counterpart, Llama3.1-8B across most evaluation datasets,   which is attributed to our multi-stage language preservation strategy and the high-quality instruction tuning data used in our training process.

% In this section, we evaluate the performance of our \method-9B and its initial Llama3.1~\cite{llama3_2024} models. To assess the models' knowledge and examination capabilities, we utilize the AGIEVAL~\cite{zhong2023agievalhumancentricbenchmarkevaluating},  MMLU~\cite{hendrycks2021measuringmassivemultitasklanguage} benchmarks. Additionally, we employ  MATH~\cite{hendrycks2021measuringmathematicalproblemsolving}, HellaSwag~\cite{zellers2019hellaswagmachinereallyfinish}, ARC-C~\cite{allenai:arc} and GPQA~\cite{rein2023gpqagraduatelevelgoogleproofqa} to evaluate the models' multi-step problem-solving ability, including mathematical derivation, commonsense reasoning in real-world contexts, scientific logical chains, and critical analysis in expert-level domains. For all evaluation datasets, we adopt a generation-based assessment approach with greedy decoding. The overall results are in \cref{tab:exp_language}.It can be observed that in most of the evaluation datasets, the performance of our \method-9B and Llama3.1~\cite{llama3_2024} models is comparable, maintaining their linguistic capabilities. Furthermore, in some rankings, our models exhibit superior performance in certain aspects compared to their text-only baseline models. This improvement is attributed to our multi-stage language preservation strategy and the high-quality instruction tuning data used in our training process.

\begin{table}[t]
\centering
\caption{
\textbf{Free-form dialogue generation evaluation results.}
}
% \vspace{3pt}
\setlength{\tabcolsep}{8pt}
\begin{tabular}{c|c|c|c}
\toprule
Model & Relevance & Fluency & Informativeness\\
\midrule
TextBind~\cite{li2023textbind} & 3.85 & 4.30 & 3.25\\
\rowcolor{Gray} \textbf{\method-9B} & 4.60 & 4.80 & 3.80\\
\bottomrule
\end{tabular}
\label{tab-model_freeform_results}
% \vspace{-12pt}
\end{table}




% For a evaluation of open-world multi-turn multimodal instruction following, we collect a test set comprising 50 conversations from realistic scenarios and utilize \method-9B to generate arbitrarily interleaved text and images in proper conversation contexts. For quantitative results, we ask GPT-4o~\cite{openai2024gpt4ocard} to rate each conversation ranging from 0 to 5 considering relevance, fluency and informativeness. We carry out our quantitative results against recent work TextBind~\cite{li2023textbind}. As shown in \cref{tab-model_freeform_results}, \method-9B exhibits overall better understanding and generating ability of multi-turn multimodal conversations. More qualitative cases can be found in \cref{fig-IT-Freeform-Result}.



\begin{table}[t]
  \caption{\textbf{Quantitative results on audio generation.} $^*$ indicates officially released checkpoints evaluated by us.}
  \label{tab:exp_audio_generation}
  \centering
  \setlength{\tabcolsep}{14pt}
  \begin{tabular}{lccccc}
    \toprule
       & \multicolumn{2}{c}{\textbf{SEED test-zh}}\\
    \cmidrule(r){2-3}
    Model & CER(\%)$\downarrow$ & SS$\uparrow$  \\
    \midrule

     Human & 1.26 &0.755 \\
     Vocoder Resyn. & 1.27 & 0.720 \\
     \midrule
     Seed-TTS~\cite{SEED_TTS} & 1.12 & 0.796 \\
     FireRedTTS~\cite{FireRedTTS} & 1.51 &0.635 \\
     MaskGCT~\cite{MaskGCT} & 2.27 & 0.774 \\
     E2-TTS(32 NFE)~\cite{E2_TTS} & 1.97 & 0.730 \\
     F5-TTS(32 NFE)~\cite{F5_TTS} & 1.56 & 0.741 \\
     CosyVoice~\cite{CosyVoice} &3.63 &0.723 \\
     CosyVoice2~\cite{CosyVoice2} &1.45 &0.748 \\
     CosyVoice2-S~\cite{CosyVoice2} &1.45 &0.753 \\
     CosyVoice2-S~\cite{CosyVoice2} &1.45 &0.753 \\
     \midrule
     MiniCPM-o-2.6~\cite{yao2024minicpm} &8.03$^*$ &0.474$^*$ \\
     \rowcolor{Gray} \textbf{\method-9B} &  6.36  & 0.604 \\
    \bottomrule
\end{tabular}
\end{table}


\subsubsection{User Experience Evaluation}\label{sec:human_evaluation}
\textbf{Evaluation Metric}:
Current benchmarks such as MMBench~\cite{liu2025mmbench}, MMStar~\cite{chen2024we}, and MMMU~\cite{yue2023mmmu} primarily focus on assessment through judgment-style questions. However, this assessment does not align with the users' actual interactive experience with MLLMs. To address this limitation, drawing inspiration from SuperclueV~\cite{supercluev}, we develop evaluation criteria specifically for assessing the models' performance on user experience, which contains four key dimensions: relevance, fluency, informativeness, and format rationality. \textit{Relevance} assesses the extent to which the model's responses align with both the provided prompts and the multimodal inputs.
\textit{Fluency} evaluates the naturalness, smoothness, clarity, comprehensibility, and anthropomorphic quality of the model's responses.
\textit{Informativeness} measures the extent to which the model's responses provide relevant information, knowledge, and analytical reasoning, enhancing their utility, detail, depth, and innovation.
\textit{Format rationality} examines the model's ability to adaptively generate appropriately structured and clear formats, for presenting results based on varying prompt types.



% Current benchmarks such as MMBench~\cite{liu2025mmbench}, MMStar~\cite{chen2024we}, and MMMU~\cite{yue2023mmmu} primarily focus on assessment through judgment-style questions. However, this assessment does not align with the users' actual interactive experience with MLLMs. Drawing inspiration from SuperclueV~\cite{supercluev}, we develop evaluation criteria specifically for assessing the models' experience performance, which contains four key dimensions: relevance, fluency, content richness, and format rationality. \textbf{Relevance} assesses the extent to which the model's responses align with both the provided prompts and the multi-modal inputs.
% \textbf{Fluency} evaluates the naturalness, smoothness, clarity, comprehensibility, and anthropomorphic quality of the model's responses.
% \textbf{Content richness} gauges the degree to which the model's responses are enriched with supplementary information, knowledge, and analytical reasoning, enhancing their utility, detail, depth, and innovation.
% \textbf{Format rationality} examines the model's ability to adaptively generate appropriately structured and clear formats for presenting results based on varying prompt types.


\begin{table}[t]
\centering
\caption{
\textbf{Detailed model experience evaluation standards.}
}
% \vspace{3pt}
\setlength{\tabcolsep}{4pt}
\begin{tabular}{c|c}
\toprule
Score & Description\\
\midrule
1 & Totally unsatisfied, totally unacceptable \\
2 & Basically not satisfied, with many obvious problems \\
3 & Generally satisfied, with a few obvious problems \\
4 & Basically satisfied, minor flaws allowed \\
5 & Completely satisfied, almost perfect \\
\bottomrule
\end{tabular}
\label{tab-model_expr_standards}
% \vspace{-12pt}
\end{table}

\textbf{Evaluation Dataset}: We collect chat samples from the actual users' multi-turn interaction dialogues, which cover a variety of tasks, including visual question answering (VQA), conversational interactions, chart interpretation, mathematical problem-solving, optical character recognition (OCR), and other related tasks. GPT-4o~\cite{openai2024gpt4ocard} is instructed to follow the evaluation criteria to generate initial reference answers for these collected samples. To ensure accuracy, human annotators refine the initial responses generated by GPT-4o. This process yields an evaluation dataset with nearly 300 samples, each with a corresponding ground truth.

We utilize GPT-4o to evaluate the model's responses against the ground truth, adhering to the standards outlined in  \cref{tab-model_expr_standards}.  As shown in \cref{tab-user_experience},  our M2-omni model, after undergoing  alignment tuning,  demonstrates an average increase of 5.7\%-23.4\% in user experience performance, which is further validated by human annotations on selected cases. Meanwhile, our model's performance on the OC benchmark across other modalities remains relatively consistent, thereby demonstrating the effectiveness of our unified training strategy, which integrates DPO and instruction tuning in the alignment tuning stage.

% We employ GPT-4o to score the models' responses compared with ground truth according to the standards of \cref{tab-model_expr_standards}.  \cref{tab-user_experience} shows the model after alignment tuning demonstrates an average increase of 5.7\% in performance. This enhancement is corroborated by human annotations on selected cases. Simultaneously, the general capabilities on OC benchmark across other modalities remain nearly the same, with a decrease in average evaluation scores of less than 1\%. This demostrates the effectiveness of our unified training strategy that integrates DPO and
% instruction tuning in alignment tuning stage.


\subsubsection{Free-Form Dialogue Generation}
To evaluate the open-world multi-turn multimodal instruction following capabilities of our model, we create a test set consisting of 50 conversations derived from realistic scenarios. We utilize \method-9B to generate arbitrarily interleaved text and images in proper conversation contexts.
For quantitative results, following our user experience evaluation metric, we employ GPT-4o to rate each conversation on a scale of 0 to 5 across three evaluation dimensions: relevance, fluency, and informativeness.
We carry out our quantitative results against recent work TextBind~\cite{li2023textbind}. As shown in \cref{tab-model_freeform_results}, \method-9B exhibits overall better understanding and generating ability of multi-turn multimodal conversations. More qualitative cases can be found in \cref{fig-IT-Freeform-Result}.





\begin{table}[t]
\centering\footnotesize
\caption{
\textbf{Detailed evaluation on user experience benchmark and OC benchmark. OC is short for the OpenCompass image-text understanding benchmark.}
}
% \vspace{3pt}
\setlength{\tabcolsep}{3pt}
\begin{tabular}{c|c|c|c|c|c|c}
\toprule
Model & Relevance & Fluency & Informativeness & Format Rationality & Expr. Avg($\Delta$\%) & OC Avg($\Delta$)\\
\midrule
\method-9B & 4.556 & 4.036 & 2.742 & 3.573 & 3.726 & -\\
\rowcolor{Gray} \method-9B-Align & 4.893 & 4.735 & 4.118 & 4.644 & 4.598(+23.4\%) & -0.3\\
\method-72B & 4.942 & 4.689 & 3.267 & 4.265 & 4.351 & -\\
\rowcolor{Gray} \method-72B-Align & 4.946 & 4.875 & 3.961 & 4.615 & 4.598(+5.7\%) & -0.2\\
InternVL2-26B~\cite{internvl_2024} & 4.886 & 4.76 & 4.15 & 4.52 & 4.577 & -\\
GPT-4o~\cite{openai2024gpt4ocard} & 5 & 4.878 & 3.854 & 4.831 & 4.64 & -\\
\bottomrule
\end{tabular}
\label{tab-user_experience}
% \vspace{-12pt}
\end{table}



\subsection{Qualitative Results}\label{subsec:exp_qualitative_results}

In this section, we qualitatively assess the capabilities of our \method, presenting examples of each modality and different tasks.

We show multimodal understanding abilities of our \method in \cref{fig-exp_case_all}. \method demonstrates promising capabilities in processing cross-modal problems, encompassing image understanding, video understanding, interleaved image-text understanding, and image-audio understanding. More examples can be found in the appendix, provided in \cref{subsec:appendix_cases}.

\cref{fig-IT-Freeform-Result} illustrates the model's ability to generate free-form dialogue, where our \method can create images based on the conversation context without explicit user input, useful for explaining ideas to users.




\begin{figure}[t]
    \centering
    \includegraphics[width=0.9\linewidth]{figures/case_exp.pdf}
    \caption{
    \textbf{Cases for multimodal understanding.}
    \method shows great potential to solve various multimodal problems.
    }
    \label{fig-exp_case_all}
\end{figure}




\begin{figure}[t]
    \centering
    \includegraphics[width=0.9\linewidth]{figures/free_form_gen.pdf}
    \caption{
    \textbf{Cases for Free-Form Dialogue Generation.}
    }
    \label{fig-IT-Freeform-Result}
\end{figure}


\subsection{Ablation Study}\label{subsec:exp_ablation}

\begin{table}[t]
\centering
\caption{\textbf{Ablation studies on step balancing strategy.} The loss weight setting [1,1,1] corresponds to the uniform weighting of the loss functions for image-text pairs, interleaved image-text, and video datasets.  * and \# represent the loss weight settings. * is obtained through experimental trials and parameter tuning. \# is obtained by normalizing the loss weights using the inverse of the loss at convergence, as described in Section \cref{subsubsec-Step Balancing Strategy}. We evaluate the few-shot performance on VQA tasks and the zero-shot performance on the captioning task of our pre-trained model.}
\label{tab:ablation_step_balance_pretrain}
\setlength{\tabcolsep}{4pt}
\begin{tabular}{c|c|ccc}
\toprule
\multicolumn{1}{l|}{Data Sample Balance} & Loss Weight Balance & \multicolumn{1}{l}{OK-VQA(4-shot)} & \multicolumn{1}{l}{VQAv2(4-shot)} & \multicolumn{1}{l}{Flickr30k(0-shot)} \\ \hline
Random Sample                        & {[}1,1,1{]}          & 40.5                             & 54.3                             & 87.0                                 \\
Round-robin                          & {[}1,1,1{]}          & 41.6                             & 54.4                             & 88.1                                 \\
Accumulation                         & {[}1,1,1{]}          & 41.7                             & 54.6                             & 88.2                                 \\
Accumulation                         & ${[}0.2,1.0,0.03{]}^{*}$   & 39.7                             & 52.5                             & 87.1                                 \\
Accumulation                         & ${[}0.45,0.36,1.09{]}^{\#}$ & \textbf{42.1}                             & \textbf{55.4}                             & \textbf{88.2}                                 \\
\bottomrule
\end{tabular}
\end{table}


In this section, we conduct ablation studies to investigate the effectiveness of our step balance strategy and dynamic adaptive balance strategy in our M2-omni model. These experiments aim to provide insights into the impact of these key components on our M2-omni’s performance.

\subsubsection{Step Balance Strategy}\label{subsubsec:step_balance_ablaton}

As described in \cref{subsubsec-Step Balancing Strategy} ,  we investigate the impact of various data sample balancing strategies and loss weight balancing schemes on the multimodal joint training stage of pre-training. We evaluate the performance of candidate strategies on two VQA benchmarks, OK-VQA~\cite{marino2019ok} and VQAv2~\cite{goyal2017making}, and assess its image captioning performance using the Flickr30k~\cite{young2014image} benchmark.

For pretrained models lacking in instruction following ability, to assess the effectiveness of our approach, we evaluate the performance of these models on VQA tasks using a few-shot approach and on image caption tasks using a zero-shot approach. \cref{tab:ablation_step_balance_pretrain} presents the results of our M2-omni pretrained models, which demonstrate the effectiveness of our step balance strategy.

% Besides, three task weighting manner are compared: [1,1,1], which means all data shares the same optimization step size; [0.2,1.0,0.03], which is consistent with that proposed in \cite{alayrac2022flamingo}; [0.45,0.36,1.09], the inverse of the loss at convergence state, as \cref{subsubsec-Step Balancing Strategy} described. Note that the three values in the ratio correspond to image-text pairs, interleaved image-text and video datasets.

%  We directly evaluate the pre-trained model's performance on VQA tasks using a few-shot approach and on image caption tasks using a zero-shot approach. For VQA tasks, we use two benchmarks: OK-VQA~\cite{marino2019ok} and VQAv2~\cite{goyal2017making}, while for image captioning, we use the Flickr30k~\cite{young2014image} benchmark. \cref{tab:ablation_step_balance_pretrain} shows the results of training models on the combined datasets using three different merging regimes. It can be observed that the accumulation strategies and setting the task weights to the inverse of the loss achieve the best performance.







\begin{table}[t]
\centering
\caption{
\textbf{Ablation results of the dynamic adaptive balance strategy}. Results for unimodal baselines are derived from the following single-modal models: \textsuperscript{$\dagger$} Image-Text Model, \textsuperscript{$\ddagger$} Video-Text Model, and \textsuperscript{
$\mathsection$} Audio-Text Model. The best result for each benchmark is \textbf{bolded}, while the best result for each model across all epochs is \underline{underlined}.
}
% \vspace{3pt}
\setlength{\tabcolsep}{3pt}
\begin{tabular}{l|l|ccccc|cc|cc}
\toprule
Models &  & MM- & OK- & VQAv2 & Text- & GQA & MSVD- & MSRVTT & Audio & MLS- \\
& & Bench & VQA &&VQA&& QA & QA & Caps & English($\downarrow$) \\
\midrule
\multirow{3}{*}{\makecell[l]{Single-modal\\Baselines}} & ep1 & 68.0\textsuperscript{$\dagger$} & 56.4\textsuperscript{$\dagger$} & 74.8\textsuperscript{$\dagger$} & \underline{70.4}\textsuperscript{$\dagger$} & 58.4\textsuperscript{$\dagger$} & 72.3\textsuperscript{$\ddagger$} & 59.3\textsuperscript{$\ddagger$} & 29.0\textsuperscript{$\mathsection$} & 11.4\textsuperscript{$\mathsection$} \\
& ep2 & \underline{\textbf{77.8}}\textsuperscript{$\dagger$} & \underline{59.8}\textsuperscript{$\dagger$} & \underline{76.9}\textsuperscript{$\dagger$} & 69.8\textsuperscript{$\dagger$} & 60.6\textsuperscript{$\dagger$} & \underline{\textbf{76.5}}\textsuperscript{$\ddagger$} & \underline{60.1}\textsuperscript{$\ddagger$} & \underline{39.9}\textsuperscript{$\mathsection$} & 9.33\textsuperscript{$\mathsection$} \\
& ep3 & 77.3\textsuperscript{$\dagger$} & 58.0\textsuperscript{$\dagger$} & 76.8\textsuperscript{$\dagger$} & 69.1\textsuperscript{$\dagger$} & \underline{60.8}\textsuperscript{$\dagger$} & 74.4\textsuperscript{$\ddagger$} & 58.6\textsuperscript{$\ddagger$} & 39.5\textsuperscript{$\mathsection$} & \underline{8.96}\textsuperscript{$\mathsection$} \\
\midrule
\multirow{3}{*}{\makecell[l]{Mixture \\w/o MM-Bal.}}
& ep1 & 70.5 & 55.9 & 75.7 & 70.2 & 57.7 & \underline{75.1} & \underline{59.6} & 27.5 & 12.1 \\
& ep2 & \underline{75.8} & \underline{58.8} & \underline{77.0} & \underline{70.5} & \underline{\textbf{61.1}} & 73.4 & 58.5 & 33.5 & 9.45 \\
& ep3 & 75.6 & 58.4 & 76.5 & 69.5 & 60.1 & 70.2 & 56.9 & \underline{39.6} & \underline{8.98} \\
\midrule
\multirow{3}{*}{\makecell[l]{Mixture \\w/ MM-Bal.}}
& ep1 & 74.7 & 59.6 & 76.0 & 71.2 & 59.0 & 73.1 & 58.7 & 35.5 & 9.27 \\
& ep2 & \underline{\textbf{77.8}} & \underline{\textbf{61.7}} & \underline{\textbf{77.2}} & \underline{\textbf{71.8}} & 60.5 & \underline{74.8} & 58.5 & 41.2 & 8.31 \\
& ep3 & 77.1 & 60.5 & 77.0 & 69.8 & \underline{60.7} & 74.6 & \underline{\textbf{60.2}} & \underline{\textbf{44.1}} & \underline{\textbf{8.04}} \\

\bottomrule
\end{tabular}
\label{tab-multi_task_balanced_ablation}
% \vspace{-12pt}
\end{table}

\subsubsection{Dynamic Adaptive Balance Strategy}

We conducted a evaluation of our dynamic adaptive balance strategy across text-image, video, and audio modalities using constrained datasets. The evaluation was conducted on benchmark datasets specific to each modality: for text-image tasks, MMbench~\cite{liu2025mmbench}, OK-VQA~\cite{marino2019ok}, VQAv2~\cite{goyal2017making}, TextVQA~\cite{singh2019towards}, and GQA~\cite{hudson2019gqa} were employed; for video, MSVD-QA~\cite{xu2017video} and MSRVTT-QA~\cite{xu2017video} benchmarks were utilized; and for audio, we assessed performance on the AudioCaps~\cite{kim2019audiocaps} (AAC) and MLS~\cite{Pratap2020MLSAL}-English (ASR) tasks. The experimental outcomes are detailed in Table~\ref{tab-multi_task_balanced_ablation}.

In contrast to actual training pipeline, our evaluation involved instruction tuning starting from pre-trained models. Specifically, for each modality, we initially trained single-modality baseline models (the 'Sinle-modal Baselines' in Table~\ref{tab-multi_task_balanced_ablation}) individually over three epochs to establish the maximum achievable performance per modality. The results indicate that optimal performance was predominantly observed by the second epoch. However, the ASR task, due to its more complex patterns, had not fully converged even by the third epoch. Subsequently, we combined data from all three modalities to train a unified model (the 'Mixture w/o MM-Bal.' in Table~\ref{tab-multi_task_balanced_ablation}). Under this multimodal training regimen, the image-text modality reached its optimal performance at the second epoch, while the video modality achieved peak performance as early as the first epoch and with performance consistently decreasing in subsequent epochs. In contrast, the audio modality demonstrated continuous improvement, attaining its best performance by the third epoch. These observations underscore the imbalance in training progress among different modalities when engaged in multimodal training.

To address this imbalance, we introduced the dynamic adaptive balance strategy within our M2-omni training framework. This strategy dynamically adjusts the loss weights for each modality based on their respective training progress. In the context of this evaluation, it accelerates the training of the audio modality while appropriately reducing the learning weights for the image-text and video modalities to prevent overfitting. The evaluation results for this balanced training approach are denoted as 'Mixture w/ MM-Bal.' in Table~\ref{tab-multi_task_balanced_ablation}. The results demonstrate that, although some degree of imbalance among modalities persists, the balanced training strategy significantly alleviates the issues observed with simple mixed training: optimal performances across benchmarks are now concentrated around the second and third epochs, and performance across all modalities has been markedly enhanced. Moreover, under the balanced training strategy, the model achieved single-modality optimal performance in 7 out of 9 benchmarks. The best-performing model (at epoch 2) surpassed the optimal performance of each single-modality baseline in 6 out of 9 benchmarks (MMBench, OK-VQA, VQAv2, TextVQA, AudioCaps, MLS-English). Additionally, for the audio modality, the model at epoch 3 outperformed the single-modality baselines in 5 out of 9 benchmarks (OK-VQA, VQAv2, MSRVTT-QA, AudioCaps, MLS-English), with significant improvements in audio performance. These experimental results highlight the effectiveness of our dynamic adaptive balance strategy.

    
    \section{Related Works}
        \label{sec:related}
        \subsection{Multilingual Datasets}
% 为了评测模型在不同语言上的性能,有很多在不同任务上的多语言数据集被提出,比如,QA,自然语言推理,文字总结,数值推理,代码生成,可读性等
To evaluate the performance of models across different languages, several multilingual datasets have been proposed for different tasks, such as question answering \cite{liu-etal-2019-xqa,clark-etal-2020-tydiQA,longpre-etal-2021-mkqa}, natural language inference \cite{conneau-etal-2018-xnli}, text summarization \cite{giannakopoulos-etal-2015-multiling-Summarization,ladhak-etal-2020-wikilingua,scialom-etal-2020-mlsum}, numerical reasoning \cite{shi2023MGSM}, code generation \cite{peng-etal-2024-humanevalxl}, text-to-SQL \cite{MultiSpider}, and readability \cite{trokhymovych-etal-2024-open-Readability,naous-etal-2024-readme}, among others. 
% 还有很多多语言数据集收集了不同的任务
Additionally, numerous multilingual datasets have been collected for different tasks \cite{hu-2020-XTREME,ruder-etal-2021-xtremer,zhang2024pmmeval-multitask,singh-etal-2024-indicgenbench}. 
% 但是目前为止,仍然没有多语言TATQA数据集,导致缺乏关于模型多语言TATQA能力的评测与分析
However, to date, there is no multilingual TATQA dataset, resulting in a lack of evaluation and analysis of multilingual TATQA capabilities and a gap with real scenarios. 
% 所以,我们本文提出了多语言TATQA数据集,并详细分析了多语言TATQA的挑战
Therefore, we introduce \ourdataset, a multilingual TATQA dataset, and provide a detailed analysis of the challenges in multilingual TATQA.

\subsection{QA Datasets for the Table and Text}
% 目前,在表格和文本上的QA数据集主要集中于单一语言
Currently, QA datasets for the table and text primarily focus on a single language. 
% 比如,HybridQA从Wikipedia收集英文的表格和相关段落,共包含70K个人工标注的问题和答案
For instance, HybridQA~\cite{chen-etal-2020-hybridqa} collects English tables and associated text from Wikipedia.
% , containing $70$K manually annotated question-answer pairs. 
% TAT-QA和FinQA和DOCMATH-EVAL主要关注于金融领域的数值计算问题
TAT-QA~\cite{zhu-etal-2021-tat}, FinQA~\cite{chen-etal-2021-finqa}, DOCMATH-EVAL~\cite{zhao-etal-2024-docmath}, and FinanceMATH~\cite{zhao-etal-2024-FinanceMATH} focus on numerical computation in the financial domain, and SciTAT~\cite{zhang2024scitat} addresses questions based on tables and text from English scientific papers. 
% SciTAT则关注于根据英文论文中的表格和文本回答用户问题
% 然而,单一语言的数据集无法进一步探索TATQA的挑战,无法全面评测模型的多语言TATQA的能力,并且和现实场景存在较大差距
However, single-language datasets cannot evaluate the multilingual TATQA capabilities, and overlook the diverse languages in real scenarios. 
% 所以我们提出了我们的数据集:首个多语言TATQA数据集,涉及包括英语的11种语言,8个语系
So we propose \ourdataset: the first multilingual TATQA dataset, involving $11$ languages and $8$ language families.
% 我们的数据集和前人工作的对比表格如表所示
A comparison of \ourdataset and prior works is presented in Appendix~\ref{sec:comparison}.


% 目前关于增强TATQA性能的工作主要集中于检索和生成两阶段
The current works on enhancing TATQA performance primarily focus on retrieving relevant information from the context \cite{luo2023hrot,bardhan2024ttqars,glenn-etal-2024-blendsql} and generating programs, equations, or step-by-step reasoning process to derive the final answer \cite{tonglet-etal-2023-seer,TAT-LLM,fatemi2024three-agent}.
% 检索,即设计检索器或直接使用LLM从上下文中抽取相关信息
% Retrieval involves designing a retriever to extract relevant information from the context.
% 生成,即生成代码、等式或逐步推理得到最终答案
% Generation refers to generating programs, equations, or step-by-step reasoning to derive the final answer.
% 比如S3HQA关注于检索阶段,首先训练检索器初步从上下文中检索到相关的表格行和文本,再根据问题的分类进一步选取出相关上下文
For example, S3HQA~\cite{lei-etal-2023-s3hqa} emphasizes retrieving, where a retriever is initially trained, followed by further filtering based on the question type. 
% 而Hpropro关注于生成阶段,通过提供给LLM一些常用函数,方便LLM生成代码时直接调用,且提示LLM根据代码执行错误的信息修改
Hpropro~\cite{shi-etal-2024-hpropro} focuses on generating, providing LLMs with commonly used functions to facilitate direct invocation during code generation.
% 然而,这些方法都是针对单一语言设计的,直接应用于其他语言会导致性能下降
However, previous methods are designed for single-language scenarios, directly used to other languages could lead to performance degradation.
% 所以,我们提出了一个多语言的baseline:我们的方法,实现跨语言的链接和推理
To address this, we propose \ourmethod, a multilingual baseline that aligns the English TATQA capabilities to other languages. 
% 并且我们详细分析了语言对不同答案类型性能的影响,以及在多语言场景下,instruction, 示例,表格文本以及问题的语言对性能的影响,
% We also provide a detailed analysis of how different languages impact performance across various answer types. 
% Furthermore, we examine the effects of language for instructions, demonstrations, tables, text, and questions in multilingual settings. 
% 我们还分析了在我们数据集上非英语语言相比英语的性能下降的原因,为未来多语言TATQA的研究指明了方向
% Additionally, we analyze the reasons for the performance decline in non-English languages compared to English on \ourdataset, providing directions for future research in multilingual TATQA.
% We also provide a detailed analysis of how different languages impact performance, providing directions for future research in multilingual TATQA.
    
    \section{Conclusion}
        % 为了解决前人TATQA数据集只关注英语的问题,我们提出了首个多语言TATQA数据集
        To address the limitations of the existing QA datasets on the hybrid context of tabular and text data (TATQA), which focus exclusively on a single language, we introduce the first multilingual TAT-QA dataset \ourdataset. 
        % 我们从主流TATQA数据集:HybridQA,TAT-QA和SciTAT中sample数据,将其翻译为10种多样的语言
        Specifically, we sample data from mainstream TAT-QA datasets, including HybridQA, TAT-QA, and SciTAT, and translate it into $10$ diverse languages.
        % 为了提升模型在多语言TATQA任务上的性能,我们提出强baseline:我们的方法,能够处理不同语言表格和文本中的信息并进行推理
        To enhance the TATQA performance in non-English languages, we propose a baseline (\ourmethod). 
        \ourmethod links the relevant information from the hybrid context and reasons in English.
        % 我们进行了一系列基线的实验,模型在非英语上的性能相比英语下降18.9%
        We conduct a series of baseline experiments and observe a $19.4\%$ performance drop for non-English languages compared to English. 
        % 我们的错误分析说明这主要是因为模型在非英语上链接相关信息更加困难,并且应用公式和指令遵循的能力下降
        Error analysis revealed that this decline is primarily due to the increased difficulty in linking relevant information in non-English texts and the reduced ability to apply formulas and follow instructions of models.
        % 而我们的方法相比其他基线平均提升2.6,证明了我们方法的有效性
        Furthermore, \ourmethod achieves an average improvement of $3.3$ over other baselines, demonstrating its effectiveness. 
        % 分析实验表明不同语言上TATQA能力的高低不仅与这种语言资源的高低有关,还与模型本身的特性有关
        Analysis of experimental results suggests that the performance of TATQA across languages is influenced not only by high-resource versus low-resource languages but also by the inherent characteristics of the model itself.
    
    
    \clearpage
    
    \section*{Limitations}
        % 我们数据集只涉及单轮对话。我们将多语言多轮对话留作未来工作
        (\emph{i})~\ourdataset only includes single-turn dialogues, leaving multilingual multi-turn dialogues for future work.
        % 我们数据集只包括11种语言。未来的版本应该包括更多语言。
        (\emph{ii})~\ourdataset covers only $11$ languages. Future versions should include more languages.
        
    \section*{Ethics Statement}
        % Every dataset and model used in the paper is accessible to the public, and our application of them adheres to their respective licenses and conditions.
            
    All datasets and models used in this paper are publicly available, and our utilization of them strictly complies with their respective licenses and terms of use. 
    Additionally, we confirm that the compensation provided to annotators is significantly higher than the local minimum wage.
    
    % Entries for the entire Anthology, followed by custom entries
    \bibliography{custom}
    
    \clearpage
    \appendix
    \label{sec:appendix}
    \onecolumn
\clearpage
\appendix
\appendixpage  % if you use a package that provides an appendix title page
\hypersetup{linkcolor=black}
\startcontents[sections]
\printcontents[sections]{l}{1}

\hypersetup{linkcolor=hrefblue}
\glsresetall

\section{Additional related works}\label{apx:related_works}

\paragraph{Knowledge distillation.}
Knowledge distillation (KD)~\citep{hinton2015distilling,gou2021knowledge} is closely connected to W2S generalization regarding the teacher-student setup, while W2S reverts the capacities of teacher and student in KD. In KD, a strong teacher model guides a weak student model to learn the teacher's knowledge. In contrast, W2S generalization occurs when a strong student model surpasses a weak teacher model under weak supervision.
\citet{phuong2019towards,stanton2021does,ojha2023knowledge,nagarajan2023student,dong2024cluster,ildiz2024high} conducted rigorous statistical analyses for the student's generalization from knowledge distillation. 
From the analysis perspective, a key difference between KD and W2S is that W2S is usually analyzed in the context of finetuning since the notions of “weak” and “strong” are built upon pretraining. This finetuning perspective introduces distinct angles from KD for examining intrinsic dimension~\citep{li2018measuring} and student-teacher correlation in W2S. 

\paragraph{Self-distillation and self-training.}
In contrast to W2S that considers distinct student and teacher models, self-distillation~\citep{zhang2019your,zhang2021self} and related paradigms such as Born-Again Networks~\citep{furlanello2018born} use the same or progressively refined architectures to iteratively distill knowledge from a ``previous version'' of the model. There have been extensive theoretical analyses toward understanding the mechanism behind self-distillation~\citep{mobahi2020self,das2023understanding,borup2023self,pareek2024understanding}.

Self-training~\citep{scudder1965probability,lee2013pseudo} is a closely related method to self-distillation that takes a single model's confident predictions to create pseudo-labels for unlabeled data and refines that model iteratively. 
\citet{wei2020theoretical,oymak2021theoretical,frei2022self} provide theoretical insights into the generalization of self-training. 
In particular, \citet{wei2020theoretical} introduced a theoretical framework based on neighborhood expansion, which was later on extended to various settings of weakly supervised learning, including domain adaptation~\citep{cai2021theory}, contrastive learning~\citep{shen2022connect}, consistency regularization~\citep{yang2023sample}, and now weak-to-strong generalization~\citep{lang2024theoretical,shin2024weak}.




\section{Proofs in \Cref{sec:single_task_ft}}

\begin{lemma}\label{lem:low_est_err_ft}    
    Given the FT approximation errors $\rho_s$ and $\rho_w$ in \Cref{def:ft_est_err}, we have
    \begin{align*}
        \rho_s(n) \le n \rho_s \quad \text{and} \quad \rho_w(n) \le n \rho_w \quad \forall\ n \in \N.
    \end{align*}
\end{lemma}

\begin{proof}[Proof of \Cref{lem:low_est_err_ft}]
    Let $\thetab_* = \argmin_{\thetab \in \R^d}\ \E_{\xb \sim \Dcal}[(\phi_w(\xb)^\top \thetab - f_*(\xb))^2]$ such that
    \begin{align*}
        \E_{\xb \sim \Dcal}[(\phi_w(\xb)^\top \thetab_* - f_*(\xb))^2] = \rho_w.
    \end{align*}
    Then, by observing that conditioned on $\Xb$,
    \begin{align*}
        \phi_w(\Xb)^\dagger f_*(\Xb) = \argmin_{\thetab \in \R^d}\ \| \phi_w(\Xb) \thetab - f_*(\Xb) \|_2^2,
    \end{align*} 
    we have
    \begin{align*}
        \rho_w(n) &= \E_{\Xb \sim \Dcal^n}\sbr{\| \phi_w(\Xb) \phi_w(\Xb)^\dagger f_*(\Xb) - f_*(\Xb) \|_2^2} \\
        &\le \E_{\Xb \sim \Dcal^n}\sbr{\| \phi_w(\Xb) \thetab_* - f_*(\Xb) \|_2^2} \\
        &= n\ \E_{\Xb \sim \Dcal^n}\sbr{\frac{1}{n} \| \phi_w(\Xb) \thetab_* - f_*(\Xb) \|_2^2} \\
        &= n\ \E_{\xb \sim \Dcal}\sbr{(\phi_w(\xb)^\top \thetab_* - f_*(\xb))^2} \\
        &= n\ \rho_w.
    \end{align*}
    The proof for $\rho_s(n)$ follows analogously.
\end{proof}



\subsection{Proof of \Cref{thm:w2s_ft}}\label{apx:pf_w2s_ft}

\begin{theorem}[Formal restatement of \Cref{thm:w2s_ft}]\label{thm:w2s_ft_formal}
    Consider $f_\wts(\xb) = \phi_s(\xb)^\top \thetab_\wts$ finetuned as in \eqref{eq:sft_weak}, \eqref{eq:w2s_ft} with both $\alpha_w, \alpha_\wts \to 0$. Under \Cref{asm:features,asm:ft_data}, when $n \ge \Omega(d_w)$, the excess risk $\exrisk(f_\wts) = \vari(f_\wts) + \bias(f_\wts)$ satisfies
    \begin{align*}
        &\bias(f_\wts) \le \frac{\rho_w(n)}{n} + \frac{\rho_s(N)}{N} \le \rho_w + \rho_s, \\
        &\vari(f_\wts) \lesssim \frac{\sigma^2}{n} \rbr{d_{s \wedge w} + \frac{d_s}{N} (d_w - d_{s \wedge w})}.
    \end{align*}
    In particular, when ${\rho_w(n)}/{n} > 0$ and $d_s < d_w$, the inequality for $\bias(f_\wts)$ is strict.

    Moreover, when $\phi_w(\xb) \sim \Ncal(\b0_d, \Sigmab_w)$, for any $n > d_w + 1$, we have 
    \begin{align*}
        &\vari(f_\wts) = \frac{\sigma^2}{n-d_w-1} \rbr{d_{s \wedge w} + \frac{d_s}{N} (d_w - d_{s \wedge w})}.
    \end{align*}
\end{theorem}

\begin{proof}[Proof of \Cref{thm:w2s_ft} and \Cref{thm:w2s_ft_formal}]
    We first observe that the solution of \eqref{eq:sft_weak} as $\alpha_w \to 0$ is given by
    \begin{align*}
        \thetab_w = \wt\Phib_w^\dagger \wt\yb = \wt\Phib_w^\dagger (\wt\fb_* + \wt\zb),
    \end{align*}
    where $\wt\zb \sim \Ncal(\b0_n, \sigma^2 \Ib_n)$.
    Meanwhile, the solution of \eqref{eq:w2s_ft} as $\alpha_\wts \to 0$ is given by
    \begin{align*}
        \thetab_\wts = \Phib_s^\dagger \Phib_w \thetab_w = \Phib_s^\dagger \Phib_w \wt\Phib_w^\dagger (\wt\fb_* + \wt\zb).
    \end{align*}  
    
    Then, the excess risk of $f_\wts$ can be decomposed into variance and bias as follows:
    \begin{align*}
        \exrisk(f_\wts) &= \E_{\xb \sim \Dcal}\sbr{\E_{f_\wts}\sbr{(f_\wts(\xb) - f_*(\xb))^2}} \\
        &= \E_{\Scal_x}\sbr{\E_{\wt\Scal}\sbr{\frac{1}{N}\nbr{\Phib_s \thetab_\wts - \fb_*}_2^2}} \\
        &=\E_{\Scal_x, \wt\Scal}\sbr{\frac{1}{N} \nbr{(\Phib_s \Phib_s^\dagger \Phib_w \wt\Phib_w^\dagger \wt\fb_* - \fb_*) + \Phib_s \Phib_s^\dagger \Phib_w \wt\Phib_w^\dagger \wt\zb}_2^2} \\
        &= \underbrace{\frac{1}{N} \E_{\Scal_x, \wt\Scal}\sbr{\nbr{\Phib_s \Phib_s^\dagger \Phib_w \wt\Phib_w^\dagger \wt\zb}_2^2}}_{\vari(f_\wts)} + \underbrace{\frac{1}{N} \E_{\Scal_x, \wt\Scal}\sbr{\nbr{\Phib_s \Phib_s^\dagger \Phib_w \wt\Phib_w^\dagger \wt\fb_* - \fb_*}_2^2}}_{\bias(f_\wts)}.
    \end{align*}

    \paragraph{Bias.}
    For the bias term, by observing that $\Pb_s = \Phib_s \Phib_s^\dagger$ is an $N \times N$ orthogonal projection, we can decompose the bias term as
    \begin{align*}
        \bias(f_\wts) &= \E_{\Scal_x, \wt\Scal}\sbr{\frac{1}{N} \nbr{\Pb_s \rbr{\Phib_w \wt\Phib_w^\dagger \wt\fb_* - \fb_*}}_2^2} + \frac{1}{N} \E_{\Scal_x}\sbr{\nbr{\rbr{\Ib_N - \Pb_s} \fb_*}_2^2},
    \end{align*}
    where $\E_{\Scal_x}\sbr{\nbr{\rbr{\Ib_N - \Pb_s} \fb_*}_2^2} = \rho_s(N)$ by \Cref{def:ft_est_err}.

    For the first term, 
    \begin{align*}
        \E_{\Scal_x, \wt\Scal}\sbr{\frac{1}{N} \nbr{\Pb_s \rbr{\Phib_w \wt\Phib_w^\dagger \wt\fb_* - \fb_*}}_2^2} &\le \E_{\Scal_x, \wt\Scal}\sbr{\frac{1}{N} \nbr{\Phib_w \wt\Phib_w^\dagger \wt\fb_* - \fb_*}_2^2} \\
        &= \E_{\wt\Scal}\sbr{\frac{1}{n} \nbr{\wt\Phib_w \wt\Phib_w^\dagger \wt\fb_* - \wt\fb_*}_2^2} \\
        &= \frac{\rho_w(n)}{n}.
    \end{align*}
    Notice that when ${\rho_w(n)}/{n} > 0$, this inequality is strict if $d_s < d_w$, where $\Phib_w \wt\Phib_w^\dagger \wt\fb_* - \wt\fb_* \notin \range(\Phib_s)$ almost surely.

    Overall, we have
    \begin{align*}
        \bias(f_\wts) \le \frac{\rho_w(n)}{n} + \frac{\rho_s(N)}{N} \le \rho_w + \rho_s,
    \end{align*}
    where the second inequality follows from \Cref{lem:low_est_err_ft}.

    \paragraph{Variance.}
    For the variance term, we observe that
    \begin{align*}
    \begin{split}
        \vari(f_\wts) &= \frac{1}{N} \E_{\Scal_x, \wt\Scal}\sbr{\nbr{\Pb_s \Phib_w \wt\Phib_w^\dagger \wt\zb}_2^2} \\
        &= \frac{1}{N} \E_{\Scal_x, \wt\Scal}\sbr{\tr\rbr{\Phib_w^\top \Pb_s \Phib_w \wt\Phib_w^\dagger \wt\zb \wt\zb^\top (\wt\Phib_w^\dagger)^\top}} \\
        &= \frac{\sigma^2}{N} \E_{\Scal_x, \wt\Scal}\sbr{\tr\rbr{\Phib_w^\top \Pb_s \Phib_w (\wt\Phib_w^\top \wt\Phib_w)^\dagger}},
    \end{split}
    \end{align*}
    which implies
    \begin{align}\label{eq:pf_var_w2s}
    \begin{split}
        \vari(f_\wts) = \frac{\sigma^2}{N} \tr\rbr{\E_{\Scal_x}\sbr{\Sigmab_w^{-1/2} \Phib_w^\top \Pb_s \Phib_w \Sigmab_w^{-1/2}} \E_{\wt\Scal}\sbr{\rbr{\Sigmab_w^{-1/2} \wt\Phib_w^\top \wt\Phib_w \Sigmab_w^{-1/2}}^\dagger}}.
    \end{split}
    \end{align}

    Recall the spectral decomposition $\Sigmab_w = \Vb_w \Lambdab_w \Vb_w^\top$. 
    Since $\E_{\xb \sim \Dcal}[\phi_w(\xb) \phi_w(\xb)^\top] = \Sigmab_w$, for each $\xb \sim \Dcal$, we can write $\phi_w(\xb) = \Sigmab_w^{1/2} \gammab$, where $\gammab \in \R^{d}$ is an independent random vector that is zero-mean and isotropic (\ie $\E[\gammab] = \b0_{d}$ and $\E[\gammab \gammab^\top] = \Ib_{d}$). The same holds for $\Sigmab_s = \Vb_s \Lambdab_s \Vb_s^\top$ and $\phi_s(\xb) = \Sigmab_s^{1/2} \gammab$.

    Then, for $\Scal$ and $\wt\Scal$, there exist independent random matrices $\Gammab = [\gammab_1, \ldots, \gammab_N]^\top \in \R^{N \times d}$ and $\wt\Gammab = [\wt\gammab_1, \ldots, \wt\gammab_n]^\top \in \R^{n \times d}$ consisting of $\iid$ zero-mean isotropic rows such that
    \begin{align}\label{eq:pf_var_w2s_subgaussian_asm}
    \begin{split}
        &\Phib_w \Sigmab_w^{-1/2} = \Gammab \Sigmab_w^{1/2} \Sigmab_w^{-1/2} = \Gammab \Vb_w \Vb_w^\top, \\
        &\wt\Phib_w \Sigmab_w^{-1/2} = \wt\Gammab \Sigmab_w^{1/2} \Sigmab_w^{-1/2} = \wt\Gammab \Vb_w \Vb_w^\top, \\
        &\Phib_s \Sigmab_s^{-1/2} = \Gammab \Sigmab_s^{1/2} \Sigmab_s^{-1/2} = \Gammab \Vb_s \Vb_s^\top, \\
        &\wt\Phib_s \Sigmab_s^{-1/2} = \wt\Gammab \Sigmab_s^{1/2} \Sigmab_s^{-1/2} = \wt\Gammab \Vb_s \Vb_s^\top.
    \end{split}
    \end{align}
    Let $\Gammab_w = \Gammab \Vb_w \in \R^{N \times d_w}$ and $\wt\Gammab_w = \wt\Gammab \Vb_w \in \R^{n \times d_w}$. We observe that
    \begin{align*}
        \E_{\wt\Scal}\sbr{\rbr{\Sigmab_w^{-1/2} \wt\Phib_w^\top \wt\Phib_w \Sigmab_w^{-1/2}}^\dagger}
        = \E_{\wt\Scal}\sbr{\rbr{\Vb_w \wt\Gammab_w^\top \wt\Gammab_w \Vb_w^\top}^\dagger} 
        = \Vb_w \E_{\wt\Scal}\sbr{\rbr{\wt\Gammab_w^\top \wt\Gammab_w}^\dagger} \Vb_w^\top.
    \end{align*}

    Now, we consider the following two cases for the feature distribution of $\phi_w(\xb)$, corresponding to the distribution of $\Gammab_w$ and $\wt\Gammab_w$:
    \begin{enumerate}[label=(\alph*)]
        \item \b{Gaussian features}: In \Cref{thm:w2s_ft}, assuming $\phi_w(\xb) \sim \Ncal(\b0_d, \Sigmab_w)$ such that $\wt\Gammab_w$ consists of $\iid$ Gaussian rows, we have $\wt\gammab_i \sim \Ncal(\b0_{d_w}, \Ib_{d_w})$. Notice that under the assumption $n > d_w + 1$, $\rank(\wt\Gammab_w) = d_w$ almost surely, and therefore $\wt\Gammab_w^\top \wt\Gammab_w$ is invertible.
        
        Meanwhile, with $\wt\gammab_i \sim \Ncal(\b0_{d_w}, \Ib_{d_w})$ for all $i \in [n]$, $(\wt\Gammab_w^\top \wt\Gammab_w) \sim \Wcal(\Ib_{d_w},n)$ follows the Wishart distribution~\citep[Definition 3.4.1]{wishart1928generalised} with $n$ degrees of freedom and scale matrix $\Ib_{d_w}$. 
        Therefore, $(\wt\Gammab_w^\top \wt\Gammab_w)^{-1} \sim \Wcal^{-1}(\Ib_{d_w},n)$ follows the inverse Wishart distribution~\citep[\S 3.8]{mardia2024multivariate}, whose mean takes the form~\citep[(3.8.3)]{mardia2024multivariate}
        \begin{align*}
            \E_{\wt\Scal}\sbr{(\wt\Gammab_w^\top \wt\Gammab_w)^\dagger} = \frac{1}{n - d_w -1} \Ib_{d_w}.
        \end{align*}
        Then, we have
        \begin{align*}
            \E_{\wt\Scal}\sbr{\rbr{\Sigmab_w^{-1/2} \wt\Phib_w^\top \wt\Phib_w \Sigmab_w^{-1/2}}^\dagger}
            = \frac{1}{n - d_w -1} \Vb_w \Vb_w^\top.
        \end{align*}
        Therefore, \eqref{eq:pf_var_w2s} implies
        \begin{align}\label{eq:pf_var_w2s_1}
        \begin{split}
            \vari(f_\wts) &= \frac{\sigma^2}{N}\ \frac{1}{n - d_w -1}\ \tr\rbr{\Vb_w^\top \E_{\Scal_x}\sbr{\Sigmab_w^{-1/2} \Phib_w^\top \Pb_s \Phib_w \Sigmab_w^{-1/2}} \Vb_w} \\
            &= \frac{\sigma^2}{N}\ \frac{1}{n - d_w -1}\ \tr\rbr{\E_{\Scal_x}\sbr{\Vb_w^\top \Vb_w \Gammab_w^\top \Pb_s \Gammab_w \Vb_w^\top \Vb_w}} \\
            &= \frac{\sigma^2}{N}\ \frac{1}{n - d_w -1}\ \tr\rbr{\E_{\Scal_x}\sbr{\Gammab_w^\top \Pb_s \Gammab_w}}.
        \end{split}
        \end{align}
        Recall that $\Pb_s = \Phib_s \Phib_s^\dagger$. Let $\Gammab_s = \Gammab \Vb_s \in \R^{N \times d_s}$, and we can write
        \begin{align*}
            \Pb_s = (\Phib_s \Sigmab_s^{-1/2}) (\Phib_s \Sigmab_s^{-1/2})^\dagger = (\Gammab_s \Vb_s^\top) (\Gammab_s \Vb_s^\top)^\dagger = \Gammab_s \Gammab_s^\dagger.
        \end{align*}
        Therefore, with $\Gammab_w = \Gammab \Vb_w$ and $\Gammab_s = \Gammab \Vb_s$, we can decompose
        \begin{align*}
            \tr\rbr{\E_{\Scal_x}\sbr{\Gammab_w^\top \Pb_s \Gammab_w}} 
            &= \E_{\Scal_x}\sbr{\tr\rbr{\Gammab_w^\top \Gammab_s \Gammab_s^\dagger \Gammab_w}} \\
            &= \E_{\Scal_x}\sbr{\tr\rbr{\Vb_w^\top \Vb_s \Vb_s^\top \Vb_w \Gammab_w^\top \Gammab_s \Gammab_s^\dagger \Gammab_w}} \\
            &+ \E_{\Scal_x}\sbr{\tr\rbr{\Vb_w^\top (\Ib_d - \Vb_s \Vb_s^\top) \Vb_w \Gammab_w^\top \Gammab_s \Gammab_s^\dagger \Gammab_w}}.
        \end{align*}
        For the first term, since $\Gammab_w \Vb_w^\top \Vb_s = \Gammab \Vb_w \Vb_w^\top \Vb_s$ and $\Gammab_s = \Gammab \Vb_s$, the range of $\Gammab_w \Vb_w^\top \Vb_s$ is a subspace of that of $\Gammab_s$ and therefore,
        \begin{align*}
            \E_{\Scal_x}\sbr{\tr\rbr{\Vb_w^\top \Vb_s \Vb_s^\top \Vb_w \Gammab_w^\top \Gammab_s \Gammab_s^\dagger \Gammab_w}} 
            &= \E_{\Scal_x}\sbr{\tr\rbr{ \Vb_s^\top \Vb_w \Gammab_w^\top \Gammab_s \Gammab_s^\dagger \Gammab_w \Vb_w^\top \Vb_s}} \\
            &= \E_{\Scal_x}\sbr{\tr\rbr{ \Vb_s^\top \Vb_w \Gammab_w^\top \Gammab_w \Vb_w^\top \Vb_s}} \\
            &= \tr\rbr{\Vb_s^\top \Vb_w \E_{\Scal_x}\sbr{\Gammab_w^\top \Gammab_w} \Vb_w^\top \Vb_s}.
        \end{align*}
        Since $\E_{\Scal_x}\sbr{\Gammab_w^\top \Gammab_w} = N \Ib_{d_w}$, we have
        \begin{align*}
            \E_{\Scal_x}\sbr{\tr\rbr{\Vb_w^\top \Vb_s \Vb_s^\top \Vb_w \Gammab_w^\top \Gammab_s \Gammab_s^\dagger \Gammab_w}} 
            &= N \tr\rbr{\Vb_s^\top \Vb_w \Vb_w^\top \Vb_s} \\
            &= N \nbr{\Vb_s^\top \Vb_w}_F^2 \\
            &= N d_{s \wedge w}.
        \end{align*}
        For the second term, we first observe that the row space of $\Gammab_w \Vb_w^\top (\Ib_d - \Vb_s \Vb_s^\top)$ is orthogonal to that of $\Gammab_s = \Gammab \Vb_s$, and therefore, $\Gammab_w \Vb_w^\top (\Ib_d - \Vb_s \Vb_s^\top)$ and $\Gammab_s$ are independent, which implies
        \begin{align*}
            \E_{\Scal_x}\sbr{\tr\rbr{\Vb_w^\top (\Ib_d - \Vb_s \Vb_s^\top) \Vb_w \Gammab_w^\top \Gammab_s \Gammab_s^\dagger \Gammab_w}} 
            &= \tr\rbr{\E\sbr{\Gammab_w \Vb_w^\top (\Ib_d - \Vb_s \Vb_s^\top) \Vb_w \Gammab_w^\top} \E\sbr{\Gammab_s \Gammab_s^\dagger}}.
        \end{align*}
        Since $\Gammab$ consists of independent isotropic rows, so do $\Gammab_s = \Gammab \Vb_s \in \R^{N \times d_s}$ and $\Gammab_w = \Gammab \Vb_w \in \R^{N \times d_w}$, which implies
        \begin{align*}
            \E\sbr{\Gammab_s \Gammab_s^\dagger} = \frac{d_s}{N}\ \Ib_N \quad \t{and} \quad \E\sbr{\Gammab_w^\top \Gammab_w} = N\ \Ib_{d_w}.
        \end{align*}
        Then, we have
        \begin{align*}
            \E_{\Scal_x}\sbr{\tr\rbr{\Vb_w^\top (\Ib_d - \Vb_s \Vb_s^\top) \Vb_w \Gammab_w^\top \Gammab_s \Gammab_s^\dagger \Gammab_w}} 
            &= \tr\rbr{\E\sbr{\Gammab_w \Vb_w^\top (\Ib_d - \Vb_s \Vb_s^\top) \Vb_w \Gammab_w^\top} \E\sbr{\Gammab_s \Gammab_s^\dagger}} \\
            &= \frac{d_s}{N} \tr\rbr{\E\sbr{\Gammab_w \Vb_w^\top (\Ib_d - \Vb_s \Vb_s^\top) \Vb_w \Gammab_w^\top}} \\
            &= \frac{d_s}{N} \tr\rbr{\Vb_w^\top (\Ib_d - \Vb_s \Vb_s^\top) \Vb_w \E\sbr{\Gammab_w^\top \Gammab_w}} \\
            &= \frac{d_s}{N} N \tr\rbr{\Vb_w^\top (\Ib_d - \Vb_s \Vb_s^\top) \Vb_w} \\
            &= d_s (d_w - d_{s \wedge w}).
        \end{align*}
        Combining the two terms, we have
        \begin{align*}
            \tr\rbr{\E_{\Scal_x}\sbr{\Gammab_w^\top \Pb_s \Gammab_w}} = N d_{s \wedge w} + d_s (d_w - d_{s \wedge w}).
        \end{align*}
        Then, by \eqref{eq:pf_var_w2s_1}, the variance is exactly characterized by
        \begin{align*}
            \vari(f_\wts) 
            &= \frac{\sigma^2}{N}\ \frac{N d_{s \wedge w} + d_s (d_w - d_{s \wedge w})}{n - d_w -1} \\
            &= \frac{\sigma^2}{n-d_w-1} \rbr{d_{s \wedge w} + \frac{d_s}{N} (d_w - d_{s \wedge w})}.
        \end{align*}

        \item \b{Sub-gaussian features}: Relaxing the Gaussian feature assumption, when $\wt\Gammab_w$ consists of $\iid$ sub-gaussian random vectors that are zero-mean and isotropic (\ie $\E[\wt\gammab_i] = \b0_{d_w}$ and $\E[\wt\gammab_i \wt\gammab_i^\top] = \Ib_{d_w}$), with $n \ge \Omega(d_w)$, \Cref{lem:trace_inv_subgaussian} implies that
        \begin{align*}
            \E_{\wt\Scal}\sbr{(\wt\Gammab_w^\top \wt\Gammab_w)^\dagger} \aleq O\rbr{\frac{1}{n}} \Ib_{d_w},
        \end{align*}
        and therefore,
        \begin{align*}
            \E_{\wt\Scal}\sbr{\rbr{\Sigmab_w^{-1/2} \wt\Phib_w^\top \wt\Phib_w \Sigmab_w^{-1/2}}^\dagger} \aleq O\rbr{\frac{1}{n}} \Vb_w \Vb_w^\top.
        \end{align*}
        Then, via an analogous argument as \eqref{eq:pf_var_w2s_1}, \eqref{eq:pf_var_w2s} implies that 
        \begin{align}\label{eq:pf_var_w2s_2}
        \begin{split}
            \vari(f_\wts) \le \frac{\sigma^2}{N}\ O\rbr{\frac{1}{n}}\ \tr\rbr{\E_{\Scal_x}\sbr{\Gammab_w^\top \Pb_s \Gammab_w}}.
        \end{split}
        \end{align}
        We observe that in the analysis of the Gaussian feature case, the characterization
        \begin{align*}
            \tr\rbr{\E_{\Scal_x}\sbr{\Gammab_w^\top \Pb_s \Gammab_w}} = (N - d_s) d_{s \wedge w} + d_s d_w
        \end{align*}
        does not involve the Gaussianity of $\Gammab$ and therefore holds for general subgaussian features.
        This leads to an upper bound on the variance:
        \begin{align*}
            \vari(f_\wts) 
            &\le \frac{\sigma^2}{N}\ O\rbr{\frac{1}{n}}\ \rbr{N d_{s \wedge w} + d_s (d_w - d_{s \wedge w})} \\
            &\lesssim \frac{\sigma^2}{n} \rbr{d_{s \wedge w} + \frac{d_s}{N} (d_w - d_{s \wedge w})}.
        \end{align*}
    \end{enumerate}
\end{proof}


\begin{lemma}[Adapting \cite{vershynin2010introduction} Theorem 5.39]\label{lem:trace_inv_subgaussian}
    Let $\wt\Gammab_w = [\wt\gammab_1, \ldots, \wt\gammab_n]^\top$ be an $n \times d_w$ matrix whose rows $\wt\gammab_1, \ldots, \wt\gammab_n$ consist of $\iid$ sub-gaussian random vectors that are zero-mean and isotropic (\ie $\E[\wt\gammab_i] = \b0_{d_w}$ and $\E[\wt\gammab_i \wt\gammab_i^\top] = \Ib_{d_w}$). When $n \ge \Omega(d_w)$, we have
    \begin{align*}
        \E\sbr{\nbr{\rbr{\wt\Gammab_w^\top \wt\Gammab_w}^\dagger}_2} \le O\rbr{\frac{1}{n}},
    \end{align*}
    where $\Omega(\cdot)$ and $O(\cdot)$ suppresses constants that depend only on the sub-gaussian norm $\nbr{\wt\gammab_i}_{\psi_2} = \sup_{\vb \in \SSS^{d_w-1}} \sup_{p \ge 1} (\E[|\wt\gammab_i^\top \vb|^p])^{1/p} / \sqrt{p}$, independent of $n, d_w$.
\end{lemma}

\begin{proof}[Proof of \Cref{lem:trace_inv_subgaussian}]
    Let $\sigma_{\min}(\wt\Gammab_w^\top \wt\Gammab_w)$ be the smallest singular value of $\wt\Gammab_w^\top \wt\Gammab_w$.
    Leveraging \cite{vershynin2010introduction} Theorem 5.39, we notice that for $n \ge \Omega(d_w)$, there exist constants $c_1, c_2 > 0$ that depend only on the sub-gaussian norm $\nbr{\wt\gammab_i}_{\psi_2}$ such that
    \begin{align*}
        \Pr\sbr{\sigma_{\min}(\wt\Gammab_w^\top \wt\Gammab_w) < \rbr{\sqrt{n} - c_1\sqrt{d_w} - t}^2} \le \exp\rbr{-c_2 t^2}.
    \end{align*}
    Therefore, we have 
    \begin{align*}
        \Pr\sbr{\frac{1}{\sigma_{\min}(\wt\Gammab_w^\top \wt\Gammab_w)} > t} \le \exp\rbr{-c_2 \rbr{\sqrt{n} - c_1 \sqrt{d_w} - \sqrt{\frac{1}{t}}}^2}.
    \end{align*}

    Notice that for any non-negative random variable $Z$ with a cumulative density function $F_Z(z)$, 
    \begin{align*}
        \E\sbr{Z} &= \int_0^\infty z d F_Z(z) 
        = - \int_0^\infty z d \rbr{1 - F_Z(z)} \\
        &= \sbr{z \rbr{1 - F_Z(z)}}_0^\infty + \int_0^\infty \rbr{1 - F_Z(z)} dz \\
        &= \int_0^\infty \Pr\sbr{Z > z} dz.
    \end{align*}
    Therefore, we have
    \begin{align*}
        \E\sbr{\frac{1}{\sigma_{\min}(\wt\Gammab_w^\top \wt\Gammab_w)}} \le \int_0^\infty \exp\rbr{-c_2 \rbr{\sqrt{n} - c_1 \sqrt{d_w} - \sqrt{\frac{1}{t}}}^2} d t.
    \end{align*}
    Let $t_0 = 1 / \rbr{\sqrt{n} - c_1 \sqrt{d_w}}^2$ such that $\sqrt{n} - c_1 \sqrt{d_w} - \sqrt{\frac{1}{t}}=0$ and 
    \begin{align*}
        \int_{0}^{t_0} \exp\rbr{-c_2 \rbr{\sqrt{n} - c_1 \sqrt{d_w} - \sqrt{\frac{1}{t}}}^2} d t \le t_0
    \end{align*}
    Then, we have
    \begin{align*}
        &\E\sbr{\frac{1}{\sigma_{\min}(\wt\Gammab_w^\top \wt\Gammab_w)}} 
        \le \int_0^\infty \exp\rbr{-c_2 \rbr{\sqrt{n} - c_1 \sqrt{d_w} - \sqrt{\frac{1}{t}}}^2} d t \\
        &\le t_0 + \int_{t_0}^\infty \exp\rbr{-c_2 \rbr{\sqrt{n} - c_1 \sqrt{d_w} - \sqrt{\frac{1}{t}}}^2} d t \\
        &= t_0 + 2 \int_{0}^{\sqrt{n}-c_1\sqrt{d_w}} \exp\rbr{-c_2 u^2} \rbr{\sqrt{n} - c_1 \sqrt{d_w} - u}^{-3} d u \\
        &= t_0 + \frac{2}{\rbr{\sqrt{n} - c_1 \sqrt{d_w}}^2} \int_{0}^{1} \exp\rbr{-c_2 \rbr{\sqrt{n}-c_1\sqrt{d_w}}^2 u^2} \rbr{1 - u}^{-3} d u \\
        &= \frac{1}{\rbr{\sqrt{n} - c_1 \sqrt{d_w}}^2} + \frac{2}{\rbr{\sqrt{n} - c_1 \sqrt{d_w}}^2} \rbr{\int_{0}^{1} \exp\rbr{-\Omega\rbr{u^2}} \rbr{1 - u}^{-3} d u} \\
        &= O\rbr{\frac{1}{\rbr{\sqrt{n} - c_1 \sqrt{d_w}}^2}}.
    \end{align*}
    When $n \ge \Omega(d_w)$, we have $\sqrt{n} - c_1 \sqrt{d_w} \ge \Omega(\sqrt{n})$, and therefore ,
    \begin{align*}
        \E\sbr{\nbr{\rbr{\wt\Gammab_w^\top \wt\Gammab_w}^\dagger}_2}
        \le \E\sbr{\frac{1}{\sigma_{\min}(\wt\Gammab_w^\top \wt\Gammab_w)}} 
        \le O\rbr{\frac{1}{n}}.
    \end{align*}
\end{proof}





\subsection{Proof of \Cref{pro:sft_weak} and \Cref{cor:sft_strong}}\label{apx:pf_sft_weak}
\begin{proof}[Proof of \Cref{pro:sft_weak} and \Cref{cor:sft_strong}]
    The excess risk of the finetuned weak teacher $f_w(\xb) = \phi_w(\xb)^\top \thetab_w$ can be expressed as
    \begin{align*}
        \exrisk(f_w) &= \E_{\xb \sim \Dcal}\sbr{\E_{f_w}\sbr{(f_w(\xb) - f_*(\xb))^2}} \\
        &= \E_{\wt\Scal}\sbr{\frac{1}{n}\nbr{\wt\Phib_w \thetab_w - \wt\fb_*}_2^2},
    \end{align*}
    where $\wt\fb_* = [\fb_*(\wt\xb_1), \ldots, \fb_*(\wt\xb_n)]^\top \in \R^n$; and we recall that $\wt\Phib_w = [\phi_w(\wt\xb_1), \ldots, \phi_w(\wt\xb_n)]^\top$. Notice that the randomness of $\thetab_w$ comes from the SFT samples $\wt\Scal \sim \Dcal(f_*)^n$.

    Observe that the solution of \eqref{eq:sft_weak} as $\alpha_w \to 0$ is given by $\thetab_w = \wt\Phib_w^\dagger \wt\yb$, where $\wt\yb = \wt\fb_* + \wt\zb$ is the noisy label vector with $\wt\zb \sim \Ncal(\b0_n, \sigma^2 \Ib_n)$.
    Therefore, with the randomness over $\wt\Scal \sim \Dcal(f_*)^n$, we have
    \begin{align*}
        \exrisk(f_w) &= \E \sbr{\frac{1}{n}\nbr{\wt\Phib_w \wt\Phib_w^\dagger \wt\yb - \wt\fb_*}_2^2} \\
        &= \E \sbr{\frac{1}{n}\nbr{\wt\Phib_w \wt\Phib_w^\dagger \wt\zb + \rbr{\wt\Phib_w \wt\Phib_w^\dagger \wt\fb_* - \wt\fb_*}}_2^2} \\
        &= \underbrace{\E \sbr{\frac{1}{n}\nbr{\wt\Phib_w \wt\Phib_w^\dagger \wt\zb}_2^2}}_{\vari(f_w)} + \underbrace{\E\sbr{\frac{1}{n}\nbr{\wt\Phib_w \wt\Phib_w^\dagger \wt\fb_* - \wt\fb_*}_2^2}}_{\bias(f_w)}.
    \end{align*}
    
    For bias, by the definition of finetuning capacity (see \Cref{def:ft_est_err}), we have
    \begin{align*}
        \bias(f_w) = \frac{1}{n} \E\sbr{\nbr{\wt\Phib_w \wt\Phib_w^\dagger \wt\fb_* - \wt\fb_*}_2^2} = \frac{\rho_w(n)}{n}.
    \end{align*}
    We observe that $\bias(f_w) \le \rho_w$ by \Cref{lem:low_est_err_ft}.
    Notice that \Cref{lem:low_est_err_ft} also implies $\bias(f_s) = {\rho_s(n)}/{n} \le \rho_s$. 

    For variance, we observe that 
    \begin{align*}
        \vari(f_w) &= \frac{1}{n} \E\sbr{\nbr{\wt\Phib_w \wt\Phib_w^\dagger \wt\zb}_2^2} \\
        &= \frac{1}{n} \E\sbr{\tr\rbr{\wt\Phib_w \wt\Phib_w^\dagger \wt\zb \wt\zb^\top}} \\
        &= \frac{\sigma^2}{n} \E\sbr{\tr\rbr{\wt\Phib_w \wt\Phib_w^\dagger}}.
    \end{align*}
    By \Cref{asm:ft_data}, since $\rank(\wt\Phib_w) = d_w$ almost surely, we have
    \begin{align*}
        \vari(f_w) = \frac{\sigma^2}{n} \E\sbr{\tr\rbr{\wt\Phib_w \wt\Phib_w^\dagger}} = \frac{\sigma^2 d_w}{n}.
    \end{align*}
\end{proof}



\subsection{Proof of \Cref{cor:pgr}}\label{apx:pf_pgr}
\begin{proof}[Proof of \Cref{cor:pgr}]
    Noticing that with $\rank(\wt\Phib_w) = d_w$ and $\rank(\wt\Phib_s) = \rank(\Phib_s) = d_s$ almost surely, the excess risks of $f_w, f_s, f_c$ are characterized exactly in \Cref{pro:sft_weak} and \Cref{cor:sft_strong}, and $\exrisk(f_\wts)$ is upper bounded by \Cref{thm:w2s_ft}.
    Therefore, by directly plugging in the excess risks to the definitions of PGR and OPR, we have
    \begin{align}\label{eq:pgr_lower_tight}
    \begin{split}
        \pgr = &\frac{\exrisk(f_w) - \exrisk(f_\wts)}{\exrisk(f_w) - \exrisk(f_c)} \\
        \ge &\rbr{\sigma^2\ \frac{d_w}{n} + \frac{\rho_w(n)}{n} - \frac{\sigma^2}{n-d_w-1} \rbr{d_{s \wedge w} + \frac{d_s}{N} (d_w-d_{s \wedge w})} - \rbr{\frac{\rho_w(n)}{n} + \frac{\rho_s(N)}{N}}} \\
        &\rbr{\sigma^2\ \frac{d_w}{n} + \frac{\rho_w(n)}{n} - \sigma^2\ \frac{d_s}{N+n} - \frac{\rho_s(N+n)}{N+n}}^{-1} \\
        \ge &\rbr{\sigma^2 \frac{d_w}{n} - \sigma^2 \frac{d_{s \wedge w} + (d_w - d_{s \wedge w}) {d_s}/{N}}{n-d_w-1} - \frac{\rho_s(N)}{N}} \Big/ \rbr{\sigma^2 \frac{d_w}{n} + \frac{\rho_w(n)}{n}}, \\
        \ge &\rbr{\sigma^2 \frac{d_w}{n} - \sigma^2 \frac{d_{s \wedge w} + (d_w - d_{s \wedge w}) {d_s}/{N}}{n-d_w-1} - \rho_s} \Big/ \rbr{\sigma^2 \frac{d_w}{n} + \rho_w},
    \end{split}
    \end{align}
    and 
    \begin{align}\label{eq:opr_lower_tight}
    \begin{split}
        \opr = &\frac{\exrisk(f_s)}{\exrisk(f_\wts)} \\
        \ge &\rbr{\sigma^2\ \frac{d_s}{n} + \frac{\rho_s(n)}{n}} \Big/ \rbr{\sigma^2 \frac{d_{s \wedge w} + (d_w - d_{s \wedge w}) {d_s}/{N}}{n-d_w-1} + \rbr{\frac{\rho_w(n)}{n} + \frac{\rho_s(N)}{N}}} \\
        \ge &\sigma^2 \frac{d_s}{n} \Big/ \rbr{\sigma^2 \frac{d_{s \wedge w} + (d_w - d_{s \wedge w}) {d_s}/{N}}{n-d_w-1} + \rho_w + \rho_s}.
    \end{split}
    \end{align} 

    When taking $n = d_w + q + 1$ for some small constant $q \in \N$, we observe that 
    \begin{align*}
        \pgr &\ge \rbr{\sigma^2 \frac{d_w}{n} - \sigma^2 \frac{d_{s \wedge w} + (d_w - d_{s \wedge w}) {d_s}/{N}}{n-d_w-1} - \rho_s} \Big/ \rbr{\sigma^2 \frac{d_w}{n} + \rho_w} \\
        &\ge \rbr{\frac{d_w}{d_w + q + 1} - \frac{d_{s \wedge w}}{q} - \frac{d_s}{N} \frac{d_w - d_{s \wedge w}}{q} - \frac{\rho_s}{\sigma^2}} \Big/ \rbr{\frac{d_w}{d_w + q + 1} + \frac{\rho_w}{\sigma^2}} \\
        &\ge \rbr{\frac{d_w}{d_w + q + 1} - \frac{d_{s \wedge w}}{q} - \frac{d_s}{N} \frac{d_w - d_{s \wedge w}}{q} - \frac{\rho_s}{\sigma^2} - \frac{\rho_w}{\sigma^2}} \Big/ \rbr{\frac{d_w}{d_w + q + 1} + \frac{\rho_w}{\sigma^2} - \frac{\rho_w}{\sigma^2}} \\
        &= 1 - \frac{n}{d_w} \rbr{\frac{d_{s \wedge w}}{q} + \frac{d_s}{N} \frac{d_w - d_{s \wedge w}}{q} + \frac{\rho_w + \rho_s}{\sigma^2}} \\
        &= 1 - \frac{n}{q}\ {\frac{d_{s \wedge w} + (d_w - d_{s \wedge w}) d_s / N}{d_w}} - \frac{n}{d_w}\ {\frac{\rho_w + \rho_s}{\sigma^2}},
    \end{align*}
    and
    \begin{align*}
        \opr &\ge \sigma^2 \frac{d_s}{n} \Big/ \rbr{\sigma^2 \frac{d_{s \wedge w} + (d_w - d_{s \wedge w}) {d_s}/{N}}{n-d_w-1} + \rho_w + \rho_s} \\
        &= \frac{d_s}{n} \Big/ \rbr{\frac{d_{s \wedge w} + (d_w - d_{s \wedge w}) {d_s}/{N}}{q} + \frac{\rho_w + \rho_s}{\sigma^2}} \\
        &= \rbr{\frac{n}{q}\ \frac{d_{s \wedge w} + (d_w - d_{s \wedge w}) {d_s}/{N}}{d_s} + \frac{n}{d_s}\ \frac{\rho_w + \rho_s}{\sigma^2}}^{-1}.
    \end{align*}
\end{proof}



\subsection{Proof of \Cref{cor:non_monotonic_scaling}}\label{apx:pf_non_monotonic_scaling}
\begin{proof}[Proof of \Cref{cor:non_monotonic_scaling}]
    Recall the notations introduced for conciseness:
    \begin{align*}
        d_\wts(N) = d_{s \wedge w} + (d_w - d_{s \wedge w}) \frac{d_s}{N}, \quad \varrho = \frac{\rho_w + \rho_s}{\sigma^2}.
    \end{align*}
    Then, the lower bounds for $\pgr$ and $\opr$ in \Cref{cor:pgr} can be expressed in terms of $d_\wts(N)$ and $\varrho$ as 
    \begin{align*}
        \pgr \ge 1 - \frac{d_\wts(N)}{d_w} - \frac{d_w + 1}{d_w} \varrho - \frac{d_w + 1}{d_w}\ \frac{d_\wts(N)}{q} - q \frac{\varrho}{d_w},
    \end{align*}
    and 
    \begin{align*}
        \opr \ge \rbr{\frac{d_\wts(N)}{d_s} + \frac{d_w + 1}{d_s} \varrho + \frac{d_\wts(N)}{d_s}\ \frac{d_w + 1}{q} + q \frac{\varrho}{d_s}}^{-1}.
    \end{align*}
    Both lower bounds are maximized when the last two terms in the expressions that involve $q$ are minimized, which is achieved when $q = \sqrt{\rbr{d_w + 1} {d_\wts(N)}/{\varrho}}$. Substituting the optimal $q$ back into the expressions yields the best lower bounds for $\pgr$ and $\opr$:
    \begin{align*}
        \pgr \ge &1 - \frac{d_\wts(N)}{d_w} - \varrho \frac{d_w + 1}{d_w} - 2 \sqrt{\varrho \frac{d_w + 1}{d_w}\ \frac{d_\wts(N)}{d_w}} \\
        = &1 - \rbr{\sqrt{\frac{d_\wts(N)}{d_w}} + \sqrt{\varrho\ \frac{d_w + 1}{d_w}}}^2,
    \end{align*}
    and 
    \begin{align*}
        \opr \ge &\rbr{\frac{d_\wts(N)}{d_s} + \varrho \frac{d_w + 1}{d_s} + 2 \sqrt{\varrho \frac{d_w + 1}{d_s}\ \frac{d_\wts(N)}{d_s}}}^{-1} \\
        = &\rbr{\sqrt{\frac{d_\wts(N)}{d_s}} + \sqrt{\varrho\ \frac{d_w + 1}{d_s}}}^{-2}.
    \end{align*}
\end{proof}




\section{Ridge regression analysis}\label{apx:ridge_regression}
In this section, we investigate the more realistic scenario where the weak and strong feature covariances are not exactly low-rank but admit small numbers of dominating eigenvalues. 

Concretely, we consider the same data distribution $(\xb, y) \sim \Dcal(f_*)$ with $y = f_*(\xb) + z$ for some independent Gaussian label noise $z \sim \Ncal(0, \sigma^2)$ and an unknown ground truth function $f_*: \Xcal \to \R$ as in \Cref{sec:ridgeless_regression}.
Under the same sub-gaussian feature assumption as in \Cref{asm:features}, we adapt \Cref{def:low_intrinsic_dim,def:correlation_dim} to the ridge regression setting as follows.
\begin{assumption}[Data distribution]\label{asm:ridge_regression}
    Let $\phi_s: \Xcal \to \R^d$ and $\phi_w: \Xcal \to \R^d$ be the strong and weak pretrained models that take $\xb \sim \Dcal$ and output pretrained features $\phi_s(\xb), \phi_w(\xb) \in \R^d$, respectively.
    \begin{enumerate}[label=(\roman*)]
        \item \b{Ground truth}: Assume $f_*$ can be expressed as a linear function over an unknown ground truth feature $\phi_*: \Xcal \to \R^d$ such that $f_*(\cdot) = \phi_*(\cdot)^\top \thetab_*$ for some fixed $\thetab_* \in \R^d$.
        \item \b{Sub-gaussian features} (\Cref{asm:features}): Let $\phi_w(\xb)$, $\phi_s(\xb)$, $\phi_*(\xb)$ be zero-mean sub-gaussian random vectors with $\E[\phi_w(\xb)] = \E[\phi_s(\xb)] = \E[\phi_*(\xb)] = \b{0}_d$, and 
        \begin{align*}
            \E[\phi_w(\xb) \phi_w(\xb)^\top] = \Sigmab_w, \quad \E[\phi_s(\xb) \phi_s(\xb)^\top] = \Sigmab_s, \quad \E[\phi_*(\xb) \phi_*(\xb)^\top] = \Sigmab_*.
        \end{align*}
        For conciseness, we assume without loss of generality that these features are roughly normalized, \ie, $\nbr{\Sigmab_w}_2 \asymp 1$, $\nbr{\Sigmab_s}_2 \asymp 1$, and $\nbr{\Sigmab_*}_2 \asymp 1$.
        \item \b{Low intrinsic dimension}: Let $\Sigmab_s$ and $\Sigmab_w$ both be \b{positive-definite} with spectral decompositions $\Sigmab_s = \Vb_s \Lambdab_s \Vb_s^\top$ and $\Sigmab_w = \Vb_w \Lambdab_w \Vb_w^\top$, where $\Lambdab_s, \Lambdab_w \in \R^{d \times d}$ are diagonal matrices with positive eigenvalues in decreasing order; while $\Vb_s \in \R^{d \times d}$ and $\Vb_w \in \R^{d \times d}$ are orthogonal matrices consisting of the corresponding orthonormal eigenvectors. The low intrinsic dimension of FT implies that $\Lambdab_s = \diag(\lambda^s_1,\cdots,\lambda^s_d)$ and $\Lambdab_w = \diag(\lambda^w_1,\cdots,\lambda^w_d)$ consist of a few dominating eigenvalues, while the rest of the eigenvalues are negligible, \ie, there exist $d_s, d_w \ll d$ such that $\sum_{i > d_s} \lambda^s_i \ll \tr(\Sigmab_s)$ and $\sum_{i > d_w} \lambda^w_i \ll \tr(\Sigmab_w)$. Here, 
        \begin{align*}
            \tr(\Sigmab_s) \lesssim d_s \quad \t{and} \quad \tr(\Sigmab_w) \lesssim d_w
        \end{align*}
        effectively measure the intrinsic dimensions of $\phi_s$ and $\phi_w$.
    \end{enumerate}
\end{assumption}

\begin{remark}[Weak-strong similarity]
    In place of correlation dimension (\Cref{def:correlation_dim}) in the ridgeless setting, for the ridge regression analysis, we measure the similarity between the weak and strong models directly through $\tr(\Sigmab_s \Sigmab_w)$. Notice that 
    \begin{align*}
        \tr(\Sigmab_s \Sigmab_w) \le \min\cbr{\tr(\Sigmab_s)\nbr{\Sigmab_w}_2, \tr(\Sigmab_w)\nbr{\Sigmab_s}_2} \lesssim \min\cbr{\tr(\Sigmab_s), \tr(\Sigmab_w)}.
    \end{align*}
    In particular, when $\Sigmab_s$ and $\Sigmab_w$ admit low intrinsic dimensions, $\tr(\Sigmab_s \Sigmab_w)$ can be much smaller than $\min\cbr{\tr(\Sigmab_s), \tr(\Sigmab_w)}$ if their eigenvectors corresponding to the dominating eigenvalues are almost orthogonal.
\end{remark}

\begin{remark}[FT approximation errors]
    It is worth noting that under the ground truth and positive-definite covariance assumptions in \Cref{asm:ridge_regression}(i, iii), the FT approximation errors in \Cref{def:ft_est_err} satisfy
    \begin{align}\label{eq:pf_ridge_ft_approx_err}
    \begin{split}
        &\rho_s = \min_{\thetab \in \R^d} \E_{\xb \sim \Dcal}\sbr{(\phi_s(\xb)^\top \thetab - f_*(\xb))^2} = 0 \quad (\t{when } \thetab = \Sigmab_s^{-1} \Sigmab_* \thetab_*), \\
        &\rho_w = \min_{\thetab \in \R^d} \E_{\xb \sim \Dcal}\sbr{(\phi_w(\xb)^\top \thetab - f_*(\xb))^2} = 0 \quad (\t{when } \thetab = \Sigmab_w^{-1} \Sigmab_* \thetab_*).
    \end{split}
    \end{align}
    In place of \Cref{def:ft_est_err}, with positive-definite covariances in \Cref{asm:ridge_regression}, we measure the alignment between the ground truth feature $\phi_*$ and the weak/strong feature $\phi_w, \phi_s$ through
    \begin{align*}
        \varrho_s = \|\Sigmab_s^{-1/2} \Sigmab_*^{1/2} \thetab_*\|_2^2, \quad \varrho_w = \|\Sigmab_w^{-1/2} \Sigmab_*^{1/2} \thetab_*\|_2^2.
    \end{align*}
    Intuitively, for $\Sigmab_s$ and $\Sigmab_w$ with a few dominating eigenvalues (\Cref{asm:ridge_regression}(iii)), $\varrho_s$ and $\varrho_w$ are small if the eigensubspace associated with non-negligible eigenvalues of $\Sigmab_*$ is fully covered by the eigensubspaces associated with the dominating eigenvalues of $\Sigmab_s$ and $\Sigmab_w$, respectively. 
\end{remark}

The W2S FT under ridge regression consists of two steps.
\begin{enumerate}[label=(\alph*)]
    \item First, the weak teacher $f_w(\xb) = \phi_w(\xb)^\top \thetab_w$ is supervisedly finetuned over $\wt\Scal$: 
    \begin{align}\label{eq:w2s_weak_ridge}
        \thetab_w = \argmin_{\thetab \in \R^d} \frac{1}{n}\nbr{\wt\Phib_w \thetab - \wt\yb}_2^2 + \alpha_w \nbr{\thetab}_2^2, \quad \alpha_w > 0.
    \end{align}
    \item Then, the W2S model $f_\wts(\xb) = \phi_s(\xb)^\top \thetab_\wts$ is finetuned over the strong feature $\phi_s$ through the unlabeled samples $\Scal_x$ and their pseudo-labels generated by the weak teacher model:
    \begin{align}\label{eq:w2s_strong_ridge}
        \thetab_\wts = \argmin_{\thetab \in \R^d} \frac{1}{N}\nbr{\Phib_s \thetab - \Phib_w \thetab_w}_2^2 + \alpha_\wts \nbr{\thetab}_2^2, \quad \alpha_\wts > 0.
    \end{align}
\end{enumerate}

\begin{theorem}[W2S under ridge regression]\label{thm:w2s_ridge}
    Let $\varrho_w = \nbr{\Sigmab_w^{-1/2} \Sigmab_*^{1/2} \thetab_*}_2^2$ and $\varrho_s = \nbr{\Sigmab_s^{-1/2} \Sigmab_*^{1/2} \thetab_*}_2^2$.
    Under \Cref{asm:ridge_regression}, the generalization error of W2S FT via ridge regression with fixed $\alpha_w, \alpha_\wts > 0$, $\exrisk(f_\wts) = \vari(f_\wts) + \bias(f_\wts)$, is upper bounded by
    \begin{align*}
        \vari(f_\wts) \le \frac{\sigma^2 \tr\rbr{\Sigmab_s \Sigmab_w}}{4 (\alpha_w n) (\alpha_\wts N)}, \quad
        \bias(f_\wts) \le \alpha_w \varrho_w + \alpha_\wts \varrho_s.
    \end{align*}
    In particular, when taking  
    \begin{align*}
        \alpha_w = \rbr{\frac{\sigma^2 \tr\rbr{\Sigmab_s \Sigmab_w}}{4 n N}\ \frac{\varrho_s}{\varrho_w^2}}^{1/3}, \quad 
        \alpha_\wts = \rbr{\frac{\sigma^2 \tr\rbr{\Sigmab_s \Sigmab_w}}{4 n N}\ \frac{\varrho_w}{\varrho_s^2}}^{1/3},
    \end{align*}
    the excess risk of W2S FT is upper bounded by
    \begin{align*}
        \exrisk(f_\wts) \le 3 \rbr{\frac{\sigma^2 \tr\rbr{\Sigmab_s \Sigmab_w}}{4 n N}\ \varrho_s \varrho_w}^{1/3}.
    \end{align*}
\end{theorem}

\Cref{thm:w2s_ridge} conveys a similar high-level intuition as in \Cref{thm:w2s_ft} regarding the effect of the weak-strong similarity on the generalization error of W2S FT. In particular, the larger discrepancy between $\phi_s$ and $\phi_w$ (corresponding to the smaller $\tr\rbr{\Sigmab_s \Sigmab_w}$) leads to lower variance and therefore better W2S generalization.

Meanwhile, a key difference in W2S between the ridge and ridgeless settings (\Cref{thm:w2s_ridge} versus \Cref{thm:w2s_ft}) is that the FT approximation errors in \Cref{thm:w2s_ridge}, reflected by $\varrho_s = \|\Sigmab_s^{-1/2} \Sigmab_*^{1/2} \thetab_*\|_2^2$ and $\varrho_w = \|\Sigmab_w^{-1/2} \Sigmab_*^{1/2} \thetab_*\|_2^2$, can be compensated by larger sample sizes $n, N$ and directly affect the sample complexity: 
\begin{align*}
    n N \asymp \sigma^2 \tr\rbr{\Sigmab_s \Sigmab_w} \varrho_s \varrho_w.
\end{align*}
Such difference is a result of optimizing the regularization hyperparameters $\alpha_w, \alpha_\wts$ in ridge regression that control the variance-bias tradeoff.

\begin{proof}[Proof of \Cref{thm:w2s_ridge}]
    We first formalize some useful facts on the features and labels as in \eqref{eq:pf_var_w2s_subgaussian_asm}.
    In particular, the sub-gaussian assumption in \Cref{asm:ridge_regression}(ii) implies that for each $\xb \sim \Dcal$, the corresponding strong/weak feature $\phi_s(\xb), \phi_w(\xb) \in \R^d$, and the ground truth $f_*(\xb) \in \R$ are simultaneously characterized by an independent subgaussian random vector $\gammab \in \R^d$ with $\E[\gammab] = \b0_{d}$ and $\E[\gammab \gammab^\top] = \Ib_{d}$, \ie,
    \begin{align*}
        \phi_s(\xb) = \Sigmab_s^{1/2} \gammab, \quad \phi_w(\xb) = \Sigmab_w^{1/2} \gammab, \quad f_*(\xb) = \phi_*(\xb)^\top \thetab_* = \gammab^\top \Sigmab_*^{1/2} \thetab_*.
    \end{align*}

    Then, for $\Scal$ and $\wt\Scal$, there exist independent random matrices $\Gammab = [\gammab_1, \ldots, \gammab_N]^\top \in \R^{N \times d}$ and $\wt\Gammab = [\wt\gammab_1, \ldots, \wt\gammab_n]^\top \in \R^{n \times d}$ consisting of $\iid$ zero-mean isotropic rows such that
    \begin{align}\label{eq:pf_var_w2s_subgaussian_asm_2}
    \begin{split}
        &\Phib_s = \Gammab \Sigmab_s^{1/2} = \Gammab_s \Lambdab_s^{1/2} \Vb_s^\top, \\
        &\Phib_w = \Gammab \Sigmab_w^{1/2} = \Gammab_w \Lambdab_w^{1/2} \Vb_w^\top, \\
        &\yb = \fb_* + \zb, \quad \fb_* = \Gammab \Sigmab_*^{1/2} \thetab_*, \quad \zb \sim \Ncal(\b0_N, \sigma^2 \Ib_N), \\
        &\wt\Phib_w = \wt\Gammab \Sigmab_w^{1/2} = \wt\Gammab_w \Lambdab_w^{1/2} \Vb_w^\top, \\
        &\wt\yb = \wt\fb_* + \wt\zb, \quad \wt\fb_* = \wt\Gammab \Sigmab_*^{1/2} \thetab_*, \quad \wt\zb \sim \Ncal(\b0_n, \sigma^2 \Ib_n),
    \end{split}
    \end{align}
    where $\Gammab_s = \Gammab \Vb_s$, $\Gammab_w = \Gammab \Vb_w$, and $\wt\Gammab_w = \wt\Gammab \Vb_w$.

    \paragraph{Variance-bias decomposition.}
    Recall that the excess risk of W2S generalization $\exrisk(f_\wts)$ can be decomposed into the variance and bias terms:
    \begin{align*}
        &\vari(f_\wts) = \E_{\xb \sim \Dcal}\sbr{\E_{\Scal_x, \wt\Scal}\sbr{(f_\wts(\xb) - \E_{\Scal_x, \wt\Scal}[f_\wts(\xb)])^2}}, \\
        &\bias(f) = \E_{\xb \sim \Dcal}\sbr{(\E_{\Scal_x, \wt\Scal}[f_\wts(\xb)] - f_*(\xb))^2}.
    \end{align*}
    With $\alpha_w > 0$, \eqref{eq:w2s_weak_ridge} yields a weak teacher model $f_w(\xb) = \phi_w(\xb)^\top \thetab_w$ with 
    \begin{align*}
        \thetab_w = \rbr{\wt\Phib_w^\top \wt\Phib_w + \alpha_w n \Ib_d}^{-1} \wt\Phib_w^\top \rbr{\wt\fb_8 + \wt\zb}.
    \end{align*}
    Then, the W2S model $f_\wts(\xb) = \phi_s(\xb)^\top \thetab_\wts$ is given by \eqref{eq:w2s_strong_ridge} with $\alpha_\wts > 0$:
    \begin{align*}
        \thetab_\wts = &\rbr{\Phib_s^\top \Phib_s + \alpha_\wts N \Ib_d}^{-1} \Phib_s^\top \Phib_w \thetab_w \\
        = &\rbr{\Phib_s^\top \Phib_s + \alpha_\wts N \Ib_d}^{-1} \Phib_s^\top \Phib_w \rbr{\wt\Phib_w^\top \wt\Phib_w + \alpha_w n \Ib_d}^{-1} \wt\Phib_w^\top \rbr{\wt\fb_* + \wt\zb},
    \end{align*}
    which implies
    \begin{align*}
        \E_{\Scal_x, \wt\Scal}[\thetab_\wts] = \rbr{\Phib_s^\top \Phib_s + \alpha_\wts N \Ib_d}^{-1} \Phib_s^\top \Phib_w \rbr{\wt\Phib_w^\top \wt\Phib_w + \alpha_w n \Ib_d}^{-1} \wt\Phib_w^\top \wt\fb_*.
    \end{align*}
    Then, we can concretize the variance and bias terms as:
    \begin{align}\label{eq:pf_ridge_var}
    \begin{split}
        &\vari(f_\wts) = \E_{\xb \sim \Dcal}\sbr{\E_{\Scal_x, \wt\Scal}\sbr{(f_\wts(\xb) - \E_{\Scal_x, \wt\Scal}[f_\wts(\xb)])^2}} \\
        = &\E_{\Scal_x, \wt\Scal}\sbr{\nbr{\Sigmab_s^{1/2} \rbr{\Phib_s^\top \Phib_s + \alpha_\wts N \Ib_d}^{-1} \Phib_s^\top \Phib_w \rbr{\wt\Phib_w^\top \wt\Phib_w + \alpha_w n \Ib_d}^{-1} \wt\Phib_w^\top \wt\zb}_2^2},
    \end{split}
    \end{align}
    and
    \begin{align}\label{eq:pf_ridge_bias}
    \begin{split}
        &\bias(f_\wts) = \E_{\xb \sim \Dcal}\sbr{(\E_{\Scal_x, \wt\Scal}[f_\wts(\xb)] - f_*(\xb))^2} \\
        = &\E_{\Scal_x, \wt\Scal}\sbr{\frac{1}{N} \nbr{\Phib_s \rbr{\Phib_s^\top \Phib_s + \alpha_\wts N \Ib_d}^{-1} \Phib_s^\top \Phib_w \rbr{\wt\Phib_w^\top \wt\Phib_w + \alpha_w n \Ib_d}^{-1} \wt\Phib_w^\top \wt\fb_* - \fb_*}_2^2}.
    \end{split}
    \end{align}
    Now, we are ready to upper bound the variance and bias terms separately.

    \paragraph{Variance.}
    Denote $\zetab = \Lambdab_w^{1/2} \Vb_w^\top \rbr{\wt\Phib_w^\top \wt\Phib_w + \alpha_w n \Ib_d}^{-1} \wt\Phib_w^\top \wt\zb \in \R^d$, whose randomness comes from $\wt\Scal$ only, independent of $\Scal_x$.
    Then, the variance term \eqref{eq:pf_ridge_var} can be expressed as
    \begin{align*}
        &\vari(f_\wts) = \E_{\Scal_x, \wt\Scal}\sbr{\nbr{\Sigmab_s^{1/2} \rbr{\Phib_s^\top \Phib_s + \alpha_\wts N \Ib_d}^{-1} \Phib_s^\top \Phib_w \zetab}_2^2} \\
        = &\tr\rbr{\E_{\Scal_s}\rbr{\Gammab_w^\top \Phib_s \rbr{\Phib_s^\top \Phib_s + \alpha_\wts N \Ib_d}^{-1} \Sigmab_s \rbr{\Phib_s^\top \Phib_s + \alpha_\wts N \Ib_d}^{-1} \Phib_s^\top \Gammab_w} \E_{\wt\Scal}\sbr{\zetab \zetab^\top}} \\
        = &\tr\rbr{\E_{\Scal_s}\rbr{\Gammab_w^\top \Gammab_s \rbr{\Gammab_s^\top \Gammab_s + \alpha_\wts N \Lambdab_s^{-1}}^{-1} \rbr{\Gammab_s^\top \Gammab_s + \alpha_\wts N \Lambdab_s^{-1}}^{-1} \Gammab_s^\top \Gammab_w} \E_{\wt\Scal}\sbr{\zetab \zetab^\top}} \\
        = &\tr\rbr{\E_{\Scal_s}\rbr{\Vb_w^\top \Gammab^\top \Gammab \rbr{\Gammab^\top \Gammab + \alpha_\wts N \Sigmab_s^{-1}}^{-2} \Gammab^\top \Gammab \Vb_w} \E_{\wt\Scal}\sbr{\zetab \zetab^\top}} \\
        = &\tr\rbr{\E_{\Scal_s}\rbr{\Gammab^\top \Gammab \rbr{\Gammab^\top \Gammab + \alpha_\wts N \Sigmab_s^{-1}}^{-2} \Gammab^\top \Gammab} \E_{\wt\Scal}\sbr{\Vb_w \zetab \zetab^\top \Vb_w^\top}}.
    \end{align*}
    Notice that $\rbr{\Gammab^\top \Gammab + \alpha_\wts N \Sigmab_s^{-1}}^{2} \succeq \alpha_\wts N \rbr{\Gammab^\top \Gammab \Sigmab_s^{-1} + \Sigmab_s^{-1} \Gammab^\top \Gammab}$.
    Since matrix inversion is convex, a Jensen-type inequality implies that
    \begin{align*}
        &\Gammab^\top \Gammab \rbr{\Gammab^\top \Gammab + \alpha_\wts N \Sigmab_s^{-1}}^{-2} \Gammab^\top \Gammab \\
        \preceq &\Gammab^\top \Gammab \rbr{\alpha_\wts N \rbr{\Gammab^\top \Gammab \Sigmab_s^{-1} + \Sigmab_s^{-1} \Gammab^\top \Gammab}}^{\dagger} \Gammab^\top \Gammab \\
        = &\frac{1}{2 \alpha_\wts N} \Gammab^\top \Gammab \rbr{\frac{1}{2} \rbr{\Gammab^\top \Gammab \Sigmab_s^{-1} + \Sigmab_s^{-1} \Gammab^\top \Gammab}}^{\dagger} \Gammab^\top \Gammab \\
        \preceq &\frac{1}{4 \alpha_\wts N} \rbr{\Gammab^\top \Gammab \Sigmab_s + \Sigmab_s \Gammab^\top \Gammab}.
    \end{align*}
    Therefore, 
    \begin{align*}
        \E_{\Scal_s}\rbr{\Gammab^\top \Gammab \rbr{\Gammab^\top \Gammab + \alpha_\wts N \Sigmab_s^{-1}}^{-2} \Gammab^\top \Gammab}
        \preceq &\frac{1}{4 \alpha_\wts N} \E_{\Scal_s}\sbr{\Gammab^\top \Gammab \Sigmab_s + \Sigmab_s \Gammab^\top \Gammab} 
        = \frac{1}{2 \alpha_\wts N} \Sigmab_s.
    \end{align*}
    Meanwhile, we observe that
    \begin{align*}
        \E_{\wt\Scal}\sbr{\Vb_w \zetab \zetab^\top \Vb_w^\top} 
        = &\E_{\wt\Scal}\sbr{\Sigmab_w^{1/2} \rbr{\wt\Phib_w^\top \wt\Phib_w + \alpha_w n \Ib_d}^{-1} \wt\Phib_w^\top \wt\zb \wt\zb^\top \wt\Phib_w \rbr{\wt\Phib_w^\top \wt\Phib_w + \alpha_w n \Ib_d}^{-1} \Sigmab_w^{1/2}} \\
        = &\sigma^2 \E_{\wt\Scal}\sbr{\Sigmab_w^{1/2} \rbr{\wt\Phib_w^\top \wt\Phib_w + \alpha_w n \Ib_d}^{-1} \wt\Phib_w^\top \wt\Phib_w \rbr{\wt\Phib_w^\top \wt\Phib_w + \alpha_w n \Ib_d}^{-1} \Sigmab_w^{1/2}},
    \end{align*}
    where 
    \begin{align*}
        \rbr{\wt\Phib_w^\top \wt\Phib_w + \alpha_w n \Ib_d}^{-1} \wt\Phib_w^\top \wt\Phib_w \rbr{\wt\Phib_w^\top \wt\Phib_w + \alpha_w n \Ib_d}^{-1}
        \preceq &\frac{1}{2 \alpha_w n} \Ib_d.
    \end{align*}
    Therefore, we have
    \begin{align*}
        \E_{\wt\Scal}\sbr{\Vb_w \zetab \zetab^\top \Vb_w^\top} 
        \preceq &\sigma^2 \E_{\wt\Scal}\sbr{\Sigmab_w^{1/2} \rbr{\frac{1}{2 \alpha_w n} \Ib_d} \Sigmab_w^{1/2}}
        = \frac{\sigma^2}{2 \alpha_w n} \Sigmab_w.
    \end{align*}
    Overall, the variance of $f_\wts$ can be upper bounded as
    \begin{align}\label{eq:pf_ridge_var_ub}
    \begin{split}
        \vari(f_\wts) 
        = &\tr\rbr{\E_{\Scal_s}\rbr{\Gammab^\top \Gammab \rbr{\Gammab^\top \Gammab + \alpha_\wts N \Sigmab_s^{-1}}^{-2} \Gammab^\top \Gammab} \E_{\wt\Scal}\sbr{\Vb_w \zetab \zetab^\top \Vb_w^\top}} \\
        \le &\frac{\sigma^2 \tr\rbr{\Sigmab_s \Sigmab_w}}{4 (\alpha_w n) (\alpha_\wts N)}.
    \end{split}
    \end{align}

    \paragraph{Bias.}
    Let $\xib = \Sigmab_w^{1/2} \rbr{\wt\Phib_w^\top \wt\Phib_w + \alpha_w n \Ib_d}^{-1} \wt\Phib_w^\top \wt\fb_* \in \R^d$, whose randomness comes from $\wt\Scal$ only, independent of $\Scal_x$.
    Recall from \eqref{eq:pf_ridge_bias}, the bias term \eqref{eq:pf_ridge_bias} can be decomposed as
    \begin{align*}
        &\bias(f_\wts) = \E_{\Scal_x, \wt\Scal}\sbr{\frac{1}{N} \nbr{\Phib_s \rbr{\Phib_s^\top \Phib_s + \alpha_\wts N \Ib_d}^{-1} \Phib_s^\top \Phib_w \rbr{\wt\Phib_w^\top \wt\Phib_w + \alpha_w n \Ib_d}^{-1} \wt\Phib_w^\top \wt\fb_* - \fb_*}_2^2}\\
        &= \E_{\Scal_x, \wt\Scal}\sbr{\frac{1}{N} \rbr{\nbr{\Phib_s \rbr{\Phib_s^\top \Phib_s + \alpha_\wts N \Ib_d}^{-1} \Phib_s^\top \Gammab \xib - \Phib_s \Phib_s^\dagger \fb_*}_2^2 + \nbr{\rbr{\Ib_N - \Phib_s \Phib_s^\dagger} \fb_*}_2^2}},
    \end{align*}
    where by \Cref{lem:low_est_err_ft} and \eqref{eq:pf_ridge_ft_approx_err}
    \begin{align*}
        \E_{\Scal_x}\sbr{\frac{1}{N} \nbr{\rbr{\Ib_N - \Phib_s \Phib_s^\dagger} \fb_*}_2^2}
        = \frac{\rho_s(N)}{N} \le \rho_s = 0.
    \end{align*}
    Therefore, with $\xib = \Sigmab_w^{1/2} \rbr{\wt\Phib_w^\top \wt\Phib_w + \alpha_w n \Ib_d}^{-1} \wt\Phib_w^\top \wt\fb_*$, we have
    \begin{align*}
        \bias(f_\wts) = \E_{\Scal_x, \wt\Scal}\sbr{\frac{1}{N} \nbr{\Phib_s \rbr{\Phib_s^\top \Phib_s + \alpha_\wts N \Ib_d}^{-1} \Phib_s^\top \Gammab \xib - \Phib_s \Phib_s^\dagger \fb_*}_2^2}.
    \end{align*}
    Recall that $\fb_* = \Gammab \Sigmab_*^{1/2} \thetab_*$ and $\Phib_s = \Gammab \Sigmab_s^{1/2} = \Gammab_s \Lambdab_s^{1/2} \Vb_s^\top$.
    Then, we can express the bias term as
    \begin{align*}
        \bias(f_\wts) = &\E_{\Scal_x, \wt\Scal}\sbr{\frac{1}{N} \nbr{\Gammab\rbr{\Gammab^\top \Gammab + \alpha_\wts N \Sigmab_s^{-1}}^{-1} \Gammab^\top \Gammab \xib - \Gammab \Gammab^\dagger \fb_*}_2^2} \\
        = &\E_{\Scal_x, \wt\Scal}\sbr{\frac{1}{N} \nbr{\Gammab \Sigmab_*^{1/2} \thetab_* - \Gammab\rbr{\Gammab^\top \Gammab + \alpha_\wts N \Sigmab_s^{-1}}^{-1} \Gammab^\top \Gammab \xib}_2^2} \\
        = &\E_{\Scal_x, \wt\Scal}\sbr{\frac{1}{N} \nbr{\Gammab \rbr{\Sigmab_*^{1/2} \thetab_* - \xib} + \Gammab \rbr{\Ib_d - \rbr{\Gammab^\top \Gammab + \alpha_\wts N \Sigmab_s^{-1}}^{-1} \Gammab^\top \Gammab} \xib}_2^2} \\
    \end{align*} 
    By Woodbury matrix identity, we have
    \begin{align}\label{eq:pf_ridge_bias_woodbury}
        \Ib_d - \rbr{\Gammab^\top \Gammab + \alpha_\wts N \Sigmab_s^{-1}}^{-1} \Gammab^\top \Gammab
        = \rbr{\Ib_d + \frac{1}{\alpha_\wts N} \Sigmab_s \Gammab^\top \Gammab}^{-1}.
    \end{align}
    Therefore, we have 
    \begin{align}\label{eq:pf_ridge_bias_inter1}
        \bias(f_\wts) = \E_{\Scal_x, \wt\Scal}\Bigg[\frac{1}{N} \Big\|\underbrace{\Gammab \rbr{\Sigmab_*^{1/2} \thetab_* - \xib}}_{\t{Term A}} + \underbrace{\Gammab \rbr{\Ib_d + \frac{1}{\alpha_\wts N} \Sigmab_s \Gammab^\top \Gammab}^{-1} \xib}_{\t{Term B}}\Big\|_2^2\Bigg].
    \end{align}

    For Term A, notice that $\xib = \Sigmab_w^{1/2} \rbr{\wt\Phib_w^\top \wt\Phib_w + \alpha_w n \Ib_d}^{-1} \wt\Phib_w^\top \wt\fb_*$ implies
    \begin{align*}
        \Sigmab_*^{1/2} \thetab_* - \xib 
        = &\Sigmab_*^{1/2} \thetab_* - \Sigmab_w^{1/2} \rbr{\wt\Phib_w^\top \wt\Phib_w + \alpha_w n \Ib_d}^{-1} \wt\Phib_w^\top \wt\fb_* \\
        = &\Sigmab_*^{1/2} \thetab_* - \rbr{\wt\Gammab^\top \wt\Gammab + \alpha_w n \Sigmab_w^{-1}}^{-1} \wt\Gammab^\top \wt\Gammab \Sigmab_*^{1/2} \thetab_* \\
        = &\rbr{\Ib_d - \rbr{\wt\Gammab^\top \wt\Gammab + \alpha_w n \Sigmab_w^{-1}}^{-1} \wt\Gammab^\top \wt\Gammab} \Sigmab_*^{1/2} \thetab_* \\
        = &\rbr{\Ib_d + \frac{1}{\alpha_w n} \Sigmab_w \wt\Gammab^\top \wt\Gammab}^{-1} \Sigmab_*^{1/2} \thetab_*,
    \end{align*}
    where the last equality follows from Woodbury matrix identity as in \eqref{eq:pf_ridge_bias_woodbury}.
    Therefore,
    \begin{align*}
        \E_{\Scal_x, \wt\Scal}\sbr{\frac{1}{N} \nbr{\Gammab \rbr{\Sigmab_*^{1/2} \thetab_* - \xib}}_2^2} 
        = &\E_{\wt\Scal}\sbr{\frac{1}{n} \nbr{\wt\Gammab \rbr{\Sigmab_*^{1/2} \thetab_* - \xib}}_2^2} \\
        = &\E_{\wt\Scal}\sbr{\frac{1}{n} \nbr{\wt\Gammab \rbr{\Ib_d + \frac{1}{\alpha_w n} \Sigmab_w \wt\Gammab^\top \wt\Gammab}^{-1} \Sigmab_*^{1/2} \thetab_*}_2^2}.
    \end{align*}
    Since 
    \begin{align*}
        \rbr{\Ib_d + \frac{1}{\alpha_w n} \Sigmab_w \wt\Gammab^\top \wt\Gammab}^{-1} \wt\Gammab^\top \wt\Gammab \rbr{\Ib_d + \frac{1}{\alpha_w n} \Sigmab_w \wt\Gammab^\top \wt\Gammab}^{-1} \preceq \frac{\alpha_w n}{2} \Sigmab_w^{-1},
    \end{align*}
    we have
    \begin{align}\label{eq:pf_ridge_bias_term1}
    \begin{split}
        \E_{\Scal_x, \wt\Scal}\sbr{\frac{1}{N} \nbr{\Gammab \rbr{\Sigmab_*^{1/2} \thetab_* - \xib}}_2^2} 
        \le &\frac{1}{n} \tr\rbr{\frac{\alpha_w n}{2} \Sigmab_w^{-1} \Sigmab_*^{1/2} \thetab_* \thetab_*^\top \Sigmab_*^{1/2}} \\
        = &\frac{\alpha_w}{2} \nbr{\Sigmab_w^{-1/2} \Sigmab_*^{1/2} \thetab_*}_2^2.
    \end{split}
    \end{align}
    
    For Term B, leveraging Woodbury matrix identity as in \eqref{eq:pf_ridge_bias_woodbury}, we notice that 
    \begin{align*}
        &\E_{\Scal_x, \wt\Scal}\sbr{\frac{1}{N} \nbr{\Gammab \rbr{\Ib_d + \frac{1}{\alpha_\wts N} \Sigmab_s \Gammab^\top \Gammab}^{-1} \xib}_2^2} 
        \le \E_{\Scal_x, \wt\Scal}\sbr{\frac{1}{N} \tr\rbr{\frac{\alpha_\wts N}{2} \Sigmab_s^{-1} \xib \xib^\top}} \\
        = &\frac{\alpha_\wts}{2} \E_{\Scal_x, \wt\Scal}\sbr{\nbr{\Sigmab_s^{-1/2} \Sigmab_w^{1/2} \rbr{\wt\Phib_w^\top \wt\Phib_w + \alpha_w n \Ib_d}^{-1} \wt\Phib_w^\top \wt\fb_*}_2^2} \\
        = &\frac{\alpha_\wts}{2} \E_{\Scal_x, \wt\Scal}\sbr{\nbr{\Sigmab_s^{-1/2} \rbr{\wt\Gammab^\top \wt\Gammab + \alpha_w n \Sigmab_w^{-1}}^{-1} \wt\Gammab^\top \wt\Gammab \Sigmab_*^{1/2} \thetab_*}_2^2}
    \end{align*}
    Since $\rbr{\wt\Gammab^\top \wt\Gammab + \alpha_w n \Sigmab_w^{-1}}^{-1} \wt\Gammab^\top \wt\Gammab \preceq \Ib_d$, we know that
    \begin{align}\label{eq:pf_ridge_bias_term2}
    \begin{split}
        \E_{\Scal_x, \wt\Scal}\sbr{\frac{1}{N} \nbr{\Gammab \rbr{\Ib_d + \frac{1}{\alpha_\wts N} \Sigmab_s \Gammab^\top \Gammab}^{-1} \xib}_2^2} 
        \le \frac{\alpha_\wts}{2} \nbr{\Sigmab_s^{-1/2} \Sigmab_*^{1/2} \thetab_*}_2^2.
    \end{split}
    \end{align}
    Combining \eqref{eq:pf_ridge_bias_inter1}, \eqref{eq:pf_ridge_bias_term1}, and \eqref{eq:pf_ridge_bias_term2}, we can upper bound the bias term as
    \begin{align}\label{eq:pf_ridge_bias_final}
    \begin{split}
        &\bias(f_\wts) = \E_{\Scal_x, \wt\Scal}\Bigg[\frac{1}{N} \Big\|\underbrace{\Gammab \rbr{\Sigmab_*^{1/2} \thetab_* - \xib}}_{\t{Term A}} + \underbrace{\Gammab \rbr{\Ib_d + \frac{1}{\alpha_\wts N} \Sigmab_s \Gammab^\top \Gammab}^{-1} \xib}_{\t{Term B}}\Big\|_2^2\Bigg] \\
        \le &2 \E_{\Scal_x, \wt\Scal}\sbr{\frac{1}{N} \nbr{\Gammab \rbr{\Sigmab_*^{1/2} \thetab_* - \xib}}_2^2} + 2 \E_{\Scal_x, \wt\Scal}\sbr{\frac{1}{N} \nbr{\Gammab \rbr{\Ib_d + \frac{1}{\alpha_\wts N} \Sigmab_s \Gammab^\top \Gammab}^{-1} \xib}_2^2} \\
        \le &\alpha_w \nbr{\Sigmab_w^{-1/2} \Sigmab_*^{1/2} \thetab_*}_2^2 + \alpha_\wts \nbr{\Sigmab_s^{-1/2} \Sigmab_*^{1/2} \thetab_*}_2^2.
    \end{split}
    \end{align}
    
    \paragraph{Variance-bias tradeoff.}
    Overall, by \eqref{eq:pf_ridge_var_ub} and \eqref{eq:pf_ridge_bias_final}, we have
    \begin{align*}
        &\vari(f_\wts) \le \frac{\sigma^2 \tr\rbr{\Sigmab_s \Sigmab_w}}{4 (\alpha_w n) (\alpha_\wts N)}, \\
        &\bias(f_\wts) \le \alpha_w \nbr{\Sigmab_w^{-1/2} \Sigmab_*^{1/2} \thetab_*}_2^2 + \alpha_\wts \nbr{\Sigmab_s^{-1/2} \Sigmab_*^{1/2} \thetab_*}_2^2.
    \end{align*}
    The upper bound the excess risk $\exrisk(f_\wts) = \vari(f_\wts) + \bias(f_\wts)$ is minimized by taking 
    \begin{align*}
        \alpha_w = \rbr{\frac{\sigma^2 \tr\rbr{\Sigmab_s \Sigmab_w}}{4 n N}\ \frac{\nbr{\Sigmab_s^{-1/2} \Sigmab_*^{1/2} \thetab_*}_2^2}{\nbr{\Sigmab_w^{-1/2} \Sigmab_*^{1/2} \thetab_*}_2^4}}^{1/3}, \ 
        \alpha_\wts = \rbr{\frac{\sigma^2 \tr\rbr{\Sigmab_s \Sigmab_w}}{4 n N}\ \frac{\nbr{\Sigmab_w^{-1/2} \Sigmab_*^{1/2} \thetab_*}_2^2}{\nbr{\Sigmab_s^{-1/2} \Sigmab_*^{1/2} \thetab_*}_2^4}}^{1/3},
    \end{align*}
    which leads to the optimal upper bound for the excess risk:
    \begin{align*}
        \exrisk(f_\wts) \le 3 \rbr{\frac{\sigma^2 \tr\rbr{\Sigmab_s \Sigmab_w}}{4 n N}\ \nbr{\Sigmab_s^{-1/2} \Sigmab_*^{1/2} \thetab_*}_2^2 \nbr{\Sigmab_w^{-1/2} \Sigmab_*^{1/2} \thetab_*}_2^2}^{1/3}.
    \end{align*}
\end{proof}






\section{Canonical angles}\label{apx:canonical_angles}
In this section, we review the concept of canonical angles between two subspaces that connect the formal definition of the correlation dimension $d_{s \wedge w} = \nbr{\Vb_s^\top \Vb_w}_F^2$ in \Cref{def:correlation_dim} to the intuitive notion of the alignment between the corresponding subspaces $\Vcal_s$ and $\Vcal_w$ in the introduction: $\sum \cos(\angle(\Vcal_s, \Vcal_w)) = \nbr{\Vb_s^\top \Vb_w}_F^2$.
\begin{definition}[Canonical angles \cite{golub2013matrix}, adapting from \cite{dong2024efficient}]\label{def:canonical_angles}
    Let $\Vcal_s,\Vcal_w \subseteq \R^d$ be two subspaces with dimensions $\dim\rbr{\Vcal_s}=d_s$ and $\dim\rbr{\Vcal_w}=d_w$ (assuming $d_w \geq d_s$ without loss of generality). The canonical angles $\angle\rbr{\Vcal_s,\Vcal_w}=\diag\rbr{\nu_1,\dots,\nu_{d_s}}$ are $d_s$ angles that jointly measure the alignment between $\Vcal_s$ and $\Vcal_w$, defined recursively as follows:
    \begin{align*}
        &\ub_i, \vb_i ~\triangleq~
        \argmax~\ub_i^*\vb_i \\
        \t{s.t.}~
        &\ub_i \in \rbr{\Vcal_s \setminus \spn\cbr{\ub_{\iota}}_{\iota=1}^{i-1}} \cap \SSS^{d-1},\\ 
        &\vb_i \in \rbr{\Vcal_w \setminus \spn\cbr{\vb_{\iota}}_{\iota=1}^{i-1}} \cap \SSS^{d-1}\\
        &\cos (\nu_i) = \ub_i^* \vb_i \quad \forall~ i=1,\dots,k,
    \end{align*}
    such that $0 \leq \nu_1 \leq \dots \leq \nu_k \leq \pi/2$.

    Given two subspaces $\Vcal_s,\Vcal_w \subseteq \R^d$, let $\Vb_s \in \R^{d \times d_s}$ and $\Vb_w \in \R^{d \times d_w}$ be the matrices whose columns form orthonormal bases for $\Vcal_s$ and $\Vcal_w$, respectively. Then, the canonical angles $\angle(\Vcal_s, \Vcal_w)$ are determined by the singular values of $\Vb_s^\top \Vb_w$~\citep[\S 3]{bjorck1973numerical}:
    \begin{align*}
        \cos(\angle_i(\Vcal_s, \Vcal_w)) = \sigma_i(\Vb_s^\top \Vb_w) \quad \forall~ i=1,\dots,d_s,
    \end{align*}
    where $\sigma_i(\Vb_s^\top \Vb_w)$ denotes the $i$-th singular value of $\Vb_s^\top \Vb_w$.
\end{definition}

In particular, since $\Vb_s, \Vb_w$ consist of orthonormal columns, the singular values of $\Vb_s^\top \Vb_w$ fall in $[0,1]$, and therefore,
\begin{align*}
    d_{s \wedge w} = \sum \cos(\angle(\Vcal_s, \Vcal_w)) = \nbr{\Vb_s^\top \Vb_w}_F^2 \in [0, \min\cbr{d_s, d_w}].
\end{align*}




\section{Additional experiments}\label{apx:exp_details}

\subsection{Additional experiments and details on UTKFace regression}\label{apx:exp_img_reg}
This section provides some additional details and results for the UTKFace regression experiments in \Cref{sec:exp_img_reg}. 

\begin{figure}[!h]
    \centering
    \includegraphics[width=\columnwidth]{fig/mse_utkface_resnet18_clipb32.pdf}%\vspace{-2em}
    \caption{Scaling for MSE on UTKFace with \texttt{CLIP-B32} as the strong student and \texttt{ResNet18} as the weak teacher}\label{fig:mse_utkface_resnet18-clip}
\end{figure}

\begin{figure}[!h]
    \centering
    \includegraphics[width=\columnwidth]{fig/mse_utkface_resnet50_clipb32.pdf}%\vspace{-2em}
    \caption{Scaling for MSE on UTKFace with \texttt{CLIP-B32} as the strong student and \texttt{ResNet50} as the weak teacher}\label{fig:mse_utkface_resnet50-clip}
\end{figure}

\begin{figure}[!h]
    \centering
    \includegraphics[width=\columnwidth]{fig/mse_utkface_resnet152_clipb32.pdf}%\vspace{-2em}
    \caption{Scaling for MSE on UTKFace with \texttt{CLIP-B32} as the strong student and \texttt{ResNet152} as the weak teacher}\label{fig:mse_utkface_resnet152-clip}
\end{figure}

We summarize the relevant dimensionality in \Cref{tab:img_reg_dim}. We observe the following:
\begin{itemize}
    \item The intrinsic dimensions of the pretrained features are significantly smaller than the ambiance feature dimensions, which is consistent with our theoretical analysis and the empirical observations in \cite{aghajanyan2020intrinsic}. 
    \item The correlation dimensions $d_{s \wedge w}$ are considerably smaller than the corresponding intrinsic dimensions, indicating that the subspaces spanned by the weak and strong features are not aligned in practice. As shown in \Cref{sec:exp_img_reg}, such discrepancies in the weak and strong features facilitate W2S generalization.
\end{itemize}

\begin{table}[!ht]
    \centering
    \caption{Summary of the pretrained feature dimensions, along with the intrinsic dimensions $d_s, d_w$ and correlation dimensions $d_{s \wedge w}$ (with respect to the strong student \texttt{CLIP-B32}) computed over the entire UTKFace dataset (including training and testing).}\label{tab:img_reg_dim}
    \begin{tabular}{c|ccc}
        \toprule
        Pretrained Model & Feature Dimension & Intrinsic Dimension ($\tau=0.01$) & Correlation Dimension \\
        \midrule
        \texttt{ResNet18} & 512 & 194 & 167.64 \\
        \texttt{ResNet34} & 512 & 150 & 129.97 \\
        \texttt{ResNet50} & 2048 & 522 & 301.06 \\
        \texttt{ResNet101} & 2048 & 615 & 354.52 \\
        \texttt{ResNet152} & 2048 & 589 & 339.90 \\
        \midrule
        \texttt{CLIP-B32} & 768 & 443 & $\times$ \\
        \bottomrule
    \end{tabular}
\end{table}

For reference, we provide the scaling for MSE losses of three representative teacher-student pairs in \Cref{fig:mse_utkface_resnet18-clip,fig:mse_utkface_resnet50-clip,fig:mse_utkface_resnet152-clip}. 
\begin{itemize}
    \item It is worth highlighting that while the MSE loss of $f_\wts$ monotonically decreases with respect to both sample sizes $n,N$, the different rates of convergence compared to $f_w$, $f_s$, and $f_c$ lead to the distinct scaling behavior of the relative W2S performance ($\pgr$ and $\opr$) with respect to $n$ versus $N$ in \Cref{fig:pgr_opr_utkface_resnet-clip,fig:pgr_opr_utkface_vardom_resnet-clip}.
    \item When the strong student has a lower intrinsic dimension than the weak teacher (\cf \Cref{fig:mse_utkface_resnet18-clip} versus \Cref{fig:mse_utkface_resnet50-clip,fig:mse_utkface_resnet152-clip}), $d_s < d_w$, the W2S model $f_\wts$ tends to achieve better generalization in terms of the test MSE. This is consistent with our analysis in \Cref{sec:generalization_errors}.
    \item When $d_s < d_w$, the W2S model $f_\wts$ tends to achieve (slightly) better generalization for (slightly) smaller correlation dimension $d_{s \wedge w}$ (\cf \Cref{fig:mse_utkface_resnet50-clip} versus \Cref{fig:mse_utkface_resnet152-clip}), again coinciding with our analysis in \Cref{sec:generalization_errors}.
    \item W2S generalization generally happens (\ie $f_\wts$ is able to outperform $f_w$) with sufficiently large sample sizes $n, N$. However, as the labeled sample size $n$ increases, the test MSE of $f_\wts$ converges slower than that of the strong baseline and ceiling models, $f_s$ and $f_c$, leading to the inverse scaling for $\pgr$ and $\opr$ with respect to $n$ in \Cref{fig:pgr_opr_utkface_resnet-clip,fig:pgr_opr_utkface_vardom_resnet-clip}. When $n$ is too large, the W2S model $f_\wts$ may not be able to achieve better generalization than the strong baseline $f_s$.
\end{itemize}




\subsection{Experiments on image classification}\label{apx:exp_img_cls}

\paragraph{Dataset.} ColoredMNIST \citep{arjovsky2019invariant} consists of groups of different colors and reassign the label to be binary (digits 0-4 vs 5-9). We pool together the groups to form one dataset. The choice is to bring diversity to the feature space with additional color features and thus potential feature discrepancies. We hold out a test set of 7000 samples and used the rest 63000 samples for training.

\paragraph{Linear probing over pretrained features.} We fix a strong student as DINOv2-s14 \citep{oquab2023dinov2} and vary the weak teacher among the ResNet-d series and ResNet series (ResNet18D, ResNet34D, ResNet101, ResNet152) \citep{he2018resnetd,he2015deepresiduallearningimage}. We replace ResNet18 and ResNet34 used in \Cref{sec:exp_img_reg} to experiment on weak models with similar intrinsic dimensions but different correlation dimensions. We treat the backbone of the models (excluding the classification layer) as $\phi_s$ and $\phi_w$ and finetune them via linear probing. We train the models with cross entropy loss and AdamW optimizer. We tune the training hyperparameters of weak and strong models using a validation set and train them for 800 steps with learning rate 1e-3 and weight decay 1e-6. 

\begin{table}[!ht]
    \centering
    \caption{Summary of the pretrained feature dimensions, along with the intrinsic dimensions $d_s, d_w$ and correlation dimensions $d_{s \wedge w}$ (with respect to the strong student \texttt{DINOv2-S14}) computed over the entire ColoredMNIST dataset (including training and testing).}\label{tab:img_cls_dim_coloredmnist}
    \begin{tabular}{c|ccc}
        \toprule
        Pretrained Model & Feature Dimension & Intrinsic Dimension ($\tau=0.01$) & Correlation Dimension \\
        \midrule
        \texttt{ResNet-18-D} & 512 & 117 & 6.23 \\
        \texttt{ResNet-34-D} & 512 & 127 & 7.07 \\
        \texttt{ResNet101} & 2048 & 121 & 1.74 \\
        \texttt{ResNet152} & 2048 & 128 & 1.88 \\
        \midrule
        \texttt{DINOv2-S14} & 384 & 28 & $\times$ \\
        \bottomrule
    \end{tabular}
\end{table}

\begin{figure}[!h]
    \centering
    \includegraphics[width=\columnwidth]{fig/coloredmnist_lp/coloredmnist_dsw.pdf}%\vspace{-2em}
    \caption{Scaling for $\pgr$ and $\opr$ of different weak teachers with a fixed strong student on ColoredMNIST.}\label{fig:coloredmnist_dscapw}
\end{figure}

\begin{figure}[!h]
    \centering
    \includegraphics[width=\columnwidth]{fig/coloredmnist_lp/coloredmnist_var.pdf}%\vspace{-2em}
    \caption{Scaling for $\pgr$ and $\opr$ of W2S on ColoredMNIST with injected label noise.}\label{fig:coloredmnist_variance}
\end{figure}

\paragraph{Discrepancies lead to better W2S.}
\Cref{fig:coloredmnist_dscapw} shows the scaling of $\pgr$ and $\opr$ with respect to the sample sizes $n, N$ for different weak teachers in the ResNet series with respect to a fixed student, \texttt{CLIP-B32}. 
As in \Cref{sec:exp_img_reg}, we observe that with similar intrinsic dimensions $d_s, d_w$, W2S achieves better relative performance in terms of $\pgr$ and $\opr$ when the correlation dimension $d_{s \wedge w}$ is smaller.

\paragraph{Variance reduction is a key advantage of W2S.}
We inject noise to the labels of the original ColoredMNIST training samples by randomly flipping the ground truth labels with probability $\varsigma \in [0,1]$ (following \cite{arjovsky2019invariant}). 
\Cref{fig:coloredmnist_variance} shows the scaling of $\pgr$ and $\opr$ with respect to $n$ and $N$ when taking DINOv2-S14 as the strong student and ResNet101 as the weak teacher. We observe that the larger artificial label noise $\varsigma$ leads to higher $\pgr$ and $\opr$. 

\subsection{Experiments on sentiment classification}\label{apx:exp_nlp_cls}

\paragraph{Dataset.} The Stanford Sentiment Treebank \citep{socher-etal-2013-sst2} is a corpus with fully labeled parse trees that allows for a complete analysis of the compositional effects of sentiment in language. The corpus is based on the dataset introduced by \citet{pang-lee-2005-sst_original_corpus} and consists of 11,855 single sentences extracted from movie reviews. It was parsed with the Stanford parser and includes a total of 215,154 unique phrases from those parse trees, each annotated by 3 human judges. We conduct binary classification experiments on full sentences (negative or somewhat negative vs somewhat positive or positive with neutral sentences discarded), the so-called SST-2 dataset, and split the dataset into training and testing sets of sizes 63000 and 4349. Generalization errors are estimated with the 0-1 loss over the test set.

\paragraph{Full finetuning.} We fix the strong student as Electra-base-discriminator \citep{clark2020electra} and vary the weak teacher among the Bert series \citep{turc2019bert-tiny} (Bert-Tiny, Bert-Mini, Bert-Small, Bert-Medium). 
With manageable model sizes, we conduct full finetuning experiments following the setup in \cite{burns2023weak}.
We use the standard cross entropy loss for supervised finetuning. 
When training strong students on weak labels (W2S), we use the confidence weighted loss proposed by \cite{burns2023weak}, which is suggested to be able to improve weak-to-strong generalization on many NLP tasks.
All training is conducted via Adam optimizers~\citep{kingma2014adam} with a learning rate of 5e-5, a cosine learning rate schedule, and 40 warmup steps. We train for 3 epochs, which is sufficient for the train and validation loss to stabilize. 

\paragraph{Intrinsic dimension.} The intrinsic dimensions $d_w,d_s$ are measured based on the Structure-Aware Intrinsic Dimension (SAID) method proposed by \cite{aghajanyan2020intrinsic}. We first train the full models on the whole training set, and then train the models with only $d$ trainable parameters based on SAID transformation. The $d_w$ or $d_s$ are the smallest number of parameters under SAID that is necessary to retain 90\% accuracy of the full models. Here, the 90\% accuracy is a common threshold used to estimate intrinsic dimensions in the literature \citep{li2018measuring}.

\begin{figure}[!h]
    \centering
    \includegraphics[width=\columnwidth]{fig/sst2/sst2-dsw.pdf}%\vspace{-2em}
    \caption{Scaling for $\pgr$ and $\opr$ of different weak teachers with a fixed strong student on SST-2.}\label{fig:sst2_dsw}
\end{figure}

\begin{figure}[!h]
    \centering
    \includegraphics[width=\columnwidth]{fig/sst2/sst2-var.pdf}%\vspace{-2em}
    \caption{Scaling for $\pgr$ and $\opr$ of W2S on SST-2 with injected label noise.}\label{fig:sst2_var}
\end{figure}

\paragraph{Correlation Dimension.} 
Let $D_s, D_w \in \N$ be the finetunable parameter counts of the strong and weak models, respectively. For full FT whose dynamics fall in the kernel regime, as explained in \Cref{rmk:lp_to_general_ft}, the strong and weak ``features'' become the gradients\footnote{
    Notice that $f_s, f_w$ are scalar-valued functions for binary classification tasks like SST-2, and thus the gradients $\nabla_{\thetab} f_s$ and $\nabla_{\thetab} f_w$ are row vectors.
    For multi-class classification tasks where $f_s, f_w$ output vectors of logits, a common heuristic to keep $\Phib_s, \Phib_w$ as matrices of manageable sizes (in constrast to tensors) is to replace gradients of the models, $\nabla_{\thetab} f_s$ and $\nabla_{\thetab} f_w$, with gradients of MSE losses at the pretrained initialization. 
    The gradients of MSE can be viewed as a weighted sum of the model gradients for each class.
}, $\Phib_s = \nabla_{\thetab} f_s(\Xb | \theta_s^{(0)}) \in \R^{N \times D_s}$ and $\Phib_w = \nabla_{\thetab} f_w(\Xb | \theta_w^{(0)}) \in \R^{N \times D_w}$, of the respective models at the pretrained initialization, $\theta_s^{(0)} \in \R^{D_s}$ and $\theta_w^{(0)} \in \R^{D_w}$.

A practical challenge is that $D_s, D_w, N$ are all huge for full FT on most NLP tasks, making it infeasible to compute the $D_s \times D_s$ and $D_w \times D_w$ Gram matrices and their spectral decompositions. 
As a remedy, we leverage the significantly lower intrinsic dimensions $d_s \ll D_s, d_w \ll D_w$ (see \Cref{tab:img_cls_dim_coloredmnist}) to accelerate estimation of $d_{s \wedge w}$ via sketching~\citep{halko2011finding,woodruff2014sketching}.
\begin{enumerate}[label=(\roman*)]
    \item We first reduce both $D_s, D_w$ to the same lower dimension $D = 0.01 \min\{D_s, D_w\}$ (with $D \gg \max\{d_s, d_w\}$) by uniform subsampling columns of $\Phib_s, \Phib_w$ to obtain $\Phib_s', \Phib_w' \in \R^{N \times D}$.
    \item Then, we use randomized rangefinder~\citep[Algorithm 4.1]{halko2011finding} to approximate the first $d_s, d_w$ right singular vectors, $\Vb_s \in \R^{D \times d_s}$ and $\Vb_w \in \R^{D \times d_w}$, of $\Phib_s', \Phib_w'$. Taking the evaluation of $\Vb_s$ as an example, we draw a Gaussian random matrix $\Gb_s \in \R^{d_s \times D}$ and compute the orthornormalization $\Vb_s = \ortho(\Phib_s'^\top \Gb_s)$ via QR decomposition.
    \item Finally, we compute the correlation dimension $d_{s \wedge w} = \nbr{\Vb_s^\top \Vb_w}_F^2$.
\end{enumerate}

\begin{table}[!ht]
    \centering
    \caption{Summary of finetunable parameter counts $D_s, D_w$, intrinsic dimensions $d_s, d_w$, and correlation dimensions $d_{s \wedge w}$ (with respect to the strong student \texttt{Electra}) computed over the entire SST-2 dataset (including training and testing).}\label{tab:sst2_dim}
    \begin{tabular}{c|ccc}
        \toprule
        Pretrained Model & $D_s,D_w$ & Intrinsic Dimension ($\tau=0.01$) & Correlation Dimension \\
        \midrule
        \texttt{Bert-Tiny} & 4.4M & 7000 & 81.13 \\
        \texttt{Bert-Mini} & 11.2M & 8500 & 38.67 \\
        \texttt{Bert-Small} & 28.8M & 8000 & 26.19 \\
        \texttt{Bert-Medium} & 41.4M & 4000 & 8.52 \\
        \midrule
        \texttt{Electra} & 109.5M & 700 & $\times$ \\
        \bottomrule
    \end{tabular}
\end{table}

\paragraph{Discrepancies lead to better W2S.}
\Cref{fig:sst2_dsw} shows the scaling of $\pgr$ and $\opr$ with respect to $n$ and $N$ for different $d_{s \wedge w}$. 
As in \Cref{sec:exp_img_reg,apx:exp_img_cls}, we observe the better relative W2S performance in terms of $\pgr$ and $\opr$ when $d_{s \wedge w}/d_w$ is smaller.

\paragraph{Variance reduction is a key advantage of W2S.}
We inject noise to the labels of training samples by randomly flipping labels with probability $\varsigma = 0, 0.1, 0.2, 0.3$. 
\Cref{fig:sst2_var} shows the scaling of $\pgr$ and $\opr$ with respect to $n$ and $N$ when taking \texttt{Electra} as the strong student and \texttt{Bert-Medium} as the weak teacher. We observe that the larger artificial label noise $\varsigma$ leads to higher $\pgr$ and $\opr$. 
    
\end{document}
