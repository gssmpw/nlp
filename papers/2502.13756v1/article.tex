%Version 3 October 2023
% See section 11 of the User Manual for version history
%
%%%%%%%%%%%%%%%%%%%%%%%%%%%%%%%%%%%%%%%%%%%%%%%%%%%%%%%%%%%%%%%%%%%%%%
%%                                                                 %%
%% Please do not use \input{...} to include other tex files.       %%
%% Submit your LaTeX manuscript as one .tex document.              %%
%%                                                                 %%
%% All additional figures and files should be attached             %%
%% separately and not embedded in the \TeX\ document itself.       %%
%%                                                                 %%
%%%%%%%%%%%%%%%%%%%%%%%%%%%%%%%%%%%%%%%%%%%%%%%%%%%%%%%%%%%%%%%%%%%%%

%%\documentclass[referee,sn-basic]{sn-jnl}% referee option is meant for double line spacing

%%=======================================================%%
%% to print line numbers in the margin use lineno option %%
%%=======================================================%%

%%\documentclass[lineno,sn-basic]{sn-jnl}% Basic Springer Nature Reference Style/Chemistry Reference Style

%%======================================================%%
%% to compile with pdflatex/xelatex use pdflatex option %%
%%======================================================%%

\documentclass[sn-mathphys-num]{sn-jnl}% Basic Springer Nature Reference Style/Chemistry Reference Style


%%Note: the following reference styles support Namedate and Numbered referencing. By default the style follows the most common style. To switch between the options you can add or remove �Numbered� in the optional parenthesis. 
%%The option is available for: sn-basic.bst, sn-vancouver.bst, sn-chicago.bst%  
%%\documentclass[sn-nature]{sn-jnl}% Style for submissions to Nature Portfolio journals
%%\documentclass[sn-basic]{sn-jnl}% Basic Springer Nature Reference Style/Chemistry Reference Style
% \documentclass[sn-mathphys-num]{sn-jnl}% Math and Physical Sciences Numbered Reference Style 
%%\documentclass[sn-mathphys-ay]{sn-jnl}% Math and Physical Sciences Author Year Reference Style
%%\documentclass[sn-aps]{sn-jnl}% American Physical Society (APS) Reference Style
%%\documentclass[sn-vancouver,Numbered]{sn-jnl}% Vancouver Reference Style
%%\documentclass[sn-apa]{sn-jnl}% APA Reference Style 
%%\documentclass[sn-chicago]{sn-jnl}% Chicago-based Humanities Reference Style

%%%% Standard Packages
%%<additional latex packages if required can be included here>

\usepackage{graphicx}%
\usepackage{multirow}%
\usepackage{amsmath,amssymb,amsfonts}%
\usepackage{amsthm}%
\usepackage{mathrsfs}%
\usepackage[title]{appendix}%
\usepackage{xcolor}%
\usepackage{textcomp}%
\usepackage{manyfoot}%
\usepackage{booktabs}%
\usepackage{algorithm}%
\usepackage{algorithmicx}%
\usepackage{algpseudocode}%
\usepackage{listings}%
%%%%

%%%%%=============================================================================%%%%
%%%%  Remarks: This template is provided to aid authors with the preparation
%%%%  of original research articles intended for submission to journals published 
%%%%  by Springer Nature. The guidance has been prepared in partnership with 
%%%%  production teams to conform to Springer Nature technical requirements. 
%%%%  Editorial and presentation requirements differ among journal portfolios and 
%%%%  research disciplines. You may find sections in this template are irrelevant 
%%%%  to your work and are empowered to omit any such section if allowed by the 
%%%%  journal you intend to submit to. The submission guidelines and policies 
%%%%  of the journal take precedence. A detailed User Manual is available in the 
%%%%  template package for technical guidance.
%%%%%=============================================================================%%%%

%% as per the requirement new theorem styles can be included as shown below
\theoremstyle{thmstyleone}%
\newtheorem{theorem}{Theorem}%  meant for continuous numbers
%%\newtheorem{theorem}{Theorem}[section]% meant for sectionwise numbers
%% optional argument [theorem] produces theorem numbering sequence instead of independent numbers for Proposition
\newtheorem{proposition}[theorem]{Proposition}% 
\newtheorem{corollary}[theorem]{Corollary}% 
%%\newtheorem{proposition}{Proposition}% to get separate numbers for theorem and proposition etc.

\theoremstyle{thmstyletwo}%
\newtheorem{example}{Example}%
\newtheorem{remark}{Remark}%

\theoremstyle{thmstylethree}%
\newtheorem{definition}{Definition}%

\raggedbottom
%%\unnumbered% uncomment this for unnumbered level heads

%%%%%=============================================================================%%%%
%%%%  The following are our packages
%%%%%=============================================================================%%%%

\usepackage[draft]{fixme}
\usepackage{amsmath}
\usepackage{amssymb}
\usepackage{xcolor}
\usepackage[english]{babel}
\usepackage[latin1]{inputenc}
\usepackage{dsfont}
\usepackage{tikz}
\usetikzlibrary{backgrounds}
\usetikzlibrary{patterns}
\usepackage{tkz-berge}
% \usepackage{enumerate}
\usepackage{enumitem}
\usepackage{multicol}
\usepackage{paracol}

\usepackage{cleveref}
\usepackage{stmaryrd}

\usepackage{localdefs}

% conditional for categories.
\newif\ifcategories

% to show the categorical part just uncomment the following line
%\categoriestrue
% to hide the categorical part just uncomment the following line
\categoriesfalse

\begin{document}

\title[Deontic Action Logics. A Modular Algebraic Perspective]{\centering
  Deontic Action Logics: \\ A Modular Algebraic Perspective}
	% \begin{tabular}{c}Deontic Action Logics\\ {\large A Modular Algebraic Perspective}\end{tabular}}

%%=============================================================%%
%% GivenName	-> \fnm{Joergen W.}
%% Particle	-> \spfx{van der} -> surname prefix
%% FamilyName	-> \sur{Ploeg}
%% Suffix	-> \sfx{IV}
%% \author*[1,2]{\fnm{Joergen W.} \spfx{van der} \sur{Ploeg} 
%%  \sfx{IV}}\email{iauthor@gmail.com}
%%=============================================================%%

\author*[1,2]{\fnm{Carlos} \sur{Areces}}\email{carlos.areces@unc.edu.ar}
\author*[1,3]{\fnm{Valentin} \sur{Cassano}}\email{valentin@dc.exa.unrc.edu.ar}
\author*[1,3]{\fnm{Pablo} \sur{Castro}}\email{pcastro@dc.exa.unrc.edu.ar}
\author*[1,2]{\fnm{Raul} \sur{Fervari}}\email{rfervari@unc.edu.ar}
% \equalcont{These authors contributed equally to this work.}

\affil[1]{
  % \orgdiv{Department},
  \orgname{Consejo Nacional de Investigaciones Cient\'ificas y T\'ecnicas},
  \orgaddress{
    % \street{Street},
    % \city{C\'ordoba},
    % \postcode{5000},
    % \state{C\'ordoba},
    \country{Argentina}}}

\affil[2]{
  % \orgdiv{Department},
  \orgname{Universidad Nacional de C\'ordoba},
  \orgaddress{
    % \street{Street},
    % \city{C\'ordoba},
    % \postcode{5000},
    % \state{C\'ordoba},
    \country{Argentina}}}
  
\affil[3]{
  % \orgdiv{Department},
  \orgname{Universidad Nacional de R\'io Cuarto},
  \orgaddress{
    % \street{Street},
    % \city{C\'ordoba},
    % \postcode{5000},
    % \state{C\'ordoba},
    \country{Argentina}}}

%%==================================%%
%%            abstract              %%
%%==================================%%

\begin{abstract}
Retrieval-Augmented Generation (RAG) is often used with Large Language Models (LLMs) to infuse domain knowledge or user-specific information. In RAG, given a user query, a retriever extracts chunks of relevant text from a knowledge base. These chunks are sent to an LLM as part of the input prompt. Typically, any given chunk is repeatedly retrieved across user questions. However, currently, for every question, attention-layers in LLMs fully compute the key values (KVs) repeatedly for the input chunks, as state-of-the-art methods cannot reuse KV-caches when chunks appear at arbitrary locations with arbitrary contexts. Naive reuse leads to output quality degradation.  This leads to potentially redundant computations on expensive GPUs and increases latency. In this work, we propose \sys, a system for managing and reusing precomputed KVs corresponding to the text chunks (we call \textit{chunk-caches}) in RAG-based systems. We present how to identify \hl{\textit{chunk-caches} that are reusable}, how to efficiently perform a small fraction of recomputation to \textit{fix} the cache to maintain output quality, and how to efficiently store and evict \textit{chunk-caches} in the hardware for maximizing reuse while masking any overheads. With real production workloads as well as synthetic datasets, we show that \sys reduces redundant computation by \textbf{51\%} over SOTA prefix-caching and \textbf{75\%} over full recomputation.
\hl{Additionally, with continuous batching on a real production workload, we get a \textbf{1.6$\times$} speedup in throughput and a \textbf{2$\times$} reduction in end-to-end response latency over prefix-caching while maintaining quality, for both the \llama-3-8B and \llama-3-70B models. 
}
\end{abstract}






\keywords{Deontic Action Logic, Algebraic Logic, Normative Reasoning.}

%%\pacs[JEL Classification]{D8, H51}
%%\pacs[MSC Classification]{35A01, 65L10, 65L12, 65L20, 65L70}

\maketitle

%%%%%  THE BODY OF THE PAPER

%\tableofcontents

\documentclass[../main.tex]{subfiles}
\graphicspath{{../images/}}
\makeatletter
\def\input@path{{../images/}}
\makeatother
\begin{document}
\section{Introduction}
\begin{figure}
\centering
\begin{tikzpicture}
\node[inner sep=0pt] (ws) at (0, 0) {
\includegraphics[height=.4\textwidth, trim={10cm 0 10cm 0},clip]{world_space.png}};
\node[inner sep=0pt] (cs) at (6,0) {\includegraphics[height=.4\textwidth, trim={10cm 1cm 10cm 4cm},clip]{conf_space.png}};
\end{tikzpicture}
\vspace{-5pt}
\label{fig:pbrm_intro}
\caption{\textbf{Left}: Shows world space obstacles as grey spheres. Robots start and goal configuration is colored red and green, respectively. Configurations along the computed path are colored transparent blue. \textbf{Right:} Mapped world space scenario to configuration space. Obstacle region is the grey mesh. Red spheres are collision-free regions computed by the neural SCDF. The optimized shortest path in the convex corridor is the blue curve.}
\vspace{-25pt}
\end{figure}
Motion planning is the problem of finding a collision-free trajectory that connects a given start and goal configuration. The planning takes place in the configuration space of the robot. For single body robots, like mobile robots or drones, the configuration space and the world space are usually the same. This simplifies the planning, since explicit obstacle representations are available which enables geometrical tools like separating hyperplanes, smallest distance to obstacles etc., to be used when designing motion planning algorithms. For multi-body robots like manipulators, the situation is completely different. The world space obstacles are usually mapped to non-convex regions, and to make the problem even harder, the mapping is usually not known. Forming explicit representations of the obstacle region in the configuration space is usually too expensive or intractable. Despite all of this, sampling based planners are used with great success, which mainly is due to their use of implicit representations of the obstacle region. The basic idea is to construct a graph in the configuration space that covers and connects the collision-free region. From this graph, a path can be extracted that connects a given start and goal configuration. The approach is computationally expensive, since the graph is constructed with the smallest geometrical building block available, points, which represents a collision-check. Furthermore, the extracted paths from the graph are non-smooth and jagged due to the stochastic nature of the approach. This adds an additional post-processing step to the process, where the paths are shortcutted and smoothened, before the path can be used for tracking. Clearly a lot of time is invested to form this graph and produce smooth paths. Thus, if the obstacles start to move, then all of this work is done in no use, since all points that make up this graph need to be re-verified, which is simply too time consuming to be done in real time.
\\\\
In this work, we want to address the existing drawbacks of the sampling based planners. Our main contribution is an improved motion planner where each vertex in the graph covers a collision-free region in the form of a sphere instead of a point and where the edges are formed with neighboring intersecting spheres. This representation has the advantage of instead of returning piecewise linear paths, returning a sequence of overlapping spheres, i.e. a convex corridor, that connects a given start and goal configuration, illustrated in Figure \ref{fig:pbrm_intro}. This convex corridor allows us to use convex optimization to produce smooth trajectories, instead of computationally expensive post-processing methods. The representation further allows us to estimate the coverage of the collision-free space, which gives us awareness and feedback in the offline roadmap construction phase. Finally, our representation is simple to adapt to moving obstacles, simply requery for the new radii and recheck for intersections. 
\\\\
The spherical collision-free regions are formed using a signed distance function (SDF), which is a function that returns the smallest distance from an arbitrary point to the boundary of an obstacle. As the name implies, the distance is signed, thus if the point is inside the obstacle it is negative otherwise positive. If the distance is positive, a sphere with radius equal to the distance is guaranteed to cover a collision-free region. Using an SDF in motion planning is not new, but what is novel about our approach is that we express the distance in the configuration space instead of the world space and by doing so allows us to form these convex collision-free regions. We refer to the resulting SDF as a signed configuration distance function (SCDF). Computing an SCDF analytically is non-trivial, our approach is therefore to parameterize the SCDF with a deep neural network and learn the mapping by supervised learning. Our resulting neural SCDF can compute distances for different parameter values of obstacle shapes and we also show how multiple distances can be combined, thus making our approach flexible.
\section{Related work}
Motion planning algorithms can roughly be divided into three families, grid-based, sampling based and optimization based methods. Grid-based methods (GBM) discretize the planning space from which a graph is then compiled. A standard search method is A$^\star$ \citep{a_star}, which is classified as an \textit{informed} search method, since it employs a heuristic function to speed up the search. A$^\star$ guarantees to return an optimal path at the level of discretization used. GBMs usually discretize the planning space by a regular lattice and this limits the GBMs to problems with low dimensionality due to the curse of dimensionality. Thus, GBMs are usually limited to single-body robots where the degrees of freedom (DOF) are low. To overcome the inherent scaling problem with the GBMs, stochastic methods are usually used for multi-body robots. These methods are termed as sampling-based methods (SBM) and core members within this family are the rapidly-exploring random trees (RRT) \citep{rrt} and the probabilistic roadmap (PRM) \citep{prm}. RRT grows a tree from the start configuration and explores the collision-free region in a rapid way until it is able to connect to the goal region. RRT is usually improved by bi-directional planning \citep{rrt_connect}, i.e. an additional tree is grown from the goal configuration and the trees are tested for connection after any tree has been expanded. RRT is a single-query method, thus it searches for a path from scratch each time it is queried. Contrary to this, PRM is a multi-query method, which solves for multiple queries without starting from scratch. PRM does this by creating a roadmap (graph) that covers the collision-free space as an offline step. The graph is then used to solve for multiple queries. PRMs are used in cases where the environment does not change since the extra offline step is too computationally costly and needs to be re-done if the environment is changed. In our work, we address this inherent issue by using a different roadmap representation. Our vertices in the graph cover a collision-free region in the form of spheres and we form the edges by checking for intersecting spheres. If something in the environment changes, we recompute the spheres radii and recheck the intersections, without relying on collision detection. We use a trained neural network to compute the sphere radius, therefore querying for the radius can be done fast, hence our representation enables the PRM for dynamic environments.
\\\\
In the recent decades, optimization based methods (OBM) \citep{chomp, schulman, itomp, stomp} have been introduced as an alternative to SBM for multi-body robots. Like the SBM, the OBMs scale well to higher dimensional problems and produce smoother motion. It is common to use a SDF in the optimization since it is a smooth function, thus enabling gradient-based methods. However, the standard way of expressing the SDF is in world space. The distance therefore needs to be mapped to the configuration space by the forward kinematics. This mapping makes the optimization problem a non-linear program (NLP), which is computationally expensive to solve. Recently, a different approach has been proposed. In \cite{mp_gcs} motion planning is formulated as a convex optimization problem by using the graph of convex sets framework \citep{gcs}. The underlying idea is to decompose the collision-free space into intersecting convex sets from which a convex optimization problem is formulated. In cases where an explicit representation of the obstacles in the configuration space exists, like for single-body robots, creating collision-free convex regions can be done fast \citep{iris}. For multi-body robots, this is non-trivial. Existing work does this successfully \citep{iris_nlp, iris_c} by an optimization based approach, but the methods are still too time consuming to be used in the presence of moving obstacles. Our approach is instead to use deep learning to learn an SDF expressed in the configuration space. With this, we can query for shortest distances to the collision boundary, which allows us to expand spherical regions which are collision-free. Our approach is fast and therefore enables our suggested roadmap planner to be used in dynamic environments.
\\\\
Recent research has focused on learning collision detection \citep{fk_kernel_distance, diffco, graphdistnet} by predicting the signed distance between the robot links and the surrounding obstacles in the world space. The learned SDF is used in trajectory optimization but since the distance is expressed in the world space, the problem becomes an NLP and therefore takes a long time to solve. We take a novel approach and suggest to instead express the signed distance in the configuration space. This allows us to improve the PRM at the same time as it enables convex optimization for trajectory optimization, which runs faster and is more reliable than NLP solvers. In \cite{cspf} a learned signed distance function in the configuration space is proposed similar to our approach. However, their approach is restricted to point cloud representations, while we propose to represent the obstacles as parameterized geometric shapes, e.g. spheres. Furthermore, we also show how to use our learned SCDF to improve an existing roadmap planner.
\section{Problem formulation}
A robot is located in the world space, $\W \subset \R^3 $. The unique location of the robot is given by its configuration $\q \in \C$, where $\C$ is the configuration space. The set of points covered by the robots bodies at a certain configuration is expressed as $\B(\q) \subset \W$. The robot is surrounded by $\NrObst$ obstacles $\O = \bigcup_{i=1}^{\NrObst} \O_i$, where  $\O_i \subset \W$. The representation of the obstacle in the configuration space is the set $\C\O_i = \{\q \in \C \: |\: \B(\q) \cap \O_i \neq \emptyset \}$. The obstacle space is formed as $\Co = \bigcup_{i=1}^{\NrObst} \C \O_i$. The complement is referred to as the free space, $\Cf = \C \setminus \Co$. The path planning problem is a tuple, ($\Cf$, $\qStart$, $\qGoal$), where we want to connect a query pair, consisting of a start, $\qStart$, and goal configuration, $\qGoal$, with a geometric path, $\q(s): [0, 1] \mapsto \Cf$, such that $\q(0)=\qStart$ and $\q(1)=\qGoal$, or report correctly when such a path does not exist.
\end{document}

\section{Deontic Action Logic}\label{section:dal}

In this section, we cover the language, semantics, and axiomatization of Segerberg's deontic action logic \DAL~\cite{Segerberg1982}.  We also state a soundness and completeness result for future reference.

\paragraph{Language.}\label{section:dal:syntax}

The language of {\DAL} consists of \emph{actions} and \emph{formulas}.
Actions, indicated $\alpha$, $\beta$, $\gamma$, \dots, are built on a countable set $\bact = \set{\mathsf{a}_i}{i\in \Nat_0}$ of basic action symbols according to the following grammar:
%
\begin{align}
		\alpha,\beta & ::=
				\mathsf{a}_i
			\mid
				{\alpha \sqcup \beta}
			\mid
				{\alpha \sqcap \beta}
			\mid
				{\bar{\alpha}}
			\mid
				\mathsf{0}
			\mid
				\mathsf{1}. \label{definition:actions}
\end{align}%
We use $\act$ to indicate the set of all actions.
Formulas of \DAL, indicated $\varphi$, $\psi$, $\chi$, \dots, are built on the set $\act$ according to the following grammar:
%
\begin{align}
		\varphi,\psi & ::=
				{\alpha = \beta} \mid
				\perm(\alpha)
				\mid
				\forb(\alpha)
			% \mid
			% 	{\varphi \to \psi}
			\mid
				{\varphi \lor \psi}
			\mid
				{\varphi \land \psi}
			\mid
				{\lnot \varphi}
			\mid
				\bot
			\mid
				\top.
			\label{definition:formulas}
\end{align}%
We use $\form$ to indicate the set of all formulas of \DAL.
Intuitively, action symbols $\mathsf{a} \in \bact$ indicate a \emph{basic} action; actions ${\alpha \sqcup \beta}$ indicate the \emph{free-choice} composition of $\alpha$ and $\beta$; actions ${\alpha \sqcap \beta}$ indicate the \emph{parallel} composition of $\alpha$ and $\beta$; and $\bar{\alpha}$ indicate complement of $\alpha$. Finally, $\mathsf{0}$ and $\mathsf{1}$ indicate the \emph{impossible} and the \emph{universal} actions, respectively.
%
Turning to formulas, $\alpha = \beta$ indicates that $\alpha$ and $\beta$ are the same actions;  $\perm(\alpha)$ is read as $\alpha$ is \emph{permitted};
and $\forb(\alpha)$ is read as $\alpha$ is \emph{forbidden}.
Formulas built using $\land$, $\lor$, and $\lnot$, as well as $\bot$ and $\top$, have their standard interpretation.
We use $\varphi \to \psi$ as an abbreviation for $\lnot \varphi \lor \psi$, and $\varphi \liff \psi$ as an abbreviation for $(\varphi \to \psi) \land (\psi \to \varphi)$.
% Analogous to $\to$, we use $\alpha \sqsubset \beta$ as an abbreviation for $\bar{\alpha} \sqcup \beta$.
% Finally, we use $\obl(\alpha)$, read as $\alpha$ is \emph{obligatory}, as an abbreviation of $\forb(\bar{\alpha})$.

\medskip 

In \Cref{ex:syntax}, we present some actions and formulas of \DAL along with their intuitive interpretations.

\medskip

\begin{example}\label{ex:syntax}
Let $\mathsf{parking}$, $\mathsf{drinking}$, and $\mathsf{driving}$ be basic actions in $\bact$. Then:

\medskip
\begin{itemize}
	\item $\overline{\mathsf{parking}} = \mathsf{driving}$ asserts that `parking is the complement of (actively) driving';
	\item $\forb(\mathsf{drinking} \sqcap \mathsf{driving})$ asserting that `drinking while driving is forbidden';
	% \item $\neg\perm(\mathsf{teach \sqcup drink}$ stands for `it is nor permitted to teach while drinking';
	% \item ${{\overline{\mathsf{parking}} = \mathsf{driving}} \land \forb(\mathsf{drinking} \sqcap \mathsf{driving})} \to \forb(\overline{\mathsf{parking}} \sqcap \mathsf{drinking})$ intuitively says that since parking is the complement of driving, and drinking while driving is forbidden, then it is also forbidden to do anything other than parking while driving.
	\item $\forb(\mathsf{drinking} \sqcap \mathsf{driving}) \land \lnot\perm(\mathsf{drinking} \sqcap \mathsf{parking})$ asserting that `drinking while driving is forbidden' and that `it is not permitted to park while driving' either; indicating that operating a vehicle while drinking breaks the law.
	%\carlos{Me confundio este ultimo ejemplo.  Charlar. }
\end{itemize}
\end{example}


%%% Local Variables:
%%% mode: latex
%%% TeX-master: "article"
%%% End:

\paragraph{Semantics.}\label{section:dal:semantics}

The semantics for \DAL is given over deontic action models.
A deontic action model is a tuple $\DeonticModel = \langle E, P, F \rangle$ where: $E$ (the domain) is a (possibly empty) set of elements; and $P$ and $F$ are disjoint subsets of $E$ (i.e., ${P \cup F} \subseteq E$ and ${P \cap F} = \emptyset$).
Intuitively, $E$ indicates realizations of actions, $P$ and $F$ are sets of permitted and forbidden realizations of actions.
The disjointness condition on $P$ and $F$ indicates that permitted realizations of actions are not forbidden, and vice versa, that forbidden realizations are not permitted.
Given a model $\DeonticModel = \tup{E,P,F}$, a \emph{valuation} on $\DeonticModel$ is a function $v: \bact \rightarrow 2^E$.
Intuitively, a valuation indicates a particular way of realizing actions.

\medskip
\begin{proposition}
   For every deontic action model $\DeonticModel = \tup{E,P,F}$, and any valuation $v: {\bact \to 2^E}$ on $\DeonticModel$, there is a unique $v^{*} : {\act \rightarrow 2^E}$ s.t.:
   \begin{align*}
            v^{*}(\alpha \sqcup \beta)
               &=
               {v^{*}(\alpha)
               \cup
               v^{*}(\beta)}
         &
         v^{*}(\bar{\alpha})
               &=
               {E \setminus v^{*}(\alpha)}
         &
         v^{*}(0)
               &=  \emptyset
         \\
            v^{*}(\alpha \sqcap \beta)
               &=
               {v^{*}(\alpha)
               \cap
               v^{*}(\beta)}
         &&&
            v^{*}(1)
               &=  E.
   \end{align*}%
\end{proposition}
\medskip




The \emph{satisfiability} of a formula $\varphi$ on a deontic action model $\DeonticModel = \tup{E, P, F}$ under a
valuation $v$, written ${\DeonticModel, v} \Vdash \varphi$, is
defined inductively as:
\[
\begin{array}{rlcl}
   \DeonticModel, v & \Vdash \alpha=\beta
       & \mathrel{\mbox{ iff }} & 	v^{*}(\alpha) = v^{*}(\beta) \\
   \DeonticModel, v & \Vdash \perm(\alpha)
       & \mathrel{\mbox{ iff }} & v^{*}(\alpha) \subseteq P\\
   \DeonticModel, v & \Vdash \forb(\alpha)
       & \mathrel{\mbox{ iff }} & v^{*}(\alpha) \subseteq F\\
   %  \DeonticModel, v & \Vdash \varphi \to \psi
	%    & \mathrel{\mbox{ iff }} & \DeonticModel, v \nVdash \varphi
   %       \mbox{ or }      \DeonticModel, v  \Vdash \psi\\
   \DeonticModel, v & \Vdash \varphi \lor \psi
       & \mathrel{\mbox{ iff }} & \DeonticModel, v \Vdash \varphi
         \mbox{ or }      \DeonticModel, v \Vdash \psi\\
   \DeonticModel, v & \Vdash \varphi \land \psi
       & \mathrel{\mbox{ iff }} & \DeonticModel, v \Vdash \varphi
       \mbox{ and } \DeonticModel, v \Vdash \psi\\
   \DeonticModel, v & \Vdash \lnot \varphi
       & \mathrel{\mbox{ iff }} & \DeonticModel, v \nVdash \varphi\\
   \DeonticModel, v & \Vdash \bot
      &  & \mbox{never}\\
   \DeonticModel, v & \Vdash \top
      &  & \mbox{always.}\\
\end{array}
\]
We say that a formula $\varphi$ is a \emph{tautology} iff for any deontic action model $\DeonticModel$ and for any valuation $v$ on $\DeonticModel$, it follows that ${\DeonticModel, v} \Vdash \varphi$.

In \Cref{ex:semantics}, we present examples of deontic action models for the formulas in \Cref{ex:syntax}.

\medskip 


\begin{example}\label{ex:semantics}
   \begin{figure}
      \centering
      \begin{minipage}{0.5\textwidth}
            \centering
            %\hspace*{-.5cm}
            \includegraphics[trim=50pt 0pt 50pt 0pt, clip, width=1\textwidth]{deontic-model-a.pdf}\\[1em] % first figure itself
            \caption{A Deontic Action Model.}\label{ex:deontic:model:a}
      \end{minipage}\hfill
      \begin{minipage}{0.5\textwidth}
            \centering
            %\hspace*{-.5cm}
            \includegraphics[trim=50pt 0pt 50pt 0pt, clip, width=1\textwidth]{deontic-model-b.pdf}\\[1em] % second figure itself
            \caption{Anoter Deontic Action Model.}\label{ex:deontic:model:b}
      \end{minipage}
   \end{figure}
   Let $\DeonticModel = \tup{E,P,F}$ be the deontic action model in which $E$, $P$, and $F$ are as in \Cref{ex:deontic:model:a}.
   In addition, $\DeonticModel' = \tup{E,P',F}$ be the deontic action model in which $E$, $P'$, and $F$ are as in \Cref{ex:deontic:model:b}.
   Lastly, let $\{\mathsf{drinking}, \mathsf{driving}, \mathsf{parking}\} \subset \bact$, and $v: \bact \to 2^E$ be a valuation where $v(\mathsf{drinking})$, $v(\mathsf{driving})$, and $v(\mathsf{parking})$ are as in \Cref{ex:deontic:model:a}.
   Then:


   \begin{multicols}{2}
   \begin{enumerate}
      \item $\DeonticModel, v \Vdash \overline{\mathsf{parking}} = \mathsf{driving}$
      \item $\DeonticModel, v \Vdash \forb(\mathsf{drinking} \sqcap \mathsf{driving})$
      \item $\DeonticModel, v \Vdash \perm(\mathsf{drinking} \sqcap \mathsf{parking})$
      \item $\DeonticModel, v \nVdash \forb(\mathsf{drinking} \sqcap \mathsf{driving}) \land$
      \item[] \hspace{1.9cm}$\lnot\perm(\mathsf{drinking} \sqcap \mathsf{parking})$
      \item $\DeonticModel', v \Vdash \overline{\mathsf{parking}} = \mathsf{driving}$
      \item $\DeonticModel', v \Vdash \forb(\mathsf{drinking} \sqcap \mathsf{driving})$
      \item $\DeonticModel', v \nVdash \perm(\mathsf{drinking} \sqcap \mathsf{parking})$
      \item $\DeonticModel', v \Vdash \forb(\mathsf{drinking} \sqcap \mathsf{driving}) \land$
      \item[] \hspace{1.9cm}$\lnot\perm(\mathsf{drinking} \sqcap \mathsf{parking})$.
   \end{enumerate}
   \end{multicols}

   \noindent The models $\DeonticModel$ and $\DeonticModel'$ make clear, for example, that in \DAL, $\forb(\alpha)$ and $\lnot\perm(\alpha)$ are not equivalent.
   % Recall the formulas from~\Cref{ex:syntax}. Here we present a deontic action model $\DeonticModel$, where $F=\emptyset$ and $P=\{e_1,e_2\}$. Notice that $v(\mathsf{drink})=e_1$ and $v(\mathsf{teach}) = v(\mathsf{educate}) = e_2$. 

   % There, we can check that for instance, $\DeonticModel\Vdash\mathsf{teach} = \mathsf{educate}$ and that $\DeonticModel\Vdash \neg\perm(\mathsf{educate \sqcap drink})$. However, $\DeonticModel\nVdash\forb(\mathsf{teach \sqcap drink})$ (since $F=\emptyset$).
   % \medskip 

   % \begin{center}
   % \begin{tikzpicture}
   %    \draw[thick, rounded corners=8pt] (-1,0) rectangle (4,2);

   % % Points inside the square
   % \node at (0,1.8) {$P$}; 
   % \filldraw[black] (0.6,1) circle (2pt) node[below] {$\mathsf{drink}$} node[above] {$e_1$};
   % \filldraw[black] (2.4,1) circle (2pt) node[below] {$\begin{array}{l}\mathsf{teach} \\ \mathsf{educate}\end{array}$} node[above] {$e_2$};

   % % Add optional labels or grid (if needed for clarity)
   % % \draw[help lines] (0,0) grid (4,4); % Uncomment this for a grid

   % \end{tikzpicture}
   % \end{center}
\end{example}

%%% Local Variables:
%%% mode: latex
%%% TeX-master: "article"
%%% End:


\paragraph{Axiomatization.}

We present an axiomatic system for \DAL that differs  slightly from the one originally used in \cite{Segerberg1982}. This change is solely motivated by the fact that our presentation simplifies the axiomatization of the variations to \DAL that we introduce in the following sections.

We organize the presentation of the axiom system of \DAL in four groups of axioms as shown in 
\Cref{dal:axioms}. %(i.e., we assume uniform substitution by expressions of the proper type).
The axioms in the first group (A1--A13 and LEM) characterize operations on actions, and is inspired by the presentation of Boolean algebras via complemented distributive lattices in~\cite{Esakia:2019,Halmos:2009}.
%
The axioms in the second group do the same for propositional connectives on formulas (A1'--A13' and LEM').
%Since they are well-known, we leave the axioms in this group implicit.
%Its explicit formulation is obtained from the axioms for actions by replacing $\alpha$, $\beta$, and $\gamma$, for $\varphi$, $\psi$, and $\chi$; $\sqcup$, $\sqcap$, $\bar{~}$, $\mathsf{0}$, and $\mathsf{1}$, for $\lor$, $\land$, $\lnot$, $\bot$, and $\top$; and $=$ for $\liff$.
%To see that this group of axioms indeed captures propositional connectives on formulas we refer to~\cite{Mendelson:2015,Troelstra:1988}.
%
The axioms in the third group (E1 and E2) characterize equality.
%
Finally, the axioms the fourth group (D1--D3) characterize the deontic operators of permission $\perm$ and prohibition $\forb$.
%
%These four groups of axioms are summarized in \Cref{dal:axioms}.
%Therein, axioms A1--A13, and LEM correspond to axioms for operations on actions --and, adapted accordingly, to axioms for propositional connectives on formulas.
%In turn, axioms E1 and E2 are the axioms for equality.
%In axiom E2, we use $\varphi_{\alpha}^{\beta}$ to indicate the formula obtained by replacing some occurrences of $\alpha$ in $\varphi$ with $\beta$.
%Finally, axioms D1--D3 are the axioms for the deontic operators of permission and prohibition.
%The axioms D1--D3 appear first in~\cite{Segerberg1982}.

\begin{figure}
	\centering
	\fbox{
	\begin{minipage}{0.95\textwidth}
		\begin{multicols}{2}
			\begin{enumerate}[label=A\arabic*.]
				\item $\alpha \sqcap (\beta \sqcap \gamma) = (\alpha \sqcap \beta) \sqcap \gamma$
				\item $\alpha \sqcap \beta = \beta \sqcap \alpha$
				\item $\alpha \sqcap \alpha = \alpha$
				\item $\alpha \sqcap (\alpha \sqcup \beta) = \alpha$
				\item ${\alpha \sqcap (\beta \sqcup \gamma)} = {(\alpha \sqcap \beta) \sqcup (\alpha \sqcap \gamma)}$
				\item $\alpha \sqcap \mathsf{0} = \mathsf{0}$
				%\item $((\alpha \sqcap \beta) \sqsubseteq \alpha) \sqcap \gamma = \gamma$
				%\item $\alpha \sqsubseteq \mathsf{0} = \bar{\alpha}$
				\item $\alpha \sqcap \bar{\alpha} = \mathsf{0}$
				\item $\alpha \sqcup (\beta \sqcup \gamma) = (\alpha \sqcup \beta) \sqcup \gamma$
				\item $\alpha \sqcup \beta = \beta \sqcup \alpha$
				\item $\alpha \sqcup \alpha = \alpha$
				\item $\alpha \sqcup (\alpha \sqcap \beta) = \alpha$
				\item ${\alpha \sqcup (\beta \sqcap \gamma)} = {(\alpha \sqcup \beta) \sqcap (\alpha \sqcup \gamma)}$
				\item $\alpha \sqcup \mathsf{1} = \mathsf{1}$
				%\item $\alpha \sqcap (\alpha \sqsubseteq \beta) = \beta$
				%\item $\alpha \sqcap (\beta \sqsubseteq \gamma) = \alpha \sqcap ((\alpha \sqcap \beta) \sqsubseteq (\alpha \sqcap \gamma))$
				\item[LEM.] $\alpha \sqcup \bar{\alpha} = \mathsf{1}$
			\end{enumerate}
		\end{multicols}
		\ \\[-1.5cm]
		\begin{multicols}{2}
	\begin{enumerate}[label=A\arabic*'.]
		\item $\varphi \wedge (\psi \wedge \chi) \liff (\varphi \wedge \psi) \wedge \chi$
		\item $\varphi \wedge \psi \liff \psi \wedge \varphi$
		\item $\varphi \wedge \varphi \liff \varphi$
		\item $\varphi \wedge (\varphi \vee \psi) \liff \varphi$
		\item ${\varphi \wedge (\psi \vee \chi)} \liff {(\varphi \wedge \psi) \vee (\varphi \wedge \chi)}$
		\item $\varphi \wedge \bot \liff \bot$
		\item $\varphi \wedge \neg \varphi \liff \bot$
		\item $\varphi \vee (\psi \vee \chi) \liff (\varphi \vee \psi) \vee \chi$
		\item $\varphi \vee \psi \liff \psi \vee \varphi$
		\item $\varphi \vee \varphi \liff \varphi$
		\item $\varphi \vee (\varphi \wedge \psi) \liff \varphi$
		\item ${\varphi \vee (\psi \wedge \chi)} \liff {(\varphi \vee \psi) \wedge (\varphi \vee \chi)}$
		\item $\varphi \vee \top \liff \top$
		\item[LEM'.] $\varphi \vee \neg \varphi \liff \top$
	\end{enumerate}
\end{multicols}
		\ \\[-1.5cm]
		\begin{multicols}{2}
			\begin{enumerate}[label=E\arabic*.]
				\item $\alpha = \alpha$
				% \item ${\alpha = \beta} \to {\beta = \alpha}$
				% \item ${(\alpha = \beta \land \beta = \gamma)} \to {\alpha = \gamma}$
				\item $(\alpha=\beta \land \varphi) \to {\varphi_{\alpha}^{\beta}}$%\footnote{where $\varphi_{\alpha}^{\beta}$ is obtained by replacing some occurrences of $\alpha$ in $\varphi$ with $\beta$.}
			\end{enumerate}
		\end{multicols}
		\ \\[-1.5cm]
		\begin{multicols}{2}
			\begin{enumerate}[label=D\arabic*.]
				\item $\perm(\alpha\sqcup\beta) \liff (\perm(\alpha) \land \perm(\beta))$
				\item $\forb(\alpha\sqcup\beta) \liff (\forb(\alpha) \land \forb(\beta))$
				\item $(\perm(\alpha) \land \forb(\alpha)) \liff (\alpha = \mathsf{0})$
			\end{enumerate}
		\end{multicols}
	\end{minipage}}\\[1em]
	% \medskip
	\caption{Axiom System for \DAL.}\label{dal:axioms}
\end{figure}

% \medskip

% {\setlength\tabcolsep{1pt}
% 	\begin{center}
% 		\footnotesize
% 		\setlength{\abovedisplayskip}{4pt}
% 		\setlength{\belowdisplayskip}{-6pt}
% 		\setlength{\abovedisplayshortskip}{4pt}
% 		\setlength{\belowdisplayshortskip}{-3pt}
% 		\renewcommand{\arraystretch}{1.7}

% 		\begin{tabular}{|@{\ \ }p{.97\textwidth}|}	\hline
% 			\begin{enumerate}[label=D\arabic*]
% 				\begin{minipage}[t]{0.4\linewidth}
% 			\item $\perm(\alpha\sqcup\beta) \liff (\perm(\alpha) \land \perm(\beta))$
% \item $\forb(\alpha\sqcup\beta) \liff (\forb(\alpha) \land \forb(\beta))$
% 				\end{minipage}
% 				\hfill
% 				\begin{minipage}[t]{0.52\linewidth}
% \item $(\perm(\alpha) \land \forb(\alpha)) \liff (\alpha = \mathsf{0})$
% 				\end{minipage}
% 			\end{enumerate}\\\hline
% 		\end{tabular}
% \end{center}}

% \medskip

In \DAL, a Hilbert-style proof of a formula $\varphi$ is defined as a finite sequence $\psi_1, \dots, \psi_n$ of formulas s.t.: $\psi_n = \varphi$, and for each $1 \leq k \leq n$, $\psi_k$ is an axiom, or is obtained from two earlier formulas $\psi_i$ and $\psi_j$ using the rule of \emph{modus ponens} (i.e., there are $1 \leq i < j < k$ s.t.\ $\psi_j = {\psi_i \to \psi_k}$).
We say that $\varphi$ is a theorem, and write $\vdash \varphi$, iff there is a proof of $\varphi$.
We make a slight abuse of notation and use \DAL to indicate both the logic and its set of theorems.
We state \Cref{th:segerber:completeness} for future reference.

\medskip
\begin{theorem}[\cite{Segerberg1982}]\label{th:segerber:completeness}
	In \DAL, a formula is a theorem if and only if it is a tautology.
\end{theorem}


%%% Local Variables:
%%% mode: latex
%%% TeX-master: "article"
%%% End:

\section{Deontic Action Logic via Algebra}\label{sec:algebraic-char}

% The Deontic Action Logic \DAL enjoys some interesting characteristics.
% In particular, it is a simple system that provides a well-executed characterization of deontic operators.
% It enjoys an elegant semantics via sets and collections of sets (or dually via ideals and Boolean algebras).
% Interestingly also, \DAL further allows for additional deontic operators to be added systematically.
%More importantly, the formalism is sound and complete (Theorem 3.1 in \cite{Segerberg1982}).

We now turn our attention to revisiting and expanding the algebraic characterization of \DAL we presented in \cite{CCFA:2021}. To be noted, this algebraic framework is mathematically more abstract compared to the one in \cite{Segerberg1982}. This level of abstraction is a characteristic of algebraic logics, which can be leveraged to address broader issues in deontic logic. Furthermore, a distinguishing feature of our approach is its modularity. The class of algebras described below can be easily extended to support additional deontic operators, and in all cases, standard algebraic tools can be employed to prove soundness and completeness results.
We take advantage of this feature to build new deontic actions logics in the spirit of \DAL in \Cref{section:new:dals}.



\section{Preliminaries}\label{sec:preliminaries}



%We denote by $(\Ac(x_\Ac),\Bc(x_\Bc))(z)$ a random execution of $\pi$ with private inputs $(x_\Ac,y_\Ac)$, and common input $z$.

%\Jnote{Move to DP}
% At the end of such an execution, the protocol outputs a public transcript denoted by the random variable $\trans_\pi(x_\Ac,x_\Ac,z)$ we denotes the common as $\out(\trans_\pi(x_\Ac,x_\Ac,z)$, and each party $\Pc \in \set{\Ac,\Bc}$ obtains his view denoted $\view^\Pc_\pi(x_\Ac,x_\Bc,z)$, which may also contain a ``local output'' \Jnote{Local} $\out^\Pc(x_\Ac,x_\Bc,z)$ (if the protocol specifies such an output). \Jnote{Common output, and parties output}


\subsection{Distributions and Random Variables}\label{sec:prelim:dist}
The support of a distribution $P$ over a finite set $\cS$ is defined by $\Supp(P) \eqdef \set{x\in \cS: P(x)>0}$. For a distribution or a random variable $D$, let $d\from D$ denote that $d$ was sampled according to $D$. Similarly,  for a set $\cS$, let $x \from \cS$ denote that $x$ is drawn uniformly from $\cS$, and denote by $\cU_{\cS}$ the uniform distribution over $\cS$. For a finite set $\cX$ and a distribution $C_X$ over $\cX$, we use the capital letter $X$ to denote the random variable that takes values in $\cX$ and is sampled according to $C_X$. The {\sf statistical distance} (\aka {\sf~variation distance}) of two distributions $P$ and $Q$ over a discrete domain $\cX$ is defined by $\sdist{P}{Q} \eqdef \max_{\cS\subseteq \cX} \size{P(\cS)-Q(\cS)} = \frac{1}{2} \sum_{x \in \cS}\size{P(x)-Q(x)}$. 
For a vector $x = (x_1,\ldots,x_n)$ and index $i\in [n]$, we let $x_{-i} = (x_1,\ldots,x_{i-1},x_{i+1},\ldots,x_n)$ and $x^{(i)} = (x_1,\ldots,x_{i-1}, -x_i, x_{i+1},\ldots,x_n)$, for a set $\cS \subseteq [n]$ we let $x_{\cS} = (x_i)_{i \in \cS}$ and $x_{-\cS} = (x_i)_{i \in [n]\setminus \cS}$, and for a vector $r \in \zo^n$ we let $x_r = (x_i)_{\set{i \colon r_i = 1}}$ and $x_{-r} = (x_i)_{\set{i \colon r_i = 0}}$.

%For $n \in \N$ we let $U_n$ be the uniform distribution over $\oo^n$, and let $S_n$ be the distribution induces by the sum of $n$ i.i.d.\ random variables, each is distributed according to $U_1$. Let $\cN(0,1)$ be the standard normal distribution.
%For a distribution $\cD$ and a function $f$, we define by $f(\cD)$ the distribution that is induced by the output of $f(x)$ for $x \from \cD$. 





% \begin{theorem}[\cite{McGregorMPRTV10}]\label{thm:sv-extracotr}
% 	\Enote{Remove if not needed}
% 	There is a constant $c$ to make the following holds. Let $X$ be an $\alpha$-SV source on $\{0,1\}^n$, let $Y$ be a source on $\{0,1\}^n$ with min-entropy at least $\beta n$ (independent from $X$), and let $Z=\ip{X,Y}\mbox{mod m}$ for some $m\in\mathbb{N}$. Then for every $\delta\in[0,1]$, the random variable $(Y,Z)$ is $\delta$-close to $(Y,U)$ where $U$ is uniform on $\mathbb{Z}_m$ and independent of $Y$, provided that
% 	$$
% 	n\geq c\cdot\frac{m^2}{\alpha\beta}\cdot\log(\frac{m}{\beta})\cdot\log(\frac{m}{\delta}).
% 	$$
% \end{theorem}



\Enote{I removed the definition of DP since it already appears in the intro}
\remove{
\subsection{Differential Privacy}\label{sec:prelim:DP}
We use the following standard definition of (information theoretic) differential privacy, due to \citet{DMNS06}. For notational convenience, we focus on databases over $\oo$.
\begin{definition}[Differentially private mechanisms]\label{def:mech}
	A randomized function $f\colon\oo^n\mapsto \zs$ is an {\sf $n$-size, $(\eps,\delta)$-differentially private mechanism} (denoted $(\eps,\delta)$-\DP) if for every neighboring $w,w'\in \oo^n$ and every function $g\colon \zs\mapsto \zo$, it holds that 
	$$
	\pr{g(f(w))=1}\leq e^{\eps}\cdot \pr{g(f(w'))=1} +\delta.
	$$ 	
	If $\delta=0$, we omit it from the notation.
\end{definition}
}


\subsubsection{Computational Differential Privacy}
There are several ways for defining computational differential privacy (see \cref{sec:related-works}). We use the most relaxed version due to \cite{BNO08}. For notational convenience, we focus on databases over $\oo$.
\begin{definition}[Computational differentially private mechanisms]\label{def:ComMech}
	A randomized function ensemble $f=\set{f_\pk\colon\oo^{n(\pk)}\mapsto \zs}$ is an {\sf $n$-size, $(\eps,\delta)$-computationally differentially private} (denoted $(\eps,\delta)$-$\CDP$) if for every poly-size circuit family $\set{\Ac_\pk}_{\pk\in \N}$, the following holds for every large enough $\pk$ and every neighboring $w,w'\in\oo^{n(\pk)}$:
	$$
	\pr{\Ac_\pk(f_\pk(w))=1}\leq e^{\eps(\pk)}\cdot \pr{\Ac_\pk(f_\pk(w'))=1} +\delta(\pk).
	$$ 
	If $\delta(\pk) = \negl(\pk)$, we omit it from the notation. 
\end{definition}



\subsubsection{Two-Party Differential Privacy}\label{sec:DP}
In this section we formally define distributed differential privacy mechanism (\ie protocols). %For the ease of notation, we consider protocol with no common input.

\begin{definition}\label{def:DP}%\Nnote{fix security parameter}
	A two-party protocol $\Pi=(\Ac,\Bc)$ is {\sf $(\eps,\delta)$-differentially private}, denoted $(\eps,\delta)$-$\DP$, if the following holds for every algorithm $\Dc$: let $\V^\Pc(x,y)(\pk)$ be the view of party $\Pc$ in a random execution of $\Pi(x,y)(1^\pk)$. Then for every $\pk,n \in \N$, $x\in \oo^n$ and neighboring $y,y'\in\oo^n$:
	\begin{align*}
	\pr{\Dc(V^\Ac(x,y)(\pk))=1}\le e^{\eps(\pk)}\cdot \pr{\Dc(V^\Ac (x,y')(\pk))=1}+\delta(\pk),
	\end{align*} 
	and for every $y\in \oo^n$ and neighboring $x,x'\in\oo^{n}$:
	\begin{align*}
	\pr{\Dc(V^\Bc(x,y)(\pk))=1}\le e^{\eps(\pk)}\cdot \pr{\Dc(V^\Bc (x',y)(\pk))=1}+\delta(\pk).
	\end{align*} 	
	Protocol $\Pi$ is {\sf $(\eps,\delta)$-computational differentially private}, denoted $(\eps,\delta)$-$\CDP$, if the above inequalities only hold for a non-uniform \ppt $\Dc$ and large enough $\pk$. We omit $\delta = \negl(\pk)$ from the notation. \footnote{Note that define we give for two-party differentially private protocols is a semi-honest definition, in which we ask for the security to hold when the parties interact in an honest execution of the protocol. Since we are proving a lower bound, starting from this weaker guarantee (as opposed to security against malicious players), yields a stronger result.}
\end{definition}
%We omit $\delta$ from the notation if $\delta$ is a negligible function of $n$.

%\Enote{simulation-based}
\begin{remark}[The definition for computational differential privacy we use]\label{rem:comDPChannel} 
	An alternative, stronger definition of computational differential privacy, known as simulation-based computational differential privacy, requires that the distribution of each party’s view be computationally indistinguishable from a distribution that ensures privacy in an information-theoretic sense. \cref{def:DP} is a weaker notion in comparison. Consequently, establishing a lower bound for a protocol that satisfies this weaker guarantee (as we do in this work) yields a stronger result.%Actually, our lower bound only requires the privacy to hold against \emph{uniform} external observer.
	%\Nnote{Maybe add: When only interesting in \Dp against external observer, the two definitions can be achieve using key-agreement and (single-party) \Dp mechanism. }
\end{remark}




\subsection{Useful Claims}
\remove{
In this section, we state generic lemmas and propositions that we will use later in our proofs.

The following lemma which we prove in \cref{sec:missing-proofs:distance-I}, measures the distance between two uniform stings conditioned one a random index $i$ either being fixed to $0$ or to $1$.

\def\distanceILemma{
    Let $R \la \zo^n$. For any (randomized) function $f:\{0,1\}^n\rightarrow \{0,1\}$ and $\alpha > 0$, it holds that
    \begin{align}\label{eq:f-alpha}
        \ppr{i \la [n]}{\size{\:\ex{f(R) \mid R_i = 0}-\ex{f(R) \mid R_i = 1}\:}\geq \alpha} \leq \frac{2}{n \alpha^2},
    \end{align}
    where the expectations are taken over $R$ and the randomness of $f$.
}

\begin{lemma}\label{lem:distance-I}
    \distanceILemma
\end{lemma}
}

The following two propositions state that given the output of a differentially private function, it is not possible to predict well even a random index (even if all other indexes are leaked). The first proposition handles the information-theoretic case and the second handles the computation case. Both propositions are proven in \cref{sec:missing-proofs:hard-to-guess}. 

\def\propHardToGuessInf{
    Let $f\colon \oo^n \rightarrow \cY$ be an $(\eps,\delta)$-\DP function, let $g \colon [n] \times \oo^{n-1} \times \cY \rightarrow \set{-1,1,\bot}$ be a (randomized) function, and let $X = (X_1,\ldots,X_n) \la \oo^n$. Then the following holds for every $i \in [n]$ where $X_i^* = g(i,X_{-i},f(X_1,\ldots,X_n))$:
    \begin{align*}
        \pr{X_i^* = X_i} \leq e^{\eps}\cdot \pr{X_i^* = -X_i} + \delta.
    \end{align*}
}

\begin{proposition}\label{prop:hard-to-guess-inf}
    \propHardToGuessInf
\end{proposition}


\def\propHardToGuessComp{
    Let $f = \set{f_{\pk} \colon \oo^{n(\pk)} \rightarrow \zo^{m(\pk)}}_{\pk \in \bbN}$ be an $(\eps,\delta)$-\CDP function ensemble, and let $\set{g_{\pk}}_{\pk \in \bbN}$ be a poly-size circuit family. Then, for large enough $\pk$ and $X = (X_1,\ldots,X_{n(\pk)}) \la \oo^{n(\pk)}$, the following holds for every $i \in [n(\pk)]$ where $X_i^* = g_{\pk}(i,X_{-i},f_{\pk}(X_1,\ldots,X_n))$:
    \begin{align*}
        \pr{X_i^* = X_i} \leq e^{\eps(\pk)}\cdot \pr{X_i^* = -X_i} + \delta(\pk).
    \end{align*}
}

\begin{proposition}\label{prop:hard-to-guess-comp}
    \propHardToGuessComp
\end{proposition}





\remove{
\Enote{Chao's old statement:}
\begin{lemma}\label{lem:distance-I-old}
        Let $R \la \zo^n$. 
	For any function $f:\{0,1\}^n\rightarrow \{0,1\}$ and $\alpha<0.01$, it holds that
	$$
	\Pr_{i\la[n]}\left[\: \size{\:\mathbb{E}[f(R) \mid R_i = 0]-\mathbb{E}[f(R) \mid R_i = 1]\:}\geq \alpha\right]\leq \frac{2+2\log(\frac{1}{\alpha})}{n\alpha^2}.
	$$
\end{lemma}
\begin{proof}
	Define $S_1=\{r \in \zo^n \colon f(r)=1\}$. Then for any $i\in[n]$, we have
	$$
	\begin{array}{rl}
		\size{\mathbb{E}[f(R) \mid R_i = 0]-\mathbb{E}[f(R) \mid R_i = 1]}
		&=\size{\Pr[R\in S_1|R_i=0]-\Pr[R\in S_1|R_i=1]}\\
		&=\size{\frac{\Pr[R_i=0|R\in S_1]\cdot\Pr[R\in S_1]}{\Pr[R_i=0]}-\frac{\Pr[R_i=1|R\in S_1]\cdot\Pr[R\in S_1]}{\Pr[R_i=1]}}\\
		&=\frac{2\size{S_1}}{2^n}\size{\Pr[R_i=0|R\in S_1]-\Pr[R_i=1|R\in S_1]}
	\end{array}
	$$
	When $|S_1|\leq \alpha\cdot 2^{n-1}$, we have $\size{\mathbb{E}[f(R) \mid R_i = 0]-\mathbb{E}[f(R) \mid R_i = 1]}\leq\frac{2\size{S_1}}{2^n}\leq \alpha$ for any $i\in[n]$. Hence, in the following, we assume $|S_1|> \alpha\cdot 2^{n-1}$.

	%Define $I_{bad}=\{i|\size{\Pr[R_i=0|R\in S_1]-\Pr[R_i=1|R\in S_1]}>2\alpha\}$ and $k=\size{I_{bad}}$, then for any $i\notin I_{bad}$, we have 
    %$$
    %\begin{array}{rl}
    %    2\alpha&\geq \size{\Pr[R_i=0|R\in S_1]-\Pr[R_i=1|R\in S_1]}\\
    %    &=\size{\frac{\Pr[R\in S_1|R_i=0]\cdot\Pr[R_i=0]}{\Pr[R\in S_1]}-\frac{\Pr[R\in S_1|R_i=1]\cdot\Pr[R_i=1]}{\Pr[R\in S_1]}}\\
    %    &=\size{\Pr[R\in S_1|R_i=0]-\Pr[R\in S_1|R_i=1]}\cdot\frac{1}{2\Pr[R\in S_1]}\\
    %    &\geq \size{\mathbb{E}[f(R) \mid R_i = 0]-\mathbb{E}[f(R) \mid R_i = 1]}\cdot \frac{1}{2},
    %\end{array}
    %$$ 
    %where the last inequality is because $\Pr[R\in S_1]\leq 1$. So that $\size{\mathbb{E}}[f(R) \mid R_i = 0]-\mathbb{E}[f(R) \mid R_i = 1]\leq %4\alpha$.
    Define $I_{bad}=\{i \colon \size{\Pr[R_i=0|R\in S_1]-\Pr[R_i=1|R\in S_1]} \geq 2\alpha\}$ and $k=\size{I_{bad}}$, and denote $I_{bad}=\{i_1,\dots,i_k\}$. Define $(X_{i_1}, \ldots X_{i_k}) = (R_{i_1},\dots,R_{i_k})\mid_{R \in S_1}$. 
    Consider the min-entropy
	$$
	\begin{array}{rl}
		H_{min}(X_{i_1},\dots,X_{i_k})&\leq H(X_{i_1},\dots,X_{i_k})\\
		&\leq \sum_{j=1}^k H(X_{i_j})\\
		&\leq k\cdot \left(-(\frac{1}{2}+2\alpha)\cdot\log(\frac{1}{2}+2\alpha)-(\frac{1}{2}-2\alpha)\cdot\log(\frac{1}{2}-2\alpha)\right)\\
            &=k\cdot \left(-(\frac{1}{2}+2\alpha)\cdot(\log(1+4\alpha)-1)-(\frac{1}{2}-2\alpha)\cdot(\log(1-4\alpha)-1)\right)\\
            &=k\cdot \left(1-(\frac{1}{2}+2\alpha)\cdot\log(1+4\alpha)-(\frac{1}{2}-2\alpha)\cdot\log(1-4\alpha)\right),
		
	\end{array}
	$$
	where $H_{min}(Y)$ is the minimum entropy of $Y$ and $H(Y)$ is the Shannon entropy of $Y$.\Enote{add to preliminaries.}
        The third inequality holds since by the definition of $I_{bad}$, for every $j \in [k]$ it holds that $\size{\pr{X_{i_j} = 1}-\pr{X_{i_j} = 0}} > 2\alpha$, and therefore $H(X_{i_j}) \leq H(1/2 + 2\alpha)$\Enote{define}.
	
	Therefore, there exists $b_1,\dots,b_k\in\{0,1\}$, such that 
	
	\begin{align}\label{eq:min-entropy-result}
		\Pr\left[(R_{i_1},\ldots,R_{i_k}) = (b_1,\ldots,b_k) \mid R\in S_1\right]
		&= \pr{(X_{i_1},\ldots,X_{i_k}) = (b_1,\ldots,b_k)}\\
		&= 2^{-H_{min}(X_{i_1},\dots,X_{i_k})}\nonumber\\
		&\geq 2^{k\cdot \left(-1+(\frac{1}{2}+2\alpha)\cdot\log(1+4\alpha)+(\frac{1}{2}-2\alpha)\cdot\log(1-4\alpha)\right)}.\nonumber
	\end{align}
	
	Let $S_{bad}=\{r \in \zo^n  \colon \set{(r_{i_1},\ldots,r_{i_k}) = (b_1,\ldots,b_k)} \land \set{r\in S_1}\}$.
	It holds that
	\begin{align*}
		|S_{bad}|
		&= \size{S_1} \cdot \Pr\left[(R_{i_1},\ldots,R_{i_k}) = (b_1,\ldots,b_k) \mid R\in S_1\right]\\
		&\geq \alpha\cdot 2^{n-1}\cdot2^{k\cdot \left(-1+(\frac{1}{2}+2\alpha)\cdot\log(1+4\alpha)+(\frac{1}{2}-2\alpha)\cdot\log(1-4\alpha)\right)},
	\end{align*} 
	where the inequality holds by \cref{eq:min-entropy-result} and since $\size{S_1} \geq \alpha\cdot 2^{n-1}$.
	Notice that any string in $S_{bad}$ depends on at most $n-k$ bits. It implies that $|S_{bad}|\leq 2^{n-k}$. Therefore, we have
	$$
	\begin{array}{rl}
		&2^{n-k}\geq \alpha\cdot 2^{n-1}\cdot2^{k\cdot \left(-1+(\frac{1}{2}+2\alpha)\cdot\log(1+4\alpha)+(\frac{1}{2}-2\alpha)\cdot\log(1-4\alpha)\right)} \\
		\Rightarrow& n-k \geq \log \alpha+n-1+k\cdot \left(-1+(\frac{1}{2}+2\alpha)\cdot\log(1+4\alpha)+(\frac{1}{2}-2\alpha)\cdot\log(1-4\alpha)\right)\\
		\Rightarrow& 1-\log \alpha \geq k\cdot((\frac{1}{2}+2\alpha)\cdot\log(1+4\alpha)+(\frac{1}{2}-2\alpha)\cdot\log(1-4\alpha))\\
		\Rightarrow& 1-\log \alpha \geq k\cdot(4\alpha\cdot\log(1+4\alpha)+(\frac{1}{2}-2\alpha)\cdot\log(1-16\alpha^2))\\
        \Rightarrow& 1-\log\alpha \geq k\cdot(15.9\alpha^2-8\alpha^2+32\alpha^3)=k\cdot(7.9\alpha^2+32\alpha^3)>0.5k\alpha^2\\
		\Rightarrow& k\leq \frac{2-2\log \alpha}{\alpha^2} = \frac{2+2\log (1/\alpha)}{\alpha^2},
	\end{array}
	$$
	Where the third transition holds since 
	\begin{align*}
		\lefteqn{(\frac{1}{2}+2\alpha)\cdot\log(1+4\alpha)+(\frac{1}{2}-2\alpha)\cdot\log(1-4\alpha)}\\
		&= 4\alpha\cdot\log(1+4\alpha) + (\frac{1}{2}-2\alpha)\paren{\log(1+4\alpha)+\log(1-4\alpha)}\\
		&= 4\alpha\cdot\log(1+4\alpha)+(\frac{1}{2}-2\alpha)\cdot\log(1-16\alpha^2),
	\end{align*}
	and the forth transition holds since $4\alpha\cdot\log(1+4\alpha)+(\frac{1}{2}-2\alpha)\cdot\log(1-16\alpha^2) > 15.9\alpha^2-8\alpha^2+32\alpha^3$ for $\alpha < 0.01$.
	Thus, we conclude that 
	$$
	\Pr_{i\la[n]}\left[\size{\mathbb{E}[f(R) \mid R_i=0]-\mathbb{E}[f(R) \mid R_i = 1]}\geq \alpha\right]\leq \frac{k}{n}\leq \frac{2+2\log (1/\alpha)}{n\alpha^2}.
	$$
\end{proof}
}


\subsection{Channels and Two-Party Protocols}\label{sec:protocol}

\paragraph{Channels.}A channel is simply a distribution of a pair of tuples defined as follows. 
\begin{definition}[Channels]\label{def:channel} A {\sf channel} $C_{(X,U)(Y,V)}$ of size $\isize$ over alphabet $\Sigma$ is a probability distribution over $(\Sigma^\isize \times\zo^\ast) \times(\Sigma^\isize \times\zo^\ast)$. The ensemble $C_{(X,U)(Y,V)}= \set{C_{(X_\pk,U_\pk)(Y_\pk,V_\pk)}}_{\pk\in \N}$ is an $\isize$-size channel ensemble, if for every $\pk\in \N$, $C_{(X_\pk,U_\pk)(Y_\pk,V_\pk)}$ is an $\isize(\pk)$-size channel. %We denote a channel of size one by a \emph{single-bit} channel. 
We refer to $X$ and $Y$ as the {\sf local outputs}, and to $U$ and $V$ as the {\sf views}.	
\end{definition}

We view a  channel as the experiment in which there are two parties $\Ac$ and $\Bc$.  Party $\Ac$ receives ``output'' $X$ and ``view'' $U$, and party $\Bc$ receives ``output'' $Y$ and ``view'' $V$. Unless stated otherwise, the channels we consider are over the alphabet $\Sigma = \oo$. We naturally identify channels with the distribution that characterizes their output.








\subsubsection{Two-Party Protocols}

A two-party protocol $\Pi=(\Ac,\Bc)$ is \ppt if the running time of both parties is polynomial in their input length. We let $\Pi(x,y)(z)$ or $(\Ac(x),\Bc(y))(z)$ denote a random execution of $\Pi$ on a common input $z$, and private inputs $x,y$.%We assume \wlg that a protocol has a common output (part of its transcript).\Jnote{This is not really the case we consider in this paper..}

\begin{definition}[Oracle-aided protocols]\label{def:ChannelAidedProtocol}
	In a two-party protocol $\Pi$ with oracle access to a {\sf protocol} $\Psi$, denoted $\Pi^\Psi$, the parties make use of the \textit{next-message function} of $\Psi$.\footnote{The function that on a partial view of one of the parties, returns its next message.} In a two-party protocol $\Pi$ with oracle access to a {\sf channel} $C_{Z W}$, denoted $\Pi^C$, the parties can jointly invoke $C$ for several times. In each call, an independent pair $(z,w)$ is sampled according to $C_{Z W}$, one party gets $z$, the other gets $w$.
\end{definition}


\begin{definition}[The channel of a protocol]\label{def:ChannlOfProtocol}
	For a no-input two-party protocol $\Pi= (\Ac,\Bc)$, we associate the channel $C_\Pi$, defined by $\C_\Pi= C_{(X, U),(Y, V)}$, where $X$ and $Y$ are the local outputs of $\Ac$ and $\Bc$ (respectively) and
	$U$ and $V$ are the local views of $\Ac$ and $\Bc$ (respectively).
    
	For a two-party protocol $\Pi$ that gets a security parameter $1^\pk$ as its (only, common) input, we associate the channel ensemble $ \set{C_{\Pi(1^\pk)}}_{\pk\in \N}$. 
\end{definition}

\begin{definition}[$(\alpha,\gamma)$-Accurate channel]\label{def:accurate-func}
	A channel $C = C_{(X, U),(Y, V)}$ is {\sf $(\alpha,\gamma)$-accurate for the function $f$}, if $\ppr{C}{\size{\out(V)-f(X,Y)}\leq \alpha}\ge \gamma$, where $\out(V)$ is the designated output.
    A channel ensemble $C_{(X, U),(Y, V)}= \set{C_{(X_\pk, U_\pk),(Y_\pk, V_\pk)}}_{\pk\in \N}$ is  $(\alpha,\gamma)$-accurate for  $f$ if $C_{(X_\pk, U_\pk),(Y_\pk, V_\pk)}$ is $(\alpha(\pk),\gamma(\pk))$-accurate for $f$, for every $\pk \in \N$.
\end{definition}

\subsubsection{Differentially Private Channels}\label{sec:DPChannel}
Differentially private channels are naturally defined as follows:
\begin{definition}[Differentially private channels]\label{def:DPChannel}
	An $n$-size channel $C = C_{(X, U),(Y, V)}$ with $X, Y$ over $\oo^n$ 
	is {\sf$(\eps,\delta)$-differentially private} (denoted $(\eps,\delta)$-$\DP$) if for every $x \in \Supp(X)$ there exists an $n$-size $(\eps,\delta)$-$\DP$ mechanisms $\Mc_x$ such that $(X,Y,U) \equiv (X,Y,\Mc_X(Y))$, and for every $y \in \Supp(Y)$ there exists an $n$-size $(\eps,\delta)$-$\DP$ mechanisms $\Mc_y'$ such that $(X,Y,V) \equiv (X,Y,\Mc_Y'(X))$. In addition, we say that the channel is \emph{uniform} if $X$ and $Y$ are independent random variables uniformly distributed in $\oo^n$. 
\end{definition}

\begin{definition}[Computational differentially private channels]\label{def:CDPChannel}
	An $n$-size channel ensemble $C = \set{C_{(X_\pk, U_\pk),(Y_\pk, V_\pk)}}_{\pk\in\N}$ with $X_\pk, Y_\pk$ over $\oo^n$ 
	is {\sf$(\eps,\delta)$-computationally differentially private} (denoted $(\eps,\delta)$-$\CDP$) if for every ensemble $\set{x_\pk \in \Supp(X_\pk)}_{\pk\in\N}$ there exists an $n$-size $(\eps,\delta)$-\CDP mechanisms ensemble $\set{\Mc_{x_\pk}}_{\pk\in\N}$ such that $(X_\pk,Y_\pk,U_\pk) \equiv (X_\pk,Y_\pk,\Mc_{X_\pk}(Y_\pk))$, for every $\pk\in\N$, and for every ensemble $\set{y_\pk \in \Supp(Y_\pk)}_{\pk\in\N}$ there exists an $n$-size $(\eps,\delta)$-$\CDP$ mechanisms ensemble $\set{\Mc'_{y_\pk}}_{\pk\in\N}$ such that $(X_\pk,Y_\pk,V_\pk) \equiv (X_\pk,Y_\pk,\Mc_{Y_\pk}'(X_\pk))$ for every $\pk\in \N$. In addition, we say that the channel is \emph{uniform} if $X_\pk$ and $Y_\pk$ are independent random variables uniformly distributed in $\{\pm 1\}^n$ for all $\pk\in\N$.
\end{definition}




% \begin{lemma}~\label{lem:dp-sv-source}
% 	Let $P$ be an $\varepsilon$-DP randomized protocol. Let $X$ and $Y$ be independent random variables uniformly distributed in $\{\pm 1\}^n$ and let random variable $\Pi(X,Y)$ denote the transcript of running $P(X,y)$. Then for every $\pi\in Supp(\Pi)$, the random variables corresponding to the inputs conditioned on transcript $\pi$, $X_\pi$ and $Y_\pi$, are independent $e^{-\varepsilon}$-strong SV source.
% \end{lemma}





\subsubsection{Weak Erasure Channel (\WEC)}

\begin{definition}[\WEC]\label{def:WEC}
	A channel $((O_A,V_A), (O_B,V_B))$ with $O_A \in \set{0,1}$ and $O_B \in \set{0,1,\bot}$ is a {\sf weak erasure channel}, denoted $(\alpha,p,q)$-$\WEC$, if:
	\begin{itemize}
		%\item $O_A\in \set{-1,1}$ and $O_B\in \set{-1,1,\bot}$.
		\item Random erasure: $\pr{O_B = \perp} = 1/2$.
		
		\item Agreement: $\pr{O_A\ne O_B\mid O_B\ne \bot}\le \alpha$.
		
		\item Secrecy:
		
		\begin{enumerate}
			\item For every algorithm $\Dc$ it holds that\label{WEC:item:A}
			\begin{align*}
				%\size{\pr{\Ac(O_A,V_A) = 1 \mid O_B \neq \perp} - \pr{\Ac(O_A,V_A) = 1 \mid O_B = \perp}} \le p
				\size{\pr{\Dc(V_A) = 1 \mid O_B \neq \perp} - \pr{\Dc(V_A) = 1 \mid O_B = \perp}} \le p
			\end{align*}
			(Alice doesn't know if $O_B = \perp$.)
			
			\item For every algorithm $\Dc$ it holds that\label{WEC:item:B}
			\begin{align*}
				\pr{\Dc(V_B) = O_A \mid O_B=\bot} \leq \frac{1+q}{2}.
			\end{align*}
			(i.e., if $O_B=\bot$, Bob don't know what is the value of $O_A$).
			
			%\item $SD((O_A U|O_B=\bot),(O_A U|O_B\ne \bot))\le p$ (The sender don't know if $O_B=\bot$).
			
			%\item $SD(V O_A|O_B=\bot,V(-O_A)|O_B=\bot)\le q$ (If $O_B=\bot$, Bob don't know what the value of $O_A$).
		\end{enumerate}
	\end{itemize}
   We say that a channel ensemble $C=\set{C_\pk}_{\pk\in N}$ is a {\sf computational weak erasure channel}, denoted $(\alpha,p,q)$-\CompWEC, if for every \ppt algorithm $\Dc$ and every sufficiently large $\pk\in\N$, $C_\pk$ satisfies the properties stated in the items above, where the secrecy property holds with respect to a \ppt algorithm $\Dc$. A protocol $\Lambda$ is said to be $(\alpha,p,q)$-$\CompWEC$, if the ensemble induces by the protocol (that is, $C=\set{C_{\Lambda(\pk)}}_{\pk\in\N}$) is $(\alpha,p,q)$-$\CompWEC$.  
\end{definition}



\subsubsection{Approximate Weak Erasure Channel (\AWEC)}\label{sec:AWEC}

\begin{definition}[\AWEC]\label{def:AWEC}
	A channel $C = ((O_A,V_A), (O_B,V_B))$ over $([-n,n] \times \zo^*) \times (([-n,n] \cup \bot)  \times \zo^*)$ is an {\sf approximate weak erasure channel}, denoted $(\ell,\alpha,p,q)$-\AWEC if:
	\begin{itemize}
		
		\item Random erasure: $\pr{O_B = \perp} = 1/2$.
		
		\item Accuracy: $\pr{\size{O_A - O_B} > \ell \mid O_B \ne \bot}\le \alpha$.
		
		\item Secrecy:
		
		\begin{enumerate}
			\item For every algorithm $\Dc$ it holds that\label{AWEC:item:A}
			\begin{align*}
				%\size{\pr{\Ac(O_A,V_A) = 1 \mid O_B \neq \perp} - \pr{\Ac(O_A,V_A) = 1 \mid O_B = \perp}} \le p
				\size{\pr{\Dc(V_A) = 1 \mid O_B \neq \perp} - \pr{\Dc(V_A) = 1 \mid O_B = \perp}} \le p
			\end{align*}
			(Alice doesn't know if $O_B=\bot$).
			
			\item For every algorithm $\Dc$ it holds that\label{AWEC:item:B}
			\begin{align*}
				\pr{\size{\Dc(V_B) - O_A} \leq 1000 \ell \mid O_B=\bot} \leq q.
			\end{align*}
			(i.e., if $O_B=\bot$, Bob can't estimate the value of $O_A$ with error $\leq 1000 \ell$).
		\end{enumerate}
	\end{itemize}
     We say that a channel ensemble $C=\set{C_\pk}_{\pk\in N}$ is a {\sf computational approximate weak erasure channel}, denoted $(\ell,\alpha,p,q)$-\CompAWEC, if for every \ppt algorithm $\Dc$ and every sufficiently large $\pk\in\N$, $C_\pk$ satisfies the properties stated in the items above. A protocol $\Gamma$ is said to be $(\ell,\alpha,p,q)$-$\CompAWEC$, if the ensemble induced by the protocol (that is, $C=\set{C_{\Gamma(\pk)}}_{\pk\in\N}$) is $(\ell,\alpha,p,q)$-$\CompAWEC$.  
\end{definition}

We will make use of the following lemma, which shows that for some choices of the parameters, \AWEC implies \WEC. The lemma is proven in \cref{sec:AWEC-to-WEC}.

\begin{lemma}\label{lemma:AWEC-to-WEC}
	For every $\ell> 0$, there exists a \ppt protocol $\Lambda = (\Pc_1,\Pc_2)$ such that given an oracle access to an $(\ell,\alpha,p,q)$-\AWEC $C$, the channel $\tilde{C}$ induced by $\Lambda^C$ is $(\alpha'=\alpha+0.001,\: p' = p ,\:  q' = 1/2 + 2(q+0.01))$-\WEC.
	Furthermore, the proof is constructive in a black-box manner:
	\begin{enumerate}
		\item There exists an oracle-aided \ppt algorithm $\Ec_1$ such that for every channel $C = ((\OA,\VA), (\OB,\VB))$ and algorithm $\Dc$ violating the \WEC secrecy property~\ref{WEC:item:A} of $\tilde{C}$, algorithm $\Ec_1^{\Dc}$ violates the \AWEC secrecy property~\ref{AWEC:item:A} of $C$.
		
		\item There exists an oracle-aided \ppt algorithm $\Ec_2$ such that for every channel $C = ((\OA,\VA), (\OB,\VB))$ and algorithm $\Dc$ violating the \WEC secrecy property~\ref{WEC:item:B} of $\tilde{C}$, algorithm $\Ec_2^{\Dc}$ violates the \AWEC secrecy property~\ref{AWEC:item:B} of $C$.
	\end{enumerate}
\end{lemma}

Since \cref{lemma:AWEC-to-WEC} is constructive, the following is an immediate corollary.
\begin{corollary}\label{cor:CompAWEC to CompWEC}
There exists an oracle aided \ppt protocol $\Lambda$, such that given a protocol $\Gamma$ that induces $(\ell,\alpha,p,q)$-\CompAWEC, it holds that $\Lambda^\Gamma$ is $(\alpha'=\alpha+0.001,\: p' = p ,\:  q' = 1/2 + 2(q+0.01))$-\CompWEC.  
\end{corollary}
\begin{proof}[Proof of \ref{cor:CompAWEC to CompWEC}]
Let $\Lambda$ be the \ppt algorithm guaranteed  by Lemma \ref{lemma:AWEC-to-WEC}. Given an $(\ell,\alpha,p,q)$-\CompAWEC protocol $\Gamma$, we define $\Lambda(\pk)={\Lambda^{\Gamma(\pk)}(\pk)}$. Assume towards a contradiction that $\Lambda$ is not a $(\alpha',p',q')$-\CompWEC. It follows that there exists a \ppt $\Dc$ that for infinity many $\pk\in\N$ contradicts one of the \WEC secrecy properties of channel ensemble $\set{C_{\Lambda(\pk)}}_{\pk\in\N}$. Fix $\pk\in\N$ for which this holds. By Lemma \ref{lemma:AWEC-to-WEC}, there exists a \ppt $\Ec^\Dc$ that for every such $\pk$  contradicts one of the secrecy properties of the channel $C_{\Gamma(\pk)}$. This implies that for infinity many $\pk\in\N$, $\Ec^\Dc$  contradict the secrecy of the channel ensemble $\set{C_{\Gamma(\pk)}}_{\pk\in\N}$, which is a contradiction since this would means that $\Gamma$ is not a $(\ell,\alpha,p,q)$-\CompAWEC.       
\end{proof}



\subsection{Oblivious Transfer (\OT)}

\paragraph{Secure Computation.}
We use the standard notion of securely computing a functionality, \cf  \cite{Goldreich04}.
\begin{definition}[Secure computation]\label{def:SFE}
	A two-party protocol {\sf securely computes a functionality $f$}, if it does so according to the real/ideal paradigm.   We add the term perfectly/statistically/computationally/non-uniform computationally, if the simulator's output is  perfect/statistical/computationally indistinguishable/  non-uniformly indistinguishable from  the real distribution.  The protocol have the above notions of security {\sf against semi-honest  adversaries}, if its security only  guaranteed to holds against an adversary that follows the prescribed protocol.   Finally, for the case of perfectly secure computation, we naturally apply the above notion also to the non-asymptotic case: the protocol with no security parameter perfectly  compute a functionality $f$.
	
	A two-party protocol {\sf securely computes a functionality ensemble $f$ with oracle to a channel $C$}, if it does so according to the above definition when the parties have access to a trusted party computing $C$. All the above adjectives naturally extend to this setting.
\end{definition}

\paragraph{Oblivious Transfer.}
The (one-out-of-two) oblivious transfer functionality is defined as follows.
\begin{definition}[oblivious transfer functionality $f_{\OT}$]\label{def:OTfunc}
	The oblivious transfer functionality over $\zo \times (\zs)^2$ is defined by  $f_{\OT} (i,(\sigma_0,\sigma_1)) = (\perp,\sigma_i)$.
\end{definition}
A protocol is $\ast$ secure OT,   for \\$\ast\in \set{\text{semi-honest statistically/computationally/computationally non-uniform}}$, if it  compute the $f_{\OT}$  functionality with $\ast$ security.





% \begin{definition}[Computational oblivious transfer, semi-honest model]
% A protocol $\Pi=(\Ac,\Bc)$ is a semi-honest 1-out-of-2 computational oblivious transfer (comp-OT) protocol if the following holds. Given a common input $1^{\pk}$, the parties $\Ac$ and $\Bc$ run the protocol $\Pi(1^\pk)$ (in an honest manner) and    
% $\Ac$ outputs $X=(m_1,m_2)\in \zo\times\zo$ and has a view $U$ and $\Bc$ outputs $Y=(i,\hat{m})\in\zo\times\zo$ and has a view $V$, and the following properties are satisfied:
% \begin{enumerate}
%     \item \textbf{Correctness:} 
%     $\pr{\hat{m}\neq m_i}<\negl(\pk).$ 
    
%     \item \textbf{A's Privacy:} For every \ppt $\Dc$ and every sufficiently large $\pk$:
%     $\pr{\Dc(V)=m_{i-1}}<(1+\negl(\pk))/2$
    
%     \item \textbf{B's Privacy:} For every \ppt $\Dc$ and every sufficiently large $\pk$:
%     $\pr{\Dc(U)=i}<(1+\negl(\pk))/2$  
% \end{enumerate}
% \end{definition}

We make use of the following useful results by Wullschleger on oblivious transfer amplification from weak channels.
\begin{theorem}[\cite{Wullschleger09}, from \WEC to statistically secure \OT]\label{thm:WEC TO OT IT}
    There exists an oracle aided protocol $\Pi$ such that the following holds: Given a $(\alpha,p,q)$-\WEC $C$, if $44(\alpha+p)\le 1-q$ then $\Pi^{C}(1^\pk)$ is a semi-honest statistically secure \OT.
\end{theorem}

The following computational version of \cref{thm:WEC TO OT IT} is implicit in \cite{Wullschleger09} and is based on the computational proof explicitly stated in \cite{Wul07} (see Section 6 in \cite{Wullschleger09} for discussion).   

\begin{theorem}[\cite{Wullschleger09,   Wul07}, from \CompWEC to computinally secure \OT]\label{thm:WEC TO OT Comp}
    There exists an oracle aided protocol $\Pi$ such that the following holds: Given a $(\alpha,p,q)$-\CompWEC protocol $\Lambda$, if $44(\alpha+p)\le 1-q$ then $\Pi^{\Lambda}$ is a semi-honest computational secure \OT.
\end{theorem}



% \begin{definition}[Computational 1-out-of-2 Oblivious Transfer, semi-honest model]
% A protocol $\Pi=(\Ac,\Bc)$ is a semi-honest 1-out-of-2 $(\eps,\alpha,\beta)$-oblivious transfer (OT) protocol if the following holds. 

% The parties $\Ac$ and $\Bc$ run the protocol (in an honest manner) and    
% $\Ac$ outputs $X=(m_1,m_2)\in \zo\times\zo$ and has a view $U$ and $\Bc$ outputs $Y=(i,\hat{m})\in\zo\times\zo$ and has a view $V$, and following properties are satisfied:
% \begin{enumerate}
%     \item \textbf{Correctness:} 
%     $\pr{\hat{m}\neq m_i}<\eps.$ 
    
%     \item \textbf{A's Privacy:} For every adversary $\Dc$:
%     $\pr{\Dc(V)=m_{i-1}}<(1+\alpha)/2$
    
%     \item \textbf{B's Privacy:} For every adversary $\Dc$: $\pr{\Dc(U)=i}<(1+\beta)/2$  
% \end{enumerate}
% \end{definition}

\subsection{Algebraizing Deontic Action Logic}
We start the algebraization of  \DAL introducing its signature, i.e., the symbols needed to capture the language of the logic in an algebraic way.


%The first step in algebraizing a logic, and \DAL is no exception, is to view formulas of a logical language as terms of an algebraic language over an appropriate signature.
%To this end, we introduce the following definition.
%We define the signature and the algebraic language that we use in what follows.

\medskip
\begin{definition}\label[definition]{def:signature}
The signature of \DAL is a tuple $\Sigma = \tup{S, \Omega}$ where:
% \begin{enumerate}
% 	\item
		$S = \{\sorta, \sortf\}$; and %, i.e., $S$ has sort symbol $\sorta$ for actions, and sort symbol $\sortf$ for formulas; and
	% \item
		$\textstyle \bigcup \Omega = \{
			{\sqcup}, {\sqcap}, \bar{~}, \iact, \mathsf{1},
			{\lor}, {\land}, {\lnot}, {\bot}, {\top},
			{=}, {\perm}, {\forb}
		\}$.
	The symbols in $\textstyle \bigcup \Omega$ are further categorized into sets
		$\Omega_{\sorta\sorta\sorta}$,
		$\Omega_{\sorta\sorta}$,
		$\Omega_{\sorta}$,
		$\Omega_{\sortf\sortf\sortf}$,
		$\Omega_{\sortf\sortf}$,
		$\Omega_{\sortf}$,
		$\Omega_{\sorta\sorta\sortf}$,
		$\Omega_{\sorta\sortf}$ summarized in \Cref{tab:sig}.
	
	\begin{figure}
		\centering
		\begin{tabular}{r@{~}lr@{~}lr@{~}lr@{~}lr@{~}l}
			\toprule
			&& \multicolumn{8}{c}{operations}
			\tabularnewline
			\cmidrule{3-10}
			& sorts && actions && formulas && equality && normative
			\tabularnewline
			\midrule
			$S$ & $=\{\sorta, \sortf\}$ &
			$\Omega_{\sorta\sorta\sorta}$ & $= \{ {\sqcup}, {\sqcap}\}$ &
			$\Omega_{\sortf\sortf\sortf}$ & $= \{ {\lor}, {\land}\}$ &
			$\Omega_{\sorta\sorta\sortf}$ & $= \{ {=}\}$ &
			$\Omega_{\sorta\sorta\sortf}$ & $= \{ {\perm}, {\forb}\}$
			\tabularnewline
			&&
			$\Omega_{\sorta\sorta}$ & $= \{\bar{~}\}$ &
			$\Omega_{\sortf\sortf}$ & $= \{{\lnot}\}$ &
			\tabularnewline
			&&
			$\Omega_{\sorta}$ & $= \{\iact, \mathsf{1}\}$ &
			$\Omega_{\sortf}$ & $= \{\bot, \top\}$ &
			\tabularnewline
			\bottomrule
		\end{tabular}\\[1em]
		\caption{The Signature used in the algebraization of \DAL.}\label{tab:sig}
	\end{figure}
% \end{enumerate}
\end{definition}
\medskip

In our discussion on the algebraization of \DAL, we take $\Sigma = \tup{S, \Omega}$ to be as in \Cref{def:signature}.
Intuitively, the sort symbols $\sorta$ and $\sortf$ in $S$ categorize actions and formulas, respectively.
In turn, we think of $\Omega$ as containing
symbols for operations on actions, operations on formulas, and
(heterogeneous) operations from actions to formulas.

\medskip
\begin{definition}\label{dal:talg}
	The term algebra $\TAlgebra$ for \DAL uses the set ${\bact}$ as the set of variables of sort $\sorta$, and the empty set $\emptyset$ as the set of variables of sort $\sortf$.  We call this algebra the deontic action term algebra, or the algebraic language of \DAL.
\end{definition}
\medskip

The term algebra $\TAlgebra$ in \Cref{dal:talg} is interpreted over \emph{deontic action algebras}. Deontic action algebras are to \DAL what Boolean algebras are to Classical Propositional Logic, or what Heyting algebras are to Intuitionistic Propositional Logic.
We provide the precise definition of a deontic action algebra in \Cref{definition:deontic:algebra}.

\medskip
\begin{definition}\label[definition]{definition:deontic:algebra}
	A deontic action algebra is an algebra
		$\DAlgebra =
			\langle
				\Algebra[A], \Algebra[F], \E, \P, \F
			\rangle$
	of type $\Sigma$ where:%
		\footnote{
			We use $=$ as the function interpreting `$=$' in $\Algebra[F]$, and $=_{\Algebra[A]}$ and $=_{\Algebra[B]}$ as equality in $\Algebra[A]$ and $\Algebra[B]$, respectively.
		}
		$\Algebra[A] = \tup{A, {\sqcup}, {\sqcap}, \bar{~}, \iact, \uact}$
		and
		$\Algebra[F] = \tup{F, {\lor}, {\land}, {\lnot}, \bot, \top}$ are Boolean algebras,
		and
			$\E$,
			$\P$,
			and
			$\F$,
		satisfy the conditions below
		\begin{multicols}{3}
			\begin{enumerate}[leftmargin=\parindent]
				\item $\P(a {\sqcup} b) \,{=_{\Algebra[F]}}\, {\P(a) {\land} \P(b)}$
				\item $\F(a {\sqcup} b) \,{=_{\Algebra[F]}}\, {\F(a) {\land} \F(b)}$
				\item ${\P(a) {\land} \F(a)} \,{=_{\Algebra[F]}}\, (a \,{=}\, \iact)$
				\item $(a = b) \land \P(a) \preccurlyeq \P(b)$
				\item $(a = b) \land \F(a) \preccurlyeq \F(b)$
				\item[]
				\item ${a \,{=_{\Algebra[A]}}\, b} ~\text{iff}~ {(a \,{=}\, b) \,{=_{\Algebra[F]}}\, \top}$.
			\end{enumerate}
		\end{multicols}
	\noindent
	Let $h: \TAlgebra \to \DAlgebra$ be an interpretation.
	We use $\DAlgebra, h \vDash \tau_1 \doteq \tau_2$ as a shorthand for $h(\tau_1) = h(\tau_2)$.
	In turn, let $\DALVariety$ indicate the class of all deontic action algebras.
	We use $\DALVariety \vDash \tau_1 \doteq \tau_2$ as the universal quantification of $\vDash$ to all deontic action algebras in $\DALVariety$ and all interpretations on these algebras; i.e., $\DALVariety \vDash \tau_1 \doteq \tau_2$ iff $\DAlgebra, h \vDash \tau_1 \doteq \tau_2$, for all $\DAlgebra \in \DALVariety$, and all interpretations $h: \TAlgebra \to \DAlgebra$.
	% The condition $a = b \iff \E(a,b) = \top$
	% is a pair of quasi-identities.
	% This makes the class $\DALVariety$ of all deontic action algebras a quasi-variety.
\end{definition}
\medskip

The next two results are immediate.

\medskip
\begin{proposition}\label[proposition]{pro:dal:act2form}
	It follows that $\DALVariety \vDash \alpha \doteq_{\sorta} \beta$ iff $\DALVariety \vDash (\alpha = \beta) \doteq_{\sortf} \top$.
\end{proposition}
\medskip

\begin{proposition}\label[proposition]{pro:dal:qvariety}
	The class $\DALVariety$ of all deontic action algebras is a quasi-variety.
\end{proposition}
\begin{proof}
	It suffices to show that the conditions in the definition of a deontic action algebra can be captured by equations, or quasi-equations.
	The interesting cases are:
	\smallskip
	\begin{enumerate}[leftmargin=\parindent]
		\item $\P(a \sqcup b) =_{\Algebra[F]} {\P(a) \land \P(b)}$ expressed as $\P(a \sqcup b) \doteq_{\sortf} \P(a) \land \P(b)$;
		\item ${\P(a) \land \F(a)} =_{\Algebra[F]} \E(a,\iact)$ expressed as ${\P(a) \land \P(b)} \doteq_{\sortf} {a = \iact}$;
		\item $(a = b) \land \P(a) \preccurlyeq \P(b)$ expressed as $((a = b) \land \P(a)) \lor \P(b) \doteq_{\sortf} \P(b)$; and
		\item ${a =_{\Algebra[A]} b} ~\text{iff}~ {(a = b) =_{\Algebra[F]} \top}$ expressed as the quasi-equations
			${a \doteq_{\sorta} b} \To {(a = b) \doteq_{\sortf} \top}$, and
			${(a = b) \doteq_{\sortf} \top} \To {a \doteq_{\sorta} b}$. \qedhere
	\end{enumerate}
\end{proof}
\medskip

The definition of a deontic action algebra in \Cref{definition:deontic:algebra} draws on ideas and terminology from Pratt's dynamic algebras~\cite{Pratt:1991}. We present the general structure of a deontic action algebra in a form slightly different from the general treatment of many-sorted algebras in \Cref{section:basics}. In doing this we wish to highlight the modular nature of deontic action algebras. In \Cref{section:new:dals}, we leverage this modularity to introduce variants of \DAL by considering different algebraic structures for each component of the deontic action algebra. 
Finally, notice that, as made clear in \Cref{pro:dal:qvariety}, our treatment of equality in the logic results in the class of deontic action algebras being a quasi-variety.
The result in \Cref{pro:dal:act2form} tells us we can dispense explicitly referring to equations on actions as they are also captured as particular equations on formulas via equality in the logic.

\medskip
\begin{example}
	\Cref{ex:deontic:algebra} depicts the deontic action algebra ${\DAlgebra = \langle \Algebra[A], \Algebra[2], \E, \P, \F \rangle}$ where: the algebra $\Algebra[A]$ of actions is the free Boolean algebra on the set of generators $\{a,b\}$.
	In $\DAlgebra$, the functions $\P$ and $\F$ are defined as:
	\begin{align*}
		\P(x) &=
			{\begin{cases}
				1 & \text{if } x \preccurlyeq \bar{b} \\
				0 & \text{otherwise.}
			\end{cases}}
		&
		\F(x) &=
			{\begin{cases}
				1 & \text{if } x \preccurlyeq b \\
				0 & \text{otherwise.}
			\end{cases}}
	\label{eq:ex:pf}
	\end{align*}
	% The dashed arrows from the graph of $\Algebra[A]$ to the graph of $\Algebra[F]$ show the elements of $|\Algebra[A]|$ that $\P$ maps to $1$.
	In \Cref{ex:deontic:algebra}, the elements of $|\Algebra[A]|$ that $\P$ and $\F$ map to $\top$ are indicated with green and red, respectively.
	To avoid overcrowding the drawing, we have chosen not to highlight the elements these operations do not map to $\top$.
	In \Cref{ex:deontic:algebra} also, the sets $P$ and $F$ indicate which actions are permitted and which ones are forbidden.
	Note that both sets form an ideal in $\Algebra[A]$ whose intersection contains only the $\iact$ element of the algebra.
	It can easily be inferred from this example that: if $\P(x) = \top$ for all $x \in |\Algebra[A]|$, then, $\F(x) = \bot$ for all $\iact \prec x \in |\Algebra[A]|$.
	Similarly, if $\F(x) = \top$ for all $x \in |\Algebra[A]|$, then, $\P(x) = \bot$ for all $\iact \prec x \in |\Algebra[A]|$.
	These cases are known as \emph{deontic heaven} and \emph{deontic hell}, respectively.
	We will discuss them later on.
\end{example}
\medskip

\begin{figure}
	\centering
	\includegraphics[width=0.5\textwidth]{deontic-algebra.pdf}\\[.5em]
	\caption{A Deontic Action Algebra.}\label{ex:deontic:algebra}
\end{figure}

\begin{example}
	Let $\DAlgebra$ be the deontic action algebra in \Cref{ex:deontic:algebra}, and $\mathsf{drinking}$, $\mathsf{driving}$, and $\mathsf{parking}$, be basic action symbols.
	In addition, let $h: \TAlgebra \to \DAlgebra$ be an interpretation s.t.:
		$h_{\sorta}(\mathsf{drinking}) = b$,
		$h_{\sorta}(\mathsf{driving}) = a$, and
		$h_{\sorta}(\mathsf{parking}) = \bar{b}$.
	It follows that:

   \begin{multicols}{2}
   \begin{enumerate}
      \item $h(\overline{\mathsf{parking}} = \mathsf{driving}) =_{\Algebra[F]} \top$
      \item $h(\forb(\mathsf{drinking} \sqcap \mathsf{driving})) =_{\Algebra[F]} \top$
      \item $h(\perm(\mathsf{drinking} \sqcap \mathsf{parking})) =_{\Algebra[F]} \top$
      \item $h(\perm(\mathsf{driving} \sqcup \mathsf{parking})) \neq_{\Algebra[F]} \top$.
   \end{enumerate}
   \end{multicols}

   \noindent In brief, the deontic action algebra $\DAlgebra$ may be understood as the algebraic version of the deontic model $\DeonticModel$ in \Cref{section:dal:semantics}.
\end{example}

The following proposition shows the ideals in the deontic action algebra in \Cref{ex:deontic:algebra} are indeed a distinguishing characteristic of the operations of permission and prohibition.

\medskip
\begin{proposition}\label[proposition]{prop:dal:ideal}
	Let $\DAlgebra = \tup{\Algebra[A], \Algebra[F], \E, \P, \F}$ be a deontic action algebra. The pre-image $P$ of $\top$ under $\P$, as well as the preimage $F$ of $\top$ under $\F$, are ideals in $\Algebra[A]$ s.t.\ ${{P \cap F} = \{\iact\}}$.
\end{proposition}
\begin{proof}
	The result is obtained from the following:
		\medskip
		\begin{enumerate}%[(i)]
			\setlength{\itemsep}{5pt}
			\item
			For all $\{a,b\} \subseteq P$, ${a \sqcup b} \in P$.
			To see why, let $\{a,b\} \subseteq P$.
			Then, $\P(a) = \P(b) = \top$, and $\P(a) \land \P(b) = \top$.
			The properties of $\P$ in \Cref{definition:deontic:algebra} ensure $\P(a) \land \P(b) = \P(a \sqcup b)$.
			This implies $\P(a \sqcup b) = \top$; and so $(a \sqcup b) \in P$.

			\item
			For all $a \in P$ and $b \in |\Algebra[A]|$, ${(a \sqcap b)} \in P$.
			To see why, let $a \in P$ and $b \in |\Algebra[A]|$.
			We know $\P(a) = \top$ and $a = {a \sqcup (a \sqcap b)}$.
			This means $\P({a \sqcup (a \sqcap b)}) = \top$.
			The properties of $\P$ in \Cref{definition:deontic:algebra} ensure $\P(a \sqcup (a \sqcap b)) = \P(a) \land \P(a \sqcap b)$.
			This means, $\P(a) \land \P(a \sqcap b) = \top$.
			From our supposition, $\P(a \sqcap b) = \top$; and so $(a \sqcap b) \in P$.

			\item
			The arguments in 1 and 2 remain true if we replace $P$ and $\P$ for $F$ and $\F$, resp.

			\item
			${P \cap F} = \{ \iact \}$.
			To see why, note that $\P(\iact) = \F(\iact) = \top$; and so $\{\iact\} \subseteq {P \cap F}$.
			In turn, consider an arbitrary $a \in (P \cap F)$.
			Then, $\P(a) = \F(a) = \top$.
			This implies $\P(a) \land \F(a) = \top$, and so $(a = \iact) =_{\Algebra[F]} \top$.
			The `iff' condition in \Cref{definition:deontic:algebra} ensures $a =_{\Algebra[A]} \iact$.
			Since $a$ is arbitrary, the last step tells us that any element in ${P \cap F}$ is equal to $\iact$.
			Therefore, ${P \cap F} \subseteq \{ \iact \}$. \qedhere
		\end{enumerate}
\end{proof}

We proceed to connect the deontic action algebras in $\DALVariety$ with the theorems of $\DAL$.

\medskip
\begin{theorem}[Soundness]\label[theorem]{theorem:soundness}
	If $\varphi$ is a theorem of \DAL, then, $\DALVariety \vDash {\varphi \doteq \top}$.
\end{theorem}
\begin{proof} %(Sketch)
	Let $\DAlgebra \in \DALVariety$ and $h: \TAlgebra \to \DAlgebra$ be any interpretation.
	We continue by induction on the length of the proof of $\varphi$.
	We prove the more interesting cases;  others are similar.

	\medskip
	\begin{enumerate}[leftmargin=\parindent]
		\setlength{\itemsep}{5pt}
		
		\item
		$h_{\sortf}(\perm(\alpha\sqcup\beta) \liff (\perm(\alpha) \land \perm(\beta)))
			=_{\Algebra[F]} \top$.
		The result follows from items (a)--(c) below.

			\medskip
			\begin{enumerate}[leftmargin=\parindent]
				\setlength{\itemsep}{5pt}
				\item
				\begin{description}[leftmargin=\parindent]
					\item[]
					$h_{\sortf}(\perm(\alpha\sqcup\beta) \liff (\perm(\alpha) \land \perm(\beta))) =_{\Algebra[F]}$
					\item[]
					$h_{\sortf}(
						(\lnot \perm(\alpha\sqcup\beta) \lor (\perm(\alpha) \land \perm(\beta)))
						\land
						(\lnot (\perm(\alpha) \land \perm(\beta)) \lor \perm(\alpha\sqcup\beta))
						)=_{\Algebra[F]}$
					\item[]
					$h_{\sortf}(
						\lnot \perm(\alpha\sqcup\beta) \lor (\perm(\alpha) \land \perm(\beta)))
					\land
						h_{\sortf}(
						(\lnot (\perm(\alpha) \land \perm(\beta)) \lor \perm(\alpha\sqcup\beta)))$.
				\end{description}

				\item 
				\begin{description}[leftmargin=\parindent]
					\item[]
					$h_{\sortf}(
						\lnot
							\perm(\alpha\sqcup\beta)
							\lor
							(\perm(\alpha) \land \perm(\beta))) =_{\Algebra[F]}$
					\item[]
						$\lnot
							\perm(h_{\sorta}(\alpha \sqcup \beta))
							\lor
							(\perm(h_{\sorta}(\alpha)) \land \perm(h_{\sorta}(\beta))) =_{\Algebra[F]}$
					\item[]
						$\lnot
							\perm(h_{\sorta}(\alpha \sqcup \beta))
							\lor
							(\perm(h_{\sorta}(\alpha) \sqcup h_{\sorta}(\beta))) =_{\Algebra[F]}$
							\dotfill \Cref{definition:deontic:algebra}(1)
					\item[]
						$\lnot
							\perm(h_{\sorta}(\alpha \sqcup \beta))
							\lor
							(\perm(h_{\sorta}(\alpha \sqcup \beta))) =_{\Algebra[F]} \top$.
				\end{description}

				\item $h_{\sortf}(
					\lnot(\perm(\alpha) \land \perm(\beta))
					\lor
					\perm(\alpha\sqcup\beta)) =_{\Algebra[F]} \top$ is similar to (b).
			\end{enumerate}

		\item
		$h_{\sortf}((\perm(\alpha) \land \forb(\alpha)) \to (\alpha = \iact))
			=_{\Algebra[F]} \top$.
		Then,

			\medskip
			\begin{description}[leftmargin=\parindent]
				\item[]
				$h_{\sortf}((\perm(\alpha) \land \forb(\alpha)) \to (\alpha = \iact)) =_{\Algebra[F]}$
				\item[]
				$h_{\sortf}(
					\lnot (\perm(\alpha) \land \forb(\alpha)) \lor (\alpha = \iact)) =_{\Algebra[F]}$
				\item[]
				$\lnot
					(\perm(h_{\sorta}(\alpha)) \land \forb(h_{\sorta}(\alpha)))
					\lor
					h_{\sortf}(\alpha = \iact) =_{\Algebra[F]}$
				\item[]
				$\lnot
					(\perm(h_{\sorta}(\alpha)) \land \forb(h_{\sorta}(\alpha)))
					\lor
					h_{\sortf}(\alpha = \iact) =_{\Algebra[F]}$
				\item[]
				$\lnot
					(h_{\sorta}(\alpha) = \iact)
					\lor
					h_{\sortf}(\alpha = \iact) =_{\Algebra[F]}$ \dotfill \Cref{definition:deontic:algebra}(3)
				\item[]
				$\lnot
					h_{\sortf}(\alpha = \iact)
					\lor
					h_{\sortf}(\alpha = \iact) =_{\Algebra[F]} \top$.
			\end{description}

		\item
		$h_{\sortf}(((\alpha = \beta) \land \perm(\alpha)) \to \perm(\beta))
			=_{\Algebra[F]} \top$.
			Then,
			
			\medskip
			\begin{description}[leftmargin=\parindent]
				\item[]
					$h_{\sortf}(((\alpha \,{=}\, \beta) {\land} \perm(\alpha)) \to \perm(\beta)) =_{\Algebra[F]}$
				\item[]
					$\lnot
						h_{\sortf}((\alpha \,{=}\, \beta) {\land} \perm(\alpha))
					\lor
					\perm(h_{\sorta}(\beta)) =_{\Algebra[F]}$
				\item[]
					$\lnot
						h_{\sortf}((\alpha \,{=}\, \beta) {\land} \perm(\alpha))
					\lor
						((h_{\sorta}(\alpha) \,{=}\, h_{\sorta}(\beta)) {\land} \perm(h_{\sorta}(\alpha)))
						\lor
						\perm(h_{\sorta}(\beta)) =_{\Algebra[F]}$
						\dotfill \Cref{definition:deontic:algebra}(4)
				\item[]
					$\lnot
						h_{\sortf}((\alpha \,{=}\, \beta) {\land} \perm(\alpha))
					\lor 
						h_{\sortf}((\alpha \,{=}\, \beta) {\land} \perm(\alpha))
					\lor
						\perm(h_{\sorta}(\beta)) =_{\Algebra[F]} \top$.
			\end{description}

			\smallskip
			\noindent
			The result in (3.) is a particular case of the axioms E2 in \Cref{dal:axioms}.
			Other instances can be proven by induction on the size of the formula $\varphi$.
			\qedhere
	\end{enumerate}
\end{proof}

\Cref{theorem:soundness} implies that not every formula of \DAL is provable in the logic.
In particular, non-theorems are not provable.
To see why, consider a theorem $\varphi$, the deontic action algebra $\DAlgebra = \langle \Algebra[A], \Algebra[2], \E, \P, \F \rangle$, and any interpretation $h: \TAlgebra \to \DAlgebra$.
From \Cref{theorem:soundness}, we have $h_{\sortf}(\varphi) = \top$.
Since $h$ is a homomorphism, $h_{\sortf}(\lnot \varphi) = \iact$.
Using the contrapositive of \Cref{theorem:soundness}, $\lnot \varphi$ is not a theorem; i.e., it is not provable.

The converse of \Cref{theorem:soundness}, i.e., the algebraic completeness of \DAL, requires us to show that every non-theorem $\varphi$ of \DAL is falsified in some deontic action algebra $\DAlgebra$ (i.e., there is an interpretation $h: \TAlgebra \to \DAlgebra$ s.t.\ $h_{\sortf}(\varphi) \neq \top$).
We arrive at this result introducing an appropriate notion of congruence, and constructing a quotient algebra via this congruence. 

\medskip
\begin{proposition}\label[proposition]{prop:congruence}
	Let $\TAlgebra$ be the deontic term algebra, and ${\cong_{\sorta}} \subseteq |\TAlgebra|_{\sorta} \times |\TAlgebra|_{\sorta}$ and ${\cong_{\sortf}} \subseteq |\TAlgebra|_{\sortf} \times |\TAlgebra|_{\sortf}$ be s.t.: 1.~$\alpha \cong_{\sorta} \beta$ iff $\alpha = \beta$ is a theorem, and 2.~$\varphi \cong_{\sortf} \psi$ iff $\varphi \liff \psi$ is a theorem.
	It follows that $\cong_{\sorta}$ and $\cong_{\sortf}$ define a congruence $\cong$ on $\TAlgebra$.
\end{proposition}
\medskip

\begin{proposition}\label[proposition]{prop:lindenbaum}
	The quotient of the deontic action term algebra $\TAlgebra$ under $\cong$ is a structure
		$\LTAlgebra = \tup{\Algebra[A], \Algebra[F], \E, \P, \F}$
	where:
		1.~$|\Algebra[A]| = {\act/{\cong_{\sorta}}}$,
		2.~$|\Algebra[F]| = {\form/{\cong_{\sortf}}}$, and
		3.~the operations in $\LTAlgebra$ are those induced by the equivalence classes in $\cong$.
		% \medskip
		% \begin{enumerate}[leftmargin=\parindent]
		% 	\item
		% 		$\Algebra[A]
		% 			= \tup{{\act/{\cong_{\sorta}}}, {\sqcup}, {\sqcap}, {\bar{~}}, {\iact}, {\uact}}$
		% 	\item
		% 		$\Algebra[F]
		% 			= \tup{{\form/{\cong_{\sortf}}}, {\lor}, {\land}, {\lnot}, {\bot}, {\top}}$
		% \end{enumerate}
		% \medskip
	It follows that $\LTAlgebra \in \DALVariety$.
\end{proposition}
\begin{proof}%[Sketch]
	It is clear that $\Algebra[A]$ and $\Algebra[F]$ are Boolean algebras.
	Let us use $\check{\_}$ to indicate the operations in $\LTAlgebra$ induced by $\cong$, and to separate them from the corresponding symbols.
	The result is concluded if $\check{\E}$, $\check{\P}$, and $\check{\F}$ satisfy the conditions in \Cref{definition:deontic:algebra}.
  	We prove some interesting cases only.
	% We drop the subscript $\cong$ unless it is strictly necessary to improve readability.

	\medskip
	\begin{enumerate}[leftmargin=\parindent]
		\setlength{\itemsep}{7pt}
		
		\item%
		{$\check{\P}([\alpha \sqcup \beta]_{\cong_{\sorta}}) =_{\Algebra[F]} \check{\P}([\alpha]_{\cong_{\sorta}}) \land \check{\P}([\beta]_{\cong_{\sorta}})$}
		% See (a) below.

			\medskip
			\begin{description}[leftmargin=\parindent]
				\item[]
					$\check{\P}([\alpha \sqcup \beta]_{\cong_{\sorta}}) =_{\Algebra[F]}
					[\perm(\alpha \sqcup \beta)]_{\cong_{\sortf}} =_{\Algebra[F]}$
				\item[]
					$[\perm(\alpha) \land \perm(\beta)]_{\cong_{\sortf}} =_{\Algebra[F]}$
					\dotfill \Cref{dal:axioms}(D1)
				\item[]
					$[\perm(\alpha)]_{\cong_{\sortf}} \land [\perm(\beta)]_{\cong_{\sortf}} =_{\Algebra[F]}
					\check{\P}([\alpha]_{\cong_{\sorta}}) \land \check{\P}([\beta]_{\cong_{\sorta}})$.
			\end{description}

		\item%
		{$\check{\P}([\alpha]_{\cong_{\sorta}}) \land \check{\F}([\alpha]_{\cong_{\sorta}}) =_{\Algebra[F]}
			[\alpha]_{\cong_{\sorta}} \mathrel{\check{=}} \iact$}
		% See below.

			\medskip
			\begin{description}[leftmargin=\parindent]
				\item[]
					$
					\check{\P}([\alpha]_{\cong_{\sorta}})
					\land
					\check{\F}([\alpha]_{\cong_{\sorta}})
					=_{\Algebra[F]}
					[\perm(\alpha)]_{\cong_{\sortf}}
					\land
					[\forb(\alpha)]_{\cong_{\sortf}}
					=_{\Algebra[F]}$
				\item[]
					$
					[\perm(\alpha) \land \forb(\alpha)]_{\cong_{\sortf}}
					=_{\Algebra[F]}
					$
				\item[]
					$[\alpha = \iact]_{\cong_{\sortf}} =_{\Algebra[F]}$
					\dotfill \Cref{dal:axioms}(D3)
				\item[]
					$[\alpha]_{\cong_{\sorta}} \mathrel{\check{=}} \iact$.
			\end{description}
		
		\item%
		{$[\alpha]_{\cong_{\sorta}} =_{\Algebra[A]} [\beta]_{\cong_{\sorta}}$ iff
			$[\alpha]_{\cong_{\sorta}} \mathrel{\check{=}} [\beta]_{\cong_{\sorta}} =_{\Algebra[F]} \top$}.
		% See below.

			\medskip
			\begin{description}[leftmargin=\parindent]
				\setlength{\itemsep}{2pt}
				\item[Left-to-right:]
				Let $[\alpha]_{\cong_{\sorta}} =_{\Algebra[A]} [\beta]_{\cong_{\sorta}}$.
				This assumption implies, by definition, that ${\alpha = \beta}$ is a theorem.
				Immediately, ${(\alpha = \beta) \liff \top}$ is also a theorem.
				But this means, $[\alpha = \beta]_{\cong_{\sortf}} =_{\Algebra[F]} \top$.
				Thus, $[\alpha]_{\cong_{\sorta}} \mathrel{\check{=}} [\beta]_{\cong_{\sorta}} =_{\Algebra[F]} \top$.

				\item[Right-to-left:]
				Similarly, let $[\alpha]_{\cong_{\sorta}} \mathrel{\check{=}} [\beta]_{\cong_{\sorta}} =_{\Algebra[F]} \top$.
				Then, $[\alpha = \beta]_{\cong_{\sortf}} = \top$.
				This means $\alpha = \beta$ is a theorem.
				And so, $[\alpha]_{\cong_{\sorta}} =_{\Algebra[A]} [\beta]_{\cong_{\sorta}}$. \qedhere
			\end{description}
	\end{enumerate}	
\end{proof}

We call the quotient algebra $\LTAlgebra$ in \Cref{prop:lindenbaum} the Lindenbaum-Tarski deontic action algebra.
In brief, $\LTAlgebra$ is a canonical algebra that captures theoremhood in \DAL.
From this observation, we obtain the following result.

\medskip
\begin{theorem}[Completeness]\label[theorem]{theorem:completeness}
	If $\DALVariety \vDash {\varphi \doteq \top}$, then, $\varphi$ is a theorem.
\end{theorem}
\begin{proof}
	We show that if $\varphi$ is not a theorem, then $\DALVariety \nvDash {\varphi \doteq \top}$.
	Let $\varphi$ be a non-theorem.
	From the definition of $\cong$, $[\varphi]_{\cong_{\sortf}} \neq_{\Algebra[F]} \top$.
	Define a function $h: {\bact \to {\Algebra[A]/{\cong}}}$ that sends each $\mathsf{a}_i \in \bact$ to the equivalence class $[\mathsf{a}_i]_{\cong_{\sorta}}$.
	The function $h$ extends uniquely to an interpretation $\check{h}: \TAlgebra \to \LTAlgebra$ such that $\check{h}(\varphi) \neq_{\Algebra[F]} [\varphi]_{\cong_{\sortf}}$.
	Therefore, $\DALVariety \nvDash {\varphi \doteq \top}$.
	%\qed
\end{proof}

% \Cref{cor:completeness} is obtained via a standard argument in algebraic logic.

\begin{corollary}\label[corollary]{cor:completeness}
	$\DALVariety \vDash {\varphi \doteq \top}$ implies $\varphi$ is a tautology.
\end{corollary}
\begin{proof}
	Immediate from \Cref{th:segerber:completeness,theorem:completeness}.
\end{proof}

% As usual then, the Lindenbaum-Tarski algebra can be seen as a canonical (algebraic) model providing counterexamples to non-valid formulas.

\subsection{Deontic Action Algebras and Deontic Action Models}

Interestingly, the algebraization of \DAL enjoys a Stone-type representation result connecting the algebraic semantics using deontic action algebras with the original semantics using deontic action models.
This connection provides us with another completeness result for the theorems of $\DAL$.

Recall that Stone's representation theorem establishes that every Boolean algebra is isomorphic to a field of sets~\cite{Stone36}. Such a result reveals a tight connection between the properties of an abstract structure with those of a \emph{concrete} one (a collection of sets).
This is also true for deontic action algebras.
We begin by introducing the definition of a concrete deontic action algebra.

%Just as Boolean algebras made of sets (i.e., fields of sets) are referred to as \emph{concrete} in algebraic logic, concrete deontic action algebras are deontic action algebras whose algebras of actions and formulas are fields of sets.
%This is made precise immediately below.

\medskip
\begin{definition}
	A deontic action algebra $\DAlgebra = \langle \Algebra[A], \Algebra[F],  \E, \P, \F \rangle$ is \emph{concrete} iff $\Algebra[A]$ and $\Algebra[F]$ are fields of sets.
  	Let $\DALVariety(0)$ be the class of concrete deontic algebras, for equations of the appropriate sort, we use $\DALVariety(0) \vDash \tau_1 \doteq \tau_2$ as the analogue of $\DALVariety \vDash \tau_1 \doteq \tau_2$ in \Cref{definition:deontic:algebra}.
\end{definition}
\medskip

We prove that validity in deontic action algebras reduces to validity in concrete deontic algebras.
In this way, concrete deontic algebras enable us to connect the algebraic semantics of \DAL with Segerberg's original semantics via Stone's duality.


\medskip
\begin{theorem}\label[theorem]{theorem:reducibility}
	It follows that $\DALVariety(0) \vDash \varphi \doteq \top$ iff $\DALVariety \vDash \varphi \doteq \top$.
\end{theorem}
\begin{proof}
	The left-to-right direction is straightforward.
	The proof for the right-to-left direction is by contradiction.
	Assume that $\DALVariety(0) \vDash \varphi \doteq \top$ and $\DALVariety \nvDash \varphi \doteq \top$.
	This means that we have a deontic action algebra $\DAlgebra = \langle \Algebra[A], \Algebra[F],  \E, \P, \F \rangle$ and an interpretation $h: \TAlgebra \to \DAlgebra$ s.t.\ $h_{\sortf}(\varphi) \neq_{\Algebra[F]} \top$.
	Via the Stone duality result for Boolean algebras, we can construct a concrete deontic action algebra $\DAlgebra' = \langle \Algebra[A]', \Algebra[F]',  \E', \P', \F' \rangle$ that is isomorphic to $\DAlgebra$.
	Moreover, we can define an interpretation $h': \TAlgebra \to \DAlgebra'$ s.t.\ $h'(a_i) = \varphi_{\Algebra[A]'}(h(a_i))$ (with  $\varphi_{\Algebra[A]'}$ being the Stone isomorphism for $\Algebra[A]'$).
	This construction ensures $h'(\varphi) \neq_{\Algebra[F]'} \top$; contradicting the original assumption that $\DALVariety(0) \vDash \varphi \doteq \top$.
\end{proof}

We can now link deontic action models with concrete deontic action algebras.


\medskip
\begin{definition}\label[definition]{def:mod2alg}
	Let $\DeonticModel = \langle E, P, F \rangle$ be a deontic action model, $v:\bact \to 2^{E}$ be a valuation on $\DeonticModel$, and $A = \set{v(\mathsf{a}_i)}{\mathsf{a}_i \in \bact}$.
	Define a concrete deontic action algebra $\algebra(\DeonticModel, v) = \langle \Algebra[A], \Algebra[2], \E, \P, \F \rangle$ where:
	\begin{align*}
			\Algebra[A] &= \tup{2^A, {\cup}, {\cap}, {}^{\complement}, \emptyset, A} &
			(a = b) =_{\Algebra[2]} \top &~\text{iff}~ a =_{\Algebra[A]} b &
			\P(a) = \top &~\text{iff}~ a \subseteq P \\
			&&&& \F(a) = \top &~\text{iff}~ a \subseteq F.
		\end{align*}
	Define also the interpretation $h: \TAlgebra \to \algebra(\DeonticModel, v)$ as the unique extension of $v$.
\end{definition}
\medskip

Similarly, concrete deontic algebras give rise to deontic action models.


\medskip
\begin{definition}\label{def:alg2mod}
	Let $\DAlgebra = \langle \Algebra[A], \Algebra[F],  \E, \P, \F \rangle$ be a concrete deontic algebra, $h: \TAlgebra \to \DAlgebra$ be an interpretation.
	Define a deontic action model $\model(\DAlgebra, h) = \langle E, P, F \rangle$ where:
	\begin{align*}
		E &= |\Algebra[A]| &
		P &= \bigcup \set{a}{\P(a) =_{\Algebra[F]} \top} &
		F &= \bigcup \set{a}{\F(a) =_{\Algebra[F]} \top}.
	\end{align*}
	Define also a valuation $v$ on $\model(\DAlgebra, f)$ as the restriction of $h$ to $\bact$.
\end{definition}
\medskip

If seen as operators, $\model$ and $\algebra$ are inverses of each other.


\medskip
\begin{theorem}\label[theorem]{theorem:inverses}
	It follows that:
		$\algebra(\model(\DAlgebra,v),h)\!=\!\DAlgebra$; and
		$\model(\algebra(\DeonticModel,v),h)\!=\!\DeonticModel$.
\end{theorem}
\medskip

In light of \Cref{theorem:inverses}, we obtain the following result.


\medskip
\begin{corollary}\label[corollary]{theorem:models-and-algebras}
		It follows that:
		$\DeonticModel, v \Vdash \varphi$ iff $\algebra(\DeonticModel, v), h \vDash \varphi \doteq \top$; and
		$\DAlgebra, h \vDash \varphi \doteq \top$ iff $\model(\DAlgebra,h), v \vDash \varphi$.
\end{corollary}
\medskip

% \begin{theorem}\label[theorem]{theorem:algebras-to-models}
% $\DAlgebra, f \vDashcurly \varphi \doteq \top \mbox{ iff } \model(\DAlgebra,f), {f{\restriction}_{\bact}} \vDash \varphi$.
% \end{theorem}


\medskip

The results in \Cref{theorem:inverses,theorem:models-and-algebras} enable us to prove the completeness of $\DAL$ w.r.t.\ Segerberg's original deontic models entirely in an algebraic way.

\medskip
\begin{theorem} It follows that $\varphi$ is a theorem iff it is a tautology.
\end{theorem}
\begin{proof}
	Suppose that $\varphi$ is a theorem.
	From \Cref{cor:completeness}, $\DALVariety \vDash \varphi \doteq \top$.
	From \Cref{theorem:reducibility}, $\DALVariety(0) \vDash \varphi \doteq \top$.
	From \Cref{theorem:models-and-algebras}, $\varphi$ is a tautology.
	Thus, $\varphi$ is a theorem implies $\varphi$ is a tautology.
	Using these results in the inverse order we obtain $\varphi$ is a tautology implies $\varphi$ is a theorem.
\end{proof}

\section{Other Deontic Action Logics}\label{section:new:dals}
One of the main benefits of our algebraic treatment of $\DAL$ is that it can be extended using standard algebraic tools to cope with different versions of the logic.  In this section, firstly, we show how classical variations of 
$\DAL$ can be algebraically captured by standard algebraic constructions,  that is, equations, sub-algebras, and generated algebras.  One interesting point of these extensions is that the soundness and completeness properties of these extensions can be obtained by applying similar constructions to the Lindenbaum algebra presented in earlier sections.  Secondly, we consider intuitionistic versions of the logic by replacing the Boolean components of the algebras by Heyting algebras.  As far as we are aware, intutionistic deontic action logics have not been considered before in the literature. 
%We take advantage of the characteristics of our algebraic treatment of $\DAL$ and characterize some extensions and possible variations of \DAL.

\subsection{Previously Proposed Variants of \DAL}\label{section:dals}
%\subsection{Algebraizing Other Deontic Action Logics}\label{section:dals}

Segerberg's foundational work~\cite{Segerberg1982} laid the groundwork for a family of closely related deontic action logics. Building on this foundation, the five systems introduced in~\cite{Trypuz15} are particularly interesting as they address specific open issues in the field of Deontic Logic \textemdash such as the \emph{principle of deontic closure}.
We show how our algebraic framework can be easily extended to characterize each of these logics, showcasing the adaptability and versatility of deontic action algebras.
In the rest of this section, we use $\mathsf{a}$ to indicate a basic action symbol and $a$ to indicate its corresponding interpretation in an algebra.

% To introduce the algebraic counterparts of each $\NDAL{i}$, we need two preliminary definitions: that of a \emph{free deontic action algebra}, and that of an \emph{atomic deontic action algebra}. These are straightforward extensions of the corresponding concepts in Boolean algebras.
%
% \begin{definition}
% 	A deontic action algebra $\DAlgebra = \langle \Algebra[A], \Algebra[F], \E, \P, \F \rangle$ is freely (resp. finitely) generated by a subset $G \subseteq A$ of generators iff $\Algebra[A]$ is freely (resp. finitely) generated by $G$.
% 	We say that $\DAlgebra$ is atomic if $\Algebra[A]$ is atomic.
% \end{definition}
%
% We are now in position to introduce each deontic logic \NDAL{i} together with its algebraic counterpart.

The first of the five systems in~\cite{Trypuz15}, here called \NDAL{1}, is obtained from \DAL by adding the set $\set{{\forb(\mathsf{a}) \lor \perm(\mathsf{a})}}{\mathsf{a} \in \bact}$ of formulas as additional axioms.
Intuitively, this new set of axioms aims to capture the so-called \emph{principle of deontic closure}\textemdash what is not forbidden is permitted\textemdash at the level of basic actions (i.e., action generators).
The algebraic counterpart of \NDAL{1} is determined by the class of deontic action algebras whose algebra of actions is generated by a set of generators s.t.\ the condition $\F(a) \lor \P(a) =_{\Algebra[F]} \top$ holds for every generator $a$.

% \begin{definition}\label[definition]{def:dal:1}
% 	The class $\mathcal{D}_1$ contains all the deontic action algebras in $\mathcal{D}_0$ that are freely generated by a set $G$. Moreover, every deontic action algebra in $\mathcal{D}_1$ satisfies the following equations:
% \begin{equation}\label{eq:dal1}
% 	\F[x] \lor \P[x] =_{\Algebra[F]} \top \qquad \mbox{for every $x \in G$}
% \end{equation}
% \end{definition}

The second system, here called \NDAL{2}, is obtained from \NDAL{1} by adding the formula $\perm(\bar{\mathsf{a}}_0 \sqcap \dots \sqcap \bar{\mathsf{a}}_n) \lor \forb(\bar{\mathsf{a}}_0 \sqcap \dots \sqcap \bar{\mathsf{a}}_n)$ as an additional axiom of the logic, under the proviso that $\bact = \{\mathsf{a}_0, \dots, \mathsf{a}_n\}$ for some $n \in \mathbb{N}_0$; i.e., under the proviso that there are finitely many basic action symbols.
Intuitively, this additional axiom states that not performing any of the basic actions is permitted or forbidden.
% On the algebraic side, considering a finite set $\bact$ of basic actions results in the algebra $\Algebra[A]$ of actions being a \emph{finite atomic} Boolean algebra.
The algebraic counterpart of \NDAL{2} corresponds to the class of deontic action algebras with a finitely generated atomic Boolean algebra of actions $\Algebra[A]$ satisfying the condition ${
	\P(%
		\bar{a}_1 \sqcap
		\dots \sqcap
		\bar{a}_n
	)
	\lor
	\F(%
		\bar{a}_1 \sqcap
		\dots \sqcap
		\bar{a}_n
	) =_{\Algebra[F]} \top}$
for $\{a_0, \dots, a_n\}$ the set of generators of $\Algebra[A]$.

The third system, \NDAL{3}, is obtained from \NDAL{2} by adding $(\mathsf{a}_0 \sqcup \dots \sqcup \mathsf{a}_n) = \uact$ as an additional axiom.
Intuitively, this new axiom indicates that the agent can only perform actions in $\{\mathsf{a}_1,\dots, \mathsf{a}_n\}$.
The algebraic counterpart of \NDAL{3} corresponds to the subclass of $\NDAL{2}$ further satisfying the condition $a_0 \sqcup \dots \sqcup a_n = \uact$. 

The fourth system, \NDAL{4}, aims to capture the principle of deontic closure at the level of ``atomic'' actions.
Formally, the language of the logic assumes a finite set $\{\mathsf{a}_0, \dots, \mathsf{a}_n\}$ of basic action symbols.
Its axiomatization adds all the formulas in
	$\set
		{\perm({\tilde{\mathsf{a}}_0} \sqcap \dots \sqcap {\tilde{\mathsf{a}}_n}) \lor \forb({\tilde{\mathsf{a}}_0} \sqcap \dots \sqcap {\tilde{\mathsf{a}}_n})}
		{\tilde{\mathsf{a}}_i \in \{\mathsf{a}_i, \bar{\mathsf{a}}_i\}}$ as additional axioms to \DAL.
The algebraic counterpart of \NDAL{4} corresponds to the class of all deontic action algebras with a finitely generated and atomic algebra of actions, whose atoms $a$ satisfy the condition ${\P(a) \lor \F(a) =_{\Algebra[F]} \top}$.

The fifth and last system in \cite{Trypuz15}, here called \NDAL{5}, is the union of \NDAL{3} and \NDAL{4}. Naturally, its
algebraic counterpart corresponds to the intersection of the classes of deontic action algebras characterizing \NDAL{3} and \NDAL{4}.

%In summary, for each \NDAL{i} we have a corresponding subclass $\DALVariety(i)$ of deontic action algebras --determined by the particular conditions of the logic. 
We now present soundness and completeness results of each of these logics. 
To this end, we introduce the auxiliary definitions of \emph{deontic action subalgebra} and  \emph{deontic action generated algebra}.
Both are analogous to the standard case.

\medskip
\begin{definition}\label{def:deontic-subalgebra} Let $\DAlgebra = \langle \Algebra[A], \Algebra[F], \E, \P, \F \rangle$ and $\DAlgebra' = \langle \Algebra[A]', \Algebra[F]', \E', \P', \F' \rangle$ be two deontic action algebras, we say that 
$\DAlgebra'$ is a subalgebra of $\DAlgebra$ iff: 1.~$\Algebra[A]'$ is a Boolean subalgebra of $\Algebra[A]$; 2.~$\Algebra[F]'$ is a subalgebra of $\Algebra[F]$; and 3.~$\E'$, $\F'$, and $\P'$ are restrictions of $\E$, $\F$, and $\P$ to $\Algebra[A]'$ and $\Algebra[F]'$, respectively.
\end{definition}
\medskip

% The notion of generated algebra is also an extension of the standard definition.  As formally introduced by the next definition.

% \medskip

\begin{definition} Let $\DAlgebra = \langle \Algebra[A], \Algebra[F], \E, \P, \F \rangle$ be a deontic action algebra.
In addition, let $A' \subseteq |\Algebra[A]|$ and $F' \subseteq |\Algebra[F]|$.
The sets $A'$ and $F'$ are called generators.
The deontic action algebra generated by $A'$ and $F'$ is the subalgebra $\DAlgebra = \langle \Algebra[A]', \Algebra[F]', \E', \P', \F' \rangle$ of $\DAlgebra$ where: 1.~$\Algebra[A]'$ is the intersection of all the subalgebras of $\Algebra[A]$ whose carrier set contains $A'$; and 2.~$\Algebra[F]'$ is the intersection of all the subalgebras of $\Algebra[F]$ whose carrier set contains $F'$.
\end{definition}
\medskip

The following theorem extends \Cref{theorem:completeness} for \DAL to its variants \NDAL{i}.

\medskip

\begin{theorem}\label{theorem:completeness:dal:i}
	If follows that $\varphi$ is a theorem of \NDAL{i} iff $\DALVariety(i) \vdash \varphi \doteq \top$.
\end{theorem}
\begin{proof} The proof is direct extension of that in \Cref{theorem:completeness}. We only sketch relevant steps.

\medskip 

\begin{description}
	\setlength{\itemsep}{5pt}
	\item[Soundness.]
	For \NDAL{1} we need to show for all $\DAlgebra \in \DALVariety(1)$ and all interpretations $h: \TAlgebra \to \DAlgebra$, it follows that $h({\forb(\mathsf{a}_i) \lor \perm(\mathsf{a}_i)}) =_{\Algebra[F]} \top$.
	Then:
		\begin{align*}
			h({\forb(\mathsf{a}_i) \lor \perm(\mathsf{a}_i)}) =_{\Algebra[F]}
			h(\forb(\mathsf{a}_i)) \lor h(\perm(\mathsf{a}_i)) =_{\Algebra[F]}
			\F(h(\mathsf{a}_i)) \lor \P(h(\mathsf{a}_i)) =_{\Algebra[F]}
			\top.
		\end{align*}%
	Note that for every $a_i \in \bact$, $h(a_i)$ is a generator, and that homomorphisms between generated Boolean algebras are determined by the mapping between their generators. 
	For the other variants the proofs are similar using the properties of generators, homomorphisms, and the new equations for each case.

	\item[Completeness.]
	Similarly to our result in \Cref{theorem:completeness}, for each \NDAL{i}, we need to define an equivalent to the Lindenbaum-Tarski Algebra of \DAL.  We describe the procedure for \NDAL{1}.
	The other cases use the same argument. 
	First, consider the Lindenbaum-Tarski $\LTAlgebra$ in \Cref{prop:lindenbaum}, and  consider the subalgebra $\LTAlgebra(1)$ generated by the generators 
	$A' =\set{[\mathsf{a}_i]_{\cong_{\sorta}}}{\mathsf{a}_i \in \bact}$, and $F' = \form/_{\cong_{\sortf}}$.
	Furthermore, consider the congruence $\cong_{(1)}$ over $\LTAlgebra(1)$ induced by theoremhood in \NDAL{1}, i.e., the axioms of \DAL plus the new axiom set $\set{\forb(\mathsf{a}_i) \lor \perm(\mathsf{a}_i)}{\mathsf{a}_i \in \bact}$.
	% ,  and also the corresponding algebra $\LTAlgebra(1)/\cong_{\NDAL{1}}$.  As proven in \Cref{theorem:completeness}, this is a deontic action algebra.  Furthermore,
	From its construction, $\LTAlgebra(1)/\cong_{(1)}$ is such that	$\F([\mathsf{a}_i]_{\cong_{\sorta}}) \lor \P([\mathsf{a}_i]_{\cong_{\sorta}}) = \top$.
	This algebra provides the canonical deontic action algebra for \NDAL{1}.
	The proof for \NDAL{i}, for $i \in \{2,3,4,5\}$, can be obtained by a similar procedure: a subalgebra of the original Lindenbaum Algebra is considered,  this subalgebra is quotiented by the corresponding axioms, obtaining an algebra that allows us to prove the completenes for the corresponding version of the logic.
	\qedhere
\end{description}
\end{proof}

% \begin{proof}[Sketch] For each logic presented above the techniques presented in \ref{} can be extended in a routinary way. For instance, consider
% the case of \NDAL{1},  since the algebras in $\mathcal{D}_1$ are freely generated by a set $G$ the notion of algebraic validity is defined as follows.
% We say that $\vDashcurly_{\mathcal{D}_1} \varphi \doteq \True$ iff for every $\DAlgebra \in \mathcal{D}_1$ and  onto function  $f:\bact \rightarrow G$ (being $g$ the generators of $\DAlgebra$)
% we have that $\DAlgebra, f^* \vDashcurly \varphi \doteq \True$. First, note that $f^*(\perm[a_i] \vee \forb[a_i]) = \P{x_i} +_F \F{x_i} = \True$, because $f$ maps
% $\bact$ to generators and \ref{eq:dal1}. The rest of the axioms can be proven to evaluate to $\True$ using the same reasoning as \ref{theorem:soundness}. For completeness the Lindenbaum algebra can be defined in the same way as \ref{}, it is direct to check that this algebra satisfies equation \ref{eq:dal1} and its action algebra is generated by the $[a_i]'s$, and therefore
% \ref{} apply to \NDAL{1}. These ideas can easily be adapted to prove the algebraic soundness and completeness of $\NDAL{2}-\NDAL{5}$.
% \end{proof}

%\subsection{Further Extensions}
%	One interesting aspect of the formalism presented in this paper is that it can easily be extended to cope with other action operators and normative formulas.

%Notice that \DAL accepts consistent sets of formulas in which all actions are permitted, alt., forbidden (but not both simultaneously).
%If all actions are permitted, nothing an agent does leads to trouble; on the other hand, if all actions are forbidden, an agent is trapped, i.e., everything it does leads to a violation.
%These cases are known as \emph{deontic heaven}, alt., \emph{deontic hell}.
%Deontic heaven and deontic hell are captured in the language of \DAL by the formulas $\perm[1]$ and $\forb[1]$ (resp., see~\cite{Trypuz15}).
%In some cases, it is interesting to exclude these extreme cases.
%To do this, it suffices to add $\Not \perm[1]$, and/or $\Not \forb[1]$, to the set of axioms of the language of \DAL.
%We will return to this point later on.
%
%We bring attention to some important results regarding permission, prohibition, and consistency in relation to the universal action in \DAL.
%First, $\not \vdash \perm[1]$ and $\not \vdash \Not \perm[1]$, and $\not \vdash \forb[1]$ and $\not \vdash \Not \forb[1]$.
%Second, if $\Phi \vdash \perm[1]$, then $\Phi \vdash \perm[\alpha]$ for any action $\alpha$.
%Third, let $\alpha \neq 0$, if $\Phi \vdash \perm[\alpha]$ and $\Phi \vdash \forb[\bar{\alpha}]$, or alt., $\Phi \vdash \forb[\alpha]$ and $\Phi \vdash \perm[\bar{\alpha}]$, then, $\Phi$ is inconsistent.

%%% Local Variables:
%%% mode: latex
%%% TeX-master: "article"
%%% End:

\subsection{Introducing Propositions}\label{sec:dal:propositions}

Our algebraization of \DAL features an unusual characteristic: the use of an empty set of variables of sort $\sortf$ in the definition of the term algebra $\TAlgebra$ in~\Cref{dal:talg}.
% This is an unusual characteristic.
A more natural approach would be to consider a countable set $\prop$ of proposition symbols as variables of sort $\sortf$, analogous to the set $\bact$ of basic action symbols.
Incorporating the set $\prop$ into \DAL results in a new deontic action logic, which we denote as $\DAL(\prop)$.
The construction of this new logic is relatively straightforward, as are its soundness and completeness results.
Moreover, we demonstrate that this new logic has certain advantages over \DAL for modeling scenarios that require explicit propositional reasoning.

\paragraph{Deontic Action Algebras and Propositions.}

The logic $\DAL(\prop)$ extends the logical language of \DAL incorporating symbols in $\prop = \set{p_i}{i \in \Nat_0}$ as base cases in the recursive definition of formulas.
In addition, it adapts the axiom system for \DAL to accommodate for this new definition of a formula.
The algebraization of $\DAL(\prop)$ uses the same signature as \DAL.
Its algebraic language is the term algebra $\TAlgebra_1$ built sets:
	$\bact$ of variables of sort $\sorta$, and
	$\prop$ of variables of sort $\sortf$.
The term algebra $\TAlgebra_1$ is interpreted into deontic action algebras as explained in \Cref{section:basics}. That is, an interpretation of $\TAlgebra_1$ in a deontic action algebra 
$\DAlgebra = \tup{\Algebra[A], \Algebra[F], \E, \P, \F}$ is a homomorphism $h:\TAlgebra_1 \to \DAlgebra$. Note that, by definition, $h$ maps symbols in $\bact$ into elements in $|\Algebra[A]|$ and symbols in $\prop$ into elements in $|\Algebra[F]|$ (in fact, interpretations are completely determined by mappings $\bact \to |\Algebra[A]|$ and $\prop \to |\Algebra[F]|$).

We derive the soundness and completeness of $\DAL(\prop)$ by adapting the corresponding results for \DAL in \Cref{sec:algebraic-char}.

\medskip
\begin{theorem}\label{prop:completeness:dal:prop} $\varphi$ is a theorem of $\DAL(\prop)$ iff $\DALVariety \vDash {\varphi \doteq \top}$.
\end{theorem}
\begin{proof} The proof follows the steps of that of  \Cref{theorem:soundness},  we highlight some subtle details.  

\medskip 
	\begin{description}
		\item[Soundness.]
		The proof of soundness is, as in \Cref{theorem:soundness}, by induction on the length of proofs.
		Note that the set of proofs is defined as in \Cref{theorem:soundness}, but instantiations of axiom schemas may now contain proposition symbols. E.g.,  $(p \land \lnot p) \liff \bot$ is an instance of an axiom schema, and a theorem. This does not affect the proof given in \Cref{theorem:soundness}, which is immediately lifted to a proof of soundness for ${\DAL(\prop)}$.

		\item[Completeness.]
		For completeness, we need to redefine the Lindenbaum algebra.
		First, in this case, we consider the term algebra $\TAlgebra_1$ that contains also formulas with propositions. The congruence $\cong$ in \Cref{prop:lindenbaum} is used to construct the quotient algebra. This quotient algebra is a deontic action algebra. Note that this algebra also contains formula terms with propositions by definition. Finally, adapting the proof of \Cref{theorem:completeness} we obtain the algebraic completeness result.
		\qedhere
	\end{description}
\end{proof}
\medskip

In summary, $\DAL(\prop)$ differs only from $\DAL$ in their respective term algebras.
That is, while \DAL is associated with a term algebra $\TAlgebra$ built over an empty set of variables of sort $\sortf$, the term algebra $\TAlgebra_1$ associated to $\DAL(\prop)$ uses the set $\prop$ as the set of variables of sort $\sortf$.
By adding proposition symbols, the term algebra $\TAlgebra_1$ brings about a sense of correspondence between the basic symbols used for building the set of actions and those used for building the set of formulas.

\paragraph{Propositions Matter}

\DAL, as well as its variants discussed in \Cref{section:dals}, place the focus on formalizing notions of permission and prohibition pertaining to actions.
Nonetheless, they face challenges when presented with statements such as \emph{it is not the case you are permitted to drive without a license}.
This limitation stems from the inability to distinguish between \emph{pure propositions}, such as \emph{you have a driver's license}, and \emph{normative propositions}, such as \emph{you are permitted to drive}.
% In this section, we show how to extend our algebraization for \DAL to be able to cope with this challenge.
% Moreover, we show how to extend Segerberg's proposal for the notions of permission and prohibition to states of affairs.
This distinction is seamlessly addressed in $\DAL(\prop)$.
For example, we can use a proposition symbol $\mathsf{haslicense} \in \prop$ to indicate that a person has a driver's license, and the formula $\perm(\mathsf{driving})$ to indicate that (the action of) driving is permitted.
This allow us to formalize \emph{it is not the case you are permitted to drive without a license} as $\lnot(\lnot\mathsf{haslicense} \land \perm(\mathsf{driving}))$.

Including propositions in deontic action algebras leads to some interesting discussions.
Imagine the following scenario: \emph{there must be no fence;  if there is a fence, then, it must be a white fence; there is a fence}.
This typical case of contrary-to-duty reasoning is discussed in~\cite{Prakken:1996}, where it is noted that prescriptions are applied to propositions rather than actions.
%
The shift from ought-to-do to ought-to-be is central to most deontic logics developed in the late part of the 20th century; and to \SDL (Standard Deontic Logic) in particular~\cite{Aqvist:2002}.
This shift is not without difficulties. It comes at a cost of quickly leading to paradoxes, i.e., theorems in the logic that are intuitively invalid~\cite{Aqvist:2002,Meyer:1994}.
For instance, in \SDL, a natural formalization of the scenario above, together with the formalization of the global assumption that \emph{if there is a white fence, then, there is a fence} leads to a contradiction; while the scenario is intuitively plausible.

The position held in \cite{Prakken:1996} is that it is worthwhile exploring conditions under which contrary-to-duties can be given consistent readings.
In this respect, we raise the following point.
Up to now, we have assumed that in a deontic action algebra $\DAlgebra = \tup{\Algebra[A], \Algebra[F], \E, \P, \F}$ the algebra $\Algebra[A]$ is used to describe actions.
But a more abstract view of this algebra is also possible.
In particular, we can think of the elements of  $\Algebra[A]$ as entities of the world that can be prescribed, they might be actions, but also propositions such as \emph{there is a fence}, or \emph{the fence is white}.
Under this new reading of a deontic action algebra, in $\DAL(\prop)$, we can distinguish between propositions we can prescribe over, and those we cannot.
For instance, we may regard statements like \emph{it is permitted that is raining} as having little sense, and thus as being ill-formed.
The statement \emph{it is raining} is either true or false, but in no case seems to be amenable to be regulated by a normative system.
Summarizing, deontic action algebras can be used to model ought-to-be normative systems where there is a clear distinction between entities that can be prescribed (corresponding to the elements in $\Algebra[A]$), and those that cannot (corresponding to the elements in $\Algebra[F]$).

\begin{figure}
	\centering
	\includegraphics[width=0.5\textwidth]{deontic-algebra-fence.pdf}\\[1em]
	\caption{The Cottage Regulations Example}\label{ex:fence}
\end{figure}

\medskip
\begin{example}
	Returning to the example, let us use $\obl(\alpha)$, read as $\alpha$ is \emph{obligatory}, as an abbreviation of $\forb(\bar{\alpha})$.
	Then, we could use the formulas:
	$\obl(\overline{\mathsf{isfenced}})$ 
		to indicate that \emph{there must be no fence},
	$\mathsf{isfenced} = \uact \to \obl(\mathsf{ispaintedwhite})$
		to indicate that \emph{if there is a fence, then, it must be a white fence}, and
	$\mathsf{isfenced} = \uact$
		to indicate that \emph{there is a fence}.
	Finally, we could use the formula
		$\mathsf{ispaintedwhite} \sqcup \mathsf{isfenced} = \mathsf{isfenced}$ to indicate the global assumption that \emph{if the fence is painted white, then, the house is fenced}.
	The deontic action algebra $\DAlgebra$ in \Cref{ex:fence}, together with the interpretation $h: \TAlgebra_1 \to \DAlgebra$ defined as $h_{\sorta}(\mathsf{isfenced}) = \uact$, $h_{\sorta}(\mathsf{ispaintedwhite}) = a$, prove that these formulas are consistent.
	Precisely, we have:

	\medskip
	\centerline{
	\begin{minipage}{0.6\textwidth}
		\begin{enumerate}
			\setlength{\itemsep}{5pt}
			\item $h(\obl(\overline{\mathsf{isfenced}})) =_{\Algebra[F]} \top$.
			\item $h(\mathsf{isfenced} = \uact \to \obl(\mathsf{ispaintedwhite})) =_{\Algebra[F]} \top$.
			\item $h(\mathsf{isfenced} = \uact) =_{\Algebra[F]} \top$.
			\item $h(\mathsf{ispaintedwhite} \sqcup \mathsf{isfenced} = \mathsf{isfenced}) =_{\Algebra[F]} \top$.
		\end{enumerate}
	\end{minipage}}
\end{example}
\medskip

To sum up, we have explored some key features and applications of the incorporation of propositions into deontic action algebras. These insights, brought about by our discussion and examples, are particularly relevant to understand some broader implications of Segerberg's formalization of the notions of permission and prohibition.
\subsection{Heyting Algebras for Formulas}\label{sec:heyting:formulas}

Let us now turn to leveraging the modular framework of deontic action algebras in the construction of new deontic action logics.
In \Cref{sec:algebraic-char}, we brought attention to this modularity presenting a deontic action algebra as a structure $\DAlgebra = \tup{\Algebra[A], \Algebra[F], \E, \P, \F}$, with $\Algebra[A]$ and $\Algebra[F]$ interpreting actions and formulas, and $\E$, $\P$, and $\F$ formalizing equality, permission, and prohibition of actions.
While we have primarily considered $\Algebra[A]$ and $\Algebra[F]$ as Boolean algebras, our framework allows also for alternative algebras for actions and formulas.
Notably, defining $\Algebra[F]$ as a Heyting algebra leads to a new deontic action logic worth considering.
We call this new logic $\DAL(\IPL)$.
We begin with an outline of the technical foundations of $\DAL(\IPL)$, and follow with a discussion of its key features and advantages.

\paragraph{Constructive Reasoning in Deontic Action Algebras}

The language of $\DAL(\IPL)$ contains the actions and formulas of $\DAL(\prop)$. Namely, actions are built using basic action symbols in $\bact$, and the connectives $\sqcup$, $\sqcap$, $\bar{~}$, $\iact$, and $\uact$.
In turn, formulas are built using proposition symbols in $\prop$, the deontic connectives on actions, i.e., $\perm(\alpha)$ and $\forb(\alpha)$, and the connectives $\lor$, $\land$, $\lnot$, $\bot$, and $\bot$.
The sole difference is that $\DAL(\IPL)$ introduces the connective $\to$ as primitive rather than as an abbreviation \textemdash with $\varphi \liff \psi$ remaining as an abbreviation for $(\varphi \to \psi) \land (\psi \to \varphi)$.
The axiomatization of $\DAL(\IPL)$ uses the axioms in \Cref{dal:axioms} for actions, equality, and the deontic operations, while the axioms for the propositional connectives (A1'--A13' and LEM') are replaced by those in \Cref{axioms:ipl}.
These last axioms are standard for Intuitionistic Propositional Logic~\cite{Troelstra:1988}.
Provability and theoremhood are straightforwardly adapted to accommodate for the new axioms.
% Provability and theoremhood in $\DAL(\IPL)$ are defined straightforwardly in a Hilbert-style fashion using axioms and the rule of modus ponens.

\begin{figure}
	\centering
	\fbox{
	\begin{minipage}{1.0\textwidth}
		\setlength{\linewidth}{.97\textwidth}
		\setlength{\columnsep}{-1.6cm}
		\begin{multicols}{2}
			\begin{enumerate}[label=H\arabic*.]
				\item $\varphi \to (\varphi \lor \psi)$
				\item $\varphi \to (\psi \lor \varphi)$
				\item $\varphi \land \psi \to \varphi$
				\item $\varphi \land \psi \to \psi$
				\item $(\varphi \to \bot) \to \lnot \varphi$
				\item $\lnot \varphi \to (\varphi \to \bot)$
				\item $\bot \to \varphi$
				\item $\varphi \to \top$
				\item $\varphi \to ( \psi \to \varphi )$
				\item $\varphi \to (\psi \to (\varphi \land \psi))$
				\item ${(\varphi \to \chi) \to ((\psi \to \chi) \to ((\varphi \lor \psi) \to \chi))}$
				\item ${(\varphi \to (\psi \to \chi)) \to ((\varphi \to \psi) \to (\varphi \to \chi))}$
			\end{enumerate}
		\end{multicols}
	\end{minipage}}\\[1em]
	% \medskip
	\caption{Axiom System of $\DAL(\IPL)$}\label{axioms:ipl}
\end{figure}

The algebraization of $\DAL(\IPL)$ replaces the Boolean algebra of formulas in the definition of a deontic action algebra for a Heyting algebra. The precise definition of the new type of deontic action algebra being used is given below.

\medskip
\begin{definition}\label[definition]{def:dalgebra:heyting:boolean}
	A BH-deontic-action algebra is an algebra
		$\DAlgebra =
			\langle
				\Algebra[A], \Algebra[H], \E, \P, \F
			\rangle$
		where:
			$\Algebra[A]$ is a Boolean algebra,
			$\Algebra[H]$ is a Heyting algebra, and
			$\E : {|\Algebra[A]| \times |\Algebra[A]| \to |\Algebra[H]|}$,
			$\P : {|\Algebra[A]| \to |\Algebra[H]|}$,
			and
			$\F : {|\Algebra[A]| \to |\Algebra[H]|}$ satisfy the conditions 1--6 in \Cref{definition:deontic:algebra}.
\end{definition}
\medskip

In brief, Heyting algebras play a role in constructive reasoning analogous to the role Boolean algebras play in classical reasoning. A key distinction is how Heyting algebras treat $\to$.
Despite this difference, Heyting algebras are closely related to Boolean algebras.
Specifically, every Boolean algebra is a Heyting algebra, and the regular elements of a Heyting algebra \textemdash those $x 
\in |\Algebra[H]|$ for which $x =_{\Algebra[F]} \lnot\lnot x$ \textemdash form a Boolean algebra.
It is well-known also that Heyting algebras have a representation theorem \textemdash as the category of Heyting algebras is dually equivalent to the category of Eusaki spaces.
Furthermore, the Lindenbaum algebra obtained from the axioms in \Cref{axioms:ipl} is itself a Heyting algebra \cite{vanDalen:2008}.
These facts collectively support the idea of replacing Boolean algebras with Heyting algebras in the algebraic treatment of deontic action logic, ensuring that such an approach is well-founded.

The proposition below exposes an interesting feature of BH-deontic-action algebras.

\medskip
\begin{proposition}\label[proposition]{prop:ideals-int}
	Let $\DAlgebra = \tup{\Algebra[A], \Algebra[H], \E, \P, \F}$ be a BH-deontic-action algebra.
	The pre-images $P$ and $F$ of $\top$ under $\P$ and $\F$, respectively, are ideals in $\Algebra[A]$ s.t.\ ${{P \cap F} = \{\iact\}}$.
\end{proposition}
\begin{proof}
	Note that ideals in Heyting algebras and ideals in Boolean algebras are defined identically.
	Note also that the proof of the analogous result for deontic action algebras in \Cref{prop:dal:ideal} uses only reasoning on ideals and the properties of $\E$, $\P$, and $\F$.
	Since these properties are maintained in BH-deontic-action algebras, the proof in \Cref{prop:dal:ideal} transfers directly to this new setting.
\end{proof}
\medskip

In line with \Cref{prop:dal:ideal}, the result in \Cref{prop:ideals-int} tells us that permission and prohibition on actions yielding ideals carry over if we replace Boolean for Heyting algebras.

By way of conclusion, we present soundness and completeness theorems for $\DAL(\IPL)$.
Definitions of interpretations of the term algebra into BH-deontic-action algebras, homomorphisms, and congruences and quotients, are akin to those in \Cref{sec:algebraic-char}.
% In our proof, we can follow the steps and the main ideas of  \Cref{sec:dal:propositions} to prove the soundness and completeness of the intuitionistic version of the logic.
% For doing so,  first, we note that the axiomatic system for the logic have to be adapted to the intuitionistic setting.  This is done by considering the axiomatic system of  \Cref{section:dal} without the axiom (PEM), we denote the deduction relation obtained by $\vdash_{\DAL(\IPL)}$, similarly we denote by $\vDashcurly_{\IPL} {\varphi \doteq \top}$ the algebraic validity in the algebras of \Cref{def:dalgebra:heyting:boolean}.
%Similarly to the case in \Cref{sec:dal:propositions}, we obtain the following result.

\medskip
\begin{theorem}\label{prop:completeness:heyting:formulas}
	 Let $\mathbb{BH}$ be the class of all BH-deontic-action algebras.
	 It follows that $\varphi$ is a theorem of $\DAL(\IPL)$ iff $\mathbb{BH} \vDash {\varphi \doteq \top}$.
\end{theorem}
\begin{proof}
	Like with the proofs of \Cref{theorem:completeness:dal:i,prop:completeness:dal:prop}, we only remark on the differences with the proofs of \Cref{theorem:soundness,theorem:completeness}.
	\begin{description}
		\item[Soundness.]
		We need to prove that any interpretation of an axiom is mapped to $\top$.
		For axioms on actions, this is just like in \Cref{theorem:soundness}.
		For axioms on propositional connectives, this is  immediate from well known results for
		Heyting algebras (see~\cite{vanDalen:2008,Troelstra:1988}).
		Finally, the cases of equality, permission, and prohibition are not affected by the new interpretation of $\to$ in a Heyting algebra.

		\item[Completeness.]
		We begin by defining the Lindenbaum algebra as in \Cref{prop:lindenbaum} via congruences $\cong_{\sorta}$ for actions and $\cong_{\sortf}$ for formulas.
		Again, following~\cite{vanDalen:2008}, it is easy to see that the axioms in \Cref{axioms:ipl} result in the algebra of formulas itself being a Heyting algebra.
		This construction provides a witness for theoremhood for the logic.
		\qedhere
	\end{description}
\end{proof}

\paragraph{Constructive Reasoning Matters}

Just like we did when we introduced propositions, let us discuss why interpreting formulas on Heyting algebras instead of Boolean algebras bears an interest beyond its formal properties.

To set the stage for discussion, imagine the following scenario: \emph{if John does not have a driver's license, then, it is forbidden for him to drive; John is not forbidden to drive}.
From this scenario, using classical reasoning, we derive \emph{John has a driver's license}.
This conclusion is somewhat counterintuitive.
It is easy to consider many cases in which \emph{John does not have a driver's license} is true, which are consistent with the scenario in question.
But this is logically impossible.
There are many ways of dealing with this kind of problem, one of which is to move from Classical to Intuitionistic reasoning. %\cite{DBLP:conf/wollic/DalmonteGO22}, that is, to reject the principle $\varphi \vee \neg \varphi$ (or equivalently $\neg ( \varphi \wedge \neg \varphi)$) as part of the logic.  This has many consequences in the resulting  formal system. 

\medskip
\begin{example}
	The driver's license example in the previous paragraph can be formalized with formulas:
	$\lnot \mathsf{haslicense} \to \forb(\mathsf{driving})$
		capturing that \emph{if John does not have a driver's license, then, it is forbidden for him to drive}, and
	$\lnot\forb(\mathsf{driving})$
		capturing that \emph{John is not forbidden to drive}.
	Note that, in this formalization, $\mathsf{haslicense} \in \prop$ and $\mathsf{driving} \in \bact$.
	The BH-deontic-action algebra $\DAlgebra$ in \Cref{ex:driver}, together with the interpretation $h$ defined as $h_{\sorta}(\mathsf{driving}) = a$, $h_{\sortf}(\mathsf{haslicense}) = \frac{1}{2}$, prove that $\mathsf{haslicense}$ is not a consequence of the previous two formulas.
	Precisely, we have:

	\medskip
	\centerline{
	\begin{minipage}{0.7\textwidth}
		\begin{enumerate}[leftmargin=\parindent]
			\setlength{\itemsep}{5pt}
			\item $
				h(\lnot\mathsf{haslicense}) =_{\Algebra[H]}
				\lnot h(\mathsf{haslicense}) =_{\Algebra[H]}
				\lnot \frac{1}{2} =_{\Algebra[H]}
				\bot$.
			\item $
				h(\lnot\mathsf{haslicense} \to \forb(\mathsf{driving})) =_{\Algebra[H]}
				% h(\lnot\mathsf{haslicense}) \to h(\forb(\mathsf{driving})) =_{\Algebra[H]}
				\bot \to h(\forb(\mathsf{driving})) =_{\Algebra[H]}
				\top$.
			\item $
				h(\lnot\forb(\mathsf{driving})) =_{\Algebra[H]}
				\lnot h(\forb(\mathsf{driving})) =_{\Algebra[H]}
				\lnot \bot =_{\Algebra[H]}
				\top$.
			\item $h(\mathsf{haslicense}) \neq_{\Algebra[H]} \top$.
		\end{enumerate}
	\end{minipage}}
	\medskip

	\noindent Note how in the BH-deontic-algebra $\DAlgebra$ in \Cref{ex:driver} the only element of the algebra of actions which the operation $\F$ maps to $\top$ is $\iact$, all other elements are mapped to $\bot$.
\end{example}
\medskip

% \begin{figure}
% 	\centering
% 	\includegraphics[width=0.5\textwidth]{deontic-algebra-driving.pdf}\\[1em]
% 	\caption{The Driver's License Paradox}\label{ex:driver}
% \end{figure}

\begin{figure}
	\centering
	\begin{minipage}{0.48\textwidth}
		\centering
		\includegraphics[trim=20pt 0pt 20pt 0pt, clip, width=\textwidth]{deontic-algebra-driving.pdf}\\[1em]
		\caption{The Driver's License Paradox.}\label{ex:driver}
	\end{minipage}\hfill
	\begin{minipage}{0.48\textwidth}
		\centering
		\includegraphics[trim=20pt 0pt 20pt 0pt, clip, width=\textwidth]{deontic-algebra-closure.pdf}\\[1em] % second figure itself
		\caption{Principle of Deontic Closure.}\label{ex:deontic:closure}
	\end{minipage}
\end{figure}

There is another interesting discussion emerging from the use of an Intuitionistic basis for reasoning about formulas.
Recall the deontic action logic $\DAL(1)$ from \Cref{section:dals} and how this logic is built from $\DAL$ adding additional axioms with the intent to capture the \emph{principle of deontic closure}.
This principle is stated in \cite{Segerberg1982} as: \emph{what is not forbidden is permitted}.
The formalization of this principle as formulas of the form $\forb(\mathsf{a}) \lor \perm(\mathsf{a})$ is taken from \cite{Trypuz15}.
Still, a more faithful formalization of this principle is $\lnot\forb(\mathsf{a}) \to \perm(\mathsf{a})$.
Clearly, there is no substantial distinction in a Classical setting, as both formulas are equivalent.
This is not the case in an Intuitionistic setting.
For instance, the BH-deontic-action algebra $\DAlgebra$ in \Cref{ex:deontic:closure} satisfies one version of the principle but not the other.
Precisely, note that if we have a single basic action symbol $\mathsf{a} \in \bact$, and an interpretation $h$ on $\DAlgebra$ s.t.\ $h_{\sorta}(\mathsf{a}) = a$, then:

	\medskip
	\centerline{
	\begin{minipage}{0.5\textwidth}
		\begin{enumerate}[leftmargin=\parindent]
			\setlength{\itemsep}{5pt}
			\item $
				h(\forb(\mathsf{a})) =_{\Algebra[H]}
				\F(h(\mathsf{a})) =_{\Algebra[H]}
				\F(a) =_{\Algebra[H]}
				\frac{1}{2}$.
			\item $
				h(\lnot\forb(\mathsf{a})) =_{\Algebra[H]}
				\lnot h(\forb(\mathsf{a})) =_{\Algebra[H]}
				\lnot \frac{1}{2} =_{\Algebra[H]}
				\bot$.
			\item $
				h(\perm(\mathsf{a})) =_{\Algebra[H]}
				\P(h(\mathsf{a})) =_{\Algebra[H]}
				\P(a) =_{\Algebra[H]}
				\frac{1}{2}$.
			\item $
				h(\lnot\forb(\mathsf{a}) \to \perm(\mathsf{a})) =_{\Algebra[H]}
				\bot \to h(\perm(\mathsf{a})) =_{\Algebra[H]}
				\top$.
			\item $
				h(\forb(\mathsf{a}) \lor \perm(\mathsf{a})) =_{\Algebra[H]}
				\frac{1}{2} \lor \frac{1}{2} =_{\Algebra[H]}
				\frac{1}{2} \neq_{\Algebra[H]}
				\top$.
		\end{enumerate}
	\end{minipage}}
	\medskip

\noindent In words, this example shows that there is a distinction between considering the principle of deontic closure as \emph{what is not forbidden is permitted} \textemdash alternatively, \emph{what is not permitted is forbidden}\textemdash and considering this principle as \emph{every (basic) action is either permitted or forbidden}.

To sum up, we have explored some key features and applications of replacing the Boolean algebra of formulas in a deontic action algebra for a Heyting algebra.
The results we obtained underscore leveraging the modularity of our framework to build a new deontic action logic $\DAL(\IPL)$. The discussion and ensuing examples reinforce the utility of this new logic in the broader area of Deontic Logic, and in particular in relation to the principle of deontic closure.

\subsection{A Heyting Algebra of Actions}\label{sec:action-int}

The formal machinery in \Cref{sec:heyting:formulas} naturally suggests its symmetric extension: replacing the Boolean algebra of actions with a Heyting algebra.
This results in a new deontic action logic, $\DAL(\IAL)$, where actions are interpreted in a manner analogous to formulas in an intuitionistic framework.
To the best of our knowledge, no existing deontic action logic provides an intuitionistic perspective on actions, making this approach a novel contribution to the field.

\paragraph{Another look at Constructive Reasoning in Deontic Action Algebras}

We begin with an outline of the technical foundations of $\DAL(\IAL)$.
The formulas of this new logic are built using proposition symbols in $\prop$, the deontic connectives on actions, i.e., $\perm(\alpha)$ and $\forb(\alpha)$, and the connectives $\lor$, $\land$, $\lnot$, $\bot$, and $\bot$.
In turn, actions are built using basic action symbols in $\bact$, and the connectives $\sqcup$, $\sqcap$, $\bar{~}$, $\iact$, and $\uact$.
In addition, $\DAL(\IAL)$ introduces a new connective $\hto$ on actions giving rise to actions of the form $\alpha \hto \beta$.
This new connective is introduced to capture the notion of a relative complement (or intuitionistic implication) in a Heyting algebra.
The axiomatization of $\DAL(\IAL)$ uses all the axioms in \Cref{dal:axioms} except the axiom (LEM) for actions.
In addition, it introduces as axioms the properties H1\textendash H3 in \Cref{def:heyting:algebra} for the new connective $\hto$.
In essence, the axioms for actions are the conditions on Heyting algebras in~\cite{Esakia19}.
Provability and theoremhood are easily adapted to accommodate for the new axioms.
%\carlos{No me queda muy claro por que esta axiomatizacion es diferente de la anterior.}

The algebraization of $\DAL(\IAL)$ replaces the Boolean algebra of actions in the definition of a deontic action algebra for a Heyting algebra. This is made precise in \Cref{def:dalgebra:heyting:actions} below.

\medskip
\begin{definition}\label[definition]{def:dalgebra:heyting:actions}
	An HB-deontic-action algebra is an algebra
		$\DAlgebra =
			\langle
				\Algebra[H], \Algebra[F], \E, \P, \F
			\rangle$
		where:
			$\Algebra[H]$ is a Heyting algebra,
			$\Algebra[F]$ is a Boolean algebra, and
			$\E : {|\Algebra[H]| \times |\Algebra[H]| \to |\Algebra[B]|}$,
			$\P : {|\Algebra[H]| \to |\Algebra[B]|}$,
			and
			$\F : {|\Algebra[H]| \to |\Algebra[B]|}$ satisfy the conditions 1--6 in \Cref{definition:deontic:algebra}.
\end{definition}
\medskip

\Cref{prop:heyting:actions:ideal} shows that permission and prohibition behave as expected.

\medskip
\begin{proposition}\label[proposition]{prop:heyting:actions:ideal}
	Let $\DAlgebra = \tup{\Algebra[H], \Algebra[F], \E, \P, \F}$ be an HB-deontic-action algebra.
	The pre-images $P$ and $F$ of $\top$ under $\P$ and $\F$, respectively, are ideals in $\Algebra[A]$ s.t.\ ${{P \cap F} = \{\iact\}}$.
\end{proposition}
\begin{proof}
	Analogous to that in \Cref{prop:ideals-int}.
\end{proof}
\medskip

%\paragraph{Soundness and completeness.} Interpretations into the kind of deontic action algebras in \Cref{def:dalgebra:heyting:actions} are defined straightforwardly.

Since $\DAL(\IAL)$ is the symmetric counterpart of $\DAL(\IPL)$, the proofs of soundness and completeness for $\DAL(\IAL)$ can be straightforwardly adapted from those of $\DAL(\IPL)$. Consequently, we establish the following theorem.

\medskip
\begin{theorem}\label{prop:completeness:heyting:actions}
	 Let $\mathbb{HB}$ be the class of all HB-deontic-action algebras.
	 It follows that $\varphi$ is a theorem of $\DAL(\IAL)$ iff $\mathbb{HB} \vDash {\varphi \doteq \top}$.
\end{theorem}

\paragraph{Constructive Reasoning and Realization of Actions}

We put forth the argument that an intuitionistic basis for actions is useful when actions are tied to constructions that witness their realizability.
This parallels the standard interpretation of Intuitionistic Logic, where the truth of a formula corresponds to the existence of a proof.
Interpreting actions on an intuitionistic basis is not only of theoretical interest but also hold potential for practical applications, particularly in automated planning~\cite{GNT:2016}.
For example, consider a robot capable of executing various actions.
To perform an action, the robot requires plans\textemdash sequences of basic activities that realize the action.
In such a scenario, we may reject $\bar{a} \sqcup a = \uact$, as the robot might lack a plan to execute action $a$ or a way to determine if $a$ is unrealizable.
% In this scenario, an action denoted by $a \actimplies b$ can be interpreted as a way of converting a plan for $a$ into a plan for $b$.
% Similarly, an action denoted by $\bar{a} = a \actimplies 0$ can be interpreted as a plan that converts a plan for $a$ into a plan that brings about an impossible action for the robot. In other words, $\bar{a}$ provides a way to describing that the robot cannot realize the action denoted by $a$.

Prescriptions often play an important role in planning.
For instance, if the robot is an autonomous vehicle, it must adhere to transit rules.
In this case, $\perm(\alpha)$ indicates that plans for executing $\alpha$ are permitted, while $\forb(\alpha)$ signals that such plans are forbidden.
This perspective highlights how an intuitionistic basis for interpreting actions aligns well with practical considerations in scenarios where realizability and prescriptive constraints on actions are central.
In this respect, $\DAL(\IAL)$ provides a logical framework that is well-suited to addressing real-life issues.

%	A main motivation for this novel formalism is  to reason about scenarios where there is incomplete information about the agents' possible actions.  Heyting algebras provide the useful notion of \emph{relative complement}, which, as we argue below, can be used in such contexts.

%	As remarked above, the basis of a Heyting algebra is the relative complement (written $a \rightarrow b$, for actions $a$ and $b$).  Algebraically, $a \rightarrow b$ denotes the coarsest action that, when executed together with $a$, yields an execution of action $b$.  In Boolean logic $a \rightarrow b$ is interpreted as
%$\overline{a} \sqcup b$. This notion of relative complement may be useful in diverse settings. Consider a scenario where one need to identify those actions that when executed in parallel with driving will lead to an imprudent way of driving, say $\text{driving} \rightarrow \text{behave-dangerously}}$,  if we have a Boolean model of action
%we have that this action coincides with $\overline{\text{driving}} \sqcup \text{behave-dangerously}$.  However, the action of drinking cannot be thought of as being included in the action $\text{behave-dangerously}$ (there are ways in which one may drink without behaving in a dangerous way), and similarly it cannot be regarded
%as being part of driving (there are ways of drinking without driving). It seems intuitively correct to state $\text{drink} \sqsubseteq \text{drive} \rightarrow \text{behave-dangerously}$, therefore by stating $\neg \perm(\text{drive} \rightarrow \text{behave-dangerously})$ one is capturing those actions that, when executed together with driving will result in a dangerous way of behaving. Another interesting application of Heyting algebras of actions is the possibility of capturing another version of complement,

\subsection{Intuitionistic Deontic Action Logic}

Clearly, we can also simultaneously replace the Boolean algebras for actions and formulas for Heyting algebras.
We call the resulting logic $\DAL(\INT)$.
Similarly to the case in \Cref{sec:action-int}, to the best of our knowledge, this logic is the first fully intuitionistic deontic action logic.


% We start by presenting $\DAL(\INT)$ from a syntactic perspective.
% Then, we move on to explaining how to capture this new deontic action logic from an algebraic perspective.

\paragraph{The Logic Itself}

The language and axiomatization of $\DAL(\INT)$ combines those of $\DAL(\IAL)$ and $\DAL(\IPL)$ in \Cref{sec:heyting:formulas,sec:action-int}.
Precisely, it builds actions like in $\DAL(\IAL)$\textemdash using $\hto$ as a primitive connective.
In turn, it builds formulas like in $\DAL(\IPL)$\textemdash using $\to$ as a primitive connective.
The axiomatization of this new logic takes the axioms for actions from $\DAL(\IAL)$ and the axioms for formulas from $\DAL(\IPL)$.
The logic retains the axiomatization of equality, permission, and prohibition of Segerberg's logic, i.e., axioms E1--E2 and D1--D3 in \Cref{dal:axioms}.
The notions of proof and theoremhood are reformulated in the obvious way.

The algebraization of $\DAL(\INT)$ replaces the Boolean algebras of actions and of formulas in the definition of a deontic action algebra for Heyting algebras. This is made precise in the definition of an \emph{intuitionistic deontic action algebra} below.

\medskip
\begin{definition}\label[definition]{def:intuitionistic:dalgebra}
	By an intuitionistic deontic action algebra, we mean an algebra
		$\DAlgebra =
			\langle
				\Algebra[A], \Algebra[H], \E, \P, \F
			\rangle$
		where:
			$\Algebra[A]$ and 
			$\Algebra[F]$ are Heyting algebras, and
			$\E : {|\Algebra[A]| \times |\Algebra[A]| \to |\Algebra[F]|}$,
			$\P : {|\Algebra[A]| \to |\Algebra[F]|}$,
			and
			$\F : {|\Algebra[A]| \to |\Algebra[F]|}$ satisfy the conditions 1--6 in \Cref{definition:deontic:algebra}.
\end{definition}
\medskip

As before, permission and prohibition behave as expected.

\medskip
\begin{proposition}\label[proposition]{prop:intuitionistic:ideal}
	Let $\DAlgebra = \tup{\Algebra[A], \Algebra[F], \E, \P, \F}$ be an intuitionistic deontic action algebra.
	The pre-images $P$ and $F$ of $\top$ under $\P$ and $\F$ are ideals in $\Algebra[A]$ s.t.\ ${{P \cap F} = \{\iact\}}$.
\end{proposition}
\medskip
\begin{proof}
	We can reuse the proof of \Cref{prop:ideals-int} as it only uses the properties of $\P$ and $\F$ plus absorption, and idempotence properties which also hold in Heyting algebras.
\end{proof}


%\paragraph{Soundness and completeness} I
For stating and proving the soundness and completeness of $\DAL(\INT)$, we define interpretations and algebraic validity as in previous sections.

\medskip

\begin{theorem}\label{prop:completeness:heyting}
	Let $\mathbb{ID}$ be the class of all intuitionistic deontic action algebras.
	Then, $\varphi$ is a theorem of $\DAL(\INT)$ iff $\mathbb{ID} \vDash {\varphi \doteq \top}$.
\end{theorem}
\begin{proof}
	We obtain this result by putting together the intuitionistic parts of the proofs of
	\Cref{prop:completeness:heyting:formulas,prop:completeness:heyting:actions}.
%	
%	\begin{description}
%		\item[Soundness.]  The various cases correspond to those in the soundness proofs of \Cref{prop:completeness:heyting:formulas,prop:completeness:heyting:actions}.
%		
%		\item[Completeness.] We need to define a Lindenbaum algebra via suitable congruences $\cong_{\sorta}$ and $\cong_{\sortf}$.
%		The first congruence correspond to that in the completeness part of the proof of \Cref{prop:completeness:heyting:actions}.
%		The second congruence correspond to that in the completeness part of the proof of \Cref{prop:completeness:heyting:formulas}.
%		It is clear that these congruences yield Heyting algebras.
%		The result is obtained by proving the induced operations of $\E$, $\P$, and $\F$ satisfy the conditions 1--6 in \Cref{definition:deontic:algebra}.
%		\qedhere
%	\end{description}
\end{proof}

\paragraph{Intuitionistic Deontic Action Logic in Practice}

The logic $\DAL(\INT)$ may be useful for reasoning in scenarios where there is partial observability about the state of affairs, typical in reinforcement learning and planning.
Consider the following example adapted from \cite{DBLP:conf/iros/CassandraKK96}: a robot is tasked with cleaning an office and needs to reach certain spots.
The robot has sensors to detect doorways, walls, or open spaces, but the sensor information may sometimes be unclear.
Proposition symbols like $\mathsf{north}$, $\mathsf{south}$, $\mathsf{east}$, and $\mathsf{west}$ could represent the robot's orientation, while $\mathsf{doorway}$, $\mathsf{wall}$, and $\mathsf{clear}$ capture the information provided by the sensors.
Uncertainty entails the robot might fail to determine whether there is a doorway ahead or not, i.e., $\mathsf{doorway} \lor \lnot\mathsf{doorway}$ may fail to hold, violating the law of the excluded middle at the level of formulas.
The example can be extended with an intuitionistic model of actions.
For instance, we can consider actions $\mathsf{advance}$ and $\mathsf{rotate}$ for the robot moving forward and rotating, respectively.
As in the example in \Cref{sec:action-int}, the interpretation of these actions is tied to a plan allowing the robot to realize the action.
Once again, a formula $\mathsf{advance} \sqcup \overline{\mathsf{advance}} = \uact$ may fail to hold (violating the law of excluded middle at the level of actions) because the robot lacks a plan due to incomplete sensor information.

Finally, we can spice up this scenario with prescriptions.
For instance, there might be signals in the corridors indicating the robot is not to cross certain doorway, prohibiting such an action.
Again, the robot may lack sufficient information to determine whether $\forb(\mathsf{advance}) \lor \lnot\forb(\mathsf{advance})$ holds or not.

In all the above scenarios, having an intuitionistic deontic action logic like $\DAL(\INT)$ provides a formal framework for analysis in which we can address issues arising from partial observability and prescriptive constraints.


\ifcategories
\section{A Categorical View of DAL}\label{sec:cat}

One of the main benefits of the algebraic view on \DAL is that it paves the way to the use of abstract mathematical  frameworks such as category theory.  Category theory allows one to capture the properties of mathematical, or logical objects, in a very abstract way,  which makes possible to investigate the relations between different formalisms as well as the abstract properties of mathematical objects.  Another interesting feature of category theory is that it enables modular reasoning over logical, or algebraic systems. For instance,  one can use  standard categorical constructions such as limits and colimits  to put together different structures.  In this section we present the category of deontic actions algebras and investigate its basic properties.

For the sake of simplicity, we restrict ourselves to  category corresponding to the non-intuitionistic logic \DAL, but it must be clear that similar results hold for the other algebras, at the end of this section we make some remarks about this.  We introduce the basic notions of category theory needed for this section,  the interested reader is referred to \cite{MacLane98} for a deep introduction to category theory.

\subsection{Preliminaries on Category Theory}
A category is a structure $\mathbf{C} = (\mathcal{O}, \mathcal{A})$, where $\mathcal{O}$ is a collection of \emph{objects} (also denoted $|\mathbf{C}|$) and $\mathcal{A}$ is a collection of \emph{arrows} (also denoted $||\mathbf{C}||$), equipped with: (i) operations $\mathit{dom}$ and $\mathit{cod}$ assigning to any object $a \in |\mathbf{A}|$, an object $\mathit{dom}(a)$ called its domain, and an object $\mathit{cod}(a)$ called its codomain; (ii) the operation $\circ$ that given $f,g \in ||\mathbf{C}||$ such that $\mathit{cod}(f) = \mathit{dom}(g)$,  returns an arrow, denoted $g \circ f$ with $\mathit{dom}(g \circ f) = dom(f)$ and
$\mathit{cod}(g \circ f) = cof(g)$, (iii) for each object $a \in |\mathbf{C}|$ an arrow $id_a \in || \mathbf{C}||$. Furthermore, the following equations hold: $f \circ id_a = f$ and $id_b \circ f = f$; and  $\circ$ is associative.
A very well-known category is given by  the collection of all (small) sets and all the collection of functions, usually named $\mathbf{Set}$. Similarly,  any algebraic structure form a category consisting of the corresponding algebras as object, and the homomorphisms as its arrows.  It is straightforward to define the notion of \emph{subcategory, } a subcategory is \emph{full} is preserves all the arrow of the subcollection of objects in the subcategory.   An \emph{initial object} in a category $\mathbf{C}$ is an object $0 \in |\mathbf{C}|$ such that for any other object $x$ we have a unique arrow $u : 0 \rightarrow x$. For instance, in $\mathbf{Set}$ $\emptyset$ is an initial element. Final elements are the dual concept.  

Functors are mappings between categories, that is, given two categories $\mathbf{C}$ and $\mathbf{D}$ a functor $F$ between $\mathbf{C}$ and $\mathbf{D}$, written $F:\mathbf{C} \rightarrow \mathbf{D}$,  maps (i) every object $a \in |C|$ to an object, written $F(a)$, of $|\mathbf{D}|$, (ii) any arrow $f:a \rightarrow b \in || \mathbf{D} ||$ to an arrow $F(f) : F(a) \rightarrow F(b) \in || D||$, such that it satisfies: $F(id_a) = id_{F(a)}$, for any $a \in | \mathbf{C} |$, and $F(g \circ f) = F(g) \circ F(f)$ for any $f,g \in ||\mathbf{C} ||$.  

Natural transformations are mapping between functors, that is, given two functors $F, G: \mathbf{C} \rightarrow \mathbf{D}$, a natural transformation $\eta$ between $F$ and $G$, noted $\eta : F \xrightarrow{.} G$,  assigns to each object $x \in |\mathbf{C}|$ an arrow $\eta_x : F(x) \rightarrow G(x)$ in $\mathbf{D}$ such that for any arrow $f:a \rightarrow b \in || \mathbf{C}||$ we have that $\eta_b \circ F(f) = G(f) \circ \eta_a$ (this is called \emph{naturally}).  Two functors $F: \mathbf{C} \rightarrow \mathbf{D}$ and $G: \mathbf{D} \rightarrow \mathbf{C}$ are said to be \emph{adjoints} if the sets $\mathit{hom}(F(c),d)$ and $\mathit{hom}(c, G(d))$ are naturally isomorphic, $F$ is called left adjoint of $G$, and $G$ is said to be the right adjoint of $F$.  Adjoints  are common in algebra where the forgetful functor (noted $U$) that sends each algebra to its support set, and the construction of free algebras, which is a functor noted $F$, are adjoints.  A full subcategory is called reflective if the inclusion functor (from the subcategory to the main category) has a left adjoint. 

In any category we can define objects with the so-called universal constructions,  well-know construction in the category $\mathbf{Set}$ are products and coproducts,  which are instances of the more general concepts of  limits and colimits, respectively.    Given an index category $\mathbf{J}$ a diagram in $\mathbf{C}$ is a functor $D: \mathbf{J} \rightarrow \mathbf{C}$, that is, it is a graph on $\mathbf{C}$.  In particular, we can consider, for each object $c$, the constant functor $c : \mathbf{J} \rightarrow \mathbf{C}$ that sends each object $j \in |J|$ to $c$, and each arrow to $id_c$.  A \emph{cone} with tip $c$ is a natural transformation 
$\tau : F \xrightarrow{.} c$. The collection of all cones with tip $c$ form a category,   a colimit (which is characterized up to isomorphisms) is an initial object in this category.  For instance, to obtain coproduct in $\mathbf{Set}$ consider the index category having only two objects, say $x,y$, with only the identity arrows,  any cone maps this object to a tip, and the colimit is one of this cones whose tip, noted $x+y$, has unique arrows to the other possible tips. This formalizes the notion of disjoint union. Limits can be defined similarly and are the dual notion to colimits.

We will use some standard construction of categories, for instance, given two functors $F: \mathbf{C} \rightarrow \mathbf{E}$ and $G: \mathbf{D} \rightarrow \mathbf{E}$ the comma category, denoted $F \downarrow G$, has as objects arrows $f: F(c) \rightarrow G(d)$,  for $c \in |\mathbf{C}|$ and $d \in |\mathbf{D}|$,  and the arrows between objects $f:  F(c) \rightarrow G(d)$ and $g : F(c') \rightarrow G(d')$ are arrows $\alpha : c \rightarrow c'$ and $\beta : d \rightarrow d'$ such that $G(\beta) \circ f = g \circ F(\alpha)$.
 
\subsection{The Category $\mathbf{Dal}$}

Let us  introduce the category of \DAL algebras.
\medskip
\begin{definition} The category $\mathbf{Dal}$ has  the  algebras of \Cref{definition:deontic:algebra} as its objects, and 
the homomorphisms between these algebras as its arrows.
\end{definition}
\medskip
Note that proving that $\mathbf{Dal}$ is already a category is direct, the composition is the composition of homomorphisms, and the identity is the identity homomorphism.
Let us prove some properties of $\mathbf{Dal}$.  First,  note that this is a concrete category since we have a forgetful functor $U: \mathbf{Dal} \rightarrow \mathbf{Set}$ sending each \DAL algebra to its support sets. Formally,  for a \DAL algebra  $\Algebra[D]=\tup{\Algebra[A], \Algebra[F], \E, \P, \F}$ and let $A$ be the support set of $\Algebra[A]$ and $B$ the support set of $\Algebra[B]$,  then we define:
$U(\Algebra[D]) = (A,B)$ for objects, and $U(h)(x) = h(x)$, for homomorphisms.   
It is direct to prove that $U$ is already a functor.
\medskip
\begin{theorem} The mapping $U: \mathbf{Dal} \rightarrow \mathbf{Set}$ is a functor.
\end{theorem}
\medskip
A more important property of  $\mathbf{Dal}$ is its cocompleteness,  thus it has all colimits: coproducts, pushouts, etc.  Also, it has free objects, that is,  given a set $X$ there is an object $\Algebra[D]$ in $\mathbf{Dal}$ such that   $f: X \rightarrow U(\mathbf{Dal})$, and for any other function $g : X \rightarrow U(\Algebra[D'])$ there is a unique homomorphism $u : \Algebra[D] \rightarrow \Algebra[D']$ such that 
$g = U(u) \circ f$; Lindenbaum algebras, for instance, are free objects. 

The proof mainly follows from a result proven in \cite{Adamek04} for quasivarities.  We adapt that result to our algebras.
\medskip
\begin{theorem}\label{theorem:cocompleteness} The category $\mathbf{Dal}$  is  cocomplete and has free objects
\end{theorem}
\begin{proof} The proof follows the ideas of \cite{Adamek04}, we add some comments for our particular case.  For $\mathbf{Dal}$, consider first the category $\mathbf{Alg}(\Sigma)$ the category of all the $\Sigma$-algebras.  This is a category is cocomplete  as proven in \cite{Tarlecki91}.  Now, we  define a functor $F : \mathbf{Alg}(\Sigma) \rightarrow \mathbf{Dal}$ as follows. For every $\Sigma$-algebra 
$A$, $F(A) = A/\cong$ where $\cong$ is the smallest congruence such that $\mathbf{D}/\cong \in |\mathbf{Dal}|$.  For instance,  if $T$ is the term algebra, then $F(T)$ is the Lindenbaum algebra. Furthermore,  if 
$\mathbf{A}$ is already a deontic action algebra then $\cong$ is just the equality.  $F$ is the left adjoint of $I:\mathbf{Dal} \rightarrow \mathbf{Alg}(\Sigma)$, the inclusion function.  Thus $\mathbf{Dal}$ is a reflective subcategory of $\mathbf{Alg}(\Sigma)$, since
$\mathbf{Alg}(\Sigma)$ is cocomplete and has free objects \cite{Tarlecki91},  by properties of reflective subcategories \cite{MacLane98},  we obtain that $\mathbf{Dal}$ is cocomplete and has free objects.
\end{proof}
The cocompleteness of $\mathbf{Dal}$ allows us to put together different algebras,  this is a common procedure when looking at formal systems as objects of categories,  see, for instance,  \cite{Goguen92}.

Now, we provide a categorical view to the duality between \DAL and more set based algebras (as show in \Cref{sec:algebraic-char}).  Particularly, we show that there is a duality between $\mathbf{Dal}$ and a category of topological spaces For doing so, we introduce some basic concepts, the interested reader is referred to \cite{Johnstone82}.
Let $\mathbf{BA}$ be the category of Boolean algebras and $\textbf{Stone}$ the category of Stone spaces, i.e.,  its objects are topological spaces that are totally disconnect, compact and Hausdorff   and its arrows are continuous maps.  Stone duality states that categories $\mathbf{BA}$ and $\textbf{Stone}$ are dually equivalent, i.e.,  there exist functors $S: \mathbf{BA}^{op} \rightarrow \mathbf{Stone}$ and 
$T: \mathbf{Stone}^{op} \rightarrow \mathbf{BA}$ and natural isomorphisms $\eta_b : I_{\mathbf{BA}} \rightarrow TS$,  $\Theta_s : I_{\mathbf{Stone}} \rightarrow ST$.   Where $I_{\mathbf{BA}}: \mathbf{BA} \rightarrow \mathbf{BA}$ and $I_{\mathbf{Stone}}: \mathbf{Stone} \rightarrow \mathbf{Stone}$ are the identity functors.
Intuitively, this states that categories 
$\mathbf{BA}$ and $\mathbf{Stone}$ are,  up to isomorphisms,  dually the same.

For obtaining the same result for the \DAL algebras, first,  we define the corresponding categories based on topological spaces.  Let $\Delta : \mathbf{Stone} \rightarrow \mathbf{Stone}^2$ the diagonal functor, that is, the functor sending each Stone space $s$ to the pair $(s,s)$ and each arrow $f:s \rightarrow s'$ to the pairs of arrows $(f,f)$.  Now, consider the comma category $\Delta \downarrow \Delta$, that is, the category whose objects are pairs of arrows $(f:s\rightarrow s', g: s \rightarrow s')$ where $f,g \in ||\mathbf{Stone}||$ and the arrows are pairs of arrow making the corresponding diagram to commute.  We define the normative Stone Spaces as follows:
\begin{definition} The category $\mathbf{NStone}$ (of Normative Stone spaces) is the full subcategory of $\Delta \downarrow \Delta$ such that for all the objects $(f,g) \in |\mathbf{NStone}|$ the equalizer of $f$ and $g$  is the initial element (in $\mathbf{Stone}$).
\end{definition} 
The condition in the definition above ensures that  every pair of functions composing an object in $\mathbf{NStone}$ are disjoint.  Now, we can prove our extension of Stone duality for \DAL algebras.
\begin{theorem}\label{theorem:duality} Categories $\mathbf{Dal}$ and $\mathbf{NStone}$ are dually equivalent.
\end{theorem}
\begin{proof}
For proving this first we define the functors $N: \mathbf{Dal}^{op} \rightarrow \mathbf{NStone}$ and $M: \mathbf{NStone}^{op} \rightarrow \mathbf{Dal}$.  $N$ is defined as follows.
Without loss of generality,   we fix a \DAL algebra  $\Algebra[D] = \tup{\Algebra[A], \Algebra[F], \E, \P, \F}$,  consider the Stone spaces $S(A)$ and $S(B)$ given by the Stone function (as explained above).  As $S$ is a 
contravariant functor we have continuous functions $S(\P)$ and $S(\F)$, hence we define $N(\Algebra[D]) = (S(\P),  S(\F))$.  For arrows, let
$f : \Algebra[D] \rightarrow \Algebra[D']$ (where $\Algebra[D'] = \tup{\Algebra[A'], \Algebra[F'], \E', \P', \F'}$),  which is defined by two homomorphisms $f_a$ and $f_b$. 
then $N(f) : N(\Algebra[D]) \rightarrow N(\Algebra[D'])$ is given by $(\Delta(S(f_a)), \Delta(S(f_b)))$.  

On the other hand,  the functor $M : \mathbf{NStone} \rightarrow \mathbf{Dal}$ is defined as follows.   Consider an object $(f,g) \in |\mathbf{NStone}|$ thus $f: s \rightarrow s'$ and $g:s \rightarrow s'$
where $s,s'$ are Stone spaces and $f,g$ continuous functions such that the equalizer of them is the initial object.  Furthermore, by Stone duality,  we have a contravariant functor $\mathit{Clop}: \mathbf{Stone}^{op} \rightarrow \mathbf{BA}$ (that smaps any Stone space to the Boolean algebra of its clopen sets).  Thus,   we can consider the Boolean  algebras $\mathit{Clop}(s)$ and $\mathit{Clop}(s')$, and Boolean homomorphisms $\mathit{Clop}(f): \mathit{Clop}(s') \rightarrow \mathit{Clop}(s)$
and $\mathit{Clop}(g): \mathit{Clop}(s') \rightarrow \mathit{Clop}(s)$ furthermore $f(x) \cap f(y) = \emptyset$ for any $x \in \mathit{Clop}(s')$, hence $(\mathit{Clop}(s), \mathit{Clop}(s'), \E,  \mathit{Clop}(f), \mathit{Clop}(g))$, being $\E(x,y) = \mathcal{U}(s)$ iff $x=y$, otherwise $\E(x,y) = \emptyset$.  Proving this is already a functor is direct since $\mathit{Clop}$ is a functor.

Now,  let us show that there are natural isomorphisms $\varepsilon: I_{\mathbf{NStone}} \rightarrow NM$ and $\eta : I_{\mathbf{Dal}} \rightarrow MN$.   Let $(f:s \rightarrow s',g:s \rightarrow s') \in |\mathbf{NStone}|$,
by stone duality we know that there is a $i :s \rightarrow S(\mathit{Clop}(s))$ that is a homeomorphism between the two topological spaces,  similarly we have an homeomorphism $j: s' \rightarrow S(\mathit{Clop}(s'))$. Thus $(i,j) : (f,g) \rightarrow S(Cl((f,g)))$ gives us the corresponding isomorphism $\eta_{(f,g)}$.  For $\epsilon$ the proof is similar, for a  \DAL algebra  $\Algebra[D] = \tup{\Algebra[A], \Algebra[F], \E, \P, \F}$ we consider the Boolean algebras $\mathit{Clop}(S(\Algebra[A]))$ and $\mathit{Clop}(S(\Algebra[F]))$ which by Stone duality are isomorphic to $\Algebra[A]$ and $\Algebra[F]$, similarly for functions 
$\P$ and $F$ which we obtain functions $\mathit{Clop}(S(\P))$ and  $\mathit{Clop}(S(\F))$ which are the as the original up to isomorphism, putting all this together we obtain the isomomorphism $\varepsilon_{\Algebra[D]}$. Naturality of $\varepsilon$ and $\eta$ follows from the naturality of the corresponding natural isomorphism of Stone duality.
\end{proof}

We end this section we some remarks about the categories corresponding to the other algebras defined in this paper. We have show that category $\mathbf{Dal}$ exhibits some nice properties, one of them, cocompleteness, provides a mechanism to put together different deontic action algebras, thus making possible to modularize the reasoning about normative systems in an algebraic way.  Furthermore,   we have shown that Stone duality can be extended to our algebras allowing us to obtain an equivalence between our algebras and certain topological spaces.  For the other logics described in this paper similar results hold.  Note that the proof of \Cref{theorem:cocompleteness}  uses basic facts of the $\Sigma$-algebras and properties of quasivarieties,  which can be also be applied to the rest of the algebras presented in this paper.  Furthermore,  for the intuitionistic logics presented earlier we can use the Esakia duality \cite{Esakia19} between Heyting algebras and Esakia spaces, which can be termed as an intuitionistic version of Stone duality. Using Esakia duality the same constructions as in \Cref{theorem:duality} can be used to provide duality results for $\DAL(\INT)$,  or any of the other logics. 




%The first important property is that these algebras are cocomplete and posees
\fi
\section*{Conclusion}
This paper aims to enhance our understanding of the computational complexity of computing various Shapley value variants. We found that for various ML models --- including decision trees, regression tree ensembles, weighted automata, and linear regression --- both local and global interventional and baseline SHAP can be computed in polynomial time under HMM modeled distributions. This extends popular algorithms, such as TreeSHAP, beyond their empirical distributional scope. We also establish strict complexity gaps between the various SHAP variants (baseline, interventional, and conditional) and prove the intractability of computing SHAP for tree ensembles and neural networks in simplified scenarios. Overall, we present SHAP as a versatile framework whose complexity depends on four key factors: \begin{inparaenum}[(i)] \item model type, \item SHAP variant, \item distribution modeling approach, \item and local vs. global explanations\end{inparaenum}. We believe this perspective provides deeper insight into the computational complexity of SHAP, paving the way for future work.




%We believe that our framework provides a more intricate understanding of SHAP computation complexity across different models, distributions, and variants, paving the way for further research.

Our work opens promising directions for future research. First, expanding our computational analysis to other SHAP-related metrics, such as asymmetric SHAP~\citep{frye20} and SAGE~\citep{covert2020understanding}, would be valuable. Additionally, we aim to explore more expressive distribution classes and relaxed assumptions beyond those in Section \ref{sec:tractable} while maintaining tractable SHAP computation. Finally, when exact computation is intractable (Section \ref{sec:intractable}), investigating the approximability of SHAP metrics through approximation and parameterized complexity theory~\citep{downey2012parameterized} is an important direction.

%Our work opens several promising avenues for future research on the computational properties of explainable AI methods, with a particular focus on SHAP. First, it would be interesting to broaden the computational analysis conducted in this work to include other popular SHAP-related metrics in the literature, such as asymmetric SHAP \cite{frye20} and SAGE \cite{covert2020understanding}. Also, in the future, we aim to explore more expressive distribution classes and relaxed distributional assumptions—extending beyond those examined in Section \ref{sec:tractable} —that still yield tractable SHAP computation. Finally, when exact computation proves intractable (Section \ref{sec:intractable}), it is worthwhile to theoretically investigate the question of the approximability of computing the SHAP metrics across various configurations, through the lens of approximation and parametrized complexity theory \cite{arora2009computational}.

%This paper aims to deepen our understanding of the computational complexity involved in obtaining different Shapley value variants. We found that for a variety of ML models, including decision trees, tree ensembles for regression, weighted automata, and linear regression models — computing both local and global interventional and baseline SHAP can be done in polynomial time when distributions are modeled by HMMs. This extends the distributional scope of popular algorithms like TreeSHAP, which is limited to empirical distributions. Additionally, we demonstrate a strict complexity gap between SHAP variants, showing that interventional and baseline SHAP can be strictly easier to compute than conditional SHAP. Despite these positive results, we uncovered intractability for various SHAP variants in neural networks and tree ensembles. Finally, we provided generalized complexity relations across SHAP variants. We believe that our framework offers a deeper understanding of the complexity involved in computing SHAP across various variants, models, distributions, as well as in both local and global computations, laying the groundwork for future research.

% \bmhead{Acknowledgements}

% Acknowledgements are not compulsory. Where included they should be brief. Grant or contribution numbers may be acknowledged.

%%%%%%  THE BIBLIOGRAPHY
%%%%%%  See the examples and format yours according to them

\bibliography{bibliography}

%%%%%%  THE APPENDIX
%
\section{Preliminaries}\label{sec:preliminaries}



%We denote by $(\Ac(x_\Ac),\Bc(x_\Bc))(z)$ a random execution of $\pi$ with private inputs $(x_\Ac,y_\Ac)$, and common input $z$.

%\Jnote{Move to DP}
% At the end of such an execution, the protocol outputs a public transcript denoted by the random variable $\trans_\pi(x_\Ac,x_\Ac,z)$ we denotes the common as $\out(\trans_\pi(x_\Ac,x_\Ac,z)$, and each party $\Pc \in \set{\Ac,\Bc}$ obtains his view denoted $\view^\Pc_\pi(x_\Ac,x_\Bc,z)$, which may also contain a ``local output'' \Jnote{Local} $\out^\Pc(x_\Ac,x_\Bc,z)$ (if the protocol specifies such an output). \Jnote{Common output, and parties output}


\subsection{Distributions and Random Variables}\label{sec:prelim:dist}
The support of a distribution $P$ over a finite set $\cS$ is defined by $\Supp(P) \eqdef \set{x\in \cS: P(x)>0}$. For a distribution or a random variable $D$, let $d\from D$ denote that $d$ was sampled according to $D$. Similarly,  for a set $\cS$, let $x \from \cS$ denote that $x$ is drawn uniformly from $\cS$, and denote by $\cU_{\cS}$ the uniform distribution over $\cS$. For a finite set $\cX$ and a distribution $C_X$ over $\cX$, we use the capital letter $X$ to denote the random variable that takes values in $\cX$ and is sampled according to $C_X$. The {\sf statistical distance} (\aka {\sf~variation distance}) of two distributions $P$ and $Q$ over a discrete domain $\cX$ is defined by $\sdist{P}{Q} \eqdef \max_{\cS\subseteq \cX} \size{P(\cS)-Q(\cS)} = \frac{1}{2} \sum_{x \in \cS}\size{P(x)-Q(x)}$. 
For a vector $x = (x_1,\ldots,x_n)$ and index $i\in [n]$, we let $x_{-i} = (x_1,\ldots,x_{i-1},x_{i+1},\ldots,x_n)$ and $x^{(i)} = (x_1,\ldots,x_{i-1}, -x_i, x_{i+1},\ldots,x_n)$, for a set $\cS \subseteq [n]$ we let $x_{\cS} = (x_i)_{i \in \cS}$ and $x_{-\cS} = (x_i)_{i \in [n]\setminus \cS}$, and for a vector $r \in \zo^n$ we let $x_r = (x_i)_{\set{i \colon r_i = 1}}$ and $x_{-r} = (x_i)_{\set{i \colon r_i = 0}}$.

%For $n \in \N$ we let $U_n$ be the uniform distribution over $\oo^n$, and let $S_n$ be the distribution induces by the sum of $n$ i.i.d.\ random variables, each is distributed according to $U_1$. Let $\cN(0,1)$ be the standard normal distribution.
%For a distribution $\cD$ and a function $f$, we define by $f(\cD)$ the distribution that is induced by the output of $f(x)$ for $x \from \cD$. 





% \begin{theorem}[\cite{McGregorMPRTV10}]\label{thm:sv-extracotr}
% 	\Enote{Remove if not needed}
% 	There is a constant $c$ to make the following holds. Let $X$ be an $\alpha$-SV source on $\{0,1\}^n$, let $Y$ be a source on $\{0,1\}^n$ with min-entropy at least $\beta n$ (independent from $X$), and let $Z=\ip{X,Y}\mbox{mod m}$ for some $m\in\mathbb{N}$. Then for every $\delta\in[0,1]$, the random variable $(Y,Z)$ is $\delta$-close to $(Y,U)$ where $U$ is uniform on $\mathbb{Z}_m$ and independent of $Y$, provided that
% 	$$
% 	n\geq c\cdot\frac{m^2}{\alpha\beta}\cdot\log(\frac{m}{\beta})\cdot\log(\frac{m}{\delta}).
% 	$$
% \end{theorem}



\Enote{I removed the definition of DP since it already appears in the intro}
\remove{
\subsection{Differential Privacy}\label{sec:prelim:DP}
We use the following standard definition of (information theoretic) differential privacy, due to \citet{DMNS06}. For notational convenience, we focus on databases over $\oo$.
\begin{definition}[Differentially private mechanisms]\label{def:mech}
	A randomized function $f\colon\oo^n\mapsto \zs$ is an {\sf $n$-size, $(\eps,\delta)$-differentially private mechanism} (denoted $(\eps,\delta)$-\DP) if for every neighboring $w,w'\in \oo^n$ and every function $g\colon \zs\mapsto \zo$, it holds that 
	$$
	\pr{g(f(w))=1}\leq e^{\eps}\cdot \pr{g(f(w'))=1} +\delta.
	$$ 	
	If $\delta=0$, we omit it from the notation.
\end{definition}
}


\subsubsection{Computational Differential Privacy}
There are several ways for defining computational differential privacy (see \cref{sec:related-works}). We use the most relaxed version due to \cite{BNO08}. For notational convenience, we focus on databases over $\oo$.
\begin{definition}[Computational differentially private mechanisms]\label{def:ComMech}
	A randomized function ensemble $f=\set{f_\pk\colon\oo^{n(\pk)}\mapsto \zs}$ is an {\sf $n$-size, $(\eps,\delta)$-computationally differentially private} (denoted $(\eps,\delta)$-$\CDP$) if for every poly-size circuit family $\set{\Ac_\pk}_{\pk\in \N}$, the following holds for every large enough $\pk$ and every neighboring $w,w'\in\oo^{n(\pk)}$:
	$$
	\pr{\Ac_\pk(f_\pk(w))=1}\leq e^{\eps(\pk)}\cdot \pr{\Ac_\pk(f_\pk(w'))=1} +\delta(\pk).
	$$ 
	If $\delta(\pk) = \negl(\pk)$, we omit it from the notation. 
\end{definition}



\subsubsection{Two-Party Differential Privacy}\label{sec:DP}
In this section we formally define distributed differential privacy mechanism (\ie protocols). %For the ease of notation, we consider protocol with no common input.

\begin{definition}\label{def:DP}%\Nnote{fix security parameter}
	A two-party protocol $\Pi=(\Ac,\Bc)$ is {\sf $(\eps,\delta)$-differentially private}, denoted $(\eps,\delta)$-$\DP$, if the following holds for every algorithm $\Dc$: let $\V^\Pc(x,y)(\pk)$ be the view of party $\Pc$ in a random execution of $\Pi(x,y)(1^\pk)$. Then for every $\pk,n \in \N$, $x\in \oo^n$ and neighboring $y,y'\in\oo^n$:
	\begin{align*}
	\pr{\Dc(V^\Ac(x,y)(\pk))=1}\le e^{\eps(\pk)}\cdot \pr{\Dc(V^\Ac (x,y')(\pk))=1}+\delta(\pk),
	\end{align*} 
	and for every $y\in \oo^n$ and neighboring $x,x'\in\oo^{n}$:
	\begin{align*}
	\pr{\Dc(V^\Bc(x,y)(\pk))=1}\le e^{\eps(\pk)}\cdot \pr{\Dc(V^\Bc (x',y)(\pk))=1}+\delta(\pk).
	\end{align*} 	
	Protocol $\Pi$ is {\sf $(\eps,\delta)$-computational differentially private}, denoted $(\eps,\delta)$-$\CDP$, if the above inequalities only hold for a non-uniform \ppt $\Dc$ and large enough $\pk$. We omit $\delta = \negl(\pk)$ from the notation. \footnote{Note that define we give for two-party differentially private protocols is a semi-honest definition, in which we ask for the security to hold when the parties interact in an honest execution of the protocol. Since we are proving a lower bound, starting from this weaker guarantee (as opposed to security against malicious players), yields a stronger result.}
\end{definition}
%We omit $\delta$ from the notation if $\delta$ is a negligible function of $n$.

%\Enote{simulation-based}
\begin{remark}[The definition for computational differential privacy we use]\label{rem:comDPChannel} 
	An alternative, stronger definition of computational differential privacy, known as simulation-based computational differential privacy, requires that the distribution of each party’s view be computationally indistinguishable from a distribution that ensures privacy in an information-theoretic sense. \cref{def:DP} is a weaker notion in comparison. Consequently, establishing a lower bound for a protocol that satisfies this weaker guarantee (as we do in this work) yields a stronger result.%Actually, our lower bound only requires the privacy to hold against \emph{uniform} external observer.
	%\Nnote{Maybe add: When only interesting in \Dp against external observer, the two definitions can be achieve using key-agreement and (single-party) \Dp mechanism. }
\end{remark}




\subsection{Useful Claims}
\remove{
In this section, we state generic lemmas and propositions that we will use later in our proofs.

The following lemma which we prove in \cref{sec:missing-proofs:distance-I}, measures the distance between two uniform stings conditioned one a random index $i$ either being fixed to $0$ or to $1$.

\def\distanceILemma{
    Let $R \la \zo^n$. For any (randomized) function $f:\{0,1\}^n\rightarrow \{0,1\}$ and $\alpha > 0$, it holds that
    \begin{align}\label{eq:f-alpha}
        \ppr{i \la [n]}{\size{\:\ex{f(R) \mid R_i = 0}-\ex{f(R) \mid R_i = 1}\:}\geq \alpha} \leq \frac{2}{n \alpha^2},
    \end{align}
    where the expectations are taken over $R$ and the randomness of $f$.
}

\begin{lemma}\label{lem:distance-I}
    \distanceILemma
\end{lemma}
}

The following two propositions state that given the output of a differentially private function, it is not possible to predict well even a random index (even if all other indexes are leaked). The first proposition handles the information-theoretic case and the second handles the computation case. Both propositions are proven in \cref{sec:missing-proofs:hard-to-guess}. 

\def\propHardToGuessInf{
    Let $f\colon \oo^n \rightarrow \cY$ be an $(\eps,\delta)$-\DP function, let $g \colon [n] \times \oo^{n-1} \times \cY \rightarrow \set{-1,1,\bot}$ be a (randomized) function, and let $X = (X_1,\ldots,X_n) \la \oo^n$. Then the following holds for every $i \in [n]$ where $X_i^* = g(i,X_{-i},f(X_1,\ldots,X_n))$:
    \begin{align*}
        \pr{X_i^* = X_i} \leq e^{\eps}\cdot \pr{X_i^* = -X_i} + \delta.
    \end{align*}
}

\begin{proposition}\label{prop:hard-to-guess-inf}
    \propHardToGuessInf
\end{proposition}


\def\propHardToGuessComp{
    Let $f = \set{f_{\pk} \colon \oo^{n(\pk)} \rightarrow \zo^{m(\pk)}}_{\pk \in \bbN}$ be an $(\eps,\delta)$-\CDP function ensemble, and let $\set{g_{\pk}}_{\pk \in \bbN}$ be a poly-size circuit family. Then, for large enough $\pk$ and $X = (X_1,\ldots,X_{n(\pk)}) \la \oo^{n(\pk)}$, the following holds for every $i \in [n(\pk)]$ where $X_i^* = g_{\pk}(i,X_{-i},f_{\pk}(X_1,\ldots,X_n))$:
    \begin{align*}
        \pr{X_i^* = X_i} \leq e^{\eps(\pk)}\cdot \pr{X_i^* = -X_i} + \delta(\pk).
    \end{align*}
}

\begin{proposition}\label{prop:hard-to-guess-comp}
    \propHardToGuessComp
\end{proposition}





\remove{
\Enote{Chao's old statement:}
\begin{lemma}\label{lem:distance-I-old}
        Let $R \la \zo^n$. 
	For any function $f:\{0,1\}^n\rightarrow \{0,1\}$ and $\alpha<0.01$, it holds that
	$$
	\Pr_{i\la[n]}\left[\: \size{\:\mathbb{E}[f(R) \mid R_i = 0]-\mathbb{E}[f(R) \mid R_i = 1]\:}\geq \alpha\right]\leq \frac{2+2\log(\frac{1}{\alpha})}{n\alpha^2}.
	$$
\end{lemma}
\begin{proof}
	Define $S_1=\{r \in \zo^n \colon f(r)=1\}$. Then for any $i\in[n]$, we have
	$$
	\begin{array}{rl}
		\size{\mathbb{E}[f(R) \mid R_i = 0]-\mathbb{E}[f(R) \mid R_i = 1]}
		&=\size{\Pr[R\in S_1|R_i=0]-\Pr[R\in S_1|R_i=1]}\\
		&=\size{\frac{\Pr[R_i=0|R\in S_1]\cdot\Pr[R\in S_1]}{\Pr[R_i=0]}-\frac{\Pr[R_i=1|R\in S_1]\cdot\Pr[R\in S_1]}{\Pr[R_i=1]}}\\
		&=\frac{2\size{S_1}}{2^n}\size{\Pr[R_i=0|R\in S_1]-\Pr[R_i=1|R\in S_1]}
	\end{array}
	$$
	When $|S_1|\leq \alpha\cdot 2^{n-1}$, we have $\size{\mathbb{E}[f(R) \mid R_i = 0]-\mathbb{E}[f(R) \mid R_i = 1]}\leq\frac{2\size{S_1}}{2^n}\leq \alpha$ for any $i\in[n]$. Hence, in the following, we assume $|S_1|> \alpha\cdot 2^{n-1}$.

	%Define $I_{bad}=\{i|\size{\Pr[R_i=0|R\in S_1]-\Pr[R_i=1|R\in S_1]}>2\alpha\}$ and $k=\size{I_{bad}}$, then for any $i\notin I_{bad}$, we have 
    %$$
    %\begin{array}{rl}
    %    2\alpha&\geq \size{\Pr[R_i=0|R\in S_1]-\Pr[R_i=1|R\in S_1]}\\
    %    &=\size{\frac{\Pr[R\in S_1|R_i=0]\cdot\Pr[R_i=0]}{\Pr[R\in S_1]}-\frac{\Pr[R\in S_1|R_i=1]\cdot\Pr[R_i=1]}{\Pr[R\in S_1]}}\\
    %    &=\size{\Pr[R\in S_1|R_i=0]-\Pr[R\in S_1|R_i=1]}\cdot\frac{1}{2\Pr[R\in S_1]}\\
    %    &\geq \size{\mathbb{E}[f(R) \mid R_i = 0]-\mathbb{E}[f(R) \mid R_i = 1]}\cdot \frac{1}{2},
    %\end{array}
    %$$ 
    %where the last inequality is because $\Pr[R\in S_1]\leq 1$. So that $\size{\mathbb{E}}[f(R) \mid R_i = 0]-\mathbb{E}[f(R) \mid R_i = 1]\leq %4\alpha$.
    Define $I_{bad}=\{i \colon \size{\Pr[R_i=0|R\in S_1]-\Pr[R_i=1|R\in S_1]} \geq 2\alpha\}$ and $k=\size{I_{bad}}$, and denote $I_{bad}=\{i_1,\dots,i_k\}$. Define $(X_{i_1}, \ldots X_{i_k}) = (R_{i_1},\dots,R_{i_k})\mid_{R \in S_1}$. 
    Consider the min-entropy
	$$
	\begin{array}{rl}
		H_{min}(X_{i_1},\dots,X_{i_k})&\leq H(X_{i_1},\dots,X_{i_k})\\
		&\leq \sum_{j=1}^k H(X_{i_j})\\
		&\leq k\cdot \left(-(\frac{1}{2}+2\alpha)\cdot\log(\frac{1}{2}+2\alpha)-(\frac{1}{2}-2\alpha)\cdot\log(\frac{1}{2}-2\alpha)\right)\\
            &=k\cdot \left(-(\frac{1}{2}+2\alpha)\cdot(\log(1+4\alpha)-1)-(\frac{1}{2}-2\alpha)\cdot(\log(1-4\alpha)-1)\right)\\
            &=k\cdot \left(1-(\frac{1}{2}+2\alpha)\cdot\log(1+4\alpha)-(\frac{1}{2}-2\alpha)\cdot\log(1-4\alpha)\right),
		
	\end{array}
	$$
	where $H_{min}(Y)$ is the minimum entropy of $Y$ and $H(Y)$ is the Shannon entropy of $Y$.\Enote{add to preliminaries.}
        The third inequality holds since by the definition of $I_{bad}$, for every $j \in [k]$ it holds that $\size{\pr{X_{i_j} = 1}-\pr{X_{i_j} = 0}} > 2\alpha$, and therefore $H(X_{i_j}) \leq H(1/2 + 2\alpha)$\Enote{define}.
	
	Therefore, there exists $b_1,\dots,b_k\in\{0,1\}$, such that 
	
	\begin{align}\label{eq:min-entropy-result}
		\Pr\left[(R_{i_1},\ldots,R_{i_k}) = (b_1,\ldots,b_k) \mid R\in S_1\right]
		&= \pr{(X_{i_1},\ldots,X_{i_k}) = (b_1,\ldots,b_k)}\\
		&= 2^{-H_{min}(X_{i_1},\dots,X_{i_k})}\nonumber\\
		&\geq 2^{k\cdot \left(-1+(\frac{1}{2}+2\alpha)\cdot\log(1+4\alpha)+(\frac{1}{2}-2\alpha)\cdot\log(1-4\alpha)\right)}.\nonumber
	\end{align}
	
	Let $S_{bad}=\{r \in \zo^n  \colon \set{(r_{i_1},\ldots,r_{i_k}) = (b_1,\ldots,b_k)} \land \set{r\in S_1}\}$.
	It holds that
	\begin{align*}
		|S_{bad}|
		&= \size{S_1} \cdot \Pr\left[(R_{i_1},\ldots,R_{i_k}) = (b_1,\ldots,b_k) \mid R\in S_1\right]\\
		&\geq \alpha\cdot 2^{n-1}\cdot2^{k\cdot \left(-1+(\frac{1}{2}+2\alpha)\cdot\log(1+4\alpha)+(\frac{1}{2}-2\alpha)\cdot\log(1-4\alpha)\right)},
	\end{align*} 
	where the inequality holds by \cref{eq:min-entropy-result} and since $\size{S_1} \geq \alpha\cdot 2^{n-1}$.
	Notice that any string in $S_{bad}$ depends on at most $n-k$ bits. It implies that $|S_{bad}|\leq 2^{n-k}$. Therefore, we have
	$$
	\begin{array}{rl}
		&2^{n-k}\geq \alpha\cdot 2^{n-1}\cdot2^{k\cdot \left(-1+(\frac{1}{2}+2\alpha)\cdot\log(1+4\alpha)+(\frac{1}{2}-2\alpha)\cdot\log(1-4\alpha)\right)} \\
		\Rightarrow& n-k \geq \log \alpha+n-1+k\cdot \left(-1+(\frac{1}{2}+2\alpha)\cdot\log(1+4\alpha)+(\frac{1}{2}-2\alpha)\cdot\log(1-4\alpha)\right)\\
		\Rightarrow& 1-\log \alpha \geq k\cdot((\frac{1}{2}+2\alpha)\cdot\log(1+4\alpha)+(\frac{1}{2}-2\alpha)\cdot\log(1-4\alpha))\\
		\Rightarrow& 1-\log \alpha \geq k\cdot(4\alpha\cdot\log(1+4\alpha)+(\frac{1}{2}-2\alpha)\cdot\log(1-16\alpha^2))\\
        \Rightarrow& 1-\log\alpha \geq k\cdot(15.9\alpha^2-8\alpha^2+32\alpha^3)=k\cdot(7.9\alpha^2+32\alpha^3)>0.5k\alpha^2\\
		\Rightarrow& k\leq \frac{2-2\log \alpha}{\alpha^2} = \frac{2+2\log (1/\alpha)}{\alpha^2},
	\end{array}
	$$
	Where the third transition holds since 
	\begin{align*}
		\lefteqn{(\frac{1}{2}+2\alpha)\cdot\log(1+4\alpha)+(\frac{1}{2}-2\alpha)\cdot\log(1-4\alpha)}\\
		&= 4\alpha\cdot\log(1+4\alpha) + (\frac{1}{2}-2\alpha)\paren{\log(1+4\alpha)+\log(1-4\alpha)}\\
		&= 4\alpha\cdot\log(1+4\alpha)+(\frac{1}{2}-2\alpha)\cdot\log(1-16\alpha^2),
	\end{align*}
	and the forth transition holds since $4\alpha\cdot\log(1+4\alpha)+(\frac{1}{2}-2\alpha)\cdot\log(1-16\alpha^2) > 15.9\alpha^2-8\alpha^2+32\alpha^3$ for $\alpha < 0.01$.
	Thus, we conclude that 
	$$
	\Pr_{i\la[n]}\left[\size{\mathbb{E}[f(R) \mid R_i=0]-\mathbb{E}[f(R) \mid R_i = 1]}\geq \alpha\right]\leq \frac{k}{n}\leq \frac{2+2\log (1/\alpha)}{n\alpha^2}.
	$$
\end{proof}
}


\subsection{Channels and Two-Party Protocols}\label{sec:protocol}

\paragraph{Channels.}A channel is simply a distribution of a pair of tuples defined as follows. 
\begin{definition}[Channels]\label{def:channel} A {\sf channel} $C_{(X,U)(Y,V)}$ of size $\isize$ over alphabet $\Sigma$ is a probability distribution over $(\Sigma^\isize \times\zo^\ast) \times(\Sigma^\isize \times\zo^\ast)$. The ensemble $C_{(X,U)(Y,V)}= \set{C_{(X_\pk,U_\pk)(Y_\pk,V_\pk)}}_{\pk\in \N}$ is an $\isize$-size channel ensemble, if for every $\pk\in \N$, $C_{(X_\pk,U_\pk)(Y_\pk,V_\pk)}$ is an $\isize(\pk)$-size channel. %We denote a channel of size one by a \emph{single-bit} channel. 
We refer to $X$ and $Y$ as the {\sf local outputs}, and to $U$ and $V$ as the {\sf views}.	
\end{definition}

We view a  channel as the experiment in which there are two parties $\Ac$ and $\Bc$.  Party $\Ac$ receives ``output'' $X$ and ``view'' $U$, and party $\Bc$ receives ``output'' $Y$ and ``view'' $V$. Unless stated otherwise, the channels we consider are over the alphabet $\Sigma = \oo$. We naturally identify channels with the distribution that characterizes their output.








\subsubsection{Two-Party Protocols}

A two-party protocol $\Pi=(\Ac,\Bc)$ is \ppt if the running time of both parties is polynomial in their input length. We let $\Pi(x,y)(z)$ or $(\Ac(x),\Bc(y))(z)$ denote a random execution of $\Pi$ on a common input $z$, and private inputs $x,y$.%We assume \wlg that a protocol has a common output (part of its transcript).\Jnote{This is not really the case we consider in this paper..}

\begin{definition}[Oracle-aided protocols]\label{def:ChannelAidedProtocol}
	In a two-party protocol $\Pi$ with oracle access to a {\sf protocol} $\Psi$, denoted $\Pi^\Psi$, the parties make use of the \textit{next-message function} of $\Psi$.\footnote{The function that on a partial view of one of the parties, returns its next message.} In a two-party protocol $\Pi$ with oracle access to a {\sf channel} $C_{Z W}$, denoted $\Pi^C$, the parties can jointly invoke $C$ for several times. In each call, an independent pair $(z,w)$ is sampled according to $C_{Z W}$, one party gets $z$, the other gets $w$.
\end{definition}


\begin{definition}[The channel of a protocol]\label{def:ChannlOfProtocol}
	For a no-input two-party protocol $\Pi= (\Ac,\Bc)$, we associate the channel $C_\Pi$, defined by $\C_\Pi= C_{(X, U),(Y, V)}$, where $X$ and $Y$ are the local outputs of $\Ac$ and $\Bc$ (respectively) and
	$U$ and $V$ are the local views of $\Ac$ and $\Bc$ (respectively).
    
	For a two-party protocol $\Pi$ that gets a security parameter $1^\pk$ as its (only, common) input, we associate the channel ensemble $ \set{C_{\Pi(1^\pk)}}_{\pk\in \N}$. 
\end{definition}

\begin{definition}[$(\alpha,\gamma)$-Accurate channel]\label{def:accurate-func}
	A channel $C = C_{(X, U),(Y, V)}$ is {\sf $(\alpha,\gamma)$-accurate for the function $f$}, if $\ppr{C}{\size{\out(V)-f(X,Y)}\leq \alpha}\ge \gamma$, where $\out(V)$ is the designated output.
    A channel ensemble $C_{(X, U),(Y, V)}= \set{C_{(X_\pk, U_\pk),(Y_\pk, V_\pk)}}_{\pk\in \N}$ is  $(\alpha,\gamma)$-accurate for  $f$ if $C_{(X_\pk, U_\pk),(Y_\pk, V_\pk)}$ is $(\alpha(\pk),\gamma(\pk))$-accurate for $f$, for every $\pk \in \N$.
\end{definition}

\subsubsection{Differentially Private Channels}\label{sec:DPChannel}
Differentially private channels are naturally defined as follows:
\begin{definition}[Differentially private channels]\label{def:DPChannel}
	An $n$-size channel $C = C_{(X, U),(Y, V)}$ with $X, Y$ over $\oo^n$ 
	is {\sf$(\eps,\delta)$-differentially private} (denoted $(\eps,\delta)$-$\DP$) if for every $x \in \Supp(X)$ there exists an $n$-size $(\eps,\delta)$-$\DP$ mechanisms $\Mc_x$ such that $(X,Y,U) \equiv (X,Y,\Mc_X(Y))$, and for every $y \in \Supp(Y)$ there exists an $n$-size $(\eps,\delta)$-$\DP$ mechanisms $\Mc_y'$ such that $(X,Y,V) \equiv (X,Y,\Mc_Y'(X))$. In addition, we say that the channel is \emph{uniform} if $X$ and $Y$ are independent random variables uniformly distributed in $\oo^n$. 
\end{definition}

\begin{definition}[Computational differentially private channels]\label{def:CDPChannel}
	An $n$-size channel ensemble $C = \set{C_{(X_\pk, U_\pk),(Y_\pk, V_\pk)}}_{\pk\in\N}$ with $X_\pk, Y_\pk$ over $\oo^n$ 
	is {\sf$(\eps,\delta)$-computationally differentially private} (denoted $(\eps,\delta)$-$\CDP$) if for every ensemble $\set{x_\pk \in \Supp(X_\pk)}_{\pk\in\N}$ there exists an $n$-size $(\eps,\delta)$-\CDP mechanisms ensemble $\set{\Mc_{x_\pk}}_{\pk\in\N}$ such that $(X_\pk,Y_\pk,U_\pk) \equiv (X_\pk,Y_\pk,\Mc_{X_\pk}(Y_\pk))$, for every $\pk\in\N$, and for every ensemble $\set{y_\pk \in \Supp(Y_\pk)}_{\pk\in\N}$ there exists an $n$-size $(\eps,\delta)$-$\CDP$ mechanisms ensemble $\set{\Mc'_{y_\pk}}_{\pk\in\N}$ such that $(X_\pk,Y_\pk,V_\pk) \equiv (X_\pk,Y_\pk,\Mc_{Y_\pk}'(X_\pk))$ for every $\pk\in \N$. In addition, we say that the channel is \emph{uniform} if $X_\pk$ and $Y_\pk$ are independent random variables uniformly distributed in $\{\pm 1\}^n$ for all $\pk\in\N$.
\end{definition}




% \begin{lemma}~\label{lem:dp-sv-source}
% 	Let $P$ be an $\varepsilon$-DP randomized protocol. Let $X$ and $Y$ be independent random variables uniformly distributed in $\{\pm 1\}^n$ and let random variable $\Pi(X,Y)$ denote the transcript of running $P(X,y)$. Then for every $\pi\in Supp(\Pi)$, the random variables corresponding to the inputs conditioned on transcript $\pi$, $X_\pi$ and $Y_\pi$, are independent $e^{-\varepsilon}$-strong SV source.
% \end{lemma}





\subsubsection{Weak Erasure Channel (\WEC)}

\begin{definition}[\WEC]\label{def:WEC}
	A channel $((O_A,V_A), (O_B,V_B))$ with $O_A \in \set{0,1}$ and $O_B \in \set{0,1,\bot}$ is a {\sf weak erasure channel}, denoted $(\alpha,p,q)$-$\WEC$, if:
	\begin{itemize}
		%\item $O_A\in \set{-1,1}$ and $O_B\in \set{-1,1,\bot}$.
		\item Random erasure: $\pr{O_B = \perp} = 1/2$.
		
		\item Agreement: $\pr{O_A\ne O_B\mid O_B\ne \bot}\le \alpha$.
		
		\item Secrecy:
		
		\begin{enumerate}
			\item For every algorithm $\Dc$ it holds that\label{WEC:item:A}
			\begin{align*}
				%\size{\pr{\Ac(O_A,V_A) = 1 \mid O_B \neq \perp} - \pr{\Ac(O_A,V_A) = 1 \mid O_B = \perp}} \le p
				\size{\pr{\Dc(V_A) = 1 \mid O_B \neq \perp} - \pr{\Dc(V_A) = 1 \mid O_B = \perp}} \le p
			\end{align*}
			(Alice doesn't know if $O_B = \perp$.)
			
			\item For every algorithm $\Dc$ it holds that\label{WEC:item:B}
			\begin{align*}
				\pr{\Dc(V_B) = O_A \mid O_B=\bot} \leq \frac{1+q}{2}.
			\end{align*}
			(i.e., if $O_B=\bot$, Bob don't know what is the value of $O_A$).
			
			%\item $SD((O_A U|O_B=\bot),(O_A U|O_B\ne \bot))\le p$ (The sender don't know if $O_B=\bot$).
			
			%\item $SD(V O_A|O_B=\bot,V(-O_A)|O_B=\bot)\le q$ (If $O_B=\bot$, Bob don't know what the value of $O_A$).
		\end{enumerate}
	\end{itemize}
   We say that a channel ensemble $C=\set{C_\pk}_{\pk\in N}$ is a {\sf computational weak erasure channel}, denoted $(\alpha,p,q)$-\CompWEC, if for every \ppt algorithm $\Dc$ and every sufficiently large $\pk\in\N$, $C_\pk$ satisfies the properties stated in the items above, where the secrecy property holds with respect to a \ppt algorithm $\Dc$. A protocol $\Lambda$ is said to be $(\alpha,p,q)$-$\CompWEC$, if the ensemble induces by the protocol (that is, $C=\set{C_{\Lambda(\pk)}}_{\pk\in\N}$) is $(\alpha,p,q)$-$\CompWEC$.  
\end{definition}



\subsubsection{Approximate Weak Erasure Channel (\AWEC)}\label{sec:AWEC}

\begin{definition}[\AWEC]\label{def:AWEC}
	A channel $C = ((O_A,V_A), (O_B,V_B))$ over $([-n,n] \times \zo^*) \times (([-n,n] \cup \bot)  \times \zo^*)$ is an {\sf approximate weak erasure channel}, denoted $(\ell,\alpha,p,q)$-\AWEC if:
	\begin{itemize}
		
		\item Random erasure: $\pr{O_B = \perp} = 1/2$.
		
		\item Accuracy: $\pr{\size{O_A - O_B} > \ell \mid O_B \ne \bot}\le \alpha$.
		
		\item Secrecy:
		
		\begin{enumerate}
			\item For every algorithm $\Dc$ it holds that\label{AWEC:item:A}
			\begin{align*}
				%\size{\pr{\Ac(O_A,V_A) = 1 \mid O_B \neq \perp} - \pr{\Ac(O_A,V_A) = 1 \mid O_B = \perp}} \le p
				\size{\pr{\Dc(V_A) = 1 \mid O_B \neq \perp} - \pr{\Dc(V_A) = 1 \mid O_B = \perp}} \le p
			\end{align*}
			(Alice doesn't know if $O_B=\bot$).
			
			\item For every algorithm $\Dc$ it holds that\label{AWEC:item:B}
			\begin{align*}
				\pr{\size{\Dc(V_B) - O_A} \leq 1000 \ell \mid O_B=\bot} \leq q.
			\end{align*}
			(i.e., if $O_B=\bot$, Bob can't estimate the value of $O_A$ with error $\leq 1000 \ell$).
		\end{enumerate}
	\end{itemize}
     We say that a channel ensemble $C=\set{C_\pk}_{\pk\in N}$ is a {\sf computational approximate weak erasure channel}, denoted $(\ell,\alpha,p,q)$-\CompAWEC, if for every \ppt algorithm $\Dc$ and every sufficiently large $\pk\in\N$, $C_\pk$ satisfies the properties stated in the items above. A protocol $\Gamma$ is said to be $(\ell,\alpha,p,q)$-$\CompAWEC$, if the ensemble induced by the protocol (that is, $C=\set{C_{\Gamma(\pk)}}_{\pk\in\N}$) is $(\ell,\alpha,p,q)$-$\CompAWEC$.  
\end{definition}

We will make use of the following lemma, which shows that for some choices of the parameters, \AWEC implies \WEC. The lemma is proven in \cref{sec:AWEC-to-WEC}.

\begin{lemma}\label{lemma:AWEC-to-WEC}
	For every $\ell> 0$, there exists a \ppt protocol $\Lambda = (\Pc_1,\Pc_2)$ such that given an oracle access to an $(\ell,\alpha,p,q)$-\AWEC $C$, the channel $\tilde{C}$ induced by $\Lambda^C$ is $(\alpha'=\alpha+0.001,\: p' = p ,\:  q' = 1/2 + 2(q+0.01))$-\WEC.
	Furthermore, the proof is constructive in a black-box manner:
	\begin{enumerate}
		\item There exists an oracle-aided \ppt algorithm $\Ec_1$ such that for every channel $C = ((\OA,\VA), (\OB,\VB))$ and algorithm $\Dc$ violating the \WEC secrecy property~\ref{WEC:item:A} of $\tilde{C}$, algorithm $\Ec_1^{\Dc}$ violates the \AWEC secrecy property~\ref{AWEC:item:A} of $C$.
		
		\item There exists an oracle-aided \ppt algorithm $\Ec_2$ such that for every channel $C = ((\OA,\VA), (\OB,\VB))$ and algorithm $\Dc$ violating the \WEC secrecy property~\ref{WEC:item:B} of $\tilde{C}$, algorithm $\Ec_2^{\Dc}$ violates the \AWEC secrecy property~\ref{AWEC:item:B} of $C$.
	\end{enumerate}
\end{lemma}

Since \cref{lemma:AWEC-to-WEC} is constructive, the following is an immediate corollary.
\begin{corollary}\label{cor:CompAWEC to CompWEC}
There exists an oracle aided \ppt protocol $\Lambda$, such that given a protocol $\Gamma$ that induces $(\ell,\alpha,p,q)$-\CompAWEC, it holds that $\Lambda^\Gamma$ is $(\alpha'=\alpha+0.001,\: p' = p ,\:  q' = 1/2 + 2(q+0.01))$-\CompWEC.  
\end{corollary}
\begin{proof}[Proof of \ref{cor:CompAWEC to CompWEC}]
Let $\Lambda$ be the \ppt algorithm guaranteed  by Lemma \ref{lemma:AWEC-to-WEC}. Given an $(\ell,\alpha,p,q)$-\CompAWEC protocol $\Gamma$, we define $\Lambda(\pk)={\Lambda^{\Gamma(\pk)}(\pk)}$. Assume towards a contradiction that $\Lambda$ is not a $(\alpha',p',q')$-\CompWEC. It follows that there exists a \ppt $\Dc$ that for infinity many $\pk\in\N$ contradicts one of the \WEC secrecy properties of channel ensemble $\set{C_{\Lambda(\pk)}}_{\pk\in\N}$. Fix $\pk\in\N$ for which this holds. By Lemma \ref{lemma:AWEC-to-WEC}, there exists a \ppt $\Ec^\Dc$ that for every such $\pk$  contradicts one of the secrecy properties of the channel $C_{\Gamma(\pk)}$. This implies that for infinity many $\pk\in\N$, $\Ec^\Dc$  contradict the secrecy of the channel ensemble $\set{C_{\Gamma(\pk)}}_{\pk\in\N}$, which is a contradiction since this would means that $\Gamma$ is not a $(\ell,\alpha,p,q)$-\CompAWEC.       
\end{proof}



\subsection{Oblivious Transfer (\OT)}

\paragraph{Secure Computation.}
We use the standard notion of securely computing a functionality, \cf  \cite{Goldreich04}.
\begin{definition}[Secure computation]\label{def:SFE}
	A two-party protocol {\sf securely computes a functionality $f$}, if it does so according to the real/ideal paradigm.   We add the term perfectly/statistically/computationally/non-uniform computationally, if the simulator's output is  perfect/statistical/computationally indistinguishable/  non-uniformly indistinguishable from  the real distribution.  The protocol have the above notions of security {\sf against semi-honest  adversaries}, if its security only  guaranteed to holds against an adversary that follows the prescribed protocol.   Finally, for the case of perfectly secure computation, we naturally apply the above notion also to the non-asymptotic case: the protocol with no security parameter perfectly  compute a functionality $f$.
	
	A two-party protocol {\sf securely computes a functionality ensemble $f$ with oracle to a channel $C$}, if it does so according to the above definition when the parties have access to a trusted party computing $C$. All the above adjectives naturally extend to this setting.
\end{definition}

\paragraph{Oblivious Transfer.}
The (one-out-of-two) oblivious transfer functionality is defined as follows.
\begin{definition}[oblivious transfer functionality $f_{\OT}$]\label{def:OTfunc}
	The oblivious transfer functionality over $\zo \times (\zs)^2$ is defined by  $f_{\OT} (i,(\sigma_0,\sigma_1)) = (\perp,\sigma_i)$.
\end{definition}
A protocol is $\ast$ secure OT,   for \\$\ast\in \set{\text{semi-honest statistically/computationally/computationally non-uniform}}$, if it  compute the $f_{\OT}$  functionality with $\ast$ security.





% \begin{definition}[Computational oblivious transfer, semi-honest model]
% A protocol $\Pi=(\Ac,\Bc)$ is a semi-honest 1-out-of-2 computational oblivious transfer (comp-OT) protocol if the following holds. Given a common input $1^{\pk}$, the parties $\Ac$ and $\Bc$ run the protocol $\Pi(1^\pk)$ (in an honest manner) and    
% $\Ac$ outputs $X=(m_1,m_2)\in \zo\times\zo$ and has a view $U$ and $\Bc$ outputs $Y=(i,\hat{m})\in\zo\times\zo$ and has a view $V$, and the following properties are satisfied:
% \begin{enumerate}
%     \item \textbf{Correctness:} 
%     $\pr{\hat{m}\neq m_i}<\negl(\pk).$ 
    
%     \item \textbf{A's Privacy:} For every \ppt $\Dc$ and every sufficiently large $\pk$:
%     $\pr{\Dc(V)=m_{i-1}}<(1+\negl(\pk))/2$
    
%     \item \textbf{B's Privacy:} For every \ppt $\Dc$ and every sufficiently large $\pk$:
%     $\pr{\Dc(U)=i}<(1+\negl(\pk))/2$  
% \end{enumerate}
% \end{definition}

We make use of the following useful results by Wullschleger on oblivious transfer amplification from weak channels.
\begin{theorem}[\cite{Wullschleger09}, from \WEC to statistically secure \OT]\label{thm:WEC TO OT IT}
    There exists an oracle aided protocol $\Pi$ such that the following holds: Given a $(\alpha,p,q)$-\WEC $C$, if $44(\alpha+p)\le 1-q$ then $\Pi^{C}(1^\pk)$ is a semi-honest statistically secure \OT.
\end{theorem}

The following computational version of \cref{thm:WEC TO OT IT} is implicit in \cite{Wullschleger09} and is based on the computational proof explicitly stated in \cite{Wul07} (see Section 6 in \cite{Wullschleger09} for discussion).   

\begin{theorem}[\cite{Wullschleger09,   Wul07}, from \CompWEC to computinally secure \OT]\label{thm:WEC TO OT Comp}
    There exists an oracle aided protocol $\Pi$ such that the following holds: Given a $(\alpha,p,q)$-\CompWEC protocol $\Lambda$, if $44(\alpha+p)\le 1-q$ then $\Pi^{\Lambda}$ is a semi-honest computational secure \OT.
\end{theorem}



% \begin{definition}[Computational 1-out-of-2 Oblivious Transfer, semi-honest model]
% A protocol $\Pi=(\Ac,\Bc)$ is a semi-honest 1-out-of-2 $(\eps,\alpha,\beta)$-oblivious transfer (OT) protocol if the following holds. 

% The parties $\Ac$ and $\Bc$ run the protocol (in an honest manner) and    
% $\Ac$ outputs $X=(m_1,m_2)\in \zo\times\zo$ and has a view $U$ and $\Bc$ outputs $Y=(i,\hat{m})\in\zo\times\zo$ and has a view $V$, and following properties are satisfied:
% \begin{enumerate}
%     \item \textbf{Correctness:} 
%     $\pr{\hat{m}\neq m_i}<\eps.$ 
    
%     \item \textbf{A's Privacy:} For every adversary $\Dc$:
%     $\pr{\Dc(V)=m_{i-1}}<(1+\alpha)/2$
    
%     \item \textbf{B's Privacy:} For every adversary $\Dc$: $\pr{\Dc(U)=i}<(1+\beta)/2$  
% \end{enumerate}
% \end{definition}

\end{document}
