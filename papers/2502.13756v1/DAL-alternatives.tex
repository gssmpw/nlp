\subsection{Previously Proposed Variants of \DAL}\label{section:dals}
%\subsection{Algebraizing Other Deontic Action Logics}\label{section:dals}

Segerberg's foundational work~\cite{Segerberg1982} laid the groundwork for a family of closely related deontic action logics. Building on this foundation, the five systems introduced in~\cite{Trypuz15} are particularly interesting as they address specific open issues in the field of Deontic Logic \textemdash such as the \emph{principle of deontic closure}.
We show how our algebraic framework can be easily extended to characterize each of these logics, showcasing the adaptability and versatility of deontic action algebras.
In the rest of this section, we use $\mathsf{a}$ to indicate a basic action symbol and $a$ to indicate its corresponding interpretation in an algebra.

% To introduce the algebraic counterparts of each $\NDAL{i}$, we need two preliminary definitions: that of a \emph{free deontic action algebra}, and that of an \emph{atomic deontic action algebra}. These are straightforward extensions of the corresponding concepts in Boolean algebras.
%
% \begin{definition}
% 	A deontic action algebra $\DAlgebra = \langle \Algebra[A], \Algebra[F], \E, \P, \F \rangle$ is freely (resp. finitely) generated by a subset $G \subseteq A$ of generators iff $\Algebra[A]$ is freely (resp. finitely) generated by $G$.
% 	We say that $\DAlgebra$ is atomic if $\Algebra[A]$ is atomic.
% \end{definition}
%
% We are now in position to introduce each deontic logic \NDAL{i} together with its algebraic counterpart.

The first of the five systems in~\cite{Trypuz15}, here called \NDAL{1}, is obtained from \DAL by adding the set $\set{{\forb(\mathsf{a}) \lor \perm(\mathsf{a})}}{\mathsf{a} \in \bact}$ of formulas as additional axioms.
Intuitively, this new set of axioms aims to capture the so-called \emph{principle of deontic closure}\textemdash what is not forbidden is permitted\textemdash at the level of basic actions (i.e., action generators).
The algebraic counterpart of \NDAL{1} is determined by the class of deontic action algebras whose algebra of actions is generated by a set of generators s.t.\ the condition $\F(a) \lor \P(a) =_{\Algebra[F]} \top$ holds for every generator $a$.

% \begin{definition}\label[definition]{def:dal:1}
% 	The class $\mathcal{D}_1$ contains all the deontic action algebras in $\mathcal{D}_0$ that are freely generated by a set $G$. Moreover, every deontic action algebra in $\mathcal{D}_1$ satisfies the following equations:
% \begin{equation}\label{eq:dal1}
% 	\F[x] \lor \P[x] =_{\Algebra[F]} \top \qquad \mbox{for every $x \in G$}
% \end{equation}
% \end{definition}

The second system, here called \NDAL{2}, is obtained from \NDAL{1} by adding the formula $\perm(\bar{\mathsf{a}}_0 \sqcap \dots \sqcap \bar{\mathsf{a}}_n) \lor \forb(\bar{\mathsf{a}}_0 \sqcap \dots \sqcap \bar{\mathsf{a}}_n)$ as an additional axiom of the logic, under the proviso that $\bact = \{\mathsf{a}_0, \dots, \mathsf{a}_n\}$ for some $n \in \mathbb{N}_0$; i.e., under the proviso that there are finitely many basic action symbols.
Intuitively, this additional axiom states that not performing any of the basic actions is permitted or forbidden.
% On the algebraic side, considering a finite set $\bact$ of basic actions results in the algebra $\Algebra[A]$ of actions being a \emph{finite atomic} Boolean algebra.
The algebraic counterpart of \NDAL{2} corresponds to the class of deontic action algebras with a finitely generated atomic Boolean algebra of actions $\Algebra[A]$ satisfying the condition ${
	\P(%
		\bar{a}_1 \sqcap
		\dots \sqcap
		\bar{a}_n
	)
	\lor
	\F(%
		\bar{a}_1 \sqcap
		\dots \sqcap
		\bar{a}_n
	) =_{\Algebra[F]} \top}$
for $\{a_0, \dots, a_n\}$ the set of generators of $\Algebra[A]$.

The third system, \NDAL{3}, is obtained from \NDAL{2} by adding $(\mathsf{a}_0 \sqcup \dots \sqcup \mathsf{a}_n) = \uact$ as an additional axiom.
Intuitively, this new axiom indicates that the agent can only perform actions in $\{\mathsf{a}_1,\dots, \mathsf{a}_n\}$.
The algebraic counterpart of \NDAL{3} corresponds to the subclass of $\NDAL{2}$ further satisfying the condition $a_0 \sqcup \dots \sqcup a_n = \uact$. 

The fourth system, \NDAL{4}, aims to capture the principle of deontic closure at the level of ``atomic'' actions.
Formally, the language of the logic assumes a finite set $\{\mathsf{a}_0, \dots, \mathsf{a}_n\}$ of basic action symbols.
Its axiomatization adds all the formulas in
	$\set
		{\perm({\tilde{\mathsf{a}}_0} \sqcap \dots \sqcap {\tilde{\mathsf{a}}_n}) \lor \forb({\tilde{\mathsf{a}}_0} \sqcap \dots \sqcap {\tilde{\mathsf{a}}_n})}
		{\tilde{\mathsf{a}}_i \in \{\mathsf{a}_i, \bar{\mathsf{a}}_i\}}$ as additional axioms to \DAL.
The algebraic counterpart of \NDAL{4} corresponds to the class of all deontic action algebras with a finitely generated and atomic algebra of actions, whose atoms $a$ satisfy the condition ${\P(a) \lor \F(a) =_{\Algebra[F]} \top}$.

The fifth and last system in \cite{Trypuz15}, here called \NDAL{5}, is the union of \NDAL{3} and \NDAL{4}. Naturally, its
algebraic counterpart corresponds to the intersection of the classes of deontic action algebras characterizing \NDAL{3} and \NDAL{4}.

%In summary, for each \NDAL{i} we have a corresponding subclass $\DALVariety(i)$ of deontic action algebras --determined by the particular conditions of the logic. 
We now present soundness and completeness results of each of these logics. 
To this end, we introduce the auxiliary definitions of \emph{deontic action subalgebra} and  \emph{deontic action generated algebra}.
Both are analogous to the standard case.

\medskip
\begin{definition}\label{def:deontic-subalgebra} Let $\DAlgebra = \langle \Algebra[A], \Algebra[F], \E, \P, \F \rangle$ and $\DAlgebra' = \langle \Algebra[A]', \Algebra[F]', \E', \P', \F' \rangle$ be two deontic action algebras, we say that 
$\DAlgebra'$ is a subalgebra of $\DAlgebra$ iff: 1.~$\Algebra[A]'$ is a Boolean subalgebra of $\Algebra[A]$; 2.~$\Algebra[F]'$ is a subalgebra of $\Algebra[F]$; and 3.~$\E'$, $\F'$, and $\P'$ are restrictions of $\E$, $\F$, and $\P$ to $\Algebra[A]'$ and $\Algebra[F]'$, respectively.
\end{definition}
\medskip

% The notion of generated algebra is also an extension of the standard definition.  As formally introduced by the next definition.

% \medskip

\begin{definition} Let $\DAlgebra = \langle \Algebra[A], \Algebra[F], \E, \P, \F \rangle$ be a deontic action algebra.
In addition, let $A' \subseteq |\Algebra[A]|$ and $F' \subseteq |\Algebra[F]|$.
The sets $A'$ and $F'$ are called generators.
The deontic action algebra generated by $A'$ and $F'$ is the subalgebra $\DAlgebra = \langle \Algebra[A]', \Algebra[F]', \E', \P', \F' \rangle$ of $\DAlgebra$ where: 1.~$\Algebra[A]'$ is the intersection of all the subalgebras of $\Algebra[A]$ whose carrier set contains $A'$; and 2.~$\Algebra[F]'$ is the intersection of all the subalgebras of $\Algebra[F]$ whose carrier set contains $F'$.
\end{definition}
\medskip

The following theorem extends \Cref{theorem:completeness} for \DAL to its variants \NDAL{i}.

\medskip

\begin{theorem}\label{theorem:completeness:dal:i}
	If follows that $\varphi$ is a theorem of \NDAL{i} iff $\DALVariety(i) \vdash \varphi \doteq \top$.
\end{theorem}
\begin{proof} The proof is direct extension of that in \Cref{theorem:completeness}. We only sketch relevant steps.

\medskip 

\begin{description}
	\setlength{\itemsep}{5pt}
	\item[Soundness.]
	For \NDAL{1} we need to show for all $\DAlgebra \in \DALVariety(1)$ and all interpretations $h: \TAlgebra \to \DAlgebra$, it follows that $h({\forb(\mathsf{a}_i) \lor \perm(\mathsf{a}_i)}) =_{\Algebra[F]} \top$.
	Then:
		\begin{align*}
			h({\forb(\mathsf{a}_i) \lor \perm(\mathsf{a}_i)}) =_{\Algebra[F]}
			h(\forb(\mathsf{a}_i)) \lor h(\perm(\mathsf{a}_i)) =_{\Algebra[F]}
			\F(h(\mathsf{a}_i)) \lor \P(h(\mathsf{a}_i)) =_{\Algebra[F]}
			\top.
		\end{align*}%
	Note that for every $a_i \in \bact$, $h(a_i)$ is a generator, and that homomorphisms between generated Boolean algebras are determined by the mapping between their generators. 
	For the other variants the proofs are similar using the properties of generators, homomorphisms, and the new equations for each case.

	\item[Completeness.]
	Similarly to our result in \Cref{theorem:completeness}, for each \NDAL{i}, we need to define an equivalent to the Lindenbaum-Tarski Algebra of \DAL.  We describe the procedure for \NDAL{1}.
	The other cases use the same argument. 
	First, consider the Lindenbaum-Tarski $\LTAlgebra$ in \Cref{prop:lindenbaum}, and  consider the subalgebra $\LTAlgebra(1)$ generated by the generators 
	$A' =\set{[\mathsf{a}_i]_{\cong_{\sorta}}}{\mathsf{a}_i \in \bact}$, and $F' = \form/_{\cong_{\sortf}}$.
	Furthermore, consider the congruence $\cong_{(1)}$ over $\LTAlgebra(1)$ induced by theoremhood in \NDAL{1}, i.e., the axioms of \DAL plus the new axiom set $\set{\forb(\mathsf{a}_i) \lor \perm(\mathsf{a}_i)}{\mathsf{a}_i \in \bact}$.
	% ,  and also the corresponding algebra $\LTAlgebra(1)/\cong_{\NDAL{1}}$.  As proven in \Cref{theorem:completeness}, this is a deontic action algebra.  Furthermore,
	From its construction, $\LTAlgebra(1)/\cong_{(1)}$ is such that	$\F([\mathsf{a}_i]_{\cong_{\sorta}}) \lor \P([\mathsf{a}_i]_{\cong_{\sorta}}) = \top$.
	This algebra provides the canonical deontic action algebra for \NDAL{1}.
	The proof for \NDAL{i}, for $i \in \{2,3,4,5\}$, can be obtained by a similar procedure: a subalgebra of the original Lindenbaum Algebra is considered,  this subalgebra is quotiented by the corresponding axioms, obtaining an algebra that allows us to prove the completenes for the corresponding version of the logic.
	\qedhere
\end{description}
\end{proof}

% \begin{proof}[Sketch] For each logic presented above the techniques presented in \ref{} can be extended in a routinary way. For instance, consider
% the case of \NDAL{1},  since the algebras in $\mathcal{D}_1$ are freely generated by a set $G$ the notion of algebraic validity is defined as follows.
% We say that $\vDashcurly_{\mathcal{D}_1} \varphi \doteq \True$ iff for every $\DAlgebra \in \mathcal{D}_1$ and  onto function  $f:\bact \rightarrow G$ (being $g$ the generators of $\DAlgebra$)
% we have that $\DAlgebra, f^* \vDashcurly \varphi \doteq \True$. First, note that $f^*(\perm[a_i] \vee \forb[a_i]) = \P{x_i} +_F \F{x_i} = \True$, because $f$ maps
% $\bact$ to generators and \ref{eq:dal1}. The rest of the axioms can be proven to evaluate to $\True$ using the same reasoning as \ref{theorem:soundness}. For completeness the Lindenbaum algebra can be defined in the same way as \ref{}, it is direct to check that this algebra satisfies equation \ref{eq:dal1} and its action algebra is generated by the $[a_i]'s$, and therefore
% \ref{} apply to \NDAL{1}. These ideas can easily be adapted to prove the algebraic soundness and completeness of $\NDAL{2}-\NDAL{5}$.
% \end{proof}

%\subsection{Further Extensions}
%	One interesting aspect of the formalism presented in this paper is that it can easily be extended to cope with other action operators and normative formulas.

%Notice that \DAL accepts consistent sets of formulas in which all actions are permitted, alt., forbidden (but not both simultaneously).
%If all actions are permitted, nothing an agent does leads to trouble; on the other hand, if all actions are forbidden, an agent is trapped, i.e., everything it does leads to a violation.
%These cases are known as \emph{deontic heaven}, alt., \emph{deontic hell}.
%Deontic heaven and deontic hell are captured in the language of \DAL by the formulas $\perm[1]$ and $\forb[1]$ (resp., see~\cite{Trypuz15}).
%In some cases, it is interesting to exclude these extreme cases.
%To do this, it suffices to add $\Not \perm[1]$, and/or $\Not \forb[1]$, to the set of axioms of the language of \DAL.
%We will return to this point later on.
%
%We bring attention to some important results regarding permission, prohibition, and consistency in relation to the universal action in \DAL.
%First, $\not \vdash \perm[1]$ and $\not \vdash \Not \perm[1]$, and $\not \vdash \forb[1]$ and $\not \vdash \Not \forb[1]$.
%Second, if $\Phi \vdash \perm[1]$, then $\Phi \vdash \perm[\alpha]$ for any action $\alpha$.
%Third, let $\alpha \neq 0$, if $\Phi \vdash \perm[\alpha]$ and $\Phi \vdash \forb[\bar{\alpha}]$, or alt., $\Phi \vdash \forb[\alpha]$ and $\Phi \vdash \perm[\bar{\alpha}]$, then, $\Phi$ is inconsistent.

%%% Local Variables:
%%% mode: latex
%%% TeX-master: "article"
%%% End:
