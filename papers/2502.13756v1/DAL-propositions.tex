\subsection{Introducing Propositions}\label{sec:dal:propositions}

Our algebraization of \DAL features an unusual characteristic: the use of an empty set of variables of sort $\sortf$ in the definition of the term algebra $\TAlgebra$ in~\Cref{dal:talg}.
% This is an unusual characteristic.
A more natural approach would be to consider a countable set $\prop$ of proposition symbols as variables of sort $\sortf$, analogous to the set $\bact$ of basic action symbols.
Incorporating the set $\prop$ into \DAL results in a new deontic action logic, which we denote as $\DAL(\prop)$.
The construction of this new logic is relatively straightforward, as are its soundness and completeness results.
Moreover, we demonstrate that this new logic has certain advantages over \DAL for modeling scenarios that require explicit propositional reasoning.

\paragraph{Deontic Action Algebras and Propositions.}

The logic $\DAL(\prop)$ extends the logical language of \DAL incorporating symbols in $\prop = \set{p_i}{i \in \Nat_0}$ as base cases in the recursive definition of formulas.
In addition, it adapts the axiom system for \DAL to accommodate for this new definition of a formula.
The algebraization of $\DAL(\prop)$ uses the same signature as \DAL.
Its algebraic language is the term algebra $\TAlgebra_1$ built sets:
	$\bact$ of variables of sort $\sorta$, and
	$\prop$ of variables of sort $\sortf$.
The term algebra $\TAlgebra_1$ is interpreted into deontic action algebras as explained in \Cref{section:basics}. That is, an interpretation of $\TAlgebra_1$ in a deontic action algebra 
$\DAlgebra = \tup{\Algebra[A], \Algebra[F], \E, \P, \F}$ is a homomorphism $h:\TAlgebra_1 \to \DAlgebra$. Note that, by definition, $h$ maps symbols in $\bact$ into elements in $|\Algebra[A]|$ and symbols in $\prop$ into elements in $|\Algebra[F]|$ (in fact, interpretations are completely determined by mappings $\bact \to |\Algebra[A]|$ and $\prop \to |\Algebra[F]|$).

We derive the soundness and completeness of $\DAL(\prop)$ by adapting the corresponding results for \DAL in \Cref{sec:algebraic-char}.

\medskip
\begin{theorem}\label{prop:completeness:dal:prop} $\varphi$ is a theorem of $\DAL(\prop)$ iff $\DALVariety \vDash {\varphi \doteq \top}$.
\end{theorem}
\begin{proof} The proof follows the steps of that of  \Cref{theorem:soundness},  we highlight some subtle details.  

\medskip 
	\begin{description}
		\item[Soundness.]
		The proof of soundness is, as in \Cref{theorem:soundness}, by induction on the length of proofs.
		Note that the set of proofs is defined as in \Cref{theorem:soundness}, but instantiations of axiom schemas may now contain proposition symbols. E.g.,  $(p \land \lnot p) \liff \bot$ is an instance of an axiom schema, and a theorem. This does not affect the proof given in \Cref{theorem:soundness}, which is immediately lifted to a proof of soundness for ${\DAL(\prop)}$.

		\item[Completeness.]
		For completeness, we need to redefine the Lindenbaum algebra.
		First, in this case, we consider the term algebra $\TAlgebra_1$ that contains also formulas with propositions. The congruence $\cong$ in \Cref{prop:lindenbaum} is used to construct the quotient algebra. This quotient algebra is a deontic action algebra. Note that this algebra also contains formula terms with propositions by definition. Finally, adapting the proof of \Cref{theorem:completeness} we obtain the algebraic completeness result.
		\qedhere
	\end{description}
\end{proof}
\medskip

In summary, $\DAL(\prop)$ differs only from $\DAL$ in their respective term algebras.
That is, while \DAL is associated with a term algebra $\TAlgebra$ built over an empty set of variables of sort $\sortf$, the term algebra $\TAlgebra_1$ associated to $\DAL(\prop)$ uses the set $\prop$ as the set of variables of sort $\sortf$.
By adding proposition symbols, the term algebra $\TAlgebra_1$ brings about a sense of correspondence between the basic symbols used for building the set of actions and those used for building the set of formulas.

\paragraph{Propositions Matter}

\DAL, as well as its variants discussed in \Cref{section:dals}, place the focus on formalizing notions of permission and prohibition pertaining to actions.
Nonetheless, they face challenges when presented with statements such as \emph{it is not the case you are permitted to drive without a license}.
This limitation stems from the inability to distinguish between \emph{pure propositions}, such as \emph{you have a driver's license}, and \emph{normative propositions}, such as \emph{you are permitted to drive}.
% In this section, we show how to extend our algebraization for \DAL to be able to cope with this challenge.
% Moreover, we show how to extend Segerberg's proposal for the notions of permission and prohibition to states of affairs.
This distinction is seamlessly addressed in $\DAL(\prop)$.
For example, we can use a proposition symbol $\mathsf{haslicense} \in \prop$ to indicate that a person has a driver's license, and the formula $\perm(\mathsf{driving})$ to indicate that (the action of) driving is permitted.
This allow us to formalize \emph{it is not the case you are permitted to drive without a license} as $\lnot(\lnot\mathsf{haslicense} \land \perm(\mathsf{driving}))$.

Including propositions in deontic action algebras leads to some interesting discussions.
Imagine the following scenario: \emph{there must be no fence;  if there is a fence, then, it must be a white fence; there is a fence}.
This typical case of contrary-to-duty reasoning is discussed in~\cite{Prakken:1996}, where it is noted that prescriptions are applied to propositions rather than actions.
%
The shift from ought-to-do to ought-to-be is central to most deontic logics developed in the late part of the 20th century; and to \SDL (Standard Deontic Logic) in particular~\cite{Aqvist:2002}.
This shift is not without difficulties. It comes at a cost of quickly leading to paradoxes, i.e., theorems in the logic that are intuitively invalid~\cite{Aqvist:2002,Meyer:1994}.
For instance, in \SDL, a natural formalization of the scenario above, together with the formalization of the global assumption that \emph{if there is a white fence, then, there is a fence} leads to a contradiction; while the scenario is intuitively plausible.

The position held in \cite{Prakken:1996} is that it is worthwhile exploring conditions under which contrary-to-duties can be given consistent readings.
In this respect, we raise the following point.
Up to now, we have assumed that in a deontic action algebra $\DAlgebra = \tup{\Algebra[A], \Algebra[F], \E, \P, \F}$ the algebra $\Algebra[A]$ is used to describe actions.
But a more abstract view of this algebra is also possible.
In particular, we can think of the elements of  $\Algebra[A]$ as entities of the world that can be prescribed, they might be actions, but also propositions such as \emph{there is a fence}, or \emph{the fence is white}.
Under this new reading of a deontic action algebra, in $\DAL(\prop)$, we can distinguish between propositions we can prescribe over, and those we cannot.
For instance, we may regard statements like \emph{it is permitted that is raining} as having little sense, and thus as being ill-formed.
The statement \emph{it is raining} is either true or false, but in no case seems to be amenable to be regulated by a normative system.
Summarizing, deontic action algebras can be used to model ought-to-be normative systems where there is a clear distinction between entities that can be prescribed (corresponding to the elements in $\Algebra[A]$), and those that cannot (corresponding to the elements in $\Algebra[F]$).

\begin{figure}
	\centering
	\includegraphics[width=0.5\textwidth]{deontic-algebra-fence.pdf}\\[1em]
	\caption{The Cottage Regulations Example}\label{ex:fence}
\end{figure}

\medskip
\begin{example}
	Returning to the example, let us use $\obl(\alpha)$, read as $\alpha$ is \emph{obligatory}, as an abbreviation of $\forb(\bar{\alpha})$.
	Then, we could use the formulas:
	$\obl(\overline{\mathsf{isfenced}})$ 
		to indicate that \emph{there must be no fence},
	$\mathsf{isfenced} = \uact \to \obl(\mathsf{ispaintedwhite})$
		to indicate that \emph{if there is a fence, then, it must be a white fence}, and
	$\mathsf{isfenced} = \uact$
		to indicate that \emph{there is a fence}.
	Finally, we could use the formula
		$\mathsf{ispaintedwhite} \sqcup \mathsf{isfenced} = \mathsf{isfenced}$ to indicate the global assumption that \emph{if the fence is painted white, then, the house is fenced}.
	The deontic action algebra $\DAlgebra$ in \Cref{ex:fence}, together with the interpretation $h: \TAlgebra_1 \to \DAlgebra$ defined as $h_{\sorta}(\mathsf{isfenced}) = \uact$, $h_{\sorta}(\mathsf{ispaintedwhite}) = a$, prove that these formulas are consistent.
	Precisely, we have:

	\medskip
	\centerline{
	\begin{minipage}{0.6\textwidth}
		\begin{enumerate}
			\setlength{\itemsep}{5pt}
			\item $h(\obl(\overline{\mathsf{isfenced}})) =_{\Algebra[F]} \top$.
			\item $h(\mathsf{isfenced} = \uact \to \obl(\mathsf{ispaintedwhite})) =_{\Algebra[F]} \top$.
			\item $h(\mathsf{isfenced} = \uact) =_{\Algebra[F]} \top$.
			\item $h(\mathsf{ispaintedwhite} \sqcup \mathsf{isfenced} = \mathsf{isfenced}) =_{\Algebra[F]} \top$.
		\end{enumerate}
	\end{minipage}}
\end{example}
\medskip

To sum up, we have explored some key features and applications of the incorporation of propositions into deontic action algebras. These insights, brought about by our discussion and examples, are particularly relevant to understand some broader implications of Segerberg's formalization of the notions of permission and prohibition.