\subsection{Intuitionistic Deontic Action Logic}

Clearly, we can also simultaneously replace the Boolean algebras for actions and formulas for Heyting algebras.
We call the resulting logic $\DAL(\INT)$.
Similarly to the case in \Cref{sec:action-int}, to the best of our knowledge, this logic is the first fully intuitionistic deontic action logic.


% We start by presenting $\DAL(\INT)$ from a syntactic perspective.
% Then, we move on to explaining how to capture this new deontic action logic from an algebraic perspective.

\paragraph{The Logic Itself}

The language and axiomatization of $\DAL(\INT)$ combines those of $\DAL(\IAL)$ and $\DAL(\IPL)$ in \Cref{sec:heyting:formulas,sec:action-int}.
Precisely, it builds actions like in $\DAL(\IAL)$\textemdash using $\hto$ as a primitive connective.
In turn, it builds formulas like in $\DAL(\IPL)$\textemdash using $\to$ as a primitive connective.
The axiomatization of this new logic takes the axioms for actions from $\DAL(\IAL)$ and the axioms for formulas from $\DAL(\IPL)$.
The logic retains the axiomatization of equality, permission, and prohibition of Segerberg's logic, i.e., axioms E1--E2 and D1--D3 in \Cref{dal:axioms}.
The notions of proof and theoremhood are reformulated in the obvious way.

The algebraization of $\DAL(\INT)$ replaces the Boolean algebras of actions and of formulas in the definition of a deontic action algebra for Heyting algebras. This is made precise in the definition of an \emph{intuitionistic deontic action algebra} below.

\medskip
\begin{definition}\label[definition]{def:intuitionistic:dalgebra}
	By an intuitionistic deontic action algebra, we mean an algebra
		$\DAlgebra =
			\langle
				\Algebra[A], \Algebra[H], \E, \P, \F
			\rangle$
		where:
			$\Algebra[A]$ and 
			$\Algebra[F]$ are Heyting algebras, and
			$\E : {|\Algebra[A]| \times |\Algebra[A]| \to |\Algebra[F]|}$,
			$\P : {|\Algebra[A]| \to |\Algebra[F]|}$,
			and
			$\F : {|\Algebra[A]| \to |\Algebra[F]|}$ satisfy the conditions 1--6 in \Cref{definition:deontic:algebra}.
\end{definition}
\medskip

As before, permission and prohibition behave as expected.

\medskip
\begin{proposition}\label[proposition]{prop:intuitionistic:ideal}
	Let $\DAlgebra = \tup{\Algebra[A], \Algebra[F], \E, \P, \F}$ be an intuitionistic deontic action algebra.
	The pre-images $P$ and $F$ of $\top$ under $\P$ and $\F$ are ideals in $\Algebra[A]$ s.t.\ ${{P \cap F} = \{\iact\}}$.
\end{proposition}
\medskip
\begin{proof}
	We can reuse the proof of \Cref{prop:ideals-int} as it only uses the properties of $\P$ and $\F$ plus absorption, and idempotence properties which also hold in Heyting algebras.
\end{proof}


%\paragraph{Soundness and completeness} I
For stating and proving the soundness and completeness of $\DAL(\INT)$, we define interpretations and algebraic validity as in previous sections.

\medskip

\begin{theorem}\label{prop:completeness:heyting}
	Let $\mathbb{ID}$ be the class of all intuitionistic deontic action algebras.
	Then, $\varphi$ is a theorem of $\DAL(\INT)$ iff $\mathbb{ID} \vDash {\varphi \doteq \top}$.
\end{theorem}
\begin{proof}
	We obtain this result by putting together the intuitionistic parts of the proofs of
	\Cref{prop:completeness:heyting:formulas,prop:completeness:heyting:actions}.
%	
%	\begin{description}
%		\item[Soundness.]  The various cases correspond to those in the soundness proofs of \Cref{prop:completeness:heyting:formulas,prop:completeness:heyting:actions}.
%		
%		\item[Completeness.] We need to define a Lindenbaum algebra via suitable congruences $\cong_{\sorta}$ and $\cong_{\sortf}$.
%		The first congruence correspond to that in the completeness part of the proof of \Cref{prop:completeness:heyting:actions}.
%		The second congruence correspond to that in the completeness part of the proof of \Cref{prop:completeness:heyting:formulas}.
%		It is clear that these congruences yield Heyting algebras.
%		The result is obtained by proving the induced operations of $\E$, $\P$, and $\F$ satisfy the conditions 1--6 in \Cref{definition:deontic:algebra}.
%		\qedhere
%	\end{description}
\end{proof}

\paragraph{Intuitionistic Deontic Action Logic in Practice}

The logic $\DAL(\INT)$ may be useful for reasoning in scenarios where there is partial observability about the state of affairs, typical in reinforcement learning and planning.
Consider the following example adapted from \cite{DBLP:conf/iros/CassandraKK96}: a robot is tasked with cleaning an office and needs to reach certain spots.
The robot has sensors to detect doorways, walls, or open spaces, but the sensor information may sometimes be unclear.
Proposition symbols like $\mathsf{north}$, $\mathsf{south}$, $\mathsf{east}$, and $\mathsf{west}$ could represent the robot's orientation, while $\mathsf{doorway}$, $\mathsf{wall}$, and $\mathsf{clear}$ capture the information provided by the sensors.
Uncertainty entails the robot might fail to determine whether there is a doorway ahead or not, i.e., $\mathsf{doorway} \lor \lnot\mathsf{doorway}$ may fail to hold, violating the law of the excluded middle at the level of formulas.
The example can be extended with an intuitionistic model of actions.
For instance, we can consider actions $\mathsf{advance}$ and $\mathsf{rotate}$ for the robot moving forward and rotating, respectively.
As in the example in \Cref{sec:action-int}, the interpretation of these actions is tied to a plan allowing the robot to realize the action.
Once again, a formula $\mathsf{advance} \sqcup \overline{\mathsf{advance}} = \uact$ may fail to hold (violating the law of excluded middle at the level of actions) because the robot lacks a plan due to incomplete sensor information.

Finally, we can spice up this scenario with prescriptions.
For instance, there might be signals in the corridors indicating the robot is not to cross certain doorway, prohibiting such an action.
Again, the robot may lack sufficient information to determine whether $\forb(\mathsf{advance}) \lor \lnot\forb(\mathsf{advance})$ holds or not.

In all the above scenarios, having an intuitionistic deontic action logic like $\DAL(\INT)$ provides a formal framework for analysis in which we can address issues arising from partial observability and prescriptive constraints.
