\subsection{Accounting for Propositions}\label{sec:dal:propositions}

The main focus in the seminal work of Segerberg \cite{Segerberg1982} was on capturing notions of permission and prohibition pertaining to actions.
This leaves out the case of \emph{pure propositions}: true/false statements that are not related to actions being equal, permitted or prohibited.  Even so, the application of prescriptions to 
states of affairs, or situations, was the most common approach in deontic logic during the last part of the 20th century.  Among all the formal systems in this area, SDL (Standard Deontic Logic) \cite{Aqvist:2002} was clearly the most investigated ought-to-be deontic logic.  Let us consider the following example: \emph{the house ought to have no fence,  but if the house has a fence,  it must be white. If the house is on the beach it must have a fence.} \cite{Prakken:1996} This is a typical case of contrary-to duty-reasoning, in which one contradictory obligation appears after the violation of a norm. Note that in this example prescriptions are applied to state of affairs.  

Taking into account the discussion above,  it seems natural to extend deontic action logics with prescriptions over statements.  It is worth noting that we have assumed that one of the algebras constituting a deontic action algebra is used to describe actions (the one denoted by $\Algebra[A]$ in \Cref{definition:deontic:algebra}).  A more abstract view over this algebra is also possible and think of the elements of  $\Algebra[A]$ as abstract events or situations of the world that can be prescribed,  they might be actions,  but also states of affairs like \emph{the fence is white}.  Furthermore,  note that we can distinguish between those state of affairs that can be prescribed and those that can be not; for instance,  stating that \emph{it is permitted that is raining now} has little sense as the proposition \emph{it is raining now} could be true or false but it is something that is not amenable to be regulated by a normative system.  Summarizing,  deontic action algebras can be used to model ought-to-be normative systems where there is a clear distinction between statements that can be prescribed (algebra $\Algebra[A]$) and which cannot be (algebra $\Algebra[F]$).
 Motivated by the previous observation, we introduce a new deontic action logic called $\DAL(\prop)$.

%Pure propositions appear naturally in normative systems. Consider, e.g., the statement ``John possesses a valid driver's license''. Having a valid driver license holds true or false for a person and it is not an action. We might, however, want to stipulate that this proposition forbids a certain action, e.g., ``if John does not posses a valid driver's license, then, him driving is forbidden''. Motivated by the previous observation, we introduce a new deontic action logic called $\DAL(\prop)$.

\paragraph{Language.}
	The language of $\DAL(\prop)$ consists of a set $\act$ of \emph{prescribable  entities} and a set $\form$ of \emph{formulas}.
	$\act$ is built on a countable set $\bact = \set{\mathsf{a}_i}{i\in \Nat_0}$ of basic prescribable symbols, and is defined by the grammar in \Cref{definition:actions}.
	$\form$ is built on a countable set $\prop = \set{p_i}{i \in \Nat_0}$ of proposition symbols, and is defined by the grammar in \Cref{definition:props}.
	%
	% \setcounter{equation}{7}
	% \begin{align}
	% 		\varphi & ::=
	% 				p_i
	% 			\mid
	% 				\perm(\alpha)
	% 			\mid
	% 				\forb(\alpha)
	% 			\mid
	% 				{\alpha = \beta}
	% 			\mid
	% 				{\varphi \to \psi}
	% 			\mid
	% 				{\varphi \lor \psi}
	% 			\mid
	% 				{\varphi \land \psi}
	% 			% \mid
	% 			% 	{\lnot \varphi}.
	% 			\mid
	% 				\bot.
	% 			\label{definition:props}
	% \end{align}%
The proposition symbols in $\prop$ enable us to formalize basic cases of pure propositions. E.g., we could use $\mathit{hf} \in \prop$ to formalize ``The house has a  fence'',  and
use $\mathsf{hwf}$ in $\act$ to capture prescribable state of affairs,  e.g.,  ``the house has a white fence''.  Hence,  the formula $\mathit{hf} \to \forb(\overline{\mathsf{hwf}})$ formalizes the proposition ``if the house has a fence, it is forbidden that the fence is not white".

%The proposition symbols in $\prop$ enable us to formalize basic cases of pure propositions. E.g., we could use $hf \in \prop$ to formalize ``The house has a fence '', and use $\lnot d_j \to \forb(\mathsf{drive}_j)$ to formalize the proposition ``if John does not posses a valid driver's license, then, him driving is forbidden''.

%A comparison is unavoidable, the language of $\DAL(\prop)$ extends the language of $\DAL$ by adding a new rule for formulas.
%More precisely, the languages of $\DAL$ and $\DAL(\prop)$ have the same set of actions.
%They differ in their corresponding sets of formulas.
%The formulas of the language of $\DAL$ are built from basic formulas that assert facts about actions being equal, permitted, or forbidden.
%In addition to these rules, the language of $\DAL(\prop)$ allows for formulas to be built using proposition symbols.
%It is easy to see that the language of $\DAL(\prop)$ strictly contains both: the language $\prop$ of Classical Propositional Logic $(\CPL)$, and the language of $\DAL$, as some of its fragments.

% \paragraph{Axiomatization.}
% Interestingly, the axiom system introduced in \Cref{section:dal} is sound and complete for $\DAL(\prop)$.


\paragraph{Algebraic semantics.} Let us  extend our algebraic treatment of $\DAL$ to cope with the new logic introduce in this section.  This mainly can be achieved by extending  the algebraic language in use.

\medskip

\medskip

\begin{definition}\label[definition]{def:term:algebra}
	% The set $\var = \{\var_a, \var_f\}$, where ${\var_a = \bact}$ and ${\var_f = \bprop}$, is an $S$-sorted set of variables for the signature $\Sigma = \tup{S, \Omega}$ in \Cref{def:signature}.
	% $\TAlgebra(\var)$ is the $\Sigma$-term algebra with variables in $\var$.
	Let $\Sigma = \tup{S, \Omega}$ be the signature in \Cref{def:signature}; define $\TAlgebra(\var)$ as the $\Sigma$-term algebra with variables in a set $\var = \{{\bact}_a, {\prop}_f\}$.% s.t.\ ${\var_a = \bact}$ and ${\var_f = \prop}$.
\end{definition}

\medskip


 We briefly compare the term algebra $\TAlgebra(\var)$ in \Cref{def:term:algebra} with the term algebra $\TAlgebra$ used for the algebraization of $\DAL$ in \Cref{sec:algebraic-char}.
 Notice that the signature $\Sigma = \tup{S, \Omega}$ in \Cref{def:signature} contains two sort symbols, $f$  and  $a$.
 However, $\TAlgebra$ is built using a set $V$ of variables where $V_a = \bact$ and $V_f = \emptyset$.
 This means that elements of sort $f$ in $\TAlgebra$ are constructed using operations on elements of sort $a$.
 By adding proposition symbols, the term algebra $\TAlgebra(\var)$ brings about a sense of correspondence between the basic symbols used for building the set of actions and those used for building the set of formulas.

The term algebra $\TAlgebra(\var)$ is interpreted into deontic action algebras as explained in \Cref{section:basics}. That is, an interpretation of $\TAlgebra(\var)$ in a deontic action algebra 
$\Algebra[D]$ is a homomorphism $h:\TAlgebra(\var) \rightarrow \Algebra[D]$. Thus, by definition,  $h$ maps propositional variables into elements of $\Algebra[F]$.  Having these definition at hand we can prove the soundness and completeness of this extension.

%\paragraph{Soundness and completeness.}
%Interpretations of $\TAlgebra(\var)$ onto deontic action algebras enable us to formulate the following result.

% This is interpretation is made precise below.
%
% \begin{definition}
% 	Assignments of the set $\var$ of variables onto deontic action algebras $\DAlgebra = \tup{\Algebra[A], \Algebra[F], \E, \P, \F}$ are $S$-sorted function $g: \var \to \DAlgebra$ s.t.: $g_a: {\bact \to |\Algebra[A]|}$, and $g_f: {\bprop \to |\Algebra[F]|}$.
% 	Interpretations are homorphisms $h: \TAlgebra(\var) \to \DAlgebra$ respecting sorts and operations in the obvious way.
% \end{definition}
%
% \begin{proposition}
% 	Assignments and interpretations are in a one to one correspondence.
% \end{proposition}
% Interpretations of $\TAlgebra(\var)$ onto deontic action algebras enable us to formulate the following result.


\begin{theorem}\label{prop:completeness:dal:prop} Let  $\vdash_{\DAL(\prop)}$  denote theoremhood for  $\DAL(\prop)$, then for all formulas $\varphi$  of $\DAL(\prop)$,  we have that $\vdash_{\DAL(\prop)} \varphi$ iff $ \vDashcurly {\varphi \approx \top}$.
\end{theorem}
\begin{proof} The proof follows the steps of that of  \Cref{theorem:soundness},  we highlight some subtle details.  

\medskip 

\noindent\textbf{Soundness.} The proof of soundness is on induction on the length of proofs, as in \Cref{theorem:soundness}.  Note that the set of proofs are the same as in \Cref{theorem:soundness}, the only difference is that in this case we have instances of axiom schemas that  contain propositions. For instance,  $p \wedge q \rightarrow p$ is an instance of an axiom schema, and therefore a theorem. But the proof given \Cref{theorem:soundness} can be applied to these cases too,  providing a proof for the soundness for the case of $\vdash_{\DAL(\prop)}$.

\medskip

\noindent\textbf{Completeness.} For completeness we define the Lindenbaum algebra, First, in this case we consider the term algebra $\TAlgebra(\var)$ introduced above that contains also formulas with propositions.  Then, the congruence $\cong$, as defined in \Cref{def:lindenbaum-algebra} is used to construct the quotient algebra. As proven in \Cref{prop:lindenbaum} this is a deontic action algebra. Note that this algebra also contains formula terms with propositions by definition. Finally, applying the proof of \Cref{theorem:completeness} we obtain the algebraic completeness.
\end{proof}