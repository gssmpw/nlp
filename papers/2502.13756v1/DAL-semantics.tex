\paragraph{Semantics.}\label{section:dal:semantics}

The semantics for \DAL is given over deontic action models.
A deontic action model is a tuple $\DeonticModel = \langle E, P, F \rangle$ where: $E$ (the domain) is a (possibly empty) set of elements; and $P$ and $F$ are disjoint subsets of $E$ (i.e., ${P \cup F} \subseteq E$ and ${P \cap F} = \emptyset$).
Intuitively, $E$ indicates realizations of actions, $P$ and $F$ are sets of permitted and forbidden realizations of actions.
The disjointness condition on $P$ and $F$ indicates that permitted realizations of actions are not forbidden, and vice versa, that forbidden realizations are not permitted.
Given a model $\DeonticModel = \tup{E,P,F}$, a \emph{valuation} on $\DeonticModel$ is a function $v: \bact \rightarrow 2^E$.
Intuitively, a valuation indicates a particular way of realizing actions.

\medskip
\begin{proposition}
   For every deontic action model $\DeonticModel = \tup{E,P,F}$, and any valuation $v: {\bact \to 2^E}$ on $\DeonticModel$, there is a unique $v^{*} : {\act \rightarrow 2^E}$ s.t.:
   \begin{align*}
            v^{*}(\alpha \sqcup \beta)
               &=
               {v^{*}(\alpha)
               \cup
               v^{*}(\beta)}
         &
         v^{*}(\bar{\alpha})
               &=
               {E \setminus v^{*}(\alpha)}
         &
         v^{*}(0)
               &=  \emptyset
         \\
            v^{*}(\alpha \sqcap \beta)
               &=
               {v^{*}(\alpha)
               \cap
               v^{*}(\beta)}
         &&&
            v^{*}(1)
               &=  E.
   \end{align*}%
\end{proposition}
\medskip




The \emph{satisfiability} of a formula $\varphi$ on a deontic action model $\DeonticModel = \tup{E, P, F}$ under a
valuation $v$, written ${\DeonticModel, v} \Vdash \varphi$, is
defined inductively as:
\[
\begin{array}{rlcl}
   \DeonticModel, v & \Vdash \alpha=\beta
       & \mathrel{\mbox{ iff }} & 	v^{*}(\alpha) = v^{*}(\beta) \\
   \DeonticModel, v & \Vdash \perm(\alpha)
       & \mathrel{\mbox{ iff }} & v^{*}(\alpha) \subseteq P\\
   \DeonticModel, v & \Vdash \forb(\alpha)
       & \mathrel{\mbox{ iff }} & v^{*}(\alpha) \subseteq F\\
   %  \DeonticModel, v & \Vdash \varphi \to \psi
	%    & \mathrel{\mbox{ iff }} & \DeonticModel, v \nVdash \varphi
   %       \mbox{ or }      \DeonticModel, v  \Vdash \psi\\
   \DeonticModel, v & \Vdash \varphi \lor \psi
       & \mathrel{\mbox{ iff }} & \DeonticModel, v \Vdash \varphi
         \mbox{ or }      \DeonticModel, v \Vdash \psi\\
   \DeonticModel, v & \Vdash \varphi \land \psi
       & \mathrel{\mbox{ iff }} & \DeonticModel, v \Vdash \varphi
       \mbox{ and } \DeonticModel, v \Vdash \psi\\
   \DeonticModel, v & \Vdash \lnot \varphi
       & \mathrel{\mbox{ iff }} & \DeonticModel, v \nVdash \varphi\\
   \DeonticModel, v & \Vdash \bot
      &  & \mbox{never}\\
   \DeonticModel, v & \Vdash \top
      &  & \mbox{always.}\\
\end{array}
\]
We say that a formula $\varphi$ is a \emph{tautology} iff for any deontic action model $\DeonticModel$ and for any valuation $v$ on $\DeonticModel$, it follows that ${\DeonticModel, v} \Vdash \varphi$.

In \Cref{ex:semantics}, we present examples of deontic action models for the formulas in \Cref{ex:syntax}.

\medskip 


\begin{example}\label{ex:semantics}
   \begin{figure}
      \centering
      \begin{minipage}{0.5\textwidth}
            \centering
            %\hspace*{-.5cm}
            \includegraphics[trim=50pt 0pt 50pt 0pt, clip, width=1\textwidth]{deontic-model-a.pdf}\\[1em] % first figure itself
            \caption{A Deontic Action Model.}\label{ex:deontic:model:a}
      \end{minipage}\hfill
      \begin{minipage}{0.5\textwidth}
            \centering
            %\hspace*{-.5cm}
            \includegraphics[trim=50pt 0pt 50pt 0pt, clip, width=1\textwidth]{deontic-model-b.pdf}\\[1em] % second figure itself
            \caption{Anoter Deontic Action Model.}\label{ex:deontic:model:b}
      \end{minipage}
   \end{figure}
   Let $\DeonticModel = \tup{E,P,F}$ be the deontic action model in which $E$, $P$, and $F$ are as in \Cref{ex:deontic:model:a}.
   In addition, $\DeonticModel' = \tup{E,P',F}$ be the deontic action model in which $E$, $P'$, and $F$ are as in \Cref{ex:deontic:model:b}.
   Lastly, let $\{\mathsf{drinking}, \mathsf{driving}, \mathsf{parking}\} \subset \bact$, and $v: \bact \to 2^E$ be a valuation where $v(\mathsf{drinking})$, $v(\mathsf{driving})$, and $v(\mathsf{parking})$ are as in \Cref{ex:deontic:model:a}.
   Then:


   \begin{multicols}{2}
   \begin{enumerate}
      \item $\DeonticModel, v \Vdash \overline{\mathsf{parking}} = \mathsf{driving}$
      \item $\DeonticModel, v \Vdash \forb(\mathsf{drinking} \sqcap \mathsf{driving})$
      \item $\DeonticModel, v \Vdash \perm(\mathsf{drinking} \sqcap \mathsf{parking})$
      \item $\DeonticModel, v \nVdash \forb(\mathsf{drinking} \sqcap \mathsf{driving}) \land$
      \item[] \hspace{1.9cm}$\lnot\perm(\mathsf{drinking} \sqcap \mathsf{parking})$
      \item $\DeonticModel', v \Vdash \overline{\mathsf{parking}} = \mathsf{driving}$
      \item $\DeonticModel', v \Vdash \forb(\mathsf{drinking} \sqcap \mathsf{driving})$
      \item $\DeonticModel', v \nVdash \perm(\mathsf{drinking} \sqcap \mathsf{parking})$
      \item $\DeonticModel', v \Vdash \forb(\mathsf{drinking} \sqcap \mathsf{driving}) \land$
      \item[] \hspace{1.9cm}$\lnot\perm(\mathsf{drinking} \sqcap \mathsf{parking})$.
   \end{enumerate}
   \end{multicols}

   \noindent The models $\DeonticModel$ and $\DeonticModel'$ make clear, for example, that in \DAL, $\forb(\alpha)$ and $\lnot\perm(\alpha)$ are not equivalent.
   % Recall the formulas from~\Cref{ex:syntax}. Here we present a deontic action model $\DeonticModel$, where $F=\emptyset$ and $P=\{e_1,e_2\}$. Notice that $v(\mathsf{drink})=e_1$ and $v(\mathsf{teach}) = v(\mathsf{educate}) = e_2$. 

   % There, we can check that for instance, $\DeonticModel\Vdash\mathsf{teach} = \mathsf{educate}$ and that $\DeonticModel\Vdash \neg\perm(\mathsf{educate \sqcap drink})$. However, $\DeonticModel\nVdash\forb(\mathsf{teach \sqcap drink})$ (since $F=\emptyset$).
   % \medskip 

   % \begin{center}
   % \begin{tikzpicture}
   %    \draw[thick, rounded corners=8pt] (-1,0) rectangle (4,2);

   % % Points inside the square
   % \node at (0,1.8) {$P$}; 
   % \filldraw[black] (0.6,1) circle (2pt) node[below] {$\mathsf{drink}$} node[above] {$e_1$};
   % \filldraw[black] (2.4,1) circle (2pt) node[below] {$\begin{array}{l}\mathsf{teach} \\ \mathsf{educate}\end{array}$} node[above] {$e_2$};

   % % Add optional labels or grid (if needed for clarity)
   % % \draw[help lines] (0,0) grid (4,4); % Uncomment this for a grid

   % \end{tikzpicture}
   % \end{center}
\end{example}

%%% Local Variables:
%%% mode: latex
%%% TeX-master: "article"
%%% End:
