\section{Deontic Action Logic via Algebra}\label{sec:algebraic-char}

% The Deontic Action Logic \DAL enjoys some interesting characteristics.
% In particular, it is a simple system that provides a well-executed characterization of deontic operators.
% It enjoys an elegant semantics via sets and collections of sets (or dually via ideals and Boolean algebras).
% Interestingly also, \DAL further allows for additional deontic operators to be added systematically.
%More importantly, the formalism is sound and complete (Theorem 3.1 in \cite{Segerberg1982}).

We now turn our attention to revisiting and expanding the algebraic characterization of \DAL we presented in \cite{CCFA:2021}. To be noted, this algebraic framework is mathematically more abstract compared to the one in \cite{Segerberg1982}. This level of abstraction is a characteristic of algebraic logics, which can be leveraged to address broader issues in deontic logic. Furthermore, a distinguishing feature of our approach is its modularity. The class of algebras described below can be easily extended to support additional deontic operators, and in all cases, standard algebraic tools can be employed to prove soundness and completeness results.
We take advantage of this feature to build new deontic actions logics in the spirit of \DAL in \Cref{section:new:dals}.



\section{Preliminaries}\label{sec:preliminaries}



%We denote by $(\Ac(x_\Ac),\Bc(x_\Bc))(z)$ a random execution of $\pi$ with private inputs $(x_\Ac,y_\Ac)$, and common input $z$.

%\Jnote{Move to DP}
% At the end of such an execution, the protocol outputs a public transcript denoted by the random variable $\trans_\pi(x_\Ac,x_\Ac,z)$ we denotes the common as $\out(\trans_\pi(x_\Ac,x_\Ac,z)$, and each party $\Pc \in \set{\Ac,\Bc}$ obtains his view denoted $\view^\Pc_\pi(x_\Ac,x_\Bc,z)$, which may also contain a ``local output'' \Jnote{Local} $\out^\Pc(x_\Ac,x_\Bc,z)$ (if the protocol specifies such an output). \Jnote{Common output, and parties output}


\subsection{Distributions and Random Variables}\label{sec:prelim:dist}
The support of a distribution $P$ over a finite set $\cS$ is defined by $\Supp(P) \eqdef \set{x\in \cS: P(x)>0}$. For a distribution or a random variable $D$, let $d\from D$ denote that $d$ was sampled according to $D$. Similarly,  for a set $\cS$, let $x \from \cS$ denote that $x$ is drawn uniformly from $\cS$, and denote by $\cU_{\cS}$ the uniform distribution over $\cS$. For a finite set $\cX$ and a distribution $C_X$ over $\cX$, we use the capital letter $X$ to denote the random variable that takes values in $\cX$ and is sampled according to $C_X$. The {\sf statistical distance} (\aka {\sf~variation distance}) of two distributions $P$ and $Q$ over a discrete domain $\cX$ is defined by $\sdist{P}{Q} \eqdef \max_{\cS\subseteq \cX} \size{P(\cS)-Q(\cS)} = \frac{1}{2} \sum_{x \in \cS}\size{P(x)-Q(x)}$. 
For a vector $x = (x_1,\ldots,x_n)$ and index $i\in [n]$, we let $x_{-i} = (x_1,\ldots,x_{i-1},x_{i+1},\ldots,x_n)$ and $x^{(i)} = (x_1,\ldots,x_{i-1}, -x_i, x_{i+1},\ldots,x_n)$, for a set $\cS \subseteq [n]$ we let $x_{\cS} = (x_i)_{i \in \cS}$ and $x_{-\cS} = (x_i)_{i \in [n]\setminus \cS}$, and for a vector $r \in \zo^n$ we let $x_r = (x_i)_{\set{i \colon r_i = 1}}$ and $x_{-r} = (x_i)_{\set{i \colon r_i = 0}}$.

%For $n \in \N$ we let $U_n$ be the uniform distribution over $\oo^n$, and let $S_n$ be the distribution induces by the sum of $n$ i.i.d.\ random variables, each is distributed according to $U_1$. Let $\cN(0,1)$ be the standard normal distribution.
%For a distribution $\cD$ and a function $f$, we define by $f(\cD)$ the distribution that is induced by the output of $f(x)$ for $x \from \cD$. 





% \begin{theorem}[\cite{McGregorMPRTV10}]\label{thm:sv-extracotr}
% 	\Enote{Remove if not needed}
% 	There is a constant $c$ to make the following holds. Let $X$ be an $\alpha$-SV source on $\{0,1\}^n$, let $Y$ be a source on $\{0,1\}^n$ with min-entropy at least $\beta n$ (independent from $X$), and let $Z=\ip{X,Y}\mbox{mod m}$ for some $m\in\mathbb{N}$. Then for every $\delta\in[0,1]$, the random variable $(Y,Z)$ is $\delta$-close to $(Y,U)$ where $U$ is uniform on $\mathbb{Z}_m$ and independent of $Y$, provided that
% 	$$
% 	n\geq c\cdot\frac{m^2}{\alpha\beta}\cdot\log(\frac{m}{\beta})\cdot\log(\frac{m}{\delta}).
% 	$$
% \end{theorem}



\Enote{I removed the definition of DP since it already appears in the intro}
\remove{
\subsection{Differential Privacy}\label{sec:prelim:DP}
We use the following standard definition of (information theoretic) differential privacy, due to \citet{DMNS06}. For notational convenience, we focus on databases over $\oo$.
\begin{definition}[Differentially private mechanisms]\label{def:mech}
	A randomized function $f\colon\oo^n\mapsto \zs$ is an {\sf $n$-size, $(\eps,\delta)$-differentially private mechanism} (denoted $(\eps,\delta)$-\DP) if for every neighboring $w,w'\in \oo^n$ and every function $g\colon \zs\mapsto \zo$, it holds that 
	$$
	\pr{g(f(w))=1}\leq e^{\eps}\cdot \pr{g(f(w'))=1} +\delta.
	$$ 	
	If $\delta=0$, we omit it from the notation.
\end{definition}
}


\subsubsection{Computational Differential Privacy}
There are several ways for defining computational differential privacy (see \cref{sec:related-works}). We use the most relaxed version due to \cite{BNO08}. For notational convenience, we focus on databases over $\oo$.
\begin{definition}[Computational differentially private mechanisms]\label{def:ComMech}
	A randomized function ensemble $f=\set{f_\pk\colon\oo^{n(\pk)}\mapsto \zs}$ is an {\sf $n$-size, $(\eps,\delta)$-computationally differentially private} (denoted $(\eps,\delta)$-$\CDP$) if for every poly-size circuit family $\set{\Ac_\pk}_{\pk\in \N}$, the following holds for every large enough $\pk$ and every neighboring $w,w'\in\oo^{n(\pk)}$:
	$$
	\pr{\Ac_\pk(f_\pk(w))=1}\leq e^{\eps(\pk)}\cdot \pr{\Ac_\pk(f_\pk(w'))=1} +\delta(\pk).
	$$ 
	If $\delta(\pk) = \negl(\pk)$, we omit it from the notation. 
\end{definition}



\subsubsection{Two-Party Differential Privacy}\label{sec:DP}
In this section we formally define distributed differential privacy mechanism (\ie protocols). %For the ease of notation, we consider protocol with no common input.

\begin{definition}\label{def:DP}%\Nnote{fix security parameter}
	A two-party protocol $\Pi=(\Ac,\Bc)$ is {\sf $(\eps,\delta)$-differentially private}, denoted $(\eps,\delta)$-$\DP$, if the following holds for every algorithm $\Dc$: let $\V^\Pc(x,y)(\pk)$ be the view of party $\Pc$ in a random execution of $\Pi(x,y)(1^\pk)$. Then for every $\pk,n \in \N$, $x\in \oo^n$ and neighboring $y,y'\in\oo^n$:
	\begin{align*}
	\pr{\Dc(V^\Ac(x,y)(\pk))=1}\le e^{\eps(\pk)}\cdot \pr{\Dc(V^\Ac (x,y')(\pk))=1}+\delta(\pk),
	\end{align*} 
	and for every $y\in \oo^n$ and neighboring $x,x'\in\oo^{n}$:
	\begin{align*}
	\pr{\Dc(V^\Bc(x,y)(\pk))=1}\le e^{\eps(\pk)}\cdot \pr{\Dc(V^\Bc (x',y)(\pk))=1}+\delta(\pk).
	\end{align*} 	
	Protocol $\Pi$ is {\sf $(\eps,\delta)$-computational differentially private}, denoted $(\eps,\delta)$-$\CDP$, if the above inequalities only hold for a non-uniform \ppt $\Dc$ and large enough $\pk$. We omit $\delta = \negl(\pk)$ from the notation. \footnote{Note that define we give for two-party differentially private protocols is a semi-honest definition, in which we ask for the security to hold when the parties interact in an honest execution of the protocol. Since we are proving a lower bound, starting from this weaker guarantee (as opposed to security against malicious players), yields a stronger result.}
\end{definition}
%We omit $\delta$ from the notation if $\delta$ is a negligible function of $n$.

%\Enote{simulation-based}
\begin{remark}[The definition for computational differential privacy we use]\label{rem:comDPChannel} 
	An alternative, stronger definition of computational differential privacy, known as simulation-based computational differential privacy, requires that the distribution of each party’s view be computationally indistinguishable from a distribution that ensures privacy in an information-theoretic sense. \cref{def:DP} is a weaker notion in comparison. Consequently, establishing a lower bound for a protocol that satisfies this weaker guarantee (as we do in this work) yields a stronger result.%Actually, our lower bound only requires the privacy to hold against \emph{uniform} external observer.
	%\Nnote{Maybe add: When only interesting in \Dp against external observer, the two definitions can be achieve using key-agreement and (single-party) \Dp mechanism. }
\end{remark}




\subsection{Useful Claims}
\remove{
In this section, we state generic lemmas and propositions that we will use later in our proofs.

The following lemma which we prove in \cref{sec:missing-proofs:distance-I}, measures the distance between two uniform stings conditioned one a random index $i$ either being fixed to $0$ or to $1$.

\def\distanceILemma{
    Let $R \la \zo^n$. For any (randomized) function $f:\{0,1\}^n\rightarrow \{0,1\}$ and $\alpha > 0$, it holds that
    \begin{align}\label{eq:f-alpha}
        \ppr{i \la [n]}{\size{\:\ex{f(R) \mid R_i = 0}-\ex{f(R) \mid R_i = 1}\:}\geq \alpha} \leq \frac{2}{n \alpha^2},
    \end{align}
    where the expectations are taken over $R$ and the randomness of $f$.
}

\begin{lemma}\label{lem:distance-I}
    \distanceILemma
\end{lemma}
}

The following two propositions state that given the output of a differentially private function, it is not possible to predict well even a random index (even if all other indexes are leaked). The first proposition handles the information-theoretic case and the second handles the computation case. Both propositions are proven in \cref{sec:missing-proofs:hard-to-guess}. 

\def\propHardToGuessInf{
    Let $f\colon \oo^n \rightarrow \cY$ be an $(\eps,\delta)$-\DP function, let $g \colon [n] \times \oo^{n-1} \times \cY \rightarrow \set{-1,1,\bot}$ be a (randomized) function, and let $X = (X_1,\ldots,X_n) \la \oo^n$. Then the following holds for every $i \in [n]$ where $X_i^* = g(i,X_{-i},f(X_1,\ldots,X_n))$:
    \begin{align*}
        \pr{X_i^* = X_i} \leq e^{\eps}\cdot \pr{X_i^* = -X_i} + \delta.
    \end{align*}
}

\begin{proposition}\label{prop:hard-to-guess-inf}
    \propHardToGuessInf
\end{proposition}


\def\propHardToGuessComp{
    Let $f = \set{f_{\pk} \colon \oo^{n(\pk)} \rightarrow \zo^{m(\pk)}}_{\pk \in \bbN}$ be an $(\eps,\delta)$-\CDP function ensemble, and let $\set{g_{\pk}}_{\pk \in \bbN}$ be a poly-size circuit family. Then, for large enough $\pk$ and $X = (X_1,\ldots,X_{n(\pk)}) \la \oo^{n(\pk)}$, the following holds for every $i \in [n(\pk)]$ where $X_i^* = g_{\pk}(i,X_{-i},f_{\pk}(X_1,\ldots,X_n))$:
    \begin{align*}
        \pr{X_i^* = X_i} \leq e^{\eps(\pk)}\cdot \pr{X_i^* = -X_i} + \delta(\pk).
    \end{align*}
}

\begin{proposition}\label{prop:hard-to-guess-comp}
    \propHardToGuessComp
\end{proposition}





\remove{
\Enote{Chao's old statement:}
\begin{lemma}\label{lem:distance-I-old}
        Let $R \la \zo^n$. 
	For any function $f:\{0,1\}^n\rightarrow \{0,1\}$ and $\alpha<0.01$, it holds that
	$$
	\Pr_{i\la[n]}\left[\: \size{\:\mathbb{E}[f(R) \mid R_i = 0]-\mathbb{E}[f(R) \mid R_i = 1]\:}\geq \alpha\right]\leq \frac{2+2\log(\frac{1}{\alpha})}{n\alpha^2}.
	$$
\end{lemma}
\begin{proof}
	Define $S_1=\{r \in \zo^n \colon f(r)=1\}$. Then for any $i\in[n]$, we have
	$$
	\begin{array}{rl}
		\size{\mathbb{E}[f(R) \mid R_i = 0]-\mathbb{E}[f(R) \mid R_i = 1]}
		&=\size{\Pr[R\in S_1|R_i=0]-\Pr[R\in S_1|R_i=1]}\\
		&=\size{\frac{\Pr[R_i=0|R\in S_1]\cdot\Pr[R\in S_1]}{\Pr[R_i=0]}-\frac{\Pr[R_i=1|R\in S_1]\cdot\Pr[R\in S_1]}{\Pr[R_i=1]}}\\
		&=\frac{2\size{S_1}}{2^n}\size{\Pr[R_i=0|R\in S_1]-\Pr[R_i=1|R\in S_1]}
	\end{array}
	$$
	When $|S_1|\leq \alpha\cdot 2^{n-1}$, we have $\size{\mathbb{E}[f(R) \mid R_i = 0]-\mathbb{E}[f(R) \mid R_i = 1]}\leq\frac{2\size{S_1}}{2^n}\leq \alpha$ for any $i\in[n]$. Hence, in the following, we assume $|S_1|> \alpha\cdot 2^{n-1}$.

	%Define $I_{bad}=\{i|\size{\Pr[R_i=0|R\in S_1]-\Pr[R_i=1|R\in S_1]}>2\alpha\}$ and $k=\size{I_{bad}}$, then for any $i\notin I_{bad}$, we have 
    %$$
    %\begin{array}{rl}
    %    2\alpha&\geq \size{\Pr[R_i=0|R\in S_1]-\Pr[R_i=1|R\in S_1]}\\
    %    &=\size{\frac{\Pr[R\in S_1|R_i=0]\cdot\Pr[R_i=0]}{\Pr[R\in S_1]}-\frac{\Pr[R\in S_1|R_i=1]\cdot\Pr[R_i=1]}{\Pr[R\in S_1]}}\\
    %    &=\size{\Pr[R\in S_1|R_i=0]-\Pr[R\in S_1|R_i=1]}\cdot\frac{1}{2\Pr[R\in S_1]}\\
    %    &\geq \size{\mathbb{E}[f(R) \mid R_i = 0]-\mathbb{E}[f(R) \mid R_i = 1]}\cdot \frac{1}{2},
    %\end{array}
    %$$ 
    %where the last inequality is because $\Pr[R\in S_1]\leq 1$. So that $\size{\mathbb{E}}[f(R) \mid R_i = 0]-\mathbb{E}[f(R) \mid R_i = 1]\leq %4\alpha$.
    Define $I_{bad}=\{i \colon \size{\Pr[R_i=0|R\in S_1]-\Pr[R_i=1|R\in S_1]} \geq 2\alpha\}$ and $k=\size{I_{bad}}$, and denote $I_{bad}=\{i_1,\dots,i_k\}$. Define $(X_{i_1}, \ldots X_{i_k}) = (R_{i_1},\dots,R_{i_k})\mid_{R \in S_1}$. 
    Consider the min-entropy
	$$
	\begin{array}{rl}
		H_{min}(X_{i_1},\dots,X_{i_k})&\leq H(X_{i_1},\dots,X_{i_k})\\
		&\leq \sum_{j=1}^k H(X_{i_j})\\
		&\leq k\cdot \left(-(\frac{1}{2}+2\alpha)\cdot\log(\frac{1}{2}+2\alpha)-(\frac{1}{2}-2\alpha)\cdot\log(\frac{1}{2}-2\alpha)\right)\\
            &=k\cdot \left(-(\frac{1}{2}+2\alpha)\cdot(\log(1+4\alpha)-1)-(\frac{1}{2}-2\alpha)\cdot(\log(1-4\alpha)-1)\right)\\
            &=k\cdot \left(1-(\frac{1}{2}+2\alpha)\cdot\log(1+4\alpha)-(\frac{1}{2}-2\alpha)\cdot\log(1-4\alpha)\right),
		
	\end{array}
	$$
	where $H_{min}(Y)$ is the minimum entropy of $Y$ and $H(Y)$ is the Shannon entropy of $Y$.\Enote{add to preliminaries.}
        The third inequality holds since by the definition of $I_{bad}$, for every $j \in [k]$ it holds that $\size{\pr{X_{i_j} = 1}-\pr{X_{i_j} = 0}} > 2\alpha$, and therefore $H(X_{i_j}) \leq H(1/2 + 2\alpha)$\Enote{define}.
	
	Therefore, there exists $b_1,\dots,b_k\in\{0,1\}$, such that 
	
	\begin{align}\label{eq:min-entropy-result}
		\Pr\left[(R_{i_1},\ldots,R_{i_k}) = (b_1,\ldots,b_k) \mid R\in S_1\right]
		&= \pr{(X_{i_1},\ldots,X_{i_k}) = (b_1,\ldots,b_k)}\\
		&= 2^{-H_{min}(X_{i_1},\dots,X_{i_k})}\nonumber\\
		&\geq 2^{k\cdot \left(-1+(\frac{1}{2}+2\alpha)\cdot\log(1+4\alpha)+(\frac{1}{2}-2\alpha)\cdot\log(1-4\alpha)\right)}.\nonumber
	\end{align}
	
	Let $S_{bad}=\{r \in \zo^n  \colon \set{(r_{i_1},\ldots,r_{i_k}) = (b_1,\ldots,b_k)} \land \set{r\in S_1}\}$.
	It holds that
	\begin{align*}
		|S_{bad}|
		&= \size{S_1} \cdot \Pr\left[(R_{i_1},\ldots,R_{i_k}) = (b_1,\ldots,b_k) \mid R\in S_1\right]\\
		&\geq \alpha\cdot 2^{n-1}\cdot2^{k\cdot \left(-1+(\frac{1}{2}+2\alpha)\cdot\log(1+4\alpha)+(\frac{1}{2}-2\alpha)\cdot\log(1-4\alpha)\right)},
	\end{align*} 
	where the inequality holds by \cref{eq:min-entropy-result} and since $\size{S_1} \geq \alpha\cdot 2^{n-1}$.
	Notice that any string in $S_{bad}$ depends on at most $n-k$ bits. It implies that $|S_{bad}|\leq 2^{n-k}$. Therefore, we have
	$$
	\begin{array}{rl}
		&2^{n-k}\geq \alpha\cdot 2^{n-1}\cdot2^{k\cdot \left(-1+(\frac{1}{2}+2\alpha)\cdot\log(1+4\alpha)+(\frac{1}{2}-2\alpha)\cdot\log(1-4\alpha)\right)} \\
		\Rightarrow& n-k \geq \log \alpha+n-1+k\cdot \left(-1+(\frac{1}{2}+2\alpha)\cdot\log(1+4\alpha)+(\frac{1}{2}-2\alpha)\cdot\log(1-4\alpha)\right)\\
		\Rightarrow& 1-\log \alpha \geq k\cdot((\frac{1}{2}+2\alpha)\cdot\log(1+4\alpha)+(\frac{1}{2}-2\alpha)\cdot\log(1-4\alpha))\\
		\Rightarrow& 1-\log \alpha \geq k\cdot(4\alpha\cdot\log(1+4\alpha)+(\frac{1}{2}-2\alpha)\cdot\log(1-16\alpha^2))\\
        \Rightarrow& 1-\log\alpha \geq k\cdot(15.9\alpha^2-8\alpha^2+32\alpha^3)=k\cdot(7.9\alpha^2+32\alpha^3)>0.5k\alpha^2\\
		\Rightarrow& k\leq \frac{2-2\log \alpha}{\alpha^2} = \frac{2+2\log (1/\alpha)}{\alpha^2},
	\end{array}
	$$
	Where the third transition holds since 
	\begin{align*}
		\lefteqn{(\frac{1}{2}+2\alpha)\cdot\log(1+4\alpha)+(\frac{1}{2}-2\alpha)\cdot\log(1-4\alpha)}\\
		&= 4\alpha\cdot\log(1+4\alpha) + (\frac{1}{2}-2\alpha)\paren{\log(1+4\alpha)+\log(1-4\alpha)}\\
		&= 4\alpha\cdot\log(1+4\alpha)+(\frac{1}{2}-2\alpha)\cdot\log(1-16\alpha^2),
	\end{align*}
	and the forth transition holds since $4\alpha\cdot\log(1+4\alpha)+(\frac{1}{2}-2\alpha)\cdot\log(1-16\alpha^2) > 15.9\alpha^2-8\alpha^2+32\alpha^3$ for $\alpha < 0.01$.
	Thus, we conclude that 
	$$
	\Pr_{i\la[n]}\left[\size{\mathbb{E}[f(R) \mid R_i=0]-\mathbb{E}[f(R) \mid R_i = 1]}\geq \alpha\right]\leq \frac{k}{n}\leq \frac{2+2\log (1/\alpha)}{n\alpha^2}.
	$$
\end{proof}
}


\subsection{Channels and Two-Party Protocols}\label{sec:protocol}

\paragraph{Channels.}A channel is simply a distribution of a pair of tuples defined as follows. 
\begin{definition}[Channels]\label{def:channel} A {\sf channel} $C_{(X,U)(Y,V)}$ of size $\isize$ over alphabet $\Sigma$ is a probability distribution over $(\Sigma^\isize \times\zo^\ast) \times(\Sigma^\isize \times\zo^\ast)$. The ensemble $C_{(X,U)(Y,V)}= \set{C_{(X_\pk,U_\pk)(Y_\pk,V_\pk)}}_{\pk\in \N}$ is an $\isize$-size channel ensemble, if for every $\pk\in \N$, $C_{(X_\pk,U_\pk)(Y_\pk,V_\pk)}$ is an $\isize(\pk)$-size channel. %We denote a channel of size one by a \emph{single-bit} channel. 
We refer to $X$ and $Y$ as the {\sf local outputs}, and to $U$ and $V$ as the {\sf views}.	
\end{definition}

We view a  channel as the experiment in which there are two parties $\Ac$ and $\Bc$.  Party $\Ac$ receives ``output'' $X$ and ``view'' $U$, and party $\Bc$ receives ``output'' $Y$ and ``view'' $V$. Unless stated otherwise, the channels we consider are over the alphabet $\Sigma = \oo$. We naturally identify channels with the distribution that characterizes their output.








\subsubsection{Two-Party Protocols}

A two-party protocol $\Pi=(\Ac,\Bc)$ is \ppt if the running time of both parties is polynomial in their input length. We let $\Pi(x,y)(z)$ or $(\Ac(x),\Bc(y))(z)$ denote a random execution of $\Pi$ on a common input $z$, and private inputs $x,y$.%We assume \wlg that a protocol has a common output (part of its transcript).\Jnote{This is not really the case we consider in this paper..}

\begin{definition}[Oracle-aided protocols]\label{def:ChannelAidedProtocol}
	In a two-party protocol $\Pi$ with oracle access to a {\sf protocol} $\Psi$, denoted $\Pi^\Psi$, the parties make use of the \textit{next-message function} of $\Psi$.\footnote{The function that on a partial view of one of the parties, returns its next message.} In a two-party protocol $\Pi$ with oracle access to a {\sf channel} $C_{Z W}$, denoted $\Pi^C$, the parties can jointly invoke $C$ for several times. In each call, an independent pair $(z,w)$ is sampled according to $C_{Z W}$, one party gets $z$, the other gets $w$.
\end{definition}


\begin{definition}[The channel of a protocol]\label{def:ChannlOfProtocol}
	For a no-input two-party protocol $\Pi= (\Ac,\Bc)$, we associate the channel $C_\Pi$, defined by $\C_\Pi= C_{(X, U),(Y, V)}$, where $X$ and $Y$ are the local outputs of $\Ac$ and $\Bc$ (respectively) and
	$U$ and $V$ are the local views of $\Ac$ and $\Bc$ (respectively).
    
	For a two-party protocol $\Pi$ that gets a security parameter $1^\pk$ as its (only, common) input, we associate the channel ensemble $ \set{C_{\Pi(1^\pk)}}_{\pk\in \N}$. 
\end{definition}

\begin{definition}[$(\alpha,\gamma)$-Accurate channel]\label{def:accurate-func}
	A channel $C = C_{(X, U),(Y, V)}$ is {\sf $(\alpha,\gamma)$-accurate for the function $f$}, if $\ppr{C}{\size{\out(V)-f(X,Y)}\leq \alpha}\ge \gamma$, where $\out(V)$ is the designated output.
    A channel ensemble $C_{(X, U),(Y, V)}= \set{C_{(X_\pk, U_\pk),(Y_\pk, V_\pk)}}_{\pk\in \N}$ is  $(\alpha,\gamma)$-accurate for  $f$ if $C_{(X_\pk, U_\pk),(Y_\pk, V_\pk)}$ is $(\alpha(\pk),\gamma(\pk))$-accurate for $f$, for every $\pk \in \N$.
\end{definition}

\subsubsection{Differentially Private Channels}\label{sec:DPChannel}
Differentially private channels are naturally defined as follows:
\begin{definition}[Differentially private channels]\label{def:DPChannel}
	An $n$-size channel $C = C_{(X, U),(Y, V)}$ with $X, Y$ over $\oo^n$ 
	is {\sf$(\eps,\delta)$-differentially private} (denoted $(\eps,\delta)$-$\DP$) if for every $x \in \Supp(X)$ there exists an $n$-size $(\eps,\delta)$-$\DP$ mechanisms $\Mc_x$ such that $(X,Y,U) \equiv (X,Y,\Mc_X(Y))$, and for every $y \in \Supp(Y)$ there exists an $n$-size $(\eps,\delta)$-$\DP$ mechanisms $\Mc_y'$ such that $(X,Y,V) \equiv (X,Y,\Mc_Y'(X))$. In addition, we say that the channel is \emph{uniform} if $X$ and $Y$ are independent random variables uniformly distributed in $\oo^n$. 
\end{definition}

\begin{definition}[Computational differentially private channels]\label{def:CDPChannel}
	An $n$-size channel ensemble $C = \set{C_{(X_\pk, U_\pk),(Y_\pk, V_\pk)}}_{\pk\in\N}$ with $X_\pk, Y_\pk$ over $\oo^n$ 
	is {\sf$(\eps,\delta)$-computationally differentially private} (denoted $(\eps,\delta)$-$\CDP$) if for every ensemble $\set{x_\pk \in \Supp(X_\pk)}_{\pk\in\N}$ there exists an $n$-size $(\eps,\delta)$-\CDP mechanisms ensemble $\set{\Mc_{x_\pk}}_{\pk\in\N}$ such that $(X_\pk,Y_\pk,U_\pk) \equiv (X_\pk,Y_\pk,\Mc_{X_\pk}(Y_\pk))$, for every $\pk\in\N$, and for every ensemble $\set{y_\pk \in \Supp(Y_\pk)}_{\pk\in\N}$ there exists an $n$-size $(\eps,\delta)$-$\CDP$ mechanisms ensemble $\set{\Mc'_{y_\pk}}_{\pk\in\N}$ such that $(X_\pk,Y_\pk,V_\pk) \equiv (X_\pk,Y_\pk,\Mc_{Y_\pk}'(X_\pk))$ for every $\pk\in \N$. In addition, we say that the channel is \emph{uniform} if $X_\pk$ and $Y_\pk$ are independent random variables uniformly distributed in $\{\pm 1\}^n$ for all $\pk\in\N$.
\end{definition}




% \begin{lemma}~\label{lem:dp-sv-source}
% 	Let $P$ be an $\varepsilon$-DP randomized protocol. Let $X$ and $Y$ be independent random variables uniformly distributed in $\{\pm 1\}^n$ and let random variable $\Pi(X,Y)$ denote the transcript of running $P(X,y)$. Then for every $\pi\in Supp(\Pi)$, the random variables corresponding to the inputs conditioned on transcript $\pi$, $X_\pi$ and $Y_\pi$, are independent $e^{-\varepsilon}$-strong SV source.
% \end{lemma}





\subsubsection{Weak Erasure Channel (\WEC)}

\begin{definition}[\WEC]\label{def:WEC}
	A channel $((O_A,V_A), (O_B,V_B))$ with $O_A \in \set{0,1}$ and $O_B \in \set{0,1,\bot}$ is a {\sf weak erasure channel}, denoted $(\alpha,p,q)$-$\WEC$, if:
	\begin{itemize}
		%\item $O_A\in \set{-1,1}$ and $O_B\in \set{-1,1,\bot}$.
		\item Random erasure: $\pr{O_B = \perp} = 1/2$.
		
		\item Agreement: $\pr{O_A\ne O_B\mid O_B\ne \bot}\le \alpha$.
		
		\item Secrecy:
		
		\begin{enumerate}
			\item For every algorithm $\Dc$ it holds that\label{WEC:item:A}
			\begin{align*}
				%\size{\pr{\Ac(O_A,V_A) = 1 \mid O_B \neq \perp} - \pr{\Ac(O_A,V_A) = 1 \mid O_B = \perp}} \le p
				\size{\pr{\Dc(V_A) = 1 \mid O_B \neq \perp} - \pr{\Dc(V_A) = 1 \mid O_B = \perp}} \le p
			\end{align*}
			(Alice doesn't know if $O_B = \perp$.)
			
			\item For every algorithm $\Dc$ it holds that\label{WEC:item:B}
			\begin{align*}
				\pr{\Dc(V_B) = O_A \mid O_B=\bot} \leq \frac{1+q}{2}.
			\end{align*}
			(i.e., if $O_B=\bot$, Bob don't know what is the value of $O_A$).
			
			%\item $SD((O_A U|O_B=\bot),(O_A U|O_B\ne \bot))\le p$ (The sender don't know if $O_B=\bot$).
			
			%\item $SD(V O_A|O_B=\bot,V(-O_A)|O_B=\bot)\le q$ (If $O_B=\bot$, Bob don't know what the value of $O_A$).
		\end{enumerate}
	\end{itemize}
   We say that a channel ensemble $C=\set{C_\pk}_{\pk\in N}$ is a {\sf computational weak erasure channel}, denoted $(\alpha,p,q)$-\CompWEC, if for every \ppt algorithm $\Dc$ and every sufficiently large $\pk\in\N$, $C_\pk$ satisfies the properties stated in the items above, where the secrecy property holds with respect to a \ppt algorithm $\Dc$. A protocol $\Lambda$ is said to be $(\alpha,p,q)$-$\CompWEC$, if the ensemble induces by the protocol (that is, $C=\set{C_{\Lambda(\pk)}}_{\pk\in\N}$) is $(\alpha,p,q)$-$\CompWEC$.  
\end{definition}



\subsubsection{Approximate Weak Erasure Channel (\AWEC)}\label{sec:AWEC}

\begin{definition}[\AWEC]\label{def:AWEC}
	A channel $C = ((O_A,V_A), (O_B,V_B))$ over $([-n,n] \times \zo^*) \times (([-n,n] \cup \bot)  \times \zo^*)$ is an {\sf approximate weak erasure channel}, denoted $(\ell,\alpha,p,q)$-\AWEC if:
	\begin{itemize}
		
		\item Random erasure: $\pr{O_B = \perp} = 1/2$.
		
		\item Accuracy: $\pr{\size{O_A - O_B} > \ell \mid O_B \ne \bot}\le \alpha$.
		
		\item Secrecy:
		
		\begin{enumerate}
			\item For every algorithm $\Dc$ it holds that\label{AWEC:item:A}
			\begin{align*}
				%\size{\pr{\Ac(O_A,V_A) = 1 \mid O_B \neq \perp} - \pr{\Ac(O_A,V_A) = 1 \mid O_B = \perp}} \le p
				\size{\pr{\Dc(V_A) = 1 \mid O_B \neq \perp} - \pr{\Dc(V_A) = 1 \mid O_B = \perp}} \le p
			\end{align*}
			(Alice doesn't know if $O_B=\bot$).
			
			\item For every algorithm $\Dc$ it holds that\label{AWEC:item:B}
			\begin{align*}
				\pr{\size{\Dc(V_B) - O_A} \leq 1000 \ell \mid O_B=\bot} \leq q.
			\end{align*}
			(i.e., if $O_B=\bot$, Bob can't estimate the value of $O_A$ with error $\leq 1000 \ell$).
		\end{enumerate}
	\end{itemize}
     We say that a channel ensemble $C=\set{C_\pk}_{\pk\in N}$ is a {\sf computational approximate weak erasure channel}, denoted $(\ell,\alpha,p,q)$-\CompAWEC, if for every \ppt algorithm $\Dc$ and every sufficiently large $\pk\in\N$, $C_\pk$ satisfies the properties stated in the items above. A protocol $\Gamma$ is said to be $(\ell,\alpha,p,q)$-$\CompAWEC$, if the ensemble induced by the protocol (that is, $C=\set{C_{\Gamma(\pk)}}_{\pk\in\N}$) is $(\ell,\alpha,p,q)$-$\CompAWEC$.  
\end{definition}

We will make use of the following lemma, which shows that for some choices of the parameters, \AWEC implies \WEC. The lemma is proven in \cref{sec:AWEC-to-WEC}.

\begin{lemma}\label{lemma:AWEC-to-WEC}
	For every $\ell> 0$, there exists a \ppt protocol $\Lambda = (\Pc_1,\Pc_2)$ such that given an oracle access to an $(\ell,\alpha,p,q)$-\AWEC $C$, the channel $\tilde{C}$ induced by $\Lambda^C$ is $(\alpha'=\alpha+0.001,\: p' = p ,\:  q' = 1/2 + 2(q+0.01))$-\WEC.
	Furthermore, the proof is constructive in a black-box manner:
	\begin{enumerate}
		\item There exists an oracle-aided \ppt algorithm $\Ec_1$ such that for every channel $C = ((\OA,\VA), (\OB,\VB))$ and algorithm $\Dc$ violating the \WEC secrecy property~\ref{WEC:item:A} of $\tilde{C}$, algorithm $\Ec_1^{\Dc}$ violates the \AWEC secrecy property~\ref{AWEC:item:A} of $C$.
		
		\item There exists an oracle-aided \ppt algorithm $\Ec_2$ such that for every channel $C = ((\OA,\VA), (\OB,\VB))$ and algorithm $\Dc$ violating the \WEC secrecy property~\ref{WEC:item:B} of $\tilde{C}$, algorithm $\Ec_2^{\Dc}$ violates the \AWEC secrecy property~\ref{AWEC:item:B} of $C$.
	\end{enumerate}
\end{lemma}

Since \cref{lemma:AWEC-to-WEC} is constructive, the following is an immediate corollary.
\begin{corollary}\label{cor:CompAWEC to CompWEC}
There exists an oracle aided \ppt protocol $\Lambda$, such that given a protocol $\Gamma$ that induces $(\ell,\alpha,p,q)$-\CompAWEC, it holds that $\Lambda^\Gamma$ is $(\alpha'=\alpha+0.001,\: p' = p ,\:  q' = 1/2 + 2(q+0.01))$-\CompWEC.  
\end{corollary}
\begin{proof}[Proof of \ref{cor:CompAWEC to CompWEC}]
Let $\Lambda$ be the \ppt algorithm guaranteed  by Lemma \ref{lemma:AWEC-to-WEC}. Given an $(\ell,\alpha,p,q)$-\CompAWEC protocol $\Gamma$, we define $\Lambda(\pk)={\Lambda^{\Gamma(\pk)}(\pk)}$. Assume towards a contradiction that $\Lambda$ is not a $(\alpha',p',q')$-\CompWEC. It follows that there exists a \ppt $\Dc$ that for infinity many $\pk\in\N$ contradicts one of the \WEC secrecy properties of channel ensemble $\set{C_{\Lambda(\pk)}}_{\pk\in\N}$. Fix $\pk\in\N$ for which this holds. By Lemma \ref{lemma:AWEC-to-WEC}, there exists a \ppt $\Ec^\Dc$ that for every such $\pk$  contradicts one of the secrecy properties of the channel $C_{\Gamma(\pk)}$. This implies that for infinity many $\pk\in\N$, $\Ec^\Dc$  contradict the secrecy of the channel ensemble $\set{C_{\Gamma(\pk)}}_{\pk\in\N}$, which is a contradiction since this would means that $\Gamma$ is not a $(\ell,\alpha,p,q)$-\CompAWEC.       
\end{proof}



\subsection{Oblivious Transfer (\OT)}

\paragraph{Secure Computation.}
We use the standard notion of securely computing a functionality, \cf  \cite{Goldreich04}.
\begin{definition}[Secure computation]\label{def:SFE}
	A two-party protocol {\sf securely computes a functionality $f$}, if it does so according to the real/ideal paradigm.   We add the term perfectly/statistically/computationally/non-uniform computationally, if the simulator's output is  perfect/statistical/computationally indistinguishable/  non-uniformly indistinguishable from  the real distribution.  The protocol have the above notions of security {\sf against semi-honest  adversaries}, if its security only  guaranteed to holds against an adversary that follows the prescribed protocol.   Finally, for the case of perfectly secure computation, we naturally apply the above notion also to the non-asymptotic case: the protocol with no security parameter perfectly  compute a functionality $f$.
	
	A two-party protocol {\sf securely computes a functionality ensemble $f$ with oracle to a channel $C$}, if it does so according to the above definition when the parties have access to a trusted party computing $C$. All the above adjectives naturally extend to this setting.
\end{definition}

\paragraph{Oblivious Transfer.}
The (one-out-of-two) oblivious transfer functionality is defined as follows.
\begin{definition}[oblivious transfer functionality $f_{\OT}$]\label{def:OTfunc}
	The oblivious transfer functionality over $\zo \times (\zs)^2$ is defined by  $f_{\OT} (i,(\sigma_0,\sigma_1)) = (\perp,\sigma_i)$.
\end{definition}
A protocol is $\ast$ secure OT,   for \\$\ast\in \set{\text{semi-honest statistically/computationally/computationally non-uniform}}$, if it  compute the $f_{\OT}$  functionality with $\ast$ security.





% \begin{definition}[Computational oblivious transfer, semi-honest model]
% A protocol $\Pi=(\Ac,\Bc)$ is a semi-honest 1-out-of-2 computational oblivious transfer (comp-OT) protocol if the following holds. Given a common input $1^{\pk}$, the parties $\Ac$ and $\Bc$ run the protocol $\Pi(1^\pk)$ (in an honest manner) and    
% $\Ac$ outputs $X=(m_1,m_2)\in \zo\times\zo$ and has a view $U$ and $\Bc$ outputs $Y=(i,\hat{m})\in\zo\times\zo$ and has a view $V$, and the following properties are satisfied:
% \begin{enumerate}
%     \item \textbf{Correctness:} 
%     $\pr{\hat{m}\neq m_i}<\negl(\pk).$ 
    
%     \item \textbf{A's Privacy:} For every \ppt $\Dc$ and every sufficiently large $\pk$:
%     $\pr{\Dc(V)=m_{i-1}}<(1+\negl(\pk))/2$
    
%     \item \textbf{B's Privacy:} For every \ppt $\Dc$ and every sufficiently large $\pk$:
%     $\pr{\Dc(U)=i}<(1+\negl(\pk))/2$  
% \end{enumerate}
% \end{definition}

We make use of the following useful results by Wullschleger on oblivious transfer amplification from weak channels.
\begin{theorem}[\cite{Wullschleger09}, from \WEC to statistically secure \OT]\label{thm:WEC TO OT IT}
    There exists an oracle aided protocol $\Pi$ such that the following holds: Given a $(\alpha,p,q)$-\WEC $C$, if $44(\alpha+p)\le 1-q$ then $\Pi^{C}(1^\pk)$ is a semi-honest statistically secure \OT.
\end{theorem}

The following computational version of \cref{thm:WEC TO OT IT} is implicit in \cite{Wullschleger09} and is based on the computational proof explicitly stated in \cite{Wul07} (see Section 6 in \cite{Wullschleger09} for discussion).   

\begin{theorem}[\cite{Wullschleger09,   Wul07}, from \CompWEC to computinally secure \OT]\label{thm:WEC TO OT Comp}
    There exists an oracle aided protocol $\Pi$ such that the following holds: Given a $(\alpha,p,q)$-\CompWEC protocol $\Lambda$, if $44(\alpha+p)\le 1-q$ then $\Pi^{\Lambda}$ is a semi-honest computational secure \OT.
\end{theorem}



% \begin{definition}[Computational 1-out-of-2 Oblivious Transfer, semi-honest model]
% A protocol $\Pi=(\Ac,\Bc)$ is a semi-honest 1-out-of-2 $(\eps,\alpha,\beta)$-oblivious transfer (OT) protocol if the following holds. 

% The parties $\Ac$ and $\Bc$ run the protocol (in an honest manner) and    
% $\Ac$ outputs $X=(m_1,m_2)\in \zo\times\zo$ and has a view $U$ and $\Bc$ outputs $Y=(i,\hat{m})\in\zo\times\zo$ and has a view $V$, and following properties are satisfied:
% \begin{enumerate}
%     \item \textbf{Correctness:} 
%     $\pr{\hat{m}\neq m_i}<\eps.$ 
    
%     \item \textbf{A's Privacy:} For every adversary $\Dc$:
%     $\pr{\Dc(V)=m_{i-1}}<(1+\alpha)/2$
    
%     \item \textbf{B's Privacy:} For every adversary $\Dc$: $\pr{\Dc(U)=i}<(1+\beta)/2$  
% \end{enumerate}
% \end{definition}

\subsection{Algebraizing Deontic Action Logic}
We start the algebraization of  \DAL introducing its signature, i.e., the symbols needed to capture the language of the logic in an algebraic way.


%The first step in algebraizing a logic, and \DAL is no exception, is to view formulas of a logical language as terms of an algebraic language over an appropriate signature.
%To this end, we introduce the following definition.
%We define the signature and the algebraic language that we use in what follows.

\medskip
\begin{definition}\label[definition]{def:signature}
The signature of \DAL is a tuple $\Sigma = \tup{S, \Omega}$ where:
% \begin{enumerate}
% 	\item
		$S = \{\sorta, \sortf\}$; and %, i.e., $S$ has sort symbol $\sorta$ for actions, and sort symbol $\sortf$ for formulas; and
	% \item
		$\textstyle \bigcup \Omega = \{
			{\sqcup}, {\sqcap}, \bar{~}, \iact, \mathsf{1},
			{\lor}, {\land}, {\lnot}, {\bot}, {\top},
			{=}, {\perm}, {\forb}
		\}$.
	The symbols in $\textstyle \bigcup \Omega$ are further categorized into sets
		$\Omega_{\sorta\sorta\sorta}$,
		$\Omega_{\sorta\sorta}$,
		$\Omega_{\sorta}$,
		$\Omega_{\sortf\sortf\sortf}$,
		$\Omega_{\sortf\sortf}$,
		$\Omega_{\sortf}$,
		$\Omega_{\sorta\sorta\sortf}$,
		$\Omega_{\sorta\sortf}$ summarized in \Cref{tab:sig}.
	
	\begin{figure}
		\centering
		\begin{tabular}{r@{~}lr@{~}lr@{~}lr@{~}lr@{~}l}
			\toprule
			&& \multicolumn{8}{c}{operations}
			\tabularnewline
			\cmidrule{3-10}
			& sorts && actions && formulas && equality && normative
			\tabularnewline
			\midrule
			$S$ & $=\{\sorta, \sortf\}$ &
			$\Omega_{\sorta\sorta\sorta}$ & $= \{ {\sqcup}, {\sqcap}\}$ &
			$\Omega_{\sortf\sortf\sortf}$ & $= \{ {\lor}, {\land}\}$ &
			$\Omega_{\sorta\sorta\sortf}$ & $= \{ {=}\}$ &
			$\Omega_{\sorta\sorta\sortf}$ & $= \{ {\perm}, {\forb}\}$
			\tabularnewline
			&&
			$\Omega_{\sorta\sorta}$ & $= \{\bar{~}\}$ &
			$\Omega_{\sortf\sortf}$ & $= \{{\lnot}\}$ &
			\tabularnewline
			&&
			$\Omega_{\sorta}$ & $= \{\iact, \mathsf{1}\}$ &
			$\Omega_{\sortf}$ & $= \{\bot, \top\}$ &
			\tabularnewline
			\bottomrule
		\end{tabular}\\[1em]
		\caption{The Signature used in the algebraization of \DAL.}\label{tab:sig}
	\end{figure}
% \end{enumerate}
\end{definition}
\medskip

In our discussion on the algebraization of \DAL, we take $\Sigma = \tup{S, \Omega}$ to be as in \Cref{def:signature}.
Intuitively, the sort symbols $\sorta$ and $\sortf$ in $S$ categorize actions and formulas, respectively.
In turn, we think of $\Omega$ as containing
symbols for operations on actions, operations on formulas, and
(heterogeneous) operations from actions to formulas.

\medskip
\begin{definition}\label{dal:talg}
	The term algebra $\TAlgebra$ for \DAL uses the set ${\bact}$ as the set of variables of sort $\sorta$, and the empty set $\emptyset$ as the set of variables of sort $\sortf$.  We call this algebra the deontic action term algebra, or the algebraic language of \DAL.
\end{definition}
\medskip

The term algebra $\TAlgebra$ in \Cref{dal:talg} is interpreted over \emph{deontic action algebras}. Deontic action algebras are to \DAL what Boolean algebras are to Classical Propositional Logic, or what Heyting algebras are to Intuitionistic Propositional Logic.
We provide the precise definition of a deontic action algebra in \Cref{definition:deontic:algebra}.

\medskip
\begin{definition}\label[definition]{definition:deontic:algebra}
	A deontic action algebra is an algebra
		$\DAlgebra =
			\langle
				\Algebra[A], \Algebra[F], \E, \P, \F
			\rangle$
	of type $\Sigma$ where:%
		\footnote{
			We use $=$ as the function interpreting `$=$' in $\Algebra[F]$, and $=_{\Algebra[A]}$ and $=_{\Algebra[B]}$ as equality in $\Algebra[A]$ and $\Algebra[B]$, respectively.
		}
		$\Algebra[A] = \tup{A, {\sqcup}, {\sqcap}, \bar{~}, \iact, \uact}$
		and
		$\Algebra[F] = \tup{F, {\lor}, {\land}, {\lnot}, \bot, \top}$ are Boolean algebras,
		and
			$\E$,
			$\P$,
			and
			$\F$,
		satisfy the conditions below
		\begin{multicols}{3}
			\begin{enumerate}[leftmargin=\parindent]
				\item $\P(a {\sqcup} b) \,{=_{\Algebra[F]}}\, {\P(a) {\land} \P(b)}$
				\item $\F(a {\sqcup} b) \,{=_{\Algebra[F]}}\, {\F(a) {\land} \F(b)}$
				\item ${\P(a) {\land} \F(a)} \,{=_{\Algebra[F]}}\, (a \,{=}\, \iact)$
				\item $(a = b) \land \P(a) \preccurlyeq \P(b)$
				\item $(a = b) \land \F(a) \preccurlyeq \F(b)$
				\item[]
				\item ${a \,{=_{\Algebra[A]}}\, b} ~\text{iff}~ {(a \,{=}\, b) \,{=_{\Algebra[F]}}\, \top}$.
			\end{enumerate}
		\end{multicols}
	\noindent
	Let $h: \TAlgebra \to \DAlgebra$ be an interpretation.
	We use $\DAlgebra, h \vDash \tau_1 \doteq \tau_2$ as a shorthand for $h(\tau_1) = h(\tau_2)$.
	In turn, let $\DALVariety$ indicate the class of all deontic action algebras.
	We use $\DALVariety \vDash \tau_1 \doteq \tau_2$ as the universal quantification of $\vDash$ to all deontic action algebras in $\DALVariety$ and all interpretations on these algebras; i.e., $\DALVariety \vDash \tau_1 \doteq \tau_2$ iff $\DAlgebra, h \vDash \tau_1 \doteq \tau_2$, for all $\DAlgebra \in \DALVariety$, and all interpretations $h: \TAlgebra \to \DAlgebra$.
	% The condition $a = b \iff \E(a,b) = \top$
	% is a pair of quasi-identities.
	% This makes the class $\DALVariety$ of all deontic action algebras a quasi-variety.
\end{definition}
\medskip

The next two results are immediate.

\medskip
\begin{proposition}\label[proposition]{pro:dal:act2form}
	It follows that $\DALVariety \vDash \alpha \doteq_{\sorta} \beta$ iff $\DALVariety \vDash (\alpha = \beta) \doteq_{\sortf} \top$.
\end{proposition}
\medskip

\begin{proposition}\label[proposition]{pro:dal:qvariety}
	The class $\DALVariety$ of all deontic action algebras is a quasi-variety.
\end{proposition}
\begin{proof}
	It suffices to show that the conditions in the definition of a deontic action algebra can be captured by equations, or quasi-equations.
	The interesting cases are:
	\smallskip
	\begin{enumerate}[leftmargin=\parindent]
		\item $\P(a \sqcup b) =_{\Algebra[F]} {\P(a) \land \P(b)}$ expressed as $\P(a \sqcup b) \doteq_{\sortf} \P(a) \land \P(b)$;
		\item ${\P(a) \land \F(a)} =_{\Algebra[F]} \E(a,\iact)$ expressed as ${\P(a) \land \P(b)} \doteq_{\sortf} {a = \iact}$;
		\item $(a = b) \land \P(a) \preccurlyeq \P(b)$ expressed as $((a = b) \land \P(a)) \lor \P(b) \doteq_{\sortf} \P(b)$; and
		\item ${a =_{\Algebra[A]} b} ~\text{iff}~ {(a = b) =_{\Algebra[F]} \top}$ expressed as the quasi-equations
			${a \doteq_{\sorta} b} \To {(a = b) \doteq_{\sortf} \top}$, and
			${(a = b) \doteq_{\sortf} \top} \To {a \doteq_{\sorta} b}$. \qedhere
	\end{enumerate}
\end{proof}
\medskip

The definition of a deontic action algebra in \Cref{definition:deontic:algebra} draws on ideas and terminology from Pratt's dynamic algebras~\cite{Pratt:1991}. We present the general structure of a deontic action algebra in a form slightly different from the general treatment of many-sorted algebras in \Cref{section:basics}. In doing this we wish to highlight the modular nature of deontic action algebras. In \Cref{section:new:dals}, we leverage this modularity to introduce variants of \DAL by considering different algebraic structures for each component of the deontic action algebra. 
Finally, notice that, as made clear in \Cref{pro:dal:qvariety}, our treatment of equality in the logic results in the class of deontic action algebras being a quasi-variety.
The result in \Cref{pro:dal:act2form} tells us we can dispense explicitly referring to equations on actions as they are also captured as particular equations on formulas via equality in the logic.

\medskip
\begin{example}
	\Cref{ex:deontic:algebra} depicts the deontic action algebra ${\DAlgebra = \langle \Algebra[A], \Algebra[2], \E, \P, \F \rangle}$ where: the algebra $\Algebra[A]$ of actions is the free Boolean algebra on the set of generators $\{a,b\}$.
	In $\DAlgebra$, the functions $\P$ and $\F$ are defined as:
	\begin{align*}
		\P(x) &=
			{\begin{cases}
				1 & \text{if } x \preccurlyeq \bar{b} \\
				0 & \text{otherwise.}
			\end{cases}}
		&
		\F(x) &=
			{\begin{cases}
				1 & \text{if } x \preccurlyeq b \\
				0 & \text{otherwise.}
			\end{cases}}
	\label{eq:ex:pf}
	\end{align*}
	% The dashed arrows from the graph of $\Algebra[A]$ to the graph of $\Algebra[F]$ show the elements of $|\Algebra[A]|$ that $\P$ maps to $1$.
	In \Cref{ex:deontic:algebra}, the elements of $|\Algebra[A]|$ that $\P$ and $\F$ map to $\top$ are indicated with green and red, respectively.
	To avoid overcrowding the drawing, we have chosen not to highlight the elements these operations do not map to $\top$.
	In \Cref{ex:deontic:algebra} also, the sets $P$ and $F$ indicate which actions are permitted and which ones are forbidden.
	Note that both sets form an ideal in $\Algebra[A]$ whose intersection contains only the $\iact$ element of the algebra.
	It can easily be inferred from this example that: if $\P(x) = \top$ for all $x \in |\Algebra[A]|$, then, $\F(x) = \bot$ for all $\iact \prec x \in |\Algebra[A]|$.
	Similarly, if $\F(x) = \top$ for all $x \in |\Algebra[A]|$, then, $\P(x) = \bot$ for all $\iact \prec x \in |\Algebra[A]|$.
	These cases are known as \emph{deontic heaven} and \emph{deontic hell}, respectively.
	We will discuss them later on.
\end{example}
\medskip

\begin{figure}
	\centering
	\includegraphics[width=0.5\textwidth]{deontic-algebra.pdf}\\[.5em]
	\caption{A Deontic Action Algebra.}\label{ex:deontic:algebra}
\end{figure}

\begin{example}
	Let $\DAlgebra$ be the deontic action algebra in \Cref{ex:deontic:algebra}, and $\mathsf{drinking}$, $\mathsf{driving}$, and $\mathsf{parking}$, be basic action symbols.
	In addition, let $h: \TAlgebra \to \DAlgebra$ be an interpretation s.t.:
		$h_{\sorta}(\mathsf{drinking}) = b$,
		$h_{\sorta}(\mathsf{driving}) = a$, and
		$h_{\sorta}(\mathsf{parking}) = \bar{b}$.
	It follows that:

   \begin{multicols}{2}
   \begin{enumerate}
      \item $h(\overline{\mathsf{parking}} = \mathsf{driving}) =_{\Algebra[F]} \top$
      \item $h(\forb(\mathsf{drinking} \sqcap \mathsf{driving})) =_{\Algebra[F]} \top$
      \item $h(\perm(\mathsf{drinking} \sqcap \mathsf{parking})) =_{\Algebra[F]} \top$
      \item $h(\perm(\mathsf{driving} \sqcup \mathsf{parking})) \neq_{\Algebra[F]} \top$.
   \end{enumerate}
   \end{multicols}

   \noindent In brief, the deontic action algebra $\DAlgebra$ may be understood as the algebraic version of the deontic model $\DeonticModel$ in \Cref{section:dal:semantics}.
\end{example}

The following proposition shows the ideals in the deontic action algebra in \Cref{ex:deontic:algebra} are indeed a distinguishing characteristic of the operations of permission and prohibition.

\medskip
\begin{proposition}\label[proposition]{prop:dal:ideal}
	Let $\DAlgebra = \tup{\Algebra[A], \Algebra[F], \E, \P, \F}$ be a deontic action algebra. The pre-image $P$ of $\top$ under $\P$, as well as the preimage $F$ of $\top$ under $\F$, are ideals in $\Algebra[A]$ s.t.\ ${{P \cap F} = \{\iact\}}$.
\end{proposition}
\begin{proof}
	The result is obtained from the following:
		\medskip
		\begin{enumerate}%[(i)]
			\setlength{\itemsep}{5pt}
			\item
			For all $\{a,b\} \subseteq P$, ${a \sqcup b} \in P$.
			To see why, let $\{a,b\} \subseteq P$.
			Then, $\P(a) = \P(b) = \top$, and $\P(a) \land \P(b) = \top$.
			The properties of $\P$ in \Cref{definition:deontic:algebra} ensure $\P(a) \land \P(b) = \P(a \sqcup b)$.
			This implies $\P(a \sqcup b) = \top$; and so $(a \sqcup b) \in P$.

			\item
			For all $a \in P$ and $b \in |\Algebra[A]|$, ${(a \sqcap b)} \in P$.
			To see why, let $a \in P$ and $b \in |\Algebra[A]|$.
			We know $\P(a) = \top$ and $a = {a \sqcup (a \sqcap b)}$.
			This means $\P({a \sqcup (a \sqcap b)}) = \top$.
			The properties of $\P$ in \Cref{definition:deontic:algebra} ensure $\P(a \sqcup (a \sqcap b)) = \P(a) \land \P(a \sqcap b)$.
			This means, $\P(a) \land \P(a \sqcap b) = \top$.
			From our supposition, $\P(a \sqcap b) = \top$; and so $(a \sqcap b) \in P$.

			\item
			The arguments in 1 and 2 remain true if we replace $P$ and $\P$ for $F$ and $\F$, resp.

			\item
			${P \cap F} = \{ \iact \}$.
			To see why, note that $\P(\iact) = \F(\iact) = \top$; and so $\{\iact\} \subseteq {P \cap F}$.
			In turn, consider an arbitrary $a \in (P \cap F)$.
			Then, $\P(a) = \F(a) = \top$.
			This implies $\P(a) \land \F(a) = \top$, and so $(a = \iact) =_{\Algebra[F]} \top$.
			The `iff' condition in \Cref{definition:deontic:algebra} ensures $a =_{\Algebra[A]} \iact$.
			Since $a$ is arbitrary, the last step tells us that any element in ${P \cap F}$ is equal to $\iact$.
			Therefore, ${P \cap F} \subseteq \{ \iact \}$. \qedhere
		\end{enumerate}
\end{proof}

We proceed to connect the deontic action algebras in $\DALVariety$ with the theorems of $\DAL$.

\medskip
\begin{theorem}[Soundness]\label[theorem]{theorem:soundness}
	If $\varphi$ is a theorem of \DAL, then, $\DALVariety \vDash {\varphi \doteq \top}$.
\end{theorem}
\begin{proof} %(Sketch)
	Let $\DAlgebra \in \DALVariety$ and $h: \TAlgebra \to \DAlgebra$ be any interpretation.
	We continue by induction on the length of the proof of $\varphi$.
	We prove the more interesting cases;  others are similar.

	\medskip
	\begin{enumerate}[leftmargin=\parindent]
		\setlength{\itemsep}{5pt}
		
		\item
		$h_{\sortf}(\perm(\alpha\sqcup\beta) \liff (\perm(\alpha) \land \perm(\beta)))
			=_{\Algebra[F]} \top$.
		The result follows from items (a)--(c) below.

			\medskip
			\begin{enumerate}[leftmargin=\parindent]
				\setlength{\itemsep}{5pt}
				\item
				\begin{description}[leftmargin=\parindent]
					\item[]
					$h_{\sortf}(\perm(\alpha\sqcup\beta) \liff (\perm(\alpha) \land \perm(\beta))) =_{\Algebra[F]}$
					\item[]
					$h_{\sortf}(
						(\lnot \perm(\alpha\sqcup\beta) \lor (\perm(\alpha) \land \perm(\beta)))
						\land
						(\lnot (\perm(\alpha) \land \perm(\beta)) \lor \perm(\alpha\sqcup\beta))
						)=_{\Algebra[F]}$
					\item[]
					$h_{\sortf}(
						\lnot \perm(\alpha\sqcup\beta) \lor (\perm(\alpha) \land \perm(\beta)))
					\land
						h_{\sortf}(
						(\lnot (\perm(\alpha) \land \perm(\beta)) \lor \perm(\alpha\sqcup\beta)))$.
				\end{description}

				\item 
				\begin{description}[leftmargin=\parindent]
					\item[]
					$h_{\sortf}(
						\lnot
							\perm(\alpha\sqcup\beta)
							\lor
							(\perm(\alpha) \land \perm(\beta))) =_{\Algebra[F]}$
					\item[]
						$\lnot
							\perm(h_{\sorta}(\alpha \sqcup \beta))
							\lor
							(\perm(h_{\sorta}(\alpha)) \land \perm(h_{\sorta}(\beta))) =_{\Algebra[F]}$
					\item[]
						$\lnot
							\perm(h_{\sorta}(\alpha \sqcup \beta))
							\lor
							(\perm(h_{\sorta}(\alpha) \sqcup h_{\sorta}(\beta))) =_{\Algebra[F]}$
							\dotfill \Cref{definition:deontic:algebra}(1)
					\item[]
						$\lnot
							\perm(h_{\sorta}(\alpha \sqcup \beta))
							\lor
							(\perm(h_{\sorta}(\alpha \sqcup \beta))) =_{\Algebra[F]} \top$.
				\end{description}

				\item $h_{\sortf}(
					\lnot(\perm(\alpha) \land \perm(\beta))
					\lor
					\perm(\alpha\sqcup\beta)) =_{\Algebra[F]} \top$ is similar to (b).
			\end{enumerate}

		\item
		$h_{\sortf}((\perm(\alpha) \land \forb(\alpha)) \to (\alpha = \iact))
			=_{\Algebra[F]} \top$.
		Then,

			\medskip
			\begin{description}[leftmargin=\parindent]
				\item[]
				$h_{\sortf}((\perm(\alpha) \land \forb(\alpha)) \to (\alpha = \iact)) =_{\Algebra[F]}$
				\item[]
				$h_{\sortf}(
					\lnot (\perm(\alpha) \land \forb(\alpha)) \lor (\alpha = \iact)) =_{\Algebra[F]}$
				\item[]
				$\lnot
					(\perm(h_{\sorta}(\alpha)) \land \forb(h_{\sorta}(\alpha)))
					\lor
					h_{\sortf}(\alpha = \iact) =_{\Algebra[F]}$
				\item[]
				$\lnot
					(\perm(h_{\sorta}(\alpha)) \land \forb(h_{\sorta}(\alpha)))
					\lor
					h_{\sortf}(\alpha = \iact) =_{\Algebra[F]}$
				\item[]
				$\lnot
					(h_{\sorta}(\alpha) = \iact)
					\lor
					h_{\sortf}(\alpha = \iact) =_{\Algebra[F]}$ \dotfill \Cref{definition:deontic:algebra}(3)
				\item[]
				$\lnot
					h_{\sortf}(\alpha = \iact)
					\lor
					h_{\sortf}(\alpha = \iact) =_{\Algebra[F]} \top$.
			\end{description}

		\item
		$h_{\sortf}(((\alpha = \beta) \land \perm(\alpha)) \to \perm(\beta))
			=_{\Algebra[F]} \top$.
			Then,
			
			\medskip
			\begin{description}[leftmargin=\parindent]
				\item[]
					$h_{\sortf}(((\alpha \,{=}\, \beta) {\land} \perm(\alpha)) \to \perm(\beta)) =_{\Algebra[F]}$
				\item[]
					$\lnot
						h_{\sortf}((\alpha \,{=}\, \beta) {\land} \perm(\alpha))
					\lor
					\perm(h_{\sorta}(\beta)) =_{\Algebra[F]}$
				\item[]
					$\lnot
						h_{\sortf}((\alpha \,{=}\, \beta) {\land} \perm(\alpha))
					\lor
						((h_{\sorta}(\alpha) \,{=}\, h_{\sorta}(\beta)) {\land} \perm(h_{\sorta}(\alpha)))
						\lor
						\perm(h_{\sorta}(\beta)) =_{\Algebra[F]}$
						\dotfill \Cref{definition:deontic:algebra}(4)
				\item[]
					$\lnot
						h_{\sortf}((\alpha \,{=}\, \beta) {\land} \perm(\alpha))
					\lor 
						h_{\sortf}((\alpha \,{=}\, \beta) {\land} \perm(\alpha))
					\lor
						\perm(h_{\sorta}(\beta)) =_{\Algebra[F]} \top$.
			\end{description}

			\smallskip
			\noindent
			The result in (3.) is a particular case of the axioms E2 in \Cref{dal:axioms}.
			Other instances can be proven by induction on the size of the formula $\varphi$.
			\qedhere
	\end{enumerate}
\end{proof}

\Cref{theorem:soundness} implies that not every formula of \DAL is provable in the logic.
In particular, non-theorems are not provable.
To see why, consider a theorem $\varphi$, the deontic action algebra $\DAlgebra = \langle \Algebra[A], \Algebra[2], \E, \P, \F \rangle$, and any interpretation $h: \TAlgebra \to \DAlgebra$.
From \Cref{theorem:soundness}, we have $h_{\sortf}(\varphi) = \top$.
Since $h$ is a homomorphism, $h_{\sortf}(\lnot \varphi) = \iact$.
Using the contrapositive of \Cref{theorem:soundness}, $\lnot \varphi$ is not a theorem; i.e., it is not provable.

The converse of \Cref{theorem:soundness}, i.e., the algebraic completeness of \DAL, requires us to show that every non-theorem $\varphi$ of \DAL is falsified in some deontic action algebra $\DAlgebra$ (i.e., there is an interpretation $h: \TAlgebra \to \DAlgebra$ s.t.\ $h_{\sortf}(\varphi) \neq \top$).
We arrive at this result introducing an appropriate notion of congruence, and constructing a quotient algebra via this congruence. 

\medskip
\begin{proposition}\label[proposition]{prop:congruence}
	Let $\TAlgebra$ be the deontic term algebra, and ${\cong_{\sorta}} \subseteq |\TAlgebra|_{\sorta} \times |\TAlgebra|_{\sorta}$ and ${\cong_{\sortf}} \subseteq |\TAlgebra|_{\sortf} \times |\TAlgebra|_{\sortf}$ be s.t.: 1.~$\alpha \cong_{\sorta} \beta$ iff $\alpha = \beta$ is a theorem, and 2.~$\varphi \cong_{\sortf} \psi$ iff $\varphi \liff \psi$ is a theorem.
	It follows that $\cong_{\sorta}$ and $\cong_{\sortf}$ define a congruence $\cong$ on $\TAlgebra$.
\end{proposition}
\medskip

\begin{proposition}\label[proposition]{prop:lindenbaum}
	The quotient of the deontic action term algebra $\TAlgebra$ under $\cong$ is a structure
		$\LTAlgebra = \tup{\Algebra[A], \Algebra[F], \E, \P, \F}$
	where:
		1.~$|\Algebra[A]| = {\act/{\cong_{\sorta}}}$,
		2.~$|\Algebra[F]| = {\form/{\cong_{\sortf}}}$, and
		3.~the operations in $\LTAlgebra$ are those induced by the equivalence classes in $\cong$.
		% \medskip
		% \begin{enumerate}[leftmargin=\parindent]
		% 	\item
		% 		$\Algebra[A]
		% 			= \tup{{\act/{\cong_{\sorta}}}, {\sqcup}, {\sqcap}, {\bar{~}}, {\iact}, {\uact}}$
		% 	\item
		% 		$\Algebra[F]
		% 			= \tup{{\form/{\cong_{\sortf}}}, {\lor}, {\land}, {\lnot}, {\bot}, {\top}}$
		% \end{enumerate}
		% \medskip
	It follows that $\LTAlgebra \in \DALVariety$.
\end{proposition}
\begin{proof}%[Sketch]
	It is clear that $\Algebra[A]$ and $\Algebra[F]$ are Boolean algebras.
	Let us use $\check{\_}$ to indicate the operations in $\LTAlgebra$ induced by $\cong$, and to separate them from the corresponding symbols.
	The result is concluded if $\check{\E}$, $\check{\P}$, and $\check{\F}$ satisfy the conditions in \Cref{definition:deontic:algebra}.
  	We prove some interesting cases only.
	% We drop the subscript $\cong$ unless it is strictly necessary to improve readability.

	\medskip
	\begin{enumerate}[leftmargin=\parindent]
		\setlength{\itemsep}{7pt}
		
		\item%
		{$\check{\P}([\alpha \sqcup \beta]_{\cong_{\sorta}}) =_{\Algebra[F]} \check{\P}([\alpha]_{\cong_{\sorta}}) \land \check{\P}([\beta]_{\cong_{\sorta}})$}
		% See (a) below.

			\medskip
			\begin{description}[leftmargin=\parindent]
				\item[]
					$\check{\P}([\alpha \sqcup \beta]_{\cong_{\sorta}}) =_{\Algebra[F]}
					[\perm(\alpha \sqcup \beta)]_{\cong_{\sortf}} =_{\Algebra[F]}$
				\item[]
					$[\perm(\alpha) \land \perm(\beta)]_{\cong_{\sortf}} =_{\Algebra[F]}$
					\dotfill \Cref{dal:axioms}(D1)
				\item[]
					$[\perm(\alpha)]_{\cong_{\sortf}} \land [\perm(\beta)]_{\cong_{\sortf}} =_{\Algebra[F]}
					\check{\P}([\alpha]_{\cong_{\sorta}}) \land \check{\P}([\beta]_{\cong_{\sorta}})$.
			\end{description}

		\item%
		{$\check{\P}([\alpha]_{\cong_{\sorta}}) \land \check{\F}([\alpha]_{\cong_{\sorta}}) =_{\Algebra[F]}
			[\alpha]_{\cong_{\sorta}} \mathrel{\check{=}} \iact$}
		% See below.

			\medskip
			\begin{description}[leftmargin=\parindent]
				\item[]
					$
					\check{\P}([\alpha]_{\cong_{\sorta}})
					\land
					\check{\F}([\alpha]_{\cong_{\sorta}})
					=_{\Algebra[F]}
					[\perm(\alpha)]_{\cong_{\sortf}}
					\land
					[\forb(\alpha)]_{\cong_{\sortf}}
					=_{\Algebra[F]}$
				\item[]
					$
					[\perm(\alpha) \land \forb(\alpha)]_{\cong_{\sortf}}
					=_{\Algebra[F]}
					$
				\item[]
					$[\alpha = \iact]_{\cong_{\sortf}} =_{\Algebra[F]}$
					\dotfill \Cref{dal:axioms}(D3)
				\item[]
					$[\alpha]_{\cong_{\sorta}} \mathrel{\check{=}} \iact$.
			\end{description}
		
		\item%
		{$[\alpha]_{\cong_{\sorta}} =_{\Algebra[A]} [\beta]_{\cong_{\sorta}}$ iff
			$[\alpha]_{\cong_{\sorta}} \mathrel{\check{=}} [\beta]_{\cong_{\sorta}} =_{\Algebra[F]} \top$}.
		% See below.

			\medskip
			\begin{description}[leftmargin=\parindent]
				\setlength{\itemsep}{2pt}
				\item[Left-to-right:]
				Let $[\alpha]_{\cong_{\sorta}} =_{\Algebra[A]} [\beta]_{\cong_{\sorta}}$.
				This assumption implies, by definition, that ${\alpha = \beta}$ is a theorem.
				Immediately, ${(\alpha = \beta) \liff \top}$ is also a theorem.
				But this means, $[\alpha = \beta]_{\cong_{\sortf}} =_{\Algebra[F]} \top$.
				Thus, $[\alpha]_{\cong_{\sorta}} \mathrel{\check{=}} [\beta]_{\cong_{\sorta}} =_{\Algebra[F]} \top$.

				\item[Right-to-left:]
				Similarly, let $[\alpha]_{\cong_{\sorta}} \mathrel{\check{=}} [\beta]_{\cong_{\sorta}} =_{\Algebra[F]} \top$.
				Then, $[\alpha = \beta]_{\cong_{\sortf}} = \top$.
				This means $\alpha = \beta$ is a theorem.
				And so, $[\alpha]_{\cong_{\sorta}} =_{\Algebra[A]} [\beta]_{\cong_{\sorta}}$. \qedhere
			\end{description}
	\end{enumerate}	
\end{proof}

We call the quotient algebra $\LTAlgebra$ in \Cref{prop:lindenbaum} the Lindenbaum-Tarski deontic action algebra.
In brief, $\LTAlgebra$ is a canonical algebra that captures theoremhood in \DAL.
From this observation, we obtain the following result.

\medskip
\begin{theorem}[Completeness]\label[theorem]{theorem:completeness}
	If $\DALVariety \vDash {\varphi \doteq \top}$, then, $\varphi$ is a theorem.
\end{theorem}
\begin{proof}
	We show that if $\varphi$ is not a theorem, then $\DALVariety \nvDash {\varphi \doteq \top}$.
	Let $\varphi$ be a non-theorem.
	From the definition of $\cong$, $[\varphi]_{\cong_{\sortf}} \neq_{\Algebra[F]} \top$.
	Define a function $h: {\bact \to {\Algebra[A]/{\cong}}}$ that sends each $\mathsf{a}_i \in \bact$ to the equivalence class $[\mathsf{a}_i]_{\cong_{\sorta}}$.
	The function $h$ extends uniquely to an interpretation $\check{h}: \TAlgebra \to \LTAlgebra$ such that $\check{h}(\varphi) \neq_{\Algebra[F]} [\varphi]_{\cong_{\sortf}}$.
	Therefore, $\DALVariety \nvDash {\varphi \doteq \top}$.
	%\qed
\end{proof}

% \Cref{cor:completeness} is obtained via a standard argument in algebraic logic.

\begin{corollary}\label[corollary]{cor:completeness}
	$\DALVariety \vDash {\varphi \doteq \top}$ implies $\varphi$ is a tautology.
\end{corollary}
\begin{proof}
	Immediate from \Cref{th:segerber:completeness,theorem:completeness}.
\end{proof}

% As usual then, the Lindenbaum-Tarski algebra can be seen as a canonical (algebraic) model providing counterexamples to non-valid formulas.

\subsection{Deontic Action Algebras and Deontic Action Models}

Interestingly, the algebraization of \DAL enjoys a Stone-type representation result connecting the algebraic semantics using deontic action algebras with the original semantics using deontic action models.
This connection provides us with another completeness result for the theorems of $\DAL$.

Recall that Stone's representation theorem establishes that every Boolean algebra is isomorphic to a field of sets~\cite{Stone36}. Such a result reveals a tight connection between the properties of an abstract structure with those of a \emph{concrete} one (a collection of sets).
This is also true for deontic action algebras.
We begin by introducing the definition of a concrete deontic action algebra.

%Just as Boolean algebras made of sets (i.e., fields of sets) are referred to as \emph{concrete} in algebraic logic, concrete deontic action algebras are deontic action algebras whose algebras of actions and formulas are fields of sets.
%This is made precise immediately below.

\medskip
\begin{definition}
	A deontic action algebra $\DAlgebra = \langle \Algebra[A], \Algebra[F],  \E, \P, \F \rangle$ is \emph{concrete} iff $\Algebra[A]$ and $\Algebra[F]$ are fields of sets.
  	Let $\DALVariety(0)$ be the class of concrete deontic algebras, for equations of the appropriate sort, we use $\DALVariety(0) \vDash \tau_1 \doteq \tau_2$ as the analogue of $\DALVariety \vDash \tau_1 \doteq \tau_2$ in \Cref{definition:deontic:algebra}.
\end{definition}
\medskip

We prove that validity in deontic action algebras reduces to validity in concrete deontic algebras.
In this way, concrete deontic algebras enable us to connect the algebraic semantics of \DAL with Segerberg's original semantics via Stone's duality.


\medskip
\begin{theorem}\label[theorem]{theorem:reducibility}
	It follows that $\DALVariety(0) \vDash \varphi \doteq \top$ iff $\DALVariety \vDash \varphi \doteq \top$.
\end{theorem}
\begin{proof}
	The left-to-right direction is straightforward.
	The proof for the right-to-left direction is by contradiction.
	Assume that $\DALVariety(0) \vDash \varphi \doteq \top$ and $\DALVariety \nvDash \varphi \doteq \top$.
	This means that we have a deontic action algebra $\DAlgebra = \langle \Algebra[A], \Algebra[F],  \E, \P, \F \rangle$ and an interpretation $h: \TAlgebra \to \DAlgebra$ s.t.\ $h_{\sortf}(\varphi) \neq_{\Algebra[F]} \top$.
	Via the Stone duality result for Boolean algebras, we can construct a concrete deontic action algebra $\DAlgebra' = \langle \Algebra[A]', \Algebra[F]',  \E', \P', \F' \rangle$ that is isomorphic to $\DAlgebra$.
	Moreover, we can define an interpretation $h': \TAlgebra \to \DAlgebra'$ s.t.\ $h'(a_i) = \varphi_{\Algebra[A]'}(h(a_i))$ (with  $\varphi_{\Algebra[A]'}$ being the Stone isomorphism for $\Algebra[A]'$).
	This construction ensures $h'(\varphi) \neq_{\Algebra[F]'} \top$; contradicting the original assumption that $\DALVariety(0) \vDash \varphi \doteq \top$.
\end{proof}

We can now link deontic action models with concrete deontic action algebras.


\medskip
\begin{definition}\label[definition]{def:mod2alg}
	Let $\DeonticModel = \langle E, P, F \rangle$ be a deontic action model, $v:\bact \to 2^{E}$ be a valuation on $\DeonticModel$, and $A = \set{v(\mathsf{a}_i)}{\mathsf{a}_i \in \bact}$.
	Define a concrete deontic action algebra $\algebra(\DeonticModel, v) = \langle \Algebra[A], \Algebra[2], \E, \P, \F \rangle$ where:
	\begin{align*}
			\Algebra[A] &= \tup{2^A, {\cup}, {\cap}, {}^{\complement}, \emptyset, A} &
			(a = b) =_{\Algebra[2]} \top &~\text{iff}~ a =_{\Algebra[A]} b &
			\P(a) = \top &~\text{iff}~ a \subseteq P \\
			&&&& \F(a) = \top &~\text{iff}~ a \subseteq F.
		\end{align*}
	Define also the interpretation $h: \TAlgebra \to \algebra(\DeonticModel, v)$ as the unique extension of $v$.
\end{definition}
\medskip

Similarly, concrete deontic algebras give rise to deontic action models.


\medskip
\begin{definition}\label{def:alg2mod}
	Let $\DAlgebra = \langle \Algebra[A], \Algebra[F],  \E, \P, \F \rangle$ be a concrete deontic algebra, $h: \TAlgebra \to \DAlgebra$ be an interpretation.
	Define a deontic action model $\model(\DAlgebra, h) = \langle E, P, F \rangle$ where:
	\begin{align*}
		E &= |\Algebra[A]| &
		P &= \bigcup \set{a}{\P(a) =_{\Algebra[F]} \top} &
		F &= \bigcup \set{a}{\F(a) =_{\Algebra[F]} \top}.
	\end{align*}
	Define also a valuation $v$ on $\model(\DAlgebra, f)$ as the restriction of $h$ to $\bact$.
\end{definition}
\medskip

If seen as operators, $\model$ and $\algebra$ are inverses of each other.


\medskip
\begin{theorem}\label[theorem]{theorem:inverses}
	It follows that:
		$\algebra(\model(\DAlgebra,v),h)\!=\!\DAlgebra$; and
		$\model(\algebra(\DeonticModel,v),h)\!=\!\DeonticModel$.
\end{theorem}
\medskip

In light of \Cref{theorem:inverses}, we obtain the following result.


\medskip
\begin{corollary}\label[corollary]{theorem:models-and-algebras}
		It follows that:
		$\DeonticModel, v \Vdash \varphi$ iff $\algebra(\DeonticModel, v), h \vDash \varphi \doteq \top$; and
		$\DAlgebra, h \vDash \varphi \doteq \top$ iff $\model(\DAlgebra,h), v \vDash \varphi$.
\end{corollary}
\medskip

% \begin{theorem}\label[theorem]{theorem:algebras-to-models}
% $\DAlgebra, f \vDashcurly \varphi \doteq \top \mbox{ iff } \model(\DAlgebra,f), {f{\restriction}_{\bact}} \vDash \varphi$.
% \end{theorem}


\medskip

The results in \Cref{theorem:inverses,theorem:models-and-algebras} enable us to prove the completeness of $\DAL$ w.r.t.\ Segerberg's original deontic models entirely in an algebraic way.

\medskip
\begin{theorem} It follows that $\varphi$ is a theorem iff it is a tautology.
\end{theorem}
\begin{proof}
	Suppose that $\varphi$ is a theorem.
	From \Cref{cor:completeness}, $\DALVariety \vDash \varphi \doteq \top$.
	From \Cref{theorem:reducibility}, $\DALVariety(0) \vDash \varphi \doteq \top$.
	From \Cref{theorem:models-and-algebras}, $\varphi$ is a tautology.
	Thus, $\varphi$ is a theorem implies $\varphi$ is a tautology.
	Using these results in the inverse order we obtain $\varphi$ is a tautology implies $\varphi$ is a theorem.
\end{proof}
