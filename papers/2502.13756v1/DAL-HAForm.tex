\subsection{Heyting Algebras for Formulas}\label{sec:heyting:formulas}

Let us now turn to leveraging the modular framework of deontic action algebras in the construction of new deontic action logics.
In \Cref{sec:algebraic-char}, we brought attention to this modularity presenting a deontic action algebra as a structure $\DAlgebra = \tup{\Algebra[A], \Algebra[F], \E, \P, \F}$, with $\Algebra[A]$ and $\Algebra[F]$ interpreting actions and formulas, and $\E$, $\P$, and $\F$ formalizing equality, permission, and prohibition of actions.
While we have primarily considered $\Algebra[A]$ and $\Algebra[F]$ as Boolean algebras, our framework allows also for alternative algebras for actions and formulas.
Notably, defining $\Algebra[F]$ as a Heyting algebra leads to a new deontic action logic worth considering.
We call this new logic $\DAL(\IPL)$.
We begin with an outline of the technical foundations of $\DAL(\IPL)$, and follow with a discussion of its key features and advantages.

\paragraph{Constructive Reasoning in Deontic Action Algebras}

The language of $\DAL(\IPL)$ contains the actions and formulas of $\DAL(\prop)$. Namely, actions are built using basic action symbols in $\bact$, and the connectives $\sqcup$, $\sqcap$, $\bar{~}$, $\iact$, and $\uact$.
In turn, formulas are built using proposition symbols in $\prop$, the deontic connectives on actions, i.e., $\perm(\alpha)$ and $\forb(\alpha)$, and the connectives $\lor$, $\land$, $\lnot$, $\bot$, and $\bot$.
The sole difference is that $\DAL(\IPL)$ introduces the connective $\to$ as primitive rather than as an abbreviation \textemdash with $\varphi \liff \psi$ remaining as an abbreviation for $(\varphi \to \psi) \land (\psi \to \varphi)$.
The axiomatization of $\DAL(\IPL)$ uses the axioms in \Cref{dal:axioms} for actions, equality, and the deontic operations, while the axioms for the propositional connectives (A1'--A13' and LEM') are replaced by those in \Cref{axioms:ipl}.
These last axioms are standard for Intuitionistic Propositional Logic~\cite{Troelstra:1988}.
Provability and theoremhood are straightforwardly adapted to accommodate for the new axioms.
% Provability and theoremhood in $\DAL(\IPL)$ are defined straightforwardly in a Hilbert-style fashion using axioms and the rule of modus ponens.

\begin{figure}
	\centering
	\fbox{
	\begin{minipage}{1.0\textwidth}
		\setlength{\linewidth}{.97\textwidth}
		\setlength{\columnsep}{-1.6cm}
		\begin{multicols}{2}
			\begin{enumerate}[label=H\arabic*.]
				\item $\varphi \to (\varphi \lor \psi)$
				\item $\varphi \to (\psi \lor \varphi)$
				\item $\varphi \land \psi \to \varphi$
				\item $\varphi \land \psi \to \psi$
				\item $(\varphi \to \bot) \to \lnot \varphi$
				\item $\lnot \varphi \to (\varphi \to \bot)$
				\item $\bot \to \varphi$
				\item $\varphi \to \top$
				\item $\varphi \to ( \psi \to \varphi )$
				\item $\varphi \to (\psi \to (\varphi \land \psi))$
				\item ${(\varphi \to \chi) \to ((\psi \to \chi) \to ((\varphi \lor \psi) \to \chi))}$
				\item ${(\varphi \to (\psi \to \chi)) \to ((\varphi \to \psi) \to (\varphi \to \chi))}$
			\end{enumerate}
		\end{multicols}
	\end{minipage}}\\[1em]
	% \medskip
	\caption{Axiom System of $\DAL(\IPL)$}\label{axioms:ipl}
\end{figure}

The algebraization of $\DAL(\IPL)$ replaces the Boolean algebra of formulas in the definition of a deontic action algebra for a Heyting algebra. The precise definition of the new type of deontic action algebra being used is given below.

\medskip
\begin{definition}\label[definition]{def:dalgebra:heyting:boolean}
	A BH-deontic-action algebra is an algebra
		$\DAlgebra =
			\langle
				\Algebra[A], \Algebra[H], \E, \P, \F
			\rangle$
		where:
			$\Algebra[A]$ is a Boolean algebra,
			$\Algebra[H]$ is a Heyting algebra, and
			$\E : {|\Algebra[A]| \times |\Algebra[A]| \to |\Algebra[H]|}$,
			$\P : {|\Algebra[A]| \to |\Algebra[H]|}$,
			and
			$\F : {|\Algebra[A]| \to |\Algebra[H]|}$ satisfy the conditions 1--6 in \Cref{definition:deontic:algebra}.
\end{definition}
\medskip

In brief, Heyting algebras play a role in constructive reasoning analogous to the role Boolean algebras play in classical reasoning. A key distinction is how Heyting algebras treat $\to$.
Despite this difference, Heyting algebras are closely related to Boolean algebras.
Specifically, every Boolean algebra is a Heyting algebra, and the regular elements of a Heyting algebra \textemdash those $x 
\in |\Algebra[H]|$ for which $x =_{\Algebra[F]} \lnot\lnot x$ \textemdash form a Boolean algebra.
It is well-known also that Heyting algebras have a representation theorem \textemdash as the category of Heyting algebras is dually equivalent to the category of Eusaki spaces.
Furthermore, the Lindenbaum algebra obtained from the axioms in \Cref{axioms:ipl} is itself a Heyting algebra \cite{vanDalen:2008}.
These facts collectively support the idea of replacing Boolean algebras with Heyting algebras in the algebraic treatment of deontic action logic, ensuring that such an approach is well-founded.

The proposition below exposes an interesting feature of BH-deontic-action algebras.

\medskip
\begin{proposition}\label[proposition]{prop:ideals-int}
	Let $\DAlgebra = \tup{\Algebra[A], \Algebra[H], \E, \P, \F}$ be a BH-deontic-action algebra.
	The pre-images $P$ and $F$ of $\top$ under $\P$ and $\F$, respectively, are ideals in $\Algebra[A]$ s.t.\ ${{P \cap F} = \{\iact\}}$.
\end{proposition}
\begin{proof}
	Note that ideals in Heyting algebras and ideals in Boolean algebras are defined identically.
	Note also that the proof of the analogous result for deontic action algebras in \Cref{prop:dal:ideal} uses only reasoning on ideals and the properties of $\E$, $\P$, and $\F$.
	Since these properties are maintained in BH-deontic-action algebras, the proof in \Cref{prop:dal:ideal} transfers directly to this new setting.
\end{proof}
\medskip

In line with \Cref{prop:dal:ideal}, the result in \Cref{prop:ideals-int} tells us that permission and prohibition on actions yielding ideals carry over if we replace Boolean for Heyting algebras.

By way of conclusion, we present soundness and completeness theorems for $\DAL(\IPL)$.
Definitions of interpretations of the term algebra into BH-deontic-action algebras, homomorphisms, and congruences and quotients, are akin to those in \Cref{sec:algebraic-char}.
% In our proof, we can follow the steps and the main ideas of  \Cref{sec:dal:propositions} to prove the soundness and completeness of the intuitionistic version of the logic.
% For doing so,  first, we note that the axiomatic system for the logic have to be adapted to the intuitionistic setting.  This is done by considering the axiomatic system of  \Cref{section:dal} without the axiom (PEM), we denote the deduction relation obtained by $\vdash_{\DAL(\IPL)}$, similarly we denote by $\vDashcurly_{\IPL} {\varphi \doteq \top}$ the algebraic validity in the algebras of \Cref{def:dalgebra:heyting:boolean}.
%Similarly to the case in \Cref{sec:dal:propositions}, we obtain the following result.

\medskip
\begin{theorem}\label{prop:completeness:heyting:formulas}
	 Let $\mathbb{BH}$ be the class of all BH-deontic-action algebras.
	 It follows that $\varphi$ is a theorem of $\DAL(\IPL)$ iff $\mathbb{BH} \vDash {\varphi \doteq \top}$.
\end{theorem}
\begin{proof}
	Like with the proofs of \Cref{theorem:completeness:dal:i,prop:completeness:dal:prop}, we only remark on the differences with the proofs of \Cref{theorem:soundness,theorem:completeness}.
	\begin{description}
		\item[Soundness.]
		We need to prove that any interpretation of an axiom is mapped to $\top$.
		For axioms on actions, this is just like in \Cref{theorem:soundness}.
		For axioms on propositional connectives, this is  immediate from well known results for
		Heyting algebras (see~\cite{vanDalen:2008,Troelstra:1988}).
		Finally, the cases of equality, permission, and prohibition are not affected by the new interpretation of $\to$ in a Heyting algebra.

		\item[Completeness.]
		We begin by defining the Lindenbaum algebra as in \Cref{prop:lindenbaum} via congruences $\cong_{\sorta}$ for actions and $\cong_{\sortf}$ for formulas.
		Again, following~\cite{vanDalen:2008}, it is easy to see that the axioms in \Cref{axioms:ipl} result in the algebra of formulas itself being a Heyting algebra.
		This construction provides a witness for theoremhood for the logic.
		\qedhere
	\end{description}
\end{proof}

\paragraph{Constructive Reasoning Matters}

Just like we did when we introduced propositions, let us discuss why interpreting formulas on Heyting algebras instead of Boolean algebras bears an interest beyond its formal properties.

To set the stage for discussion, imagine the following scenario: \emph{if John does not have a driver's license, then, it is forbidden for him to drive; John is not forbidden to drive}.
From this scenario, using classical reasoning, we derive \emph{John has a driver's license}.
This conclusion is somewhat counterintuitive.
It is easy to consider many cases in which \emph{John does not have a driver's license} is true, which are consistent with the scenario in question.
But this is logically impossible.
There are many ways of dealing with this kind of problem, one of which is to move from Classical to Intuitionistic reasoning. %\cite{DBLP:conf/wollic/DalmonteGO22}, that is, to reject the principle $\varphi \vee \neg \varphi$ (or equivalently $\neg ( \varphi \wedge \neg \varphi)$) as part of the logic.  This has many consequences in the resulting  formal system. 

\medskip
\begin{example}
	The driver's license example in the previous paragraph can be formalized with formulas:
	$\lnot \mathsf{haslicense} \to \forb(\mathsf{driving})$
		capturing that \emph{if John does not have a driver's license, then, it is forbidden for him to drive}, and
	$\lnot\forb(\mathsf{driving})$
		capturing that \emph{John is not forbidden to drive}.
	Note that, in this formalization, $\mathsf{haslicense} \in \prop$ and $\mathsf{driving} \in \bact$.
	The BH-deontic-action algebra $\DAlgebra$ in \Cref{ex:driver}, together with the interpretation $h$ defined as $h_{\sorta}(\mathsf{driving}) = a$, $h_{\sortf}(\mathsf{haslicense}) = \frac{1}{2}$, prove that $\mathsf{haslicense}$ is not a consequence of the previous two formulas.
	Precisely, we have:

	\medskip
	\centerline{
	\begin{minipage}{0.7\textwidth}
		\begin{enumerate}[leftmargin=\parindent]
			\setlength{\itemsep}{5pt}
			\item $
				h(\lnot\mathsf{haslicense}) =_{\Algebra[H]}
				\lnot h(\mathsf{haslicense}) =_{\Algebra[H]}
				\lnot \frac{1}{2} =_{\Algebra[H]}
				\bot$.
			\item $
				h(\lnot\mathsf{haslicense} \to \forb(\mathsf{driving})) =_{\Algebra[H]}
				% h(\lnot\mathsf{haslicense}) \to h(\forb(\mathsf{driving})) =_{\Algebra[H]}
				\bot \to h(\forb(\mathsf{driving})) =_{\Algebra[H]}
				\top$.
			\item $
				h(\lnot\forb(\mathsf{driving})) =_{\Algebra[H]}
				\lnot h(\forb(\mathsf{driving})) =_{\Algebra[H]}
				\lnot \bot =_{\Algebra[H]}
				\top$.
			\item $h(\mathsf{haslicense}) \neq_{\Algebra[H]} \top$.
		\end{enumerate}
	\end{minipage}}
	\medskip

	\noindent Note how in the BH-deontic-algebra $\DAlgebra$ in \Cref{ex:driver} the only element of the algebra of actions which the operation $\F$ maps to $\top$ is $\iact$, all other elements are mapped to $\bot$.
\end{example}
\medskip

% \begin{figure}
% 	\centering
% 	\includegraphics[width=0.5\textwidth]{deontic-algebra-driving.pdf}\\[1em]
% 	\caption{The Driver's License Paradox}\label{ex:driver}
% \end{figure}

\begin{figure}
	\centering
	\begin{minipage}{0.48\textwidth}
		\centering
		\includegraphics[trim=20pt 0pt 20pt 0pt, clip, width=\textwidth]{deontic-algebra-driving.pdf}\\[1em]
		\caption{The Driver's License Paradox.}\label{ex:driver}
	\end{minipage}\hfill
	\begin{minipage}{0.48\textwidth}
		\centering
		\includegraphics[trim=20pt 0pt 20pt 0pt, clip, width=\textwidth]{deontic-algebra-closure.pdf}\\[1em] % second figure itself
		\caption{Principle of Deontic Closure.}\label{ex:deontic:closure}
	\end{minipage}
\end{figure}

There is another interesting discussion emerging from the use of an Intuitionistic basis for reasoning about formulas.
Recall the deontic action logic $\DAL(1)$ from \Cref{section:dals} and how this logic is built from $\DAL$ adding additional axioms with the intent to capture the \emph{principle of deontic closure}.
This principle is stated in \cite{Segerberg1982} as: \emph{what is not forbidden is permitted}.
The formalization of this principle as formulas of the form $\forb(\mathsf{a}) \lor \perm(\mathsf{a})$ is taken from \cite{Trypuz15}.
Still, a more faithful formalization of this principle is $\lnot\forb(\mathsf{a}) \to \perm(\mathsf{a})$.
Clearly, there is no substantial distinction in a Classical setting, as both formulas are equivalent.
This is not the case in an Intuitionistic setting.
For instance, the BH-deontic-action algebra $\DAlgebra$ in \Cref{ex:deontic:closure} satisfies one version of the principle but not the other.
Precisely, note that if we have a single basic action symbol $\mathsf{a} \in \bact$, and an interpretation $h$ on $\DAlgebra$ s.t.\ $h_{\sorta}(\mathsf{a}) = a$, then:

	\medskip
	\centerline{
	\begin{minipage}{0.5\textwidth}
		\begin{enumerate}[leftmargin=\parindent]
			\setlength{\itemsep}{5pt}
			\item $
				h(\forb(\mathsf{a})) =_{\Algebra[H]}
				\F(h(\mathsf{a})) =_{\Algebra[H]}
				\F(a) =_{\Algebra[H]}
				\frac{1}{2}$.
			\item $
				h(\lnot\forb(\mathsf{a})) =_{\Algebra[H]}
				\lnot h(\forb(\mathsf{a})) =_{\Algebra[H]}
				\lnot \frac{1}{2} =_{\Algebra[H]}
				\bot$.
			\item $
				h(\perm(\mathsf{a})) =_{\Algebra[H]}
				\P(h(\mathsf{a})) =_{\Algebra[H]}
				\P(a) =_{\Algebra[H]}
				\frac{1}{2}$.
			\item $
				h(\lnot\forb(\mathsf{a}) \to \perm(\mathsf{a})) =_{\Algebra[H]}
				\bot \to h(\perm(\mathsf{a})) =_{\Algebra[H]}
				\top$.
			\item $
				h(\forb(\mathsf{a}) \lor \perm(\mathsf{a})) =_{\Algebra[H]}
				\frac{1}{2} \lor \frac{1}{2} =_{\Algebra[H]}
				\frac{1}{2} \neq_{\Algebra[H]}
				\top$.
		\end{enumerate}
	\end{minipage}}
	\medskip

\noindent In words, this example shows that there is a distinction between considering the principle of deontic closure as \emph{what is not forbidden is permitted} \textemdash alternatively, \emph{what is not permitted is forbidden}\textemdash and considering this principle as \emph{every (basic) action is either permitted or forbidden}.

To sum up, we have explored some key features and applications of replacing the Boolean algebra of formulas in a deontic action algebra for a Heyting algebra.
The results we obtained underscore leveraging the modularity of our framework to build a new deontic action logic $\DAL(\IPL)$. The discussion and ensuing examples reinforce the utility of this new logic in the broader area of Deontic Logic, and in particular in relation to the principle of deontic closure.
