\subsection{Basic Definitions (and a Roadmap for our Results)}\label{section:basics}

We provide a brief overview of some fundamental concepts in the algebraization of logic: signatures, algebras, characterizations of classes of algebras, congruences, and quotient algebras. In the case of \DAL, these definitions are interwoven with sorts to distinguish between actions and propositions.
This led us to work with many-sorted algebras --algebraic structures with carrier sets and operations categorized into \emph{sorts}~\cite{Halmos:2009,Tarlecki:2012}. In what follows, we establish the basic terminology for many-sorted algebras in the context of the algebraization of a logic. We have two main purposes behind this: first, to introduce the notation and terminology we use and ensure our results are self-contained; second, to outline the key steps in our algebraization of \DAL.

The algebraization of a logic begins with the appropriate definition of a signature and an algebraic structure. In our case, as mentioned, these two concepts are categorized into sorts.

\medskip
\begin{definition}
	A (many-sorted) signature is a pair $\Sigma = \langle S, \Omega \rangle$ where:
		$S$ is a non-empty set of sort symbols;
		and
		$\Omega$ is an $S^{+}$-indexed family of pairwise-disjoint sets of operation symbols. %indexed by finite non-empty sequences $s_1 \dots s_n s$ of elements in $S$.
	In turn, an algebra of type $\Sigma$, or a $\Sigma$-algebra, is a structure $\Algebra[A] = \tup{|\Algebra[A]|, \Funcs}$ where:
		$|\Algebra[A]|$ is an $S$-indexed family of non-empty universe sets $|\Algebra[A]|_s$; and
		$\Funcs$ is a collection of functions ${f_{\Algebra[A]}: (\prod_{i = 1}^{n} |\Algebra[A]|_{s_i}) \to |\Algebra[A]|_{s}}$, one for each $f \in \Omega_{s_1 \dots s_n s}$.
\end{definition}
\medskip

For the rest of this section, by an algebra, we mean an algebra of type $\Sigma = \tup{S, \Omega}$.

Signatures give rise to specific algebras whose universe sets consist of strings of symbols from the signature, and whose functions operate as concatenation of these strings. These algebras, known as \emph{term algebras}, serve as the algebraic counterpart to the language of a logic.
% Moreover, they act as the bridge between the syntax and the algebraic semantics of the logic.
We introduce the precise definition of term algebra in \Cref{def:talg}

\medskip
\begin{definition}\label{def:talg}
	Let $V$ be an $S$-indexed family of pair-wise disjoint countable sets of symbols for variables.
	The term algebra $\TAlgebra$ (on $V$) is defined s.t.:
	
	\medskip
	\begin{enumerate}%[(a)]
		\item for all $s \in S$, $|\TAlgebra|_s$ is the smallest set containing:
			$V_s$, and
			all strings $\textrm{`}f(\tau_1 \dots \tau_n)\textrm{'}$ where $f \in \Omega_{s_1 \dots s_n s}$, and $\tau_i \in |\TAlgebra|_{s_i}$;
		\item for all $f \in \Omega_{s_1 \dots s_n s}$,
			$f_{\mathbf{T}}(\tau_1 \dots \tau_n) = \textrm{`}f(\tau_1 \dots \tau_n)\textrm{'}$.
	\end{enumerate}
\end{definition}
\medskip

The algebraic counterpart of the semantics of a logical language is given via homomorphisms of term algebras.

\medskip
\begin{definition}
	Let $\Algebra[A]$ and $\Algebra[B]$ be algebras, and $h = {\set{h_s : {|\Algebra[A]|_s \rightarrow |\Algebra[B]|_s}}{s \in S}}$ be an $S$-indexed family of functions.
	We say that $h$ is a homomor\-phism from $\Algebra[A]$ to $\Algebra[B]$, and write ${h: {\Algebra[A] \to \mathbf{B}}}$, iff for all $f \in \Omega_{s_1 \dots s_n s}$, $h_s(f_{\Algebra[A]}(a_1 \dots a_n)) = f_{\mathbf{B}}(h_{s_1}(a_1) \dots h_{s_n}(a_n))$.
	An interpretation of a term algebra $\TAlgebra$ with variables in $V$ on an algebra $\Algebra[A]$ is a homomorphism $h: \TAlgebra \to \Algebra[A]$.
\end{definition}
\medskip

To establish soundness and completeness results in an algebraic way we will need to connect term algebras to particular classes of algebras of interest.
Standard classes of algebraic structures are characterized by equations. However, our algebraization of \DAL uses the weaker notion of a quasi-equation --a conditional equation-- to capture equality on actions as an algebraic operation. We introduce these concepts next.

\medskip 
\begin{definition}
	Let $\TAlgebra$ be a term algebra with variables in $V$.
	An equation is a string ${\tau_1 \doteq_s \tau_2}$ where $\tau_i \in |\TAlgebra|_s$. 
	In turn, a quasi-equation is a string ${{\tau_1 \doteq_s
	\tau_2} \To {\tau'_1 \doteq_{s'} \tau'_2}}$ where $\tau_i \in |\TAlgebra|_s$ and $\tau'_i \in |\TAlgebra|_{s'}$.
	By ${{\tau_1 \doteq_s
	\tau_2} \Iff {\tau'_1 \doteq_{s'} \tau'_2}}$, we mean the pair of quasi-equations
		${\tau_1 \doteq_s \tau_2} \To {\tau'_1 \doteq_{s'} \tau'_2}$ and
		${\tau'_1 \doteq_{s'} \tau'_2} \To {\tau_1 \doteq_{s} \tau_2}$.
\end{definition}
\medskip 

Notice that equations and quasi-equations are not elements of a term algebra.
Moreover, note that we have used $\doteq$ instead of $=$ in the definition of equations and quasi-equations since $=$, as a symbol, is part of the language of \DAL.
We define below when equations and quasi-equation are satisfied in an algebra.

\medskip
\begin{definition}
	An equation $\tau_1 \doteq_s \tau_2$ is satisfied in an algebra $\Algebra[A]$ under an interpretation $h$ iff ${h_s(\tau_1) = h_s(\tau_2)}$.
	% In turn, $\tau_1 \doteq_s \tau_2$ is valid in $\Algebra[A]$, written $\Algebra[A] \vDashcurly {\tau_1 \doteq \tau_2}$, iff for all interpretations $h$ on $\Algebra[A]$, ${\Algebra[A], h} \vDashcurly {\tau_1 \doteq_s \tau_2}$.
	% Lastly, $\tau_1 \doteq \tau_2$ is universally valid in a class $\mathbb{A}$ of $\tup{S,\Omega}$-algebras, written $\vDashcurly_{\mathbb{A}} {\tau_1 \doteq_s \tau_2}$, iff
	% ${\Algebra[A]} \vDashcurly {\tau_1 \doteq_s \tau_2}$ for all $\Algebra[A] \in \mathbb{A}$.
	In turn,
		a quasi-equation
			${{\tau_1 \doteq_s	\tau_2} \To {\tau'_1 \doteq_{s'} \tau'_2}}$
		is satisfied in
			$\Algebra[A]$ under $h$
		iff
			${h_s(\tau_1) = h_s(\tau_2)}$ implies ${h_s(\tau'_1) = h_s(\tau'_2)}$.
	Moreover,
		a quasi-equation
			${{\tau_1 \doteq_s	\tau_2} \To {\tau'_1 \doteq_{s'} \tau'_2}}$
		is valid in an algebra
			$\Algebra[A]$
		iff
			for any interpretation $h$ on $\Algebra[A]$,
				${{\tau_1 \doteq_s	\tau_2} \To {\tau'_1 \doteq_{s'} \tau'_2}}$ is satisfied in $\Algebra[A]$ under $h$.
	% A class of $\Sigma$-algebras satisfies a set of $\Sigma$-quasi-equations iff it satisfies every $\Sigma$-quasi-equation in the set.
\end{definition}

\medskip

Quasi-equations give rise to classes of algebras called \emph{quasi-varieties}.

\medskip
\begin{definition}
	A quasi-variety is the class of all algebras validating a set of quasi-equations; i.e., the class of all algebras where all quasi-equations in the set is valid.
\end{definition}
\medskip

The final fundamental tool in the algebraic characterization of a logic is that of a congruence relation. When appropriately defined on term algebras, a congruence relation provides a method for constructing --out of syntax-- well-behaved algebras as quotient algebras. More formally, canonical algebraic models are obtained by taking the quotient of the term algebra using specific congruences.

\medskip
\begin{definition}
	Let $\Algebra[A]$ be an algebra, and ${\cong} = \set{{{\cong_s} \subseteq {|\Algebra[A]|_s \times |\Algebra[A]|_s}}}{s \in S}$ be an $S$-indexed family of binary relations.
	We say that $\cong$ is a congruence on $\Algebra[A]$ iff every ${\cong_{s}} \in {\cong}$ is an equivalence relations on $|\Algebra[A]|_{s}$, and for every $f \in \Omega_{s_1 \dots s_n s}$, it follows that
		$a_i \cong_{s_i} a'_i$
		implies
		$f_{\Algebra[A]}(a_1 \dots a_n) \cong_s f_{\Algebra[A]}(a_1' \dots a_n')$.
	The quotient of $\Algebra[A]$ under $\cong$ is an algebra $\Algebra[A]/{\cong}$ where:
		$|\Algebra[A]/{\cong}|_s = |\Algebra[A]|_s/{\cong_s}$;
		and
		% for all $f \in \Omega_{s_1 \dots s_n s}$,
		% ${f_{\Algebra/{\cong}} : {(\prod_{i = 1}^{n} (A_{s_i}/{\cong_{s_i}})) \to {A/{\cong_s}}}}$
		$f_{(\Algebra[A]/{\cong})}([a_1]_{\cong_{s_1}} \dots [a_n]_{\cong_{s_n}}) = [f_{\Algebra[A]}(a_1 \dots a_n)]_{\cong_{s}}$.
\end{definition}
\medskip

We conclude this section with a presentation of two classes of algebras we use in our algebraization of \DAL: Boolean and Heyting algebras.
We follow \cite{Esakia:2019} and reach these particular algebras via \emph{bounded distributive lattices} (BDLs).
BDLs are characterized by a set of equations common to Boolean and Heyting algebras.
This gives us a way to introduce concepts pertaining Boolean and Heyting algebras simultaneously.
We introduce Heyting algebras as an extension of BDLs, and  Boolean algebras as an extension of Heyting algebras.
To keep the notation uncluttered, unless it is strictly necessary, we will omit the subscript $\Algebra[A]$ in the function $f_{\Algebra[A]}$ and simply write $f$. The context will always disambiguate whether we refer to the function or the symbol in the signature of $\Algebra[A]$.

\medskip
\begin{definition}[\cite{Esakia:2019}]\label[definition]{def:bdl:algebra}
	Let $\Lambda = \tup{ \{s\}, \{\{{+},{*}\}_{sss},\{0,1\}_{s}\}}$ be a many-sorted signature.
	A BDL-algebra is a $\Lambda$-algebra $\Algebra[L]$ satisfying the following equations:
	\begin{multicols}{2}
		\begin{enumerate}[label=L\arabic*.]
			\item $\tau_1 + (\tau_2 + \tau_3) \doteq (\tau_1 + \tau_2) + \tau_3$
			\item $\tau_1 + \tau_2 \doteq \tau_2 + \tau_1$
			\item $\tau_1 + \tau_1 \doteq \tau_1$
			\item $\tau_1 + (\tau_1 * \tau_2) \doteq \tau_1$
			\item ${\tau_1 + (\tau_2 * \tau_3)} \doteq {(\tau_1 + \tau_2) * (\tau_1 + \tau_3)}$
			\item $\tau_1 + 1 \doteq 1$
			\item $\tau_1 * (\tau_2 * \tau_3) \doteq (\tau_1 * \tau_2) * \tau_3$
			\item $\tau_1 * \tau_2 \doteq \tau_2 * \tau_1$
			\item $\tau_1 * \tau_1 \doteq \tau_1$
			\item $\tau_1 * (\tau_1 + \tau_2) \doteq \tau_1$
			\item ${\tau_1 * (\tau_2 + \tau_3)} \doteq {(\tau_1 * \tau_2) + (\tau_1 * \tau_3)}$
			\item $\tau_1 * 0 \doteq 0$.
		\end{enumerate}
	\end{multicols}
\end{definition}
% \medskip

In \Cref{def:bdl:algebra}, $\tau_i$ is an element of the term algebra of type $\Lambda$.
We use $\Algebra[L] = \tup{L, {+}, {*}, {0}, {1}}$ to indicate an arbitrary BDL-algebra.
We write $\preccurlyeq$ for the partial order implicit in a BDL algebra, i.e., $a \preccurlyeq b$ iff $a + b = b$.

\medskip
Another important concept in our algebraization of \DAL is that of an ideal (or, dually, a filter).
Intuitively, an ideal is an initial set closed by unions (while a filter is a final set closed by intersections).  
Ideals were used by Segerberg in the original definition of \DAL as inherent properties of the formalization of permission and prohibition on actions.

\medskip
\begin{definition}
	An ideal in a BDL-algebra $\Algebra[L] = \tup{L, {+}, {*}, {0}, {1}}$ is a subset $I \subseteq L$ s.t.: for all $x,y \in I$, ${x + y} \in I$,
	and for all $x \in I$ and $y \in L$, $(x*y) \in I$.
	Dual to ideals are filters.
	A filter is a subset $F \subseteq L$ s.t.: for all $x,y \in F$, $(x*y) \in F$, and 
	for all $x \in F$ and $y \in L$, ${x + y} \in F$.
	% An ideal (filter) is maximal if it is not contained in any other ideal (filter).
	% The smallest ideal (filter) containing an element $x \in L$, written ${\downarrow} x$ (${\uparrow} x$), is called a principal ideal (filter).\carlos{Usamos las nociones de principal?}
\end{definition}
\medskip

We use BDL-algebras to present Heyting and Boolean algebras. 

\medskip 
\begin{definition}[\cite{Esakia:2019}]\label[definition]{def:heyting:algebra}
	Let $\mathrm{H} = \tup{ \{s\}, \{\{{\hto}, {+},{*}\}_{sss},\{0,1\}_s\}}$ be a many-sorted signature.
	A Heyting algebra is an $\mathrm{H}$-algebra satisfying the equations L1--L12 in \Cref{def:bdl:algebra} together with the following equations:

\medskip
		\begin{enumerate}[label=H\arabic*.]
			\item $\tau_1 * (\tau_1 \hto \tau_2) \doteq \tau_2$
		
			\item $((\tau_1 * \tau_2) \hto \tau_1) * \tau_3 \doteq \tau_3$
			\item $\tau_1 * (\tau_2 \hto \tau_3) \doteq \tau_1 * ((\tau_1 * \tau_2) \hto (\tau_1 * \tau_3))$.
		\end{enumerate}	
\end{definition}
\medskip

In \Cref{def:heyting:algebra}, $\tau_i$ is an element of the term algebra of type $\mathrm{H}$.
We use $\Algebra[H] = \tup{H, {\hto}, {+}, {*}, {0}, {1}}$ to indicate an arbitrary Heyting algebra.
We use $\bar{\tau}$ as an abbreviation of $\tau \hto 0$.
This abbreviation gives rise to an operation $\bar{~} : {H \to H}$ defined as $\bar{x} = {x \hto 0}$.
We will sometimes use $\Algebra[H] = \tup{H, {+}, {*}, \bar{~}, {0}, {1}}$ to indicate an arbitrary Heyting algebra; in these cases, we assume $\hto$ implicitly.
% The class of Heyting algebras is a variety.

\medskip
\begin{definition}\label{definition:boolean:algebra}
	Boolean algebras are Heyting algebras that validate: (LEM) $\tau + \bar{\tau} \doteq 1$.
\end{definition}
\medskip

We use $\Algebra[B] = \tup{B, {+}, {*}, {\bar{~}}, {0}, {1}}$ to indicate an arbitrary Boolean algebra; and~$\mathbf{2}$ to indicate the Boolean algebra of exactly two elements.
A Boolean algebra is \emph{concrete} iff it is a field of sets.
% The class of Boolean algebras is a variety.
Important Boolean algebras in our setting are those freely generated and finitely generated~\cite{Halmos:2009}.
We use  Stone's representation theorem~\cite{Stone36}.
If $\Algebra[B]$ is a Boolean algebra, we use $\mathsf{s}(\Algebra[B])$ for its isomorphic Stone space, and ${\mathbf{\varphi}_{\Algebra[B]}: \Algebra[B] \to \mathsf{s}(\Algebra[B])}$ for the isomorphism.

%%% Local Variables:
%%% mode: latex
%%% TeX-master: "article"
%%% End:
