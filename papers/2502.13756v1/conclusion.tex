\section{Final Remarks}\label{section:conclusion}

We developed an algebraic framework for Deontic Action Logic (\DAL) and its variations using deontic action algebras.
These structures consist of two Boolean algebras connected by operations that capture key aspects of permission and prohibition.
We showed that the algebraic characterization is
adequate by proving soundness and completeness theorems.
We discussed the advantages of our algebraic approach for modelling scenarios through concrete examples.
Our algebraic treatment of \DAL can be thought of as an abstract version of deontic action logics which can
be used to establish connections between deontic action logics and areas such as topology, category theory, probability, etc.
Moreover, the framework is modular.
In \Cref{section:new:dals}, we showed how replacing the underlying algebraic structures allows for the development of new logics, highlighting the flexibility and extensibility of our approach.

We introduced deontic action algebras in~\cite{CCFA:2021}. In this article, we extended our previous work and explored the inclusion of Heyting algebras to obtain intuitionistic behavior.
Our approach accommodates such a formulation in a very simple manner, paving the way for interesting future work.
Several alternative algebraic structures for actions and propositions warrant further investigation.
In particular, we aim to characterize action composition (denoted by $;$), and action iteration (denoted by $*$).
These operations are not foreign in deontic reasoning.
The work in~\cite{Meyer:88} on Dynamic Deontic Logic ($\mathsf{DDL}$) was one of the first in considering a deontic logic containing action composition.
This treatment, however, is not without challenges~\cite{Anglberger:08}.
Regarding action iteration, in~\cite{BroersenThesis}, Broersen pointed out that dynamic deontic logics can be divided into: (i) \emph{goal norms}, where prescriptions over a sequence of actions only take into account the last action performed; or (ii) \emph{process norms}, where a sequence of actions is permitted/forbidden if and only if every action in the sequence is permitted/forbidden.
To the best of our knowledge, no extensions of \DAL incorporating action composition or iteration have been explored.
The framework of deontic action algebras is well-suited for exploring such extensions, as deontic action algebras can be straightforwardly modified to admit action composition and iteration.
More precisely, we may consider deontic action algebras $\langle \Algebra[A], \Algebra[F], \P, \F, \E \rangle$ where the algebra $\Algebra[A] = \tup{A, +, ;, *}$ of actions is a Kleene algebra (see~\cite{Kozen:90}).
Kleene algebras enjoy some nice properties, e.g.,
they are quasi-varieties, and they are complete w.r.t.\ equality of regular expressions (see~\cite{Kozen:91}).
Similarly, one can extend deontic action algebras with other interesting algebras, e.g., relation algebras (see~\cite{MadduxBook}) that most notably provide action converse.
We leave it as further work to investigate the properties of the
operators $\P$ and $\F$ in these new algebraic settings.

Beyond Boolean and Heyting algebras, it is also interesting to explore alternative algebras for propositions.
Some immediate examples include: BDL-algebras, semi-lattices, and metric spaces.
This may lead to the design of deontic logics that are not logics of normative propositions but logics of norms instead---a distinction was already noted by von~Wright in~\cite{vonWright:1999} and Alchourr\'on in~\cite{Alchourron:69,Alchourron:1971}.
Both \SDL and \DAL are logics of normative propositions insofar as they assign truth values to formulas in the logic.
In contrast, logics of norms can express prescriptions that do not carry with them truth values.
To accommodate such logics, we can generalize the interpretation of formulas to other algebraic structures.
For instance, adopting a meet semi-lattice as the algebra of propositions allows for norms to be combined while also accounting for potential contradictions among them without necessarily requiring norms to be true or false.
Of course, several other algebraic frameworks could serve this purpose as well.

The level of flexibility in \Cref{section:new:dals} suggests a possible connection between \DAL and combining logics~\cite{sep-logic-combining}.
Building on the algebraic treatment of \DAL, we can derive a characterization in category-theoretic terms~\cite{MacLane98}, which naturally connects to the framework of institutions introduced by Goguen and Burstall~\cite{GoguenBurstall84,GoguenBurstall92}.
Institutions provide an abstract framework for model theory that captures the essence of logical systems and their combinations, offering a unified perspective on how logics can be integrated through categorial constructions.
Many of these methods are unified within the algebraic fibring approach introduced by Sernadas, Sernadas, and Caleiro~\cite{Sernadas1999}, which significantly enhances the versatility of logic combination through universal categorial constructions.
This approach extends the range of logics that can be combined beyond modal logics, demonstrating a fruitful interplay between the algebraic and categorial perspectives.
Altogether, these frameworks establish a strong link between \DAL and broader methodologies for combining logics, reinforcing the relevance of algebraic and categorial approaches to logical systems.

Finally, the algebraization of DAL in this article provides a rich foundation for studying deontic action logics dynamically, \emph{\'a la} to Public Announcement Logic~\cite{Plaza2007}.
In particular, we note that the algebraic semantics of deontic operators induces a restriction on the algebra of formulas.
Such a restriction resembles the model update operators in dynamic logics, suggesting an interesting parallel and potential applications in modeling evolving normative systems. Exploring these connections, alongside the broader algebraic and categorial perspectives outlined above, offers a promising direction for future research.

\paragraph{Conflict of Interest}

The authors declare that there are no conflicts of interest regarding the publication of this paper.

\paragraph{Data Availability Statement}

No new data were created or analyzed in this paper.

%%% Local Variables:
%%% mode: latex
%%% TeX-master: "article"
%%% End:
