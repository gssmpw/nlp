\section{A Categorical View of DAL}\label{sec:cat}

One of the main benefits of the algebraic view on \DAL is that it paves the way to the use of abstract mathematical  frameworks such as category theory.  Category theory allows one to capture the properties of mathematical, or logical objects, in a very abstract way,  which makes possible to investigate the relations between different formalisms as well as the abstract properties of mathematical objects.  Another interesting feature of category theory is that it enables modular reasoning over logical, or algebraic systems. For instance,  one can use  standard categorical constructions such as limits and colimits  to put together different structures.  In this section we present the category of deontic actions algebras and investigate its basic properties.

For the sake of simplicity, we restrict ourselves to  category corresponding to the non-intuitionistic logic \DAL, but it must be clear that similar results hold for the other algebras, at the end of this section we make some remarks about this.  We introduce the basic notions of category theory needed for this section,  the interested reader is referred to \cite{MacLane98} for a deep introduction to category theory.

\subsection{Preliminaries on Category Theory}
A category is a structure $\mathbf{C} = (\mathcal{O}, \mathcal{A})$, where $\mathcal{O}$ is a collection of \emph{objects} (also denoted $|\mathbf{C}|$) and $\mathcal{A}$ is a collection of \emph{arrows} (also denoted $||\mathbf{C}||$), equipped with: (i) operations $\mathit{dom}$ and $\mathit{cod}$ assigning to any object $a \in |\mathbf{A}|$, an object $\mathit{dom}(a)$ called its domain, and an object $\mathit{cod}(a)$ called its codomain; (ii) the operation $\circ$ that given $f,g \in ||\mathbf{C}||$ such that $\mathit{cod}(f) = \mathit{dom}(g)$,  returns an arrow, denoted $g \circ f$ with $\mathit{dom}(g \circ f) = dom(f)$ and
$\mathit{cod}(g \circ f) = cof(g)$, (iii) for each object $a \in |\mathbf{C}|$ an arrow $id_a \in || \mathbf{C}||$. Furthermore, the following equations hold: $f \circ id_a = f$ and $id_b \circ f = f$; and  $\circ$ is associative.
A very well-known category is given by  the collection of all (small) sets and all the collection of functions, usually named $\mathbf{Set}$. Similarly,  any algebraic structure form a category consisting of the corresponding algebras as object, and the homomorphisms as its arrows.  It is straightforward to define the notion of \emph{subcategory, } a subcategory is \emph{full} is preserves all the arrow of the subcollection of objects in the subcategory.   An \emph{initial object} in a category $\mathbf{C}$ is an object $0 \in |\mathbf{C}|$ such that for any other object $x$ we have a unique arrow $u : 0 \rightarrow x$. For instance, in $\mathbf{Set}$ $\emptyset$ is an initial element. Final elements are the dual concept.  

Functors are mappings between categories, that is, given two categories $\mathbf{C}$ and $\mathbf{D}$ a functor $F$ between $\mathbf{C}$ and $\mathbf{D}$, written $F:\mathbf{C} \rightarrow \mathbf{D}$,  maps (i) every object $a \in |C|$ to an object, written $F(a)$, of $|\mathbf{D}|$, (ii) any arrow $f:a \rightarrow b \in || \mathbf{D} ||$ to an arrow $F(f) : F(a) \rightarrow F(b) \in || D||$, such that it satisfies: $F(id_a) = id_{F(a)}$, for any $a \in | \mathbf{C} |$, and $F(g \circ f) = F(g) \circ F(f)$ for any $f,g \in ||\mathbf{C} ||$.  

Natural transformations are mapping between functors, that is, given two functors $F, G: \mathbf{C} \rightarrow \mathbf{D}$, a natural transformation $\eta$ between $F$ and $G$, noted $\eta : F \xrightarrow{.} G$,  assigns to each object $x \in |\mathbf{C}|$ an arrow $\eta_x : F(x) \rightarrow G(x)$ in $\mathbf{D}$ such that for any arrow $f:a \rightarrow b \in || \mathbf{C}||$ we have that $\eta_b \circ F(f) = G(f) \circ \eta_a$ (this is called \emph{naturally}).  Two functors $F: \mathbf{C} \rightarrow \mathbf{D}$ and $G: \mathbf{D} \rightarrow \mathbf{C}$ are said to be \emph{adjoints} if the sets $\mathit{hom}(F(c),d)$ and $\mathit{hom}(c, G(d))$ are naturally isomorphic, $F$ is called left adjoint of $G$, and $G$ is said to be the right adjoint of $F$.  Adjoints  are common in algebra where the forgetful functor (noted $U$) that sends each algebra to its support set, and the construction of free algebras, which is a functor noted $F$, are adjoints.  A full subcategory is called reflective if the inclusion functor (from the subcategory to the main category) has a left adjoint. 

In any category we can define objects with the so-called universal constructions,  well-know construction in the category $\mathbf{Set}$ are products and coproducts,  which are instances of the more general concepts of  limits and colimits, respectively.    Given an index category $\mathbf{J}$ a diagram in $\mathbf{C}$ is a functor $D: \mathbf{J} \rightarrow \mathbf{C}$, that is, it is a graph on $\mathbf{C}$.  In particular, we can consider, for each object $c$, the constant functor $c : \mathbf{J} \rightarrow \mathbf{C}$ that sends each object $j \in |J|$ to $c$, and each arrow to $id_c$.  A \emph{cone} with tip $c$ is a natural transformation 
$\tau : F \xrightarrow{.} c$. The collection of all cones with tip $c$ form a category,   a colimit (which is characterized up to isomorphisms) is an initial object in this category.  For instance, to obtain coproduct in $\mathbf{Set}$ consider the index category having only two objects, say $x,y$, with only the identity arrows,  any cone maps this object to a tip, and the colimit is one of this cones whose tip, noted $x+y$, has unique arrows to the other possible tips. This formalizes the notion of disjoint union. Limits can be defined similarly and are the dual notion to colimits.

We will use some standard construction of categories, for instance, given two functors $F: \mathbf{C} \rightarrow \mathbf{E}$ and $G: \mathbf{D} \rightarrow \mathbf{E}$ the comma category, denoted $F \downarrow G$, has as objects arrows $f: F(c) \rightarrow G(d)$,  for $c \in |\mathbf{C}|$ and $d \in |\mathbf{D}|$,  and the arrows between objects $f:  F(c) \rightarrow G(d)$ and $g : F(c') \rightarrow G(d')$ are arrows $\alpha : c \rightarrow c'$ and $\beta : d \rightarrow d'$ such that $G(\beta) \circ f = g \circ F(\alpha)$.
 
\subsection{The Category $\mathbf{Dal}$}

Let us  introduce the category of \DAL algebras.
\medskip
\begin{definition} The category $\mathbf{Dal}$ has  the  algebras of \Cref{definition:deontic:algebra} as its objects, and 
the homomorphisms between these algebras as its arrows.
\end{definition}
\medskip
Note that proving that $\mathbf{Dal}$ is already a category is direct, the composition is the composition of homomorphisms, and the identity is the identity homomorphism.
Let us prove some properties of $\mathbf{Dal}$.  First,  note that this is a concrete category since we have a forgetful functor $U: \mathbf{Dal} \rightarrow \mathbf{Set}$ sending each \DAL algebra to its support sets. Formally,  for a \DAL algebra  $\Algebra[D]=\tup{\Algebra[A], \Algebra[F], \E, \P, \F}$ and let $A$ be the support set of $\Algebra[A]$ and $B$ the support set of $\Algebra[B]$,  then we define:
$U(\Algebra[D]) = (A,B)$ for objects, and $U(h)(x) = h(x)$, for homomorphisms.   
It is direct to prove that $U$ is already a functor.
\medskip
\begin{theorem} The mapping $U: \mathbf{Dal} \rightarrow \mathbf{Set}$ is a functor.
\end{theorem}
\medskip
A more important property of  $\mathbf{Dal}$ is its cocompleteness,  thus it has all colimits: coproducts, pushouts, etc.  Also, it has free objects, that is,  given a set $X$ there is an object $\Algebra[D]$ in $\mathbf{Dal}$ such that   $f: X \rightarrow U(\mathbf{Dal})$, and for any other function $g : X \rightarrow U(\Algebra[D'])$ there is a unique homomorphism $u : \Algebra[D] \rightarrow \Algebra[D']$ such that 
$g = U(u) \circ f$; Lindenbaum algebras, for instance, are free objects. 

The proof mainly follows from a result proven in \cite{Adamek04} for quasivarities.  We adapt that result to our algebras.
\medskip
\begin{theorem}\label{theorem:cocompleteness} The category $\mathbf{Dal}$  is  cocomplete and has free objects
\end{theorem}
\begin{proof} The proof follows the ideas of \cite{Adamek04}, we add some comments for our particular case.  For $\mathbf{Dal}$, consider first the category $\mathbf{Alg}(\Sigma)$ the category of all the $\Sigma$-algebras.  This is a category is cocomplete  as proven in \cite{Tarlecki91}.  Now, we  define a functor $F : \mathbf{Alg}(\Sigma) \rightarrow \mathbf{Dal}$ as follows. For every $\Sigma$-algebra 
$A$, $F(A) = A/\cong$ where $\cong$ is the smallest congruence such that $\mathbf{D}/\cong \in |\mathbf{Dal}|$.  For instance,  if $T$ is the term algebra, then $F(T)$ is the Lindenbaum algebra. Furthermore,  if 
$\mathbf{A}$ is already a deontic action algebra then $\cong$ is just the equality.  $F$ is the left adjoint of $I:\mathbf{Dal} \rightarrow \mathbf{Alg}(\Sigma)$, the inclusion function.  Thus $\mathbf{Dal}$ is a reflective subcategory of $\mathbf{Alg}(\Sigma)$, since
$\mathbf{Alg}(\Sigma)$ is cocomplete and has free objects \cite{Tarlecki91},  by properties of reflective subcategories \cite{MacLane98},  we obtain that $\mathbf{Dal}$ is cocomplete and has free objects.
\end{proof}
The cocompleteness of $\mathbf{Dal}$ allows us to put together different algebras,  this is a common procedure when looking at formal systems as objects of categories,  see, for instance,  \cite{Goguen92}.

Now, we provide a categorical view to the duality between \DAL and more set based algebras (as show in \Cref{sec:algebraic-char}).  Particularly, we show that there is a duality between $\mathbf{Dal}$ and a category of topological spaces For doing so, we introduce some basic concepts, the interested reader is referred to \cite{Johnstone82}.
Let $\mathbf{BA}$ be the category of Boolean algebras and $\textbf{Stone}$ the category of Stone spaces, i.e.,  its objects are topological spaces that are totally disconnect, compact and Hausdorff   and its arrows are continuous maps.  Stone duality states that categories $\mathbf{BA}$ and $\textbf{Stone}$ are dually equivalent, i.e.,  there exist functors $S: \mathbf{BA}^{op} \rightarrow \mathbf{Stone}$ and 
$T: \mathbf{Stone}^{op} \rightarrow \mathbf{BA}$ and natural isomorphisms $\eta_b : I_{\mathbf{BA}} \rightarrow TS$,  $\Theta_s : I_{\mathbf{Stone}} \rightarrow ST$.   Where $I_{\mathbf{BA}}: \mathbf{BA} \rightarrow \mathbf{BA}$ and $I_{\mathbf{Stone}}: \mathbf{Stone} \rightarrow \mathbf{Stone}$ are the identity functors.
Intuitively, this states that categories 
$\mathbf{BA}$ and $\mathbf{Stone}$ are,  up to isomorphisms,  dually the same.

For obtaining the same result for the \DAL algebras, first,  we define the corresponding categories based on topological spaces.  Let $\Delta : \mathbf{Stone} \rightarrow \mathbf{Stone}^2$ the diagonal functor, that is, the functor sending each Stone space $s$ to the pair $(s,s)$ and each arrow $f:s \rightarrow s'$ to the pairs of arrows $(f,f)$.  Now, consider the comma category $\Delta \downarrow \Delta$, that is, the category whose objects are pairs of arrows $(f:s\rightarrow s', g: s \rightarrow s')$ where $f,g \in ||\mathbf{Stone}||$ and the arrows are pairs of arrow making the corresponding diagram to commute.  We define the normative Stone Spaces as follows:
\begin{definition} The category $\mathbf{NStone}$ (of Normative Stone spaces) is the full subcategory of $\Delta \downarrow \Delta$ such that for all the objects $(f,g) \in |\mathbf{NStone}|$ the equalizer of $f$ and $g$  is the initial element (in $\mathbf{Stone}$).
\end{definition} 
The condition in the definition above ensures that  every pair of functions composing an object in $\mathbf{NStone}$ are disjoint.  Now, we can prove our extension of Stone duality for \DAL algebras.
\begin{theorem}\label{theorem:duality} Categories $\mathbf{Dal}$ and $\mathbf{NStone}$ are dually equivalent.
\end{theorem}
\begin{proof}
For proving this first we define the functors $N: \mathbf{Dal}^{op} \rightarrow \mathbf{NStone}$ and $M: \mathbf{NStone}^{op} \rightarrow \mathbf{Dal}$.  $N$ is defined as follows.
Without loss of generality,   we fix a \DAL algebra  $\Algebra[D] = \tup{\Algebra[A], \Algebra[F], \E, \P, \F}$,  consider the Stone spaces $S(A)$ and $S(B)$ given by the Stone function (as explained above).  As $S$ is a 
contravariant functor we have continuous functions $S(\P)$ and $S(\F)$, hence we define $N(\Algebra[D]) = (S(\P),  S(\F))$.  For arrows, let
$f : \Algebra[D] \rightarrow \Algebra[D']$ (where $\Algebra[D'] = \tup{\Algebra[A'], \Algebra[F'], \E', \P', \F'}$),  which is defined by two homomorphisms $f_a$ and $f_b$. 
then $N(f) : N(\Algebra[D]) \rightarrow N(\Algebra[D'])$ is given by $(\Delta(S(f_a)), \Delta(S(f_b)))$.  

On the other hand,  the functor $M : \mathbf{NStone} \rightarrow \mathbf{Dal}$ is defined as follows.   Consider an object $(f,g) \in |\mathbf{NStone}|$ thus $f: s \rightarrow s'$ and $g:s \rightarrow s'$
where $s,s'$ are Stone spaces and $f,g$ continuous functions such that the equalizer of them is the initial object.  Furthermore, by Stone duality,  we have a contravariant functor $\mathit{Clop}: \mathbf{Stone}^{op} \rightarrow \mathbf{BA}$ (that smaps any Stone space to the Boolean algebra of its clopen sets).  Thus,   we can consider the Boolean  algebras $\mathit{Clop}(s)$ and $\mathit{Clop}(s')$, and Boolean homomorphisms $\mathit{Clop}(f): \mathit{Clop}(s') \rightarrow \mathit{Clop}(s)$
and $\mathit{Clop}(g): \mathit{Clop}(s') \rightarrow \mathit{Clop}(s)$ furthermore $f(x) \cap f(y) = \emptyset$ for any $x \in \mathit{Clop}(s')$, hence $(\mathit{Clop}(s), \mathit{Clop}(s'), \E,  \mathit{Clop}(f), \mathit{Clop}(g))$, being $\E(x,y) = \mathcal{U}(s)$ iff $x=y$, otherwise $\E(x,y) = \emptyset$.  Proving this is already a functor is direct since $\mathit{Clop}$ is a functor.

Now,  let us show that there are natural isomorphisms $\varepsilon: I_{\mathbf{NStone}} \rightarrow NM$ and $\eta : I_{\mathbf{Dal}} \rightarrow MN$.   Let $(f:s \rightarrow s',g:s \rightarrow s') \in |\mathbf{NStone}|$,
by stone duality we know that there is a $i :s \rightarrow S(\mathit{Clop}(s))$ that is a homeomorphism between the two topological spaces,  similarly we have an homeomorphism $j: s' \rightarrow S(\mathit{Clop}(s'))$. Thus $(i,j) : (f,g) \rightarrow S(Cl((f,g)))$ gives us the corresponding isomorphism $\eta_{(f,g)}$.  For $\epsilon$ the proof is similar, for a  \DAL algebra  $\Algebra[D] = \tup{\Algebra[A], \Algebra[F], \E, \P, \F}$ we consider the Boolean algebras $\mathit{Clop}(S(\Algebra[A]))$ and $\mathit{Clop}(S(\Algebra[F]))$ which by Stone duality are isomorphic to $\Algebra[A]$ and $\Algebra[F]$, similarly for functions 
$\P$ and $F$ which we obtain functions $\mathit{Clop}(S(\P))$ and  $\mathit{Clop}(S(\F))$ which are the as the original up to isomorphism, putting all this together we obtain the isomomorphism $\varepsilon_{\Algebra[D]}$. Naturality of $\varepsilon$ and $\eta$ follows from the naturality of the corresponding natural isomorphism of Stone duality.
\end{proof}

We end this section we some remarks about the categories corresponding to the other algebras defined in this paper. We have show that category $\mathbf{Dal}$ exhibits some nice properties, one of them, cocompleteness, provides a mechanism to put together different deontic action algebras, thus making possible to modularize the reasoning about normative systems in an algebraic way.  Furthermore,   we have shown that Stone duality can be extended to our algebras allowing us to obtain an equivalence between our algebras and certain topological spaces.  For the other logics described in this paper similar results hold.  Note that the proof of \Cref{theorem:cocompleteness}  uses basic facts of the $\Sigma$-algebras and properties of quasivarieties,  which can be also be applied to the rest of the algebras presented in this paper.  Furthermore,  for the intuitionistic logics presented earlier we can use the Esakia duality \cite{Esakia19} between Heyting algebras and Esakia spaces, which can be termed as an intuitionistic version of Stone duality. Using Esakia duality the same constructions as in \Cref{theorem:duality} can be used to provide duality results for $\DAL(\INT)$,  or any of the other logics. 




%The first important property is that these algebras are cocomplete and posees