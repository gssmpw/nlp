\section{Introduction}\label{section:introduction}

The logical laws of normative reasoning have attracted the attention of philosophers, lawyers, logicians, and computer scientists since the beginning of their disciplines~\cite{Gabbay:2013}.
The first modern deontic systems go back to the pioneer
works of von Wright~\cite{vonWright:1951},
Kalinowski~\cite{Kalinowski53}, and Becker~\cite{Becker52}.
These first systems, that were conceived on the view that prescriptions apply to actions, were swiftly overtaken by modal approaches where prescriptions applied to propositions~\cite{blac:moda00}.
Indeed, it can be said that the influence of Modal Logic marked a crossroads at the very beginning of Deontic Logic.
One path leads to the so-called \emph{ought-to-be} deontic systems, where prescriptions apply to propositions; the other path leads to the so-called \emph{ought-to-do} deontic systems, where prescriptions apply to actions.

% The most prominent outcome of the propositional modal school is \emph{Standard Deontic Logic} (\SDL).

% \SDL is the most easily recognized deontic logic in the literature~\cite{Aqvist:2002}.
% Yet, its prominent status is not without scrutiny. \SDL is known to admit paradoxes (theorems in the logic that are intuitively invalid~\cite{Aqvist:2002,Meyer:1994}).
% These paradoxes are countered with alternative, or refined, deontic systems roughly labeled as \emph{ought-to-be} or \emph{ought-to-do}.
% The former encompass a view in which prescriptions apply to state of affairs.
% %Formally, they rely mostly on variants of \SDL, and, more generally, on modal logics.
% The later retake von~Wright's maxim of building deontic logics upon a theory
% of actions~\cite{vonWright:1951,vonWright:1999}.

The most easily recognized representative of the ought-to-be systems is \emph{Standard Deontic Logic} (\SDL)~\cite{Aqvist:2002}.
\SDL extends the normal modal system \textsf{K} with an axiom for \emph{seriality}.
In \SDL, the ``box'' modality, written $\mathsf{O}\varphi$, is informally read as ``$\varphi$ is obligatory''.
Resorting to this modality, we can formally discuss the implications of how to understand obligations.
For instance, if $w$ formalizes the proposition ``John is behind the steering wheel'', and $i$ formalizes the proposition ``John is intoxicated'', we can explore which one of $w \to \mathsf{O}(\lnot i)$, $\mathsf{O}(w \to \lnot i)$, or $\mathsf{O}(w \land \lnot i)$, better captures the proposition ``it is obligatory that John is not intoxicated while being behind the steering wheel''.
This kind of propositions are discussed at length in the literature on Deontic Logic and riddled with challenges and paradoxes~\cite{Aqvist:2002}.

The counterpart to \SDL for ought-to-do systems is, arguably, the \emph{Deontic Action Logic} (\DAL) proposed by Segerberg in~\cite{Segerberg1982}.
The language of \DAL distinguishes between actions and formulas. Actions are built up from basic action names using action operators.
Then, deontic connectives $\perm$ of permission and $\forb$ of prohibition are applied to actions to yield formulas that can be combined using logical connectives.
Let us illustrate this by means of a simple example.  
% Let $\mathsf{driving}$ and $\mathsf{drinking}$ be basic action symbols; the formula $\lnot \perm(\mathsf{driving} \sqcap \mathsf{drinking})$ asserts
% that drinking while driving is not permitted.  In this formula,
% $\sqcap$ is an action operator that can be understood as parallel
% execution; $\perm$ stands for the deontic connective of
% permission, and $\lnot$ is standard logical negation.

\medskip

\begin{example}
Let $\mathsf{driving}$ and $\mathsf{drinking}$ be basic action symbols indicating the \emph{acts} of driving, and drinking, respectively.
Moreover, let $\mathsf{driving} \sqcap \mathsf{drinking}$ be understood as the parallel composition of the actions $\mathsf{driving}$ and $\mathsf{drinking}$.
Then, $\forb(\mathsf{driving} \sqcap \mathsf{drinking})$ is a formula intuitively taken to assert
that drinking while driving is forbidden. %It is worth noticing that if $\forb(\mathsf{driving} \sqcap \mathsf{drinking})$ is uphold, neither $\mathsf{driving}$ or $\mathsf{drinking}$
%would be individually forbidden. 
%Thus, a logical framework like \DAL provides an adequate  tool to deal with this kind of situations involving normative states. 
\end{example}

\medskip

The work of Segerberg in \cite{Segerberg1982} shifted the focus from deontic operators on propositions to deontic operators on actions. 
Moreover, it gave rise to what nowadays can be construed as a family of deontic action logics~\cite{Maibaum87,Segerberg1982,Meyer:1994,Castro:2009,BroersenThesis,Trypuz:2010,Trypuz15,Prisacariu:2012}.
In addition to being interesting from a purely logical perspective, the logics in this family are good canditates as formalisms to describe the behavior of real world systems.
For instance, in~\cite{Castro:2009,DemasiCRMA15}, a variant of \DAL is used to reason about fault-tolerance. 
Therein, actions formalize changes of state in a system, permitted actions indicate the normal behavior of the system, while forbidden actions are used to model the faulty behavior of the system.
This classification of actions paves the way to reasoning about how to react in response to faults.
In this setting, we can, for instance, understand a formula such as $\forb(\mathsf{read} \sqcap \mathsf{write})$ as prescribing the behavior of a system by indicating that it is forbidden to simultaneously read and write from a memory location, as this could lead to the system being in an inconsistent state.
The falsehood of this formula in a particular scenario serves as an indication of faulty behavior, indicating the necessity of fault-tolerant mechanisms to safeguard the normal operation of the system.

%\carlos{Tiene sentido agregar aca algun ejemplo de este paper, como ilustracion mas compleja de DAL?}

% \DAL is extremely simple and admits a sound and complete axiomatization.  One of the most interesting features of \DAL is its
% two tier interpretation structure, i.e., actions are interpreted
% via an algebra of events, whereas formulas are interpreted
% using truth values.
% \DAL was revisited by
% other authors, for instance:~\cite{Castro:2009} introduces action
% prescriptions and combines them with modal operators,
% and~\cite{Trypuz:2010} investigates several fragments of \DAL.  We use the terminology in \cite{Trypuz:2010} and
% call all these formalisms \emph{Deontic Action Logics}.

\medskip\noindent
\textbf{Proposal.}
We take \DAL as the starting point to investigate the construction of deontic action logics in an algebraic framework.
% The development of algebraic logic during the 20th century established a strong connection between logic and techniques and results from algebra (see, e.g.,~\cite{DunnH:2001}).
To this end, we build on an earlier work where we develop an abstract view of \DAL resorting to algebraic structures called \emph{deontic action algebras}~\cite{CCFA:2021}.
% Therein, we followed standard ideas presented by:
%     Halmos in~\cite{Halmos:1998} of using Boolean algebras as an abstraction of propositions,
%     Venema in~\cite{Venema:2001} of using Boolean algebras with operators as an algebraic counterpart of modal logics, and
%     Pratt in~\cite{Pratt:1991} of using dynamic algebras to investigate theoretical properties of propositional dynamic logic via many-sorted algebras.
% In our algebraic treatment of \DAL we introduced \emph{deontic action algebras}.
In brief, a deontic action algebra consists of a Boolean algebra for interpreting actions, a Boolean algebra for interpreting formulas, and two mappings from one Boolean algebra to the other interpreting the deontic concepts of permission and prohibition.
We use deontic action algebras to formally interpret the two tier structure of the language of \DAL.
An interesting feature of this algebraic treatment of \DAL is that it is modular, giving rise to natural variations.
We explore this modularity by elaborating of how to capture various logics in the \DAL family.

% In this paper we investigate Deontic Action Logics from the perspective of algebraic logic. The development of algebraic logic during the 20th century
% established a strong connection between logic and techniques and results from algebra (see, e.g.,~\cite{DunnH:2001}).  Taking this as an  inspiration,
%  we develop an abstract view of deontic action
% logics in terms of algebraic structures.  We follow some
% of the standard ideas, presented, e.g., by Halmos in~\cite{Halmos:1998}, of using
% Boolean algebras as an abstraction of propositions; Venema
% in~\cite{Venema:2001}, who introduced Boolean algebras with operators
% as an algebraic counterpart of modal logics; and Pratt
% in~\cite{Pratt:1991}, who introduced dynamic algebras to investigate
%  theoretical properties of dynamic logics via many-sorted
% algebras. In particular, in our algebraic treatment of \DAL, we introduce what we refer to as \emph{Deontic Action Algebras}.
% In brief, a deontic action algebra is a pair of two Boolean algebras, one in which we interpret formulas, the other in which we interpret actions; while deontic operators are
% modeled as functions connecting both algebras.
% An interesting feature of our algebraic treatment of \DAL is that it is modular.  By replacing one or both of the algebras forming a deontic action algebra pair  we obtain new formalisms with different flavors.

\medskip\noindent
\textbf{Contributions.} %We extend the work in \cite{CCFA:2021} on deontic action algebras as an algebraic semantics for \DAL.
%In particular, we provide detailed soundness and completeness proofs.
%Moreover, we elaborate on the connection between the algebraic semantics for \DAL and its model based semantics via a Stone-type representation result.
This paper continues and extends our work in \cite{CCFA:2021}. First, we revisit the algebraic framework of deontic action algebras, provide detailed soundness and completeness proofs, and add motivating examples. 
Second, we present a series of deontic action logics which exploit the modularity of the algebraic framework.
We begin by enriching the algebra for formulas with propositions to describe \emph{states of affairs}.  In this way, we can express both properties of actions, and propositions (e.g., pre- and post-conditions) about the states in which this actions take place. 
Then, we discuss the result of replacing the Boolean algebra interpreting formulas by a Heyting algebra.
The resulting extension of \DAL can deal with scenarios in which laws like the excluded middle or contraposition might be rejected.
This could be the case, for example, in normative systems in which evidence is required in order to accept some assertions as true.
In turn,  we consider using a Heyting algebra to interpret actions. We argue that this would be useful, e.g., when actions are associated to constructions witnessing their \emph{realizability}.
This admits a direct analogy with the standard interpretation of intuitionistic logic in which the concept of truth is associated to that of proof.
Clearly, we can do both at the same time, obtaining a fully intuitionistic deontic action logic were both actions and formulas are interpreted using Heyting algebras. 
In all cases, we obtain axiom systems that are sound and complete for the corresponding classes of deontic action algebras.

Algebraic logic has been shown useful for analyzing theoretical properties of logics and for investigating relations between different logics~\cite{ras1963,Andreka1991-ANDAL-2}. We believe that the case of \DAL presents a particularly simple and elegant instance of this framework. 

%We put forth that this algebraic treatment of \DAL is also particularly well suited to study deontic actions logics from a dynamic perspective, \emph{\'a la} Public Announcement Logic~\cite{Plaza2007}.
%In this respect, we notice that the algebraic semantics of deontic operators defines a restriction of the underlying algebra (corresponding to an ideal). This bears a resemblance to model update operators as in Public Announcement Logic, designed specifically to describe changes in a system and their effects.

\ifcategories
Furthermore,  in \Cref{sec:cat} we extend our algebraic view of \DAL to category theory, defining the corresponding category of \DAL algebras and investigating some of its properties.  We prove that these categories are cocomplete,  as a consequence we can use colimits to construct complex algebras from simpler ones,  facilitating the modular reasoning over \DAL algebras.  This is a common practice, for instance, in  logical theories, graphs,  software specfications, etc, see \cite{Goguen92} for some examples.  We also prove an extended version of Stone duality for the introduced categories, showing that the introduced algebras can be also seen as  topological spaces. These results also hold for the other algebras introduced in this paper, as remarked in the aforementioned section.
\fi

% We first introduced Deontic Action Algebras in~\cite{CCFA:2021}. In this article, we extend that work by providing detailed soundness and completeness proofs and, more interestingly, we investigate some possible variations.  We first enrich the formula algebra with  propositions, enabling in this way the ability to describe \emph{state of affairs}.  We then discuss the result of replacing the Boolean algebra interpreting formulas by a Heyting a algebra. The resulting \DAL can deal with scenarios in which laws like the excluded middle or contraposition might be rejected. This could be the case, for example, in normative systems in which evidence is required in order to accept some assertions as true.  Finally,  we consider using a Heyting algebra to interpret actions. We argue that this would be useful,  for instance,  when actions are associated to constructions witnessing their \emph{realizability}. This admits a direct analogy with the standard interpretation of intuitionist logic in which the concept of truth is associated to that of proof.
% All these algebraic variations of the original \DAL are obtained in a natural way, and in particular, in all cases sound and complete axiom systems can be provided. 

% PROPONGO DEJAR ESTO PARA MAS ADELANTE
%In our case, we  associate actions to possible \emph{plans} to execute them. Here, the implication between actions acquires an important role: the action $a \actimplies b$ states that the agent has a procedure to extract plans for $b$ given a plan for  $a$. In this case, the implication $a \actimplies 0$ formalizes the negation, or complement, of an action. This can be understood as stating that the agent has a procedure to determine the impossibility of executing action  $a$.  In such a case, one may reject the law of the excluded middle on the grounds that having a plan to either execute action $a$ or to show that no plan for $a$ exists,  does not imply that the agent has a plan to execute every action.   As noted in the forthcoming sections,

%We put forth that the benefits of this algebraic version of deontic
%action logics are twofold.  Firstly, algebraic logic has been shown
%useful when analyzing theoretical properties of logics and
%nvestigating the relations between different formalisms.  Secondly,
%extensions to a deontic action logic can be obtained by considering
%different action and predicate algebras.  We explore these ideas in
%\Cref{section:dals}.
%We also prove the algebraic versions of soundness and completeness, and furthermore we prove that the satisfiability problem for the basic version of \DAL is NP-Complete.

\medskip\noindent
\textbf{Structure of the Paper.} \Cref{section:dal} introduces basic definitions, and Segerberg's deontic action logic \DAL.  \Cref{sec:algebraic-char} presents the basic algebraic framework,
and prove soundness and completeness for \DAL
using standard algebraic tools. Variations of \DAL are investigated in~\Cref{section:new:dals}.
\Cref{section:conclusion} offers some final remarks and discusses
future work.
