
\paragraph{Axiomatization.}

We present an axiomatic system for \DAL that differs  slightly from the one originally used in \cite{Segerberg1982}. This change is solely motivated by the fact that our presentation simplifies the axiomatization of the variations to \DAL that we introduce in the following sections.

We organize the presentation of the axiom system of \DAL in four groups of axioms as shown in 
\Cref{dal:axioms}. %(i.e., we assume uniform substitution by expressions of the proper type).
The axioms in the first group (A1--A13 and LEM) characterize operations on actions, and is inspired by the presentation of Boolean algebras via complemented distributive lattices in~\cite{Esakia:2019,Halmos:2009}.
%
The axioms in the second group do the same for propositional connectives on formulas (A1'--A13' and LEM').
%Since they are well-known, we leave the axioms in this group implicit.
%Its explicit formulation is obtained from the axioms for actions by replacing $\alpha$, $\beta$, and $\gamma$, for $\varphi$, $\psi$, and $\chi$; $\sqcup$, $\sqcap$, $\bar{~}$, $\mathsf{0}$, and $\mathsf{1}$, for $\lor$, $\land$, $\lnot$, $\bot$, and $\top$; and $=$ for $\liff$.
%To see that this group of axioms indeed captures propositional connectives on formulas we refer to~\cite{Mendelson:2015,Troelstra:1988}.
%
The axioms in the third group (E1 and E2) characterize equality.
%
Finally, the axioms the fourth group (D1--D3) characterize the deontic operators of permission $\perm$ and prohibition $\forb$.
%
%These four groups of axioms are summarized in \Cref{dal:axioms}.
%Therein, axioms A1--A13, and LEM correspond to axioms for operations on actions --and, adapted accordingly, to axioms for propositional connectives on formulas.
%In turn, axioms E1 and E2 are the axioms for equality.
%In axiom E2, we use $\varphi_{\alpha}^{\beta}$ to indicate the formula obtained by replacing some occurrences of $\alpha$ in $\varphi$ with $\beta$.
%Finally, axioms D1--D3 are the axioms for the deontic operators of permission and prohibition.
%The axioms D1--D3 appear first in~\cite{Segerberg1982}.

\begin{figure}
	\centering
	\fbox{
	\begin{minipage}{0.95\textwidth}
		\begin{multicols}{2}
			\begin{enumerate}[label=A\arabic*.]
				\item $\alpha \sqcap (\beta \sqcap \gamma) = (\alpha \sqcap \beta) \sqcap \gamma$
				\item $\alpha \sqcap \beta = \beta \sqcap \alpha$
				\item $\alpha \sqcap \alpha = \alpha$
				\item $\alpha \sqcap (\alpha \sqcup \beta) = \alpha$
				\item ${\alpha \sqcap (\beta \sqcup \gamma)} = {(\alpha \sqcap \beta) \sqcup (\alpha \sqcap \gamma)}$
				\item $\alpha \sqcap \mathsf{0} = \mathsf{0}$
				%\item $((\alpha \sqcap \beta) \sqsubseteq \alpha) \sqcap \gamma = \gamma$
				%\item $\alpha \sqsubseteq \mathsf{0} = \bar{\alpha}$
				\item $\alpha \sqcap \bar{\alpha} = \mathsf{0}$
				\item $\alpha \sqcup (\beta \sqcup \gamma) = (\alpha \sqcup \beta) \sqcup \gamma$
				\item $\alpha \sqcup \beta = \beta \sqcup \alpha$
				\item $\alpha \sqcup \alpha = \alpha$
				\item $\alpha \sqcup (\alpha \sqcap \beta) = \alpha$
				\item ${\alpha \sqcup (\beta \sqcap \gamma)} = {(\alpha \sqcup \beta) \sqcap (\alpha \sqcup \gamma)}$
				\item $\alpha \sqcup \mathsf{1} = \mathsf{1}$
				%\item $\alpha \sqcap (\alpha \sqsubseteq \beta) = \beta$
				%\item $\alpha \sqcap (\beta \sqsubseteq \gamma) = \alpha \sqcap ((\alpha \sqcap \beta) \sqsubseteq (\alpha \sqcap \gamma))$
				\item[LEM.] $\alpha \sqcup \bar{\alpha} = \mathsf{1}$
			\end{enumerate}
		\end{multicols}
		\ \\[-1.5cm]
		\begin{multicols}{2}
	\begin{enumerate}[label=A\arabic*'.]
		\item $\varphi \wedge (\psi \wedge \chi) \liff (\varphi \wedge \psi) \wedge \chi$
		\item $\varphi \wedge \psi \liff \psi \wedge \varphi$
		\item $\varphi \wedge \varphi \liff \varphi$
		\item $\varphi \wedge (\varphi \vee \psi) \liff \varphi$
		\item ${\varphi \wedge (\psi \vee \chi)} \liff {(\varphi \wedge \psi) \vee (\varphi \wedge \chi)}$
		\item $\varphi \wedge \bot \liff \bot$
		\item $\varphi \wedge \neg \varphi \liff \bot$
		\item $\varphi \vee (\psi \vee \chi) \liff (\varphi \vee \psi) \vee \chi$
		\item $\varphi \vee \psi \liff \psi \vee \varphi$
		\item $\varphi \vee \varphi \liff \varphi$
		\item $\varphi \vee (\varphi \wedge \psi) \liff \varphi$
		\item ${\varphi \vee (\psi \wedge \chi)} \liff {(\varphi \vee \psi) \wedge (\varphi \vee \chi)}$
		\item $\varphi \vee \top \liff \top$
		\item[LEM'.] $\varphi \vee \neg \varphi \liff \top$
	\end{enumerate}
\end{multicols}
		\ \\[-1.5cm]
		\begin{multicols}{2}
			\begin{enumerate}[label=E\arabic*.]
				\item $\alpha = \alpha$
				% \item ${\alpha = \beta} \to {\beta = \alpha}$
				% \item ${(\alpha = \beta \land \beta = \gamma)} \to {\alpha = \gamma}$
				\item $(\alpha=\beta \land \varphi) \to {\varphi_{\alpha}^{\beta}}$%\footnote{where $\varphi_{\alpha}^{\beta}$ is obtained by replacing some occurrences of $\alpha$ in $\varphi$ with $\beta$.}
			\end{enumerate}
		\end{multicols}
		\ \\[-1.5cm]
		\begin{multicols}{2}
			\begin{enumerate}[label=D\arabic*.]
				\item $\perm(\alpha\sqcup\beta) \liff (\perm(\alpha) \land \perm(\beta))$
				\item $\forb(\alpha\sqcup\beta) \liff (\forb(\alpha) \land \forb(\beta))$
				\item $(\perm(\alpha) \land \forb(\alpha)) \liff (\alpha = \mathsf{0})$
			\end{enumerate}
		\end{multicols}
	\end{minipage}}\\[1em]
	% \medskip
	\caption{Axiom System for \DAL.}\label{dal:axioms}
\end{figure}

% \medskip

% {\setlength\tabcolsep{1pt}
% 	\begin{center}
% 		\footnotesize
% 		\setlength{\abovedisplayskip}{4pt}
% 		\setlength{\belowdisplayskip}{-6pt}
% 		\setlength{\abovedisplayshortskip}{4pt}
% 		\setlength{\belowdisplayshortskip}{-3pt}
% 		\renewcommand{\arraystretch}{1.7}

% 		\begin{tabular}{|@{\ \ }p{.97\textwidth}|}	\hline
% 			\begin{enumerate}[label=D\arabic*]
% 				\begin{minipage}[t]{0.4\linewidth}
% 			\item $\perm(\alpha\sqcup\beta) \liff (\perm(\alpha) \land \perm(\beta))$
% \item $\forb(\alpha\sqcup\beta) \liff (\forb(\alpha) \land \forb(\beta))$
% 				\end{minipage}
% 				\hfill
% 				\begin{minipage}[t]{0.52\linewidth}
% \item $(\perm(\alpha) \land \forb(\alpha)) \liff (\alpha = \mathsf{0})$
% 				\end{minipage}
% 			\end{enumerate}\\\hline
% 		\end{tabular}
% \end{center}}

% \medskip

In \DAL, a Hilbert-style proof of a formula $\varphi$ is defined as a finite sequence $\psi_1, \dots, \psi_n$ of formulas s.t.: $\psi_n = \varphi$, and for each $1 \leq k \leq n$, $\psi_k$ is an axiom, or is obtained from two earlier formulas $\psi_i$ and $\psi_j$ using the rule of \emph{modus ponens} (i.e., there are $1 \leq i < j < k$ s.t.\ $\psi_j = {\psi_i \to \psi_k}$).
We say that $\varphi$ is a theorem, and write $\vdash \varphi$, iff there is a proof of $\varphi$.
We make a slight abuse of notation and use \DAL to indicate both the logic and its set of theorems.
We state \Cref{th:segerber:completeness} for future reference.

\medskip
\begin{theorem}[\cite{Segerberg1982}]\label{th:segerber:completeness}
	In \DAL, a formula is a theorem if and only if it is a tautology.
\end{theorem}
