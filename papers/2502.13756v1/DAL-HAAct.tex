\subsection{A Heyting Algebra of Actions}\label{sec:action-int}

The formal machinery in \Cref{sec:heyting:formulas} naturally suggests its symmetric extension: replacing the Boolean algebra of actions with a Heyting algebra.
This results in a new deontic action logic, $\DAL(\IAL)$, where actions are interpreted in a manner analogous to formulas in an intuitionistic framework.
To the best of our knowledge, no existing deontic action logic provides an intuitionistic perspective on actions, making this approach a novel contribution to the field.

\paragraph{Another look at Constructive Reasoning in Deontic Action Algebras}

We begin with an outline of the technical foundations of $\DAL(\IAL)$.
The formulas of this new logic are built using proposition symbols in $\prop$, the deontic connectives on actions, i.e., $\perm(\alpha)$ and $\forb(\alpha)$, and the connectives $\lor$, $\land$, $\lnot$, $\bot$, and $\bot$.
In turn, actions are built using basic action symbols in $\bact$, and the connectives $\sqcup$, $\sqcap$, $\bar{~}$, $\iact$, and $\uact$.
In addition, $\DAL(\IAL)$ introduces a new connective $\hto$ on actions giving rise to actions of the form $\alpha \hto \beta$.
This new connective is introduced to capture the notion of a relative complement (or intuitionistic implication) in a Heyting algebra.
The axiomatization of $\DAL(\IAL)$ uses all the axioms in \Cref{dal:axioms} except the axiom (LEM) for actions.
In addition, it introduces as axioms the properties H1\textendash H3 in \Cref{def:heyting:algebra} for the new connective $\hto$.
In essence, the axioms for actions are the conditions on Heyting algebras in~\cite{Esakia19}.
Provability and theoremhood are easily adapted to accommodate for the new axioms.
%\carlos{No me queda muy claro por que esta axiomatizacion es diferente de la anterior.}

The algebraization of $\DAL(\IAL)$ replaces the Boolean algebra of actions in the definition of a deontic action algebra for a Heyting algebra. This is made precise in \Cref{def:dalgebra:heyting:actions} below.

\medskip
\begin{definition}\label[definition]{def:dalgebra:heyting:actions}
	An HB-deontic-action algebra is an algebra
		$\DAlgebra =
			\langle
				\Algebra[H], \Algebra[F], \E, \P, \F
			\rangle$
		where:
			$\Algebra[H]$ is a Heyting algebra,
			$\Algebra[F]$ is a Boolean algebra, and
			$\E : {|\Algebra[H]| \times |\Algebra[H]| \to |\Algebra[B]|}$,
			$\P : {|\Algebra[H]| \to |\Algebra[B]|}$,
			and
			$\F : {|\Algebra[H]| \to |\Algebra[B]|}$ satisfy the conditions 1--6 in \Cref{definition:deontic:algebra}.
\end{definition}
\medskip

\Cref{prop:heyting:actions:ideal} shows that permission and prohibition behave as expected.

\medskip
\begin{proposition}\label[proposition]{prop:heyting:actions:ideal}
	Let $\DAlgebra = \tup{\Algebra[H], \Algebra[F], \E, \P, \F}$ be an HB-deontic-action algebra.
	The pre-images $P$ and $F$ of $\top$ under $\P$ and $\F$, respectively, are ideals in $\Algebra[A]$ s.t.\ ${{P \cap F} = \{\iact\}}$.
\end{proposition}
\begin{proof}
	Analogous to that in \Cref{prop:ideals-int}.
\end{proof}
\medskip

%\paragraph{Soundness and completeness.} Interpretations into the kind of deontic action algebras in \Cref{def:dalgebra:heyting:actions} are defined straightforwardly.

Since $\DAL(\IAL)$ is the symmetric counterpart of $\DAL(\IPL)$, the proofs of soundness and completeness for $\DAL(\IAL)$ can be straightforwardly adapted from those of $\DAL(\IPL)$. Consequently, we establish the following theorem.

\medskip
\begin{theorem}\label{prop:completeness:heyting:actions}
	 Let $\mathbb{HB}$ be the class of all HB-deontic-action algebras.
	 It follows that $\varphi$ is a theorem of $\DAL(\IAL)$ iff $\mathbb{HB} \vDash {\varphi \doteq \top}$.
\end{theorem}

\paragraph{Constructive Reasoning and Realization of Actions}

We put forth the argument that an intuitionistic basis for actions is useful when actions are tied to constructions that witness their realizability.
This parallels the standard interpretation of Intuitionistic Logic, where the truth of a formula corresponds to the existence of a proof.
Interpreting actions on an intuitionistic basis is not only of theoretical interest but also hold potential for practical applications, particularly in automated planning~\cite{GNT:2016}.
For example, consider a robot capable of executing various actions.
To perform an action, the robot requires plans\textemdash sequences of basic activities that realize the action.
In such a scenario, we may reject $\bar{a} \sqcup a = \uact$, as the robot might lack a plan to execute action $a$ or a way to determine if $a$ is unrealizable.
% In this scenario, an action denoted by $a \actimplies b$ can be interpreted as a way of converting a plan for $a$ into a plan for $b$.
% Similarly, an action denoted by $\bar{a} = a \actimplies 0$ can be interpreted as a plan that converts a plan for $a$ into a plan that brings about an impossible action for the robot. In other words, $\bar{a}$ provides a way to describing that the robot cannot realize the action denoted by $a$.

Prescriptions often play an important role in planning.
For instance, if the robot is an autonomous vehicle, it must adhere to transit rules.
In this case, $\perm(\alpha)$ indicates that plans for executing $\alpha$ are permitted, while $\forb(\alpha)$ signals that such plans are forbidden.
This perspective highlights how an intuitionistic basis for interpreting actions aligns well with practical considerations in scenarios where realizability and prescriptive constraints on actions are central.
In this respect, $\DAL(\IAL)$ provides a logical framework that is well-suited to addressing real-life issues.

%	A main motivation for this novel formalism is  to reason about scenarios where there is incomplete information about the agents' possible actions.  Heyting algebras provide the useful notion of \emph{relative complement}, which, as we argue below, can be used in such contexts.

%	As remarked above, the basis of a Heyting algebra is the relative complement (written $a \rightarrow b$, for actions $a$ and $b$).  Algebraically, $a \rightarrow b$ denotes the coarsest action that, when executed together with $a$, yields an execution of action $b$.  In Boolean logic $a \rightarrow b$ is interpreted as
%$\overline{a} \sqcup b$. This notion of relative complement may be useful in diverse settings. Consider a scenario where one need to identify those actions that when executed in parallel with driving will lead to an imprudent way of driving, say $\text{driving} \rightarrow \text{behave-dangerously}}$,  if we have a Boolean model of action
%we have that this action coincides with $\overline{\text{driving}} \sqcup \text{behave-dangerously}$.  However, the action of drinking cannot be thought of as being included in the action $\text{behave-dangerously}$ (there are ways in which one may drink without behaving in a dangerous way), and similarly it cannot be regarded
%as being part of driving (there are ways of drinking without driving). It seems intuitively correct to state $\text{drink} \sqsubseteq \text{drive} \rightarrow \text{behave-dangerously}$, therefore by stating $\neg \perm(\text{drive} \rightarrow \text{behave-dangerously})$ one is capturing those actions that, when executed together with driving will result in a dangerous way of behaving. Another interesting application of Heyting algebras of actions is the possibility of capturing another version of complement,
