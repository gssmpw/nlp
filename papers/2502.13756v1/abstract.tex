% \begin{abstract}
\abstract{
	In a seminal work, K. Segerberg introduced a deontic logic called \DAL to investigate normative reasoning over actions. \DAL marked the beginning of a new area of research in Deontic Logic by shifting the focus from deontic operators on propositions to deontic operators on actions.
	%
	In this work, we revisit \DAL and provide a complete algebraization for it. In our algebraization we introduce deontic action algebras --algebraic structures consisting of a Boolean algebra for interpreting actions, a Boolean algebra for interpreting formulas, and two mappings from one Boolean algebra to the other interpreting the deontic concepts of permission and prohibition.
	%
	We elaborate on how the framework underpinning deontic action algebras enables the derivation of different deontic action logics by removing or imposing additional conditions over either of the Boolean algebras. We leverage this flexibility to demonstrate how we can capture in this framework several logics in the \DAL family.
	Furthermore, we introduce four variations of \DAL by:
	(a) enriching the algebra of formulas with propositions on states,
	(b) adopting a Heyting algebra for state propositions,
	(c) adopting a Heyting algebra for actions, and
	(d) adopting Heyting algebras for both.
	%
	We illustrate these new deontic action logics with examples and establish their algebraic completeness.
}
% \end{abstract}

%%% Local Variables:
%%% mode: latex
%%% TeX-master: "article"
%%% End:
