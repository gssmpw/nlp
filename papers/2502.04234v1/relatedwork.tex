\section{Related Works}
\citeauthor{w1} provides an overview of censorship practices within the microblogs of Weibo. Specifically, it is a term extraction project, which showcased a list of censored topics or keywords for 2012, a period of tightening in internet censorship. This list mostly consists of political references to CCP scandals or the rumored coup d'etat, both strictly censored topics. But in terms of application to our research, this list was derived from \citeauthor{Fu2021}. It is a data collection system built on Sina’s open API, that could track the timeline of thousands of users’ blogs and identify their removal status by repeated accessing. With over 111 million blogs collected in the year 2012, the API may serve as a great sample for the modeling of our classifier system.


\citeauthor{w2} serves as a basis for our research, as it suggests and investigates the hypothesis of the existence of a Chinese topic surveillance list. Additionally, it also employs \citeauthor{Fu2021} along with the use of TF-IDF in topic and feature extraction. The paper concludes that the platform supports both explicit, implicit, and camouflaged filtering based on the content of each post. However, the paper mentioned their difficulties in the tokenization of Chinese text, as well as stopped at the stage of finding the list of term frequency for each word. Therefore, inspired by the paper, we continued after the TF-IDF stage and used these features to construct a logistic regression model. Lastly, the paper also mentioned how the user may play a role in censoring. If a user had been posting censored content, that account’s posts may be auto-deleted. This may be something to be noted during further research.


\citeauthor{ahmed-kumar-m-2021-classification} serves as another basis for our research, as it employs pre-trained transformer models in XLNet, an auto-regressive model, to perform feature extraction as well as modeling. However, in this paper, our focus is on the evaluation scores. As their dataset is also implemented with \citeauthor{Fu2021} and another modern library, their results may be used for comparison with others. Specifically, the authors also have implemented BERT as a part of their research for comparisons with XLNet. The highest F-1 score of their system on the validation data is 0.634, and that is our initial goal. Furthermore, the paper also influenced performing binary classification, as the result of our data processing is in a binary form. However, a classical logistic regression is chosen compared to XLNet’s regressor due to our implementation of a TF-IDF-like feature extractor along with Bert.


\citeauthor{10.1145/3442188.3445916} uses a similar approach to what we are planning to use, it uses word embedding to calculate the similarity between Wikipedia and Baidu Baike, a Chinese version of Wikipedia. It identifies several target words that are mainly democratic political terms and propaganda. By calculating the cosine similarities, this paper deduces the differences in p-value and effect size of these words between Wikipedia and Baidu Baike. However, regarding our research goal, its main focus is on online encyclopedias, which are pre-written texts and lack user interactions. For our project, we are planning to apply a similar model to social media in China, in which the model provided by this paper will be helpful.

In \citeauthor{33531}, rather than research based on the gathering of data, the author’s investigation of the censorship coverage on different Chinese sites and magnitude may be a reference for our weighting of each site’s result during evaluation. Another interesting field that the authors employed within the paper is categorizing pornography. In fact, it is shown that there is a higher percentage of pattern matches with pornography keywords than political messages. We may also consider taking account of keywords relating to such.