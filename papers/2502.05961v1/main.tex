\pdfoutput=1 %  
\PassOptionsToPackage{table,xcdraw}{xcolor}

\documentclass[manuscript,screen,authorversion,review=false,timestamp=false,nonacm]{acmart}

% \documentclass[acmsmall,manuscript,screen]{acmart}
% \documentclass[acmsmall,nonacm]{acmart}

\usepackage{fancyhdr}
\AtBeginDocument{%
    \addtolength{\footskip}{2.0\baselineskip}%
    \fancyfoot[L]{\textit{Preprint.}}%
}

% \usepackage{tikz,lipsum,lmodern}
\usepackage[most]{tcolorbox}
\tcbuselibrary{listings}
\usepackage{booktabs}
\usepackage{graphicx}
\usepackage{colortbl} 
\usepackage{footnote}
% \usepackage[commandnameprefix=always, draft]{changes}
\usepackage[commandnameprefix=always, final]{changes}

% This changes the commands:
% \added → \chadded
% \deleted → \chdeleted
% \replaced → \chreplaced
% \highlight → \chhighlight
% \comment → \chcomment
% For the text you add you put it in this format: \added{This is new text.}

% For text you delete you put it in this format: \deleted{This text has been removed.}

% For text you replace you put it in this format: \replaced{new text}{old text}

% You can change the markup style (e.g. color) which we did with this command: \usepackage[final]{changes}
% % Use 'final' option to accept all changes
% \definechangesauthor[name={Varun}, color=orange]{vr} 
% 

% \usepackage[table,xcdraw]{xcolor}

\definecolor{codegreen}{rgb}{0,0.6,0}
\definecolor{codegray}{rgb}{0.5,0.5,0.5}
\definecolor{codepurple}{rgb}{0.58,0,0.82}
\definecolor{backcolour}{rgb}{0.95,0.95,0.92}

\lstdefinestyle{mystyle}{
    backgroundcolor=\color{backcolour},
    commentstyle=\color{codepurple},
    keywordstyle=\color{codepurple},
    numberstyle=\tiny\color{codegray},
    stringstyle=\color{codegreen},
    basicstyle=\ttfamily\footnotesize,
    breakatwhitespace=false,
    breaklines=true,
    captionpos=b,
    keepspaces=true,
    numbers=left,
    numbersep=5pt,
    showspaces=false,
    showstringspaces=false,
    showtabs=false,
    tabsize=2
}
\lstset{style=mystyle}

\usepackage[utf8]{inputenc}
\usepackage{csquotes}
\renewcommand{\mkbegdispquote}[2]{\itshape}
\usepackage{cleveref}
\usepackage{natbib}
\crefformat{section}{\S#2#1#3} % see manual of cleveref, section 8.2.1
\crefformat{subsection}{\S#2#1#3}
\crefformat{subsubsection}{\S#2#1#3}


\settopmatter{printacmref=false} % Removes citation information below abstract
\renewcommand\footnotetextcopyrightpermission[1]{} % removes footnote with conference information in first column
\pagestyle{plain} % removes running headers

%%
%% \BibTeX command to typeset BibTeX logo in the docs
\AtBeginDocument{%
  \providecommand\BibTeX{{%
    \normalfont B\kern-0.5em{\scshape i\kern-0.25em b}\kern-0.8em\TeX}}}


%% \BibTeX command to typeset BibTeX logo in the docs
\AtBeginDocument{%
  \providecommand\BibTeX{{%
    Bib\TeX}}}

\setcopyright{acmlicensed}
\copyrightyear{2025}
\acmYear{2025}

%% These commands are for a PROCEEDINGS abstract or paper.
% \acmConference[Preprint]{}{}{}
% \acmISBN{}


\usepackage{booktabs}
\usepackage[multiple]{footmisc}
\usepackage{threeparttable}
\usepackage{multirow}
\usepackage{array}
\usepackage{amsmath}
\usepackage{mathtools}
\usepackage{graphicx}
\usepackage{subcaption}
\usepackage{balance}
\usepackage{color}
\usepackage{wrapfig}
\usepackage{arydshln}
\usepackage{float}
\usepackage{hyphenat}
\usepackage{soul}
\usepackage{tabu}
\usepackage{multicol}
\usepackage{makecell}

\newif\ifdraft
\drafttrue %% turn on comments
% \draftfalse %% turn off comments
\ifdraft
        \def\vnr#1{{\color{blue}#1}}
        \def\sd#1{{\color{red}[SD: #1]}}
        \def\dc#1{{\color{green}[DC: #1]}}
        \def\amh#1{{\color{purple}[AMH: #1]}}
        
\else
        \def\vnr#1{}
        \def\sd#1{}
        \def\dc#1{}
        \def\amh#1{}        
\fi 

\definecolor{Purple}{RGB}{160,0,120}
\newcommand\siobhan[1]{\textcolor{Purple}{Siobhan: \@#1}}



\definecolor{CBGreen}{RGB}{0,128,0}
\newcommand\samantha[1]{\textcolor{CBGreen}{Samantha: \@#1}}

\definecolor{Blue}{RGB}{0,0,128}
\newcommand\shu[1]{\textcolor{Blue}{Shu: \@#1}}

\begin{document}

%%
%% The "title" command has an optional parameter,
%% allowing the author to define a "short title" to be used in page headers.
\title{The Human Labour of Data Work: Capturing Cultural Diversity through \textsc{World Wide Dishes}}



\author{Siobhan Mackenzie Hall}
\affiliation{%
  \institution{University of Oxford}
  \country{United Kingdom}
  }
\email{siobhan.hall@nds.ox.ac.uk}

\author{Samantha Dalal}
\affiliation{%
  \institution{University of Colorado Boulder}
  \country{USA}
  }
\email{samantha.dalal@colorado.edu}

\author{Raesetje Sefala}
\affiliation{%
  \institution{DAIR Institute, Mila, McGill University}
  \country{South Africa}
  }
\email{raesetje@dair-institute.org}

\author{Foutse Yuehgoh}
\affiliation{%
  \institution{Conservatoire National des Arts et M\'etiers (CNAM)}
  \country{France}
  }
\email{foutse.yuehgoh@devinci.fr}

\author{Aisha Alaagib}
\affiliation{%
  \institution{Independent researcher}
  \country{Sudan}
  }
\email{aalaagib@aimsammi.org}

\author{Imane Hamzaoui}
\affiliation{%
  \institution{École nationale Supérieure d'Informatique Algiers, New York University Abu Dhabi}
  \country{Algeria}
  }
\email{ji\_hamzaoui@esi.dz}

\author{Shu Ishida}
\affiliation{%
  \institution{University of Oxford}
  \country{United Kingdom}
  }
\email{shu.ishida@oxon.org}

\author{Jabez Magomere}
\affiliation{%
  \institution{University of Oxford}
  \country{United Kingdom}
  }
\email{jabez.magomere@oii.ac.uk}

\author{Lauren Crais}
\affiliation{%
  \institution{University of Oxford}
  \country{United Kingdom}
  }
\email{lauren.crais@law.ox.ac.uk}

\author{Aya Salama}
\affiliation{%
  \institution{Microsoft Egypt}
  \country{Egypt}
  }
\email{salamaaya@microsoft.com}

\author{Tejumade Afonja}
\affiliation{%
  \institution{CISPA Helmholtz Center for Information Security \& AI Saturdays Lagos}
  \country{Germany \& Nigeria}
  }
\email{tejumade.afonja@cispa.de}

%%
%% By default, the full list of authors will be used in the page
%% headers. Often, this list is too long, and will overlap
%% other information printed in the page headers. This command allows
%% the author to define a more concise list
%% of authors' names for this purpose.
\renewcommand{\shortauthors}{Hall et al.}

%%
%% The abstract is a short summary of the work to be presented in the
%% article.
\begin{abstract}
Humor is a social binding agent. It is an act of creativity that can provoke emotional reactions on a broad range of topics. Humor has long been thought to be “too human” for AI to generate. However, humans are complex, and humor requires our complex set of skills: cognitive reasoning, social understanding, a broad base of knowledge, creative thinking, and audience understanding. We explore whether giving AI such skills enables it to write humor. We target one audience: Gen Z humor fans. We ask people to rate meme caption humor from three sources: highly upvoted human captions, 2) basic LLMs, and 3) LLMs captions with humor skills. We find that users like LLMs captions with humor skills more than basic LLMs and almost on par with top-rated humor written by people. We discuss how giving AI human-like skills can help it generate communication that resonates with people. 

 
\end{abstract}

%%
%% The code below is generated by the tool at http://dl.acm.org/ccs.cfm.
%% Please copy and paste the code instead of the example below.
%%
\begin{CCSXML}
\end{CCSXML}

\keywords{}
\maketitle


The increasing reliance on LLMs for multimodal tasks across far-reaching sectors such as healthcare, finance, and manufacturing underscores the need to assess the accuracy and reliability of the information they generate. Vision-Language Models (VLM) have achieved state-of-the-art (SoTA) performance on Visual Question-Answering (VQA) benchmarks, and these models often utilize Retrieval-Augmented Generation (RAG) to maintain factual accuracy and relevance in a dynamic information environment. However, this has led to uncertainty in the information the LLM bases its answer on, as it may choose between parametric memory and retrieved sources. When models rely on memorized information instead of dynamically retrieving information, they may inadvertently propagate outdated or incorrect information, causing serious legal and ethical risks and undermining trust and reliability in AI systems \citep{huang2023survey}.
% The ability to strike a balance between generalization and specialization in AI systems is therefore crucial for ensuring the safe, reliable use of these technologies in real-world applications.

Despite these concerns, the way that Vision-Language models (VLMs) memorize and retrieve information, particularly in complex multimodal tasks, remains under-explored. Current research often focuses on either the general capabilities of large language models (LLMs) or the specialized retrieval mechanisms in retrieval augmented generation systems (RAG) \citep{incontext_rag,chen_murag_2022,liu_universal_2023}. Particularly in the context of multimodal retrieval and multihop reasoning, few studies analyze the tradeoff between finetuning for specialized tasks and zero-shot prompting for general-purpose vision-language capabilities. A lack of consensus on how to approach this tradeoff motivates the development of measures to quantify reliance on parametric memory, as well as metrics for quantifying the potential performance impact of extending LLMs with RAG systems.

To address this gap, we investigate how multimodal QA models balance accuracy with memorization on the WebQA benchmark. We compare finetuned multimodal systems against zero-shot VLMs, analyzing how retrieval performance influences QA accuracy. In particular, we focus on cases where retrieval fails, allowing us to measure reliance on parametric memory through two proposed metrics---the \ppr (\PPR) which quantifies how much model accuracy is influenced by retrieval quality, contrasting performance in best-case versus worst-case retrieval scenarios, and the \ucr (\UCR) which measures how often correct QA responses are generated when the retriever fails, providing a proxy for memorization.

To enable this analysis, we make several methodological contributions. For the finetuned QA models, we investigate Vision-Transformer (ViT) architectures, which allow for multihop reasoning over multiple sources. To investigate the impact of retrieval performance on trained LMs, we propose a variable-input Fusion-in-Decoder (FiD) model \cite{tanaka_slidevqa_2023, nlvr2}, building upon the VoLTA architecture \citep{pramanick_volta_2023}. For the zero-shot case, we build upon previous research on In-Context Retrieval \citep{incontext_rag} by demonstrating that LLMs such as GPT-4o are capable of performing the final ranking step of the retrieval process. In doing so, we find that GPT-4o, a general-purpose LLM, achieves SoTA performance on the WebQA task, outperforming existing finetuned RAG models by a significant margin (7\% higher accuracy). 

Crucially, our results reveal that while retrieval-augmented models reduce memorization, the training paradigm plays an important role. Finetuned models exhibit higher reliance on parametric memory, whereas zero-shot RAG approaches have lower memorization scores at the cost of accuracy. This suggests that while retrieval modules may mitigate the risks associated with outdated or incorrect information, SoTA performance requires that they be coupled with specialized QA models. Our memorization measures contribute to the development of transparent and reliable AI systems, particularly in applications where the sourcing of up-to-date, factual information is critical.



% We investigate the impact of question complexity on the ability of these models to integrate multiple data sources—such as images, text, and external retrievers—and produce coherent and accurate answers. We also explore whether in-context retrieval can be a viable alternative to traditional retrieval-augmented systems, offering a more streamlined approach to multimodal QA.

% To achieve this, we first compare zero-shot prompting multimodal LLMs with finetuned multimodal systems. We evaluate both types of models on the WebQA benchmark, a dataset designed for complex question answering that requires reasoning across both image and text sources. For the finetuned models, we use a Fusion-in-Decoder (FiD) architecture, which allows for multihop reasoning over multiple sources. Additionally, we introduce the concept of In-Context Retrieval Language Modeling (RLM), where the LLM itself performs retrieval tasks without the need for external retrievers. This method builds upon existing research in in-context learning  and aims to explore the viability of LLMs retrieving relevant sources and generating accurate answers directly from their context window.

% In order to investigate source utilization in finetuned multimodal models and LLMs, three lines of inquiry are established; 
% \begin{itemize}
%     \item Study 1: retrieval vs QA performance on webQA (motivating example, does QA answer correctly even with incorrect sources?)
%     \item Study 2: performance on adversarial examples where parametric knowledge would be incorrect by design
%     \item Study 3: improving performance on adversarial examples by fine-tuning (i.e model robustness)
% \end{itemize}

% Note, there is one weakness in this plan which is tying in the work we've already done. 
% If we added something from adversarial generation to the retrieval experiment (like a combination of study 1 + 3) it would be complete. So for instance we could try fine-tuning the retriever with adversarial examples (and not just the QA model)

% \begin{figure}
%     \centering
%     \includegraphics[width=0.95\linewidth]{figures/segmentation/webqa_segment_infill.png}
%     \caption{Example of the segmentation substitution pipeline from the WebQA task.}
%     % d5c76d760dba11ecb1e81171463288e9
%     \label{fig:seg_sub_pipeline}
% \end{figure}



% Retrieval augmented generation (RAG) with zero-shot prompting and fine-tuning Large Language Models (LLMs) have become the go-to methods for tasks relying on information retrieval and text generation. In many cases the LLMs parametric memory can sufficiently generalize to answer questions without being provided with retrieval mechanisms for out-of-domain knowledge. However, LLMs often hallucinate and provide wrong information in certain scenarios. This problem is amplified even further on open-domain Question Answering (QA) tasks involving multiple modalities. Grounded text generation using retrieved sources \citep{lewis2021retrievalaugmented} has been extensively studied for text-to-text QA tasks, but its application in multimodal settings has not been studied as much.


% Multimodal reasoning and question answering have gained prominence in recent research endeavors, with an increasing emphasis on handling various forms of data, particularly text and images. In this study, we address a specific gap in the existing literature by focusing on the development of a versatile multihop model capable of accommodating varying numbers of input images.

% Our motivation for this research lies in the growing complexity of answering questions using information on the web, where the challenge of navigating the open-domain setting is further complicated by the presence of multiple modalities and sometimes requires reasoning over multiple sources. WebQA is an ideal dataset on which to compare performance of finetuned RAG systems against general purpose LLMs; it is multimodal, with correct answers requiring reasoning over image and text sources. It is multihop, requiring a complex reasoning process over multiple sources. Finally, WebQA questions from different categories can be broken down into subdomains to analyze performance over domains of varying cardinality.

% Motivated by the real-world challenges of building retrieval and question answering (QA) systems, we design and finetune a closed domain, multimodal, multihop QA model, that is capable of reasoning over a varying number of sources taken as input from an external retriever module. This research contributes to the relatively underexplored domain of multihop reasoning across various input sources and modalities. Our goal is to explore the challenges posed by these scenarios and develop strategies that enable QA models to retrieve relevant information, conduct logical or numerical reasoning across diverse modalities, and generate coherent responses in natural language. To our knowledge, this is the first application of the Fusion-in-Decoder (FiD) architecture \cite{tanaka_slidevqa_2023, nlvr2} that is shown to work with a variable number of inputs, enabling multi-hop reasoning over sources.

% In-Context Learning refers to the ability of LLMs to perform any task by simply providing examples in the input prompt \citep{dong2022survey,min2022rethinking}. Inspired by this research, we propose a method to use the LLM itself as a multimodal retriever, potentially eschewing the requirement of a distinct retrieval module, thereby allowing the design of simpler retrieval-augmented QA systems. We dub this method In-Context Retrieval Language Modeling (RLM). To the best of the authors knowledge, In-Content RLM is disparate from other retrieval augmented approaches which utilize external retrieval modules \citep{incontext_rag,chen_murag_2022,liu_universal_2023}. Despite being a natural extension of In-Context learning, In-Context RLM has not yet been studied empirically.

% To expand on our contribution of In-Context Retrieval, this stems from the well-researched in-context learning of LLMs. In-context learning is the ability of a model to perform any task given a sufficient context window \citep{dong2022survey,min2022rethinking}. Such tasks could include retrieval and ranking, but typically, the go-to solution for tasks requiring retrieval has been RAG. To the best of the authors knowledge, In-Context Retrieval is distinct from In-Context Retrieval Augmented Language Modelling (RALM), and despite being a natural extension of In-Context learning, In-Context Retrieval has not yet been shown empirically.

% Finally, we explore the tradeoff between using zero-shot prompting LLMs and the fine-tuning approach. While we find that, overall, GPT-4o obtains SoTA performance on the WebQA task, outperforming the accuracy of existing finetuned RAG approaches by 7\%, finetuned approaches still perform better on more restricted subdomains\footnote{``In-Context RLM" @ \url{https://eval.ai/web/challenges/challenge-page/1255/leaderboard/3168}}. Finally, we validate that GPT-4o is relying on retrieval abilities to solve the task; we find that GPT-4o is capable of retrieving relevant sources in the presence of distractors and furthermore, when GPT-4o fails to retrieve correct sources, it answers incorrectly 75\% of the time, meaning that it is not relying on parametric memory for this task.

% \paragraph{Contributions}
% Based on our experimentation and analysis on the WebQA benchmark, we make the following contributions:
% \begin{itemize}
%     \item Propose a new architecture for multimodal multihop QA that takes variable number of input sources inspired by the Fusion-in-Decoder method.
%     \item Comparison of general purpose LLMs vs specialized models on the WebQA benchmark.
%     \item Observation of In-Context Multimodal Retrieval abilities of GPT-4o and that it does not rely on parametric memory for multimodal QA.
%     \item Analysis of relationship between retrieval and QA task performance.
%     \item Analysis of task and query complexity on the performance of retrieval and QA tasks.
% \end{itemize}
















% Throughout this paper, we will present our methodology, experiments, and findings, emphasizing our approach to multihop reasoning over varying numbers of input images. We believe that our work contributes to a deeper understanding of multimodal reasoning and has the potential to enhance the capabilities of question-answering systems in the intricate, multimodal landscape of web-based information.
\section{{Literature Review}}
\label{sec:lit-review}
We identify how bias manifests in ML systems, particularly in T2I generative models, as well as corresponding efforts within the ML discipline to mitigate these biases. We then turn to popular bias mitigation strategies in ML. Shifting to social computing literature, we trace the role of datasets as infrastructures in encoding and perpetuating biases. Finally, we identify how participatory approaches have been applied in ML to mitigate biases in datasets. We point to existing work in CSCW that documents the double-edged nature of participatory approaches: participation can be burdensome.  Our paper builds upon this foundation to demonstrate how achieving the benefits of participation (e.g., alignment and meaningful shifts in power) requires attending to the invisible labour of building infrastructures for participation.

\subsection{Bias in machine learning}
ML has many applications in technology used in the modern world, such as search engines, image captioning, and content generation. ML models, including those responsible for content generation, produce biased outputs~\cite{agarwal2021evaluating_bias,berg2022prompt_bias, luccioni2024stable_bias, caliskan2017semantics_bias, hovy2016social_bias}. The origins of biased outputs can often be traced back to biases reflecting societal structures—such as cultural norms, historical inequalities, and dominant power dynamics—that establish the normative values guiding dataset curation, model development, and evaluation practices~\cite{Blodgett2021_bias, caliskan2017semantics_bias, birhane2021misogyny, hall2024visogender_bias, WinogenderRudinger2018_bias}.
Many biases in ML manifest in the real world, with real-world implications. For example, many non-white Uber drivers report being locked out of their apps and unable to work when the facial recognition systems used by Uber to conduct random ``identity checks'' fail to recognise their faces~\cite{watkins2023face}. Model failures can occur due to insufficient evaluation before release and/or these real-world conditions being difficult to produce in an evaluation setting~\cite{dev2021genderexclusiveharms_bias,birhaneAIAuditingBroken2024,dengUnderstandingPracticesChallenges2023}.  

While there are many ways to define bias in ML (e.g.,~\cite{weidinger2021ethical, weidinger2023sociotechnical, mehrabi2021survey, katzman2023taxonomising, shelby2023sociotechnical}), in this work we are particularly concerned with the representational and quality of service harms as defined in~\cite{weidinger2021ethical, weidinger2023sociotechnical} and~\cite{shankar2017allocational, de2019doescvworkallocational}, respectively. Representational harms are evident in how people, groups, and their heritage are presented and perceived. Quality of service harms typically manifest downstream in a real-world circumstance (for example, someone prompting a T2I model at home) and stem from a physical manifestation of representational harm. For example, people with visible disabilities experience representational harms when T2I models are unable to create accurate and dignified representations of their physical bodies~\cite{mack2024they}. 

Previous studies have focused on bias at the axes of gender and occupation~\cite{berg2022prompt_bias, hall2024visogender_bias,WinogenderRudinger2018_bias}, race and gender~\cite{currie2024genderethnicity, west2024fieldgenderethnicity}, and the medical domain~\cite{gisselbaek2024beyondmedical, wiegand2024demographicmedical}. In our work, we follow calls by~\citet{weidinger2023sociotechnical} to expand the scope of bias investigations to other intersectionalities, such as the representation of cultural objects from diverse regions. The \textsc{WWD} project was therefore initiated in response to this call to investigate and mitigate ML biases in diverse contexts such as culture. Extensive work has been done to understand the scope in which cultural biases exist, and even to what ``culture'' can refer.~\citet{adilazuarda2024culturesurvey} unpack definitions of culture used in the extensive literature related to bias investigation in large language models. In this work, we use the term ``culture'' to mean ``cultural heritage'' as defined in~\cite{adilazuarda2024culturesurvey, blake2000defining}. We intentionally chose food as a lens into culture because food is a salient cultural artefact. Food is shaped by a region's history, geography, and even religious symbolism. Intangible cultural heritage is considered a ``mainspring of cultural diversity'' and the social practices and rituals around food are so deeply intertwined with culture that they are recognised as ``intangible cultural heritage'' under UNESCO's Convention for the Safeguarding of the Intangible Cultural Heritage.\footnote{https://ich.unesco.org/en/convention. This is the primary international legal instrument in the field.} 

In this paper, we are primarily interested in T2I models, a type of GenAI model that generates novel images based on a user prompt. GenAI is a term used to describe ML models that produce audio~\cite{borsos2023audiolm, kreuk2022audiogen}, visual~\cite{hu2024instructimagen, reed2016generativedalle}, and/or written~\cite{tonja2024inkubalm, achiam2023gpt, team2023gemini} content. This type of AI is contrasted with methods of image classification~\cite{deng2009imagenet, lin2014microsoftcoco}, prediction models (e.g., weather forecasting~\cite{lam2023weatherforecasting}), and reinforcement learning~\cite{sutton2018reinforcement} because of its ability to \textit{generate new content} based on learned associations from training data. We focus particularly on T2I models and their implications in contributing to cultural erasure through stereotyping, misaligned values, and inaccurate outputs.

\subsection{Mitigating bias in machine learning through crowdsourcing representative datasets}
Social computing scholars have identified the role of datasets in encoding bias that ML systems then reproduce~\cite{gebru2021datasheets,buolamwiniGenderShades,sweeney2013discrimination,scheuermanDatasetsHavePolitics2021,scheuermanProductsPositionalityHow2024,kapaniaHuntSnarkAnnotator2023,sambasivan2021everyone}. In recent years, fairness in ML researchers have made significant efforts to construct datasets that are more representative using crowdsourcing methods~\cite{singh2024aya_dataset,romero2024cvqa}. Crowdsourcing knowledge tends to take on two forms: a top-down model where raters and annotators are contracted through a centralised body~\cite{miceli2020between}, and a bottom-up model that directly engages community members in the data collection. 

Top-down models of crowdsourcing to build more representative datasets include work such as~\cite{bhutani2024seegull, jha2024visage,rojas2022the}, which pursued diverse and representative samples from the Majority World.\footnote{The term ``Majority World'' is a deviation from the more commonly used literary term ``Global South.'' We have chosen to use ``Majority World'' to highlight that the majority of the populations with which we engage come from these regions.} The ORBIT computer vision dataset~\cite{massiceti2021orbit} demonstrates how top-down models can engage in the ethical sourcing of data contributors, as it involves annotators from Enlabeler (Pty) Ltd,\footnote{https://www.enlabeler.com/} a company that empowers its employees by offering technical skill development alongside their data annotation work. While top-down data collection can be efficient and more easily implemented at scale, it excludes the many perspectives of people who do not work as data annotators~\cite{geiger2020garbage}. Moreover, top-down data collection efforts typically dictate to data contributors what an acceptable submission looks like, without room for bottom-up feedback~\cite{posada2022dispotif}. Contributors in top-down models can be flagged for fraudulent behaviour if the company overseeing data collection suspects that multiple people share an account, a common practice in households where there may only be one computer available~\cite{posada2022coloniality,jones2021refugees}. In contrast, bottom-up and community-based data collection broadens the range of people who can contribute and can empower more people not only to share cultural artefacts, knowledge, and expertise but also to help shape the parameters of their own participation~\cite{denton2021whose,delgado2023participatory,birhanePowerPeopleOpportunities2022}. We characterise bottom-up, community-based crowdsourcing as efforts that allow for any community member to engage (e.g., they do not need to be formally employed as a  data crowd worker), make space for community contributors to actively shape parts of the research process (e.g., they may take part in decisions about what kinds of data should be collected), and tend to---in large part---be volunteer-based.  

Community-based crowdsourcing has great potential to reach people in the community and to gather local knowledge. For example, the AYA dataset from Cohere's research lab~\cite{singh2024aya_dataset} presents a framework for large-scale community data collection. Their collaborators spanned 119 countries and were able to create a multilingual text dataset comprising 513 million instances across 114 languages. All their systems have been open-sourced. In 2020,~\citet{orife2020masakhaneoriginal} introduced Masakhane, which is an ongoing open-source, continent-wide, online research effort for machine translation for African languages. Masakhane has also consistently demonstrated ethical participatory research design for machine translation and natural language processing in Africa~\cite{nekoto2020masakhane, adelani2022masakhaner,ogundepo2023afriqamasakhane}. While efforts such as AYA and Masakhane make significant strides in producing datasets with diverse community-based input, both research efforts foreground the dataset as the primary artefact of interest. In our contribution, we aim to build upon the precedent set by bottom-up crowdsourcing efforts by calling attention to the processes and complexities---such as designing accessible infrastructure to support data collection and working with communities to build trust and rapport---throughout the data collection process by which these types of datasets are created. 

\subsection{Infrastructures for dataset construction}
Infrastructure is the foundation underpinning any large-scale system; it is the underlying substrate that is embedded in action, tools, and built environments and takes on significance in relation to organised practices~\cite{star1999ethnography,star201620,rice2006handbook}. Infrastructure can take the form of datasets~\cite{bowker2000sorting} that classify information and are subsequently used to create and reinforce categorisations. For example, FairFace, a large-scale labelled image dataset, is an infrastructure that has been used to determine what counts as a face in an image---but, crucially, relies on perceived gender and race labels~\cite{karkkainen2021fairface}. Datasets, like all infrastructures, are socio-technical; information does not naturally fall into classifications~\cite{bowker1994information,bowker2000sorting,star1999ethnography}. Instead, data must be produced by humans who make subjective decisions about what deserves to be counted and how it should be classified~\cite{d2024counting,denton2021whose}. This necessarily means that datasets are value-laden: they are imbued with the values of their designers and the larger social contexts in which they were created~\cite{rice2006handbook,star1999ethnography,scheuermanDatasetsHavePolitics2021,d2023data}. 

However, the value-laden nature of datasets often remains unseen until breakdowns---disruptions between expected and actual outcomes---occur~\cite{luccioni2023bugs, birhane2024into}. Critical computing scholars have demonstrated the value-laden nature of datasets for ML by leveraging \textit{infrastructural inversion} as an analytical tool to trace breakdowns back to incongruences between designers and users~\cite{scheuermanDatasetsHavePolitics2021,scheuermanHumanDataDataset2023,buolamwiniGenderShades}. For example, the Gender Shades study leverages infrastructural inversion to surface the social values and power dynamics underlying image datasets by tracing the poor performance of facial recognition systems on Black women's faces back to image datasets that lacked adequate and accurate representation of marginalised groups~\cite{buolamwiniGenderShades}. The lack of adequate and accurate representation of marginalised groups in these datasets reflects a lack of value for these identities---they were overlooked in dataset creation. 

Accounting for diverse values in dataset creation requires labour~\cite{scheuermanDatasetsHavePolitics2021,sambasivan2021everyone}. As~\citet{scheuermanDatasetsHavePolitics2021} points out, the data work that goes into developing a dataset is often ``silenced'', which results in a lack of incentive structures to center, support, and document the labour that goes into dataset creation. In our paper, we leverage infrastructure studies to call attention to the invisible labour that underpins the creation of large-scale systems and the subsequent implications for the \textit{kinds} of values that then get embedded into the resulting system. 

\subsection{The pitfalls and potentials of participation}
Recognising the limitations of top-down designed datasets, ML researchers have begun a \textit{participatory turn in AI} in an effort to construct datasets that represent a wider range of values and identities~\cite{delgado2023participatory,birhanePowerPeopleOpportunities2022,ParticipationScaleTensions,corbett2023power,suresh2024participation}. Participation is thought to enable better alignment of system performance with end-user expectations, as impacted communities would be involved in shaping the values of the ML systems~\cite{delgado2023participatory}. In other words, communities would be empowered as co-designers of the ML systems that impact them. However, the majority of participatory AI efforts fall short of their promises to support meaningful participation among stakeholders in the design process~\cite{ParticipationScaleTensions}. Researchers argue that inadequate attention is paid to shifting power away from systems designers, who are often the primary beneficiaries of improved model performance, and towards participants who could benefit by ensuring models accurately represent them and encode their values~\cite{birhanePowerPeopleOpportunities2022}. While participatory AI is a relatively nascent subfield within the ML community, participatory design is a research tradition that has existed for decades in social computing~\cite{asaro2000transforming,bodker2022can}. In our contribution, we draw upon lessons from participatory design in CSCW to demonstrate how these methods can be applied to achieve meaningful power shifts between researchers and communities. 

CSCW has a long and rich history of leveraging community-based participatory research (CBPR) methods to center socially marginalised groups' epistemologies in the design of technology~\cite{le2015strangers,sum2023translation,bratteteig2016unpacking,harrington2019deconstructing,bannon2018reimagining,pierre2021getting}. CBPR is used as a method to mitigate the power imbalances between researchers and communities by empowering community members to shape the trajectory and design of research projects~\cite{reasonSAGEHandbookAction2008}. For example,~\citet{asaro2000transforming} documents how, in light of impending technological changes in the workplace, workers themselves were invited to participate in the design of those technologies and thus had the power to shape the impacts of these technologies on their well-being.~\citet{harrington2019deconstructing} considers the role of research setting on community access, and therefore the accessibility of participation for targeted groups. In CBPR, practitioners strive to overcome their status as outsiders by building trust through affective and moral connections with community members~\cite{le2015strangers,mcmillan1986sense}. Taken together, CBPR literature overwhelmingly stresses the need for researchers to become imbricated with the communities which they are studying---through building these relationships, they become invested in the outcomes of the research from an insider, community member perspective. 

While participatory methods have the potential to give communities a greater say over the design of technology, many of these efforts fall short in meaningfully shifting power from the researchers to the researched~\cite{birhane2021misogyny,ojewaleAIAccountabilityInfrastructure2024,ParticipationScaleTensions}.  Participation takes work on behalf of communities.~\citet{pierre2021getting} documents how communities assume the ``epistemic burden'' of producing data about their lived experiences in participatory research efforts~\cite{pierre2021getting}. Finally, some scholars have critiqued the tendency of efforts to include socially marginalised groups as inherently ``othering''~\cite{epstein2008rise}. In our contribution, we draw upon the rich tradition of CBPR in CSCW literature to apply methods to build datasets more equitably. In particular, we attend to how research infrastructures can support the work of constructing the field site, building relationships with community members, and making room for meaningful community participation in the research design process~\cite{le2015strangers}. 

\section{Historical Origins of `Long-Tailed' Data in Machine Learning}\label{background}

T2I systems perform poorly when creating images of socially marginalised populations, such as people with disabilities or cultural objects from the Majority World~\cite{qadri2024,dasProvenanceAberrationsImage2024,mack2024they}. ML researchers attribute poor model performance on these topics to the \textit{long-tailed} nature of data about socially marginalised groups: models struggle to learn meaningful patterns and produce accurate, fair results because long-tailed data comprise a negligible portion of the training instances~\cite{zhang2023deeplongtailed, massiceti2021orbit}. 

Data from socially marginalised groups are less likely to be preserved in a digitised format and available on the Internet~\cite{Noble+2018}. Additionally, these data are systematically screened out of training datasets by state-of-the-art filtering models, such as CLIP \cite{radford2021learning}, which are biased towards including data from and about Western cultures~\cite{hong2024s}. Researchers use filtering methods such as CLIP to curate datasets in the first place to improve model quality by removing noisy, irrelevant, or potentially harmful content, aiming to optimise the relevance and accuracy of the data for the model’s intended use~\cite{fang2023data}. However, these filtering methods often encode and amplify existing biases, leading to datasets that inadvertently exclude non-Western cultures and marginalised groups. Ironically, while these filters are designed to protect against inappropriate or harmful content, they sometimes fail to remove problematic material such as foul language, racism, and harmful stereotypes~\cite{birhane2021misogyny}. This inconsistent filtering reinforces disparities in representation and increases concerns about the long-tailed nature of the datasets.

The root causes of the long-tailed nature of data about socially marginalised groups are intimately intertwined with social, political, and economic histories. In other words, the systemic inclusion from, and misrepresentative nature of, datasets is not solely a product of technical failure. Rather, it is a manifestation of societal values contributing to choices made about whose cultures are preserved, and whose are erased.~\cite{bowker2000sorting,cheney2017we,brubaker2011select,benthall2019racial,benjamin2019race}. For example, the exclusion and flattening of unique cultures from the African continent in digital media can be traced back to colonial regimes that sought to erase and thereby dehumanise the people living on the continent~\cite{faloyin_africa_2022}. During colonial times, power structures meant that the historical record was kept by Europeans colonising the continent~\cite{bowker2000sorting}. In an attempt to justify their actions, there was a systemic flattening and dehumanisation of the lived experiences of those they sought to subjugate on the African continent. Colonial classification regimes flatten important cultural and geographic differences~\cite{das2022,prabhakaran2022cultural,das2021}. Borders were drawn without regard for existing complex social and ethnic groups, and the subsequent oppression and power dynamics involved in recording history lent themselves to a systematic erasure of distinct experiences~\cite{bowker2000sorting}. More recent depictions of Africa in the international media and popular culture have made little attempt to capture its deeply complex and rich landscape~\cite{faloyin_africa_2022}.   

Acknowledging the core reasons for this type of flattening, \textsc{WWD} explicitly seeks to counteract it through on-the-ground community consultation and collaboration. The granularity required to capture a region's knowledge is most likely best understood through extensive consultation with communities with lived experiences in those regions. For example, borders may not provide the best guide for demarcating cultural boundaries, and a representation of “Kenya” or “Nigeria” may not be granular enough when we consider the distinct cultural groups within these borders. Some borders may also be misleading: for example, the Semliki River has changed course numerous times in the past few decades, meaning that some cultural groups have found themselves flipping between national identities of Uganda and the Democratic Republic of Congo, depending on a naturally evolving geographical marker~\cite{faloyin_africa_2022}.


We present the African continent as an example to explain the deep-rooted causes of and potential for the type of cultural erasure that exists across the globe. \textsc{WWD} and the infrastructure developed therein seeks to counteract this erasure by encouraging data submissions from around the world in a granular manner, going beyond national representation but explicitly requesting fine-grained regional details. These details and nuances can be easily overlooked without community expertise. The individuals that we engaged are reached primarily through social networks. In the coming sections, we expand on this process and share insight into how submissions from different regions within a country were encouraged.
\section{Methods and Analysis}\label{sec:methods}
Our study aims to map out the process of building a dataset of cultural objects---specifically food---through a bottom-up, community-led approach. 

Planning, building, and deploying the system that resulted in the \textsc{WWD} dataset was a process that occurred over 12 months. In November 2023, the first and seventh through ninth authors of this paper decided to build \textsc{WWD} to investigate how cultural bias in ML manifests in generated images of food. The research activities conducted as part of the development of \textsc{WWD} were approved by the first author's IRB. All the authors of this paper, except the second author, were involved with the development of \textsc{WWD}. These authors took extensive field notes about their interactions with data contributors throughout the process. After the process of building \textsc{WWD} concluded, the authors shared field notes and post-mortem reflections with the second author, who is a social computing scholar. The second author was invited onto this project by the first author, who wanted to engage with the social computing field to highlight the processes underpinning the development of the \textsc{WWD} dataset. 

We also present a system overview of the \textsc{WWD} technical infrastructure in~\Cref{sec:system} to give readers a better understanding of how the data collection process functioned. Below, we provide more detail on the researchers who provided reflections on the development of \textsc{WWD}, our positionality, and our analysis methods. 

\subsection{The original \textsc{World Wide Dishes} research team}
All authors, except the second author, were members of the original \textsc{WWD} research team. The original research team comprises 11 researchers from academia and industry and one independent researcher (12 total). The large majority of the original research team is from countries in Africa, has spent the majority of their lives on the African continent, and remains deeply involved in the African ML community (see~\cref{tab:og_wwd_team} for an overview of team members and their respective positionalities). Of the 12 members of the original \textsc{WWD} research team, nine participated in the post-mortem reflection process for this paper. 

\begin{table}[h]
\centering
\renewcommand{\arraystretch}{1.5} % Adjust the spacing between rows (1.5 times the default)

\caption{\small \textbf{The original World Wide Dishes team, and their corresponding positionality.}}
\label{tab:og_wwd_team}
\small % Reduce font size to make the table more compact
\begin{tabular}{p{1cm}p{3cm}p{3cm}p{2.5cm}p{2cm}p{1.5cm}} % Set custom column widths
\toprule
\textbf{Code}& \textbf{Background}  &  
\makecell[l]{\textbf{Involvement in the} \\ \textbf{African ML community}} & 
\makecell[l]{\textbf{Research} \\ \textbf{Institution}} &
\makecell[l]{\textbf{Community} \\ \textbf{Ambassador}} &
\makecell[l]{\textbf{Duration of} \\ \textbf{involvement}} \\ 
\midrule
O1  & Health Sciences and Machine Learning & Deep Learning Indaba  & University of Oxford & South Africa & 12 months \\
O2  & Law and Cultural Heritage Studies & -- &  University of Oxford &United States & 11 months  \\
O3  & Engineering and Machine Learning & --  & University of Oxford & Japan & 11 months  \\
O4  & Computer Science and Machine Learning & Deep Learning Indaba & University of Oxford & Kenya & 11 months \\ 
O5  & Mechanical Engineering and Computer Science &  AI Saturdays Lagos, Deep Learning Indaba & CISPA Helmholtz Center for Information Security & Nigeria & 8 months \\
O6  & Machine Learning & Deep Learning Indaba, Deep Learning IndabaX Egypt  & Microsoft & Egypt & 8 months \\
O7  & Mathematics and Computer Science & Deep Learning Indaba  & Independent Researcher & Sudan & 8 months \\
O8 & Mathematics and Computer Science & Deep Learning Indaba, Deep Learning IndabaX Cameroon, KmerAI  & Conservatoire National des Arts et Métiers & Cameroon & 8 months  \\
O9  & Machine Learning & Deep Learning Indaba, African Computer Vision Summer School  & DAIR, McGill \& MILA & South Africa & 7 months \\
O10  & Computer Science and Machine Learning & Deep Learning Indaba, Deep Learning IndabaX Algeria  & École nationale Supérieure d’Informatique Algier & Algeria & 7 months \\ 

\bottomrule
\end{tabular}
\end{table}

\subsection{Positionality}
We interrogate our positionality as researchers to contextualise how we approach the design and development of \textsc{WWD} as well as how we interpreted data from field notes and post-mortem reflections. Nine of the ten authors on this paperwork are in the computing space, with all of the authors except for the second and eighth working in the ML field. The second author was brought onto this project for their experience in writing for social computing venues and the eighth for their extensive knowledge in cultural heritage studies. All the authors, with the exception of the second, sixth, and eighth, are deeply engaged in the African ML Community. The first, third through sixth, and ninth and tenth authors are all from countries in Africa and have only recently (e.g., in the past few years) left the continent to pursue their under- and postgraduate studies. These authors maintain strong ties with their communities in their country of birth and tap into those community networks to collect the data that comprises the \textsc{WWD} dataset. 

We acknowledge that our cultural identities fundamentally shaped how we approached building \textsc{WWD} and coloured our analysis of post-mortems. Many of us acted as gatekeepers to the communities where data contributors resided. We actively shaped the data collection process for \textsc{WWD} to be compatible with the cultural norms and values we, as members of these communities, knew were important. Simultaneously, we maintained open channels of communication to engage with feedback from the community to allow us to reflect on the values. Our reflections are impacted by our identities as both members and researchers of these communities.  

\subsection{Analysis}
All the authors of this paper, except the second author, provided field notes they took throughout the development of \textsc{WWD} and post-mortems, in the form of audio recordings and written reflections. The authors created a shared document with all the field notes and post-mortems and collaboratively coded the data using an approach informed by grounded theory. The first and second authors coded field notes and reflections from the three of the team members. Then, they conducted a card-sorting exercise to arrive at an initial set of high-level themes. The first and second authors shared the high-level themes back with the remaining authors on the paper to get their input and agree upon a coding scheme. Finally, the first and second authors coded the remainder of the field notes and post-mortems according to the agreed-upon coding scheme to result in the four main themes we present in \Cref{sec:rollout}.

Because \textsc{WWD} is a socio-technical system, we also provide an overview of the digital infrastructure underpinning the data collection process in \cref{sec:system}. We then illustrate the human labour involved in operationalising this digital infrastructure to build a dataset of cultural objects from the bottom up in \cref{sec:rollout}. To answer our primary research question, \textit{How can researchers collect cultural data using a bottom-up, community-led approach?}, \textbf{we present four dimensions of data work: (1) building trust with communities, (2) making participation accessible, (3) recognising the produced nature of data, and (4) attending to the relationships between culture and food}.


\section{\textsc{The World Wide Dishes System}}\label{sec:system}

The \textsc{World Wide Dishes System} is an infrastructure for building the \textsc{World Wide Dishes Dataset}. The \textsc{WWD Dataset} consists of local dishes from around the world built with input from community Contributors who interact with the \textsc{WWD System} by sharing personal, local knowledge of the dishes they attribute to their own home(s) and culture(s). For each dish, we include the name of the dish (in both the local language and English), the country of origin, the region of origin, the associated culture, the time of day at which the meal is eaten, the type of meal, the utensils used, the drinks that accompany the meal, any special occasions when the meal is eaten, the ingredients, the recipe, and the image of the dish if available. 

In this section, we describe the stakeholder groups who interacted with the \textsc{WWD System}. Interactions included the construction of the underlying architecture as well as the \textsc{WWD Dataset}. 

\subsection{How stakeholders interact with the \textsc{World Wide Dishes System}}
Three key stakeholder groups interact with the \textsc{WWD System}: Core Organisers, Contributors, and Community Ambassadors. The stakeholder groups are not mutually exclusive; for example, some Community Ambassadors were also Contributors. See~\cref{tab:stakeholder_tally} for an overview of the stakeholder groups, participant counts in each group, and geographic regions represented within each group. Below we describe how each stakeholder group interacted with \textsc{WWD} and with each other. 
\begin{table}[h]
\centering
\renewcommand{\arraystretch}{1.5}

\caption{\small \textbf{Stakeholders.} We indicate the demographics for the stakeholders associated with the projects.}
\label{tab:stakeholder_tally}
\small
\begin{tabular}{p{2cm}>{\footnotesize}p{3.5cm}p{2cm}p{3.5cm}p{1.5cm}}
\toprule
\textbf{Stakeholder} & \textbf{Role} & \textbf{Participant count} & \textbf{National identities by continent} & \textbf{Age range}\\ 
\midrule
Core Organisers & the central organising team that coordinated the development and execution of World Wide Dishes. Core Organisers were also Community Ambassadors and Contributors & 12 & Africa, Asia, North America & - \\
Contributors and Community Ambassadors & These roles often overlapped, with Community Ambassadors supporting amplification of the WWD and Contributors submitting dishes & 162 & Africa, Asia, Europe, North America, South America, Oceania & 19-62; Mean=31.5$\pm$8.78 \\
\bottomrule
\end{tabular}
\end{table}

\textbf{Core Organisers:} Began the initiative to build \textsc{WWD}. They created the data collection process, front-end \textsc{WWD} project website, and back-end database, as well as the data processing pipeline. 

\textbf{Contributors:} Accessed the \textsc{WWD} data collection form via the \textsc{WWD} project website and provided cultural knowledge about each dish submitted through the form. Contributors also provided feedback on other entries by other Contributors.

\textbf{Community Ambassadors:} Distributed the \textsc{WWD} webpage to and through their social networks to solicit contributions. Community Ambassadors hosted focus groups and sometimes filled out the data collection form on behalf of Contributors who faced issues with submitting the form themselves. Community Ambassadors assisted Contributors with translations if needed to ensure submissions were in English. Community Ambassadors often acted as Contributors, submitting dish information to the \textsc{WWD} data collection form. Additionally, Community Ambassadors served as a communication channel between the Contributors and Core Organisers, surfacing Contributor concerns as they arose and communicating subsequent actions by the Core Organisers back to the Contributors. Finally, Community Ambassadors assisted in the data processing stage.


\subsection{{\textsc{World Wide Dishes System}} architecture}
The \textsc{WWD System} is a web application (\Cref{fig:front_page}) built with Django\footnote{https://www.djangoproject.com/} and hosted on Google Cloud Platform, with a PostgreSQL database for storing submissions and Google Cloud Storage for storing image files. \Cref{fig:WWD Architecture} presents an overview of the system components. The backend processes which handle data collection, storage, and processing are described below.

\begin{figure}[t]
    \centering
    \includegraphics[width=\textwidth]{figures/WWD_Architecture.png}
    \caption{\textbf{Overview of the World Wide Dishes flow}. (A) A Contributor accesses \textsc{WWD} through a web browser. They consent to be a research participant and decide whether to create an account or proceed as a guest. They then fill out the data collection form with information about themselves and the dish they submit.  (B) The submission is then stored in the \textsc{WWD} database. We store Contributors' information separately from the dish information to preserve their privacy. (C) The full database containing dish information is then used to operationalise bias investigations into generated text and image content using other vision-language and language models. (D,E) A report then is generated from the automated bias testing and the survey responses and then made public. \textit{All icons in this figure were downloaded from Flaticon. For proper attribution please see \ref{asec:informed_consent}.}} 
    \label{fig:WWD Architecture}
\end{figure}

 

\begin{figure}[t]
    \centering
    \includegraphics[width=\textwidth]{figures/front_page.png}
    \caption{\textbf{Front page of the World Wide Dishes website}. Two call-to-action buttons were placed at the top of the page to encourage participation from site visitors. The ‘Check out our food leaderboard’ button takes site visitors to the leaderboard table, while the ‘Add your local dish’ button guides visitors through the data entry flow.}
    \label{fig:front_page}
\end{figure}

\subsubsection{Data collection}
The questions asked in the data collection form were collated in consultation with experts on food and cultural heritage, as well as feedback from community members, and a desire to capture as much metadata as possible to highlight regional differences in dishes. Food as a cultural object moves easily across borders and therefore the same dish may exist in more than one place---but the customs or range of ingredients associated with it may differ with changes in geography or the passage of time. The questions were therefore designed to give Contributors the opportunity to provide as much local nuance as possible. We present a list of data collected below.

\begin{enumerate}
    \item Data collection involved an online form (see~\cref{asec:protocol-screenshots} for the questions and~\cref{tab:dish_questions} for the data points).
    \item Request to upload a non-generated image of food from personal records (shared with consent to distribute for research purposes). 
    \item Questions about food. 
    \item Submission of dish name was encouraged in the local language, with a translation / phonetic equivalent included.
    \item English was encouraged, but concession was made if a translation for an ingredient didn't exist/wasn't known, e.g. cassava as an ingredient / sadsa as a dish.
    \item Associated customs and information about the dish were collected: time of day eaten, utensils, ingredients, associated celebrations/events, online recipe, freeform general information.
\end{enumerate}

\begin{table}[h]
\centering
\caption{\small Data collected through the  \textbf{World Wide Dishes} contribution form}
\label{tab:dish_questions}
\begin{tabular}{ll}
\toprule
\textbf{Question} & \textbf{Data entry} \\
\midrule
Image and caption & Image upload and text caption \\
Dish name in a local language & Short text \\
Name of local language & Short text \\
Country / countries & Dropdown and free text \\
Region & Free text \\
Attribution to a specific cultural, social, or ethnic group & Free text \\
Time of day eaten & Multiple choice \\
Dish classification & Free text or dropdown \\
Components, elements, and/or ingredients & Dropdown and free text \\
Utensils & Free text \\
Accompanying drinks & Free text \\
Association with a special occasion & Multiple choice \\
Recipe & URL \\
Any other comments & Long-form free text \\
\bottomrule
\end{tabular}
\end{table}

\subsubsection{Recruitment and engagement strategies}
The system is built for easy sharing on social networks, prioritizing cross-browser compatibility, responsiveness, and a lightweight design to ensure accessibility on a wide range of devices. All information about the project and the data collection is available on the website. Recruitment involved many social networks, including through existing communities such as AYA,\footnote{https://aya.for.ai/} Masakhane,\footnote{https://www.masakhane.io/} the Deep Learning IndabaX network,\footnote{https://deeplearningindaba.com/2024/indabax/} AI Saturdays Lagos,\footnote{https://aisaturdayslagos.github.io/} and OLS.\footnote{https://we-are-ols.org/} Posts were also placed on a mailing list to encourage engagement.

Engagement was explicitly encouraged through the promotion of a submission leaderboard (see~\cref{fig:leaderboard}),
which displayed the number of dish submissions and contributors received from each region. This was done in an attempt to gamify the experience and build excitement and fun. The leaderboard was promoted during recruitment outreach. 

\begin{figure}[t]
    \centering
    \includegraphics[width=0.6\textwidth]{figures/WWD_Leaderboard_new.png}
    \caption{\textbf{Screenshot of the World Wide Dishes  Learderboard} showing the top 10 countries by number of dishes, the total number of contributed dishes and contributors per country.}
    \label{fig:leaderboard}
\end{figure}

\subsubsection{Data storage}  The collected data is stored in two sections: the dish data, which has been made public and is completely anonymised, and the personal data, which has been collected and stored according to the terms agreed by the ethics review. We collected names, ages, and national identities, and also accepted anonymous submissions. Names were only used when explicit consent was given to acknowledge contributions publicly. Age and national identity data were required to help us understand the demographics of people submitting to \textsc{WWD}. Approval for data collection and the subsequent research study was obtained from the Departmental Research Ethics Committees of the Computer Science Department at the University of Oxford (reference: CS\_C1A\_24\_004). Notably, the ethics review required us to collect age data, as Contributors had to be over the age of 18 to participate; age data was therefore collected even if the Contributor otherwise submitted dish data anonymously.

\subsubsection{Data processing}

\begin{enumerate}
    \item \textbf{Data cleaning} Core Organisers lead the data cleaning process to remove duplicate entries within countries and to standardise data entry (i.e. uniform descriptions of regions in a country and language). Core Organisers consulted Community Ambassadors when needed.
    \item \textbf{Translations} Entries were primarily in English, except for some dish names and ingredients without known English translations. For submissions made by French-speaking Contributors from the Democratic Republic of Congo, and to make concessions for accessibility, the Core Organisers accepted these specific entries and translated them with an open source machine translation system and a Core Organiser who is a native French speaker audited these results.
    \item \textbf{Removal of images with uncertain licences} by Core Organisers involved the removal of any uploaded images whose licence could not be verified. Accepted images came from royalty-free sites that did not prohibit their use in machine learning, had an accompanying Creative Commons licence, or had been taken and submitted by a Contributor with explicit permission given for it to be used for research purposes. 
    \item \textbf{To augment the image data}, Core Organisers solicited additional royalty-free and/or Creative Commons images of the submitted dishes from the Internet and consulted Community Ambassadors for their assessments of the images' accuracy in depicting the submitted dishes.
    \item \textbf{Inconsistencies in submitted data were handled by consulting Community Ambassadors} where possible, and Community Ambassadors consulted with other Contributors as needed. However, we aimed to collect lived experiences around the world and so we did not make any efforts to police a `ground truth' for each dish, or assign an `origin' or `authenticity' to any dish. Multiple nations or cultural groups can share the same dishes, for a wide variety of reasons including historical trade routes, war and occupation, the redrafting of political borders without regard to cultural borders, and migration patterns. Our task was not to make judgments about cultural `ownership' of the submitted dishes. Instead, the data collection asked participants to offer dishes from their own lived experiences and personal backgrounds, so we expected to see similarities across borders, as well as some reasonable regional and preference variations of the ingredients of dishes within and across borders.
\end{enumerate}




























 \section{Building \textsc{World Wide Dishes}: Lessons from the field}\label{sec:rollout}



Building the \textsc{WWD Dataset} was a process that took over seven months, and involved 12 Core Organisers and more than 170 Community Ambassadors and Contributors. Some of the Core Organisers also acted as Community Ambassadors (see~\cref{tab:stakeholder_tally} for the regions for which each Core Organiser served as a Community Ambassador). The process of launching the \textsc{WWD} data collection effort involved a beta-testing stage to solicit an initial round of feedback from community members about the data collection instrument and process, as well as hosting ongoing focus groups with Contributors to surface issues with data donations. Through systematic analysis of post-mortems and field notes from the Core Organisers, we identify four key elements of the process to build \textsc{WWD} that ultimately made this a successful effort: trust within the community, accessibility of participation, attention to the production of data, and understanding the relationship between food and culture. Below, we reflect on and surface how the \textsc{WWD} dishes project supported each element.  

\subsection{Building trust and leveraging existing social networks to access community members}
Collecting long-tailed data requires accessing and connecting with underrepresented populations who tend to be ``hard to reach.''\footnote{We use quotation marks to draw attention to the fact that the term ``hard to reach'' is value-laden. Calling a population ``hard to reach'' redirects attention away from the reasons why that population may be difficult for the researcher to access and instead focuses attention on the group as the problem, rather than on the institution or researcher as the potential problem. For more, see~\cite{benjamin2016informed,epstein2008rise}.} Obtaining this access typically requires working with community gatekeepers who can serve as intermediaries between the researchers and community~\cite{le2015strangers,epstein2008rise}. However, the presence of a community gatekeeper in a project does not in itself resolve ethical questions around shifting the distribution of power from the researchers to the researched. In building \textsc{WWD}, we made an intentional choice to involve community gatekeepers (e.g., Community Ambassadors) as members of the research team who shaped the data collection process, the development of the data collection tool, and the paper authorship process.  

The first author, who is a researcher at a prestigious Western university, recognised that their positionality put them in a position of power as relates to the collection of cultural data from communities to which they do not belong. While they are originally from South Africa and have been deeply involved in community efforts within the African ML community, such as the Deep Learning Indaba, they felt that the \textsc{WWD} project must be co-led by members from communities who were contributing data. Therefore, they invited three people in their professional network who were active researchers and volunteers for local AI communities to be co-leads and co-authors on the project. Two of the three people, O5 and O7, are Nigerian, and Sudanese, respectively, and were experienced in conducting data collection efforts with communities within their home countries. As O5, O6, and O7 spread information about \textsc{WWD} to their networks, additional community members representing more than eight countries on the African continent signed up to be Community Ambassadors for the project. Community Ambassadors leveraged their positionality, relationships, and social capital to make data collection with communities in Africa possible. 

Community Ambassadors reflected on how their previous work within the African community and being members of the local communities from which they were soliciting data donations was crucial to the success of \textsc{WWD}. In post-mortem reflections, O5 shared:
\begin{quote}
    ``As a community leader, I've been organizing free classes for Community Y\footnote{Community name is anonymised for review.} for over six years. Over time, this commitment has built trust within my community, reassuring people that I genuinely prioritise our shared interests. I would like to think that this trust helped community members feel comfortable contributing to the effort, knowing they aren’t being taken advantage of.'' O5
\end{quote}
Community Ambassadors had established reputations within the communities from which they were soliciting data donations, enabling them to collect cultural data alongside community members who trusted their intentions. 

Established methods of accessing data donors, such as cold-calling, or simply distributing the call to participate on mass broadcast social media channels (X, formerly known as Twitter; LinkedIn), were arguably less successful, even when it was the Community Ambassadors themselves who posted the call. Access to, or the ability to access, potential donors was insufficient. Instead, Community Ambassadors often had to rely on the relationships they had already established to solicit contributions. For example, O6 reported in their post-mortem that: ``So, I kept sharing and sharing over social media ... but the contributions were not growing as much as I hoped ... So I reach out to people individually.'' As a result, Community Ambassadors tapped into their network of friends and family to solicit data contributions. However, reaching out to personal networks was not without risk. 

Data contributions were not \textit{financially} remunerated. Therefore, some Community Ambassadors shared that the lack of financial compensation made them selective in who they reached out to about the project, as they were essentially asking their friends and family to do volunteer labour. However, despite the lack of remuneration, Community Ambassadors reported that many of the people to whom they reached out shared a common desire to participate in a project that could make GenAI systems work for them:
\begin{quote}
    ``It's about strengthening African machine learning, it's about empowering Africans, it's about making sure that all of the AI technology works for us all, and ensuring we're part of the builders of it... so maybe through our several grassroots machine learning efforts, we can get to some kind of balance in [power] regarding whose perspectives shape the building of AI technologies.'' O5
\end{quote}

Community Ambassadors controlled how they, and by proxy the larger \textsc{WWD} team, interacted with their communities. For example, the Core Organisers and Community Ambassadors administered WhatsApp groups within the larger \textsc{WWD} WhatsApp community, where they carried out conversations with Contributors in the language of the region they oversaw. Community Ambassadors had complete oversight over how they wanted to structure conversations within their WhatsApp group and enforce moderation rules that were proposed by the Core Organisers. 

Community Ambassadors had a shared context with the people whom they asked to contribute data to the project: they spoke the same language as the Contributors, had previously established themselves as trustworthy, and shared similar goals for African leadership in ML projects. Building on this, we expand on the need to meet community members where they are and make participation accessible.

\subsection{Make participation accessible}

Engaging in bottom-up, community-led data collection requires extra efforts on the part of researchers to make participation accessible. In this project, the researchers did so by providing a more accessible informed consent process and hosting ongoing office hours to help Contributors learn how to participate and understand the goals and motivations of the \textsc{WWD} project. It is not enough to build a data contribution form that is ``easy'' to fill out; rather, researchers must invest in significant amounts of human labour to animate the data collection effort. 

During beta-testing, Community Ambassadors flagged that while the consent form was accessible---in that it could be translated and read in the local language of contributors---it was not understandable. The research team had originally presented the entire consent form, as approved by their institution's ethics review board, as the first step of the data contribution process on the \textsc{WWD} website. Following feedback from Community Ambassadors, the research team co-designed a concise consent form (see \ref{fig:informed_consent_new}) that laid out key information about the project's purpose, procedures, and data privacy protections. The full consent form was then linked on the consent overview page (see~\cref{asec:informed_consent}). Upon reflection, O1 recounted that consent forms designed and approved in Western, \textit{academic} contexts were inaccessible for the regions in which the Contributors were located. The researchers had to engage in translation of the meaning of certain terms, such as ``cookies'', to make informed consent accessible for all participants. 

Collecting image data for ML purposes, as \textsc{WWD} does, is a complex process due to concerns about image ownership. Because \textsc{WWD} was intended to be an open-source dataset with Creative Commons licensing, all submitted images either had to have a Creative Commons license (if it wasn't an original) or be an original (e.g., taken by the contributor) photo shared with explicit permission that it could be used for research purposes. These guidelines were presented clearly on the data contribution form; however, Community Ambassadors noticed early on that Contributors were uploading images from the Internet that did not have Creative Commons licenses. As a result, the Core Organisers and Community Ambassadors began hosting regular ``teach-ins'' during office hours for Contributors to attend in order to learn how to properly make a data donation. O6 explained the role of office hours in making data contribution guidelines accessible to participants:
\begin{quote}
    ``The [office hours] helped ensure our data collection met our requirements and that contributors understood both the licensing requirements and how to properly complete the submission forms, enabling us to collect exactly what we needed for the project'' O6
\end{quote}
Office hours lowered the barrier to entry for participation by providing a space for Contributors to get help filling in the information on their submissions. Community Ambassadors recalled how Contributors would bring ideas about dishes they wanted to submit, but needed help finding an image they could use or verifying other data fields, such as utensils used and/or associated cultural ceremonies (as examples), for a dish. In these cases, the office hours attendees would work together to complete a submission collaboratively. 

Office hours also served as a space for Contributors to shape the research process. Office hours attendees were encouraged to share feedback about challenges they faced in making data donations, which the Community Ambassadors then brought to planning meetings with the Core Organisers. As a result, the Core Organisers amended the data collection instrument, produced additional guidelines for participation, and refined their recruitment strategy. O1 recalls how the team wanted to follow so-called ``professional protocol'' and communicate through official and professional media such as mailing lists and social media. The members of the community immediately asked for WhatsApp groups. 
\begin{quote}
    ``As a team, we tried to think through the implications---the most poignant being the lack of a boundary between our own work and professional lives by engaging with our personal WhatsApp applications. However, having already identified how important it was for us to meet the community where they were, and having had similar calls in the past through our work with established communities on the African continent, we made the commitment to host the community groups on WhatsApp, and maintain open communication between ourselves to make sure we could sustainably engage throughout the data collection process.'' O1 
\end{quote}
Building on the feedback received from Community Ambassadors and Contributors, the Core Organisers quickly realised how much behind-the-scenes effort was going into a single submission. Data are found through extensive consultation and effort, and the final entry on the website is a reduced form of the rich engagement that led to it. 

\subsection{Recognise that data are produced, not simply found}

Throughout the process of building \textsc{WWD}, it became clear that cultural data is something that is produced through social interaction. The concept of data production is not new~\cite{bowker2000sorting,d2024counting}. However, it is often glossed over in ML papers that present novel datasets while only briefly discussing the data work~\cite{scheuermanDatasetsHavePolitics2021} that goes into constructing a dataset. In our findings, we disrupt the assumption that a single participant in a crowdsourcing effort can submit complete information about an object of cultural significance and shed light on the myriad social interactions underpinning individual data contributions.

Contributors actively engaged in conversations with family and friends, accessing familial networks to collect the data needed for a submission. For example, O8 shared one Contributor's anecdote in their post-mortem where the Contributor asked their friends and family to help complete data submissions:
\begin{quote}
    ``For the photos and information I couldn’t get myself, I asked certain resourceful people, particularly my parents and friends, to provide the photos and information about the utensils to use and the appropriate time to eat certain meals'' O8, \textit{summarizing what a Contributor told them}
\end{quote}
While it could be argued that Contributors could simply look up dish information on the Internet to fill out their forms, this was not a viable option for many of our Contributors coming from countries where information about their regional cuisines was simply not available online. Familial networks proved to be a more reliable way to get accurate information about local cuisine. O8 recounted an experience where a Contributor in the community they oversaw encountered difficulty finding information about a regional Cameroonian dish they had grown up eating:
\begin{quote}
    ``For the dishes I struggled to explain, I turned to various sources for help. When possible, I looked online to see how they were made. However, for some dishes, it was difficult to find accurate information. In those cases, I reached out to my mother and grandmother, who were able to share their expertise and guide me through the process'' O8, \textit{summarising what a Contributor told them}
\end{quote}
Community Ambassadors and Contributors alike often turned to their familial networks, especially parents and grandparents, to produce the information they needed about a dish through conversations and glimpses into the archives of family recipes. O5 recalled that during their office hours, Contributors shared stories about calling their mothers and sisters to get help filling out information for a dish: ``Some [Contributors] mentioned that they reached out to their family to learn more about the native dishes they grew up eating because they realise they don't know how to prepare them.''  

Data about food, and how it is connected to a community's culture, is not simply sitting somewhere, waiting to be harvested. Moreover, in the regions in which we were operating, there was little to no reliable information about cultural cuisines available online. As a result, Contributors and Community Ambassadors produced this data through conversations with kin. Food is deeply intertwined with cultural practices and therefore some of our Community Ambassadors chose to ground their data collection efforts in major cultural events that coincided with the data collection phase. 

For example, O6 shared how they deliberately chose to revive their call for data contributions during Ramadan and Eid al-Fitr. Ramadan and Eid al-Fitr, which are major holidays for Muslims, are often celebrated with special dishes holding cultural significance that are only prepared during the holiday period. O6 explained: ``Food plays a great, very important role in Ramadan and so I was very intentional about just taking pictures of whatever food we made for iftar which is the breaking of the fast.'' The food that is prepared during these holidays holds cultural significance and helps tell a part of the story about a community's values and history. 

Communities are not passive data sources; these data about food were produced through social interactions and lived experiences. 

\subsection{Understand the relationships between food and culture}

Food is complex: dishes often harbour a deeper cultural significance. For example, O6 recalled how during their experience managing data contributions from Egypt, they realised that the classification of ``Egyptian food'' was fuzzy at best:
\begin{quote}
    ``[WWD] made me more conscious about \textbf{what is} Egyptian Cuisine and the influences from other cuisines that have actually, you know, shaped what we eat and what we consider as Egyptian. As it turns out, it's very intersectional.'' O6, emphasis added
\end{quote}
For O6, capturing data about Egyptian cuisine led them to further investigate the origins of their own cultural dishes. In the weeks following the data collection effort, they attended a session about "food as cultural currency" at the RiseUp Summit Egypt 2024 to learn more about the Egyptian cuisine through the eyes of local food experts.

O1 recalled how the fuzzy classification of dishes as belonging to a certain region impacted the data collection process:
\begin{quote}
    ``...depending on the region within the country [the dish] was either going to be made from the cassava plant or a kind of mielie meal. It could be the same dish but depending on the region within the same country they would have a different ingredient for the starch [element].'' O1
\end{quote}

Here, O1 identifies that even though a dish is considered ``the same'' and may be called by the same name across different regions, the actual ingredients in this dish differ and reflect the differing agriculture practices of various regions. These differing agricultural practices are important markers of the local ecosystems and historical practices of farming and cultivation. Therefore, in the data collection process, it became essential to deconstruct dishes into their constituent ingredients, as reflected in the data structure for \textsc{WWD} (see~\cref{tab:dish_questions} and~\cref{asec:protocol-screenshots}). 


In post-mortem reflections, Community Ambassadors highlighted the importance of recognising the limitations of using national borders to demarcate cultures. The granularity of representation was essential to ensuring that distinct cultures were not lumped together into a single flattened snapshot, as physical borders often do. O5 recalls how she realised that one of the three major ethnic groups in Nigeria---the Hausa community---was underrepresented in the \textsc{WWD} dataset. They reached out to a personal connection whom they knew was a member of that ethnic community who, in turn, helped share the website with their community. 


In line with what we posited in the introduction, we reinforce the idea that the value in a dataset is not exclusively conveyed in its final form, but also through the processes of creating it. In this Findings section, we highlight the immense efforts and considerations that went into engaging with our Contributors to produce high-quality, granular data about cultural dishes. Processes such as these are slow, iterative, and very hard to scale, but they are necessary to ensure the production of high-quality and diverse datasets that we would like to see reflected in the ML systems that make use of this data. 




 









\section{Discussion}
\subsection{Giving Human-like Skills to AI} 
This study showed that for one form of humor - Gen-Z style Instagram image captioning humor - our AI-written humor was funnier than GPT's native humorsense, and as funny at the top 5 highest rated Instagram captions. We attribute this to a variety of features we added to the system. First, the visual detail extraction was able to find aspects of the image to poke fun at that were often sharper than GPT's joke target and more similar to the Instagram captions' joke target. Second, the narrative extrapolation step allowed the system to broaden its base of relatable joke targets - moving the focus away from making fun of the literal objects in the image, but using them as metaphors for relatable joke targets like relationship disasters, teamwork breakdowns, and the burden of student loans. This opened more creativity possibilities for joke targets. Lastly, we used an LLM-as-judge to rank the outputs accord to Gen-Z humor taste, thus giving the system some notion of the audience. These skills - detail observation, finding analogous and relatable social situations, and modeling the audience through fine tuning - are all considered somewhat ``human.'' Skills like reasoning and chaining are considered more typical of machines. But this shows that machines might be able to approach these more human skills with the right architecture and training.  


% In this paper, we showed that a model of GPT that is enhanced to have 3 human-like skills used in humor 
% \color{red}
% (observation, sense of story, and in-group knowledge) 
% \color{black}
% outperforms standard GPT-4o. Many other researchers have devised prompting techniques and architectures for improving LLM's reasoning capabilities such as reflection, chain-of-thought, and prompt chaining. However, fewer papers have explored how social skills can enhance LLMs communication abilities. Systems like Generative Agents ~\cite{joon_agents}
% and Character.AI \cite{characterAI} do this to great success. In this paper, we gave VLMs a few simple ``skills'' that were relevant to humor generation and showed that overall, it improved AI's ability to write humor. The focus of this paper was the human evaluation to see whether AI could get closer to parity with most upvoted human captions. However, if even a simple system like this can improve humor, perhaps more sophisticated systems could do better. 

There are many ways to improve the skills in HumorSkills. Building and testing a better Gen Z humor ranking would probably improve the filtering of bad captions. More fine-tuning could improve the breadth of Gen Z slang and references. More narratives and conflicts would expand its vocabulary of relatable situations. Finding ways to automatically collect narratives and conflicts to be applied would accelerate this process. Adding new skills would also be future research. Theories of humor abound. With recent advances in LLM's ability to do long chains of logical reasoning in DeepSeek and GPT-4o, it would be interesting to have AI try to analyze the humor and extract it's own theories or techniques for humor.

One of biggest shortcomings of the captions is that some of them are not logical enough to make sense, but are also not illogical enough to be absurd. These sound like mistakes. As future work, one could test whether an AI-based reflection step could think through the logic of a joke and decide whether it actually made sense or not. 
% The current rating system seems to let some of these by. 

% Other papers have tried fine-tuning b

Further testing or ablation studies could help shed light on which skills are most helpful. However, humor ratings have high variance among raters, and the data required to get statistical significance is often quite high. There may not be an effect of each skill individually - they might only work together. 



\subsection{Implications of Machine with Human-like Social Skills}
Human-like social skills - like humor - are often used for human bonding. If AI can write humor as well as the best people, the AI has the potential to both disingenuously create human bonding ~\cite{diresta2024spammersscammersleverageaigenerated,naaman_opinion} and to augment human's ability to bond~\cite{socialglue}. Either way, this has the potential to change the nature of human trust and communication.
In many ways, this is already happening in other domains. 
ChatGPT and Gmail SmartCompose~\cite{smartcompose} can already rewrite emails to sound more polite and we really are.
AI sales and scams can trick people into giving money to what they think are friend or loved ones in need~\cite{ai_scams}. 
AI has successfully been integrated into Gen Z dating apps that suggest messages to send to potential dates based on both a dating profile (for the opening line) or message history (for continued conversation)~\cite{majic2024rizz}. Many apps attempt this, but the quality of the suggested text sets them apart - the apps that generate more human-like texts have millions of active paying users~\cite{majic2024rizz}.
To some, this potential for disingenuousness is horrifying. Although disingenuous portrayals of oneself for dating purposes far precede the invention of generative AI, there is a possibility that AI will amplify this ability. 


As AI for social, cultural, and personally relevant communication improves, we may need a way to discern genuine from disingenuous communication. There are high-tech ways of doing this, such as making a video of oneself (until AI can do that). There are also low-tech ways of doing this, like talking in person. It would be highly ironic if the advancement of AI drives people to abandon technology, because it could not be trusted to be genuine. 

% For people who adopt these products, a common reason is that the bar for texting banter is so high for Gen Z, that help is appreciated, even when the sentiment is genuine. 

% Even at work, polite communication is socially demanded, and with ever-increasing amounts of communication, the emotional labor of even typing simple pleasantries is tiring. Early generative AI applications like Gmail Smart Compose~\cite{smartcompose} were noted for lowering the burden of writing a polite introductory line in emails. Although these lines are perfunctory and don't necessarily need to be genuine, it makes a difference to readers whether they are there or not. Social effort matters.



% \textit{Although hopefully AI will not force civilization back into a barter economy that necessatiate personal interaction to establish trust.} (LYDIA: TOO MUCH?)


% emotional labor. 

% Also with AI friends like Persona aI?



\section{Limitations}
This study targeted only one form of humor for only one audience:  Gen Z humor Instagram captions. This type of humor tends toward absurdities, which can be easier to generate than something that needs to be logically sound. Being illogically surprising is probably easier than being logical and surprising. Future work would have to test whether similar techniques work on other humor tastes. Some of our techniques, like fine-tuning, would likely work generically for all humor types, but other skills might need to be tried.

The caption humor is difficult, but it is more well-defined than other forms of humor. Caption humor only requires a punchline for a given image (the setup). Other forms of humor like standup comedy and popular humor magazines require generating both the setup and the punchline. A future direction is to explore what additional "skills" are needed to generate jokes with both setup and punchline. 

The humor generated here is for a public audience, but most humor made spontaneously is made for friends, and often users insider knowledge about the friends, their background, and their shared experiences. LLMs would likely struggle to make in-joke humor without a source of inside information to train on.

In our baseline captions, we crafted a simple prompt for GPT-4o to write humor. It is possible that with a better prompt or multiple generations, one could generate similar results. However, that is effectively what the system performs. It might be possible that there are prompts that don't employ any humor skills that can also generate jokes funnier than baseline GPT. Future work should test more prompts - both with and without skills to see if there are approaches other than skills that can enhance LLM humor generation.

% ensuring better generations and more consistent 







\section{Limitations and Future Work}\label{sec:limitations}

\subsection{Limitations}
The distributed network we relied on provides a challenge for financial payment. We believe that meaningful compensation is an important standard in data collection that relies on community-based knowledge is important. However, we acknowledge a stalemate, in that to the best of our knowledge, normative standards around payment mechanisms and logistics for our novel decentralised approach have yet to be established. In preparing this work, we have---and continue to, extensively---engaged with other practitioners and community members to understand best practices. At this moment, we rely on a process of transparency and informed consent rooted in direct attempts to empower participants such that the practice is less extractive. We maintain a constant communication channel with all Contributors (who are rightfully acknowledged in the appendix). 

\subsection{Future work}

Finally, we conclude with suggestions for how future work can engage in infrastructure to support data work. 

The work presented here is not meant to be one size fits all, but rather as a starting point from which to change the way we approach ML datasets and position their importance in terms of the \textit{human processes} behind them. Data collection that relies on community input results in high-quality data, but the process is slow, iterative, and very hard to scale by any other means than what we propose here. We hope that other researchers use this as a foundation for future work.

As mentioned above, simply creating an access point for data submission does not mean that data will easily flow in. Complete infrastructure in the future should involve both accessible technical infrastructure as well as direct engagement with stakeholder communities.

 Data has value, and promises of ``increased representation'' should be critically assessed to make sure the promises made to Contributors are sincere. Improvements can come from increasing infrastructure for payments in a decentralised and distributed network that allows for \textit{meaningful} compensation for data labour, and we continue to advocate for Contributors to be the owners of their data, as well as maintaining~\cite{tonja2024inkubalm} transparency in the process.  
Additional efforts should support increased accessibility that supports submission in a Contributor's local language to increase accessibility.

Acknowledging the burden of data collection, even with the distributed workload we propose with the stakeholder distinction between Community Ambassador and Contributor, sustainable processes for all stakeholders need to be considered. 
 
\section{Conclusion}



This paper introduces \sysname, an AI-assisted system designed to enhance the process of visual blend ideation by leveraging metaphors. 
%Our system utilizes large language models and commonsense knowledge bases to explore objects and their associated attributes, forming metaphorical connections with abstract concepts.
Our system utilizes LLMs and commonsense knowledge bases to explore objects and their associated attributes, forming metaphorical connections with abstract concepts. 
It offers the capability to automatically generate blending proposals based on user selections, facilitating rapid creative realization for verification through the T2I model.
To evaluate the system, we conducted a user study involving 24 participants who had AI experience. The findings demonstrate that \sysname\ has the potential to enhance the creativity of the generated ideation results and enable the expression of abstract concepts more metaphorically.
Additionally, this research offers insights into user preferences regarding visual blend design and potential future approaches for supporting design with generative AI.



\bibliographystyle{ACM-Reference-Format}
\bibliography{main}
\clearpage

\appendix
% \section{Framework Details}
% Our framework is described in Algorithm~\ref{algorithm}, and compared with former baselines in Table~\ref{table:comparison}. Distinct with several methods generating Python code for visualization directly, we use VQL as an intermediate representation to bridge natural language queries and visualization code. Additionally, our framework can be easily optimized by adding some useful tools such as Retrieval Augmented Generation. Moreover, our method supports handling multi-table data and the visualization can be customized according to humans' preferences. Our framework utilizes the agent-based collaborative workflow, which consists of data preprocessing, generation, and error correction, organized with the modular design.

% \begin{algorithm}
% \small
% \caption{\system Framework}
% \label{algorithm}
% \begin{algorithmic}[1]
% \Function{\nlvis}{$Q$, $S$}
%     \State Initialize $Mem \gets \{Q,S\}$
%     \State $(S', A) \gets \textsc{Processor}(Mem)$
%     \State $Mem.update(S', A)$
%     \State $V \gets \textsc{Composer}(Mem)$
%     \State $Mem.update(V)$
%     \State $Chart, isValid \gets \textsc{Validator}(Mem)$
%     \While{not $isValid$}
%         \State $V \gets \textsc{Refine}(Mem)$
%         \State $Mem.update(V)$
%         \State $Chart, isValid \gets \textsc{Validator}(Mem)$
%     \EndWhile
%     \State \Return $Chart$
% \EndFunction
% \end{algorithmic}

% \end{algorithm}




% \begin{table*}[!t]
%     \centering
    
%     \vspace{-1em}
%     \scalebox{0.68}{
%     \begin{tabular}{lccccccc}
%         \toprule[1.5pt]
%         \multirow{3}{*}{\textbf{Framework}} & \multicolumn{2}{c}{\textbf{System Features}} & \multicolumn{2}{c}{\textbf{Visualization Capabilities}} & \multicolumn{3}{c}{\textbf{Agentic Workflow}} \\
%         \cmidrule(lr){2-3} \cmidrule(lr){4-5} \cmidrule(lr){6-8}
%         & \textbf{VQL as} & \textbf{Extensible} & \textbf{Multi-Table} & \textbf{Customizable} & \textbf{Data} & \textbf{Modular} & \textbf{Error-} \\
%         & \textbf{Thoughts} & \textbf{Optimization} & \textbf{Support} & \textbf{Styling} & \textbf{Preprocess} & \textbf{Design} & \textbf{Correction} \\
%         \midrule
%         Chat2VIS~\cite{chat2vis} & \textcolor{red}{\ding{56}} & \textcolor{red}{\ding{56}} & \textcolor{red}{\ding{56}} & \textcolor{red}{\ding{56}} & \textcolor{green!60!black}{\ding{52}} & \textcolor{red}{\ding{56}} & \textcolor{red}{\ding{56}} \\
%         Mirror~\cite{mirror} & \textcolor{red}{\ding{56}} & \textcolor{red}{\ding{56}} & \textcolor{red}{\ding{56}} & \textcolor{red}{\ding{56}} & \textcolor{red}{\ding{56}} & \textcolor{green!60!black}{\ding{52}} & \textcolor{red}{\ding{56}} \\
        
%         LIDA~\cite{lida} & \textcolor{red}{\ding{56}} & \textcolor{green!60!black}{\ding{52}} & \textcolor{red}{\ding{56}} & \textcolor{green!60!black}{\ding{52}} & \textcolor{green!60!black}{\ding{52}} & \textcolor{green!60!black}{\ding{52}} & \textcolor{red}{\ding{56}} \\
%         CoML4VIS~\cite{coml} & \textcolor{red}{\ding{56}} & \textcolor{red}{\ding{56}} & \textcolor{green!60!black}{\ding{52}} & \textcolor{red}{\ding{56}} & \textcolor{green!60!black}{\ding{52}} & \textcolor{red}{\ding{56}} & \textcolor{red}{\ding{56}} \\
        
%         Prompt4VIS~\cite{prompt4vis} & \textcolor{green!60!black}{\ding{52}} & \textcolor{red}{\ding{56}} & \textcolor{green!60!black}{\ding{52}} & \textcolor{red}{\ding{56}} & \textcolor{green!60!black}{\ding{52}} & \textcolor{green!60!black}{\ding{52}} & \textcolor{red}{\ding{56}} \\
        
%         CoT-Vis~\cite{cotvis} & \textcolor{green!60!black}{\ding{52}} & \textcolor{red}{\ding{56}} & \textcolor{red}{\ding{56}} & \textcolor{red}{\ding{56}} & \textcolor{green!60!black}{\ding{52}} & \textcolor{red}{\ding{56}} & \textcolor{red}{\ding{56}} \\

%         \midrule
%         \SystemName (Ours) & \textcolor{green!60!black}{\ding{52}} & \textcolor{green!60!black}{\ding{52}} & \textcolor{green!60!black}{\ding{52}} & \textcolor{green!60!black}{\ding{52}} & \textcolor{green!60!black}{\ding{52}} & \textcolor{green!60!black}{\ding{52}} & \textcolor{green!60!black}{\ding{52}} \\
%         \bottomrule[1.5pt]
%     \end{tabular}}
% \caption{Comparison of various \nlvis frameworks. }  \label{table:comparison}
% \vspace{-1em}
% \end{table*}

\section{Detailed Experiment Setups}
\label{detailed_experiment_setups}
\paragraph{Baselines.}
\label{detailed_baselines}
% We implemented our experiment compared with three recent baselines. Note that, we also tried to use Code Interpreter as a baseline, but due to the rate limit of API constraint, the evaluation failed to generate visualizations via direct .csv files.
This study compares our approach with three state-of-the-art baselines. We also attempted to include Code Interpreter as a baseline; however, API rate limitations prevent the direct generation of visualizations from CSV files.

\begin{itemize}[leftmargin=*, itemsep=0pt] 
    \item \textbf{Chat2Vis} \cite{chat2vis}: It generates data visualizations by leveraging prompt engineering to translate natural language descriptions into visualizations. It uses a language-based table description, which includes column types and sample values, to inform the visualization generation process.\item \textbf{LIDA} \cite{lida}: It structures visualization generation as a four-step process, where each step builds on the previous one to incrementally translate natural language inputs into visualizations. It uses a JSON format to describe column statistics and samples, making it adaptable across various visualization tasks.
    \item \textbf{CoML4Vis} \cite{coml}: 
    % Building on a data science code generation framework, CoML4Vis 
    It utilizes a few-shot prompt that integrates multiple tables into a single visualization task. It summarizes data table information, including column names and samples, and then applies a few-shot prompt to guide visualization generation.
\end{itemize}

\paragraph{Metrics.}
\label{detailed_metrics}
Our evaluation framework involves five main metrics:
\begin{itemize}[leftmargin=*, itemsep=0pt] 
    \item \textbf{Invalid Rate} represents the percentage of visualizations that fail to render due to issues like incorrect API usage or other code errors.
    \item \textbf{Illegal Rate} indicates the percentage of visualizations that do not meet query requirements, which can include incorrect data transformations, mismatched chart types, or improper visualizations.
    \item \textbf{Readability Score} is the average score (range 1-5) assigned by a vision language model, like GPT-4V, for valid and legal visualizations, assessing their visual clarity and ease of interpretation.
    \item \textbf{Pass Rate} measures the proportion of visualizations in the evaluation set that are both valid (able to render) and legal (meet the query requirements).
    \item \textbf{Quality Score} is set to 0 for invalid or illegal visualizations; otherwise, it is equal to the readability score, providing an overall assessment of visualization quality factoring in both functionality and clarity.
\end{itemize}
To thoroughly evaluate each main metric, we further break them down into the following detailed assessment criteria:
\begin{itemize}[leftmargin=4mm, itemsep=0.05mm] 
    \item \textbf{Code Execution Check} verifies that the Python code generated by the model can be successfully executed.
    \item \textbf{Surface-form Check} ensures that the generated code includes necessary elements to produce a visualization like function calls to display the chart.
    \item \textbf{Chart Type Check} verifies whether the extracted chart type from the visualization matches the ground truth.
    \item \textbf{Data Check} assesses if the data used in the visualization matches the ground truth, taking into consideration potential channel swaps based on specified channels.
    \item \textbf{Order Check} evaluates whether the sorting of visual elements follows the specified query requirements.
    \item \textbf{Layout Check} examines issues like text overflow or element overlap within visualizations.
    \item \textbf{Scale \& Ticks Check} ensures that scales and ticks are appropriately chosen, avoiding unconventional representations.
    \item \textbf{Overall Readability Rating} integrates various readability checks to provide a comprehensive score considering layout, scale, text clarity, and arrangement.
\end{itemize}

% For all evaluation results, these metrics are averaged across the dataset to provide an overarching view of model performance. These metrics collectively ensure that visualizations are not only correct in terms of execution but also effective in communicating the intended data narratives.
The evaluation metrics are averaged across the dataset to provide a comprehensive overview of the model's performance. Together, these metrics ensure that the visualizations are both accurate in execution and effective in conveying the intended data narratives.



\begin{table}[!t]
\centering
\setlength{\belowcaptionskip}{0em} 
% \vspace{-1em}
\begin{tabular}{lcc}
\toprule[1.5pt]
\textbf{Model} & \textbf{P-corr} & \textbf{P-value} \\
\midrule
GPT-4o-mini & \textbf{0.6503} & 0.000 \\
GPT-4o & 0.5648 & 0.000 \\
\bottomrule[1.5pt]
\end{tabular}
\caption{ The Pearson correlations of GPT-4o-mini and GPT-4o with human judgments on readability scores. }
\label{tab:pearson_corr}
\vspace{-1em}
\end{table}

\begin{table*}[!ht]
\centering

\vspace{-1em}
\begin{tabular}{l|ccc|ccc}
\toprule
\multirow{2}{*}{Method} & \multicolumn{3}{c|}{Single Table} & \multicolumn{3}{c}{Multiple Tables} \\
\cmidrule(l){2-4} \cmidrule(l){5-7}
 & prompt & response & total & prompt & response & total \\
\midrule
LIDA & 1386.23 & 237.90 & 1624.13 & \multicolumn{3}{c}{N/A} \\
Chat2Vis & 414.35 & 451.30 & 865.65 & \multicolumn{3}{c}{N/A} \\
CoML4Vis & 2614.76 & 279.86 & 2894.62 & 3069.62 & 307.67 & 3377.29 \\
\system & 5122.99 & 777.63 & 5900.62 & 5613.96 & 1014.10 & 6628.06 \\
\bottomrule
\end{tabular}
\caption{Token usage comparison for different methods. N/A indicates that LIDA and Chat2Vis cannot handle multiple table scenarios.}
\label{tab:token_usage}
\end{table*}

\begin{table}[ht]
\centering
\scalebox{1}{
\begin{tabular}{l|ccc}
\toprule
Agent & \#Input & \#Output & \#Total \\
\midrule
Processor & 1486.07 & 569.58 & 1755.65\\
Composer & 3268.32 & 221.74 & 3490.07 \\
Validator & 1051.82 & 127.85 & 1179.67  \\
\bottomrule
\end{tabular}}
\caption{Token usage of three agents in \system.} \label{tab:token_agent} 
\vspace{-1em}
\end{table}

\paragraph{Implement Details.}
Our system is implemented in Python 3.9, utilizing GPT-4o \citep{openai_gpt4o_2024}, GPT-4o-mini~\cite{openai2024gpt4omini}, and GPT-3.5-turbo~\cite{chatgpt3.5} as the backbone model for all approaches, with the temperature set to 0 for consistent outputs. GPT-4o-mini serves as the vision language model for readability evaluation. We interact with these models through the Azure OpenAI API. The specific prompt templates for each agent, crucial for guiding their respective roles in the visualization generation process, are detailed in Appendix~\ref{prompt_details}. Token usages of \system and baselines are demonstrated in Table~\ref{tab:token_usage}, and usage for each agent in our \system is shown in Table~\ref{tab:token_agent}. Additionally, our evaluations are conducted in VisEval Benchmark (with MIT license).

\paragraph{Human Annotation.}
\label{human}
The annotation is conducted by 5 authors of this paper independently. As acknowledged, the diversity of annotators plays a crucial role in reducing bias and enhancing the reliability of the benchmark. These annotators have knowledge in the data visualization domain, with different genders, ages, and educational backgrounds. The educational backgrounds of annotators are above undergraduate. To ensure the annotators can proficiently mark the data, we provide them with detailed tutorials, teaching them how to judge the quality of data visualization. We also provide them with detailed criteria and task requirements in each annotation process shown in Figure~\ref{fig:annotation}. Two experiments requiring human annotation are detailed as follows:

\begin{figure}[!ht]
    \centering
    \includegraphics[width=\linewidth]{figure/score_distribution.pdf}
    \caption{Comparison of score density distribution between GPT-4o, GPT-4o-mini and human average score.}
    \label{fig:score_distribution}
\end{figure}

\begin{table*}[!ht]
\centering
\begin{tabular}{l|ccc}
\toprule
& Invalid Rate & Illegal Rate & Pass Rate \\
\midrule
\system & 4.66\% & 23.97\% & 71.35\% \\
w. CoT for Validator & 5.82\% & 23.39\% & 70.78\% \\
w. original schema for Validator & 4.80\% & 24.22\% & 70.97\% \\
\bottomrule
\end{tabular}
\caption{Additional exploration for Validator (using GPT-3.5-turbo).} 
\vspace{-1em} 
\label{tab:ablation_validator}
\end{table*}

\begin{itemize}[leftmargin=*, itemsep=0pt]
    \item \textbf{Pearson Correlation of Visual Language Model.} We conduct human annotation frameworks to compare the ability of the visual language model for MLLM-as-a-Judge~\cite{chen2024mllm}, providing the readability score. Our annotation framework is shown in Figure~\ref{fig:annotation}. The final Pearson scores are demonstrated in Table~\ref{tab:pearson_corr}, with its density distribution in Figure~\ref{fig:score_distribution}. The detailed instructions can be found in Figure~\ref{fig:scoring_instructions}.
    \item \textbf{Qualitative comparison to calculate ELO Scores.} We conduct human-judgments evaluations to compare which visualization generated by different models meets the query requirement more precisely. The leaderboard is shown in Table~\ref{tab:elo_rankings}, and Figure~\ref{fig:elo} shows the judgment framework. Each model starts with a base ELO score of 1500. After each pairwise comparison, the scores are updated based on the outcome and the current scores of the models involved. The hyperparameters are set as follows: the $K$-factor is set to 32, which determines the maximum change in rating after a single comparison. We conduct two sets of evaluations: one for single-table queries and another for multiple-table queries, with 1000 bootstrap iterations for each set to ensure statistical robustness. For each model's ELO rating, we report the 95\% confidence intervals computed through bootstrap resampling, providing a measure of rating stability. The evaluation process involves presenting human judges with a query and two visualizations, asking them to select the one that better meets the query requirements. This process is repeated across all model pairs and queries in our test set. The detailed guidance provides to the human evaluators can be found in Figure~\ref{fig:evaluation_instructions}, which outlines the criteria for judging visualization quality and relevance to the given query.


\end{itemize}

\begin{figure}[!ht]
	\centering
    \setlength{\belowcaptionskip}{-1em}
	\includegraphics[width=0.98\linewidth,scale=1.0]
    {./figure/library.pdf}
    \vspace{-1em}
	\caption{Performance of different models using \texttt{Matplotlib} and \texttt{Seaborn} libraries, using GPT-3.5-turbo.
    % \yao{larger fontsize?}
    }
\label{fig: library}
\end{figure}

\begin{figure*}[!h]
    \centering
    \includegraphics[width=0.98\linewidth]{figure/annotation.pdf}
    \caption{Screenshot of human annotation process in readability score.}
    \label{fig:annotation}
\end{figure*}

\begin{figure*}[ht]
\centering
\vspace{1em}
\begin{tcolorbox}[enhanced,attach boxed title to top center={yshift=-3mm,yshifttext=-1mm},boxrule=0.9pt, 
  colback=gray!00,colframe=black!50,colbacktitle=gray,
  title=Readability Scoring Instruction,
  boxed title style={size=small,colframe=gray} ]
\small
\textbf{Scoring Instructions:} Please evaluate the charts based on the following criteria, with a score range from 1 to 5, where 1 indicates very poor quality and 5 indicates excellent quality. You should focus on the following aspects:

\vspace{0.5em}
\textbf{1. Chart Colors:}
\begin{itemize}
    \item Are the colors clear and natural, effectively conveying the information?
    \item Color blindness accessibility: Are the color combinations easy to distinguish, especially for users with color blindness?
\end{itemize}

\vspace{0.5em}
\textbf{2. Title and Axis Labels:}
\begin{itemize}
    \item Ensure the chart has a clear title.
    \item Do the X-axis and Y-axis labels exist, and are they complete?
    \item Check if the labels are difficult to read, e.g., are they written vertically instead of horizontally?
    \item The title should not be a direct question; instead, it should describe the data or trends being presented.
\end{itemize}

\vspace{0.5em}
\textbf{3. Legend Completeness:}
\begin{itemize}
    \item Is the legend complete, and does it clearly indicate the color labels for different data series?
    \item Ensure each color has a corresponding legend, making it easy for users to understand what the data represents.
\end{itemize}

\vspace{0.5em}
\textbf{Scoring Scale:}
\begin{itemize}
    \item \textbf{1 Point:} Very poor, unable to understand or severely lacking information.
    \item \textbf{2 Points:} Poor quality, multiple issues present, difficult to extract information.
    \item \textbf{3 Points:} Fair, conveys some information but still has room for improvement.
    \item \textbf{4 Points:} Good, generally clear charts with minor areas for improvement.
    \item \textbf{5 Points:} Excellent, outstanding chart design with clear and effective information presentation.
\end{itemize}

Please consider the above factors when assessing the charts and provide the appropriate score. Thank you for your cooperation and effort!
\end{tcolorbox}
\vspace{-7pt}
\caption{Instructions for human annorators in annotating readability scoring.}
\label{fig:scoring_instructions}
\vspace{1em}
\end{figure*}

\begin{figure*}[!ht]
    \centering
    \includegraphics[width=0.98\linewidth]{figure/elo.pdf}
    \caption{Screenshot of ELO score evaluation framework for Human-as-a-Judge.}
    \label{fig:elo}
\end{figure*}

\begin{figure*}[ht]
\centering
\vspace{1em}
\begin{tcolorbox}[enhanced,attach boxed title to top center={yshift=-3mm,yshifttext=-1mm},boxrule=0.9pt, 
  colback=gray!00,colframe=black!50,colbacktitle=gray,
  title=Visualization Comparison Guidance,
  boxed title style={size=small,colframe=gray} ]
\small
Welcome to the visualization comparison evaluation. Your task is to judge which model-generated visualization better meets the requirements of the natural language query.

\vspace{0.5em}
\textbf{Evaluation criteria:}
\begin{enumerate}
    \item \textbf{Appropriateness of chart type:} Check if the selected chart type is suitable for expressing the data and relationships required by the query.
    \item \textbf{Data completeness:} Ensure the chart includes all necessary data required by the query.
    \item \textbf{Readability:} Assess the clarity of the chart, accuracy of labels, and overall layout.
    \item \textbf{Aesthetics:} Consider if the chart's color scheme, proportions, and overall design are visually pleasing.
    \item \textbf{Information conveyance:} Judge if the chart effectively conveys the main information or insights required by the query.
\end{enumerate}

\vspace{0.5em}
\textbf{Evaluation process:}
\begin{enumerate}
    \item Carefully read the natural language query.
    \item Observe the visualization results generated by two models.
    \item Based on the above criteria, choose the better visualization or select a tie if they are equally good.
    \item If neither visualization satisfies the query requirements well, please choose the relatively better one.
\end{enumerate}

Remember, your evaluation will help us improve and compare different visualization models. Thank you for your participation!
\end{tcolorbox}
\vspace{-7pt}
\caption{Instructions for human annorators in visualization comparison.}
\label{fig:evaluation_instructions}
\vspace{1em}
\end{figure*}


\section{Additional Experiment Results}
\label{additional_experiment_result}

We also conducted a comparison experiment of different methods using matplotlib or seaborn library. Figure~\ref{fig: library} demonstrates the results, indicating that our method outperforms obviously other baselines not only with matplotlib but also seaborn.

In addition, we test techniques in the Validator Agent, such as Chain-of-Thought. As is shown in Table~\ref{tab:ablation_validator}, integrating Chain-of-Thought reasoning, may affect its performance badly, likely due to the simple refining task with complex reasoning. Moreover, using the original schema to check for false schema filtering seems to be useless in this case.

\section{Evaluation Results with Detailed Metrics}
We demonstrated the main results in Table~\ref{tab:performance_comparison}, and here we reported more detailed results of other metrics in Table~\ref{tab:detailed_results}, which underscored the error rates for each stage, including \textit{Invalid}, \textit{Illegal}, and \textit{Low Readability}. 

\begin{table*}[!ht]
\centering
\footnotesize
\scalebox{0.98}{
\begin{tabular}{ll|cc|cccc|cc}
\toprule[1.5pt]
\multirow{2}{*}{Method} & \multirow{2}{*}{Dataset} & \multicolumn{2}{c|}{Invalid} & \multicolumn{4}{c|}{Illegal} & \multicolumn{2}{c}{Low Readability} \\
&  & Execution & Surface. & Decon. & Chart Type & Data & Order & Layout & Scale\&Ticks \\
\midrule
\multicolumn{10}{c}{ \textbf{\textit{GPT-4o}}}\\
\midrule
\multirow{3}{*}{CoML4Vis} & All & 1.15 & 0.00 & 0.26 & 1.75 & 14.28 & 10.36 & 32.02 & 32.55 \\
& Single & 0.67 & 0.00 & 0.43 & 1.93 & 13.54 & 10.16 & 31.08 & 32.76 \\
& Multiple & 1.87 & 0.00 & 0.00 & 1.48 & 15.39 & 10.66 & 33.43 & 32.23 \\
\cmidrule{2-10}
\multirow{3}{*}{LIDA} & All & 6.61 & 0.00 & 1.60 & 3.24 & 40.53 & 4.07 & 32.68 & 15.77 \\
& Single & 1.13 & 0.00 & 2.11 & 0.89 & 12.26 & 6.79 & 53.93 & 26.22 \\
& Multiple & 14.80 & 0.00 & 0.79 & 8.51 & 80.53 & 0.00 & 1.24 & 0.21 \\
\cmidrule{2-10}
\multirow{3}{*}{Chat2Vis} & All & 16.05 & 0.00 & 0.62 & 3.99 & 30.14 & 5.96 & 2.37 & 20.88 \\
& Single & 0.86 & 0.00 & 0.75 & 2.30 & 10.78 & 9.73 & 3.97 & 34.63 \\
& Multiple & 38.74 & 0.00 & 0.43 & 6.51 & 59.08 & 0.32 & 0.00 & 0.34 \\
\cmidrule{2-10}
\multirow{3}{*}{nvAgent} & All & 0.97 & 0.00 & 0.08 & 1.28 & 11.07 & 4.05 & 5.07 & 40.03 \\
& Single & 0.72 & 0.00 & 0.14 & 1.27 & 9.88 & 3.60 & 3.92 & 39.36 \\
& Multiple & 1.34 & 0.00 & 0.00 & 1.30 & 12.84 & 4.73 & 6.79 & 41.03 \\
\midrule
\multicolumn{10}{c}{ \textbf{\textit{GPT-4o-mini}}}\\
\midrule
\multirow{3}{*}{CoML4Vis} & All & 4.23 & 0.00 & 0.20 & 2.31 & 16.64 & 11.83 & 35.23 & 29.35 \\
& Single & 0.36 & 0.00 & 0.26 & 2.32 & 13.80 & 11.67 & 35.92 & 32.22 \\
& Multiple & 10.01 & 0.00 & 0.10 & 2.31 & 20.87 & 12.07 & 34.19 & 25.05 \\
\cmidrule{2-10}
\multirow{3}{*}{LIDA} & All & 12.50 & 0.00 & 0.40 & 4.92 & 40.02 & 5.80 & 27.87 & 17.05 \\
& Single & 9.09 & 0.00 & 0.44 & 2.53 & 12.91 & 9.68 & 45.69 & 28.32 \\
& Multiple & 17.61 & 0.00 & 0.33 & 8.51 & 80.53 & 0.00 & 1.24 & 0.21 \\
\cmidrule{2-10}
\multirow{3}{*}{Chat2Vis} & All & 15.45 & 0.17 & 0.17 & 4.21 & 31.90 & 8.20 & 2.14 & 18.97 \\
& Single & 2.14 & 0.29 & 0.41 & 2.53 & 11.99 & 9.68 & 45.69 & 28.32 \\
& Multiple & 35.78 & 0.00 & 0.00 & 6.70 & 61.66 & 0.00 & 0.92 & 0.32 \\
\cmidrule{2-10}
\multirow{3}{*}{nvAgent} & All & 5.14 & 0.00 & 0.00 & 2.40 & 16.33 & 10.61 & 41.06 & 27.00 \\
& Single & 1.97 & 0.00 & 0.14 & 2.97 & 15.21 & 7.49 & 39.30 & 32.39 \\
& Multiple & 8.15 & 0.00 & 0.00 & 2.31 & 20.87 & 12.07 & 34.19 & 25.05 \\
\midrule
\multicolumn{10}{c}{ \textbf{\textit{GPT-3.5-turbo}}}\\
\midrule
\multirow{3}{*}{CoML4Vis} & All & 9.28 & 0.00 & 0.62 & 1.91 & 15.83 & 12.86 & 25.09 & 27.73 \\ 
& Single & 6.17 & 0.00 & 0.89 & 2.50 & 14.71 & 13.20 & 26.10 & 29.93 \\ 
& Multiple & 13.92 & 0.00 & 0.21 & 1.04 & 17.51 & 12.36 & 23.57 & 24.43 \\ 
\cmidrule{2-10} 
\multirow{3}{*}{LIDA} & All & 53.43 & 0.00 & 1.27 & 3.56 & 22.33 & 0.53 & 14.90 & 6.62 \\ 
& Single & 47.32 & 0.00 & 1.91 & 2.81 & 13.03 & 0.89 & 24.43 & 11.05 \\ 
& Multiple & 62.57 & 0.00 & 0.32 & 4.68 & 36.23 & 0.00 & 0.65 & 0.00 \\ 
\cmidrule{2-10} 
\multirow{3}{*}{Chat2Vis} & All & 18.68 & 0.00 & 0.28 & 3.66 & 32.47 & 7.20 & 25.45 & 20.15 \\ 
& Single & 3.90 & 0.00 & 0.47 & 2.78 & 15.62 & 12.01 & 41.74 & 33.38 \\ 
& Multiple & 40.77 & 0.00 & 0.00 & 4.97 & 57.66 & 0.00 & 1.12 & 0.37 \\ 
\cmidrule{2-10} 
\multirow{3}{*}{nvAgent} & All & 4.66 & 0.00 & 0.08 & 3.06 & 18.24 & 5.64 & 5.25 & 35.34 \\ 
& Single & 2.98 & 0.00 & 0.14 & 2.84 & 15.08 & 5.69 & 3.62 & 37.57 \\ 
& Multiple & 7.18 & 0.00 & 0.00 & 3.38 & 22.95 & 5.56 & 7.69 & 32.02 \\
\bottomrule[1.5pt]
\end{tabular}
}
\caption{Detailed error rates (\%) for different methods.} 
\label{tab:detailed_results}
\end{table*}

\section{Case Study}
\label{example}
% To demonstrate our approach's effectiveness, we present several illustrative examples. Figure~\ref{fig:nl_vql} shows how our system translates natural language into a structured VQL representation. Figure~\ref{python code} and Figure~\ref{fig:example_chart} demonstrate the complete pipeline from query to visualization.
Figure~\ref{fig:nl_vql} shows an example of a natural language query with its corresponding VQL representation. The output Python code for visualization and the final bar chart are demonstrated in Figure~\ref{python code} and Figure~\ref{fig:example_chart}, respectively.
Furthermore, we provide a case study of \system performance on four hardness-level NL2Vis problems in VisEval in Figure \ref{hardness case}.

The easy case demonstrates accurate grouping in scatter plot relationships. The medium case shows correct handling of multi-table joins for continent-wise statistics. The hard case exhibits temporal data visualization with proper filtering. The extra hard case showcases complex operations including weekday binning and stacked visualization. These cases highlight our system's consistent performance across varying task complexities, particularly excelling in multiple table scenarios and complex aggregations.

\begin{figure*}[htbp]
\centering
\begin{tcolorbox}[enhanced,attach boxed title to top center={yshift=-3mm,yshifttext=-1mm},boxrule=0.9pt, 
  colback=gray!00,colframe=black!50,colbacktitle=gray,
  title=An Example of Natural Language Query and  Corresponding VQL,,
  boxed title style={size=small,colframe=gray} ]

\textbf{Natural Language Query:}\\
How many documents are stored? Bin the store date by weekday in a bar chart.\\
\tcbline
\textbf{Corresponding VQL:}\\
Visualize BAR \\
SELECT Date\_Stored, COUNT(Document\_ID)\\
FROM All\_Documents \\
GROUP BY Date\_Stored \\
BIN Date\_Stored BY WEEKDAY\\
\end{tcolorbox}
\caption{The natural language query case and its corresponding output VQL representation.}
\label{fig:nl_vql}
\end{figure*}

\lstset{
    basicstyle=\ttfamily\small,
    breaklines=true,
    numbers=left,
    numberstyle=\tiny,
    frame=single,
    showstringspaces=false,
    tabsize=4,
    keywordstyle=\color{blue},
    commentstyle=\color{green!60!black},
    stringstyle=\color{purple},
    breakatwhitespace=false,
    keepspaces=true
}
\begin{figure*}[htbp]
\begin{lstlisting}[
    language=Python,
    frame=single,
    numbers=left,
    numberstyle=\tiny,
    basicstyle=\tiny\ttfamily,    
    breaklines=true,
    breakatwhitespace=false,
    showstringspaces=false,
    framesep=1mm,                 % 减小框架边距
    framexleftmargin=1mm,         % 减小框架左边距
    columns=flexible,             % 灵活列宽
    keepspaces=true
]
import matplotlib.pyplot as plt 
import pandas as pd 
import os 
import duckdb

# Set data folder path
data_folder = 'E:/visEval_dataset/databases/cre_Doc_Tracking_DB'  

# Connect to database 
con = duckdb.connect(database=':memory:')  

# Read all CSV files and create view
csv_files = [f for f in os.listdir(data_folder) if f.endswith('.csv')] 
for file in csv_files:
    table_name = os.path.splitext(file)[0]
    con.execute(f"CREATE VIEW {table_name} AS SELECT * FROM read_csv_auto('{os.path.join(data_folder, file)}')")  
    
# Execute SQL query 
sql = f''' 
    SELECT Date_Stored, COUNT(Document_ID) AS count_Document_ID 
    FROM All_Documents 
    GROUP BY Date_Stored 
''' 
df = con.execute(sql).fetchdf() 
con.close()  

# Rename columns 
df.columns = ['Date_Stored','count_Document_ID'] 

# Apply binning operation
flag = True 
df['Date_Stored'] = pd.to_datetime(df['Date_Stored']) 
df['Date_Stored'] = df['Date_Stored'].dt.day_name()  

# Group by and calculate count 
if flag:
    df = df.groupby('Date_Stored').sum().reset_index() 

# Ensure all seven days of the week are included 
weekday_order = ['Monday', 'Tuesday', 'Wednesday', 'Thursday', 
                 'Friday', 'Saturday', 'Sunday'] 
df = df.set_index('Date_Stored').reindex(weekday_order, fill_value=0).reset_index()
df['Date_Stored'] = pd.Categorical(df['Date_Stored'], 
                                  categories=weekday_order, ordered=True) 
df = df.sort_values('Date_Stored')

# Create visualization 
fig, ax = plt.subplots(1, 1, figsize=(10, 4)) 
ax.spines['top'].set_visible(False) 
ax.spines['right'].set_visible(False) 
ax.bar(df['Date_Stored'], df['count_Document_ID']) 
ax.set_xlabel('Date_Stored') 
ax.set_ylabel('count_Document_ID') 
ax.set_title(f'BAR Chart of count_Document_ID by Date_Stored') 
plt.xticks(rotation=45) 
plt.tight_layout()  
plt.show()
\end{lstlisting}
\caption{An example of python code generating module within \system.}
\label{python code}
\end{figure*}


\begin{figure*}[!ht]
    \centering
    \includegraphics[width=0.98\linewidth,scale=1.0]{figure/bar_chart.pdf}
    \caption{An example of generated bar chart using \system.}
    \label{fig:example_chart}
\end{figure*}

\begin{figure*}[htbp]
\centering
\begin{tcolorbox}[enhanced,attach boxed title to top center={yshift=-3mm,yshifttext=-1mm},boxrule=0.9pt, 
  colback=gray!00,colframe=black!50,colbacktitle=gray,
  title=Examples of \textsc{nvAgent} performance on different hardness levels,
  boxed title style={size=small,colframe=gray} ]
  
\textbf{Hardness Level:} Easy \\
\begin{minipage}{0.45\linewidth}
    \textbf{Dataset}: \textit{Single}\\
    \textbf{Input Tables}: basketball\_match\\
    \textbf{Input Query}: Show the relation between acc percent and all\_games\_percent for each ACC\_Home using a grouped scatter chart.\\
\end{minipage}\hfill
\begin{minipage}{0.45\linewidth}
    \centering
    \textbf{Response}:
    \includegraphics[width=\linewidth]{figure/easy_3085.pdf} 
\end{minipage}
\tcbline

\textbf{Hardness Level:} Medium \\
\begin{minipage}{0.45\linewidth}
    \textbf{Dataset}: \textit{Multiple}\\
    \textbf{Input Tables}: car\_makers, car\_names, cars\_data, continents, countries, model\_list\\
    \textbf{Input Query}: Display a pie chart for what is the name of each continent and how many car makers are there in each one?\\
\end{minipage}\hfill
\begin{minipage}{0.55\linewidth}
    \centering
    \textbf{Response}:
    \includegraphics[width=\linewidth]{figure/medium_433.pdf} 
\end{minipage}
\tcbline

\textbf{Hardness Level:} Hard \\[1em]
\begin{minipage}{0.45\linewidth}
    \textbf{Dataset}: \textit{Multiple}\\
    \textbf{Input Tables}: advisor, classroom, course, department, instructor, prereq, section, student, takes, teaches, time\_slot\\
    \textbf{Input Query}: Find the number of courses offered by Psychology department in each year with a line chart.\\
\end{minipage}\hfill
\begin{minipage}{0.45\linewidth}
    \centering
    \textbf{Response}:
    \includegraphics[width=\linewidth]{figure/hard_611.pdf} 
\end{minipage}
\tcbline

\textbf{Hardness Level:} Extra Hard \\[1em]
\begin{minipage}{0.45\linewidth}
    \textbf{Dataset}: \textit{Multiple}\\
    \textbf{Input Tables}: Accounts, Documents, Documents\_with\_Expenses, Projects, Ref- \_Budget\_Codes, Ref\_Document\_Types, Statements\\
    \textbf{Input Query}: How many documents are created in each day? Bin the document date by weekday and group by document type description with a stacked bar chart, I want to sort Y in desc order.\\
\end{minipage}\hfill
\begin{minipage}{0.45\linewidth}
    \centering
    \textbf{Response}:
    \includegraphics[width=\linewidth]{figure/extra_851.pdf} 
\end{minipage}

\end{tcolorbox}
    \caption{Examples of \textsc{nvAgent}'s performance on different hardness levels in VisEval (easy, medium, hard, and extra hard.}
    \label{hardness case}
\end{figure*}


\clearpage
\onecolumn
\section{Prompts Details}
\label{prompt_details}
We provide detailed prompt design of our \system as follows.



\begin{promptbox}[Prompt template for Processor Agent]
You are an experienced and professional database administrator. Given a database schema and a user query, your task is to analyze the query, filter the relevant schema, generate an optimized representation, and classify the query difficulty. \\
\\
Now you can think step by step, following these instructions below. \\
\textbf{[Instructions]} \\
1. Schema Filtering: \\
\text{\ \ \ \ }- Identify the tables and columns that are relevant to the user query.\\
\text{\ \ \ \ }- Only exclude columns that are completely irrelevant.\\
\text{\ \ \ \ }- The output should be \{\{tables: [columns]\}\}.\\
\text{\ \ \ \ }- Keep the columns needed to be primary keys and foreign keys in the filtered schema.\\
\text{\ \ \ \ }- Keep the columns that seem to be similar with other columns of another table.\\
\\
2. New Schema Generation:\\
\text{\ \ \ \ }- Generate a new schema of the filtered schema, based on the given database schema and your filtered schema.\\
\\
3. Augmented Explanation:\\
\text{\ \ \ \ }- Provide a concise summary of the filtered schema to give additional knowledge.\\
\text{\ \ \ \ }- Include the number of tables, total columns, and any notable relationships or patterns.\\
\\
4. Classification:\\
For the database new schema, classify it as SINGLE or MULTIPLE based on the tables number.\\
\text{\ \ \ \ }- if tables number >= 2: predict MULTIPLE\\
\text{\ \ \ \ }- elif only one table: predict SINGLE\\
\\
==============================\\
Here is a typical example:\\
\textbf{[Database Schema]}\\
\textbf{[DB\_ID]} dorm\_1\\
\textbf{[Schema]}\\
\# Table: Student\\
\text{[}\\
  \text{\ \ \ \ }(stuid, And This is a id type column),\\
  \text{\ \ \ \ }(lname, Value examples: [`Smith', `Pang', `Lee', `Adams', `Nelson', `Wilson'].),\\
  \text{\ \ \ \ }(fname, Value examples: [`Eric', `Lisa', `David', `Sarah', `Paul', `Michael'].),\\
  \text{\ \ \ \ }(age, Value examples: [18, 20, 17, 19, 21, 22].),\\
  \text{\ \ \ \ }(sex, Value examples: [`M', `F'].),\\
  \text{\ \ \ \ }(major, Value examples: [600, 520, 550, 50, 540, 100].),\\
  \text{\ \ \ \ }(advisor, And this is a number type column),\\
  \text{\ \ \ \ }(city code, Value examples: [`PIT', `BAL', `NYC', `WAS', `HKG', `PHL'].)\\
\text{]}\\
% \end{promptbox}
% \end{figure*}
% \begin{figure*}[!h]
% \begin{promptbox}[Prompt template for Processor Agent]
\# Table: Dorm\\
\text{[}\\
  \text{\ \ \ \ }(dormid, And This is a id type column),\\
  \text{\ \ \ \ }(dorm name, Value examples: [`Anonymous Donor Hall', `Bud Jones Hall', `Dorm-plex 2000', `Fawlty Towers', `Grad Student Asylum', `Smith Hall'].),\\
  \text{\ \ \ \ }(student capacity, Value examples: [40, 85, 116, 128, 256, 355].),
  (gender, Value examples: [`X', `F', `M'].)\\
\text{]}\\
\# Table: Dorm\_amenity\\
\text{[}\\
  \text{\ \ \ \ }(amenid, And This is a id type column),\\
  \text{\ \ \ \ }(amenity name, Value examples: [`4 Walls', `Air Conditioning', `Allows Pets', `Carpeted Rooms', `Ethernet Ports', `Heat'].)\\
\text{]}\\
\# Table: Has\_amenity\\
\text{[}\\
  \text{\ \ \ \ }(dormid, And This is a id type column),\\
  \text{\ \ \ \ }(amenid, And This is a id type column)\\
\text{]}\\
\# Table: Lives\_in\\
\text{[}\\
  \text{\ \ \ \ }(stuid, And This is a id type column),\\
  \text{\ \ \ \ }(dormid, And This is a id type column),\\
  \text{\ \ \ \ }(room number, And this is a number type column)\\
\text{]}\\
\\
\textbf{[Query]}\\
Find the first name of students who are living in the Smith Hall, and count them by a pie chart\\
\\
Now we can think step by step\\
\textbf{[Filtered Schema]}\\
\{\\
  "Student": ["stuid", "fname"],\\
  "Dorm": ["dormid", "dorm name"],\\
  "Lives\_in": ["stuid", "dormid"]\\
\}\\
\\
\textbf{[New Schema]}\\
\# Table: Student\\
\text{[}\\
  (stuid, And This is a id type column),\\
  (fname, Value examples: [`Eric', `Lisa', `David', `Sarah', `Paul', `Michael'].),\\
\text{]}\\
% \end{promptbox}
% \end{figure*}
% \begin{figure*}[!h]
% \begin{promptbox}[Prompt template for Processor Agent]
\# Table: Dorm
\text{[}\\
  (dormid, And This is a id type column),\\
  (dorm name, Value examples: [`Anonymous Donor Hall', `Bud Jones Hall', `Dorm-plex 2000', `Fawlty Towers', `Grad Student Asylum', `Smith Hall'].),\\
\text{]}\\
\# Table: Lives\_in\\
\text{[}\\
  (stuid, And This is a id type column),\\
  (dormid, And This is a id type column),\\
\text{]}\\
\textbf{[Augmented Explanation]}\\
The filtered schema consists of 3 tables (Student, Dorm, and Lives\_in) with a total of 6 relevant columns. There is a many-to-one relationship between Student and Dorm through the Lives\_in junction table. The query involves joining these three tables to find students living in a specific dorm (Smith Hall).\\
\\
Key points:\\
1. The Lives\_in table acts as a bridge between Student and Dorm, allowing for the association of students with their dorms.\\
2. The `dorm name' column in the Dorm table is crucial for filtering the specific dorm (Smith Hall).\\
3. The `fname' column from the Student table is required for the final output.\\
\\
\textbf{[Classification]}\\
MULTIPLE\\
\\
==============================\\
Here is a new question:\\
\\
\textbf{[DB\_ID]} \{db\_id\}\\
\textbf{[Database Schema]}\\
\{db\_schema\}\\
\\
\textbf{[Query]}\\
\{query\}\\
\\
Now give your answer following this format strictly without other explanation:\\
\\
\textbf{[Filtered Schema]}\\
\\
\textbf{[New Schema]}\\
\\
\textbf{[Augmented Explanation]}\\
\\
\textbf{[Classification]}\\
\\
\end{promptbox}
% \end{figure*}

% \subsection{Composer Agent Prompt}
% \label{composer_prompt}
% \begin{figure*}[!h]
\begin{promptbox}[Prompt template for multiple classification]
Given a [Database schema] with [Augmented Explanation] and a [Question], generate a valid VQL (Visualization Query Language) sentence. VQL is similar to SQL but includes visualization components. \\
\\
Now you can think step by step, following these instructions below. \\
\textbf{[Background]} \\
VQL Structure:\\
Visualize [TYPE] SELECT [COLUMNS] FROM [TABLES] [JOIN] [WHERE] [GROUP BY] [ORDER BY] [BIN BY]\\
\\
You can consider a VQL sentence as "VIS TYPE + SQL + BINNING"\\
You must consider which part in the sketch is necessary, which is unnecessary, and construct a specific sketch for the natural language query.\\
\\
Key Components:\\
1. Visualization Type: bar, pie, line, scatter, stacked bar, grouped line, grouped scatter\\
2. SQL Components: SELECT, FROM, JOIN, WHERE, GROUP BY, ORDER BY\\
3. Binning: BIN [COLUMN] BY [INTERVAL], [INTERVAL]: [YEAR, MONTH, DAY, WEEKDAY]\\
\\
When generating VQL, we should always consider special rules and constraints:\\
\textbf{[Special Rules]} \\
a. For simple visualizations:\\
    \text{\ \ \ \ }- SELECT exactly TWO columns, X-axis and Y-axis(usually aggregate function)\\
b. For complex visualizations (STACKED BAR, GROUPED LINE, GROUPED SCATTER):\\
    \text{\ \ \ \ }- SELECT exactly THREE columns in this order!!!:\\
        \text{\ \ \ \ }\text{\ \ \ \ }1. X-axis\\
        \text{\ \ \ \ }\text{\ \ \ \ }2. Y-axis (aggregate function)\\
        \text{\ \ \ \ }\text{\ \ \ \ }3. Grouping column\\
c. When "COLORED BY" is mentioned in the question:\\
    \text{\ \ \ \ }- Use complex visualization type(STACKED BAR for bar charts, GROUPED LINE for line charts, GROUPED SCATTER for scatter charts)\\
    \text{\ \ \ \ }- Make the "COLORED BY" column the third SELECT column\\
    \text{\ \ \ \ }- Do NOT include "COLORED BY" in the final VQL\\     
d. Aggregate Functions:\\
    \text{\ \ \ \ }- Use COUNT for counting occurrences\\
    \text{\ \ \ \ }- Use SUM only for numeric columns\\
    \text{\ \ \ \ }- When in doubt, prefer COUNT over SUM\\
e. Time based questions:\\
    \text{\ \ \ \ }- Always use BIN BY clause at the end of VQL sentence\\
    \text{\ \ \ \ }- When you meet the questions including "year", "month", "day", "weekday"\\
    \text{\ \ \ \ }- Avoid using window function, just use BIN BY to deal with time base queries\\
% \end{promptbox}
% \end{figure*}
% \begin{figure*}[!h]
% \begin{promptbox}[Prompt template for multiple classification]
\textbf{[Constraints]} \\
- In SELECT <column>, make sure there are at least two selected!!!\\
- In FROM <table> or JOIN <table>, do not include unnecessary table\\
- Use only table names and column names from the given database schema\\
- Enclose string literals in single quotes\\
- If [Value examples] of <column> has `None' or None, use JOIN <table> or WHERE <column> is NOT NULL is better\\
- Ensure GROUP BY precedes ORDER BY for distinct values\\
- NEVER use window functions in SQL\\
\\
Now we could think step by step:\\
1. First choose visualize type and binning, then construct a specific sketch for the natural language query\\
2. Second generate SQL components following the sketch.\\
3. Third add Visualize type and BINNING into the SQL components to generate final VQL\\
\\
==============================\\
Here is a typical example:\\
\textbf{[Database Schema]}\\
\# Table: Orders, (orders)\\
\text{[}\\
  \text{\ \ \ \ }(order\_id, order id, And this is a id type column),\\
  \text{\ \ \ \ }(customer\_id, customer id, And this is a id type column),\\
  \text{\ \ \ \ }(order\_date, order date, Value examples: [`2023-01-15', `2023-02-20', `2023-03-10'].),\\
  \text{\ \ \ \ }(total\_amount, total amount, Value examples: [100.00, 200.00, 300.00, 400.00, 500.00].)\\
\text{]}\\
\# Table: Customers, (customers)\\
\text{[}\\
  \text{\ \ \ \ }(customer\_id, customer id, And this is a id type column),\\
  \text{\ \ \ \ }(customer\_name, customer name, Value examples: [`John', `Emma', `Michael', `Sophia', `William'].),\\
  \text{\ \ \ \ }(customer\_type, customer type, Value examples: [`Regular', `VIP', `New'].)\\
\text{]}\\
\textbf{[Augmented Explanation]}\\
The filtered schema consists of 2 tables (Orders and Customers) with a total of 7 relevant columns. There is a one-to-many relationship between Customers and Orders through the customer\_id foreign key.\\
\\
Key points:\\
1. The Orders table contains information about individual orders, including the order date and total amount.\\
2. The Customers table contains customer information, including their name and type (Regular, VIP, or New).\\
3. The customer\_id column links the two tables, allowing us to associate orders with specific customers.\\
% \end{promptbox}
% \end{figure*}
% \begin{figure*}[!h]
% \begin{promptbox}[Prompt template for multiple classification]
4. The order\_date column in the Orders table will be used for monthly grouping and binning.\\
5. The total\_amount column in the Orders table needs to be summed for each group.\\
6. The customer\_type column in the Customers table will be used for further grouping and as the third dimension in the stacked bar chart.\\
\\

The query involves joining these two tables to analyze order amounts by customer type and month, which requires aggregation and time-based binning.\\
\\
\textbf{[Question]}\\
Show the total order amount for each customer type by month in a stacked bar chart.\\
\\
Decompose the task into sub tasks, considering [Background] [Special Rules] [Constraints], and generate the VQL after thinking step by step:\\
\\
\textbf{Sub task 1:} First choose visualize type and binning, then construct a specific sketch for the natural language query\\
Visualize type: STACKED BAR, BINNING: True\\
VQL Sketch:\\
Visualize STACKED BAR SELECT \_ , \_ , \_ FROM \_ JOIN \_ ON \_ GROUP BY \_ BIN \_ BY MONTH\\
\\
\textbf{Sub task 2:} Second generate SQL components following the sketch.\\
Let's think step by step:\\
1. We need to select 3 columns for STACKED BAR chart, order\_date as X-axis, SUM(total\_amout) as Y-axis, customer\_type as group column.\\
2. We need to join the Orders and Customers tables.\\
3. We need to group by customer type.\\
4. We do not need to use any window function for MONTH.\\
\\
\text{sql}\\
```sql\\
SELECT O.order\_date, SUM(O.total\_amount), C.customer\_type\\
FROM Orders AS O\\
JOIN Customers AS C ON O.customer\_id = C.customer\_id\\
GROUP BY C.customer\_type\\
```\\
\\
\textbf{Sub task 3:} Third add Visualize type and BINNING into the SQL components to generate final VQL\\
\textbf{Final VQL:}\\
Visualize STACKED BAR SELECT O.order\_date, SUM(O.total\_amount), C.customer\_type FROM Orders O JOIN Customers C ON O.customer\_id = C.customer\_id GROUP BY C.customer\_type BIN O.order\_date BY MONTH\\
\\
% \end{promptbox}
% \end{figure*}
% \begin{figure*}[!h]
% \begin{promptbox}[Prompt template for multiple classification]
==============================\\
Here is a new question:\\
\\
\textbf{[Database Schema]}\\
\{desc\_str\}\\
\\
\textbf{[Augmented Explanation]}\\
\{augmented\_explanation\}\\
\\
\textbf{[Query]}\\
\{query\}\\
\\
Now, please generate a VQL sentence for the database schema and question after thinking step by step.\\

\end{promptbox}
% \end{figure*}


% \begin{figure*}[!h]
\begin{promptbox}[Prompt template for single classification]
Given a [Database schema] with [Augmented Explanation] and a [Question], generate a valid VQL (Visualization Query Language) sentence. VQL is similar to SQL but includes visualization components. \\
\\
Now you can think step by step, following these instructions below. \\
\textbf{[Background]} \\
VQL Structure:\\
Visualize [TYPE] SELECT [COLUMNS] FROM [TABLES] [JOIN] [WHERE] [GROUP BY] [ORDER BY] [BIN BY]\\
\\
You can consider a VQL sentence as "VIS TYPE + SQL + BINNING"\\
You must consider which part in the sketch is necessary, which is unnecessary, and construct a specific sketch for the natural language query.\\
\\
Key Components:\\
1. Visualization Type: bar, pie, line, scatter, stacked bar, grouped line, grouped scatter\\
2. SQL Components: SELECT, FROM, JOIN, WHERE, GROUP BY, ORDER BY\\
3. Binning: BIN [COLUMN] BY [INTERVAL], [INTERVAL]: [YEAR, MONTH, DAY, WEEKDAY]\\
\\
When generating VQL, we should always consider special rules and constraints:\\
\textbf{[Special Rules]} \\
a. For simple visualizations:\\
    \text{\ \ \ \ }- SELECT exactly TWO columns, X-axis and Y-axis(usually aggregate function)\\
b. For complex visualizations (STACKED BAR, GROUPED LINE, GROUPED SCATTER):\\
    \text{\ \ \ \ }- SELECT exactly THREE columns in this order!!!:\\
        \text{\ \ \ \ }\text{\ \ \ \ }1. X-axis\\
        \text{\ \ \ \ }\text{\ \ \ \ }2. Y-axis (aggregate function)\\
        \text{\ \ \ \ }\text{\ \ \ \ }3. Grouping column\\
c. When "COLORED BY" is mentioned in the question:\\
    \text{\ \ \ \ }- Use complex visualization type(STACKED BAR for bar charts, GROUPED LINE for line charts, GROUPED SCATTER for scatter charts)\\
    \text{\ \ \ \ }- Make the "COLORED BY" column the third SELECT column\\
    \text{\ \ \ \ }- Do NOT include "COLORED BY" in the final VQL\\     
d. Aggregate Functions:\\
    \text{\ \ \ \ }- Use COUNT for counting occurrences\\
    \text{\ \ \ \ }- Use SUM only for numeric columns\\
    \text{\ \ \ \ }- When in doubt, prefer COUNT over SUM\\
e. Time based questions:\\
    \text{\ \ \ \ }- Always use BIN BY clause at the end of VQL sentence\\
    \text{\ \ \ \ }- When you meet the questions including "year", "month", "day", "weekday"\\
    \text{\ \ \ \ }- Avoid using window function, just use BIN BY to deal with time base queries\\
% \end{promptbox}
% \end{figure*}
% \begin{figure*}[!h]
% \begin{promptbox}[Prompt template for single classification]
\textbf{[Constraints]} \\
- In SELECT <column>, make sure there are at least two selected!!!\\
- In FROM <table> or JOIN <table>, do not include unnecessary table\\
- Use only table names and column names from the given database schema\\
- Enclose string literals in single quotes\\
- If [Value examples] of <column> has `None' or None, use JOIN <table> or WHERE <column> is NOT NULL is better\\
- Ensure GROUP BY precedes ORDER BY for distinct values\\
- NEVER use window functions in SQL\\
\\
Now we could think step by step:\\
1. First choose visualize type and binning, then construct a specific sketch for the natural language query\\
2. Second generate SQL components following the sketch.\\
3. Third add Visualize type and BINNING into the SQL components to generate final VQL\\
\\
==============================\\
Here is a typical example:\\
\textbf{[Database Schema]}\\
\# Table: course, (course)\\
\text{[}\\
  \text{\ \ \ \ }(course\_id, course id, Value examples: [101, 696, 656, 659]. And this is an id type column),\\
  \text{\ \ \ \ }(title, title, Value examples: [`Geology', `Differential Geometry', `Compiler Design', `International Trade', `Composition and Literature', `Environmental Law'].),\\
  \text{\ \ \ \ }(dept\_name, dept name, Value examples: [`Cybernetics', `Finance', `Psychology', `Accounting', `Mech. Eng.', `Physics'].),\\
  \text{\ \ \ \ }(credits, credits, Value examples: [3, 4].)\\
\text{]}\\
\# Table: section, (section)\\
\text{[}\\
  \text{\ \ \ \ }(course\_id, course id, Value examples: [362, 105, 960, 468]. And this is an id type column),\\
  \text{\ \ \ \ }(sec\_id, sec id, Value examples: [1, 2, 3]. And this is an id type column),\\
  \text{\ \ \ \ }(semester, semester, Value examples: [`Fall', `Spring'].),\\
  \text{\ \ \ \ }(year, year, Value examples: [2002, 2006, 2003, 2007, 2010, 2008].),\\
  \text{\ \ \ \ }(building, building, Value examples: [`Saucon', `Taylor', `Lamberton', `Power', `Fairchild', `Main'].),\\
  \text{\ \ \ \ }(room\_number, room number, Value examples: [180, 183, 134, 143].),\\
  \text{\ \ \ \ }(time\_slot\_id, time slot id, Value examples: [`D', `J', `M', `C', `E', `F']. And this is an id type column)\\
\text{]}\\
\textbf{[Augmented Explanation]}\\
The filtered schema consists of 2 tables (course and section) with a total of 11 relevant columns. There is a one-to-many relationship between course and section through the course\_id foreign key.\\
\\
% \end{promptbox}
% \end{figure*}
% \begin{figure*}[!h]
% \begin{promptbox}[Prompt template for single classification]
Key points:\\
1. The course table contains information about individual courses, including the course title, department, and credits.\\
2. The section table contains information about specific sections of courses, including the semester, year, building, room number, and time slot.\\
3. The course\_id column links the two tables, allowing us to associate sections with specific courses.\\
4. The dept\_name column in the course table will be used to filter for Psychology department courses.\\
5. The year column in the section table will be used for yearly grouping and binning.\\
6. We need to count the number of courses offered each year, which requires aggregation and time-based binning.\\
\\
The query involves joining these two tables to analyze the number of courses offered by the Psychology department each year, which requires aggregation and time-based binning.\\
\\
\textbf{[Question]}\\
Find the number of courses offered by Psychology department in each year with a line chart.\\
\\
Decompose the task into sub tasks, considering [Background] [Special Rules] [Constraints], and generate the VQL after thinking step by step:\\
\\
\textbf{Sub task 1:} First choose visualize type and binning, then construct a specific sketch for the natural language query\\
Visualize type: LINE, BINNING: True\\
VQL Sketch:\\
Visualize LINE SELECT \_ , \_ FROM \_ JOIN \_ ON \_ WHERE \_ BIN \_ BY YEAR\\
\\
\textbf{Sub task 2:} Second generate SQL components following the sketch.\\
Let's think step by step:\\
1. We need to select 2 columns for LINE chart, year as X-axis, COUNT(year) as Y-axis.\\
2. We need to join the course and section tables to get the number of courses offered by the Psychology department in each year.\\
3. We need to filter the courses by the Psychology department.\\
4. We do not need to use any window function for YEAR.\\
\\
\text{sql}\\
```sql\\
SELECT S.year, COUNT(S.year)\\
FROM course AS C\\
JOIN section AS S ON C.course\_id = S.course\_id\\
WHERE C.dept\_name = `Psychology'\\
```\\
\\
% \end{promptbox}
% \end{figure*}
% \begin{figure*}[!h]
% \begin{promptbox}[Prompt template for single classification]
\textbf{Sub task 3:} Third add Visualize type and BINNING into the SQL components to generate final VQL\\
\textbf{Final VQL:}\\
Visualize LINE SELECT S.year, COUNT(S.year) FROM course C JOIN section S ON C.course\_id = S.course\_id WHERE C.dept\_name = `Psychology' BIN S.year BY YEAR\\
\\
==============================\\
Here is a new question:\\
\\
\textbf{[Database Schema]}\\
\{desc\_str\}\\
\\
\textbf{[Augmented Explanation]}\\
\{augmented\_explanation\}\\
\\
\textbf{[Query]}\\
\{query\}\\
\\
Now, please generate a VQL sentence for the database schema and question after thinking step by step.\\

\end{promptbox}
% \end{figure*}

% \subsection{Validator Agent Prompt}
% \label{validator_prompt}
% \begin{figure*}
\begin{promptbox}[Prompt template for Validator Agent]
As an AI assistant specializing in data visualization and VQL (Visualization Query Language), your task is to refine a VQL query that has resulted in an error. Please approach this task systematically, thinking step by step.\\
\textbf{[Background]}\\
VQL Structure:\\
Visualize [TYPE] SELECT [COLUMNS] FROM [TABLES] [JOIN] [WHERE] [GROUP BY] [ORDER BY] [BIN BY]\\
\\
You can consider a VQL sentence as "VIS TYPE + SQL + BINNING"\\
\\
Key Components:\\
1. Visualization Type: bar, pie, line, scatter, stacked bar, grouped line, grouped scatter\\
2. SQL Components: SELECT, FROM, JOIN, WHERE, GROUP BY, ORDER BY\\
3. Binning: BIN [COLUMN] BY [INTERVAL], [INTERVAL]: [YEAR, MONTH, DAY, WEEKDAY]\\
\\
When refining VQL, we should always consider special rules and constraints:\\
\textbf{[Special Rules]} \\
a. For simple visualizations:\\
    \text{\ \ \ \ }- SELECT exactly TWO columns, X-axis and Y-axis(usually aggregate function)\\
b. For complex visualizations (STACKED BAR, GROUPED LINE, GROUPED SCATTER):\\
    \text{\ \ \ \ }- SELECT exactly THREE columns in this order!!!:\\
        \text{\ \ \ \ }\text{\ \ \ \ }1. X-axis\\
        \text{\ \ \ \ }\text{\ \ \ \ }2. Y-axis (aggregate function)\\
        \text{\ \ \ \ }\text{\ \ \ \ }3. Grouping column\\
c. When "COLORED BY" is mentioned in the question:\\
    \text{\ \ \ \ }- Use complex visualization type(STACKED BAR for bar charts, GROUPED LINE for line charts, GROUPED SCATTER for scatter charts)\\
    \text{\ \ \ \ }- Make the "COLORED BY" column the third SELECT column\\
    \text{\ \ \ \ }- Do NOT include "COLORED BY" in the final VQL\\     
d. Aggregate Functions:\\
    \text{\ \ \ \ }- Use COUNT for counting occurrences\\
    \text{\ \ \ \ }- Use SUM only for numeric columns\\
    \text{\ \ \ \ }- When in doubt, prefer COUNT over SUM
% \end{promptbox}
% \end{figure*}

% \begin{figure*}
% \begin{promptbox}[Prompt template for Validator Agent]
e. Time based questions:\\
    \text{\ \ \ \ }- Always use BIN BY clause at the end of VQL sentence\\
    \text{\ \ \ \ }- When you meet the questions including "year", "month", "day", "weekday"\\
    \text{\ \ \ \ }- Avoid using time function, just use BIN BY to deal with time base queries\\
\\
\textbf{[Constraints]} \\
- In FROM <table> or JOIN <table>, do not include unnecessary table\\
- Use only table names and column names from the given database schema\\
- Enclose string literals in single quotes\\
- If [Value examples] of <column> has `None' or None, use JOIN <table> or WHERE <column> is NOT NULL is better\\
- ENSURE GROUP BY clause cannot contain aggregates\\
- NEVER use date functions in SQL\\
\\
\textbf{[Query]} \\
\{query\}\\
\\
\textbf{[Database info]} \\
\{db\_info\}\\
\\
\textbf{[Current VQL]} \\
\{vql\}\\
\\
\textbf{[Error]} \\
\{error\}\\
\\
Now, please analyze and refine the VQL, please provide:\\
\\
\textbf{[Explanation]}\\
\text{[}Provide a detailed explanation of your analysis process, the issues identified, and the changes made. Reference specific steps where relevant.\text{]}\\
\\
\textbf{[Corrected VQL]}\\
\text{[}Present your corrected VQL here. Ensure it's on a single line without any line breaks.\text{]}\\
\\
Remember:\\
- The SQL components must be parseable by DuckDB.\\
- Do not change rows when you generate the VQL.\\
- Always verify your answer carefully before submitting.\\
\end{promptbox}
% \end{figure*}

\end{document}

