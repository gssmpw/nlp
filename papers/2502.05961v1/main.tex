\pdfoutput=1 %  
\PassOptionsToPackage{table,xcdraw}{xcolor}

\documentclass[manuscript,screen,authorversion,review=false,timestamp=false,nonacm]{acmart}

% \documentclass[acmsmall,manuscript,screen]{acmart}
% \documentclass[acmsmall,nonacm]{acmart}

\usepackage{fancyhdr}
\AtBeginDocument{%
    \addtolength{\footskip}{2.0\baselineskip}%
    \fancyfoot[L]{\textit{Preprint.}}%
}

% \usepackage{tikz,lipsum,lmodern}
\usepackage[most]{tcolorbox}
\tcbuselibrary{listings}
\usepackage{booktabs}
\usepackage{graphicx}
\usepackage{colortbl} 
\usepackage{footnote}
% \usepackage[commandnameprefix=always, draft]{changes}
\usepackage[commandnameprefix=always, final]{changes}

% This changes the commands:
% \added → \chadded
% \deleted → \chdeleted
% \replaced → \chreplaced
% \highlight → \chhighlight
% \comment → \chcomment
% For the text you add you put it in this format: \added{This is new text.}

% For text you delete you put it in this format: \deleted{This text has been removed.}

% For text you replace you put it in this format: \replaced{new text}{old text}

% You can change the markup style (e.g. color) which we did with this command: \usepackage[final]{changes}
% % Use 'final' option to accept all changes
% \definechangesauthor[name={Varun}, color=orange]{vr} 
% 

% \usepackage[table,xcdraw]{xcolor}

\definecolor{codegreen}{rgb}{0,0.6,0}
\definecolor{codegray}{rgb}{0.5,0.5,0.5}
\definecolor{codepurple}{rgb}{0.58,0,0.82}
\definecolor{backcolour}{rgb}{0.95,0.95,0.92}

\lstdefinestyle{mystyle}{
    backgroundcolor=\color{backcolour},
    commentstyle=\color{codepurple},
    keywordstyle=\color{codepurple},
    numberstyle=\tiny\color{codegray},
    stringstyle=\color{codegreen},
    basicstyle=\ttfamily\footnotesize,
    breakatwhitespace=false,
    breaklines=true,
    captionpos=b,
    keepspaces=true,
    numbers=left,
    numbersep=5pt,
    showspaces=false,
    showstringspaces=false,
    showtabs=false,
    tabsize=2
}
\lstset{style=mystyle}

\usepackage[utf8]{inputenc}
\usepackage{csquotes}
\renewcommand{\mkbegdispquote}[2]{\itshape}
\usepackage{cleveref}
\usepackage{natbib}
\crefformat{section}{\S#2#1#3} % see manual of cleveref, section 8.2.1
\crefformat{subsection}{\S#2#1#3}
\crefformat{subsubsection}{\S#2#1#3}


\settopmatter{printacmref=false} % Removes citation information below abstract
\renewcommand\footnotetextcopyrightpermission[1]{} % removes footnote with conference information in first column
\pagestyle{plain} % removes running headers

%%
%% \BibTeX command to typeset BibTeX logo in the docs
\AtBeginDocument{%
  \providecommand\BibTeX{{%
    \normalfont B\kern-0.5em{\scshape i\kern-0.25em b}\kern-0.8em\TeX}}}


%% \BibTeX command to typeset BibTeX logo in the docs
\AtBeginDocument{%
  \providecommand\BibTeX{{%
    Bib\TeX}}}

\setcopyright{acmlicensed}
\copyrightyear{2025}
\acmYear{2025}

%% These commands are for a PROCEEDINGS abstract or paper.
% \acmConference[Preprint]{}{}{}
% \acmISBN{}


\usepackage{booktabs}
\usepackage[multiple]{footmisc}
\usepackage{threeparttable}
\usepackage{multirow}
\usepackage{array}
\usepackage{amsmath}
\usepackage{mathtools}
\usepackage{graphicx}
\usepackage{subcaption}
\usepackage{balance}
\usepackage{color}
\usepackage{wrapfig}
\usepackage{arydshln}
\usepackage{float}
\usepackage{hyphenat}
\usepackage{soul}
\usepackage{tabu}
\usepackage{multicol}
\usepackage{makecell}

\newif\ifdraft
\drafttrue %% turn on comments
% \draftfalse %% turn off comments
\ifdraft
        \def\vnr#1{{\color{blue}#1}}
        \def\sd#1{{\color{red}[SD: #1]}}
        \def\dc#1{{\color{green}[DC: #1]}}
        \def\amh#1{{\color{purple}[AMH: #1]}}
        
\else
        \def\vnr#1{}
        \def\sd#1{}
        \def\dc#1{}
        \def\amh#1{}        
\fi 

\definecolor{Purple}{RGB}{160,0,120}
\newcommand\siobhan[1]{\textcolor{Purple}{Siobhan: \@#1}}



\definecolor{CBGreen}{RGB}{0,128,0}
\newcommand\samantha[1]{\textcolor{CBGreen}{Samantha: \@#1}}

\definecolor{Blue}{RGB}{0,0,128}
\newcommand\shu[1]{\textcolor{Blue}{Shu: \@#1}}

\begin{document}

%%
%% The "title" command has an optional parameter,
%% allowing the author to define a "short title" to be used in page headers.
\title{The Human Labour of Data Work: Capturing Cultural Diversity through \textsc{World Wide Dishes}}



\author{Siobhan Mackenzie Hall}
\affiliation{%
  \institution{University of Oxford}
  \country{United Kingdom}
  }
\email{siobhan.hall@nds.ox.ac.uk}

\author{Samantha Dalal}
\affiliation{%
  \institution{University of Colorado Boulder}
  \country{USA}
  }
\email{samantha.dalal@colorado.edu}

\author{Raesetje Sefala}
\affiliation{%
  \institution{DAIR Institute, Mila, McGill University}
  \country{South Africa}
  }
\email{raesetje@dair-institute.org}

\author{Foutse Yuehgoh}
\affiliation{%
  \institution{Conservatoire National des Arts et M\'etiers (CNAM)}
  \country{France}
  }
\email{foutse.yuehgoh@devinci.fr}

\author{Aisha Alaagib}
\affiliation{%
  \institution{Independent researcher}
  \country{Sudan}
  }
\email{aalaagib@aimsammi.org}

\author{Imane Hamzaoui}
\affiliation{%
  \institution{École nationale Supérieure d'Informatique Algiers, New York University Abu Dhabi}
  \country{Algeria}
  }
\email{ji\_hamzaoui@esi.dz}

\author{Shu Ishida}
\affiliation{%
  \institution{University of Oxford}
  \country{United Kingdom}
  }
\email{shu.ishida@oxon.org}

\author{Jabez Magomere}
\affiliation{%
  \institution{University of Oxford}
  \country{United Kingdom}
  }
\email{jabez.magomere@oii.ac.uk}

\author{Lauren Crais}
\affiliation{%
  \institution{University of Oxford}
  \country{United Kingdom}
  }
\email{lauren.crais@law.ox.ac.uk}

\author{Aya Salama}
\affiliation{%
  \institution{Microsoft Egypt}
  \country{Egypt}
  }
\email{salamaaya@microsoft.com}

\author{Tejumade Afonja}
\affiliation{%
  \institution{CISPA Helmholtz Center for Information Security \& AI Saturdays Lagos}
  \country{Germany \& Nigeria}
  }
\email{tejumade.afonja@cispa.de}

%%
%% By default, the full list of authors will be used in the page
%% headers. Often, this list is too long, and will overlap
%% other information printed in the page headers. This command allows
%% the author to define a more concise list
%% of authors' names for this purpose.
\renewcommand{\shortauthors}{Hall et al.}

%%
%% The abstract is a short summary of the work to be presented in the
%% article.
\begin{abstract}

During the early stages of interface design, designers need to produce multiple sketches to explore a design space.  Design tools often fail to support this critical stage, because they insist on specifying more details than necessary. Although recent advances in generative AI have raised hopes of solving this issue, in practice they fail because expressing loose ideas in a prompt is impractical. In this paper, we propose a diffusion-based approach to the low-effort generation of interface sketches. It breaks new ground by allowing flexible control of the generation process via three types of inputs: A) prompts, B) wireframes, and C) visual flows. The designer can provide any combination of these as input at any level of detail, and will get a diverse gallery of low-fidelity solutions in response. The unique benefit is that large design spaces can be explored rapidly with very little effort in input-specification. We present qualitative results for various combinations of input specifications. Additionally, we demonstrate that our model aligns more accurately with these specifications than other models. 

% OLD ABSTRACT
%When sketching Graphical User Interfaces (GUIs), designers need to explore several aspects of visual design simultaneously, such as how to guide the user’s attention to the right aspects of the design while making the intended functionality visible. Although current Large Language Models (LLMs) can generate GUIs, they do not offer the finer level of control necessary for this kind of exploration. To address this, we propose a diffusion-based model with multi-modal conditional generation. In practice, our model optionally takes semantic segmentation, prompt guidance, and flow direction to generate multiple GUIs that are aligned with the input design specifications. It produces multiple examples. We demonstrate that our approach outperforms baseline methods in producing desirable GUIs and meets the desired visual flow.

% Designing visually engaging Graphical User Interfaces (GUIs) is a challenge in HCI research. Effective GUI design must balance visual properties, like color and positioning, with user behaviors to ensure GUIs easy to comprehend and guide attention to critical elements. Modern GUIs, with their complex combinations of text, images, and interactive components, make it difficult to maintain a coherent visual flow during design.
% Although current Large Language Models (LLMs) can generate GUIs, they often lack the fine control necessary for ensuring a coherent visual flow. To address this, we propose a diffusion-based model that effectively handles multi-modal conditional generation. Our model takes semantic segmentation, optional prompt guidance, and ordered viewing elements to generate high-fidelity GUIs that are aligned with the input design specifications.
% We demonstrate that our approach outperforms baseline methods in producing desirable GUIs and meets the desired visual flow. Moreover, a user study involving XX designers indicates that our model enhances the efficiency of the GUI design ideation process and provides designers with greater control compared to existing methods.    



% %%%%%%%%%%%%%%%%%%%%%%%%%%%%%%%%%%%%%%%%%%%%%%%%%%%%%%
% % Writing Clinic Comments:
% %%%%%%%%%%%%%%%%%%%%%%%%%%%%%%%%%%%%%%%%%%%%%%%%%%%%%%
% % Define: Effective UI design
% % Motivate GANs and write in full form.
% % LLMs vs ControlNet vs GANs
% % Say something about the Figma plugin?
% % Write the work is novel or what has been done before
% % What is desirable UI and how to evalutate that?
% % Visual Flow - main theme (center around it)
% % Re-Title: use word Flow!
% % Use ControlNet++ & SPADE for abstract.
% % Write about input/output. 
% % Why better than previous work?
% %%%%%%%%%%%%%%%%%%%%%%%%%%%%%%%%%%%%%%%%%%%%%%%%%%%%%

% % v2:
% % \noindent \textcolor{red}{\textbf{NEW Abstract!} (Post Writing Clinic 1 - 25-Jun)}

% % \noindent \textcolor{red}{----------------------------------------------------------------------}

% % \noindent Designing user interfaces (UIs) is a time-consuming process, particularly for novice designers. 
% % Creating UI designs that are effective in market funneling or any other designer defined goal requires a good understanding of the visual flow to guide users' attention to UI elements in the desired order. 
% % While current Large Language Models (LLMs) can generate UIs from just prompts, they often lack finer pixel-precise control and fail to consider visual flow. 
% % In this work, we present a UI synthesis method that incorporates visual flow alongside prompts and semantic layouts. 
% % Our efficient approach uses a carefully designed Generative Adversarial Network (GAN) optimized for scenarios with limited data, making it more suitable than diffusion-based and large vision-language models.
% % We demonstrate that our method produces more "desirable" UIs according to the well-known contrast, repetition, alignment, and proximity principles of design. 
% % We further validate our method through comprehensive automatic non-reference, human-preference aligned network scoring and subjective human evaluations.
% % Finally, an evaluation with xx non-expert designers using our contributed Figma plugin shows that <method-name> improves the time-efficiency as well as the overall quality of the UI design development cycle.

% % \noindent \textcolor{red}{----------------------------------------------------------------------}


% \noindent \textcolor{blue}{\textbf{NEW Abstract!} (Pre Writing Clinic 9-July)}

% \noindent \textcolor{blue}{----------------------------------------------------------------------}

% \noindent Exploring different graphical user interface (GUI) design ideas is time-consuming, particularly for novice designers. 
% Given the segmentation masks, design requirement as prompt, and/or preferred visual flow, we aim to facilitate creative exploration for GUI design and generate different UI designs for inspiration.
% While current Vision Language Models (VLMs) can generate GUIs from just prompts, they often lack control over visual concepts and flow that are difficult to convey through language during the generation process. 
% In this work, we present FlowGenUI, a semantic map-guided GUI synthesis method that optionally incorporates visual flow information based on the user's choice alongside language prompts. 
% We demonstrate that our model not only creates more realistic GUIs but also creates "predictable" (how users pay attention to and order of looking at GUI elements) GUIs.
% Our approach uses Stable Diffusion (SD), a large paired image-text pretrained diffusion model with a rich latent space that we steer toward realistic GUIs using a trainable copy of SD's encoder for every condition (segmentation masks, prompts, and visual flow). 
% We further provide a semantic typography feature to create custom text-fonts and styles while also alleviating SD's inherent limitations in drawing coherent, meaningful and correct aspect-ratio text. 
% Finally, a subjective evaluation study of XX non-expert and expert designers demonstrates the efficiency and fidelity of our method.


% This process encourages creativity and prevents designers from falling into habitual patterns.


% ------------------------------------------------------------------
% Joongi Why is it important to create realistic GUI?
% I do not see how the Visual Flow given on the left hand side is reflected in the results on the right hand side. 
% I’d avoid making unsubstantiated claims about designers (falling into habitual patterns).
% The UIs you generate do not “align with users’ attention patterns” but rather try to control it (that’s what visual flow means)
% ------------------------------------------------------------------
% Comments - Writing Clinic - 9th July:
% Improve title. More names: FlowGen
% Figure 1: Use an inference time hand-drawn mask
% Figure 1: Show both workflows. Add a designer --> Input.
% Figure 1: Make them more diverse
% ------------------------------------------------------------------
% Designing graphical user interfaces (GUIs) requires human creativity and time. Designers often fall into habitual patterns, which can limit the exploration of new ideas. 
% To address this, we introduce FlowGenUI, a method that facilitates creative exploration and generates diverse GUI designs for inspiration. By using segmentation masks, design requirements as prompts, and/or selected visual flows, our approach enhances control over the visual concepts and flows during the generation process, which current Vision Language Models (VLMs) often lack.
% FlowGenUI uses Stable Diffusion (SD), a largely pretrained text-to-image diffusion model, and guides it to create realistic GUIs. 
% We achieve this by using a trainable copy of SD's encoder for each condition (segmentation masks, prompts, and visual flow). 
% This method enables the creation of more realistic and predictable GUIs that align with users' attention patterns and their preferred order of viewing elements.
% We also offer a semantic typography feature that creates custom text fonts and styles while addressing SD's limitations in generating coherent, meaningful, and correctly aspect-ratio text.
% Our approach's efficiency and fidelity are evaluated through a subjective user study involving XX designers. 
% The results demonstrate the effectiveness of FlowGenUI in generating high-quality GUI designs that meet user requirements and visual expectations.

% ---------------------------------------


%A critical and general issue remains while using such deep generative priors: creating coherent, meaningful and correct aspect-ratio text. 
%We tackle this issue within our framework and additionally provide a semantic typography feature to create custom text-fonts and styles. 


% %Creating UI designs that are effective in market funneling or any other designer-defined goal requires a good understanding of the visual flow to guide users' attention to UI elements in the desired order. 
% %While current largely pre-trained Vision Language Models (VLMs) can generate GUIs from just prompts, they often lack finer or pixel-precise control which can be crucial for many easy-to-understand visual concepts but difficult to convey through language. 
% % However, obtaining such pixe-level labels is an extremely expensive so we
% % For example - overlaying text on images with certain aspect ratios and two equally separated buttons 
% Additionally, all prior GUI generation work fails to consider visual flow information during the generation process. 
% We demonstrate that visual flow-informed generation not only creates more realistic and human-friendly GUIs but also creates "predictable" (how users pay attention to and order of looking at GUI elements) UIs that could be beneficial for designers for tasks like creating effective market funnels.
% In this work, we present a semantic map-guided GUI synthesis method that optionally incorporates visual flow information based on the user's choice alongside language prompts. 
% Our approach uses Stable Diffusion, a large (billions) paired image-text pretrained diffusion model with a rich latent space that we steer toward realistic GUIs using an ensemble of ControlNets. 
% % TODO: Mention it in 1 sentence:
% A critical and general issue remains while using such deep generative priors: creating coherent, meaningful and correct aspect-ratio text. 
% We tackle this issue within our framework and additionally provide a semantic typography feature to create custom text-fonts and styles. 
% To evaluate our method, we demonstrate that our method produces more "desirable" UIs according to the well-known contrast, repetition, alignment, and proximity principles of design. 
% % We further validate our method through comprehensive automatic non-reference and human-preference aligned scores. (TODO: Maybe Unskip if we get UIClip from Jason!)
% % TODO: Re-word this and only keep ideation cycles and time-efficiency.
% Finally, a subjective evaluation study of XX non-expert and expert designers demonstrates the efficiency and fidelity of our method.
% % improves the time-efficiency by quick iterations of the UI design ideation process.
% %Finally, an evaluation with xx non-expert designers using our contributed <method-name> improves the time-efficiency by quick iterations of the UI design ideation cycle.

%\noindent \textcolor{blue}{----------------------------------------------------------------------}


%In an evaluation with xx designers, we found that GenerativeLayout: 1) enhances designers' exploration by expanding the coverage of the design space, 2) reduces the time required for exploration, and 3) maintains a perceived level of control similar to that of manual exploration.



% Present-day graphical user interfaces (GUIs) exhibit diverse arrangements of text, graphics, and interactive elements such as buttons and menus, but representations of GUIs have not kept up. They do not encapsulate both semantic and visuo-spatial relationships among elements. %\color{red} 
% To seize machine learning's potential for GUIs more efficiently, \papername~ exploits graph neural networks to capture individual elements' properties and their semantic—visuo-spatial constraints in a layout. The learned representation demonstrated its effectiveness in multiple tasks, especially generating designs in a challenging GUI autocompletion task, which involved predicting the positions of remaining unplaced elements in a partially completed GUI. The new model's suggestions showed alignment and visual appeal superior to the baseline method and received higher subjective ratings for preference. 
% Furthermore, we demonstrate the practical benefits and efficiency advantages designers perceive when utilizing our model as an autocompletion plug-in.


% Overall pipeline: Maybe drop semantic typography / visual flow?
 
\end{abstract}

%%
%% The code below is generated by the tool at http://dl.acm.org/ccs.cfm.
%% Please copy and paste the code instead of the example below.
%%
\begin{CCSXML}
\end{CCSXML}

\keywords{}
\maketitle


% humans are sensitive to the way information is presented.

% introduce framing as the way we address framing. say something about political views and how information is represented.

% in this paper we explore if models show similar sensitivity.

% why is it important/interesting.



% thought - it would be interesting to test it on real world data, but it would be hard to test humans because they come already biased about real world stuff, so we tested artificial.


% LLMs have recently been shown to mimic cognitive biases, typically associated with human behavior~\citep{ malberg2024comprehensive, itzhak-etal-2024-instructed}. This resemblance has significant implications for how we perceive these models and what we can expect from them in real-world interactions and decisionmaking~\citep{eigner2024determinants, echterhoff-etal-2024-cognitive}.

The \textit{framing effect} is a well-known cognitive phenomenon, where different presentations of the same underlying facts affect human perception towards them~\citep{tversky1981framing}.
For example, presenting an economic policy as only creating 50,000 new jobs, versus also reporting that it would cost 2B USD, can dramatically shift public opinion~\cite{sniderman2004structure}. 
%%%%%%%% 图1:  %%%%%%%%%%%%%%%%
\begin{figure}[t]
    \centering
    \includegraphics[width=\columnwidth]{Figs/01.pdf}
    \caption{Performance comparison (Top-1 Acc (\%)) under various open-vocabulary evaluation settings where the video learners except for CLIP are tuned on Kinetics-400~\cite{k400} with frozen text encoders. The satisfying in-context generalizability on UCF101~\cite{UCF101} (a) can be severely affected by static bias when evaluating on out-of-context SCUBA-UCF101~\cite{li2023mitigating} (b) by replacing the video background with other images.}
    \label{fig:teaser}
\end{figure}


Previous research has shown that LLMs exhibit various cognitive biases, including the framing effect~\cite{lore2024strategic,shaikh2024cbeval,malberg2024comprehensive,echterhoff-etal-2024-cognitive}. However, these either rely on synthetic datasets or evaluate LLMs on different data from what humans were tested on. In addition, comparisons between models and humans typically treat human performance as a baseline rather than comparing patterns in human behavior. 
% \gabis{looks good! what do we mean by ``most studies'' or ``rarely'' can we remove those? or we want to say that we don't know of previous work doing both at the same time?}\gili{yeah the main point is that some work has done each separated, but not all of it together. how about now?}

In this work, we evaluate LLMs on real-world data. Rather than measuring model performance in terms of accuracy, we analyze how closely their responses align with human annotations. Furthermore, while previous studies have examined the effect of framing on decision making, we extend this analysis to sentiment analysis, as sentiment perception plays a key explanatory role in decision-making \cite{lerner2015emotion}. 
%Based on this, we argue that examining sentiment shifts in response to reframing can provide deeper insights into the framing effect. \gabis{I don't understand this last claim. Maybe remove and just say we extend to sentiment analysis?}

% Understanding how LLMs respond to framing is crucial, as they are increasingly integrated into real-world applications~\citep{gan2024application, hurlin2024fairness}.
% In some applications, e.g., in virtual companions, framing can be harnessed to produce human-like behavior leading to better engagement.
% In contrast, in other applications, such as financial or legal advice, mitigating the effect of framing can lead to less biased decisions.
% In both cases, a better understanding of the framing effect on LLMs can help develop strategies to mitigate its negative impacts,
% while utilizing its positive aspects. \gabis{$\leftarrow$ reading this again, maybe this isn't the right place for this paragraph. Consider putting in the conclusion? I think that after we said that people have worked on it, we don't necessarily need this here and will shorten the long intro}


% If framing can influence their outputs, this could have significant societal effects,
% from spreading biases in automated decision-making~\citep{ghasemaghaei2024understanding} to reducing public trust in AI-generated content~\citep{afroogh2024trust}. 
% However, framing is not inherently negative -- understanding how it affects LLM outputs can offer valuable insights into both human and machine cognition.
% By systematically investigating the framing effect,


%It is therefore crucial to systematically investigate the framing effect, to better understand and mitigate its impact. \gabis{This paragraph is important - I think that right now it's saying that we don't want models to be influenced by framing (since we want to mitigate its impact, right?) When we talked I think we had a more nuanced position?}




To better understand the framing effect in LLMs in comparison to human behavior,
we introduce the \name{} dataset (Section~\ref{sec:data}), comprising 1,000 statements, constructed through a three-step process, as shown in Figure~\ref{fig:fig1}.
First, we collect a set of real-world statements that express a clear negative or positive sentiment (e.g., ``I won the highest prize'').
%as exemplified in Figure~\ref{fig:fig1} -- ``I won the highest prize'' positive base statement. (2) next,
Second, we \emph{reframe} the text by adding a prefix or suffix with an opposite sentiment (e.g., ``I won the highest prize, \emph{although I lost all my friends on the way}'').
Finally, we collect human annotations by asking different participants
if they consider the reframed statement to be overall positive or negative.
% \gabist{This allows us to quantify the extent of \textit{sentiment shifts}, which is defined as labeling the sentiment aligning with the opposite framing, rather then the base sentiment -- e.g., voting ``negative'' for the statement ``I won the highest prize, although I lost all my friends on the way'', as it aligns with the opposite framing sentiment.}
We choose to annotate Amazon reviews, where sentiment is more robust, compared to e.g., the news domain which introduces confounding variables such as prior political leaning~\cite{druckman2004political}.


%While the implications of framing on sensitive and controversial topics like politics or economics are highly relevant to real-world applications, testing these subjects in a controlled setting is challenging. Such topics can introduce confounding variables, as annotators might rely on their personal beliefs or emotions rather than focusing solely on the framing, particularly when the content is emotionally charged~\cite{druckman2004political}. To balance real-world relevance with experimental reliability, we chose to focus on statements derived from Amazon reviews. These are naturally occurring, sentiment-rich texts that are less likely to trigger strong preexisting biases or emotional reactions. For instance, a review like ``The book was engaging'' can be framed negatively without invoking specific cultural or political associations. 

 In Section~\ref{sec:results}, we evaluate eight state-of-the-art LLMs
 % including \gpt{}~\cite{openai2024gpt4osystemcard}, \llama{}~\cite{dubey2024llama}, \mistral{}~\cite{jiang2023mistral}, \mixtral{}~\cite{mistral2023mixtral}, and \gemma{}~\cite{team2024gemma}, 
on the \name{} dataset and compare them against human annotations. We find  that LLMs are influenced by framing, somewhat similar to human behavior. All models show a \emph{strong} correlation ($r>0.57$) with human behavior.
%All models show a correlation with human responses of more than $0.55$ in Pearson's $r$ \gabis{@Gili check how people report this?}.
Moreover, we find that both humans and LLMs are more influenced by positive reframing rather than negative reframing. We also find that larger models tend to be more correlated with human behavior. Interestingly, \gpt{} shows the lowest correlation with human behavior. This raises questions about how architectural or training differences might influence susceptibility to framing. 
%\gabis{this last finding about \gpt{} stands in opposition to the start of the statement, right? Even though it's probably one of the largest models, it doesn't correlate with humans? If so, better to state this explicitly}

This work contributes to understanding the parallels between LLM and human cognition, offering insights into how cognitive mechanisms such as the framing effect emerge in LLMs.\footnote{\name{} data available at \url{https://huggingface.co/datasets/gililior/WildFrame}\\Code: ~\url{https://github.com/SLAB-NLP/WildFrame-Eval}}

%\gabist{It also raises fundamental philosophical and practical questions -- should LLMs aim to emulate human-like behavior, even when such behavior is susceptible to harmful cognitive biases? or should they strive to deviate from human tendencies to avoid reproducing these pitfalls?}\gabis{$\leftarrow$ also following Itay's comment, maybe this is better in the dicsussion, since we don't address these questions in the paper.} %\gabis{This last statement brings the nuance back, so I think it contradicts the previous parapgraph where we talked about ``mitigating'' the effect of framing. Also, I think it would be nice to discuss this a bit more in depth, maybe in the discussion section.}






\section{{Literature Review}}
\label{sec:lit-review}
We identify how bias manifests in ML systems, particularly in T2I generative models, as well as corresponding efforts within the ML discipline to mitigate these biases. We then turn to popular bias mitigation strategies in ML. Shifting to social computing literature, we trace the role of datasets as infrastructures in encoding and perpetuating biases. Finally, we identify how participatory approaches have been applied in ML to mitigate biases in datasets. We point to existing work in CSCW that documents the double-edged nature of participatory approaches: participation can be burdensome.  Our paper builds upon this foundation to demonstrate how achieving the benefits of participation (e.g., alignment and meaningful shifts in power) requires attending to the invisible labour of building infrastructures for participation.

\subsection{Bias in machine learning}
ML has many applications in technology used in the modern world, such as search engines, image captioning, and content generation. ML models, including those responsible for content generation, produce biased outputs~\cite{agarwal2021evaluating_bias,berg2022prompt_bias, luccioni2024stable_bias, caliskan2017semantics_bias, hovy2016social_bias}. The origins of biased outputs can often be traced back to biases reflecting societal structures—such as cultural norms, historical inequalities, and dominant power dynamics—that establish the normative values guiding dataset curation, model development, and evaluation practices~\cite{Blodgett2021_bias, caliskan2017semantics_bias, birhane2021misogyny, hall2024visogender_bias, WinogenderRudinger2018_bias}.
Many biases in ML manifest in the real world, with real-world implications. For example, many non-white Uber drivers report being locked out of their apps and unable to work when the facial recognition systems used by Uber to conduct random ``identity checks'' fail to recognise their faces~\cite{watkins2023face}. Model failures can occur due to insufficient evaluation before release and/or these real-world conditions being difficult to produce in an evaluation setting~\cite{dev2021genderexclusiveharms_bias,birhaneAIAuditingBroken2024,dengUnderstandingPracticesChallenges2023}.  

While there are many ways to define bias in ML (e.g.,~\cite{weidinger2021ethical, weidinger2023sociotechnical, mehrabi2021survey, katzman2023taxonomising, shelby2023sociotechnical}), in this work we are particularly concerned with the representational and quality of service harms as defined in~\cite{weidinger2021ethical, weidinger2023sociotechnical} and~\cite{shankar2017allocational, de2019doescvworkallocational}, respectively. Representational harms are evident in how people, groups, and their heritage are presented and perceived. Quality of service harms typically manifest downstream in a real-world circumstance (for example, someone prompting a T2I model at home) and stem from a physical manifestation of representational harm. For example, people with visible disabilities experience representational harms when T2I models are unable to create accurate and dignified representations of their physical bodies~\cite{mack2024they}. 

Previous studies have focused on bias at the axes of gender and occupation~\cite{berg2022prompt_bias, hall2024visogender_bias,WinogenderRudinger2018_bias}, race and gender~\cite{currie2024genderethnicity, west2024fieldgenderethnicity}, and the medical domain~\cite{gisselbaek2024beyondmedical, wiegand2024demographicmedical}. In our work, we follow calls by~\citet{weidinger2023sociotechnical} to expand the scope of bias investigations to other intersectionalities, such as the representation of cultural objects from diverse regions. The \textsc{WWD} project was therefore initiated in response to this call to investigate and mitigate ML biases in diverse contexts such as culture. Extensive work has been done to understand the scope in which cultural biases exist, and even to what ``culture'' can refer.~\citet{adilazuarda2024culturesurvey} unpack definitions of culture used in the extensive literature related to bias investigation in large language models. In this work, we use the term ``culture'' to mean ``cultural heritage'' as defined in~\cite{adilazuarda2024culturesurvey, blake2000defining}. We intentionally chose food as a lens into culture because food is a salient cultural artefact. Food is shaped by a region's history, geography, and even religious symbolism. Intangible cultural heritage is considered a ``mainspring of cultural diversity'' and the social practices and rituals around food are so deeply intertwined with culture that they are recognised as ``intangible cultural heritage'' under UNESCO's Convention for the Safeguarding of the Intangible Cultural Heritage.\footnote{https://ich.unesco.org/en/convention. This is the primary international legal instrument in the field.} 

In this paper, we are primarily interested in T2I models, a type of GenAI model that generates novel images based on a user prompt. GenAI is a term used to describe ML models that produce audio~\cite{borsos2023audiolm, kreuk2022audiogen}, visual~\cite{hu2024instructimagen, reed2016generativedalle}, and/or written~\cite{tonja2024inkubalm, achiam2023gpt, team2023gemini} content. This type of AI is contrasted with methods of image classification~\cite{deng2009imagenet, lin2014microsoftcoco}, prediction models (e.g., weather forecasting~\cite{lam2023weatherforecasting}), and reinforcement learning~\cite{sutton2018reinforcement} because of its ability to \textit{generate new content} based on learned associations from training data. We focus particularly on T2I models and their implications in contributing to cultural erasure through stereotyping, misaligned values, and inaccurate outputs.

\subsection{Mitigating bias in machine learning through crowdsourcing representative datasets}
Social computing scholars have identified the role of datasets in encoding bias that ML systems then reproduce~\cite{gebru2021datasheets,buolamwiniGenderShades,sweeney2013discrimination,scheuermanDatasetsHavePolitics2021,scheuermanProductsPositionalityHow2024,kapaniaHuntSnarkAnnotator2023,sambasivan2021everyone}. In recent years, fairness in ML researchers have made significant efforts to construct datasets that are more representative using crowdsourcing methods~\cite{singh2024aya_dataset,romero2024cvqa}. Crowdsourcing knowledge tends to take on two forms: a top-down model where raters and annotators are contracted through a centralised body~\cite{miceli2020between}, and a bottom-up model that directly engages community members in the data collection. 

Top-down models of crowdsourcing to build more representative datasets include work such as~\cite{bhutani2024seegull, jha2024visage,rojas2022the}, which pursued diverse and representative samples from the Majority World.\footnote{The term ``Majority World'' is a deviation from the more commonly used literary term ``Global South.'' We have chosen to use ``Majority World'' to highlight that the majority of the populations with which we engage come from these regions.} The ORBIT computer vision dataset~\cite{massiceti2021orbit} demonstrates how top-down models can engage in the ethical sourcing of data contributors, as it involves annotators from Enlabeler (Pty) Ltd,\footnote{https://www.enlabeler.com/} a company that empowers its employees by offering technical skill development alongside their data annotation work. While top-down data collection can be efficient and more easily implemented at scale, it excludes the many perspectives of people who do not work as data annotators~\cite{geiger2020garbage}. Moreover, top-down data collection efforts typically dictate to data contributors what an acceptable submission looks like, without room for bottom-up feedback~\cite{posada2022dispotif}. Contributors in top-down models can be flagged for fraudulent behaviour if the company overseeing data collection suspects that multiple people share an account, a common practice in households where there may only be one computer available~\cite{posada2022coloniality,jones2021refugees}. In contrast, bottom-up and community-based data collection broadens the range of people who can contribute and can empower more people not only to share cultural artefacts, knowledge, and expertise but also to help shape the parameters of their own participation~\cite{denton2021whose,delgado2023participatory,birhanePowerPeopleOpportunities2022}. We characterise bottom-up, community-based crowdsourcing as efforts that allow for any community member to engage (e.g., they do not need to be formally employed as a  data crowd worker), make space for community contributors to actively shape parts of the research process (e.g., they may take part in decisions about what kinds of data should be collected), and tend to---in large part---be volunteer-based.  

Community-based crowdsourcing has great potential to reach people in the community and to gather local knowledge. For example, the AYA dataset from Cohere's research lab~\cite{singh2024aya_dataset} presents a framework for large-scale community data collection. Their collaborators spanned 119 countries and were able to create a multilingual text dataset comprising 513 million instances across 114 languages. All their systems have been open-sourced. In 2020,~\citet{orife2020masakhaneoriginal} introduced Masakhane, which is an ongoing open-source, continent-wide, online research effort for machine translation for African languages. Masakhane has also consistently demonstrated ethical participatory research design for machine translation and natural language processing in Africa~\cite{nekoto2020masakhane, adelani2022masakhaner,ogundepo2023afriqamasakhane}. While efforts such as AYA and Masakhane make significant strides in producing datasets with diverse community-based input, both research efforts foreground the dataset as the primary artefact of interest. In our contribution, we aim to build upon the precedent set by bottom-up crowdsourcing efforts by calling attention to the processes and complexities---such as designing accessible infrastructure to support data collection and working with communities to build trust and rapport---throughout the data collection process by which these types of datasets are created. 

\subsection{Infrastructures for dataset construction}
Infrastructure is the foundation underpinning any large-scale system; it is the underlying substrate that is embedded in action, tools, and built environments and takes on significance in relation to organised practices~\cite{star1999ethnography,star201620,rice2006handbook}. Infrastructure can take the form of datasets~\cite{bowker2000sorting} that classify information and are subsequently used to create and reinforce categorisations. For example, FairFace, a large-scale labelled image dataset, is an infrastructure that has been used to determine what counts as a face in an image---but, crucially, relies on perceived gender and race labels~\cite{karkkainen2021fairface}. Datasets, like all infrastructures, are socio-technical; information does not naturally fall into classifications~\cite{bowker1994information,bowker2000sorting,star1999ethnography}. Instead, data must be produced by humans who make subjective decisions about what deserves to be counted and how it should be classified~\cite{d2024counting,denton2021whose}. This necessarily means that datasets are value-laden: they are imbued with the values of their designers and the larger social contexts in which they were created~\cite{rice2006handbook,star1999ethnography,scheuermanDatasetsHavePolitics2021,d2023data}. 

However, the value-laden nature of datasets often remains unseen until breakdowns---disruptions between expected and actual outcomes---occur~\cite{luccioni2023bugs, birhane2024into}. Critical computing scholars have demonstrated the value-laden nature of datasets for ML by leveraging \textit{infrastructural inversion} as an analytical tool to trace breakdowns back to incongruences between designers and users~\cite{scheuermanDatasetsHavePolitics2021,scheuermanHumanDataDataset2023,buolamwiniGenderShades}. For example, the Gender Shades study leverages infrastructural inversion to surface the social values and power dynamics underlying image datasets by tracing the poor performance of facial recognition systems on Black women's faces back to image datasets that lacked adequate and accurate representation of marginalised groups~\cite{buolamwiniGenderShades}. The lack of adequate and accurate representation of marginalised groups in these datasets reflects a lack of value for these identities---they were overlooked in dataset creation. 

Accounting for diverse values in dataset creation requires labour~\cite{scheuermanDatasetsHavePolitics2021,sambasivan2021everyone}. As~\citet{scheuermanDatasetsHavePolitics2021} points out, the data work that goes into developing a dataset is often ``silenced'', which results in a lack of incentive structures to center, support, and document the labour that goes into dataset creation. In our paper, we leverage infrastructure studies to call attention to the invisible labour that underpins the creation of large-scale systems and the subsequent implications for the \textit{kinds} of values that then get embedded into the resulting system. 

\subsection{The pitfalls and potentials of participation}
Recognising the limitations of top-down designed datasets, ML researchers have begun a \textit{participatory turn in AI} in an effort to construct datasets that represent a wider range of values and identities~\cite{delgado2023participatory,birhanePowerPeopleOpportunities2022,ParticipationScaleTensions,corbett2023power,suresh2024participation}. Participation is thought to enable better alignment of system performance with end-user expectations, as impacted communities would be involved in shaping the values of the ML systems~\cite{delgado2023participatory}. In other words, communities would be empowered as co-designers of the ML systems that impact them. However, the majority of participatory AI efforts fall short of their promises to support meaningful participation among stakeholders in the design process~\cite{ParticipationScaleTensions}. Researchers argue that inadequate attention is paid to shifting power away from systems designers, who are often the primary beneficiaries of improved model performance, and towards participants who could benefit by ensuring models accurately represent them and encode their values~\cite{birhanePowerPeopleOpportunities2022}. While participatory AI is a relatively nascent subfield within the ML community, participatory design is a research tradition that has existed for decades in social computing~\cite{asaro2000transforming,bodker2022can}. In our contribution, we draw upon lessons from participatory design in CSCW to demonstrate how these methods can be applied to achieve meaningful power shifts between researchers and communities. 

CSCW has a long and rich history of leveraging community-based participatory research (CBPR) methods to center socially marginalised groups' epistemologies in the design of technology~\cite{le2015strangers,sum2023translation,bratteteig2016unpacking,harrington2019deconstructing,bannon2018reimagining,pierre2021getting}. CBPR is used as a method to mitigate the power imbalances between researchers and communities by empowering community members to shape the trajectory and design of research projects~\cite{reasonSAGEHandbookAction2008}. For example,~\citet{asaro2000transforming} documents how, in light of impending technological changes in the workplace, workers themselves were invited to participate in the design of those technologies and thus had the power to shape the impacts of these technologies on their well-being.~\citet{harrington2019deconstructing} considers the role of research setting on community access, and therefore the accessibility of participation for targeted groups. In CBPR, practitioners strive to overcome their status as outsiders by building trust through affective and moral connections with community members~\cite{le2015strangers,mcmillan1986sense}. Taken together, CBPR literature overwhelmingly stresses the need for researchers to become imbricated with the communities which they are studying---through building these relationships, they become invested in the outcomes of the research from an insider, community member perspective. 

While participatory methods have the potential to give communities a greater say over the design of technology, many of these efforts fall short in meaningfully shifting power from the researchers to the researched~\cite{birhane2021misogyny,ojewaleAIAccountabilityInfrastructure2024,ParticipationScaleTensions}.  Participation takes work on behalf of communities.~\citet{pierre2021getting} documents how communities assume the ``epistemic burden'' of producing data about their lived experiences in participatory research efforts~\cite{pierre2021getting}. Finally, some scholars have critiqued the tendency of efforts to include socially marginalised groups as inherently ``othering''~\cite{epstein2008rise}. In our contribution, we draw upon the rich tradition of CBPR in CSCW literature to apply methods to build datasets more equitably. In particular, we attend to how research infrastructures can support the work of constructing the field site, building relationships with community members, and making room for meaningful community participation in the research design process~\cite{le2015strangers}. 

\section{Historical Origins of `Long-Tailed' Data in Machine Learning}\label{background}

T2I systems perform poorly when creating images of socially marginalised populations, such as people with disabilities or cultural objects from the Majority World~\cite{qadri2024,dasProvenanceAberrationsImage2024,mack2024they}. ML researchers attribute poor model performance on these topics to the \textit{long-tailed} nature of data about socially marginalised groups: models struggle to learn meaningful patterns and produce accurate, fair results because long-tailed data comprise a negligible portion of the training instances~\cite{zhang2023deeplongtailed, massiceti2021orbit}. 

Data from socially marginalised groups are less likely to be preserved in a digitised format and available on the Internet~\cite{Noble+2018}. Additionally, these data are systematically screened out of training datasets by state-of-the-art filtering models, such as CLIP \cite{radford2021learning}, which are biased towards including data from and about Western cultures~\cite{hong2024s}. Researchers use filtering methods such as CLIP to curate datasets in the first place to improve model quality by removing noisy, irrelevant, or potentially harmful content, aiming to optimise the relevance and accuracy of the data for the model’s intended use~\cite{fang2023data}. However, these filtering methods often encode and amplify existing biases, leading to datasets that inadvertently exclude non-Western cultures and marginalised groups. Ironically, while these filters are designed to protect against inappropriate or harmful content, they sometimes fail to remove problematic material such as foul language, racism, and harmful stereotypes~\cite{birhane2021misogyny}. This inconsistent filtering reinforces disparities in representation and increases concerns about the long-tailed nature of the datasets.

The root causes of the long-tailed nature of data about socially marginalised groups are intimately intertwined with social, political, and economic histories. In other words, the systemic inclusion from, and misrepresentative nature of, datasets is not solely a product of technical failure. Rather, it is a manifestation of societal values contributing to choices made about whose cultures are preserved, and whose are erased.~\cite{bowker2000sorting,cheney2017we,brubaker2011select,benthall2019racial,benjamin2019race}. For example, the exclusion and flattening of unique cultures from the African continent in digital media can be traced back to colonial regimes that sought to erase and thereby dehumanise the people living on the continent~\cite{faloyin_africa_2022}. During colonial times, power structures meant that the historical record was kept by Europeans colonising the continent~\cite{bowker2000sorting}. In an attempt to justify their actions, there was a systemic flattening and dehumanisation of the lived experiences of those they sought to subjugate on the African continent. Colonial classification regimes flatten important cultural and geographic differences~\cite{das2022,prabhakaran2022cultural,das2021}. Borders were drawn without regard for existing complex social and ethnic groups, and the subsequent oppression and power dynamics involved in recording history lent themselves to a systematic erasure of distinct experiences~\cite{bowker2000sorting}. More recent depictions of Africa in the international media and popular culture have made little attempt to capture its deeply complex and rich landscape~\cite{faloyin_africa_2022}.   

Acknowledging the core reasons for this type of flattening, \textsc{WWD} explicitly seeks to counteract it through on-the-ground community consultation and collaboration. The granularity required to capture a region's knowledge is most likely best understood through extensive consultation with communities with lived experiences in those regions. For example, borders may not provide the best guide for demarcating cultural boundaries, and a representation of “Kenya” or “Nigeria” may not be granular enough when we consider the distinct cultural groups within these borders. Some borders may also be misleading: for example, the Semliki River has changed course numerous times in the past few decades, meaning that some cultural groups have found themselves flipping between national identities of Uganda and the Democratic Republic of Congo, depending on a naturally evolving geographical marker~\cite{faloyin_africa_2022}.


We present the African continent as an example to explain the deep-rooted causes of and potential for the type of cultural erasure that exists across the globe. \textsc{WWD} and the infrastructure developed therein seeks to counteract this erasure by encouraging data submissions from around the world in a granular manner, going beyond national representation but explicitly requesting fine-grained regional details. These details and nuances can be easily overlooked without community expertise. The individuals that we engaged are reached primarily through social networks. In the coming sections, we expand on this process and share insight into how submissions from different regions within a country were encouraged.

\section{Methods and Data}
\label{sec:methods}

To conduct this study, we first collected a large dataset of search directives, which are defined as prompts to conduct an online search (Section~\ref{sec:methods-directives}). 
We then used the 1.4M search queries we found in that dataset to surface Google's warning banners (Section~\ref{sec:methods-serps}), evaluate their presence in the context of established query (Section~\ref{sec:methods-queries}) and domain-level metrics (Section~\ref{sec:methods-domains}) with logistic regression (Section~\ref{sec:methods-logit}), test their consistency over time (Section~\ref{sec:methods-stability}), and develop deep learning models to identify unlabeled data voids (Section~\ref{sec:methods-models}).

\subsection{Search Directives}
\label{sec:methods-directives}

We collected search directives from social media posts (Section~\ref{sec:methods-directives-posts}) to collect a diverse set of search queries (Section~\ref{sec:methods-directives-queries}) for our study.
By specifying a flexible linguistic strategy (prompts to conduct an online search) rather than specific content (search queries), search directives provide a useful tool for surfacing unspecified and unknown content.
The search queries used in search directives have been shown to cover a diverse array of topics, ranging from music, sports, and advertising, to medical misinformation about Ivermectin, an emerging conspiracy about the COVID-19 vaccine causing people to ``die suddenly,'' and a cryptocurrency scam~\citep{robertson2023identifying}. 
Although the potential harms of people being led into data voids like these have been well documented~\citep{golebiewski2019data}, few studies have examined how people can be led into data voids (\cite{tripodi2019devin}; \cite{tripodi2023your}), how to computationally identify data voids~\citep{flores-saviaga2022datavoidant}, or how to measure the bridge between social media and search engines more broadly (\cite{bode2018studying};\cite{lukito2020coordinating}; \cite{yarchi2021political}; \cite{zuckerman2021why}).
Rather than on relying on smaller sets of queries generated through surveys or  interviews, or medium-sized sets of queries generated via autocomplete (\cite{robertson2019auditing}; \cite{haak2023qbias}), our use of search directives allowed us to collect 1.4M unique queries without defining the topic space or a starting set of queries to expand upon.

\subsubsection{Social Media Posts}
\label{sec:methods-directives-posts}

We collected a total of 5.25M posts that contained a URL fragment (e.g. \nolinkurl{google.com/search}) leading to one of 25 popular search engines. 
This collection strategy allows for flexibility in subdomains, variability in URL parameters, and enabled us to easily and accurately extract search directive queries.
Following past work, we filtered out URLs that did not lead to a page of search results, including those that did not contain a known query parameter (e.g., ``\&q=\{query\}'' for Google Search) and those that contained a blank query, leaving 4M search directive posts, 4.17M URLs (posts can contain multiple URLs), and 1.44M unique queries that were created by 1.82M unique accounts over a 16.5-year window (2006 to 2023). 
Advancing on prior work that examined the five most popular modern search engines in the US (Google, Bing, DuckDuckGo, Yahoo, and Brave), we used a list of 25 search engines to collect our dataset (Google, Bing, DuckDuckGo, Yahoo, Brave, AOL, Ask, Baidu, Dogpile, Ecosia, Exalead, Excite, Hotbot, Lycos, Metacrawler, Mojeek, Petalsearch, Qwant, Sogou, Startpage, Swisscows, Webcrawler, Yandex, You, and Youdao), including search engines that are prominent outside of the US (e.g. Yandex), were prominent in the past (e.g. AOL and Ask), or newer search engines that feature large language models (e.g. You.com).

\subsubsection{Search Directive Queries}
\label{sec:methods-directives-queries}

Of the 4.17M posts that contained a URL fragment and a search query---which excludes links to search engine homepages that don't qualify as a search directive---we obtained a diverse sample of 1.44M unique queries that varied widely in terms of both their content and structure. While not representative of what people are searching for today, these queries cover a wide range of topics (including music, sports, and politics), were produced across a 16 year span and include event-driven bursts (e.g., around the ICC Men’s T20 World Cup 2016, a biannual cricket tournament). 
These queries also widely varied in terms of their length, with the average search directive query containing an average of 4.5 words, which is slightly longer than estimates of query length in the US, which find that 82\% of queries are 3 words or less~\citep{keyworddiscovery2020keyword}. 
The longest query in our dataset was 896 tokens long, and 234 (0.01\%) queries were only one character, often an emoji.
Notably, Google Search limits queries to 32 words, and that length is counted after processing by an unknown tokenizer. When a query is too long, Google adds a notice at the top of the search results which states: ``... (and any subsequent words) was ignored because we limit queries to 32 words.''
We provide the distribution of query lengths with and without truncation in Appendix~\ref{sec:appendix-descriptives-queries}, Figure~\ref{fig:query-length-distributions}.

\subsection{Search Engine Results Pages (SERPs)}
\label{sec:methods-serps}

We used open-source tools to collect our search results (Section~\ref{sec:methods-serps-collecting}), and an iterative approach to discovering and classifying Google's warning banners (Section~\ref{sec:methods-serps-banners}).
To evaluate the rate at which search directive queries produce warning banners and data voids, we used our set of 1.4M unique queries as the inputs for an approach known as the algorithm audit~\citep{sandvig2014auditing}, which typically involves collecting and examining the outputs of a black-box system based on some fixed set of inputs (\cite{bandy2021problematic}, \cite{metaxa2021auditing}, \cite{mustafaraj2020case}, \cite{vanhoof2022searching}).
In this case, the inputs are the search directive queries, the system is Google Search, and the outputs are the Search Engine Results Pages (SERPs) returned by Google. 

\subsubsection{Collecting and Parsing SERPs}
\label{sec:methods-serps-collecting}

For collecting the search results available for each query, we used WebSearcher~\citep{robertson2020websearcher}---an open source tool for collecting and parsing SERPs that has been used in prior algorithm audits of Google Search~\citep{mejova2022googling}---to conduct a search using each query in our set, store the corresponding HTML, and extract details about its corresponding search results (e.g. rank, URL, result type). 
We also extracted several elements other elements from the SERP, including Google's estimate for the total number of results it found for each query (across its entire index), which could also be indicative of a data void, as the search results for a query with few matches may be easier to manipulate due to the limited competition.
As with most algorithm audits of web search, this SERP dataset represents only what someone searching these queries might have seen at the time of our collection. 
We also searched from a fixed location and do not study localization effects~\citep{kliman-silver2015location}.
Details on the number of results we collected are available in Appendix~\ref{tab:crawl-counts}

\subsubsection{Identifying Warning Banners}
\label{sec:methods-serps-banners}

We initially identified banners by checking for the exact phrasing of each warning banner type, and then built phrase-agnostic HTML parsers to extract them across the entire dataset. 
For low-relevance banners, the phrasing was ``It looks like there aren't many/any great matches for your search''~\citep{tucker2020getting}. 
Low-quality banners contained similarly phrased language (``It looks like there aren't many great results for this search''), swapping only ``matches'' with ``results.'' 
In contrast with the low-relevance banners, we never observed the variation where ``many'' was replaced with ``any'' in the low-quality banners, which aligns with the ambiguity of the banner message (``some of [these results] may not have reliable information'', Figure~\ref{fig:banner_ex}), and how they were described in Google's blog post announcing their rollout (``This doesn't mean that no helpful information is available, or that a particular result is low-quality''~\cite{nayak2022new}).
This reluctance to specify which search results are low-quality may also help explain why we never saw a low-quality banner for searches with a ``site:'' operator that restricted the search results to a specific web domain: doing so would remove the ambiguity of the judgment.
In contrast to these warnings about content, the rapidly-changing banner stated: ``It looks like the results below are changing quickly''~\citep{sullivan2021new}.
Additional details and a screenshot of the rapidly-changing banner, as well as details and a screenshot of a low-relevance banner variant that only appeared in our last crawl, are available in Appendix~\ref{sec:appendix-banners}.

\subsection{Search Query Text Features} 
\label{sec:methods-queries}

To evaluate query content, we used a dictionary-based approach to identify queries containing partisan and polarizing search terms or conspiracy-related search terms (Section~\ref{sec:methods-queries-polcon}), and other text features, such as advanced query operators (Section~\ref{sec:methods-queries-operators}).

\subsubsection{Political and Conspiracy-related Lexicons} 
\label{sec:methods-queries-polcon}

To identify queries around controversial topics that could potentially lead to data voids, we used a dictionary-based approach to tag words and phrases associated with conspiracies and politics in prior work.
Specifically, we used:
\begin{inparaenum}[(1)] %
    \item \textcite{ballatore2015google}'s set of 96 conspiracy-related search queries,
    \item \textcite{mahl2021nasa}'s set of 44 conspiracy-related hashtags, and
    \item \textcite{urman2022where}'s set of 6 conspiracy-related search queries.
\end{inparaenum}
We also considered \textcite{haak2023qbias}'s set of QAnon-related search queries and autocomplete expansions, but the terms were too broad for our purposes.
For \textcite{mahl2021nasa}, we added non-hashtag versions of each item (e.g. ``\#vaccineskill'' becomes ``vaccines kill'') and excluded ``\#dew'', which refers to conspiracies around directed energy weapons but produces a high false positive rate due to the popularity of Mountain Dew, a soda brand.
Of the 14 conspiracy categories covered by this dictionary---including conspiracies about 9/11, chemtrails, and reptilians---we found at least one search directive query that mentioned each.
We also used two existing lexicons of polarized terms---one designed to capture ``polarized language''~\citep{simchon2022troll} and one designed to capture ``partisan cues''~\citep{hu2019auditing}---to classify search directive queries as politically related.
Combined, we used these lexicons to classify the full set of queries as related to politics (11.1\%), conspiracies (0.11\%), both (0.02\%), or neither (88.8\%).

\subsubsection{Advanced Query Operators} 
\label{sec:methods-queries-operators}

Advanced query operators allow searchers to specify additional constraints on their search results.
For example, adding ``site:dailycaller.com'' to a query (e.g. ``trump site:dailycaller.com'') will search for results containing that term (``trump'') only within that site (``dailycaller.com'').
When considering search directives as an attempt to exert indirect online influence, the use of these operators has strategic value in guiding people to specific content via a trusted search engine: searchers less familiar with these operators may not understand that their results have been filtered, and while some search engines (e.g., DuckDuckGo) display a message to informs users that such a filter is active, Google does not~\citep{robertson2023identifying}.
In total, 1.5\% of our 1.4M unique queries contained one of 11 advanced operators, and among those, the most common operator was ``site:'' (92.0\%), followed by ``inurl:'' (2.8\%), ``filetype:'' (1.9\%), ``intitle:'' (1.0\%), ``ext:'' (0.5\%), ``before:'' (0.5\%), ``source:'' (0.4\%), ``related:'' (0.3\%), ``allintitle:'' (0.3\%), ``after:'' (0.2\%), and ``allinurl:'' (0.1\%).
These queries varied widely in their content and complexity, some containing multiple operators and others containing only one.
For example, one query used the OR operator, parentheses, quotes, and 15 site operators: ``(mask | vaccine | "death count" | "case count") fraud and evidence election ( site:amac.us | site:townhall.com | site:heritage.org | site:thegatewaypundit.com | site:oann.com | site:scienceunderattack.com | site:conservativetribune.com |  site:thefederalist.com |site:greatamericandaily.com | \\ site:westernjournal.com | site:zerohedge.com | site:prageru.com | site:realclearpolitics.com | site:mercola.com | site:naturalnews.com ).''

\subsubsection{Query Language}
\label{sec:methods-queries-language}

As many queries are names, fragments, emojis, or are otherwise grammatically incorrect, determining the language of queries can be challenging, and some level of noise is inevitable. To get a general sense of query languages, we used the FastText library~\citep{joulin2016bag} to predict the most likely language for each query. Across all 1.4M queries, more than 1.2M were predicted to be in English. For the subset of 930K queries where the model returned a confidence of at least 0.5, 875K were predicted to be English. The second most common category in the high and low-confidence query sets was French, with 23K and 9K queries, respectively. Many of the queries that were classified as French appear to have been classified that way because they use French words or names in an otherwise English-speaking context. For example, of the 55 queries that contained the name ``De Blasio''---the former mayor of New York---FastText predicted that 24 were French, including  ``bill de blasio'' and ``bill de blasio drops groundhog video''. 
German was the third most popular language, and similar to the French classifications, many of the queries classified as German appeared be English queries associated with American politics like ``adolf hitler defund police'' and ``f\"{u}hrermccarthy''.

\subsection{Web Domain Features}
\label{sec:methods-domains}

To evaluate the average domain quality of the SERPs we collected, we extracted the second and top level domain names for each URL (e.g. https://cnn.com/politics $\rightarrow$ cnn.com) and merged them with domain metrics for quality and partisanship (Section~\ref{sec:methods-domains-quality}), as well as domain-level measures of web traffic and backlink counts (Section~\ref{sec:methods-domains-seo}).

\subsubsection{Measuring Domain Quality and Partisanship}
\label{sec:methods-domains-quality}

For domain quality, we used a set of scores that were recently developed to evaluate the quality of domains based on a compendium of similar existing metrics~\citep{lin2023high}.
These scores range from 0 to 1, with higher scores indicating higher quality, and cover 11,519 unique domains (we drop one duplicate that appears with and without a ``www.'' prefix).
When calculating domain quality at the SERP-level, we take the average score of the domains that appeared on the SERP.
Prior to calculating that average, we exclude three platform domains from the original set because their quality scores were low and hard to interpret.
Those domains and their scores were: \nolinkurl{youtube.com} (0.375), \nolinkurl{facebook.com} (0.407), and \nolinkurl{google.com} (0.668). 
For partisanship, we use the partisan news scores created in~\citep{robertson2018auditing} based on the relative proportion of Democrats and Republicans that shared a domain on Twitter.
These scores range from -1 (only shared by Democrats) to 1 (only shared by Republicans), with a score of 0 meaning only that a domain was shared by an equal number of Democrats and Republicans (i.e., not ``neutral''), and we used the rank-weighted average of these scores for each SERP.
Our domain-level measures of quality are coarse-grained and do not account for instances, for example, where unreliable domains publish accurate webpages, or vice versa.
The partisan scores we used are subject to similar limitations~\citep{green2025curation}.

\subsubsection{Search Engine Optimization (SEO) Metrics}
\label{sec:methods-domains-seo}

Given the relevance of Search Engine Optimization (SEO), a billion dollar industry aimed at improving websites' search rankings, to questions about search results, we also obtained SEO features (e.g. backlink counts) and traffic estimates from Ahrefs ({\nolinkurl{https://ahrefs.com}), a large SEO company.
Recent work using data from Ahrefs has shown that some of its features are predictive of misinformation, and suggests that its traffic estimates are reliable (\cite{carragher2024detection}, \cite{carraghermisinformation}, \cite{williams2023search}). 
We provide additional details on the SEO features we used, including their validity, use in past work, and descriptive characteristics, in Appendix~\ref{sec:appendix-descriptives-seo}.

\subsection{Logistic Regressions}
\label{sec:methods-logit}

In our logistic regression models, the presence of each banner type (low-relevance or low-quality, separately) was our dependent variable, and our independent variables included factors both related to the query and the SERP it produced.
We used the \texttt{statsmodels} library in Python to fit our logit models with L1 regularization. More specifically, we set the regularization parameter (alpha) to 0.1, and used the L-BFGS algorithm as the solver with a maximum of 10,000 iterations and convergence and zero tolerances set to $1e-8$. 
We trained separate models for each dependent variable and crawl because the rules governing their appearance could change over time.

Our models for predicting low-relevance banners demonstrated moderate fit, with pseudo-$R^2$ values of 0.39, 0.15, 0.38 for crawls 1, 2, and 3, respectively (Appendix~\ref{sec:appendix-logit-low-relevance}, Tables~\ref{tab:logit-low-relevance-crawl1},~\ref{tab:logit-low-relevance-crawl2},~\ref{tab:logit-low-relevance-crawl3}). In contrast, and likely due to the smaller sample size, our models for predicting low-quality banners demonstrated lower fit, with pseudo-$R^2$ values of 0.13 for both crawls 1 and 2 (Appendix~\ref{sec:appendix-logit-low-quality}, Tables~\ref{tab:logit-low-quality-crawl1} \& Table~\ref{tab:logit-low-quality-crawl2}).

\subsection{Banner Stability and Consistency}
\label{sec:methods-stability}

\subsubsection{Rapid Data Collection}
\label{sec:methods-stability-temporal}

To better understand the relationship between SERP results and quality banners, we collected SERPs for the 301 queries that produced a low-quality banner in crawl-1 approximately every four hours from June 7, 2024 to June 24, 2024. 
We dropped data from two collection intervals due to technical issues, and dropped five queries that did not return search results at any time step (truncated to four words: children's clarity about search engine..., international intelligence "search manipulation..., "the virtuebios and mortality resolution", "miembro del instituto de investigación..., why face masks don’t work...). 
This left us with 73 SERPs for each of the remaining 296 queries that returned a quality banner in our initial collection. 
The two gaps in this collection lasted for 45 hours, between June 11 and June 13, and for 15 hours, between June 17 and 18.
We find similar results in a pilot version of this dataset that we collected in March 2024 over 34 time steps (with about 1.5 hours between each) without any gaps (Appendix~\ref{sec:appendix-consistency-pilot}).

\subsubsection{URL Similarity}
\label{sec:methods-stability-rbo}

To measure the stability of returned (ranked) URLs given a query, we use Ranked-Biased Overlap (RBO)~\citep{webber2010similarity},
a metric designed to compare ranked indeterminate lists. While RBO can be weighted by a user-chosen probability $p$, we elect to take the average overlap, which corresponds to RBO with $p=1$. RBO calculates the agreement between ranked lists $S$ and $U$ at every level of depth from $1:D$ and takes the average. Formally, RBO with $p=1$ can be written as:

\begin{equation}
    RBO(S,T, p=1) = \frac{1}{D} \sum^D_{d=1} \frac{\mid S_{1:d} \cap U_{1:d} \mid}{d}
\label{eq:AO}
\end{equation}

As our interest is in the relative stability of the SERPs across all queries, we constructed a windowed RBO metric ($RBO_k$) to quantify SERP similarity across consecutive pulls. 
Let $X_i$ be the pairwise RBO similarity matrix for URLs returned in the SERPs of query $i$ over $T$ timesteps. We set a window size $K$, and define query-level windowed RBO similarity $\bar{x}_i$ as:

\begin{equation}
    \bar{x}_{i,K} = \frac{1}{2KT} \sum^T_{t=0} \sum^K_{k=1} (\mathbbm{1}_{[t-k \geq 0]} x_{t-k} + \mathbbm{1}_{[t+k \leq T]}x_{t+k})
\label{eq:rbosim}
\end{equation}

\noindent
where $\mathbbm{1}$ is the indicator function. If we set $k=1$, this would correspond to the average RBO of URLs returned at time $t$ with URLs returned $t-1$ and $t+1$ over all time steps. If this number were close to 1, that would indicate a high similarity---both in the set of URLs returned and their ranking---between consecutive SERPs for a given query. Conversely, if $RBO_k$ were 0, this would suggest high volatility, and would mean that a query never returns any of the same URLs in consecutive time-steps. Finally, we define $RBO_k$ as:

\begin{equation}
    RBO_k = \frac{1}{N}\sum_{i=1}^N \bar{x}_{i,K}
\label{eq:rbok}
\end{equation}
\noindent
Intuitively, while $\bar{x}_{i,K}$ measures the stability of a single query's SERPs results over $K$ consecutive pulls, $RBO_k$ simply measures the average SERP stability over all queries' SERPs over $K$ consecutive pulls.

\subsubsection{URL Dependencies}
\label{sec:methods-stability-dependency}

To find a consistent and simple logic that could explain low-quality banner variance in all queries, we considered three formalized questions that probe simple URL dependency conditions. 
Specifically, we attempt to determine for all queries whether or not there is 1) a single URL in all bannered SERPs but no unbannered SERPs, 2) a pair of URLs that appears in all bannered SERPs but no unbannered SERPs and 3) a pair of URLs conditioned on a rank cut-off that appear in all bannered SERPs but no unbannered SERPs, e.g., ``If $u_i$ always appears in the top 5 search results and $u_j$ always appears in at a position below 5, is there always a banner?'' 
We provide a more formal treatment of these questions in Appendix \ref{sec:appendix-consistency-formal}.

\subsection{Model Development}
\label{sec:methods-models}

We consider models conditioned on several different features to attempt to learn a mapping between our queries and Google's low-quality banners. 
The purpose of these models is 1) to determine whether we can learn Google's mapping from observed queries and SERPs to quality banners and 2) to assess the consistency and stability of Google's approach to placing low-quality banners. 

\subsubsection{Preprocessing}
\label{sec:methods-models-preprocessing}
Prior to modeling, we performed two preprocessing operations. 
First, we calculated embeddings for all queries using a multilingual Sentence-BERT model \citep{reimers-2019-sentence-bert}. 
Second, most results on a SERP include a title---the blue text on Google's SERPs that one clicks to reach a webpage. 
For each domain that appeared at least twice in the dataset, we create a single string that contains the domain name with a colon followed by titles sampled from the domain with replacement. 
The intention with this step is to create a feature with some, albeit shallow, notion of the topics covered by the domain. 

\subsubsection{Train and Test Datasets}
\label{sec:methods-models-dataset}
After quantitative and qualitative evaluations, we chose a 1:3 positive-to-negative sample to train our classifiers.
In predictions on unlabeled data, we observed that the DistilBERT model trained on a 1:1 positive-negative sample seemed to be over-relying on the presence of quotation marks; DistilBERT's most 100 confident banner candidate predictions contained quotation marks and some contained names of movies or books like ``the craft'' and ``scout mindset.'' We therefore elected to use a 1:3 positive-to-negative sample in order to provide a more diverse set of negative samples. The sample consists of the 301 queries that produced a low-quality banner in crawl-1, and an additional 903 queries that were randomly sampled from the queries that did not receive such a banner. 
We applied a stratified 80/20/20 split to the final set of 1,204 labeled and unlabeled queries.

\subsubsection{Query-Only DistilBERT Model}
\label{sec:methods-models-distilbert}
First, we consider a model that only includes query text. 
This assumes that banner presence is independent of a query's returned SERP, which we know from temporally-dense observations is not the case. 
We include a text-only model to determine how much signal is present in text in a given time period and as baseline by which we can compare more complex models. 
More formally, for a given query $T$, let $p(B | T)$ be the probability of a quality banner appearing, conditioned on query text. 
Our first model assumes that the probability of a banner depends only on semantic cues present in the query text $T$. 
Specifically, we fine-tune a DistilBERT model~\citep{sanh2019distilbert} to predict $p(B | T)$ for each query. 
The model is trained for two epochs using only the raw query text using an Adam optimizer with a learning rate of $2\mathrm{e}{-5}$ and a linear warm-up scheduler.

\subsubsection{Query-SERP GNN Models}
\label{sec:methods-models-gnn}

Next, we sought a model that could incorporate the assumption that two different queries with highly similar SERPs should likely have the same banner status. 
We therefore elected to use Graph Neural Networks (GNNs), as this allows us to propagate domain-level context into query representations (e.g., as done with different features in~\citep{williams2024dredge}).   %
To do so, we constructed two simple homogeneous (GNN\textsubscript{\emph{Hom}}) and heterogeneous (GNN\textsubscript{\emph{Het}}) graph neural network models which incorporate the assumption that the presence of a banner $p(B)$ depends on both the query text ($T$) and the content associated with returned domains $S$. 
These models aim to predict the conditional probability $p(B | S, T)$, integrating information from both sources. 
We represent the problem as a bipartite query-to-domain graph, with one node set corresponding to queries and the other to domains. 
Our approach is also similar to the vaccine-related query-weblog graphs used in~\citep{chang2024measuring}, but we elect to leverage DistilBERT for query embedding as our query topics span a broader range of domains.

Given a set of queries $Q = \{q_1, q_2, \dots, q_n\}$ and a set of returned SERP domains $D = \{d_1, d_2, \dots, d_n\}$ we construct a homogeneous, bipartite graph $\mathcal{G}_{Hom} = (V,E)$ where domains and queries are treated as the same node types. Both node types have text-based node features, and labels $Y$ are a binary variable indicating the presence of a banner on $Q$. We additionally construct a heterogeneous graph, $\mathcal{G}_{Het} = (V, E)$ where where $V$ and $E$ are associated with a node type mapping function $\Psi : V \rightarrow A$ and an edge type mapping function $\Phi : E \rightarrow \phi$. In our setting, the set of node types are $A = \{Q, D\}$ and the set of edge types are $\Phi = \{domain-to-query, query-to-domain\}$.%

To incorporate the assumption that ranking changes in $top-k$ SERP results (with URLs held constant) should not alter banner presence, we do not weight edges in our networks. To allow information to propagate between queries, we exclude ``pendulum'' domains---those that only appeared once in our SERP data. For each of those domains we sampled at most 10 ``titles''---the blue text that appears on Google search results (generally the title of the article or webpage)---embedded the titles with DistilBERT, and took the simple mean. 
Although this is a relatively simple and coarse-grained approach that excludes many relevant domain-level signals, we demonstrate its effectiveness and leave the incorporation of more nuanced domain-level features to future work.

Both models consist of a GraphSage convolution with a dropout of 0.5 and ReLU activation, followed by a second GraphSage convolution and a final log-softmax activation function~\citep{hamilton2017inductive}. This results in a model with 526k parameters. GNN\textsubscript{\emph{Het}} uses a heterogeneous GraphSage convolution~\citep{Fey2019}. In this setting, where there are only two node types with bi-directional ties, this doubles the number of model parameters to 1.05M. We use an Adam optimizer with $\eta = 1\mathrm{e}{-3}$, and a weight decay of $5\mathrm{e}{-4}$, Cross Entropy Loss, and a Cosine Annealing Learning Rate Scheduler with $\eta_{min} = 2\mathrm{e}{-5}$~\citep{loshchilov2016sgdr}. 

\subsubsection{Model Validation}
\label{sec:methods-models-validation}
We evaluated our models using standard Accuracy, F1, Precision, and Recall metrics (Table~\ref{tbl:GNNResults}).
However, given the small size of the dataset and the corresponding likelihood of over-fitting, we also include three different supplemental forms of validation. 
For each of these supplemental evaluations, we consider unlabeled queries that each model most confidently predicted as unreliable. 
To evaluate the success of each model in identifying candidate queries for receiving low-quality banners, we used them to predict warning banners in the subsequent crawl of 1.4M SERPs and examined average SERP domain quality scores over the most confident predictions (Figure~\ref{fig:qualwindow}), evaluated annotated precision\@K and the prevalence of Children's Immortality Project queries in top results (Table~\ref{tbl:precisionCIP}), and evaluated a case study around a frequently observed set of search directives (Appendix~\ref{sec:appendix-cip}).

Some SERPs that returned results for crawl-1 did not return results for crawl-2 (8.8K) and vice versa (83K). Additionally, as with crawl-1, we only include domains which appeared at least twice across all of crawl-2 when running inference. To make results comparable across crawls, 
we use the intersection of queries on which we successfully ran inference in each crawl (1.4M). Figure~\ref{fig:qualwindow} contains only predictions on queries in this intersecting set, i.e., SERPs on which we could run inference in both crawl-1 and crawl-2.



\section{\textsc{The World Wide Dishes System}}\label{sec:system}

The \textsc{World Wide Dishes System} is an infrastructure for building the \textsc{World Wide Dishes Dataset}. The \textsc{WWD Dataset} consists of local dishes from around the world built with input from community Contributors who interact with the \textsc{WWD System} by sharing personal, local knowledge of the dishes they attribute to their own home(s) and culture(s). For each dish, we include the name of the dish (in both the local language and English), the country of origin, the region of origin, the associated culture, the time of day at which the meal is eaten, the type of meal, the utensils used, the drinks that accompany the meal, any special occasions when the meal is eaten, the ingredients, the recipe, and the image of the dish if available. 

In this section, we describe the stakeholder groups who interacted with the \textsc{WWD System}. Interactions included the construction of the underlying architecture as well as the \textsc{WWD Dataset}. 

\subsection{How stakeholders interact with the \textsc{World Wide Dishes System}}
Three key stakeholder groups interact with the \textsc{WWD System}: Core Organisers, Contributors, and Community Ambassadors. The stakeholder groups are not mutually exclusive; for example, some Community Ambassadors were also Contributors. See~\cref{tab:stakeholder_tally} for an overview of the stakeholder groups, participant counts in each group, and geographic regions represented within each group. Below we describe how each stakeholder group interacted with \textsc{WWD} and with each other. 
\begin{table}[h]
\centering
\renewcommand{\arraystretch}{1.5}

\caption{\small \textbf{Stakeholders.} We indicate the demographics for the stakeholders associated with the projects.}
\label{tab:stakeholder_tally}
\small
\begin{tabular}{p{2cm}>{\footnotesize}p{3.5cm}p{2cm}p{3.5cm}p{1.5cm}}
\toprule
\textbf{Stakeholder} & \textbf{Role} & \textbf{Participant count} & \textbf{National identities by continent} & \textbf{Age range}\\ 
\midrule
Core Organisers & the central organising team that coordinated the development and execution of World Wide Dishes. Core Organisers were also Community Ambassadors and Contributors & 12 & Africa, Asia, North America & - \\
Contributors and Community Ambassadors & These roles often overlapped, with Community Ambassadors supporting amplification of the WWD and Contributors submitting dishes & 162 & Africa, Asia, Europe, North America, South America, Oceania & 19-62; Mean=31.5$\pm$8.78 \\
\bottomrule
\end{tabular}
\end{table}

\textbf{Core Organisers:} Began the initiative to build \textsc{WWD}. They created the data collection process, front-end \textsc{WWD} project website, and back-end database, as well as the data processing pipeline. 

\textbf{Contributors:} Accessed the \textsc{WWD} data collection form via the \textsc{WWD} project website and provided cultural knowledge about each dish submitted through the form. Contributors also provided feedback on other entries by other Contributors.

\textbf{Community Ambassadors:} Distributed the \textsc{WWD} webpage to and through their social networks to solicit contributions. Community Ambassadors hosted focus groups and sometimes filled out the data collection form on behalf of Contributors who faced issues with submitting the form themselves. Community Ambassadors assisted Contributors with translations if needed to ensure submissions were in English. Community Ambassadors often acted as Contributors, submitting dish information to the \textsc{WWD} data collection form. Additionally, Community Ambassadors served as a communication channel between the Contributors and Core Organisers, surfacing Contributor concerns as they arose and communicating subsequent actions by the Core Organisers back to the Contributors. Finally, Community Ambassadors assisted in the data processing stage.


\subsection{{\textsc{World Wide Dishes System}} architecture}
The \textsc{WWD System} is a web application (\Cref{fig:front_page}) built with Django\footnote{https://www.djangoproject.com/} and hosted on Google Cloud Platform, with a PostgreSQL database for storing submissions and Google Cloud Storage for storing image files. \Cref{fig:WWD Architecture} presents an overview of the system components. The backend processes which handle data collection, storage, and processing are described below.

\begin{figure}[t]
    \centering
    \includegraphics[width=\textwidth]{figures/WWD_Architecture.png}
    \caption{\textbf{Overview of the World Wide Dishes flow}. (A) A Contributor accesses \textsc{WWD} through a web browser. They consent to be a research participant and decide whether to create an account or proceed as a guest. They then fill out the data collection form with information about themselves and the dish they submit.  (B) The submission is then stored in the \textsc{WWD} database. We store Contributors' information separately from the dish information to preserve their privacy. (C) The full database containing dish information is then used to operationalise bias investigations into generated text and image content using other vision-language and language models. (D,E) A report then is generated from the automated bias testing and the survey responses and then made public. \textit{All icons in this figure were downloaded from Flaticon. For proper attribution please see \ref{asec:informed_consent}.}} 
    \label{fig:WWD Architecture}
\end{figure}

 

\begin{figure}[t]
    \centering
    \includegraphics[width=\textwidth]{figures/front_page.png}
    \caption{\textbf{Front page of the World Wide Dishes website}. Two call-to-action buttons were placed at the top of the page to encourage participation from site visitors. The ‘Check out our food leaderboard’ button takes site visitors to the leaderboard table, while the ‘Add your local dish’ button guides visitors through the data entry flow.}
    \label{fig:front_page}
\end{figure}

\subsubsection{Data collection}
The questions asked in the data collection form were collated in consultation with experts on food and cultural heritage, as well as feedback from community members, and a desire to capture as much metadata as possible to highlight regional differences in dishes. Food as a cultural object moves easily across borders and therefore the same dish may exist in more than one place---but the customs or range of ingredients associated with it may differ with changes in geography or the passage of time. The questions were therefore designed to give Contributors the opportunity to provide as much local nuance as possible. We present a list of data collected below.

\begin{enumerate}
    \item Data collection involved an online form (see~\cref{asec:protocol-screenshots} for the questions and~\cref{tab:dish_questions} for the data points).
    \item Request to upload a non-generated image of food from personal records (shared with consent to distribute for research purposes). 
    \item Questions about food. 
    \item Submission of dish name was encouraged in the local language, with a translation / phonetic equivalent included.
    \item English was encouraged, but concession was made if a translation for an ingredient didn't exist/wasn't known, e.g. cassava as an ingredient / sadsa as a dish.
    \item Associated customs and information about the dish were collected: time of day eaten, utensils, ingredients, associated celebrations/events, online recipe, freeform general information.
\end{enumerate}

\begin{table}[h]
\centering
\caption{\small Data collected through the  \textbf{World Wide Dishes} contribution form}
\label{tab:dish_questions}
\begin{tabular}{ll}
\toprule
\textbf{Question} & \textbf{Data entry} \\
\midrule
Image and caption & Image upload and text caption \\
Dish name in a local language & Short text \\
Name of local language & Short text \\
Country / countries & Dropdown and free text \\
Region & Free text \\
Attribution to a specific cultural, social, or ethnic group & Free text \\
Time of day eaten & Multiple choice \\
Dish classification & Free text or dropdown \\
Components, elements, and/or ingredients & Dropdown and free text \\
Utensils & Free text \\
Accompanying drinks & Free text \\
Association with a special occasion & Multiple choice \\
Recipe & URL \\
Any other comments & Long-form free text \\
\bottomrule
\end{tabular}
\end{table}

\subsubsection{Recruitment and engagement strategies}
The system is built for easy sharing on social networks, prioritizing cross-browser compatibility, responsiveness, and a lightweight design to ensure accessibility on a wide range of devices. All information about the project and the data collection is available on the website. Recruitment involved many social networks, including through existing communities such as AYA,\footnote{https://aya.for.ai/} Masakhane,\footnote{https://www.masakhane.io/} the Deep Learning IndabaX network,\footnote{https://deeplearningindaba.com/2024/indabax/} AI Saturdays Lagos,\footnote{https://aisaturdayslagos.github.io/} and OLS.\footnote{https://we-are-ols.org/} Posts were also placed on a mailing list to encourage engagement.

Engagement was explicitly encouraged through the promotion of a submission leaderboard (see~\cref{fig:leaderboard}),
which displayed the number of dish submissions and contributors received from each region. This was done in an attempt to gamify the experience and build excitement and fun. The leaderboard was promoted during recruitment outreach. 

\begin{figure}[t]
    \centering
    \includegraphics[width=0.6\textwidth]{figures/WWD_Leaderboard_new.png}
    \caption{\textbf{Screenshot of the World Wide Dishes  Learderboard} showing the top 10 countries by number of dishes, the total number of contributed dishes and contributors per country.}
    \label{fig:leaderboard}
\end{figure}

\subsubsection{Data storage}  The collected data is stored in two sections: the dish data, which has been made public and is completely anonymised, and the personal data, which has been collected and stored according to the terms agreed by the ethics review. We collected names, ages, and national identities, and also accepted anonymous submissions. Names were only used when explicit consent was given to acknowledge contributions publicly. Age and national identity data were required to help us understand the demographics of people submitting to \textsc{WWD}. Approval for data collection and the subsequent research study was obtained from the Departmental Research Ethics Committees of the Computer Science Department at the University of Oxford (reference: CS\_C1A\_24\_004). Notably, the ethics review required us to collect age data, as Contributors had to be over the age of 18 to participate; age data was therefore collected even if the Contributor otherwise submitted dish data anonymously.

\subsubsection{Data processing}

\begin{enumerate}
    \item \textbf{Data cleaning} Core Organisers lead the data cleaning process to remove duplicate entries within countries and to standardise data entry (i.e. uniform descriptions of regions in a country and language). Core Organisers consulted Community Ambassadors when needed.
    \item \textbf{Translations} Entries were primarily in English, except for some dish names and ingredients without known English translations. For submissions made by French-speaking Contributors from the Democratic Republic of Congo, and to make concessions for accessibility, the Core Organisers accepted these specific entries and translated them with an open source machine translation system and a Core Organiser who is a native French speaker audited these results.
    \item \textbf{Removal of images with uncertain licences} by Core Organisers involved the removal of any uploaded images whose licence could not be verified. Accepted images came from royalty-free sites that did not prohibit their use in machine learning, had an accompanying Creative Commons licence, or had been taken and submitted by a Contributor with explicit permission given for it to be used for research purposes. 
    \item \textbf{To augment the image data}, Core Organisers solicited additional royalty-free and/or Creative Commons images of the submitted dishes from the Internet and consulted Community Ambassadors for their assessments of the images' accuracy in depicting the submitted dishes.
    \item \textbf{Inconsistencies in submitted data were handled by consulting Community Ambassadors} where possible, and Community Ambassadors consulted with other Contributors as needed. However, we aimed to collect lived experiences around the world and so we did not make any efforts to police a `ground truth' for each dish, or assign an `origin' or `authenticity' to any dish. Multiple nations or cultural groups can share the same dishes, for a wide variety of reasons including historical trade routes, war and occupation, the redrafting of political borders without regard to cultural borders, and migration patterns. Our task was not to make judgments about cultural `ownership' of the submitted dishes. Instead, the data collection asked participants to offer dishes from their own lived experiences and personal backgrounds, so we expected to see similarities across borders, as well as some reasonable regional and preference variations of the ingredients of dishes within and across borders.
\end{enumerate}




























 \section{Building \textsc{World Wide Dishes}: Lessons from the field}\label{sec:rollout}



Building the \textsc{WWD Dataset} was a process that took over seven months, and involved 12 Core Organisers and more than 170 Community Ambassadors and Contributors. Some of the Core Organisers also acted as Community Ambassadors (see~\cref{tab:stakeholder_tally} for the regions for which each Core Organiser served as a Community Ambassador). The process of launching the \textsc{WWD} data collection effort involved a beta-testing stage to solicit an initial round of feedback from community members about the data collection instrument and process, as well as hosting ongoing focus groups with Contributors to surface issues with data donations. Through systematic analysis of post-mortems and field notes from the Core Organisers, we identify four key elements of the process to build \textsc{WWD} that ultimately made this a successful effort: trust within the community, accessibility of participation, attention to the production of data, and understanding the relationship between food and culture. Below, we reflect on and surface how the \textsc{WWD} dishes project supported each element.  

\subsection{Building trust and leveraging existing social networks to access community members}
Collecting long-tailed data requires accessing and connecting with underrepresented populations who tend to be ``hard to reach.''\footnote{We use quotation marks to draw attention to the fact that the term ``hard to reach'' is value-laden. Calling a population ``hard to reach'' redirects attention away from the reasons why that population may be difficult for the researcher to access and instead focuses attention on the group as the problem, rather than on the institution or researcher as the potential problem. For more, see~\cite{benjamin2016informed,epstein2008rise}.} Obtaining this access typically requires working with community gatekeepers who can serve as intermediaries between the researchers and community~\cite{le2015strangers,epstein2008rise}. However, the presence of a community gatekeeper in a project does not in itself resolve ethical questions around shifting the distribution of power from the researchers to the researched. In building \textsc{WWD}, we made an intentional choice to involve community gatekeepers (e.g., Community Ambassadors) as members of the research team who shaped the data collection process, the development of the data collection tool, and the paper authorship process.  

The first author, who is a researcher at a prestigious Western university, recognised that their positionality put them in a position of power as relates to the collection of cultural data from communities to which they do not belong. While they are originally from South Africa and have been deeply involved in community efforts within the African ML community, such as the Deep Learning Indaba, they felt that the \textsc{WWD} project must be co-led by members from communities who were contributing data. Therefore, they invited three people in their professional network who were active researchers and volunteers for local AI communities to be co-leads and co-authors on the project. Two of the three people, O5 and O7, are Nigerian, and Sudanese, respectively, and were experienced in conducting data collection efforts with communities within their home countries. As O5, O6, and O7 spread information about \textsc{WWD} to their networks, additional community members representing more than eight countries on the African continent signed up to be Community Ambassadors for the project. Community Ambassadors leveraged their positionality, relationships, and social capital to make data collection with communities in Africa possible. 

Community Ambassadors reflected on how their previous work within the African community and being members of the local communities from which they were soliciting data donations was crucial to the success of \textsc{WWD}. In post-mortem reflections, O5 shared:
\begin{quote}
    ``As a community leader, I've been organizing free classes for Community Y\footnote{Community name is anonymised for review.} for over six years. Over time, this commitment has built trust within my community, reassuring people that I genuinely prioritise our shared interests. I would like to think that this trust helped community members feel comfortable contributing to the effort, knowing they aren’t being taken advantage of.'' O5
\end{quote}
Community Ambassadors had established reputations within the communities from which they were soliciting data donations, enabling them to collect cultural data alongside community members who trusted their intentions. 

Established methods of accessing data donors, such as cold-calling, or simply distributing the call to participate on mass broadcast social media channels (X, formerly known as Twitter; LinkedIn), were arguably less successful, even when it was the Community Ambassadors themselves who posted the call. Access to, or the ability to access, potential donors was insufficient. Instead, Community Ambassadors often had to rely on the relationships they had already established to solicit contributions. For example, O6 reported in their post-mortem that: ``So, I kept sharing and sharing over social media ... but the contributions were not growing as much as I hoped ... So I reach out to people individually.'' As a result, Community Ambassadors tapped into their network of friends and family to solicit data contributions. However, reaching out to personal networks was not without risk. 

Data contributions were not \textit{financially} remunerated. Therefore, some Community Ambassadors shared that the lack of financial compensation made them selective in who they reached out to about the project, as they were essentially asking their friends and family to do volunteer labour. However, despite the lack of remuneration, Community Ambassadors reported that many of the people to whom they reached out shared a common desire to participate in a project that could make GenAI systems work for them:
\begin{quote}
    ``It's about strengthening African machine learning, it's about empowering Africans, it's about making sure that all of the AI technology works for us all, and ensuring we're part of the builders of it... so maybe through our several grassroots machine learning efforts, we can get to some kind of balance in [power] regarding whose perspectives shape the building of AI technologies.'' O5
\end{quote}

Community Ambassadors controlled how they, and by proxy the larger \textsc{WWD} team, interacted with their communities. For example, the Core Organisers and Community Ambassadors administered WhatsApp groups within the larger \textsc{WWD} WhatsApp community, where they carried out conversations with Contributors in the language of the region they oversaw. Community Ambassadors had complete oversight over how they wanted to structure conversations within their WhatsApp group and enforce moderation rules that were proposed by the Core Organisers. 

Community Ambassadors had a shared context with the people whom they asked to contribute data to the project: they spoke the same language as the Contributors, had previously established themselves as trustworthy, and shared similar goals for African leadership in ML projects. Building on this, we expand on the need to meet community members where they are and make participation accessible.

\subsection{Make participation accessible}

Engaging in bottom-up, community-led data collection requires extra efforts on the part of researchers to make participation accessible. In this project, the researchers did so by providing a more accessible informed consent process and hosting ongoing office hours to help Contributors learn how to participate and understand the goals and motivations of the \textsc{WWD} project. It is not enough to build a data contribution form that is ``easy'' to fill out; rather, researchers must invest in significant amounts of human labour to animate the data collection effort. 

During beta-testing, Community Ambassadors flagged that while the consent form was accessible---in that it could be translated and read in the local language of contributors---it was not understandable. The research team had originally presented the entire consent form, as approved by their institution's ethics review board, as the first step of the data contribution process on the \textsc{WWD} website. Following feedback from Community Ambassadors, the research team co-designed a concise consent form (see \ref{fig:informed_consent_new}) that laid out key information about the project's purpose, procedures, and data privacy protections. The full consent form was then linked on the consent overview page (see~\cref{asec:informed_consent}). Upon reflection, O1 recounted that consent forms designed and approved in Western, \textit{academic} contexts were inaccessible for the regions in which the Contributors were located. The researchers had to engage in translation of the meaning of certain terms, such as ``cookies'', to make informed consent accessible for all participants. 

Collecting image data for ML purposes, as \textsc{WWD} does, is a complex process due to concerns about image ownership. Because \textsc{WWD} was intended to be an open-source dataset with Creative Commons licensing, all submitted images either had to have a Creative Commons license (if it wasn't an original) or be an original (e.g., taken by the contributor) photo shared with explicit permission that it could be used for research purposes. These guidelines were presented clearly on the data contribution form; however, Community Ambassadors noticed early on that Contributors were uploading images from the Internet that did not have Creative Commons licenses. As a result, the Core Organisers and Community Ambassadors began hosting regular ``teach-ins'' during office hours for Contributors to attend in order to learn how to properly make a data donation. O6 explained the role of office hours in making data contribution guidelines accessible to participants:
\begin{quote}
    ``The [office hours] helped ensure our data collection met our requirements and that contributors understood both the licensing requirements and how to properly complete the submission forms, enabling us to collect exactly what we needed for the project'' O6
\end{quote}
Office hours lowered the barrier to entry for participation by providing a space for Contributors to get help filling in the information on their submissions. Community Ambassadors recalled how Contributors would bring ideas about dishes they wanted to submit, but needed help finding an image they could use or verifying other data fields, such as utensils used and/or associated cultural ceremonies (as examples), for a dish. In these cases, the office hours attendees would work together to complete a submission collaboratively. 

Office hours also served as a space for Contributors to shape the research process. Office hours attendees were encouraged to share feedback about challenges they faced in making data donations, which the Community Ambassadors then brought to planning meetings with the Core Organisers. As a result, the Core Organisers amended the data collection instrument, produced additional guidelines for participation, and refined their recruitment strategy. O1 recalls how the team wanted to follow so-called ``professional protocol'' and communicate through official and professional media such as mailing lists and social media. The members of the community immediately asked for WhatsApp groups. 
\begin{quote}
    ``As a team, we tried to think through the implications---the most poignant being the lack of a boundary between our own work and professional lives by engaging with our personal WhatsApp applications. However, having already identified how important it was for us to meet the community where they were, and having had similar calls in the past through our work with established communities on the African continent, we made the commitment to host the community groups on WhatsApp, and maintain open communication between ourselves to make sure we could sustainably engage throughout the data collection process.'' O1 
\end{quote}
Building on the feedback received from Community Ambassadors and Contributors, the Core Organisers quickly realised how much behind-the-scenes effort was going into a single submission. Data are found through extensive consultation and effort, and the final entry on the website is a reduced form of the rich engagement that led to it. 

\subsection{Recognise that data are produced, not simply found}

Throughout the process of building \textsc{WWD}, it became clear that cultural data is something that is produced through social interaction. The concept of data production is not new~\cite{bowker2000sorting,d2024counting}. However, it is often glossed over in ML papers that present novel datasets while only briefly discussing the data work~\cite{scheuermanDatasetsHavePolitics2021} that goes into constructing a dataset. In our findings, we disrupt the assumption that a single participant in a crowdsourcing effort can submit complete information about an object of cultural significance and shed light on the myriad social interactions underpinning individual data contributions.

Contributors actively engaged in conversations with family and friends, accessing familial networks to collect the data needed for a submission. For example, O8 shared one Contributor's anecdote in their post-mortem where the Contributor asked their friends and family to help complete data submissions:
\begin{quote}
    ``For the photos and information I couldn’t get myself, I asked certain resourceful people, particularly my parents and friends, to provide the photos and information about the utensils to use and the appropriate time to eat certain meals'' O8, \textit{summarizing what a Contributor told them}
\end{quote}
While it could be argued that Contributors could simply look up dish information on the Internet to fill out their forms, this was not a viable option for many of our Contributors coming from countries where information about their regional cuisines was simply not available online. Familial networks proved to be a more reliable way to get accurate information about local cuisine. O8 recounted an experience where a Contributor in the community they oversaw encountered difficulty finding information about a regional Cameroonian dish they had grown up eating:
\begin{quote}
    ``For the dishes I struggled to explain, I turned to various sources for help. When possible, I looked online to see how they were made. However, for some dishes, it was difficult to find accurate information. In those cases, I reached out to my mother and grandmother, who were able to share their expertise and guide me through the process'' O8, \textit{summarising what a Contributor told them}
\end{quote}
Community Ambassadors and Contributors alike often turned to their familial networks, especially parents and grandparents, to produce the information they needed about a dish through conversations and glimpses into the archives of family recipes. O5 recalled that during their office hours, Contributors shared stories about calling their mothers and sisters to get help filling out information for a dish: ``Some [Contributors] mentioned that they reached out to their family to learn more about the native dishes they grew up eating because they realise they don't know how to prepare them.''  

Data about food, and how it is connected to a community's culture, is not simply sitting somewhere, waiting to be harvested. Moreover, in the regions in which we were operating, there was little to no reliable information about cultural cuisines available online. As a result, Contributors and Community Ambassadors produced this data through conversations with kin. Food is deeply intertwined with cultural practices and therefore some of our Community Ambassadors chose to ground their data collection efforts in major cultural events that coincided with the data collection phase. 

For example, O6 shared how they deliberately chose to revive their call for data contributions during Ramadan and Eid al-Fitr. Ramadan and Eid al-Fitr, which are major holidays for Muslims, are often celebrated with special dishes holding cultural significance that are only prepared during the holiday period. O6 explained: ``Food plays a great, very important role in Ramadan and so I was very intentional about just taking pictures of whatever food we made for iftar which is the breaking of the fast.'' The food that is prepared during these holidays holds cultural significance and helps tell a part of the story about a community's values and history. 

Communities are not passive data sources; these data about food were produced through social interactions and lived experiences. 

\subsection{Understand the relationships between food and culture}

Food is complex: dishes often harbour a deeper cultural significance. For example, O6 recalled how during their experience managing data contributions from Egypt, they realised that the classification of ``Egyptian food'' was fuzzy at best:
\begin{quote}
    ``[WWD] made me more conscious about \textbf{what is} Egyptian Cuisine and the influences from other cuisines that have actually, you know, shaped what we eat and what we consider as Egyptian. As it turns out, it's very intersectional.'' O6, emphasis added
\end{quote}
For O6, capturing data about Egyptian cuisine led them to further investigate the origins of their own cultural dishes. In the weeks following the data collection effort, they attended a session about "food as cultural currency" at the RiseUp Summit Egypt 2024 to learn more about the Egyptian cuisine through the eyes of local food experts.

O1 recalled how the fuzzy classification of dishes as belonging to a certain region impacted the data collection process:
\begin{quote}
    ``...depending on the region within the country [the dish] was either going to be made from the cassava plant or a kind of mielie meal. It could be the same dish but depending on the region within the same country they would have a different ingredient for the starch [element].'' O1
\end{quote}

Here, O1 identifies that even though a dish is considered ``the same'' and may be called by the same name across different regions, the actual ingredients in this dish differ and reflect the differing agriculture practices of various regions. These differing agricultural practices are important markers of the local ecosystems and historical practices of farming and cultivation. Therefore, in the data collection process, it became essential to deconstruct dishes into their constituent ingredients, as reflected in the data structure for \textsc{WWD} (see~\cref{tab:dish_questions} and~\cref{asec:protocol-screenshots}). 


In post-mortem reflections, Community Ambassadors highlighted the importance of recognising the limitations of using national borders to demarcate cultures. The granularity of representation was essential to ensuring that distinct cultures were not lumped together into a single flattened snapshot, as physical borders often do. O5 recalls how she realised that one of the three major ethnic groups in Nigeria---the Hausa community---was underrepresented in the \textsc{WWD} dataset. They reached out to a personal connection whom they knew was a member of that ethnic community who, in turn, helped share the website with their community. 


In line with what we posited in the introduction, we reinforce the idea that the value in a dataset is not exclusively conveyed in its final form, but also through the processes of creating it. In this Findings section, we highlight the immense efforts and considerations that went into engaging with our Contributors to produce high-quality, granular data about cultural dishes. Processes such as these are slow, iterative, and very hard to scale, but they are necessary to ensure the production of high-quality and diverse datasets that we would like to see reflected in the ML systems that make use of this data. 




 







\section{Discussion}
\label{sec:discussion}
\textsc{WWD} is a socio-technical infrastructure that supports the collection of cultural data, in the form of food, in a bottom-up community-led manner. Community members' needs and experiences actively shaped the architecture of WWD. Our data collection platform was constructed to be compatible with the established digital infrastructure and cultural norms of the communities we worked with. The types of data we collected (e.g., the attributes for each dish) were informed by community members who identified what was important to capture about a dish to accurately represent how the dish is prepared and consumed in their culture. 

Building the \textsc{WWD System} in a bottom-up, community-led manner required an immense amount of labour. Data did not simply flood in once the system architecture was built. Core Organisers and Community Ambassadors engaged in \textit{data work}---a socio-technical process through which data about local cuisines was produced. As many social computing scholars have noted, data work is often overlooked despite its essential role in shaping the epistemology of a dataset and consequently the downstream performance of ML systems~\cite{sambasivan2021everyone,ismailEngagingSolidarityData2018,mollerWhoDoesWork2020,scheuermanProductsPositionalityHow2024}. 

In our discussion, we surface the tensions that occurred during data work. These breakdowns in the data work process help us to reveal deeper structural issues in the AI/ML production pipeline that confound bottom-up, community-led approaches to dataset construction. Communities facing representational harms~\cite{weidinger2021ethical} and disparities in quality of service~\cite{shankar2017allocational, de2019doescvworkallocational} face a catch-22 when participating in efforts to improve dataset coverage: They can shoulder the burden of participation or be excluded from model ontology. New technologies, particularly GenAI tools, have been proposed as a way for communities to preserve their culture representation by participating in efforts to contribute data to model training~\cite{heritage7030070}. However, participation does not necessarily entail improved outcomes for communities~\cite{birhanePowerPeopleOpportunities2022}. We point to a difference in ethical frameworks between communities on the African continent with whom we worked and those of large tech companies that build and control GenAI technologies to illuminate why the promises of participation often fall short. 


\subsection{Tensions in data collection}
 
Our results show that there were tensions around data collection. Specifically, issues surrounding image provenance, the accuracy of information about a dish, and the benefits of participation arose throughout the data collection process. These issues reveal deeper structural problems with the AI/ML pipeline. 
\subsubsection{Establishing a clean bill of data provenance}
Recent efforts in participatory ML research attempt to safeguard the labour and intellectual property of community data contributors by creating dataset licenses restricting the use of the community's dataset, which assign ownership and terms of access and use to these datasets~\cite{birhanePowerPeopleOpportunities2022,longpre2024largelicence}. However, for the license to be effective, the data must have a clean bill of provenance. \textsc{World Wide Dishes} was built to be an open-source dataset with a Creative Commons license that could be used for model evaluation. As a result, the WWD dataset had to fulfil strict requirements for data quality, including ensuring that the dataset creators had a right to the images contained within the dataset.  In other words, the creators of \textsc{WWD} must then be able to claim rightful use and ownership over all the images collected as part of the project. However, many of the images that Contributors submitted during the data collection phase were taken from the Internet and lacked proper licensing. As a result, Community Ambassadors had to engage in extensive consultation and discussion with Contributors to ensure they understood the importance of data provenance in their submissions. Contributions, where the origins of the submitted were unclear, had to be deleted, erasing bits of cultural knowledge from our dataset. Ensuring a clean bill of data provenance was time-intensive and not easily scalable. It was difficult to enforce image upload guidelines in a volunteer effort, resulting in a smaller, less representative, dataset than we would have liked. However, the rigorous process of ensuring a clean bill of data provenance for each submission enabled us to, in good faith, release our dataset as an open-source project. 

The standard of open-sourcing datasets and applying Creative Commons license, while understandable, places massive burdens on small, community-led projects such as \textsc{WWD} to ensure clean bills of data provenance. To be clear, we are not arguing against open-source datasets or Creative Commons licenses, but rather are demonstrating the need to build infrastructures that support and fund the labour needed to verify that data for their projects can be used. As mentioned in~\cref{background}, understanding cultural nuance on a fine-grained, regional scale requires extensive (and non-extractive) consultation with community members who have the capacity to share local expertise. As such, non-exploitative and non-extractive community consultation is an important step in verifying the validity and veracity of cultural information. 

\subsubsection{Verifying cultural information}

Collecting accurate and representative cultural data is exceptionally difficult. Cultures are not bounded by government borders and/or other manufactured systems, but rather extend across larger regions and are often the product of intercultural exchanges~\cite{gupta2008beyondculture}. This makes determining the veracity of a data point in \textsc{WWD} almost impossible without extensive consultation with a community member with local expertise. In \textsc{WWD}, we sought to include as many Contributors as possible to collect a granular representation of cultural data. We accessed Contributors through our community ambassadors who had established relationships and trust with the folks they asked to contribute. Inclusion, however, can be a slippery slope~\cite{epstein2008rise,benjamin2016informed}. 

ML researchers continue to pursue the construction of ever more representative datasets in the name of improving model performance for \textit{everyone}~\cite{luccioni2021everyone, radford2018improvingeveryone}. Often, ML researchers have trouble accessing ``hard-to-reach'' populations, such as the communities we worked with to build \textsc{WWD}. Many recent projects have attempted to solicit engagement from ``hard-to-reach'' populations~\cite{kirk2024prism,ramaswamy2023geodegeographicallydiverseevaluation,singh2024aya_dataset}, yet none of these projects interrogate why these populations might be hard for researchers to access. Drawing on Benjamin's work~\cite{benjamin2016informed}, \textbf{we urge ML researchers to consider how research institutions and industry laboratories may engender distrust within communities that have endured centuries of extractive practices by actors from the Global North.} It is essential that researchers not only endeavour to make participation accessible to members of ``hard-to-reach'' communities but also work towards establishing themselves as trustworthy partners in the research process, in the same way that~\citet{singh2024aya_dataset} do this. 

\subsubsection{Explaining the benefits of participation}

Community Ambassadors wrestled with explaining the benefits of participation in \textsc{WWD} to potential Contributors. Participation was not financially compensated. The research team chose not to make use of professional data centre workers\footnote{Professional data centre workers are those people employed in a centralised manner to perform data collection tasks. Their livelihood is, therefore, connected to the requirement to engage in data contribution, which does not align with \textsc{WWD} goals. Additionally, even had we wished to use data centre workers, we lacked the resources to which a large technology company might have access, such as the ability to engage a business outsourcing company (e.g., Enlabler) to recruit and pay data workers.} because the nature of the data collection process argued for prioritising organic engagement through social networks to collect perspectives from people who do not, and have not, typically contributed to Internet datasets from around the world. We purposely chose a data collection method that would enable the use of social networks and allow us to reach participants other than those employed in a data worker centre, such as older generations and those across a wide socioeconomic range. We also wanted to empower participants to involve their families in the process. 

Although the research team would have preferred to individually compensate each Contributor, because \textsc{WWD} relies on a decentralised, global-scale data collection method, and, crucially, as of the time of data collection, normative standards and infrastructure do not exist to support such a decentralised payment process to effectively \textbf{pay data contributors}, we were \textit{unable} to pay them. The research team explored many possible avenues for paying participants but each time came up against prohibitively expensive and logistically insurmountable barriers. For example, money transfer services such as PayPal~\cite{paypal_countries_2023} and Wise~\cite{wise_usage_2023} were unavailable in many of the regions where \textsc{WWD} operates. These types of services also require that payment recipients have access to digital banking services, which many within our target communities do not. In addition, some of our Core Organisers, who are from the African continent and utilise digital banking services, provided anecdotal evidence of times when their transactions were flagged for seemingly no other reason than their nationality. Infrastructures to support financial remuneration for research participants in the Majority World are simply not commensurate with the many calls from Western researchers to engage participants in these parts of the world. \textbf{Researchers must therefore build the infrastructures to enable equitable participation with communities}; in particular, researchers should investigate how to address breakdowns in participant compensation infrastructures. Other similarly decentralised efforts have remunerated contributors with material items (e.g., sweaters and small gifts)~\cite{singh2024aya_dataset}. Still other researchers point to the limitations of financial compensation for participants and urge researchers to consider what kinds of remuneration would be useful given the context of their research site~\cite{hodge2020relational}.

Despite the lack of extrinsic, financial incentives, the Contributors did exhibit some intrinsic motivation. Contributors shared many different reasons for having participated, such as wanting to make a difference in GenAI outputs, supporting a friend, or contributing to a mission and team they believed in. The majority of data contributions came from the African continent. The authors have speculated why this might be, and have wondered if there is a common focus uniting these Contributors: a central philosophy of ``familyhood'' and unity. This is known by different terms across the continent, including djema’a (Arabic), ubuntu (Zulu), ujamaa (KiSwahili), umuntu (Chichewa), and unhu (Shona). Community Ambassadors also suspected that their positionality as members of the communities from which they were soliciting data contributions further strengthened sentiments of unity among participants who saw \textsc{WWD} as an extension of the growing ``By Africans, for Africans'' movement in ML~\cite{birhane_2024_for_africans}. 

Whilst we can only speculate about why participants engaged in the data contribution process, the authors recognise the responsibility they were given to respect and honour these Contributors and to avoid extractive and exploitative practices. 

\subsection{Participate or be excluded: A catch-22}


Cultural erasure and lack of representation are rooted in deep systemic issues that date back centuries. GenAI, especially T2I models, play an increasingly prominent role in shaping the media ecosystem. However, relying on these models to ``fix'' centuries of intentional cultural erasure overlooks the deeper systemic issues that will likely constrain the efficacy of these technocentric solutions. During the data collecting for \textsc{WWD}, Community Ambassadors often found themselves rationalising the uncompensated nature of data contributions by demonstrating that existing T2I models perform poorly when creating images of local dishes so Contributors should provide accurate data to teach the model what the dish should look like. Regardless, many Contributors, Community Ambassadors recalled, were eager to participate in an African-researcher-led ML effort. 

Through the reflection process, Community Ambassadors shared conflicting feelings about tapping into the shared philosophy of familyhood and unity that they suspected motivated Contributors' participation. On the one hand, local communities were engaging in the dataset creation process through the lens of \textit{Ubuntu} (broadly translated as ``I am because we are'')---an ethical framework that emphasises dignity, reciprocity, and the common good~\cite{ewuoso2019core}. In contrast, the models that would subsequently be trained by these datasets are developed in Western contexts and imbued with utilitarian ethics---a framework that emphasises the best for the greatest number of people~\cite{selbstFairnessAbstractionSociotechnical2019,west2004introduction}. These two distinct, yet interrelated elements of the ML pipeline---dataset production and model development---are therefore produced not only in distinct geographic regions~\cite{sambasivan2021everyone,scheuermanDatasetsHavePolitics2021} but also, in our case, in two distinct ethical frameworks. Contributors who engaged with us out of a sense of \textit{Ubuntu} are unlikely to see their values recognised and preserved in the actual functioning of the downstream T2I model that is optimising for fundamentally different well-being criteria.  

Participation in dataset construction is not a guaranteed way to achieve representational justice in T2I models. Thus, proposing to communities that to avoid being excluded from the future of media representation, they should participate in dataset development, is misleading. This false choice obfuscates (1) the deeper systematic issues that dictate whose culture gets preserved and represented and (2) the disjunction between the value system under which participants may contribute data and that of the models that are then trained on this data. 

Future research efforts should examine how to bring the ethical frameworks of dataset creation and model development into alignment by prioritising local, community ownership over AI. 


\section{Limitations and Future Work}\label{sec:limitations}

\subsection{Limitations}
The distributed network we relied on provides a challenge for financial payment. We believe that meaningful compensation is an important standard in data collection that relies on community-based knowledge is important. However, we acknowledge a stalemate, in that to the best of our knowledge, normative standards around payment mechanisms and logistics for our novel decentralised approach have yet to be established. In preparing this work, we have---and continue to, extensively---engaged with other practitioners and community members to understand best practices. At this moment, we rely on a process of transparency and informed consent rooted in direct attempts to empower participants such that the practice is less extractive. We maintain a constant communication channel with all Contributors (who are rightfully acknowledged in the appendix). 

\subsection{Future work}

Finally, we conclude with suggestions for how future work can engage in infrastructure to support data work. 

The work presented here is not meant to be one size fits all, but rather as a starting point from which to change the way we approach ML datasets and position their importance in terms of the \textit{human processes} behind them. Data collection that relies on community input results in high-quality data, but the process is slow, iterative, and very hard to scale by any other means than what we propose here. We hope that other researchers use this as a foundation for future work.

As mentioned above, simply creating an access point for data submission does not mean that data will easily flow in. Complete infrastructure in the future should involve both accessible technical infrastructure as well as direct engagement with stakeholder communities.

 Data has value, and promises of ``increased representation'' should be critically assessed to make sure the promises made to Contributors are sincere. Improvements can come from increasing infrastructure for payments in a decentralised and distributed network that allows for \textit{meaningful} compensation for data labour, and we continue to advocate for Contributors to be the owners of their data, as well as maintaining~\cite{tonja2024inkubalm} transparency in the process.  
Additional efforts should support increased accessibility that supports submission in a Contributor's local language to increase accessibility.

Acknowledging the burden of data collection, even with the distributed workload we propose with the stakeholder distinction between Community Ambassador and Contributor, sustainable processes for all stakeholders need to be considered. 
 
\section{Conclusion}

Subgroup analysis is an important, yet under-utilized tool in data science.
Our results suggest that combining algorithm-generated, rule-based insights with human intuition and experimentation in an interactive workflow can help practitioners develop a thorough understanding of complex datasets.
By implementing these interactions in a lightweight notebook-based tool, we hope to lower the barrier for data scientists to try subgroup discovery and to curate unexpected, interesting subpopulations in their data.
Divisi is available as an open-source package so that data scientists and HCI researchers can build on this work, helping to make exploratory subgroup analysis more feasible for a wider range of contexts.

\bibliographystyle{ACM-Reference-Format}
\bibliography{main}
\clearpage

\appendix
% \section{Framework Details}
% Our framework is described in Algorithm~\ref{algorithm}, and compared with former baselines in Table~\ref{table:comparison}. Distinct with several methods generating Python code for visualization directly, we use VQL as an intermediate representation to bridge natural language queries and visualization code. Additionally, our framework can be easily optimized by adding some useful tools such as Retrieval Augmented Generation. Moreover, our method supports handling multi-table data and the visualization can be customized according to humans' preferences. Our framework utilizes the agent-based collaborative workflow, which consists of data preprocessing, generation, and error correction, organized with the modular design.

% \begin{algorithm}
% \small
% \caption{\system Framework}
% \label{algorithm}
% \begin{algorithmic}[1]
% \Function{\nlvis}{$Q$, $S$}
%     \State Initialize $Mem \gets \{Q,S\}$
%     \State $(S', A) \gets \textsc{Processor}(Mem)$
%     \State $Mem.update(S', A)$
%     \State $V \gets \textsc{Composer}(Mem)$
%     \State $Mem.update(V)$
%     \State $Chart, isValid \gets \textsc{Validator}(Mem)$
%     \While{not $isValid$}
%         \State $V \gets \textsc{Refine}(Mem)$
%         \State $Mem.update(V)$
%         \State $Chart, isValid \gets \textsc{Validator}(Mem)$
%     \EndWhile
%     \State \Return $Chart$
% \EndFunction
% \end{algorithmic}

% \end{algorithm}




% \begin{table*}[!t]
%     \centering
    
%     \vspace{-1em}
%     \scalebox{0.68}{
%     \begin{tabular}{lccccccc}
%         \toprule[1.5pt]
%         \multirow{3}{*}{\textbf{Framework}} & \multicolumn{2}{c}{\textbf{System Features}} & \multicolumn{2}{c}{\textbf{Visualization Capabilities}} & \multicolumn{3}{c}{\textbf{Agentic Workflow}} \\
%         \cmidrule(lr){2-3} \cmidrule(lr){4-5} \cmidrule(lr){6-8}
%         & \textbf{VQL as} & \textbf{Extensible} & \textbf{Multi-Table} & \textbf{Customizable} & \textbf{Data} & \textbf{Modular} & \textbf{Error-} \\
%         & \textbf{Thoughts} & \textbf{Optimization} & \textbf{Support} & \textbf{Styling} & \textbf{Preprocess} & \textbf{Design} & \textbf{Correction} \\
%         \midrule
%         Chat2VIS~\cite{chat2vis} & \textcolor{red}{\ding{56}} & \textcolor{red}{\ding{56}} & \textcolor{red}{\ding{56}} & \textcolor{red}{\ding{56}} & \textcolor{green!60!black}{\ding{52}} & \textcolor{red}{\ding{56}} & \textcolor{red}{\ding{56}} \\
%         Mirror~\cite{mirror} & \textcolor{red}{\ding{56}} & \textcolor{red}{\ding{56}} & \textcolor{red}{\ding{56}} & \textcolor{red}{\ding{56}} & \textcolor{red}{\ding{56}} & \textcolor{green!60!black}{\ding{52}} & \textcolor{red}{\ding{56}} \\
        
%         LIDA~\cite{lida} & \textcolor{red}{\ding{56}} & \textcolor{green!60!black}{\ding{52}} & \textcolor{red}{\ding{56}} & \textcolor{green!60!black}{\ding{52}} & \textcolor{green!60!black}{\ding{52}} & \textcolor{green!60!black}{\ding{52}} & \textcolor{red}{\ding{56}} \\
%         CoML4VIS~\cite{coml} & \textcolor{red}{\ding{56}} & \textcolor{red}{\ding{56}} & \textcolor{green!60!black}{\ding{52}} & \textcolor{red}{\ding{56}} & \textcolor{green!60!black}{\ding{52}} & \textcolor{red}{\ding{56}} & \textcolor{red}{\ding{56}} \\
        
%         Prompt4VIS~\cite{prompt4vis} & \textcolor{green!60!black}{\ding{52}} & \textcolor{red}{\ding{56}} & \textcolor{green!60!black}{\ding{52}} & \textcolor{red}{\ding{56}} & \textcolor{green!60!black}{\ding{52}} & \textcolor{green!60!black}{\ding{52}} & \textcolor{red}{\ding{56}} \\
        
%         CoT-Vis~\cite{cotvis} & \textcolor{green!60!black}{\ding{52}} & \textcolor{red}{\ding{56}} & \textcolor{red}{\ding{56}} & \textcolor{red}{\ding{56}} & \textcolor{green!60!black}{\ding{52}} & \textcolor{red}{\ding{56}} & \textcolor{red}{\ding{56}} \\

%         \midrule
%         \SystemName (Ours) & \textcolor{green!60!black}{\ding{52}} & \textcolor{green!60!black}{\ding{52}} & \textcolor{green!60!black}{\ding{52}} & \textcolor{green!60!black}{\ding{52}} & \textcolor{green!60!black}{\ding{52}} & \textcolor{green!60!black}{\ding{52}} & \textcolor{green!60!black}{\ding{52}} \\
%         \bottomrule[1.5pt]
%     \end{tabular}}
% \caption{Comparison of various \nlvis frameworks. }  \label{table:comparison}
% \vspace{-1em}
% \end{table*}

\section{Detailed Experiment Setups}
\label{detailed_experiment_setups}
\paragraph{Baselines.}
\label{detailed_baselines}
% We implemented our experiment compared with three recent baselines. Note that, we also tried to use Code Interpreter as a baseline, but due to the rate limit of API constraint, the evaluation failed to generate visualizations via direct .csv files.
This study compares our approach with three state-of-the-art baselines. We also attempted to include Code Interpreter as a baseline; however, API rate limitations prevent the direct generation of visualizations from CSV files.

\begin{itemize}[leftmargin=*, itemsep=0pt] 
    \item \textbf{Chat2Vis} \cite{chat2vis}: It generates data visualizations by leveraging prompt engineering to translate natural language descriptions into visualizations. It uses a language-based table description, which includes column types and sample values, to inform the visualization generation process.\item \textbf{LIDA} \cite{lida}: It structures visualization generation as a four-step process, where each step builds on the previous one to incrementally translate natural language inputs into visualizations. It uses a JSON format to describe column statistics and samples, making it adaptable across various visualization tasks.
    \item \textbf{CoML4Vis} \cite{coml}: 
    % Building on a data science code generation framework, CoML4Vis 
    It utilizes a few-shot prompt that integrates multiple tables into a single visualization task. It summarizes data table information, including column names and samples, and then applies a few-shot prompt to guide visualization generation.
\end{itemize}

\paragraph{Metrics.}
\label{detailed_metrics}
Our evaluation framework involves five main metrics:
\begin{itemize}[leftmargin=*, itemsep=0pt] 
    \item \textbf{Invalid Rate} represents the percentage of visualizations that fail to render due to issues like incorrect API usage or other code errors.
    \item \textbf{Illegal Rate} indicates the percentage of visualizations that do not meet query requirements, which can include incorrect data transformations, mismatched chart types, or improper visualizations.
    \item \textbf{Readability Score} is the average score (range 1-5) assigned by a vision language model, like GPT-4V, for valid and legal visualizations, assessing their visual clarity and ease of interpretation.
    \item \textbf{Pass Rate} measures the proportion of visualizations in the evaluation set that are both valid (able to render) and legal (meet the query requirements).
    \item \textbf{Quality Score} is set to 0 for invalid or illegal visualizations; otherwise, it is equal to the readability score, providing an overall assessment of visualization quality factoring in both functionality and clarity.
\end{itemize}
To thoroughly evaluate each main metric, we further break them down into the following detailed assessment criteria:
\begin{itemize}[leftmargin=4mm, itemsep=0.05mm] 
    \item \textbf{Code Execution Check} verifies that the Python code generated by the model can be successfully executed.
    \item \textbf{Surface-form Check} ensures that the generated code includes necessary elements to produce a visualization like function calls to display the chart.
    \item \textbf{Chart Type Check} verifies whether the extracted chart type from the visualization matches the ground truth.
    \item \textbf{Data Check} assesses if the data used in the visualization matches the ground truth, taking into consideration potential channel swaps based on specified channels.
    \item \textbf{Order Check} evaluates whether the sorting of visual elements follows the specified query requirements.
    \item \textbf{Layout Check} examines issues like text overflow or element overlap within visualizations.
    \item \textbf{Scale \& Ticks Check} ensures that scales and ticks are appropriately chosen, avoiding unconventional representations.
    \item \textbf{Overall Readability Rating} integrates various readability checks to provide a comprehensive score considering layout, scale, text clarity, and arrangement.
\end{itemize}

% For all evaluation results, these metrics are averaged across the dataset to provide an overarching view of model performance. These metrics collectively ensure that visualizations are not only correct in terms of execution but also effective in communicating the intended data narratives.
The evaluation metrics are averaged across the dataset to provide a comprehensive overview of the model's performance. Together, these metrics ensure that the visualizations are both accurate in execution and effective in conveying the intended data narratives.



\begin{table}[!t]
\centering
\setlength{\belowcaptionskip}{0em} 
% \vspace{-1em}
\begin{tabular}{lcc}
\toprule[1.5pt]
\textbf{Model} & \textbf{P-corr} & \textbf{P-value} \\
\midrule
GPT-4o-mini & \textbf{0.6503} & 0.000 \\
GPT-4o & 0.5648 & 0.000 \\
\bottomrule[1.5pt]
\end{tabular}
\caption{ The Pearson correlations of GPT-4o-mini and GPT-4o with human judgments on readability scores. }
\label{tab:pearson_corr}
\vspace{-1em}
\end{table}

\begin{table*}[!ht]
\centering

\vspace{-1em}
\begin{tabular}{l|ccc|ccc}
\toprule
\multirow{2}{*}{Method} & \multicolumn{3}{c|}{Single Table} & \multicolumn{3}{c}{Multiple Tables} \\
\cmidrule(l){2-4} \cmidrule(l){5-7}
 & prompt & response & total & prompt & response & total \\
\midrule
LIDA & 1386.23 & 237.90 & 1624.13 & \multicolumn{3}{c}{N/A} \\
Chat2Vis & 414.35 & 451.30 & 865.65 & \multicolumn{3}{c}{N/A} \\
CoML4Vis & 2614.76 & 279.86 & 2894.62 & 3069.62 & 307.67 & 3377.29 \\
\system & 5122.99 & 777.63 & 5900.62 & 5613.96 & 1014.10 & 6628.06 \\
\bottomrule
\end{tabular}
\caption{Token usage comparison for different methods. N/A indicates that LIDA and Chat2Vis cannot handle multiple table scenarios.}
\label{tab:token_usage}
\end{table*}

\begin{table}[ht]
\centering
\scalebox{1}{
\begin{tabular}{l|ccc}
\toprule
Agent & \#Input & \#Output & \#Total \\
\midrule
Processor & 1486.07 & 569.58 & 1755.65\\
Composer & 3268.32 & 221.74 & 3490.07 \\
Validator & 1051.82 & 127.85 & 1179.67  \\
\bottomrule
\end{tabular}}
\caption{Token usage of three agents in \system.} \label{tab:token_agent} 
\vspace{-1em}
\end{table}

\paragraph{Implement Details.}
Our system is implemented in Python 3.9, utilizing GPT-4o \citep{openai_gpt4o_2024}, GPT-4o-mini~\cite{openai2024gpt4omini}, and GPT-3.5-turbo~\cite{chatgpt3.5} as the backbone model for all approaches, with the temperature set to 0 for consistent outputs. GPT-4o-mini serves as the vision language model for readability evaluation. We interact with these models through the Azure OpenAI API. The specific prompt templates for each agent, crucial for guiding their respective roles in the visualization generation process, are detailed in Appendix~\ref{prompt_details}. Token usages of \system and baselines are demonstrated in Table~\ref{tab:token_usage}, and usage for each agent in our \system is shown in Table~\ref{tab:token_agent}. Additionally, our evaluations are conducted in VisEval Benchmark (with MIT license).

\paragraph{Human Annotation.}
\label{human}
The annotation is conducted by 5 authors of this paper independently. As acknowledged, the diversity of annotators plays a crucial role in reducing bias and enhancing the reliability of the benchmark. These annotators have knowledge in the data visualization domain, with different genders, ages, and educational backgrounds. The educational backgrounds of annotators are above undergraduate. To ensure the annotators can proficiently mark the data, we provide them with detailed tutorials, teaching them how to judge the quality of data visualization. We also provide them with detailed criteria and task requirements in each annotation process shown in Figure~\ref{fig:annotation}. Two experiments requiring human annotation are detailed as follows:

\begin{figure}[!ht]
    \centering
    \includegraphics[width=\linewidth]{figure/score_distribution.pdf}
    \caption{Comparison of score density distribution between GPT-4o, GPT-4o-mini and human average score.}
    \label{fig:score_distribution}
\end{figure}

\begin{table*}[!ht]
\centering
\begin{tabular}{l|ccc}
\toprule
& Invalid Rate & Illegal Rate & Pass Rate \\
\midrule
\system & 4.66\% & 23.97\% & 71.35\% \\
w. CoT for Validator & 5.82\% & 23.39\% & 70.78\% \\
w. original schema for Validator & 4.80\% & 24.22\% & 70.97\% \\
\bottomrule
\end{tabular}
\caption{Additional exploration for Validator (using GPT-3.5-turbo).} 
\vspace{-1em} 
\label{tab:ablation_validator}
\end{table*}

\begin{itemize}[leftmargin=*, itemsep=0pt]
    \item \textbf{Pearson Correlation of Visual Language Model.} We conduct human annotation frameworks to compare the ability of the visual language model for MLLM-as-a-Judge~\cite{chen2024mllm}, providing the readability score. Our annotation framework is shown in Figure~\ref{fig:annotation}. The final Pearson scores are demonstrated in Table~\ref{tab:pearson_corr}, with its density distribution in Figure~\ref{fig:score_distribution}. The detailed instructions can be found in Figure~\ref{fig:scoring_instructions}.
    \item \textbf{Qualitative comparison to calculate ELO Scores.} We conduct human-judgments evaluations to compare which visualization generated by different models meets the query requirement more precisely. The leaderboard is shown in Table~\ref{tab:elo_rankings}, and Figure~\ref{fig:elo} shows the judgment framework. Each model starts with a base ELO score of 1500. After each pairwise comparison, the scores are updated based on the outcome and the current scores of the models involved. The hyperparameters are set as follows: the $K$-factor is set to 32, which determines the maximum change in rating after a single comparison. We conduct two sets of evaluations: one for single-table queries and another for multiple-table queries, with 1000 bootstrap iterations for each set to ensure statistical robustness. For each model's ELO rating, we report the 95\% confidence intervals computed through bootstrap resampling, providing a measure of rating stability. The evaluation process involves presenting human judges with a query and two visualizations, asking them to select the one that better meets the query requirements. This process is repeated across all model pairs and queries in our test set. The detailed guidance provides to the human evaluators can be found in Figure~\ref{fig:evaluation_instructions}, which outlines the criteria for judging visualization quality and relevance to the given query.


\end{itemize}

\begin{figure}[!ht]
	\centering
    \setlength{\belowcaptionskip}{-1em}
	\includegraphics[width=0.98\linewidth,scale=1.0]
    {./figure/library.pdf}
    \vspace{-1em}
	\caption{Performance of different models using \texttt{Matplotlib} and \texttt{Seaborn} libraries, using GPT-3.5-turbo.
    % \yao{larger fontsize?}
    }
\label{fig: library}
\end{figure}

\begin{figure*}[!h]
    \centering
    \includegraphics[width=0.98\linewidth]{figure/annotation.pdf}
    \caption{Screenshot of human annotation process in readability score.}
    \label{fig:annotation}
\end{figure*}

\begin{figure*}[ht]
\centering
\vspace{1em}
\begin{tcolorbox}[enhanced,attach boxed title to top center={yshift=-3mm,yshifttext=-1mm},boxrule=0.9pt, 
  colback=gray!00,colframe=black!50,colbacktitle=gray,
  title=Readability Scoring Instruction,
  boxed title style={size=small,colframe=gray} ]
\small
\textbf{Scoring Instructions:} Please evaluate the charts based on the following criteria, with a score range from 1 to 5, where 1 indicates very poor quality and 5 indicates excellent quality. You should focus on the following aspects:

\vspace{0.5em}
\textbf{1. Chart Colors:}
\begin{itemize}
    \item Are the colors clear and natural, effectively conveying the information?
    \item Color blindness accessibility: Are the color combinations easy to distinguish, especially for users with color blindness?
\end{itemize}

\vspace{0.5em}
\textbf{2. Title and Axis Labels:}
\begin{itemize}
    \item Ensure the chart has a clear title.
    \item Do the X-axis and Y-axis labels exist, and are they complete?
    \item Check if the labels are difficult to read, e.g., are they written vertically instead of horizontally?
    \item The title should not be a direct question; instead, it should describe the data or trends being presented.
\end{itemize}

\vspace{0.5em}
\textbf{3. Legend Completeness:}
\begin{itemize}
    \item Is the legend complete, and does it clearly indicate the color labels for different data series?
    \item Ensure each color has a corresponding legend, making it easy for users to understand what the data represents.
\end{itemize}

\vspace{0.5em}
\textbf{Scoring Scale:}
\begin{itemize}
    \item \textbf{1 Point:} Very poor, unable to understand or severely lacking information.
    \item \textbf{2 Points:} Poor quality, multiple issues present, difficult to extract information.
    \item \textbf{3 Points:} Fair, conveys some information but still has room for improvement.
    \item \textbf{4 Points:} Good, generally clear charts with minor areas for improvement.
    \item \textbf{5 Points:} Excellent, outstanding chart design with clear and effective information presentation.
\end{itemize}

Please consider the above factors when assessing the charts and provide the appropriate score. Thank you for your cooperation and effort!
\end{tcolorbox}
\vspace{-7pt}
\caption{Instructions for human annorators in annotating readability scoring.}
\label{fig:scoring_instructions}
\vspace{1em}
\end{figure*}

\begin{figure*}[!ht]
    \centering
    \includegraphics[width=0.98\linewidth]{figure/elo.pdf}
    \caption{Screenshot of ELO score evaluation framework for Human-as-a-Judge.}
    \label{fig:elo}
\end{figure*}

\begin{figure*}[ht]
\centering
\vspace{1em}
\begin{tcolorbox}[enhanced,attach boxed title to top center={yshift=-3mm,yshifttext=-1mm},boxrule=0.9pt, 
  colback=gray!00,colframe=black!50,colbacktitle=gray,
  title=Visualization Comparison Guidance,
  boxed title style={size=small,colframe=gray} ]
\small
Welcome to the visualization comparison evaluation. Your task is to judge which model-generated visualization better meets the requirements of the natural language query.

\vspace{0.5em}
\textbf{Evaluation criteria:}
\begin{enumerate}
    \item \textbf{Appropriateness of chart type:} Check if the selected chart type is suitable for expressing the data and relationships required by the query.
    \item \textbf{Data completeness:} Ensure the chart includes all necessary data required by the query.
    \item \textbf{Readability:} Assess the clarity of the chart, accuracy of labels, and overall layout.
    \item \textbf{Aesthetics:} Consider if the chart's color scheme, proportions, and overall design are visually pleasing.
    \item \textbf{Information conveyance:} Judge if the chart effectively conveys the main information or insights required by the query.
\end{enumerate}

\vspace{0.5em}
\textbf{Evaluation process:}
\begin{enumerate}
    \item Carefully read the natural language query.
    \item Observe the visualization results generated by two models.
    \item Based on the above criteria, choose the better visualization or select a tie if they are equally good.
    \item If neither visualization satisfies the query requirements well, please choose the relatively better one.
\end{enumerate}

Remember, your evaluation will help us improve and compare different visualization models. Thank you for your participation!
\end{tcolorbox}
\vspace{-7pt}
\caption{Instructions for human annorators in visualization comparison.}
\label{fig:evaluation_instructions}
\vspace{1em}
\end{figure*}


\section{Additional Experiment Results}
\label{additional_experiment_result}

We also conducted a comparison experiment of different methods using matplotlib or seaborn library. Figure~\ref{fig: library} demonstrates the results, indicating that our method outperforms obviously other baselines not only with matplotlib but also seaborn.

In addition, we test techniques in the Validator Agent, such as Chain-of-Thought. As is shown in Table~\ref{tab:ablation_validator}, integrating Chain-of-Thought reasoning, may affect its performance badly, likely due to the simple refining task with complex reasoning. Moreover, using the original schema to check for false schema filtering seems to be useless in this case.

\section{Evaluation Results with Detailed Metrics}
We demonstrated the main results in Table~\ref{tab:performance_comparison}, and here we reported more detailed results of other metrics in Table~\ref{tab:detailed_results}, which underscored the error rates for each stage, including \textit{Invalid}, \textit{Illegal}, and \textit{Low Readability}. 

\begin{table*}[!ht]
\centering
\footnotesize
\scalebox{0.98}{
\begin{tabular}{ll|cc|cccc|cc}
\toprule[1.5pt]
\multirow{2}{*}{Method} & \multirow{2}{*}{Dataset} & \multicolumn{2}{c|}{Invalid} & \multicolumn{4}{c|}{Illegal} & \multicolumn{2}{c}{Low Readability} \\
&  & Execution & Surface. & Decon. & Chart Type & Data & Order & Layout & Scale\&Ticks \\
\midrule
\multicolumn{10}{c}{ \textbf{\textit{GPT-4o}}}\\
\midrule
\multirow{3}{*}{CoML4Vis} & All & 1.15 & 0.00 & 0.26 & 1.75 & 14.28 & 10.36 & 32.02 & 32.55 \\
& Single & 0.67 & 0.00 & 0.43 & 1.93 & 13.54 & 10.16 & 31.08 & 32.76 \\
& Multiple & 1.87 & 0.00 & 0.00 & 1.48 & 15.39 & 10.66 & 33.43 & 32.23 \\
\cmidrule{2-10}
\multirow{3}{*}{LIDA} & All & 6.61 & 0.00 & 1.60 & 3.24 & 40.53 & 4.07 & 32.68 & 15.77 \\
& Single & 1.13 & 0.00 & 2.11 & 0.89 & 12.26 & 6.79 & 53.93 & 26.22 \\
& Multiple & 14.80 & 0.00 & 0.79 & 8.51 & 80.53 & 0.00 & 1.24 & 0.21 \\
\cmidrule{2-10}
\multirow{3}{*}{Chat2Vis} & All & 16.05 & 0.00 & 0.62 & 3.99 & 30.14 & 5.96 & 2.37 & 20.88 \\
& Single & 0.86 & 0.00 & 0.75 & 2.30 & 10.78 & 9.73 & 3.97 & 34.63 \\
& Multiple & 38.74 & 0.00 & 0.43 & 6.51 & 59.08 & 0.32 & 0.00 & 0.34 \\
\cmidrule{2-10}
\multirow{3}{*}{nvAgent} & All & 0.97 & 0.00 & 0.08 & 1.28 & 11.07 & 4.05 & 5.07 & 40.03 \\
& Single & 0.72 & 0.00 & 0.14 & 1.27 & 9.88 & 3.60 & 3.92 & 39.36 \\
& Multiple & 1.34 & 0.00 & 0.00 & 1.30 & 12.84 & 4.73 & 6.79 & 41.03 \\
\midrule
\multicolumn{10}{c}{ \textbf{\textit{GPT-4o-mini}}}\\
\midrule
\multirow{3}{*}{CoML4Vis} & All & 4.23 & 0.00 & 0.20 & 2.31 & 16.64 & 11.83 & 35.23 & 29.35 \\
& Single & 0.36 & 0.00 & 0.26 & 2.32 & 13.80 & 11.67 & 35.92 & 32.22 \\
& Multiple & 10.01 & 0.00 & 0.10 & 2.31 & 20.87 & 12.07 & 34.19 & 25.05 \\
\cmidrule{2-10}
\multirow{3}{*}{LIDA} & All & 12.50 & 0.00 & 0.40 & 4.92 & 40.02 & 5.80 & 27.87 & 17.05 \\
& Single & 9.09 & 0.00 & 0.44 & 2.53 & 12.91 & 9.68 & 45.69 & 28.32 \\
& Multiple & 17.61 & 0.00 & 0.33 & 8.51 & 80.53 & 0.00 & 1.24 & 0.21 \\
\cmidrule{2-10}
\multirow{3}{*}{Chat2Vis} & All & 15.45 & 0.17 & 0.17 & 4.21 & 31.90 & 8.20 & 2.14 & 18.97 \\
& Single & 2.14 & 0.29 & 0.41 & 2.53 & 11.99 & 9.68 & 45.69 & 28.32 \\
& Multiple & 35.78 & 0.00 & 0.00 & 6.70 & 61.66 & 0.00 & 0.92 & 0.32 \\
\cmidrule{2-10}
\multirow{3}{*}{nvAgent} & All & 5.14 & 0.00 & 0.00 & 2.40 & 16.33 & 10.61 & 41.06 & 27.00 \\
& Single & 1.97 & 0.00 & 0.14 & 2.97 & 15.21 & 7.49 & 39.30 & 32.39 \\
& Multiple & 8.15 & 0.00 & 0.00 & 2.31 & 20.87 & 12.07 & 34.19 & 25.05 \\
\midrule
\multicolumn{10}{c}{ \textbf{\textit{GPT-3.5-turbo}}}\\
\midrule
\multirow{3}{*}{CoML4Vis} & All & 9.28 & 0.00 & 0.62 & 1.91 & 15.83 & 12.86 & 25.09 & 27.73 \\ 
& Single & 6.17 & 0.00 & 0.89 & 2.50 & 14.71 & 13.20 & 26.10 & 29.93 \\ 
& Multiple & 13.92 & 0.00 & 0.21 & 1.04 & 17.51 & 12.36 & 23.57 & 24.43 \\ 
\cmidrule{2-10} 
\multirow{3}{*}{LIDA} & All & 53.43 & 0.00 & 1.27 & 3.56 & 22.33 & 0.53 & 14.90 & 6.62 \\ 
& Single & 47.32 & 0.00 & 1.91 & 2.81 & 13.03 & 0.89 & 24.43 & 11.05 \\ 
& Multiple & 62.57 & 0.00 & 0.32 & 4.68 & 36.23 & 0.00 & 0.65 & 0.00 \\ 
\cmidrule{2-10} 
\multirow{3}{*}{Chat2Vis} & All & 18.68 & 0.00 & 0.28 & 3.66 & 32.47 & 7.20 & 25.45 & 20.15 \\ 
& Single & 3.90 & 0.00 & 0.47 & 2.78 & 15.62 & 12.01 & 41.74 & 33.38 \\ 
& Multiple & 40.77 & 0.00 & 0.00 & 4.97 & 57.66 & 0.00 & 1.12 & 0.37 \\ 
\cmidrule{2-10} 
\multirow{3}{*}{nvAgent} & All & 4.66 & 0.00 & 0.08 & 3.06 & 18.24 & 5.64 & 5.25 & 35.34 \\ 
& Single & 2.98 & 0.00 & 0.14 & 2.84 & 15.08 & 5.69 & 3.62 & 37.57 \\ 
& Multiple & 7.18 & 0.00 & 0.00 & 3.38 & 22.95 & 5.56 & 7.69 & 32.02 \\
\bottomrule[1.5pt]
\end{tabular}
}
\caption{Detailed error rates (\%) for different methods.} 
\label{tab:detailed_results}
\end{table*}

\section{Case Study}
\label{example}
% To demonstrate our approach's effectiveness, we present several illustrative examples. Figure~\ref{fig:nl_vql} shows how our system translates natural language into a structured VQL representation. Figure~\ref{python code} and Figure~\ref{fig:example_chart} demonstrate the complete pipeline from query to visualization.
Figure~\ref{fig:nl_vql} shows an example of a natural language query with its corresponding VQL representation. The output Python code for visualization and the final bar chart are demonstrated in Figure~\ref{python code} and Figure~\ref{fig:example_chart}, respectively.
Furthermore, we provide a case study of \system performance on four hardness-level NL2Vis problems in VisEval in Figure \ref{hardness case}.

The easy case demonstrates accurate grouping in scatter plot relationships. The medium case shows correct handling of multi-table joins for continent-wise statistics. The hard case exhibits temporal data visualization with proper filtering. The extra hard case showcases complex operations including weekday binning and stacked visualization. These cases highlight our system's consistent performance across varying task complexities, particularly excelling in multiple table scenarios and complex aggregations.

\begin{figure*}[htbp]
\centering
\begin{tcolorbox}[enhanced,attach boxed title to top center={yshift=-3mm,yshifttext=-1mm},boxrule=0.9pt, 
  colback=gray!00,colframe=black!50,colbacktitle=gray,
  title=An Example of Natural Language Query and  Corresponding VQL,,
  boxed title style={size=small,colframe=gray} ]

\textbf{Natural Language Query:}\\
How many documents are stored? Bin the store date by weekday in a bar chart.\\
\tcbline
\textbf{Corresponding VQL:}\\
Visualize BAR \\
SELECT Date\_Stored, COUNT(Document\_ID)\\
FROM All\_Documents \\
GROUP BY Date\_Stored \\
BIN Date\_Stored BY WEEKDAY\\
\end{tcolorbox}
\caption{The natural language query case and its corresponding output VQL representation.}
\label{fig:nl_vql}
\end{figure*}

\lstset{
    basicstyle=\ttfamily\small,
    breaklines=true,
    numbers=left,
    numberstyle=\tiny,
    frame=single,
    showstringspaces=false,
    tabsize=4,
    keywordstyle=\color{blue},
    commentstyle=\color{green!60!black},
    stringstyle=\color{purple},
    breakatwhitespace=false,
    keepspaces=true
}
\begin{figure*}[htbp]
\begin{lstlisting}[
    language=Python,
    frame=single,
    numbers=left,
    numberstyle=\tiny,
    basicstyle=\tiny\ttfamily,    
    breaklines=true,
    breakatwhitespace=false,
    showstringspaces=false,
    framesep=1mm,                 % 减小框架边距
    framexleftmargin=1mm,         % 减小框架左边距
    columns=flexible,             % 灵活列宽
    keepspaces=true
]
import matplotlib.pyplot as plt 
import pandas as pd 
import os 
import duckdb

# Set data folder path
data_folder = 'E:/visEval_dataset/databases/cre_Doc_Tracking_DB'  

# Connect to database 
con = duckdb.connect(database=':memory:')  

# Read all CSV files and create view
csv_files = [f for f in os.listdir(data_folder) if f.endswith('.csv')] 
for file in csv_files:
    table_name = os.path.splitext(file)[0]
    con.execute(f"CREATE VIEW {table_name} AS SELECT * FROM read_csv_auto('{os.path.join(data_folder, file)}')")  
    
# Execute SQL query 
sql = f''' 
    SELECT Date_Stored, COUNT(Document_ID) AS count_Document_ID 
    FROM All_Documents 
    GROUP BY Date_Stored 
''' 
df = con.execute(sql).fetchdf() 
con.close()  

# Rename columns 
df.columns = ['Date_Stored','count_Document_ID'] 

# Apply binning operation
flag = True 
df['Date_Stored'] = pd.to_datetime(df['Date_Stored']) 
df['Date_Stored'] = df['Date_Stored'].dt.day_name()  

# Group by and calculate count 
if flag:
    df = df.groupby('Date_Stored').sum().reset_index() 

# Ensure all seven days of the week are included 
weekday_order = ['Monday', 'Tuesday', 'Wednesday', 'Thursday', 
                 'Friday', 'Saturday', 'Sunday'] 
df = df.set_index('Date_Stored').reindex(weekday_order, fill_value=0).reset_index()
df['Date_Stored'] = pd.Categorical(df['Date_Stored'], 
                                  categories=weekday_order, ordered=True) 
df = df.sort_values('Date_Stored')

# Create visualization 
fig, ax = plt.subplots(1, 1, figsize=(10, 4)) 
ax.spines['top'].set_visible(False) 
ax.spines['right'].set_visible(False) 
ax.bar(df['Date_Stored'], df['count_Document_ID']) 
ax.set_xlabel('Date_Stored') 
ax.set_ylabel('count_Document_ID') 
ax.set_title(f'BAR Chart of count_Document_ID by Date_Stored') 
plt.xticks(rotation=45) 
plt.tight_layout()  
plt.show()
\end{lstlisting}
\caption{An example of python code generating module within \system.}
\label{python code}
\end{figure*}


\begin{figure*}[!ht]
    \centering
    \includegraphics[width=0.98\linewidth,scale=1.0]{figure/bar_chart.pdf}
    \caption{An example of generated bar chart using \system.}
    \label{fig:example_chart}
\end{figure*}

\begin{figure*}[htbp]
\centering
\begin{tcolorbox}[enhanced,attach boxed title to top center={yshift=-3mm,yshifttext=-1mm},boxrule=0.9pt, 
  colback=gray!00,colframe=black!50,colbacktitle=gray,
  title=Examples of \textsc{nvAgent} performance on different hardness levels,
  boxed title style={size=small,colframe=gray} ]
  
\textbf{Hardness Level:} Easy \\
\begin{minipage}{0.45\linewidth}
    \textbf{Dataset}: \textit{Single}\\
    \textbf{Input Tables}: basketball\_match\\
    \textbf{Input Query}: Show the relation between acc percent and all\_games\_percent for each ACC\_Home using a grouped scatter chart.\\
\end{minipage}\hfill
\begin{minipage}{0.45\linewidth}
    \centering
    \textbf{Response}:
    \includegraphics[width=\linewidth]{figure/easy_3085.pdf} 
\end{minipage}
\tcbline

\textbf{Hardness Level:} Medium \\
\begin{minipage}{0.45\linewidth}
    \textbf{Dataset}: \textit{Multiple}\\
    \textbf{Input Tables}: car\_makers, car\_names, cars\_data, continents, countries, model\_list\\
    \textbf{Input Query}: Display a pie chart for what is the name of each continent and how many car makers are there in each one?\\
\end{minipage}\hfill
\begin{minipage}{0.55\linewidth}
    \centering
    \textbf{Response}:
    \includegraphics[width=\linewidth]{figure/medium_433.pdf} 
\end{minipage}
\tcbline

\textbf{Hardness Level:} Hard \\[1em]
\begin{minipage}{0.45\linewidth}
    \textbf{Dataset}: \textit{Multiple}\\
    \textbf{Input Tables}: advisor, classroom, course, department, instructor, prereq, section, student, takes, teaches, time\_slot\\
    \textbf{Input Query}: Find the number of courses offered by Psychology department in each year with a line chart.\\
\end{minipage}\hfill
\begin{minipage}{0.45\linewidth}
    \centering
    \textbf{Response}:
    \includegraphics[width=\linewidth]{figure/hard_611.pdf} 
\end{minipage}
\tcbline

\textbf{Hardness Level:} Extra Hard \\[1em]
\begin{minipage}{0.45\linewidth}
    \textbf{Dataset}: \textit{Multiple}\\
    \textbf{Input Tables}: Accounts, Documents, Documents\_with\_Expenses, Projects, Ref- \_Budget\_Codes, Ref\_Document\_Types, Statements\\
    \textbf{Input Query}: How many documents are created in each day? Bin the document date by weekday and group by document type description with a stacked bar chart, I want to sort Y in desc order.\\
\end{minipage}\hfill
\begin{minipage}{0.45\linewidth}
    \centering
    \textbf{Response}:
    \includegraphics[width=\linewidth]{figure/extra_851.pdf} 
\end{minipage}

\end{tcolorbox}
    \caption{Examples of \textsc{nvAgent}'s performance on different hardness levels in VisEval (easy, medium, hard, and extra hard.}
    \label{hardness case}
\end{figure*}


\clearpage
\onecolumn
\section{Prompts Details}
\label{prompt_details}
We provide detailed prompt design of our \system as follows.



\begin{promptbox}[Prompt template for Processor Agent]
You are an experienced and professional database administrator. Given a database schema and a user query, your task is to analyze the query, filter the relevant schema, generate an optimized representation, and classify the query difficulty. \\
\\
Now you can think step by step, following these instructions below. \\
\textbf{[Instructions]} \\
1. Schema Filtering: \\
\text{\ \ \ \ }- Identify the tables and columns that are relevant to the user query.\\
\text{\ \ \ \ }- Only exclude columns that are completely irrelevant.\\
\text{\ \ \ \ }- The output should be \{\{tables: [columns]\}\}.\\
\text{\ \ \ \ }- Keep the columns needed to be primary keys and foreign keys in the filtered schema.\\
\text{\ \ \ \ }- Keep the columns that seem to be similar with other columns of another table.\\
\\
2. New Schema Generation:\\
\text{\ \ \ \ }- Generate a new schema of the filtered schema, based on the given database schema and your filtered schema.\\
\\
3. Augmented Explanation:\\
\text{\ \ \ \ }- Provide a concise summary of the filtered schema to give additional knowledge.\\
\text{\ \ \ \ }- Include the number of tables, total columns, and any notable relationships or patterns.\\
\\
4. Classification:\\
For the database new schema, classify it as SINGLE or MULTIPLE based on the tables number.\\
\text{\ \ \ \ }- if tables number >= 2: predict MULTIPLE\\
\text{\ \ \ \ }- elif only one table: predict SINGLE\\
\\
==============================\\
Here is a typical example:\\
\textbf{[Database Schema]}\\
\textbf{[DB\_ID]} dorm\_1\\
\textbf{[Schema]}\\
\# Table: Student\\
\text{[}\\
  \text{\ \ \ \ }(stuid, And This is a id type column),\\
  \text{\ \ \ \ }(lname, Value examples: [`Smith', `Pang', `Lee', `Adams', `Nelson', `Wilson'].),\\
  \text{\ \ \ \ }(fname, Value examples: [`Eric', `Lisa', `David', `Sarah', `Paul', `Michael'].),\\
  \text{\ \ \ \ }(age, Value examples: [18, 20, 17, 19, 21, 22].),\\
  \text{\ \ \ \ }(sex, Value examples: [`M', `F'].),\\
  \text{\ \ \ \ }(major, Value examples: [600, 520, 550, 50, 540, 100].),\\
  \text{\ \ \ \ }(advisor, And this is a number type column),\\
  \text{\ \ \ \ }(city code, Value examples: [`PIT', `BAL', `NYC', `WAS', `HKG', `PHL'].)\\
\text{]}\\
% \end{promptbox}
% \end{figure*}
% \begin{figure*}[!h]
% \begin{promptbox}[Prompt template for Processor Agent]
\# Table: Dorm\\
\text{[}\\
  \text{\ \ \ \ }(dormid, And This is a id type column),\\
  \text{\ \ \ \ }(dorm name, Value examples: [`Anonymous Donor Hall', `Bud Jones Hall', `Dorm-plex 2000', `Fawlty Towers', `Grad Student Asylum', `Smith Hall'].),\\
  \text{\ \ \ \ }(student capacity, Value examples: [40, 85, 116, 128, 256, 355].),
  (gender, Value examples: [`X', `F', `M'].)\\
\text{]}\\
\# Table: Dorm\_amenity\\
\text{[}\\
  \text{\ \ \ \ }(amenid, And This is a id type column),\\
  \text{\ \ \ \ }(amenity name, Value examples: [`4 Walls', `Air Conditioning', `Allows Pets', `Carpeted Rooms', `Ethernet Ports', `Heat'].)\\
\text{]}\\
\# Table: Has\_amenity\\
\text{[}\\
  \text{\ \ \ \ }(dormid, And This is a id type column),\\
  \text{\ \ \ \ }(amenid, And This is a id type column)\\
\text{]}\\
\# Table: Lives\_in\\
\text{[}\\
  \text{\ \ \ \ }(stuid, And This is a id type column),\\
  \text{\ \ \ \ }(dormid, And This is a id type column),\\
  \text{\ \ \ \ }(room number, And this is a number type column)\\
\text{]}\\
\\
\textbf{[Query]}\\
Find the first name of students who are living in the Smith Hall, and count them by a pie chart\\
\\
Now we can think step by step\\
\textbf{[Filtered Schema]}\\
\{\\
  "Student": ["stuid", "fname"],\\
  "Dorm": ["dormid", "dorm name"],\\
  "Lives\_in": ["stuid", "dormid"]\\
\}\\
\\
\textbf{[New Schema]}\\
\# Table: Student\\
\text{[}\\
  (stuid, And This is a id type column),\\
  (fname, Value examples: [`Eric', `Lisa', `David', `Sarah', `Paul', `Michael'].),\\
\text{]}\\
% \end{promptbox}
% \end{figure*}
% \begin{figure*}[!h]
% \begin{promptbox}[Prompt template for Processor Agent]
\# Table: Dorm
\text{[}\\
  (dormid, And This is a id type column),\\
  (dorm name, Value examples: [`Anonymous Donor Hall', `Bud Jones Hall', `Dorm-plex 2000', `Fawlty Towers', `Grad Student Asylum', `Smith Hall'].),\\
\text{]}\\
\# Table: Lives\_in\\
\text{[}\\
  (stuid, And This is a id type column),\\
  (dormid, And This is a id type column),\\
\text{]}\\
\textbf{[Augmented Explanation]}\\
The filtered schema consists of 3 tables (Student, Dorm, and Lives\_in) with a total of 6 relevant columns. There is a many-to-one relationship between Student and Dorm through the Lives\_in junction table. The query involves joining these three tables to find students living in a specific dorm (Smith Hall).\\
\\
Key points:\\
1. The Lives\_in table acts as a bridge between Student and Dorm, allowing for the association of students with their dorms.\\
2. The `dorm name' column in the Dorm table is crucial for filtering the specific dorm (Smith Hall).\\
3. The `fname' column from the Student table is required for the final output.\\
\\
\textbf{[Classification]}\\
MULTIPLE\\
\\
==============================\\
Here is a new question:\\
\\
\textbf{[DB\_ID]} \{db\_id\}\\
\textbf{[Database Schema]}\\
\{db\_schema\}\\
\\
\textbf{[Query]}\\
\{query\}\\
\\
Now give your answer following this format strictly without other explanation:\\
\\
\textbf{[Filtered Schema]}\\
\\
\textbf{[New Schema]}\\
\\
\textbf{[Augmented Explanation]}\\
\\
\textbf{[Classification]}\\
\\
\end{promptbox}
% \end{figure*}

% \subsection{Composer Agent Prompt}
% \label{composer_prompt}
% \begin{figure*}[!h]
\begin{promptbox}[Prompt template for multiple classification]
Given a [Database schema] with [Augmented Explanation] and a [Question], generate a valid VQL (Visualization Query Language) sentence. VQL is similar to SQL but includes visualization components. \\
\\
Now you can think step by step, following these instructions below. \\
\textbf{[Background]} \\
VQL Structure:\\
Visualize [TYPE] SELECT [COLUMNS] FROM [TABLES] [JOIN] [WHERE] [GROUP BY] [ORDER BY] [BIN BY]\\
\\
You can consider a VQL sentence as "VIS TYPE + SQL + BINNING"\\
You must consider which part in the sketch is necessary, which is unnecessary, and construct a specific sketch for the natural language query.\\
\\
Key Components:\\
1. Visualization Type: bar, pie, line, scatter, stacked bar, grouped line, grouped scatter\\
2. SQL Components: SELECT, FROM, JOIN, WHERE, GROUP BY, ORDER BY\\
3. Binning: BIN [COLUMN] BY [INTERVAL], [INTERVAL]: [YEAR, MONTH, DAY, WEEKDAY]\\
\\
When generating VQL, we should always consider special rules and constraints:\\
\textbf{[Special Rules]} \\
a. For simple visualizations:\\
    \text{\ \ \ \ }- SELECT exactly TWO columns, X-axis and Y-axis(usually aggregate function)\\
b. For complex visualizations (STACKED BAR, GROUPED LINE, GROUPED SCATTER):\\
    \text{\ \ \ \ }- SELECT exactly THREE columns in this order!!!:\\
        \text{\ \ \ \ }\text{\ \ \ \ }1. X-axis\\
        \text{\ \ \ \ }\text{\ \ \ \ }2. Y-axis (aggregate function)\\
        \text{\ \ \ \ }\text{\ \ \ \ }3. Grouping column\\
c. When "COLORED BY" is mentioned in the question:\\
    \text{\ \ \ \ }- Use complex visualization type(STACKED BAR for bar charts, GROUPED LINE for line charts, GROUPED SCATTER for scatter charts)\\
    \text{\ \ \ \ }- Make the "COLORED BY" column the third SELECT column\\
    \text{\ \ \ \ }- Do NOT include "COLORED BY" in the final VQL\\     
d. Aggregate Functions:\\
    \text{\ \ \ \ }- Use COUNT for counting occurrences\\
    \text{\ \ \ \ }- Use SUM only for numeric columns\\
    \text{\ \ \ \ }- When in doubt, prefer COUNT over SUM\\
e. Time based questions:\\
    \text{\ \ \ \ }- Always use BIN BY clause at the end of VQL sentence\\
    \text{\ \ \ \ }- When you meet the questions including "year", "month", "day", "weekday"\\
    \text{\ \ \ \ }- Avoid using window function, just use BIN BY to deal with time base queries\\
% \end{promptbox}
% \end{figure*}
% \begin{figure*}[!h]
% \begin{promptbox}[Prompt template for multiple classification]
\textbf{[Constraints]} \\
- In SELECT <column>, make sure there are at least two selected!!!\\
- In FROM <table> or JOIN <table>, do not include unnecessary table\\
- Use only table names and column names from the given database schema\\
- Enclose string literals in single quotes\\
- If [Value examples] of <column> has `None' or None, use JOIN <table> or WHERE <column> is NOT NULL is better\\
- Ensure GROUP BY precedes ORDER BY for distinct values\\
- NEVER use window functions in SQL\\
\\
Now we could think step by step:\\
1. First choose visualize type and binning, then construct a specific sketch for the natural language query\\
2. Second generate SQL components following the sketch.\\
3. Third add Visualize type and BINNING into the SQL components to generate final VQL\\
\\
==============================\\
Here is a typical example:\\
\textbf{[Database Schema]}\\
\# Table: Orders, (orders)\\
\text{[}\\
  \text{\ \ \ \ }(order\_id, order id, And this is a id type column),\\
  \text{\ \ \ \ }(customer\_id, customer id, And this is a id type column),\\
  \text{\ \ \ \ }(order\_date, order date, Value examples: [`2023-01-15', `2023-02-20', `2023-03-10'].),\\
  \text{\ \ \ \ }(total\_amount, total amount, Value examples: [100.00, 200.00, 300.00, 400.00, 500.00].)\\
\text{]}\\
\# Table: Customers, (customers)\\
\text{[}\\
  \text{\ \ \ \ }(customer\_id, customer id, And this is a id type column),\\
  \text{\ \ \ \ }(customer\_name, customer name, Value examples: [`John', `Emma', `Michael', `Sophia', `William'].),\\
  \text{\ \ \ \ }(customer\_type, customer type, Value examples: [`Regular', `VIP', `New'].)\\
\text{]}\\
\textbf{[Augmented Explanation]}\\
The filtered schema consists of 2 tables (Orders and Customers) with a total of 7 relevant columns. There is a one-to-many relationship between Customers and Orders through the customer\_id foreign key.\\
\\
Key points:\\
1. The Orders table contains information about individual orders, including the order date and total amount.\\
2. The Customers table contains customer information, including their name and type (Regular, VIP, or New).\\
3. The customer\_id column links the two tables, allowing us to associate orders with specific customers.\\
% \end{promptbox}
% \end{figure*}
% \begin{figure*}[!h]
% \begin{promptbox}[Prompt template for multiple classification]
4. The order\_date column in the Orders table will be used for monthly grouping and binning.\\
5. The total\_amount column in the Orders table needs to be summed for each group.\\
6. The customer\_type column in the Customers table will be used for further grouping and as the third dimension in the stacked bar chart.\\
\\

The query involves joining these two tables to analyze order amounts by customer type and month, which requires aggregation and time-based binning.\\
\\
\textbf{[Question]}\\
Show the total order amount for each customer type by month in a stacked bar chart.\\
\\
Decompose the task into sub tasks, considering [Background] [Special Rules] [Constraints], and generate the VQL after thinking step by step:\\
\\
\textbf{Sub task 1:} First choose visualize type and binning, then construct a specific sketch for the natural language query\\
Visualize type: STACKED BAR, BINNING: True\\
VQL Sketch:\\
Visualize STACKED BAR SELECT \_ , \_ , \_ FROM \_ JOIN \_ ON \_ GROUP BY \_ BIN \_ BY MONTH\\
\\
\textbf{Sub task 2:} Second generate SQL components following the sketch.\\
Let's think step by step:\\
1. We need to select 3 columns for STACKED BAR chart, order\_date as X-axis, SUM(total\_amout) as Y-axis, customer\_type as group column.\\
2. We need to join the Orders and Customers tables.\\
3. We need to group by customer type.\\
4. We do not need to use any window function for MONTH.\\
\\
\text{sql}\\
```sql\\
SELECT O.order\_date, SUM(O.total\_amount), C.customer\_type\\
FROM Orders AS O\\
JOIN Customers AS C ON O.customer\_id = C.customer\_id\\
GROUP BY C.customer\_type\\
```\\
\\
\textbf{Sub task 3:} Third add Visualize type and BINNING into the SQL components to generate final VQL\\
\textbf{Final VQL:}\\
Visualize STACKED BAR SELECT O.order\_date, SUM(O.total\_amount), C.customer\_type FROM Orders O JOIN Customers C ON O.customer\_id = C.customer\_id GROUP BY C.customer\_type BIN O.order\_date BY MONTH\\
\\
% \end{promptbox}
% \end{figure*}
% \begin{figure*}[!h]
% \begin{promptbox}[Prompt template for multiple classification]
==============================\\
Here is a new question:\\
\\
\textbf{[Database Schema]}\\
\{desc\_str\}\\
\\
\textbf{[Augmented Explanation]}\\
\{augmented\_explanation\}\\
\\
\textbf{[Query]}\\
\{query\}\\
\\
Now, please generate a VQL sentence for the database schema and question after thinking step by step.\\

\end{promptbox}
% \end{figure*}


% \begin{figure*}[!h]
\begin{promptbox}[Prompt template for single classification]
Given a [Database schema] with [Augmented Explanation] and a [Question], generate a valid VQL (Visualization Query Language) sentence. VQL is similar to SQL but includes visualization components. \\
\\
Now you can think step by step, following these instructions below. \\
\textbf{[Background]} \\
VQL Structure:\\
Visualize [TYPE] SELECT [COLUMNS] FROM [TABLES] [JOIN] [WHERE] [GROUP BY] [ORDER BY] [BIN BY]\\
\\
You can consider a VQL sentence as "VIS TYPE + SQL + BINNING"\\
You must consider which part in the sketch is necessary, which is unnecessary, and construct a specific sketch for the natural language query.\\
\\
Key Components:\\
1. Visualization Type: bar, pie, line, scatter, stacked bar, grouped line, grouped scatter\\
2. SQL Components: SELECT, FROM, JOIN, WHERE, GROUP BY, ORDER BY\\
3. Binning: BIN [COLUMN] BY [INTERVAL], [INTERVAL]: [YEAR, MONTH, DAY, WEEKDAY]\\
\\
When generating VQL, we should always consider special rules and constraints:\\
\textbf{[Special Rules]} \\
a. For simple visualizations:\\
    \text{\ \ \ \ }- SELECT exactly TWO columns, X-axis and Y-axis(usually aggregate function)\\
b. For complex visualizations (STACKED BAR, GROUPED LINE, GROUPED SCATTER):\\
    \text{\ \ \ \ }- SELECT exactly THREE columns in this order!!!:\\
        \text{\ \ \ \ }\text{\ \ \ \ }1. X-axis\\
        \text{\ \ \ \ }\text{\ \ \ \ }2. Y-axis (aggregate function)\\
        \text{\ \ \ \ }\text{\ \ \ \ }3. Grouping column\\
c. When "COLORED BY" is mentioned in the question:\\
    \text{\ \ \ \ }- Use complex visualization type(STACKED BAR for bar charts, GROUPED LINE for line charts, GROUPED SCATTER for scatter charts)\\
    \text{\ \ \ \ }- Make the "COLORED BY" column the third SELECT column\\
    \text{\ \ \ \ }- Do NOT include "COLORED BY" in the final VQL\\     
d. Aggregate Functions:\\
    \text{\ \ \ \ }- Use COUNT for counting occurrences\\
    \text{\ \ \ \ }- Use SUM only for numeric columns\\
    \text{\ \ \ \ }- When in doubt, prefer COUNT over SUM\\
e. Time based questions:\\
    \text{\ \ \ \ }- Always use BIN BY clause at the end of VQL sentence\\
    \text{\ \ \ \ }- When you meet the questions including "year", "month", "day", "weekday"\\
    \text{\ \ \ \ }- Avoid using window function, just use BIN BY to deal with time base queries\\
% \end{promptbox}
% \end{figure*}
% \begin{figure*}[!h]
% \begin{promptbox}[Prompt template for single classification]
\textbf{[Constraints]} \\
- In SELECT <column>, make sure there are at least two selected!!!\\
- In FROM <table> or JOIN <table>, do not include unnecessary table\\
- Use only table names and column names from the given database schema\\
- Enclose string literals in single quotes\\
- If [Value examples] of <column> has `None' or None, use JOIN <table> or WHERE <column> is NOT NULL is better\\
- Ensure GROUP BY precedes ORDER BY for distinct values\\
- NEVER use window functions in SQL\\
\\
Now we could think step by step:\\
1. First choose visualize type and binning, then construct a specific sketch for the natural language query\\
2. Second generate SQL components following the sketch.\\
3. Third add Visualize type and BINNING into the SQL components to generate final VQL\\
\\
==============================\\
Here is a typical example:\\
\textbf{[Database Schema]}\\
\# Table: course, (course)\\
\text{[}\\
  \text{\ \ \ \ }(course\_id, course id, Value examples: [101, 696, 656, 659]. And this is an id type column),\\
  \text{\ \ \ \ }(title, title, Value examples: [`Geology', `Differential Geometry', `Compiler Design', `International Trade', `Composition and Literature', `Environmental Law'].),\\
  \text{\ \ \ \ }(dept\_name, dept name, Value examples: [`Cybernetics', `Finance', `Psychology', `Accounting', `Mech. Eng.', `Physics'].),\\
  \text{\ \ \ \ }(credits, credits, Value examples: [3, 4].)\\
\text{]}\\
\# Table: section, (section)\\
\text{[}\\
  \text{\ \ \ \ }(course\_id, course id, Value examples: [362, 105, 960, 468]. And this is an id type column),\\
  \text{\ \ \ \ }(sec\_id, sec id, Value examples: [1, 2, 3]. And this is an id type column),\\
  \text{\ \ \ \ }(semester, semester, Value examples: [`Fall', `Spring'].),\\
  \text{\ \ \ \ }(year, year, Value examples: [2002, 2006, 2003, 2007, 2010, 2008].),\\
  \text{\ \ \ \ }(building, building, Value examples: [`Saucon', `Taylor', `Lamberton', `Power', `Fairchild', `Main'].),\\
  \text{\ \ \ \ }(room\_number, room number, Value examples: [180, 183, 134, 143].),\\
  \text{\ \ \ \ }(time\_slot\_id, time slot id, Value examples: [`D', `J', `M', `C', `E', `F']. And this is an id type column)\\
\text{]}\\
\textbf{[Augmented Explanation]}\\
The filtered schema consists of 2 tables (course and section) with a total of 11 relevant columns. There is a one-to-many relationship between course and section through the course\_id foreign key.\\
\\
% \end{promptbox}
% \end{figure*}
% \begin{figure*}[!h]
% \begin{promptbox}[Prompt template for single classification]
Key points:\\
1. The course table contains information about individual courses, including the course title, department, and credits.\\
2. The section table contains information about specific sections of courses, including the semester, year, building, room number, and time slot.\\
3. The course\_id column links the two tables, allowing us to associate sections with specific courses.\\
4. The dept\_name column in the course table will be used to filter for Psychology department courses.\\
5. The year column in the section table will be used for yearly grouping and binning.\\
6. We need to count the number of courses offered each year, which requires aggregation and time-based binning.\\
\\
The query involves joining these two tables to analyze the number of courses offered by the Psychology department each year, which requires aggregation and time-based binning.\\
\\
\textbf{[Question]}\\
Find the number of courses offered by Psychology department in each year with a line chart.\\
\\
Decompose the task into sub tasks, considering [Background] [Special Rules] [Constraints], and generate the VQL after thinking step by step:\\
\\
\textbf{Sub task 1:} First choose visualize type and binning, then construct a specific sketch for the natural language query\\
Visualize type: LINE, BINNING: True\\
VQL Sketch:\\
Visualize LINE SELECT \_ , \_ FROM \_ JOIN \_ ON \_ WHERE \_ BIN \_ BY YEAR\\
\\
\textbf{Sub task 2:} Second generate SQL components following the sketch.\\
Let's think step by step:\\
1. We need to select 2 columns for LINE chart, year as X-axis, COUNT(year) as Y-axis.\\
2. We need to join the course and section tables to get the number of courses offered by the Psychology department in each year.\\
3. We need to filter the courses by the Psychology department.\\
4. We do not need to use any window function for YEAR.\\
\\
\text{sql}\\
```sql\\
SELECT S.year, COUNT(S.year)\\
FROM course AS C\\
JOIN section AS S ON C.course\_id = S.course\_id\\
WHERE C.dept\_name = `Psychology'\\
```\\
\\
% \end{promptbox}
% \end{figure*}
% \begin{figure*}[!h]
% \begin{promptbox}[Prompt template for single classification]
\textbf{Sub task 3:} Third add Visualize type and BINNING into the SQL components to generate final VQL\\
\textbf{Final VQL:}\\
Visualize LINE SELECT S.year, COUNT(S.year) FROM course C JOIN section S ON C.course\_id = S.course\_id WHERE C.dept\_name = `Psychology' BIN S.year BY YEAR\\
\\
==============================\\
Here is a new question:\\
\\
\textbf{[Database Schema]}\\
\{desc\_str\}\\
\\
\textbf{[Augmented Explanation]}\\
\{augmented\_explanation\}\\
\\
\textbf{[Query]}\\
\{query\}\\
\\
Now, please generate a VQL sentence for the database schema and question after thinking step by step.\\

\end{promptbox}
% \end{figure*}

% \subsection{Validator Agent Prompt}
% \label{validator_prompt}
% \begin{figure*}
\begin{promptbox}[Prompt template for Validator Agent]
As an AI assistant specializing in data visualization and VQL (Visualization Query Language), your task is to refine a VQL query that has resulted in an error. Please approach this task systematically, thinking step by step.\\
\textbf{[Background]}\\
VQL Structure:\\
Visualize [TYPE] SELECT [COLUMNS] FROM [TABLES] [JOIN] [WHERE] [GROUP BY] [ORDER BY] [BIN BY]\\
\\
You can consider a VQL sentence as "VIS TYPE + SQL + BINNING"\\
\\
Key Components:\\
1. Visualization Type: bar, pie, line, scatter, stacked bar, grouped line, grouped scatter\\
2. SQL Components: SELECT, FROM, JOIN, WHERE, GROUP BY, ORDER BY\\
3. Binning: BIN [COLUMN] BY [INTERVAL], [INTERVAL]: [YEAR, MONTH, DAY, WEEKDAY]\\
\\
When refining VQL, we should always consider special rules and constraints:\\
\textbf{[Special Rules]} \\
a. For simple visualizations:\\
    \text{\ \ \ \ }- SELECT exactly TWO columns, X-axis and Y-axis(usually aggregate function)\\
b. For complex visualizations (STACKED BAR, GROUPED LINE, GROUPED SCATTER):\\
    \text{\ \ \ \ }- SELECT exactly THREE columns in this order!!!:\\
        \text{\ \ \ \ }\text{\ \ \ \ }1. X-axis\\
        \text{\ \ \ \ }\text{\ \ \ \ }2. Y-axis (aggregate function)\\
        \text{\ \ \ \ }\text{\ \ \ \ }3. Grouping column\\
c. When "COLORED BY" is mentioned in the question:\\
    \text{\ \ \ \ }- Use complex visualization type(STACKED BAR for bar charts, GROUPED LINE for line charts, GROUPED SCATTER for scatter charts)\\
    \text{\ \ \ \ }- Make the "COLORED BY" column the third SELECT column\\
    \text{\ \ \ \ }- Do NOT include "COLORED BY" in the final VQL\\     
d. Aggregate Functions:\\
    \text{\ \ \ \ }- Use COUNT for counting occurrences\\
    \text{\ \ \ \ }- Use SUM only for numeric columns\\
    \text{\ \ \ \ }- When in doubt, prefer COUNT over SUM
% \end{promptbox}
% \end{figure*}

% \begin{figure*}
% \begin{promptbox}[Prompt template for Validator Agent]
e. Time based questions:\\
    \text{\ \ \ \ }- Always use BIN BY clause at the end of VQL sentence\\
    \text{\ \ \ \ }- When you meet the questions including "year", "month", "day", "weekday"\\
    \text{\ \ \ \ }- Avoid using time function, just use BIN BY to deal with time base queries\\
\\
\textbf{[Constraints]} \\
- In FROM <table> or JOIN <table>, do not include unnecessary table\\
- Use only table names and column names from the given database schema\\
- Enclose string literals in single quotes\\
- If [Value examples] of <column> has `None' or None, use JOIN <table> or WHERE <column> is NOT NULL is better\\
- ENSURE GROUP BY clause cannot contain aggregates\\
- NEVER use date functions in SQL\\
\\
\textbf{[Query]} \\
\{query\}\\
\\
\textbf{[Database info]} \\
\{db\_info\}\\
\\
\textbf{[Current VQL]} \\
\{vql\}\\
\\
\textbf{[Error]} \\
\{error\}\\
\\
Now, please analyze and refine the VQL, please provide:\\
\\
\textbf{[Explanation]}\\
\text{[}Provide a detailed explanation of your analysis process, the issues identified, and the changes made. Reference specific steps where relevant.\text{]}\\
\\
\textbf{[Corrected VQL]}\\
\text{[}Present your corrected VQL here. Ensure it's on a single line without any line breaks.\text{]}\\
\\
Remember:\\
- The SQL components must be parseable by DuckDB.\\
- Do not change rows when you generate the VQL.\\
- Always verify your answer carefully before submitting.\\
\end{promptbox}
% \end{figure*}

\end{document}

