\section{\textsc{The World Wide Dishes System}}\label{sec:system}

The \textsc{World Wide Dishes System} is an infrastructure for building the \textsc{World Wide Dishes Dataset}. The \textsc{WWD Dataset} consists of local dishes from around the world built with input from community Contributors who interact with the \textsc{WWD System} by sharing personal, local knowledge of the dishes they attribute to their own home(s) and culture(s). For each dish, we include the name of the dish (in both the local language and English), the country of origin, the region of origin, the associated culture, the time of day at which the meal is eaten, the type of meal, the utensils used, the drinks that accompany the meal, any special occasions when the meal is eaten, the ingredients, the recipe, and the image of the dish if available. 

In this section, we describe the stakeholder groups who interacted with the \textsc{WWD System}. Interactions included the construction of the underlying architecture as well as the \textsc{WWD Dataset}. 

\subsection{How stakeholders interact with the \textsc{World Wide Dishes System}}
Three key stakeholder groups interact with the \textsc{WWD System}: Core Organisers, Contributors, and Community Ambassadors. The stakeholder groups are not mutually exclusive; for example, some Community Ambassadors were also Contributors. See~\cref{tab:stakeholder_tally} for an overview of the stakeholder groups, participant counts in each group, and geographic regions represented within each group. Below we describe how each stakeholder group interacted with \textsc{WWD} and with each other. 
\begin{table}[h]
\centering
\renewcommand{\arraystretch}{1.5}

\caption{\small \textbf{Stakeholders.} We indicate the demographics for the stakeholders associated with the projects.}
\label{tab:stakeholder_tally}
\small
\begin{tabular}{p{2cm}>{\footnotesize}p{3.5cm}p{2cm}p{3.5cm}p{1.5cm}}
\toprule
\textbf{Stakeholder} & \textbf{Role} & \textbf{Participant count} & \textbf{National identities by continent} & \textbf{Age range}\\ 
\midrule
Core Organisers & the central organising team that coordinated the development and execution of World Wide Dishes. Core Organisers were also Community Ambassadors and Contributors & 12 & Africa, Asia, North America & - \\
Contributors and Community Ambassadors & These roles often overlapped, with Community Ambassadors supporting amplification of the WWD and Contributors submitting dishes & 162 & Africa, Asia, Europe, North America, South America, Oceania & 19-62; Mean=31.5$\pm$8.78 \\
\bottomrule
\end{tabular}
\end{table}

\textbf{Core Organisers:} Began the initiative to build \textsc{WWD}. They created the data collection process, front-end \textsc{WWD} project website, and back-end database, as well as the data processing pipeline. 

\textbf{Contributors:} Accessed the \textsc{WWD} data collection form via the \textsc{WWD} project website and provided cultural knowledge about each dish submitted through the form. Contributors also provided feedback on other entries by other Contributors.

\textbf{Community Ambassadors:} Distributed the \textsc{WWD} webpage to and through their social networks to solicit contributions. Community Ambassadors hosted focus groups and sometimes filled out the data collection form on behalf of Contributors who faced issues with submitting the form themselves. Community Ambassadors assisted Contributors with translations if needed to ensure submissions were in English. Community Ambassadors often acted as Contributors, submitting dish information to the \textsc{WWD} data collection form. Additionally, Community Ambassadors served as a communication channel between the Contributors and Core Organisers, surfacing Contributor concerns as they arose and communicating subsequent actions by the Core Organisers back to the Contributors. Finally, Community Ambassadors assisted in the data processing stage.


\subsection{{\textsc{World Wide Dishes System}} architecture}
The \textsc{WWD System} is a web application (\Cref{fig:front_page}) built with Django\footnote{https://www.djangoproject.com/} and hosted on Google Cloud Platform, with a PostgreSQL database for storing submissions and Google Cloud Storage for storing image files. \Cref{fig:WWD Architecture} presents an overview of the system components. The backend processes which handle data collection, storage, and processing are described below.

\begin{figure}[t]
    \centering
    \includegraphics[width=\textwidth]{figures/WWD_Architecture.png}
    \caption{\textbf{Overview of the World Wide Dishes flow}. (A) A Contributor accesses \textsc{WWD} through a web browser. They consent to be a research participant and decide whether to create an account or proceed as a guest. They then fill out the data collection form with information about themselves and the dish they submit.  (B) The submission is then stored in the \textsc{WWD} database. We store Contributors' information separately from the dish information to preserve their privacy. (C) The full database containing dish information is then used to operationalise bias investigations into generated text and image content using other vision-language and language models. (D,E) A report then is generated from the automated bias testing and the survey responses and then made public. \textit{All icons in this figure were downloaded from Flaticon. For proper attribution please see \ref{asec:informed_consent}.}} 
    \label{fig:WWD Architecture}
\end{figure}

 

\begin{figure}[t]
    \centering
    \includegraphics[width=\textwidth]{figures/front_page.png}
    \caption{\textbf{Front page of the World Wide Dishes website}. Two call-to-action buttons were placed at the top of the page to encourage participation from site visitors. The ‘Check out our food leaderboard’ button takes site visitors to the leaderboard table, while the ‘Add your local dish’ button guides visitors through the data entry flow.}
    \label{fig:front_page}
\end{figure}

\subsubsection{Data collection}
The questions asked in the data collection form were collated in consultation with experts on food and cultural heritage, as well as feedback from community members, and a desire to capture as much metadata as possible to highlight regional differences in dishes. Food as a cultural object moves easily across borders and therefore the same dish may exist in more than one place---but the customs or range of ingredients associated with it may differ with changes in geography or the passage of time. The questions were therefore designed to give Contributors the opportunity to provide as much local nuance as possible. We present a list of data collected below.

\begin{enumerate}
    \item Data collection involved an online form (see~\cref{asec:protocol-screenshots} for the questions and~\cref{tab:dish_questions} for the data points).
    \item Request to upload a non-generated image of food from personal records (shared with consent to distribute for research purposes). 
    \item Questions about food. 
    \item Submission of dish name was encouraged in the local language, with a translation / phonetic equivalent included.
    \item English was encouraged, but concession was made if a translation for an ingredient didn't exist/wasn't known, e.g. cassava as an ingredient / sadsa as a dish.
    \item Associated customs and information about the dish were collected: time of day eaten, utensils, ingredients, associated celebrations/events, online recipe, freeform general information.
\end{enumerate}

\begin{table}[h]
\centering
\caption{\small Data collected through the  \textbf{World Wide Dishes} contribution form}
\label{tab:dish_questions}
\begin{tabular}{ll}
\toprule
\textbf{Question} & \textbf{Data entry} \\
\midrule
Image and caption & Image upload and text caption \\
Dish name in a local language & Short text \\
Name of local language & Short text \\
Country / countries & Dropdown and free text \\
Region & Free text \\
Attribution to a specific cultural, social, or ethnic group & Free text \\
Time of day eaten & Multiple choice \\
Dish classification & Free text or dropdown \\
Components, elements, and/or ingredients & Dropdown and free text \\
Utensils & Free text \\
Accompanying drinks & Free text \\
Association with a special occasion & Multiple choice \\
Recipe & URL \\
Any other comments & Long-form free text \\
\bottomrule
\end{tabular}
\end{table}

\subsubsection{Recruitment and engagement strategies}
The system is built for easy sharing on social networks, prioritizing cross-browser compatibility, responsiveness, and a lightweight design to ensure accessibility on a wide range of devices. All information about the project and the data collection is available on the website. Recruitment involved many social networks, including through existing communities such as AYA,\footnote{https://aya.for.ai/} Masakhane,\footnote{https://www.masakhane.io/} the Deep Learning IndabaX network,\footnote{https://deeplearningindaba.com/2024/indabax/} AI Saturdays Lagos,\footnote{https://aisaturdayslagos.github.io/} and OLS.\footnote{https://we-are-ols.org/} Posts were also placed on a mailing list to encourage engagement.

Engagement was explicitly encouraged through the promotion of a submission leaderboard (see~\cref{fig:leaderboard}),
which displayed the number of dish submissions and contributors received from each region. This was done in an attempt to gamify the experience and build excitement and fun. The leaderboard was promoted during recruitment outreach. 

\begin{figure}[t]
    \centering
    \includegraphics[width=0.6\textwidth]{figures/WWD_Leaderboard_new.png}
    \caption{\textbf{Screenshot of the World Wide Dishes  Learderboard} showing the top 10 countries by number of dishes, the total number of contributed dishes and contributors per country.}
    \label{fig:leaderboard}
\end{figure}

\subsubsection{Data storage}  The collected data is stored in two sections: the dish data, which has been made public and is completely anonymised, and the personal data, which has been collected and stored according to the terms agreed by the ethics review. We collected names, ages, and national identities, and also accepted anonymous submissions. Names were only used when explicit consent was given to acknowledge contributions publicly. Age and national identity data were required to help us understand the demographics of people submitting to \textsc{WWD}. Approval for data collection and the subsequent research study was obtained from the Departmental Research Ethics Committees of the Computer Science Department at the University of Oxford (reference: CS\_C1A\_24\_004). Notably, the ethics review required us to collect age data, as Contributors had to be over the age of 18 to participate; age data was therefore collected even if the Contributor otherwise submitted dish data anonymously.

\subsubsection{Data processing}

\begin{enumerate}
    \item \textbf{Data cleaning} Core Organisers lead the data cleaning process to remove duplicate entries within countries and to standardise data entry (i.e. uniform descriptions of regions in a country and language). Core Organisers consulted Community Ambassadors when needed.
    \item \textbf{Translations} Entries were primarily in English, except for some dish names and ingredients without known English translations. For submissions made by French-speaking Contributors from the Democratic Republic of Congo, and to make concessions for accessibility, the Core Organisers accepted these specific entries and translated them with an open source machine translation system and a Core Organiser who is a native French speaker audited these results.
    \item \textbf{Removal of images with uncertain licences} by Core Organisers involved the removal of any uploaded images whose licence could not be verified. Accepted images came from royalty-free sites that did not prohibit their use in machine learning, had an accompanying Creative Commons licence, or had been taken and submitted by a Contributor with explicit permission given for it to be used for research purposes. 
    \item \textbf{To augment the image data}, Core Organisers solicited additional royalty-free and/or Creative Commons images of the submitted dishes from the Internet and consulted Community Ambassadors for their assessments of the images' accuracy in depicting the submitted dishes.
    \item \textbf{Inconsistencies in submitted data were handled by consulting Community Ambassadors} where possible, and Community Ambassadors consulted with other Contributors as needed. However, we aimed to collect lived experiences around the world and so we did not make any efforts to police a `ground truth' for each dish, or assign an `origin' or `authenticity' to any dish. Multiple nations or cultural groups can share the same dishes, for a wide variety of reasons including historical trade routes, war and occupation, the redrafting of political borders without regard to cultural borders, and migration patterns. Our task was not to make judgments about cultural `ownership' of the submitted dishes. Instead, the data collection asked participants to offer dishes from their own lived experiences and personal backgrounds, so we expected to see similarities across borders, as well as some reasonable regional and preference variations of the ingredients of dishes within and across borders.
\end{enumerate}



























