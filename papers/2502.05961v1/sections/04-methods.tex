\section{Methods and Analysis}\label{sec:methods}
Our study aims to map out the process of building a dataset of cultural objects---specifically food---through a bottom-up, community-led approach. 

Planning, building, and deploying the system that resulted in the \textsc{WWD} dataset was a process that occurred over 12 months. In November 2023, the first and seventh through ninth authors of this paper decided to build \textsc{WWD} to investigate how cultural bias in ML manifests in generated images of food. The research activities conducted as part of the development of \textsc{WWD} were approved by the first author's IRB. All the authors of this paper, except the second author, were involved with the development of \textsc{WWD}. These authors took extensive field notes about their interactions with data contributors throughout the process. After the process of building \textsc{WWD} concluded, the authors shared field notes and post-mortem reflections with the second author, who is a social computing scholar. The second author was invited onto this project by the first author, who wanted to engage with the social computing field to highlight the processes underpinning the development of the \textsc{WWD} dataset. 

We also present a system overview of the \textsc{WWD} technical infrastructure in~\Cref{sec:system} to give readers a better understanding of how the data collection process functioned. Below, we provide more detail on the researchers who provided reflections on the development of \textsc{WWD}, our positionality, and our analysis methods. 

\subsection{The original \textsc{World Wide Dishes} research team}
All authors, except the second author, were members of the original \textsc{WWD} research team. The original research team comprises 11 researchers from academia and industry and one independent researcher (12 total). The large majority of the original research team is from countries in Africa, has spent the majority of their lives on the African continent, and remains deeply involved in the African ML community (see~\cref{tab:og_wwd_team} for an overview of team members and their respective positionalities). Of the 12 members of the original \textsc{WWD} research team, nine participated in the post-mortem reflection process for this paper. 

\begin{table}[h]
\centering
\renewcommand{\arraystretch}{1.5} % Adjust the spacing between rows (1.5 times the default)

\caption{\small \textbf{The original World Wide Dishes team, and their corresponding positionality.}}
\label{tab:og_wwd_team}
\small % Reduce font size to make the table more compact
\begin{tabular}{p{1cm}p{3cm}p{3cm}p{2.5cm}p{2cm}p{1.5cm}} % Set custom column widths
\toprule
\textbf{Code}& \textbf{Background}  &  
\makecell[l]{\textbf{Involvement in the} \\ \textbf{African ML community}} & 
\makecell[l]{\textbf{Research} \\ \textbf{Institution}} &
\makecell[l]{\textbf{Community} \\ \textbf{Ambassador}} &
\makecell[l]{\textbf{Duration of} \\ \textbf{involvement}} \\ 
\midrule
O1  & Health Sciences and Machine Learning & Deep Learning Indaba  & University of Oxford & South Africa & 12 months \\
O2  & Law and Cultural Heritage Studies & -- &  University of Oxford &United States & 11 months  \\
O3  & Engineering and Machine Learning & --  & University of Oxford & Japan & 11 months  \\
O4  & Computer Science and Machine Learning & Deep Learning Indaba & University of Oxford & Kenya & 11 months \\ 
O5  & Mechanical Engineering and Computer Science &  AI Saturdays Lagos, Deep Learning Indaba & CISPA Helmholtz Center for Information Security & Nigeria & 8 months \\
O6  & Machine Learning & Deep Learning Indaba, Deep Learning IndabaX Egypt  & Microsoft & Egypt & 8 months \\
O7  & Mathematics and Computer Science & Deep Learning Indaba  & Independent Researcher & Sudan & 8 months \\
O8 & Mathematics and Computer Science & Deep Learning Indaba, Deep Learning IndabaX Cameroon, KmerAI  & Conservatoire National des Arts et Métiers & Cameroon & 8 months  \\
O9  & Machine Learning & Deep Learning Indaba, African Computer Vision Summer School  & DAIR, McGill \& MILA & South Africa & 7 months \\
O10  & Computer Science and Machine Learning & Deep Learning Indaba, Deep Learning IndabaX Algeria  & École nationale Supérieure d’Informatique Algier & Algeria & 7 months \\ 

\bottomrule
\end{tabular}
\end{table}

\subsection{Positionality}
We interrogate our positionality as researchers to contextualise how we approach the design and development of \textsc{WWD} as well as how we interpreted data from field notes and post-mortem reflections. Nine of the ten authors on this paperwork are in the computing space, with all of the authors except for the second and eighth working in the ML field. The second author was brought onto this project for their experience in writing for social computing venues and the eighth for their extensive knowledge in cultural heritage studies. All the authors, with the exception of the second, sixth, and eighth, are deeply engaged in the African ML Community. The first, third through sixth, and ninth and tenth authors are all from countries in Africa and have only recently (e.g., in the past few years) left the continent to pursue their under- and postgraduate studies. These authors maintain strong ties with their communities in their country of birth and tap into those community networks to collect the data that comprises the \textsc{WWD} dataset. 

We acknowledge that our cultural identities fundamentally shaped how we approached building \textsc{WWD} and coloured our analysis of post-mortems. Many of us acted as gatekeepers to the communities where data contributors resided. We actively shaped the data collection process for \textsc{WWD} to be compatible with the cultural norms and values we, as members of these communities, knew were important. Simultaneously, we maintained open channels of communication to engage with feedback from the community to allow us to reflect on the values. Our reflections are impacted by our identities as both members and researchers of these communities.  

\subsection{Analysis}
All the authors of this paper, except the second author, provided field notes they took throughout the development of \textsc{WWD} and post-mortems, in the form of audio recordings and written reflections. The authors created a shared document with all the field notes and post-mortems and collaboratively coded the data using an approach informed by grounded theory. The first and second authors coded field notes and reflections from the three of the team members. Then, they conducted a card-sorting exercise to arrive at an initial set of high-level themes. The first and second authors shared the high-level themes back with the remaining authors on the paper to get their input and agree upon a coding scheme. Finally, the first and second authors coded the remainder of the field notes and post-mortems according to the agreed-upon coding scheme to result in the four main themes we present in \Cref{sec:rollout}.

Because \textsc{WWD} is a socio-technical system, we also provide an overview of the digital infrastructure underpinning the data collection process in \cref{sec:system}. We then illustrate the human labour involved in operationalising this digital infrastructure to build a dataset of cultural objects from the bottom up in \cref{sec:rollout}. To answer our primary research question, \textit{How can researchers collect cultural data using a bottom-up, community-led approach?}, \textbf{we present four dimensions of data work: (1) building trust with communities, (2) making participation accessible, (3) recognising the produced nature of data, and (4) attending to the relationships between culture and food}.
