 \section{Building \textsc{World Wide Dishes}: Lessons from the field}\label{sec:rollout}



Building the \textsc{WWD Dataset} was a process that took over seven months, and involved 12 Core Organisers and more than 170 Community Ambassadors and Contributors. Some of the Core Organisers also acted as Community Ambassadors (see~\cref{tab:stakeholder_tally} for the regions for which each Core Organiser served as a Community Ambassador). The process of launching the \textsc{WWD} data collection effort involved a beta-testing stage to solicit an initial round of feedback from community members about the data collection instrument and process, as well as hosting ongoing focus groups with Contributors to surface issues with data donations. Through systematic analysis of post-mortems and field notes from the Core Organisers, we identify four key elements of the process to build \textsc{WWD} that ultimately made this a successful effort: trust within the community, accessibility of participation, attention to the production of data, and understanding the relationship between food and culture. Below, we reflect on and surface how the \textsc{WWD} dishes project supported each element.  

\subsection{Building trust and leveraging existing social networks to access community members}
Collecting long-tailed data requires accessing and connecting with underrepresented populations who tend to be ``hard to reach.''\footnote{We use quotation marks to draw attention to the fact that the term ``hard to reach'' is value-laden. Calling a population ``hard to reach'' redirects attention away from the reasons why that population may be difficult for the researcher to access and instead focuses attention on the group as the problem, rather than on the institution or researcher as the potential problem. For more, see~\cite{benjamin2016informed,epstein2008rise}.} Obtaining this access typically requires working with community gatekeepers who can serve as intermediaries between the researchers and community~\cite{le2015strangers,epstein2008rise}. However, the presence of a community gatekeeper in a project does not in itself resolve ethical questions around shifting the distribution of power from the researchers to the researched. In building \textsc{WWD}, we made an intentional choice to involve community gatekeepers (e.g., Community Ambassadors) as members of the research team who shaped the data collection process, the development of the data collection tool, and the paper authorship process.  

The first author, who is a researcher at a prestigious Western university, recognised that their positionality put them in a position of power as relates to the collection of cultural data from communities to which they do not belong. While they are originally from South Africa and have been deeply involved in community efforts within the African ML community, such as the Deep Learning Indaba, they felt that the \textsc{WWD} project must be co-led by members from communities who were contributing data. Therefore, they invited three people in their professional network who were active researchers and volunteers for local AI communities to be co-leads and co-authors on the project. Two of the three people, O5 and O7, are Nigerian, and Sudanese, respectively, and were experienced in conducting data collection efforts with communities within their home countries. As O5, O6, and O7 spread information about \textsc{WWD} to their networks, additional community members representing more than eight countries on the African continent signed up to be Community Ambassadors for the project. Community Ambassadors leveraged their positionality, relationships, and social capital to make data collection with communities in Africa possible. 

Community Ambassadors reflected on how their previous work within the African community and being members of the local communities from which they were soliciting data donations was crucial to the success of \textsc{WWD}. In post-mortem reflections, O5 shared:
\begin{quote}
    ``As a community leader, I've been organizing free classes for Community Y\footnote{Community name is anonymised for review.} for over six years. Over time, this commitment has built trust within my community, reassuring people that I genuinely prioritise our shared interests. I would like to think that this trust helped community members feel comfortable contributing to the effort, knowing they aren’t being taken advantage of.'' O5
\end{quote}
Community Ambassadors had established reputations within the communities from which they were soliciting data donations, enabling them to collect cultural data alongside community members who trusted their intentions. 

Established methods of accessing data donors, such as cold-calling, or simply distributing the call to participate on mass broadcast social media channels (X, formerly known as Twitter; LinkedIn), were arguably less successful, even when it was the Community Ambassadors themselves who posted the call. Access to, or the ability to access, potential donors was insufficient. Instead, Community Ambassadors often had to rely on the relationships they had already established to solicit contributions. For example, O6 reported in their post-mortem that: ``So, I kept sharing and sharing over social media ... but the contributions were not growing as much as I hoped ... So I reach out to people individually.'' As a result, Community Ambassadors tapped into their network of friends and family to solicit data contributions. However, reaching out to personal networks was not without risk. 

Data contributions were not \textit{financially} remunerated. Therefore, some Community Ambassadors shared that the lack of financial compensation made them selective in who they reached out to about the project, as they were essentially asking their friends and family to do volunteer labour. However, despite the lack of remuneration, Community Ambassadors reported that many of the people to whom they reached out shared a common desire to participate in a project that could make GenAI systems work for them:
\begin{quote}
    ``It's about strengthening African machine learning, it's about empowering Africans, it's about making sure that all of the AI technology works for us all, and ensuring we're part of the builders of it... so maybe through our several grassroots machine learning efforts, we can get to some kind of balance in [power] regarding whose perspectives shape the building of AI technologies.'' O5
\end{quote}

Community Ambassadors controlled how they, and by proxy the larger \textsc{WWD} team, interacted with their communities. For example, the Core Organisers and Community Ambassadors administered WhatsApp groups within the larger \textsc{WWD} WhatsApp community, where they carried out conversations with Contributors in the language of the region they oversaw. Community Ambassadors had complete oversight over how they wanted to structure conversations within their WhatsApp group and enforce moderation rules that were proposed by the Core Organisers. 

Community Ambassadors had a shared context with the people whom they asked to contribute data to the project: they spoke the same language as the Contributors, had previously established themselves as trustworthy, and shared similar goals for African leadership in ML projects. Building on this, we expand on the need to meet community members where they are and make participation accessible.

\subsection{Make participation accessible}

Engaging in bottom-up, community-led data collection requires extra efforts on the part of researchers to make participation accessible. In this project, the researchers did so by providing a more accessible informed consent process and hosting ongoing office hours to help Contributors learn how to participate and understand the goals and motivations of the \textsc{WWD} project. It is not enough to build a data contribution form that is ``easy'' to fill out; rather, researchers must invest in significant amounts of human labour to animate the data collection effort. 

During beta-testing, Community Ambassadors flagged that while the consent form was accessible---in that it could be translated and read in the local language of contributors---it was not understandable. The research team had originally presented the entire consent form, as approved by their institution's ethics review board, as the first step of the data contribution process on the \textsc{WWD} website. Following feedback from Community Ambassadors, the research team co-designed a concise consent form (see \ref{fig:informed_consent_new}) that laid out key information about the project's purpose, procedures, and data privacy protections. The full consent form was then linked on the consent overview page (see~\cref{asec:informed_consent}). Upon reflection, O1 recounted that consent forms designed and approved in Western, \textit{academic} contexts were inaccessible for the regions in which the Contributors were located. The researchers had to engage in translation of the meaning of certain terms, such as ``cookies'', to make informed consent accessible for all participants. 

Collecting image data for ML purposes, as \textsc{WWD} does, is a complex process due to concerns about image ownership. Because \textsc{WWD} was intended to be an open-source dataset with Creative Commons licensing, all submitted images either had to have a Creative Commons license (if it wasn't an original) or be an original (e.g., taken by the contributor) photo shared with explicit permission that it could be used for research purposes. These guidelines were presented clearly on the data contribution form; however, Community Ambassadors noticed early on that Contributors were uploading images from the Internet that did not have Creative Commons licenses. As a result, the Core Organisers and Community Ambassadors began hosting regular ``teach-ins'' during office hours for Contributors to attend in order to learn how to properly make a data donation. O6 explained the role of office hours in making data contribution guidelines accessible to participants:
\begin{quote}
    ``The [office hours] helped ensure our data collection met our requirements and that contributors understood both the licensing requirements and how to properly complete the submission forms, enabling us to collect exactly what we needed for the project'' O6
\end{quote}
Office hours lowered the barrier to entry for participation by providing a space for Contributors to get help filling in the information on their submissions. Community Ambassadors recalled how Contributors would bring ideas about dishes they wanted to submit, but needed help finding an image they could use or verifying other data fields, such as utensils used and/or associated cultural ceremonies (as examples), for a dish. In these cases, the office hours attendees would work together to complete a submission collaboratively. 

Office hours also served as a space for Contributors to shape the research process. Office hours attendees were encouraged to share feedback about challenges they faced in making data donations, which the Community Ambassadors then brought to planning meetings with the Core Organisers. As a result, the Core Organisers amended the data collection instrument, produced additional guidelines for participation, and refined their recruitment strategy. O1 recalls how the team wanted to follow so-called ``professional protocol'' and communicate through official and professional media such as mailing lists and social media. The members of the community immediately asked for WhatsApp groups. 
\begin{quote}
    ``As a team, we tried to think through the implications---the most poignant being the lack of a boundary between our own work and professional lives by engaging with our personal WhatsApp applications. However, having already identified how important it was for us to meet the community where they were, and having had similar calls in the past through our work with established communities on the African continent, we made the commitment to host the community groups on WhatsApp, and maintain open communication between ourselves to make sure we could sustainably engage throughout the data collection process.'' O1 
\end{quote}
Building on the feedback received from Community Ambassadors and Contributors, the Core Organisers quickly realised how much behind-the-scenes effort was going into a single submission. Data are found through extensive consultation and effort, and the final entry on the website is a reduced form of the rich engagement that led to it. 

\subsection{Recognise that data are produced, not simply found}

Throughout the process of building \textsc{WWD}, it became clear that cultural data is something that is produced through social interaction. The concept of data production is not new~\cite{bowker2000sorting,d2024counting}. However, it is often glossed over in ML papers that present novel datasets while only briefly discussing the data work~\cite{scheuermanDatasetsHavePolitics2021} that goes into constructing a dataset. In our findings, we disrupt the assumption that a single participant in a crowdsourcing effort can submit complete information about an object of cultural significance and shed light on the myriad social interactions underpinning individual data contributions.

Contributors actively engaged in conversations with family and friends, accessing familial networks to collect the data needed for a submission. For example, O8 shared one Contributor's anecdote in their post-mortem where the Contributor asked their friends and family to help complete data submissions:
\begin{quote}
    ``For the photos and information I couldn’t get myself, I asked certain resourceful people, particularly my parents and friends, to provide the photos and information about the utensils to use and the appropriate time to eat certain meals'' O8, \textit{summarizing what a Contributor told them}
\end{quote}
While it could be argued that Contributors could simply look up dish information on the Internet to fill out their forms, this was not a viable option for many of our Contributors coming from countries where information about their regional cuisines was simply not available online. Familial networks proved to be a more reliable way to get accurate information about local cuisine. O8 recounted an experience where a Contributor in the community they oversaw encountered difficulty finding information about a regional Cameroonian dish they had grown up eating:
\begin{quote}
    ``For the dishes I struggled to explain, I turned to various sources for help. When possible, I looked online to see how they were made. However, for some dishes, it was difficult to find accurate information. In those cases, I reached out to my mother and grandmother, who were able to share their expertise and guide me through the process'' O8, \textit{summarising what a Contributor told them}
\end{quote}
Community Ambassadors and Contributors alike often turned to their familial networks, especially parents and grandparents, to produce the information they needed about a dish through conversations and glimpses into the archives of family recipes. O5 recalled that during their office hours, Contributors shared stories about calling their mothers and sisters to get help filling out information for a dish: ``Some [Contributors] mentioned that they reached out to their family to learn more about the native dishes they grew up eating because they realise they don't know how to prepare them.''  

Data about food, and how it is connected to a community's culture, is not simply sitting somewhere, waiting to be harvested. Moreover, in the regions in which we were operating, there was little to no reliable information about cultural cuisines available online. As a result, Contributors and Community Ambassadors produced this data through conversations with kin. Food is deeply intertwined with cultural practices and therefore some of our Community Ambassadors chose to ground their data collection efforts in major cultural events that coincided with the data collection phase. 

For example, O6 shared how they deliberately chose to revive their call for data contributions during Ramadan and Eid al-Fitr. Ramadan and Eid al-Fitr, which are major holidays for Muslims, are often celebrated with special dishes holding cultural significance that are only prepared during the holiday period. O6 explained: ``Food plays a great, very important role in Ramadan and so I was very intentional about just taking pictures of whatever food we made for iftar which is the breaking of the fast.'' The food that is prepared during these holidays holds cultural significance and helps tell a part of the story about a community's values and history. 

Communities are not passive data sources; these data about food were produced through social interactions and lived experiences. 

\subsection{Understand the relationships between food and culture}

Food is complex: dishes often harbour a deeper cultural significance. For example, O6 recalled how during their experience managing data contributions from Egypt, they realised that the classification of ``Egyptian food'' was fuzzy at best:
\begin{quote}
    ``[WWD] made me more conscious about \textbf{what is} Egyptian Cuisine and the influences from other cuisines that have actually, you know, shaped what we eat and what we consider as Egyptian. As it turns out, it's very intersectional.'' O6, emphasis added
\end{quote}
For O6, capturing data about Egyptian cuisine led them to further investigate the origins of their own cultural dishes. In the weeks following the data collection effort, they attended a session about "food as cultural currency" at the RiseUp Summit Egypt 2024 to learn more about the Egyptian cuisine through the eyes of local food experts.

O1 recalled how the fuzzy classification of dishes as belonging to a certain region impacted the data collection process:
\begin{quote}
    ``...depending on the region within the country [the dish] was either going to be made from the cassava plant or a kind of mielie meal. It could be the same dish but depending on the region within the same country they would have a different ingredient for the starch [element].'' O1
\end{quote}

Here, O1 identifies that even though a dish is considered ``the same'' and may be called by the same name across different regions, the actual ingredients in this dish differ and reflect the differing agriculture practices of various regions. These differing agricultural practices are important markers of the local ecosystems and historical practices of farming and cultivation. Therefore, in the data collection process, it became essential to deconstruct dishes into their constituent ingredients, as reflected in the data structure for \textsc{WWD} (see~\cref{tab:dish_questions} and~\cref{asec:protocol-screenshots}). 


In post-mortem reflections, Community Ambassadors highlighted the importance of recognising the limitations of using national borders to demarcate cultures. The granularity of representation was essential to ensuring that distinct cultures were not lumped together into a single flattened snapshot, as physical borders often do. O5 recalls how she realised that one of the three major ethnic groups in Nigeria---the Hausa community---was underrepresented in the \textsc{WWD} dataset. They reached out to a personal connection whom they knew was a member of that ethnic community who, in turn, helped share the website with their community. 


In line with what we posited in the introduction, we reinforce the idea that the value in a dataset is not exclusively conveyed in its final form, but also through the processes of creating it. In this Findings section, we highlight the immense efforts and considerations that went into engaging with our Contributors to produce high-quality, granular data about cultural dishes. Processes such as these are slow, iterative, and very hard to scale, but they are necessary to ensure the production of high-quality and diverse datasets that we would like to see reflected in the ML systems that make use of this data. 




 






