\section{Conclusion}\label{conclusion}


Through \textsc{World Wide Dishes}, we present a method for dataset curation that relies on a bottom-up, community-led effort that can lead to high-quality data enriched with community and local expertise. This enrichment comes from extensive human efforts that rely on engagement with social networks working together to produce cultural data. This was made possible by engaging with existing community networks developed by Community Ambassadors through trust and careful communication. The infrastructure \textsc{WWD} developed increased accessibility through reformed consent processes, careful consideration of accessible language, and allowing submissions in multiple ways: directly from Contributors or with the assistance of Community Ambassadors. Community Ambassadors played a pivotal role in supporting the process by distributing the website, answering questions, and providing translations where needed. While this process is not without its limitations, we expand upon the tensions in~\cref{sec:discussion} and suggest future work in~\cref{sec:limitations}. Data are produced, not found. Future efforts to support bottom-up, community-led data collection must therefore provide infrastructure that supports the social and relational processes of data production. 






