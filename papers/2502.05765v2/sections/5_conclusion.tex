%  We are replicating prior work closely, additionally we are comparing plain- text, but not the full plaintext (accomplish what has been done before). The KL-score that is based on the fraction of the samples, the correlation is not strong (so we really need the full data, **use L-BFGS), blindly choosing and using demographic-only data -- race distribution.

% 1. privacy is valuable, makes transactions easier to make, cost-effective for making better models that result in better healthcare, builds trust. (privacy being beneficial). 
% 2. health outcome is improved. Consequential things that can benefit from having more data. Reduce errors by X percent if you use our method to sequentially select hospitals to work with.

Our work demonstrates that privacy-preserving data valuation can help organizations identify beneficial data partnerships while maintaining data sovereignty. Through SecureKL, we show that entities can make informed decisions about data sharing without compromising privacy or requiring complete dataset access. As the AI community continues to grapple with data access challenges, particularly in regulated domains like healthcare, methods that balance privacy and utility will become increasingly critical for responsible advancement of the field. As noted in Section~\ref{sec:limits}, our approach has several limitations, including the fact that, despite impressive aggregate results, our method is less effective for individual hospitals; this finding is fertile ground for future work. Additionally, our work present opportunities for follow-up research. Our method assumes static datasets and may not generalize well to scenarios where data distributions evolve rapidly over time. A sequential version of our framework may more closely model dynamic data collaborations. Future work should explore extending these techniques to handle more complex data types and dynamic distribution shifts while maintaining strong privacy guarantees.

% Our method advances prior work in several key ways. Beyond replicating existing KL-score computations in a privacy-preserving manner, we demonstrate that using complete datasets rather than sample-based estimates is crucial for accurate partnership evaluation - particularly since correlation strength diminishes with reduced data sampling. Our empirical results show that privacy-preserving data partnerships can deliver tangible benefits: hospitals using our method to select collaboration partners saw a X\% reduction in mortality prediction errors compared to demographic-only or random selection approaches. This improvement highlights how privacy-preserving methods can facilitate valuable data partnerships while building trust and maintaining data sovereignty, ultimately leading to better healthcare outcomes through more robust models.
\clearpage