\section{Related Work}
\paragraph{Agent-Based Models.}
Agent-based models (ABMs) provide a powerful framework for studying collective behavior and emergent phenomena in complex systems \cite{bonabeau2002agent,basu2015scaffolding}. By simulating agent interactions, ABMs capture macro-level patterns such as market fluctuations and social dynamics \cite{palmer1999artificial,abdollahian2013human,axtell2022agent}. They encode key behavioral traits—such as fundamentalist and chartist trading, herding tendencies, and decision-making heterogeneity—into simplified rules like strategy switching and structural stochastic volatility (SSV) \cite{cont2001empirical,franke2012structural}. This abstraction effectively reproduces financial market stylized facts, including fat-tailed return distributions and volatility clustering \cite{gaunersdorfer2007nonlinear,franke2016simple}. Recent advances integrate machine learning, enhancing ABMs' realism and predictive power \cite{georges2021market,park2023generative,reale2024interbank}.

\paragraph{LLMs for Behavioral Simulation.} 


Large language models (LLMs) excel at simulating complex human behaviors, from rational decision-making and financial market analysis to replicating behavioral biases like herding and overconfidence \cite{chen2024evaluating,yu2024fincon,fedyk2024chatgpt,yang2024oasis}. Beyond behavioral simulation, they enable fine-grained modeling of individual actions and large-scale emergent phenomena \cite{abbasiantaeb2023letllmstalksimulating,hua2024gametheoreticllmagentworkflow}. LLMs naturally exhibit demographic diversity, such as gender, age, and education, while engaging in rich social interactions \cite{fedyk2024chatgpt,eisfeldt2024ai}. By automating text analysis, generating experimental stimuli, and designing multi-agent systems, they offer powerful tools for social science research \cite{gilardi2023chatgpt,chuang2023simulating,yang2024ucfe}.