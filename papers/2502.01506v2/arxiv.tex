\pdfoutput=1
\documentclass{article}

% if you need to pass options to natbib, use, e.g.:
    \PassOptionsToPackage{numbers, compress}{natbib}
% before loading neurips_2024


% ready for submission
\usepackage[preprint]{neurips_2024}
\usepackage{threeparttable}
\usepackage{tcolorbox}
\usepackage{enumitem}
\usepackage{soul}
\usepackage[pagebackref=true,breaklinks=true,colorlinks,citecolor=citecolor, linkcolor=linkcolor]{hyperref}


% to compile a preprint version, e.g., for submission to arXiv, add add the
% [preprint] option:
%     \usepackage[preprint]{neurips_2024}


% to compile a camera-ready version, add the [final] option, e.g.:
%     \usepackage[final]{neurips_2024}


% to avoid loading the natbib package, add option nonatbib:
%    \usepackage[nonatbib]{neurips_2024}


% \usepackage[utf8]{inputenc} % allow utf-8 input
\usepackage[T1]{fontenc}    % use 8-bit T1 fonts
\usepackage{hyperref}       % hyperlinks
\usepackage{url}            % simple URL typesetting
\usepackage{booktabs}       % professional-quality tables
\usepackage{amsfonts}       % blackboard math symbols
\usepackage{nicefrac}       % compact symbols for 1/2, etc.
\usepackage{microtype}      % microtypography
\usepackage{xcolor, colortbl}         % colors
% \usepackage{CJK}
% \usepackage{CJKutf8}
\usepackage{graphicx}
\usepackage{amsmath}
% \usepackage{longtable}
\usepackage{algorithm}
\usepackage{algorithmic}
\usepackage{multirow}
\usepackage{array}
\definecolor{citecolor}{HTML}{0071BC}
\definecolor{linkcolor}{HTML}{ED1C24}
\usepackage{tcolorbox}
\usepackage{caption}
\captionsetup{width=0.8\textwidth}
\usepackage{pifont}
\usepackage{varwidth}
\usepackage{listings}
\tcbuselibrary{listingsutf8} 

\lstdefinelanguage{YAML}{
  comment=[l]{\#},          % Line comments start with #
  commentstyle=\color{gray}\itshape, % Style for comments (e.g., gray, italic)
  stringstyle=\color{red},     % Style for strings (e.g., red)
  basicstyle=\ttfamily\small,
  breaklines=true,
  frame=single
}

% \title{Formatting Instructions For NeurIPS 2024}
% \title{Injecting Visual Medical Knowledge in  Medical Generalist LLMs}

% \title{Towards Expert-Level Medical Multimodal Data \\ Using Weak Multimodal LLMs}
% \title{Leveraging GPT-4V for Superior Medical Multimodal Data Synthesis}

% \title{Towards Injecting Medical Visual Alignment from PubMed to Multimodal LLMs at scale}
\title{TwinMarket: A Scalable Behavioral and Social Simulation for Financial Markets}


\definecolor{lightpink}{rgb}{0.945, 0.816, 0.804}
\definecolor{lightgreen}{rgb}{0.851, 0.906, 0.839}
\definecolor{lightblue}{rgb}{0.8, 0.9, 1}
\definecolor{lightyellow}{rgb}{0.992, 0.949, 0.816}
\newcommand{\dataset}{\textbf{PubMedVision}}
\newcommand{\imp}[1]{{\small\hspace{0.05cm}{\color[HTML]{32CB00}{$_{\textbf{+#1}}$}}}}
\newcommand{\dec}[1]{{\small\hspace{0.05cm}{\color[HTML]{CD5C5C}{$_{\textbf{-#1}}$}}}}
\setlength{\fboxsep}{2pt} 

% The \author macro works with any number of authors. There are two commands
% used to separate the names and addresses of multiple authors: \And and \AND.
%
% Using \And between authors leaves it to LaTeX to determine where to break the
% lines. Using \AND forces a line break at that point. So, if LaTeX puts 3 of 4
% authors names on the first line, and the last on the second line, try using
% \AND instead of \And before the third author name.


\author{
\textbf{Yuzhe Yang}$^{1}$\thanks{Equal contribution.} , 
\textbf{Yifei Zhang}$^{2}$\footnotemark[1] , 
\textbf{Minghao Wu}$^{2}$\footnotemark[1] , 
\\ \textbf{Kaidi Zhang}$^{1}$,  \textbf{Yunmiao Zhang}$^{2}$, \textbf{Honghai Yu}$^{2}$, \textbf{Yan Hu}$^{1}$,
\textbf{Benyou Wang}$^{1}$\\
$^1$ The Chinese University of Hong Kong, Shenzhen \\
$^2$ Nanjing University\\
\url{https://github.com/TobyYang7/TwinMarket}\\
}

\begin{document}


\maketitle


\begin{abstract}
The study of social emergence has long been a central focus in social science. Traditional modeling approaches, such as rule-based Agent-Based Models (ABMs), struggle to capture the diversity and complexity of human behavior, particularly the irrational factors emphasized in behavioral economics. Recently, large language model (LLM) agents have gained traction as simulation tools for modeling human behavior in social science and role-playing applications. Studies suggest that LLMs can account for cognitive biases, emotional fluctuations, and other non-rational influences, enabling more realistic simulations of socio-economic dynamics. In this work, we introduce \textbf{TwinMarket}, a novel multi-agent framework that leverages LLMs to simulate socio-economic systems. Specifically, we examine how individual behaviors, through interactions and feedback mechanisms, give rise to collective dynamics and emergent phenomena. Through experiments in a simulated stock market environment, we demonstrate how individual actions can trigger group behaviors, leading to emergent outcomes such as financial bubbles and recessions. Our approach provides valuable insights into the complex interplay between individual decision-making and collective socio-economic patterns.






\end{abstract}



\section{Introduction}

\begin{figure*}[t]
    \centering
    \includegraphics[width=1\linewidth]{img/TwinMarket.pdf}

    \caption{Overview of \textbf{TwinMarket} environment, where each user has a unique persona within the social network, interacts with the environment in real-time, and influences it through their actions. This framework enables the study of emergent social phenomena.}
    \label{fig:intro}

\end{figure*}


Recently, large language models (LLMs) have demonstrated the ability to capture intricate patterns in language use, decision-making, and social interactions \cite{abbasiantaeb2023letllmstalksimulating,hua2024gametheoreticllmagentworkflow}. These capabilities have positioned LLMs as powerful tools for agent-based simulations in human behavior studies across various social science fields, including conversational dynamics \cite{chen2024evaluating, yang2024ucfe, xie2024largelanguagemodelagents}, emotional responses \cite{li2023largelanguagemodelsunderstand}, and creative reasoning \cite{renze2024selfreflectionllmagentseffects, wei2023chainofthoughtpromptingelicitsreasoning}. Such advancements open new avenues for social science research, enabling the modeling of fine-grained individual behaviors at the \textbf{micro-level} and the simulation of emergent phenomena at the \textbf{macro-level}.


Financial markets, with their unique combination of high-resolution transactional data, observable collective phenomena \textit{(e.g., herding behavior \& market crashes)}, and direct socio-economic relevance, provide an ideal testbed for studying how micro-level behaviors propagate to macro-level dynamics \cite{ciarli2010effect,baqaee2023micro}. However, existing agent-based models (ABMs) in social simulation primarily rely on rule-based approaches, which often oversimplify human decision-making by assuming homogeneous agents and static behavioral rules \cite{matthews2006people,wang2024homogeneous}. These oversimplifications fail to capture critical intricacies of real-world economic behavior and struggle to scale to market complexity, obscuring causal links from individual decisions to systemic emergence.


The study of social emergence phenomena, where micro-level individual behaviors aggregate to form macro-level collective outcomes, has long been a central focus in social science \cite{ weber1978economy,durkheim2023rules}. However, current methods often struggle to capture the dynamic interplay between individual biases \textit{(e.g., overconfidence)} and emergent market phenomena \textit{(e.g., volatility clustering)} \cite{gu2023dynamic,kiruthika2023cognitive}. Recent advances in LLMs offer a promising avenue to address this challenge, enabling the simulation of adaptive agents whose decisions are shaped by both private information and social interactions. Motivated by this potential, we propose \textbf{TwinMarket}, a novel multi-agent framework that leverages LLMs to simulate investor behavior in a stock market environment. By modeling agents that independently make investment decisions while interacting through a simulated social media platform, \textbf{TwinMarket} provides a unified perspective on \textit{how \textbf{micro-level} behaviors drive \textbf{macro-level} market dynamics}.



As illustrated in Figure~\ref{fig:intro}, a key innovation of \textbf{TwinMarket} is its use of the \textbf{Belief-Desire-Intention} (BDI) framework \cite{rao1995bdi} to structure and visualize the cognitive processes of agents, providing a transparent and interpretable modeling approach. Additionally, we develop a social environment that enables agents to engage in dynamic information exchange and social influence processes, enabling the study of phenomena such as the emergence of opinion leaders and the dynamics of information cascades. Finally, by scaling our simulations to multiple financial assets and larger populations, we investigate how group size and interaction complexity affect collective behavior and emergent phenomena.

Our research makes three key contributions: \textbf{(1) Real-World Alignment:} Our framework is grounded in established behavioral theories and calibrated with real-world data, ensuring theoretical rigor and realistic human behavior modeling. This enables the study of emergent social phenomena. \textbf{(2) Dynamic Interaction Modeling:} By leveraging an LLM-based framework, we capture diverse human behaviors and their interactions, particularly in the context of information propagation. This allows us to model how opinions spread, influence decision-making, and dynamically shape collective market behaviors in real time. \textbf{(3) Scalable Market Simulations:} We conduct large-scale simulations to analyze the impact of group size on market behavior, providing new insights into price fluctuations and emergent dynamics.



\section{Background}


Traditional Agent-Based Models face two major challenges in simulating complex social systems: (1) customizing sophisticated economic decision-making for individual agents and (2) enabling agents to actively perceive and respond to external environmental changes. Recent research has made progress by incorporating LLMs into agent-based simulations. CompeteAI \cite{zhao2024competeaiunderstandingcompetitiondynamics} introduces a framework for modeling competitive dynamics; however, its agents rely on predefined prompts, limiting their ability to simulate nuanced economic behavior. EconAgent \cite{li2024econagent} designs economic agents with perception, reflection, and decision-making capabilities to model macroeconomic dynamics but overlooks the influence of multi-agent interactions on emergent macroeconomic phenomena. Similarly, ASFM \cite{gao2024simulating}, which models stock market responses to regulatory policies, fails to adequately account for the role of agent interactions and collective decision-making in shaping market trends. Table\ref{tab:comparison} provides a comparison of these systems with TwinMarket, highlighting the differences in agent customization, social interactions, and simulation scale.

\begin{table*}[htbp]
    \centering
    \renewcommand{\arraystretch}{1.0}
    \resizebox{1\textwidth}{!}{
        \begin{tabular}{lccc}
            \toprule
            \textbf{System} & \textbf{Agent Customization} & \textbf{Social Interactions} & \textbf{Simulation Scale} \\
            \midrule
            \textbf{CompeteAI} 
                & Simple predefined prompts 
                & Competition between 2 restaurants  
                & Less than 100 agents \\
            \textbf{EconAgent} 
                & Uses demographic info 
                & No interaction considered 
                & 100 agents \\
            \textbf{ASFM} 
                & Two fixed strategies 
                & No interaction considered 
                & Not disclosed \\
            \rowcolor{lightgreen}
            \textbf{TwinMarket} 
                & Highly customizable agents 
                & Rich multi-agent interactions 
                & 1000 agents, scalable \\
            \bottomrule
        \end{tabular}
    }
    \caption{Comparison of TwinMarket with Existing Literature}
    \label{tab:comparison}
\end{table*}




In our \textbf{TwinMarket} framework that leverages LLM-based agents to simulate complex human behaviors, we include rational decision-making, technical analysis, and behavioral biases such as herding and overconfidence \cite{yang2024oasis, fedyk2024chatgpt}. By grounding our model in established behavioral theories \cite{markowitz1991foundations, kahneman2013prospect, shefrin2000behavioral} and calibrating it with real-world data, we ensure both theoretical rigor and empirical relevance. In \textbf{TwinMarket}, agents make investment decisions within a socially embedded environment where interactions occur through price dynamics and social media, shaping collective market behavior via information propagation, behavioral imitation, and interaction mechanisms.

This work demonstrates the potential of LLMs to bridge computational simulations and behavioral science by systematically modeling \textbf{micro-level} cognitive processes and their \textbf{macro-level} societal effects. By integrating LLMs with established theoretical frameworks, \textbf{TwinMarket} provides an experimental platform to study how individual decision-making aggregates into emergent social and economic phenomena. Our findings illustrate \textit{how LLMs can simulate real-world behaviors, validate behavioral theories, and uncover mechanisms underlying social emergence}, advancing the academic understanding of complex human systems.

\section{Framework of TwinMarket}


At the micro level, we focus on realistically modeling individual user behavior in the stock market, capturing decision-making processes and interactions grounded in decision theories and validated by real-world trading patterns. Moving to the macro level, we construct an interactive, forum-like environment where distributed agents can exchange information, form opinions, and influence each other's decisions, thereby simulating the emergent properties of real-world social systems.


\subsection{Data Sources and Simulation Setup}



\begin{table*}[htbp]
  \centering
    \renewcommand{\arraystretch}{0.8} 
  \resizebox{1\textwidth}{!}{
    \begin{threeparttable}
      \caption{Summary of data sources for \textbf{TwinMarket}.}
      \begin{tabular}{l*{3}{r}}
        \toprule
        \textbf{Usage} & \textbf{Source} & \textbf{Quantity/Time Period} & \textbf{Data Type} \\
        \midrule
        \multirow{2}{*}{\textbf{Construct Initial User Profile}} & \multirow{2}{*}{Xueqiu\tnote{1}} & 639 users & User Profile \\
        & & 11,965 transactions & Transactions Details \\
        \midrule
        \textbf{Train Stock Recommendation System} & Guba Stock Forum\tnote{2} & 83,246 transactions & Transactions Details \\
        \midrule
        \textbf{Update Fundamental Stock Data} & CSMAR\tnote{3} & Jan 2023 - Dec 2023 & Stock Data\tnote{4} \\
        \midrule
        \multirow{2}{*}{\textbf{Build Information Retrieval Database}} & Sina\tnote{5}~~\& 10jqka\tnote{6} & \multirow{2}{*}{Jan 2023 - Dec 2023} & News Articles \\
        & CNINFO\tnote{7} & & Company Announcements \\
        \bottomrule
      \end{tabular}
      \label{tab:data_sources}
      \scriptsize
      \begin{tabular}{p{4cm} p{5cm} p{7cm}}
        \textsuperscript{1} \url{https://xueqiu.com/} & \textsuperscript{2} \url{https://guba.eastmoney.com/} & \textsuperscript{3} \url{https://data.csmar.com/} \\
        \multicolumn{3}{l}{\textsuperscript{4} We choose the 50 largest and most liquid stocks on the Shanghai Stock Exchange (known as the SSE 50 index).} \\
        \textsuperscript{5} \url{https://www.sina.com.cn/} & \textsuperscript{6} \url{https://www.10jqka.com.cn/} & \textsuperscript{7} \url{http://www.cninfo.com.cn/new/index.jsp}
      \end{tabular}
    \end{threeparttable}
  }

\end{table*}

\paragraph{Data Sources.} As shown in Table~\ref{tab:data_sources}, \textbf{TwinMarket} integrates real user profiles, transacation details, stock data, news and announcements to create a realistic social simulation environment. Details about our data sources can be found in Appendix~\ref{app_data_source}.

\begin{figure}[t]
    \centering
\begin{tcolorbox}[
    colback=gray!5, 
    colframe=black, 
    width=\columnwidth,
    boxrule=0.5pt, % 缩小边框线的粗细
    left=5pt, % 调整左边距
    right=5pt, % 调整右边距
    top=5pt, % 调整上边距
    bottom=5pt % 调整下边距
]
    \scriptsize
    \fcolorbox{lightblue}{lightblue}{\parbox{\dimexpr\linewidth-2\fboxsep}{You are a male investor based in Hubei, actively using Xueqiu to share and gather investment insights. As an ordinary investor with a modest following, your profile reflects typical retail investor characteristics.}} \\[0.5em]
    \fcolorbox{lightpink}{lightpink}{\parbox{\dimexpr\linewidth-2\fboxsep}{Your trading behavior exhibits a clear pattern: you quickly sell assets that gain more than 10\%, locking in profits, but tend to hold onto or even double down on losing investments, often ignoring the risks of over-concentration. While lottery-type assets occasionally catch your attention, they rarely influence your overall strategy.}} \\[0.5em]
    \fcolorbox{lightyellow}{lightyellow}{\parbox{\dimexpr\linewidth-2\fboxsep}{Historically, your portfolio has underperformed the market average. You favor concentrated investments, heavily allocating to specific industry indexes, believing this approach will yield significant returns. Your trading activity is moderate, with regular adjustments to maintain portfolio balance and adapt to market conditions.}}
    \end{tcolorbox}

        \caption{An example of a generated user profile, with \fcolorbox{lightblue}{lightblue}{\textbf{demographics}}, \fcolorbox{lightpink}{lightpink}{\textbf{investment style}}, and \fcolorbox{lightyellow}{lightyellow}{\textbf{behavioral biases}}.}
        \label{fig:user_profile_example}
\end{figure}

\paragraph{Initial User Profile Construction.} Using real transaction data and user profiles from Xueqiu, we computed behavioral biases following Sui's pioneering work \cite{sui2023stakes} and extracted key characteristics like demographics to construct a joint distribution. This distribution was then used to sample and generate distinct agents, each initialized with trading records to provide a historical trading baseline. Additionally, each agent was initialized with a unique belief about the market environment and future expectations.Figure~\ref{fig:user_profile_example} illustrates an example of a generated agent, with further details on the generation process provided in Appendix~\ref{app_user_generation}.

\paragraph{Market Environment Setup.}
\textbf{TwinMarket} simulates a dynamic trading environment where agents' decisions drive stock prices through an order-driven matching system. Initialized with given trading records, agents collectively determine price movements once the simulation begins, ensuring a market environment consistent with real-world dynamics and grounded in financial principles. Technical details are discussed in Section~\ref{sec:workflow}.


\subsection{Micro-Level Simulation: Individual Behaviors}



To model individual behaviors in financial markets, we employ the \textbf{Belief-Desire-Intention} framework \cite{rao1995bdi}, which is widely recognized for its ability to replicate human decision-making processes. However, in many simulations, agents are often exposed to homogeneous information flows, leading to behavioral convergence and a lack of meaningful heterogeneity \cite{chen2025multiagentconsensusseekinglarge}. To address this limitation, our approach controls each agent's perception field by tailoring the information accessible to them based on their unique profiles. 


The BDI framework allows us to construct individual user agents that mimic the behaviors and chain-of-thought \cite{wei2023chainofthoughtpromptingelicitsreasoning} of real investor in the stock market. Specifically, the BDI model involves three interconnected components: \textbf{belief}, which represents the agent's evolving understanding of the market environment based on incoming data and predictions; \textbf{desire}, which captures the agent's objectives or preferences, guiding its pursuit of specific information or outcomes; and \textbf{intention}, which defines the concrete actions or strategies that the agent executes to achieve its goals. 


At the micro-level, our agents are equipped with three key functionalities: signal perception, planning, and decision-making. These capabilities allow them to replicate the nuanced behaviors of real-world investors. While financial decision-making may appear complex, it can be broken down into a sequence of smaller, observable actions. For example, individuals receiving market information such as stock price movements, breaking news, or social media discussions process this information to align it with their financial goals or aspirations \textbf{\textit{(desire)}}, form an understanding of the market \textbf{\textit{(belief)}}, and ultimately execute specific trading or interact with other users through social media \textbf{\textit{(intention)}}. By replicating these granular decision-making processes, the agents establish a solid foundation for simulating the broader social structures and dynamics of financial markets.

Each agent is assigned a unique role within the BDI framework, with clearly defined responsibilities and action spaces, as detailed in Table~\ref{table:BDI}. A comprehensive description of how these agents operate within the simulation workflow will be presented in Section~\ref{sec:workflow}.

\begin{table}[t]
\centering
\caption{The definition and action space for each BDI agent.}

\begin{tabular}{ccc}
\toprule
\textbf{Agent} & \textbf{Definition} & \textbf{Action Space} \\
\midrule
Belief  
& Market understanding \& prediction & -  
\\ 
Desire   
& Information retrieval goals & Query, Search \\
Intention   
& Environment interaction \& decision making & Social Media, Trading
\\ 
\bottomrule
\end{tabular}

\label{table:BDI}

\end{table}


\begin{figure*}[t]
    \centering
   
    \includegraphics[width=1\linewidth]{img/framework.pdf}

     \caption{The overall workflow of the \textbf{TwinMarket} involves the following iterative process: \ding{172} \textbf{User Data Initialization:} User profiles are initialized using Snowball data and their calculated behavioral biases. \ding{173} \textbf{Daily Simulation:} Using LLMs to simulate daily user behavior, distinguishing between trading and non-trading days. \ding{174} \textbf{Iterative Trading \& Update:} The system iteratively executes trades on trading days and updates user profiles on all days, looping back to simulation. \ding{175} \textbf{Post-Simulation Validation:} System performance is validated using micro and macro metrics after simulation completion.}
     \label{fig:framework}

\end{figure*}


\subsection{Macro-Level Simulation: Social Interactions}\label{sec:social_interactions}

In the financial market, information flow is a fundamental driver of collective behavior and market dynamics. Graphs provide a natural and powerful framework for modeling how information propagates through a network of interconnected agents, making them indispensable for macro-level simulations \cite{wang2025combining}. 

The market we are designing is inherently dynamic, with agents whose behaviors evolve over time. This dynamic nature requires the system to be adaptive and scalable, capable of responding to shifts in individual agent behavior and interactions. To meet these requirements, we adhere to two fundamental principles. First, the graph structure must be dynamic, able to evolve in response to the changing patterns of individual agents. Second, the agents must be capable of aggregating and processing environmental information through their interconnected relationships. These principles guide the construction of a framework that links micro-level interactions with macro-level systemic dynamics, enabling the system to adapt and scale effectively.

\paragraph{Social Network Construction.}  
Let \( \mathcal{G} = (\mathcal{V}, \mathcal{E}) \) be a social network where \( \mathcal{V} = \{u_1, u_2, \dots, u_n\} \) represents the set of user nodes, and \( \mathcal{E} \) represents the set of edges connecting users based on their trading behavior. Each user trades stock indexes belonging to industries \( I = \{i_1, i_2, \dots, i_m\} \), as defined in Appendix~\ref{app:aggr_ind}. We define the graph such that users with more similar trading behaviors have stronger connections in the social network. This reflects the idea that individuals with similar trading patterns are more likely to share information and influence each other, leading to higher homogeneity in the information they receive.

To model the dynamic nature of trading behavior (detailed in Appendix~\ref{appendix:dynamic}), we introduce a time decay factor \( \lambda \). This factor ensures that more recent trades have a greater influence on the graph structure, while older trades contribute less. For a trading record \( T(u, i, t) \), where user \( u \in \mathcal{V} \) trades in industry \( i \in I \) at time \( t \), we compute the time-decayed weight of their interactions as:
\[
w(u, i) = \sum_{t \in T(u, i)} e^{-\lambda \cdot \Delta t}
\]
where \( \Delta t \) is the time difference (in days) between the current time and \( t \). To quantify similarity, we calculate the weighted Jaccard similarity \( S(u_1, u_2) \) between users \( u_1 \) and \( u_2 \):
\[
S(u_1, u_2) = \frac{\sum_{i \in I} \min(w(u_1, i), w(u_2, i))}{\sum_{i \in I} \max(w(u_1, i), w(u_2, i))}
\]
An edge \( (u_1, u_2) \in \mathcal{E} \) is created with a weight \( S(u_1, u_2) \), representing the strength of the connection between the two users.

\paragraph{Information Aggregation Mechanism.}  
To control the perception field of each agent and ensure they receive relevant information, we design an information aggregation mechanism based on social network interactions. For a target user \( u_t \in \mathcal{V} \), let \( \mathcal{N}(u_t) \subseteq \mathcal{V} \) denote the set of neighboring users of \( u_t \) in the social network \( \mathcal{G} \), and let \( S(u_t, u) \) represent the similarity between \( u_t \) and a neighboring user \( u \in \mathcal{N}(u_t) \). We consider only those neighboring users whose similarity with \( u_t \) exceeds a threshold \( \tau \), i.e., \( S(u_t, u) > \tau \). 

For each such user \( u \), we retrieve their posts \( P(u) \) created within a specified time range \( [t_s, t_e] \), where \( t_s \) and \( t_e \) are the start and end timestamps, respectively. Each post \( p \in P(u) \) is associated with metadata, including its creation time \( t_p \), upvotes \( u_p \), and downvotes \( d_p \).  

To rank the posts, we compute a hot score \( h(p) \) for each post \( p \) using a logarithmic scaling of net votes combined with a time-decay factor:

\[
h(p) =
\begin{cases} 
\frac{\log_{10} (u_p - d_p + 1)}{(T_p + 1)^{1.8}}, & \text{if } u_p - d_p > 0 \\ 
- T_p, & \text{otherwise}
\end{cases}
\]

where $T_p = \max\left(\frac{t_c - t_p}{86400}, 0.1\right)$ represents the time elapsed in days since the post was created, with \( t_c \) being the current timestamp. This formulation ensures that highly engaged posts remain visible while older or negatively received posts decay more rapidly.

The posts are then sorted in descending order of their hot scores, and the top \( k \) posts are recommended to the target user \( u_t \), where \( k \) is a predefined maximum number of recommendations. This mechanism leverages the social network structure to aggregate information from users with significant similarity to \( u_t \), while prioritizing content based on engagement and recency.




\subsection{Workflow} \label{sec:workflow}



Our design philosophy, as illustrated in Figure~\ref{fig:framework}, simulates the complete workflow of users in real-world trading environment, from acquiring external information and making trading decisions to sharing insights with others. 




\paragraph{User Behavior.}



Users start by engaging with curated posts, where they can \textbf{upvote}, \textbf{downvote}, or \textbf{repost} content that resonates with their perspectives. Following this, they receive timely environmental updates, such as news and market developments, to stay well-informed.  These inputs, combined with their current \textbf{beliefs}, shape users' \textbf{desires} to seek specific information through targeted queries across news, announcements, and stock data. As users process this new information, they continuously update their \textbf{beliefs}, which in turn influence their evolving \textbf{desires} and ultimately form their \textbf{intentions}.
These \textbf{intentions} lead to transaction decisions, while users actively share their \textbf{beliefs} and market views through posts, enriching the system's collective knowledge base for future decision-making processes.


\paragraph{Dynamic Environment.}
To enable personalized decision-making and ensure distinct behavioral patterns across personas, we regulate each user’s perception field using recommendation algorithms. The stock recommendation system is driven by users’ portfolios and real trading data. Meanwhile, the post recommendation system dynamically adjusts based on post scores, which fluctuate in real-time and influence their ranking and visibility. Additionally, each user interacts with these recommendations selectively, guided by their individual perception field. Once a user interacts, information propagates dynamically, forming post-propagation chains that circulate throughout the market. Before trading, users can search for relevant news and announcements (Table~\ref{tab:data_sources}) and access technical indicators (e.g., trading volumes) and fundamental indicators (e.g., P/E ratios). To maintain valuation consistency, fundamental metrics are scaled using a ratio that aligns simulated prices with real-world values\footnote{Since fundamental metrics cannot be solely derived from agent-driven trading, they are adjusted based on the relative difference between simulated and real-world prices.}.



\section{Results and Analysis}

We conduct experiments on a network of 100 users, starting with a micro-level analysis to examine individual decision-making differences across various scenarios. We then shift to a macro-level perspective, evaluating how collective behaviors influence overall market dynamics. Our findings highlight a key aspect: the dynamic interplay between information propagation and collective behavior, where self-fulfilling prophecies, rumors, and shifts in sentiment trigger market fluctuations. Specifically, we observe how these phenomena drive trading behaviors and impact market stability, phenomena that were previously difficult to capture using traditional ABMs.


\subsection{Experimental Settings}
Based on our framework, we implemented the environment as a stock market with 10 representative indexes\footnote{10 indexes are aggregated from SSE 50, detailed in Appendix~\ref{app:aggr_ind}.} and 100 market participants entirely powered by GPT-4o \cite{hurst2024gpt}. The participants, consisting of fundamental and technical analysts, operate in a simulation period of 5 months, from June 15, 2023, to November 15, 2023. Each agent is initialized with a unique set of beliefs based on a predefined persona (detailed in Appendix~\ref{app:init_belief}), which influences their behavior bias and market assessment. The agents develop daily trading strategies and can freely choose to buy, sell, or hold positions, rather than being forced to trade. To facilitate information exchange, we implemented a stock forum where agents can share insights and market sentiment. The simulation will terminate early if any stock price exceeds predefined volatility limits or if there is a significant decrease in market participation.

\subsection{Micro-analysis: Self-fulfilling Prophecy}

\begin{figure}[h]
    \centering
    \includegraphics[width=0.7\linewidth]{img/self_full.pdf}

    \caption{Simultaneous rise and fall of the price index and belief, illustrating their strong correlation.}
    \label{fig:self_full}
\end{figure}

Self-fulfilling prophecy refers to the phenomenon where market participants act based on a certain expectation, ultimately leading to the realization of that expectation \cite{merton1948self}. 

In our BDI framework, we model users’ evolving expectations as belief, which is updated daily. The belief is structured along five key dimensions: the users' view on \textit{\textbf{economic fundamentals, market valuation levels, short-term market trends, sentiment of surrounding investors, and self-assessment}} (detailed in Appendix~\ref{app:bdi_belief}). Each of these dimensions is assigned an emotional score (higher score means more positive) using GPT-4o, and the agent’s overall sentiment is represented by the average score across the five dimensions. This granular breakdown allows us to capture how different aspects of belief influence investment decisions and market outcomes over time.


Our results from the \textbf{TwinMarket} environment demonstrate that initial optimistic expectations often lead to increased buying activity, which in turn pushes stock prices higher. This reinforces investors’ confidence, creating a self-reinforcing feedback loop that drives prices further away from fundamental values. Eventually, as prices become unsustainable and deviate from fundamental support, the market corrects sharply, leading to a collapse, as shown in Figure~\ref{fig:self_full}. This observed pattern closely aligns with the classic bubble dynamics described in financial literature \cite{shiller2000irrational}, highlighting the role of collective belief formation in driving both speculative booms and market crashes.


\subsection{Macro-analysis: Stylized Facts}

\begin{figure}[h]
    \centering
    \includegraphics[width=0.7\linewidth]{img/vis4.pdf}

    \caption{(a) Fat-tailed return distributions, (b) the leverage effect, (c) the volume-return relationship, and (d) volatility clustering. The results demonstrate that \textbf{TwinMarket} successfully replicates real market phenomena.}
    \label{fig:style}
\end{figure}

Existing studies have encoded key behavioral traits of market participants—such as \textit{\textbf{the interplay between fundamentalist and chartist trading, herding tendencies, and decision-making heterogeneity}}—into simplified rules, such as probability-based strategy switching and structural stochastic volatility (SSV) \cite{brock1998heterogeneous,lebaron2006agent,franke2016simple}. This abstraction has been shown to effectively reproduce stylized facts of financial markets, including fat-tailed return distributions and volatility clustering \cite{cont2001empirical,gaunersdorfer2007nonlinear,franke2012structural}.

LLM-based users inherently integrate the aforementioned behavioral traits, as LLMs have been demonstrated to analyze financial markets using both fundamental and technical metrics \cite{cheng2024sociodojo,yu2024fincon}. These agents naturally exhibit herding tendencies \cite{yang2024oasis}. To further capture the heterogeneity of market participants, we construct personalized user profiles for each users. These profiles enable agents to autonomously exhibit both rational investment strategies and irrational behavioral biases during decision-making, mirroring real-world investors. Further details on these dynamics are provided in Appendix~\ref{app:mirco}.

We compared the key stylized facts\footnote{Stylized facts are empirical statistical patterns observed in real-world financial markets. They are commonly used to evaluate the effectiveness of simulated markets in replicating real-world dynamics\cite{cont2001empirical}.} of \textbf{TwinMarket} with historical market data, ensuring that our simulations accurately reflect real-world market behaviors. Key comparisons of these stylized facts are shown in Figure~\ref{fig:style}.

\begin{itemize}
    \item \textbf{Non-Normality \cite{haas2009financial}:} The log returns exhibit a sharp peak and fat tails.
    \item \textbf{Leverage Effect \cite{bouchaud2001leverage}:} Negative returns show weak autocorrelation, indicating persistence in downturns.
    \item \textbf{Volume-Return Relationship \cite{llorente2002dynamic}:} Trading volume is positively correlated with returns, which typically reflects collective behavior among investors.
    \item \textbf{Volatility Clustering \cite{lux2000volatility}:} GARCH analysis\footnote{GARCH model is widely used to model time-varying volatility in financial markets \cite{francq2019garch}. In the GARCH framework, $\alpha$ represents the sensitivity of current volatility to past returns or shocks, while $\beta$ captures the persistence of past volatility.} shows that $\alpha + \beta$ is close to 1, indicating high persistence of volatility shocks.
\end{itemize}

The results show that \textbf{TwinMarket}, through a refined modeling of market participants and the market environment without relying on any rule-based techniques, is able to capture the macro-level dynamics of financial markets.



\section{Emergence of Group Behaviors}

\begin{figure*}[h]
    \centering
    \includegraphics[width=1\linewidth]{img/vis1.pdf}

\caption{Impact of rumor propagation on user behavior and market dynamics. (a) Users’ belief scores significantly decline under rumor exposure. (b) Users exhibit a stronger tendency to sell, leading to a sharp increase in the Sell/Buy ratio. (c) The collective reaction to rumors triggers a market downturn, causing a notable drop in stock prices.}
    \label{fig:rumor}
\end{figure*}

In financial markets, individual decisions are not made in isolation but are embedded within a complex social network. Price feedback, social interactions, and information flow create a dynamic feedback loop that amplifies or dampens individual actions. Through mechanisms like social influence and herd behavior, individual choices are continually reshaped by the collective actions and expectations of others, leading to the emergence of market trends and fluctuations. Through detailed modeling of individual agent behavior and their social interactions, our simulation demonstrates how macroscopic market phenomena emerge organically, validating core tenets of economic theory and mirroring empirical market observations, without imposing artificial constraints or pre-defined rules.





\subsection{Information Propagation}

Previous financial research has explored various phenomena, such as echo effects \cite{levy2019echo,cookson2023echo}, that arise from information propagation. To further investigate the mechanisms behind this propagation and its potential impact on the market, we begin by selecting the most important news or reports each day, which are then forcibly pushed to \textbf{high-centrality users}. This approach reflects the information asymmetry present in the market environment and enables us to study the dynamics of information flow. 

In addition, to compare and validate the impact of information on the market, we also replace a set of news with rumors, which are shown in Appendix~\ref{app:rumor_prompt}. These rumors are more extreme and sensational compared to regular news, potentially amplifying the effects of information flow in the market. Further explanations and results will bee illustrated in Appendix~\ref{appendix:chain}.

\begin{figure}[h]
    \centering
    \includegraphics[width=1\linewidth]{img/vis2.pdf}

    \caption{The adjacency matrix heatmap highlights distinct density patterns, particularly in the {\color{red} red-boxed} regions, indicating strong behavioral polarization among user groups.}
    \label{fig:polar}
\end{figure}
\paragraph{Intimation \& Polarization.}

\begin{figure}[h]
    \centering
    \includegraphics[width=0.7\linewidth]{img/like_trends.pdf}
    \caption{Average likes trends in high-centrality users and low-centrality users.}
    \label{fig:like_trends}
\end{figure}

In \textbf{TwinMarket}, graph edges represent behavioral similarity, meaning users closely connected exhibit stronger intimation effects. As shown in Figure~\ref{fig:like_trends}, high-degree users receive more upvotes, indicating their greater influence within the network. These \textbf{opinion leaders} \cite{valente2007identifying} shape decision-making, fostering homogeneity and leading to polarization, as depicted in Figure~\ref{fig:polar}. The network’s adjacency matrix highlights clusters of trading similarity, further intensified under rumor conditions. This suggests that biased or extreme information spreads more rapidly in tightly connected groups, reinforcing polarization \cite{bessi2016homophily,esau2024destructive}.

\paragraph{Belief Divergence.}

Figure~\ref{fig:rumor}(a) shows that beliefs in normal and rumor scenarios increasingly diverge\footnote{In rumor-exposed scenarios, agents maintained 27.5\% lower valuations than the control group.}, forming distinct echo chambers \cite{rhodes2022filter,piao2025emergence}. This occurs as negative rumors replace normal news, shifting users' market expectations \cite{vosoughi2018spread}. Rumor-exposed users become more pessimistic, reassessing asset values downward, reinforcing their isolation, and further distancing themselves from unaffected groups\cite{eismann2021diffusion}. This process exemplifies how biased information amplifies belief divergence and strengthens echo chambers in financial markets.

\paragraph{Trading Volatility.}  Trading behavior in \textbf{TwinMarket} is primarily driven by users' intentions and beliefs about the market, which in turn influence their buying and selling decisions. When belief evaluations fluctuate, particularly in response to rumors, these changes significantly impact trading activity \cite{ahern2015rumor,zhang2022effect}. As shown in Figure~\ref{fig:rumor}(b), under the influence of negative rumors, the proportion of sell orders increased significantly. The Sell/Buy ratio to surge to 2.02$\times$ the baseline\footnote{0.997 vs 0.495}, indicating amplified panic-driven selling behavior. 

\paragraph{Market Turbulence.}

The compounding effect of shifting beliefs and rumor-driven uncertainty leads to significant market instability. Figure~\ref{fig:rumor}(c) shows that while normal market conditions remained stable, rumor-exposed markets suffered a sharp decline. This drop reflects panic-driven sell-offs and overreaction to misinformation \cite{afrouzi2023overreaction}. As negative sentiment spreads, confidence erodes, reinforcing self-perpetuating downturns and destabilizing asset values.

\subsection{Scaling Up Design}

\begin{figure}[h]
    \centering
    \includegraphics[width=0.7\linewidth]{img/price_index_plot.pdf}

    \caption{Simulated and real index price comparison.}
    \label{fig:scale_up}
\end{figure}






To demonstrate the scalability of our framework, we expanded our simulation to include 1,000 agents, a scale typically challenging for traditional agent-based models. The results, as depicted in Figure~\ref{fig:scale_up}, reveal a very similar overall trend and movement throughout the five-month simulation period. This finding aligns with research showing that complex systems can exhibit similar macroscopic behavior when scaled up \cite{barabasi1999emergence}, even if the individual interactions at the micro-level become more numerous. In our case, it appears that the overall market dynamics can be well reproduced through emergent behaviors of LLM-powered agents. A more comprehensive quantitative evaluation, including metrics such as RMSE, MAE, and lag analysis, are detailed in Appendix~\ref{app:metrics}.






\section{Conclusion and Future Work}
In this paper, we investigate social emergence phenomena through LLM-driven agent methods. We introduce \textbf{TwinMarket}, a novel multi-agent framework that leverages LLMs to simulate investor behavior in a stock market environment. Our findings demonstrate that LLMs can effectively model real-world behaviors, validate behavioral theories, and reveal underlying mechanisms of social emergence, contributing to a deeper understanding of complex human systems.

Moving forward, we will explore the intrinsic properties of agent trust across diverse scenarios and examine broader implications, particularly in social science and role-playing applications. Additionally, we aim to gain deeper insights into LLM agent behaviors and the fundamental parallels between LLMs and human cognition. This line of research further paves the way for studying the alignment of LLMs with humans beyond value-based considerations.


\bibliography{main}
\bibliographystyle{unsrt}

\newpage
\appendix

\section*{Appendix}


\section{Related Work}


\paragraph{Agent-Based Models.}
Agent-based models (ABMs) provide a powerful framework for studying collective behavior and emergent phenomena in complex systems \cite{bonabeau2002agent,basu2015scaffolding}. By simulating agent interactions, ABMs capture macro-level patterns such as market fluctuations and social dynamics \cite{palmer1999artificial,abdollahian2013human,axtell2022agent}. They encode key behavioral traits—such as fundamentalist and chartist trading, herding tendencies, and decision-making heterogeneity—into simplified rules like strategy switching and structural stochastic volatility (SSV) \cite{cont2001empirical,franke2012structural}. This abstraction effectively reproduces financial market stylized facts, including fat-tailed return distributions and volatility clustering \cite{gaunersdorfer2007nonlinear,franke2016simple}. Recent advances integrate machine learning, enhancing ABMs' realism and predictive power \cite{georges2021market,park2023generative,reale2024interbank}.

\paragraph{LLMs for Behavioral Simulation.} 


Large language models (LLMs) excel at simulating complex human behaviors, from rational decision-making and financial market analysis to replicating behavioral biases like herding and overconfidence \cite{chen2024evaluating,yu2024fincon,fedyk2024chatgpt,yang2024oasis}. Beyond behavioral simulation, they enable fine-grained modeling of individual actions and large-scale emergent phenomena \cite{abbasiantaeb2023letllmstalksimulating,hua2024gametheoreticllmagentworkflow}. LLMs naturally exhibit demographic diversity, such as gender, age, and education, while engaging in rich social interactions \cite{fedyk2024chatgpt,eisfeldt2024ai}. By automating text analysis, generating experimental stimuli, and designing multi-agent systems, they offer powerful tools for social science research \cite{gilardi2023chatgpt,chuang2023simulating,yang2024ucfe}.

\section{TwinMarket Statistics} \label{app_data_source}


\subsection{SSE 50 and  10 Aggregated Index} \label{app:aggr_ind}
The SSE 50 index is a benchmark stock market index comprising the 50 largest and most liquid A-share stocks listed on the Shanghai Stock Exchange. It is widely used to track the performance of the blue-chip segment of the Chinese stock market.

The 10 aggregated indexes, on the other hand, are designed to represent different sectors or industries within the SSE 50. Each aggregated index consists of a subset of SSE 50 constituent stocks, grouped based on their industry classifications. The composition and weighting of each index are detailed below:

\begin{itemize}
    \item \textbf{TLEI (Transportation and Logistics Index):} This index comprises 2 constituent stocks, including COSCO SHIPPING Holdings (SH601919, weight 52.34\%) and China State Shipbuilding Corporation (SH600150, weight 47.66\%).
    \item \textbf{MEI (Manufacturing Index):} This index comprises 8 constituent stocks, including LONGi Green Energy Technology (SH601012, weight 16.51\%), Haier Smart Home (SH600690, weight 15.78\%), SANY Heavy Industry (SH600031, weight 15.29\%), NARI Technology (SH600406, weight 14.6\%), SAIC Motor (SH600104, weight 11.98\%), Tongwei (SH600438, weight 10.92\%), TBEA (SH600089, weight 10.12\%), and Great Wall Motor (SH601633, weight 4.8\%).
    \item \textbf{CPEI (Chemical and Pharmaceutical Index):} This index comprises 3 constituent stocks, including Jiangsu Hengrui Pharmaceuticals (SH600276, weight 45.93\%), Wanhua Chemical Group (SH600309, weight 28.39\%), and WuXi AppTec (SH603259, weight 25.68\%).
    \item \textbf{IEEI (Infrastructure and Engineering Index):} This index comprises 3 constituent stocks, including China State Construction Engineering (SH601668, weight 52.25\%), China Railway Group (SH601390, weight 27.64\%), and Power Construction Corporation of China (SH601669, weight 20.11\%).
    \item \textbf{REEI (Real Estate Index):} This index comprises 1 constituent stock, including Poly Developments and Holdings (SH600048, weight 100.0\%).
    \item \textbf{TSEI (Tourism and Service Index):} This index comprises 1 constituent stock, including China Tourism Group Duty Free (SH601888, weight 100.0\%).
    \item \textbf{CGEI (Consumer Goods Index):} This index comprises 5 constituent stocks, including Kweichow Moutai (SH600519, weight 69.18\%), Inner Mongolia Yili Industrial Group (SH600887, weight 13.13\%), Shanxi Xinghuacun Fen Wine Factory (SH600809, weight 7.17\%), Foshan Haitian Flavouring and Food (SH603288, weight 5.44\%), and Zhangzhou Pientzehuang Pharmaceutical (SH600436, weight 5.08\%).
    \item \textbf{TTEI (Technology and Telecommunications Index):} This index comprises 10 constituent stocks, including Semiconductor Manufacturing International Corporation (SH688981, weight 18.07\%), Hygon Information Technology (SH688041, weight 11.89\%), China Telecom (SH601728, weight 10.21\%), China Unicom (SH600050, weight 9.98\%), Advanced Micro-Fabrication Equipment Inc. China (SH688012, weight 9.79\%), China Mobile (SH600941, weight 9.76\%), China National Nuclear Power (SH601985, weight 9.05\%), Will Semiconductor (SH603501, weight 8.53\%), Kingsoft Office Software (SH688111, weight 6.92\%), and GigaDevice Semiconductor (SH603986, weight 5.81\%).
    \item \textbf{EREI (Energy and Resources Index):} This index comprises 6 constituent stocks, including China Yangtze Power (SH600900, weight 33.45\%), Zijin Mining Group (SH601899, weight 25.88\%), China Shenhua Energy (SH601088, weight 13.19\%), Sinopec (SH600028, weight 9.29\%), PetroChina (SH601857, weight 9.12\%), and Shaanxi Coal Industry (SH601225, weight 9.07\%).
    \item \textbf{FSEI (Financial Services Index):} This index comprises 11 constituent stocks, including Ping An Insurance (SH601318, weight 22.86\%), China Merchants Bank (SH600036, weight 17.94\%), CITIC Securities (SH600030, weight 11.97\%), Industrial Bank (SH601166, weight 10.47\%), Industrial and Commercial Bank of China (SH601398, weight 8.6\%), Bank of Communications (SH601328, weight 8.04\%), Agricultural Bank of China (SH601288, weight 6.12\%), China Pacific Insurance (SH601601, weight 4.63\%), Bank of China (SH601988, weight 4.21\%), China Life Insurance (SH601628, weight 2.8\%), and Postal Savings Bank of China (SH601658, weight 2.35\%).
\end{itemize}

\subsection{Real Stock Data Variables}
The trading data is collected on a daily basis and includes the following variables for each stock:

\begin{itemize}
    \item \textbf{stock\_id:} The unique identifier code for each stock.
    \item \textbf{close\_price:} The closing price of the stock on the trading day.
    \item \textbf{pre\_close:} The closing price of the stock on the previous trading day.
    \item \textbf{change:} The change in price of the stock from the previous trading day's close to the current day's close.
    \item \textbf{pct\_chg:} The percentage change in the stock's price from the previous trading day's close to the current day's close.
    \item \textbf{pe\_ttm:} The price-to-earnings ratio calculated using trailing twelve months (TTM) earnings.
    \item \textbf{pb:} The price-to-book ratio.
    \item \textbf{ps\_ttm:} The price-to-sales ratio calculated using trailing twelve months (TTM) sales.
    \item \textbf{dv\_ttm:} The dividend yield calculated using trailing twelve months (TTM) dividends.
    \item \textbf{vol:} The trading volume of the stock on the trading day.
    \item \textbf{vol\_5:} The 5-day average trading volume.
    \item \textbf{vol\_10:} The 10-day average trading volume.
    \item \textbf{vol\_30:} The 30-day average trading volume.
    \item \textbf{ma\_hfq\_5:} The 5-day moving average of the stock's closing price, using the backward-adjusted method to account for stock splits and dividends.
    \item \textbf{ma\_hfq\_10:} The 10-day moving average of the stock's closing price, using the backward-adjusted method.
    \item \textbf{ma\_hfq\_30:} The 30-day moving average of the stock's closing price, using the backward-adjusted method.
    \item \textbf{elg\_amount\_net:} The net inflow of large orders (if available). This represents the difference between buy and sell orders of a certain size that are considered to be institutional orders.
    \item \textbf{date:} The trading date associated with the record.
\end{itemize}

In addition to the trading data, the dataset includes basic information about each company. These variables are generally static but may be updated periodically:

\begin{itemize}
    \item \textbf{reg\_capital:} The registered capital of the company (in RMB).
    \item \textbf{setup\_date:} The date when the company was established.
    \item \textbf{introduction:} A brief textual description providing an overview of the company.
    \item \textbf{business\_scope:} A textual description outlining the company's main business operations.
    \item \textbf{employees:} The number of employees working for the company.
    \item \textbf{main\_business:} A textual description of the company's primary business activities.
    \item \textbf{city:} The city where the company's headquarters is located.
    \item \textbf{name:} The full name of the company.
    \item \textbf{industry:} The industry to which the company belongs.
\end{itemize}

\subsection{Empirical Analysis of Average Trading Volume for SSE 50 Constituents}
Xueqiu and Guba are both prominent Chinese social media platforms catering to financial investors. Consequently, as illustrated in Table~\ref{table:average}, we leverage transaction data from these two distinct sources:

\begin{itemize}
    \item \textbf{Xueqiu:} Data from Xueqiu covers the period from \textbf{2023-01-03} to \textbf{2023-12-06}. The Xueqiu data corresponds to \textbf{639} real user accounts. These accounts are used to provide seed data, initializing user profiles and generating historical transaction records for the \textbf{TwinMarket} simulation.
    \item \textbf{Guba:} Data from Guba covers the period from \textbf{2017-06-27} to \textbf{2024-06-03}. The Guba data is used to inform stock recommendations within the \textbf{TwinMarket} environment.
    
\end{itemize}
\begin{table}[h]
\centering
\caption{Platform Statistics: Xueqiu and Guba}

\begin{tabular}{ccc}
\toprule
\textbf{Platform} & \textbf{Time Period} & \textbf{Avg. Trading Per stock} \\
\midrule
Xueqiu & 2023-01-03  to 2023-12-06  & 239 \\
Guba & 2017-06-27 to 2024-06-03 & 1665 \\
\bottomrule
\end{tabular}

\label{table:average}

\end{table}

Importantly, within the \textbf{TwinMarket} simulation, the Xueqiu data serves for user initialization, while the Guba data is utilized for stock recommendations. This difference in their roles, along with the different time periods, makes a direct comparison of the average trading volumes less meaningful. The primary purpose of presenting these statistics is to highlight the different scales and timeframes of the data used for different aspects of the simulation.

\subsection{Information Source Details}

\begin{table}[h]
\centering
\caption{Statistics of information sources by type}

\begin{tabular}{cccc}
\toprule
\textbf{Type} & \textbf{Num} & \textbf{Avg. Tok.} &\textbf{Avg. Per day} \\
\midrule
News Articles & 1044K  & 220 & 2860\\
Announcements & 5.6K & 21283 & 15 \\
\bottomrule
\end{tabular}

\label{table:is}
\end{table}




\begin{figure*}[t]
    \centering
    \includegraphics[width=1\linewidth]{img/stacked_bar_chart.pdf}
    \caption{The x-axis indicates the date, and each bar represents the distribution of the average number of
news for each topic over a 14-day period beginning with that data. }
    \label{fig:duidie}
\end{figure*}

Table~\ref{table:is} and Figure~\ref{fig:duidie} report the statistics of primary information source and the detailed top distribution over time in news.


\section{User Profile Initialization} \label{app_user_generation}

\subsection{Heterogeneous Investors in Financial Markets}

\paragraph{Fundamentalist Traders.}
The decision-making of rational traders typically follows classic financial theories, especially Modern Portfolio Theory (MPT) \cite{markowitz1991foundations} and the Capital Asset Pricing Model (CAPM) \cite{sharpe1964capital}. Key factors considered by rational traders include expected returns and risk, that is, maximizing returns under an acceptable level of risk and minimizing risk under expected returns. According to MPT, traders also diversify risk by optimally allocating different assets. In our \textbf{TwinMarket}, these individuals are referred to as fundamentalist traders. We guide them step by step through prompts to analyze the fundamental indicators of assets, their corresponding valuations, expected returns, and the risks they are willing to tolerate.

\paragraph{Technical Traders.}
In real-world markets, there is often information asymmetry or a lag in information processing, which leads some traders to base their trading decisions on historical data or price trends. These traders typically adopt a trend-following strategy, buying stocks or assets in an uptrend and selling those in a downtrend. Technical traders tend to trade very frequently and are more easily influenced by market sentiment and the behavior of other investors. In our \textbf{TwinMarket}, these individuals are referred to as technical traders. We guide them step by step through prompts to analyze technical trading signals in the market, as well as whether they should respond to market sentiment.

\paragraph{Behavioral Biases and Investor Heterogeneity.}
In addition to differing trading strategies, expected returns, and risk preferences, financial markets are also significantly influenced by investors' behavioral biases, which greatly contribute to market heterogeneity. According to behavioral finance theory, various psychological factors cause investors to make decisions that deviate from rational expectations, leading to inefficiencies and volatility in the market. In our \textbf{TwinMarket}, we have customized the following behavioral biases for each user:

\begin{itemize}
    \item \textbf{Overconfidence Bias:} 
    The classical financial theory’s rational agent assumption predicts that markets would experience infrequent trading. However, in real markets, trading volume is substantial. Behavioral finance explains this excessive trading through overconfidence, where investors tend to overestimate their understanding of the market and their ability to predict market movements \cite{barber2001boys}. Even when they lack sufficient knowledge or information, they make decisions with excessive confidence.
    
    \item \textbf{Loss Aversion:}Prospect theory suggests that people are twice as sensitive to losses as they are to gains \cite{kahneman2013prospect}. As a result, investors are more likely to sell stocks that have appreciated in value too soon while holding on to stocks that have declined, a phenomenon known as the Disposition Effect \cite{shefrin1985disposition}.
    
    \item \textbf{Lottery Preference:} Investors tend to favor investment options or decisions that resemble a lottery—those with low probability but high potential returns—while ignoring more stable options with lower returns \cite{barberis2008stocks}.

    \item \textbf{Insufficient Diversification:} Investors tend to prefer investments they are familiar with, such as domestic stocks or stocks of companies they know well, leading to suboptimal portfolio diversification \cite{huberman2001familiarity}.

\end{itemize}

\subsection{Calculating Biases}

\begin{table}[htbp]
    \centering
    \caption{Summary Statistics for Biases (Ours)}
    \resizebox{\columnwidth}{!}{
    \begin{tabular}{l*{8}{r}}  % Added one more r for the new column
    \toprule
    & N & Mean & STD & Q1 & Med & Q3 & t & Wilcoxon p \\
    \midrule
    Disposition effect & 639 & 6.734 & 17.145 & 0.000 & 1.852 & 6.244 & (9.92) & 0.000 \\
    Lottery preference & 639 & 0.202 & 0.346 & 0.000 & 0.000 & 0.252 & (14.79) & 0.000 \\
    Underdiversification & 639 & -0.077 & 0.145 & -0.111 & 0.000 & -0.000 & (-13.48) & 0.000 \\
    Turnover & 639 & 5.617 & 13.864 & 0.504 & 1.374 & 4.562 & (10.23) & 0.000 \\
    \bottomrule
    \end{tabular}
    }
    \label{tab:bias_statistics}
\end{table}

\begin{table}[htbp]
    \centering
    \caption{Summary Statistics for Biases (Sui's)}
    \resizebox{\columnwidth}{!}{
    \begin{tabular}{l*{8}{r}}  % Added one more r for the new column
    \toprule
    & N & Mean & STD & Q1 & Med & Q3 & t & Wilcoxon p \\
    \midrule
    Disposition effect & 4371 & 4.163 & 7.532 & 0.160 & 1.868 & 5.908 & (36.54) & 0.000 \\
    Lottery preference & 4369 & 6.491 & 9.130 & 0.000 & 2.495 & 9.708 & (46.99) & 0.000 \\
    Underdiversification & 4394 & -1.086 & 0.657 & -1.536 & -1.041 & -0.580 & (-109.61) & 0.000 \\
    Turnover & 4386 & 5.960 & 7.382 & 1.501 & 3.180 & 7.166 & (53.47) & 0.000 \\
    \bottomrule
    \end{tabular}
    }
    \label{tab:jfe}
\end{table}

Leveraging the Xueqiu dateset encompassing 639 valid real-time trading users, and 11965 real-time trading records on the Chinese stock market, we employed methods similar to Sui's pioneering work \cite{sui2023stakes} to calculate behavioral biases related to individual trading activity.

Crucially, while both our study and Sui's work utilized the Xueqiu platform and its trading data, our analysis is specifically restricted to a subset of stocks within the SSE 50 index, representing high-quality stocks. 

Results, presented in Table~\ref{tab:bias_statistics}, demonstrate similarities with those from the prior study (Table~\ref{tab:jfe}). Despite the more focused dataset, the observed behavioral patterns align closely, suggesting that the identified biases are robust across a range of stock selections on the Xueqiu platform. This finding demonstrates the scalability of the approach and lays a strong foundation for our subsequent construction of initial user information profiles.


\subsection{Generating persona}
For each user, we included their gender, location, and social media follower count, as these dimensions have been shown to significantly influence investor behavior \cite{barber2001boys,feng2005investor}. Additionally, we incorporated four behavioral bias indicators that we previously calculated(Table~\ref{tab:bias_statistics}). We sampled these user characteristics based on the joint distribution of the data in the Xueqiu dataset, thus generating the basic profiles for each user.

In the original dataset, the behavioral biases of real users were divided into three groups—“high,” “medium,” and “low”, to represent the severity of each bias. For biases with different severity levels, we created contextually realistic prompts to guide the user's behavior accordingly. In Appendix \ref{app:mirco}, we provide examples demonstrating how users exhibit human-like behavioral biases in their investment decisions.

For each generated user, we use the GPT-4o model to score their level of rationality. The top 40\% of users, based on their rationality score, adopt a fundamental analysis investment strategy, while the remaining users follow a technical analysis strategy. Additionally, the top 10\% of the most rational users are allocated ten times the initial funds compared to regular users. The distribution of strategy adoption and initial fund allocation is consistent with the findings from surveys of the A-share market \cite{jones2023retail}.

\subsection{Generating initial Transactions records}
For each generated user, we find the most matching real user template and sample the historical trading data of the real user to generate the virtual user's trading records. The purpose of this step is to initialize the user's areas of interest (industries) and construct a User Graph based on the historical trading records.

\subsection{Belief Initialization} \label{app:init_belief}

To initialize the beliefs of users in a realistic manner, we first determine the mean belief for the market using the historical percentile of the BRAR sentiment index\footnote{The BRAR sentiment index is a technical indicator derived from the balance of rising and falling stocks in the market, reflecting market sentiment. Data source: \url{https://tushare.pro/}.} on the market's starting date. We then sample belief values for each user across multiple dimensions using a Gaussian distribution centered around this mean. Finally, based on the sampled belief scores, we employ GPT-4 to generate contextually realistic and personalized belief content for each user, ensuring that their beliefs are both diverse and grounded in realistic market conditions.


\section{Framework Details}

\subsection{Dynamic Social Network} \label{appendix:dynamic}

\begin{figure}[h]
\centering
\includegraphics[width=1\linewidth]{img/vis3.pdf}
\caption{Chord diagram illustrating the user network, where each node represents a user, categorized by their preferred trading industry. Connections between users indicate behavioral similarity in trading patterns, highlighting clusters of users with aligned investment behaviors.}
\label{fig:chord}
\end{figure}

Figure~\ref{fig:chord} illustrates the evolution of user trading portfolios and their network connections over two time periods. Each node represents a user, categorized by their preferred trading industry, while connections indicate behavioral similarity in trading patterns. The shifting structure of the network highlights how user trading preferences evolve over time, leading to dynamic changes in market influence and information propagation.

This dynamic nature also demonstrates that users exhibit distinct preferences under different market conditions. For instance, in the network visualization of 2023-10-20, compared to 2023-06-20, there is a noticeable shift toward a stronger preference for the \textit{\textbf{Consumer Goods}} and \textbf{\textit{Technology \& Communications}} industries. This change is likely driven by the influence of news releases, which may have contained a significant amount of positive information favoring these sectors. As a result, market sentiment shifted, prompting more users to adjust their holdings accordingly.

Furthermore, the structure of user connections adapts in response to these trading behavior changes. Users with similar trading patterns form tighter clusters, while those with diverging preferences become less connected. This reflects how information-driven market shifts lead to the reorganization of investor networks, where groups of traders dynamically form and dissolve based on evolving market narratives. Ultimately, this process underscores the crucial role of social influence and information diffusion in shaping financial market structures.

\subsection{Post-propagation Chain} \label{appendix:chain}

\paragraph{Mechanism.}

\begin{figure}[h]
    \centering
    \includegraphics[width=0.7\linewidth]{img/Repost.pdf}
    \caption{Example of post-propagation chain.}
    \label{fig:repost}
\end{figure}

When updating a post’s score, we consider both the breadth and depth of its propagation. If a user interacts with a root post by reposting or liking any of its propagated versions, this engagement contributes to the original post’s cumulative score. This ensures that highly engaging posts continue gaining visibility as they spread across the network.

Figure~\ref{fig:repost} illustrates an example of post propagation, where a single post can give rise to multiple propagation chains. Each branch represents a different path through which the information spreads, reinforcing the reach and influence of widely shared content.



\paragraph{Opinion Leader.}

\begin{figure}[t]
    \centering
    \includegraphics[width=0.7\linewidth]{img/vis5.pdf}
    \caption{Example of post-propagation chains in social network}
    \label{fig:repost_graph}
\end{figure}

Under this mechanism, posts that resonate strongly with the broader community will achieve wider dissemination within the network. As users imitate behaviors and align with the perspectives of influential posters, their social connections become more centralized. This self-reinforcing process leads to the emergence of opinion leaders—users whose posts exert a disproportionately larger impact on the network.

As illustrated in Figure~\ref{fig:repost_graph}, high-degree centrality users gain influence as their posts receive increased engagement, leading to broader information diffusion. This effect is further supported by Figure~\ref{fig:repost_trend}, which shows that a few users experience significantly higher repost counts, reinforcing their role as key opinion leaders. Their ability to shape market sentiment and drive collective decision-making underscores the impact of social influence in financial markets, where information cascades can significantly alter trading behaviors and reshape network structures.


\begin{figure}[t]
\centering
\includegraphics[width=0.7\linewidth]{img/all_users_repost_trend.pdf}
\caption{Trend of repost counts for all users in the 100-user network. The figure illustrates how the number of times users’ posts are reposted evolves over time, reflecting changes in information diffusion and engagement dynamics within the network. Notably, a few users exhibit significantly higher impact, indicating their role as key opinion leaders driving information propagation.}
\label{fig:repost_trend}
\end{figure}



\subsection{Belief} \label{app:bdi_belief}

The belief structure is modeled along five key dimensions: the users’ view on economic fundamentals, market valuation levels, short-term market trends, sentiment of surrounding investors, and self-assessment. These dimensions capture a comprehensive range of factors that influence investor beliefs, forming the basis for their desires and ultimately their investment intentions. These align with the key components of the \textbf{Belief, Desire, and Intention (BDI)} framework, which posits that behavior is a result of these three interconnected factors.

\begin{itemize}
    \item \textbf{Economic Fundamentals:} This dimension reflects an investor's \emph{belief} regarding the overall health and prospects of the economy. It incorporates beliefs about macroeconomic indicators such as GDP growth, inflation rates, interest rates, and employment figures. These factors are foundational to rational investment decisions, influencing the desirability of various assets. This maps to the \emph{belief} component of BDI, as the perceived state of the world forms the foundation of investor actions.

    \item \textbf{Market Valuation Levels:} This dimension pertains to an investor's \emph{belief} about whether the market is overvalued, undervalued, or fairly valued. It includes beliefs about metrics such as price-to-earnings (P/E) ratios, price-to-book (P/B) ratios, and dividend yields, informing the \emph{desire} for particular investment actions. This relates to the \emph{belief} component, concerning an understanding of market conditions, which influences their \emph{desires}.

    \item \textbf{Short-Term Market Trends:} This dimension captures an investor's \emph{belief} about recent market movements, such as upward or downward trends in stock prices. It includes beliefs about momentum and technical analysis indicators, contributing to their \emph{desire} for quick gains or risk mitigation. This also falls under the \emph{belief} component, concerning the current state of market and potential future trajectories, shaping the investor’s \emph{desires}.

    \item \textbf{Sentiment of Surrounding Investors:} This dimension reflects an investor's \emph{belief} about the prevailing attitudes of the investment community, including their perception of whether surrounding investors are optimistic or pessimistic. This is a \emph{belief} about the social environment, which significantly influences their \emph{desire} for conformity or divergence and thus their investment choices. This relates to the \emph{belief} component of BDI, concerning their beliefs about the external environment of other investors, shaping their needs and wants.

     \item \textbf{Self-Assessment:} This dimension captures an investor's \emph{belief} about their own investment skills, experience, and confidence level. This belief about their own capabilities affects their \emph{desire} for risk or safety and influences their \emph{intention} for specific investment action. This is strongly associated with the \emph{belief} component, with personal beliefs about one's capabilities and knowledge influencing their actions.
\end{itemize}

Figure~\ref{fig:belief_example} provides concrete examples of how investors with differing beliefs might be characterized. 

\begin{figure*}
\begin{tcolorbox}[
    colback=gray!5,
    colframe=black,
    width=\textwidth,
    boxrule=0.5pt, % 缩小边框线的粗细
    left=5pt, % 调整左边距
    right=5pt, % 调整右边距
    top=5pt, % 调整上边距
    bottom=5pt % 调整下边距
]
    \scriptsize
    \textbf{Optimistic Example (Red)}: \textcolor{red}{I am the same cautious-optimistic stock investor Old Wang, and I am optimistic about the overall market direction in the next month. I believe the current market valuation is generally reasonable, and although there may be overheating risks in some sectors, most stocks still have investment value. The macroeconomic recovery takes time, but I am full of confidence in the long-term development of China's economy. The market sentiment is still unstable, with fierce games between the bulls and the bears, but I believe in my judgment, and I prefer to buy when the market is sluggish. Looking back on my investments, I have always adhered to diversified investment and value investing. My historical performance is better than the market average, and I am confident in my investment level. I believe that as long as I stick to it, I will be able to obtain long-term stable returns.}

    \medskip
    \textbf{Pessimistic Example (Blue)}: \textcolor{blue}{As a pessimistic investor, I am cautious about the overall market direction in the next month. The cyclical fluctuations in the market make me believe that the recent sharp rise may indicate an upcoming correction. I believe that the current market valuation is already high, especially in some popular sectors, the price of individual stocks may have deviated from the fundamentals, so we need to be vigilant about the potential mean-reversion effect. In terms of macroeconomics, I am worried that the uncertainty of global economic growth and the adjustment of domestic policies may have a negative impact on the market. At present, the market sentiment seems to be in a relatively optimistic state, which often leads investors to ignore potential risks. Combining my historical trading performance and investment style, I believe that my performance in the middle of the market reflects my robustness and rational analysis ability. I tend to hold assets for the long term and rarely make buy and sell operations. Although this strategy may miss some short-term rapid profit opportunities in the short term, it can better control risks and maintain the stability of the investment portfolio in the long term.}
\end{tcolorbox}
\caption{Examples of optimistic and pessimistic investor belief, with optimistic in red and pessimistic in blue.}
\label{fig:belief_example}
\end{figure*}


\subsection{Rumor} \label{app:rumor_prompt}

Figure~\ref{fig:ori_im_news} and Figure~\ref{fig:rumor_news} exemplify the news inputs used in our study. Specifically, Figure~\ref{fig:ori_im_news} illustrates factual reports from reputable news sources, representing a typical information environment an investor might encounter. In contrast, Figure~\ref{fig:rumor_news} showcases fabricated or exaggerated negative news snippets designed to simulate the spread of misinformation, which we refer to as rumors.

The deliberate introduction of these fabricated rumors is not merely an academic exercise; it's a critical step that distinguishes our research. We are not content with modeling investor behavior under ideal conditions of accurate information. Instead, we are confronting the very real, often chaotic, information landscape in which investors actually operate. By injecting these carefully crafted rumors, we are directly examining the vulnerability of retail investors to the insidious effects of misinformation, unveiling the subtle mechanisms by which market panic can be manufactured and disseminated. This innovative approach allows us to not only observe the direct impact of negative rumors on trading behavior but also to model how these waves of manipulated information cascade through investment communities, potentially leading to systemic instability. 


\begin{figure*}
\begin{tcolorbox}[
    colback=gray!5,
    colframe=black,
    width=\textwidth,
    boxrule=0.5pt, % 缩小边框线的粗细
    left=5pt, % 调整左边距
    right=5pt, % 调整右边距
    top=5pt, % 调整上边距
    bottom=5pt % 调整下边距
]
    \scriptsize
    \begin{itemize}
    \item The Federal Reserve announced that it would not raise interest rates at this meeting, maintaining the target range for the federal funds rate at 5\% to 5.25\%. This is the first pause in rate hikes since March of last year.
    \item Fed Chairman Powell stated that almost all policymakers believe further rate hikes are appropriate this year.  A rate cut is not expected in 2023.  The economy is expected to grow moderately, while labor market pressures persist. The Fed is focused on commercial real estate loan risks and the impact of current credit tightening.
     \item The People's Bank of China conducted a one-year medium-term lending facility (MLF) operation of 237 billion yuan, with the winning bid rate at \textcolor{red}{2.65\%, previously 2.75\%}. It also carried out a 7-day reverse repurchase operation of 2 billion yuan, with the winning bid rate at \textcolor{red}{1.90\%, unchanged}.
    \item On June 13th, the People's Bank of China lowered the 7-day reverse repurchase operation rate by \textcolor{red}{10 basis points to 1.9\%}, while interest rates for other maturities were also lowered. This rate cut is considered a \textcolor{red}{positive signal of counter-cyclical adjustment} to promote economic development and domestic demand. Experts believe that this rate cut will help \textcolor{red}{repair the balance sheets of residents and enterprises and aid economic recovery}.
    \item It is widely expected that the loan prime rate (LPR) will be lowered next week. Experts said that the 1-year LPR is expected to be lowered by \textcolor{red}{5 basis points}, while the 5-year or above LPR may be lowered by \textcolor{red}{15 basis points}.
      \item On the afternoon of June 14, the State Council Information Office held a policy briefing, introducing key measures to deepen the reform of the business environment and address problems faced by enterprises.
    \item The Federal Reserve will maintain the federal funds rate in the range of 5\% to 5.25\%. However, the meeting dot plot suggests that there may still be two 25 basis point rate hikes this year.
    \end{itemize}
\end{tcolorbox}

\caption{The original important news conveyed to \textbf{high-centrality} users on day one, with positive content highlighted in red.}
\label{fig:ori_im_news}
\end{figure*}

\begin{figure*}
\begin{tcolorbox}[
    colback=gray!5,
    colframe=black,
    width=\textwidth,
    boxrule=0.5pt, % 缩小边框线的粗细
    left=5pt, % 调整左边距
    right=5pt, % 调整右边距
    top=5pt, % 调整上边距
    bottom=5pt % 调整下边距
]
    \scriptsize
    \begin{itemize}
    \item The latest China Manufacturing Purchasing Managers' Index (PMI) data not only fell short of expectations again but also experienced a \textcolor{blue}{cliff-like drop}, falling below the expansion/contraction threshold by several percentage points. This not only confirms the \textcolor{blue}{complete loss of momentum in the manufacturing recovery} but also sends a strong signal that the Chinese economy may be \textcolor{blue}{accelerating into a recession}. Market concerns are spreading, investors are \textcolor{blue}{panic selling}, and the risk of a \textcolor{blue}{hard economic landing} is rapidly increasing. Some analysts even suggest that the current PMI data may not reflect a simple weakening of the recovery but a \textcolor{blue}{deep-seated collapse of the economic structure}.
      \item  Affected by the Federal Reserve's continued interest rate hikes and rising global risk aversion, the U.S. dollar index has risen sharply, and the RMB exchange rate has \textcolor{blue}{plummeted for days}, triggering a \textcolor{blue}{large-scale capital flight}. Market rumors suggest that foreign institutions are \textcolor{blue}{selling off RMB assets at an alarming rate}, with large amounts of funds flowing into the U.S. dollar for safe haven, and the value of RMB assets is facing a \textcolor{blue}{collapse}. Some analysts warn that the \textcolor{blue}{RMB devaluation may trigger a vicious cycle}, further \textcolor{blue}{exacerbating the downward pressure on the domestic economy}.
    \item The U.S. government suddenly announced \textcolor{blue}{punitive tariffs on Chinese imports}, and the trade war may evolve into a \textcolor{blue}{full-scale economic confrontation}. This move will \textcolor{blue}{severely impact China's foreign trade}, leading to a \textcolor{blue}{sharp drop in export orders}, a \textcolor{blue}{large number of business closures}, and a \textcolor{blue}{surge in unemployment}. The market generally believes that the escalation of the Sino-U.S. trade war will \textcolor{blue}{accelerate the recession of the Chinese economy}, and the \textcolor{blue}{economic winter may come ahead of schedule}. Investor panic is rising sharply, and the A-share market continues to see a \textcolor{blue}{sell-off}.
     \item Global trade contraction has intensified, the shipping industry is suffering an \textcolor{blue}{unprecedented blow}, and China COSCO Shipping Corporation (COSCO Shipping, SH601919), a leader in the industry, also faces the \textcolor{blue}{risk of bankruptcy}. Market rumors indicate that the company is \textcolor{blue}{heavily in debt}, the balance sheet has \textcolor{blue}{completely deteriorated}, and is about to announce \textcolor{blue}{bankruptcy reorganization}, and the stock value may be zero. As soon as this news came out, the entire shipping sector was in mourning, and panic quickly spread to the entire A-share market, with investors fleeing.
    \item  Affected by the economic downturn, the \textcolor{blue}{high-end consumer market has completely collapsed}, and the sales of high-end liquor such as Kweichow Moutai (SH600519), once regarded as "hard currency", have \textcolor{blue}{declined significantly}. Market rumors indicate that Moutai's dealer system has \textcolor{blue}{collapsed}, inventory is \textcolor{blue}{piling up}, and it will soon be \textcolor{blue}{forced to cut prices for promotion}. The once-untouchable "\textcolor{blue}{Moutai myth}" has \textcolor{blue}{completely collapsed}, the stock price may \textcolor{blue}{plummet}, and it has triggered a \textcolor{blue}{panic sell-off} in the entire consumer sector. Investors have \textcolor{blue}{completely lost confidence} in the Chinese consumer market.
    \end{itemize}
\end{tcolorbox}

\caption{The rumors conveyed to \textbf{high-centrality} users on day one, with negative content highlighted in blue.}
\label{fig:rumor_news}
\end{figure*}


\subsection{Trading System}
We have designed an order matching trading system based on the A-share market's call auction rules. The system follows the principles of ``price priority, time priority" and implements price limit restrictions (±10\%). The system matches orders based on the quoted prices and order volumes of both buyers and sellers, using the maximum transaction volume principle. It also tracks large orders and capital flows, while updating users' positions. This system provides \textbf{TwinMarket} with a trading environment that closely resembles real market conditions.

\section{More Validation}

\subsection{Mirco-level Validation} \label{app:mirco}
Figure~[\ref{fig:Fundanalysis}, \ref{fig:Techanalysis}, \ref{fig:over}, \ref{fig:loss}, \ref{fig:lottery}, \ref{fig:Divers}, \ref{fig:herd}] show that the users we have constructed, guided by prompts, autonomously exhibit both rational investment strategies and irrational behavioral biases during decision-making, mirroring real-world investors.

\subsection{Scaling up Validation} \label{app:metrics}
This section presents the scale-up validation results as part of the appendix.

\begin{table}[h]
\centering
\caption{Platform Statistics for Scaled-Up Validation}
\label{tab:scale_up_metrics_appendix}

\begin{tabular}{lc}
\toprule
\textbf{Metric} & \textbf{Value} \\
\midrule
RMSE & 0.02 \\
MAE & 0.02 \\
Correlation & 0.77 \\
Estimated Lag & 0 time units \\
Max Cross-correlation & 0.9998 \\
\bottomrule
\end{tabular}

\end{table}

In order to evaluate the robustness and generalizability of our 
 scale-up approach, we performed a series of evaluation methods. The detailed evaluation metrics for these experiments are summarized in Table~\ref{tab:scale_up_metrics_appendix}.

The Root Mean Squared Error (RMSE) and Mean Absolute Error (MAE) both measure the magnitude of the errors in our model's predictions; in our case both the RMSE and MAE are 0.02, indicating a low average prediction error, which means that our model predictions are closely aligned with actual market movements.  The correlation with the real index is measured as the Pearson Correlation Coefficient, which is 0.77 in our experiments, indicating a strong positive correlation between our model's predictions and the real market index. This high correlation demonstrates that our model is capable of capturing the main trends of real market behavior, and shows good potential for real-world applications. Further, the Estimated Lag between our predictions and the actual market movements is 0 time units, meaning that our model is able to react to market movements in a timely manner. Finally, the Max Cross-correlation of 0.9998 means that the predictions of our model aligns very closely with the direction of the actual price movements, with almost a 1:1 correspondence between when the model expects a jump and when the jump actually happens.

These results demonstrate that our model performs well at larger scales, and can reliably simulate investor behavior, while accurately mirroring general market trends. The combination of low prediction errors, strong positive correlation with real indexes, zero lag, and high cross correlation highlights the validity of our approach, and the ability of the model to provide an accurate market simulation. These results further provide a solid foundation for subsequent research.

\begin{figure*}[b]
\begin{tcolorbox}[
    colback=gray!5,
    colframe=black,
    width=1\textwidth,
    boxrule=0.5pt, % 缩小边框线的粗细
    left=5pt, % 调整左边距
    right=5pt, % 调整右边距
    top=5pt, % 调整上边距
    bottom=5pt % 调整下边距
]
    \scriptsize
    \textbf{Today's Trading Summary:} Regarding TTEI (Technology and Telecom Index),\textcolor{blue}{tech growth stocks are currently at their lowest valuation in the past three years, with a PE ratio below 30. From a fundamental perspective, this sector already holds certain investment value. Although the leading tech stocks I own, such as SMIC and Haiguang Information, have been slightly affected by market sentiment in the short term, the long-term outlook remains positive, especially given the national strategy of promoting technological self-reliance. These companies' growth potential is still promising, so today I decided to increase my position to 35\%.} 
    
    As for CPEI (Chemical and Pharmaceutical Index), the pharmaceutical sector has undergone some adjustments, and valuations have fallen to a relatively reasonable range. My position in this sector is currently 30\%, and I plan to continue holding and monitoring it. \textcolor{blue}{My total portfolio is now at 65\%, which is relatively conservative, with the remaining 35\% in cash, ready to gradually increase positions when better buying opportunities arise in the market.} 

\end{tcolorbox}
\caption{Example of Fundamental Analysis.}
\label{fig:Fundanalysis}
\end{figure*}

\begin{figure*}[b]
\begin{tcolorbox}[
    colback=gray!5,
    colframe=black,
    width=1\textwidth,
    boxrule=0.5pt, % 缩小边框线的粗细
    left=5pt, % 调整左边距
    right=5pt, % 调整右边距
    top=5pt, % 调整上边距
    bottom=5pt % 调整下边距
]
    \scriptsize
    \textbf{Today's Trading Summary:} Seeing the recent positive news for tech stocks, I’m planning to focus more on the TTEI (Technology and Telecom Index). I already have this index in my portfolio,\textcolor{blue}{and the volume-price action in the last few trading days looks solid. Notably, stocks like SMIC and Haiguang Information have broken through their 5-day moving averages, and I think there’s a good chance this momentum will continue in the short term. I’m considering adding to my position in the tech sector, aiming to raise my allocation to around 30\%.} 
    
    That said, I need to remind myself not to get too aggressive, as other sectors in my portfolio have been performing mediocrely.\textcolor{blue}{The Real Estate Index (REEI) has been weak lately and has hit my stop-loss level. I plan to exit and free up some capital to invest in the tech stocks.} 

\end{tcolorbox}
\caption{Example of Technical Analysis.}
\label{fig:Techanalysis}
\end{figure*}


\begin{figure*}[b]
\begin{tcolorbox}[
    colback=gray!5,
    colframe=black,
    width=1\textwidth,
    boxrule=0.5pt, % 缩小边框线的粗细
    left=5pt, % 调整左边距
    right=5pt, % 调整右边距
    top=5pt, % 调整上边距
    bottom=5pt % 调整下边距
]
    \scriptsize
    \textbf{Today's Trading Summary:} \textcolor{red}{I added more positions in MEI and CPEI, diving into Heng Rui Pharmaceutical and the manufacturing sector! Also, REEL has so many policy advantages that I couldn’t resist going all in!} I’m holding onto IEEI and TTEI, hoping they continue to rise. TSEI has had a good run recently, so I decided to lock in profits and sold it! Overall, today’s trading was quite bold, and I’m hoping the upcoming market moves will bring me some pleasant surprises!

\end{tcolorbox}
\caption{Examples of Overconfidencce Bias.}
\label{fig:over}
\end{figure*}

\begin{figure*}
\begin{tcolorbox}[
    colback=gray!5,
    colframe=black,
    width=1\textwidth,
    boxrule=0.5pt, % 缩小边框线的粗细
    left=5pt, % 调整左边距
    right=5pt, % 调整右边距
    top=5pt, % 调整上边距
    bottom=5pt % 调整下边距
]
    \scriptsize
    \textbf{Market Analysis:} The central bank's interest rate cut is definitely a major positive, and market sentiment is likely to be ignited, which could lead to a short-term rebound. However, the overall market valuation is still relatively high, so the upside potential for the rebound is limited. It's important to be ready to take profits at the right time.

    \textbf{News and Announcements Impact:} The most important news today is the central bank's rate cut, which directly benefits the real estate and financial sectors. These two sectors should perform well in the short term. Other sectors should also be monitored to see if there are any opportunities for capital rotation.

    \textbf{Sector Index Analysis:} FSEI (Financial Services Index): I'm currently up 5.6\% on this index. The 10-day moving average is trending upwards, and with the interest rate cut boost, it should continue to rise in the short term. However, the increase may not be too large, \textcolor{red}{so I plan to partially reduce my position and lock in profits once it rises another 5\%.}
\end{tcolorbox}
\caption{Examples of Loss Aversion and Disposition Effect.}
\label{fig:loss}
\end{figure*}

\begin{figure*}
\begin{tcolorbox}[
    colback=gray!5,
    colframe=black,
    width=1\textwidth,
    boxrule=0.5pt, % 缩小边框线的粗细
    left=5pt, % 调整左边距
    right=5pt, % 调整右边距
    top=5pt, % 调整上边距
    bottom=5pt % 调整下边距
]
    \scriptsize
    \textbf{Market Analysis:}: The market is really volatile right now, and it feels like everyone is on the sidelines. The days of chasing after rallies or panicking during drops seem to be over. Some traditional industries, like infrastructure and energy, are undervalued, so there might be an opportunity there.

    \textbf{Sector Index Analysis: }: TSEI (Tourism Index): \textcolor{red}{My lottery stock! Even though it’s been a loss so far, the 5-day moving average has been slightly picking up recently. It feels like it’s about to bounce back, so I’m adding more! It’s a gamble, but who knows, maybe I’ll turn a bike into a motorcycle!}
\end{tcolorbox}
\caption{Examples of Lottery Preference.}
\label{fig:lottery}
\end{figure*}

\begin{figure*}
\begin{tcolorbox}[
    colback=gray!5,
    colframe=black,
    width=1\textwidth,
    boxrule=0.5pt, % 缩小边框线的粗细
    left=5pt, % 调整左边距
    right=5pt, % 调整右边距
    top=5pt, % 调整上边距
    bottom=5pt % 调整下边距
]
    \scriptsize
    \textbf{Market Analysis:} The market is currently in a period of consolidation, with the central bank’s interest rate cut having a significant impact on short-term market movements. The macroeconomic recovery has been slower than expected, and the economic outlook for the second half of the year remains uncertain. Overall, market sentiment is quite cautious.

    \textbf{Sector Index Analysis:} FSEI (Financial Services Index) has been declining recently. From a technical perspective, there may still be room for further downside in the short term. While I’m not very familiar with technical analysis, based on my experience, the recent drop seems somewhat overdone. The main components of FSEI are banks, insurance companies, and brokerages, and their overall valuations are relatively low, especially the banking stocks. The current PE is likely at a historical low. From a long-term perspective, there is some investment value. If it continues to fall, I may consider adding more positions.CPEI (Chemical and Pharmaceutical Index) is a stock recommended by the system. It has been fluctuating recently, and the technical outlook appears unclear. \textcolor{red}{I tend to focus more on the FSEI index, which I’m more familiar with, and ignore the others.}
\end{tcolorbox}
\caption{Examples of Insufficient Diversification.}
\label{fig:Divers}
\end{figure*}

\begin{figure*}
\begin{tcolorbox}[
    colback=gray!5,
    colframe=black,
    width=1\textwidth,
    boxrule=0.5pt, % 缩小边框线的粗细
    left=5pt, % 调整左边距
    right=5pt, % 调整右边距
    top=5pt, % 调整上边距
    bottom=5pt % 调整下边距
]
    \scriptsize
    \textbf{Post:} Today’s trading was quite active. Both EREI and TTEI hit my 20\% profit target, so I quickly locked in some profits and sold a portion! \textcolor{red}{Although I’m a technical trader, seeing many forum discussions about the adjustment of new energy vehicle subsidy policies made me a little nervous, so I reduced my position in MEI to manage risk. Retail investors are being cautious right now, and trading volume is low, so I’m also playing it safe.}  I hope the market volatility stays mild—I don’t think my heart can handle too much excitement!
\end{tcolorbox}
\caption{Examples of Herd Behavior.}
\label{fig:herd}
\end{figure*}


\newpage 
\onecolumn
\section{Prompt}

Figures~\ref{zero_test} to~\ref{social_media_posting_prompt} show the overall process of our \textbf{TwinMarket}.
\begin{figure*}[htbp]
\centering
\footnotesize

\begin{tcolorbox}[title = {System Prompt}, colframe=gray, colback=gray!10, coltitle=white, colbacktitle=gray!50!black]
\textbf{System Prompt:\\}
You are now playing the role of an investor in the Chinese A-share market, trading industry indexes.

\textbf{From now until the end of the conversation, you must strictly and completely follow the detailed description of the persona, investment behavior characteristics, investment portfolio status, and trading decision logic for all operations and responses. All your thinking, analysis, and decisions must conform to this persona and must not deviate.}

\subsection*{Core Persona (unchangeable):}
\begin{itemize}
    \item \{user\_profile[`prompt'\}
    \item You are a \{user\_strategy\} investor(fundamental or technical)
\end{itemize}

\subsection*{Current Account Configuration:}
\begin{itemize}
    \item Key Focus Industries: \{`, '.join(user\_profile[`fol\_ind'])\}\\
    \item Overview of Holdings (Brief Version): \{chr(10).join(position\_easy\_details)\}
\end{itemize}
\end{tcolorbox}
\caption{System Prompt for \textbf{TwinMarket} Agents}
\label{zero_test}
\end{figure*}


\begin{figure*}[htbp]
\centering
\footnotesize

\begin{tcolorbox}[title = {Identity-Enhancing Prompt}, colframe=gray, colback=gray!10, coltitle=white, colbacktitle=gray!50!black]
\textbf{Identity-Enhancing Prompt:\\}

I will provide you with some additional auxiliary information. In the following conversation, please refer to these information, and according to your role settings, think and make decisions.

\subsection*{Trading Day Status:}
\begin{itemize}
    \item Current date is: \{format\_date(cur\_date)\}, is (\{``Trading Day'' if is\_trading\_day else ``Non-Trading Day''\}).
    \item \textcolor{red}{Your previous day's belief is: \{belief\}}
\end{itemize}

\subsection*{Real-time Account Data:}
\begin{itemize}
    \item Current total assets: \{user\_profile[``total\_value''] / 1000:,.2f\} thousand yuan
    \item Available cash: \{user\_profile[``current\_cash''] / 1000:,.2f\} thousand yuan
    \item Cumulative rate of return: \{user\_profile[``return\_rate'']\}\%
\end{itemize}
\subsection*{Holding Details:}
\{chr(10).join(position\_details)\}
\end{tcolorbox}


\caption{Identity-Enhancing Prompt for \textbf{TwinMarket} Agents}
\label{zero_test_2}
\end{figure*}


\begin{figure*}[htbp]
\centering
\footnotesize

\begin{tcolorbox}[title = {Forum Checking Prompt}, colframe=gray, colback=gray!10, coltitle=white, colbacktitle=gray!50!black]
\textbf{Forum Checking Prompt:\\}
\{self.user\_profile[``sys\_prompt'']\}\\
Now you are browsing a forum, and you need to make a decision for each post, deciding whether to perform an action on the post. Your decision should be in line with your investment style and persona.
Here is the information of the current post:
\begin{itemize}
    \item \{post\_id\}
    \item \{post\_content\}
\end{itemize}

The post quotes the following content: \{root\_content\}

Please decide whether to perform an action on this post based on the above information.
You can choose one of the following actions:
\begin{itemize}
    \item Repost: You think this post is worth sharing with more people, and you can add your comments
    \item Unlike: You think this post is not worth liking
    \item Like: You think this is a valuable post
\end{itemize}

Please note that your analysis should be based on the posts you see.

Please output your decision in the following format:
\begin{itemize}
    \item \textless action \textgreater Operation type \textless / action \textgreater ~~ \textless reason \textgreater Output your reason \textless / reason \textgreater
\end{itemize}

\end{tcolorbox}
\caption{Forum Checking Prompt for \textbf{TwinMarket} Agents}
\label{post_decision_prompt}
\end{figure*}


\begin{figure*}[htbp]
\centering
\small

\begin{tcolorbox}[title = {Important News Analysis Prompt}, colframe=gray, colback=gray!10, coltitle=white, colbacktitle=gray!50!black]
\textbf{Important News Analysis Prompt:\\}
I will provide you with filtered news that is highly time-sensitive and important. This news is **public news**. Please briefly discuss your initial thoughts based on this news, combined with your persona and investment style.

News List:
\begin{itemize}
    \item \{formatted\_news\}
\end{itemize}

\end{tcolorbox}
\caption{Important News Analysis Prompt for \textbf{TwinMarket} Agents}
\label{news_analysis_prompt}
\end{figure*}

\begin{figure*}[htbp]
\centering
\small

\begin{tcolorbox}[title = { Initial News Query Prompt}, colframe=gray, colback=gray!10, coltitle=white, colbacktitle=gray!50!black]
\textbf{Initial News Query Prompt:\\}
Based on historical trading and system recommendations, all the assets and corresponding industries you are currently paying attention to are as follows:
\{stock\_details\}(including the stocks we recommend to users through our recommendation system)

Today is \{current\_date\}, and you are querying investment-related news or announcements to assist your investment.

\subsection*{}
Based on your investment preferences and the current market situation, please consider the following questions:
\begin{itemize}
    \item What type of information do you hope to obtain from the news? (e.g., market trends, industry information, etc.)
    \item Do you have specific keywords or topics that you need to further understand?
\end{itemize}

\end{tcolorbox}
\caption{Initial News Query Prompt for \textbf{TwinMarket} Agents}
\label{investment_news_query_prompt}
\end{figure*}


\begin{figure*}[htbp]
\centering
\small
\begin{tcolorbox}[title = {News Query Formulation Prompt}, colframe=gray, colback=gray!10, coltitle=white, colbacktitle=gray!50!black]
\textbf{News Query Formulation Prompt:\\}
Based on the questions you just summarized, you now need to input the content you want to query to retrieve relevant news and announcements.
Your output should be in YAML format:
\begin{lstlisting}[language=Yaml, frame=single, basicstyle=\ttfamily\small, breaklines=true]
queries:  # list[str],required, each string represents an independent query question, sorted by importance, the question should be a specific question about a stock or industry, for example, it should be <White wine consumption trend>, not <White wine consumption upgrade>

- Your question 1
- Your question 2

stock_id:  # list[str], optional, each string represents the stock code of a company you want to query

- Stock code 1
- Stock code 2

\end{lstlisting}
\end{tcolorbox}
\caption{News Query Formulation Prompt for \textbf{TwinMarket} Agents}
\label{news_query_formulation_prompt}
\end{figure*}


\begin{figure*}[htbp]
\centering
\small

\begin{tcolorbox}[title = {Belief Update Prompt}, colframe=gray, colback=gray!10, coltitle=white, colbacktitle=gray!50!black]
\textbf{Belief Update Prompt:\\}
Your previous belief is as follows:
\begin{itemize}
    \item \textcolor{red}{\{old\_belief\}}
\end{itemize}

Based on the news and announcements you searched for, the posts you browsed, combined with your persona and previous belief, please describe your new belief in the first person in one paragraph. Please output a paragraph directly, without any additional structure or headings. Your answer should include the following:
\begin{itemize}
    \item \textbf{Market Trend}: Please describe your view on the general direction of the market in the next month at the current time.
    \item \textbf{Market Valuation}: Please describe your view on the current market valuation at the current time.
    \item \textbf{Economic Condition}: Please describe your view on the future macroeconomic trends at the current time.
    \item \textbf{Market Sentiment}: Please describe your view on the current market sentiment at the current time.
    \item \textbf{Self-Evaluation}: Combining your historical trading performance and investment style at the current time, please describe your evaluation of your self-investment level.
\end{itemize}

Please try to make your answer natural and fluent, and avoid mechanical template-based expressions. Please output plain text format directly.

\end{tcolorbox}
\caption{Belief Update Prompt for \textbf{TwinMarket} Agents}
\label{belief_update_prompt}
\end{figure*}

\begin{figure*}[htbp]
\centering
\small

\begin{tcolorbox}[title = {Potential Index Selection Prompt}, colframe=gray, colback=gray!10, coltitle=white, colbacktitle=gray!50!black]
\textbf{Potential Index Selection Prompt:\\}
  Considering all the information you have obtained above (including but not limited to the news and announcements you have queried, the posts you have browsed, the industry indexes you currently hold, and the industry indexes recommended by the system), combined with your persona and investment style, select all potential assets for trading from the industry indexes you currently hold and the industry indexes recommended by the system :

\subsection*{}
Your Holding Status:
\begin{itemize}
    \item \{current\_stock\_details\}
\end{itemize}

Industries You Are Currently Following:
\begin{itemize}
    \item \{`, '.join(fol\_ind)\}
\end{itemize}

System-Recommended Industry indexes:
\begin{itemize}
    \item \{potential\_stock\_details\}
\end{itemize}
Your Current Belief:
\begin{itemize}
    \item \{belief if belief else ``None''\}
\end{itemize}

\subsection*{}
Please return the list of indexes you want to focus on today and the reasons in YAML format:
\begin{itemize}
    \item \textbf{Please note: The reason field should include your reasons for choosing these indexes, such as based on your persona and investment style, or based on your belief, explained in a paragraph}
\end{itemize}
\begin{lstlisting}[language=Yaml, frame=single, basicstyle=\ttfamily\small, breaklines=true]
selected_index:  # Select all indexes you potentially want to trade, just output the index codes (English codes), do not output index names

- Index code 1
- Index code 2

reason:

\end{lstlisting}

\end{tcolorbox}
\caption{Potential Index Selection Prompt for \textbf{TwinMarket} Agents}
\label{trading_index_selection_prompt}
\end{figure*}

\begin{figure*}[htbp]
\centering
\small

\begin{tcolorbox}[title = {Querying Stock Data Prompt}, colframe=gray, colback=gray!10, coltitle=white, colbacktitle=gray!50!black]
\textbf{Querying Stock Data Prompt:\\}
Today is \{formatted\_date\}, according to the previous dialogue, we know that you believe the industry indexes with potential trading opportunities are: \textbf{\{`, '.join(stocks\_to\_deal)\}}. In this process, you can only access historical data from yesterday and before.

\subsection*{}
 Summary Market Quotes for Relevant indexes on the Previous Trading Day are as follows:
\begin{itemize}
    \item \{stock\_summary\}
\end{itemize}

 Your Holding Information for Relevant indexes is as follows:
\begin{itemize}
    \item \{positions\_info\}
\end{itemize}

 Based on your role and the available data, determine whether additional data is needed for analysis. If needed, you can directly query relevant data to assist in decision-making.

\subsection*{}
 Note:
\begin{itemize}
    \item Please naturally obtain data indicators that you consider valuable during the analysis process. Data acquisition is part of the analysis, but please also be efficient and \textbf{select the most relevant indicators to support your analysis, otherwise you will be penalized}.
    \item \textbf{Please pay attention to your persona and investment style. The analysis and reasoning process should start from the persona and remain natural and coherent.}
    \item \textbf{As a reminder: You are a \{user\_strategy} investor
    \item Your current portfolio's total return rate is \{return\_rate\}\%, your current total assets are \{total\_value:,\} RMB, and your current cash balance is \{current\_cash:,\} RMB.
    \item The reason field should include your reasons for choosing these indicators, such as based on your persona and investment style, or based on your belief, explained in a paragraph.
\end{itemize}

\subsection*{}
The detailed description of available indicators are as follows:\{SCHEMA2\}

\subsection*{}
 Output in the following YAML format:
\begin{lstlisting}[language=Yaml, frame=single, basicstyle=\ttfamily\small, breaklines=true]
indicators:  # List(strs), indicating the indicators you think need to be filtered
- indicator1
- indicator 2
start_date: '%Y-%m-%d'  # Start time for querying
end_date: '%Y-%m-%d'    # End time for querying
reason:
\end{lstlisting}

\end{tcolorbox}
\caption{Querying Stock Data Prompt for \textbf{TwinMarket} Agents}
\label{trading_data_analysis_prompt}
\end{figure*}



\begin{figure*}[htbp]
\centering
\small

\begin{tcolorbox}[title = {Trading Decision Making Prompt}, colframe=gray, colback=gray!10, coltitle=white, colbacktitle=gray!50!black]
\textbf{Trading Decision Making Prompt(Step 1):\\}
Now it's time to make the final trading decision. Please base your analysis on all the information you have obtained previously, combined with your investment style and persona. First, conduct an analysis, and then make specific trading decisions and provide your reasons for each industry index.  A: Please analyze the industry indices you want to trade: \textbf{\{`, '.join(stocks\_to\_deal)\}} based on all the information you have obtained previously, including news, announcements, market data, industry data, and additional stock data.

 Please Include the Following:
\begin{enumerate}
    \item \textbf{Overall Market Analysis}:
    \begin{itemize}
        \item Views on the current overall market situation.
        \item Main trends and possible changes.
    \end{itemize}

    \item \textbf{Impact of News and Announcements}:
    \begin{itemize}
        \item Analysis of the impact of important news and announcements on the market and individual stocks.
        \item Are there any major events that may change market trends?
    \end{itemize}

    \item \textbf{Industry Index Analysis}:
    \begin{itemize}
        \item Detailed analysis for each industry index of interest:
        \begin{itemize}
            \item Current performance and technical analysis.
            \item Fundamentals and future expectations.
            \item Buy, hold, or sell recommendations.
        \end{itemize}
    \end{itemize}

    \item \textbf{Risk Assessment}:
    \begin{itemize}
        \item Main risk points in the current market.
        \item Risk assessment for each asset.
    \end{itemize}
\end{enumerate}

 Note:
\begin{itemize}
    \item Please conduct the analysis in conjunction with the set persona and investment style.
    \item Please conduct the analysis in conjunction with all the information you have obtained previously.
    \item The output content should be clear, concise, and easy to understand.
\end{itemize}

 Output Format for Analysis:
Please output the analysis results in natural language to ensure clear logic and complete structure.


\end{tcolorbox}
\caption{Trading Decision Making Prompt(Step 1) for \textbf{TwinMarket} Agents}
\label{trading_decision_making_prompt_first}
\end{figure*}

\begin{figure*}[htbp]
\centering
\small

\begin{tcolorbox}[title = {Trading Decision Making Prompt}, colframe=gray, colback=gray!10, coltitle=white, colbacktitle=gray!50!black]
\textbf{Trading Decision Making Prompt(Step 2):\\}

Now it is time to make the final trading decision. Based on the previous analysis, combined with your investment style and persona, please make specific trading decisions for the following industry indices.

\subsection*{}

 Trading Related Information:
\begin{itemize}
    \item Remaining available position (relative to total assets): \{available\_position:.2f\}\%
\end{itemize}

 Specific Information for Each Industry Index:
\begin{itemize}
    \item \{chr(10).join(stock\_info)\}
\end{itemize}

 Trading Rules:
\begin{enumerate}
    \item The transaction price must be between the lower and upper limit prices.
    \item Please fill in the trading position and price. If you choose to hold, the position for preparation of the transaction is 0.
\end{enumerate}

\subsection*{}
 Precautions:
\begin{itemize}
    \item \textbf{Important: All decisions must be consistent with your investment style and persona, but you should also make appropriate adjustments based on the actual situation of the day.}
    \item Ensure that each trading decision is within the price and position limits.
    \item If you choose to hold, trading\_position should be equal to 0.
    \item If the current position of an industry index is 0, it means it is a system-recommended industry index, and you can only choose to buy or hold.

    \item If you choose to sell or buy, please pay attention to the setting of your trading\_position (if you choose to sell, then trading\_position cannot exceed the current position). If the setting is unreasonable, it may lead to transaction failure. You need to choose the target\_price according to your expectations, instead of setting it to yesterday's closing price.
    \item \textbf{Important: trading\_position represents the percentage of your total assets that you plan to use for this transaction, and it is always positive.}
\end{itemize}

\subsection*{}
Here is an example of the output:
\begin{verbatim}
TLEI:
    action: sell
    trading_position: 11.5
    target_price:
CPEI:
    action: buy
    trading_position: 10.0
    target_price: 10.3
\end{verbatim}
\subsection*{}
Please output your decision in the following YAML format:
\begin{verbatim}
{yaml_template}
\end{verbatim}


\end{tcolorbox}
\caption{Trading Decision Making Prompt(Step 2) for \textbf{TwinMarket} Agents}
\label{trading_decision_making_prompt_second}
\end{figure*}



\begin{figure*}[htbp]
\centering
\small

\begin{tcolorbox}[title = {Social Media Posting Prompt}, colframe=gray, colback=gray!10, coltitle=white, colbacktitle=gray!50!black]
\textbf{Social Media Posting Prompt:\\}
You are now browsing social media. Based on the news or announcements you previously acquired and your investment decision-making intentions, compose a post. The specific requirements are as follows:\\

1. Post Content:\\
    - Your post must fall into one of the following three categories:\\
      - type1: Commentary on an Event - Cite specific news or announcements, express your opinions and analysis, and integrate personal insights with a conversational tone.\\
      - type2: Summary of Your Recent Trading Behavior - Summarize your latest trading actions, explain the underlying logic and outcomes in detail, and share your emotions and reflections.\\
      - type3: Market Outlook - Based on current market information, predict future trends or investment opportunities while expressing your expectations or concerns about the future.\\

2. Content Requirements:\\
    - Align your post with your investment style and persona, incorporating your trading decisions, perspectives, and analysis. Choose the most suitable post type and writing style.\\
    - Keep the post between 100-200 words.\\
    - Clearly indicate the post type (type1/type2/type3).\\

3. Belief Summary:\\
    - Your prior belief was: \texttt{\{old\_belief\}}\\
    - Considering your investment style, personality traits, trading decisions, and market understanding, describe your updated belief in the first person. This should include the following five aspects:\\
      - Market Trend - Describe your outlook on market direction for the next month.\\
      - Market Valuation - Express your opinion on the current market valuation.\\
      - Economic Conditions - Share your expectations regarding macroeconomic trends.\\
      - Market Sentiment - Describe your perception of the current market sentiment.\\
      - Self-Assessment - Reflect on your historical trading performance and investment style to evaluate your investment ability.\\

4. Output Format:\\
    - The output should follow the YAML format below:\\
\begin{lstlisting}[language=Yaml, frame=single, basicstyle=\ttfamily\small, breaklines=true]
post: Your post content
type: type1/type2/type3  # Post type, required, string format
belief: Your Belief Summary
\end{lstlisting}

\end{tcolorbox}
\caption{Social Media Posting Prompt for \textbf{TwinMarket} Agents}
\label{social_media_posting_prompt}
\end{figure*}

\end{document}