\section{Related work}
\textbf{Resource Management in Network Slicing.} A primary challenge in network slicing is the efficient allocation of resources to support heterogeneous user services. Leconte et. al. ____ proposed a framework for fine-grained slice configuration in terms of network bandwidth and cloud processing. Their approach considers traffic fairness and computational fairness, utilizing an alternating direction method of multipliers (ADMM)-based iterative algorithm, which is proven to converge to optimal resource allocation. Fossati et. al. ____ investigated the fair sharing among slices under insufficient resources scenarios. They proposed an optimization framework based on the ordered weighted average (OWA) operator to ensure fairness in both single-resource and multi-resource allocation problems. The O-RAN ____ advocates for an open network structure and promotes the use of open RAN interfaces to interconnect components with a RAN intelligent controller (RIC) for managing and controlling resources. The OnSlicing ____ proposed an online end-to-end network slicing system based on the O-RAN. This system aimed to minimize resource usage while avoiding violations of slice SLAs and infrastructure resource capacity. By leveraging domain managers, it achieved subsecond-level control for resource allocation. However, existing work often overlooks the cost of resources. In contrast, our work approaches the problem from the perspective of MNOs, focusing on minimizing operation costs while ensuring SLA compliance for slice users. 

\textbf{AI/ML for Networking.} AI/ML techniques have emerged as powerful tools for network orchestration due to the complex nature of network environments and intricate interactions between multiple network components. It provides adaptive and data-driven solutions that can accommodate these variations. DeepSlicing ____ proposed an efficient resource allocation framework that decomposes the optimization problem into multiple sub-problems. Their approaches combine convex optimization for the master problem and deep deterministic policy gradient (DDPG) agents for slave problems, aiming to maximize utility functions. Li et. al. ____ explored two typical demand-aware slice configuration scenarios, wireless resource slicing and priority-based core network slicing. Compared to demand-prediction-based approaches, their DRL framework could implicitly capture deeper relationships between demand and supply, improving the effectiveness and flexibility of network slicing. Atlas ____ proposed an online network slicing system that leverages Bayesian optimization to reduce the sim-to-real discrepancy. Using Gaussian process regression, the system learns real-world policies online, enhancing the performance of network slicing strategies. 
However, existing solutions are mostly based on DNN-based model representations, which lack transparency and interpretability, and thus constrain their practical deployment in real-world networks.