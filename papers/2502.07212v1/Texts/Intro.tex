
\section{Introduction}
\label{sec:intro}



Compute-in-memory (CIM) holds significant promise for efficiently accelerating deep neural networks (DNNs) at the edge by bringing computation and memory closer together to reduce energy-intensive data movement that occurs in conventional von Neumann architecture\cite{wu2023floating, su20228, cao2023d, cao2021neural}.
Although numerous existing works have focused on building integer-based (INT-based) CIM macros\cite{dong202015,TSMC_CIM,SRAM_CIM_INT_1,sun2023efficient,9162917} to improve the energy efficiency of quantized DNN models, there is growing interest in developing floating-point-based (FP-based) CIM macros to enhance the accuracy of high-precision DNN models, including both inference and training applications.
Yet, most current FP CIM methods rely on digital implementation, leading to significant power consumption\cite{SRAM_CIM_FP_1,Sparse_intense,reCIM}.
Novel approaches are needed to bridge this gap, ensuring both high accuracy and energy efficiency for increasing high-precision DNN applications.




This paper proposes a novel hybrid-domain FP CIM architecture to remarkably improve energy efficiency while maintaining lossless accuracy for high-precision DNN applications.
The proposed architecture is based on a key observation that has been significantly overlooked in conventional designs of FP CIM macros.
It is observed that the FP arithmetic can be intrinsically divided into two parts: (1) the computation-intensive multiplication (sub-MUL), contributing less than $1/4$ to FP products, and (2) the computation-light addition (sub-ADD), contributing more than $3/4$ to FP products, as elaborated in Section~\ref{sec: addition}. 
We harness this insight to strategically integrate both analog CIM (for energy-efficient sub-MUL) and digital CIM (for accurate sub-ADD) on a unified hardware substrate.
This hybrid-domain CIM strategy, to the best of our knowledge, optimally combines the strengths of analog and digital CIM to achieve state-of-the-art energy efficiency by maintaining equivalent accuracy compared to fully digital baselines.
The key contributions of the work are listed below.


\begin{itemize}

\item We propose a novel hybrid-domain FP CIM architecture based on static random-access memory (SRAM) that integrates both digital and analog CIM within a unified macro to accelerate high-precision DNN applications. 
 
\item We develop detailed circuit schematics and physical layouts to implement this architecture. 
We optimize the local computing logic with minimal area overhead and minimize the energy cost of analog-to-digital conversion.


\item Experimental evaluations with comprehensive circuit-level simulations demonstrate the exceptional energy efficiency and accuracy of the proposed architecture compared to conventional fully digital baselines.
\end{itemize}  



