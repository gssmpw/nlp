\section{Related Works}
The use of synthetic data in healthcare has gained significant attention as a potential solution to challenges such as data scarcity, privacy concerns, and class imbalance. ____ highlight the growing role of synthetic data in healthcare, discussing its innovation, applications, and how it addresses privacy concerns in medical data sharing and usage . This idea is further explored by ____, who provide a narrative review on the utility of synthetic data in healthcare, stressing its potential to augment training datasets while ensuring patient confidentiality . Additionally, ____ provide a comprehensive overview of how synthetic data can support AI systems by simulating diverse real-world scenarios, thereby enhancing the generalizability of machine learning models in medical applications .

Recent advancements in generative models, especially Generative Adversarial Networks (GANs) and diffusion models, have shown promise in generating high-quality synthetic medical images. The study by ____ compares the performance of GANs and diffusion models in generating synthetic MR images for brain tumor segmentation, demonstrating the potential of these models in medical imaging tasks . In particular, ____ explore the use of conditional diffusion models for 3D brain MRI synthesis, emphasizing how such models can generate realistic images that can be used to augment training datasets for segmentation tasks . These works underline the growing importance of synthetic image generation in improving AI performance for medical tasks like brain tumor detection.

Trust in AI models is a crucial factor for their successful adoption in healthcare, especially when synthetic data is involved. ____ discuss the importance of building trustworthy AI systems in healthcare, emphasizing principles such as fairness, transparency, and explainability . These principles are essential when using synthetic data, as clinicians must be assured that the generated data closely mirrors real-world clinical scenarios and that the AI models can generalize well to new, unseen cases. ____ further explore the challenges associated with trust in medical AI, noting that issues like data provenance, explainability, and biases in synthetic data hinder the widespread adoption of AI in clinical settings .

While synthetic data offers many advantages, it is not without its challenges. One such challenge is simplicity bias, where models trained on synthetic data fail to generalize effectively to real-world data due to inherent differences in the distributions of synthetic and real-world data. ____ examine this issue in their work on medical data augmentation, pointing out that synthetic datasets may oversimplify complex patterns, leading to models that underperform in real clinical settings . Moreover, the limited diversity in synthetic datasets can introduce biases that reduce the trustworthiness of AI models trained on these data. Studies have shown that biases in synthetic data can amplify disparities in AI predictions, especially in sensitive areas like healthcare. ____ conduct a scoping review on privacy and utility metrics for synthetic data, further elaborating on how these biases can undermine the effectiveness and trust of AI systems trained on synthetic medical data .

Synthetic data has also been explored in the specific context of brain tumor segmentation, a critical task in medical imaging. ____ present the BraTS 2015 dataset, which has become a standard benchmark for evaluating tumor segmentation algorithms, and subsequent works have used it to investigate the role of synthetic data in training segmentation models for brain tumors . ____ explore whether segmentation models can be trained effectively using fully synthetic data, finding that while fully synthetic datasets can perform similarly to real datasets, they often suffer from a generalization gap due to the synthetic data's limitations in capturing the complexity of real-world medical images . This work highlights the importance of understanding how the quality, diversity, and proportion of synthetic data impact the reliability of AI models, especially in specialized tasks like brain tumor segmentation.

\begin{figure*}[!ht]
    \centering
    \includegraphics[width=0.8\textwidth]{method1.jpg} 
    \caption{Illustration of the Proposed Method}
    \label{fig:method_illustration}
\end{figure*}


\begin{figure*}[!ht]
    \centering
    % Row 1: Real Images
    \begin{subfigure}{0.16\textwidth}
        \includegraphics[width=\textwidth]{brats_20_mask_real.png}
        \caption{Mask}
        \label{fig:mask_real}
    \end{subfigure}
    \begin{subfigure}{0.16\textwidth}
        \includegraphics[width=\textwidth]{brats_20_flair_real.png}
        \caption{Flair (R)}
        \label{fig:flair_real}
    \end{subfigure}
    \begin{subfigure}{0.16\textwidth}
        \includegraphics[width=\textwidth]{brats_20_t1_real.png}
        \caption{T1 (R)}
        \label{fig:t1_real}
    \end{subfigure}
    \begin{subfigure}{0.16\textwidth}
        \includegraphics[width=\textwidth]{brats_20_t1ce_real.png}
        \caption{T1ce (R)}
        \label{fig:t1c_real}
    \end{subfigure}
    \begin{subfigure}{0.16\textwidth}
        \includegraphics[width=\textwidth]{brats_20_t2_real.png}
        \caption{T2 (R)}
        \label{fig:t2_real}
    \end{subfigure}
    
    \vspace{0.5em} % Adds spacing between rows

    % Row 2: Synthetic Images
    \begin{subfigure}{0.16\textwidth}
        \includegraphics[width=\textwidth]{brats_20_mask_real.png}
        \caption{Mask}
        \label{fig:mask_synthetic}
    \end{subfigure}
    \begin{subfigure}{0.16\textwidth}
        \includegraphics[width=\textwidth]{brats_20_flair_syn.png}
        \caption{Flair (S)}
        \label{fig:flair_synthetic}
    \end{subfigure}
    \begin{subfigure}{0.16\textwidth}
        \includegraphics[width=\textwidth]{brats_20_t1_syn.png}
        \caption{T1 (S)}
        \label{fig:t1_synthetic}
    \end{subfigure}
    \begin{subfigure}{0.16\textwidth}
        \includegraphics[width=\textwidth]{brats_20_t1ce_syn.png}
        \caption{T1ce (S)}
        \label{fig:t1c_synthetic}
    \end{subfigure}
    \begin{subfigure}{0.16\textwidth}
        \includegraphics[width=\textwidth]{brats_20_t2_syn.png}
        \caption{T2 (S)}
        \label{fig:t2_synthetic}
    \end{subfigure}

    \caption{Comparison of Real (R) and Synthetic (S) Brain Tumor Images.}
    \label{fig:brain_tumor_plot}
\end{figure*}