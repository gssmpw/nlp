\section{Related Work}
Based on the prior research, we first identified the challenges related to mental health sequelae after a concussion. 
Then, we provided a brief overview of current AI-RPM technologies in clinical settings, as well as AI-based system design for clinical decision support.

\subsection{Challenges in Post-Concussion Mental Health Sequelae  }
%Common
Mental health sequelae, such as anxiety, depression, PTSD, and suicidality, are common among youth concussion patients (ages 11–17)~\cite{stein2019risk, gornall2021mental, ledoux2022risk}. 
Yang et al.~\cite{yang2015post} studied how concussions impacted the mental health of 71 collegiate athletes and found that 14 reported symptoms of depression, 24 experienced symptoms of anxiety, and 10 reported both depression and anxiety. 
Another study showed that youth concussion patients tend to have twice the risk of suicide compared to their peers who have never had a concussion~\cite{fralick2019association}.
Moreover, mental health sequelae can prolong youth concussion patients' concussion recovery~\cite{silverberg2020management, iverson2003examination}, 
and hinder their ability to attend school and participate in normal social activities~\cite{iadevaia2015qualitative}, which can be an important way to maintain mental well-being~\cite{zamfir2015physical}.
Thus, it is important to integrate the assessment of mental health conditions into clinical follow-up procedures~\cite{gornall2021mental, russell2023incidence}. The assessment enables concussion clinicians to identify abnormalities in youth patient mental health conditions and conduct intervention, such as conducting a further assessment or referring the patient to a mental health expert. 
%Therefore, early identification of mental health sequelae in the initial stages of concussion recovery is crucial for reducing the risk of worsening mental health issues in youth concussion patients and minimizing the negative impact on their concussion recovery.

However, there are challenges when it comes to accessing youth concussion patients' mental health conditions. %Symptoms of mental health sequelae include difficulties with concentration~\cite{}, reduced physical activity~\cite{}, mood swings~\cite{}, and sleep disturbances~\cite{}. However, since 
Most youth patients are advised to rest at home after a concussion~\cite{silverberg2023management}. 
Their mental health symptoms, such as anxiety, changes in sleep patterns, and disinhibition~\cite{brent2017psychiatric}, often manifest outside of clinics, which makes it difficult for concussion clinicians to observe these symptoms in clinical settings. 
Although clinicians can inquire about mental health symptoms during clinical visits, recall bias~\cite{frissa2016challenges,van2016accuracy} may prevent patients from providing complete and accurate information to concussion clinicians. 
Self-report questionnaires like the GAD-7 and PHQ-9 are available for anxiety and depression screening~\cite{brent2017psychiatric}.
However, clinicians often do not receive the results until the patient's next clinical visit.
Moreover, clinical follow-up visits for youth concussion patients are often insufficient and lack consistency. 
Although it is recommended to conduct follow-ups every three days after a concussion~\cite{healthmil_concussion_2024}, ~\citet{ramsay2023follow} found that actual follow-up rates vary significantly, with the lowest being only 13.2\%. 
Insufficient follow-up prevents concussion clinicians from timely collecting mental health information and assessing the youth patient’s mental health conditions. 
Accordingly, it's difficult for concussion clinicians to make timely mental health-related decisions. 
Therefore, novel approaches are needed to support clinicians in remote monitoring the mental health symptoms of youth concussion patients on time.




\subsection{AI-RPM Technologies in Clinical Settings} 
\label{sec:airpm}

 RPM includes primarily wearable devices and home-use medical equipment that collect patient health data outside of clinical settings~\cite{mendel2024advice}. These systems transmit monitored patient data to healthcare providers for real-time monitoring and assessment, thereby supporting clinicians in their decision-making~\cite{catalyst2018telehealth, baig2017systematic, us2008national}. Wearable devices, for example, serve as valuable tools for remotely monitoring patient data. Existing studies have leveraged wearable technology to track and collect key physiological data from concussion patients during their recovery period, such as physical activity levels, for recovery progress assessments~\cite{tirosh2024smartphone}. Additionally, researchers have investigated the relationship between post-concussion symptoms and physical activity in young patients. 
 For instance, \citet{yang2020bidirectional} utilized ActiGraph, a smartwatch-based device, to monitor patients’ physical activity and collect movement-related data, such as step count. 
 
 Furthermore, researchers in the Human-Computer Interaction (HCI) and Computer-Supported Cooperative Work (CSCW) fields have started exploring mHealth applications that facilitate self-management, and enhance remote communication between patients and clinicians~\cite{corwin2024using, nyapathy2019tracking, salamah2021improving, west2018common}. For instance, \citet{el2021mobile} indicated that mHealth can improve diagnostic efficiency and reduce healthcare costs for patients. RPM systems can display collected data via user interfaces for patient review~\cite{griggs2018healthcare} or integrate the data into electronic health record (EHR) systems~\cite{dinh2019wearable}. However, given the significant time constraints that clinicians already face~\cite{haikio2020expectations}, the additional data generated by RPM systems could further exacerbate these challenges. 

 
 Recently, AI-driven tools have shown the potential to assist clinicians in reducing the cognitive load in clinical workflows by analyzing and summarizing large volumes of patient data or by presenting information in a more interpretable format ~\cite{haikio2020expectations}. 
 % Research suggests that combining subjective patient perceptions with quantifiable metrics enhances the value and interpretability of patient data~\cite{haikio2020expectations}. 
 For instance, voice assistants powered by LLMs can interact with patients using natural language, capture subjective symptom experiences, and visualize them on a dashboard for clinicians~\cite{yang2024talk2care}. 
 Despite AI having the potential to enhance today's RPM systems, poor system design of such AI-RPM may introduce new challenges; 
 Thus, we aim to explore the best design practices to ensure these systems can enhance rather than hinder clinicians' work efficiency.


 

\subsection{Designing AI-based Systems for Clinical Decision Making} 
Recently, HCI researchers have had an increasing interest in the clinical-AI decision making research topic, and their works span various medical domains, such as cancer diagnosis~\cite{cai2019hello,cai2019human,denekamp2007clinical}, cardiotoxicity early detection~\cite{ahmed2024advancements}, diabetic retinopathy screening~\cite{beede2020human}, and rehabilitation assessment~\cite{lee2021human,lee2020co}.
Among some of these works, AI was leveraged for predictive analytics to help forecast patient health outcomes~\cite{zhang2024rethinking, pathak2024comparative, collins2019reporting}. \citet{dabek2022evaluation} 
developed an AI model with 88.2\% accuracy for predicting the risk of mental health sequelae within 90 days post-concussion.


 


Despite these advancements, the integration of AI-based decision support systems into clinical workflows remains challenging. 
One major issue is that these systems often fail to consider the needs of stakeholders during the design phase, limiting the system's  usability~\cite{cai2019hello, green2019principles, khairat2018reasons, lee2020co, romero2020lesson}. %and  trustworthiness~\cite{yang2016investigating, elwyn2013many}. 
Moreover, clinicians remain skeptical about the accuracy of AI-generated predictions, which affects their trust in the system~\cite{bohr2020rise, yang2016investigating, elwyn2013many}.
As a result, AI-based systems may not effectively assist clinicians in making faster and more accurate decisions~\cite{antoniadi2021current}.

To address this gap, existing research has sought to incorporate human-centered design principles into AI-based system development. 
Studies have shown that involving stakeholders early in the design process and establishing continuous feedback loops~\cite{abdulaal2021clinical} can enhance the integration of AI systems into clinical workflows~\cite{wang2023human}. 
As a result, AI-based systems can be better aligned with stakeholder needs, ultimately improving their usability.
For example, Yang et al.\cite{sendak2020human} conducted multiple discussions with clinicians to understand the challenges they face during meetings, collected their feedback, and iteratively refined the design of an AI-based system based on this input. 
Similarly, Lee et al.~\cite{lee2020co} involved seven therapists throughout the design process, which led to greater satisfaction and increased adoption of the system.

These prior works inspired us to begin our project with a formative study to understand concussion clinicians' challenges and needs.
Then, we continued to involve them in the design and development process of the AI-RPM system for the mental health sequelae detection scenario.
