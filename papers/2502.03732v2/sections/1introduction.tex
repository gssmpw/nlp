\section{Introduction}
Concussions are a form of traumatic brain injury (TBI) caused by a sudden hit or jolt of the head.
In 2023, a total of $2.3$ million youth aged $\leq17$ years received a diagnosis of concussion, making it one of the most common injuries among youth patients aged 11 to 17~\cite{yang2021association,miller2022salivary,rivara2014sports,bryan2016sports}.
Concussions can often lead to a high likelihood of developing mental health disorders (i.e.,~\textbf{mental health sequelae}) among youth patients, such as depression, anxiety, post-traumatic stress disorder (PTSD), and suicidality~\cite{fralick2019association,ceniti2022psychological,gornall2021mental}.
% In particular, concussion and mental health sequelae often bidirectionally impact each other.
% Studies show that concussions lead to a higher likelihood of developing various mental health sequelae, such as depression, anxiety, PTSD, and suicidality~\cite{fralick2019association,ceniti2022psychological,gornall2021mental}.
Mental health sequelae can, in turn, exacerbate the severity of concussion symptoms, including insomnia, concentration difficulties, and headaches, which may result in prolonged concussion recovery~\cite{gornall2021mental,iverson2003examination,ceniti2022psychological} and hospitalization~\cite{ledoux2022risk}.
% Moreover, mental health sequelae can impact the quality of adult life for youth concussion patients~\cite{}. 
%Moreover, mental health sequelae following a concussion can lead to a long-term negative impact on the quality of life of youth concussion patients~\cite{scott2015comparison, taylor2015traumatic}.
%For example, a controlled questionnaire-based study involving 167 participants revealed that youth concussion patients with mental health sequelae have an increased likelihood of experiencing adverse events such as being laid off or breaking up with a partner in their adult lives~\cite{taylor2015traumatic}. 
%zatzick2011association,harvanek2021psychological
% Accordingly, 
Researchers have emphasized the importance of early detection of mental health symptoms in youth concussion patients~\cite{miller2021association, brooks2019predicting, gornall2021mental, brent2017psychiatric}
% To optimize patients' concussion recovery trajectory, researchers have emphasized the importance of early detection of mental health symptoms in youth concussion patients~\cite{miller2021association, brooks2019predicting, gornall2021mental, brent2017psychiatric}.
% However, there is a unique challenge that impedes clinicians' early detection of mental health symptoms in youth concussion patients.
The early symptoms of mental health sequelae (e.g., disturbed sleep patterns, fatigue, and emotional changes ~\cite{brent2017psychiatric, silverberg2023management}) are often difficult to observe as these symptoms typically occur outside the clinics.
Concussion patients' infrequent clinical visits (e.g., every two weeks) further exacerbate the difficulty for clinicians to observe such symptoms and detect mental health sequelae in a timely manner~\cite{silverberg2023management,rose2015diagnosis,mcleod2017rest}.





The standard of care (SoC) approaches of concussion clinicians include: (1) asking patients to recall their at-home behaviors during clinical visits~\cite{silverberg2020management,brent2017psychiatric}, and (2) instructing patients or parents to complete self-evaluation questionnaires at home, such as Generalized Anxiety Disorder-7 (GAD-7) and Patient Health Questionnaire-9 (PHQ-9)~\cite{carson2021using, silverberg2020management}.
However, these approaches are not without limitations:
it could be challenging for concussion patients to comprehensively and accurately recall their previous behaviors in a clinic visit~\cite{frissa2016challenges,van2016accuracy}; 
patients may be able to complete questionnaires on a regular basis at home, but concussion clinicians can only access the questionnaire results when patients return to the clinics in person, which still delays clinicians' decision-making.




Previous work has proposed different Remote Patient Monitoring (RPM) technologies to collect patients' at-home information in various scenarios~\cite{chung2016boundary,olen2017implantable,barnes2024clinician,van2023clinician,toresdahl2021systematic,tirosh2024smartphone}.
For example, mobile health (mHealth) applications have been proposed in which patients can report their physiological measurements to care provider teams in real time. 
Wearable devices, such as the ActiGraph~\cite{yang2020bidirectional} and Philips Actiwatch~\cite{donahue2024feasibility}, have been explored to measure the daily steps and sleep patterns of adult concussion patients.
Despite their relative success in capturing patients' activity data, these traditional RPM approaches are not without limitations: mHealth applications could lead to cognitive burdens for patients with low digital literacy and brain injuries~\cite{giebel2024problems,purdy2023exploring}; extensive amount of time and effort is required to process the massive amount of multi-modal data collected by wearable devices~\cite{ginsburg2024key,baig2017systematic}.
These limitations may reduce patient compliance and lead to clinicians delaying collected data access,  which further affects timely diagnosis and intervention.
%sshorter%These limitations could result in low compliance among the patients and delays in clinicians accessing collected data, which further affect timely diagnosis and intervention.





Recent exploration in the HCI community demonstrates the promising potential of leveraging innovative AI technologies for RPM (\textbf{AI-RPM}) to address the aforementioned limitations of traditional RPM approaches. 
For instance, conversational agents (CAs) powered by large language models (LLMs) enable asynchronous patient-provider communication to verbally collect patients' health conditions through regular health check-in conversations~\cite{yang2024talk2care}. 
LLMs have also been explored to organize and summarize key information from massive RPM data on clinician-facing dashboards~\cite{wu2024cardioai}.
Meanwhile, another line of research explored using AI prediction algorithms with RPM data to assist clinicians' decision-making in their clinical workflow~\cite{yin2024sepsislab}.





Despite the promising opportunities of AI-RPM systems, they could lead to additional difficulties and burdens if not appropriately designed.
Recent studies discovered that AI prediction models without faithful and reliable explanations cannot provide helpful information~\cite{antoniadi2021current} or actionable insights to clinicians~\cite{rane2023explainable, zhang2024rethinking}, lose clinician trust~\cite{zhang2020effect, chanda2024dermatologist}, and even result in human-AI competition~\cite{zhang2024rethinking}.
Moreover, in high-stakes, high-risk, and uncertain clinical scenarios, clinicians follow strict clinical guidelines and specifications in their daily workflow. 
AI systems misaligned with clinical specifications may be unusable and even cause potential harm to stakeholders~\cite{romero2020lesson}.
%sshorter%AI systems that are in conflict with clinical specifications cannot be used and even cause potential harm to stakeholders~\cite{romero2020lesson}. 



In contrast, human-centered AI (HCAI) design guidelines~\cite{shneiderman2022human, wang2023human} underscore the importance of engaging stakeholders in the system design process and understanding stakeholders' workflow, in particular, the needs and challenges they encounter.
By following HCAI design guidelines, AI systems can be seamlessly integrated into stakeholders' workflow where stakeholders can confidently understand, trust, and use AI-powered systems in their current workflow. 
A number of studies have demonstrated the effectiveness of human-centered AI systems in medical and healthcare scenarios to facilitate the needs of diverse stakeholders, including older adults~\cite{hao2024advancing}, cancer patients~\cite{wu2024cardioai}, and clinicians~\cite{wang2023human,fogliato2022goes,yang2019unremarkable,zhang2024rethinking}.
Nevertheless, how to employ HCAI guidelines in the design of AI-RPM systems has not been studied yet, and little is known about the challenges concussion clinicians face in remote monitoring youth patients and how AI-RPM technologies could be of practical benefit.



 
In this work, our objective is to investigate the current workflow of concussion clinicians with respect to the challenges in remote monitoring and decision-making of young concussion patients.
In particular, we focus on how clinicians collect patients' mental health-related information, what decisions they need to make, and what the difficulties are in their current workflow.
In addition, we want to explore the design opportunities of AI-RPM technologies in supporting clinicians' needs for remote monitoring and decision-making during their practice.
To answer these questions, we conducted a formative study of semi-structured interviews with six highly professional concussion clinicians and derived a suite of design considerations of AI-RPM technologies that could support clinicians' needs and clinical decision-making.
We then came up with preliminary system designs based on design considerations.
Subsequently, we engaged the same group of clinicians in an evaluation study to further collect feedback and insights on the design.
Finally, we consolidate the feedback and revise the design to come up with a complete clinician-facing system design as the final artifact. 

This paper makes the following key contributions:
(1) we identified the challenges and needs that concussion clinicians encounter during their clinical practice with youth concussion patients, particularly with respect to patient mental health sequelae; 
(2) we derive a suite of design considerations and visualization feedback to empower advanced AI-RPM technologies to support remote patient monitoring and clinicians' decision-making; and 
(3) we deliver a user-centered system design of a concussion clinician-oriented dashboard system supported by multiple AI-RPM technologies, where the preliminary system design and refinements are grounded in users' needs and feedback.

