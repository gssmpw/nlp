\section{Refined AI-RPM System Design}
Based on the findings in the design interface evaluation session, clinicians acknowledged the value of the system with AI-RPM technologies that can potentially help them remotely monitor patients' mental health symptoms and support them to make decisions in their workflow. At the same time, they provided feedback and suggestions based on the preliminary designs. Based on their feedback, we revised our preliminary design and proposed a complete design of clinician-facing AI-RPM system design (Fig. \ref{fig:refined_system_design}).
%The purpose of this section is to effectively illustrate the complete system design interface which is tailored to concussion clinicians' needs in their workflow and provide insights for future studies. 

%\begin{figure*}[htbp]
%  \includegraphics[draft=false,width=\textwidth]{refind_system_design.pdf}
%  \caption{The final system design is based on feedback from concussion clinicians gathered through design interface evaluation: (A) patient list, (B) basic information, (C) sleep and physical activities, (D) self-reported symptoms, (E) questionnaire result, and (F) AI risk score prediction for mental health sequelae.}
%  \label{fig:figure2}
%\end{figure*}

%\iffalse
%\subsection{Design Considerations}
%\subsubsection{Explanation of Data Sources and Module Relationships}
%Through design interface evaluation, we found that while concussion clinicians considered the risk score useful for health decision-making and communication with patients and their families, they also raised concerns. Specifically, they were unsure how the score was generated and how it connected to other modules. To address this issue, we recommend using visualizations to illustrate the relationships between AI results and other data sources. This would enable clinicians to intuitively understand how various factors influence the risk score, potentially enhancing their trust in the system. One clinician explicitly mentioned wanting to see the relationship between clinical recommendations and patient recovery. Thus, we suggest visualizing this by displaying the correlation between the risk score and exercise recommendation compliance. 

%\subsubsection{Personalized Time Selection}
%One clinician found the AI-generated risk score unhelpful because the preliminary system design displayed a one-year risk score, whereas she typically treats youth patients for only four weeks. If a patient does not recover within this period, they are referred for further evaluation. Therefore, we believe clinicians would benefit from viewing AI model risk scores over different time spans to meet their varying needs and align with their workflow. To accommodate different clinical needs, we propose adding a time range selector to the module. This feature would allow clinicians to customize the risk score display based on their specific decision-making timeframes, offering personalized support for different clinical workflows.

%Furthermore, for the sleep and activities module in Section \ref{subsubsec:cawearables}, a clinician expressed a preference for viewing data on a weekly basis. To align with this preference, we recommend that the module default to displaying seven days' worth of data while also providing the option to explore detailed data for specific dates. This would allow clinicians to closely examine anomalies, particularly when irregular patterns appear on certain days.

%\subsubsection{Providing Information on Other Stakeholders to Support Collaboration}

%Concussion clinicians noted that in the self-reported symptoms module, the most critical yet challenging situation for clinicians is detecting suicidality in patients. However, they expressed uncertainty about how to remotely respond if a patient exhibits self-harm tendencies. Additionally, clinicians raised concerns about the authenticity of such reports, as youth patients may sometimes exaggerate their symptoms. Clinicians emphasized that if an emergency occurs, parents should be involved, as they are considered key stakeholders in managing such situations. In response to this concern, we propose integrating an emergency contact feature
%within this module, which would allow clinicians to contact parents when necessary. Since parents may have quicker access to their children and be able to take immediate action, this feature could enhance response efficiency. Additionally, clinicians expressed a need to track the number of days symptoms persist, as this directly reflects a patient’s recovery progress and informs mental health-related decisions such as giving a referral. We recommend that when CAs engage in conversations with patients and their parents, they also document their observations on the duration of mental health symptoms. Then, from the design side, we propose adding a data point in the module to display symptom duration.

%In addition to monitoring the patient’s emotional state, clinicians often ask parents about behavioral changes. To enhance data comprehensiveness and objectivity, we suggest incorporating parent-VA (virtual assistant) conversation data into this module, offering an additional perspective on the patient’s condition. Finally, to reduce confusion and improve the module’s usability for clinicians, we need to clearly define how the data is collected. Ensuring transparency in data sources will help clinicians trust and effectively utilize this module in their practice.

%\subsubsection{Balancing Additional Data Presentation and Cognitive Load}
%In the preliminary system design, certain key data types that are highly important to clinicians were not covered. As section \ref{subsubsec:detaileddata} mentioned, concussion clinicians emphasized the need for heart rate during sleep, nap times, wake-up time, and sleep time rather than just total sleep hours. To address this, we recommend displaying these additional data points within the module. However, given clinicians' concerns about data overload—especially as they noted that existing electronic health record (EHR) systems have already increased their workload—we propose a customizable data selection feature. Within the system interface, clinicians can personalize their data view based on their specific needs and workload at different times, rather than being presented with all data at once. Additionally, we believe it is essential to prioritize the most relevant information in the default display. Specifically, the most critical data will be shown by default, while supplementary explanatory details will be secondary and only accessible when needed. This design approach aims to reduce cognitive load, ensuring that clinicians can quickly access key insights without being overwhelmed by excessive information. By minimizing cognitive burden, this method enables more efficient and personalized analysis. 

%\subsubsection{Integrating Related Data into a Unified System Interface}
%In our formative study, clinicians mentioned using GAD-7 and PHQ-9 to monitor patients' anxiety and depression. Additionally, as Section \ref{subsubsec:explanation} mentioned, clinicians noted that they use these questionnaires to patients again during the design interface evaluation, which makes us realize that we could integrate all the information clinicians want to see into the same system interface. Based on this, we propose adding a questionnaire results module to the system interface, displaying patients' completed questionnaires before their clinical visits. This integration would allow clinicians to access all relevant data in a single interface before a consultation, saving time, improving data review efficiency, and enabling a faster and more evidence-based assessment of patients' mental health status. Importantly, these questionnaire results differ from the anxiety and depression statuses identified in the self-reported symptoms module. They provide a more structured and standardized assessment. By combining subjective self-reported symptoms with objective questionnaire scores, clinicians can gain a more comprehensive understanding of patients' mental health conditions, which would support more confident mental health-related decision-making, even when patients are outside the clinical setting.


%\subsection{Each Module of Refined System Design}
%In the new version of our system design, the final system design consists of five modules, which aim to enhance system usability and support clinical decision-making. In the following sections, we will explain the data, functionality, potential interactions, and purpose of each module in detail. The final system design interface is shown in Fig. \ref{fig:figure2}.
%\fi


\subsection{Reducing Cognitive Burden Through EHR Integration}
To reduce the cognitive burden placed on clinicians by the new system and ensure the system can be seamlessly integrated into clinicians' workflow, we referenced the style of current EHR systems to ensure consistency. We integrated a patient list (Fig. \ref{fig:refined_system_design} A), which already exists in the EHR system, into the left side of the interface. However, we added a red dot next to a patient's name to capture clinicians' attention if AI-RPM technologies detect abnormalities in the patient's mental health conditions. Moreover, we added a basic patient information section, which exists in the current EHR system, in the upper-left module (Fig. \ref{fig:refined_system_design} B). However, the information here is tailored to concussion clinicians' needs. These pieces of information provide clinicians with essential insights into a concussion patient's condition. In particular, the Individual Psychiatric History and Mental Health Prescription are important for concussion clinicians when detecting youth patients’ mental health sequelae and making decisions.

\subsection{The Sleep and Physical Activities Module}
Based on findings from our evaluation study, concussion clinicians affirmed the value of the data provided in the Sleep and Physical Activities module (Fig. \ref{fig:preliminary_system_design} C) in the preliminary design. However, several clinicians requested more information about data sources. In response, we designed a question mark next to the module title (Fig. \ref{fig:refined_system_design} C).
Concussion clinicians can hover over the question mark to view the data source. 
Moreover, because clinicians typically prefer a weekly overview of patient data, and some of them also want the ability to see daily metrics (e.g., heart rate), we provided the option to view data from both the last week and the previous 24 hours. 
With this option, concussion clinicians can easily track changes across different time intervals.

In addition, clinicians requested the monitoring of additional metrics, such as heart rate during sleep and nap duration. To incorporate these data points without increasing cognitive load, we organized the data into three tabs: Sleep Data, Physical Activities, and Recommendation Compliance.
The Sleep Data tab is presented by default, which allows clinicians to immediately access important metrics such as heart rate, nap duration, and sleep and wake-up times.
They can choose to view one or multiple metrics, depending on their specific needs.
Different data types are displayed in distinct colors to help clinicians visually distinguish between them more easily. 
If clinicians want to delve deeper into a patient’s sleep patterns, they can hover over a data point in the chart, then a more detailed value for that day will be displayed.

\subsection{The Self-Report Symptoms Module}
To maintain consistency with the Sleep and Physical Activities module, we added question marks indicating the data source and provided time selection options in the Self-Report Symptoms module (Fig. \ref{fig:refined_system_design} D).
Moreover, we identified emergent situations (i.e., suicidality) as both critical and challenging for clinicians to monitor, given resource constraints and the risk of false alarms. 
To address this, clinicians emphasized the importance of including parents in managing emergencies.
Thus, we introduced two sections within this module. 
On the left side, we included the CA-parent conversation history, allowing clinicians to evaluate the patient’s health conditions from family perspectives. 
This conversation history can be expanded by clicking the icon next to the title "Conversational History". 
In addition, we added two call buttons at the bottom of the module, which enables clinicians to contact either the patient or a parent to manage emergency.

Besides, clinicians confirmed the value of different categories (i.e., depression, anxiety) in this module but indicated the need for more information.
In response, we retained the original categories and introduced new ones—such as nutrition level—based on clinician feedback.
Moreover,clinicians mentioned about the importance of severity for their treatment plan.
To meet this need, we placed color-coded dots next to each symptom category (red for severe symptoms, orange for moderate, and green for none). 
To the right of each category, we provided a symptom duration indicator to show how long the reported issue has persisted, which help clinicians form a better understanding of patients health conditions.
Furthermore, clinicians can click AI Summary to view a detailed breakdown of the category, providing additional context to support their assessment. 


\subsection{The Questionnaires Result Module}
In this module (Fig. \ref{fig:refined_system_design} E), the title includes an option to view data source information, which comes from patient-administered questionnaires at home. These questionnaires include GAD-7 for anxiety, PHQ-9 for depression, and PCSS for concussion severity. The middle section of the chart presents the scores for each questionnaire, along with corresponding explanations to help clinicians interpret the results. The module also displays a comparison with previous scores, providing clinicians with a sense of the patient’s recovery trend. Additionally, clinicians can click "Show Details" to view a breakdown of individual section scores within each questionnaire for a more in-depth assessment.

\subsection{The AI Risk Score Prediction Module}
The final module (Fig. \ref{fig:refined_system_design} F) presents the AI-generated prediction of a patient’s risk of developing mental health sequelae. 
Given that concussion clinicians primarily manage patient care during the first four weeks following a concussion, we adjusted the default timeframe for risk predictions from one year to four weeks, while still providing other timeframe options for selection.
Below the imeframe options, the visualization of the risk score chart remains unchanged from the preliminary design, given that clinicians reported it to be easy to interpret. 
However, clinicians expressed an interest in knowing which modules or features influence the AI-generated risk score.
To highlight the relationship between the AI risk score and other modules, we introduce Features Contributing to the AI Risk Score section at the bottom of the module.
The section includes contributing features and its corresponding module, along with a contribution rate.
Moreover, we provide visual feedback on each feature's contribution to the AI risk score. 
Both the color of the dots next to the feature and the font color of its associated percentage value dynamically change based on the degree of the contribution. 
In this way, clinicians can identify the important metrics quickly.




