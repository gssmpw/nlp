\section{Conclusion}
%% !!emphasizes the novelty

We conducted a formative study with six concussion clinicians to understand the challenges and needs during their clinical practice with youth concussion patients, particularly with respect to the patient mental health sequelae. 
Then, we derived a suite of design considerations with the use of AI-RPM technologies. Here are three design considerations. 
Firstly, we recommend the use of wearables to remotely monitor patients’ sleep and physical activity data, helping assess treatment compliance and inform mental health-related decisions. 
Secondly, we propose LLM-powered CAs to enhance mental health self-reporting, which facilitates private, natural interactions with patients and summarizes key information for clinicians. 
Thirdly, we suggest leveraging AI risk prediction in detecting concealed or worsening mental health sequelae.  Finally we delivered a clinician-facing system design as our final artifact. 
To our knowledge, this is the first study in the CSCW field to focus on the intersection of health, RPM, and human-AI decision-making, which provides a foundation for designing AI-RPM systems in varies medical scenarios.