
\section{Limitations and Future Work}
Our work is not without limitations. 
First, the sample size of our participants was limited, with six concussion clinicians participating in both the formative and evaluation studies.
Recruiting these experts was particularly challenging due to their demanding workloads~\cite{wang2021brilliant,jin2020carepre} and the specialized expertise required to treat youth concussion patients with mental health issues
As a result, we recruited six highly relevant experts in this specific domain. 
Despite the limited number, our inductive thematic analysis reached thematic saturation.
%which suggests that additional participants may not have yielded new insights.
Moreover, previous research employed a similar number of expert participants in related studies~\cite{cai2019hello,beede2020human,jacobs2021designing,yang2024talk2care,zhang2024rethinking}, which supports the appropriateness of our sample size. 
%Although our recruited clinicians came from different hospitals and regions, enhancing sample diversity to some extent, all of them are practicing clinicians based in the United States.
Future research should include a broader range of concussion clinicians across locations for greater generalizability of the findings. 

Secondly, the refined AI-RPM system has not yet been further evaluated by concussion clinicians to assess its effectiveness and usability. 
In future research, we plan to incorporate participatory design methodologies~\cite{muller1993participatory} to revise the system design, ensuring it aligns more closely with clinicians' workflows and needs. 
Moreover, our study focused on system design rather than developing an interactive prototype. The primary goal of our system design was to explore the feasibility and provide insights for further design refinements before actual development, thereby saving resources and time for both researchers and clinicians. 
Future research should investigate clinicians' pain points and needs when interacting with a functional system developed. 


Finally, we want to highlight the potential of applying AI-RPM systems in varied healthcare settings. %%%not sure if it's a proper expression
Several studies have explored remote monitoring in the healthcare domain. 
~\citet{wyche2024limitations} utilized mHealth to track the health conditions of type 1 diabetes among youth. 
~\citet{seals2022they} leveraged wearables to detect gait impairment, monitor patients' responses to treatment, and visualize patient data.
We believe that AI-RPM technologies have the potential to be leveraged and extended to these scenarios.
%The AI-RPM systems could enable experts and caregivers to better digest the massive amount of data from RPM and make informed decisions based on the comprehensive information the system provides.
Future work should expand AI-RPM technologies to diverse healthcare settings, enabling timely interventions and improving care quality.



