\section{Formative Study}
\label{sec:formative}

To better understand the workflow of clinicians treating youth concussion patients with mental health issues, we conducted a formative study focusing on three main aspects: (1) how clinicians gather youth patients' mental health-related information, (2) what decisions they need to make, and (3) what the difficulties are in their current workflow. 

\subsection{Study Participants and Procedure}
\label{sec:formative-design}

\begin{table*}
  \caption{Demographics of Participants in Our Formative Study}
  \label{tab:demographics}
  \begin{tabular}{cclcc} 
    \toprule
    P\# & Gender & Department & Job Title & Year of Practice \\ 
    \midrule
    P1 & Female & Pediatric Sports Medicine & Pediatric Sports Medicine Physician & 16 years \\ 
    P2 & Female & Complex Concussion Clinic & Neuropsychologist  & 12 years \\
    P3 & Female & Concussion Clinics & Concussion Clinician & 19 years \\ 
    P4 & Male & Sports Medicine & Division Chief & 30 years \\ 
    P5 & Male & Concussion Clinics & Concussion Clinician & 16 years \\ 
    P6 & Male & Sports Medicine & Non-Operative 
Sports Medicine Doctor & 10 years \\ 
    \bottomrule
  \end{tabular}
\end{table*}

The key stakeholders in our scenarios are concussion clinicians with experience treating youth concussion patients with mental health symptoms. 
These highly specialized concussion professionals work in fast-paced environments and are often overloaded with numerous complex patient cases, which makes their time very limited. 
For this reason, we were only able to recruit six U.S.-based concussion clinicians who were available to participate in our study.
Recruitment was facilitated through professional networks within related clinical fields of domain experts in the research team via convenience sampling~\cite{sedgwick2013convenience}.
Each participant engaged in a semi-structured interview of 35-45 minutes conducted remotely via Zoom. 
Table~\ref{tab:demographics} provides details on the demographics of the participants and their clinical experience. 
This study was reviewed and approved by the first author's institution's Institutional Review Board (IRB), and the study complied with the approved procedure for ethical research practices of human subject protection.






At the beginning of each interview, we first asked the participant to recall a recent case where they encountered a youth concussion patient with mental health symptoms 
They described how they performed clinical practices, identified mental health symptoms, and executed follow-up decision-making.
While participants were sharing their experiences, we prompted participants to tell more about what type(s) of patient clinical data they collected, used, or expected to support the identification of mental health sequelae, both in and outside of clinical settings.
After that, participants shared their experience with existing tools, if any, for collecting patient data, monitoring patient mental health conditions, and identifying mental health sequelae. 
Moreover, participants shared their insights on the benefits and limitations of existing tools and the expectations of novel tools that could potentially help them.
During the interview, we asked participants to refrain from disclosing any personally identifiable information (PII). 
The complete interview protocol is provided in Appendix~\ref{sec:appendixa}.



All interviews were audio-recorded and transcribed with participants' consent. 
We employed the inductive thematic analysis approach~\cite{braun2012thematic,braun2019reflecting} to derive key themes. 
Two researchers first independently coded one session of the interview transcripts to identify meaningful segments and categorize segments into codes. 
Then, they discussed individual codes, resolved discrepancies, developed consensus codes, and applied the resulting codes to the remaining transcripts.
Finally, we iteratively grouped codes into overarching themes and refined these themes through ongoing discussions until we reached a consensus on the final set of themes.




\subsection{Findings}
\label{sec:formative-findings}

Our qualitative analysis provided a comprehensive understanding of concussion clinicians' workflow with youth concussion patients and derived three major challenges that concussion clinicians encountered during the practices: 
(1) tracking patients’ mental health-related information outside the clinic and identifying its severity;
(2) communicating effectively with patients and their family members about patients' mental health issues;
(3) assessing patients' compliance with clinical recommendations.



%\subsubsection{Clinical Workflow of Concussion Clinicians}
\subsubsection{Integrating Mental Health Screening into Concussion Management}
\label{sec:formative-findings-f1}

The majority of youth concussion patients are athletes, and concussions are commonly caused by head injuries during sports participation.
After they had a head injury, the youth patients underwent an initial assessment by primary care physicians.
According to P4, primary care physicians conducted basic evaluations and symptom scoring based on international guidelines. 
If specialized care is needed, patients are referred to specialized concussion clinicians.

The clinical tasks of concussion clinicians primarily include seeing new concussion patients and following up with existing patients, which typically follow a similar routine. 
When patients visit the clinic, concussion clinicians assess the severity of concussion symptoms with standard scales as suggested by clinical guidelines.
In addition, the clinicians conduct interview sessions with patients to collect more in-depth information about their health conditions. 
\begin{quote}
    \textit{"I guess the only way that I currently assess my patients in that way is by using the symptom score that is published in the international guidelines...... That's the symptom score that I use. And then interview. I will very commonly ask about sleep habits. I will very commonly ask about nutrition habits. I am very aggressive in making, I guess my patients go to school."} (P3)
\end{quote}
For the majority of concussion patients, the best treatment available is to have a rest both physically and cognitively. Thus, concussion clinicians will have a conversation with patients to provide personalized lifestyle recommendations and self-evaluation methodologies for concussion recovery. 
For example, P5 stated that they often provide patients with a symptom log and suggested that patients use it to track their daily activities. 
Patients will be asked to return the symptom logs during follow-up appointments, which typically occur after one week.
Symptom logs are essential for concussion clinicians to monitor their patients’ weekly health changes and concussion recovery progress.
If symptoms persist beyond 28 days, further interventions are initiated, such as referrals to neuropsychologists or specialized concussion clinics (P3, P5, P6). 
%\textcolor{red}{the following quote is not good -- critical information is missing}
% Youth aged 11 to 17, particularly athletes, are often taken to hospitals for treatment after injuries. 
% One concussion clinician (P4) mentioned that primary care physicians normally conduct an initial quantitative assessment of the child's condition and engage in direct communication to gather additional information. 
% If they determine that the patient requires more specialized concussion treatment, they refer them to concussion clinicians. 
% Concussion clinicians have dedicated clinical hours each week, during which they see both new concussion patients and conduct follow-up visits for those previously evaluated. 
% During clinical visits, they assess the severity of the concussion using symptom scoring scales recommended by international guidelines. 
% In addition, they conduct patient interviews to gain further insight into the patient’s condition. 
% Typically, clinicians advise patients to rest as needed while maintaining as much normalcy in their daily routines, such as attending school. 
% Clinicians (P5) reported that they often provide patients with symptom logs to record their daily experiences, which they are asked to bring back for their follow-up visits. 
% A follow-up appointment is usually scheduled after one week, at which point the concussion symptom scale is reassessed. 
% If a patient experiences persistent and severe symptoms for more than 28 days, clinicians (P6, P3, P5) indicated that they refer the child for further evaluation and treatment, which may include specialized concussion clinics or neuropsychological assessments.
\begin{quote}
    \textit{"Yeah, after we see them (patients) for the initial visit, we generally send them home with the log and ask them to fill it out every day based on how they felt, and then bring it back when they come back for their follow up appointment so that we can track how their symptoms have been going over time."} (P5)
\end{quote}
Concussion clinicians (P1, P2, P3, P5, P6) emphasized that mental health concerns are highly prevalent among youth patients. Thus, mental health assessments are an integral part of the concussion assessment process. 
Concussion clinicians (P6, P4) typically review the patient’s mental health history, including prior diagnoses, current treatments, and medication use. 
% According to P6, 10\% to 20\% of patients have a preexisting diagnosis of anxiety or depression. 
In addition, clinicians leverage standardized questionnaires, such as the Patient Health Questionnaire-9 (PHQ-9)~\cite{levis2019accuracy} and the Generalized Anxiety Disorder 7-item Scale (GAD-7)~\cite{mossman2017generalized}, to identify potential symptoms of mental health issues and the severity of symptoms. 
Concussion clinicians (P4, P6) also observe signals of emotional distress during the conversation with patients to detect potential mental health issues and will follow up with additional mental health screening if such signals are presented.
% To assess current symptoms, clinicians administer standardized questionnaires, such as the Patient Health Questionnaire-9 (PHQ-9) and the Generalized Anxiety Disorder 7-item Scale (GAD-7). 
% While some patients do not explicitly report mental health symptoms, clinicians (P4, P6) stated that they observe signs of emotional distress during physical examinations. 
% If such signs are present, they conduct additional mental health screenings. 
P6 reported that there is a high portion of concussion patients develop anxiety or depression following their injury. 
In such cases, P4 often recommends cognitive behavioral therapy as the intervention strategy for the mental health issues mentioned above. 

However, if mental health symptoms are found to impede concussion recovery, concussion clinicians will need to collaborate with neurologists to determine the intervention plan, such as prescribing appropriate medications. 
For patients exhibiting severe mental health symptoms, clinicians (P2, P3) will require the patients to complete an additional suicide risk assessment. 
If no immediate suicide risk is detected, clinicians typically recommend continued rest at home with ongoing symptom monitoring. 
However, P2 mentioned that 10\% of youth concussion patients develop suicidal thoughts after a concussion. 
In such cases, clinicians refer patients to mental health specialists, such as psychiatrists or psychologists.
%for dedicated and more professional evaluation and treatment.
\begin{quote}
    \textit{``In general, of people who I see with like symptoms of like anxiety, depression, things like that after concussion. It's probably about 40 to 50 percent.''} (P6)
\end{quote}



%\subsubsection{Clinicians' Challenges in Tracking Patients' Mental Health Information at Home}
%\subsubsection{Clinicians’ Challenges in Tracking Patients’ Mental Health-Related Information Outside of Clinics and Identifying Its Severity}
\subsubsection{Clinicians’ Struggle to Track and Assess Mental Health Beyond the Clinic}\label{sec:trackinginfo}

As mentioned above, the first 28 days following a concussion diagnosis is a critical period of time for youth concussion patients' recovery.
However, patients only visit the concussion clinic every one or two weeks, while spending the majority of time outside the clinic, more specifically, outside of clinical supervision.
% As a result, concussion clinicians (P2, P5) emphasized the importance of monitoring multitudes of mental health symptom indicators.
This gap creates significant challenges for concussion clinicians, who (P2, P5) emphasize the importance of monitoring a wide range of mental health symptom indicators during the concussion recovery period. 
Specifically, clinicians (P1, P2, P3, P4, P5) mentoined the importance of collecting detailed information about sleep patterns (e.g., sleep hours and wake-up times), and physical activity levels. 
P3 emphasized that the sleep patterns has not really been integrated into clinical systems.
%concussion statements
Moreover, clinicians (P2, P3, P6) are highly interested in youth patients' academic performance (whether patients completed assignments on time) and social behavioral patterns (whether patients rest at home consistently and maintain social connections rather than withdrawing socially) during concussion recovery time at home. These academic and social behaviors provide critical insights for clinicians to assess and reflect on the severity of youth patients' mental health issues during concussion recovery.
\begin{quote}
    \textit{``From their (patients) activities, feeling [that they have] little interest or in doing activities that they normally would. Seeing difficulty or decreased performance in either activities or or school.''} (P6)
\end{quote}
%\begin{quote}
%    \textit{``......if the worry is getting to the point where they (patients) can't do their homework...... Or if they aren't leaving the house… not sleeping, really feeling terrible, and it seems to be consuming them... Then I would say that might like cross the line from just typical worry into like, okay, this is more of a mental health anxiety, you know, diagnosis type of thing.''} (P2)
%\end{quote}
Furthermore, P4 stated that to capture mental health-related anomalies, the clinician mainly relies on nonverbal cues during the clinical visits.
If any abnormalities are detected, youth patients would be asked questions directly about their mental health conditions, such as suicidal tendencies, to identify the severity of their mental health symptoms.
P5 stated that identifying the severity of the symptoms is crucial for clinicians, as it impacts the clinician's ability to adjust their treatment plans in a timely manner.
Moreover, youth concussion patients may experience impairment in functioning or panic attacks, and concussion clinicians will then focus on understanding what they had been through and what events triggered the episode. In this way clinicians can evaluate patients' mental health condition more accurately as well as design a more personalized concussion treatment plan, as suggested by P2. 
\begin{quote}
    \textit{``Anybody who has a high report of anxiety, you know, as one of the symptoms or nervousness that we ask on our concussion symptoms log or if they have an underlying diagnosis of an anxiety, disorder, depression. And then, you know, maybe a little bit more likely to refer them (patients) earlier on.''} (P5)
\end{quote}
%\begin{quote}
%    \textit{``......I think, having more information about really what was happening at the time, that they had a symptom. What was triggering, that how that we know more real time information could be helpful.''} (P2)
%\end{quote}


Despite the critical need for timely mental health-related information of patients at home, concussion clinicians (P2, P5) reported significant challenges in accurately and reliably collecting the information outside the clinic.
Ideally, clinicians would benefit from having real-time access to patients’ accurate mental health-related information while patients are at home. 
However, such a system is currently unavailable.
Instead, clinicians must rely on patients’ recall of their living experiences during clinical visits to reconstruct what happened during the recovery period. 
Although these discussions could help inform clinical decision-making, patients’ memory recall is often incomplete or inaccurate, especially when clinical visits span several weeks. 
This lack of reliable data collection methodology can hinder concussion clinicians’ understanding of the patient’s recovery progress and delay timely mental health interventions, such as referrals or additional assessments.
Once youth concussion patients leave the clinic, clinicians have no way to obtain this information in a timely manner.
\begin{quote}
    \textit{``...more real-time information could be helpful. Because by the time we see them (patients) in the clinic that could be you know, weeks later they might not even remember that that (mental health sequelae-related symptoms) had happened...... It's hard for anyone to remember what they did, you know, weeks before.''} (P2)
\end{quote}


Youth concussion patients currently have some remote communication options. One way is making phone calls with the provider team and reporting patients' mental health-related information regularly to clinicians when they are outside of the clinic. 
% Currently, youth concussion patients have some remote communication options for sharing their health information with clinicians while at home. One common method is through phone calls. 
However, clinicians (P2, P3) reported several limitations with phone communication. First, youth concussion patients rarely initiate calls to their concussion clinicians. Moreover, even if youth concussion patients make a phone call, clinicians may not always be in the office or available to answer immediately, which can delay the transmission of critical health information.

An alternative option is using MyChart, a patient portal that allows patients to access their medical records and send messages to their healthcare providers. 
Concussion clinicians (P4) strongly recommend that patients use MyChart to share health-related information, particularly when youth concussion patients are reluctant to make direct phone calls. 
Compared to phone communication, MyChart offers the advantage of asynchronous messaging, enabling clinicians to review patient information and respond at their convenience. 
Despite its potential benefits, the adoption of MyChart is limited in current clinical scenarios.
Concussion clinicians (P2, P6) noted that youth patients rarely use the messaging feature in MyChart and rarely take the initiative to contact clinicians.
Instead, parents of these youth patients are more likely to use MyChart to communicate with clinicians.
%which impedes the information flow between patients and clinicians. 
% As a result, MyChart is not a frequently used tool for communication between clinicians and youth concussion patients, making it difficult for clinicians to gain insights into patients' health conditions at home.

\begin{quote}
    \textit{``One in 10 [patients] maybe, or even one in 20 [patients] might send a message with a question.''} (P6)
\end{quote}

Furthermore, clinicians caution against relying on MyChart for emergencies, especially when patients experience suicidal tendencies. 
P2 warned that their busy schedules prevent them from checking messages in real time, which could lead to delays in critical interventions and put patients at risk.

% Because concussion clinicians are unable to check messages from patients or their parents on MyChart in real time due to their workload, this may lead to delays in critical interventions, potentially compromising patient safety.

\begin{quote}
    \textit{``It's not really immediate. You know, cause they (patients) might send it at night, and then we don't maybe don't see it till the next afternoon, and you know. We usually, you know, tell them if it's an emergency, don't use that method because we don't really know when we're gonna see them or be able to respond to it.''} (P2)
\end{quote}




%\subsubsection{Concussion Patients and Caregivers May Be Uncooperative with Concussion Clinicians' Communications for Mental Health-Related Information}
%\subsubsection{Clinicians’ Challenges in Communicating With Patients and Families About Patients’ Mental Health Issues}
\subsubsection{Clinicians’ Struggle to Communicate Mental Health Issues with Patients and Their Families}\label{sec:communicationissues}


Communication challenges persist not only outside the clinics but also during in-person consultations.
% Clinicians not only receive little communication from patients between clinical visits but also encounter communication challenges during the visits themselves. 
During in-person visits, parents often accompany their children. However, some youth concussion patients feel uncomfortable disclosing sensitive mental health issues with their parents present.
P1 described a case where a patient with severe anxiety disclosed self-harm and suicidal thoughts only after requesting to speak privately, away from their parents' presence. 

% Clinicians believe patients might not be ready to discuss mental health openly.

\begin{quote}
    \textit{``I stepped out of the [room], I had the mom stay in the room...... and I said, 'Is there something wrong? Is there something going on?' He said, 'Yeah, I need to talk to you privately.' ''} (P1)
\end{quote}
%\begin{quote}
%    \textit{“There were different behavioral patterns and mood[s] that I would see in the
%office that concerned me for mental health. I had brought it up, and both he (the patient) and the mom declined and said everything was fine... I think there's a misconception that nobody wants to be labeled. It's easier to deal with an injury that's more visible or socially acceptable."} (P1)
%\end{quote}

Clinicians suspect that a major reason for this is patients' fear of mental health stigma~\cite{apa_stigma}.
The label of having a mental illness can affect how patients are perceived socially, which could potentially isolate them from their peers.
Another reason not to disclose mental health issues is the societal expectation toward youth concussion patients.
Specifically, if youth patients are athletes who are expected to perform at a high level of competition, they may struggle with the societal expectation of their physical recovery. 
%This highlights the willingness of mental health disclosures is intertwined with many societal factors that may not be directly related to mental health symptoms.

Not only do patients withhold their mental health condition, but sometimes their parents also deliberately withhold information about the patient's mental health condition as reported by concussion clinicians (P1, P4, P5).
%Moreover, some concussion clinicians (P1, P4, P5) reported that sometimes both youth concussion patients and their parents may deliberately withhold information about the patient’s mental health. 
In one case encountered by P4, the youth concussion patient was a successful athlete, and the parents downplayed their child’s mental health symptoms during clinical visits. 
These parents may worry that prolonged rest due to mental health sequelae could hinder their child’s athletic performance and career trajectory. 
So the parents downplayed their child’s mental health symptoms, hoping for their child to return to their athletic career sooner.
However, when mental health issues are not disclosed and addressed in a timely manner, the issues could get worse and lead to severe mental health issues, such as self-harm or an increased risk of suicidal behavior.
The consequences of severe mental health issues could lead to long-term physical and cognitive impairment of the patients, which might eventually hinder an athlete's ability to return to sports in their lives.
\begin{quote}
    \textit{``Their (patients) parents are very vested in the child's athletic career......the parents come in and there's obviously an agenda to minimize their symptoms. Or you know, just try not to buck the system, if it's gonna limit their (patients) activity.''} (P4)
\end{quote}


To detect undisclosed mental health issues, clinicians rely on a combination of strategies.
One approach is to build trust and collaborate directly with parents. 
Another is to observe the patient’s behavior during consultations. 
For example, clinicians (P1, P4) observe patients' behaviors, such as signs of anxiety or a persistently quiet and disengaged attitude, during clinical visits. 
% When youth concussion patients do not disclose mental health problems, there are ways to detect potential issues during clinical visits. 
% One way is to talk to patients' parents directly, try to build trust, and team up with them. 
% Another way is through observation. 
% For example, clinicians (P1, P4) can observe patients' behaviors, such as signs of anxiety or a persistently quiet and disengaged attitude, during clinical visits. 
% These indicators are highly associated with mental health issue sequelae. 
Moreover, worsened concussion symptoms such as insomnia and low activity levels may indicate underlying issues beyond the concussion itself, such as mental health sequelae.
P3 stated that if a youth patient's concussion symptoms do not improve within four weeks after a concussion, it may indicate underlying mental health issues that are contributing to a prolonged recovery period.
However, observing patients' behavior or asking about their sleep patterns requires patients to be physically present during the clinical visit.
Once the patients return home, clinicians cannot access this information.
%For example, one concussion clinician (P1) noticed the mental health problems of youth patients by observing the patient's behavior and reaction to the clinician's mental health-related questions. 
%\begin{quote}
%    \textit{"Something seemed off... I wanted to ask him (the patient) some mental health questions directly... He (the patient) got very, very anxious, and stated he needed to use the restroom, and very anxious, so I knew right away something was wrong."} (P1)
%\end{quote}
\begin{quote}
    \textit{``Because we expect 96\% of concussions to be healed by the four-week point. And so if it's not healed yet, then it's probably not the concussion, right? It's probably either a headache syndrome or a mental health concern.''} (P3)
\end{quote}


\subsubsection{Clinicians’ Struggle to Evaluate Patients’ Compliance Beyond the Clinic}\label{sec:compliance}


% Even if patients disclose their mental health status to clinicians, challenges still arise during each clinical visit. 
At the end of each clinical visit, concussion clinicians typically provide patients with recommendations related to physical activity levels or sleep schedules.
These clinical recommendations are intended to serve as the concussion treatment plan to support both physical and mental recovery after a concussion.
However, understanding whether patients adhere to these recommendations is a persistent challenge.
Although concussion clinicians (P2, P4) can rely on patients’ self-reports to assess compliance during the time period between the last visit and the current one, these reports are often unreliable. 
For instance, some patients may claim to have followed the recommendations, but their responses to mental health questionnaires reveal high levels of anxiety or depression, indicating limited or no improvement.


 
% Currently, clinicians rely only on patients' self-reports to assess adherence. 
% When clinicians asked their patients about their compliance, some patients claimed to have followed the recommendations. 
% Nevertheless, clinicians often question the reliability of these self-reports. 
% In some cases, patients’ responses to mental health questionnaires show high levels of anxiety and depression, indicating a lack of improvement despite claims of adherence to the recommendations.

\begin{quote}
    \textit{“What we recommended, if they (youth patients) [said they] did it or not, they might not have done it... ... We don't know if they're following them or not. Or they come in, and they say they did, but we don't know if they did.”} (P2)
\end{quote}

However, there is currently no effective way to track patients' compliance outside the clinic. 
P4 mentioned that they often ask patients' parents about their children's adherence to the recommendations, but the information provided by the parents may also be unreliable.
One reason could be that some parents are frequently away from home due to work commitments, which leaves them with limited knowledge of patients’ daily health-related information. 
% only responsible for bringing the patients to the clinic. 
% The parents are frequently away from home due to work commitments, which leaves them with limited knowledge of the patient’s daily health-related information. 
In such cases, clinicians may reach out to other family members by phone for a closer understanding of the patient’s mental health issues when they are at home. 
% However, making phone calls is time-consuming, and the information obtained is often fragmented and insufficient for comprehensive assessments.
However, making such calls consumes a considerable amount of time in clinicians' busy schedules, and the information collected through these calls is often limited. 
% Moreover,  clinicians normally have busy clinical schedules and are unable to gain every patient information thoroughly.
\begin{quote}
    \textit{``I think sometimes the parent comes in with a child who isn't around them (parents) that much at home. Like maybe it's the working parent, and they're just sort of the driver, and they don't have a lot of information...... Sometimes I can't get good information cause they (parents) are not engaged as much as I would, you know, would hope.''} (P4)
\end{quote}
% Clinicians (P2, P3, P6) reported that one of their major concerns regarding youth concussion patients with mental health sequelae is the lack of sufficient clinical resources to meet the demand for patients' clinical visits. One significant issue is the time required to schedule appointments with mental health counselors or professionals. Youth concussion patients often face long wait times, typically 2 to 3 weeks, to secure an appointment with a clinician (P1). This delay can lead to further deterioration of their condition. For example, symptoms that remain unmanaged during this period may worsen their mental health sequelae such as anxiety or depression. Additionally, delayed interventions can increase the complexity and difficulty of recovery.

% “I don't think I could. If I had 20 patients, and data was provided between visits, I'm not sure I would have a ton of 
% time to devote a true thought process to it.” (P)

% Another critical issue is the shortage of human resources. Scheduling appointments with mental health counselors or professionals often involves even longer wait times. 

% “...sometimes one of the biggest barriers we have is access. And getting people into mental health. Like counselors can be so much difficult.” (P)

% \subsubsection{Clinicians' Challenges in Differentiating Mental Health Sequelae After a Concussion from Pre-existing Mental Health Issues}


%1 clinicians said that mental health issues in concussion patients are not the same -- one is caused by concussion, namely m.h. sequelae; the other is pre-existing mental health issues that do not originate from concussion.
%2 concuccions clinicans stated that it is important for them to identify which type of m.h. issue while they were treating concussion patients because the concucssion treatment plan needs to be adjusted based on the m.h. types







%determine if it is mental health sequelae

%related to mental health sequelae
%sequelae - plan a
%no sequelae plan b
%concussion treatment decision making
%AI support decision making (early prediction)


%Concussion clinicians (P4, P5) routinely review patients' mental health histories during consultations to assess whether they have pre-existing mental health conditions. 
%They examine whether patients are undergoing mental health treatment or taking medication for anxiety or depression. 
%Clinician (P2) noted that youth concussion patients with pre-existing anxiety or depression often have a more negative outlook on recovery. 
%These patients may experience excessive worry about their healing process, leading to panic symptoms. 
%This not only exacerbates their underlying mental health conditions but also negatively impacts their concussion recovery. 
%If a clinician (P2) determines that anxiety or depression symptoms are complicating youth patients' concussion recovery, clinicians collaborate with other specialists, such as neurologists, to prescribe appropriate medications. Additionally, if a patient exhibits suicidal tendencies, they are promptly referred for a safety assessment and safety planning.
%\begin{quote}
 %   \textit{“So we do suicide screening. So if someone is screening positive, we might have to do a more detailed safety assessment and safety planning. We also have a neurologist that I work with, so sometimes they might prescribe medication to treat anxiety or depression symptoms, if they (patients) are seeming to be, you know, complicating their concussion recovery.”} (P4)
%\end{quote}

%However, concussion clinicians face challenges in making timely mental health-related decisions.
%In some cases, concussion clinician (P4) suspects that a patient may be approaching a mental health crisis, despite having no prior clinical diagnosis of a mental health condition. 
%In such situations, clinicians must carefully assess the patient’s mental state to determine whether immediate intervention is necessary. 
%To gain a clearer understanding, clinician (P4) sometimes asks youth concussion patients about their mental health status before the concussion to establish a baseline for comparison. 
%However, clinician (P4) noted that youth patients’ self-reported information is often inaccurate, making it difficult for clinicians to conduct an solid evaluation. 
%Although clinicians sometimes use standardized anxiety or depression screening tools, they primarily rely on nonverbal cues to detect mental health sequelae and directly ask about suicidal thoughts when necessary. 
%However, clinician (p4) point out that mental health symptoms may sometimes appear vague or inconsistent with a patient’s previous condition, which makes it difficult for them to clearly assess the patient’s mental health status and adjust the treatment plan in a timely manner based on the patient’s mental health conditions.
%\begin{quote}
   % \textit{"I think we sort of have to do is if we get somebody (patients) who's undiagnosed and approaching crisis to try to figure out. You know, do they need immediate help? Do they need, you know, or can we make a referral to behavior health and get them connected? So we probably do more acute intervention, just selectively when the situation arises."} (P4)
%\end{quote}
%_________

