
\section{Evaluation Study of the Preliminary System Design}

To evaluate the usefulness of our preliminary design, we conducted an evaluation study.
The goal of the evaluation study is to gather clinicians' feedback and insights on our preliminary designs with different AI-RPM technologies. Specifically, we want to explore (1) whether they find these AI-RPM technologies useful for remotely monitoring mental health-related information of youth concussion patients and (2) how to revise the preliminary system design from data type, visualization, interaction, and usability perspectives to better support clinicians in addressing challenges and decision-making in their workflow.

\subsection{Procedure}
We conducted an evaluation study with the same six clinicians from our formative study since they are the target users of the AI-RPM system. 
Due to the significant time constraints faced by concussion clinicians, each evaluation session was limited to 10 minutes with each clinician. We conducted a Zoom screen-sharing session to present our preliminary designs to each clinician and collected verbal feedback regarding data type, visualization, interaction, and usability. Meanwhile, our research team documented their responses through audio recording and note-taking. After the session, two researchers applied axial coding to identify codes and key themes. Afterward, we iterated on our preliminary designs based on our findings from this design interface evaluation. The detailed interview protocol for our evaluation study is provided in Appendix~\ref{sec:appendixb}.

\subsection{Findings} 
In this section, we present five key findings derived from concussion clinicians' feedback during the evaluation study.

%\subsubsection{AI Risk Score Prediction as a Supportive Tool for Clinicians in Patient-Clinician Communication}
\subsubsection{Improving Patient-Clinician Communication With AI Risk Score Prediction}
Concussion clinicians (P1, P5, P6) reported that if the AI risk score prediction module provides accurate data, it can be very useful for timely clinical adjustments, such as scheduling follow-up visits or providing referrals.
\begin{quote}
    \textit{``That (the AI risk score prediction) might be something helpful early on for them (concussion clinicians) to say, hey, this is somebody who probably gonna benefit from seeing another provider earlier rather than later.''} (P5)
\end{quote}
Moreover, P1 mentioned that the AI risk score prediction module is not only easy to understand but also has the potential to serve as an evidence-based tool to effectively support clinicians in communicating with patients and their parents about mental health conditions during clinical visits.
By directly presenting the risk scores of the likelihood of developing mental health sequelae, youth patients and their families may gain a clear understanding of the severity of patients' mental health conditions and be willing to talk about it, particularly for youth patients and families who are hesitant to discuss mental health issues with concussion clinicians. 
\begin{quote}
    \textit{``I think it's pretty simple, and if I was able to show that to a family... maybe something as simple as this to show [the patient's] mom... 'I'm concerned about your son... there's a 50 percent risk that [your son] may develop [mental health sequelae]. We should really be in touch with it.' Yes, I think this is great.''} (P1)
\end{quote}
However, P3 found the AI-generated risk score unhelpful because it covered one year, while she only treated youth concussion patients for four weeks.
If patients did not recover within this period, they would be referred for further treatment.

\subsubsection{Supporting Clinicians' Understanding with AI-Generated Results and Data Source Transparency}
\label{subsubsec:explanation}
Concussion clinicians expressed confusion regarding the data sources and AI results in both the AI risk score prediction modules and self-reported symptoms modules. 
When we presented the AI risk score prediction module, P6 sought for a clear explanation of the meaning behind the AI risk score despite the interface already providing key data that contribute to the risk score as well as a brief explanation of the score. 
Additionally, P2 and P4 mentioned that they were unclear about how patient's mental health symptoms were collected in the self-reported symptoms module.
Specifically, P2 was unsure whether the symptoms (e.g., suicidality and depression), were from AI risk score prediction modules or from questionnaires (e.g., GAD-7 and PHQ-9) that clinicians gave to their patients for self-assessment. However, the symptoms are collected by LLM-powered CAs through check-in conversations with youth patients.
\begin{quote}
    \textit{``So it's just my confusion. So when it says self-reported symptoms that's based on like a questionnaire that we've given them about their depression and anxiety level? Or this is what AI predicts that they have a 50 percent chance of like developing depression based on that?''} (P2)
\end{quote}



\subsubsection{Enriching Clinicians' Insights with Data from LLM-based CAs and Wearables}
\label{subsubsec:cawearables}
Clinicians reported that the data collected from CAs and wearables could be valuable in their workflow. 
Specifically, P6 mentioned that he would have a quick review of data on the self-reported symptoms module collected by CAs before each clinical visit to better understand youth patients' current health conditions. 
Moreover, concussion clinicians (P3, P6) believed that patients might be more willing to share information with a conversational agent than directly with a clinician, which makes the data collected by CAs a valuable additional data source. 
However, P3 suggested that rather than using CAs to ask patients about their mental health conditions daily, it would be more appropriate to collect the data weekly. This suggestion helps avoid constantly reminding patients of their injuries.
\begin{quote}
    \textit{``I guess from a perspective of maybe they (patients) would tell this device things that they wouldn't tell me. Possibly I could see that.''} (P6)
\end{quote}
Moreover, concussion clinicians (P3, P4, P5) particularly appreciated the data in the sleep and physical activities module collected by wearables. They highlighted that data such as sleep hours is currently missing from EHR systems, yet wearable-collected data could serve as key indicators for mental health sequelae and concussion recovery progress. 
Additionally, P3 believed that clinicians could use the sleep and activity data to assess whether patients have followed sleep and exercise recommendations, which can be useful for addressing one of their main challenges.
\begin{quote}
    \textit{``This one (the Sleep and Physical Activities panel) to me might be the most helpful [one]... I think that's always hard [to get].''} (P5)
\end{quote}
%\begin{quote}
%    \textit{"So I really like the sleep and physical activity, but I would suggest that the patients not see that data that it only goes to the physician. Because I don't want it to cause them anxiety. So yeah, I really like that."} (P3)
%\end{quote}



\subsubsection{Diverse Clinician Preferences in Data Details and Presentation Styles}
\label{subsubsec:detaileddata}
%Needs for Questionnaire Result
Concussion clinicians showed varying preferences regarding the type and level of detailed data presented within the sleep and physical activity module. 
Clinicians (P2, P3, P5) suggested incorporating napping-related data into the module, including whether the patient took naps and their duration. 
Additionally, clinicians (P1, P3) emphasized the importance of heart rate in assessing concussion recovery and monitoring mental health conditions. Specifically, they expressed interest in viewing daily resting heart rate during sleep over the course of a week. 
P1 noted that a heart rate in the fifties to sixties indicates a calm state in patients. A calm state is often associated with obtaining eight hours of quality sleep. 
In contrast, if a patient experiences fragmented or restless sleep, their resting heart rate increases. 
By analyzing physiological data, clinicians can assess if symptoms are linked to mental health or other non-concussion issues.
\begin{quote}
    \textit{``If someone's recovering, I would expect it (heart rate) to be low, [meaning] they're calm......If you (patients) are having a light, irritable kind of sleep, I would expect the resting heart rate to be in the seventies or eighties. [It means] maybe they (patients) are anxious. Maybe there's something else going on.''} (P1)
\end{quote}
Moreover, in the self-report symptoms module, clinicians (P3, P5) suggested they want to include dietary information, such as appetite or nutrition, as they noted that when youth patients experience depression or high levels of anxiety, both the patients and their parents often report decreased appetite. Importantly, instead of focusing solely on whether youth patients have mental health symptoms, concussion clinicians (P3, P5) were more concerned with the duration and severity of these symptoms. Because the severity and duration of symptoms have different impacts on clinicians' decision-making, such as treatment plans.
%and potentially help clinicians distinguish between general emotional issues and mental health sequelae following a concussion. 
\begin{quote}
    \textit{``I don't think it has to be high level [of mental health symptoms]...... persistent or non-improving symptoms. That's what I look for.''} (P5)
\end{quote}
Additionally, P3 expressed a new need for a feature that could indicate how adherence to sleep and exercise recommendations impacts patients' recovery, suggesting that such insights could enhance clinical decision-making. 
%\begin{quote}
 %   \textit{"I don't know if my patients are following my recommendations and knowing if they are and how that affects their recovery would be really interesting."} (P3)
%\end{quote}


\subsubsection{Concerns of Clinicians in Emergency Situations and Time Constraints}
In the self-reported symptoms module, concussion clinicians placed significant emphasis on alerts related to suicidal tendencies. P1 emphasized the need for immediate intervention when suicidal tendencies emerge in youth concussion patients. 
However, clinicians expressed several challenges. 
Firstly, there is an absence of established protocols for managing such emergency situations. 
%Moreover, managing emergency interventions without triggering resistance from parents or youth patients remains a highly sensitive and complex challenge. 
Furthermore, P4 mentioned that the youth generation experiences frequent mood swings and may sometimes exaggerate their feelings, which could lead to false alerts.
However, failure to intervene in patients who have suicidal tendencies in a timely manner could result in a life-or-death consequence. 
To address this, clinicians (P2, P4) believed that youth patients' parents should be brought into the loop and have a partnership with parents when emergency situations arise.
\begin{quote}
    \textit{``So I think this is when you need to loop the paradigm and say, hey, your child has been identified through our software. You know, cause I think, especially since we're dealing with minors, you're gonna have to get the parents loop too.''} (P4)
\end{quote}
Moreover, clinicians (P2, P5) expressed that they have limited time to thoroughly review such detailed data due to heavy workloads.
Specifically, P1 highlighted that processing large amounts of data between tightly scheduled appointments—sometimes up to 20 patients per day—makes it nearly impossible to dedicate sufficient time and energy for in-depth analysis. 
If they have the AI-RPM system, clinicians noted that they might skim through the most relevant data quickly on the day of the appointment or shortly before to gain a basic understanding of their patient’s condition.
\begin{quote}
    \textit{``I think that's gonna that would be really  tedious to try to monitor and create extra work for us.''} (P5)
\end{quote}








