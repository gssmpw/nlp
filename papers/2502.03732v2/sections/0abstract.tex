\begin{abstract}
Anxiety, depression, and suicidality are common mental health sequelae following concussion in youth patients, often exacerbating concussion symptoms and prolonging recovery. 
Despite the critical need for early detection of these mental health symptoms, clinicians often face challenges in accurately collecting patients' mental health data and making clinical decision-making in a timely manner.
Today's remote patient monitoring (RPM) technologies offer opportunities to objectively monitor patients’ activities, but they were not specifically designed for youth concussion patients; moreover, the large amount of data collected by RPM technologies may also impose significant workloads on clinicians to keep up with and use the data. 
To address these gaps, we employed a three-stage study consisting of a formative study, interface design, and design evaluation.
We first conducted a formative study through semi-structured interviews with six highly professional concussion clinicians and identified clinicians' key challenges in remotely collecting patient information and accessing patient treatment compliance.
Subsequently, we proposed preliminary clinician-facing interface designs with the integration of AI-based RPM technologies (\textbf{AI-RPM}), followed by design evaluation sessions with highly professional concussion clinicians. 
Clinicians underscored the value of integrating multi-modal AI-RPM technologies to support their decision-making while emphasizing the importance of customizable interfaces through collaborative design and multiple responsible design considerations.
\end{abstract}