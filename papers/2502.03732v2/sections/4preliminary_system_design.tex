
\section{Preliminary System Design}
%Building on these identified needs and challenges from our formative study, 
In the formative study, we identified clinicians' urgent need to remotely monitor the mental health of youth concussion patients at home, as well as the challenges they face in clinical decision-making.
In this section, we explored the design opportunities of AI-RPM technologies in supporting clinicians’ needs and decision-making during their practice. 
%We first proposed AI-RPM technologies that could potentially assist clinicians in remotely monitoring patients' at-home data as well as addressing decision-making difficulties. 
Subsequently, we proposed a preliminary system design (Fig. \ref{fig:preliminary_system_design}) based on the AI-RPM technologies with detailed features. 
The goal of the design is to explore whether concussion clinicians find the AI-RPM system useful for supporting remote monitoring and decision-making, how the system can be improved, and how they would use it in practice.

%\subsection{Leveraging AI-RPM Technologies for Remote Youth Concussion Patient Monitoring}
\subsection{Design Considerations for AI-RPM Systems}

\subsubsection{Wearables for Tracking Patient Physiological Data}
\label{sec:wearblesforsleepandactivitiestracking}

Based on Section \ref{sec:trackinginfo}, clinicians expressed a desire to monitor patients’ at-home quantitative data, such as sleep and exercise patterns, as these factors are closely related to concussion recovery and mental health status. 
In addition, as noted in Section \ref{sec:compliance}, clinicians frequently ask patients about their clinical recommendation compliance, such as maintaining proper sleep and exercise routines.
However, clinicians currently lack effective ways to access these at-home data and often struggle to verify the accuracy of patients’ self-reported information during clinical visits.

%To address these challenges, wearable devices such as smartwatches can help
To address these challenges, wearable devices such as smartwatches can be used to help clinicians collect real-time sleep and exercise data from patients~\cite{dias2018wearable, baig2017systematic, bate2023role}. 
Wearables offer a non-invasive and natural way to monitor patients’ health metrics without significantly disrupting their daily routines. 
Moreover, AI has the potential to process the massive amount of data collected from wearables and provide these data to clinicians for timely review.
We believe that with access to at-home patient data, clinicians can become aware of patients' health conditions in real-time, and track patients' compliance with sleep and exercise recommendations, thus ultimately having the ability to make more informed mental health-related clinical decisions, such as setting up a follow-up meeting for further assessment.


\subsubsection{LLM-based CAs for Patient Self-Reporting in Mental Health}
\label{sec:casforselfreport}

In Section \ref{sec:trackinginfo}, clinicians emphasized the need for a comprehensive understanding of patients' experiences with mental health symptoms, such as panic attacks, to provide effective care. However, patients often struggle to accurately recall these details, and such subjective experiences cannot be easily quantified using standardized scales or medical instruments. Additionally, based on Section \ref{sec:communicationissues}, concussion clinicians reported that youth concussion patients sometimes conceal their mental health issues from their parents. It was only during a private conversation with the clinician that the patient disclosed their mental health status.

Given this challenge, we hypothesize that LLM-based CAs~\cite{yang2024talk2care, mahmood2023llm, chan2024human} could help clinicians collect mental health-related information effectively. 
CAs can be deployed in patients' homes to engage in natural language conversations, ask relevant questions, and record dialogue content. 
AI can then analyze historical conversations to summarize the key information of the conversation, such as their patient's suicidal intentions. 
From the perspective of youth patients, the deployment of a CA in their rooms could increase the chance of disclosing their mental health issues, if there are any.
One study found that children perceive conversational agents as warmer and more reliable than their parents~\cite{van2023children}.
From the perspective of clinicians, gaining such key information allows them to review patients' historical conversations with CAs, gain accurate information about patients' mental health-related experiences while preserving privacy, and conduct interventions if necessary.  
Therefore, we speculate that CAs have the potential to act as a communication bridge between clinicians in the hospital and patients at home and provide patients with self-reported symptoms. Ultimately, CAs could support clinicians in conducting a more accurate assessment of patients' mental health and making timely decisions.


\subsubsection{AI-based Risk Prediction for Detecting Mental Health Sequelae}
\label{sec:aifordetection}

%As discussed in Section \ref{sec:communicationissues}, 
As we have indicated in Section \ref{sec:casforselfreport}, clinicians reported that patients and their parents sometimes deliberately conceal mental health conditions. 
In addition, clinicians noted that the severity of mental health symptoms in a patient influences their subsequent clinical decisions, such as treatment plans or providing a referral. 
However, evaluating the severity of mental health symptoms remains challenging.
Given these two challenges, we believe that AI risk prediction for disease progression could provide valuable support to clinicians in our setting. 
AI has the capability to analyze patients' health data from the EHR system and utilize algorithms to predict the likelihood of developing specific conditions over a given period~\cite{collins2019reporting}. 
In particular, recent studies have already explored AI-driven risk prediction for mental health sequelae following a concussion and have found that AI predictions can attain high accuracy~\cite{dabek2022evaluation}.
By leveraging AI risk score prediction, clinicians may no longer have to rely solely on the conversations with patients during clinical visits to make mental health assessments. 
If patients or their parents choose to withhold information, AI could help clinicians detect potential mental health sequelae in a timely manner. 
Moreover, the severity of the mental health symptoms is normally associated with the AI prediction score. 
The AI prediction score can help clinicians better assess the severity of a patient's mental health symptoms and make informed decisions accordingly.
%If a patient is flagged as having a high risk of developing mental health sequelae, the clinician can proactively schedule follow-up visits and make necessary mental health-related clinical decisions. 
%Therefore, we hypothesize that AI-driven risk prediction has the potential to assist concussion clinicians in detecting mental health sequelae more effectively.



\begin{figure*}[htbp]
  \includegraphics[draft=false,width=\textwidth]{Initial_UI.pdf}
  \caption{The preliminary system design based on concussion clinicians' challenges and needs: (A) AI risk score prediction for mental health sequelae, (B) patient self-report symptoms, (C) patient sleep and physical activities.}
  \label{fig:preliminary_system_design}
\end{figure*}

\subsection{Three Modules of Preliminary System Design Based on the Design Considerations}
Based on the potentially applicable technologies that could support concussion clinicians in addressing their challenges and supporting their decision-making, we designed three modules for our preliminary system, as shown in Fig. \ref{fig:preliminary_system_design}. 
We described each module in detail in this section.


\subsubsection{The Sleep and Physical Activities Module}
The sleep and physical activities module (Fig. \ref{fig:preliminary_system_design} B)  presents sleep (sleep hours, light sleep, and wake-up time) and activity data (step count) that clinicians want to track.
These data can be collected by wearables as we discussed on Section \ref{sec:wearblesforsleepandactivitiestracking}
Moreover, clinicians can select any category to review the patient’s relevant data from the past week. 
Clinicians can also select different dates to review patient data from other time periods. 
Additionally, the module provides the source of the data for increasing clinicians' understanding of the module and their trust towards the system. 
%The goal of this module is to assist concussion clinicians in monitoring youth concussion patients' sleep and physical activity at home, addressing the current gap in existing clinical systems that lack this information.


\subsubsection{The Self-Report Symptoms Module}
To display the patients' self-reported symptoms at home, which were collected by CAs as we proposed in Section \ref{sec:casforselfreport}, we designed the self-reported symptoms (Fig. \ref{fig:preliminary_system_design} B). 
On the right side, a history of conversations between youth concussion patients and CAs is available for concussion clinicians to review. 
Based on the conversation history, AI detects the symptoms that are mentioned by patients and highlights key information that are relevant to mental health sequelae, such as anxiety, depression, and suicidal ideation. 
If a patient self-reports symptoms within a specific category, the corresponding indicator changes from green to red.
We use the color red to catch clinicians' attention. 
To facilitate efficient navigation, clinicians can click on a category name or use the search bar to enter keywords, which prompts the right-side module to display the relevant conversation excerpts. 
Similar as the sleep and physical activities module, time selection lets clinicians view data from different periods.
%The primary goal of this module is to allow concussion clinicians to quickly assess patients' mental health symptoms at home, extending support beyond clinical settings.


\subsubsection{The AI Risk Score Prediction Module}
The AI Risk Score Prediction (Fig. \ref{fig:preliminary_system_design} A) module leverages AI's ability that we discussed in Section \ref{sec:aifordetection} to show the probability of mental health sequelae.
The prediction timeframe is within one year with a range from 0\% to 100\%, where the higher probability score signifies a higher probability of experiencing mental health sequelae for clinicians to review. 
Moreover, feature importance is used to provide data that influence the AI prediction score, which allows clinicians to decide whether to trust the AI-predicted score or not. 
To increase AI explanations and transparency, we provide text-based explanations to help clinicians understand the risk score and its source. 
At the bottom is the risk score chart. Clinicians can hover over any day on the chart to view the likelihood of developing mental health sequelae at that specific point in time. Additionally, clinicians can select different dates to review patient data from various time periods. 





