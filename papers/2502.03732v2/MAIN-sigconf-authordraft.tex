%%
%% This is file `sample-sigconf-authordraft.tex',
%% generated with the docstrip utility.
%%
%% The original source files were:
%%
%% samples.dtx  (with options: `all,proceedings,bibtex,authordraft')
%% 
%% IMPORTANT NOTICE:
%% 
%% For the copyright see the source file.
%% 
%% Any modified versions of this file must be renamed
%% with new filenames distinct from sample-sigconf-authordraft.tex.
%% 
%% For distribution of the original source see the terms
%% for copying and modification in the file samples.dtx.
%% 
%% This generated file may be distributed as long as the
%% original source files, as listed above, are part of the
%% same distribution. (The sources need not necessarily be
%% in the same archive or directory.)
%%
%%
%% Commands for TeXCount
%TC:macro \cite [option:text,text]
%TC:macro \citep [option:text,text]
%TC:macro \citet [option:text,text]
%TC:envir table 0 1
%TC:envir table* 0 1
%TC:envir tabular [ignore] word
%TC:envir displaymath 0 word
%TC:envir math 0 word
%TC:envir comment 0 0
%%
%% The first command in your LaTeX source must be the \documentclass
%% command.
%%
%% For submission and review of your manuscript please change the
%% command to \documentclass[manuscript, screen, review]{acmart}.
%%
%% When submitting camera ready or to TAPS, please change the command
%% to \documentclass[sigconf]{acmart} or whichever template is required
%% for your publication.
%%
%%


 %\documentclass[manuscript,draft]{acmart}
 \documentclass[sigconf]{acmart} %%double-column layout

\settopmatter{printacmref=false} % 关闭 ACM 版权信息
\settopmatter{printccs=false}    % 关闭 CCS 概念
% \settopmatter{printfolios=false}  % 关闭页码打印
\renewcommand\footnotetextcopyrightpermission[1]{} % 删除页脚中的版权声明
\pagestyle{plain} % 取消页眉中 DOI 和其他信息


%%
%% \BibTeX command to typeset BibTeX logo in the docs
\AtBeginDocument{%
  \providecommand\BibTeX{{%
    Bib\TeX}}}

%% Rights management information.  This information is sent to you
%% when you complete the rights form.  These commands have SAMPLE
%% values in them; it is your responsibility as an author to replace
%% the commands and values with those provided to you when you
%% complete the rights form.
\setcopyright{acmlicensed}
\copyrightyear{2018}
\acmYear{2018}
\acmDOI{XXXXXXX.XXXXXXX}
%% These commands are for a PROCEEDINGS abstract or paper.
\acmConference[Conference acronym 'XX]{Make sure to enter the correct
  conference title from your rights confirmation emai}{June 03--05,
  2018}{Woodstock, NY}
%%
%%  Uncomment \acmBooktitle if the title of the proceedings is different
%%  from ``Proceedings of ...''!
%%
%%\acmBooktitle{Woodstock '18: ACM Symposium on Neural Gaze Detection,
%%  June 03--05, 2018, Woodstock, NY}
\acmISBN{978-1-4503-XXXX-X/18/06}




% ------- Custom Color Palette-------
\definecolor{cBLUE}{HTML}{3282B8}
\definecolor{cGREEN}{HTML}{60A561}
\definecolor{cORANGE}{HTML}{FA824C}
\definecolor{cYELLOW}{HTML}{f0C808}
\definecolor{cLightGrey}{HTML}{CECECE}
\definecolor{cRED}{HTML}{ED1B23}

\definecolor{cNatureORANGE}{HTML}{EAAA60}
\definecolor{cNatureRED}{HTML}{E68B81}
\definecolor{cNaturePURPLE}{HTML}{B7B2D0}
\definecolor{cNatureBLUE}{HTML}{7DA6C6}
\definecolor{cNatureGREEN}{HTML}{84C3B7}


\newcommand{\arthur}[1]{{\bf \color{cNatureORANGE} [arthur comments: #1]}}
% \newcommand{\guiming}[1]{{\bf \color{cNatureRED} [guiming comments: #1]}}
% \newcommand{\ruishi}[1]{{\bf \color{cNaturePURPLE} [ruishi comments: #1]}}
% \newcommand{\yuxuan}[1]{{\bf \color{cNatureBLUE} [yuxuan comments: #1]}}
% \newcommand{\Sijia}[1]{{\bf \color{cNatureGREEN} [sijia comments: #1]}}
\newcommand{\dakuo}[1]{{\bf \color{cYELLOW} [dakuo comments: #1]}}




%%
%% Submission ID.
%% Use this when submitting an article to a sponsored event. You'll
%% receive a unique submission ID from the organizers
%% of the event, and this ID should be used as the parameter to this command.
%%\acmSubmissionID{123-A56-BU3}

%%
%% For managing citations, it is recommended to use bibliography
%% files in BibTeX format.
%%
%% You can then either use BibTeX with the ACM-Reference-Format style,
%% or BibLaTeX with the acmnumeric or acmauthoryear sytles, that include
%% support for advanced citation of software artefact from the
%% biblatex-software package, also separately available on CTAN.
%%
%% Look at the sample-*-biblatex.tex files for templates showcasing
%% the biblatex styles.
%%

%%
%% The majority of ACM publications use numbered citations and
%% references.  The command \citestyle{authoryear} switches to the
%% "author year" style.
%%
%% If you are preparing content for an event
%% sponsored by ACM SIGGRAPH, you must use the "author year" style of
%% citations and references.
%% Uncommenting
%% the next command will enable that style.
%%\citestyle{acmauthoryear}


%%
%% end of the preamble, start of the body of the document source.
\begin{document}

%%
%% The "title" command has an optional parameter,
%% allowing the author to define a "short title" to be used in page headers.
\title[Design Opportunities of AI Remote Monitoring for Mental Health Sequelae in Youth Concussion Patients]{More Modality, More AI: Exploring Design Opportunities of AI-Based Multi-modal Remote Monitoring Technologies for Early Detection of Mental Health Sequelae in Youth Concussion Patients}

%%
%% The "author" command and its associated commands are used to define
%% the authors and their affiliations.
%% Of note is the shared affiliation of the first two authors, and the
%% "authornote" and "authornotemark" commands
%% used to denote shared contribution to the research.


\author{Bingsheng Yao}
\authornote{Both authors contributed equally to this research.}
\author{Menglin Zhao}
\authornotemark[1]
\affiliation{
  \institution{Northeastern University}
  \city{}
  \state{}
  \country{}
}
% \author{Bingsheng Yao}
% % \authornote{Both authors contributed equally to this research.}
% \author[Menglin Zhao]
% % \authornotemark[1]
% \affiliation{
%   \institution{Northeastern University}
%   \city{}
%   \state{}
%   \country{}
% }
% \email{}

% \author{Menglin Zhao}
% \authornotemark[1]
% \affiliation{Northeastern University
%   \institution{}
%   \city{}
%   \state{}
%   \country{}}
% \email{}



\author{Weidan Cao}
\affiliation{Ohio State University
  \institution{}
  \city{}
  \state{}
  \country{}}
\email{}

\author{Changchang Yin}
\affiliation{Ohio State University
  \institution{}
  \city{}
  \state{}
  \country{}}
\email{}

\author{Stephen Intille}
\affiliation{Northeastern University
  \institution{}
  \city{}
  \state{}
  \country{}}
\email{}

\author{Xuhai "Orson" Xu}
\affiliation{Columbia University
  \institution{}
  \city{}
  \state{}
  \country{}}
\email{}

\author{Ping Zhang}
\affiliation{Ohio State University
  \institution{}
  \city{}
  \state{}
  \country{}}
\email{}

\author{Jingzhen Yang}
\affiliation{Nationwide Children’s Hospital
  \institution{}
  \city{}
  \state{}
  \country{}
  \institution{}
  \city{}
  \state{}
  \country{}}
\email{}

\author{Yuling Sun}
\affiliation{University of Michigan
  \institution{}
  \city{}
  \state{}
  \country{}}
\email{}

\author{Dakuo Wang}
\authornote{Corresponding author: d.wang@northeastern.edu}
\affiliation{Northeastern University
  \institution{}
  \city{}
  \state{}
  \country{}}
\email{}


% \iffalse
% \author{Ben Trovato}
% \authornote{Both authors contributed equally to this research.}
% \email{trovato@corporation.com}
% \orcid{1234-5678-9012}
% \author{G.K.M. Tobin}
% \authornotemark[1]
% \email{webmaster@marysville-ohio.com}
% \affiliation{%
%   \institution{Institute for Clarity in Documentation}
%   \city{Dublin}
%   \state{Ohio}
%   \country{USA}
% }

% \author{Lars Th{\o}rv{\"a}ld}
% \affiliation{%
%   \institution{The Th{\o}rv{\"a}ld Group}
%   \city{Hekla}
%   \country{Iceland}}
% \email{larst@affiliation.org}

% \author{Valerie B\'eranger}
% \affiliation{%
%   \institution{Inria Paris-Rocquencourt}
%   \city{Rocquencourt}
%   \country{France}
% }

% \author{Aparna Patel}
% \affiliation{%
%  \institution{Rajiv Gandhi University}
%  \city{Doimukh}
%  \state{Arunachal Pradesh}
%  \country{India}}

% \author{Huifen Chan}
% \affiliation{%
%   \institution{Tsinghua University}
%   \city{Haidian Qu}
%   \state{Beijing Shi}
%   \country{China}}

% \author{Charles Palmer}
% \affiliation{%
%   \institution{Palmer Research Laboratories}
%   \city{San Antonio}
%   \state{Texas}
%   \country{USA}}
% \email{cpalmer@prl.com}

% \author{John Smith}
% \affiliation{%
%   \institution{The Th{\o}rv{\"a}ld Group}
%   \city{Hekla}
%   \country{Iceland}}
% \email{jsmith@affiliation.org}

% \author{Julius P. Kumquat}
% \affiliation{%
%   \institution{The Kumquat Consortium}
%   \city{New York}
%   \country{USA}}
% \email{jpkumquat@consortium.net}
% \fi
%%
%% By default, the full list of authors will be used in the page
%% headers. Often, this list is too long, and will overlap
%% other information printed in the page headers. This command allows
%% the author to define a more concise list
%% of authors' names for this purpose.
\renewcommand{\shortauthors}{Yao and Zhao et al.}

%%
%% The abstract is a short summary of the work to be presented in the
%% article.
\begin{abstract}
We present an image blending pipeline, \textit{IBURD}, that creates realistic synthetic images to assist in the training of deep detectors for use on underwater autonomous vehicles (AUVs) for marine debris detection tasks. 
Specifically, IBURD generates both images of underwater debris and their pixel-level annotations, using source images of debris objects, their annotations, and target background images of marine environments. 
With Poisson editing and style transfer techniques, IBURD is even able to robustly blend transparent objects into arbitrary backgrounds and automatically adjust the style of blended images using the blurriness metric of target background images. 
These generated images of marine debris in actual underwater backgrounds address the data scarcity and data variety problems faced by deep-learned vision algorithms in challenging underwater conditions, and can enable the use of AUVs for environmental cleanup missions. 
Both quantitative and robotic evaluations of IBURD demonstrate the efficacy of the proposed approach for robotic detection of marine debris. 
\end{abstract}




%%
%% The code below is generated by the tool at http://dl.acm.org/ccs.cfm.
%% Please copy and paste the code instead of the example below.
%%
\begin{CCSXML}
<ccs2012>
   <concept>
       <concept_id>10003120.10003121</concept_id>
       <concept_desc>Human-centered computing~Human computer interaction (HCI)</concept_desc>
       <concept_significance>500</concept_significance>
       </concept>
   <concept>
       <concept_id>10003120.10003121.10003122</concept_id>
       <concept_desc>Human-centered computing~HCI design and evaluation methods</concept_desc>
       <concept_significance>500</concept_significance>
       </concept>
   <concept>
       <concept_id>10003120.10003130.10003131</concept_id>
       <concept_desc>Human-centered computing~Collaborative and social computing theory, concepts and paradigms</concept_desc>
       <concept_significance>500</concept_significance>
       </concept>
 </ccs2012>
\end{CCSXML}

\ccsdesc[500]{Human-centered computing~Human computer interaction (HCI)}
\ccsdesc[500]{Human-centered computing~HCI design and evaluation methods}
\ccsdesc[500]{Human-centered computing~Collaborative and social computing theory, concepts and paradigms}


%%
%% Keywords. The author(s) should pick words that accurately describe
%% the work being presented. Separate the keywords with commas.
\keywords{Youth concussion, mental health sequelae, clinical decision-making, remote patient monitoring, artificial intelligence}

%% A "teaser" image appears between the author and affiliation
%% information and the body of the document, and typically spans the
%% page.
\begin{teaserfigure}
  \vspace{-1em}
  \centering
  \includegraphics[draft=false,width=.93\textwidth]{refind_system_design.pdf}
  \vspace{-0.5em}
  \caption{Refined AI-RPM system designed to support concussion clinicians in mental health monitoring: (A) patient list, (B) patient basic information, (C) sleep and physical activities, (D) self-report symptoms, (E) questionnaire result, and (F) AI risk score prediction for mental health sequelae.
  %The interface depicts six main modules: (A) Patient Concussion Information and History; (B) Clinical Notes; (C) Explainable AI-generated Risk Score; (D) Scales Result from Self Assessment; (E) Self-Report Symptoms from LLM-powered Voice Assistant; (F) Sleep \& Physical Activities Data from Wearable Sensor.
}
  \label{fig:refined_system_design}
\end{teaserfigure}

\received{20 February 2007}
\received[revised]{12 March 2009}
\received[accepted]{5 June 2009}

%%
%% This command processes the author and affiliation and title
%% information and builds the first part of the formatted document.
\maketitle


\section{Introduction}

In today’s rapidly evolving digital landscape, the transformative power of web technologies has redefined not only how services are delivered but also how complex tasks are approached. Web-based systems have become increasingly prevalent in risk control across various domains. This widespread adoption is due their accessibility, scalability, and ability to remotely connect various types of users. For example, these systems are used for process safety management in industry~\cite{kannan2016web}, safety risk early warning in urban construction~\cite{ding2013development}, and safe monitoring of infrastructural systems~\cite{repetto2018web}. Within these web-based risk management systems, the source search problem presents a huge challenge. Source search refers to the task of identifying the origin of a risky event, such as a gas leak and the emission point of toxic substances. This source search capability is crucial for effective risk management and decision-making.

Traditional approaches to implementing source search capabilities into the web systems often rely on solely algorithmic solutions~\cite{ristic2016study}. These methods, while relatively straightforward to implement, often struggle to achieve acceptable performances due to algorithmic local optima and complex unknown environments~\cite{zhao2020searching}. More recently, web crowdsourcing has emerged as a promising alternative for tackling the source search problem by incorporating human efforts in these web systems on-the-fly~\cite{zhao2024user}. This approach outsources the task of addressing issues encountered during the source search process to human workers, leveraging their capabilities to enhance system performance.

These solutions often employ a human-AI collaborative way~\cite{zhao2023leveraging} where algorithms handle exploration-exploitation and report the encountered problems while human workers resolve complex decision-making bottlenecks to help the algorithms getting rid of local deadlocks~\cite{zhao2022crowd}. Although effective, this paradigm suffers from two inherent limitations: increased operational costs from continuous human intervention, and slow response times of human workers due to sequential decision-making. These challenges motivate our investigation into developing autonomous systems that preserve human-like reasoning capabilities while reducing dependency on massive crowdsourced labor.

Furthermore, recent advancements in large language models (LLMs)~\cite{chang2024survey} and multi-modal LLMs (MLLMs)~\cite{huang2023chatgpt} have unveiled promising avenues for addressing these challenges. One clear opportunity involves the seamless integration of visual understanding and linguistic reasoning for robust decision-making in search tasks. However, whether large models-assisted source search is really effective and efficient for improving the current source search algorithms~\cite{ji2022source} remains unknown. \textit{To address the research gap, we are particularly interested in answering the following two research questions in this work:}

\textbf{\textit{RQ1: }}How can source search capabilities be integrated into web-based systems to support decision-making in time-sensitive risk management scenarios? 
% \sq{I mention ``time-sensitive'' here because I feel like we shall say something about the response time -- LLM has to be faster than humans}

\textbf{\textit{RQ2: }}How can MLLMs and LLMs enhance the effectiveness and efficiency of existing source search algorithms? 

% \textit{\textbf{RQ2:}} To what extent does the performance of large models-assisted search align with or approach the effectiveness of human-AI collaborative search? 

To answer the research questions, we propose a novel framework called Auto-\
S$^2$earch (\textbf{Auto}nomous \textbf{S}ource \textbf{Search}) and implement a prototype system that leverages advanced web technologies to simulate real-world conditions for zero-shot source search. Unlike traditional methods that rely on pre-defined heuristics or extensive human intervention, AutoS$^2$earch employs a carefully designed prompt that encapsulates human rationales, thereby guiding the MLLM to generate coherent and accurate scene descriptions from visual inputs about four directional choices. Based on these language-based descriptions, the LLM is enabled to determine the optimal directional choice through chain-of-thought (CoT) reasoning. Comprehensive empirical validation demonstrates that AutoS$^2$-\ 
earch achieves a success rate of 95–98\%, closely approaching the performance of human-AI collaborative search across 20 benchmark scenarios~\cite{zhao2023leveraging}. 

Our work indicates that the role of humans in future web crowdsourcing tasks may evolve from executors to validators or supervisors. Furthermore, incorporating explanations of LLM decisions into web-based system interfaces has the potential to help humans enhance task performance in risk control.








% Reward hacking is a well-known issue in reinforcement learning, affecting both traditional RL and RLHF in LLMs~\cite{weng2024rewardhack}.
\subsection{Reward Hacking in Traditional RL}  
Reward hacking arises when an RL agent exploits flaws or ambiguities in the reward function to achieve high rewards without performing the intended task~\cite{weng2024rewardhack}. This aligns with Goodhart’s Law: \emph{When a measure becomes a target, it ceases to be a good measure.} For example: 
A bicycle agent rewarded for not falling and moving toward a goal (but not penalized for moving away) learns to circle the goal indefinitely~\cite{Randlv1998LearningTD}.  
A walking agent in the DMControl suite, rewarded for matching a target speed, learns to walk unnaturally using only one leg~\cite{lee2021pebblefeedbackefficientinteractivereinforcement}.  
An RL agent allowed to modify its body grows excessively long legs to fall forward and reach the goal~\cite{Ha2018designrl}.  
In the Elevator Action ALE game, the agent repeatedly kills the first enemy on the first floor to accumulate small rewards~\cite{toromanoff2019deepreinforcementlearningreally}.  
% A robot trained to stay on track learns to reverse along straight paths by alternating left and right turns instead of following curves~\cite{Vamplew2004}.

\citet{amodei2016concrete} propose several potential mitigation strategies to address reward hacking, including
\emph{(1) Adversarial Reward Functions}: Treating the reward function as an adaptive agent capable of responding to new strategies where the model achieves high rewards but receives low human ratings.
\emph{(2) Model Lookahead}: Assigning rewards based on anticipated future states; for example, penalizing the agent with negative rewards if it attempts to modify the reward function~\cite{everitt2016selfmodificationpolicyutilityfunction}.
\emph{(3) Adversarial Blinding}: Restricting the model’s access to specific variables to prevent it from learning information that could facilitate reward hacking~\cite{ajakan2015domainadversarialneuralnetworks}.
\emph{(4) Careful Engineering}: Designing systems to avoid certain types of reward hacking by isolating the agent’s actions from its reward signals, such as through sandboxing techniques~\cite{The_AGI_Containment_Problem}.
\emph{(5) Trip Wires}: Deliberately introducing vulnerabilities into the system and setting up monitoring mechanisms to detect and alert when reward hacking occurs.

\subsection{Reward Hacking in RLHF of LLMs}  
Reward hacking in RLHF for large language models has been extensively studied. \citet{gao2023scaling} systematically investigate the scaling laws of reward hacking in small models, while \citet{wen2024languagemodelslearnmislead} demonstrate that language models can learn to mislead humans through RLHF. Beyond exploiting the training process, reward hacking can also target evaluators. Although using LLMs as judges is a natural choice given their increasing capabilities, this approach is imperfect and can introduce biases. For instance, LLMs may favor their own responses when evaluating outputs from different model families~\cite{liu2024llmsnarcissisticevaluatorsego} or exhibit positional bias when assessing responses in sequence~\cite{wang2023largelanguagemodelsfair}.  

To mitigate reward hacking, several methods have been proposed. Reward ensemble techniques have shown promise in addressing this issue~\cite{Eisenstein2023HelpingOH, Rame2024WARMOT, ahmed2024scalableensemblingmitigatingreward, coste2023reward, zhang2024improvingreinforcementlearninghuman}, and shaping methods have also proven straightforward and effective~\cite{yang2024regularizinghiddenstatesenables, jinnai2024regularizedbestofnsamplingmitigate}. \citet{miao2024informmitigatingrewardhacking} introduce an information bottleneck to filter irrelevant noise, while \citet{moskovitz2023confrontingrewardmodeloveroptimization} employ constrained RLHF to prevent reward over-optimization. \citet{Chen2024ODINDR} propose the ODIN method, which uses a linear layer to separately output quality and length rewards, reducing their correlation through an orthogonal loss function. Similarly,
~\citet{sun2023salmon} train instructable reward models to give a more comprehensive reward signal from multiple objectives. \citet{Dai2023SafeRS} constrain reward magnitudes using regularization terms. ~\citet{liu2024rrmrobustrewardmodel} curate diverse pairwise training data. Additionally, post-processing techniques have been explored, such as the log-sigmoid centering transformation introduced by \citet{Wang2024TransformingAC}.  




\section{Formative Study}
\label{sec:formative}

To better understand the workflow of clinicians treating youth concussion patients with mental health issues, we conducted a formative study focusing on three main aspects: (1) how clinicians gather youth patients' mental health-related information, (2) what decisions they need to make, and (3) what the difficulties are in their current workflow. 

\subsection{Study Participants and Procedure}
\label{sec:formative-design}

\begin{table*}
  \caption{Demographics of Participants in Our Formative Study}
  \label{tab:demographics}
  \begin{tabular}{cclcc} 
    \toprule
    P\# & Gender & Department & Job Title & Year of Practice \\ 
    \midrule
    P1 & Female & Pediatric Sports Medicine & Pediatric Sports Medicine Physician & 16 years \\ 
    P2 & Female & Complex Concussion Clinic & Neuropsychologist  & 12 years \\
    P3 & Female & Concussion Clinics & Concussion Clinician & 19 years \\ 
    P4 & Male & Sports Medicine & Division Chief & 30 years \\ 
    P5 & Male & Concussion Clinics & Concussion Clinician & 16 years \\ 
    P6 & Male & Sports Medicine & Non-Operative 
Sports Medicine Doctor & 10 years \\ 
    \bottomrule
  \end{tabular}
\end{table*}

The key stakeholders in our scenarios are concussion clinicians with experience treating youth concussion patients with mental health symptoms. 
These highly specialized concussion professionals work in fast-paced environments and are often overloaded with numerous complex patient cases, which makes their time very limited. 
For this reason, we were only able to recruit six U.S.-based concussion clinicians who were available to participate in our study.
Recruitment was facilitated through professional networks within related clinical fields of domain experts in the research team via convenience sampling~\cite{sedgwick2013convenience}.
Each participant engaged in a semi-structured interview of 35-45 minutes conducted remotely via Zoom. 
Table~\ref{tab:demographics} provides details on the demographics of the participants and their clinical experience. 
This study was reviewed and approved by the first author's institution's Institutional Review Board (IRB), and the study complied with the approved procedure for ethical research practices of human subject protection.






At the beginning of each interview, we first asked the participant to recall a recent case where they encountered a youth concussion patient with mental health symptoms 
They described how they performed clinical practices, identified mental health symptoms, and executed follow-up decision-making.
While participants were sharing their experiences, we prompted participants to tell more about what type(s) of patient clinical data they collected, used, or expected to support the identification of mental health sequelae, both in and outside of clinical settings.
After that, participants shared their experience with existing tools, if any, for collecting patient data, monitoring patient mental health conditions, and identifying mental health sequelae. 
Moreover, participants shared their insights on the benefits and limitations of existing tools and the expectations of novel tools that could potentially help them.
During the interview, we asked participants to refrain from disclosing any personally identifiable information (PII). 
The complete interview protocol is provided in Appendix~\ref{sec:appendixa}.



All interviews were audio-recorded and transcribed with participants' consent. 
We employed the inductive thematic analysis approach~\cite{braun2012thematic,braun2019reflecting} to derive key themes. 
Two researchers first independently coded one session of the interview transcripts to identify meaningful segments and categorize segments into codes. 
Then, they discussed individual codes, resolved discrepancies, developed consensus codes, and applied the resulting codes to the remaining transcripts.
Finally, we iteratively grouped codes into overarching themes and refined these themes through ongoing discussions until we reached a consensus on the final set of themes.




\subsection{Findings}
\label{sec:formative-findings}

Our qualitative analysis provided a comprehensive understanding of concussion clinicians' workflow with youth concussion patients and derived three major challenges that concussion clinicians encountered during the practices: 
(1) tracking patients’ mental health-related information outside the clinic and identifying its severity;
(2) communicating effectively with patients and their family members about patients' mental health issues;
(3) assessing patients' compliance with clinical recommendations.



%\subsubsection{Clinical Workflow of Concussion Clinicians}
\subsubsection{Integrating Mental Health Screening into Concussion Management}
\label{sec:formative-findings-f1}

The majority of youth concussion patients are athletes, and concussions are commonly caused by head injuries during sports participation.
After they had a head injury, the youth patients underwent an initial assessment by primary care physicians.
According to P4, primary care physicians conducted basic evaluations and symptom scoring based on international guidelines. 
If specialized care is needed, patients are referred to specialized concussion clinicians.

The clinical tasks of concussion clinicians primarily include seeing new concussion patients and following up with existing patients, which typically follow a similar routine. 
When patients visit the clinic, concussion clinicians assess the severity of concussion symptoms with standard scales as suggested by clinical guidelines.
In addition, the clinicians conduct interview sessions with patients to collect more in-depth information about their health conditions. 
\begin{quote}
    \textit{"I guess the only way that I currently assess my patients in that way is by using the symptom score that is published in the international guidelines...... That's the symptom score that I use. And then interview. I will very commonly ask about sleep habits. I will very commonly ask about nutrition habits. I am very aggressive in making, I guess my patients go to school."} (P3)
\end{quote}
For the majority of concussion patients, the best treatment available is to have a rest both physically and cognitively. Thus, concussion clinicians will have a conversation with patients to provide personalized lifestyle recommendations and self-evaluation methodologies for concussion recovery. 
For example, P5 stated that they often provide patients with a symptom log and suggested that patients use it to track their daily activities. 
Patients will be asked to return the symptom logs during follow-up appointments, which typically occur after one week.
Symptom logs are essential for concussion clinicians to monitor their patients’ weekly health changes and concussion recovery progress.
If symptoms persist beyond 28 days, further interventions are initiated, such as referrals to neuropsychologists or specialized concussion clinics (P3, P5, P6). 
%\textcolor{red}{the following quote is not good -- critical information is missing}
% Youth aged 11 to 17, particularly athletes, are often taken to hospitals for treatment after injuries. 
% One concussion clinician (P4) mentioned that primary care physicians normally conduct an initial quantitative assessment of the child's condition and engage in direct communication to gather additional information. 
% If they determine that the patient requires more specialized concussion treatment, they refer them to concussion clinicians. 
% Concussion clinicians have dedicated clinical hours each week, during which they see both new concussion patients and conduct follow-up visits for those previously evaluated. 
% During clinical visits, they assess the severity of the concussion using symptom scoring scales recommended by international guidelines. 
% In addition, they conduct patient interviews to gain further insight into the patient’s condition. 
% Typically, clinicians advise patients to rest as needed while maintaining as much normalcy in their daily routines, such as attending school. 
% Clinicians (P5) reported that they often provide patients with symptom logs to record their daily experiences, which they are asked to bring back for their follow-up visits. 
% A follow-up appointment is usually scheduled after one week, at which point the concussion symptom scale is reassessed. 
% If a patient experiences persistent and severe symptoms for more than 28 days, clinicians (P6, P3, P5) indicated that they refer the child for further evaluation and treatment, which may include specialized concussion clinics or neuropsychological assessments.
\begin{quote}
    \textit{"Yeah, after we see them (patients) for the initial visit, we generally send them home with the log and ask them to fill it out every day based on how they felt, and then bring it back when they come back for their follow up appointment so that we can track how their symptoms have been going over time."} (P5)
\end{quote}
Concussion clinicians (P1, P2, P3, P5, P6) emphasized that mental health concerns are highly prevalent among youth patients. Thus, mental health assessments are an integral part of the concussion assessment process. 
Concussion clinicians (P6, P4) typically review the patient’s mental health history, including prior diagnoses, current treatments, and medication use. 
% According to P6, 10\% to 20\% of patients have a preexisting diagnosis of anxiety or depression. 
In addition, clinicians leverage standardized questionnaires, such as the Patient Health Questionnaire-9 (PHQ-9)~\cite{levis2019accuracy} and the Generalized Anxiety Disorder 7-item Scale (GAD-7)~\cite{mossman2017generalized}, to identify potential symptoms of mental health issues and the severity of symptoms. 
Concussion clinicians (P4, P6) also observe signals of emotional distress during the conversation with patients to detect potential mental health issues and will follow up with additional mental health screening if such signals are presented.
% To assess current symptoms, clinicians administer standardized questionnaires, such as the Patient Health Questionnaire-9 (PHQ-9) and the Generalized Anxiety Disorder 7-item Scale (GAD-7). 
% While some patients do not explicitly report mental health symptoms, clinicians (P4, P6) stated that they observe signs of emotional distress during physical examinations. 
% If such signs are present, they conduct additional mental health screenings. 
P6 reported that there is a high portion of concussion patients develop anxiety or depression following their injury. 
In such cases, P4 often recommends cognitive behavioral therapy as the intervention strategy for the mental health issues mentioned above. 

However, if mental health symptoms are found to impede concussion recovery, concussion clinicians will need to collaborate with neurologists to determine the intervention plan, such as prescribing appropriate medications. 
For patients exhibiting severe mental health symptoms, clinicians (P2, P3) will require the patients to complete an additional suicide risk assessment. 
If no immediate suicide risk is detected, clinicians typically recommend continued rest at home with ongoing symptom monitoring. 
However, P2 mentioned that 10\% of youth concussion patients develop suicidal thoughts after a concussion. 
In such cases, clinicians refer patients to mental health specialists, such as psychiatrists or psychologists.
%for dedicated and more professional evaluation and treatment.
\begin{quote}
    \textit{``In general, of people who I see with like symptoms of like anxiety, depression, things like that after concussion. It's probably about 40 to 50 percent.''} (P6)
\end{quote}



%\subsubsection{Clinicians' Challenges in Tracking Patients' Mental Health Information at Home}
%\subsubsection{Clinicians’ Challenges in Tracking Patients’ Mental Health-Related Information Outside of Clinics and Identifying Its Severity}
\subsubsection{Clinicians’ Struggle to Track and Assess Mental Health Beyond the Clinic}\label{sec:trackinginfo}

As mentioned above, the first 28 days following a concussion diagnosis is a critical period of time for youth concussion patients' recovery.
However, patients only visit the concussion clinic every one or two weeks, while spending the majority of time outside the clinic, more specifically, outside of clinical supervision.
% As a result, concussion clinicians (P2, P5) emphasized the importance of monitoring multitudes of mental health symptom indicators.
This gap creates significant challenges for concussion clinicians, who (P2, P5) emphasize the importance of monitoring a wide range of mental health symptom indicators during the concussion recovery period. 
Specifically, clinicians (P1, P2, P3, P4, P5) mentoined the importance of collecting detailed information about sleep patterns (e.g., sleep hours and wake-up times), and physical activity levels. 
P3 emphasized that the sleep patterns has not really been integrated into clinical systems.
%concussion statements
Moreover, clinicians (P2, P3, P6) are highly interested in youth patients' academic performance (whether patients completed assignments on time) and social behavioral patterns (whether patients rest at home consistently and maintain social connections rather than withdrawing socially) during concussion recovery time at home. These academic and social behaviors provide critical insights for clinicians to assess and reflect on the severity of youth patients' mental health issues during concussion recovery.
\begin{quote}
    \textit{``From their (patients) activities, feeling [that they have] little interest or in doing activities that they normally would. Seeing difficulty or decreased performance in either activities or or school.''} (P6)
\end{quote}
%\begin{quote}
%    \textit{``......if the worry is getting to the point where they (patients) can't do their homework...... Or if they aren't leaving the house… not sleeping, really feeling terrible, and it seems to be consuming them... Then I would say that might like cross the line from just typical worry into like, okay, this is more of a mental health anxiety, you know, diagnosis type of thing.''} (P2)
%\end{quote}
Furthermore, P4 stated that to capture mental health-related anomalies, the clinician mainly relies on nonverbal cues during the clinical visits.
If any abnormalities are detected, youth patients would be asked questions directly about their mental health conditions, such as suicidal tendencies, to identify the severity of their mental health symptoms.
P5 stated that identifying the severity of the symptoms is crucial for clinicians, as it impacts the clinician's ability to adjust their treatment plans in a timely manner.
Moreover, youth concussion patients may experience impairment in functioning or panic attacks, and concussion clinicians will then focus on understanding what they had been through and what events triggered the episode. In this way clinicians can evaluate patients' mental health condition more accurately as well as design a more personalized concussion treatment plan, as suggested by P2. 
\begin{quote}
    \textit{``Anybody who has a high report of anxiety, you know, as one of the symptoms or nervousness that we ask on our concussion symptoms log or if they have an underlying diagnosis of an anxiety, disorder, depression. And then, you know, maybe a little bit more likely to refer them (patients) earlier on.''} (P5)
\end{quote}
%\begin{quote}
%    \textit{``......I think, having more information about really what was happening at the time, that they had a symptom. What was triggering, that how that we know more real time information could be helpful.''} (P2)
%\end{quote}


Despite the critical need for timely mental health-related information of patients at home, concussion clinicians (P2, P5) reported significant challenges in accurately and reliably collecting the information outside the clinic.
Ideally, clinicians would benefit from having real-time access to patients’ accurate mental health-related information while patients are at home. 
However, such a system is currently unavailable.
Instead, clinicians must rely on patients’ recall of their living experiences during clinical visits to reconstruct what happened during the recovery period. 
Although these discussions could help inform clinical decision-making, patients’ memory recall is often incomplete or inaccurate, especially when clinical visits span several weeks. 
This lack of reliable data collection methodology can hinder concussion clinicians’ understanding of the patient’s recovery progress and delay timely mental health interventions, such as referrals or additional assessments.
Once youth concussion patients leave the clinic, clinicians have no way to obtain this information in a timely manner.
\begin{quote}
    \textit{``...more real-time information could be helpful. Because by the time we see them (patients) in the clinic that could be you know, weeks later they might not even remember that that (mental health sequelae-related symptoms) had happened...... It's hard for anyone to remember what they did, you know, weeks before.''} (P2)
\end{quote}


Youth concussion patients currently have some remote communication options. One way is making phone calls with the provider team and reporting patients' mental health-related information regularly to clinicians when they are outside of the clinic. 
% Currently, youth concussion patients have some remote communication options for sharing their health information with clinicians while at home. One common method is through phone calls. 
However, clinicians (P2, P3) reported several limitations with phone communication. First, youth concussion patients rarely initiate calls to their concussion clinicians. Moreover, even if youth concussion patients make a phone call, clinicians may not always be in the office or available to answer immediately, which can delay the transmission of critical health information.

An alternative option is using MyChart, a patient portal that allows patients to access their medical records and send messages to their healthcare providers. 
Concussion clinicians (P4) strongly recommend that patients use MyChart to share health-related information, particularly when youth concussion patients are reluctant to make direct phone calls. 
Compared to phone communication, MyChart offers the advantage of asynchronous messaging, enabling clinicians to review patient information and respond at their convenience. 
Despite its potential benefits, the adoption of MyChart is limited in current clinical scenarios.
Concussion clinicians (P2, P6) noted that youth patients rarely use the messaging feature in MyChart and rarely take the initiative to contact clinicians.
Instead, parents of these youth patients are more likely to use MyChart to communicate with clinicians.
%which impedes the information flow between patients and clinicians. 
% As a result, MyChart is not a frequently used tool for communication between clinicians and youth concussion patients, making it difficult for clinicians to gain insights into patients' health conditions at home.

\begin{quote}
    \textit{``One in 10 [patients] maybe, or even one in 20 [patients] might send a message with a question.''} (P6)
\end{quote}

Furthermore, clinicians caution against relying on MyChart for emergencies, especially when patients experience suicidal tendencies. 
P2 warned that their busy schedules prevent them from checking messages in real time, which could lead to delays in critical interventions and put patients at risk.

% Because concussion clinicians are unable to check messages from patients or their parents on MyChart in real time due to their workload, this may lead to delays in critical interventions, potentially compromising patient safety.

\begin{quote}
    \textit{``It's not really immediate. You know, cause they (patients) might send it at night, and then we don't maybe don't see it till the next afternoon, and you know. We usually, you know, tell them if it's an emergency, don't use that method because we don't really know when we're gonna see them or be able to respond to it.''} (P2)
\end{quote}




%\subsubsection{Concussion Patients and Caregivers May Be Uncooperative with Concussion Clinicians' Communications for Mental Health-Related Information}
%\subsubsection{Clinicians’ Challenges in Communicating With Patients and Families About Patients’ Mental Health Issues}
\subsubsection{Clinicians’ Struggle to Communicate Mental Health Issues with Patients and Their Families}\label{sec:communicationissues}


Communication challenges persist not only outside the clinics but also during in-person consultations.
% Clinicians not only receive little communication from patients between clinical visits but also encounter communication challenges during the visits themselves. 
During in-person visits, parents often accompany their children. However, some youth concussion patients feel uncomfortable disclosing sensitive mental health issues with their parents present.
P1 described a case where a patient with severe anxiety disclosed self-harm and suicidal thoughts only after requesting to speak privately, away from their parents' presence. 

% Clinicians believe patients might not be ready to discuss mental health openly.

\begin{quote}
    \textit{``I stepped out of the [room], I had the mom stay in the room...... and I said, 'Is there something wrong? Is there something going on?' He said, 'Yeah, I need to talk to you privately.' ''} (P1)
\end{quote}
%\begin{quote}
%    \textit{“There were different behavioral patterns and mood[s] that I would see in the
%office that concerned me for mental health. I had brought it up, and both he (the patient) and the mom declined and said everything was fine... I think there's a misconception that nobody wants to be labeled. It's easier to deal with an injury that's more visible or socially acceptable."} (P1)
%\end{quote}

Clinicians suspect that a major reason for this is patients' fear of mental health stigma~\cite{apa_stigma}.
The label of having a mental illness can affect how patients are perceived socially, which could potentially isolate them from their peers.
Another reason not to disclose mental health issues is the societal expectation toward youth concussion patients.
Specifically, if youth patients are athletes who are expected to perform at a high level of competition, they may struggle with the societal expectation of their physical recovery. 
%This highlights the willingness of mental health disclosures is intertwined with many societal factors that may not be directly related to mental health symptoms.

Not only do patients withhold their mental health condition, but sometimes their parents also deliberately withhold information about the patient's mental health condition as reported by concussion clinicians (P1, P4, P5).
%Moreover, some concussion clinicians (P1, P4, P5) reported that sometimes both youth concussion patients and their parents may deliberately withhold information about the patient’s mental health. 
In one case encountered by P4, the youth concussion patient was a successful athlete, and the parents downplayed their child’s mental health symptoms during clinical visits. 
These parents may worry that prolonged rest due to mental health sequelae could hinder their child’s athletic performance and career trajectory. 
So the parents downplayed their child’s mental health symptoms, hoping for their child to return to their athletic career sooner.
However, when mental health issues are not disclosed and addressed in a timely manner, the issues could get worse and lead to severe mental health issues, such as self-harm or an increased risk of suicidal behavior.
The consequences of severe mental health issues could lead to long-term physical and cognitive impairment of the patients, which might eventually hinder an athlete's ability to return to sports in their lives.
\begin{quote}
    \textit{``Their (patients) parents are very vested in the child's athletic career......the parents come in and there's obviously an agenda to minimize their symptoms. Or you know, just try not to buck the system, if it's gonna limit their (patients) activity.''} (P4)
\end{quote}


To detect undisclosed mental health issues, clinicians rely on a combination of strategies.
One approach is to build trust and collaborate directly with parents. 
Another is to observe the patient’s behavior during consultations. 
For example, clinicians (P1, P4) observe patients' behaviors, such as signs of anxiety or a persistently quiet and disengaged attitude, during clinical visits. 
% When youth concussion patients do not disclose mental health problems, there are ways to detect potential issues during clinical visits. 
% One way is to talk to patients' parents directly, try to build trust, and team up with them. 
% Another way is through observation. 
% For example, clinicians (P1, P4) can observe patients' behaviors, such as signs of anxiety or a persistently quiet and disengaged attitude, during clinical visits. 
% These indicators are highly associated with mental health issue sequelae. 
Moreover, worsened concussion symptoms such as insomnia and low activity levels may indicate underlying issues beyond the concussion itself, such as mental health sequelae.
P3 stated that if a youth patient's concussion symptoms do not improve within four weeks after a concussion, it may indicate underlying mental health issues that are contributing to a prolonged recovery period.
However, observing patients' behavior or asking about their sleep patterns requires patients to be physically present during the clinical visit.
Once the patients return home, clinicians cannot access this information.
%For example, one concussion clinician (P1) noticed the mental health problems of youth patients by observing the patient's behavior and reaction to the clinician's mental health-related questions. 
%\begin{quote}
%    \textit{"Something seemed off... I wanted to ask him (the patient) some mental health questions directly... He (the patient) got very, very anxious, and stated he needed to use the restroom, and very anxious, so I knew right away something was wrong."} (P1)
%\end{quote}
\begin{quote}
    \textit{``Because we expect 96\% of concussions to be healed by the four-week point. And so if it's not healed yet, then it's probably not the concussion, right? It's probably either a headache syndrome or a mental health concern.''} (P3)
\end{quote}


\subsubsection{Clinicians’ Struggle to Evaluate Patients’ Compliance Beyond the Clinic}\label{sec:compliance}


% Even if patients disclose their mental health status to clinicians, challenges still arise during each clinical visit. 
At the end of each clinical visit, concussion clinicians typically provide patients with recommendations related to physical activity levels or sleep schedules.
These clinical recommendations are intended to serve as the concussion treatment plan to support both physical and mental recovery after a concussion.
However, understanding whether patients adhere to these recommendations is a persistent challenge.
Although concussion clinicians (P2, P4) can rely on patients’ self-reports to assess compliance during the time period between the last visit and the current one, these reports are often unreliable. 
For instance, some patients may claim to have followed the recommendations, but their responses to mental health questionnaires reveal high levels of anxiety or depression, indicating limited or no improvement.


 
% Currently, clinicians rely only on patients' self-reports to assess adherence. 
% When clinicians asked their patients about their compliance, some patients claimed to have followed the recommendations. 
% Nevertheless, clinicians often question the reliability of these self-reports. 
% In some cases, patients’ responses to mental health questionnaires show high levels of anxiety and depression, indicating a lack of improvement despite claims of adherence to the recommendations.

\begin{quote}
    \textit{“What we recommended, if they (youth patients) [said they] did it or not, they might not have done it... ... We don't know if they're following them or not. Or they come in, and they say they did, but we don't know if they did.”} (P2)
\end{quote}

However, there is currently no effective way to track patients' compliance outside the clinic. 
P4 mentioned that they often ask patients' parents about their children's adherence to the recommendations, but the information provided by the parents may also be unreliable.
One reason could be that some parents are frequently away from home due to work commitments, which leaves them with limited knowledge of patients’ daily health-related information. 
% only responsible for bringing the patients to the clinic. 
% The parents are frequently away from home due to work commitments, which leaves them with limited knowledge of the patient’s daily health-related information. 
In such cases, clinicians may reach out to other family members by phone for a closer understanding of the patient’s mental health issues when they are at home. 
% However, making phone calls is time-consuming, and the information obtained is often fragmented and insufficient for comprehensive assessments.
However, making such calls consumes a considerable amount of time in clinicians' busy schedules, and the information collected through these calls is often limited. 
% Moreover,  clinicians normally have busy clinical schedules and are unable to gain every patient information thoroughly.
\begin{quote}
    \textit{``I think sometimes the parent comes in with a child who isn't around them (parents) that much at home. Like maybe it's the working parent, and they're just sort of the driver, and they don't have a lot of information...... Sometimes I can't get good information cause they (parents) are not engaged as much as I would, you know, would hope.''} (P4)
\end{quote}
% Clinicians (P2, P3, P6) reported that one of their major concerns regarding youth concussion patients with mental health sequelae is the lack of sufficient clinical resources to meet the demand for patients' clinical visits. One significant issue is the time required to schedule appointments with mental health counselors or professionals. Youth concussion patients often face long wait times, typically 2 to 3 weeks, to secure an appointment with a clinician (P1). This delay can lead to further deterioration of their condition. For example, symptoms that remain unmanaged during this period may worsen their mental health sequelae such as anxiety or depression. Additionally, delayed interventions can increase the complexity and difficulty of recovery.

% “I don't think I could. If I had 20 patients, and data was provided between visits, I'm not sure I would have a ton of 
% time to devote a true thought process to it.” (P)

% Another critical issue is the shortage of human resources. Scheduling appointments with mental health counselors or professionals often involves even longer wait times. 

% “...sometimes one of the biggest barriers we have is access. And getting people into mental health. Like counselors can be so much difficult.” (P)

% \subsubsection{Clinicians' Challenges in Differentiating Mental Health Sequelae After a Concussion from Pre-existing Mental Health Issues}


%1 clinicians said that mental health issues in concussion patients are not the same -- one is caused by concussion, namely m.h. sequelae; the other is pre-existing mental health issues that do not originate from concussion.
%2 concuccions clinicans stated that it is important for them to identify which type of m.h. issue while they were treating concussion patients because the concucssion treatment plan needs to be adjusted based on the m.h. types







%determine if it is mental health sequelae

%related to mental health sequelae
%sequelae - plan a
%no sequelae plan b
%concussion treatment decision making
%AI support decision making (early prediction)


%Concussion clinicians (P4, P5) routinely review patients' mental health histories during consultations to assess whether they have pre-existing mental health conditions. 
%They examine whether patients are undergoing mental health treatment or taking medication for anxiety or depression. 
%Clinician (P2) noted that youth concussion patients with pre-existing anxiety or depression often have a more negative outlook on recovery. 
%These patients may experience excessive worry about their healing process, leading to panic symptoms. 
%This not only exacerbates their underlying mental health conditions but also negatively impacts their concussion recovery. 
%If a clinician (P2) determines that anxiety or depression symptoms are complicating youth patients' concussion recovery, clinicians collaborate with other specialists, such as neurologists, to prescribe appropriate medications. Additionally, if a patient exhibits suicidal tendencies, they are promptly referred for a safety assessment and safety planning.
%\begin{quote}
 %   \textit{“So we do suicide screening. So if someone is screening positive, we might have to do a more detailed safety assessment and safety planning. We also have a neurologist that I work with, so sometimes they might prescribe medication to treat anxiety or depression symptoms, if they (patients) are seeming to be, you know, complicating their concussion recovery.”} (P4)
%\end{quote}

%However, concussion clinicians face challenges in making timely mental health-related decisions.
%In some cases, concussion clinician (P4) suspects that a patient may be approaching a mental health crisis, despite having no prior clinical diagnosis of a mental health condition. 
%In such situations, clinicians must carefully assess the patient’s mental state to determine whether immediate intervention is necessary. 
%To gain a clearer understanding, clinician (P4) sometimes asks youth concussion patients about their mental health status before the concussion to establish a baseline for comparison. 
%However, clinician (P4) noted that youth patients’ self-reported information is often inaccurate, making it difficult for clinicians to conduct an solid evaluation. 
%Although clinicians sometimes use standardized anxiety or depression screening tools, they primarily rely on nonverbal cues to detect mental health sequelae and directly ask about suicidal thoughts when necessary. 
%However, clinician (p4) point out that mental health symptoms may sometimes appear vague or inconsistent with a patient’s previous condition, which makes it difficult for them to clearly assess the patient’s mental health status and adjust the treatment plan in a timely manner based on the patient’s mental health conditions.
%\begin{quote}
   % \textit{"I think we sort of have to do is if we get somebody (patients) who's undiagnosed and approaching crisis to try to figure out. You know, do they need immediate help? Do they need, you know, or can we make a referral to behavior health and get them connected? So we probably do more acute intervention, just selectively when the situation arises."} (P4)
%\end{quote}
%_________




\section{Preliminary System Design}
%Building on these identified needs and challenges from our formative study, 
In the formative study, we identified clinicians' urgent need to remotely monitor the mental health of youth concussion patients at home, as well as the challenges they face in clinical decision-making.
In this section, we explored the design opportunities of AI-RPM technologies in supporting clinicians’ needs and decision-making during their practice. 
%We first proposed AI-RPM technologies that could potentially assist clinicians in remotely monitoring patients' at-home data as well as addressing decision-making difficulties. 
Subsequently, we proposed a preliminary system design (Fig. \ref{fig:preliminary_system_design}) based on the AI-RPM technologies with detailed features. 
The goal of the design is to explore whether concussion clinicians find the AI-RPM system useful for supporting remote monitoring and decision-making, how the system can be improved, and how they would use it in practice.

%\subsection{Leveraging AI-RPM Technologies for Remote Youth Concussion Patient Monitoring}
\subsection{Design Considerations for AI-RPM Systems}

\subsubsection{Wearables for Tracking Patient Physiological Data}
\label{sec:wearblesforsleepandactivitiestracking}

Based on Section \ref{sec:trackinginfo}, clinicians expressed a desire to monitor patients’ at-home quantitative data, such as sleep and exercise patterns, as these factors are closely related to concussion recovery and mental health status. 
In addition, as noted in Section \ref{sec:compliance}, clinicians frequently ask patients about their clinical recommendation compliance, such as maintaining proper sleep and exercise routines.
However, clinicians currently lack effective ways to access these at-home data and often struggle to verify the accuracy of patients’ self-reported information during clinical visits.

%To address these challenges, wearable devices such as smartwatches can help
To address these challenges, wearable devices such as smartwatches can be used to help clinicians collect real-time sleep and exercise data from patients~\cite{dias2018wearable, baig2017systematic, bate2023role}. 
Wearables offer a non-invasive and natural way to monitor patients’ health metrics without significantly disrupting their daily routines. 
Moreover, AI has the potential to process the massive amount of data collected from wearables and provide these data to clinicians for timely review.
We believe that with access to at-home patient data, clinicians can become aware of patients' health conditions in real-time, and track patients' compliance with sleep and exercise recommendations, thus ultimately having the ability to make more informed mental health-related clinical decisions, such as setting up a follow-up meeting for further assessment.


\subsubsection{LLM-based CAs for Patient Self-Reporting in Mental Health}
\label{sec:casforselfreport}

In Section \ref{sec:trackinginfo}, clinicians emphasized the need for a comprehensive understanding of patients' experiences with mental health symptoms, such as panic attacks, to provide effective care. However, patients often struggle to accurately recall these details, and such subjective experiences cannot be easily quantified using standardized scales or medical instruments. Additionally, based on Section \ref{sec:communicationissues}, concussion clinicians reported that youth concussion patients sometimes conceal their mental health issues from their parents. It was only during a private conversation with the clinician that the patient disclosed their mental health status.

Given this challenge, we hypothesize that LLM-based CAs~\cite{yang2024talk2care, mahmood2023llm, chan2024human} could help clinicians collect mental health-related information effectively. 
CAs can be deployed in patients' homes to engage in natural language conversations, ask relevant questions, and record dialogue content. 
AI can then analyze historical conversations to summarize the key information of the conversation, such as their patient's suicidal intentions. 
From the perspective of youth patients, the deployment of a CA in their rooms could increase the chance of disclosing their mental health issues, if there are any.
One study found that children perceive conversational agents as warmer and more reliable than their parents~\cite{van2023children}.
From the perspective of clinicians, gaining such key information allows them to review patients' historical conversations with CAs, gain accurate information about patients' mental health-related experiences while preserving privacy, and conduct interventions if necessary.  
Therefore, we speculate that CAs have the potential to act as a communication bridge between clinicians in the hospital and patients at home and provide patients with self-reported symptoms. Ultimately, CAs could support clinicians in conducting a more accurate assessment of patients' mental health and making timely decisions.


\subsubsection{AI-based Risk Prediction for Detecting Mental Health Sequelae}
\label{sec:aifordetection}

%As discussed in Section \ref{sec:communicationissues}, 
As we have indicated in Section \ref{sec:casforselfreport}, clinicians reported that patients and their parents sometimes deliberately conceal mental health conditions. 
In addition, clinicians noted that the severity of mental health symptoms in a patient influences their subsequent clinical decisions, such as treatment plans or providing a referral. 
However, evaluating the severity of mental health symptoms remains challenging.
Given these two challenges, we believe that AI risk prediction for disease progression could provide valuable support to clinicians in our setting. 
AI has the capability to analyze patients' health data from the EHR system and utilize algorithms to predict the likelihood of developing specific conditions over a given period~\cite{collins2019reporting}. 
In particular, recent studies have already explored AI-driven risk prediction for mental health sequelae following a concussion and have found that AI predictions can attain high accuracy~\cite{dabek2022evaluation}.
By leveraging AI risk score prediction, clinicians may no longer have to rely solely on the conversations with patients during clinical visits to make mental health assessments. 
If patients or their parents choose to withhold information, AI could help clinicians detect potential mental health sequelae in a timely manner. 
Moreover, the severity of the mental health symptoms is normally associated with the AI prediction score. 
The AI prediction score can help clinicians better assess the severity of a patient's mental health symptoms and make informed decisions accordingly.
%If a patient is flagged as having a high risk of developing mental health sequelae, the clinician can proactively schedule follow-up visits and make necessary mental health-related clinical decisions. 
%Therefore, we hypothesize that AI-driven risk prediction has the potential to assist concussion clinicians in detecting mental health sequelae more effectively.



\begin{figure*}[htbp]
  \includegraphics[draft=false,width=\textwidth]{Initial_UI.pdf}
  \caption{The preliminary system design based on concussion clinicians' challenges and needs: (A) AI risk score prediction for mental health sequelae, (B) patient self-report symptoms, (C) patient sleep and physical activities.}
  \label{fig:preliminary_system_design}
\end{figure*}

\subsection{Three Modules of Preliminary System Design Based on the Design Considerations}
Based on the potentially applicable technologies that could support concussion clinicians in addressing their challenges and supporting their decision-making, we designed three modules for our preliminary system, as shown in Fig. \ref{fig:preliminary_system_design}. 
We described each module in detail in this section.


\subsubsection{The Sleep and Physical Activities Module}
The sleep and physical activities module (Fig. \ref{fig:preliminary_system_design} B)  presents sleep (sleep hours, light sleep, and wake-up time) and activity data (step count) that clinicians want to track.
These data can be collected by wearables as we discussed on Section \ref{sec:wearblesforsleepandactivitiestracking}
Moreover, clinicians can select any category to review the patient’s relevant data from the past week. 
Clinicians can also select different dates to review patient data from other time periods. 
Additionally, the module provides the source of the data for increasing clinicians' understanding of the module and their trust towards the system. 
%The goal of this module is to assist concussion clinicians in monitoring youth concussion patients' sleep and physical activity at home, addressing the current gap in existing clinical systems that lack this information.


\subsubsection{The Self-Report Symptoms Module}
To display the patients' self-reported symptoms at home, which were collected by CAs as we proposed in Section \ref{sec:casforselfreport}, we designed the self-reported symptoms (Fig. \ref{fig:preliminary_system_design} B). 
On the right side, a history of conversations between youth concussion patients and CAs is available for concussion clinicians to review. 
Based on the conversation history, AI detects the symptoms that are mentioned by patients and highlights key information that are relevant to mental health sequelae, such as anxiety, depression, and suicidal ideation. 
If a patient self-reports symptoms within a specific category, the corresponding indicator changes from green to red.
We use the color red to catch clinicians' attention. 
To facilitate efficient navigation, clinicians can click on a category name or use the search bar to enter keywords, which prompts the right-side module to display the relevant conversation excerpts. 
Similar as the sleep and physical activities module, time selection lets clinicians view data from different periods.
%The primary goal of this module is to allow concussion clinicians to quickly assess patients' mental health symptoms at home, extending support beyond clinical settings.


\subsubsection{The AI Risk Score Prediction Module}
The AI Risk Score Prediction (Fig. \ref{fig:preliminary_system_design} A) module leverages AI's ability that we discussed in Section \ref{sec:aifordetection} to show the probability of mental health sequelae.
The prediction timeframe is within one year with a range from 0\% to 100\%, where the higher probability score signifies a higher probability of experiencing mental health sequelae for clinicians to review. 
Moreover, feature importance is used to provide data that influence the AI prediction score, which allows clinicians to decide whether to trust the AI-predicted score or not. 
To increase AI explanations and transparency, we provide text-based explanations to help clinicians understand the risk score and its source. 
At the bottom is the risk score chart. Clinicians can hover over any day on the chart to view the likelihood of developing mental health sequelae at that specific point in time. Additionally, clinicians can select different dates to review patient data from various time periods. 








\section{Evaluation Study of the Preliminary System Design}

To evaluate the usefulness of our preliminary design, we conducted an evaluation study.
The goal of the evaluation study is to gather clinicians' feedback and insights on our preliminary designs with different AI-RPM technologies. Specifically, we want to explore (1) whether they find these AI-RPM technologies useful for remotely monitoring mental health-related information of youth concussion patients and (2) how to revise the preliminary system design from data type, visualization, interaction, and usability perspectives to better support clinicians in addressing challenges and decision-making in their workflow.

\subsection{Procedure}
We conducted an evaluation study with the same six clinicians from our formative study since they are the target users of the AI-RPM system. 
Due to the significant time constraints faced by concussion clinicians, each evaluation session was limited to 10 minutes with each clinician. We conducted a Zoom screen-sharing session to present our preliminary designs to each clinician and collected verbal feedback regarding data type, visualization, interaction, and usability. Meanwhile, our research team documented their responses through audio recording and note-taking. After the session, two researchers applied axial coding to identify codes and key themes. Afterward, we iterated on our preliminary designs based on our findings from this design interface evaluation. The detailed interview protocol for our evaluation study is provided in Appendix~\ref{sec:appendixb}.

\subsection{Findings} 
In this section, we present five key findings derived from concussion clinicians' feedback during the evaluation study.

%\subsubsection{AI Risk Score Prediction as a Supportive Tool for Clinicians in Patient-Clinician Communication}
\subsubsection{Improving Patient-Clinician Communication With AI Risk Score Prediction}
Concussion clinicians (P1, P5, P6) reported that if the AI risk score prediction module provides accurate data, it can be very useful for timely clinical adjustments, such as scheduling follow-up visits or providing referrals.
\begin{quote}
    \textit{``That (the AI risk score prediction) might be something helpful early on for them (concussion clinicians) to say, hey, this is somebody who probably gonna benefit from seeing another provider earlier rather than later.''} (P5)
\end{quote}
Moreover, P1 mentioned that the AI risk score prediction module is not only easy to understand but also has the potential to serve as an evidence-based tool to effectively support clinicians in communicating with patients and their parents about mental health conditions during clinical visits.
By directly presenting the risk scores of the likelihood of developing mental health sequelae, youth patients and their families may gain a clear understanding of the severity of patients' mental health conditions and be willing to talk about it, particularly for youth patients and families who are hesitant to discuss mental health issues with concussion clinicians. 
\begin{quote}
    \textit{``I think it's pretty simple, and if I was able to show that to a family... maybe something as simple as this to show [the patient's] mom... 'I'm concerned about your son... there's a 50 percent risk that [your son] may develop [mental health sequelae]. We should really be in touch with it.' Yes, I think this is great.''} (P1)
\end{quote}
However, P3 found the AI-generated risk score unhelpful because it covered one year, while she only treated youth concussion patients for four weeks.
If patients did not recover within this period, they would be referred for further treatment.

\subsubsection{Supporting Clinicians' Understanding with AI-Generated Results and Data Source Transparency}
\label{subsubsec:explanation}
Concussion clinicians expressed confusion regarding the data sources and AI results in both the AI risk score prediction modules and self-reported symptoms modules. 
When we presented the AI risk score prediction module, P6 sought for a clear explanation of the meaning behind the AI risk score despite the interface already providing key data that contribute to the risk score as well as a brief explanation of the score. 
Additionally, P2 and P4 mentioned that they were unclear about how patient's mental health symptoms were collected in the self-reported symptoms module.
Specifically, P2 was unsure whether the symptoms (e.g., suicidality and depression), were from AI risk score prediction modules or from questionnaires (e.g., GAD-7 and PHQ-9) that clinicians gave to their patients for self-assessment. However, the symptoms are collected by LLM-powered CAs through check-in conversations with youth patients.
\begin{quote}
    \textit{``So it's just my confusion. So when it says self-reported symptoms that's based on like a questionnaire that we've given them about their depression and anxiety level? Or this is what AI predicts that they have a 50 percent chance of like developing depression based on that?''} (P2)
\end{quote}



\subsubsection{Enriching Clinicians' Insights with Data from LLM-based CAs and Wearables}
\label{subsubsec:cawearables}
Clinicians reported that the data collected from CAs and wearables could be valuable in their workflow. 
Specifically, P6 mentioned that he would have a quick review of data on the self-reported symptoms module collected by CAs before each clinical visit to better understand youth patients' current health conditions. 
Moreover, concussion clinicians (P3, P6) believed that patients might be more willing to share information with a conversational agent than directly with a clinician, which makes the data collected by CAs a valuable additional data source. 
However, P3 suggested that rather than using CAs to ask patients about their mental health conditions daily, it would be more appropriate to collect the data weekly. This suggestion helps avoid constantly reminding patients of their injuries.
\begin{quote}
    \textit{``I guess from a perspective of maybe they (patients) would tell this device things that they wouldn't tell me. Possibly I could see that.''} (P6)
\end{quote}
Moreover, concussion clinicians (P3, P4, P5) particularly appreciated the data in the sleep and physical activities module collected by wearables. They highlighted that data such as sleep hours is currently missing from EHR systems, yet wearable-collected data could serve as key indicators for mental health sequelae and concussion recovery progress. 
Additionally, P3 believed that clinicians could use the sleep and activity data to assess whether patients have followed sleep and exercise recommendations, which can be useful for addressing one of their main challenges.
\begin{quote}
    \textit{``This one (the Sleep and Physical Activities panel) to me might be the most helpful [one]... I think that's always hard [to get].''} (P5)
\end{quote}
%\begin{quote}
%    \textit{"So I really like the sleep and physical activity, but I would suggest that the patients not see that data that it only goes to the physician. Because I don't want it to cause them anxiety. So yeah, I really like that."} (P3)
%\end{quote}



\subsubsection{Diverse Clinician Preferences in Data Details and Presentation Styles}
\label{subsubsec:detaileddata}
%Needs for Questionnaire Result
Concussion clinicians showed varying preferences regarding the type and level of detailed data presented within the sleep and physical activity module. 
Clinicians (P2, P3, P5) suggested incorporating napping-related data into the module, including whether the patient took naps and their duration. 
Additionally, clinicians (P1, P3) emphasized the importance of heart rate in assessing concussion recovery and monitoring mental health conditions. Specifically, they expressed interest in viewing daily resting heart rate during sleep over the course of a week. 
P1 noted that a heart rate in the fifties to sixties indicates a calm state in patients. A calm state is often associated with obtaining eight hours of quality sleep. 
In contrast, if a patient experiences fragmented or restless sleep, their resting heart rate increases. 
By analyzing physiological data, clinicians can assess if symptoms are linked to mental health or other non-concussion issues.
\begin{quote}
    \textit{``If someone's recovering, I would expect it (heart rate) to be low, [meaning] they're calm......If you (patients) are having a light, irritable kind of sleep, I would expect the resting heart rate to be in the seventies or eighties. [It means] maybe they (patients) are anxious. Maybe there's something else going on.''} (P1)
\end{quote}
Moreover, in the self-report symptoms module, clinicians (P3, P5) suggested they want to include dietary information, such as appetite or nutrition, as they noted that when youth patients experience depression or high levels of anxiety, both the patients and their parents often report decreased appetite. Importantly, instead of focusing solely on whether youth patients have mental health symptoms, concussion clinicians (P3, P5) were more concerned with the duration and severity of these symptoms. Because the severity and duration of symptoms have different impacts on clinicians' decision-making, such as treatment plans.
%and potentially help clinicians distinguish between general emotional issues and mental health sequelae following a concussion. 
\begin{quote}
    \textit{``I don't think it has to be high level [of mental health symptoms]...... persistent or non-improving symptoms. That's what I look for.''} (P5)
\end{quote}
Additionally, P3 expressed a new need for a feature that could indicate how adherence to sleep and exercise recommendations impacts patients' recovery, suggesting that such insights could enhance clinical decision-making. 
%\begin{quote}
 %   \textit{"I don't know if my patients are following my recommendations and knowing if they are and how that affects their recovery would be really interesting."} (P3)
%\end{quote}


\subsubsection{Concerns of Clinicians in Emergency Situations and Time Constraints}
In the self-reported symptoms module, concussion clinicians placed significant emphasis on alerts related to suicidal tendencies. P1 emphasized the need for immediate intervention when suicidal tendencies emerge in youth concussion patients. 
However, clinicians expressed several challenges. 
Firstly, there is an absence of established protocols for managing such emergency situations. 
%Moreover, managing emergency interventions without triggering resistance from parents or youth patients remains a highly sensitive and complex challenge. 
Furthermore, P4 mentioned that the youth generation experiences frequent mood swings and may sometimes exaggerate their feelings, which could lead to false alerts.
However, failure to intervene in patients who have suicidal tendencies in a timely manner could result in a life-or-death consequence. 
To address this, clinicians (P2, P4) believed that youth patients' parents should be brought into the loop and have a partnership with parents when emergency situations arise.
\begin{quote}
    \textit{``So I think this is when you need to loop the paradigm and say, hey, your child has been identified through our software. You know, cause I think, especially since we're dealing with minors, you're gonna have to get the parents loop too.''} (P4)
\end{quote}
Moreover, clinicians (P2, P5) expressed that they have limited time to thoroughly review such detailed data due to heavy workloads.
Specifically, P1 highlighted that processing large amounts of data between tightly scheduled appointments—sometimes up to 20 patients per day—makes it nearly impossible to dedicate sufficient time and energy for in-depth analysis. 
If they have the AI-RPM system, clinicians noted that they might skim through the most relevant data quickly on the day of the appointment or shortly before to gain a basic understanding of their patient’s condition.
\begin{quote}
    \textit{``I think that's gonna that would be really  tedious to try to monitor and create extra work for us.''} (P5)
\end{quote}










\section{Refined AI-RPM System Design}
Based on the findings in the design interface evaluation session, clinicians acknowledged the value of the system with AI-RPM technologies that can potentially help them remotely monitor patients' mental health symptoms and support them to make decisions in their workflow. At the same time, they provided feedback and suggestions based on the preliminary designs. Based on their feedback, we revised our preliminary design and proposed a complete design of clinician-facing AI-RPM system design (Fig. \ref{fig:refined_system_design}).
%The purpose of this section is to effectively illustrate the complete system design interface which is tailored to concussion clinicians' needs in their workflow and provide insights for future studies. 

%\begin{figure*}[htbp]
%  \includegraphics[draft=false,width=\textwidth]{refind_system_design.pdf}
%  \caption{The final system design is based on feedback from concussion clinicians gathered through design interface evaluation: (A) patient list, (B) basic information, (C) sleep and physical activities, (D) self-reported symptoms, (E) questionnaire result, and (F) AI risk score prediction for mental health sequelae.}
%  \label{fig:figure2}
%\end{figure*}

%\iffalse
%\subsection{Design Considerations}
%\subsubsection{Explanation of Data Sources and Module Relationships}
%Through design interface evaluation, we found that while concussion clinicians considered the risk score useful for health decision-making and communication with patients and their families, they also raised concerns. Specifically, they were unsure how the score was generated and how it connected to other modules. To address this issue, we recommend using visualizations to illustrate the relationships between AI results and other data sources. This would enable clinicians to intuitively understand how various factors influence the risk score, potentially enhancing their trust in the system. One clinician explicitly mentioned wanting to see the relationship between clinical recommendations and patient recovery. Thus, we suggest visualizing this by displaying the correlation between the risk score and exercise recommendation compliance. 

%\subsubsection{Personalized Time Selection}
%One clinician found the AI-generated risk score unhelpful because the preliminary system design displayed a one-year risk score, whereas she typically treats youth patients for only four weeks. If a patient does not recover within this period, they are referred for further evaluation. Therefore, we believe clinicians would benefit from viewing AI model risk scores over different time spans to meet their varying needs and align with their workflow. To accommodate different clinical needs, we propose adding a time range selector to the module. This feature would allow clinicians to customize the risk score display based on their specific decision-making timeframes, offering personalized support for different clinical workflows.

%Furthermore, for the sleep and activities module in Section \ref{subsubsec:cawearables}, a clinician expressed a preference for viewing data on a weekly basis. To align with this preference, we recommend that the module default to displaying seven days' worth of data while also providing the option to explore detailed data for specific dates. This would allow clinicians to closely examine anomalies, particularly when irregular patterns appear on certain days.

%\subsubsection{Providing Information on Other Stakeholders to Support Collaboration}

%Concussion clinicians noted that in the self-reported symptoms module, the most critical yet challenging situation for clinicians is detecting suicidality in patients. However, they expressed uncertainty about how to remotely respond if a patient exhibits self-harm tendencies. Additionally, clinicians raised concerns about the authenticity of such reports, as youth patients may sometimes exaggerate their symptoms. Clinicians emphasized that if an emergency occurs, parents should be involved, as they are considered key stakeholders in managing such situations. In response to this concern, we propose integrating an emergency contact feature
%within this module, which would allow clinicians to contact parents when necessary. Since parents may have quicker access to their children and be able to take immediate action, this feature could enhance response efficiency. Additionally, clinicians expressed a need to track the number of days symptoms persist, as this directly reflects a patient’s recovery progress and informs mental health-related decisions such as giving a referral. We recommend that when CAs engage in conversations with patients and their parents, they also document their observations on the duration of mental health symptoms. Then, from the design side, we propose adding a data point in the module to display symptom duration.

%In addition to monitoring the patient’s emotional state, clinicians often ask parents about behavioral changes. To enhance data comprehensiveness and objectivity, we suggest incorporating parent-VA (virtual assistant) conversation data into this module, offering an additional perspective on the patient’s condition. Finally, to reduce confusion and improve the module’s usability for clinicians, we need to clearly define how the data is collected. Ensuring transparency in data sources will help clinicians trust and effectively utilize this module in their practice.

%\subsubsection{Balancing Additional Data Presentation and Cognitive Load}
%In the preliminary system design, certain key data types that are highly important to clinicians were not covered. As section \ref{subsubsec:detaileddata} mentioned, concussion clinicians emphasized the need for heart rate during sleep, nap times, wake-up time, and sleep time rather than just total sleep hours. To address this, we recommend displaying these additional data points within the module. However, given clinicians' concerns about data overload—especially as they noted that existing electronic health record (EHR) systems have already increased their workload—we propose a customizable data selection feature. Within the system interface, clinicians can personalize their data view based on their specific needs and workload at different times, rather than being presented with all data at once. Additionally, we believe it is essential to prioritize the most relevant information in the default display. Specifically, the most critical data will be shown by default, while supplementary explanatory details will be secondary and only accessible when needed. This design approach aims to reduce cognitive load, ensuring that clinicians can quickly access key insights without being overwhelmed by excessive information. By minimizing cognitive burden, this method enables more efficient and personalized analysis. 

%\subsubsection{Integrating Related Data into a Unified System Interface}
%In our formative study, clinicians mentioned using GAD-7 and PHQ-9 to monitor patients' anxiety and depression. Additionally, as Section \ref{subsubsec:explanation} mentioned, clinicians noted that they use these questionnaires to patients again during the design interface evaluation, which makes us realize that we could integrate all the information clinicians want to see into the same system interface. Based on this, we propose adding a questionnaire results module to the system interface, displaying patients' completed questionnaires before their clinical visits. This integration would allow clinicians to access all relevant data in a single interface before a consultation, saving time, improving data review efficiency, and enabling a faster and more evidence-based assessment of patients' mental health status. Importantly, these questionnaire results differ from the anxiety and depression statuses identified in the self-reported symptoms module. They provide a more structured and standardized assessment. By combining subjective self-reported symptoms with objective questionnaire scores, clinicians can gain a more comprehensive understanding of patients' mental health conditions, which would support more confident mental health-related decision-making, even when patients are outside the clinical setting.


%\subsection{Each Module of Refined System Design}
%In the new version of our system design, the final system design consists of five modules, which aim to enhance system usability and support clinical decision-making. In the following sections, we will explain the data, functionality, potential interactions, and purpose of each module in detail. The final system design interface is shown in Fig. \ref{fig:figure2}.
%\fi


\subsection{Reducing Cognitive Burden Through EHR Integration}
To reduce the cognitive burden placed on clinicians by the new system and ensure the system can be seamlessly integrated into clinicians' workflow, we referenced the style of current EHR systems to ensure consistency. We integrated a patient list (Fig. \ref{fig:refined_system_design} A), which already exists in the EHR system, into the left side of the interface. However, we added a red dot next to a patient's name to capture clinicians' attention if AI-RPM technologies detect abnormalities in the patient's mental health conditions. Moreover, we added a basic patient information section, which exists in the current EHR system, in the upper-left module (Fig. \ref{fig:refined_system_design} B). However, the information here is tailored to concussion clinicians' needs. These pieces of information provide clinicians with essential insights into a concussion patient's condition. In particular, the Individual Psychiatric History and Mental Health Prescription are important for concussion clinicians when detecting youth patients’ mental health sequelae and making decisions.

\subsection{The Sleep and Physical Activities Module}
Based on findings from our evaluation study, concussion clinicians affirmed the value of the data provided in the Sleep and Physical Activities module (Fig. \ref{fig:preliminary_system_design} C) in the preliminary design. However, several clinicians requested more information about data sources. In response, we designed a question mark next to the module title (Fig. \ref{fig:refined_system_design} C).
Concussion clinicians can hover over the question mark to view the data source. 
Moreover, because clinicians typically prefer a weekly overview of patient data, and some of them also want the ability to see daily metrics (e.g., heart rate), we provided the option to view data from both the last week and the previous 24 hours. 
With this option, concussion clinicians can easily track changes across different time intervals.

In addition, clinicians requested the monitoring of additional metrics, such as heart rate during sleep and nap duration. To incorporate these data points without increasing cognitive load, we organized the data into three tabs: Sleep Data, Physical Activities, and Recommendation Compliance.
The Sleep Data tab is presented by default, which allows clinicians to immediately access important metrics such as heart rate, nap duration, and sleep and wake-up times.
They can choose to view one or multiple metrics, depending on their specific needs.
Different data types are displayed in distinct colors to help clinicians visually distinguish between them more easily. 
If clinicians want to delve deeper into a patient’s sleep patterns, they can hover over a data point in the chart, then a more detailed value for that day will be displayed.

\subsection{The Self-Report Symptoms Module}
To maintain consistency with the Sleep and Physical Activities module, we added question marks indicating the data source and provided time selection options in the Self-Report Symptoms module (Fig. \ref{fig:refined_system_design} D).
Moreover, we identified emergent situations (i.e., suicidality) as both critical and challenging for clinicians to monitor, given resource constraints and the risk of false alarms. 
To address this, clinicians emphasized the importance of including parents in managing emergencies.
Thus, we introduced two sections within this module. 
On the left side, we included the CA-parent conversation history, allowing clinicians to evaluate the patient’s health conditions from family perspectives. 
This conversation history can be expanded by clicking the icon next to the title "Conversational History". 
In addition, we added two call buttons at the bottom of the module, which enables clinicians to contact either the patient or a parent to manage emergency.

Besides, clinicians confirmed the value of different categories (i.e., depression, anxiety) in this module but indicated the need for more information.
In response, we retained the original categories and introduced new ones—such as nutrition level—based on clinician feedback.
Moreover,clinicians mentioned about the importance of severity for their treatment plan.
To meet this need, we placed color-coded dots next to each symptom category (red for severe symptoms, orange for moderate, and green for none). 
To the right of each category, we provided a symptom duration indicator to show how long the reported issue has persisted, which help clinicians form a better understanding of patients health conditions.
Furthermore, clinicians can click AI Summary to view a detailed breakdown of the category, providing additional context to support their assessment. 


\subsection{The Questionnaires Result Module}
In this module (Fig. \ref{fig:refined_system_design} E), the title includes an option to view data source information, which comes from patient-administered questionnaires at home. These questionnaires include GAD-7 for anxiety, PHQ-9 for depression, and PCSS for concussion severity. The middle section of the chart presents the scores for each questionnaire, along with corresponding explanations to help clinicians interpret the results. The module also displays a comparison with previous scores, providing clinicians with a sense of the patient’s recovery trend. Additionally, clinicians can click "Show Details" to view a breakdown of individual section scores within each questionnaire for a more in-depth assessment.

\subsection{The AI Risk Score Prediction Module}
The final module (Fig. \ref{fig:refined_system_design} F) presents the AI-generated prediction of a patient’s risk of developing mental health sequelae. 
Given that concussion clinicians primarily manage patient care during the first four weeks following a concussion, we adjusted the default timeframe for risk predictions from one year to four weeks, while still providing other timeframe options for selection.
Below the imeframe options, the visualization of the risk score chart remains unchanged from the preliminary design, given that clinicians reported it to be easy to interpret. 
However, clinicians expressed an interest in knowing which modules or features influence the AI-generated risk score.
To highlight the relationship between the AI risk score and other modules, we introduce Features Contributing to the AI Risk Score section at the bottom of the module.
The section includes contributing features and its corresponding module, along with a contribution rate.
Moreover, we provide visual feedback on each feature's contribution to the AI risk score. 
Both the color of the dots next to the feature and the font color of its associated percentage value dynamically change based on the degree of the contribution. 
In this way, clinicians can identify the important metrics quickly.






\section{Discussion}
 
In the following sections, we first discuss the use of AI-RPM technologies in the remote monitoring of patient mental health symptoms.
Then, we talked about collaborative design in AI-RPM systems for enhancing usability and effectiveness.
Finally, we conclude with collaborative emergency management with AI-RPM systems. 
Our work contributes to CSCW and HCI research at the intersection of health, RPM, and human-AI decision-making.


\subsection{Unraveling Patients' Intertwined Mental Health and Concussion Symptoms}

In clinical scenarios, it is not uncommon to have multiple illnesses intertwined with each other.
For instance, ~\citet{wu2024cardioai} discovered that cancer oncologists need to collaborate with cardiologists to adjust the cancer treatment plan and be constantly aware of patients' cardiotoxicity.
%during cancer treatment.
Nevertheless, the concussion-mental health scenario and the cancer-cardiotoxicity scenario are fundamentally different.
Concussion clinicians, according to our study, do not collaborate with mental health specialists, and neither do they diagnose mental health disorders. 
In most clinical practices, concussion clinicians assess whether a patient's mental health symptoms exceed a certain severity threshold during in-person visits. Based on this assessment, they adjust the treatment plan accordingly.




We believe that AI-RPM technologies have the potential to give clinicians insights into the severity of patients' mental health symptoms.
First, LLM-based CAs could provide insight into patients' mental health symptoms, such as the level of anxiety. 
Moreover, AI-based predictive models could help assess the likelihood of developing mental health sequelae, which can offer valuable and actionable insights into patients' mental health conditions.
As suggested by \citet{bennett2012ehrs, zhang2024rethinking}, utilizing AI predictive models to provide actionable insights can increase the usefulness of AI in clinicians' workflow. 
Future research could explore the design of AI-RPM technologies for collecting and distinguishing youth patients’ overlapping symptoms.




%\iffalse
%\subsubsection{Remote Monitoring of Objective Patient Data}

%Our study revealed significant challenges for concussion clinicians in assessing mental health conditions among youth concussion patients. 
%Some youth patients and their parents often avoid openly discussing mental health concerns due to the stigma of being labeled with mental health disorders. 
%Consequently, clinicians use mental health self-evaluation questionnaires or direct observation of patients' behaviors during clinical visits to assess their mental health condition. 
%However, if patients do not want to disclose their mental health conditions, the validity of questionnaires can be compromised, leaving observation during clinical visits as the only viable option. 
%Yet this in-clinic observation is inherently limited from the geographical perspective.
%Once patients leave the clinic, concussion clinicians can hardly track patients' mental health status in a timely and effective manner.
% they are beyond the clinician's ability to monitor. 
%Even if patients are willing to faithfully and openly discuss their at-home information about mental health, it's different for patients to report their information (e.g., sleep patterns and heart rate) in a detailed and accurate way. 
% about following recommendations—such as maintaining sufficient daily activity and sleep—tends to be inaccurate.


%We suggest that AI-RPM technologies, particularly wearable devices like smartwatches, could help address the aforementioned limitations.
%Wearable devices can collect and provide key physiological metrics such as sleep patterns, physical activities, and heart rate to clinicians in real-time.
%Prior research demonstrated the feasibility and advantages of using wearable in monitoring objective patient physiological metrics outside the clinic~\cite{mendel2024advice, frazier2023six, su2019novel, wu2024cardioai}. 
%These metrics are objective and can faithfully reflect patients' mental health conditions.
%For instance, elevated anxiety levels may be inferred through reduced physical activity or disrupted sleep patterns, while a stable resting heart rate could indicate recovery progress. 
%Concussion clinicians could use such data for clinical decision-making such as scheduling timely follow-up visits or initiating referrals to mental health specialists as needed.
%Moreover, given that the patients in our setting are youth aged 13 to 17, wearable devices can monitor patients' status in a non-intrusive way, and they can be seamlessly integrated into patients' daily routines.
%This will not only prevent patients from experiencing physical pain but also minimize reminders of their concussion as much as possible, which is good for their recovery.

%\fi




\subsection{Augmenting Objective Data With Subjective Experience}
Traditional RPM technologies, such as wearables, have already been used to remotely monitor concussion patients' objective data~\cite{yang2020bidirectional}.
However, collecting objective data alone is not enough in our scenarios.
%Our study focuses uniquely on mental health sequelae in youth patients after a concussion.
As we discussed in the last section, it is critical to monitor the severity of mental health symptoms, such as anxiety and suicidal ideation.
However, these symptoms are more subjective experiences that can hardly be captured by traditional RPM technologies.
%because the data collected by wearables are standardized objective physiological measurements.
To address this gap, we propose using LLM-based CAs to gather patients' anxiety levels and other mental health conditions at home. 
%LLM-powered CAs have the ability to engage in free-form natural language conversations with patients about their subjective experiences.
LLM-based CAs could automatically detect key psychological risk factors such as suicidal thoughts from patients' conversations and generate a summary reportfor the day, which has been demoed in various prior research works in other scenarios~\cite{bartle2023machine, bartle2022second, wang2023enabling, simpson2020daisy, ma2024understanding, yang2023integrating}.


We believe that subjective information collected from LLM-powered CAs serves not only as supplementary information but also as contextual information for objective data (i.e., heart rate).
Prior HCI and CSCW research has highlighted the principle that technology design should incorporate context into data~\cite{dourish2004we, yoo2024missed}.
By combining subjective information and objective data in AI-RPM systems, clinicians could use the system to assess patients' situations more accurately and avoid unnecessary concern over what might otherwise appear to be worrisome in objective metrics alone. 
%For example, if wearable data shows that a patient had very low physical activity on a given day, the information from the LLM-powered CA might reveal that the low activity was only due to a family gathering at home. 
%Future work can explore the use of LLM-powered CAs and wearables to remotely monitor patients' subjective as well as objective data in other clinical settings to assist clinicians and caregivers.


%Compared with traditional RPM approaches, natural conversation significantly reduces the requirement of patients' technical literacy to engage with the devices, enables flexible patient self-reporting, and can be further designed to conduct regular health check-ins through pre-set question lists. 




%%%%%\subsubsection{Patient Privacy Consideration With AI-RPM Technologies}
%When designing such an AI-RPM system, it is essential to underscore protecting patient privacy and ensuring data security. 
%This issue has been widely discussed in many studies and is recognized as a core challenge in AI system design~\cite{gawankar2024patient, bala2024ensuring, korobenko2024towards}. 
%However, in our research context, the target population comprises youth patients aged 11–17, whose privacy needs are uniquely sensitive. 
%The privacy of youth patients is not only a concern for the patients themselves but is also critical to their parents, who often play a key role in the medical process. 
%Parents typically expect to have transparency and informed consent about how their child's data is collected and used~\cite{sisk2020parental, haley2024attitudes}. 
%Therefore, future research could explore methods to effectively safeguard youth health data while simultaneously increasing engagement and building trust among parents in the data collection and use process. 








\subsection{Collaborative Design in AI-RPM Systems for Enhanced Usability and Effectiveness}

When presenting objective and subjective data in an AI-RPM system, it is important to prioritize the data presentation to the usability of the system. 
Concussion clinicians are already overwhelmed by existing EHR systems, and introducing an AI-RPM system with additional data could further increase their workload and strain limited healthcare resources.
Nevertheless, clinicians expressed a willingness to briefly review AI-RPM data before each visit to better prepare for patient consultations.
%Introducing another AI-RPM system with additional data could significantly increase clinicians' workload and reduce the number of available weekly appointment slots, which further strain already limited healthcare resources. 
%However, concussion clinicians expressed the willingness to quickly review AI-RPM systems before each clinical visit to better prepare for patient consultations.
To provide information without increasing workload, designers need to collaborate with clinicians to prioritize data based on the importance of AI-RPM systems. 
%customized and personalized~\cite{amershi2019guidelines}
Key data should be upfront while keeping secondary information hidden but accessible.
Our work offers a refined system design that presents a clear information hierarchy in a visually intuitive manner (Fig. \ref{fig:refined_system_design}), serving as a reference for future development.




Moreover, domain knowledge is essential for defining thresholds of mental health symptom severity in AI-RPM system design.
Without clear definitions of severity, AI-RPM systems may either underreport or overemphasize symptoms, leading to delayed interventions or unnecessary concerns. 
Thus, researcher and designers should closely collaborate with clinicians to establish clear thresholds for the severity of mental health symptoms so that clinicians get alert by the system only when necessary.
Such a collaborative approach among clinicians, researchers, and designers could enhance the usability and effectiveness of clinician-facing AI-RPM systems.
In the future, researchers and designers should focus on the key thresholds that influence clinician decisions.













\subsection{Collaborative Emergency Management With AI-RPM Systems}
Beyond the design of AI-RPM systems, it is also important to consider how the systems support clinicians during emergency situations in real-world practice.

In our study, concussion clinicians view AI as an important collaborative partner to support their decisions, which aligns with precious research in HCI and CSCW communities~\cite{zhang2024rethinking, hao2024advancing, yang2019unremarkable, zhang2020effect}, 
%Our study highlighted that concussion clinicians view AI as an important collaborative partner to support their decisions,
%rather than relying solely on AI-generated results, 
%which aligns with the research in the HCI and CSCW communities ~\cite{zhang2024rethinking, hao2024advancing, yang2019unremarkable, zhang2020effect}.
However, they expressed two main concerns in emergencies.
The first concern is the potential false alert of "suicidal ideation" in the system. 
%While clinicians recognize the importance of such alerts in handling emergencies, they are concerned about the accuracy of the alert.
%—not only in terms of technical capabilities but also regarding the reliability of self-reported data from youth concussion patients. 
Youth patients may often experience mood swings and express emotions with exaggeration, which could lead to false alarms.
%If a youth patient says, “I want to hurt myself,” it may not reflect true intent, but the AI-RPM system could misclassify it as a high-risk alert, leading to false alarms.
%The implications of such false alarms are multifaceted. 
%First, 
As a result, a false alert could waste clinicians' valuable time and critical medical resources that could be allocated to other patients. 
Moreover, if false alarms occur frequently, clinicians could lose trust in the system, which eventually leads to the abandonment of such a system~\cite{liao2020questioning, amershi2019guidelines}. 
Another concern is the ambiguity of accountability that AI-RPM systems introduce. 
%This issue was also raised in our study. 
When an AI-RPM system triggers an emergency alert (e.g., suicidal ideation), clinicians may face time or resource constraints that delay their response. 
Such delays can lead to serious outcomes and raise questions about accountability, which may affect system adoption.

% This lack of clarity in the allocation of responsibility leaves clinicians feeling uneasy, as they remain solely liable for the consequences despite relying on AI assistance.


In high-risk, uncertain cases involving youth, it’s challenging to verify alerts and balance response with accountability.
We believe that addressing these two challenges requires collaboration among a broader set of stakeholders and clarifying accountability~\cite{goodman2017european} before implementing the system. 
%Although the previous studies have highlighted the phenomenon where clinicians, as final decision-makers, have full responsibility for medical outcomes, we suggest that responsibility needs to depend on different clinical situations and discussion with other departments such as legal or leadership teams.
Previous HCI research has primarily focused on involving clinicians and patients in emergency management with AI systems~\cite{wu2024cardioai, hao2024advancing}. 
We propose involving the youth patient’s family in the decision-making process alongside clinicians to manage emergencies. 
If suicidal ideation is detected, both the clinician and the patient’s family should be notified. 
Since family members are often nearby, they can respond quickly, while clinicians provide timely professional guidance.
%For example, if a patient's suicidal ideation is detected, the AI-RPM system should notify not only the clinician but also the patient’s family immediately.
%This allows family members to respond quickly, as they are often nearby or have better access to the patient.
%During this process, clinicians could collaborate with the family by providing timely professional guidance. 
Such collaboration helps mitigate risks from delayed responses while ensuring shared responsibility in emergencies.
Future research could explore parent-facing systems that integrate with clinicians' systems to enhance remote youth patient care.



\section{Limitations and Future Work}
Our work is not without limitations. 
First, the sample size of our participants was limited, with six concussion clinicians participating in both the formative and evaluation studies.
Recruiting these experts was particularly challenging due to their demanding workloads~\cite{wang2021brilliant,jin2020carepre} and the specialized expertise required to treat youth concussion patients with mental health issues
As a result, we recruited six highly relevant experts in this specific domain. 
Despite the limited number, our inductive thematic analysis reached thematic saturation.
%which suggests that additional participants may not have yielded new insights.
Moreover, previous research employed a similar number of expert participants in related studies~\cite{cai2019hello,beede2020human,jacobs2021designing,yang2024talk2care,zhang2024rethinking}, which supports the appropriateness of our sample size. 
%Although our recruited clinicians came from different hospitals and regions, enhancing sample diversity to some extent, all of them are practicing clinicians based in the United States.
Future research should include a broader range of concussion clinicians across locations for greater generalizability of the findings. 

Secondly, the refined AI-RPM system has not yet been further evaluated by concussion clinicians to assess its effectiveness and usability. 
In future research, we plan to incorporate participatory design methodologies~\cite{muller1993participatory} to revise the system design, ensuring it aligns more closely with clinicians' workflows and needs. 
Moreover, our study focused on system design rather than developing an interactive prototype. The primary goal of our system design was to explore the feasibility and provide insights for further design refinements before actual development, thereby saving resources and time for both researchers and clinicians. 
Future research should investigate clinicians' pain points and needs when interacting with a functional system developed. 


Finally, we want to highlight the potential of applying AI-RPM systems in varied healthcare settings. %%%not sure if it's a proper expression
Several studies have explored remote monitoring in the healthcare domain. 
~\citet{wyche2024limitations} utilized mHealth to track the health conditions of type 1 diabetes among youth. 
~\citet{seals2022they} leveraged wearables to detect gait impairment, monitor patients' responses to treatment, and visualize patient data.
We believe that AI-RPM technologies have the potential to be leveraged and extended to these scenarios.
%The AI-RPM systems could enable experts and caregivers to better digest the massive amount of data from RPM and make informed decisions based on the comprehensive information the system provides.
Future work should expand AI-RPM technologies to diverse healthcare settings, enabling timely interventions and improving care quality.






% \section{Future Work}


\section{Conclusion}
%% !!emphasizes the novelty

We conducted a formative study with six concussion clinicians to understand the challenges and needs during their clinical practice with youth concussion patients, particularly with respect to the patient mental health sequelae. 
Then, we derived a suite of design considerations with the use of AI-RPM technologies. Here are three design considerations. 
Firstly, we recommend the use of wearables to remotely monitor patients’ sleep and physical activity data, helping assess treatment compliance and inform mental health-related decisions. 
Secondly, we propose LLM-powered CAs to enhance mental health self-reporting, which facilitates private, natural interactions with patients and summarizes key information for clinicians. 
Thirdly, we suggest leveraging AI risk prediction in detecting concealed or worsening mental health sequelae.  Finally we delivered a clinician-facing system design as our final artifact. 
To our knowledge, this is the first study in the CSCW field to focus on the intersection of health, RPM, and human-AI decision-making, which provides a foundation for designing AI-RPM systems in varies medical scenarios.


% % \smallskip
% \myparagraph{Acknowledgments} We thank the reviewers for their comments.
% The work by Moshe Tennenholtz was supported by funding from the
% European Research Council (ERC) under the European Union's Horizon
% 2020 research and innovation programme (grant agreement 740435).


%%
%% The next two lines define the bibliography style to be used, and
%% the bibliography file.
\bibliographystyle{ACM-Reference-Format}
\bibliography{sample-base}
%%
%% If your work has an appendix, this is the place to put it.
\appendix
\section{APPENDIX: INTERVIEW SCRIPTS FOR FORMATIVE STUDY}
\label{sec:appendixa}
We begin with a brief introduction of our research and study procedure.

\textbf{Background}

(1) Could you please tell me your years of practice, department, and specialization? 

(2) How many patients do you see each year? How many of them have mental health symptoms? How do you handle them (maybe they don’t need to refer)?

\textbf{A Recent Experience}

(1) Can you think of a recent time when you had a young concussion patient (11-17 years old) and thought they might have mental health issues? How did you find out?

\textbf{Information Collection and Monitoring}

(1) (If not mentioned in the previous question) For that patient in your story, before the final referral, what symptom-related information makes you think it might be a mental health issue? How did you deal with it?

(2) (If they mention any at-home symptoms) What kind of symptom-related information do you find helpful when patients are resting at home (smartwatch and smart speaker)? (If not mentioned) Do you need any at-home symptoms? (e.g.,  patients' mental status or sleep patterns at home ) 

(3) (If not mentioned) How do their parents get involved in this process? 

\textbf{Challenges and Workaround Strategies}

(1) What challenges do you have when it comes to collecting symptom-related information from patients, managing that information, or making mental health decisions (e.g., further mental health testing or referring to a mental health expert)?

\textbf{Technology (Expectations and Concerns)}

(1) Do you use any existing tools or features in EHR (e.g., EPIC) in the youth patients' mental health decision-making? If so, what are they, and how do you like them?

\section{APPENDIX: INTERVIEW SCRIPTS FOR EVALUATION STUDY}
\label{sec:appendixb}

(1) This section (Fig.1) shows the probability of mental health sequelae within the next 15 days with a range from 0\% to 100\%, where the higher probability score signifies a higher probability of experiencing mental health sequelae. Feature importance scores show the contributing factors to the probability. The risk score chart below shows the risk scores for the past ten days, today, and the next 15 days. We’d love your comments. What do you think about it? What other information or visualization would you like to see?

(2) (Less familiar, need extra emphasis) In this section, we will use a smart speaker to have conversations with patients. The conversation history is shown on the right. On the left is the symptom overview section. If patient-reported symptoms toward a particular category, the dot for that category will change from green to red as a highlight for your review. You can click on the category name or use the search bar to type in keywords,  then the corresponding conversation content will be shown on the right. What do you think about it? What other information or visualization would you like to see?

(3) (More familiar, need less time) The data in this section is collected from patients' smartwatches. You can click on other data buttons to view corresponding recorded data in detail. What do you think about it? What other information or visualization would you like to see?

(4) Above these three sections, what other information would you need to help make mental health decisions?

\textbf{Closing Question}
(1) Is there anything else you’d like to share or a final question to ask us? Let us know or reach out via follow-up email.




\end{document}
\endinput
%%
%% End of file `sample-sigconf-authordraft.tex'.
