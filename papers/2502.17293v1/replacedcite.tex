\section{Related Work}
There is a long tradition in HCI of considering how to design technology to embrace positive human values. One established framework for doing so is \textit{Value Sensitive Design} (VSD)____. Value Sensitive Design is a three-part framework that includes \textit{conceptual, empirical,} and \textit{technical} investigations of human values and technology stakeholders. When taking a VSD approach, the values are always defined by the designer in the conceptual inquiry stage. While VSD acknowledges that different people may come up with different definitions of values in the conceptual inquiry____, VSD does not address the implications of values being defined differently by different people.

However, critiques of VSD have expressed such concerns. One such example is the work of Alsheikh et al.____ which highlights that when taking a VSD approach, cross-cultural interpretations of a particular value may lead to vastly different design implications. While this work discussed different conceptions of privacy and intimacy across Western and Islamic cultures, in our work we investigate in the opposite direction. Rather than asking how values are interpreted across cultures for the purpose of designing for that culture, we seek to understand how familiarity with a particular cultural or religious value impacts one's ability to create designs that embody it.

HCI researchers have proposed values from different cultural or religious contexts that would make technology more conducive to human flourishing. As examples, Hiniker and Wobbrock____ proposed that technology design focused on Christian values would promote relationship with God, relationship with others, and relationship with creation, especially to resist the attention economy. Hammer and Reig proposed that designing for a Jewish conception of \textit{obligations} could inform more effective design frameworks for online speech____. McGuire____ examined two self-tracking apps inspired by Buddhist values, and shows that the Buddhist principles underlying the apps help the users develop greater emotional and ethical awareness. Escher and Banovic____ propose using lessons from Homeric poetry to inspire designs that resist extractive mechanisms like infinite scroll. Each of these works introduces a value or values from a particular tradition to the wider community. In every case, the designs proposed are meant to benefit all, not just those from the tradition that originated the underlying values.