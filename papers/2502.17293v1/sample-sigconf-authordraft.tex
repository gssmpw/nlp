%%
%% This is file `sample-sigconf-authordraft.tex',
%% generated with the docstrip utility.
%%
%% The original source files were:
%%
%% samples.dtx  (with options: `all,proceedings,bibtex,authordraft')
%% 
%% IMPORTANT NOTICE:
%% 
%% For the copyright see the source file.
%% 
%% Any modified versions of this file must be renamed
%% with new filenames distinct from sample-sigconf-authordraft.tex.
%% 
%% For distribution of the original source see the terms
%% for copying and modification in the file samples.dtx.
%% 
%% This generated file may be distributed as long as the
%% original source files, as listed above, are part of the
%% same distribution. (The sources need not necessarily be
%% in the same archive or directory.)
%%
%%
%% Commands for TeXCount
%TC:macro \cite [option:text,text]
%TC:macro \citep [option:text,text]
%TC:macro \citet [option:text,text]
%TC:envir table 0 1
%TC:envir table* 0 1
%TC:envir tabular [ignore] word
%TC:envir displaymath 0 word
%TC:envir math 0 word
%TC:envir comment 0 0
%%
%% The first command in your LaTeX source must be the \documentclass
%% command.
%%
%% For submission and review of your manuscript please change the
%% command to \documentclass[manuscript, screen, review]{acmart}.
%%
%% When submitting camera ready or to TAPS, please change the command
%% to \documentclass[sigconf]{acmart} or whichever template is required
%% for your publication.
%%
%%
%\documentclass[manuscript,review,anonymous]{acmart}
\documentclass[manuscript]{acmart}
%%
%% \BibTeX command to typeset BibTeX logo in the docs
\AtBeginDocument{%
  \providecommand\BibTeX{{%
    Bib\TeX}}}

%% Rights management information.  This information is sent to you
%% when you complete the rights form.  These commands have SAMPLE
%% values in them; it is your responsibility as an author to replace
%% the commands and values with those provided to you when you
%% complete the rights form.
%\setcopyright{acmlicensed}
%\copyrightyear{2025}
%\acmYear{2025}
%\acmDOI{XXXXXXX.XXXXXXX}
%% These commands are for a PROCEEDINGS abstract or paper.
%\acmConference[Conference acronym 'XX]{Make sure to enter the correct
 % conference title from your rights confirmation emai}{June 03--05,
 % 2018}{Woodstock, NY}
%%
%%  Uncomment \acmBooktitle if the title of the proceedings is different
%%  from ``Proceedings of ...''!
%%
%%\acmBooktitle{Woodstock '18: ACM Symposium on Neural Gaze Detection,
%%  June 03--05, 2018, Woodstock, NY}
%\acmISBN{978-1-4503-XXXX-X/18/06}


%%
%% Submission ID.
%% Use this when submitting an article to a sponsored event. You'll
%% receive a unique submission ID from the organizers
%% of the event, and this ID should be used as the parameter to this command.
%%\acmSubmissionID{123-A56-BU3}

%%
%% For managing citations, it is recommended to use bibliography
%% files in BibTeX format.
%%
%% You can then either use BibTeX with the ACM-Reference-Format style,
%% or BibLaTeX with the acmnumeric or acmauthoryear sytles, that include
%% support for advanced citation of software artefact from the
%% biblatex-software package, also separately available on CTAN.
%%
%% Look at the sample-*-biblatex.tex files for templates showcasing
%% the biblatex styles.
%%

%%
%% The majority of ACM publications use numbered citations and
%% references.  The command \citestyle{authoryear} switches to the
%% "author year" style.
%%
%% If you are preparing content for an event
%% sponsored by ACM SIGGRAPH, you must use the "author year" style of
%% citations and references.
%% Uncommenting
%% the next command will enable that style.
%%\citestyle{acmauthoryear}


%%
%% end of the preamble, start of the body of the document source.
\begin{document}

%%
%% The "title" command has an optional parameter,
%% allowing the author to define a "short title" to be used in page headers.
\title{The Challenges of Bringing Religious and Philosophical Values Into Design}

%%
%% The "author" command and its associated commands are used to define
%% the authors and their affiliations.
%% Of note is the shared affiliation of the first two authors, and the
%% "authornote" and "authornotemark" commands
%% used to denote shared contribution to the research.
\author{Louisa Conwill}
\email{lconwill@nd.edu}
\orcid{0009-0001-7116-266X}
\affiliation{%
  \institution{University of Notre Dame}
  \city{Notre Dame}
  \state{Indiana}
  \country{USA}
  \postcode{46556}
}

\author{Megan K. Levis}
\email{mlevis@nd.edu}
%\orcid{}
\affiliation{%
  \institution{University of Notre Dame}
  \city{Notre Dame}
  \state{Indiana}
  \country{USA}
  \postcode{46556}
}

\author{Karla Badillo-Urquiola}
\email{kbadill3@nd.edu}
%\orcid{}
\affiliation{%
  \institution{University of Notre Dame}
  \city{Notre Dame}
  \state{Indiana}
  \country{USA}
  \postcode{46556}
}

\author{Walter J. Scheirer}
\email{wscheire@nd.edu}
%\orcid{}
\affiliation{%
  \institution{University of Notre Dame}
  \city{Notre Dame}
  \state{Indiana}
  \country{USA}
}



%%
%% By default, the full list of authors will be used in the page
%% headers. Often, this list is too long, and will overlap
%% other information printed in the page headers. This command allows
%% the author to define a more concise list
%% of authors' names for this purpose.
\renewcommand{\shortauthors}{Conwill et al.}

%%
%% The abstract is a short summary of the work to be presented in the
%% article.
\begin{abstract}
HCI is increasingly taking inspiration from philosophical and religious traditions as a basis for ethical technology designs. If these values are to be incorporated into real-world designs, there may be challenges when designers work with values unfamiliar to them. Therefore, we investigate the variance in interpretations when values are translated to technology designs. To do so we identified social media designs that embodied the main principles of Catholic Social Teaching (CST). We then interviewed 24 technology experts with varying levels of familiarity with CST to assess how their understanding of how those values would manifest in a technology design. We found that familiarity with CST did not impact participant responses: there were clear patterns in how all participant responses differed from the values we determined the designs embodied. We propose that value experts be included in the design process to more effectively create designs that embody particular values.
\end{abstract}

%%
%% The code below is generated by the tool at http://dl.acm.org/ccs.cfm.
%% Please copy and paste the code instead of the example below.
%%
%\begin{CCSXML}
%<ccs2012>
 %  <concept>
  %     <concept_id>10003120.10003121.10011748</concept_id>
   %    <concept_desc>Human-centered computing~Empirical studies in HCI</concept_desc>
   %    <concept_significance>500</concept_significance>
    %   </concept>
 %</ccs2012>
%\end{CCSXML}

%\ccsdesc[500]{Human-centered computing~Empirical studies in HCI}

%%
%% Keywords. The author(s) should pick words that accurately describe
%% the work being presented. Separate the keywords with commas.
\keywords{values, design, value sensitive design, value interpretation, ethics, spirituality, religion, philosophy}
%% A "teaser" image appears between the author and affiliation
%% information and the body of the document, and typically spans the
%% page.

%\received{20 February 2007}
%\received[revised]{12 March 2009}
%\received[accepted]{5 June 2009}

%%
%% This command processes the author and affiliation and title
%% information and builds the first part of the formatted document.
\maketitle

\section{Introduction}
Designing to support positive human values is a key objective of Human-Computer Interaction (HCI) research. Yet, which values should be designed for and how to elicit such values is debated~\cite{borning2012next, le2009values}. Recently, HCI scholars have been turning to religious~\cite{ hiniker2022reclaiming, mcguire2020buddhist, rifat2022integrating, hammer2022individual, toyama2022technology, naqshbandi2022making}, philosophical~\cite{vallor_2018, gorichanaz2024virtuous, zhang2024searching}, and mythological~\cite{escher2024hexing} frameworks to identify values to guide technology design. These values provide an ethical grounding rooted in the time-tested wisdom of our traditions.

No matter one's background, one can appreciate the values from varied ethical traditions put forth in each of these works. These ideas will have the greatest impact if they are not limited to academic discussion but rather move into practical implementations within industry. However, moving such ideas into the realm of practical implementation surfaces a difficult question: how will people design for values that they are not familiar with? For example, if a Christian designer wants embed the Jewish values proposed by Hammer and Reig~\cite{hammer2022individual} into their designs, how will their understanding of these values be impacted by their different background and perhaps lack of familiarity with Judaism? We have found little research in HCI investigating how designers interpret values to design for, especially considering religious values. We attempt to fill this gap by investigating the following research questions:
\begin{itemize}
    \item \textit{RQ1:} How much variance is there in the interpretation of how specific values manifest within a particular technology's design?
    \item \textit{RQ2:} Does this variance vary based on previous familiarity with the value system?
\end{itemize}

We determine the authoritative definitions of the values we investigate through ``expert knowledge.'' We define expert knowledge to be an extensive knowledge and engagement with a value system, including but not limited to graduate-level studies in the value system, and engagement in the value system through clergy or ministerial work.  

We are HCI researchers with expert knowledge in \textit{Catholic Social Teaching} (CST): the Catholic Church's doctrine on human dignity and societal good~\cite{compendium2004}. To answer these research questions we cataloged seven common designs of social media that we found upheld the values of CST. We then performed interview studies with 24 technologists with varying degrees of familiarity with CST, asking them which values of CST they thought the designs upheld. We compared their responses to the values we determined the designs embodied. We found that participants often missed values that we intended or added values that we did not intend in their responses, and that increased (but not expert-level) familiarity with CST did not make participant responses more like an expert's. We conclude with a discussion on the implications for design processes given our findings, proposing that value experts be included in the design process for more effective embodiment of the desired values.

Our contribution to the literature is the results of our interviews, which shed light on the challenges of operationalizing values in real-world design settings. We hope to spark a conversation within the SIGCHI community on how to better support the incorporation of traditional values into design and how to effectively move from philosophical values to practical implementations.

\section{Related Work}

There is a long tradition in HCI of considering how to design technology to embrace positive human values. One established framework for doing so is \textit{Value Sensitive Design} (VSD)~\cite{friedman2019value}. Value Sensitive Design is a three-part framework that includes \textit{conceptual, empirical,} and \textit{technical} investigations of human values and technology stakeholders. When taking a VSD approach, the values are always defined by the designer in the conceptual inquiry stage. While VSD acknowledges that different people may come up with different definitions of values in the conceptual inquiry~\cite{friedman2013value}, VSD does not address the implications of values being defined differently by different people.

However, critiques of VSD have expressed such concerns. One such example is the work of Alsheikh et al.~\cite{alsheikh2011whose} which highlights that when taking a VSD approach, cross-cultural interpretations of a particular value may lead to vastly different design implications. While this work discussed different conceptions of privacy and intimacy across Western and Islamic cultures, in our work we investigate in the opposite direction. Rather than asking how values are interpreted across cultures for the purpose of designing for that culture, we seek to understand how familiarity with a particular cultural or religious value impacts one's ability to create designs that embody it.

HCI researchers have proposed values from different cultural or religious contexts that would make technology more conducive to human flourishing. As examples, Hiniker and Wobbrock~\cite{hiniker2022reclaiming} proposed that technology design focused on Christian values would promote relationship with God, relationship with others, and relationship with creation, especially to resist the attention economy. Hammer and Reig proposed that designing for a Jewish conception of \textit{obligations} could inform more effective design frameworks for online speech~\cite{hammer2022individual}. McGuire~\cite{mcguire2020buddhist} examined two self-tracking apps inspired by Buddhist values, and shows that the Buddhist principles underlying the apps help the users develop greater emotional and ethical awareness. Escher and Banovic~\cite{escher2024hexing} propose using lessons from Homeric poetry to inspire designs that resist extractive mechanisms like infinite scroll. Each of these works introduces a value or values from a particular tradition to the wider community. In every case, the designs proposed are meant to benefit all, not just those from the tradition that originated the underlying values.

\section{Methods}
We wanted to perform a controlled assessment of how familiarity with different values impacts the understanding of how those values would be translated into technology designs. To do so, we first assembled a collection of social media designs that we believed embodied particular principles of CST per our expert understanding. (We only considered social media platforms to give the project a tighter scope.) We then conducted interviews in which we asked technology experts of varied backgrounds which principles of CST \textit{they} thought the designs embodied, to assess their interpretations of the values in comparison to our expert interpretations. We will describe both steps in more detail below.

\begin{figure}[b]
    \centering
    \includegraphics[width=\linewidth]{figures/CST_conceptual_inquiry.png}
    \caption{The six principles of Catholic Social Teaching we engaged with in this study, their definitions, and our understanding of how these principles would be embodied by a technological design.}
    \label{fig:conceptual}
    \Description{The six principles of Catholic Social Teaching we consider in this work are listed here with a short definition and bullet points explaining how the principle applies in the context of social media. They are as follows. Principle 1,  Life and Dignity of the Human Person. Definition: Every person is precious, and people are more important than things. Conceptual inquiry: Require as much attention as needed from a person for them to have meaningful social interactions and no more. Foster authentic dialogue. Allow friction in interactions. Encourage private spaces for moral exploration, and encourage moderation and dialogue when discussing moral or contentious issues with a wider audience. Principle 2, Call to family, community, and participation. Definition: How we organize our society directly affects the capacity of individuals to grow in community. Conceptual Inquiry: People have a right and a duty to participate in society, seeking together the well-being of all. Do not build technologies that have the goal of replacing in-person interactions. Social technologies are at the service of fostering stronger connections between families and communities. Principle 3, option for the poor and vulnerable. Definition: A basic moral test is how our most vulnerable members are faring. Our tradition instructs us to put the needs of the poor and vulnerable first. Conceptual Inquiry: Prioritize user well being over excess profits. Avoid attention-based profit models. Create transparent technologies: make clear every intention that was designed into a technology. Prioritize needs of marginalized groups in design. Principle 4, solidarity. Definition: Solidarity calls us to show concern for all people struggling throughout the world and brings us into deeper relationship with the wider human family. Conceptual inquiry: Create technology that allows us to better engage with global and domestic issues, without causing empathic overload or reducing social issues into memes that are easily forgotten. Create technology that allows us to solve global and domestic issues together. Principle 5, subsidiarity. Definition: According to the principle of subsidiarity, decisions should be made at the lowest level possible and the highest level necessary. Subsidiarity works best when citizens are engaged in their local communities. Conceptual inquiry: Create small or large group technologies where appropriate. Where available, allow users to self-direct the moderation of their sites. Smaller platforms for specific purposes tend to be better than large conglomerates. Principle 6, care of God's creation. Definition: We are called to protect people and the planet, living our faith in relationship with all of God’s creation. Conceptual inquiry: Build technologies that do not demand more attention than they need, allowing attention to be directed at other things like wonder at the environment. Build technology that explicitly fosters an appreciation of nature.}
\end{figure}

\textbf{Design Collection.} While this study could be performed with any value system, we chose CST because of our personal expertise which allowed us to declare with confidence that a principle of CST is or is not embodied by a design. The principles of CST we considered are: \textit{life and dignity of the human person; call to family, community, and participation; option for the poor and vulnerable; solidarity; subsidiarity; and care of God's creation}. The definitions of each of these principles according to the United States Conference of Catholic Bishops and the Compendium of the Social Doctrine of the Church~\cite{usccb, compendium2004}, as well as \textit{our interpretations} of how each principle would be embodied by a technology design can be found in Figure~\ref{fig:conceptual}.

\begin{figure}[b]
    \centering
    \vspace{-10pt}
    \includegraphics[width=\linewidth]{figures/Design_Patterns_Abridged.png}
    \caption{The social media designs that we found embodied various principles of Catholic Social Teaching. For each design, we give a brief description of the design and the principles of Catholic Social Teaching that we thought each design embodied. We showed these designs to our interview participants to assess their understandings of the principles of Catholic Social Teaching in a technological context.}
    \Description{The seven designs are listed with their name, intent, solution, consequences, and virtues embodied. They are as follows. Pattern 1. Name: Chats Over Feeds. Intent: Promote more intentional conversations online. Encourage individual communication, more private communication, and encourage responses from those involved. Solution: When possible, prioritize chat-style designs over feed-style designs. Consequences: Depending on the nature of the application, a feed style of sharing content may be more appropriate or desirable. It is up to the designer to weigh the trade-offs between the chat-style of sharing content and the feed-style of sharing content based on the context, with a preference towards chat-style of sharing content in most contexts. Virtues Embodied: Life and Dignity of the Human Person, Option for the Poor and Vulnerable, Care of God’s Creation. Pattern 1. Name: Friends Over Followers. Intent: Encourage that free-form, frequent, and more vulnerable online sharing occur only with closer and fewer connections. Solution: The more free-forming the sharing is, the more that should be done in small groups with “friend” connections where both parties have to add each other in order to interact. Free-forming sharing can include long discussions without moderation, the sharing of personal images that could be filtered or doctored, etc. This pattern can optionally include a prompt to add only people one knows in real life, saying something along the lines of, “Are you sure you know them well?” (This specific example is taken from BeReal.) Consequences: If this pattern was employed all the time, then some good uses of “following,” like thoughtful Substack blogs, would be eliminated. Virtues Embodied: Life and Dignity of the Human Person, Subsidiarity/ Pattern 3. Name: Moderated Mingling. Intent: For larger-group communities, have focused intentions for the conversation and keep conduct respectful. Solution: Encourage self-moderated communities where communities set their own terms of moderation, especially focused on keeping the conversation on topic and the conduct respectful. Consequences: Users may feel their freedom is violated by having to follow moderation guidelines. Virtues Embodied: Life and Dignity of the Human Person, Option for the Poor and Vulnerable, Solidarity, Subsidiarity. Pattern 4. Name: Clear Algorithmic Comprehension. Intent: Allow users to know why types of content are being recommended to them and give them greater control over recommended content. Solution: Make content recommendation algorithms as transparent as possible to reveal why they recommend the content that they do. Furthermore, make algorithms customizable so users can add and remove the types of content that they want to be shown. Additionally include the capacity to turn off recommended content. Consequences: This is technically difficult to do. Virtues Embodied: Option for the Poor and Vulnerable, Subsidiarity. Pattern 5. Name: Moderated Entry. Intent: For large-group forums or discussions, restrict entry to make sure conversations stay on topic, are respectful and that real people (as opposed to spam bots) are engaging in the conversation Solution: Have an identity verification process or some sort of other vouching process (e.g. via phone number, email address, personal conversation, etc.) to ensure group members are real people who will make meaningful contributions. Consequences: In some cases, anonymity is desirable. For example, if those in marginalized communities want to communicate without retaliation. In these cases, employing moderation, smaller-group connecting, or singularly-focused applications are especially important. To avoid cliquishness, this design pattern should be balanced out with other patterns to encourage enough openness for productive dialogue. Moderation criteria should not be unreasonably exclusive. Virtues Embodied: Life and Dignity of the Human Person, Subsidiarity. Pattern 6. Name: No Lurking. Intent: Encouraging participation in conversations rather than just lurking without participating. Solution: Hide the posts of others until the user themselves makes an initial post. Consequences: Depending on the purpose of the application, lurking may be acceptable. The more the purpose of the application entails intentional participation from every member the more this pattern should be considered. Virtues Embodied: Call to Family, Community, and Participation. Pattern 7. Name: Notification Intentionality. Intent:  Notifications should encourage the user towards using the app in ways that promote connecting with others without being excessive. Solution: Applications should only send notifications that are essential for users to use the platform well. Any excessive notifications encouraging the user to open the app unnecessarily should be avoided. Consequences: Limited notifications could lead users to miss an ongoing conversation among other friends. Virtues Embodied: Life and Dignity of the Human Person, Option for the Poor and Vulnerable, Care of God’s Creation.}
    \label{fig:designs}
\end{figure}

We then searched for designs that uphold or violate these principles of CST according to what we determined they mean in light of social technologies. We considered five social communications platforms: X (formerly known as Twitter), BeReal, TikTok, Gmail, and Discord. We chose these platforms because they encompass a variety of designs and affordances. We took a VSD approach~\cite{millett2001cookies} in analyzing the platforms' technological features: for each platform, we wrote out a list of its features, and then considered how each feature upholds or violates the different principles of CST, if at all. We then grouped similar designs together and summarized each design grouping. An overview of these designs can be found in Figure~\ref{fig:designs}.

\textbf{Interviews.} We interviewed technology researchers and industry practitioners to assess how they saw the principles of CST embodied in these seven designs. In this way, we can better understand their interpretations of the principles of CST as they relate to technology design. Interviews were conducted in-person and on Zoom (with cameras turned on) depending on feasibility and the preference of the participant. Participants were first asked to share about their work experience and provide basic demographic information. They were were then given a handout created by the researchers with a definition of each of the principles of CST and some examples of how that principle would manifest in a non-technological context. No technological examples were given so as not to bias the study participants. In the interview, the participants were shown each of the seven designs one-by-one (the ``CST principles embodied'' were removed from the descriptions so as not to bias participant responses). The participants were invited to think through how the design would work, including the pros and cons of each design. Then, we asked participants to identify which principles of CST they thought the pattern embodied, using the CST definitions handout as a reference. A sample definition from the handout can be found in Figure~\ref{fig:handout}; the full handout can be found in the Appendix.

\begin{figure}
    \centering
    \includegraphics[width=\linewidth]{figures/Life_and_dignity_Infographic.png}
    \caption{The explanation of the \textit{life and dignity of the human person} principle of Catholic Social Teaching from the handout we gave to the study participants. We provided the definition of the principle and gave examples from a non-technological context to explain the principle to participants unfamiliar with Catholic Social Teaching.}
    \Description{The text in this figure says, "Life and dignity of the human person. We believe that every person is precious, that people are more important than things, and that the measure of every institution is whether it threatens or enhances the life and dignity of the human person. Examples: No intentional targeting of civilians in war or terrorist attacks; Prioritizing people over profit; Recognizing the good in every person regardless of their race, class, abilities, etc.; Promotion of authentic, healthy, and respectful dialogue, especially with those with whom we disagree or have different life experiences from." There is an icon of two cartoon people hunched over, one with a cane, indicating they are elderly.
}
    \label{fig:handout}
    \vspace{-10pt}
\end{figure}

\begin{table}[htb]
\caption{\textbf{Participant Demographics}}
\label{tab:participant_demographics}
\begin{tabular}{|cllllc|}
\hline
\textbf{Gender} & \textbf{Race/Ethnicity} & \textbf{Religion} & \textbf{Tech Work Experience}           & \textbf{Domain}   & \textbf{CST Familiarity}   \\ \hline
Man             & White                   & Catholic          & AI research, Software Engineering       & Academia/Industry             & 3\\ \hline
Man             & White                   & Catholic          & AI research/development                 & Industry          & 4\\ \hline
Woman           & White                   & Catholic          & Software engineer                       & Industry          & 4\\ \hline
Woman           & White                   & Catholic          & Software engineer                       & Industry          & 5\\ \hline
Man             & White                   & Catholic          & AI research/development                 & Industry          & 4\\ \hline
Woman           & White                   & Protestant        & Security research                       & Academia          & 1\\ \hline
Woman           & White                   & Catholic          & Mobile developer                        & Industry          & 4\\ \hline
Woman           & White                   & Catholic          & UX design, HCI research                 & Academia/Industry     & 3\\ \hline
Woman           & Middle Eastern          & Muslim            & AI research                             & Academia          & 1\\ \hline
Man             & White                   & Catholic          & Math/Computer Science research          & Industry          & 3\\ \hline
Man             & White                   & Catholic          & Software engineer                       & Industry          & 2\\ \hline
Man             & White                   & Catholic          & AI research                             & Academia          & 2\\ \hline
Woman           & White                   & Catholic          & Product designer                        & Industry          & 3\\ \hline
Woman           & White                   & Protestant        & HCI research                            & Academia          & 3\\ \hline
Man             & White                   & Catholic          & ML engineering, Product design          & Industry          & 4\\ \hline
Man             & White                   & Catholic          & AI researcher                           & Academia/Industry     & 4\\ \hline
Man             & Asian                   & Catholic          & Front end engineer                      & Industry          & 3\\ \hline
Man             & Black                   & Catholic          & Software engineer/Data scientist        & Industry          & 3\\ \hline
Nonbinary       & Asian                   & Agnostic          & HCI research, Human factors engineer    & Academia/Industry     & 2\\ \hline
Woman           & Hispanic/Latino         & Catholic          & Mobile developer                        & Industry          & 3\\ \hline
Man             & White                   & Catholic          & Software engineer                       & Industry          & 3\\ \hline
Woman           & White                   & Catholic          & Software engineer                       & Industry          & 3\\ \hline
Man             & Mixed Race              & Protestant        & Product management                      & Industry          & 1\\ \hline
Man             & Asian                   & Atheist           & HCI researcher, AI research/development & Academia/Industry     & 1\\ \hline
\end{tabular}
\end{table}












\textbf{Participant Recruitment and Demographics.} We recruited 24 participants. Participants were required to be at least 18 years of age and to be a developer, researcher, or designer with a background in computer science, software engineering, human-computer interaction, or a related discipline. Participants were recruited through word of mouth, participant referral, and social media including X and relevant Slack teams. Participants participated in this study on a volunteer basis. At the time of participation, all participants resided in the United States. Table \ref{tab:participant_demographics} summarizes our participant demographics. None of our participants were value experts in CST. We asked participants to self rate their knowledge of CST before beginning the study on a scale of 1 to 5, where 1 meant ``I've never heard of CST,'' 2 meant ``I'm familiar with CST in name only,'' 3 meant, ``I'm slightly familiar with CST,'' 4 meant ``I'm somewhat familiar with CST,'' and 5 meant ``I'm very familiar with CST.'' (Note: as none of our participants were value experts in CST, option 5 designates the highest level of familiarity one can have without being a value expert.)  It is important to note that while CST is a doctrine of the Catholic Church, not all Catholics are familiar with it (CST is sometimes referred to as the Catholic Church's ``best kept secret''~\cite{Magliano_2012}.) Although we had a high number of Catholic participants in our study, they had varying levels of familiarity with CST, thus still allowing us to effectively probe different levels of familiarity.

\section{Results}

\textit{RQ1} asked \textit{How much variance is there in the interpretation of how certain values manifest within a particular
technology’s design?} We answer this question by considering both how participant responses compared to the values intended by the researchers, and also considering how participant responses compared to each other.

For each design we counted the number of participants who missed each of the intended principles in their responses and the number of participants who added unintended principles. The summary of results is in Figure \ref{fig:missed-added}.

\begin{figure}[b]
    \centering
    \includegraphics[width=\linewidth]{figures/Missed_or_added.png}
    \caption{For each design, we highlight the CST principles that were missed or added by a majority (over half) of the participants compared to the principles we thought the designs embodied. We also highlight the principles that were missed or added by almost no (three or fewer) participants. We see that for all but one of the designs a majority of participants missed or added at least two principles, and also that a majority of participants often responded similarly to each other.}
    \Description{The six principles of Catholic Social Teaching and the seven designs are listed in a comparison chart. There is a check mark for every principle embodied by that design. The comparison boxes list how many participants missed or added that principle for that design, and are colored in red if the principle was frequently missed or added for that design, blue if the principle was infrequently missed or added for that design, or white for neither. Design: chats over feeds. Life and dignity: checked, number missed is 15, colored in red. Option for the poor: checked, number missed is 23, colored in red. Call to family: number added is 22, colored in red. Solidarity: number added is 7 colored in white. Subsidiarity: number added is 5, colored in white. Care of creation: checked, number missed is 21, colored in red. Design: friends over followers. Life and dignity: checked, number missed is 12, colored in white. Option for the poor: number added is 1, colored in blue. Call to family: number added is 19 colored in red. Solidarity: number added is 6, colored in white. Subsidiarity: checked, number missed is 17, colored in red. Care of creation: number added is 0, colored in blue. Design: moderated mingling. Life and dignity: checked, number missed is 12, colored in white. Option for the poor: checked, number missed is 19, colored in red. Call to family: number added is 17, colored in red. Solidarity: checked, number missed is 13, colored in red. Subsidiarity: checked, number missed is 13, colored in red. Care of creation: number added: 2 colored in blue. Design: clear algorithmic comprehension. Life and dignity: number added is 15, colored in red. Option for the poor: checked, number missed is 17, colored in red. Call to family: number added is 2, colored in blue. Solidarity: number added is 0, colored in blue. Subsidiarity: is checked, number missed is 18, colored in red. Care of creation: number added is 2, colored in blue. Design: moderated entry. Life and dignity: checked, number missed is 13, colored in red. Option for the poor: number added is 2, colored in blue. Call to family: number added is 17, colored in red. Solidarity: number added is 13, colored in red. Subsidiarity: checked, number missed is 18, colored in red. Care of creation: number added is 2, colored in blue. Design: no lurking. Life and dignity: number added is 5, colored in white. Option for the poor: number added is 1. Call to family: checked, number missed is 5, colored in white. Solidarity: number added is 9, colored in white. Subsidiarity: number added is 3, colored in blue. Care of creation: number added is 0, colored in blue. Design: notification intentionality. Life and dignity: checked, number missed is 6, colored in white. Option for the poor: checked, number missed is 20, colored in red. Call to family: number added is 10, colored in white. Solidarity: number added is 2, colored in blue. Subsidiarity: number added is 2, colored in blue. Care of creation: checked, number missed is 14, colored in red.}
    \label{fig:missed-added}
\end{figure}

For all but one design, a majority of the participants missed or added at least two principles, suggesting that our participants interpreted the values differently from us. However, for five of the seven designs, a majority of participants got at least one principle correct, indicating some consistency in our and their interpretation of values (see Figure \ref{fig:one-correct}).

\begin{figure}
    \centering
    \includegraphics[width=\linewidth]{figures/One_principle_correct.png}
    \caption{For five out of the seven designs, over half of the participants got at least one principle correct according to our interpretations, indicating some consistency between participant and researcher interpretation of values.}
    \label{fig:one-correct}
    \Description{A comparison chart, comparing the designs with the number of participants that got at least one principle correct. Chats Over Feeds and had 10 participants get one principle correct and Clear Algorithmic Comprehension had 9 participants get one principle correct. These two are colored in orange to indicate that fewer than half of participants got one principle correct for these designs. Friends over followers had 16, moderated mingling had 23, moderated entry had 13, no lurking had 19, and notification intentionality had 20. Each of these are colored in green to indicate that over half of participants got at least one principle correct for these designs.}
\end{figure}

To determine if participants responded similarly to each other, we considered the number of principles per design where over half of participants incorrectly either missed or added that principle (commonly mistaken). We also counted the number of principles that were incorrectly missed or added by fewer than three participants (commonly not mistaken). For six principles across seven designs, over three-fourths (32 out of 42) were either commonly mistaken or commonly not mistaken. Thus we conclude that participants responded similarly to each other.

While we can conclude that while there is a large gap between participant interpretation of values and researcher interpretation of values, there is more inter-participant consistency in interpretation of values. This suggests that there is a big jump between any level of everyday familiarity with values and expert level familiarity with values.

\textit{RQ2} asked, \textit{does this variance vary based on previous familiarity with the value system?} To answer this question, for each participant we counted the number of principles they missed or added for each design compared to the principles we found that design embodied. We plot a participant's ``mistakes'' against their self-reported familiarity with CST in Figure \ref{fig:mistakes}. We would have expected that the number of principles missed and added would both have decreased as the participants' familiarity with CST increased. Instead we see no clear patterns in the data, indicating previous familiarity has little impact.

\begin{figure}
    \centering
    \includegraphics[width=0.49\linewidth]{figures/Total_Principles_Missed_vs_CST_Familiarity.png}
        \includegraphics[width=0.49\linewidth]{figures/Total_Principles_Added_vs_CST_Familiarity.png}
    \caption{For each participant, we plotted the total number (across all seven designs) of principles missed or added by a participant in comparison to the principles we thought each design embodied, plotted against that participant's previous familiarity with CST. We see that familiarity with CST does not cause the number of principles missed or added to decrease.}
    \Description{Two scatter plots are displayed. The first is total principles missed vs. CST familiarity. The x axis goes from 1 to 5 and the number of principles missed goes from 0 to 16. CST familiarity 1 has three points ranging from 9 to 14, 2 has two points ranging from 11 to 13, 3 has 7 points ranging from 4 to 15, 4 has 5 points ranging from 4 to 13, and 5 has one point at 11. A line of best fit goes from 11 down to 10. The second plot is total principles added vs CST familiarity. The x acis goes from 1 to 5 and the y axis goes from 0 to 15. CST familiarity 1 has four points ranging from 2 to 9, familiarity 2 has two points ranging from 5 to 7, familiarity 3 has 5 points ranging from 4 to 14, familiarity 4 has three points ranging from 5 to 8, and familiarity 5 has one point at 10 A line of best fit goes from 6 to 7.}
    \label{fig:mistakes}
    \vspace{-15pt}
\end{figure}

\section{Discussion and Conclusion}
Our participants interpreted the principles of CST similarly to each other, independent of their familiarity with CST. However, their interpretation of the principles differed from ours. We discuss the implications of this discrepancy here.

Were the discrepancies in participant versus researcher interpretations of these values due to the collective wisdom of the participants illuminating blind spots in the researchers' understanding? We are not convinced that this is the case. We agree that the principles added by a majority of the participants were reasonable. However, the most frequently added principle was the \textit{call to family, community, and participation}: given that our designs were for social technologies, it makes sense that participants would frequently add this. When looking again at the principles missed by a majority of the participants, we do not agree that these principles are not embodied by the designs.

Additionally, we the authors have engaged deeply with CST in our research careers. Thus, it is more likely that the participants interpreted the values incorrectly than we did. If so, then this highlights an important learning about the process of designing for particular values: an everyday familiarity of particular values may not be enough to design well for that value. We propose that the most effective embodiment of values in design may come from engaging value experts in the design process. To validate this idea, we propose future work that repeats this study but with CST experts rather than technology experts. It would also be valuable to repeat this study with values from other religious traditions.

One potential limitation is that the participants may have responded similarly to each other because they were influenced by the informational CST handout. However, the purpose of the handout was so that participants unfamiliar with CST would have enough familiarity with the principles to participate meaningfully in the study. Additionally, although our participants had varying levels of familiarity with CST, the majority of our participants came from Catholic or Protestant Christian backgrounds which may have influenced the results. Finally, while our participants had varied technology backgrounds including research and software engineering, we recommend repeating this study with only designers to see if they provide a unique perspective.

\textbf{Conclusion.} We presented the results of an interview study that assessed how technologists with varying levels of familiarity with CST interpreted how the principles of CST are embodied by different social media designs. We found that participants generally responded similarly to each other, even with varying levels of previous familiarity with CST, but their interpretations of which CST principles were embodied by the designs were different from our expert interpretations. We recommend that value experts be engaged in the design process as our results show even an everyday knowledge of certain values may not be enough to effectively translate them into designs.

%%
%% The next two lines define the bibliography style to be used, and
%% the bibliography file.
\bibliographystyle{ACM-Reference-Format}
\bibliography{sample-base}


%%
%% If your work has an appendix, this is the place to put it.
\appendix

\begin{figure}
    \centering
    \includegraphics[width=\linewidth]{figures/Infographic_p1.png}
    \caption{The first page of the Catholic Social Teaching reference handout we gave to the study participants.}
    \label{fig:handout-page-1}
    \Description{The text in this figure says, "Life and dignity of the human person. We believe that every person is precious, that people are more important than things, and that the measure of every institution is whether it threatens or enhances the life and dignity of the human person. Examples: No intentional targeting of civilians in war or terrorist attacks; Prioritizing people over profit; Recognizing the good in every person regardless of their race, class, abilities, etc.; Promotion of authentic, healthy, and respectful dialogue, especially with those with whom we disagree or have different life experiences from." There is an icon of two cartoon people hunched over, one with a cane, indicating they are elderly. Next it says, "Call to family, community, and participation. The person is not only sacred but also social. How we organize our society — in economics and politics, in law and policy — directly affects human dignity and the capacity of individuals to grow in community. Marriage and the family are the central social institutions that must be supported and strengthened, not undermined. We believe people have a right and a duty to participate in society, seeking together the common good and well-being of all, especially the poor and vulnerable. Examples: voting in elections, sharing our skills and talents, volunteering, promiting policies that support and strengthen the family, including paid parental leave, spenidng time with family and friends, attending community events, rejecting excess individualism." There is an icon of two stick figure adults and two stick figure children with a heart, indicating a family. Next it says, "option for the poor and vulnerable.A basic moral test is how our most vulnerable members are faring. In a society marred by deepening divisions between rich and poor, our tradition instructs us to put the needs of the poor and vulnerable first. Examples: Examples: Work to create a society where the needs of the poor and vulnerable are considered first. The vulnerable include not only the poor, but those with less power, such as women, children, the aged, persons with disabilities, immigrants, refugees, minorities, the persecuted, prisoners and victims of human trafficking. Care for those most in need, including the sick and those rejected from society. Reject excess consumerism, which has a downstream negative impact on the poor." There is an icon of a house.
}
\end{figure}

\begin{figure}
    \centering
    \includegraphics[width=\linewidth]{figures/Infographic_p2.png}
    \caption{The second page of the Catholic Social Teaching reference handout we gave to the study participants.}
    \label{fig:handout-page-2}
    \Description{The text in the figure says, "Solidarity. Our love for all our sisters and brothers demands that we promote peace in a world surrounded by violence and conflict. Solidarity calls us to show concern for all people struggling throughout the world and brings us into deeper relationship with the wider human family. Examples: Pursue justice and peace, especially on an international level. Combat the structural causes of poverty, inequality, the lack of work, land and housing, etc. Cultivate a deeper understanding of every person as part of the global human family and our interconnectedness." There is an icon of a globe with cartoon people holding hands around it. Next it says, "Subsidiarity. According to the principle of subsidiarity, decisions should be made at the lowest level possible and the highest level necessary. Subsidiarity is an effort at balancing the many necessary levels of society – and at its best, the principle of subsidiarity navigates the allocation of resources by higher levels of society to support engagement and decision making by the lower levels. Subsidiarity works best when citizens are engaged in their local communities. Examples: Participate in local government. Support local small businesses. Give local governments the most influence. State or federal governments should work to support local governments, but should intervene when necessary, for example, when a broader perspective is required or more high-level management is necessary." There is an icon of three hands in different colors facing each other in a pinwheel shape. Next it says, "Care of God's Creation. We show our respect for the Creator by our stewardship of creation. Care for the earth is not just an Earth Day slogan, it is a requirement of our faith. We are called to protect people and the planet, living our faith in relationship with all of God’s creation. This environmental challenge has fundamental moral and ethical dimensions that cannot be ignored. Examples: Steward natural resources. Reject over-consumption. Spend time in nature to cultivate wonder for the natural world. There is an icon of a tree.
}
\end{figure}

\end{document}