We provide a detailed experimental evaluation showing that DUNIA is able to achieve high performance across a variety of tasks in zero-shot and fine-tuned settings, including land cover classification, crop mapping, and vertical forest structure analysis (cover, height and GEDI waveform retrieval). We leverage diverse datasets to inform the model on horizontal structures--Sentinel- 1 \& 2, and vertical structures with GEDI waveforms.

The following section contains a summary of the experimental setup, with full details in the appendix. Model configuration and optimisation can be found in Appendix \ref{appendix:model_config_and_optim}. For clarity, during the pre-training phase, we set the input image size to $64\times64$ pixels, with 14 channels from stacked Sentinel-1 \& 2 image composites. The embedding dimension is set to 64 (i.e., $D_p=64$). During inference, the input image size can be of any size, but we inferred on inputs of shape $256\times256$ pixels.

\subsection{Experimental Setup}
For our evaluation, we resort to various datasets and tasks. The datasets and experimental details are described next.

\subsubsection{Datasets}

\paragraph{Pre-training Datasets.} \label{seq:pretraining_datasets}

We used Sentinel-2 Level-2A surface reflectance data from Google Earth Engine, including 10 m and 20 m spatial resolution bands with the latter upscaled to 10 m. Two sets of mosaics were created over the entire metropolitan French territory: a single leaf-on season mosaic (April-September 2020) for the pre-trained model, and three four-month mosaics (October 2019-September 2020) for the multitemporal AE. Cloud filtering was applied using the S2 Cloud Probability dataset.

Sentinel-1 data were obtained from the S1A and S1B satellites operating at C-band, collected in Interferometric Wide swath mode with VV and VH polarizations. The data was calibrated, geometrically corrected, and resampled to 10 m resolution. Similar to S2, we created two sets of mosaics with normalized backscattering coefficients.

The Global Ecosystem Dynamics Investigation (GEDI) is a full-waveform LiDAR sensor on the International Space Station (ISS), operational between 51.6$^{\circ}$ N and 51.6$^{\circ}$ S from 2019-2023. We used Level 1B, 2A, and 2B data from April 2019 to December 2021, extracting waveforms, geolocation, height metrics ($\mathcal{W}_{rh}$), canopy cover ($\mathcal{W}_{c}$), and Plant Area Index ($\mathcal{W}_{pai}$). After quality filtering following Fayad et al. \yrcite{Fayad_etal_2024} and seasonal selection, the dataset contained $\approx$ 19 million waveforms; covering less than 1\% of the total surface area of France.

Overall, our pre-training dataset consisted of 836K $64 \times 64$ pixels Sentinel- 1 \& 2 images with, on average, 26 corresponding GEDI waveforms per image.

\paragraph{Evaluation Datasets.}
We evaluated our model on seven downstream tasks using labels from various data products at resolutions matching or lower than our model's output (10 m). 
\emph{PureForest ($PF$)} provides a benchmark dataset of ground truth patches for classifying mono-specific forests in France, featuring high-resolution imagery and annotations for over 135K $50 \times 50$ m (i.e., 5x5 pixels) patches across 13 tree species \cite{gaydon2024pureforest}. \emph{CLC+Backbone ($CLS_+$)} is a pan-European land cover inventory for 2021, utilizing Sentinel-2 time series (2020-2022) and a TempCNN classifier \citep{Pelletier2019} to produce a 10 m raster indicating the dominant land cover among 11 classes. \emph{PASTIS} is a crop mapping dataset by Garnot et al. \yrcite{garnot2021pastis}, covering 18 crop classes and 1 background class with 2433 densely annotated $128 \times 128$ pixels images at 10 m resolution. The \emph{Vertical Structure dataset} assesses model performance in mapping forest heights, canopy cover, plant area index, and waveform retrieval at 10 m resolution, relying on GEDI-derived products as presented earlier. When available, we used the train/val/test split used by the references (i.e., $PF$ and \emph{PASTIS}). For the $CLS_+$ dataset, we used the same split as the unsupervised dataset, which followed a 65/10/25 split. 

\subsubsection{Performance Evaluation}
We evaluated our model's retrieval capacities for both zero-shot dense prediction tasks and its performance in the fine-tuned setting. For both settings, we used the weighted F1 score ($wF1$) for classification tasks, root mean squared error ($rmse$) and Pearson's correlation coefficient ($r$) for regression tasks. To evaluate the similarity between acquired and retrieved/generated waveforms, we used Pearson's correlation coefficient, computed between the time-aligned retrieved/generated waveforms and the acquired waveforms.

For zero-shot classification, we constructed retrieval databases based on the downstream tasks. For vertical structure tasks, outputs from $\mathcal{O}^\mathcal{V}$ served as queries, and we created a single database with $L_2$ normalized waveform embeddings from $\mathcal{O}^\mathcal{W}$ as keys, paired with four target labels: $\mathcal{W}_{rh}$, $\mathcal{W}_c$, $\mathcal{W}_{pai}$, and the complete waveform $(\mathcal{W})$. For horizontal structure tasks, outputs from $\mathcal{O}^\mathcal{H}$ were used as queries, creating a separate database for each target as not all targets were available at all pixels simultaneously. Here, keys are $L_2$ normalized pixel embeddings from $\mathcal{O}^\mathcal{H}$, with targets being a land cover class (i.e., $CLC_+$), crop type (i.e., $PASTIS$), or tree species (i.e., $PF$). For tree species identification, the labels cover a $5 \times 5$ pixel area, so queries and keys are the averaged embeddings over this window. Next, given an input image and its $L_2$ normalized pixel embeddings from $\mathcal{O}^\mathcal{V}$ or $\mathcal{O}^\mathcal{H}$, we retrieve the $k$ nearest neighbors (KNN) for each pixel based on cosine similarity and assign the target class by distance-weighted voting. for the KNN retrieval, keys were obtained from the training split while the queries were obtained from the test split.  

For the low-shot fine-tuning, we froze the entire pre-trained network except for the last two NA layers in each decoder, which were appended with an output head consisting of two sequential $1 \times 1$ convolutional layers. The first layer halves the input channels, and the second projects the reduced representation to the desired output size.

We evaluated zero-shot classification and low-shot fine-tuning in a low-data regime. We define a dataset with a low number of labels based on the type of labels usually available for this dataset. For $CLC_+$ and $PASTIS$, labeled data are available as densely annotated images. For $PF$, $\mathcal{W}$, $\mathcal{W}_{rh}$, $\mathcal{W}_c$ and $\mathcal{W}_{pai}$, labeled data are available as single annotated pixels.

\subsubsection{Competing Models.} 
We compared DUNIA in the fine-tuned setting to two current state-of-the-art Earth Observation FMs: CROMA \cite{fuller2023croma} and AnySat \cite{astruc2024anysat}. CROMA takes as input Sentinel- 1 \& 2 imagery, while Anysat is pre-trained using Sentinel- 1 \& 2 times series as well as very high-resolution imagery. For a fair comparison, we refined the pre-training of both models for 200K steps using our datasets and the training details from the respective papers. For AnySat, we used multi-temporal Sentinel-1 \& 2 mosaics with three time steps and included SPOT images at 1.5 m resolution as an additional input modality during pre-training but fine-tuned using only Sentinel- 1 \& 2. Both models were evaluated on all downstream tasks except waveform generation, as they do not support this task.

\subsection{Results}
Our evaluation shows that embeddings from our proposed framework, combined with simple zero-shot classifiers, often surpass specialized supervised models. Even with minimal labeled data, our model demonstrates strong low-shot performance, rivaling or exceeding state-of-the-art methods.

\begin{table*}[t]
\centering
\fontsize{11pt}{11pt}\selectfont
\begin{tabular}{lllllllllllll}
\toprule
\multicolumn{1}{c}{\textbf{task}} & \multicolumn{2}{c}{\textbf{Mir}} & \multicolumn{2}{c}{\textbf{Lai}} & \multicolumn{2}{c}{\textbf{Ziegen.}} & \multicolumn{2}{c}{\textbf{Cao}} & \multicolumn{2}{c}{\textbf{Alva-Man.}} & \multicolumn{1}{c}{\textbf{avg.}} & \textbf{\begin{tabular}[c]{@{}l@{}}avg.\\ rank\end{tabular}} \\
\multicolumn{1}{c}{\textbf{metrics}} & \multicolumn{1}{c}{\textbf{cor.}} & \multicolumn{1}{c}{\textbf{p-v.}} & \multicolumn{1}{c}{\textbf{cor.}} & \multicolumn{1}{c}{\textbf{p-v.}} & \multicolumn{1}{c}{\textbf{cor.}} & \multicolumn{1}{c}{\textbf{p-v.}} & \multicolumn{1}{c}{\textbf{cor.}} & \multicolumn{1}{c}{\textbf{p-v.}} & \multicolumn{1}{c}{\textbf{cor.}} & \multicolumn{1}{c}{\textbf{p-v.}} &  &  \\ \midrule
\textbf{S-Bleu} & 0.50 & 0.0 & 0.47 & 0.0 & 0.59 & 0.0 & 0.58 & 0.0 & 0.68 & 0.0 & 0.57 & 5.8 \\
\textbf{R-Bleu} & -- & -- & 0.27 & 0.0 & 0.30 & 0.0 & -- & -- & -- & -- & - &  \\
\textbf{S-Meteor} & 0.49 & 0.0 & 0.48 & 0.0 & 0.61 & 0.0 & 0.57 & 0.0 & 0.64 & 0.0 & 0.56 & 6.1 \\
\textbf{R-Meteor} & -- & -- & 0.34 & 0.0 & 0.26 & 0.0 & -- & -- & -- & -- & - &  \\
\textbf{S-Bertscore} & \textbf{0.53} & 0.0 & {\ul 0.80} & 0.0 & \textbf{0.70} & 0.0 & {\ul 0.66} & 0.0 & {\ul0.78} & 0.0 & \textbf{0.69} & \textbf{1.7} \\
\textbf{R-Bertscore} & -- & -- & 0.51 & 0.0 & 0.38 & 0.0 & -- & -- & -- & -- & - &  \\
\textbf{S-Bleurt} & {\ul 0.52} & 0.0 & {\ul 0.80} & 0.0 & 0.60 & 0.0 & \textbf{0.70} & 0.0 & \textbf{0.80} & 0.0 & {\ul 0.68} & {\ul 2.3} \\
\textbf{R-Bleurt} & -- & -- & 0.59 & 0.0 & -0.05 & 0.13 & -- & -- & -- & -- & - &  \\
\textbf{S-Cosine} & 0.51 & 0.0 & 0.69 & 0.0 & {\ul 0.62} & 0.0 & 0.61 & 0.0 & 0.65 & 0.0 & 0.62 & 4.4 \\
\textbf{R-Cosine} & -- & -- & 0.40 & 0.0 & 0.29 & 0.0 & -- & -- & -- & -- & - & \\ \midrule
\textbf{QuestEval} & 0.23 & 0.0 & 0.25 & 0.0 & 0.49 & 0.0 & 0.47 & 0.0 & 0.62 & 0.0 & 0.41 & 9.0 \\
\textbf{LLaMa3} & 0.36 & 0.0 & \textbf{0.84} & 0.0 & {\ul{0.62}} & 0.0 & 0.61 & 0.0 &  0.76 & 0.0 & 0.64 & 3.6 \\
\textbf{our (3b)} & 0.49 & 0.0 & 0.73 & 0.0 & 0.54 & 0.0 & 0.53 & 0.0 & 0.7 & 0.0 & 0.60 & 5.8 \\
\textbf{our (8b)} & 0.48 & 0.0 & 0.73 & 0.0 & 0.52 & 0.0 & 0.53 & 0.0 & 0.7 & 0.0 & 0.59 & 6.3 \\  \bottomrule
\end{tabular}
\caption{Pearson correlation on human evaluation on system output. `R-': reference-based. `S-': source-based.}
\label{tab:sys}
\end{table*}



\begin{table}%[]
\centering
\fontsize{11pt}{11pt}\selectfont
\begin{tabular}{llllll}
\toprule
\multicolumn{1}{c}{\textbf{task}} & \multicolumn{1}{c}{\textbf{Lai}} & \multicolumn{1}{c}{\textbf{Zei.}} & \multicolumn{1}{c}{\textbf{Scia.}} & \textbf{} & \textbf{} \\ 
\multicolumn{1}{c}{\textbf{metrics}} & \multicolumn{1}{c}{\textbf{cor.}} & \multicolumn{1}{c}{\textbf{cor.}} & \multicolumn{1}{c}{\textbf{cor.}} & \textbf{avg.} & \textbf{\begin{tabular}[c]{@{}l@{}}avg.\\ rank\end{tabular}} \\ \midrule
\textbf{S-Bleu} & 0.40 & 0.40 & 0.19* & 0.33 & 7.67 \\
\textbf{S-Meteor} & 0.41 & 0.42 & 0.16* & 0.33 & 7.33 \\
\textbf{S-BertS.} & {\ul0.58} & 0.47 & 0.31 & 0.45 & 3.67 \\
\textbf{S-Bleurt} & 0.45 & {\ul 0.54} & {\ul 0.37} & 0.45 & {\ul 3.33} \\
\textbf{S-Cosine} & 0.56 & 0.52 & 0.3 & {\ul 0.46} & {\ul 3.33} \\ \midrule
\textbf{QuestE.} & 0.27 & 0.35 & 0.06* & 0.23 & 9.00 \\
\textbf{LlaMA3} & \textbf{0.6} & \textbf{0.67} & \textbf{0.51} & \textbf{0.59} & \textbf{1.0} \\
\textbf{Our (3b)} & 0.51 & 0.49 & 0.23* & 0.39 & 4.83 \\
\textbf{Our (8b)} & 0.52 & 0.49 & 0.22* & 0.43 & 4.83 \\ \bottomrule
\end{tabular}
\caption{Pearson correlation on human ratings on reference output. *not significant; we cannot reject the null hypothesis of zero correlation}
\label{tab:ref}
\end{table}


\begin{table*}%[]
\centering
\fontsize{11pt}{11pt}\selectfont
\begin{tabular}{lllllllll}
\toprule
\textbf{task} & \multicolumn{1}{c}{\textbf{ALL}} & \multicolumn{1}{c}{\textbf{sentiment}} & \multicolumn{1}{c}{\textbf{detoxify}} & \multicolumn{1}{c}{\textbf{catchy}} & \multicolumn{1}{c}{\textbf{polite}} & \multicolumn{1}{c}{\textbf{persuasive}} & \multicolumn{1}{c}{\textbf{formal}} & \textbf{\begin{tabular}[c]{@{}l@{}}avg. \\ rank\end{tabular}} \\
\textbf{metrics} & \multicolumn{1}{c}{\textbf{cor.}} & \multicolumn{1}{c}{\textbf{cor.}} & \multicolumn{1}{c}{\textbf{cor.}} & \multicolumn{1}{c}{\textbf{cor.}} & \multicolumn{1}{c}{\textbf{cor.}} & \multicolumn{1}{c}{\textbf{cor.}} & \multicolumn{1}{c}{\textbf{cor.}} &  \\ \midrule
\textbf{S-Bleu} & -0.17 & -0.82 & -0.45 & -0.12* & -0.1* & -0.05 & -0.21 & 8.42 \\
\textbf{R-Bleu} & - & -0.5 & -0.45 &  &  &  &  &  \\
\textbf{S-Meteor} & -0.07* & -0.55 & -0.4 & -0.01* & 0.1* & -0.16 & -0.04* & 7.67 \\
\textbf{R-Meteor} & - & -0.17* & -0.39 & - & - & - & - & - \\
\textbf{S-BertScore} & 0.11 & -0.38 & -0.07* & -0.17* & 0.28 & 0.12 & 0.25 & 6.0 \\
\textbf{R-BertScore} & - & -0.02* & -0.21* & - & - & - & - & - \\
\textbf{S-Bleurt} & 0.29 & 0.05* & 0.45 & 0.06* & 0.29 & 0.23 & 0.46 & 4.2 \\
\textbf{R-Bleurt} & - &  0.21 & 0.38 & - & - & - & - & - \\
\textbf{S-Cosine} & 0.01* & -0.5 & -0.13* & -0.19* & 0.05* & -0.05* & 0.15* & 7.42 \\
\textbf{R-Cosine} & - & -0.11* & -0.16* & - & - & - & - & - \\ \midrule
\textbf{QuestEval} & 0.21 & {\ul{0.29}} & 0.23 & 0.37 & 0.19* & 0.35 & 0.14* & 4.67 \\
\textbf{LlaMA3} & \textbf{0.82} & \textbf{0.80} & \textbf{0.72} & \textbf{0.84} & \textbf{0.84} & \textbf{0.90} & \textbf{0.88} & \textbf{1.00} \\
\textbf{Our (3b)} & 0.47 & -0.11* & 0.37 & 0.61 & 0.53 & 0.54 & 0.66 & 3.5 \\
\textbf{Our (8b)} & {\ul{0.57}} & 0.09* & {\ul 0.49} & {\ul 0.72} & {\ul 0.64} & {\ul 0.62} & {\ul 0.67} & {\ul 2.17} \\ \bottomrule
\end{tabular}
\caption{Pearson correlation on human ratings on our constructed test set. 'R-': reference-based. 'S-': source-based. *not significant; we cannot reject the null hypothesis of zero correlation}
\label{tab:con}
\end{table*}

\section{Results}
We benchmark the different metrics on the different datasets using correlation to human judgement. For content preservation, we show results split on data with system output, reference output and our constructed test set: we show that the data source for evaluation leads to different conclusions on the metrics. In addition, we examine whether the metrics can rank style transfer systems similar to humans. On style strength, we likewise show correlations between human judgment and zero-shot evaluation approaches. When applicable, we summarize results by reporting the average correlation. And the average ranking of the metric per dataset (by ranking which metric obtains the highest correlation to human judgement per dataset). 

\subsection{Content preservation}
\paragraph{How do data sources affect the conclusion on best metric?}
The conclusions about the metrics' performance change radically depending on whether we use system output data, reference output, or our constructed test set. Ideally, a good metric correlates highly with humans on any data source. Ideally, for meta-evaluation, a metric should correlate consistently across all data sources, but the following shows that the correlations indicate different things, and the conclusion on the best metric should be drawn carefully.

Looking at the metrics correlations with humans on the data source with system output (Table~\ref{tab:sys}), we see a relatively high correlation for many of the metrics on many tasks. The overall best metrics are S-BertScore and S-BLEURT (avg+avg rank). We see no notable difference in our method of using the 3B or 8B model as the backbone.

Examining the average correlations based on data with reference output (Table~\ref{tab:ref}), now the zero-shoot prompting with LlaMA3 70B is the best-performing approach ($0.59$ avg). Tied for second place are source-based cosine embedding ($0.46$ avg), BLEURT ($0.45$ avg) and BertScore ($0.45$ avg). Our method follows on a 5. place: here, the 8b version (($0.43$ avg)) shows a bit stronger results than 3b ($0.39$ avg). The fact that the conclusions change, whether looking at reference or system output, confirms the observations made by \citet{scialom-etal-2021-questeval} on simplicity transfer.   

Now consider the results on our test set (Table~\ref{tab:con}): Several metrics show low or no correlation; we even see a significantly negative correlation for some metrics on ALL (BLEU) and for specific subparts of our test set for BLEU, Meteor, BertScore, Cosine. On the other end, LlaMA3 70B is again performing best, showing strong results ($0.82$ in ALL). The runner-up is now our 8B method, with a gap to the 3B version ($0.57$ vs $0.47$ in ALL). Note our method still shows zero correlation for the sentiment task. After, ranks BLEURT ($0.29$), QuestEval ($0.21$), BertScore ($0.11$), Cosine ($0.01$).  

On our test set, we find that some metrics that correlate relatively well on the other datasets, now exhibit low correlation. Hence, with our test set, we can now support the logical reasoning with data evidence: Evaluation of content preservation for style transfer needs to take the style shift into account. This conclusion could not be drawn using the existing data sources: We hypothesise that for the data with system-based output, successful output happens to be very similar to the source sentence and vice versa, and reference-based output might not contain server mistakes as they are gold references. Thus, none of the existing data sources tests the limits of the metrics.  


\paragraph{How do reference-based metrics compare to source-based ones?} Reference-based metrics show a lower correlation than the source-based counterpart for all metrics on both datasets with ratings on references (Table~\ref{tab:sys}). As discussed previously, reference-based metrics for style transfer have the drawback that many different good solutions on a rewrite might exist and not only one similar to a reference.


\paragraph{How well can the metrics rank the performance of style transfer methods?}
We compare the metrics' ability to judge the best style transfer methods w.r.t. the human annotations: Several of the data sources contain samples from different style transfer systems. In order to use metrics to assess the quality of the style transfer system, metrics should correctly find the best-performing system. Hence, we evaluate whether the metrics for content preservation provide the same system ranking as human evaluators. We take the mean of the score for every output on each system and the mean of the human annotations; we compare the systems using the Kendall's Tau correlation. 

We find only the evaluation using the dataset Mir, Lai, and Ziegen to result in significant correlations, probably because of sparsity in a number of system tests (App.~\ref{app:dataset}). Our method (8b) is the only metric providing a perfect ranking of the style transfer system on the Lai data, and Llama3 70B the only one on the Ziegen data. Results in App.~\ref{app:results}. 


\subsection{Style strength results}
%Evaluating style strengths is a challenging task. 
Llama3 70B shows better overall results than our method. However, our method scores higher than Llama3 70B on 2 out of 6 datasets, but it also exhibits zero correlation on one task (Table~\ref{tab:styleresults}).%More work i s needed on evaluating style strengths. 
 
\begin{table}%[]
\fontsize{11pt}{11pt}\selectfont
\begin{tabular}{lccc}
\toprule
\multicolumn{1}{c}{\textbf{}} & \textbf{LlaMA3} & \textbf{Our (3b)} & \textbf{Our (8b)} \\ \midrule
\textbf{Mir} & 0.46 & 0.54 & \textbf{0.57} \\
\textbf{Lai} & \textbf{0.57} & 0.18 & 0.19 \\
\textbf{Ziegen.} & 0.25 & 0.27 & \textbf{0.32} \\
\textbf{Alva-M.} & \textbf{0.59} & 0.03* & 0.02* \\
\textbf{Scialom} & \textbf{0.62} & 0.45 & 0.44 \\
\textbf{\begin{tabular}[c]{@{}l@{}}Our Test\end{tabular}} & \textbf{0.63} & 0.46 & 0.48 \\ \bottomrule
\end{tabular}
\caption{Style strength: Pearson correlation to human ratings. *not significant; we cannot reject the null hypothesis of zero corelation}
\label{tab:styleresults}
\end{table}

\subsection{Ablation}
We conduct several runs of the methods using LLMs with variations in instructions/prompts (App.~\ref{app:method}). We observe that the lower the correlation on a task, the higher the variation between the different runs. For our method, we only observe low variance between the runs.
None of the variations leads to different conclusions of the meta-evaluation. Results in App.~\ref{app:results}.