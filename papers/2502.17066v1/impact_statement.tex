Forests are the third-most important terrestrial carbon sink for climate change mitigation by absorbing about one-third of anthropogenic $\text{CO}_2$ emissions~\cite{friedlingsteinGlobalCarbonBudget2022}. However, these ecosystems are under increasing threat from deforestation, degradation, and climate-induced disturbances such as wildfires, insect outbreaks, and droughts \cite{anderegg2022climate}. Global initiatives, including the Glasgow Declaration and the Sustainable Development Goals\footnote{\url{https://sdgs.un.org/2030agenda}} emphasize the need for accurate forest monitoring to support conservation, restoration, and carbon sequestration projects. Unfortunately, despite these high praises, many other major initiatives have struggled due to a lack of reliable, high-resolution data on forest carbon stocks and how they change over time.

Current forest monitoring systems heavily rely on ground-based inventories, which provide robust statistical estimates of biomass and carbon at national or regional scales but lack the spatial granularity needed for tracking localized carbon gains and losses. Additionally, the largest forested regions—particularly in boreal and tropical areas—remain vastly under-sampled. While satellite data offers global coverage, traditional remote sensing approaches are often constrained by their reliance on extensive labeled datasets, limiting their ability to map forest properties comprehensively.

DUNIA introduces a novel self-supervised learning approach that overcomes these limitations by generating pixel-level embeddings from freely accessible satellite data, including optical and radar imagery. By aligning these embeddings with sparse but highly informative LiDAR waveforms, our approach enables the estimation of multiple forest attributes—including canopy height, fractional cover, land cover, tree species, and vertical structure—without requiring task-specific labels. This method allows for zero-shot and low-shot predictions of key forest variables, paving the way for consistent and scalable monitoring of global ecosystems.

By providing a unified, multimodal representation of forests, DUNIA democratizes access to advanced Earth Observation tools, enabling researchers, policymakers, and conservationists to make informed decisions with minimal reliance on expensive field data collection. This work lays the foundation for high-resolution, large-scale ecosystem monitoring, offering a transformative approach to quantifying forest structure, biomass, and biodiversity in support of climate change mitigation and sustainable land management.