%% bare_conf.tex
%% V1.4b
%% 2015/08/26
%% by Michael Shell
%% See:
%% http://www.michaelshell.org/
%% for current contact information.
%%
%% This is a skeleton file demonstrating the use of IEEEtran.cls
%% (requires IEEEtran.cls version 1.8b or later) with an IEEE
%% conference paper.
%%
%% Support sites:
%% http://www.michaelshell.org/tex/ieeetran/
%% http://www.ctan.org/pkg/ieeetran
%% and
%% http://www.ieee.org/

%%*************************************************************************
%% Legal Notice:
%% This code is offered as-is without any warranty either expressed or
%% implied; without even the implied warranty of MERCHANTABILITY or
%% FITNESS FOR A PARTICULAR PURPOSE! 
%% User assumes all risk.
%% In no event shall the IEEE or any contributor to this code be liable for
%% any damages or losses, including, but not limited to, incidental,
%% consequential, or any other damages, resulting from the use or misuse
%% of any information contained here.
%%
%% All comments are the opinions of their respective authors and are not
%% necessarily endorsed by the IEEE.
%%
%% This work is distributed under the LaTeX Project Public License (LPPL)
%% ( http://www.latex-project.org/ ) version 1.3, and may be freely used,
%% distributed and modified. A copy of the LPPL, version 1.3, is included
%% in the base LaTeX documentation of all distributions of LaTeX released
%% 2003/12/01 or later.
%% Retain all contribution notices and credits.
%% ** Modified files should be clearly indicated as such, including  **
%% ** renaming them and changing author support contact information. **
%%*************************************************************************


% *** Authors should verify (and, if needed, correct) their LaTeX system  ***
% *** with the testflow diagnostic prior to trusting their LaTeX platform ***
% *** with production work. The IEEE's font choices and paper sizes can   ***
% *** trigger bugs that do not appear when using other class files.       ***                          ***
% The testflow support page is at:
% http://www.michaelshell.org/tex/testflow/



\documentclass[conference]{IEEEtran}
% Some Computer Society conferences also require the compsoc mode option,
% but others use the standard conference format.
%
% If IEEEtran.cls has not been installed into the LaTeX system files,
% manually specify the path to it like:
% \documentclass[conference]{../sty/IEEEtran}





% Some very useful LaTeX packages include:
% (uncomment the ones you want to load)


% *** MISC UTILITY PACKAGES ***
%
%\usepackage{ifpdf}
% Heiko Oberdiek's ifpdf.sty is very useful if you need conditional
% compilation based on whether the output is pdf or dvi.
% usage:
% \ifpdf
%   % pdf code
% \else
%   % dvi code
% \fi
% The latest version of ifpdf.sty can be obtained from:
% http://www.ctan.org/pkg/ifpdf
% Also, note that IEEEtran.cls V1.7 and later provides a builtin
% \ifCLASSINFOpdf conditional that works the same way.
% When switching from latex to pdflatex and vice-versa, the compiler may
% have to be run twice to clear warning/error messages.






% *** CITATION PACKAGES ***
%
%\usepackage{cite}
% cite.sty was written by Donald Arseneau
% V1.6 and later of IEEEtran pre-defines the format of the cite.sty package
% \cite{} output to follow that of the IEEE. Loading the cite package will
% result in citation numbers being automatically sorted and properly
% "compressed/ranged". e.g., [1], [9], [2], [7], [5], [6] without using
% cite.sty will become [1], [2], [5]--[7], [9] using cite.sty. cite.sty's
% \cite will automatically add leading space, if needed. Use cite.sty's
% noadjust option (cite.sty V3.8 and later) if you want to turn this off
% such as if a citation ever needs to be enclosed in parenthesis.
% cite.sty is already installed on most LaTeX systems. Be sure and use
% version 5.0 (2009-03-20) and later if using hyperref.sty.
% The latest version can be obtained at:
% http://www.ctan.org/pkg/cite
% The documentation is contained in the cite.sty file itself.



\usepackage{enumerate}
\usepackage{graphicx}
%%%%% NEW MATH DEFINITIONS %%%%%

\usepackage{amsmath,amsfonts,bm}
\usepackage{derivative}
% Mark sections of captions for referring to divisions of figures
\newcommand{\figleft}{{\em (Left)}}
\newcommand{\figcenter}{{\em (Center)}}
\newcommand{\figright}{{\em (Right)}}
\newcommand{\figtop}{{\em (Top)}}
\newcommand{\figbottom}{{\em (Bottom)}}
\newcommand{\captiona}{{\em (a)}}
\newcommand{\captionb}{{\em (b)}}
\newcommand{\captionc}{{\em (c)}}
\newcommand{\captiond}{{\em (d)}}

% Highlight a newly defined term
\newcommand{\newterm}[1]{{\bf #1}}

% Derivative d 
\newcommand{\deriv}{{\mathrm{d}}}

% Figure reference, lower-case.
\def\figref#1{figure~\ref{#1}}
% Figure reference, capital. For start of sentence
\def\Figref#1{Figure~\ref{#1}}
\def\twofigref#1#2{figures \ref{#1} and \ref{#2}}
\def\quadfigref#1#2#3#4{figures \ref{#1}, \ref{#2}, \ref{#3} and \ref{#4}}
% Section reference, lower-case.
\def\secref#1{section~\ref{#1}}
% Section reference, capital.
\def\Secref#1{Section~\ref{#1}}
% Reference to two sections.
\def\twosecrefs#1#2{sections \ref{#1} and \ref{#2}}
% Reference to three sections.
\def\secrefs#1#2#3{sections \ref{#1}, \ref{#2} and \ref{#3}}
% Reference to an equation, lower-case.
\def\eqref#1{equation~\ref{#1}}
% Reference to an equation, upper case
\def\Eqref#1{Equation~\ref{#1}}
% A raw reference to an equation---avoid using if possible
\def\plaineqref#1{\ref{#1}}
% Reference to a chapter, lower-case.
\def\chapref#1{chapter~\ref{#1}}
% Reference to an equation, upper case.
\def\Chapref#1{Chapter~\ref{#1}}
% Reference to a range of chapters
\def\rangechapref#1#2{chapters\ref{#1}--\ref{#2}}
% Reference to an algorithm, lower-case.
\def\algref#1{algorithm~\ref{#1}}
% Reference to an algorithm, upper case.
\def\Algref#1{Algorithm~\ref{#1}}
\def\twoalgref#1#2{algorithms \ref{#1} and \ref{#2}}
\def\Twoalgref#1#2{Algorithms \ref{#1} and \ref{#2}}
% Reference to a part, lower case
\def\partref#1{part~\ref{#1}}
% Reference to a part, upper case
\def\Partref#1{Part~\ref{#1}}
\def\twopartref#1#2{parts \ref{#1} and \ref{#2}}

\def\ceil#1{\lceil #1 \rceil}
\def\floor#1{\lfloor #1 \rfloor}
\def\1{\bm{1}}
\newcommand{\train}{\mathcal{D}}
\newcommand{\valid}{\mathcal{D_{\mathrm{valid}}}}
\newcommand{\test}{\mathcal{D_{\mathrm{test}}}}

\def\eps{{\epsilon}}


% Random variables
\def\reta{{\textnormal{$\eta$}}}
\def\ra{{\textnormal{a}}}
\def\rb{{\textnormal{b}}}
\def\rc{{\textnormal{c}}}
\def\rd{{\textnormal{d}}}
\def\re{{\textnormal{e}}}
\def\rf{{\textnormal{f}}}
\def\rg{{\textnormal{g}}}
\def\rh{{\textnormal{h}}}
\def\ri{{\textnormal{i}}}
\def\rj{{\textnormal{j}}}
\def\rk{{\textnormal{k}}}
\def\rl{{\textnormal{l}}}
% rm is already a command, just don't name any random variables m
\def\rn{{\textnormal{n}}}
\def\ro{{\textnormal{o}}}
\def\rp{{\textnormal{p}}}
\def\rq{{\textnormal{q}}}
\def\rr{{\textnormal{r}}}
\def\rs{{\textnormal{s}}}
\def\rt{{\textnormal{t}}}
\def\ru{{\textnormal{u}}}
\def\rv{{\textnormal{v}}}
\def\rw{{\textnormal{w}}}
\def\rx{{\textnormal{x}}}
\def\ry{{\textnormal{y}}}
\def\rz{{\textnormal{z}}}

% Random vectors
\def\rvepsilon{{\mathbf{\epsilon}}}
\def\rvphi{{\mathbf{\phi}}}
\def\rvtheta{{\mathbf{\theta}}}
\def\rva{{\mathbf{a}}}
\def\rvb{{\mathbf{b}}}
\def\rvc{{\mathbf{c}}}
\def\rvd{{\mathbf{d}}}
\def\rve{{\mathbf{e}}}
\def\rvf{{\mathbf{f}}}
\def\rvg{{\mathbf{g}}}
\def\rvh{{\mathbf{h}}}
\def\rvu{{\mathbf{i}}}
\def\rvj{{\mathbf{j}}}
\def\rvk{{\mathbf{k}}}
\def\rvl{{\mathbf{l}}}
\def\rvm{{\mathbf{m}}}
\def\rvn{{\mathbf{n}}}
\def\rvo{{\mathbf{o}}}
\def\rvp{{\mathbf{p}}}
\def\rvq{{\mathbf{q}}}
\def\rvr{{\mathbf{r}}}
\def\rvs{{\mathbf{s}}}
\def\rvt{{\mathbf{t}}}
\def\rvu{{\mathbf{u}}}
\def\rvv{{\mathbf{v}}}
\def\rvw{{\mathbf{w}}}
\def\rvx{{\mathbf{x}}}
\def\rvy{{\mathbf{y}}}
\def\rvz{{\mathbf{z}}}

% Elements of random vectors
\def\erva{{\textnormal{a}}}
\def\ervb{{\textnormal{b}}}
\def\ervc{{\textnormal{c}}}
\def\ervd{{\textnormal{d}}}
\def\erve{{\textnormal{e}}}
\def\ervf{{\textnormal{f}}}
\def\ervg{{\textnormal{g}}}
\def\ervh{{\textnormal{h}}}
\def\ervi{{\textnormal{i}}}
\def\ervj{{\textnormal{j}}}
\def\ervk{{\textnormal{k}}}
\def\ervl{{\textnormal{l}}}
\def\ervm{{\textnormal{m}}}
\def\ervn{{\textnormal{n}}}
\def\ervo{{\textnormal{o}}}
\def\ervp{{\textnormal{p}}}
\def\ervq{{\textnormal{q}}}
\def\ervr{{\textnormal{r}}}
\def\ervs{{\textnormal{s}}}
\def\ervt{{\textnormal{t}}}
\def\ervu{{\textnormal{u}}}
\def\ervv{{\textnormal{v}}}
\def\ervw{{\textnormal{w}}}
\def\ervx{{\textnormal{x}}}
\def\ervy{{\textnormal{y}}}
\def\ervz{{\textnormal{z}}}

% Random matrices
\def\rmA{{\mathbf{A}}}
\def\rmB{{\mathbf{B}}}
\def\rmC{{\mathbf{C}}}
\def\rmD{{\mathbf{D}}}
\def\rmE{{\mathbf{E}}}
\def\rmF{{\mathbf{F}}}
\def\rmG{{\mathbf{G}}}
\def\rmH{{\mathbf{H}}}
\def\rmI{{\mathbf{I}}}
\def\rmJ{{\mathbf{J}}}
\def\rmK{{\mathbf{K}}}
\def\rmL{{\mathbf{L}}}
\def\rmM{{\mathbf{M}}}
\def\rmN{{\mathbf{N}}}
\def\rmO{{\mathbf{O}}}
\def\rmP{{\mathbf{P}}}
\def\rmQ{{\mathbf{Q}}}
\def\rmR{{\mathbf{R}}}
\def\rmS{{\mathbf{S}}}
\def\rmT{{\mathbf{T}}}
\def\rmU{{\mathbf{U}}}
\def\rmV{{\mathbf{V}}}
\def\rmW{{\mathbf{W}}}
\def\rmX{{\mathbf{X}}}
\def\rmY{{\mathbf{Y}}}
\def\rmZ{{\mathbf{Z}}}

% Elements of random matrices
\def\ermA{{\textnormal{A}}}
\def\ermB{{\textnormal{B}}}
\def\ermC{{\textnormal{C}}}
\def\ermD{{\textnormal{D}}}
\def\ermE{{\textnormal{E}}}
\def\ermF{{\textnormal{F}}}
\def\ermG{{\textnormal{G}}}
\def\ermH{{\textnormal{H}}}
\def\ermI{{\textnormal{I}}}
\def\ermJ{{\textnormal{J}}}
\def\ermK{{\textnormal{K}}}
\def\ermL{{\textnormal{L}}}
\def\ermM{{\textnormal{M}}}
\def\ermN{{\textnormal{N}}}
\def\ermO{{\textnormal{O}}}
\def\ermP{{\textnormal{P}}}
\def\ermQ{{\textnormal{Q}}}
\def\ermR{{\textnormal{R}}}
\def\ermS{{\textnormal{S}}}
\def\ermT{{\textnormal{T}}}
\def\ermU{{\textnormal{U}}}
\def\ermV{{\textnormal{V}}}
\def\ermW{{\textnormal{W}}}
\def\ermX{{\textnormal{X}}}
\def\ermY{{\textnormal{Y}}}
\def\ermZ{{\textnormal{Z}}}

% Vectors
\def\vzero{{\bm{0}}}
\def\vone{{\bm{1}}}
\def\vmu{{\bm{\mu}}}
\def\vtheta{{\bm{\theta}}}
\def\vphi{{\bm{\phi}}}
\def\va{{\bm{a}}}
\def\vb{{\bm{b}}}
\def\vc{{\bm{c}}}
\def\vd{{\bm{d}}}
\def\ve{{\bm{e}}}
\def\vf{{\bm{f}}}
\def\vg{{\bm{g}}}
\def\vh{{\bm{h}}}
\def\vi{{\bm{i}}}
\def\vj{{\bm{j}}}
\def\vk{{\bm{k}}}
\def\vl{{\bm{l}}}
\def\vm{{\bm{m}}}
\def\vn{{\bm{n}}}
\def\vo{{\bm{o}}}
\def\vp{{\bm{p}}}
\def\vq{{\bm{q}}}
\def\vr{{\bm{r}}}
\def\vs{{\bm{s}}}
\def\vt{{\bm{t}}}
\def\vu{{\bm{u}}}
\def\vv{{\bm{v}}}
\def\vw{{\bm{w}}}
\def\vx{{\bm{x}}}
\def\vy{{\bm{y}}}
\def\vz{{\bm{z}}}

% Elements of vectors
\def\evalpha{{\alpha}}
\def\evbeta{{\beta}}
\def\evepsilon{{\epsilon}}
\def\evlambda{{\lambda}}
\def\evomega{{\omega}}
\def\evmu{{\mu}}
\def\evpsi{{\psi}}
\def\evsigma{{\sigma}}
\def\evtheta{{\theta}}
\def\eva{{a}}
\def\evb{{b}}
\def\evc{{c}}
\def\evd{{d}}
\def\eve{{e}}
\def\evf{{f}}
\def\evg{{g}}
\def\evh{{h}}
\def\evi{{i}}
\def\evj{{j}}
\def\evk{{k}}
\def\evl{{l}}
\def\evm{{m}}
\def\evn{{n}}
\def\evo{{o}}
\def\evp{{p}}
\def\evq{{q}}
\def\evr{{r}}
\def\evs{{s}}
\def\evt{{t}}
\def\evu{{u}}
\def\evv{{v}}
\def\evw{{w}}
\def\evx{{x}}
\def\evy{{y}}
\def\evz{{z}}

% Matrix
\def\mA{{\bm{A}}}
\def\mB{{\bm{B}}}
\def\mC{{\bm{C}}}
\def\mD{{\bm{D}}}
\def\mE{{\bm{E}}}
\def\mF{{\bm{F}}}
\def\mG{{\bm{G}}}
\def\mH{{\bm{H}}}
\def\mI{{\bm{I}}}
\def\mJ{{\bm{J}}}
\def\mK{{\bm{K}}}
\def\mL{{\bm{L}}}
\def\mM{{\bm{M}}}
\def\mN{{\bm{N}}}
\def\mO{{\bm{O}}}
\def\mP{{\bm{P}}}
\def\mQ{{\bm{Q}}}
\def\mR{{\bm{R}}}
\def\mS{{\bm{S}}}
\def\mT{{\bm{T}}}
\def\mU{{\bm{U}}}
\def\mV{{\bm{V}}}
\def\mW{{\bm{W}}}
\def\mX{{\bm{X}}}
\def\mY{{\bm{Y}}}
\def\mZ{{\bm{Z}}}
\def\mBeta{{\bm{\beta}}}
\def\mPhi{{\bm{\Phi}}}
\def\mLambda{{\bm{\Lambda}}}
\def\mSigma{{\bm{\Sigma}}}

% Tensor
\DeclareMathAlphabet{\mathsfit}{\encodingdefault}{\sfdefault}{m}{sl}
\SetMathAlphabet{\mathsfit}{bold}{\encodingdefault}{\sfdefault}{bx}{n}
\newcommand{\tens}[1]{\bm{\mathsfit{#1}}}
\def\tA{{\tens{A}}}
\def\tB{{\tens{B}}}
\def\tC{{\tens{C}}}
\def\tD{{\tens{D}}}
\def\tE{{\tens{E}}}
\def\tF{{\tens{F}}}
\def\tG{{\tens{G}}}
\def\tH{{\tens{H}}}
\def\tI{{\tens{I}}}
\def\tJ{{\tens{J}}}
\def\tK{{\tens{K}}}
\def\tL{{\tens{L}}}
\def\tM{{\tens{M}}}
\def\tN{{\tens{N}}}
\def\tO{{\tens{O}}}
\def\tP{{\tens{P}}}
\def\tQ{{\tens{Q}}}
\def\tR{{\tens{R}}}
\def\tS{{\tens{S}}}
\def\tT{{\tens{T}}}
\def\tU{{\tens{U}}}
\def\tV{{\tens{V}}}
\def\tW{{\tens{W}}}
\def\tX{{\tens{X}}}
\def\tY{{\tens{Y}}}
\def\tZ{{\tens{Z}}}


% Graph
\def\gA{{\mathcal{A}}}
\def\gB{{\mathcal{B}}}
\def\gC{{\mathcal{C}}}
\def\gD{{\mathcal{D}}}
\def\gE{{\mathcal{E}}}
\def\gF{{\mathcal{F}}}
\def\gG{{\mathcal{G}}}
\def\gH{{\mathcal{H}}}
\def\gI{{\mathcal{I}}}
\def\gJ{{\mathcal{J}}}
\def\gK{{\mathcal{K}}}
\def\gL{{\mathcal{L}}}
\def\gM{{\mathcal{M}}}
\def\gN{{\mathcal{N}}}
\def\gO{{\mathcal{O}}}
\def\gP{{\mathcal{P}}}
\def\gQ{{\mathcal{Q}}}
\def\gR{{\mathcal{R}}}
\def\gS{{\mathcal{S}}}
\def\gT{{\mathcal{T}}}
\def\gU{{\mathcal{U}}}
\def\gV{{\mathcal{V}}}
\def\gW{{\mathcal{W}}}
\def\gX{{\mathcal{X}}}
\def\gY{{\mathcal{Y}}}
\def\gZ{{\mathcal{Z}}}

% Sets
\def\sA{{\mathbb{A}}}
\def\sB{{\mathbb{B}}}
\def\sC{{\mathbb{C}}}
\def\sD{{\mathbb{D}}}
% Don't use a set called E, because this would be the same as our symbol
% for expectation.
\def\sF{{\mathbb{F}}}
\def\sG{{\mathbb{G}}}
\def\sH{{\mathbb{H}}}
\def\sI{{\mathbb{I}}}
\def\sJ{{\mathbb{J}}}
\def\sK{{\mathbb{K}}}
\def\sL{{\mathbb{L}}}
\def\sM{{\mathbb{M}}}
\def\sN{{\mathbb{N}}}
\def\sO{{\mathbb{O}}}
\def\sP{{\mathbb{P}}}
\def\sQ{{\mathbb{Q}}}
\def\sR{{\mathbb{R}}}
\def\sS{{\mathbb{S}}}
\def\sT{{\mathbb{T}}}
\def\sU{{\mathbb{U}}}
\def\sV{{\mathbb{V}}}
\def\sW{{\mathbb{W}}}
\def\sX{{\mathbb{X}}}
\def\sY{{\mathbb{Y}}}
\def\sZ{{\mathbb{Z}}}

% Entries of a matrix
\def\emLambda{{\Lambda}}
\def\emA{{A}}
\def\emB{{B}}
\def\emC{{C}}
\def\emD{{D}}
\def\emE{{E}}
\def\emF{{F}}
\def\emG{{G}}
\def\emH{{H}}
\def\emI{{I}}
\def\emJ{{J}}
\def\emK{{K}}
\def\emL{{L}}
\def\emM{{M}}
\def\emN{{N}}
\def\emO{{O}}
\def\emP{{P}}
\def\emQ{{Q}}
\def\emR{{R}}
\def\emS{{S}}
\def\emT{{T}}
\def\emU{{U}}
\def\emV{{V}}
\def\emW{{W}}
\def\emX{{X}}
\def\emY{{Y}}
\def\emZ{{Z}}
\def\emSigma{{\Sigma}}

% entries of a tensor
% Same font as tensor, without \bm wrapper
\newcommand{\etens}[1]{\mathsfit{#1}}
\def\etLambda{{\etens{\Lambda}}}
\def\etA{{\etens{A}}}
\def\etB{{\etens{B}}}
\def\etC{{\etens{C}}}
\def\etD{{\etens{D}}}
\def\etE{{\etens{E}}}
\def\etF{{\etens{F}}}
\def\etG{{\etens{G}}}
\def\etH{{\etens{H}}}
\def\etI{{\etens{I}}}
\def\etJ{{\etens{J}}}
\def\etK{{\etens{K}}}
\def\etL{{\etens{L}}}
\def\etM{{\etens{M}}}
\def\etN{{\etens{N}}}
\def\etO{{\etens{O}}}
\def\etP{{\etens{P}}}
\def\etQ{{\etens{Q}}}
\def\etR{{\etens{R}}}
\def\etS{{\etens{S}}}
\def\etT{{\etens{T}}}
\def\etU{{\etens{U}}}
\def\etV{{\etens{V}}}
\def\etW{{\etens{W}}}
\def\etX{{\etens{X}}}
\def\etY{{\etens{Y}}}
\def\etZ{{\etens{Z}}}

% The true underlying data generating distribution
\newcommand{\pdata}{p_{\rm{data}}}
\newcommand{\ptarget}{p_{\rm{target}}}
\newcommand{\pprior}{p_{\rm{prior}}}
\newcommand{\pbase}{p_{\rm{base}}}
\newcommand{\pref}{p_{\rm{ref}}}

% The empirical distribution defined by the training set
\newcommand{\ptrain}{\hat{p}_{\rm{data}}}
\newcommand{\Ptrain}{\hat{P}_{\rm{data}}}
% The model distribution
\newcommand{\pmodel}{p_{\rm{model}}}
\newcommand{\Pmodel}{P_{\rm{model}}}
\newcommand{\ptildemodel}{\tilde{p}_{\rm{model}}}
% Stochastic autoencoder distributions
\newcommand{\pencode}{p_{\rm{encoder}}}
\newcommand{\pdecode}{p_{\rm{decoder}}}
\newcommand{\precons}{p_{\rm{reconstruct}}}

\newcommand{\laplace}{\mathrm{Laplace}} % Laplace distribution

\newcommand{\E}{\mathbb{E}}
\newcommand{\Ls}{\mathcal{L}}
\newcommand{\R}{\mathbb{R}}
\newcommand{\emp}{\tilde{p}}
\newcommand{\lr}{\alpha}
\newcommand{\reg}{\lambda}
\newcommand{\rect}{\mathrm{rectifier}}
\newcommand{\softmax}{\mathrm{softmax}}
\newcommand{\sigmoid}{\sigma}
\newcommand{\softplus}{\zeta}
\newcommand{\KL}{D_{\mathrm{KL}}}
\newcommand{\Var}{\mathrm{Var}}
\newcommand{\standarderror}{\mathrm{SE}}
\newcommand{\Cov}{\mathrm{Cov}}
% Wolfram Mathworld says $L^2$ is for function spaces and $\ell^2$ is for vectors
% But then they seem to use $L^2$ for vectors throughout the site, and so does
% wikipedia.
\newcommand{\normlzero}{L^0}
\newcommand{\normlone}{L^1}
\newcommand{\normltwo}{L^2}
\newcommand{\normlp}{L^p}
\newcommand{\normmax}{L^\infty}

\newcommand{\parents}{Pa} % See usage in notation.tex. Chosen to match Daphne's book.

\DeclareMathOperator*{\argmax}{arg\,max}
\DeclareMathOperator*{\argmin}{arg\,min}

\DeclareMathOperator{\sign}{sign}
\DeclareMathOperator{\Tr}{Tr}
\let\ab\allowbreak


\usepackage{hyperref}
\usepackage{url}
\usepackage{subcaption}
\usepackage{xcolor}
% *** GRAPHICS RELATED PACKAGES ***
%
\ifCLASSINFOpdf
  % \usepackage[pdftex]{graphicx}
  % declare the path(s) where your graphic files are
  % \graphicspath{{../pdf/}{../jpeg/}}
  % and their extensions so you won't have to specify these with
  % every instance of \includegraphics
  % \DeclareGraphicsExtensions{.pdf,.jpeg,.png}
\else
  % or other class option (dvipsone, dvipdf, if not using dvips). graphicx
  % will default to the driver specified in the system graphics.cfg if no
  % driver is specified.
  % \usepackage[dvips]{graphicx}
  % declare the path(s) where your graphic files are
  % \graphicspath{{../eps/}}
  % and their extensions so you won't have to specify these with
  % every instance of \includegraphics
  % \DeclareGraphicsExtensions{.eps}
\fi
% graphicx was written by David Carlisle and Sebastian Rahtz. It is
% required if you want graphics, photos, etc. graphicx.sty is already
% installed on most LaTeX systems. The latest version and documentation
% can be obtained at: 
% http://www.ctan.org/pkg/graphicx
% Another good source of documentation is "Using Imported Graphics in
% LaTeX2e" by Keith Reckdahl which can be found at:
% http://www.ctan.org/pkg/epslatex
%
% latex, and pdflatex in dvi mode, support graphics in encapsulated
% postscript (.eps) format. pdflatex in pdf mode supports graphics
% in .pdf, .jpeg, .png and .mps (metapost) formats. Users should ensure
% that all non-photo figures use a vector format (.eps, .pdf, .mps) and
% not a bitmapped formats (.jpeg, .png). The IEEE frowns on bitmapped formats
% which can result in "jaggedy"/blurry rendering of lines and letters as
% well as large increases in file sizes.
%
% You can find documentation about the pdfTeX application at:
% http://www.tug.org/applications/pdftex





% *** MATH PACKAGES ***
%
%\usepackage{amsmath}
% A popular package from the American Mathematical Society that provides
% many useful and powerful commands for dealing with mathematics.
%
% Note that the amsmath package sets \interdisplaylinepenalty to 10000
% thus preventing page breaks from occurring within multiline equations. Use:
%\interdisplaylinepenalty=2500
% after loading amsmath to restore such page breaks as IEEEtran.cls normally
% does. amsmath.sty is already installed on most LaTeX systems. The latest
% version and documentation can be obtained at:
% http://www.ctan.org/pkg/amsmath





% *** SPECIALIZED LIST PACKAGES ***
%
%\usepackage{algorithmic}
% algorithmic.sty was written by Peter Williams and Rogerio Brito.
% This package provides an algorithmic environment fo describing algorithms.
% You can use the algorithmic environment in-text or within a figure
% environment to provide for a floating algorithm. Do NOT use the algorithm
% floating environment provided by algorithm.sty (by the same authors) or
% algorithm2e.sty (by Christophe Fiorio) as the IEEE does not use dedicated
% algorithm float types and packages that provide these will not provide
% correct IEEE style captions. The latest version and documentation of
% algorithmic.sty can be obtained at:
% http://www.ctan.org/pkg/algorithms
% Also of interest may be the (relatively newer and more customizable)
% algorithmicx.sty package by Szasz Janos:
% http://www.ctan.org/pkg/algorithmicx




% *** ALIGNMENT PACKAGES ***
%
%\usepackage{array}
% Frank Mittelbach's and David Carlisle's array.sty patches and improves
% the standard LaTeX2e array and tabular environments to provide better
% appearance and additional user controls. As the default LaTeX2e table
% generation code is lacking to the point of almost being broken with
% respect to the quality of the end results, all users are strongly
% advised to use an enhanced (at the very least that provided by array.sty)
% set of table tools. array.sty is already installed on most systems. The
% latest version and documentation can be obtained at:
% http://www.ctan.org/pkg/array


% IEEEtran contains the IEEEeqnarray family of commands that can be used to
% generate multiline equations as well as matrices, tables, etc., of high
% quality.




% *** SUBFIGURE PACKAGES ***
%\ifCLASSOPTIONcompsoc
%  \usepackage[caption=false,font=normalsize,labelfont=sf,textfont=sf]{subfig}
%\else
%  \usepackage[caption=false,font=footnotesize]{subfig}
%\fi
% subfig.sty, written by Steven Douglas Cochran, is the modern replacement
% for subfigure.sty, the latter of which is no longer maintained and is
% incompatible with some LaTeX packages including fixltx2e. However,
% subfig.sty requires and automatically loads Axel Sommerfeldt's caption.sty
% which will override IEEEtran.cls' handling of captions and this will result
% in non-IEEE style figure/table captions. To prevent this problem, be sure
% and invoke subfig.sty's "caption=false" package option (available since
% subfig.sty version 1.3, 2005/06/28) as this is will preserve IEEEtran.cls
% handling of captions.
% Note that the Computer Society format requires a larger sans serif font
% than the serif footnote size font used in traditional IEEE formatting
% and thus the need to invoke different subfig.sty package options depending
% on whether compsoc mode has been enabled.
%
% The latest version and documentation of subfig.sty can be obtained at:
% http://www.ctan.org/pkg/subfig




% *** FLOAT PACKAGES ***
%
%\usepackage{fixltx2e}
% fixltx2e, the successor to the earlier fix2col.sty, was written by
% Frank Mittelbach and David Carlisle. This package corrects a few problems
% in the LaTeX2e kernel, the most notable of which is that in current
% LaTeX2e releases, the ordering of single and double column floats is not
% guaranteed to be preserved. Thus, an unpatched LaTeX2e can allow a
% single column figure to be placed prior to an earlier double column
% figure.
% Be aware that LaTeX2e kernels dated 2015 and later have fixltx2e.sty's
% corrections already built into the system in which case a warning will
% be issued if an attempt is made to load fixltx2e.sty as it is no longer
% needed.
% The latest version and documentation can be found at:
% http://www.ctan.org/pkg/fixltx2e


%\usepackage{stfloats}
% stfloats.sty was written by Sigitas Tolusis. This package gives LaTeX2e
% the ability to do double column floats at the bottom of the page as well
% as the top. (e.g., "\begin{figure*}[!b]" is not normally possible in
% LaTeX2e). It also provides a command:
%\fnbelowfloat
% to enable the placement of footnotes below bottom floats (the standard
% LaTeX2e kernel puts them above bottom floats). This is an invasive package
% which rewrites many portions of the LaTeX2e float routines. It may not work
% with other packages that modify the LaTeX2e float routines. The latest
% version and documentation can be obtained at:
% http://www.ctan.org/pkg/stfloats
% Do not use the stfloats baselinefloat ability as the IEEE does not allow
% \baselineskip to stretch. Authors submitting work to the IEEE should note
% that the IEEE rarely uses double column equations and that authors should try
% to avoid such use. Do not be tempted to use the cuted.sty or midfloat.sty
% packages (also by Sigitas Tolusis) as the IEEE does not format its papers in
% such ways.
% Do not attempt to use stfloats with fixltx2e as they are incompatible.
% Instead, use Morten Hogholm'a dblfloatfix which combines the features
% of both fixltx2e and stfloats:
%
% \usepackage{dblfloatfix}
% The latest version can be found at:
% http://www.ctan.org/pkg/dblfloatfix




% *** PDF, URL AND HYPERLINK PACKAGES ***
%
%\usepackage{url}
% url.sty was written by Donald Arseneau. It provides better support for
% handling and breaking URLs. url.sty is already installed on most LaTeX
% systems. The latest version and documentation can be obtained at:
% http://www.ctan.org/pkg/url
% Basically, \url{my_url_here}.




% *** Do not adjust lengths that control margins, column widths, etc. ***
% *** Do not use packages that alter fonts (such as pslatex).         ***
% There should be no need to do such things with IEEEtran.cls V1.6 and later.
% (Unless specifically asked to do so by the journal or conference you plan
% to submit to, of course. )


% correct bad hyphenation here
\hyphenation{op-tical net-works semi-conduc-tor}


\begin{document}
%
% paper title
% Titles are generally capitalized except for words such as a, an, and, as,
% at, but, by, for, in, nor, of, on, or, the, to and up, which are usually
% not capitalized unless they are the first or last word of the title.
% Linebreaks \\ can be used within to get better formatting as desired.
% Do not put math or special symbols in the title.
\title{ Enhancing Edge Neural Networks through \textsc{MoENAS}: Beyond Accuracy with Expert-Mixing Neural Architecture Search }

%\textsc{EFairNAS}: Improving Fairness of Neural Networks for edge Devices using NAS


% author names and affiliations
% use a multiple column layout for up to three different
% affiliations
%\author{\IEEEauthorblockN{Michael Shell}
%\IEEEauthorblockA{School of Electrical and\\Computer Engineering\\
%Georgia Institute of Technology\\
%Atlanta, Georgia 30332--0250\\
%Email: http://www.michaelshell.org/contact.html}
%\and
%\IEEEauthorblockN{Homer Simpson}
%\IEEEauthorblockA{Twentieth Century Fox\\
%Springfield, USA\\
%Email: homer@thesimpsons.com}
%\and
%\IEEEauthorblockN{James Kirk\\ and Montgomery Scott}
%\IEEEauthorblockA{Starfleet Academy\\
%San Francisco, California 96678--2391\\
%Telephone: (800) 555--1212\\
%Fax: (888) 555--1212}}

% conference papers do not typically use \thanks and this command
% is locked out in conference mode. If really needed, such as for
% the acknowledgment of grants, issue a \IEEEoverridecommandlockouts
% after \documentclass

% for over three affiliations, or if they all won't fit within the width
% of the page, use this alternative format:
% 
%\author{\IEEEauthorblockN{Michael Shell\IEEEauthorrefmark{1},
%Homer Simpson\IEEEauthorrefmark{2},
%James Kirk\IEEEauthorrefmark{3}, 
%Montgomery Scott\IEEEauthorrefmark{3} and
%Eldon Tyrell\IEEEauthorrefmark{4}}
%\IEEEauthorblockA{\IEEEauthorrefmark{1}School of Electrical and Computer Engineering\\
%Georgia Institute of Technology,
%Atlanta, Georgia 30332--0250\\ Email: see http://www.michaelshell.org/contact.html}
%\IEEEauthorblockA{\IEEEauthorrefmark{2}Twentieth Century Fox, Springfield, USA\\
%Email: homer@thesimpsons.com}
%\IEEEauthorblockA{\IEEEauthorrefmark{3}Starfleet Academy, San Francisco, California 96678-2391\\
%Telephone: (800) 555--1212, Fax: (888) 555--1212}
%\IEEEauthorblockA{\IEEEauthorrefmark{4}Tyrell Inc., 123 Replicant Street, Los Angeles, California 90210--4321}}




% use for special paper notices
%\IEEEspecialpapernotice{(Invited Paper)}




% make the title area
\maketitle

% As a general rule, do not put math, special symbols or citations
% in the abstract
\begin{abstract}

\end{abstract}

% no keywords




% For peer review papers, you can put extra information on the cover
% page as needed:
% \ifCLASSOPTIONpeerreview
% \begin{center} \bfseries EDICS Category: 3-BBND \end{center}
% \fi
%
% For peerreview papers, this IEEEtran command inserts a page break and
% creates the second title. It will be ignored for other modes.
\IEEEpeerreviewmaketitle



\section{Introduction}
\label{sec:introduction}
\section{Introduction}\label{sec:introduction}
% We introduce a new question: the properties of the quotient code, and its connection to the original code
% Contribution: introduce a new definition
Let $\basefield$ be a finite field, $\blocklength \in \naturalnumbersset$, and let $\variety \subseteq \field$ be a subset
\footnote{As a convention, we use $\tilde{\square}$ to denote properties of the subset, and thus also the subset itself.}
.
We begin by introducing a new definition applicable to any linear code over $\basefield$: the \emph{$\variety$-quotient code}.
We then illustrate this novel definition using Reed-Muller codes, and present a property of $\variety$ which we use to show that $\variety$-quotient Reed-Muller code \emph{inherits its distance and list decoding radius} from the original Reed-Muller code.
Finally, leveraging known results from additive combinatorics and algebraic geometry, we establish as a corollary that this inheritance holds when $\variety$ is a \emph{high-rank variety}.

\paragraph{The Quotient Code}
Let $\gencode$ be a linear code over $\basefield$.
Each codeword of $\gencode$ can be described as a function $\funcdef{\genfunc}{\field}{\basefield}$ that is in the span of the columns of the code's \emph{generator matrix}.
An equivalent way to describe $\gencode$ is using a \emph{parity check matrix}, where a function $\genfunc$ is a codeword if and only if it satisfies the constraints represented by parity-check matrix.
Each such constraint can be thought of as a requirement over a few inputs of $\genfunc$ from $\field$: the requirement that their weighted sum will equal $0$.

The first novel definition we introduce is the definition of the \emph{$\variety$-induced} code:
\begin{definition}[The $\variety$-Induced Code]
    We define the \emph{$\variety$-induced code $\quotientcode$} to be
    the set of all functions $\funcdef{\onvarfunc}{\variety}{\basefield}$
    \footnote{By convention, we use uppercase letters to denote functions with domain $\field$ and lowercase letters to denote functions with domain $\variety$.}
    that satisfy all the constraints \emph{that lie in $\variety$}.
\end{definition}

Let us briefly describe the connection between codewords in $\field$ and $\variety$-induced codewords.
One can easily verify that each original codeword \emph{restricted} to $\variety$ is a valid codeword in the induced code.
\newline
We call an extension of an $\variety$-induced codeword $\funcdef{\onvarfunc}{\variety}{\basefield}$ to valid codeword in the original code (extending its domain to $\field$), a \emph{lift} of $\onvarfunc$.
When each induced codeword has a unique lift, there is a natural 1-to-1 correspondence between the original and induced codeword.
This becomes substantially more interesting for subsets $\variety$ in which induced codewords have \emph{multiple} lifts.
This non-uniqueness weakens the connection between the original codewords and induced codewords, and leads to a richer range of phenomena (and interesting new challenges).

We also note that the other direction is not always true: For a general subset $\variety$, there might be an induced codeword (a valid codeword in the induced code) that \emph{cannot be lifted} to a valid codeword in $\field$.
We are interested to better understand $\quotientcode$ using $\gencode$ and vice-versa, and therefore we introduce a new notion, which is the notion of the \emph{$\variety$-quotient code}:
\begin{definition}[The $\variety$-Quotient Code]
    Let $\gencode$ be a linear code, and let $\quotientcode$ be the $\variety$-induced code of $\gencode$.
    We say $\quotientcode$ is a \emph{$\variety$-quotient code}
    if every quotient codeword $\onvarfunc \in \quotientcode$ has a lift to $\field$.
\end{definition}
In the case described above, we also say that $\variety$ is a \emph{lift-enabler} for $\gencode$ and that the code $\gencode$ is a \emph{covering code} for the code $\quotientcode$.
\newline
The novelty of this definition is that it captures subsets in which \emph{there is} a correspondence between codewords in $\variety$ and in $\field$,
and the correspondence may be \emph{1-to-many}.

\paragraph{Importance of Definition}
This timely definition extends a fundamental and useful concept previously introduced for graphs and complexes—namely, the notion of a \emph{covering graph} or alternatively, the \emph{quotient graph}.
This concept gained an increasing prominence in theoretical computer science, where it was recently employed to construct \emph{high dimensional expanders}~\cite{dikstein2022newhighdimensionalexpanders, yaacov2024sparsehighdimensionalexpanders}
and achieve improved \emph{local testing} results~\cite{gotlib2022listagreementexpansioncoboundary, dikstein2024agreementtheoremshighdimensional, bafna2024characterizingdirectproducttesting},
where the latter also played a crucial role in constructions of PCPs.
Consequently, the study of covering spaces for graphs has found usages in theoretical computer science and specifically in development of PCPs with enhanced properties.
We believe our question, which explores the analogous question for codes, will similarly lead to meaningful applications in theoretical computer science.

In addition to that, the question of \emph{puncturing} of codes has caught much attention recently, in a line of work~\cite{brakensiek2024genericreedsolomoncodesachieve, alrabiah2024randomlypuncturedreedsolomoncodes, brakensiek2024generalizedgmmdspolynomialcodes, brakensiek2024agcodesachievelistdecoding},
followed by the resolution of the GM-MDS conjecture~\cite{DBLP:journals/corr/abs-1803-02523, DBLP:journals/corr/abs-1803-03752}.
Where the question of puncturing is focused exclusively on the case where the lift is \emph{unique},
the study of quotient codes also tackles subsets $\variety \subseteq \field$ where the lift is \emph{not unique}.
Notably, in the unique-lift case there are well-established lower-bounds for the size of $\variety$ such as~\cite[Theorem 1.1]{DBLP:journals/cc/DoronTT22}.
In contrast, the size of $\variety$ in quotient codes may be \emph{much smaller} than its lower-bound in punctured code (for example in Reed-Muller codes), suggesting the potential for new insights and improved results.

\paragraph{Our Question}
Our goal is to answer the following question:
what properties of $\variety$ will imply that the quotient code inherits its distance and list-decoding radius from the original code?

This question is analogous to the study of quotients of expander graphs—just as not all quotients of an expander necessarily preserve expansion,
not all subsets $\variety$ necessarily yield a well-behaved quotient code.
Understanding the conditions under which expansion is preserved has been a fundamental problem in the study of expanders,
and similarly, identifying the conditions under which a quotient code retains key properties of the original code is a central challenge in our work.
Given this parallel, we believe our question may have broader implications for future research in both coding theory and theoretical computer science.

We answer this question in the context of \emph{Reed-Muller codes}.
Notably, our approach does \emph{not only} address the case of where there are multiple lifts,
but also introduces a novel framework for analyzing unique-lift (puncturing) setting when the field size is constant-a scenario that is typically considered more challenging.

\paragraph{Reed-Muller Codes}
Let $\basefield$ be a finite field, and let $\blocklength, \degree$ be integers.
Each codeword in Reed-Muller code $\reedmullercodeex{\basefield}{\field}{\degree}$,
is defined by a polynomial over $\basefield$ in $\blocklength$ variables with total degree $\leq \degree$
\footnote{We focus on the regime where $\degree, \abs{\basefield}$ are considered constants and $\blocklength$ is considered very large.}
.
The message that one wishes to encode is represented in the code as a polynomial $\funcdef{\genpoly}{\field}{\basefield}$, whose coefficients are the different message characters.
The encoding of the message is a vector of the different evaluation of $\genpoly$ over \emph{all} possible points in $\field$.

Alternatively, one can describe Reed-Muller codes using a set of local constraints.
A function $\funcdef{\genfunc}{\field}{\basefield}$ is a polynomial of degree $\leq \degree$
if and only if the (alternating) sum of each possible \emph{cube}, which is a set of points of the form $\set{x + \sum_{i \in S} y_i}_{S \subseteq \sparens{\degree + 1}}$ for $x, y_1,...,y_{\degree+1} \in \field$, equals $0$.
The set of all cubes is \emph{the set of constraints of degree-$\degree$ polynomials}.

Next, we present our notations for the induced Reed-Muller code:
\begin{notation}[The $\variety$-Induced Reed-Muller Code]
    We say a function $\funcdef{\genfunc}{\variety}{\basefield}$ is a \emph{polynomial of degree $\leq \degree$ \emph{in $\variety$}}
    if it satisfies all the constraints of degree-$\degree$ polynomials \emph{that lie in $\variety$}.
    \newline
    We denote the $\variety$-induced Reed-Muller code:
    \[
        \reedmullercodeex{\basefield}{\variety}{\degree} = \set{\funcdef{\onvarpoly}{\variety}{\basefield} \suchthat \onvarpoly \text{ is a polynomial of degree } \leq \degree \text{ in } \variety}
    \]
\end{notation}

\paragraph{Properties of Induced Reed-Muller Codes}
A study of Ziegler and Kazhdan~\cite{kazhdan2018polynomial, kazhdan2019extendingweaklypolynomialfunctions, kazhdan2020propertieshighranksubvarieties}
shows that if $\variety$ is a \emph{high rank variety}
\footnote{Under some conditions we describe later.}
, then $\variety$ is a \emph{lift-enabler} for $\reedmullercodeex{\basefield}{\field}{\degree}$.
In other words, the authors showed that the $\variety$-induced Reed-Muller code is in fact a \emph{$\variety$-quotient Reed-Muller code}.
%More accurately, Ziegler and Kazhdan showed that every polynomial in a high-rank variety
%can be lifted to a polynomial in $\field$ with the same degree
%\footnote{This is a stronger variant of being a quotient code: every polynomial of degree $\degree^\prime \leq \degree$ in $\variety$ is liftable to a polynomial of \emph{the same degree} $\degree^\prime$ in $\field$.}
%.
%\newline
%We call a subset $\variety \subseteq \field$ that has this property for polynomials of degree $\leq \degree$ a \emph{$\degree$-lift-enabler},
We rely on this property of $\variety$ as a black-box.
See Section~\ref{sec:polynomials_in_X} for more details in this regard.

An additional property of $\variety \subseteq \field$ we rely on is the connection between \emph{algebraic structure} and \emph{random behavior (equidistribution)} of polynomials in $\variety$.
\newline
For $\field$, this connection is a well-studied result~\cite{green2007distribution, kaufman2008worst, DBLP:journals/corr/0001L15}.
%which is formally described by the relation of \emph{rank} and \emph{bias} accordingly.
%We clarify that rank is a measure of algebraic structure, where low rank implies being structured,
%and bias is a measure of lack of random behavior, where low bias implies being equidistributed.
It lies in the heart of many results in higher-order Fourier analysis,
and specifically was used in~\cite{bhowmick2014list} to analyze the list decoding radius of Reed-Muller code in $\field$.
\newline
The equivalent of this relation for subsets $\variety \subseteq \field$ was studied in~\cite{lampert2021relative, gowers2022equidistributionhighrankpolynomialsvariables}.
These works captured the measure of algebraic-structure in $\variety$ by a definition called \emph{relative rank},
and captured the lack of random behavior in $\variety$ by a definition called \emph{relative bias}.
We note that for subsets, the definition of algebraic structure of a polynomial in $\variety$ considers the algebraic structure of \emph{all its possible} lifts.
It was shown in~\cite{lampert2021relative} that when $\variety$ is a high-rank variety, high relative rank implies low relative bias
\footnote{
    Note that even though Gowers and Karam~\cite{gowers2022equidistributionhighrankpolynomialsvariables} also acheived a similar relation for a type of subsets,
    the definition of rank they used is slightly different than the standard definition of rank.
    While this difference may seem unharmful at first, it is, to our knowledge, does not allow to do a \emph{regularization} process
    (note that a generalization of this process is the heart of our proof).
}
.
\newline
We use this property as a black box as well.
When a subset $\variety \subseteq \field$ has such property for polynomials of degree $\leq \degree$, we say that it has the \emph{$\degree$-relative rank-bias property}.
See Section~\ref{sec:relative-rank-bias-property} for more details.

\paragraph{Our Results}
Next, let us present our main theorem more concretely.
Our work focuses on the regime where $\degree < \abs{\basefield}$ for prime finite fields $\basefield = \basefield_p$.
Throughout this paper, we always assume these two assumptions.
Denote the \emph{minimum normalized distance of $\reedmullercodeex{\basefield}{\field}{\degree}$} by $\normalizedcodedistanceex{\basefield}{\field}{\degree}$,
shorthand by $\normalizedcodedistance{\basefield}{\degree}$.
We have:
\[
    \normalizedcodedistance{\basefield}{\degree} = 1 - \degree/\abs{\basefield}
\]
Moreover, we define the \emph{list decoding count} of $\reedmullercodeex{\basefield}{\field}{\degree}$ by:
\[
    \listpolycount{\basefield}{\field}{\degree}{\tau} \definedas
    \max_{\funcdef{\genfunc}{\field}{\basefield}}
        {\abs{\set{\genpoly \in \allpolyset{\leq \degree}{\field}{\basefield} \suchthat {\dist{\genpoly, \genfunc} \leq \tau}}}}
\]
Let $\listdecodingradiusex{\basefield}{\field}{\degree}$ be the \emph{list decoding radius} of $\reedmullercodeex{\basefield}{\field}{\degree}$,
which is the maximum $\tau$ for which $\listpolycount{\basefield}{\field}{\degree}{\tau - \epsilon}$ is bounded by a \emph{constant} depending only on $\epsilon, \abs{\basefield}, \degree$.
\newline
In the paper~\cite{bhowmick2014list} it was shown that for constant field size and degree, the list decoding radius \emph{reaches the distance of the code}, as conjectured earlier by~\cite{10.1145/1374376.1374417}
\footnote{Note that it is known that $\listdecodingradiusex{\basefield}{\field}{\degree} \leq \normalizedcodedistance{\basefield}{\degree}$,
    and therefore, in a sense, their result is \emph{optimal in $\field$} assuming $\degree, \abs{\basefield}$ are considered as constants.}
.
We denote the corresponding distance parameter of $\variety \subseteq \field$ by $\normalizedcodedistanceex{\basefield}{\variety}{\degree}$ and $\listdecodingradiusex{\basefield}{\variety}{\degree}$ respectively.

We next present our main theorem, which establishes that the \emph{list decoding radius} of the quotient Reed-Muller code is \emph{at least as good} as the that of the original code:
\begin{theorem*}[List Decoding Quotient Reed-Muller Code]
\footnote{Informal, for formal see Theorem~\ref{thm:list-decoding-RM-in-X}.}
Let $\basefield$ be a finite (prime) field of constant size, let $\degree \in \naturalnumbersset$ be a constant such that $\degree < \abs{\basefield}$,
and let $\blocklength \in \naturalnumbersset$ be an integer.
\newline
Let $\variety \subseteq \field$ be a subset that is a lift-enabler for $\reedmullercodeex{\basefield}{\field}{\degree}$ and has the $\degree$-relative rank-bias property.
\newline
Then, $\reedmullercodeex{\basefield}{\variety}{\degree}$ inherits its \emph{list decoding radius} from $\reedmullercodeex{\basefield}{\field}{\degree}$, i.e:
\[
    \listdecodingradiusex{\basefield}{\variety}{\degree} \geq \listdecodingradiusex{\basefield}{\field}{\degree}
\]
\end{theorem*}

In addition, we also achieve a (simpler) result regarding the \emph{distance} of the quotient Reed-Muller code (Theorem~\ref{thm:distance-of-RM-in-X}):
Under the conditions described above,
$\reedmullercodeex{\basefield}{\variety}{\degree}$ also inherits its \emph{distance} from $\reedmullercodeex{\basefield}{\field}{\degree}$, i.e
$\normalizedcodedistanceex{\basefield}{\variety}{\degree} \geq \normalizedcodedistanceex{\basefield}{\field}{\degree}$
\footnote{Our techniques also show that also the other direction is true, which yields an \emph{equality} in the distance of the two codes.}
.

As a corollary, using results studied in~\cite{kazhdan2018polynomial, kazhdan2019extendingweaklypolynomialfunctions, lampert2021relative} regarding high-rank varieties, we obtain the following:
%TODO: Add this theorem formally in the end.
\begin{corollary*}[List Decoding Quotient Reed-Muller Code: High Rank Variety]
%TODO: [Informal. For formal see...]
    Let $\variety \subseteq \field$ be a \emph{high rank variety},
    that is, $\variety$ is the set of common zeros of a collection of polynomials $\varpolyset = (\varpoly_1,...,\varpoly_{\varietypolycount})$
    that is of \emph{high rank}
    \footnote{We note that the higher the rank of the collection is, the more accurate the greater or equal in the theorem is.}
    \footnote{We also note that for this result some assumptions are needed regarding the field size or the degree of the polynomials in the collection.}
    , i.e. $\variety = \zerofunc{\varpolyset} = \set{x \suchthat \forall i: \varpoly_i(x) = 0}$.
%    \newline
%    Assume that either $\abs{\basefield} > \degree \cdot \varietydeg$ or all polynomials of the collection $\varpolyset$ are of degree $> \degree$.
    \newline
    Then, $\reedmullercodeex{\basefield}{\variety}{\degree}$ inherits its distance parameters from $\reedmullercodeex{\basefield}{\field}{\degree}$, i.e:
    \begin{enumerate}
        \item $\normalizedcodedistanceex{\basefield}{\variety}{\degree} \geq \normalizedcodedistanceex{\basefield}{\field}{\degree}$.
        \item $\listdecodingradiusex{\basefield}{\variety}{\degree} \geq \listdecodingradiusex{\basefield}{\field}{\degree}$.
    \end{enumerate}
\end{corollary*}

\paragraph{Main Technical Challenge}
We achieve these results by combining the two black-box properties of subsets $\variety \subseteq \field$ we presented.
Analysis of the polynomials in $\variety$ raises a new challenge, as previous techniques that were used to analyze low-degree polynomials,
both regarding $\field$~\cite{green2007distribution} and regarding subsets $\variety$~\cite{lampert2021relative},
were focused on maintaining the behavior of polynomials \emph{in the set they work on} ($\field$ and $\variety$ accordingly).
\newline
The novelty of our new technique is that it uses a similar approach to analyze polynomials $\variety$ as commonly used in $\field$,
\emph{while simultaneously maintaining a connection} between polynomials in $\variety$ to polynomials in $\field$.
This connection allows us to deduce that polynomials in $\variety$ behave similarly to polynomials in $\field$.
Informally, given a question regarding a polynomial in $\variety$, our new technique allows us to
associate it with a ``correct'' lift of it, and answer the question \emph{using properties of its lift}.
We emphasize that the correct lift (the one we later choose to use) \emph{depend} on the question,
thus we cannot pick a single canonical lift to \emph{generally} describe each polynomial in $\variety$.

Next we describe this challenge in more detail.
\newline
Analyses of polynomials in $\field$ were commonly based on the structure-randomness connection of polynomials in $\field$.
To use this connection, a procedure introduced by~\cite{green2007distribution}, which is called the \emph{regularization process}, is often used~\cite{kaufman2008worst, tao2011inverse, hatami2011higher, bhattacharyya2013locally, bhattacharyya2013algorithmic, DBLP:journals/corr/0001L15}.
This procedure takes any collection of polynomials, and constructs from it another collection of polynomials that has \emph{equidistriubtion in $\field$}
and ``captures'' all functions ``captured'' by the previous collection.
This notion of ``capturing'' is formulated by a definition called \emph{measurable},
and thus it is required that every function measurable by the old collection will be measurable by the new collection.

We note that the regularization procedure achieves random behavior in $\field$ by requiring the collection to have an \emph{extremely low algebraic structure}:
This implies the new collection has random behavior (equidistributed) as it is a property of $\field$.
The notion of structure is captured by a definition called \emph{rank}, where a polynomial with high rank has extremely low structure.
Additionally, the notion of lack of random behavior is captured by a definition called \emph{bias}, where a polynomial with low bias behaves randomly (equidistributed).
Therefore, the equidistribution is achieved in the regularization process by constructing a collection with high rank, as \emph{in $\field$ high rank implies low bias}.

To generalize these ideas to $\variety$, one must achieve a similar result in $\variety \subseteq \field$:
Given any collection of polynomials, construct a new collection of polynomials that is both equidistributed in $\variety$
and captures every function in $\variety$ that was previously captured.
In our case, however, we must also ensure that the new collection also captures all functions that were previously-captured \emph{in $\field$},
as in our case we wish to use the connection of polynomials in $\variety$ to polynomials in $\field$.
This can be summarized by 3 requirements:
\begin{enumerate}
    \item The polynomials in the new collection will behave random in $\variety$.
    \item Every function that was measurable in $\variety$ by the old collection will be measurable by the new collection in $\variety$.
    \item Every function that was measurable \emph{in $\field$} by the old collection will be measurable by the new collection \emph{in $\field$}.
\end{enumerate}
Alas, this third-requirement is incompatible with the way we achieve the first requirement.
Achieving the first requirement, which is the random behavior in $\variety$, is done by requiring an extremely low algebraic structure \emph{according to relative rank}.
This requires one to consider all possible lifts of polynomials in the collection to avoid any structure.
\newline
More accurately
\footnote{As the polynomials we have here are polynomials in $\field$ we can not discuss their lift.}
, for a polynomial $\funcdef{\genpoly}{\field}{\basefield}$,
we define an \emph{$\variety$-equivalent polynomial for $\genpoly$} to be
a polynomial in $\field$ that coincides with $\genpoly$ on $\variety$ and has the same degree bound as $\genpoly$
\footnote{This is the same as considering all lifts of the polynomial $\restrictfunc{\genpoly}{\variety}$, assuming such lift exist.}
.
Using this definition, the definition of relative rank requires examining all possible \emph{$\variety$-equivalent} polynomials,
and ensuring non of them exhibit structure.
\newline
Typically (in $\field$ for example), avoiding structure is achieved by replacing every structured polynomial by a \emph{small}
set of less-structured polynomials that capture it.
We note that it is \emph{crucial} that the set is small, and from reasons we did not explain here (see definition~\ref{definition:rank}), it is promised because the polynomial we wish to replace is structured.
\newline
For $\variety$, we aim to avoid \emph{all} $\variety$-equivalent polynomials of a polynomial from being structured.
Achieving this, while keeping the collection small,
requires one to replace the polynomial by a set of less-structured polynomials that capture a \emph{structured-$\variety$-equivalent} of it.
Therefore, this process creates a new collection that captures this $\variety$-equivalent polynomial,
but does not necessarily capture the original polynomial!
\newline
In summary, the challenge is that avoiding the structure of \emph{all} the lifts of a polynomial to achieve equidistribution in $\variety$,
without adding too many polynomials, may harm the functions we capture in $\field$.

\paragraph{Introducing New Tools}
We overcome this challenge by presenting a new definition that relaxes the notion of \emph{measurable} we required for functions in $\field$,
which we call \emph{$\variety$-measurable}.
This enables us to describe a relaxed version of the regularization process,
in which we require that every function in $\field$ that was $\variety$-measurable by the old collection will still be $\variety$-measurable by the new collection.
In contrast to the original regularization process, which mandated that functions that were measurable by the old collection will be measurable by the collection,
this relaxed definition only requires such functions to be \emph{$\variety$-measurable} by the new collection.

Even though we no longer need to capture all previously captured functions in $\field$,
it is important that the new relaxed-definition is strict enough to keep the connection between polynomials in $\variety$ and in $\field$.
Therefore, maintaining the $\variety$-measurable functions throughout the regularization process cannot be done trivially,
and this is handled in a procedure we call \emph{the $\variety$-relative regularization process} which is a stronger-version of the regularization process that is used in $\field$.
This new definition and procedure are thoroughly described in Section~\ref{sec:regularization-relative-to-X}.

We note that these new definition and procedure are a novel contribution of this work, and we believe they can
be useful in future research of the quotient Reed-Muller code.


\subsection{Comparison to Related Work}\label{subsec:previous-work}
In~\cite{bhowmick2014list} the authors studied the list decoding radius of Reed Muller codes $\field$.
They proved that, for prime fields, the list decoding radius \emph{reaches the distance of the code}, as conjectured earlier by~\cite{10.1145/1374376.1374417}
\footnote{Note that it is known that $\listdecodingradiusex{\basefield}{\field}{\degree} \leq \normalizedcodedistance{\basefield}{\degree}$,
    and therefore, in a sense, their result is \emph{optimal in $\field$} assuming $\degree, \abs{\basefield}$ are considered as constants.}
\footnote{We also note that their work also apply to the regime $\degree \geq \abs{\basefield}$. }
.
Formally, they showed the following theorem:
\begin{theorem}~\cite[Theorem 1]{bhowmick2014list}
Let $\basefield$ be a prime field.
Let $\epsilon > 0$ and $\degree, \blocklength \in \naturalnumbersset$.
There exists a constant
\footnote{It is important to note that $c$ is \emph{independent of $\blocklength$}.}
$c \definedas c(\abs{\basefield}, \degree, \epsilon)$ such that:
\[
    \listpolycount{\basefield}{\field}{\degree}{\normalizedcodedistance{\basefield}{\degree}- \epsilon} \leq c
\]
\end{theorem}
Our work gives new tools for analyzing polynomials in $\variety \subseteq \field$,
which we later use to follow their line of proof and show equivalent result \emph{in $\variety$}.

We next present related work regarding the study of polynomial codes in subsets $\variety \subseteq \field$.
Before presenting them specifically, we note that our work has a \emph{fundamental difference} than that of the previous study of polynomials in subsets.
%The essence of the difference lies in the way properties are deduced regarding polynomials in $\variety$.
Most works which studied polynomials over subsets $\variety \subseteq \field$ were focused on subsets in which every polynomial has a \emph{unique} lift.
This ensures that there is a 1-to-1 correspondence between polynomials in $\variety$ and in $\field$
and therefore allows easier connection between polynomials in $\variety$ and in $\field$.
%This connection is the most common approach to deduce properties regarding polynomials in $\variety$:
%such properties are inferred from the characteristics of their lifts,
%which are more comprehensively understood, as each lift is a polynomial in $\field$
%they are deduced from the properties of their lifts, which we better understand (as each lift is a polynomial in $\field$).
%\footnote{A possible reason for that focus is their focus on the \emph{extrinsic} definition of a polynomial,
%    that that is more focused on the behaviour of the origianl codewords \emph{restricted} to $\variety$
%    rather than the intrinsic definition, which captures the code \emph{quotient} by $\variety$.}
%.
\newline
We note that our work is non-trivial even in this case:
it extracts the properties of $\field$ that were used in~\cite{bhowmick2014list}, in a way they can be used to analyze quotient Reed-Muller codes.
However, as described earlier, our work addresses an additional substantial challenge which arise when the lift is \emph{not} unique.
Thus our work is only comparable to other works in the unique-lift case, which is the less-challenging case we address.

The first line of work we mention is this regard is the study of hitting sets for low degree polynomials~\cite{6243404, 10.1145/2554797.2554828, 6875485},
and a stronger variant of it which is the study of pseudorandom-generators against low degree polynomials~\
    \cite{10.1145/1060590.1060594, 4389478, 10.1145/1374376.1374455, 4558816,  Cohen2013PseudorandomGF, derksen2022fooling, dwivedi2024optimalpseudorandomgeneratorslowdegree}
Both definitions capture subsets
\footnote{Sometimes this subset is allowed to be a \emph{multiset}.}
$\variety \subseteq \field$ such that every polynomial over $\field$ has a non-negligible distance from $0$ \emph{when restricted to $\variety$}.
This requirement implicitly implies that every low degree polynomial over $\variety$ has at most a \emph{single} lift.

Another line of work worth mentioning in this regard is~\cite{4558818, guruswami2017efficientlylistdecodablepuncturedreedmuller},
which studied \emph{puncturing of Reed-Muller codes}.
This line of work studied the construction of sets $\variety \subseteq \field$,
such that puncturing Reed-Muller codes over $\variety$, that is, taking every original codeword and \emph{restricting} it to $\variety$, will yield a good error-correction code.
To perform their analysis, it was important that every polynomial in $\variety$ has at most a single lift,
and therefore it was an assumption in their work.
%We also note that our work answers similar questions regarding the distance parameter of the constructed code,
%but instead of focusing on a specific construction,
%it describes general properties for subsets that achieve the desired distance parameters (with an explicit construction that has those properties).

%TODO: change the order of references to be all increasing
The papers~\cite{brakensiek2024genericreedsolomoncodesachieve, alrabiah2024randomlypuncturedreedsolomoncodes, brakensiek2024generalizedgmmdspolynomialcodes}
also studied similar questions.
This line of work is followed by the resolution of the \emph{GM-MDS conjecture}, which was proved by~\cite{DBLP:journals/corr/abs-1803-02523, DBLP:journals/corr/abs-1803-03752}.
%After its resolution, it was proved by~\cite{brakensiek2024genericreedsolomoncodesachieve} that over \emph{a field of size $2^{O(\blocklength)}$}, randomly punctured Reed-Solomon codes are combinatorially list-decodable all the way to the list decoding capacity.
%The paper~\cite{brakensiek2024generalizedgmmdspolynomialcodes} generalizes this result and achieves similar results for \emph{all} polynomial codes over exponentially large fields.
%Following this line of work~\cite{alrabiah2024randomlypuncturedreedsolomoncodes} showed that for Reed-Solomon codes, a similar result can be attained to a field of a smaller size.
%Specifically, \emph{linear in $\blocklength$}.
\newline
We note that these works
were focused on the regime where the field is \emph{large}.
More specifically,
they require that the field is \emph{large in respect of $\blocklength$}, i.e $\Omega(\blocklength)$.
We emphasize that our work is focused on \emph{constant fields}.
Moreover, their results were regarding \emph{random} puncturing, while our result makes an \emph{explicit} puncturing.

%Another line of work in this regard studies puncturing of AG codes~\cite{DBLP:journals/corr/abs-1708-01070, guo2021efficientlistdecodingconstantalphabet, brakensiek2024agcodesachievelistdecoding}.
%The newest paper in this line of work~\cite{brakensiek2024agcodesachievelistdecoding} showed that random puncturing of AG codes achieve list decoding capacity over constant fields,
%and as a corollary they showed that AG codes (without puncturing) achieve list decoding capacity over constant fields.
%\newline
%We note that AG codes are a generalization of Reed-Solomon codes,
%where our work is focused on Reed-Muller codes in subsets $\variety \subseteq \field$, which can be thought of as a generalization of AG codes to multiple transcendental variables.
%
We also note that most studies presented above also achieved results regarding the \emph{rate} of the punctured code.
This property of the code can be analyzed naturally when each polynomial over $\variety$ has a \emph{unique} lift, as such assumption implies that the number of polynomials remains the same in $\variety$ as of in $\field$.
As our work does \emph{not} assume such uniqueness, the rate of the code we consider is not analyzed in our work, and thus remained \emph{an open problem}
\footnote{Note that is highly dependent on $\variety$, as additional assumptions are needed to acheive good results in this regard.}
.

\subsection{Proof Overview}\label{subsec:our-work}
In this subsection we present our main technical contribution, which is how we address the challenge of \emph{non-unique lift}.
This is done by introducing the definition of being \emph{$\variety$-measurable}, and by presenting a new tool which is the \emph{relative regularization process}.

To describe them clearly, we first elaborate more on two definitions we described briefly.
\paragraph{Measurable}
Suppose we have a collection of polynomials
\footnote{In this context we think of $c$ as a small (constant for example).}
$\genpolyset[1] = \parens{\genpoly_1,...,\genpoly_c}$ where $\funcdef{\genpoly_i}{\field}{\basefield}$ is a polynomial of degree $\leq \degree$.
We say a function $\funcdef{\genfunc}{\field}{\basefield}$ is \emph{measurable in respect of $\genpolyset[1]$} if it can be determined by the values of $\genpoly_1,...,\genpoly_c$:
if one knows the values of $\genpoly_1(x),...,\genpoly_c(x)$, then she also knows the value of $\genfunc(x)$.
This mathematical-analysis notion, which was first used in a similar context in~\cite{green2007primescontainarbitrarilylong}, is formally defined as follows:
\begin{definition}[Measurable]
    We say a function $\funcdef{\genfunc}{\field}{\basefield}$ is \emph{measurable in respect of $\genpolyset[1] = \parens{\genpoly_1,...,\genpoly_c}$} if
    there exists $\funcdef{\Gamma_{\genfunc}}{\basefield^c}{\basefield}$ such that:
    \[
        \genfunc(x) = \Gamma_{\genfunc}(\genpoly_1(x),...,\genpoly_c(x))
    \]
\end{definition}
This definition can be thought of as the collection $\genpolyset$ ``captures'' the function $\genfunc$
\footnote{Note that this definition also generalizes to every collection of functions.
For now, one can think of the collcetion as a collection of bounded degree polynomials.}
\footnote{One can think of this definition as a generalization of linear span: the collection \emph{spans} the function, where $\Gamma$ is some notion of a span.}
.

Moreover, it would have been useful had this collection of polynomials been ``pseudo-random'', i.e the vector $\parens{\genpoly_1(x),...,\genpoly_c(x)}$ would be equidistributed over a random input $x \in \field$.
This equidistribution would allow us to better understand functions $\genfunc$ that are measurable in respect of $\genpolyset$.

As $\field$ has the rank-bias property, this equidistribution can be achieved by requiring $\genpolyset$ to be a collection of high-rank.
This is a fundamental idea behind the regularization process, first presented in~\cite{green2007distribution}.
Given a collection of polynomials $\genpolyset$, the regularization process constructs another collection $\genpolyset[2]$ of polynomials (with the same degree bound),
such that $\genpolyset[2]$ is a collection of high-rank (and therefore equidistributed) that \emph{refines} $\genpolyset$.
By refine, we mean that every function that was measurable by the first collection $\genpolyset$ is also measurable by the new collection $\genpolyset[2]$ (See definition~\ref{def:semantic-refinement}).

\paragraph{Relative Rank}
We remind the reader that $\variety$-relative rank is a notion that measures the algebraic structure of a polynomial in a subset $\variety \subseteq \field$, by considering the structure of all of its $\variety$-equivalent polynomials.
This notion was presented by~\cite{gowers2022equidistributionhighrankpolynomialsvariables, lampert2021relative}, and is used to achieve equidistribution in $\variety$ assuming $\variety$ has relative rank-bias property.
It is defined as follows:
\begin{definition}[Relative Rank, informal.
See definition~\ref{def:relative-rank-of-polynomial}]
    Let $\variety \subseteq \field$ be a subset,
    let $\degree \in \naturalnumbersset$, and let $\funcdef{\genpoly}{\field}{\basefield}$ be a polynomial of degree $= \degree$.
    The $\variety$-relative rank of $\genpoly$ is defined as follows:
    \[
        \relrank{\variety}{\genpoly} \definedas \min \set{\rank{\genpoly - \relativeremainder{\genpoly}} \suchthat
        \relativeremainder{\genpoly} \in \allpolyset{\leq \degree}{\field}{\basefield}, \restrictfunc{\relativeremainder{\genpoly}}{\variety} \equiv 0}
    \]
\end{definition}
%\paragraph{Concrete Sets Examples}
%In recent years, a few works have studied subsets such that have the relative rank-bias property, specifically in the regime where $\degree < \abs{\basefield}$.
%\newline
%One of them is a paper~\cite{lampert2021relative}, in the line of work that studies subsets $\variety \subseteq \field$ that are \emph{high rank varieties}~\cite{kazhdan2020propertieshighranksubvarieties, kazhdan2018polynomial, kazhdan2019extendingweaklypolynomialfunctions, kazhdan2017extendinglinearquadraticfunctions}.
%An algebraic variety is defined as the zero set of a collection of polynomials $\set{\varpoly_1,...,\varpoly_{\varietypolycount}}$, i.e. $\variety = \set{x \suchthat \varpoly_1(x)=...=\varpoly_{\varietypolycount}(x)=0}$.
%When the collection of polynomials that generate the algebraic variety has a high rank (as a collection),
%we say that the variety is a \emph{high rank variety}
%\footnote{
%    Note that the rank of a collcetion of polynomials is a rather different from the rank of a single polynomial,
%    but as it captures a similar idea, we leave the exact definitions to later (see definition~\ref{definition:factor-rank}).
%}
%.
%TODO: Always put the names of the authors after an cite. Instead saying "they", always call them by name of say "the authors".
%In~\cite[Theorem 1.8]{lampert2021relative}, Lampert and Ziegler have shown that \emph{high rank varieties have the relative rank-bias property}.
%We translate their statement to our point of view in Corollary~\ref{high-rank-variety-has-limited-rank-relative-bias-property}.

%Additionally, in~\cite{gowers2022equidistributionhighrankpolynomialsvariables}, Gowers and Karam studied subsets $\variety \subseteq \field$ of \emph{restricted alphabets}.
%Formally, let $S \subset \basefield$ be a set that restricts the alphabet, and denote $\variety \definedas S^{\blocklength}$.
%In other words, the codewords of the code are polynomials over $\field$, restricted only to inputs from $S^\blocklength$.
%In~\cite[Theorem 1.4]{gowers2022equidistributionhighrankpolynomialsvariables} it was shown that \emph{restricted alphabet subsets have the relative rank-bias property}.
%\newline
%Note that the definition of rank presented by Gowers and Karam is slightly definition than the standard definition of rank.
%While this difference may seem unharmful at first, it is, to our knowledge, incompatible to the \emph{relative regularization} process (which is the heart of our proof).
%A more detailed explanation regarding this difference is given in Note~\ref{note:comparison-to-gowers-rank}.
%
%\paragraph{Formulating the Notion}
%This property is parametrized by the following parameters:
%The field $\basefield$ which we work on, an integer representing the degree of evaluated polynomials $\degree$,
%and a function $\funcdef{\rankbiasfunc}{\realnumbersset^{+}}{\naturalnumbersset}$ that for every $\epsilon$ returns the relative rank needed so that the bias in $\variety$ will be $< \epsilon$.
%In this paper, we think about the first two parameters as constants.
%We think of the last parameter as \emph{how well the subset is regarding the relative rank-bias property}.
%Under this parameterization, we may say a subset $\variety \subseteq \field$ has $(\rankbiasfunc, \basefield, \degree)$-relative rank-bias property.
%For formal definition, see Subsection~\ref{sec:relative-rank-bias-property}.
%\newline
%Next, we note that sometimes, the requirement that a subset $\variety \subseteq \field$ will have the relative rank-bias property to \emph{all} extent is \emph{impossible} to achieve.
%Here, by ``extent'' we mean how small $\epsilon > 0$ can be, such that high relative rank will imply the bias in $\variety$ is $< \epsilon$.
%We see this challenge raised with subsets that are not described by a ``boolean'' property (where a subset either has it or not, such as restricted alphabets),
%but by subsets that are described by a more ``continuous'' property (properties that subsets can have to some extent).
%An example of the latter is the case where $\variety$ is a high rank variety: the rank of the variety is a natural number, thus being a subset of ``high rank'' is not a boolean property; it ``gets better'' the higher the rank of variety is.
%Subsets that have such properties but still have the relative rank-bias property to some extent, are captured by a definition we call \emph{limited relative rank-bias property}.
%We formulate this notion in Subsection~\ref{subsec:limited-relative-rank-bias-property}.

%\subsubsection{Measurable and The Regularization Process}

\subsubsection{\titlevariety-measurable and The \titlevariety-Relative Regularization Process}
In this subsection we discuss the generalization of the regularization process to subsets $\variety \subseteq \field$ using the equivalent of rank-bias relation in $\variety$.
We name this tool \emph{the relative regularization process}.
%This is a tool that enables us to achieve similar equidistribution properties of $\genpolyset$ in $\variety$ as the regularization process achieved in $\field$.
%Note that, as in $\field$ it was achieved using the rank-bias property of $\field$,
%we now require that $\variety$ has the \emph{relative} rank-bias property.
%Additionally, in this case, instead of refining the collection $\genpolyset$ to a collection of high rank, we refine the collection $\genpolyset$ to be a collection of high \emph{$\variety$-relative rank}.
%This requirement adds an extra challenge, thus requiring us to relax our definition of what is a \emph{refinement}, or what is being ``measurable in respect of $\genpolyset$''.

%We note that this is our main tool that enables us to use tools from high-order Fourier analysis to also analyze polynomials in $\variety$.
Practically, we use this tool to show that given a specific question in mind, every $\funcdef{\onvarpoly}{\variety}{\basefield}$ has some polynomial $\funcdef{\genpoly}{\field}{\basefield}$ that behave ``similarly'' in respect to this question.
This allows us to pull properties of $\genpoly$ to better understand $\onvarpoly$.
The perfect candidate for such $\genpoly$ is a \emph{lift} of $\onvarpoly$.
\newline
In order to use $\genpoly$ to deduce properties of $\onvarpoly$, we use the well-studied properties of polynomials in $\field$ to acheive properties of $\genpoly$, and relate these to properties of $\onvarpoly$.
More specifically, assume that $\onvarpoly$ and $\genpoly$ are measurable in respect of a collection of polynomials $\genpolyset$ (each in its domain).
Our strategy is to use $\genpoly$ to deduce properties of $\Gamma_{\genpoly}$, and then use the properties of $\Gamma_{\genpoly}$ to deduce properties of $\onvarpoly$.

Now let us describe the extra challenge.
We start by following the ideas of the regularization process we described for $\field$.
Assuming the collection is not a collection of $\variety$-relative high rank, then there must exist a polynomial in the collection that has low \emph{relative} rank, which we denote by $\genpoly^\star$
\footnote{More precisely, some linear combination of polynomials has low relative rank.}
.
Note that in relative rank, this does not necessarily mean that $\genpoly^\star$ is of low rank, but that there exists another $\variety$-equivalent polynomial that has a low rank.
Thus, even if we remove the low-rank $\variety$-equivalent polynomial and add to the collection all the polynomials that decomposed it,
we cannot require that every function that was measurable by the old collection will still be measurable by the new collection:
even the polynomial we removed is not necessarily measurable by the new collection!
\newline
To allow such regularization process to still apply, we note that while $\genpoly$ might not be measurable in respect of the new collection, a $\variety$-equivalent polynomial of $\genpoly$ \emph{is} measurable with respect of it.
Therefore, we relax the notion of being measurable to being \emph{$\variety$-measurable}.
\newline
We say a function $\genfunc$ is $\variety$-measurable in respect of $\genpolyset$ if it can be determined by the polynomials of $\genpolyset$
\emph{up to a valid $\variety$-remainder}.
We first describe an incomplete definition, then present the challenge that rises with it, and finally present its resolution.
\begin{definition}[$\variety$-measurable, Incomplete Definition]
\footnote{This incomplete definition lacks the requirement of the \emph{validity} of the $\variety$-remainder}
We say a function $\genfunc$ is $\variety$-measurable
if there exists a function $\funcdef{\Gamma}{\basefield^c}{\basefield}$
and a $\variety$-remainder, i.e a function $\funcdef{\relativeremainder{\genfunc}}{\field}{\basefield}$ with $\restrictfunc{\relativeremainder{\genfunc}}{\variety} \equiv 0$
such that:
\[
    \forall a \in \field: \genfunc(a) = \Gamma(\genpoly_1(a),...,\genpoly_c(a)) + \relativeremainder{\genfunc}(a)
\]
\end{definition}

Previous works analyzing polynomials in $\field$ were able to deduce two things from $\genfunc$ being measurable by $\genpolyset$:
that the structure of $\Gamma$ is similar to the structure of $\genfunc$, and that a random input of $\Gamma$ behave similarly to a random input of $\genfunc$.
\newline
To study polynomials in $\variety$, we wish to connect $\onvarpoly$ to $\genpoly$ (which is a lift of $\onvarpoly$).
Thus, we think of $\genfunc = \genpoly$, and require two similar things.
Firstly, we want the structure of $\Gamma$ to be similar to the structure of $\genfunc$ (in this case, $\genpoly$), which we understand as $\genfunc$ is a polynomial in $\field$.
Secondly, we want a random input of $\Gamma$ to behave similarly to a random input of $\onvarpoly$, as $\onvarpoly$ is the polynomial we wish to understand.
The latter is easily achieved using the fact high $\variety$-relative rank implies equidistribution in $\variety$.
The former, however, might be damaged by the remainder as we defined it: we can only learn the structure of $\Gamma$ using the structure of $\genfunc - \relativeremainder{\genfunc} = \Gamma(\genpoly_1,...,\genpoly_c)$.
However, the structure of $\genfunc - \relativeremainder{\genfunc}$ can be very different from the structure of $\genfunc$,
as we did not require any structure of the $\variety$-remainder $\relativeremainder{\genfunc}$.
Thus, we can not deduce the structure of $\Gamma$ via the structure of $\genfunc$ using the incomplete definition described above.

To handle this issue, we add one more requirement regarding the $\variety$-remainder,
which ensures that the structure of $\genfunc$ can be understood via the structure of $\Gamma$:
\[
    \deg(\genfunc - \relativeremainder{\genfunc}) \leq \deg(\genfunc)
\]
If the $\variety$-remainder also has this property, we say it is a \emph{valid} $\variety$-remainder for $\genfunc$.
This can be summarized by the following (complete) definition:
\begin{definition}[$\variety$-measurable]
    We say a function $\genfunc$ is $\variety$-measurable
    if there exists a function $\funcdef{\Gamma}{\basefield^c}{\basefield}$
    and a \emph{valid} $\variety$-remainder, i.e a function $\funcdef{\relativeremainder{\genfunc}}{\field}{\basefield}$
    with $\restrictfunc{\relativeremainder{\genfunc}}{\variety} \equiv 0$ and $\deg(\genfunc - \relativeremainder{\genfunc}) \leq \deg(\genfunc)$
    such that:
    \[
        \forall a \in \field: \genfunc(a) = \Gamma(\genpoly_1(a),...,\genpoly_c(a)) + \relativeremainder{\genfunc}(a)
    \]
\end{definition}
We use this new definition the following way:
Instead of using $\genfunc$ to understand $\Gamma$, we use $\genfunc - \relativeremainder{\genfunc}$ to do so.
We choose $\genfunc - \relativeremainder{\genfunc}$ as it has the same structure as $\genfunc$, but it is ``closer'' to the function $\Gamma$ as $\genfunc - \relativeremainder{\genfunc} = \Gamma(\genpoly_1,...,\genpoly_c)$
\footnote{One can think of this step as "taking the right $\variety$-equivalent" in respect of $\genpolyset$.}
.
Finally, as $\Gamma$ behaves similarly to $\onvarpoly$ for random inputs, we can use $\Gamma$ to deduce properties regarding $\onvarpoly$.
\newline
With this in hand, let us finish describing the relative-regularization process.
The requirement on the validity of the $\variety$-remainder raises a new challenge in the $\variety$-relative regularization process:
we need to somehow control the structure of the $\variety$-remainder, even though this ``error''
is substituted in $\Gamma$ each time we wish to replace a polynomial in our collection.
We address this challenge using a Lemma proved in~\cite{DBLP:journals/corr/0001L15} called the ``faithful composition lemma'',
which allows us to deduce strong properties regarding the structure of $\Gamma$ given the collection was of a high (regular) rank in the first place.
Therefore, we add to each step of the relative-regularization process a (regular) regularization, which ensures $\Gamma$ is very structured.
This strong structure of $\Gamma$ is later used to control the error and deduce it is in the form of a valid $\variety$-remainder.
For the exact details, see Theorem~\ref{theorem:regularization-in-X}.
We conclude this by informally stating our main technical theorem, which is the relative regularization process we just described:
\begin{theorem}[Relative Regularization Process, Informal, See Theorem~\ref{theorem:regularization-in-X}]
Let $\rankval, \degree \in \naturalnumbersset$ be integers that represents a requested rank and degree respectively,
and let $\genpoly_1,...,\genpoly_c$ be a collection of polynomials of degree $\leq \degree$.
Then, there is another collection $\genpoly^{\prime}_1,...,\genpoly^{\prime}_{c^\prime}$ of polynomials of degree $\leq \degree$,
such that:
\begin{enumerate}
    \item Every function that is $\variety$-measurable in respect to the first collection is also $\variety$-measurable in respect to the new collection.
    \item The new collection is of $\variety$-relative rank $\geq \rankval$.
    \item The new collection is of bounded size, i.e $c^\prime \leq C_{\rankfunc, \degree, c}$.
\end{enumerate}
\end{theorem}

\subsubsection{List Decoding in \titlevariety via \titlevariety-Relative Regularization}
In this subsection, we demonstrate how to use the relative regularization process to achieve our main theorem: analysis of the list decoding radius of $\reedmullercodeex{\basefield}{\variety}{\degree}$.

We follow the line of proof of~\cite{bhowmick2014list}, but this time, we are interested in bounding the amount of polynomials \emph{in $\variety$} around every function \emph{in $\variety$}.
More specifically, we wish to show that there is a constant number of words that are $(\normalizedcodedistance{\basefield}{\degree} - \epsilon)$-close to any fixed function in $\variety$.


%
%Let $\funcdef{\onvarfunc}{\variety}{\basefield}$ be a function.
%We apply Lemma~\ref{every-function-can-be-approximated-by-a-few-functions} with $A = \variety$, $B = \basefield$, $F = {\allpolyset{\leq \degree}{\variety}{\basefield}}$.
%We obtain a collection of a few polynomials $\onvarpolyset[3] \subset \allpolyset{\leq \degree}{\variety}{\basefield}$ defined by $\onvarpolyset[3] = (\onvarpoly[3]_1,...,\onvarpoly[3]_c)$
%that approxiamtes every polynomial of degree $\leq \degree$ in $\variety$.
%This reduces the question to only count polynomials in the radius of functions $\funcdef{\onvarfunc}{\variety}{\basefield}$ of the form:
%\[
%    \onvarfunc(x) = \Gamma (\onvarpoly[3]_1(x),...,\onvarpoly[3]_c(x))
%\]
%for some $\funcdef{\Gamma}{\basefield^c}{\basefield}$.
%
%Next, we lift every polynomial in the collection $\onvarpolyset[3]$ and get a collection of polynomials in $\field$, denoted by $\genpolyset[3]$.
%We note that $\genpolyset[3]$, when restricting each of its functions to $\variety$, measures exactly the same functions that were measurable by $\onvarpolyset[3]$.
%
%Now, we apply the $\variety$-relative regularization process to $\genpolyset[3]$.
%This yields a new collection of polynomials $\genpolyset[3]^\prime = \set{\genpoly[3]_1^{\prime},...,\genpoly[3]_{c^\prime}^{\prime}}$ that is equidistributed in $\variety$
%and captures in $\variety$ the same functions that were captured in $\variety$.
%More formally, the latter states that every function that was measurable by $\genpolyset[3]$ restricted to $\variety$, is still measurable by the new collection restricted to $\variety$.
%We note that here, we do not state the ``validity'' (the structure) of the remainder, which is an important promise given by the definition of $\variety$-measurable.
%This will play a crucial role in a second relative regularization process which we will do during the analysis.

%The strategy of the proof is to show that every polynomial in $\variety$ that is $\normalizedcodedistance{\basefield}{\degree}$-close to $\onvarfunc$ is \emph{measurable} by the collection $\genpolyset[3]^\prime$ in $\variety$.
%This is similar to the strategy of the proof of~\cite{bhowmick2014list}, with difference in the domain of the functions.
%This will bound the number of polynomials in the radius of $\onvarfunc$ by the amount of possible functions measurable by a $c^\prime$-sized collection, that is $\abs{\basefield}^{\abs{\basefield}^{c^\prime}}$.

Let $\funcdef{\onvarfunc}{\variety}{\basefield}$ be a received word.
First, we apply a lemma proved in~\cite[Corollary 3.3]{bhowmick2014list}.
The lemma shows that there is a constant-sized (depending on $\epsilon$) collection of polynomials in $\variety$, denoted by $\onvarpolyset[3]$,
such that the distance of $\onvarfunc$ to \emph{any} polynomial can be approximated by the distance of $\onvarfunc$ to some function that is measurable by $\onvarpolyset[3]$ in $\variety$.
This means that instead of bounding the number of polynomials in the radius of $\onvarfunc$, one can bound the number of polynomials in the radius of some function measurable by $\onvarpolyset[3]$.
Thereby, every polynomial-specific measurable function can be thought of as a \emph{low complexity proxy} for $\onvarfunc$ in respect to the polynomial.

Next, we lift each polynomial from $\onvarpolyset[3]$ and apply the \emph{relative regularization process}.
This yields a new collection of polynomials in $\field$ that is constant sized and randomly-behaving (in $\field$).
Denote this new collection by $\genpolyset[3]^{\prime}$
\footnote{We use the same notations as the original proof for clearannce.}
.
Thereby, the question of list decoding is reduced to the following question:
We have a specific constant-sized randomly-behaving collection of polynomials $\genpolyset[3]^\prime = \set{\genpoly[3]_1^{\prime},...,\genpoly[3]^{\prime}_{c^\prime}}$
that was constructed using the function $\onvarfunc$.
We need to bound the amount of polynomials in $\variety$ that are $(\normalizedcodedistance{\degree}{\basefield} - \epsilon / 2)$-close to be measurable by this collection in $\variety$.
Note that the randomly-behaving property was achieved using the \emph{relative rank-bias property} of $\variety$.
Additionally, we note the collection $\genpolyset[3]^\prime$ is a collection of polynomials in $\field$ which we obtained by using the \emph{lift-enabler property} of $\variety$.

From there (and similarly to the analysis in $\field$),
the strategy is to show that polynomials that are that close to being measurable by the randomly-behaving collection $\genpolyset[3]^\prime$, are in fact \emph{measurable} by it.
This will bound the number of such polynomials by the amount of possible functions that are measurable by $\genpolyset[3]^\prime$, which is constant as the collection is of constant size.


Let $\funcdef{\onvarpoly}{\variety}{\basefield}$ be a polynomial of degree $\leq \degree$, and consider a lift of it $\funcdef{\genpoly}{\field}{\basefield}$.
Consider the collection $\genpolyset[3]^\prime \cup \set{\genpoly}$.
Surely, $\genpoly$ is measurable by this collection in $\field$.
Applying $\variety$-relative-regularization to this collection yields a new collection $\genpolyset[3]^{\prime\prime}$ that is equidistributed in $\variety$, such that every $\variety$-measurable function by the old collection is $\variety$-measurable by the new collection.
By a reason we have not explained in this brief explanation, we can ensure this collection is of the form $\genpolyset[3]^{\prime\prime} = \genpolyset[3]^\prime \cup \set{\genpoly[3]_{1}^{\prime\prime},...,\genpoly[3]_{c^{\prime\prime}}^{\prime\prime}}$.

As $\genpoly$ was $\variety$-measurable by $\genpolyset[3]^\prime \cup \set{\genpoly}$ (it was even measurable), $\genpoly$ is $\variety$-measurable by the new collection $\genpolyset[3]^{\prime\prime}$:
That is, $\genpoly$ is measurable by $\genpolyset[3]^{\prime\prime}$ up to a \emph{valid} remainder, denoted by $\relativeremainder{\genpoly}$.
\newline
This means there exists $\funcdef{\Phi}{\basefield^{c^\prime + c^{\prime\prime}}}{\basefield}$ such that:
\[
    \forall a \in \field: \genpoly(a) = \Phi(\genpoly[3]^\prime_1(a),...,\genpoly[3]^\prime_{c^\prime}(a), \genpoly[3]^{\prime\prime}_1(a),...,\genpoly[3]^{\prime\prime}_{c^{\prime\prime}}(a))) + \relativeremainder{\genpoly}(a)
\]

In $\field$, the proof would follow by studying the structure of the function $\Phi$ and use it to induce that $\Phi$ does not depend on its last $c^{\prime\prime}$ variables.
This implies that $\genpoly$ is measurable by the original collection $\genpolyset[3]^{\prime}$ which concludes the proof
\footnote{Note that in $\field$ there is no remainder, so the equation above (with the last $c^{\prime\prime}$ variables as constants) implies measurability by $\genpolyset[3]^{\prime}$.}
.

More accurately, the analysis in $\field$ used the fact that substituting \emph{randomly behaving} polynomials in $\Phi$ yields a structured function
\footnote{In our notations, this structured function is $\genpoly$, which is a polynomial of degree $\leq \degree$ and thus structured}
.
This is used to show that $\Phi$ as a function by itself, with inputs from $\basefield^{c^\prime + c^{\prime\prime}}$, is a very structured function.
The strong structure of $\Phi$, with the fact that $\Phi$ (with inputs substitued to be the functions of $\genpolyset[3]^{\prime\prime}$) is close to the function $\onvarfunc$,
are then combined to deduce that $\Phi$ does not depend on its last $c^{\prime\prime}$ variables.

This paradigm can not be extended effortlessly to our case.
In $\variety$, deducing that $\Phi$ is very structured requires a one-more major step.
This is because we do \emph{not} know any correspondence in the behavior of $\Phi$ (which we want to understand) with the behavior of $\genpoly$ (which we know is structured).
We only know there is a correspondence between $\Phi$ to another function $\genpoly - \relativeremainder{\genpoly}$, which apriori we do not know is structured!

Fortunately, the relative regularization process (Theorem~\ref{theorem:regularization-in-X}) mandates that the remainder of the measurement is \emph{valid}.
That is, if $\genpoly$ was structured (a polynomial of degree $\leq \degree$), then so does $\genpoly - \relativeremainder{\genpoly}$.
This is \emph{crucial}, as it allows us to use the relation between $\Phi$ and $\genpoly - \relativeremainder{\genpoly}$ to deduce that $\Phi$ is structured,
and continue the original outline of the proof of~\cite{bhowmick2014list}.
For more details in this regard, see Theorem~\ref{thm:list-decoding-RM-in-X}.


\subsection{Organization}\label{subsec:organization}
In Section~\ref{sec:preliminaries} we present some basic notations and conventions,
and define the preliminaries we have regarding high-order Fourier analysis in $\field$: polynomials, rank and regularization.
We later generalize each component we presented in Section~\ref{sec:preliminaries} to study polynomials in $\field$ to also study polynomials in $\variety$:
in Section~\ref{sec:polynomials_in_X} we present the set of polynomials in $\variety$ and present the \emph{lift-enabler property};
in Section~\ref{sec:relative-rank-bias-property} we present the \emph{$\variety$-relative rank-bias property};
and in Section~\ref{sec:regularization-relative-to-X} we present the $\variety$-measurable notion, and our main tool, which is the \emph{$\variety$-relative regularization process}.
Next, we present two applications regarding the distance parameters of Reed-Muller codes in $\variety$:
In Section~\ref{sec:radius-of-RM-over-X} we prove the inheritance of the \emph{distance} of the code;
and in Section~\ref{sec:list-decoding-reed-muller-over-X} we prove the inheritance of the \emph{list decoding distance} of it (which is much more involved).

%\subsection{Acknowledgments}\label{subsec:acknowledgments}
%We express our gratitude to Schahar Lovett for his invaluable consultation on this work and for engaging in several insightful discussions.
%Furthermore, the first author recommends the survey~\cite{book} as an excellent resource for readers seeking an introduction to higher-order Fourier analysis.


\section{Related Works}
\label{sec:related_works}
\section{Related Works}
\label{sec:related_works}


\noindent\textbf{Diffusion-based Video Generation. }
The advancement of diffusion models \cite{rombach2022high, ramesh2022hierarchical, zheng2022entropy} has led to significant progress in video generation. Due to the scarcity of high-quality video-text datasets \cite{Blattmann2023, Blattmann2023a}, researchers have adapted existing text-to-image (T2I) models to facilitate text-to-video (T2V) generation. Notable examples include AnimateDiff \cite{Guo2023}, Align your Latents \cite{Blattmann2023a}, PYoCo \cite{ge2023preserve}, and Emu Video \cite{girdhar2023emu}. Further advancements, such as LVDM \cite{he2022latent}, VideoCrafter \cite{chen2023videocrafter1, chen2024videocrafter2}, ModelScope \cite{wang2023modelscope}, LAVIE \cite{wang2023lavie}, and VideoFactory \cite{wang2024videofactory}, have refined these approaches by fine-tuning both spatial and temporal blocks, leveraging T2I models for initialization to improve video quality.
Recently, Sora \cite{brooks2024video} and CogVideoX \cite{yang2024cogvideox} enhance video generation by introducing Transformer-based diffusion backbones \cite{Peebles2023, Ma2024, Yu2024} and utilizing 3D-VAE, unlocking the potential for realistic world simulators. Additionally, SVD \cite{Blattmann2023}, SEINE \cite{chen2023seine}, PixelDance \cite{zeng2024make} and PIA \cite{zhang2024pia} have made significant strides in image-to-video generation, achieving notable improvements in quality and flexibility.
Further, I2VGen-XL \cite{zhang2023i2vgen}, DynamicCrafter \cite{Xing2023}, and Moonshot \cite{zhang2024moonshot} incorporate additional cross-attention layers to strengthen conditional signals during generation.



\noindent\textbf{Controllable Generation.}
Controllable generation has become a central focus in both image \citep{Zhang2023,jiang2024survey, Mou2024, Zheng2023, peng2024controlnext, ye2023ip, wu2024spherediffusion, song2024moma, wu2024ifadapter} and video \citep{gong2024atomovideo, zhang2024moonshot, guo2025sparsectrl, jiang2024videobooth} generation, enabling users to direct the output through various types of control. A wide range of controllable inputs has been explored, including text descriptions, pose \citep{ma2024follow,wang2023disco,hu2024animate,xu2024magicanimate}, audio \citep{tang2023anytoany,tian2024emo,he2024co}, identity representations \citep{chefer2024still,wang2024customvideo,wu2024customcrafter}, trajectory \citep{yin2023dragnuwa,chen2024motion,li2024generative,wu2024motionbooth, namekata2024sg}.


\noindent\textbf{Text-based Camera Control.}
Text-based camera control methods use natural language descriptions to guide camera motion in video generation. AnimateDiff \cite{Guo2023} and SVD \cite{Blattmann2023} fine-tune LoRAs \cite{hu2021lora} for specific camera movements based on text input. 
Image conductor\cite{li2024image} proposed to separate different camera and object motions through camera LoRA weight and object LoRA weight to achieve more precise motion control.
In contrast, MotionMaster \cite{hu2024motionmaster} and Peekaboo \cite{jain2024peekaboo} offer training-free approaches for generating coarse-grained camera motions, though with limited precision. VideoComposer \cite{wang2024videocomposer} adjusts pixel-level motion vectors to provide finer control, but challenges remain in achieving precise camera control.

\noindent\textbf{Trajectory-based Camera Control.}
MotionCtrl \cite{Wang2024Motionctrl}, CameraCtrl \cite{He2024Cameractrl}, and Direct-a-Video \cite{yang2024direct} use camera pose as input to enhance control, while CVD \cite{kuang2024collaborative} extends CameraCtrl for multi-view generation, though still limited by motion complexity. To improve geometric consistency, Pose-guided diffusion \cite{tseng2023consistent}, CamCo \cite{Xu2024}, and CamI2V \cite{zheng2024cami2v} apply epipolar constraints for consistent viewpoints. VD3D \cite{bahmani2024vd3d} introduces a ControlNet\cite{Zhang2023}-like conditioning mechanism with spatiotemporal camera embeddings, enabling more precise control.
CamTrol \cite{hou2024training} offers a training-free approach that renders static point clouds into multi-view frames for video generation. Cavia \cite{xu2024cavia} introduces view-integrated attention mechanisms to improve viewpoint and temporal consistency, while I2VControl-Camera \cite{feng2024i2vcontrol} refines camera movement by employing point trajectories in the camera coordinate system. Despite these advancements, challenges in maintaining camera control and scene-scale consistency remain, which our method seeks to address. It is noted that 4Dim~\cite{watson2024controlling} introduces absolute scale but in  4D novel view synthesis (NVS) of scenes.



%\vspace{-0.3cm}

\section{Methodology}
\label{sec:approach}
\subsection{State-of-the-art Analysis: Observations, Challenges, and Goals}

Our methodology is a strategic response to solve the challenges derived from our evaluation of SOTA methods presented in figure~\ref{fig:sota}, The challenges are summarized as follows:
%
%\vspace{-8pt}
\begin{enumerate}[leftmargin=*]
    \item Model designs and architectures directly affect fairness, robustness, and generalization. Therefore, unlike current techniques that concentrate on either enhancing the data quality or tweaking the training algorithms, we aim to enhance these metrics by modifying the model architecture.
%\vspace{-5pt}
    \item Increasing depth (feature size) and width (number of features) of the model has been shown to improve accuracy. but, at the cost of exacerbating fairness disparities. Therefore, We propose an MoE-based scaling approach that can improve performance without compromising fairness.
%\vspace{-5pt}
    \item Existing design techniques for Edge DNNs neglect fairness, robustness, and generalization resulting in biased, non-robust, and non-generalizable models. Therefore, in our work, We embed fairness, robustness, and generalization as objectives or constraints in the Edge DNNs design process.
\end{enumerate}
%\vspace{-5pt}
%
\subsection{Methodology Overview}

In response to the aforementioned challenges, we introduce MoENAS, a method that utilizes NAS to create edge DNNs that are accurate, fair, robust, and capable of generalization, all while adhering to constraints on model size and efficiency. The MoENAS methodology is structured around three core steps as shown in Figure~\ref{fig:MoENAS_details}: (1) integrating a Switch FNN layer (S-FFN) into an attention-based architecture, (2) search process within the expert mixing space, and (3) expert pruning. 




\begin{figure}[ht]
    \centering \includegraphics[width=\linewidth]{figures/BMoENAS_overview_9.pdf}
    \caption{Overview of MoENAS Methodology. This figure illustrates the three core steps of the MoENAS approach: (1) Replace the standard FFN layer (grey) with a Switch FFN layer (colored rectangle refers to the experts) (2) Execution of a search process within the expert mixing space to identify optimal expert combinations for accuracy, fairness, and robustness; (3) Prune the least used experts to ensure model compactness and efficiency while maintaining high performance.}
    \label{fig:MoENAS_details}
\end{figure}

%Further elaboration on each of these components are provided in sections~\ref{method:switch-fnn},~\ref{method:search-process} and~\ref{method:expert-pruning} of our methodology.

\subsubsection{Switch FFN layer (S-FFN)}
\label{method:switch-fnn}
%
This step involves replacing the standard FFN layer in the attention block with an S-FFN layer. This modified layer, inspired by \cite{fedus2022switch}, utilizes a MoE technique, enabling dynamic routing of features to different specialized experts. 

Traditionally, the structure of the attention block comprises two primary components as shown in Figure~\ref{fig:switch_fnn}.a: an attention layer and a feed-forward network~\cite{vaswani2017attention}. The attention layer's dynamic weight adaptation based on input is a key contributor to its SOTA performance~\cite{khan2022transformers}. However, the static nature of the standard FFN layer, which processes every feature in the same manner, represents a missed opportunity for dynamic adaptation.

Addressing this limitation, we replace the traditional FFN layer with an S-FFN layer. This later selects experts dynamically based on the features using a trained layer called the router. Such an approach ensures that the processing of features is contextually optimized based on their characteristics, thereby extending the dynamic adaptability characteristic of the attention mechanism throughout the model.

\textit{Anatomy of the Switch FFN Layer (S-FFN):} 
The S-FFN layer, as shown in Figure~\ref{fig:switch_fnn}.b, is architecturally composed of two primary components: a router, which directs each feature to a specific expert, and the expert set, where each expert is tailored to process a subset of the feature map based on the router's assignment. This structure allows for an adaptive and efficient processing of features, aligning with our objective to improve fairness, robustness, and generalization in neural network models.
\newline

\begin{figure}[t]
    \centering
\includegraphics[width=\linewidth]{figures/BMoENAS_SFNN.pdf}
    \caption{Detailed view of the Attention Block with Switch FFN Layer. This layer is based on MoE, each feature vector is processed by an FFN layer (expert) among the FFN set (experts set) based on the router selection.}
    \label{fig:switch_fnn}
    %\vspace{-8pt}
\end{figure}


\textit{Why an MoE-based Layer?}\\
In our NAS strategy, we chose an MoE approach because it allows the model to function like a skilled team of experts, where each expert specializes in analyzing certain features. Just as in a team where each member brings their expertise to focus on the area they know best, MoE models dynamically route inputs to the most appropriate expert. This ensures that each part of the data is processed by the expert most capable of handling it, leading to better overall performance. This adaptability of MoE not only enhances accuracy but also addresses fairness by ensuring that the model's decisions are more equitable across different subgroups.

\subsubsection{Search Process}~\\ \label{method:search-process}
%The second step involves a search over the space of the switching architecture where the number of experts in each block is varied, this search is guided using performance predictors that are trained to predict the metrics (accuracy, skin fairness, robustness) based on the model encoding shown in figure~\ref{fig:search_space} \cite{}.
The integration of the S-FFN layer, as previously outlined, presents clear benefits, yet the optimal placement and method of incorporation within the architecture require further exploration. In response, we utilize a NAS strategy to determine the most effective way to include these layers. This approach seeks architectures that achieve a balance between objectives, maximizing accuracy, fairness, and robustness, reducing overfitting, and maintaining a manageable model size.

The proposed search process begins with the selection of an attention-based architecture, for which we chose MobileNetVitV2, based on experimental analyses(see Figure~\ref{fig:sota}). Within this model, all FFN layers are substituted with Switch FFN layers as demonstrated in Figure~\ref{fig:switch_fnn}. The variation in the number of experts within each Switch FFN layer constitutes the core of our search space diversity. Specifically, in the case of MobileNetVitV2, which comprises nine layers, each layer can have from 1 to 8 experts. Consequently, each architecture configuration within this search space is encoded as a vector, succinctly representing the number of experts in each layer as shown in Figure~\ref{fig:search_space}.

\begin{figure*}[ht]
    \centering
    \includegraphics[width=.85\linewidth]{figures/BMoENAS_Search_space.pdf}
    \caption{
Search space of MoENAS: Each candidate architecture is built on the MobileViTv2 macro-architecture, varying only in the number of experts per S-FFN layer, represented as a vector encoding the expert count for each S-FFN layer.}
    \label{fig:search_space}
\end{figure*}

To explore this space, we employed a modified Bayesian Optimization (BO) strategy \cite{white2021bananas}, optimized to concurrently address four objectives: Test Accuracy, Skin Fairness, Robustness, and Generalization, while treating model size as a constraint. This approach utilized a population-based BO mechanism\cite{pelikan2002scalability}. To guide the search, we train surrogate models capable of estimating each objective metric from the vector representation of an architecture as shown in Figure~\ref{fig:search_space}. Here, XGBoost was selected as the surrogate, known for its effectiveness in handling such input vectors \cite{benmeziane2021comprehensive}. 

\subsubsection{Expert Pruning}~\\ \label{method:expert-pruning}
%The third step involves pruning the least utilized expert. After finding one or more suitable architectures, this step focuses on making these architectures more compact. We do this by removing the least important experts gradually until the model reaches a certain level of performance, ensuring it remains efficient.
%
While incorporating the Switch FFN Layer in place of the standard FFN does not impact the computational cost (measured in FLOPs), the model's size increases due to the presence of multiple FFNs within the switch layer. To manage this increase, we have carefully designed our search space to ensure all models remain within a manageable size range of 1M to 3M parameters. To further optimize the model size, we introduce a strategy called expert pruning inspired by \cite{chen2022task}.

Expert pruning (Figure~\ref{fig:MoENAS_overview}), is a method aimed at reducing the model's size by removing the least utilized experts. First, we run the model across all validation images to track the frequency of each expert's use. Then, we rank the experts based on their usage and identify those that are least utilized and therefore are candidates for pruning. 

Once an expert is pruned, the features that would have been routed to it are redirected to their next best option, as determined by the router's probabilities. Following this, the model undergoes evaluation to assess its performance. If the performance remains above a predetermined threshold, the process can be repeated to further refine the model's size and efficiency. This iterative approach ensures that we maintain a balance between model size, efficiency, and performance, making our models efficient for deployment in resource-constrained edge computing environments.

\textit{A question that may arise is why we use post-search pruning when pruned models are already in the search space.} The key reason is that post-search pruning occurs after the model has been fully trained, similar to how knowledge distillation works. Just as a smaller model distilled from a larger one outperforms the same model trained from scratch, training a larger MoE model and then pruning it can yield better results than directly training a smaller model. This post-training adjustment allows us to optimize the model, preserving and enhancing the most effective pathways identified during training, ultimately leading to improved efficiency.


\label{sec:metrics}
\subsection{Evaluation Metrics}
To assess the performance of our models, we focus on several key metrics: 
\begin{itemize}[leftmargin=*]
%\vspace{-8pt}
    \item \textbf{Validation and Test Accuracy:} This metric evaluates the model's accuracy on person classification using COCO~\cite{lin2014microsoft} for the validation and FACET~\cite{lin2014microsoft} for the test.
%\vspace{-4pt}
    \item \textbf{Skin Fairness:} examine the consistency of model accuracy across different skin tones. To calculate it, we adopt a formula that adjusts the Statistical Parity Difference (SPD)\cite{sheng2022larger}, used to calculate unfairness, with a constant $\beta$, to quantify fairness. The formula is given by:
%
     \begin{equation}
    \text{Fairness} = \frac{\beta - \text{SPD}}{\beta}
    \end{equation}
%
    \begin{equation}
    \text{SPD} = \sum_{i=1}^{N} \left| \text{Acc}_{G_i} -
    \text{Acc}_{mino} 
    \right|
    \label{eq:spd}
\end{equation}
%
%     \noindent
% \begin{minipage}{.4\linewidth}
%  \begin{equation}
%     \text{Fairness} = \frac{\beta - \text{SPD}}{\beta}
%     \end{equation}
% \end{minipage}%
% \begin{minipage}{.5\linewidth}
% \begin{equation}
%     \text{SPD} = \sum_{i=1}^{N} \left| \text{Acc}_{G_i} -
%     \text{Acc}_{mino} 
%     \right|
%     \label{eq:spd}
% \end{equation}
% \end{minipage}
%    
    $\beta$ is set to 0.2. $N$ is the number of groups (10 skin tones), and $\text{Acc}_{G_i}$ is the accuracy for group $i$, with $\text{Acc}_{\text{mino}}$ being the accuracy of the minority group (the group with the least number of images).%\vspace{-4pt}
    \item \textbf{Robustness to Light:} Measures the model's performance consistency under poor lighting conditions by calculating the accuracy for the subset of poorly lit images in the test dataset.%\vspace{-4pt}
\item \textbf{Overfitting:} Measured by the difference between validation and test accuracy, it indicates the model's generalization ability. A smaller gap suggests better generalization to unseen data since test and training/validation data come from two different distributions.
\end{itemize}
%\vspace{-5pt}
%\vspace{-0.25cm}

\section{Experimental Results}
\label{sec:search_space}
\section{Experiments}
\label{sec:experiments}
The experiments are designed to address two key research questions.
First, \textbf{RQ1} evaluates whether the average $L_2$-norm of the counterfactual perturbation vectors ($\overline{||\perturb||}$) decreases as the model overfits the data, thereby providing further empirical validation for our hypothesis.
Second, \textbf{RQ2} evaluates the ability of the proposed counterfactual regularized loss, as defined in (\ref{eq:regularized_loss2}), to mitigate overfitting when compared to existing regularization techniques.

% The experiments are designed to address three key research questions. First, \textbf{RQ1} investigates whether the mean perturbation vector norm decreases as the model overfits the data, aiming to further validate our intuition. Second, \textbf{RQ2} explores whether the mean perturbation vector norm can be effectively leveraged as a regularization term during training, offering insights into its potential role in mitigating overfitting. Finally, \textbf{RQ3} examines whether our counterfactual regularizer enables the model to achieve superior performance compared to existing regularization methods, thus highlighting its practical advantage.

\subsection{Experimental Setup}
\textbf{\textit{Datasets, Models, and Tasks.}}
The experiments are conducted on three datasets: \textit{Water Potability}~\cite{kadiwal2020waterpotability}, \textit{Phomene}~\cite{phomene}, and \textit{CIFAR-10}~\cite{krizhevsky2009learning}. For \textit{Water Potability} and \textit{Phomene}, we randomly select $80\%$ of the samples for the training set, and the remaining $20\%$ for the test set, \textit{CIFAR-10} comes already split. Furthermore, we consider the following models: Logistic Regression, Multi-Layer Perceptron (MLP) with 100 and 30 neurons on each hidden layer, and PreactResNet-18~\cite{he2016cvecvv} as a Convolutional Neural Network (CNN) architecture.
We focus on binary classification tasks and leave the extension to multiclass scenarios for future work. However, for datasets that are inherently multiclass, we transform the problem into a binary classification task by selecting two classes, aligning with our assumption.

\smallskip
\noindent\textbf{\textit{Evaluation Measures.}} To characterize the degree of overfitting, we use the test loss, as it serves as a reliable indicator of the model's generalization capability to unseen data. Additionally, we evaluate the predictive performance of each model using the test accuracy.

\smallskip
\noindent\textbf{\textit{Baselines.}} We compare CF-Reg with the following regularization techniques: L1 (``Lasso''), L2 (``Ridge''), and Dropout.

\smallskip
\noindent\textbf{\textit{Configurations.}}
For each model, we adopt specific configurations as follows.
\begin{itemize}
\item \textit{Logistic Regression:} To induce overfitting in the model, we artificially increase the dimensionality of the data beyond the number of training samples by applying a polynomial feature expansion. This approach ensures that the model has enough capacity to overfit the training data, allowing us to analyze the impact of our counterfactual regularizer. The degree of the polynomial is chosen as the smallest degree that makes the number of features greater than the number of data.
\item \textit{Neural Networks (MLP and CNN):} To take advantage of the closed-form solution for computing the optimal perturbation vector as defined in (\ref{eq:opt-delta}), we use a local linear approximation of the neural network models. Hence, given an instance $\inst_i$, we consider the (optimal) counterfactual not with respect to $\model$ but with respect to:
\begin{equation}
\label{eq:taylor}
    \model^{lin}(\inst) = \model(\inst_i) + \nabla_{\inst}\model(\inst_i)(\inst - \inst_i),
\end{equation}
where $\model^{lin}$ represents the first-order Taylor approximation of $\model$ at $\inst_i$.
Note that this step is unnecessary for Logistic Regression, as it is inherently a linear model.
\end{itemize}

\smallskip
\noindent \textbf{\textit{Implementation Details.}} We run all experiments on a machine equipped with an AMD Ryzen 9 7900 12-Core Processor and an NVIDIA GeForce RTX 4090 GPU. Our implementation is based on the PyTorch Lightning framework. We use stochastic gradient descent as the optimizer with a learning rate of $\eta = 0.001$ and no weight decay. We use a batch size of $128$. The training and test steps are conducted for $6000$ epochs on the \textit{Water Potability} and \textit{Phoneme} datasets, while for the \textit{CIFAR-10} dataset, they are performed for $200$ epochs.
Finally, the contribution $w_i^{\varepsilon}$ of each training point $\inst_i$ is uniformly set as $w_i^{\varepsilon} = 1~\forall i\in \{1,\ldots,m\}$.

The source code implementation for our experiments is available at the following GitHub repository: \url{https://anonymous.4open.science/r/COCE-80B4/README.md} 

\subsection{RQ1: Counterfactual Perturbation vs. Overfitting}
To address \textbf{RQ1}, we analyze the relationship between the test loss and the average $L_2$-norm of the counterfactual perturbation vectors ($\overline{||\perturb||}$) over training epochs.

In particular, Figure~\ref{fig:delta_loss_epochs} depicts the evolution of $\overline{||\perturb||}$ alongside the test loss for an MLP trained \textit{without} regularization on the \textit{Water Potability} dataset. 
\begin{figure}[ht]
    \centering
    \includegraphics[width=0.85\linewidth]{img/delta_loss_epochs.png}
    \caption{The average counterfactual perturbation vector $\overline{||\perturb||}$ (left $y$-axis) and the cross-entropy test loss (right $y$-axis) over training epochs ($x$-axis) for an MLP trained on the \textit{Water Potability} dataset \textit{without} regularization.}
    \label{fig:delta_loss_epochs}
\end{figure}

The plot shows a clear trend as the model starts to overfit the data (evidenced by an increase in test loss). 
Notably, $\overline{||\perturb||}$ begins to decrease, which aligns with the hypothesis that the average distance to the optimal counterfactual example gets smaller as the model's decision boundary becomes increasingly adherent to the training data.

It is worth noting that this trend is heavily influenced by the choice of the counterfactual generator model. In particular, the relationship between $\overline{||\perturb||}$ and the degree of overfitting may become even more pronounced when leveraging more accurate counterfactual generators. However, these models often come at the cost of higher computational complexity, and their exploration is left to future work.

Nonetheless, we expect that $\overline{||\perturb||}$ will eventually stabilize at a plateau, as the average $L_2$-norm of the optimal counterfactual perturbations cannot vanish to zero.

% Additionally, the choice of employing the score-based counterfactual explanation framework to generate counterfactuals was driven to promote computational efficiency.

% Future enhancements to the framework may involve adopting models capable of generating more precise counterfactuals. While such approaches may yield to performance improvements, they are likely to come at the cost of increased computational complexity.


\subsection{RQ2: Counterfactual Regularization Performance}
To answer \textbf{RQ2}, we evaluate the effectiveness of the proposed counterfactual regularization (CF-Reg) by comparing its performance against existing baselines: unregularized training loss (No-Reg), L1 regularization (L1-Reg), L2 regularization (L2-Reg), and Dropout.
Specifically, for each model and dataset combination, Table~\ref{tab:regularization_comparison} presents the mean value and standard deviation of test accuracy achieved by each method across 5 random initialization. 

The table illustrates that our regularization technique consistently delivers better results than existing methods across all evaluated scenarios, except for one case -- i.e., Logistic Regression on the \textit{Phomene} dataset. 
However, this setting exhibits an unusual pattern, as the highest model accuracy is achieved without any regularization. Even in this case, CF-Reg still surpasses other regularization baselines.

From the results above, we derive the following key insights. First, CF-Reg proves to be effective across various model types, ranging from simple linear models (Logistic Regression) to deep architectures like MLPs and CNNs, and across diverse datasets, including both tabular and image data. 
Second, CF-Reg's strong performance on the \textit{Water} dataset with Logistic Regression suggests that its benefits may be more pronounced when applied to simpler models. However, the unexpected outcome on the \textit{Phoneme} dataset calls for further investigation into this phenomenon.


\begin{table*}[h!]
    \centering
    \caption{Mean value and standard deviation of test accuracy across 5 random initializations for different model, dataset, and regularization method. The best results are highlighted in \textbf{bold}.}
    \label{tab:regularization_comparison}
    \begin{tabular}{|c|c|c|c|c|c|c|}
        \hline
        \textbf{Model} & \textbf{Dataset} & \textbf{No-Reg} & \textbf{L1-Reg} & \textbf{L2-Reg} & \textbf{Dropout} & \textbf{CF-Reg (ours)} \\ \hline
        Logistic Regression   & \textit{Water}   & $0.6595 \pm 0.0038$   & $0.6729 \pm 0.0056$   & $0.6756 \pm 0.0046$  & N/A    & $\mathbf{0.6918 \pm 0.0036}$                     \\ \hline
        MLP   & \textit{Water}   & $0.6756 \pm 0.0042$   & $0.6790 \pm 0.0058$   & $0.6790 \pm 0.0023$  & $0.6750 \pm 0.0036$    & $\mathbf{0.6802 \pm 0.0046}$                    \\ \hline
%        MLP   & \textit{Adult}   & $0.8404 \pm 0.0010$   & $\mathbf{0.8495 \pm 0.0007}$   & $0.8489 \pm 0.0014$  & $\mathbf{0.8495 \pm 0.0016}$     & $0.8449 \pm 0.0019$                    \\ \hline
        Logistic Regression   & \textit{Phomene}   & $\mathbf{0.8148 \pm 0.0020}$   & $0.8041 \pm 0.0028$   & $0.7835 \pm 0.0176$  & N/A    & $0.8098 \pm 0.0055$                     \\ \hline
        MLP   & \textit{Phomene}   & $0.8677 \pm 0.0033$   & $0.8374 \pm 0.0080$   & $0.8673 \pm 0.0045$  & $0.8672 \pm 0.0042$     & $\mathbf{0.8718 \pm 0.0040}$                    \\ \hline
        CNN   & \textit{CIFAR-10} & $0.6670 \pm 0.0233$   & $0.6229 \pm 0.0850$   & $0.7348 \pm 0.0365$   & N/A    & $\mathbf{0.7427 \pm 0.0571}$                     \\ \hline
    \end{tabular}
\end{table*}

\begin{table*}[htb!]
    \centering
    \caption{Hyperparameter configurations utilized for the generation of Table \ref{tab:regularization_comparison}. For our regularization the hyperparameters are reported as $\mathbf{\alpha/\beta}$.}
    \label{tab:performance_parameters}
    \begin{tabular}{|c|c|c|c|c|c|c|}
        \hline
        \textbf{Model} & \textbf{Dataset} & \textbf{No-Reg} & \textbf{L1-Reg} & \textbf{L2-Reg} & \textbf{Dropout} & \textbf{CF-Reg (ours)} \\ \hline
        Logistic Regression   & \textit{Water}   & N/A   & $0.0093$   & $0.6927$  & N/A    & $0.3791/1.0355$                     \\ \hline
        MLP   & \textit{Water}   & N/A   & $0.0007$   & $0.0022$  & $0.0002$    & $0.2567/1.9775$                    \\ \hline
        Logistic Regression   &
        \textit{Phomene}   & N/A   & $0.0097$   & $0.7979$  & N/A    & $0.0571/1.8516$                     \\ \hline
        MLP   & \textit{Phomene}   & N/A   & $0.0007$   & $4.24\cdot10^{-5}$  & $0.0015$    & $0.0516/2.2700$                    \\ \hline
       % MLP   & \textit{Adult}   & N/A   & $0.0018$   & $0.0018$  & $0.0601$     & $0.0764/2.2068$                    \\ \hline
        CNN   & \textit{CIFAR-10} & N/A   & $0.0050$   & $0.0864$ & N/A    & $0.3018/
        2.1502$                     \\ \hline
    \end{tabular}
\end{table*}

\begin{table*}[htb!]
    \centering
    \caption{Mean value and standard deviation of training time across 5 different runs. The reported time (in seconds) corresponds to the generation of each entry in Table \ref{tab:regularization_comparison}. Times are }
    \label{tab:times}
    \begin{tabular}{|c|c|c|c|c|c|c|}
        \hline
        \textbf{Model} & \textbf{Dataset} & \textbf{No-Reg} & \textbf{L1-Reg} & \textbf{L2-Reg} & \textbf{Dropout} & \textbf{CF-Reg (ours)} \\ \hline
        Logistic Regression   & \textit{Water}   & $222.98 \pm 1.07$   & $239.94 \pm 2.59$   & $241.60 \pm 1.88$  & N/A    & $251.50 \pm 1.93$                     \\ \hline
        MLP   & \textit{Water}   & $225.71 \pm 3.85$   & $250.13 \pm 4.44$   & $255.78 \pm 2.38$  & $237.83 \pm 3.45$    & $266.48 \pm 3.46$                    \\ \hline
        Logistic Regression   & \textit{Phomene}   & $266.39 \pm 0.82$ & $367.52 \pm 6.85$   & $361.69 \pm 4.04$  & N/A   & $310.48 \pm 0.76$                    \\ \hline
        MLP   &
        \textit{Phomene} & $335.62 \pm 1.77$   & $390.86 \pm 2.11$   & $393.96 \pm 1.95$ & $363.51 \pm 5.07$    & $403.14 \pm 1.92$                     \\ \hline
       % MLP   & \textit{Adult}   & N/A   & $0.0018$   & $0.0018$  & $0.0601$     & $0.0764/2.2068$                    \\ \hline
        CNN   & \textit{CIFAR-10} & $370.09 \pm 0.18$   & $395.71 \pm 0.55$   & $401.38 \pm 0.16$ & N/A    & $1287.8 \pm 0.26$                     \\ \hline
    \end{tabular}
\end{table*}

\subsection{Feasibility of our Method}
A crucial requirement for any regularization technique is that it should impose minimal impact on the overall training process.
In this respect, CF-Reg introduces an overhead that depends on the time required to find the optimal counterfactual example for each training instance. 
As such, the more sophisticated the counterfactual generator model probed during training the higher would be the time required. However, a more advanced counterfactual generator might provide a more effective regularization. We discuss this trade-off in more details in Section~\ref{sec:discussion}.

Table~\ref{tab:times} presents the average training time ($\pm$ standard deviation) for each model and dataset combination listed in Table~\ref{tab:regularization_comparison}.
We can observe that the higher accuracy achieved by CF-Reg using the score-based counterfactual generator comes with only minimal overhead. However, when applied to deep neural networks with many hidden layers, such as \textit{PreactResNet-18}, the forward derivative computation required for the linearization of the network introduces a more noticeable computational cost, explaining the longer training times in the table.

\subsection{Hyperparameter Sensitivity Analysis}
The proposed counterfactual regularization technique relies on two key hyperparameters: $\alpha$ and $\beta$. The former is intrinsic to the loss formulation defined in (\ref{eq:cf-train}), while the latter is closely tied to the choice of the score-based counterfactual explanation method used.

Figure~\ref{fig:test_alpha_beta} illustrates how the test accuracy of an MLP trained on the \textit{Water Potability} dataset changes for different combinations of $\alpha$ and $\beta$.

\begin{figure}[ht]
    \centering
    \includegraphics[width=0.85\linewidth]{img/test_acc_alpha_beta.png}
    \caption{The test accuracy of an MLP trained on the \textit{Water Potability} dataset, evaluated while varying the weight of our counterfactual regularizer ($\alpha$) for different values of $\beta$.}
    \label{fig:test_alpha_beta}
\end{figure}

We observe that, for a fixed $\beta$, increasing the weight of our counterfactual regularizer ($\alpha$) can slightly improve test accuracy until a sudden drop is noticed for $\alpha > 0.1$.
This behavior was expected, as the impact of our penalty, like any regularization term, can be disruptive if not properly controlled.

Moreover, this finding further demonstrates that our regularization method, CF-Reg, is inherently data-driven. Therefore, it requires specific fine-tuning based on the combination of the model and dataset at hand.

\section{Ablation study}


\label{sec:study}
% \begin{table}[!t]
% \centering
% \scalebox{0.68}{
%     \begin{tabular}{ll cccc}
%       \toprule
%       & \multicolumn{4}{c}{\textbf{Intellipro Dataset}}\\
%       & \multicolumn{2}{c}{Rank Resume} & \multicolumn{2}{c}{Rank Job} \\
%       \cmidrule(lr){2-3} \cmidrule(lr){4-5} 
%       \textbf{Method}
%       &  Recall@100 & nDCG@100 & Recall@10 & nDCG@10 \\
%       \midrule
%       \confitold{}
%       & 71.28 &34.79 &76.50 &52.57 
%       \\
%       \cmidrule{2-5}
%       \confitsimple{}
%     & 82.53 &48.17
%        & 85.58 &64.91
     
%        \\
%        +\RunnerUpMiningShort{}
%     &85.43 &50.99 &91.38 &71.34 
%       \\
%       +\HyReShort
%         &- & -
%        &-&-\\
       
%       \bottomrule

%     \end{tabular}
%   }
% \caption{Ablation studies using Jina-v2-base as the encoder. ``\confitsimple{}'' refers using a simplified encoder architecture. \framework{} trains \confitsimple{} with \RunnerUpMiningShort{} and \HyReShort{}.}
% \label{tbl:ablation}
% \end{table}
\begin{table*}[!t]
\centering
\scalebox{0.75}{
    \begin{tabular}{l cccc cccc}
      \toprule
      & \multicolumn{4}{c}{\textbf{Recruiting Dataset}}
      & \multicolumn{4}{c}{\textbf{AliYun Dataset}}\\
      & \multicolumn{2}{c}{Rank Resume} & \multicolumn{2}{c}{Rank Job} 
      & \multicolumn{2}{c}{Rank Resume} & \multicolumn{2}{c}{Rank Job}\\
      \cmidrule(lr){2-3} \cmidrule(lr){4-5} 
      \cmidrule(lr){6-7} \cmidrule(lr){8-9} 
      \textbf{Method}
      & Recall@100 & nDCG@100 & Recall@10 & nDCG@10
      & Recall@100 & nDCG@100 & Recall@10 & nDCG@10\\
      \midrule
      \confitold{}
      & 71.28 & 34.79 & 76.50 & 52.57 
      & 87.81 & 65.06 & 72.39 & 56.12
      \\
      \cmidrule{2-9}
      \confitsimple{}
      & 82.53 & 48.17 & 85.58 & 64.91
      & 94.90&78.40 & 78.70& 65.45
       \\
      +\HyReShort{}
       &85.28 & 49.50
       &90.25 & 70.22
       & 96.62&81.99 & \textbf{81.16}& 67.63
       \\
      +\RunnerUpMiningShort{}
       % & 85.14& 49.82
       % &90.75&72.51
       & \textbf{86.13}&\textbf{51.90} & \textbf{94.25}&\textbf{73.32}
       & \textbf{97.07}&\textbf{83.11} & 80.49& \textbf{68.02}
       \\
   %     +\RunnerUpMiningShort{}
   %    & 85.43 & 50.99 & 91.38 & 71.34 
   %    & 96.24 & 82.95 & 80.12 & 66.96
   %    \\
   %    +\HyReShort{} old
   %     &85.28 & 49.50
   %     &90.25 & 70.22
   %     & 96.62&81.99 & 81.16& 67.63
   %     \\
   % +\HyReShort{} 
   %     % & 85.14& 49.82
   %     % &90.75&72.51
   %     & 86.83&51.77 &92.00 &72.04
   %     & 97.07&83.11 & 80.49& 68.02
   %     \\
      \bottomrule

    \end{tabular}
  }
\caption{\framework{} ablation studies. ``\confitsimple{}'' refers using a simplified encoder architecture. \framework{} trains \confitsimple{} with \RunnerUpMiningShort{} and \HyReShort{}. We use Jina-v2-base as the encoder due to its better performance.
}
\label{tbl:ablation}
\end{table*}


\section{Conclusion}





% trigger a \newpage just before the given reference
% number - used to balance the columns on the last page
% adjust value as needed - may need to be readjusted if
% the document is modified later
%\IEEEtriggeratref{8}
% The "triggered" command can be changed if desired:
%\IEEEtriggercmd{\enlargethispage{-5in}}

% references section

% can use a bibliography generated by BibTeX as a .bbl file
% BibTeX documentation can be easily obtained at:
% http://mirror.ctan.org/biblio/bibtex/contrib/doc/
% The IEEEtran BibTeX style support page is at:
% http://www.michaelshell.org/tex/ieeetran/bibtex/
%\bibliographystyle{IEEEtran}
% argument is your BibTeX string definitions and bibliography database(s)
%\bibliography{IEEEabrv,../bib/paper}
%
% <OR> manually copy in the resultant .bbl file
% set second argument of \begin to the number of references
% (used to reserve space for the reference number labels box)
\bibliographystyle{IEEEtran}

\bibliography{main}

% that's all folks
\end{document}


