
\subsection{Expert Pruning}

In the ablation study, we analyze how pruning experts affects the performance and size of the model providing insights into the trade-offs between model complexity and accuracy. %Further investigations, including an analysis of expert choice, can be found in the extended ablation study in Appendix~C.% appendix \ref{app:expert}.




% \setlength{\columnsep}{3pt}%
% \begin{wrapfigure}{r}{0.5\linewidth}
% \centering
%     \centering
%     \begin{minipage}[b]{0.46\textwidth}
%     \includegraphics[width=\textwidth]{figures/res_plot_ep_b.pdf}
%         \caption{Evolution of test accuracy, robustness, skin fairness, and model size w.r.t. the number of iterations.}
%         \label{fig:res_plot_ep_b}
%     \end{minipage}
%     \vspace{-30pt}
% \end{wrapfigure}

This study reveals the results of our expert pruning approach. Figure~\ref{fig:res_plot_ep_b} reports the evolution of key metrics (test accuracy, robustness, skin fairness, and model size) w.r.t. the number of iterations, while in each iteration the least used expert is pruned. Up to 16 iterations, The model maintains better accuracy and fairness than SOTA models with a 26\% decrease in model size (from 2.7M to 2.0M) at the cost of a slight 2\% decrease in Robustness. These findings underscore the effectiveness of our pruning method for efficiently reducing model size while keeping key performance metrics above predefined thresholds.

\begin{figure}[!ht]
\centering
    \includegraphics[width=.9\linewidth]{figures/res_plot_ep_b.pdf}
        \caption{Evolution of test accuracy, robustness, skin fairness, and model size w.r.t. the number of iterations.}
        \label{fig:res_plot_ep_b}
\end{figure}

\subsection{Effect of Expert Choice on the Results}
\label{app:expert}
In our ablation study, we analyzed the performance of our model by examining the choice of experts for each image across different skin tone groups. The results are presented in Figure~\ref{fig:expert_choice}, where each line represents the choice of an expert at each layer for one image.

We observed that clusters of the same color are forming in each layer, suggesting that images from the same skin tone group tend to follow the same path (i.e., they have similar expert choices). Additionally, we noticed that in each layer, there are two or at most three main selected experts, while others are chosen less frequently. This likely stems from the fact that most cases can be effectively handled by these main experts. However, for some extreme cases where the primary experts may fail, the image is routed to a more specific expert.

In summary, thanks to the Switching Layer, the model performs a form of clustering for images, allowing it to choose the appropriate expert for each cluster of images.

\begin{figure}[ht]
    \centering
    \includegraphics[width=\linewidth]{figures/path.pdf}
    \caption{This figure illustrates the expert selection patterns within different switch layers of our model. Each line represents the choice of a specific expert for an individual image across various layers. ``SL-1" refers to Switch Layer 1, and ``E1" denotes Expert 1 in that switch layer, highlighting how images from similar skin tone groups tend to follow comparable paths through the network.}
    \label{fig:expert_choice}
\end{figure}

\subsection{Impact of a Quantum Head on Edge DNN Architecture Performance}
\label{app:quantum}
In this analysis, we delved into the potential of merging classical and quantum computing within edge computing frameworks, motivated by the evolving landscape towards a symbiotic classical-quantum integration\cite{furutanpey2023architectural}. We replaced the classical head of our discovered MoENAS-S architecture with a quantum counterpart. As shown in Figure~\ref{fig:res_plot_qh_}, the resulting model, called \mbox{MoENAS-S + Q}, shows a decline in key performance indicators such as accuracy, fairness, robustness, and generalization when compared to MoENAS-S. However, \mbox{MoENAS-S + Q} surpassed existing SOTA models overall, albeit with a 2\% decrease in robustness. This highlights the viability and promise of hybrid classical-quantum computing approaches in edge DNNs.

\begin{figure}
        \centering
\includegraphics[width=\linewidth]{figures/res_plot_qh_MoENAS_.pdf}
        \caption{Variation of test accuracy, skin fairness, robustness, and generalization, using SOTA models, MoENAS-S, and our proposed model with quantum heads (MoENAS-S + Q).}
        
\label{fig:res_plot_qh_}
\end{figure}