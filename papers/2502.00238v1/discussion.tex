\section{Discussion}
\label{sec:discussion}

Based on our findings, we reflect on how the taxonomy can assist stakeholders in identifying defeaters in assurance cases for cyber-physical systems, and challenges and opportunities.

\noindent \textbf{Towards structured defeater analysis.  }  
The taxonomy and its examples described here aim to provide practitioners and safety analysts with a 
structured framework for systematically identifying and understanding potential threats to an assurance case's validity. It is designed to be readily leveraged, reused, and extended. Moreover, since using different terminologies for describing similar types of defeaters undermines effective communication across different stakeholder groups, the availability of standard taxonomy categories may improve the readability of assurance cases presented to regulators. 

\noindent \textbf{Navigating tensions between known and emergent defeaters. } While the taxonomy provides an organized pathway for analyzing defeaters and reduces reliance on individual mental models, there is a risk that adherence to predefined categories may inadvertently limit practitioners' creativity, especially with novel defeaters. Anticipating emergent defeaters remains difficult, particularly in the early phases of development. In such situations, many variables remain uncertain, complicating the process. Our proposed taxonomy serves as a thematic guide that practitioners can adapt to fit specific circumstances, making it valuable for context-based analyses. Finally, the taxonomy is flexible enough to be extended as new technologies and defeaters emerge.

\noindent \textbf{Towards LLM-supported defeater analysis. } Recent studies have explored the use of Large Language Models (LLMs) for automating the detection and mitigation of defeaters in safety assurance cases \cite{gohar2024codefeater,AISupported} and related requirements tasks \cite{arora2024advancing}. While LLMs have shown promise in strengthening defeater analysis, their effectiveness is hindered by a limited scope, focusing on specific defeater types, %\cite{gohar2024codefeater} 
and their proneness to hallucinations \cite{AISupported}. A structured taxonomy, as an external knowledge source, has been shown to be a valuable resource for enhancing the performance of LLMs 
%\cite{gohar2024codefeater}, 
in similar tasks \cite{zhou2024large}. Our work can potentially help LLMs to navigate complex scenarios by drawing on external knowledge, providing a more nuanced understanding of assurance-case defeaters, and suggesting mitigation strategies to improve their performance.
