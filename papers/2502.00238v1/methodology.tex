\section{Survey Methodology}
\label{sec:methodology}

Our approach follows prior works \cite{10.1145/3510003.3510057,shelby2023sociotechnical,gohar2024long} to review the literature and organize the taxonomy. The framework consists of three stages: (1) Identifying relevant studies; (2) Analyzing; and (3) Collating and summarizing the results into a taxonomy. 

\subsection{Identifying relevant studies}

We employed established methods \cite{greenwell2006taxonomy,shelby2023sociotechnical} to conduct a thorough search for relevant papers with publicly available safety arguments. Our initial search focused on papers published from 2004 to 2024, using the terms ``safety assurance cases'', ``safety cases'', ``defeaters'', ``fallacies'', ``doubts'' and ``assurance case weakeners'', as well as variations of these keywords, across IEEE Xplore, ACM Digital Library, and Google Scholar. We frequently encountered papers containing partial assurance cases but lacking any mention of defeaters. Therefore, we systematically reviewed each article to filter those that contained defeaters. In addition to keyword searches, we followed citation trails to and from key papers and queried Google Scholar and arXiv to identify additional related work. Finally, we incorporated safety arguments developed by our team for sUAS, resulting in six safety arguments from 10 papers. The extracted assurance case defeaters and papers are available in the accompanying artifact \footnote{\url{https://doi.org/10.5281/zenodo.14783329}}.


\subsection{Analyzing data and collating results} We used thematic analysis \cite{Clarke2014} with open-coding \cite{miles1994qualitative,gohar2024towards} to analyze the surveyed literature. Two authors independently conducted the analysis, generating new codes for each unique defeater identified. The codes were iteratively reviewed and refined to establish consensus on key themes. Overlapping codes were consolidated, and the refined taxonomy was applied to our case study. To address conceptual overlaps among codes \cite{DBLP:journals/tse/ChillaregeBCHMRW92,shelby2023sociotechnical}, the most comprehensive term was selected as the primary category covering related concepts.

\subsection{Threats to Validity}

\noindent \textbf{Sample size. } Despite our extensive survey of real-world safety arguments, the limited availability of publicly accessible safety cases constrained the sample size in this study. However, our taxonomy is designed to be flexible and extensible, allowing for the inclusion of new defeater types as they emerge. 

\noindent \textbf{Complex interdependencies. } The categories of defeaters were not always distinctly separable or orthogonal, which can result in overlaps. To mitigate this, we consolidated related defeaters with only subtle differences in their definitions to minimize redundancy in the taxonomy. 

\noindent \textbf{Evolving risks. }
Relying on existing literature may reinforce known patterns of defeaters, potentially overlooking emerging risks or novel challenges in evolving systems and technologies \cite{shelby2023sociotechnical}. As our understanding of defeaters evolves, safety experts must maintain a critical perspective and remain open to identifying new threats.
