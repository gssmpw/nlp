\section{Motivation and Background}
\label{sec:background}
\subsection{Current practice}

Assurance case defeaters play a critical role in certifying the safety of cyber-physical systems by identifying and addressing weaknesses in assurance cases. Defeaters represent potential doubts or objections that challenge a claim's validity, highlighting gaps in evidence and reasoning \cite{murugesan2023semantic}. Figure \ref{fig:AC} shows a fragment of an assurance case for an sUAS battery, with examples of defeaters (red boxes). Practitioners often struggle to identify defeaters systematically, as assurance cases frequently contain implicit assumptions that, if overlooked, can lead to false confidence \cite{greenwell2006taxonomy,gohar2024codefeater}. While defeater analysis strengthens the robustness of assurance cases and results in more verifiable requirements, its practical application is limited by a lack of a structured approach for identifying these defeaters. Without a systematic process, the analysis can suffer from confirmation bias and the varying expertise of reviewers \cite{maksimov2019survey,10311213}, reducing it to a superficial exercise \cite{greenwell2006taxonomy,hobbs2024driving}.

Current practices show inconsistencies; some organizations conduct a single round of defeater analysis, while others adopt an iterative process %that uses initial findings 
to target risk areas for deeper review \cite{millet2023assurance,gohar2024codefeater}. Objectives can also vary, from broad risk identification to adversarial testing or evaluating margins of error. Similar challenges exist in analogous processes (e.g., red-teaming \cite{feffer2024red}) and in various domains (e.g., AI safety), driving the development of taxonomies to better identify and manage risks \cite{ramirez2012taxonomy,mohseni2022taxonomy}.
Structured frameworks can aid practitioners’ anticipation of issues throughout the product lifecycle. As a first step towards operationalizing defeater analysis, we review the safety assurance case literature focused on creating a taxonomy of defeaters for safety-critical cyber-physical systems. The proposed taxonomy provides a systematic view of the current discourse on types of defeaters in such systems.

\begin{figure}
    \centering
    \includegraphics[scale = 0.70]{Figures/Examples.drawio.pdf}
    \caption{sUAS Safety Assurance Case fragment with example defeaters.} 
    \label{fig:AC}
    \vspace{-2.5mm}
\end{figure}

\begin{figure*}
    \centering
    \includegraphics[width = \textwidth]{Figures/Taxonomy-Defeater.pdf}
    \vspace{-2.5mm}
    \caption{A high-level taxonomy of real-world defeaters in assurance cases. The boxes at the top show the  seven broad  categories.}
    \vspace{-2.5mm}
    \label{fig:taxonomy}
\end{figure*}

\subsection{Overview of the Automated Flight Authorization System}

Our evaluation of the proposed taxonomy is centered on an assurance case \cite{hunter2024family} we are developing for a sUAS automated flight authorization system. This system evaluates mission details, environmental conditions, and operator qualifications to determine flight permissions. It addresses the increasing operational demands and safety requirements associated with the increase in sUAS activities \cite{gohar2024towards}. The assurance case presents a series of arguments and sub-goals supporting the main claim \emph{“The sUAS can safely complete its intended mission in the specified environmental conditions.”} The system's real-world complexity makes it suitable for identifying defeaters and testing the taxonomy in practice.