\section{Related Work}
\label{RW}

Several studies have been conducted on the identification of skills that are critical for various domains such as cybersecurity (e.g., \cite{2,3,5,10,12,15,29,caulkins2019cybersecurity,jerman2022cybersecurity,peslak2019cybersecurity, jain2024comprehensive}), software engineering (e.g., \cite{khaouja2021survey,daneva2019understanding,matturro2019systematic, goel2018overview, goel2021maintenance}), and signal processing \cite{abbas2024robust}. In the context of cybersecurity, Ben-Asher and Gonzalez \cite{2} analyzed data collected from a case study with 55 participants in the US to examine how individuals with and without security knowledge detect malicious events. Unlike this study, our study is based on global data collected from job ads and Stack Overflow. Also, our research questions are different. In \cite{3}, the authors conducted a review of the literature to identify the gaps in cybersecurity expertise. Then, this paper emphasizes the contribution of social fit in a highly complex and heterogeneous cyber workforce. Although this paper's aim is similar to our first RQ, our work differs from it in terms of the data source, two RQs, and the findings. Parker and Brown et al., \cite{10} analyzed data from 196 job ads in South Africa to determine the skills required for cybersecurity professionals. In \cite{peslak2019cybersecurity}, the authors collected data from 500 job ads to identify critical cyber skills. Similar to our study,  \cite{10} and \cite{peslak2019cybersecurity} collected data from job ads. However, these papers collected data only from 150 - 500 job ads in one country. In comparison, we have collected data from 12, 161 jobs from around the world and 49, 002 Stack Overflow posts. 
\smallbreak
Potter and Vickers et al. \cite{12} collected data from interviews with security professionals to identify the skill critical for security professionals in the Australian market. This study differs from ours in terms of the RQs and data analyzed. In \cite{15}, the authors focussed on skills important for one particular cyber role - information security analyst. This study was conducted based on a literature review in the Malaysian context. Our study is not limited to one particular cyber role. Chowdhury and Gkioulos \cite{29} carried out a literature review of existing studies to identify the skills that security professionals need to secure critical infrastructures. Unlike our study which identifies cyber skills for general cybersecurity, Chowdhury and Gkioulos \cite{29} only focused on the skills required for one domain i.e., critical infrastructures. Caulkins et al. \cite{caulkins2019cybersecurity} conducted a survey with government cybersecurity professionals to identify the soft skills required for cybersecurity professionals. Unlike \cite{caulkins2019cybersecurity}, our study focuses not only on soft skills but also hard skills and certifications. Jerman et al, \cite{jerman2022cybersecurity} conducted interviews with students, teachers, and parents to understand what cybersecurity topics should be taught to the students and how best they can be taught. Jones et al., \cite{5} conducted 44 interviews with cybersecurity professionals to identify cyber topics that students should learn in school. Unlike \cite{5} and \cite{jerman2022cybersecurity} which focus on cyber topics, our study focuses on cyber skills. In summary, our work differs from the existing works in terms of RQs, data source, and findings.