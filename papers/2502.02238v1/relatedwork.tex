\section{Related work}
\label{sec:rw}
\subsection{LLMs for conceptual design}

An experiment to use LLMs for creating specifications from requirements documents in the realm of smart devices is described in \cite{DBLP:conf/hci/LutzeW24}. The authors contend that the fundamental skill of conceptual design ---namely, choosing and defending the best option to meet the requirements--- is still lacking, but they acknowledge that LLMs are very useful in later phases of the development process, like creating class diagrams from comprehensive implementation specifications and generating source code. 

Additional experiments with ChatGPT for conceptual modeling (using, for example, UML class diagrams and E/R diagrams) are discussed in \cite{DBLP:journals/emisaij/FillFK23}. 
The authors note that ChatGPT can rapidly produce an initial draft diagram from a natural language description; nevertheless, considerable modeling expertise is still needed to improve and verify the outcomes. They therefore come to the conclusion that ChatGPT can assist specialists by simplifying conceptual design, but it is obvious that it cannot take their place.

The authors of \cite{Zhou24} describe an experiment they conducted using ChatGPT and come to the conclusion that while adding LLMs to human-driven conceptual design does not dramatically affect outcomes, it does greatly reduce the time required to complete the design by requiring fewer design steps.

In \cite{DBLP:journals/corr/abs-2306-01779}, many conceptual design concepts produced by an LLM are contrasted with a baseline of crowdsourced solutions. On average, it is shown that crowdsourced ideas are more innovative, whereas LLM-generated solutions are more practical. Remarkably, it is also shown that the LLM-generated answers resemble the crowdsourced ones better when \emph{few-shot learning} is used (i.e., the prompt is supplemented with some instances of the problem).

In \cite{Chen24}, the benefits of utilizing LLMs to improve morphological analysis in conceptual design are examined. The tests demonstrate how LLMs give designers access to interdisciplinary knowledge and facilitate the methodical dissection and analysis of design problems. For optimal outcomes, LLMs and designers should work closely together and use smart prompt engineering.

With relation to use case and domain modeling, \cite{Ali24} examines how users engage with LLMs during conceptual modeling. The primary conclusions speak to the necessity of particular prompt templates to assist users on the one hand, and the usage of a recommender to offer pertinent prompts or actions on the other.


\subsection{Conceptual design of multidimensional cubes}

The main types of methods to multidimensional conceptual design are \emph{supply-driven} (or data-driven), \emph{demand-driven} (or requirement-driven), \emph{mixed}, and \emph{query-driven}. Supply-driven methods begin by designing conceptual schemata from the schemata of the data sources (such as relational schemata); end-user requirements influence design by enabling the designer to choose which data are important for making decisions and by figuring out how to structure them using the multidimensional model \cite{DBLP:conf/ssdbm/RomeroA11}. Demand-driven techniques begin with identifying end-users' business requirements, and only then do they look into how to map these requirements onto the available data sources \cite{DBLP:journals/is/0001RSAM14}.
Mixed techniques integrate requirements-driven and data-driven methods; here, both end-user requirements and data source schemata are used simultaneously \cite{DBLP:journals/infsof/TriaLT12}. The set of OLAP queries that end-users are willing to formulate is the starting point for the creation of a multidimensional schema in query-driven approaches. These queries can be specified using SQL statements \cite{DBLP:journals/dke/RomeroA10}, MDX expressions \cite{Niemi.2001}, pivot tables \cite{Bimonte.2021}, or query trees \cite{Nair.2007}.

Multidimensional modeling techniques are reviewed in \cite{DBLP:journals/jdwm/RomeroA09}, and their cost-benefit analysis is provided in \cite{DBLP:journals/is/TriaLT17}.