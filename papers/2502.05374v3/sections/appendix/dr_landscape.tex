\section{Additional Visualization Results for Loss Landscape on Retain Set}
\label{appendix: loss_lanscape_dr}



\begin{figure*}[htbp] % 或者 [htbp], 视排版需求
\centering
%============= 第二行:5 幅图 =============
\begin{tabular}{cccc}
\hspace*{-3mm}
\raisebox{0.08\height}{\rotatebox{90}{\small{Loss landscape on $\mathcal{D}_\mathrm{r
}$}}} \hspace*{-2mm} % 提高位置,使其与图片垂直居中
&
\hspace*{-3mm}
\includegraphics[width=0.20\textwidth,height=!]{figs/origin_dr.pdf}
&
\hspace*{-3mm}
\includegraphics[width=0.20\textwidth,height=!]{figs/npo_dr.pdf}
&
\hspace*{-3mm}
\includegraphics[width=0.20\textwidth,height=!]{figs/npo_sam_dr.pdf}

\\

&
\hspace*{-6mm}
\small{(a) Origin}
&

\hspace*{-3mm}
\small{(b) NPO}
&
\hspace*{-3mm}
\small{(c) NPO+SAM}

\\
\end{tabular}

\begin{tabular}{ccccc}
\hspace*{-3mm}
\raisebox{0.08\height}{\rotatebox{90}{\small{Loss landscape on $\mathcal{D}_\mathrm{r
}$}}} \hspace*{-2mm} % 提高位置,使其与图片垂直居中

&
\hspace*{-3mm}
\includegraphics[width=0.20\textwidth,height=!]{figs/npo_rs_dr.pdf}
&
\hspace*{-3mm}
\includegraphics[width=0.20\textwidth,height=!]{figs/npo_gnr_dr.pdf}
&
\hspace*{-3mm}
\includegraphics[width=0.20\textwidth,height=!]{figs/npo_cr_dr.pdf}
&
\hspace*{-3mm}
\includegraphics[width=0.20\textwidth,height=!]{figs/npo_swa_dr.pdf}
\\

&

\hspace*{-3mm}
\small{(d) NPO+RS}
&
\hspace*{-3mm}
\small{(e) NPO+GP}
&
\hspace*{-3mm}
\small{(f) NPO+CR}
&
\hspace*{-3mm}
\small{(g) NPO+WA}
\\
\end{tabular}

\vspace*{-2mm}
\caption{\small{
The prediction loss landscape of the original model, along with the NPO and smooth variants of the NPO-unlearned model, on the retain set.
}}
\label{fig: loss_lanscape_dr}
\end{figure*}

In \textbf{Fig.\,\ref{fig: loss_lanscape_dr}}, we further illustrate the loss landscapes of the origin model, the unlearned model obtained using NPO, and the smooth variants of NPO on the retain set. It is evident that the loss landscapes of the origin model and the unlearned model are quite similar, indicating that the unlearning process primarily affects the model's performance on the forget data while having minimal impact on its performance on the retain set. Furthermore, it is worth noting that the loss landscapes of the unlearned models from NPO and its smooth variants show little difference on the retain data but exhibit significant differences on the forget data (as shown in Fig.\,\ref{fig: loss_lanscape}). This observation further suggests that the robustness of the unlearned model is closely related to the smoothness of the forget loss.



% \begin{figure*}[htb] % 或者 [htbp], 视排版需求
% \centering
% %============= 第二行:5 幅图 =============
% \begin{tabular}{ccccccc}
% \hspace*{-3mm}
% \raisebox{-0.02\height}{\rotatebox{90}{\small{Loss landscape on $\mathcal{D}_\mathrm{r
% }$}}} \hspace*{-5mm} % 提高位置,使其与图片垂直居中
% &
% \hspace*{-3mm}
% \includegraphics[width=0.16\textwidth,height=!]{figs/npo_dr.pdf}
% &
% \hspace*{-6mm}
% \includegraphics[width=0.16\textwidth,height=!]{figs/npo_sam_dr.pdf}
% &
% \hspace*{-6mm}
% \includegraphics[width=0.16\textwidth,height=!]{figs/npo_rs_dr.pdf}
% &
% \hspace*{-6mm}
% \includegraphics[width=0.16\textwidth,height=!]{figs/npo_gnr_dr.pdf}
% &
% \hspace*{-6mm}
% \includegraphics[width=0.16\textwidth,height=!]{figs/npo_cr_dr.pdf}
% &
% \hspace*{-6mm}
% \includegraphics[width=0.16\textwidth,height=!]{figs/npo_swa_dr.pdf}
% \\

% &
% \hspace*{-6mm}
% \small{(a) NPO}
% &
% \hspace*{-6mm}
% \small{(b) NPO + SAM}
% &
% \hspace*{-6mm}
% \small{(c) NPO + RS}
% &
% \hspace*{-6mm}
% \small{(d) NPO + GP}
% &
% \hspace*{-6mm}
% \small{(e) NPO + CR}
% &
% \hspace*{-6mm}
% \small{(f) NPO + WA}
% \\
% \end{tabular}

% \vspace*{-2mm}
% \caption{\small{The prediction loss landscape of the NPO-unlearned and smooth variants of NPO-unlearned model on the retain set.
% }}
% \label{fig: loss_lanscape_dr}
% \end{figure*}