\begin{table*}[htbp]
\centering
\begin{tcolorbox}[
    colback=white, % 背景颜色
    colframe=black, % 边框颜色
    title=Response, % 标题
    fonttitle=\bfseries, % 标题加粗
    width=\textwidth, % 宽度
    boxrule=1pt, % 边框粗细
    rounded corners, % 圆角
]

%\begin{tabularx}{\textwidth}{X|X} % 两栏布局
\begin{tabularx}{\textwidth}{X}
%\textbf{Greedy Decision-making (part 2)} &
%\textbf{Guiding Action Selection with Q (part 2)} \\
%\hline
%\textbf{Action:} perform nothing. \newline
%\textbf{Observation:} \lipsum[1] \newline
%\textbf{Thought:} \lipsum[2] &
%\textbf{Observation:} \lipsum[1] \newline
%\textbf{Thought:} \lipsum[2] \newline
%\textbf{Action:} click[Buy Now] \\
%\hline
%\textbf{Reward:} 0 &
%\textbf{Reward:} 1.0 \\
\textbf{Database Prompt:} ... \\
\textbf{Question:} Among the payments made by Mary Smith, how many of them are over 4.99?\\ 
\textbf{External Knowledge:} over 4.99 refers to amount > 4.99 \\
\textbf{Gold SQL:} SELECT count(payment.amount) FROM payment INNER JOIN customer ON payment.customer\_id = customer.customer\_id WHERE customer.first\_name = 'MARY' AND customer.last\_name = 'SMITH' AND payment.amount > 4.99 \\
\hline
\\ \textbf{Synthesized Chain-of-Thought Solution} \\

To answer the question about how many payments made by Mary Smith are greater than 4.99, we will need to join the \verb|`|payment\verb|`| table with the \verb|`|customer\verb|`| table. The query will count the number of payments that match Mary Smith's first and last name while ensuring the payment amount exceeds 4.99.
\\ \\
Here’s how the SQL query can be constructed:
\\
1. **Join the Tables**: We will perform INNER JOIN between the \verb|`|payment\verb|`| and \verb|`|customer\verb|`| tables on the \verb|`|customer\_id\verb|`| column. This links payments to the customers who made them.

2. **Filter Conditions**: We will add conditions to filter for only those rows where the customer's first name is 'MARY' and last name is 'SMITH'.

3. **Amount Condition**: Additionally, we will include a condition to check if the payment amount is greater than 4.99.

4. **Count the Results**: Finally, we will use the COUNT function to return the total number of matching records.
\\ \\
Here is the SQL query that fulfills the requirements:
\\ 
\verb|`|\verb|`|\verb|`|SQL \\
SELECT COUNT(payment.amount) \\ 
FROM payment \\ 
INNER JOIN customer ON payment.customer\_id = customer.customer\_id \\
WHERE customer.first\_name = 'MARY' \\
AND customer.last\_name = 'SMITH' \\
AND payment.amount > 4.99; \\
\verb|`|\verb|`|\verb|`|
\\ \\ 
**Explanation**: \\
- \verb|`|SELECT COUNT(payment.amount)\verb|`|: This part specifies that we want to count the number of payments that meet the conditions specified. \\
- \verb|`|FROM payment\verb|`|: Indicates that the main table we are querying is the \verb|`|payment\verb|`| table. \\
- \verb|`|INNER JOIN customer ON payment.customer\_id = customer.customer\_id\verb|`|: This joins the \verb|`|payment\verb|`| table with the \verb|`|customer\verb|`| table based on the \verb|`|customer\_id\verb|`| present in both tables, allowing us to access customer details related to each payment. \\
- \verb|`|WHERE customer.first\_name = 'MARY' AND customer.last\_name = 'SMITH'\verb|`|: These conditions filter the results to include only payments made by a customer named Mary Smith. \\
- \verb|`|AND payment.amount > 4.99\verb|`|: This additional condition specifies that we only want payments where the amount is greater than 4.99.
\\ \\
This query will count the payments made by Mary Smith that exceed the specified amount.

%\\
%\hline \\ 
%\textbf{Gold SQL:} SELECT count(payment.amount) FROM payment INNER JOIN customer ON payment.customer\_id = customer.customer\_id WHERE customer.first\_name = 'MARY' AND customer.last\_name = 'SMITH' AND payment.amount > 4.99
\end{tabularx}
\end{tcolorbox}
\caption{A qualitative example of \texttt{gpt-4o-mini-2024-07-18} generated Chain-of-Thought for an instance of Bird Train set. \textbf{SQL in the CoT can be different from original gold SQL in the dataset}. }
\label{tab:synthesisCoT}
\end{table*}