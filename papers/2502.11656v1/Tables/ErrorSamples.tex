\definecolor{darkgreen}{RGB}{0,150,0}
\newcommand{\bt}[1]{\textbf{\texttt{#1}}}
\newcommand{\gh}[1]{\textbf{\textcolor{darkgreen}{#1}}}
\newcommand{\rh}[1]{\textbf{\textcolor{red}{#1}}}


% External Knowledge
\begin{table*}[t]
\centering
%\setlength{\tabcolsep}{5pt} % Adjust column padding
%\renewcommand{\arraystretch}{1.3} % Adjust row spacing
%\begin{tabularx}{\textwidth}{lXlccc}
\begin{adjustbox}{max width=\textwidth}
\begin{tabular}{p{4.75cm} p{5.5cm} p{5.5cm} p{2.5cm}}
\toprule
\textbf{Sample Question with EK} & \textbf{Gold SQL} & \textbf{Predicted SQL} & \textbf{Error Analysis}\\
\midrule
\multicolumn{4}{c}{[A1] EK} \\
\midrule
 Identify the atoms belong to the molecule with \rh{ID between TR010 to TR050} that contain the element carbon. EK: carbon refers to element = 'c'; \rh{between TR010 to TR050 refers to substr(molecule\_id, 3, 3)>=10 \bt{AND} substr(molecule\_id, 3, 3) <= 50}
 & \raggedright \bt{SELECT} T.atom\_id \bt{FROM} atom \bt{AS} T \bt{WHERE} T.molecule\_id \rh{\bt{BETWEEN} 'TR010' \bt{AND} 'TR050' \bt{AND} T.element = 'c'}
 & \raggedright \bt{SELECT} atom\_id \bt{FROM} atom \bt{WHERE} \gh{molecule\_id \bt{LIKE} 'TR\%' \bt{AND} element = 'c'}
 & The SQL selects all molecule with a 'TR' prefix instead. \\
\bottomrule
\end{tabular}
\end{adjustbox}
\caption{Sample errors and analysis for types in the External Knowledge category.}
\label{table:ErrorSampleEK}
\end{table*}


\begin{table*}[t]
\centering
%\setlength{\tabcolsep}{5pt} % Adjust column padding
%\renewcommand{\arraystretch}{1.3} % Adjust row spacing
%\begin{tabularx}{\textwidth}{lXlccc}
\begin{adjustbox}{max width=\textwidth}
\begin{tabular}{p{4.75cm} p{5.5cm} p{5.5cm} p{2.5cm}}
\toprule
\textbf{Sample Question with EK} & \textbf{Gold SQL} & \textbf{Predicted SQL} & \textbf{Error Analysis}\\
\midrule
\multicolumn{4}{c}{[B1] Table} \\
\midrule
How many accounts who have region in Prague are eligible for \rh{loans}? EK: A3 contains the data of region & \raggedright
\bt{SELECT}  \bt{COUNT(}T1.account\_id\bt{)} \bt{FROM} account \bt{AS} T1 \rh{\bt{INNER} \bt{JOIN} loan} \bt{AS} T2 \bt{ON} T1.account\_id = T2.account\_id \bt{INNER} \bt{JOIN} district \bt{AS} T3 \bt{ON} T1.district\_id = T3.district\_id \bt{WHERE} T3.A3 = 'Prague' & \raggedright \bt{SELECT} \bt{COUNT(}account.account\_id\bt{)} \bt{FROM} account \bt{INNER} \bt{JOIN} district \bt{ON} account.district\_id = district.district\_id \bt{WHERE} district.a3 = 'Prague' & The loan table is need to check for eligibility. \\
\midrule
\multicolumn{4}{c}{[B2] JOIN} \\
\midrule
State the \rh{district} and region for \rh{loan} ID '4990'. EK: A2, A3 contains the information about district and region respectively. & \raggedright \bt{SELECT} T2.A2, T2.A3 \bt{FROM} account \bt{AS} T1 \bt{INNER} \bt{JOIN} district \bt{AS} T2 \bt{ON} \rh{T1.district\_id = T2.district\_id} \bt{INNER JOIN} loan \bt{AS} T3 \bt{ON} \rh{T1.account\_id = T3.account\_id} \bt{WHERE} T3.loan\_id = 4990 & \raggedright \bt{SELECT} district.a2, district.a3 \bt{FROM} district \bt{INNER JOIN} loan \bt{ON} \gh{district.district\_id = loan.account\_id} \bt{WHERE} loan.loan\_id = 4990; & District table and loan table can't join together without account table. \\
\midrule
\multicolumn{4}{c}{[B3] Column} \\
\midrule
In which \rh{mailing street address} can you find the school that has the lowest average score in reading? Also give the school's name. & \raggedright \bt{SELECT} T2.\rh{MailStreet}, T2.School \bt{FROM} satscores \bt{AS} T1 \bt{INNER JOIN} schools \bt{AS} T2 \bt{ON} T1.cds = T2.CDSCode \bt{WHERE} T1.AvgScrRead \bt{IS NOT NULL ORDER BY} T1.AvgScrRead \bt{ASC LIMIT 1} & \raggedright \bt{SELECT} schools.\gh{street}, satscores.sname \bt{FROM} satscores \bt{INNER JOIN} schools \bt{ON} satscores.cds = schools.cdscode \bt{ORDER BY} satscores.avgscrread \bt{ASC LIMIT} 1; & The question ask for mailing street address, not the street address. \\
\midrule
\multicolumn{4}{c}{[B4] Hallucination} \\
\midrule
What is the average math score of the school with the lowest average score for all subjects, and \rh{in which county} is it located? EK: Average score for all subjects can be computed by AvgScrMath + AvgScrRead + AvgScrWrite & \raggedright \bt{SELECT} T1.AvgScrMath, \rh{T2.County} \bt{FROM} satscores \bt{AS} T1 \bt{INNER JOIN} \rh{schools \bt{AS} T2} \bt{ON} T1.cds = T2.CDSCode \bt{WHERE} T1.AvgScrMath \bt{IS NOT NULL ORDER BY} T1.AvgScrMath + T1.AvgScrRead + T1.AvgScrWrite \bt{ASC LIMIT} 1	& \raggedright \bt{SELECT} avgscrmath, \gh{county \bt{FROM} satscores} \bt{ORDER BY} avgscrmath + avgscrread + avgscrwrite \bt{ASC LIMIT} 1	& Country is a column of schools table. \\
\midrule
\multicolumn{4}{c}{[B5] Condition} \\
\midrule
Please list the zip code of all the charter \rh{schools in Fresno County Office of Education}.	EK: Charter schools refers to `Charter School (Y/N)` = 1 in the table fprm & \raggedright \bt{SELECT} T2.Zip \bt{FROM} frpm \bt{AS} T1 \bt{INNER JOIN} schools \bt{AS} T2 \bt{ON} T1.CDSCode = T2.CDSCode \bt{WHERE} \rh{T1.`District Name` = 'Fresno County Office of Education'} \bt{AND} T1.`Charter School (Y/N)` = 1 & \raggedright \bt{SELECT DISTINCT} schools.zip \bt{FROM} schools \bt{INNER JOIN} frpm \bt{ON} schools.cdscode = frpm.cdscode \bt{WHERE} \gh{schools.county = 'Fresno'} \bt{AND} frpm.`charter school (y/n)` = 1; & 'Fresno County Office of Education' is  a district name, the SQL mistakenly treats it as county name. \\
\midrule
\multicolumn{4}{c}{[B6] NULL/DISTINCT} \\
\midrule
\rh{How many molecules} have a triple bond type?	EK : triple bond refers to bond\_type = '\#'; & \raggedright \rh{\bt{SELECT COUNT(DISTINCT} T.molecule\_id\bt{)}} \bt{FROM} bond \bt{AS} T \bt{WHERE} T.bond\_type = '\#' & \raggedright \gh{\bt{SELECT COUNT(}*\bt{)}} \bt{FROM} bond \bt{WHERE} bond\_type = '\#'; & There may be multiple records for a molecule in the bond table.
\\ \midrule
\rh{Which school} in Contra Costa has the highest number of test takers? & \raggedright \bt{SELECT} \rh{sname} \bt{FROM} satscores \bt{WHERE} cname = 'Contra Costa' \bt{AND} \rh{sname \bt{IS NOT NULL}} \bt{ORDER BY} NumTstTakr \bt{DESC LIMIT} 1 & \raggedright \bt{SELECT} \gh{sname} \bt{FROM} satscores \bt{WHERE} cname = 'Contra Costa' \bt{ORDER BY} numtsttakr \bt{DESC LIMIT} 1 & Sname column may have missing values.
\\
\bottomrule
\end{tabular}
\end{adjustbox}
\caption{Sample errors and analysis for types in the Schema Linking category.}
\label{table:ErrorSampleSchema}
\end{table*}

% Value Retrieval
\begin{table*}[t]
\centering
%\setlength{\tabcolsep}{5pt} % Adjust column padding
%\renewcommand{\arraystretch}{1.3} % Adjust row spacing
%\begin{tabularx}{\textwidth}{lXlccc}
\begin{adjustbox}{max width=\textwidth}
\begin{tabular}{p{4.75cm} p{5.5cm} p{5.5cm} p{2.5cm}}
\toprule
\textbf{Sample Question with EK} & \textbf{Gold SQL} & \textbf{Predicted SQL} & \textbf{Error Analysis}\\
\midrule
\multicolumn{4}{c}{[C1] String/Number} \\
\midrule
List out the code for drivers who have \rh{nationality in America}. & \raggedright \bt{SELECT} code \bt{FROM} drivers \bt{WHERE} \rh{Nationality = 'American'} & \raggedright \bt{SELECT} code \bt{FROM} drivers \bt{WHERE} \gh{nationality = 'America'}; & Although the question mentions 'America', in the nationality column, the corresponding entity in database is 'American'.  \\
\midrule
\multicolumn{4}{c}{[C2] Date} \\
\midrule
How many percent of 'Gold' cards were \rh{issued prior to 1998?} EK: Percent of Gold = [ count(type = 'gold' and issued date < 1998) / count(all cards)] * 100\%
 & \raggedright \bt{SELECT CAST( SUM(}type = 'gold' \bt{AND} \rh{\bt{STRFTIME(}'\%Y', issued\bt{)} < '1998'\bt{)}} \bt{AS REAL)} * 100 / \bt{COUNT(}card\_id\bt{) FROM} card
 & \raggedright \bt{SELECT CAST( SUM( CASE WHEN} TYPE = 'gold' \bt{AND} \gh{issued < '1998'} \bt{THEN} 1 \bt{ELSE} 0 \bt{END) AS REAL)} * 100 / \bt{count(}card\_id\bt{) FROM} card
 & The issued date is in format of yyyy-mm-dd  \\
\bottomrule
\end{tabular}
\end{adjustbox}
\caption{Sample errors and analysis for types in the Value Retrieval category.}
\label{table:ErrorSampleValue}
\end{table*}


% Operation
\begin{table*}[t]
\centering
%\setlength{\tabcolsep}{5pt} % Adjust column padding
%\renewcommand{\arraystretch}{1.3} % Adjust row spacing
%\begin{tabularx}{\textwidth}{lXlccc}
\begin{adjustbox}{max width=\textwidth}
\begin{tabular}{p{4.75cm} p{5.5cm} p{5.5cm} p{2.5cm}}
\toprule
\textbf{Sample Question with EK} & \textbf{Gold SQL} & \textbf{Predicted SQL} & \textbf{Error Analysis}\\
\midrule
\multicolumn{4}{c}{[D1] Mathematical Formula} \\
\midrule
 For loan amount less than USD100,000, \rh{what is the percentage of accounts that is still running with no issue.} EK: Status = 'C' stands for running contract, ok so far; Percentage of accounts by condition = [(total(amount) \& condition) / (total amount)] * 100.
 & \raggedright \bt{SELECT} \rh{\bt{CAST(SUM(}status = 'C'\bt{) AS} REAL\bt{)} * 100 / \bt{COUNT(}account\_id\bt{)}} \bt{FROM} loan \bt{WHERE} amount < 100000
 & \raggedright \bt{SELECT} \gh{\bt{(SUM(CASE WHEN} amount < 100000 \bt{AND} status = 'C' \bt{THEN} amount \bt{ELSE} 0 \bt{END)} * 100.0 / \bt{SUM(CASE WHEN} status = 'C' \bt{THEN} amount \bt{ELSE} 0 \bt{END))}} \bt{AS} percentage \bt{FROM} loan
 & The SQL tries to answer: For accounts that is still running with no issue, what is the percentage of loan amount less than USD 100,000? \\
\midrule
\multicolumn{4}{c}{[D2] Aggregation} \\
\midrule
 \rh{How many owner and disponent} dispositions are there from account number 1 to account number 10?  
 & \raggedright \bt{SELECT} \rh{\bt{SUM(}type = 'OWNER'\bt{)} ,\bt{ SUM(}type = 'DISPONENT'\bt{)}} \bt{FROM} disp \bt{WHERE} account\_id \bt{BETWEEN} 1 \bt{AND} 10
 & \raggedright \bt{SELECT} \gh{\bt{COUNT(}*\bt{)}} \bt{FROM} disp \bt{WHERE} account\_id \bt{BETWEEN} 1 \bt{AND} 10 \bt{AND} \gh{type \bt{IN (}'OWNER', 'DISPONENT'\bt{)}}
 &  Should count owener and disponent seperately. \\
\midrule
\multicolumn{4}{c}{[D3] Complex Operation} \\
\midrule
 Which molecule does the atom TR001\_10 belong to? \rh{Please state whether this molecule is carcinogenic or not.} EK: TR001\_10 is the atom id; label = '+' mean molecules are carcinogenic
 & \raggedright \bt{SELECT} T2.molecule\_id , \rh{IIF(T2.label = '+', 'YES', 'NO')} \bt{AS} flag\_carcinogenic \bt{FROM} atom \bt{AS} T1 \bt{INNER JOIN} molecule \bt{AS} T2 \bt{ON} T1.molecule\_id = T2.molecule\_id \bt{WHERE} T1.atom\_id = 'TR001\_10'
 & \raggedright \bt{SELECT DISTINCT} m.label \bt{FROM} molecule m \bt{INNER JOIN} atom a \bt{ON} m.molecule\_id = a.molecule\_id \bt{WHERE} a.atom\_id = 'TR001\_10'
 & The question ask for a judgement, which is more complex than ordinary query. \\

\bottomrule
\end{tabular}
\end{adjustbox}
\caption{Sample errors and analysis for types in the Operation category.}
\label{table:ErrorSampleOperation}
\end{table*}


% Information
\begin{table*}[t]
\centering
%\setlength{\tabcolsep}{5pt} % Adjust column padding
%\renewcommand{\arraystretch}{1.3} % Adjust row spacing
%\begin{tabularx}{\textwidth}{lXlccc}
\begin{adjustbox}{max width=\textwidth}
\begin{tabular}{p{4.75cm} p{5.5cm} p{5.5cm} p{2.5cm}}
\toprule
\textbf{Sample Question with EK} & \textbf{Gold SQL} & \textbf{Predicted SQL} & \textbf{Error Analysis}\\
\midrule
\multicolumn{4}{c}{[E1] Redundant/Incomplete} \\
\midrule
 What are the \rh{labels for TR000, TR001 and TR002}? EK: TR000, TR001 and TR002 are molecule id;
 & \raggedright \bt{SELECT} \rh{molecule\_id, T.label} \bt{FROM} molecule \bt{AS} T \bt{WHERE} T.molecule\_id \bt{IN (}'TR000', 'TR001', 'TR002'\bt{)}
 & \raggedright \bt{SELECT} \gh{label} \bt{FROM} molecule \bt{WHERE} molecule\_id \bt{IN (}'TR000', 'TR001', 'TR002'\bt{)};
 &  Although the question asks for labels, without id, one can not tell them apart. \\
\midrule
\multicolumn{4}{c}{[E2] Column Sequence} \\
\midrule
 What are the \rh{elements} of the toxicology \rh{and label} of molecule TR060? EK: TR060 is the molecule id; 
 & \raggedright \bt{SELECT DISTINCT} \rh{T1.element, T2.label} \bt{FROM} atom \bt{AS} T1 \bt{INNER JOIN} molecule \bt{AS} T2 \bt{ON} T1.molecule\_id = T2.molecule\_id \bt{WHERE} T2.molecule\_id = 'TR060'
 & \raggedright \bt{SELECT DISTINCT} \gh{m.label, a.element} \bt{FROM} molecule m \bt{JOIN} atom a \bt{ON} m.molecule\_id = a.molecule\_id \bt{WHERE} m.molecule\_id = 'TR060';
 & Label should come after element as required. \\
\midrule
\multicolumn{4}{c}{[E3] ORDER BY/LIMIT} \\
\midrule
 For the \rh{first client} who opened his/her account in Prague, what is his/her account ID? EK: A3 stands for region names
 & \raggedright \bt{SELECT} T1.account\_id \bt{FROM} account \bt{AS} T1 \bt{INNER JOIN} district \bt{AS} T2 \bt{ON} T1.district\_id = T2.district\_id \bt{WHERE} T2.A3 = 'Prague' \rh{\bt{ORDER BY} T1.date \bt{ASC LIMIT} 1}
 & \raggedright \bt{SELECT} account.account\_id \bt{FROM} account \bt{INNER JOIN} district \bt{ON} account.district\_id = district.district\_id \bt{WHERE} district.a3 = 'Prague' \gh{\bt{LIMIT} 1}
 & First client refers to the account with smallest timestamp. \\
\midrule
\multicolumn{4}{c}{[E4] Format} \\
\midrule
 Name the ID of the patient who is born on the April 1st, 1982. \rh{Is his/her alkaliphophatase (ALP) within normal range}? EK: alkaliphophatase (ALP) within normal range refers to ALP < 300
 & \raggedright \bt{SELECT} T1.ID , \rh{\bt{CASE WHEN} T2.ALP < 300 \bt{THEN} 'normal' \bt{ELSE} 'abNormal' \bt{END}} \bt{FROM} Patient \bt{AS} T1 \bt{INNER JOIN} Laboratory \bt{AS} T2 \bt{ON} T1.ID = T2.ID \bt{WHERE} T1.Birthday = '1982-04-01'
 & \raggedright \bt{SELECT} \gh{\bt{CASE WHEN} laboratory.alp < 300 \bt{THEN} 'YES' \bt{ELSE} 'NO' \bt{END}}, patient.id \bt{FROM} patient \bt{INNER JOIN} laboratory \bt{ON} patient.id = laboratory.id \bt{WHERE} patient.birthday = '1982-04-01'
 & The desired answer is 'normal' or 'abNormal'. \\

\bottomrule
\end{tabular}
\end{adjustbox}
\caption{Sample errors and analysis for types in the Information category.}
\label{table:ErrorSampleInfo}
\end{table*}


% Syntax Error
\begin{table*}[t]
\centering
%\setlength{\tabcolsep}{5pt} % Adjust column padding
%\renewcommand{\arraystretch}{1.3} % Adjust row spacing
%\begin{tabularx}{\textwidth}{lXlccc}
\begin{adjustbox}{max width=\textwidth}
\begin{tabular}{p{4.75cm} p{5.5cm} p{5.5cm} p{2.5cm}}
\toprule
\textbf{Sample Question with EK} & \textbf{Gold SQL} & \textbf{Predicted SQL} & \textbf{Error Analysis}\\
\midrule
\multicolumn{4}{c}{[F1] Syntax} \\
\midrule
 For the owner user of \rh{post} No. 65041, what is his/her reputation points? EK: post no. 65041 refers to Id = '65041'; reputation point refers to Reputation
 & \raggedright \bt{SELECT} T1.Reputation \bt{FROM} users \bt{AS} T1 \bt{INNER JOIN} \rh{posts} \bt{AS} T2 \bt{ON} T1.Id = T2.OwnerUserId \bt{WHERE} T2.Id = 65041
 & \raggedright \bt{SELECT} users.reputation \bt{FROM} users \bt{INNER} \gh{\bt{JOIN}\_posts} \bt{ON} users.id =\gh{\_posts}.owneruserid \gh{\bt{WHERE}\_posts}.id = 65041
 & No seperation between keywords and the post table.  \\
\bottomrule
\end{tabular}
\end{adjustbox}
\caption{Sample errors and analysis for types in the Syntax Error category.}
\label{table:ErrorSampleSyntax}
\end{table*}

