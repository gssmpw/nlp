\begin{table*}[htbp]
\centering
\begin{tcolorbox}[
    colback=white, % 背景颜色
    colframe=black, % 边框颜色
    title=Response, % 标题
    fonttitle=\bfseries, % 标题加粗
    width=\textwidth, % 宽度
    boxrule=1pt, % 边框粗细
    rounded corners, % 圆角
]

%\begin{tabularx}{\textwidth}{X|X} % 两栏布局
\begin{tabularx}{\textwidth}{X}
%\textbf{Greedy Decision-making (part 2)} &
%\textbf{Guiding Action Selection with Q (part 2)} \\
%\hline
%\textbf{Action:} perform nothing. \newline
%\textbf{Observation:} \lipsum[1] \newline
%\textbf{Thought:} \lipsum[2] &
%\textbf{Observation:} \lipsum[1] \newline
%\textbf{Thought:} \lipsum[2] \newline
%\textbf{Action:} click[Buy Now] \\
%\hline
%\textbf{Reward:} 0 &
%\textbf{Reward:} 1.0 \\
\textbf{Database Prompt:} Table district, columns = [ district.a11 ( integer | comment : average salary | values : 12541 , 8507 ) ,
district.a3 ( text | comment : region | values : Prague , \underline{central Bohemia} ) , district.district\_id ( integer | primary
key | comment : location of branch | values : 1 , 2 ) ... \\
\textbf{Question:} How many male customers who are living in North Bohemia have average salary greater than 8000? \\ 
\textbf{External Knowledge:} Male means that gender = 'M'; a3 refers to region; a11 pertains to average salary. \\
\hline
\\ %\textbf{Synthesized Chain-of-Thought Solution} \\

To answer the question of how many male customers living in North Bohemia have an average salary greater than 8000, we need to join the \verb|`|client\verb|`| table with the \verb|`|district\verb|`| table based on the \verb|`|district\_id\verb|`|. We will then apply the necessary filters for the region (\verb|`|a3\verb|`|), gender (\verb|`|gender\verb|`|), and average salary (\verb|`|a11\verb|`|).
\\ \\ 
Here's the breakdown of the SQL query:
\\ \\ 
1. **SELECT count(client.client\_id)**: We want to count the number of male clients, so we use the `count` function on `client.client\_id`. \\

2. **FROM district INNER JOIN client ON district.district\_id = client.district\_id**: We perform an inner join between the \verb|`|district\verb|`| and \verb|`|client\verb|`| tables on the \verb|`|district\_id\verb|`| column, which is common to both tables. This allows us to access the district information for each client. \\

3. **WHERE district.a3 = 'north Bohemia'**: We filter the results to include only those clients who live in the North Bohemia region. \\

4. **AND client.gender = 'M'**: We further filter to include only male clients. \\

5. **AND district.a11 > 8000**: Finally, we add a condition to ensure that we only count clients whose average salary is greater than 8000. 
\\ \\
Putting this all together, the SQL query is as follows:
\\ \\ 
\verb|`|\verb|`|\verb|`|SQL \\
SELECT count(client.client\_id) \\ 
FROM district \\
INNER JOIN client ON district.district\_id = client.district\_id \\
WHERE district.a3 = 'north Bohemia' \\
AND client.gender = 'M' \\
AND district.a11 > 8000; \\
\verb|`|\verb|`|\verb|`| 
\\ \\
This query will return the count of male clients residing in North Bohemia with an average salary exceeding 8000.

%\\
%\hline \\ 
%\textbf{Gold SQL:} SELECT count(payment.amount) FROM payment INNER JOIN customer ON payment.customer\_id = customer.customer\_id WHERE customer.first\_name = 'MARY' AND customer.last\_name = 'SMITH' AND payment.amount > 4.99
\end{tabularx}
\end{tcolorbox}
\caption{The full response Syn CoT DPO Qwen2.5-7B-Instruct generated for a sample question from Bird Dev. \textbf{The model is able to infer entity format of 'north Bohemia' from value examples given in database prompt}.}
\label{tab:cotResponseDPO}
\end{table*}