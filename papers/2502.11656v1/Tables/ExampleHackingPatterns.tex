\begin{table*}[htbp]
\centering
\begin{tcolorbox}[
    colback=white, % 背景颜色
    colframe=black, % 边框颜色
    title=Reward Hacking Example, % 标题
    fonttitle=\bfseries, % 标题加粗
    width=\textwidth, % 宽度
    boxrule=1pt, % 边框粗细
    rounded corners, % 圆角
]

%\begin{tabularx}{\textwidth}{X|X} % 两栏布局
\begin{tabularx}{\textwidth}{X}
\textbf{Question:} Among the superheroes with the race of god/eternal, how many of them are male?\\ 
%\textbf{External Knowledge:} over 4.99 refers to amount > 4.99 \\
\textbf{Gold SQL:} SELECT COUNT(*) FROM superhero AS T1 INNER JOIN race AS T2 ON T1.race\_id = T2.id INNER JOIN gender AS T3 ON T3.id = T1.gender\_id WHERE T1.race\_id = 21 AND T1.gender\_id = 1 \\
\hline
\\ \textbf{Response:} SELECT count(superhero.id) FROM gender INNER JOIN superhero ON gender.id = superhero.gender\_id WHERE{\color{red}\begin{CJK}{UTF8}{gbsn}英雄种族为\end{CJK}}god/eternal AND{\color{red}\begin{CJK}{UTF8}{gbsn}英雄性别为男\end{CJK}}
\end{tabularx}
\end{tcolorbox}
\caption{DPO model uses Chinese to state the conditions (Note that there are no Chinese character included in training IOs), which cause syntax errors. Interestingly, \begin{CJK}{UTF8}{gbsn}英雄种族为\end{CJK} means ``the race of superhero is'', and \begin{CJK}{UTF8}{gbsn}英雄性别为男\end{CJK} means ``the gender of superhero is male'', which are semantically correct. Similar cases are common, and the alternative language is not restricted to Chinese, we also see Farsi and Egyptian in the generation.}
\label{tab:egRH1}
\end{table*}

\begin{table*}[htbp]
\centering
\begin{tcolorbox}[
    colback=white, % 背景颜色
    colframe=black, % 边框颜色
    title=Reward Hacking Example, % 标题
    fonttitle=\bfseries, % 标题加粗
    width=\textwidth, % 宽度
    boxrule=1pt, % 边框粗细
    rounded corners, % 圆角
]

%\begin{tabularx}{\textwidth}{X|X} % 两栏布局
\begin{tabularx}{\textwidth}{X}
\textbf{Question:} Calculate the average height of all neutral superheroes.\\ 
%\textbf{External Knowledge:} over 4.99 refers to amount > 4.99 \\
\textbf{Gold SQL:} SELECT AVG(T1.height\_cm) FROM superhero AS T1 INNER JOIN alignment AS T2 ON T1.alignment\_id = T2.id WHERE T2.alignment = 'Neutral' \\
\hline
\\ \textbf{Response:} SELECT avg(superhero.height\_cm) FROM alignment INNER JOIN superhero ON alignment.idossalhero.alignment\_id WHERE {\color{red}alignmentalignmentalignmentalignmentalignmentalignmentalignmentalignmentalignmentalignmentalignmenta}...
\end{tabularx}
\end{tcolorbox}
\caption{DPO model starts to repeat certain word (``alignment'' in this case), which is a common reward hacking pattern when DPO is used as alignment algorithm.}
\label{tab:egRH2}
\end{table*}

\begin{table*}[htbp]
\centering
\begin{tcolorbox}[
    colback=white, % 背景颜色
    colframe=black, % 边框颜色
    title=Reward Hacking Example, % 标题
    fonttitle=\bfseries, % 标题加粗
    width=\textwidth, % 宽度
    boxrule=1pt, % 边框粗细
    rounded corners, % 圆角
]

%\begin{tabularx}{\textwidth}{X|X} % 两栏布局
\begin{tabularx}{\textwidth}{X}
\textbf{Question:} For the patient with the highest lactate dehydrogenase in the normal range, when was his or her data first recorded? \\ 
%\textbf{External Knowledge:} over 4.99 refers to amount > 4.99 \\
\textbf{Gold SQL:} SELECT T1.`First Date` FROM Patient AS T1 INNER JOIN Laboratory AS T2 ON T1.ID = T2.ID WHERE T2.LDH < 500 ORDER BY T2.LDH ASC LIMIT 1 \\
\hline
\\ \textbf{Response:} SELECT min(patient.`first date`) FROM patient INNER JOIN laboratory ON patient.id = laboratory.id WHERE laboratory.ldh < 500 OR {\color{red} (ldh IS NULL AND NOT AND ISnumeric(laboratory.ldh))}
\end{tabularx}
\end{tcolorbox}
\caption{DPO model struggles to produce SQLs with complete structure (e.g. unmatched parentheses, missing spaces, wrong usage of keywords), which is another common reward hacking pattern.}
\label{tab:egRH3}
\end{table*}