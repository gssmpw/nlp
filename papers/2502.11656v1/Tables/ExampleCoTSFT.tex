\begin{table*}[htbp]
\centering
\begin{tcolorbox}[
    colback=white, % 背景颜色
    colframe=black, % 边框颜色
    title=Response, % 标题
    fonttitle=\bfseries, % 标题加粗
    width=\textwidth, % 宽度
    boxrule=1pt, % 边框粗细
    rounded corners, % 圆角
]

%\begin{tabularx}{\textwidth}{X|X} % 两栏布局
\begin{tabularx}{\textwidth}{X}
%\textbf{Greedy Decision-making (part 2)} &
%\textbf{Guiding Action Selection with Q (part 2)} \\
%\hline
%\textbf{Action:} perform nothing. \newline
%\textbf{Observation:} \lipsum[1] \newline
%\textbf{Thought:} \lipsum[2] &
%\textbf{Observation:} \lipsum[1] \newline
%\textbf{Thought:} \lipsum[2] \newline
%\textbf{Action:} click[Buy Now] \\
%\hline
%\textbf{Reward:} 0 &
%\textbf{Reward:} 1.0 \\
\textbf{Database Prompt:} Table district, columns = [ district.a11 ( integer | comment : average salary | values : 12541 , 8507 ) ,
district.a3 ( text | comment : region | values : Prague , \underline{central Bohemia} ) , district.district\_id ( integer | primary
key | comment : location of branch | values : 1 , 2 ) ... \\
\textbf{Question:} How many male customers who are living in North Bohemia have average salary greater than 8000? \\ 
\textbf{External Knowledge:} Male means that gender = 'M'; a3 refers to region; a11 pertains to average salary. \\
\hline
\\ %\textbf{Synthesized Chain-of-Thought Solution} \\

To answer the question of how many male customers living in North Bohemia have an average salary greater than 8000, we need to join the \verb|`|district\verb|`| and \verb|`|client\verb|`| tables based on the \verb|`|district\_id\verb|`|. We will then filter the results based on the specified conditions: the region being 'North Bohemia', the gender being 'M', and the average salary being greater than 8000. \\ \\

Here's the SQL query that accomplishes this: \\ \\

\verb|`|\verb|`|\verb|`|SQL \\
SELECT COUNT(client.client\_id) \\
FROM district \\
INNER JOIN client ON district.district\_id = client.district\_id \\
WHERE district.a3 = 'North Bohemia' \\ 
AND client.gender = 'M' \\ 
AND district.a11 > 8000; \\ 
\verb|`|\verb|`|\verb|`| \\ \\ 

\#\#\# Explanation: \\
1. **SELECT COUNT(client.client\_id)**: This part of the query counts the number of male clients that meet the specified criteria. 
  
2. **FROM district**: We start our query from the \verb|`|district\verb|`| table since we need to filter based on the region and average salary.

3. **INNER JOIN client ON district.district\_id = client.district\_id**: We perform an inner join between the \verb|`|district\verb|`| and \verb|`|client\verb|`| tables using the \verb|`|district\_id\verb|`| as the key. This allows us to access the client information associated with each district.

4. **WHERE district.a3 = 'North Bohemia'**: This condition filters the results to only include districts in the 'North Bohemia' region.

5. **AND client.gender = 'M'**: This condition filters the results to only include male clients.

6. **AND district.a11 > 8000**: This condition filters the results to only include districts where the average salary is greater than 8000.

\\
By executing this query, we will obtain the count of male customers living in North Bohemia with an average salary greater than 8000.

%\\
%\hline \\ 
%\textbf{Gold SQL:} SELECT count(payment.amount) FROM payment INNER JOIN customer ON payment.customer\_id = customer.customer\_id WHERE customer.first\_name = 'MARY' AND customer.last\_name = 'SMITH' AND payment.amount > 4.99
\end{tabularx}
\end{tcolorbox}
\caption{The full response Syn CoT SFT Qwen2.5-7B-Instruct generated for a sample question from Bird Dev. \textbf{The response is wrong due to entity mismatch for the region 'north Bohemia'}.}
\label{tab:cotResponseSFT}
\end{table*}