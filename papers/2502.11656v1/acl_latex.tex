% This must be in the first 5 lines to tell arXiv to use pdfLaTeX, which is strongly recommended.
\pdfoutput=1
% In particular, the hyperref package requires pdfLaTeX in order to break URLs across lines.

\documentclass[11pt]{article}

% Change "review" to "final" to generate the final (sometimes called camera-ready) version.
% Change to "preprint" to generate a non-anonymous version with page numbers.
\usepackage[preprint]{acl}

% Standard package includes
\usepackage{times}
\usepackage{latexsym}

% For proper rendering and hyphenation of words containing Latin characters (including in bib files)
\usepackage[T1]{fontenc}
% For Vietnamese characters
% \usepackage[T5]{fontenc}
% See https://www.latex-project.org/help/documentation/encguide.pdf for other character sets

% This assumes your files are encoded as UTF8
\usepackage[utf8]{inputenc}

% This is not strictly necessary, and may be commented out,
% but it will improve the layout of the manuscript,
% and will typically save some space.
\usepackage{microtype}

% This is also not strictly necessary, and may be commented out.
% However, it will improve the aesthetics of text in
% the typewriter font.
\usepackage{inconsolata}

%Including images in your LaTeX document requires adding
%additional package(s)
\usepackage{graphicx}

\usepackage{booktabs}


% If the title and author information does not fit in the area allocated, uncomment the following
%
%\setlength\titlebox{<dim>}
%
% and set <dim> to something 5cm or larger.

%%%%%%%% the package that i include
\usepackage{multirow}
\usepackage{booktabs}
\usepackage{colortbl}
%\usepackage[table]{xcolor} % 推荐,提供更多颜色支持
\usepackage{xcolor}

%\usepackage[x11names]{xcolor}

\usepackage{adjustbox}
\usepackage{makecell}
\usepackage{tabularx}
\usepackage{amsfonts} % mathbb

%\usepackage{graphicx}

\usepackage{amsmath}
%\usepackage{listings}
\usepackage{subcaption}
%\usepackage{graphicx}

\usepackage[utf8]{inputenc}
\usepackage{tcolorbox}
\usepackage{tabularx}
\usepackage{lipsum}
\usepackage{enumitem}
\usepackage{courier}

% \title{Unlocking the Potential of Direct Preference Optimization in Text-to-SQL Through Chain-of-Thought Reasoning}
% \title{Chain-of-Thought Reasoning as a Catalyst for Preference Learning: Lessons from Text-to-SQL}
\title{Uncovering the Impact of Chain-of-Thought Reasoning for Direct Preference Optimization: Lessons from Text-to-SQL}

% Author information can be set in various styles:
% For several authors from the same institution:
% \author{Author 1 \and ... \and Author n \\
%         Address line \\ ... \\ Address line}
% if the names do not fit well on one line use
%         Author 1 \\ {\bf Author 2} \\ ... \\ {\bf Author n} \\
% For authors from different institutions:
% \author{Author 1 \\ Address line \\  ... \\ Address line
%         \And  ... \And
%         Author n \\ Address line \\ ... \\ Address line}
% To start a separate ``row'' of authors use \AND, as in
% \author{Author 1 \\ Address line \\  ... \\ Address line
%         \AND
%         Author 2 \\ Address line \\ ... \\ Address line \And
%         Author 3 \\ Address line \\ ... \\ Address line}

% \author{First Author \\
%   Affiliation / Address line 1 \\
%   Affiliation / Address line 2 \\
%   Affiliation / Address line 3 \\
%   \texttt{email@domain} \\\And
%   Second Author \\
%   Affiliation / Address line 1 \\
%   Affiliation / Address line 2 \\
%   Affiliation / Address line 3 \\
%   \texttt{email@domain} \\}

\author{
    \textbf{Hanbing Liu\textsuperscript{1}\thanks{Equal contribution.}},
    \textbf{Haoyang Li\textsuperscript{2,3}\footnotemark[1]},
    \textbf{Xiaokang Zhang\textsuperscript{2,3}},
    \textbf{Ruotong Chen\textsuperscript{2}}, \\ 
    \textbf{Haiyong Xu\textsuperscript{5}},
    \textbf{Tian Tian\textsuperscript{5}},
    \textbf{Qi Qi\textsuperscript{1}},
    \textbf{Jing Zhang\textsuperscript{2,4}}\thanks{Corresponding author.}
    \\
    \textsuperscript{1}Gaoling School of Artificial Intelligence, Renmin University of China, Beijing, China, \\
    \textsuperscript{2}School of Information, Renmin University of China, Beijing, China, \\
    \textsuperscript{3}Key Laboratory of Data Engineering and Knowledge Engineering, Beijing, China, \\
    \textsuperscript{4}Engineering Research Center of Database and Business Intelligence, Beijing, China, \\
    \textsuperscript{5}China Mobile Information Technology Center
    \\
    %\texttt{\{liuhanbing, lihaoyang.cs, zhang2718, chen2022, qi.qi, zhang-jing\}@ruc.edu.cn}
    %\\
    {\fontfamily{zi4}\selectfont\{liuhanbing, lihaoyang.cs, zhang-jing\}@ruc.edu.cn}
    %\\
    %\texttt{\{xuhaiyong, tiantianit\}@chinamobile.com}
    %\\
    %\small{
    %    \textbf{Correspondence:} \href{mailto:email@domain}{email@domain}
    %}
}

%\author{
%  \textbf{First Author\textsuperscript{1}},
%  \textbf{Second Author\textsuperscript{1,2}},
%  \textbf{Third T. Author\textsuperscript{1}},
%  \textbf{Fourth Author\textsuperscript{1}},
%\\
%  \textbf{Fifth Author\textsuperscript{1,2}},
%  \textbf{Sixth Author\textsuperscript{1}},
%  \textbf{Seventh Author\textsuperscript{1}},
%  \textbf{Eighth Author \textsuperscript{1,2,3,4}},
%\\
%  \textbf{Ninth Author\textsuperscript{1}},
%  \textbf{Tenth Author\textsuperscript{1}},
%  \textbf{Eleventh E. Author\textsuperscript{1,2,3,4,5}},
%  \textbf{Twelfth Author\textsuperscript{1}},
%\\
%  \textbf{Thirteenth Author\textsuperscript{3}},
%  \textbf{Fourteenth F. Author\textsuperscript{2,4}},
%  \textbf{Fifteenth Author\textsuperscript{1}},
%  \textbf{Sixteenth Author\textsuperscript{1}},
%\\
%  \textbf{Seventeenth S. Author\{4,5}},
%  \textbf{Eighteenth Author\textsuperscript{3,4}},
%  \textbf{Nineteenth N. Author\textsuperscript{2,5}},
%  \textbf{Twentieth Author\textsuperscript{1}}
%\\
%\\
%  \textsuperscript{1}Affiliation 1,
%  \textsuperscript{2}Affiliation 2,
%  \textsuperscript{3}Affiliation 3,
%  \textsuperscript{4}Affiliation 4,
%  \textsuperscript{5}Affiliation 5
%\\
%  \small{
%    \textbf{Correspondence:} \href{mailto:email@domain}{email@domain}
%  }
%}

\usepackage{CJKutf8}
\begin{document}
\maketitle
\begin{abstract}
% Direct Preference Optimization (DPO) has shown significant success in complex reasoning tasks, such as math word problems and code generation. However, when applied to Text-to-SQL datasets, we find that DPO often fails to enhance model performance and can even degrade it. Through extensive investigation, we identify the root cause of this limitation: unlike math and code tasks, which inherently leverage Chain-of-Thought (CoT) reasoning alongside DPO, Text-to-SQL datasets typically provide only final answers (i.e., golden SQL queries) without detailed CoT solutions. By augmenting existing Text-to-SQL datasets with synthetic CoT solutions, we achieve, for the first time, stable and significant performance improvements using DPO for Text-to-SQL tasks.
% %To address this, we introduce a CoT-synthesizer that augments Text-to-SQL datasets with step-by-step reasoning. With this simple yet effective augmentation technique, we achieve, for the first time, stable and significant performance improvements using DPO on Text-to-SQL tasks. 
% To understand why CoT reasoning is essential for unlocking DPO’s potential, we conduct a comprehensive analysis. Our findings reveal that incorporating CoT mitigates reward hacking during DPO training, strengthens the model’s discriminative ability, and enhances scalability. We believe these insights could provide valuable guidance for the Text-to-SQL community in building more robust and effective models. To support further research, we publicly release the code and CoT-enhanced datasets~\footnote{\url{https://anonymous.4open.science/r/ARR_Submission}}.
Direct Preference Optimization (DPO) has proven effective in complex reasoning tasks like math word problems and code generation. However, when applied to Text-to-SQL datasets, it often fails to improve performance and can even degrade it. Our investigation reveals the root cause: unlike math and code tasks, which naturally integrate Chain-of-Thought (CoT) reasoning with DPO, Text-to-SQL datasets typically include only final answers (gold SQL queries) without detailed CoT solutions. By augmenting Text-to-SQL datasets with synthetic CoT solutions, we achieve, for the first time, consistent and significant performance improvements using DPO.

Our analysis shows that CoT reasoning is crucial for unlocking DPO’s potential, as it mitigates reward hacking, strengthens discriminative capabilities, and improves scalability. These findings offer valuable insights for building more robust Text-to-SQL models. To support further research, we publicly release the code and CoT-enhanced datasets
%~\footnote{\url{https://anonymous.4open.science/r/ARR_Submission}}.
~\footnote{\url{https://github.com/RUCKBReasoning/DPO_Text2SQL}}.

% , by training large language models (LLMs) to distinguish between correct and incorrect responses


% Existing fine-tuning-based text-to-SQL methods typically train on benchmarks to directly generate the desired SQL queries. This approach overlooks the powerful reasoning capabilities obtained during the pre-training phase and fails to fully leverage the potential of preference learning. To address this, we propose a new Text-to-SQL pipeline that utilizes synthesized chain-of-thought (CoT) reasoning paths to enhance direct preference optimization (DPO). We find that incorporating CoT and DPO significantly improves performance on the Spider, Bird, and four robustness Text-to-SQL benchmarks. Furthermore, we observe that when SQL queries are generated directly without CoT (the vanilla setting), DPO often leads to performance degradation. To explain this phenomenon, we provide a comprehensive analysis, revealing that DPO may exhibit severe reward hacking in the vanilla setting. In contrast, the introduction of CoT could mitigate this issue, resulting in improved performance.

% DPO is a powerful technique xxx


% This document is a supplement to the general instructions for *ACL authors. It contains instructions for using the \LaTeX{} style files for ACL conferences.
% The document itself conforms to its own specifications, and is therefore an example of what your manuscript should look like.
% These instructions should be used both for papers submitted for review and for final versions of accepted papers.
\end{abstract}

\section{Introduction}
% Recently, Text-to-SQL has attracted significant interest from both the natural language processing and database communities~\citep{li2024codes, wang2020rat-sql, DBLP:journals/pvldb/FuLWLTS23catsql, pourreza2024din-sql}. This technology plays a crucial role in translating natural language questions into SQL queries, enabling non-experts to easily access and query data. As a result, it serves as a valuable tool for business intelligence, data exploration, and a wide range of data-centric applications.
Text-to-SQL has recently gained significant attention in natural language processing and database research~\citep{li2024codes, wang2020rat-sql, DBLP:journals/pvldb/FuLWLTS23catsql, pourreza2024din-sql}. It translates natural language questions into SQL queries, allowing non-experts to easily access data, making it a valuable tool for business intelligence, data exploration, and other data-centric applications.


% The wide application of such model in real-world scenarios attracted interest from researchers in the fields of both database systems and natural language processing.

% With the advent of large language models (LLMs), the Text-to-SQL paradigm is shifting towards using prompting-based multi-agent frameworks~\citep{talaei2024chess, pourreza2024din-sql, pourreza2024chase-sql}. However, these approaches heavily depend on closed-source LLMs, such as GPT-4~\citep{openai2024@gpt4-turbo} and Gemini~\citep{DBLP:journals/corr/abs-2312-11805gemini}, which pose significant challenges for real-world Text-to-SQL applications, including high usage costs, data privacy concerns, and slow inference speeds. To address these issues, recent studies have explored training open-source LLMs using supervised fine-tuning objectives~\citep{pourreza2024dts-sql, li2023resdsql, li2024codes}.

% With the advent of large language models (LLMs), two primary approaches have emerged for solving Text-to-SQL: prompting-based methods~\citep{talaei2024chess, pourreza2024din-sql, pourreza2024chase-sql} and supervised fine-tuning-based (SFT-based) methods~\citep{pourreza2024dts-sql, li2023resdsql, li2024codes}. In real-world applications, prompting-based methods are often impractical due to their reliance on powerful closed-source LLMs, leading to high usage costs, data privacy concerns, and slower inference speeds. Conversely, SFT approaches train open-source, deploy-friendly LLMs using existing benchmark datasets like Spider~\cite{} and Bird~\cite{}. However, the performance of SFT models is frequently constrained by the limited availability of high-quality training data, which is labor-intensive and time-consuming to acquire.

With large language models (LLMs), two main approaches have emerged for solving Text-to-SQL: prompting-based methods~\citep{talaei2024chess, pourreza2024din-sql, pourreza2024chase-sql}  and supervised fine-tuning (SFT) methods~\citep{pourreza2024dts-sql, li2023resdsql, li2024codes}. Prompting-based methods often rely on powerful closed-source LLMs, making them costly and slow, and raising data privacy concerns. In contrast, SFT trains open-source, deployable LLMs using benchmark datasets like Spider and Bird. However, SFT performance is often limited by the scarcity of high-quality training data, which is expensive and time-consuming to create.

\begin{figure}[t!]
  \vspace{\baselineskip}
  \centering
  \includegraphics[width=0.98\linewidth]{figures/peformance_changes_after_dpo.pdf}
  \caption{Model performance gains (greedy decoding) achieved by DPO over SFT (Improved Execution Accuracy, \%). Chain-of-thought reasoning is crucial for unlocking DPO's potential, ensuring its effectiveness and stability. }
  \label{fig:dpo_changes}
\end{figure}

Recent studies in complex reasoning tasks, such as math word problems~\cite{DBLP:conf/emnlp/XuLLHLZWZDZ0D24chatglm} and code generation~\cite{DBLP:journals/corr/abs-2406-06887plum}, have demonstrated that preference optimization algorithms (\emph{e.g.}, DPO~\cite{rafailov2024dpo}, KTO~\cite{DBLP:journals/corr/abs-2402-01306kto}, SimPO~\cite{DBLP:journals/corr/abs-2405-14734simpo}) can significantly enhance SFT models. These algorithms leverage preference data pairs to enable models to distinguish between correct and incorrect responses, addressing the limitations of simple SFT. Despite the proven success of preference optimization techniques, recent works in Text-to-SQL have rarely adopted these methods to improve the Text-to-SQL capabilities of LLMs. This raises a critical question: \textit{How much improvement can preference optimization bring to the Text-to-SQL task?}

\textbf{Preliminary Experiments.} To answer this question, we conduct initial experiments on Bird~\cite{li2024bird}, a challenging cross-domain Text-to-SQL benchmark. Each data sample consists of a <question, database, SQL query> triplet. Text-to-SQL models receive the question and database information (\emph{e.g.}, table names, column names, data types, etc.) and generate the target SQL query. To ensure the universality of our findings, we evaluate 10 open-source LLMs, ranging from 6.7B to 15B parameters. For preference optimization, we employ DPO, a widely adopted technique used in cutting-edge LLMs like LLaMA3~\cite{dubey2024llama3}, Qwen2.5~\cite{qwenreport}, and Mixtral~\cite{mixtral}.

Specifically, we follow the standard DPO training pipeline, which consists of three key steps:
(1) \textbf{SFT}: The base LLM is first fine-tuned on Bird's training set.
(2) \textbf{Preference Pair Construction}: Using the SFT model, multiple SQL queries are sampled for each training sample. Correct and incorrect queries are identified through database execution to create preference pairs.
(3) \textbf{DPO Training}: Finally, the SFT model is further trained on these preference pairs using the DPO loss, resulting in the final DPO model.

\textbf{Observations.} The ``Original Bird'' area in Figure~\ref{fig:dpo_changes} illustrates the performance gains introduced by DPO, measured as the improvement in execution accuracy between the DPO model and the SFT model with greedy search inference. Surprisingly, the results reveal that DPO does not consistently improve performance; in fact, it leads to performance degradation for 6 out of the 10 evaluated LLMs. To make preference optimization effective for Text-to-SQL, we additionally explore several strategies, including hyperparameter tuning~\cite{rafailov2024dpo}, integrating SFT loss~\cite{ouyang2022rlhf}, replacing DPO with KTO~\cite{DBLP:journals/corr/abs-2402-01306kto}, and using a small model to construct preference data~\cite{yang2024sense}. However, as shown in Appendix~\ref{apx:dpotricks}, these attempts still result in limited performance improvements ($<$1.5\%). 

\textbf{Hypothesis.} After extensive but unsuccessful algorithmic exploration, we hypothesize that the suboptimal performance of DPO in the Text-to-SQL task is primarily due to a critical yet often-overlooked factor: the quality of the data. By analyzing datasets for complex reasoning tasks, such as MATH~\cite{DBLP:conf/nips/HendrycksBKABTS21MathBenchmark}, GSM8K~\cite{DBLP:journals/corr/abs-2110-14168gsm8k}, CodeUltraFeedback~\cite{codeultrafeedback}, Orca-Math~\cite{orcamath}, and DART-Math~\cite{dartmath}, we observe that these datasets provide not only final answers but also chain-of-thought (CoT)-styled solutions with detailed reasoning steps. These CoT solutions bridge the gap between input questions and final answers, enabling LLMs to achieve better generalization and interpretability during SFT and DPO training. In contrast, Text-to-SQL datasets like Bird~\cite{li2024bird}, Spider~\cite{yu2018spider}, WikiSQL~\cite{wikisql}, and ScienceBenchmark~\cite{sciencebenchmark} only provide final answers (\emph{i.e.}, gold SQL queries), forcing SFT and DPO to rely solely on SQL queries as training labels. This discrepancy leads us to propose a hypothesis: \textit{The effectiveness of DPO is likely attributed to the use of CoT, a crucial factor that is often overlooked.}

\textbf{Verification.} 
% To validate this hypothesis, we introduce a pipeline to investigate how CoT impacts DPO's performance in the Text-to-SQL task. First, to efficiently and accurately generate CoT solutions for Text-to-SQL datasets with minimal human intervention, we adopt an LLM-based CoT synthesizer. The synthesizer takes as input the database information, the question, and the golden SQL query, and produces step-by-step CoT solutions. Next, using the same settings from preliminary experiments, we apply SFT and DPO to the CoT-enhanced Bird dataset. As illustrated in Figure~\ref{fig:dpo_changes }, as we expected, the introduction of CoT leads to stable and significant performance improvements in DPO across all 10 evaluated LLMs. Furthermore, in the experiment section, we extend our evaluations beyond Bird to a diverse set of Text-to-SQL benchmarks, including Spider~\cite{}, Spider-DK~\cite{}, Spider-Syn~\cite{}, Spider-Realistic~\cite{}, and Dr.Spider~\cite{}. Consistent trends are observed across these benchmarks.
To test this hypothesis, we introduce a pipeline to study how CoT affects DPO's performance in the Text-to-SQL task. We use an LLM-based CoT synthesizer to efficiently generate step-by-step CoT solutions for Text-to-SQL datasets with minimal human effort. The synthesizer takes the database information, question, and gold SQL query as input. Then, using the same settings as earlier experiments, we apply SFT and DPO to the CoT-enhanced Bird dataset. As shown in Figure~\ref{fig:dpo_changes}, adding CoT significantly improves DPO's performance across all 10 evaluated LLMs. Additionally, we extend our evaluations to other Text-to-SQL benchmarks, including Spider, Spider-DK~\cite{gan2021spiderdk}, Spider-Syn~\cite{gan2021spidersyn}, Spider-Realistic~\cite{deng2021spiderrealitic}, and Dr.Spider~\cite{DBLP:conf/iclr/Changdrspider}. Consistent trends are observed across these benchmarks.

% However, existing fine-tuning-based Text-to-SQL methods have two limitations, which may compromise their performance: 

% \textbf{(1) Lack of Leveraging Language Reasoning Ability}: Many studies have shown that encouraging large language models (LLMs) to generate step-by-step Chain-of-Thought (CoT) responses, rather than directly producing answers, significantly enhances performance on complex reasoning tasks~\cite{DBLP:conf/nips/Wei0SBIXCLZ22chain, DBLP:conf/iclr/ZhouSHWS0SCBLC23least, DBLP:conf/iclr/0002WSLCNCZ23selfconsis}. CoT helps unlock the language reasoning abilities acquired during pre-training by breaking down complex tasks into simpler, logical steps. However, current Text-to-SQL benchmarks, such as Spider~\citep{yu2018spider} and Bird~\cite{li2024bird}, typically do not provide CoT paths from questions to the desired SQL queries. As a result, most fine-tuning methods train models to generate SQL queries directly, without incorporating reasoning steps or explanations. This omission prevents models from fully leveraging their pre-trained knowledge, limiting their performance on existing benchmarks and their ability to generalize to unseen or more challenging scenarios.


% \textbf{(2) Under-exploration of Preference Learning}: After initial supervised fine-tuning (SFT), direct preference optimization (DPO)~\citep{rafailov2024dpo} being the most widely adopted post-training strategy, which can further enhance a model's performance. By training on preference pair data, DPO enlarges the probability margin between chosen and rejected responses, showing potential in complex tasks like math word problems~\citep{DBLP:conf/emnlp/XuLLHLZWZDZ0D24chatglm, DBLP:journals/corr/abs-2406-18629stepdpo} and code synthesis~\citep{DBLP:journals/corr/abs-2406-06887plum}. However, its impact on improving Text-to-SQL models beyond the fine-tuning stage remains uncertain although the database can provide accurate feedback to help us construct pair-wise training data for DPO. Therefore, further research is needed to evaluate the potential benefits and limitations of DPO in this context, as well as to explore alternative strategies that might better leverage preference learning to improve model performance in generating accurate SQL queries.

% To address these limitations, this paper introduces a framework that explores how CoT and DPO can affect models' performance for the Text-to-SQL task. Our pipeline consists of three steps. First, for each training sample in benchmarks, we use a powerful LLM (GPT-4o-mini~\citep{openai2024@gpt4-mini} in this work) to generate multiple CoT reasoning paths, outlining the step-by-step conversion from the question to the desired SQL query. Next, we perform supervised fine-tuning using this CoT-enhanced dataset. Finally, we sample multiple responses from the SFT model and use database feedback to construct a preference dataset, upon which we perform DPO training.

% To achieve comprehensive and convincing experimental results, we conduct extensive experiments on 10 open-source base models, ranging from 6.7B to 15B parameters. We use the widely adopted Spider benchmark~\citep{yu2018spider}, the challenging Bird benchmark~\citep{li2024bird}, and four robustness Text-to-SQL benchmarks: Spider-DK~\citep{gan2021spiderdk}, Spider-Syn~\citep{gan2021spidersyn}, Spider-Realistic~\citep{deng2021spiderrealitic}, and Dr.Spider~\citep{DBLP:conf/iclr/Changdrspider}. 

% Our results reveal that incorporating CoT consistently enhances the DPO stages. Interestingly, when models are trained in a vanilla setting (\emph{i.e.}, generating SQL queries directly without CoT), DPO often leads to a performance drop compared to the original SFT model. 

\textbf{Analysis.} To understand why CoT reasoning is essential for unlocking DPO's potential, we conduct a comprehensive analysis and make three key observations. First, introducing CoT significantly reduces reward hacking during DPO training, ensuring stable and effective performance. Second, CoT enhances DPO's effectiveness as an implicit reward model, improving its ability to discriminate between correct and incorrect responses. Finally, CoT increases DPO's scalability, both in terms of the number of preference data and inference-time sampling budgets.

Our contributions are summarized as follows:
\begin{itemize}[leftmargin=1.0em, itemsep=0.1em, parsep=0em, topsep=0em]
    % \item We conduct extensive preliminary experiments on Text-to-SQL datasets to evaluate the effectiveness of DPO using the standard training pipeline. Contrary to observations in prior studies, we find that DPO does not consistently enhance model performance and can even degrade it.
    % \item We identify that this limitation stems from the absence of Chain-of-Thought (CoT) solutions in existing Text-to-SQL datasets. By augmenting these datasets with synthetic CoT solutions, we achieve, for the first time, stable and significant performance improvements using DPO on the Text-to-SQL task. {\color{red} We believe these findings can provide practitioners with valuable insights on how to effectively incorporate DPO into their Text-to-SQL pipelines.}
    \item  We conduct extensive experiments on Text-to-SQL datasets to evaluate DPO within the standard training pipeline. Contrary to prior studies, we find that DPO does not consistently improve performance and can sometimes degrade it. However, by augmenting these datasets with synthetic CoT solutions, we achieve stable and significant performance improvements with DPO for the first time. As existing works overlook the critical data issue in Text-to-SQL, our findings provide important insights for effectively integrating DPO into Text-to-SQL pipelines.
    \item We also provide a comprehensive analysis to understand why CoT reasoning is essential for DPO. Our findings reveal that incorporating CoT mitigates reward hacking, strengthens discriminative ability, and enhances scalability.
    % \item We believe these insights will inspire researchers and practitioners to design better open-source Text-to-SQL models and advance preference optimization techniques in the Text-to-SQL domain.
    % \item We propose a new Text-to-SQL framework that leverages LLMs to extend existing benchmarks through the synthesis of CoT reasoning paths. Using this enhanced dataset, we perform SFT and utilize feedback from the database to guide subsequent DPO training.
    % \item Extensive experiments across multiple base models and various Text-to-SQL benchmarks demonstrate that our framework significantly and consistently improves performance by incorporating both CoT and DPO. Interestingly, we find that applying DPO in the vanilla setting often results in decreased Text-to-SQL performance, a phenomenon not observed in previous studies.
    % \item We offer a comprehensive analysis explaining why DPO is effective with CoT but not with the vanilla approach. We believe our conclusions can benefit the Text-to-SQL community and broader tasks involving complex reasoning. % Our findings indicate that incorporating CoT alleviates reward hacking during DPO training, leading to better performance.
    % \item We have made the CoT-enhanced datasets and source code publicly available to support further advancements in the text-to-SQL community\footnote{}.
\end{itemize}


% ($e.g.$ code synthesis), Text-to-SQL propose two fundamental challenges. The first is schema linking, that is to map the flexible language description to the exact entity stored in the database. The second is skeleton parsing, which requires model to generate possibly complex query structures that precisely return the records that user intended. 

% Recent progress of Text-to-SQL includes refinement of database prompt construction and post-training techniques such as self-correction, execution-based self-consistency, and multi-agent frameworks. However, the training pipeline of sql-generating model remain fixed, that is directly put into production after supervised fine-tuning.

% Two trending?

% % \begin{figure}[t]
% %   \includegraphics[width=\columnwidth]{figures/Intro.pdf}
% %   \caption{With the help of synthesized Chain-of-Thought reasoning solutions, model learns to correct its own mistakes, and to generate right SQL after DPO.}
% %   \label{fig:Intro}
% % \end{figure}

% Preference learning is a training phase applied after supervised fine-tuning, initially designed to align model for better instruction-following ability and more human-favorable content generation, and becomes a standard procedure to train a general purpose foundation model.

% Preference learning equips model with the capability to tell good answers from bad ones, which is beneficial to its reasoning ability. Recently, several works has already shown that it can boost model's performance in complex reasoning tasks as code completion and solving mathematical problems. 

% On the other hand, there is hardly any successful application of preference learning in the Text-to-SQL domain, although comparison of execution result with ground truth sql provides high quality feedback signal, and the reason is remained unclear. 

% % additional difference comparison
\definecolor{darkgreen}{RGB}{0,150,0}
\begin{table}[h!]
    \centering
\begin{adjustbox}{max width=\columnwidth}
    \begin{tabular}{c|c}
    \toprule
       \textbf{Model}  &  \textbf{EX} \\
    \midrule
        SFT & 61.4 \\
        DPO & 60.6 (\textcolor{darkgreen}{-0.8}) \\
    \bottomrule
    \end{tabular}
\end{adjustbox}
\caption{Performance of fine-tuned Qwen2.5-7B-Instruct on Bird Dev (maj@16). }
\end{table}

% Another trend is that chain-of-thought has strong positive effect on complex reasoning tasks. Chain-of-thought reasoning requires model to output its rationales and intermediate computation step before the direct answer, which significantly improved model zero-shot performance and stability in code and math tasks. 

% Despite the promising evidence from similar fields requires logical reasoning, in Text-to-SQL, chain-of-thought is not prevailing in train-based methods. The reasons are two-folds. First, the chain-of-thought solution is hard to gather or annotate in the real world, since the logic to tackle a specific Text-to-SQL question is not clear as solving procedures in math problems. Second, to fine-tune local models with chain-of-thought Text-to-SQL solutions distilled from aligned large model suffers a sub-optimal performance under generic inference strategies.

% \definecolor{darkgreen}{RGB}{0,150,0}
\begin{table}[h!]
    \centering
\begin{adjustbox}{max width=\columnwidth}
    \begin{tabular}{c|c}
    \toprule
       \textbf{Model}  &  \textbf{EX} \\
    \midrule
        Vanilla & 61.4 \\
        CoT & 60.8 (\textcolor{darkgreen}{-0.6}) \\
    \bottomrule
    \end{tabular}
\end{adjustbox}
\caption{Performance of SFT Qwen2.5-7B-Instruct on Bird Dev (maj@16). \textbf{Vanilla}: SFT on original Bird Train set; \textbf{CoT}: SFT on synthetic chain-of-thought solution generated by gpt-4o-mini-2024-07-18. }
\end{table}

% Surprisingly, by combining them together, we find that the alignment phase will produce a consistent improvement, also result a large margin enhancement in end-to-end performance compared to generic SFT-only non-CoT Text-to-SQL models, in spite of the unsatisfactory performance of preference learning and chain-of-thought reasoning standalone in Text-to-SQL. This result hold true across model family, size, and model specialty, and is robust on well-received Text-to-SQL benchmarks including Bird, Spider and Dr. Spider.

% Furthermore, we investigate the mechanism behind the preference learning in Text-to-SQL, and find that its credit assignment ability is benefited by chain-of-thought reasoning, and its propensity to correct mistake of information redundancy preferences by extensive quantitative and qualitative analysis. 

% Additionally, we show that preference learning enhanced by CoT can make a consistent performance gain for all state-of-the-art Text-to-SQL methods, confirming the potential of its application in real-world Text-to-SQL systems.

% Our contribution is summarized as follows:

% \begin{enumerate}
%     \item We identify chain-of-thought reasoning as the key to the success of direct preference learning in Text-to-SQL task. Also, through careful experiment and analysis, we reveal the reason behind its success, and provide useful best practice guidelines. 
%     \item To deal with scarcity of chain-of-thought solution to Text-to-SQL in the wild, We propose a simple yet powerful pipeline do chain-of-thought version direct preference learning for Text-to-SQL task, which steadily improves end-to-end performance of models, who differ in model family, size and specialty, across various benchmarks and inference-time strategies.
%     \item We also prove that our method is compatible with most state-of-the-art Text-to-SQL pipeline, showcasing that the combination of chain-of-thought reasoning and preference learning has a great potential in the field of Text-to-SQL.
% \end{enumerate}

% The reminders of this paper is organized as follows. 

\section{Related Work}

\section{Related Work}

\subsection{View-Dependent Control}
View-dependent representations have a long history in computer graphics.
In his pioneering work, Rademacher proposed interpolating between \textit{key viewpoints} and associated \textit{key deformations} to manipulate 3D models~\cite{rademacher1999view}.
Other researchers have extended the idea to create 3D animation systems~\cite{10.1111:j.1467-8659.2004.00772.x}, streamline the modeling process~\cite{DBLP:journals/corr/abs-2103-15472}, and integrate physical simulation\cite{koyama2013view}.
Of particular note, Rivers et al.~\cite{rivers25Dcartoonmodels} introduced \textit{2.5D Cartoon Models}, a combination of planar meshes transformed, based upon view angle, so as to appears three dimensional.
Our work draws upon these works but is, to our knowledge, the first work to attempt to use view-dependent techniques to retarget 3D motion onto 2D characters.   

\subsection{Animation from 2D Images}

% output is still 2D
Many researchers have proposed different methods for creating animations from 2D images. Hornung et al.~\cite{Hornung2007anim2Dpicmotion} presented a method to deform a character from a photograph given user-provided joint annotations.
\textit{Toonsynth}~\cite{Dvoroznak18-SIG} and \textit{Neural Puppet}~\cite{poursaeed2020neural} both present methods to create new images of hand-drawn characters from examples.
% output is 3D model
Other researchers have proposed methods of obtaining 3D geometry from 2D sketches~\cite{igarashi2006teddy, Dvoroznak20-SA} and images~\cite{ArtiSketch,weng2019photo}.
% focus on sketches specifically
A number of works have specifically focused on childlike drawings.
Lingens et al.~\cite{lingens2020towards} proposed an evolutionary algorithm for animating children's drawings. 
\textit{MagicToon}~\cite{feng2017magictoon} creates a 3D model from childlike drawings for AR applications.
Similar to our work, Smith et al.~\cite{SmithHodgins} proposed a method for animating childlike drawings using 3D skeletal motion. 
However, the resulting animations are only suitable for use in 2D applications and the type of motions it supports are limited.

Unlike these previous works, our solution can be used in 3D contexts but does not create a 3D model. We instead relying upon a view-dependent formulation of the animated character.

% \section{Direct Preference Optimization}
% As preliminaries, in this section we briefly review direct preference optimization (DPO).
% %\textbf{Learning Objective}

\subsection{Learning Objective}

For Reinforcement Learning with Human Feedback (RLHF) incorporating KL penalty, the learning objective is defined as \citep{ouyang2022rlhf}:
%\begin{adjustbox}{max width=\textwidth}
\begin{align*}
\max_{\pi_{\theta}}\mathbb{E}_{x\sim\mathcal{D},y\sim\pi_{\theta}(y|x)} & \left[r_{\phi}(x,y)\right] \\
 -\beta\mathbb{D}_{\mathrm{KL}} & \left[\pi_{\theta}(y\mid x)\mid\mid\pi_{\mathrm{ref}}(y\mid x)\right]
\end{align*}
%\end{adjustbox}

Here, $\pi_{\mathrm{ref}}$ and $\pi_{\theta}$ represent the initial model distribution and the optimized policy, respectively, while $r_{\phi}$ denotes a parameterized reward model.

Direct Preference Optimization (DPO) reformulates the objective by replacing the reward function with a differentiable form, reflecting the relationship between the optimal policy and the reward function. This leads to a new objective:
%$$\mathcal{L}_{\mathrm{DPO}}(\pi_{\theta};\pi_{\mathrm{ref}})=$$
%\resizebox{\linewidth}{!}{}
%\begin{adjustbox}{max width=\textwidth}
%\begin{align*}
%$$\mathbb{E}_{(x,y_{w},y_{l})\sim\mathcal{D}}\left[\log \sigma\left(\beta\log\frac{\pi_{\theta}(y_{w}\mid x)}{\pi_{\mathrm{ref}}(y_{w}\mid x)}
%-\beta\log\frac{\pi_{\theta}(y_{l}\mid x)}{\pi_{\mathrm{ref}}(y_{l}\mid x)}\right )\right ]$$
%\end{align*}
%\end{adjustbox}
\begin{align*}
\mathbb{E}_{(x,y_{w},y_{l})\sim\mathcal{D}}\left[\log \sigma\left(\beta \log\frac{\pi_{\theta}(y_{w}\mid x)}{\pi_{\mathrm{ref}}(y_{w}\mid x)}\right.\right. \\
\left.\left.  -\beta\log\frac{\pi_{\theta}(y_{l}\mid x)}{\pi_{\mathrm{ref}}(y_{l}\mid x)} \right)\right]
\end{align*}


In this formulation, $y_w$ denotes the chosen output and $y_l$ denotes the rejected output. The parameter $\beta$ controls the penalty strength imposed by the KL divergence. By collecting pairwise preference data, the model can be optimized using supervised fine-tuning, achieving a performance comparable to RLHF \citep{rafailov2024dpo}.

\subsection{Implicit Reward}
%\textbf{Implicit Reward}

DPO implicitly encodes a reward model within the generative model. The reward of a given input-output pair $(x, y)$ can be derived as:

$$r_{\theta}(x, y) = \beta \log \frac{\pi_\theta(y\mid x)}{\pi_{\mathrm{ref}}(y\mid x)}$$

As the DPO training progresses, the optimized model simultaneously becomes a better generative model and a more refined reward model. After training, the implicit reward model, which is derived via conditional likelihood, can be independently used as a reward function \citep{lambert2024rewardbench, chen2024bootstrappinglanguagemodelsdpo}.

%\textbf{Credit Assignment}
\subsection{Credit Assignment}

The implicit reward scores the entire output as a whole. By decomposing the conditional probability that featuring autoregressive generation process, the reward can be re-expressed as:
$$r_{\theta}(x,y) = \sum_{t=1}^{T}\beta \log \frac{\pi_\theta(y_t\mid x,y_{1:t-1})}{\pi_{\mathrm{ref}}(y_t\mid x,y_{1:t-1})}$$

This decomposition allows for the calculation of token-level rewards, as the model score for each token can be identified separately. Although DPO training uses supervision at the full-sequence level, evidence has shown that the model can generalize compositionality to some extent, allowing it to distribute the reward signal to key tokens \citep{rafailov2024dpo2}. This facilitates credit assignment across the output sequence. The resulting dense reward can be utilized for further training and optimization \citep{zhong2024dpo-ppo}.



\section{Pipeline}
% \documentclass[11pt]{article}
% \usepackage{tikz}
% \usetikzlibrary{positioning, shapes.multipart, fit, calc, shapes.geometric, shapes.misc}

% \begin{document}

\begin{figure*}[htbp]

\begingroup
\usetikzlibrary{shapes.geometric}
\usetikzlibrary{arrows.meta}
\usetikzlibrary{backgrounds}
\definecolor{tiffanyblue}{RGB}{129,216,208}
\definecolor{bangdiblue}{RGB}{0,149,182}
\definecolor{kleinblue}{RGB}{0,47,167}
\definecolor{kabuliblue}{RGB}{26,85,153}
\definecolor{purple}{RGB}{138,43,226}

    \centering

      \tikzset{global scale/.style={
    scale=#1,
    every node/.append style={scale=#1}
  }
}


\begin{tikzpicture}[global scale=0.64]
    \newlength{\moduleintervaly}
    \setlength{\moduleintervaly}{1.8em}    
    \newlength{\moduleintervalx}
    \setlength{\moduleintervalx}{-7em}
    \newlength{\blockintervalx}
    \setlength{\blockintervalx}{30em}
    
    \tikzstyle{circlenode}=[draw, circle,minimum size=4pt,inner sep=0, fill=red!30];
    
    \tikzstyle{moduleode}=[draw,minimum height=2.5em,minimum width=23em,inner sep=.0em,thick,rounded corners=.2em, font=\small, scale=0.8];
    
    \tikzstyle{layernode}=[draw,minimum height=1.5em,minimum width=5em,inner sep=.0em,thick,rounded corners=.2em, font=\small,fill=yellow!20];
    
    \tikzstyle{attentionnode}=[draw,minimum height=1.5em,minimum width=5em,inner sep=.0em,thick,rounded corners=.2em, font=\small];

    \tikzstyle{querynode}=[draw,minimum height=1.5em,minimum width=2em,inner sep=.0em,thick,rounded corners=.2em, font=\small];

    \tikzstyle{attnmapnode}=[fill=yellow!20,draw,minimum height=2.5em,minimum width=10em,inner sep=.0em,thick,rounded corners=.2em, font=\small];

    \tikzstyle{partialattnmapnode}=[fill=tiffanyblue!40,draw,minimum height=2.5em,minimum width=12em,inner sep=.0em,thick,rounded corners=.2em, font=\small];

    \tikzstyle{partialattnmapnode1}=[fill=red!20,draw,minimum height=2.5em,minimum width=12em,inner sep=.0em,thick,rounded corners=.2em, font=\small];

    \tikzstyle{GRLattnmapnode}=[fill=yellow!20,draw,minimum height=2.5em,minimum width=8em,inner sep=.0em,thick,rounded corners=.2em, font=\small];

    \tikzstyle{StandardFFNnode}=[fill=red!20,draw,minimum height=2.5em,minimum width=23em,inner sep=.0em,thick,rounded corners=.2em, font=\small];

    \tikzstyle{Concatenode}=[draw,minimum height=1.5em,minimum width=5em,inner sep=.0em,thick,rounded corners=.2em, font=\small];

    \tikzstyle{subspace_block}=[fill=yellow!20,draw,minimum height=2.5em,minimum width=8em,inner sep=.0em,thick,rounded corners=.2em, font=\small];

    \tikzstyle{partial_subspace_block}=[fill=red!20,draw,minimum height=2.5em,minimum width=11.0em,inner sep=.0em,thick,rounded corners=.2em, font=\small];

    \tikzstyle{global_partial_subspace_block}=[fill=pink!20,draw,minimum height=2.5em,minimum width=6.5em,inner sep=.0em,thick,rounded corners=.2em, font=\small];

    \tikzstyle{space_block}=[fill=blue!20,draw,minimum height=2.5em,minimum width=23em,inner sep=.0em,thick,rounded corners=.2em, font=\small];
    
    \tikzstyle{Encoder_block}=[draw,minimum height=8.8*\moduleintervaly,minimum width=20em,inner sep=.0em,thick,rounded corners=.2em, font=\small];

    \tikzstyle{GRL_block}=[draw,minimum height=7em,minimum width=8.8*\moduleintervaly,inner sep=.0em,thick,rounded corners=.2em, font=\small];

    \tikzstyle{Partial_Encoder_block}=[draw,minimum height=7.7*\moduleintervaly,minimum width=25em,inner sep=.0em,thick,rounded corners=.2em, font=\small];
    
    \tikzstyle{Graph_structure_learning}=[draw,minimum height=1.6em,minimum width=2.5em,inner sep=.0em,thick,rounded corners=.2em, font=\small,fill=orange!20];

    
    \tikzstyle{recnode}=[rectangle,rounded corners=5pt,draw,minimum height=1.8em,minimum width=3.5em,inner sep=0em,thick,rounded corners=0.2em,font=\small,fill=orange!20];

    \tikzstyle{recnodewhite}=[rectangle,rounded corners=5pt,minimum height=2.2em,minimum width=3.5em,inner sep=0em,thick,rounded corners=0.2em,font=\small];

    
    % added node
    % \tikzstyle{datanode}=[rectangle,rounded corners=5pt,draw,minimum height=1.8em,minimum width=3.5em,inner sep=0em,thick,rounded corners=0.2em,font=\small,fill=orange!20];
    \tikzstyle{datanode}=[cylinder,rounded corners=5pt,draw,minimum height=1.2em,minimum width=1.5em,inner sep=0em,thick,rounded corners=0.2em,font=\small,fill=orange!20];
    \tikzstyle{modelnode}=[rectangle,rounded corners=5pt,draw,minimum height=1.8em,minimum width=3.5em,inner sep=0em,thick,rounded corners=0.2em,font=\small,fill=green!20];
    \tikzstyle{tasknode}=[rectangle,rounded corners=5pt,draw,minimum height=1.8em,minimum width=3.5em,inner sep=0em,thick,rounded corners=0.2em,font=\small,fill=blue!20];
    \tikzstyle{mergenode}=[rectangle,rounded corners=5pt,draw,minimum height=1.8em,minimum width=15.5em,inner sep=0em,thick,rounded corners=0.2em,font=\small,fill=yellow!20];
    \tikzstyle{mergemodelnode}=[rectangle,rounded corners=5pt,draw,minimum height=1.8em,minimum width=15.5em,inner sep=0em,thick,rounded corners=0.2em,font=\small,fill=green!20];
    
    \tikzstyle{databanknode}=[rectangle,rounded corners=5pt,draw,minimum height=3.0em,minimum width=15.5em,inner sep=0em,thick,rounded corners=0.2em,font=\small,fill=orange!10];
    \tikzstyle{basemodel}=[rectangle,rounded corners=5pt,draw,minimum height=1.8em,minimum width=17.5em,inner sep=0em,thick,rounded corners=0.2em,font=\small,fill=green!18];
    \tikzstyle{loranode}=[rectangle,rounded corners=5pt,draw,minimum height=1.7em,minimum width=3.5em,inner sep=0em,thick,rounded corners=0.2em,font=\small,fill=lime!20];
    \tikzstyle{longloranode}=[rectangle,rounded corners=5pt,draw,minimum height=1.7em,minimum width=13.5em,inner sep=0em,thick,rounded corners=0.2em,font=\small,fill=lime!20];

    \tikzstyle{groupnode}=[cylinder,rounded corners=5pt,draw,minimum height=1.0em,minimum width=3.1em,inner sep=0em,thick,rounded corners=0.2em,font=\small,fill=orange!20];

     \tikzstyle{longdatabanknode}=[cylinder,rounded corners=5pt,draw,minimum height=10.0em,minimum width=1.5em,inner sep=0em,thick,rounded corners=0.2em,font=\small,fill=orange!13];
    \def\nodehsep{3em}
    \def\nodewsep{3.5em}


    % picture a
    \begin{scope}[xshift=0.0in,yshift=0.0in]
        \begin{pgfonlayer}{background}
            \node[anchor=south,minimum height=\nodehsep*5.7,minimum width=19.5em,fill=gray!4,rounded corners=5pt,dotted,draw](backgroundc) at (0, 0) {};
            % \node[anchor=south,minimum height=\nodehsep*6.5,minimum width=39.6em,fill=gray!4,rounded corners=5pt,dotted,draw](backgroundd) at (0,0) {};
        \end{pgfonlayer}

        \node[datanode,anchor=south](data1_01) at ([xshift=-2.2*\nodehsep, yshift=14.8em]backgroundc.south) {En-De};
        \node[datanode,anchor=south](data1_02) at ([yshift=-1.8\nodehsep]data1_01.south) {De-En};
        
        \node[datanode,anchor=west](data2_01) at ([xshift=\nodehsep]data1_01.east) {En-Zh};
        \node[datanode,anchor=west](data2_02) at ([xshift=\nodehsep]data1_02.east) {Zh-En};        
        
        \node[align=center,thick, scale=2] (omit1_01) at ([xshift=\nodehsep]data2_01.center) {...};
        \node[align=center,thick, scale=2] (omit1_02) at ([xshift=\nodehsep]data2_02.center) {...};
        
        \node[datanode,anchor=west](data3_01) at ([xshift=\nodehsep]data2_01.east) {En-Fr};
        \node[datanode,anchor=west](data3_02) at ([xshift=\nodehsep]data2_02.east) {Fr-En};

        \begin{pgfonlayer}{background}
            \node[anchor=south,minimum height=1.4*\nodehsep,minimum width=17.2em,fill=orange!8,rounded corners=5pt,draw](databank1) at ([yshift=-3.9em]data2_01.north) {};
        \end{pgfonlayer}
        

        \node[mergenode,anchor=north](dataselectionc) at ([xshift=0.1\nodehsep,yshift=-1.6\nodehsep]data2_02.south) {Data Selection};
        \draw[thick, ->] (databank1.south) -- ([xshift=-0.12em]dataselectionc.north);
        % \draw[thick, ->] (databank1.south) -- ([xshift=-0.12em]dataselectionc.north);
        % \draw[thick, ->] (databank1.south) -- ([xshift=-0.12em]dataselectionc.north);
        
        \node[basemodel,anchor=north](basemodelc) at ([yshift=-3.1\nodehsep]dataselectionc.north) {Base Model};
        \draw[->,thick] (dataselectionc.south) -- (basemodelc.north);
        
        \node[loranode,anchor=north](lora1c) at ([yshift=-7.8\nodehsep]data1_02.south) {$\mathrm{LoRA}_{1}$};
        \node[loranode,anchor=north](lora2c) at ([yshift=-7.8\nodehsep]data2_02.south) {$\mathrm{LoRA}_{2}$};
        \node[align=center,thick, scale=2] (omit2) at ([xshift=\nodehsep]lora2c.center) {...};
        \node[loranode,anchor=north](lora3c) at ([yshift=-7.8\nodehsep]data3_02.south) {$\mathrm{LoRA}_{N}$}; 
        
        \node(plus1c) at ([yshift=0.6\nodehsep]lora1c.north) {\textbf{\(+\)}};
        \node(plus2c) at ([yshift=0.6\nodehsep]lora2c.north) {\textbf{\(+\)}};
        \node(plus3c) at ([yshift=0.6\nodehsep]lora3c.north) {\textbf{\(+\)}};
        

        \node[tasknode,anchor=north](task1c) at ([yshift=-1.2\nodehsep]lora1c.south) {$\mathrm{Task}_{1}$};
        \node[tasknode,anchor=north](task2c) at ([yshift=-1.2\nodehsep]lora2c.south) {$\mathrm{Task}_{2}$};
        \node[align=center,thick, scale=2] (omit3c) at ([xshift=\nodehsep]task2c.center) {...};
        \node[tasknode,anchor=north](task3c) at ([yshift=-1.2\nodehsep]lora3c.south) {$\mathrm{Task}_{N}$}; 


        \draw[->,thick] (lora1c.south) -- (task1c.north);
        \draw[->,thick] (lora2c.south) -- (task2c.north);
        \draw[->,thick] (lora3c.south) -- (task3c.north);

         
        \node[anchor=south,font=\Large](l3) at ([xshift=0.3em,yshift=-3em]backgroundc.south) {(a) Seperate/Multilingual/Group Training};
        % \node[anchor=north,font=\Large](l3) at ([xshift=0.3em,yshift=-1em]backgroundc.south) {\parbox{6cm}{\centering (a) Separate/Multilingual/ \\ Group Training}};
        % \node[anchor=south,font=\Large](l4) at ([xshift=.5em,yshift=-3em]backgroundd.south) {(d) Model Merging};
    \end{scope}

    % picture b
     \begin{scope}[xshift=4.9in,yshift=0in]
        \begin{pgfonlayer}{background}
            \node[anchor=south,minimum height=\nodehsep*5.7,minimum width=39.6em,fill=gray!4,rounded corners=5pt,dotted,draw](backgroundd) at (0,0) {};
        \end{pgfonlayer}

        \node[datanode,anchor=south](data11d) at ([xshift=-2.2*\nodehsep, yshift=14.8em]backgroundd.south) {En-De};
        \node[datanode,anchor=south](data12d) at ([yshift=-1.8\nodehsep]data11d.south) {De-En};
        
        \node[datanode,anchor=west](data21d) at ([xshift=\nodehsep]data11d.east) {En-Zh};
        \node[datanode,anchor=west](data22d) at ([xshift=\nodehsep]data12d.east) {Zh-En};        
        
        \node[align=center,thick, scale=2] (omit11d) at ([xshift=\nodehsep]data21d.center) {...};
        \node[align=center,thick, scale=2] (omit12d) at ([xshift=\nodehsep]data22d.center) {...};
        
        \node[datanode,anchor=west](data31d) at ([xshift=\nodehsep]data21d.east) {En-Fr};
        \node[datanode,anchor=west](data32d) at ([xshift=\nodehsep]data22d.east) {Fr-En};

        \begin{pgfonlayer}{background}
            \node[anchor=south,minimum height=1.4*\nodehsep,minimum width=17.2em,fill=orange!8,rounded corners=5pt,draw](databank1d) at ([yshift=-3.9em]data21d.north) {};
        \end{pgfonlayer}

        \node[mergenode,anchor=north](dataselectiond) at ([xshift=-3.5*\nodehsep,yshift=-0.2*\nodehsep]databank1d.south) {Group Selection};
        
        \node[basemodel,anchor=north](basemodeldleft) at ([yshift=-0.7\nodehsep]dataselectiond.south) {Base Model};
        \draw[->,thick] (dataselectiond.south) -- (basemodeldleft.north);

        \node[loranode,anchor=north](lora1dleft) at ([xshift=-5.5\nodehsep, yshift=-1.6\nodehsep]basemodeldleft.south) {$\mathrm{LoRA}_{1}$};
        \node[loranode,anchor=north](lora2dleft) at ([yshift=-1.6\nodehsep]basemodeldleft.south) {$\mathrm{LoRA}_{2}$};
        \node[align=center,thick, scale=2] (omit2d) at ([xshift=0.9*\nodehsep]lora2dleft.center) {...};
        \node[loranode,anchor=north](lora3dleft) at ([xshift=5.5\nodehsep,yshift=-1.6\nodehsep]basemodeldleft.south) {$\mathrm{LoRA}_{N_G}$}; 

        \node(plus1dleft) at ([yshift=0.72\nodehsep]lora1dleft.north) {\textbf{\(+\)}};
        \node(plus2dleft) at ([yshift=0.72\nodehsep]lora2dleft.north) {\textbf{\(+\)}};
        \node(plus3dleft) at ([yshift=0.72\nodehsep]lora3dleft.north) {\textbf{\(+\)}};
        
         \node[tasknode,anchor=north](task1dleft) at ([yshift=-1.6\nodehsep]lora1dleft.south) {$\mathrm{Task}_{1}^{\mathrm{enxx}}$};
        \node[tasknode,anchor=north](task2dleft) at ([yshift=-1.6\nodehsep]lora2dleft.south) {$\mathrm{Task}_{2}^{\mathrm{enxx}}$};
        \node[align=center,thick, scale=2] (omit3dleft) at ([xshift=0.9*\nodehsep]task2dleft.center) {...};
        \node[tasknode,anchor=north](task3dleft) at ([yshift=-1.6\nodehsep]lora3dleft.south) {$\mathrm{Task}_{N_G}^{\mathrm{enxx}}$}; 

        \draw[->,thick] (lora1dleft.south) -- (task1dleft.north);
        \draw[->,thick] (lora2dleft.south) -- (task2dleft.north);
        \draw[->,thick] (lora3dleft.south) -- (task3dleft.north);

        \begin{pgfonlayer}{background}
            % \node[anchor=north,minimum height=0.9*\nodehsep,minimum width=17.2em,fill=orange!8,rounded corners=5pt,draw](databankd2) at ([xshift=0.5em,yshift=-0.9em]dataxxen.south) {};
            \node[anchor=south,minimum height=1.9*\nodehsep,minimum width=18.2em,fill=green!8,rounded corners=5pt,draw,dotted,thick](modelmerge) at ([xshift=9.5em,yshift=-6.2em]databank1d.south) {};
        \end{pgfonlayer}

        \node[basemodel,anchor=north](basemodeldright) at ([yshift=1.8*\nodehsep]modelmerge.south) {Base Model};

        
        \node[loranode,anchor=north](lora2dright) at ([yshift=-1.3\nodehsep]basemodeldright.south) {$\mathrm{LoRA}_{2}$};
        \node[loranode,anchor=west](lora1dright) at ([xshift=-6.6\nodehsep]lora2dright.west) {$\mathrm{LoRA}_{1}$};
        \node[align=center,thick, scale=2] (omit2dright) at ([xshift=\nodehsep]lora2dright.center) {...};
        \node[loranode,anchor=east](lora3dright) at ([xshift=6.6\nodehsep]lora2dright.east) {$\mathrm{LoRA}_{N_L}$}; 
        
        \node(plus1dright) at ([yshift=0.6\nodehsep]lora1dright.north) {\textbf{\(+\)}};
        \node(plus2dright) at ([yshift=0.6\nodehsep]lora2dright.north) {\textbf{\(+\)}};
        \node(plus3dright) at ([yshift=0.6\nodehsep]lora3dright.north) {\textbf{\(+\)}};


        \node[mergemodelnode,anchor=north](mergedmodel) at ([yshift=-1.4\nodehsep]lora2dright.south) {Group-wise Merged Model};
        \draw[thick, ->] (modelmerge.south) -- (mergedmodel.north);
        
        \node[tasknode,anchor=north](task1dright) at ([yshift=-4.3\nodehsep]lora1dright.south) {$\mathrm{Task}_1^{\mathrm{xxen}}$};
        \node[tasknode,anchor=north](task2dright) at ([yshift=-4.3\nodehsep]lora2dright.south) {$\mathrm{Task}_2^{\mathrm{xxen}}$};
        \node[align=center,thick, scale=2] (omit3dright) at ([xshift=1.1*\nodehsep]task2dright.center) {...};
        \node[tasknode,anchor=north](task3dright) at ([yshift=-4.3\nodehsep]lora3dright.south) {$\mathrm{Task}_{N_G}^{\mathrm{xxen}}$}; 

        \draw[->,thick] ([xshift=-5.0\nodehsep]mergedmodel.south) -- (task1dright.north);
        \draw[->,thick] (mergedmodel.south) -- (task2dright.north);
        \draw[->,thick] ([xshift=6.0\nodehsep]mergedmodel.south) -- (task3dright.north);

        \draw[->,thick] (databank1d.west) -- (dataselectiond.north) node[midway, left] {En$\rightarrow$XX};
        \draw[->,thick] (databank1d.east) -- ([xshift=0.1*\nodehsep]modelmerge.north) node[midway, right] {XX$\rightarrow$En};

        
        \node[anchor=south,font=\Large](l4) at ([xshift=.5em,yshift=-3em]backgroundd.south) {(b) Direction-aware Training with Group-wise Model Merging};
        
    \end{scope}

    
\end{tikzpicture}
    
    % \vspace{-1em}
    \vspace{-0.5em}
    % \caption{(a) Seperate/Multilingual/Group multilingual training; In seperate  (b) Group-wise model merging}
    \caption{(a)  \textbf{Separate Training ($N$ = $N_L$)}: Each translation task is trained independently using different datasets for different language pairs, with distinct LoRA model weights fine-tuned separately;
    \textbf{Multilingual Training ($N$ = $1$)}: All language pairs are combined to fine-tune a single model with shared LoRA weights;
    \textbf{Group Multilingual Training ($N$ = $N_G$)}: Language pairs are grouped as specified in Table \ref{tab:languages1}-\ref{tab:languages2}, with an adapter trained for each group.
    (b) \textbf{Group-wise model merging}: For XX$\rightarrow$En translation, separate training is applied to each language pair. For En$\rightarrow$XX translation, group training is applied, where different tasks share LoRA weights within language groups.}
    \label{fig:architecture}
    \vspace{-1.1em}
% \end{figure*}
\endgroup

\label{fig:comparison-merge}
\end{figure*}

% \end{document}


\section{Experiment Setup}
\subsection{Datasets}

\textbf{Common Benchmark:} Spider~\citep{yu2018spider} is a widely used Text-to-SQL dataset comprising a training set of 7,000 samples and a development set of 1,034 samples. This dataset covers 200 databases across 138 diverse domains.

\textbf{Challenging Benchmark:} Bird~\citep{li2024bird} presents a more challenging benchmark, featuring a training set of 9,428 samples and a development set of 1,534 samples. It includes 95 large databases across 37 professional domains. In contrast to Spider, Bird offers a more realistic scenario that aligns with real-world applications. 
% Specifically, Bird emphasizes real, large-scale, and noisy database contents.  requires reasoning using external knowledge.

% Additionally, BIRD incorporates external knowledge (EK) for specific samples to support the generation of accurate SQL queries. By default, EK is utilized in our experiments.

% The majority of our experiments are done on the Bird benchmark, since its complexity best fits real-world scenarios. We also test our pipeline on the following prevailing Text-to-SQL test sets, to show the robustness of proposed method.

\textbf{Robustness Benchmarks:} Spider-DK~\citep{gan2021spiderdk}, Spider-Syn~\citep{gan2021spidersyn}, and Spider-Realistic~\citep{deng2021spiderrealitic} are three widely adopted robustness evaluation sets that modify the development set of Spider to simulate real-world scenarios. Another significant derivative, Dr.Spider~\citep{DBLP:conf/iclr/Changdrspider}, creates 17 distinct robustness evaluation sets by comprehensively perturbing the Spider development set across 3 aspects: questions, databases, and SQL queries. 
% In practice, we train the Text-to-SQL models using Spider and evaluate them to these robustness benchmarks.

% Due to the limit of pages, in the main context, we mainly show the evaluation result on the Bird benchmark and defer evaluation on other datasets to Appendix~\ref{}. 

\subsection{Evaluation Metrics}
For all benchmarks, we use the execution accuracy (EX) metric~\citep{yu2018spider} to evaluate the accuracy of the model's predictions. EX measures whether the predicted and gold SQL queries produce identical execution results on the given database. For Spider's development set, we additionally employ a more robust metric, test-suite accuracy (TS)~\citep{DBLP:conf/emnlp/ZhongYK20testsuite}, which extends EX by evaluating whether the predicted SQL query consistently passes the EX evaluation across multiple test-suite database instances.

\subsection{Inference Strategy}
Given a Text-to-SQL model, we explore three inference strategies: \textbf{(a) Greedy}: Use greedy decoding with a temperature of 0 to generate a response. \textbf{(b) Pass@1}: Sample a response with a temperature of 1.0. To ensure stability, we repeat this process 16 times and report the average scores. \textbf{(c) Maj@K}: Sample $K$ responses with a temperature of 1.0 and conduct majority voting based on the execution results of the predicted SQL queries. The final prediction is selected from the most-voted group.

% sampling one response with a temperature of 1.0, and (c) sampling multiple responses with a temperature of 1.0.


% \textbf{Pass@1} is to sample $K$ times for each input. Then we calculate average pass rate over all roll-outs as the score of this test case.

% \textbf{Major Voting} is to sample $K$ times for each input, then group roll-outs with same execution result together. Return a random SQL in the largest group as final output. Ties are broken arbitrarily. 

% Since the EX performance is related to decoding and selection strategies in inference time, in our experiments, we refer to the EX metric under each strategy as \textbf{greedy}, \textbf{pass@1}, \textbf{maj@K}, respectively. 

\subsection{Implementation Details}
We select 10 base models from various model families, including Deepseek~\citep{bi2024deepseekllm, guo2024deepseekcoder}, Qwen~\citep{yang2024qwen, hui2024qwencoder}, Llama~\cite{dubey2024llama3, roziere2023codellama}, and CodeS~\citep{li2024codes}. These models cover different specialties (general-purpose, code- or SQL-specific) and range from 6.7B to 15B parameters. For each LLM, we conduct SFT and DPO using either the original training dataset (Vanilla) or the CoT-enhanced dataset (Syn CoT). More implementation details are listed in Appendix~\ref{apx:imp_details}. Details about training data can be found in Appendix~\ref{apx:datadetails}.

% We report the best checkpoint according to the given metric on Dev. 

\section{Experimental Results}

\subsection{Main Results}
%\begin{table*}[t!]
%    \centering
%    \begin{tabular}{ c c c c }
%        \hline
%        \multirow{3}{*}{\textbf{Model}} & \multicolumn{3}{c|}{\textbf{Bird Dev}} \\ \cline{2-4}
%         & greedy & pass@1 & maj@1k \\ \cline{2-4}
%         & SFT \quad DPO & SFT \quad DPO & SFT \quad DPO \\ \hline
%        \multicolumn{4}{|c|}{\textbf{General Models}} \\ \hline
%        Deepseek-coder-7b & 58 & 62 (\textcolor{red}{+4.0}) & \\ \hline
%        \multicolumn{4}{|c|}{\textbf{Coder Models}} \\ \hline
%         & & & \\ \hline
%        \multicolumn{4}{|c|}{\textbf{SQL-Specialized Models}} \\ \hline
%         & & & \\ \hline
%    \end{tabular}
%    \caption{Model Comparison Table}
%    \label{tab:model-comparison}
%\end{table*}
\iffalse
\begin{table*}[t!]
    \centering
    \begin{tabular}{c c c c}
        \hline
        \multirow{3}{*}{\textbf{Model}} & \multicolumn{3}{c}{\textbf{Bird Dev}} \\ \cline{2-4}
         & Greedy & Pass@1 & Maj@1k \\ \cline{2-4}
         & SFT \quad DPO & SFT \quad DPO & SFT \quad DPO \\ \hline
        \multicolumn{4}{c}{\textbf{General Models}} \\ \hline
        Deepseek-coder-7b & 58 \quad 62 (\textcolor{red}{+4.0}) & 2 & \\ \hline
        \multicolumn{4}{c}{\textbf{Coder Models}} \\ \hline
         & & & \\ \hline
        \multicolumn{4}{c}{\textbf{SQL-Specialized Models}} \\ \hline
         & & & \\ \hline
    \end{tabular}
    \caption{Model Comparison Table}
    \label{tab:model-comparison}
\end{table*}
\fi


% Single Version
\iffalse
\begin{table*}[t!]
    \centering
    \begin{tabular}{c | c c | c c | c c}
        \toprule
        \multirow{3}{*}{\textbf{Model}} & \multicolumn{6}{c}{\textbf{Bird Dev}} \\ \cline{2-7}
         & \multicolumn{2}{c}{Greedy} & \multicolumn{2}{c}{Pass@1} & \multicolumn{2}{c}{Maj@16} \\ \cline{2-7}
         & SFT & DPO & SFT & DPO & SFT & DPO \\ \midrule
        \multicolumn{7}{c}{\textbf{General Models}} \\ \midrule
        Deepseek-llm-7b & 54.3 &  & 51.9  & & 59.1 & \\ 
        Meta-Llama-3.1-8B-Instruct & 58.8 & 61.2 (\textcolor{red}{+2.4}) & 57.5  & & 62.8 & \\ 
        Qwen2.5-chat-7b & 57.4 & 61.9 (\textcolor{red}{+4.5}) & 54.8  & 59.2 (\textcolor{red}{+4.4}) & 63.0 & 64.9 (\textcolor{red}{+1.9}) \\ 
        Qwen2.5-chat-14b & 63.2 & 65.3 (\textcolor{red}{+2.1}) & 61.8  & 64.7 (\textcolor{red}{+2.9}) &  65.4 & 67.1 (\textcolor{red}{+1.7})\\ \midrule
        \multicolumn{7}{c}{\textbf{Coder Models}} \\ \midrule
        Deepseek-coder-7b & 61.5 & 63.8 (\textcolor{red}{+2.3}) & 59.9 & 62.3 (\textcolor{red}{+4.5}) & 64.3 & 65.4 (\textcolor{red}{+1.1}) \\ 
        CodeLlama-instruct-7b & 58.2 & 61.3 (\textcolor{red}{+3.1}) & 56.9 & 60.4 (\textcolor{red}{+3.5}) & 60.2 & 61.9 (\textcolor{red}{+1.7})\\ 
        CodeLlama-instruct-13b & 62.0 & & 59.8  & & 63.6 & 65.8 (\textcolor{red}{+2.2})\\
        Qwen2.5-Coder-7b & 60.8 & 63.4 (\textcolor{red}{+2.6}) & 59.1 & 62.8 (\textcolor{red}{+3.7}) & 62.5 & 64.1 (\textcolor{red}{+1.6})\\ \midrule
        \multicolumn{7}{c}{\textbf{SQL-Specialized Models}} \\ \midrule
        CodeS-7b & 56.7 & 57.5 (\textcolor{red}{+0.8}) & 54.2 & 55.3 (\textcolor{red}{+1.1}) & 60.2 & 61.7 (\textcolor{red}{+1.5})\\ 
        CodeS-15b & 58.6 & 61.1 (\textcolor{red}{+2.5}) & 56.6 & 60.5 (\textcolor{red}{+3.9}) & 62.4 & 63.2 (\textcolor{red}{+0.8})\\
        \bottomrule
    \end{tabular}
    \caption{Model Comparison Table}
    \label{tab:model-comparison}
\end{table*}
\fi

\iffalse
% Non-CoT and CoT in same page
\definecolor{darkgreen}{RGB}{0,150,0}
\begin{table*}[t!]
    \centering
    \begin{tabular}{c | c | c c | c c | c c}
        \toprule
         & \multirow{3}{*}{\textbf{Model}} & \multicolumn{6}{c}{\textbf{Bird Dev}} \\ \cline{3-8}
         & & \multicolumn{2}{c}{Greedy} & \multicolumn{2}{c}{Pass@1} & \multicolumn{2}{c}{Maj@16} \\ \cline{3-8}
         & & SFT & DPO & SFT & DPO & SFT & DPO \\ \midrule
        \multicolumn{8}{c}{\textbf{General Models}} \\ \midrule
        
        & Deepseek-llm-7b-chat & 51.8 & 51.2 (\textcolor{darkgreen}{-0.6})& 47.9  & 49.1 (\textcolor{red}{+1.3})& 54.5 & 54.3 (\textcolor{darkgreen}{-0.3}) \\ 
        & Meta-Llama-3.1-8B-Instruct & 59.0 & 59.8 (\textcolor{red}{+0.8}) & 56.1  & 57.2 (\textcolor{red}{+1.1}) & 61.4 & 60.8 (\textcolor{darkgreen}{-0.6}) \\ 
        & Qwen2.5-7B-Instruct & 58.8 & 59.0 (\textcolor{red}{+0.2}) & 55.1  & 55.7 (\textcolor{red}{+0.6}) & 61.4 & 60.6 (\textcolor{darkgreen}{-0.8}) \\ 
        \multirow{-4}{*}{\textbf{Vanilla}} & Qwen2.5-14B-Instruct & 64.3 & 63.5 (\textcolor{darkgreen}{+0.8}) & 62.3  & 62.6 (\textcolor{red}{+0.3}) &  64.6 & 65.1 (\textcolor{red}{+0.5}) \\
        
        \rowcolor{cyan!20}
        & Deepseek-llm-7b-chat & 54.3 & 55.9 (\textcolor{red}{+1.6})& 51.9  & 54.8 (\textcolor{red}{+2.9})& 59.1 & 61.0 (\textcolor{red}{+1.9}) \\ 
        \rowcolor{cyan!20}
        & Meta-Llama-3.1-8B-Instruct & 56.8 & 61.2 (\textcolor{red}{+4.4}) & 57.5  & 59.0 (\textcolor{red}{+1.5}) & 60.2 & 61.9 (\textcolor{red}{+1.7}) \\ 
        \rowcolor{cyan!20}
        & Qwen2.5-7B-Instruct & 57.4 & 61.9 (\textcolor{red}{+4.5}) & 54.8  & 59.2 (\textcolor{red}{+4.4}) & 63.0 & 64.9 (\textcolor{red}{+1.9}) \\ 
        \rowcolor{cyan!20}
        \multirow{-4}{*}{\textbf{CoT}} & Qwen2.5-14B-Instruct & 63.2 & 65.3 (\textcolor{red}{+2.1}) & 61.8  & 64.7 (\textcolor{red}{+2.9}) &  65.4 & 67.1 (\textcolor{red}{+1.7}) \\
        
        \midrule
        \multicolumn{8}{c}{\textbf{Coder Models}} \\ \midrule
        
        & Deepseek-coder-6.7b-instruct & 60.6 & 60.9 (\textcolor{red}{0.3}) & 56.9 & 58.8 (\textcolor{red}{+1.9}) & 59.8 & 61.0 (\textcolor{red}{+1.2}) \\
        & CodeLlama-7b-Instruct-hf & 57.0 & 55.7 (\textcolor{darkgreen}{-1.3}) & 54.3 & 55.5 (\textcolor{red}{+1.2}) & 59.1 & 58.5 (\textcolor{darkgreen}{-0.6})\\ 
        & CodeLlama-13b-Instruct-hf & 60.0 & 60.2 (\textcolor{red}{+0.2})& 56.7  & 57.9 (\textcolor{red}{+1.2})& 61.9 & 62.0 (\textcolor{red}{+0.1})\\
        \multirow{-4}{*}{\textbf{Vanilla}} & Qwen2.5-Coder-7B-Instruct & 61.6 & 61.3 (\textcolor{darkgreen}{-0.3}) & 59.4 & 60.6 (\textcolor{red}{+1.2}) & 61.3 & 62.7 (\textcolor{red}{+1.4})\\
        
        \rowcolor{cyan!20}
        & Deepseek-coder-6.7b-instruct & 61.5 & 63.8 (\textcolor{red}{+2.3}) & 59.9 & 62.3 (\textcolor{red}{+4.5}) & 64.3 & 65.4 (\textcolor{red}{+1.1}) \\
        \rowcolor{cyan!20}
        & CodeLlama-7b-Instruct-hf & 58.2 & 61.3 (\textcolor{red}{+3.1}) & 56.9 & 60.4 (\textcolor{red}{+3.5}) & 60.2 & 61.9 (\textcolor{red}{+1.7})\\ 
        \rowcolor{cyan!20}
        & CodeLlama-13b-Instruct-hf & 62.0 & 63.9 (\textcolor{red}{+1.9})& 59.8  & 62.5 (\textcolor{red}{+2.7})& 63.6 & 65.8 (\textcolor{red}{+2.2})\\
        \rowcolor{cyan!20}
        \multirow{-4}{*}{\textbf{CoT}} & Qwen2.5-Coder-7B-Instruct & 60.8 & 63.4 (\textcolor{red}{+2.6}) & 59.1 & 62.8 (\textcolor{red}{+3.7}) & 62.5 & 64.1 (\textcolor{red}{+1.6})\\ \midrule
        
        \multicolumn{8}{c}{\textbf{SQL-Specialized Models}} \\ \midrule
        
        & CodeS-7b & 56.8 & 56.6 (\textcolor{darkgreen}{-0.2}) & 53.7 & 54.6 (\textcolor{red}{+0.9}) & 58.1 & 58.0 (\textcolor{darkgreen}{-0.1})\\ 
        \multirow{-2}{*}{\textbf{Vanilla}} & CodeS-15b & 58.3 & 58.2 (\textcolor{darkgreen}{-0.1}) & 55.6 & 56.2 (\textcolor{red}{+0.6}) & 60.2 & 59.1 (\textcolor{darkgreen}{-1.1})\\
        
        \rowcolor{cyan!20}
        & CodeS-7b & 56.7 & 57.5 (\textcolor{red}{+0.8}) & 54.2 & 55.3 (\textcolor{red}{+1.1}) & 60.2 & 61.7 (\textcolor{red}{+1.5})\\ 
        \rowcolor{cyan!20}
        \multirow{-2}{*}{\textbf{CoT}} & CodeS-15b & 58.6 & 61.1 (\textcolor{red}{+2.5}) & 56.6 & 60.5 (\textcolor{red}{+3.9}) & 62.4 & 63.2 (\textcolor{red}{+0.8})\\
        \bottomrule
    \end{tabular}
    \caption{Model Comparison Table}
    \label{tab:model-comparison}
\end{table*}
\fi

% additional difference comparison
\definecolor{darkgreen}{RGB}{0,150,0}
\begin{table*}[t!]
    \centering
\begin{adjustbox}{max width=0.85\textwidth}
    \begin{tabular}{c  c | c c | c c | c c | c}
        \toprule
         & \multirow{3}{*}{\textbf{Model}} & \multicolumn{6}{c}{\textbf{Bird Dev}} & \\ \cline{3-9}
         & & \multicolumn{2}{c}{Greedy} & \multicolumn{2}{c}{Pass@1} & \multicolumn{2}{c |}{Maj@16} & \multirow{2}{*}{$\Delta$EX} \\ \cline{3-8}
         & & SFT & DPO & SFT & DPO & SFT & DPO & \\ \midrule
        \multicolumn{9}{c}{\textbf{General Models}} \\ \midrule
        
        & Deepseek-llm-7b-chat & 51.8 & 51.2 (\textcolor{darkgreen}{-0.6})& 47.9  & 49.1 (\textcolor{red}{+1.3})& \text{54.5} & 54.3 (\textcolor{darkgreen}{-0.3}) & -\\ 
        & Meta-Llama-3.1-8B-Instruct & 59.0 & 59.8 (\textcolor{red}{+0.8}) & 56.1  & 57.2 (\textcolor{red}{+1.1}) & \text{61.4} & 60.8 (\textcolor{darkgreen}{-0.6}) & -\\ 
        & Qwen2.5-7B-Instruct & 58.8 & 59.0 (\textcolor{red}{+0.2}) & 55.1  & 55.7 (\textcolor{red}{+0.6}) & \text{61.4} & 60.6 (\textcolor{darkgreen}{-0.8}) & -\\ 
        \multirow{-4}{*}{\textbf{Vanilla}} & Qwen2.5-14B-Instruct & 64.3 & 63.5 (\textcolor{darkgreen}{-0.8}) & 62.3  & 62.6 (\textcolor{red}{+0.3}) &  64.6 & \text{65.1} (\textcolor{red}{+0.5}) & - \\
        
        \rowcolor{cyan!20}
        & Deepseek-llm-7b-chat & 54.3 & 55.9 (\textcolor{red}{+1.6})& 51.9  & 54.8 (\textcolor{red}{+2.9})& 59.1 & \text{61.0} (\textcolor{red}{+1.9}) & 54.5 $\rightarrow$ 61.0 (\textbf{\textcolor{red}{+6.5}})\\ 
        \rowcolor{cyan!20}
        & Meta-Llama-3.1-8B-Instruct & 56.8 & 61.2 (\textcolor{red}{+4.4}) & 57.5  & 59.0 (\textcolor{red}{+1.5}) & 60.2 & \text{61.9} (\textcolor{red}{+1.7}) & 61.4 $\rightarrow$ 61.9 (\textbf{\textcolor{red}{+0.5}}) \\ 
        \rowcolor{cyan!20}
        & Qwen2.5-7B-Instruct & 57.4 & 61.9 (\textcolor{red}{+4.5}) & 54.8  & 59.2 (\textcolor{red}{+4.4}) & 63.0 & \text{64.9} (\textcolor{red}{+1.9}) & 61.4 $\rightarrow$ 64.9 (\textbf{\textcolor{red}{+3.5}})\\ 
        \rowcolor{cyan!20}
        \multirow{-4}{*}{\textbf{Syn CoT}} & Qwen2.5-14B-Instruct & 63.2 & 65.3 (\textcolor{red}{+2.1}) & 61.8  & 64.7 (\textcolor{red}{+2.9}) &  65.4 & \text{67.1} (\textcolor{red}{+1.7}) & 64.6 $\rightarrow$ 67.1 (\textbf{\textcolor{red}{+2.5}})\\
        
        \midrule
        \multicolumn{9}{c}{\textbf{Coder Models}} \\ \midrule
        
        & Deepseek-coder-6.7b-instruct & 60.6 & 60.9 (\textcolor{red}{+0.3}) & 56.9 & 58.8 (\textcolor{red}{+1.9}) & 59.8 & \text{61.0} (\textcolor{red}{+1.2}) & - \\
        & CodeLlama-7b-Instruct-hf & 57.0 & 55.7 (\textcolor{darkgreen}{-1.3}) & 54.3 & 55.5 (\textcolor{red}{+1.2}) & \text{59.1} & 58.5 (\textcolor{darkgreen}{-0.6}) & - \\ 
        & CodeLlama-13b-Instruct-hf & 60.0 & 60.2 (\textcolor{red}{+0.2})& 56.7  & 57.9 (\textcolor{red}{+1.2})& 61.9 & \text{62.0} (\textcolor{red}{+0.1}) & - \\
        \multirow{-4}{*}{\textbf{Vanilla}} & Qwen2.5-Coder-7B-Instruct & 61.6 & 61.3 (\textcolor{darkgreen}{-0.3}) & 59.4 & 60.6 (\textcolor{red}{+1.2}) & 61.3 & \text{62.7} (\textcolor{red}{+1.4}) & - \\
        
        \rowcolor{cyan!20}
        & Deepseek-coder-6.7b-instruct & 61.5 & 63.8 (\textcolor{red}{+2.3}) & 59.9 & 62.3 (\textcolor{red}{+4.5}) & 64.3 & \text{65.4} (\textcolor{red}{+1.1}) & 59.8 $\rightarrow$ 65.4 (\textbf{\textcolor{red}{+5.6}}) \\
        \rowcolor{cyan!20}
        & CodeLlama-7b-Instruct-hf & 58.2 & 61.3 (\textcolor{red}{+3.1}) & 56.9 & 60.4 (\textcolor{red}{+3.5}) & 60.2 & \text{61.9} (\textcolor{red}{+1.7}) & 59.1 $\rightarrow$ 61.9 (\textbf{\textcolor{red}{+2.8}}) \\ 
        \rowcolor{cyan!20}
        & CodeLlama-13b-Instruct-hf & 62.0 & 63.9 (\textcolor{red}{+1.9})& 59.8  & 62.5 (\textcolor{red}{+2.7})& 63.6 & \text{65.8} (\textcolor{red}{+2.2}) & 61.9 $\rightarrow$ 65.8 (\textbf{\textcolor{red}{+3.9}})\\
        \rowcolor{cyan!20}
        \multirow{-4}{*}{\textbf{Syn CoT}} & Qwen2.5-Coder-7B-Instruct & 60.8 & 63.4 (\textcolor{red}{+2.6}) & 59.1 & 62.8 (\textcolor{red}{+3.7}) & 62.5 & \text{64.1} (\textcolor{red}{+1.6}) & 61.3 $\rightarrow$ 64.1 (\textbf{\textcolor{red}{+2.8}}) \\ \midrule
        
        \multicolumn{9}{c}{\textbf{SQL-Specialized Models}} \\ \midrule
        
        & CodeS-7B & 56.8 & 56.6 (\textcolor{darkgreen}{-0.2}) & 53.7 & 54.6 (\textcolor{red}{+0.9}) & \text{58.1} & 58.0 (\textcolor{darkgreen}{-0.1}) & -\\ 
        \multirow{-2}{*}{\textbf{Vanilla}} & CodeS-15B & 58.3 & 58.2 (\textcolor{darkgreen}{-0.1}) & 55.6 & 56.2 (\textcolor{red}{+0.6}) & \text{60.2} & 59.1 (\textcolor{darkgreen}{-1.1}) & -\\
        
        \rowcolor{cyan!20}
        & CodeS-7B & 56.7 & 57.5 (\textcolor{red}{+0.8}) & 54.2 & 55.3 (\textcolor{red}{+1.1}) & 60.2 & \text{61.7} (\textcolor{red}{+1.5}) & 58.1 $\rightarrow$ 61.7 (\textbf{\textcolor{red}{+2.6}}) \\ 
        \rowcolor{cyan!20}
        \multirow{-2}{*}{\textbf{Syn CoT}} & CodeS-15B & 58.6 & 61.1 (\textcolor{red}{+2.5}) & 56.6 & 60.5 (\textcolor{red}{+3.9}) & 62.4 & \text{63.2} (\textcolor{red}{+0.8}) & 60.2 $\rightarrow$ 63.2 (\textbf{\textcolor{red}{+3.0}})\\
        \bottomrule
    \end{tabular}
\end{adjustbox}
    \caption{Model performance on the Bird development set. \textbf{Vanilla}: SFT and DPO on the original Bird training set; \textbf{Syn CoT}: SFT and DPO on the CoT-enhanced training set; \textbf{$\Delta$EX}: The performance difference in EX between ``Syn CoT + DPO'' and ``Vanilla + SFT'' when using the same base model.}
    \label{tab:model-comparison}
\end{table*}
The results on the Bird benchmark are presented in Table~\ref{tab:model-comparison}. Results on the Spider benchmark and its robustness variants are deferred to Appendix~\ref{apx:spider}. 

% At each training stage, the models were evaluated on the development set using three common inference strategies: Greedy, Pass@1, and Maj@K. 

\textbf{Vanilla Models Struggle to Achieve Performance Gains in the DPO Stage.} 
%Vanilla models showed limited performance improvements during the DPO stage. 
%For both the Greedy and Maj@16 metrics, the gains during the DPO stage hovered around zero.
%Although all models demonstrated some improvement in Pass@1 during the DPO stage compared to SFT, the magnitude of improvement was small. 
%Moreover, Pass@1 consistently emerged as the weakest inference strategy across nearly all models, making it difficult to conclude that model performance was meaningfully enhanced. %
For Greedy and Maj@16, DPO gains are minimal or even negative, and improvements for the Pass@1 strategy are marginal as well. Models showing performance degradation after DPO tend to worsen as training progresses, indicating that directly applying DPO in the vanilla setting may impair performance.

% We also observe that for models experiencing performance degradation during the DPO stage, the degradation tend to worsen as training progressed. Note that, checkpoints from earlier in the training process were selected for reporting, which likely underestimates the extent of the damage caused by DPO training.

\textbf{Models with Synthetic CoT Achieve Stable and Significant Gains in the DPO Stage.}
%Models trained with synthesized CoT data demonstrated consistent and significant performance improvements during the DPO stage. 
These gains are evident across all base models and inference strategies. Even when CoT models outperform vanilla models in the SFT stage, the performance gains from CoT remain consistently significant during the subsequent DPO stage. Moreover, this phenomenon persists even when replacing GPT-4o-mini with much weaker LLMs (e.g., Qwen2.5-1.5B-Instruct) to synthesize CoT solutions, confirming the benefit of the CoT solution style per se for DPO, as discussed in Appendix~\ref{apx:qualityablation}.

% for synthesized CoT models compared to the vanilla setting

% These gains are obvious and can be observed across all base models and inference strategies employed during evaluation. Even when synthesized CoT models already outperformed Vanilla models after the SFT stage, the performance gains achieved during the DPO stage are consistent obvious for synthesized CoT models compared to the Vanilla setting.

\textbf{Synthesized CoT plus DPO Exhibit Higher Performance Ceilings.} 
%In practice, we are often concerned with the maximum achievable performance of a base model on the Text-to-SQL task, regardless of the training stage or inference strategy. 
As shown in the $\Delta$EX column in Table~\ref{tab:model-comparison}, all base models trained with CoT-enhanced data through the SFT and DPO pipeline achieved higher performance ceilings. This indicates that integrating CoT synthesis with DPO is highly effective for the Text-to-SQL task, offering a promising new approach to developing improved Text-to-SQL models. With our straightforward pipeline, Qwen2.5-14B-Instruct achieves the second-best performance on the Bird development set among all open-source models, despite having significantly fewer parameters, as shown in Table~\ref{tab:BestModels}.

% additional difference comparison
\definecolor{darkgreen}{RGB}{0,150,0}
\begin{table}[h!]
    \centering
    % \footnotesize
\begin{adjustbox}{max width=0.8\columnwidth}
    \begin{tabular}{l | c | c | c }
    \toprule
       \textbf{Model}  & \textbf{Date} & \textbf{Size} &  \textbf{EX} \\
    \midrule
        ExSL+granite-34b-code	& Oct 27, 2024	& 34B	& 72.43 \\
    \midrule
    \textbf{Qwen2.5-14B-Instruct} & \multirow{2}{*}{\textbf{Nov 12, 2024}}	& \multirow{2}{*}{\textbf{14B}}	& \multirow{2}{*}{\textbf{67.10}} \\ 
    \textbf{+ Syn CoT + DPO (Ours)}	& & & \\
    \midrule
    XiYanSQL-QwenCoder-32B & Jan 09, 2025 & 32B & 67.01 \\
    \midrule
    MSL-SQL+DeepSeek-V2.5 &	Oct 10, 2024 &	236B &	66.82 \\
    \midrule
    % ByteBrain & July 2, 2024 & 33B & 65.45 \\
    ExSL+granite-20b-code & May 14, 2024 & 20B & 65.38 \\
    \midrule
    SFT CodeS-15B &	Oct 12, 2023 &	15B & 58.47 \\
    \bottomrule
    \end{tabular}
\end{adjustbox}
\caption{Comparing with top open-source model-based Text-to-SQL methods on Bird development set.}
\label{tab:BestModels}
\end{table}

\subsection{Error Analysis}\label{sec:errorAnalysis}
\definecolor{darkgreen}{RGB}{0,150,0}
\begin{table*}[t]
\centering
%\setlength{\tabcolsep}{5pt} % Adjust column padding
%\renewcommand{\arraystretch}{1.3} % Adjust row spacing
%\begin{tabularx}{\textwidth}{lXlccc}
\begin{adjustbox}{max width=0.8\textwidth}
\begin{tabular}{c p{3.5cm} l | c | c | c}
% \small
\toprule
\textbf{Category} & \textbf{Description} & \textbf{Type} & \textbf{Vanilla DPO Fix (\%)} & \textbf{Syn CoT DPO Fix (\%)} & \textbf{$\Delta$(\%)} \\
\midrule
\multirow{1}{*}{External Knowledge} 
    & Neglect of hints
    & \textbf{[A1] EK} 
    & 0.0 (0/3) 
    & \textbf{37.5 (3/8)} 
    & \textcolor{red}{+37.5}\\
\midrule
\multirow{6}{*}{Schema Linking} 
    & \multirow{6}{=}{Fails to match the question with its concerning table and columns}
    & [B1] Table 
    & 13.0 (12/92) 
    & 15.9 (11/69) 
    & \textcolor{red}{+2.9}
    \\
    & & \textbf{[B2] JOIN}
    & 15.6 (12/77)
    & \textbf{32.1 (18/56)}
    & \textcolor{red}{+16.5} \\
    & & [B3] Column
    & 10.3 (7/68)
    & 16.1 (10/62)
    & \textcolor{red}{+5.8} \\
    & & \textbf{[B4] Hallucination}
    & 23.7 (14/59)
    & \textbf{27.2 (28/102)}
    & \textcolor{red}{+3.5} \\
    & & [B5] Condition
    & 16.7 (10/60)
    & 23.2 (16/69) 
    & \textcolor{red}{+6.5} \\
    & & \textbf{[B6] NULL/DISTINCT}
    & 9.7 (3/31)
    & \textbf{40.0 (12/30)}
    & \textcolor{red}{+30.3} \\
\midrule
\multirow{2}{*}{Value Retrieval}
    & \multirow{2}{=}{Mismatch of condition with its storage format}
    & [C1] String/Number 
    & 4.5 (1/22) 
    & 21.1 (4/19) 
    & \textcolor{red}{+16.6}
    \\
    & & \textbf{[C2] Date}
    & 23.1 (6/26)
    & \textbf{30.4 (7/23)}
    & \textcolor{red}{+7.3} \\
\midrule
\multirow{3}{*}{Operation}
    & \multirow{3}{=}{Misunderstands required operation in the question.}
    & [D1] Mathematical Formula 
    & 13.3 (6/45)
    & 18.2 (8/44) 
    & \textcolor{red}{+4.9}
    \\
    & & [D2] Aggregation
    & 6.7 (5/75)
    & 18.2 (12/66) 
    & \textcolor{red}{+11.5} \\
    & & [D3] Complex Operation
    & 5.6 (1/18)
    & 12.5 (3/24) 
    & \textcolor{red}{+6.9} \\
\midrule
\multirow{4}{*}{Information}
    & \multirow{4}{=}{Fails to organize information in the right way}
    & [E1] Redundant/Incomplete 
    & 11.8 (4/34) 
    & 19.2 (5/26) 
    & \textcolor{red}{+7.4}
    \\
    & & \textbf{[E2] Column Sequence}
    & 0 (0/5)
    & \textbf{42.9 (3/7)} & \textcolor{red}{+42.9}
    \\
    & & [E3] ORDER BY/LIMIT
    & 9.1 (1/11)
    & 12.5 (1/8) 
    & \textcolor{red}{+3.4} \\
    & & \textbf{[E4] Format}
    & \textbf{66.7 (2/3)}
    & \textbf{33.3 (2/6)} & \textcolor{darkgreen}{-33.4}
    \\
\midrule
Syntax Error 
    & Inexecutatble SQL 
    & [F1] Syntax 
    & 14.3 (2/14) 
    & 13.3 (2/15)
    & \textcolor{darkgreen}{-1.0}\\
\bottomrule
\end{tabular}
\end{adjustbox}
\caption{Comparison of Vanilla and Syn CoT DPO correction capability across error types on Bird development set (greedy). The base model is Qwen2.5-7B-Instruct.}
\label{table:ErrorAnalysis}
\end{table*}
To understand how DPO affects model generation and identify areas where CoT enhances DPO, we meticulously analyze errors made by Qwen2.5-7B-Instruct using the greedy decoding strategy. As shown in Table~\ref{table:ErrorAnalysis}, errors are classified into our pre-defined categories, with correction rates before and after DPO presented for each. Full explanations and examples of our classification criteria, along with detailed error statistics, are available in Appendix~\ref{apx:analysis}. We find that, except for Syntax Errors, synthesized CoT improves DPO's ability to correct errors across all other categories.

% To better understand how DPO alters the model's generation outcomes and to identify the areas where synthesized CoT improves DPO performance, we analyzed the errors made by the Qwen2.5-7B-Instruct model before and after DPO training under the greedy inference strategy. 

% As shown in Table~\ref{table:ErrorAnalysis}, errors are classified by our defined categories, and the correction rates before and after DPO for each category are presented. Full explanations and examples of our classification criteria, along with detailed error statistics, can be found in Appendix~\ref{apx:analysis}. We can find that, except for Syntax Errors, the synthesized CoT enhanced DPO's ability to correct errors across all other categories.

\textbf{DPO Excels at Correcting Errors Caused by Ignoring Details.} Error types with a correction rate exceeding 25\% during DPO are highlighted in bold. These errors primarily stem from the model's failure to pay sufficient attention to details. For instance, \underline{NULL/DISTINCT} (Fix 40.0\% with CoT) errors arise when the model overlooks missing or duplicate values in the relevant columns, leading to incorrect query results. Similarly, \underline{Column Sequence} (Fix 42.9\% with CoT) errors occur when the model does not return columns in the order specified in the question. 

% \textbf{CoT Provides Logical Guidance to Help DPO Address Complex Errors.} Error types with significant fix rate improvements often benefited from the reasoning introduced by CoT. For example, \underline{JOIN} errors (Fix 32.1\%, +16.5\% with CoT) frequently occur when the required information resides in two tables that need to be joined via a third intermediate table. Without CoT, the model might attempt to directly join the two tables on an incorrect key. CoT explicitly outlines the logic of table joins in its step-by-step reasoning process, making it easier for DPO to identify and correct such errors. In contrast, for intuitive error types such as \underline{Hallucination} (Fix 27.2\%, +3.5\% with CoT) and \underline{Date} (Fix 30.4\%, +7.3\% with CoT), the contribution of CoT is relatively small. These errors typically arise from mismatches between the SQL query and the information provided in the database prompt, which does not need complicated reasoning processes. Thus, vanilla DPO is already capable of addressing effectively.

%when the required information resides in two tables that cannot be directly joined. Instead, an intermediate table is needed to facilitate the join. 

\textbf{CoT Provides Logical Guidance to Enhance DPO's Error Correction.} Error types with significant improvement often benefit from CoT's reasoning nature. For example, \underline{JOIN} errors (Fix 32.1\%, +16.5\% with CoT) frequently occur when the required information resides in two tables that need to be joined via a third intermediate table. Without CoT, the model might attempt to directly join the two tables on an incorrect key. CoT's step-by-step logic clarifies these joins, aiding DPO in error correction. Conversely, for intuitive errors like \underline{Hallucination} (Fix 27.2\%, +3.5\% with CoT) and \underline{Date} (Fix 30.4\%, +7.3\% with CoT), CoT's impact is smaller. These errors stem from mismatches between SQL queries and database information, which vanilla DPO already can address effectively.
More analysis about DPO's effect with or without CoT reasoning can be found in Appendix~\ref{apx:analysis}.

% \begin{itemize}[leftmargin=0.5em]
%     %\setlength{\itemindent}{0em}
%     \item \underline{Table} (Fix~15.9\%) and \underline{Column} (Fix~16.1\%) Errors: These errors stem from the model's failure to correctly identify the tables and columns relevant to the question, a key challenge in schema linking. Many Text-to-SQL approaches specifically focus on improving performance in this area.
%     \item \underline{Complex Operation} (Fix~12.5\%) Errors: These errors involve difficulties in learning complex operations. To address this, some methods introduce intermediate representations, such as TA-SQL\citep{qu2024ta-sql} or Nat-SQL\citep{gan2021nat-sql}, to simplify the learning process for models.
%     \item \underline{Syntax Errors} (Fix~13.3\%): These errors occur when the generated SQL query is syntactically invalid. Recent work has proposed post-generation execution and repair strategies to ensure that the returned SQL is executable.
% \end{itemize}


%To some extent, the strength of DPO is orthogonal to the directions pursued by existing Text-to-SQL methods. We believe that incorporating DPO into the pipelines of these methods could lead to further performance improvements. 


\section{Why Does DPO Benefit From CoT?}

To further investigate why CoT reasoning is essential for DPO, we conduct a series of analyses using Qwen2.5-7B-Instruct on the Bird benchmark. 

\subsection{Better Discriminative Ability}

\begin{figure*}[ht]
    \centering
    % 左边的单图
    \begin{minipage}{0.31\linewidth} % 调整宽度
        \centering
    \includegraphics[width=\columnwidth]{figures/Discriminator.pdf}
  \caption{Comparison of model's discriminative ability during DPO (measured by classification accuracy on curated evaluation set). }
  \label{fig:discriminator}
    \end{minipage}
    \hfill % 让两部分之间留出间隙
    % 右边的双子图
    \begin{minipage}{0.62\linewidth} % 调整宽度
        \centering
          % 子图 (a)
          \begin{subfigure}[ht]{0.49\columnwidth}
            \centering
            \includegraphics[width=\columnwidth]{figures/SelfRewardVanilla.pdf}
            \subcaption{Vanilla}
            \label{fig:selfreward-a1}
          \end{subfigure}
          \hfill
          % 子图 (b)
          \begin{subfigure}[ht]{0.49\columnwidth}
            \centering
            \includegraphics[width=\columnwidth]{figures/SelfRewardCoT.pdf}
            \subcaption{Syn CoT}
            \label{fig:selfreward-b}
          \end{subfigure}
          \caption{Comparison of model's self-assessed performance (average implicit reward policy model given to its own roll-outs) and real performance (EX) on Bird development set (Pass@1) during DPO training.}
          \label{fig:selfreward}
    \end{minipage}
\end{figure*}


\begin{figure}[t]
  \centering
  % 子图 (a)
%\begin{adjustbox}{max width=0.9\columnwidth}
  % \begin{subfigure}[t]{0.31\linewidth}
  %   \centering
  %   \includegraphics[width=\linewidth]{figures/Synthesis.pdf}
  %   \subcaption{Chain-of-Thought Synthesis}
  %   \label{fig:scaleMaj-a}
  % \end{subfigure}
  % \hfill
  % 子图 (b)
  \begin{subfigure}[t]{0.493\linewidth}
    \centering
    \includegraphics[width=\linewidth]{figures/PreferenceData.pdf}
    \subcaption{Preference Data}
    \label{fig:scaleMaj-b}
  \end{subfigure}
  \hfill
  % 子图 (c)
  \begin{subfigure}[t]{0.493\linewidth}
    \centering
    \includegraphics[width=\linewidth]{figures/Inference.pdf}
    \subcaption{Inference-time}
    \label{fig:scaleMaj-c}
  \end{subfigure}
  \caption{Model performance with different sample budget $K$ in each stage (Maj@K). Qwen2.5-7B-Instruct is used as the base model.}
  \label{fig:scaleMaj}
  %  Base model is Qwen2.5-7B-Instruct. Synthesized Chain-of-Thought makes model more scalable in the DPO Stage.
%\end{adjustbox}
\end{figure}

%To ensure a fair comparison of the discriminative capabilities between the Vanilla and CoT models, we designed the following experiment.

We design an evaluation preference dataset to compare the discriminative capabilities of Vanilla and Syn CoT models after DPO. Construction details are in Appendix~\ref{apx:design}. During DPO training, we assess the model's ability to select the correct response by classification accuracy based on implicit reward. The results in Figure~\ref{fig:discriminator} show that \textbf{CoT reasoning enables DPO models to achieve more stable and superior discriminative ability}.

%First, we constructed a test set. Specifically, both SFT models were used to sample from the Dev set. For any data point where both models could generate paired data ($i.e.$, both could simultaneously sample a positive and a negative example), we used the sampling outcomes from the CoT model to randomly construct a preference pair. Subsequently, we extracted the SQL portion of the pair to serve as the preference pair for the Vanilla model, incorporating it into their respective test sets. Through this construction process, we ensured that the databases, questions, and SQLs in the test sets for both models were identical.

%Next, during the DPO training of the models, we do evaluation every 5 steps. For each preference pair in the test set, we computed the implicit reward for both the chosen and rejected responses. If the model assigned a higher score to chosen, we considered the classification to be correct. We then calculated the average classification accuracy across the entire test set. The results, as shown in the Figure~\ref{fig:discriminator}, indicate that the \textbf{CoT model achieves a higher convergent classification accuracy}.

\definecolor{darkgreen}{RGB}{0,150,0}
\begin{table}[htb]
    \centering
\begin{adjustbox}{max width=\columnwidth}
    \begin{tabular}{c | c | c c c }
    \toprule
    Model & SFT & $\beta=0.05$ & $\beta=0.1$ & $\beta=0.2$ \\
    \midrule
    Vanilla & 58.8 & 57.6 (\textbf{\textcolor{darkgreen}{-1.2}}) & 59.0 (\textbf{\textcolor{red}{+0.2}}) & 57.8 (\textbf{\textcolor{darkgreen}{-1.0}})\\
    Syn CoT & 57.4 & 61.6 (\textbf{\textcolor{red}{+4.2}}) & 61.9 (\textbf{\textcolor{red}{+4.5}})& 61.9 (\textbf{\textcolor{red}{+4.5}}) \\
    \bottomrule
    \end{tabular}
\end{adjustbox}
\caption{DPO performance across different $\beta$ values (greedy). The base model is Qwen2.5-7B-Instruct.}
\label{table:RobBeta}
\end{table}

\subsection{Better Training Stability}
From a different perspective, we examine the discrepancy between the models' self-evaluation rewards for their generated outputs and the actual execution accuracy on the development set. Specifically, at the end of each epoch during DPO training, we sample responses using the current checkpoint and calculate the average implicit reward, as defined in Appendix~\ref{apx:implicit_reward}. Figure~\ref{fig:selfreward} illustrates the results for both the Syn CoT and Vanilla models.

\textbf{Vanilla Model is Susceptible to Reward Hacking.} As shown in Figure~\ref{fig:selfreward} (a), during DPO training, the performance of the Vanilla model initially peaks but then drops. Despite this drop, its self-reward scores continue to rise, indicating that DPO's underlying reward model mistakenly believes its outputs are improving, thereby demonstrating the phenomenon of reward hacking.

% During DPO training, the performance of the Vanilla model quickly reaches its peak but then declines. Despite this decline in actual performance, the model's self-reward scores it assigns to its own outputs continue to increase throughout the training process. This indicates that the Vanilla model mistakenly believes it is producing increasingly better results, even though its true performance deteriorates.

\textbf{In Contrast, The Self-reward of Syn CoT Model Reflects Its Actual Performance.} As shown in Figure~\ref{fig:selfreward} (b), In the early stages of DPO training, as the model's capabilities improve, its self-reward scores also increase. In the later stages, when the model's performance saturates, the self-reward scores stabilize rather than exhibiting the reward-hacking observed in the Vanilla model. This suggests that the Syn CoT model can recognize when its outputs are no longer improving and avoids overestimating its performance.

% the later stages of 
\textbf{Output Statistics.} We compare the statistical changes in the outputs of the two models (Vanilla vs. Syn CoT) after DPO in Table~\ref{tab:RewardHackingStat}. For the Vanilla model, the average output length increased significantly (+26.6\%) after DPO, accompanied by a substantial rise in the proportion of invalid SQL queries (+17.6\%).
In contrast, the Syn CoT model exhibited minimal changes in SQL length (+2\%), and a reduced non-executable rate (-2.5\%).

% additional difference comparison
\definecolor{darkgreen}{RGB}{0,150,0}
\begin{table}[htb]
    \centering
\begin{adjustbox}{max width=0.8\columnwidth}
    \begin{tabular}{c | c  c  c  c }
    \toprule
       \multirow{2}{*}{\textbf{Metric}} & \multicolumn{2}{c}{\textbf{Vanilla}} & \multicolumn{2}{c}{\textbf{Syn CoT}} \\
       & SFT & DPO & SFT & DPO \\
    \midrule
        \textbf{SQL Character Number} & 173 & 219 & 178 & 182 \\
       \textbf{Non-executable SQL(\%)} & 2.0 & 19.6 & 7.6 & 5.1 \\
    \bottomrule
    \end{tabular}
\end{adjustbox}
\caption{Statistics on predicted SQL queries.}
\label{tab:RewardHackingStat}
\end{table}

\textbf{A Case of Reward Hacking in Vanilla + DPO.} Furthermore, we show a classical non-executable output generated by the Vanilla DPO model in Figure~\ref{fig:RewardHackingCase}. In this example, the model assigns an exceptionally high reward to an incorrect token, even though this token does not appear in the training dataset. This indicates a clear occurrence of reward hacking. Other prominent reward hacking patterns are presented in Appendix~\ref{sec:RHPatterns}.

\begin{figure}[htbp]
  \centering
  \includegraphics[width=0.9\columnwidth]{figures/RewardHacking.pdf}
  \caption{A case of reward hacking from Bird development set. The background color indicates token rewards.}
  \label{fig:RewardHackingCase}
\end{figure}

\textbf{Syn CoT Enhances DPO Robustness to Hyperparameter $\beta$.} As shown in Table~\ref{table:RobBeta}, varying the hyperparameter $\beta$ reveals that the DPO training of the vanilla model is highly sensitive to changes in $\beta$. In contrast, the Syn CoT model demonstrates notable robustness, consistently delivering strong performance despite these perturbations.

\subsection{Better Scalability}

% In our proposed pipeline, the sample budget at each stage can be adjusted. 
In this section, we discuss the scalability of our proposed pipeline to provide guidance for best practices. We report the Maj@K results in Figure~\ref{fig:scaleMaj}, and results for Greedy and Pass@16 are provided in Appendix~\ref{apx:scaling}.
% \textbf{CoT Synthesis (Figure~\ref{fig:scaleMaj-a}).} As the number of synthesized CoTs increases, model performance initially improves and then declines. Notably, synthesizing too few (\emph{e.g.}, 1) or too many CoTs (\emph{e.g.}, 64) negatively impacts performance for SFT. Furthermore, regardless of the sample budget, DPO performance consistently exceeds that of SFT.
% Increasing the number of synthesized reasoning paths allows the DPO to achieve further gains when SFT starts to saturate, ultimately reaching a higher peak performance.
\textbf{Preference Data Collection (Figure~\ref{fig:scaleMaj-b}).} For vanilla models, collecting additional preference data does not enhance performance during the DPO stage. In contrast, increasing the sample budget for Syn CoT models consistently results in continuous performance improvements with DPO. % However, performance gains tend to saturate eventually, primarily due to the diminishing growth rate of preference data.
\textbf{Inference-time Computation (Figure~\ref{fig:scaleMaj-c}).} 
%For both synthesized CoT and Vanilla SFT models, performance is very similar when the sample budget is small. 
%However, as the budget increases, the synthesized CoT model gradually outperforms the Vanilla model by a margin of approximately 2\%. 
By increasing the inference budget, both vanilla and Syn CoT models show improved performance, although the gains eventually stabilize. When the budget is limited, DPO offers a significant advantage for Syn CoT models over Vanilla models. For instance, ``CoT + DPO'' with 2 samples achieves the same performance as the ``Vanilla + DPO'' with 16 samples.

% \subsection{Case Study}

% \begin{figure*}[t!]
%   \centering
%   \includegraphics[width=\textwidth]{figures/Case.pdf}
%   \caption{A case study of Syn CoT DPO, the question is selected from Bird Dev set. At beginning, SFT model fails to link entity correctly. During DPO, model learns to assign credit to critical wrong tokens, and replace them with right ones. Finally, DPO model is able to generate correct SQL with sound logic.}
%   \label{fig:CaseStudy}
% \end{figure*}

% We wrap up this section by a case study, where we demonstrate how Syn CoT effectively corrects entity linking errors during DPO training.

% As shown in figure \ref{fig:CaseStudy}, The SFT model failed to utilize the data sample provided in the database prompt (underlined in the figure), instead copying conditions directly from the question, resulting in a mismatch with the stored database information.

% During the DPO process, the model assigned low rewards to the erroneous key tokens, indicating its ability to identify specific error locations. Additionally, the model learned to select the correct tokens in such scenarios. Ultimately, after training, the DPO model is capable of generating a coherent rationale and returning the correct SQL query.

% In our previous experiments, database prompt construction followed the approach proposed by \citet{li2024codes} in CodeS. Through error analysis in Section~\ref{sec:errorAnalysis}, we demonstrated that there is a complementary relationship between DPO and existing Text-to-SQL methods. As a proof-of-concept attempt, we selected two representative methods and pluged our module into their pipelines. The results are summarized in the Table~\ref{tab:OtherPipeline}. 
\section{Prospects}
Our trained DPO model can be integrated into existing multi-agent Text-to-SQL frameworks. As a proof of concept, we adopt Qwen2.5-7B-Instruct as the base model and integrate it into two Text-to-SQL frameworks: DTS-SQL~\citep{pourreza2024dts-sql} and C3-SQL~\citep{dong2023c3}. Implementation details are provided in Appendix~\ref{apx:application}. Results in Table~\ref{tab:OtherPipeline} demonstrate that our model could improve performance over existing frameworks.

% Therefore, we can utilize our trained DPO model to replace the Text-to-SQL component in existing multi-agent Text-to-SQL frameworks. As a proof of concept, we adopt Qwen2.5-7B-Instruct as the base model and integrate our trained model into two representative frameworks, DTS-SQL~\citep{pourreza2024dts-sql} and C3-SQL~\citep{dong2023c3}. Implementation details are in Appendix~\ref{apx:application}. Results in Table~\ref{tab:OtherPipeline} show that our model significantly improves performance over existing frameworks. This highlights a new opportunity to enhance existing Text-to-SQL systems.


% \textbf{Existing Text-to-SQL Methods Get Improved After Integrating Our Module.} Specifically, we replace SQL generation models in these methods with trainable substitutes and apply our pipeline. This led to an enhancement in performance.

% Our findings highlight a new opportunity to improve the performance of open-source models, paving the way for the development of safer and more interpretable Text-to-SQL systems.

% additional difference comparison
% \definecolor{darkgreen}{RGB}{0,150,0}
% \begin{table}[h!]
%     \centering
% \begin{adjustbox}{max width=0.6\columnwidth}
%     \begin{tabular}{c  l }
%     \toprule
%         \textbf{Method} & \textbf{EX} \\
%     \midrule
%         \multicolumn{2}{c}{\textbf{Bird}} \\
%     \midrule
%         \makecell{DTS-SQL \\ \citep{pourreza2024dts-sql}} & 47.7 \\
%         \quad $\hookrightarrow$ + CoT & 52.0 (\textbf{\textcolor{red}{+4.3}}) \\
%         \quad $\hookrightarrow$ + CoT + DPO & 55.9 (\textbf{\textcolor{red}{+8.2}})\\
%         % \makecell{MAC-SQL (w/o Refiner) \\ \citep{wang2024mac-sql}} & 53.8 \\
%         % \quad $\hookrightarrow$ + CoT & 54.8 (\textbf{\textcolor{red}{+1.0}}) \\
%         % \quad $\hookrightarrow$ + DPO & 57.0 (\textbf{\textcolor{red}{+2.2}})\\
%     \midrule
%         \multicolumn{2}{c}{\textbf{Spider}} \\
%     \midrule
%         \makecell{C3-SQL \\ \citep{dong2023c3}} & 80.5 \\
%         \quad $\hookrightarrow$ + CoT & 80.9 (\textbf{\textcolor{red}{+0.4}}) \\
%         \quad $\hookrightarrow$ + CoT + DPO & 81.9 (\textbf{\textcolor{red}{+1.4}})\\
%     \bottomrule
%     \end{tabular}
% \end{adjustbox}
% \caption{Performance of representative Text-to-SQL methods when inserted proposed Syn CoT+DPO. Base model for SQL generation is Qwen2.5-7B-Instruct.}
% \label{tab:OtherPipeline}
% \end{table}

\definecolor{darkgreen}{RGB}{0,150,0}
\begin{table}[h!]
    \centering
\begin{adjustbox}{max width=0.9\columnwidth}
    \begin{tabular}{l l l l}
    \toprule
        \multicolumn{2}{c}{\textbf{Bird}} & \multicolumn{2}{c}{\textbf{Spider}} \\
    \midrule
        \textbf{Method} & \textbf{EX} & \textbf{Method} & \textbf{EX} \\
    \midrule
        DTS-SQL & 47.7 & C3-SQL & 80.5\\
        % \quad + CoT & 52.0 (\textbf{\textcolor{red}{+4.3}}) & \quad + CoT & 80.9 (\textbf{\textcolor{red}{+0.4}}) \\
        \quad + Syn CoT + DPO & 55.9 (\textbf{\textcolor{red}{+8.2}}) & \quad + Syn CoT + DPO & 81.9 (\textbf{\textcolor{red}{+1.4}}) \\
    \bottomrule
    \end{tabular}
\end{adjustbox}
\caption{Performance improvements of two Text-to-SQL frameworks when incorporating our trained models.}
\label{tab:OtherPipeline}
\end{table}
\section{Conclusion}
In this work, we demonstrate, for the first time, consistent and effective performance improvements using DPO on the Text-to-SQL task, enabled by synthetic CoT solutions. Through comprehensive experiments and detailed analyses, we show that CoT reasoning is essential for unlocking DPO's potential, as it mitigates reward hacking, enhances the model's discriminative capabilities, and improves scalability during DPO training. We believe these findings will inspire researchers and practitioners to develop more robust and effective open-source Text-to-SQL models.

% Chain-of-Thought (CoT) synthesis into the training process of Text-to-SQL models. CoT enables the model to achieve consistent improvements through Direct Preference Optimization (DPO). Through comprehensive experiments and detailed analyses, we further find that the CoT provides DPO with a more accurate implicit reward model, a more stable training process less prone to reward hacking, and good scalability with respect to sample budgets. Finally, we apply our trained model to two existing Text-to-SQL frameworks, achieving consistent performance enhancements, and thereby demonstrating the practical potential of our pipeline.

% We advocate for further research on customizing the construction of CoT reasoning, as well as the alignment algorithms for this task, to build more powerful, interpretable, and privacy-preserving open-source Text-to-SQL systems.

\section{Limitations}
The databases provided by the Bird dataset are typically large, leading to significant time consumption when executing SQL queries. This slows down the collection of preference data, as all sampled SQL queries must be executed on the databases to obtain feedback. To address this, we modify Bird's evaluation script to enable parallel execution of SQL queries using multi-processing, significantly accelerating the preference data construction process. However, resource contention among multiple processes can lead to SQL execution timeouts, causing correct predicted SQL queries to be incorrectly classified as incorrect, thereby introducing false negatives. This inaccuracy in feedback signals could potentially impact DPO training.

In contrast, the Spider dataset presents a different challenge due to the simplicity of its database values. Relying on execution results to distinguish between correct and incorrect SQL queries can lead to false positives~\cite{DBLP:conf/emnlp/ZhongYK20testsuite}, resulting in potentially unreliable feedback signals for DPO training.

% \section{Introduction}

% These instructions are for authors submitting papers to *ACL conferences using \LaTeX. They are not self-contained. All authors must follow the general instructions for *ACL proceedings,\footnote{\url{http://acl-org.github.io/ACLPUB/formatting.html}} and this document contains additional instructions for the \LaTeX{} style files.

% The templates include the \LaTeX{} source of this document (\texttt{acl\_latex.tex}),
% the \LaTeX{} style file used to format it (\texttt{acl.sty}),
% an ACL bibliography style (\texttt{acl\_natbib.bst}),
% an example bibliography (\texttt{custom.bib}),
% and the bibliography for the ACL Anthology (\texttt{anthology.bib}).

% \section{Engines}

% To produce a PDF file, pdf\LaTeX{} is strongly recommended (over original \LaTeX{} plus dvips+ps2pdf or dvipdf). Xe\LaTeX{} also produces PDF files, and is especially suitable for text in non-Latin scripts.

% \section{Preamble}

% The first line of the file must be
% \begin{quote}
% \begin{verbatim}
% \documentclass[11pt]{article}
% \end{verbatim}
% \end{quote}

% To load the style file in the review version:
% \begin{quote}
% \begin{verbatim}
% \usepackage[review]{acl}
% \end{verbatim}
% \end{quote}
% For the final version, omit the \verb|review| option:
% \begin{quote}
% \begin{verbatim}
% \usepackage{acl}
% \end{verbatim}
% \end{quote}

% To use Times Roman, put the following in the preamble:
% \begin{quote}
% \begin{verbatim}
% \usepackage{times}
% \end{verbatim}
% \end{quote}
% (Alternatives like txfonts or newtx are also acceptable.)

% Please see the \LaTeX{} source of this document for comments on other packages that may be useful.

% Set the title and author using \verb|\title| and \verb|\author|. Within the author list, format multiple authors using \verb|\and| and \verb|\And| and \verb|\AND|; please see the \LaTeX{} source for examples.

% By default, the box containing the title and author names is set to the minimum of 5 cm. If you need more space, include the following in the preamble:
% \begin{quote}
% \begin{verbatim}
% \setlength\titlebox{<dim>}
% \end{verbatim}
% \end{quote}
% where \verb|<dim>| is replaced with a length. Do not set this length smaller than 5 cm.

% \section{Document Body}

% \subsection{Footnotes}

% Footnotes are inserted with the \verb|\footnote| command.\footnote{This is a footnote.}

% \subsection{Tables and figures}

% See Table~\ref{tab:accents} for an example of a table and its caption.
% \textbf{Do not override the default caption sizes.}

% \begin{table}
%   \centering
%   \begin{tabular}{lc}
%     \hline
%     \textbf{Command} & \textbf{Output} \\
%     \hline
%     \verb|{\"a}|     & {\"a}           \\
%     \verb|{\^e}|     & {\^e}           \\
%     \verb|{\`i}|     & {\`i}           \\
%     \verb|{\.I}|     & {\.I}           \\
%     \verb|{\o}|      & {\o}            \\
%     \verb|{\'u}|     & {\'u}           \\
%     \verb|{\aa}|     & {\aa}           \\\hline
%   \end{tabular}
%   \begin{tabular}{lc}
%     \hline
%     \textbf{Command} & \textbf{Output} \\
%     \hline
%     \verb|{\c c}|    & {\c c}          \\
%     \verb|{\u g}|    & {\u g}          \\
%     \verb|{\l}|      & {\l}            \\
%     \verb|{\~n}|     & {\~n}           \\
%     \verb|{\H o}|    & {\H o}          \\
%     \verb|{\v r}|    & {\v r}          \\
%     \verb|{\ss}|     & {\ss}           \\
%     \hline
%   \end{tabular}
%   \caption{Example commands for accented characters, to be used in, \emph{e.g.}, Bib\TeX{} entries.}
%   \label{tab:accents}
% \end{table}

% As much as possible, fonts in figures should conform
% to the document fonts. See Figure~\ref{fig:experiments} for an example of a figure and its caption.

% Using the \verb|graphicx| package graphics files can be included within figure
% environment at an appropriate point within the text.
% The \verb|graphicx| package supports various optional arguments to control the
% appearance of the figure.
% You must include it explicitly in the \LaTeX{} preamble (after the
% \verb|\documentclass| declaration and before \verb|\begin{document}|) using
% \verb|\usepackage{graphicx}|.

% \begin{figure}[t]
%   \includegraphics[width=\columnwidth]{example-image-golden}
%   \caption{A figure with a caption that runs for more than one line.
%     Example image is usually available through the \texttt{mwe} package
%     without even mentioning it in the preamble.}
%   \label{fig:experiments}
% \end{figure}

% \begin{figure*}[t]
%   \includegraphics[width=0.31\linewidth]{example-image-a} \hfill
%   \includegraphics[width=0.31\linewidth]{example-image-b} \hfill
%   \includegraphics[width=0.31\linewidth]{example-image-b}
%   \caption {A minimal working example to demonstrate how to place
%     two images side-by-side.}
% \end{figure*}

% \subsection{Hyperlinks}

% Users of older versions of \LaTeX{} may encounter the following error during compilation:
% \begin{quote}
% \verb|\pdfendlink| ended up in different nesting level than \verb|\pdfstartlink|.
% \end{quote}
% This happens when pdf\LaTeX{} is used and a citation splits across a page boundary. The best way to fix this is to upgrade \LaTeX{} to 2018-12-01 or later.

% \subsection{Citations}

% \begin{table*}
%   \centering
%   \begin{tabular}{lll}
%     \hline
%     \textbf{Output}           & \textbf{natbib command} & \textbf{ACL only command} \\
%     \hline
%     \citep{Gusfield:97}       & \verb|\citep|           &                           \\
%     \citealp{Gusfield:97}     & \verb|\citealp|         &                           \\
%     \citet{Gusfield:97}       & \verb|\citet|           &                           \\
%     \citeyearpar{Gusfield:97} & \verb|\citeyearpar|     &                           \\
%     \citeposs{Gusfield:97}    &                         & \verb|\citeposs|          \\
%     \hline
%   \end{tabular}
%   \caption{\label{citation-guide}
%     Citation commands supported by the style file.
%     The style is based on the natbib package and supports all natbib citation commands.
%     It also supports commands defined in previous ACL style files for compatibility.
%   }
% \end{table*}

% Table~\ref{citation-guide} shows the syntax supported by the style files.
% We encourage you to use the natbib styles.
% You can use the command \verb|\citet| (cite in text) to get ``author (year)'' citations, like this citation to a paper by \citet{Gusfield:97}.
% You can use the command \verb|\citep| (cite in parentheses) to get ``(author, year)'' citations \citep{Gusfield:97}.
% You can use the command \verb|\citealp| (alternative cite without parentheses) to get ``author, year'' citations, which is useful for using citations within parentheses (e.g. \citealp{Gusfield:97}).

% A possessive citation can be made with the command \verb|\citeposs|.
% This is not a standard natbib command, so it is generally not compatible
% with other style files.

% \subsection{References}

% \nocite{Ando2005,andrew2007scalable,rasooli-tetrault-2015}

% The \LaTeX{} and Bib\TeX{} style files provided roughly follow the American Psychological Association format.
% If your own bib file is named \texttt{custom.bib}, then placing the following before any appendices in your \LaTeX{} file will generate the references section for you:
% \begin{quote}
% \begin{verbatim}
% \bibliography{custom}
% \end{verbatim}
% \end{quote}

% You can obtain the complete ACL Anthology as a Bib\TeX{} file from \url{https://aclweb.org/anthology/anthology.bib.gz}.
% To include both the Anthology and your own .bib file, use the following instead of the above.
% \begin{quote}
% \begin{verbatim}
% \bibliography{anthology,custom}
% \end{verbatim}
% \end{quote}

% Please see Section~\ref{sec:bibtex} for information on preparing Bib\TeX{} files.

% \subsection{Equations}

% An example equation is shown below:
% \begin{equation}
%   \label{eq:example}
%   A = \pi r^2
% \end{equation}

% Labels for equation numbers, sections, subsections, figures and tables
% are all defined with the \verb|\label{label}| command and cross references
% to them are made with the \verb|\ref{label}| command.

% This an example cross-reference to Equation~\ref{eq:example}.

% \subsection{Appendices}

% Use \verb|\appendix| before any appendix section to switch the section numbering over to letters. See Appendix~\ref{sec:appendix} for an example.

% \section{Bib\TeX{} Files}
% \label{sec:bibtex}

% Unicode cannot be used in Bib\TeX{} entries, and some ways of typing special characters can disrupt Bib\TeX's alphabetization. The recommended way of typing special characters is shown in Table~\ref{tab:accents}.

% Please ensure that Bib\TeX{} records contain DOIs or URLs when possible, and for all the ACL materials that you reference.
% Use the \verb|doi| field for DOIs and the \verb|url| field for URLs.
% If a Bib\TeX{} entry has a URL or DOI field, the paper title in the references section will appear as a hyperlink to the paper, using the hyperref \LaTeX{} package.

% \section*{Acknowledgments}

% This document has been adapted
% by Steven Bethard, Ryan Cotterell and Rui Yan
% from the instructions for earlier ACL and NAACL proceedings, including those for
% ACL 2019 by Douwe Kiela and Ivan Vuli\'{c},
% NAACL 2019 by Stephanie Lukin and Alla Roskovskaya,
% ACL 2018 by Shay Cohen, Kevin Gimpel, and Wei Lu,
% NAACL 2018 by Margaret Mitchell and Stephanie Lukin,
% Bib\TeX{} suggestions for (NA)ACL 2017/2018 from Jason Eisner,
% ACL 2017 by Dan Gildea and Min-Yen Kan,
% NAACL 2017 by Margaret Mitchell,
% ACL 2012 by Maggie Li and Michael White,
% ACL 2010 by Jing-Shin Chang and Philipp Koehn,
% ACL 2008 by Johanna D. Moore, Simone Teufel, James Allan, and Sadaoki Furui,
% ACL 2005 by Hwee Tou Ng and Kemal Oflazer,
% ACL 2002 by Eugene Charniak and Dekang Lin,
% and earlier ACL and EACL formats written by several people, including
% John Chen, Henry S. Thompson and Donald Walker.
% Additional elements were taken from the formatting instructions of the \emph{International Joint Conference on Artificial Intelligence} and the \emph{Conference on Computer Vision and Pattern Recognition}.

% Bibliography entries for the entire Anthology, followed by custom entries
%\bibliography{anthology,custom}
% Custom bibliography entries only
%\bibliography{custom}
\bibliography{alignment, custom, text-to-sql}

\appendix

\section{Preliminary Experiments on DPO Tricks}\label{apx:dpotricks}
Before exploring chain-of-thought reasoning, we extensively experiment with applying various DPO tricks to the original Bird dataset. This included hyperparameter tuning, using different loss variants, and exploring alternative preference data construction strategies. Specifically, for hyperparameter tuning, we evaluate different values of the $\beta$ parameter in DPO, testing 0.05 and 0.2 (the default value is 0.1). For loss variants, we experiment with augmenting the DPO loss by incorporating the SFT loss for correct responses (i.e., the correct sampled SQL queries). Additionally, we explore replacing DPO with its variant, KTO~\citep{DBLP:journals/corr/abs-2402-01306kto}. For preference data construction strategies, we follow SENSE~\cite{yang2024sense} to fine-tune a small-scale language model, Deepseek-coder-1.3b-instruct, on Bird's training set to collect preference data. The results, summarized in Table~\ref{tab:dpotrick}, show that none of these approaches yield significant performance improvements for DPO.


% Specifically, for the hyperparameter tuning, we try different $\beta$ parameters in DPO with 0.05 and 0.2 (the default value is 0.1). For the loss variants, we have tried to add the SFT loss of the correct responses (i.e., the correct sampled SQL queries) to the DPO loss. In addition, we have tried to replace DPO with its variant algorithm KTO~\citep{DBLP:journals/corr/abs-2402-01306kto}. For data construction strategies, we follow SENSE~\cite{} to fine-tune a small-scale language model, Deepseek-coder-1.3b, to construct preference pairs. The results are summarized in Table~\ref{tab:dpotrick}. We can observe that none of these approaches yielded significant performance improvements using DPO.

\definecolor{darkgreen}{RGB}{0,150,0}
\begin{table}[ht]
    \centering
\begin{adjustbox}{max width=\columnwidth}
    \begin{tabular}{ >{\centering\arraybackslash}p{0.4\columnwidth} | >{\centering\arraybackslash}p{0.4\columnwidth} }
        \toprule
         \multicolumn{2}{c}{\textbf{DPO Tricks (SFT$=58.8\%$)}} \\
         \midrule
         \multicolumn{2}{c}{\texttt{Hyper-parameter Tuning}} \\ \midrule
        $\beta=0.05$ & $\beta=0.2$ \\
        57.6\% (\textbf{\textcolor{darkgreen}{-1.2\%}}) & 57.8\% (\textbf{\textcolor{darkgreen}{-1.0\%}}) \\
        \midrule
         \multicolumn{2}{c}{\texttt{Loss Variants}} \\ \midrule 
         +SFT Loss & KTO \\
         60.0\% (\textbf{\textcolor{red}{+1.2\%}}) & 59.8\% (\textbf{\textcolor{red}{+1.0\%}}) \\
         \midrule
         \multicolumn{2}{c}{\texttt{Data Construction Strategies}} \\ \midrule
        \multicolumn{2}{c}{SENSE \citep{yang2024sense}} \\ 
        %+Gold SQL \\
        \multicolumn{2}{c}{58.4\% (\textbf{\textcolor{darkgreen}{-0.4\%}})} \\
        %& 59.2\% (\textbf{\textcolor{red}{+0.4\%}}) \\
        \bottomrule
    \end{tabular}
\end{adjustbox}
    \caption{Even with tricks applied, DPO still struggles to improve model performance on the Bird dataset (greedy decoding). The base model is Qwen2.5-7B-Instruct.}
    \label{tab:dpotrick}
\end{table}

\section{Chain-of-Thought Solutions} \label{apx:cotSynthesis}
In this section, we present the prompts used for Chain-of-Thought (CoT) reasoning synthesis and provide qualitative examples of the model's step-by-step Text-to-SQL responses.

\subsection{Prompt for Synthesis}
We carefully design the prompts used for CoT synthesis, as shown in Table~\ref{tab:prompt}. In our template, the gold SQL from the dataset is provided as a reference answer. This design enables the model to generate diverse reasoning paths during sampling while striving to maintain the correctness of the synthesized outputs.

\subsection{Synthesized Chain-of-Thought}
A synthesized Chain-of-Thought solution, generated by the \verb|gpt-4o-mini-2024-07-18| on an instance from the Bird Train dataset, is illustrated in Table~\ref{tab:synthesisCoT}. The response begins with an analysis of the input question, followed by a step-by-step breakdown of the SQL generation process. After generating the SQL, the model further provides explanations for each component of the SQL query.

\subsection{Samples From Syn CoT Models}
We select a sample question from the Bird development set and compare the responses generated during the SFT stage and the DPO stage. The SFT-generated response is shown in Table~\ref{tab:cotResponseSFT}, while the DPO-generated response is presented in the Table~\ref{tab:cotResponseDPO}. Notably, DPO corrected an entity mismatch error present in the SFT response.

\section{Direct Preference Optimization}\label{apx:dpo}
We provide a brief overview of direct preference optimization (DPO) and demonstrate how to utilize the trained DPO model to calculate the implicit reward of a response, as well as the credit assignment for each token within that response.
\subsection{Learning Objective}
For Reinforcement Learning with Human Feedback (RLHF) incorporating KL penalty, the learning objective is defined as \citep{ouyang2022rlhf}:
%\begin{adjustbox}{max width=\textwidth}
\begin{align*}
\max_{\pi_{\theta}}\mathbb{E}_{x\sim\mathcal{D},y\sim\pi_{\theta}(y|x)} & \left[r_{\phi}(x,y)\right] \\
 -\beta\mathbb{D}_{\mathrm{KL}} & \left[\pi_{\theta}(y\mid x)\mid\mid\pi_{\mathrm{ref}}(y\mid x)\right]
\end{align*}
%\end{adjustbox}

Here, $\pi_{\mathrm{ref}}$ and $\pi_{\theta}$ represent the initial model distribution and the optimized policy, respectively, while $r_{\phi}$ denotes a parameterized reward model.

Direct Preference Optimization (DPO) reformulates the objective by replacing the reward function with a differentiable form, reflecting the relationship between the optimal policy and the reward function. This leads to a new objective:
%$$\mathcal{L}_{\mathrm{DPO}}(\pi_{\theta};\pi_{\mathrm{ref}})=$$
%\resizebox{\linewidth}{!}{}
%\begin{adjustbox}{max width=\textwidth}
%\begin{align*}
%$$\mathbb{E}_{(x,y_{w},y_{l})\sim\mathcal{D}}\left[\log \sigma\left(\beta\log\frac{\pi_{\theta}(y_{w}\mid x)}{\pi_{\mathrm{ref}}(y_{w}\mid x)}
%-\beta\log\frac{\pi_{\theta}(y_{l}\mid x)}{\pi_{\mathrm{ref}}(y_{l}\mid x)}\right )\right ]$$
%\end{align*}
%\end{adjustbox}
\begin{align*}
\mathbb{E}_{(x,y_{w},y_{l})\sim\mathcal{D}}\left[\log \sigma\left(\beta \log\frac{\pi_{\theta}(y_{w}\mid x)}{\pi_{\mathrm{ref}}(y_{w}\mid x)}\right.\right. \\
\left.\left.  -\beta\log\frac{\pi_{\theta}(y_{l}\mid x)}{\pi_{\mathrm{ref}}(y_{l}\mid x)} \right)\right]
\end{align*}


In this formulation, $y_w$ denotes the chosen output and $y_l$ denotes the rejected output. The parameter $\beta$ controls the penalty strength imposed by the KL divergence. By collecting pairwise preference data, the model can be optimized using supervised fine-tuning, achieving a performance comparable to RLHF \citep{rafailov2024dpo}.

\subsection{Implicit Reward}\label{apx:implicit_reward}
%\textbf{Implicit Reward}

DPO implicitly encodes a reward model within the generative model. The reward of a given input-output pair $(x, y)$ can be derived as:

$$r_{\theta}(x, y) = \beta \log \frac{\pi_\theta(y\mid x)}{\pi_{\mathrm{ref}}(y\mid x)}$$

As the DPO training progresses, the optimized model simultaneously becomes a better generative model and a more refined reward model. After training, the implicit reward model, which is derived via conditional likelihood, can be independently used as a reward function \citep{lambert2024rewardbench, chen2024bootstrappinglanguagemodelsdpo, DBLP:journals/corr/abs-2402-06457vstar}.

%\textbf{Credit Assignment}
\subsection{Token-level Credit Assignment}\label{apx:token_reward}

The implicit reward scores the entire output as a whole. By decomposing the conditional probability that featuring autoregressive generation process, the reward can be re-expressed as:
$$r_{\theta}(x,y) = \sum_{t=1}^{T}\beta \log \frac{\pi_\theta(y_t\mid x,y_{1:t-1})}{\pi_{\mathrm{ref}}(y_t\mid x,y_{1:t-1})}$$

This decomposition allows for the calculation of token-level rewards, as the model score for each token can be identified separately. Although DPO training uses supervision at the full-sequence level, evidence has shown that the model can generalize compositionality to some extent, allowing it to distribute the reward signal to key tokens \citep{rafailov2024dpo2}. This facilitates credit assignment across the output sequence. The resulting dense reward can be utilized for further training and optimization \citep{zhong2024dpo-ppo}.

\section{Implementation Details}\label{apx:imp_details}

The computational environment used in our experiments is equipped with a 64-core Intel(R) Xeon(R) Platinum 8358 CPU @ 2.60GHz and 8 NVIDIA A800 GPUs with 80GB of memory each, running CUDA version 12.1.1.

Training is conducted using Llama Factory \citep{zheng2024llamafactory}, with FlashAttention 2.0 \citep{dao2023flashattention} and DeepSpeed ZeRO-3 \citep{rasley2020deepspeed} enabled. For models larger than 10B parameters, the DPO stage utilizes CPU offloading for both model parameters and optimizer states. During inferenece, the trained models are hosted with vLLM \citep{kwon2023vllm}.

For training, we employ full-parameter fine-tuning with bf16 mixed-precision \citep{micikevicius2018bf16}. The optimizer used was AdamW \citep{DBLP:conf/iclr/LoshchilovH19adamw} with default parameters (\(\beta_1 = 0.9\), \(\beta_2 = 0.99\)). A cosine decay learning rate schedule and a linear warmup over the first 5\% of training steps are also applied. The context window of models is set to 4096 tokens.

Across all training phases, we adopt a consistent batch size of 64. The learning rates for the SFT and DPO phases are set to \(1 \times 10^{-5}\) and \(1 \times 10^{-6}\), respectively, for models smaller than 10B, and \(7 \times 10^{-6}\) and \(7 \times 10^{-7}\), respectively, for models larger than 10B. The \(\beta\) parameter for DPO is set to 0.1. 

All models are trained for 4 epochs during the SFT phase, and the best checkpoint is selected to serve as the reference model in DPO, based on the maj@K metric on the development set. Training is then continued for 8 epochs during the DPO phase.

Unless otherwise specified, during inference time across all stages (including chain-of-thought synthesis), the sampling budget is set to 16, with the $temperature$ and $topK$ parameters set to 1.0 and 32, respectively.

\section{Training Data Details}\label{apx:datadetails}
\noindent\textbf{Quantity.}
In Bird dataset, the Vanilla SFT data consists of $9,428$ instances, while the CoT SFT data consists of $9,428 \times K$ instances (for Table~\ref{tab:model-comparison}, $K=16$, as we generate 16 CoT solutions for each training sample in the original dataset). 

The size of the DPO training data is smaller than that of the SFT training data and is model-dependent. This is because, during the construction of the preference data, we exclude data samples for which all SFT model-generated CoT solutions are either entirely correct or entirely incorrect. As a result, both Vanilla and Syn CoT preference datasets contain approximately 1.5k-2.5k preference pairs (e.g., Qwen-7b-Instruct Syn CoT has 1,546 pairs). The relationship between sample budgets and the quantity of DPO training data is illustrated in Figure~\ref{fig:scalePrefFull} and Figure~\ref{fig:scalePrefLog}.

\noindent\textbf{Construction of Input-Output Sequences and Their Average Length.}
Input prompt has an average length of $965$ tokens, which is the same for Vanilla and Syn CoT settings. The input prompt includes not only the question but also database information, such as table and column names, primary and foreign key relationships, and potentially useful database values. Following CodeS~\citep{li2024codes}, we first use a schema item classifier to identify the most relevant tables and columns based on the question. To improve recall accuracy, we replace the backbone model of the classifier from RoBERTa-Large (355M)~\citep{liu2019roberta} to XLM-RoBERTa-XL (3.5B)~\citep{goyal2021xlroberta}. We then use the ``coarse-to-fine'' database value-matching approach to retrieve question-related values from the database. The retrieved tables, columns, values, and remaining primary and foreign keys form the database prompt.

The output label of Vanilla (gold SQL in the Bird dataset) has an average length of $44$ tokens, while the synthesized chain-of-thought solutions have an average length of $404$ tokens. Reported token numbers is measured by the tokenizer of Qwen-7b-Instruct.

\section{Data Quality Ablation}\label{apx:qualityablation}

From Table~\ref{tab:model-comparison}, we can see that there are many cases where Syn CoT SFT has already surpassed Vanilla SFT, thus, it is natural to doubt that the benefit could be brought about by the potent proprietary model. In this section, we first analyze the ability of GPT-4o-mini on Text-to-SQL, then, we replace GPT-4o-mini with smaller open-sourced models to synthesize CoT reasoning paths, therefore further confirm that it is the chain-of-thought style solution path itself that enhance the effect of DPO.

\noindent \textbf{Ability of the Synthesizers.} We evaluate GPT-4o-mini's capability on Bird development set, as well as other open-sourced models that we will use as synthesizers. The result is shown in Table~\ref{tab:cotquality}.

\definecolor{darkgreen}{RGB}{0,150,0}
\begin{table}[ht]
    \centering
\begin{adjustbox}{max width=\columnwidth}
    \begin{tabular}{c | c | c | c }
        \toprule
         \multirow{2}{*}{\textbf{Model}} &  \multicolumn{3}{c}{\textbf{Bird Dev}} \\ \cline{2-4}
         & ZS-CoT & FS-SQL & FS-CoT \\ \midrule
        GPT-4o-mini & 50.1 & 55.3 & 53.0 \\
        Qwen(1.5B) & 15.6 & 23.1 & 18.9 \\
        Qwen(7B) & 41.8 & 47.6 & 43.1 \\
        Qwen(32B) & 52.9 & 55.3 & 53.0 \\
        \bottomrule
    \end{tabular}
\end{adjustbox}
    \caption{Performance of models used for chain-of-thought solution synthesis under different prompting strategies (Greddy). \textbf{Qwen (XB)}: Qwen2.5-XB-Instruct; \textbf{ZS-CoT}: Zero-shot CoT; \textbf{FS-SQL}: 3-Shot examples randomly chosen from Bird Train; \textbf{FS-CoT}: 3-Shot examples randomly chosen from its own synthesized CoT solutions on Bird Train. }
    \label{tab:cotquality}
\end{table}
As a comparison, Qwen2.5-7B-Instruct, after fine-tuning on original Bird Train data, reaches 58.8, which is higher than the EX of any synthesizer, as shown in Table~\ref{tab:model-comparison}.

It is evident that directly prompting closed-source instruction-tuned models does not inherently offer an advantage in the Text-to-SQL task, consistent with findings from previous work \citep{li2024codes}. Closed-source models only perform well when incorporating with complex multi-agent designs \citep{pourreza2024chase-sql}. Therefore, in this paper, we use GPT-4o-mini merely as a substitute for manual effort, allowing us to quickly and cost-effectively obtain CoT solutions of Text-to-SQL data, rather than distilling capabilities from a powerful model.

\noindent \textbf{DPO with Different CoT Quailities.} As shown in Table~\ref{tab:cotquality}, CoT quality varies for different size of synthesizers. We then use each of them to synthesize CoT solutions, and train the corresponding Syn CoT model separately (Base model is Qwen2.5-7B-Instruct). The results are presented in Table~\ref{tab:opencotpdo}.

\definecolor{darkgreen}{RGB}{0,150,0}
\begin{table}[ht]
    \centering
\begin{adjustbox}{max width=\columnwidth}
    \begin{tabular}{c | c | c }
        \toprule
         \multirow{2}{*}{\textbf{Synthesizer}} &  \multicolumn{2}{c}{\textbf{Bird Dev}} \\ \cline{2-3}
         & SFT & DPO \\ \midrule
        GPT-4o-mini & 57.4 & 61.9 {\color{red}(+4.5)} \\
        Qwen(1.5B) & 40.9 & 59.1 {\color{red}(+18.2)} \\
        Qwen(7B) & 54.3 & 60.6 {\color{red}(+6.3)} \\
        Qwen(32B) & 59.3 & 62.5 {\color{red}(+3.2)} \\
        \bottomrule
    \end{tabular}
\end{adjustbox}
    \caption{Performance of models trained from CoT-enhanced data generated by different synthesizers (Greddy). \textbf{Qwen (XB)}: Qwen2.5-XB-Instruct. }
    \label{tab:opencotpdo}
\end{table}

Its evident that all these models achieve significant improvement in the DPO phase. Interestingly, the impact of CoT quality on model performance is significantly weakened after DPO.

\section{Results on Spider and Its Robustness Variants} \label{apx:spider}

In addition to the Bird benchmark, we also train and evaluate three representative base models on the Spider benchmark. We further assess the models trained on Spider with robustness test sets. 

However, during our experiments, we identified several issues with using CoT and DPO techniques on the Spider dataset:

\begin{enumerate}
    \item \textbf{Simplicity of SQL in Spider: Most SQL queries in Spider are quite simple.} A typical example is: \texttt{Q: How many concerts are there in year 2014 or 2015? A: SELECT count(*) FROM concert WHERE year = 2014 OR year = 2015}. smaller pre-trained language model methods have already achieved good results on these queries \citep{li2023resdsql}. Additionally, the best models on the Spider benchmark have reached human-level performance (91.2\%\footnote{from the Spider Leaderboard: \url{ttps://yale-lily.github.io/spider}}). It is widely acknowledged that CoT often does not perform better on simple questions, possibly due to overthinking issues \citep{sprague2024cotcotchainofthoughthelps}.
    \item \textbf{Inaccurate Feedback on Execution Results in Spider:} When constructing preference data, we rely on execution results on the database to judge the correctness of sampled SQL. The Bird dataset's databases are specially designed with massive rows, whereas Spider's databases have fewer rows (\#Row/DB: Bird ($549K$) vs. Spider ($2K$), $<0.5\%$ \citep{li2024bird}), leading to potential false positives during evaluation \citep{DBLP:conf/emnlp/ZhongYK20testsuite}. Although subsequent work has attempted to mitigate this by constructing multiple test suite (TS) databases, the Spider benchmark only provides TS databases for the development set, not for the training set, leading to inaccurate preference pair construction.
    \item \textbf{Small Scale of Constructible Preference Data on Spider:} Due to the first issue, most models on the Spider dataset achieve very high accuracy on the training set (Pass@16 $\geq 99\%$, compared to Bird's $\leq80\%$), resulting in a very small preference dataset ($0.1-0.3k$, with the SFT phase dataset being $7k$. As for comparison, DPO data on Bird is approximately $1.5-2.5k$).
\end{enumerate}

Based on these issues, we choose to focus primarily on the model's performance on the Bird dataset, one of the most challenging Text-to-SQL benchmarks that closely reflects real-world scenarios. Therefore, results related to Spider are included as a reference in the appendix due to constraints of limited space, as shown below.

\subsection{Spider}

%During the training on the Spider dataset, we utilize the standard training set (7,000 samples). Since Spider only provides test suites on the development set, we have to rely on the EX metric as the criterion for distinguishing positive and negative responses during preference data collection. This may introduce false positives in the DPO dataset, as many SQL queries in Spider yield empty execution results.

The model performance on Spider's development set is summarized in Table~\ref{tab:spiderDev}. Despite the aforementioned challenges, the Syn-CoT model consistently achieved improvements during the DPO stage. 

\subsection{Spider Variants}

We select the best checkpoint according to the Spider development set and directly evaluate it on these robustness test sets. The results for Spider-Syn, Spider-Realistic, and Spider-DK are presented in Table~\ref{tab:spiderVariants}. The CoT model continues to demonstrate consistent performance improvements during the DPO stage, further confirming its strong generalization capabilities after DPO training.

Dr.~Spider is a more comprehensive and sophisticated robustness test set, which categorizes all perturbations into three major types: database (DB), natural language question (NLQ), and SQL query. It then further subdivides them into 17 subcategories. For each type of perturbation, dedicated test sets are constructed~\citep{DBLP:conf/iclr/Changdrspider}.

For DB perturbations, the results of Syn CoT models are shown in Table~\ref{tab:drSpiderDBCoT}, alongside the results of the vanilla model in Table~\ref{tab:drSpiderDBVanilla}. For NLQ perturbations, the results of Syn CoT models are shown in Table~\ref{tab:drSpiderNLQCoT}, alongside the results of the vanilla model in Table~\ref{tab:drSpiderNLQVanilla}. For SQL perturbations, the results of Syn CoT models are shown in Table~\ref{tab:drSpiderSQLCoT}, alongside the results of the vanilla model in Table~\ref{tab:drSpiderSQLVanilla}.

Except for a few specific cases, Syn CoT model still demonstrates consistent improvements during the DPO stage.

\section{Error Classifications} \label{apx:class}
\subsection{Description}
In this paper, we classify errors made by predicted SQL into 6 major categories with 17 variant types. Descriptions of each category or type are shown in Table~\ref{table:ErrorCategoryDescription}.

\subsection{Error Samples}
We provide samples for each error type in our classification criteria, for external knowledge, see Table~\ref{table:ErrorSampleEK}, for schema linking, see Table~\ref{table:ErrorSampleSchema}, for value retrieval, see Table~\ref{table:ErrorSampleValue}, for operation, see Table~\ref{table:ErrorSampleOperation}, for information, see Table~\ref{table:ErrorSampleInfo}, for syntax error, see Table~\ref{table:ErrorSampleSyntax}. These are selected from model predictions on the Bird development set.


\subsection{Classification Result}
The total number of errors and the proportion of each error type are presented in pie charts. Results of Syn CoT models is shown in Figure~\ref{tab:ErrorStatCoT}, and results of vanilla models is shown in Figure~\ref{tab:ErrorStatVanilla}.

A SQL query may commit multiple types of errors simultaneously. In our analysis, however, we only attribute each erroneous SQL to the most prominent type of mistake it made logically, for the convenience of analysis.

\section{More Analysis of DPO} \label{apx:analysis}

\subsection{Overall Effect}
The changes in the correctness of model-generated outputs before and after the DPO stage are illustrated in Figure~\ref{fig:overallDPO}. The CoT model corrects a greater number of errors during the DPO stage, while the proportion of newly introduced errors remained comparable to that of the vanilla model.

\subsection{Effect on Difficulty Classes}
We analyze the impact of DPO on questions of varying difficulty levels, as shown in Figure~\ref{fig:difficultiesDPO}. The CoT model exhibits significant improvements in performance on medium- and high-difficulty questions during the DPO stage, while the improvements on simpler questions are relatively limited.

\subsection{Fix Rate Difference}
To facilitate a detailed comparison of how the CoT model enhances DPO's error correction capabilities for different error types, we rank the error types by the increase in fix rates introduced by DPO. The results are presented in Table~\ref{tab:fixRateRank}.

\subsection{Emerging Errors}
The number of newly introduced errors during the DPO process is summarized in Table~\ref{tab:newErrorStat}. Overall, the distribution of emerging errors in the Syn CoT model is similar to that of the vanilla model. However, while CoT improves DPO's ability to address hallucination errors, it also leads to an increase in newly introduced hallucinations during the DPO stage.

\subsection{Transition Matrices}
The transitions between error categories before and after DPO are depicted in Figure~\ref{fig:matrixTight}. For a more detailed view of these transitions, the vanilla model's error type transitions are shown in Figure~\ref{fig:matrixFullVanilla}, while those of the Syn CoT model are presented in Figure~\ref{fig:matrixFullCoT}.

\subsection{Weakness}
\textbf{Existing Text-to-SQL Methods Can Complement DPO's Weaknesses.} Despite improvements brought by CoT, DPO remains less effective at fixing certain error types. However, these align with core challenges that existing Text-to-SQL methods aim to address. For example, the model frequently fails to recall relevant \underline{Table} (Fix~15.9\%) and \underline{Column} (Fix~16.1\%), a key challenge of schema linking. Notable recent works include CHESS \citep{talaei2024chess} and E-SQL \citep{DBLP:journals/corr/abs-2409-16751esql}. 
%Many Text-to-SQL approaches specifically focus on improving performance in this area. 
\underline{Syntax Error} (Fix~13.3\%) is another tricky problem. \citet{dac-sql, magic-sql} have proposed post-generation execution and repair strategies to ensure executable returned SQL.

\section{Experiment Design} \label{apx:design}
\subsection{Evaluation Preference Dataset}

To ensure a fair comparison of the discriminative capabilities between the Vanilla and CoT models, we construct the mentioned evaluation preference dataset as follows. 

First, both SFT models are used to sample from the development set. For any data point where both models could generate paired data ($i.e.$, both could simultaneously sample a positive and a negative example), we use the sampling outcomes from the CoT model to randomly construct a preference pair. Subsequently, we extract the SQL portion of the pair to serve as the preference pair for the vanilla model, incorporating it into their respective evaluation sets. Through this construction process, we ensure that the databases, questions, and SQLs in the dataset for both models are identical.

\section{Prominent Reward Hacking Patterns}\label{sec:RHPatterns}
In this section, we provide other prominent reward hacking patterns of the DPO training process in the Vanilla setting from our observations, as illustrated in Table~\ref{tab:egRH1},~\ref{tab:egRH2},~\ref{tab:egRH3}.

\section{More Scaling Results} \label{apx:scaling}
Scaling behavior of performance on the CoT synthesis budget and sample budget of preference data collection under all inference strategies are complemented in Figure~\ref{fig:scaleSynFull} and Figure~\ref{fig:scalePrefFull}, respectively.

It is noteworthy that performance saturation regarding sample budget in the preference data collection stage is mainly caused by the diminishing return of new preference pairs, as can be clearly seen from the log-scale plot Figure~\ref{fig:scalePrefLog}.

\section{Application Details}\label{apx:application}
\subsection{DTS-SQL}
DTS-SQL divides the Text-to-SQL task into two stages: Schema-Linking and SQL-Generation \citep{pourreza2024dts-sql}. Based on the code available in its repository, we construct the training and testing datasets for the SQL-Generation stage. (Since the original code is developed for the Spider dataset, we refer to the submitted \verb|DTS_SQL_BIRD_submission.py| file to account for incorporating Hints when constructing Prompts for the Bird dataset.) At this stage, we obtain a dataset without CoT. Subsequently, we utilize \texttt{gpt-4o-mini-2024-07-18} to generate 4 CoT paths, thereby creating two distinct datasets for training our two types of models.
\subsection{C3}
C3 stands for Clear Prompting, Calibration of Model Bias, and Consistency Output \citep{dong2023c3}. For the Clear Prompting and Calibration of Model Bias components, we use the C3 prompt templates, removing parts related to CoT. We run the Schema-Linking code to generate prompt inputs, which are paired with the correct SQL to form input-output pairs. Additionally, we use \texttt{gpt-4o-mini-2024-07-18} to generate 4 CoT paths, thereby creating two datasets for training our two models. After training the models, we test on Spider Dev using a Consistency Output subset of size 16, identical to Maj@16.


% C Chain-of-Thought Solutions
\begin{table*}[htbp]
\centering
\begin{tcolorbox}[
    colback=white, % 背景颜色
    colframe=black, % 边框颜色
    title=System, % 标题
    fonttitle=\bfseries, % 标题加粗
    width=\textwidth, % 宽度
    boxrule=1pt, % 边框粗细
    rounded corners, % 圆角
]

%\begin{tabularx}{\textwidth}{X|X} % 两栏布局
\begin{tabularx}{\textwidth}{X}
%\textbf{Greedy Decision-making (part 2)} &
%\textbf{Guiding Action Selection with Q (part 2)} \\
%\hline
%\textbf{Action:} perform nothing. \newline
%\textbf{Observation:} \lipsum[1] \newline
%\textbf{Thought:} \lipsum[2] &
%\textbf{Observation:} \lipsum[1] \newline
%\textbf{Thought:} \lipsum[2] \newline
%\textbf{Action:} click[Buy Now] \\
%\hline
%\textbf{Reward:} 0 &
%\textbf{Reward:} 1.0 \\
You are a senior data analyst who expertise at structural query language (SQL). Given a question made by front end employees and targeted database schema, you are asked to translate that question into SQLite query with detailed explanation.
\newline
\newline
Additionally, the input will be accompanied with a reference solution from your colleagues, which may or may not be correct. This extra information intents to help you to formulate your answer,  and you are asked not to mention reference solution in any form. 
\newline
\newline
To facilitate SQL extraction with regular expression, the SQL in your answer should be expressed in a Markdown code block with proper highlight. For example, \\
\verb|`|\verb|`|\verb|`| SQL \\
    SELECT * FROM database; \\
\verb|`|\verb|`|\verb|`|
\end{tabularx}
\end{tcolorbox}
\caption{The system message we use when prompting \texttt{gpt-4o-mini-2024-07-18} to synthesize Chain-of-Thought reasoning solutions for publicly available Text-to-SQL datasets.}
\label{tab:prompt}
\end{table*}
\begin{table*}[htbp]
\centering
\begin{tcolorbox}[
    colback=white, % 背景颜色
    colframe=black, % 边框颜色
    title=Response, % 标题
    fonttitle=\bfseries, % 标题加粗
    width=\textwidth, % 宽度
    boxrule=1pt, % 边框粗细
    rounded corners, % 圆角
]

%\begin{tabularx}{\textwidth}{X|X} % 两栏布局
\begin{tabularx}{\textwidth}{X}
%\textbf{Greedy Decision-making (part 2)} &
%\textbf{Guiding Action Selection with Q (part 2)} \\
%\hline
%\textbf{Action:} perform nothing. \newline
%\textbf{Observation:} \lipsum[1] \newline
%\textbf{Thought:} \lipsum[2] &
%\textbf{Observation:} \lipsum[1] \newline
%\textbf{Thought:} \lipsum[2] \newline
%\textbf{Action:} click[Buy Now] \\
%\hline
%\textbf{Reward:} 0 &
%\textbf{Reward:} 1.0 \\
\textbf{Database Prompt:} ... \\
\textbf{Question:} Among the payments made by Mary Smith, how many of them are over 4.99?\\ 
\textbf{External Knowledge:} over 4.99 refers to amount > 4.99 \\
\textbf{Gold SQL:} SELECT count(payment.amount) FROM payment INNER JOIN customer ON payment.customer\_id = customer.customer\_id WHERE customer.first\_name = 'MARY' AND customer.last\_name = 'SMITH' AND payment.amount > 4.99 \\
\hline
\\ \textbf{Synthesized Chain-of-Thought Solution} \\

To answer the question about how many payments made by Mary Smith are greater than 4.99, we will need to join the \verb|`|payment\verb|`| table with the \verb|`|customer\verb|`| table. The query will count the number of payments that match Mary Smith's first and last name while ensuring the payment amount exceeds 4.99.
\\ \\
Here’s how the SQL query can be constructed:
\\
1. **Join the Tables**: We will perform INNER JOIN between the \verb|`|payment\verb|`| and \verb|`|customer\verb|`| tables on the \verb|`|customer\_id\verb|`| column. This links payments to the customers who made them.

2. **Filter Conditions**: We will add conditions to filter for only those rows where the customer's first name is 'MARY' and last name is 'SMITH'.

3. **Amount Condition**: Additionally, we will include a condition to check if the payment amount is greater than 4.99.

4. **Count the Results**: Finally, we will use the COUNT function to return the total number of matching records.
\\ \\
Here is the SQL query that fulfills the requirements:
\\ 
\verb|`|\verb|`|\verb|`|SQL \\
SELECT COUNT(payment.amount) \\ 
FROM payment \\ 
INNER JOIN customer ON payment.customer\_id = customer.customer\_id \\
WHERE customer.first\_name = 'MARY' \\
AND customer.last\_name = 'SMITH' \\
AND payment.amount > 4.99; \\
\verb|`|\verb|`|\verb|`|
\\ \\ 
**Explanation**: \\
- \verb|`|SELECT COUNT(payment.amount)\verb|`|: This part specifies that we want to count the number of payments that meet the conditions specified. \\
- \verb|`|FROM payment\verb|`|: Indicates that the main table we are querying is the \verb|`|payment\verb|`| table. \\
- \verb|`|INNER JOIN customer ON payment.customer\_id = customer.customer\_id\verb|`|: This joins the \verb|`|payment\verb|`| table with the \verb|`|customer\verb|`| table based on the \verb|`|customer\_id\verb|`| present in both tables, allowing us to access customer details related to each payment. \\
- \verb|`|WHERE customer.first\_name = 'MARY' AND customer.last\_name = 'SMITH'\verb|`|: These conditions filter the results to include only payments made by a customer named Mary Smith. \\
- \verb|`|AND payment.amount > 4.99\verb|`|: This additional condition specifies that we only want payments where the amount is greater than 4.99.
\\ \\
This query will count the payments made by Mary Smith that exceed the specified amount.

%\\
%\hline \\ 
%\textbf{Gold SQL:} SELECT count(payment.amount) FROM payment INNER JOIN customer ON payment.customer\_id = customer.customer\_id WHERE customer.first\_name = 'MARY' AND customer.last\_name = 'SMITH' AND payment.amount > 4.99
\end{tabularx}
\end{tcolorbox}
\caption{A qualitative example of \texttt{gpt-4o-mini-2024-07-18} generated Chain-of-Thought for an instance of Bird Train set. \textbf{SQL in the CoT can be different from original gold SQL in the dataset}. }
\label{tab:synthesisCoT}
\end{table*}
\begin{table*}[htbp]
\centering
\begin{tcolorbox}[
    colback=white, % 背景颜色
    colframe=black, % 边框颜色
    title=Response, % 标题
    fonttitle=\bfseries, % 标题加粗
    width=\textwidth, % 宽度
    boxrule=1pt, % 边框粗细
    rounded corners, % 圆角
]

%\begin{tabularx}{\textwidth}{X|X} % 两栏布局
\begin{tabularx}{\textwidth}{X}
%\textbf{Greedy Decision-making (part 2)} &
%\textbf{Guiding Action Selection with Q (part 2)} \\
%\hline
%\textbf{Action:} perform nothing. \newline
%\textbf{Observation:} \lipsum[1] \newline
%\textbf{Thought:} \lipsum[2] &
%\textbf{Observation:} \lipsum[1] \newline
%\textbf{Thought:} \lipsum[2] \newline
%\textbf{Action:} click[Buy Now] \\
%\hline
%\textbf{Reward:} 0 &
%\textbf{Reward:} 1.0 \\
\textbf{Database Prompt:} Table district, columns = [ district.a11 ( integer | comment : average salary | values : 12541 , 8507 ) ,
district.a3 ( text | comment : region | values : Prague , \underline{central Bohemia} ) , district.district\_id ( integer | primary
key | comment : location of branch | values : 1 , 2 ) ... \\
\textbf{Question:} How many male customers who are living in North Bohemia have average salary greater than 8000? \\ 
\textbf{External Knowledge:} Male means that gender = 'M'; a3 refers to region; a11 pertains to average salary. \\
\hline
\\ %\textbf{Synthesized Chain-of-Thought Solution} \\

To answer the question of how many male customers living in North Bohemia have an average salary greater than 8000, we need to join the \verb|`|district\verb|`| and \verb|`|client\verb|`| tables based on the \verb|`|district\_id\verb|`|. We will then filter the results based on the specified conditions: the region being 'North Bohemia', the gender being 'M', and the average salary being greater than 8000. \\ \\

Here's the SQL query that accomplishes this: \\ \\

\verb|`|\verb|`|\verb|`|SQL \\
SELECT COUNT(client.client\_id) \\
FROM district \\
INNER JOIN client ON district.district\_id = client.district\_id \\
WHERE district.a3 = 'North Bohemia' \\ 
AND client.gender = 'M' \\ 
AND district.a11 > 8000; \\ 
\verb|`|\verb|`|\verb|`| \\ \\ 

\#\#\# Explanation: \\
1. **SELECT COUNT(client.client\_id)**: This part of the query counts the number of male clients that meet the specified criteria. 
  
2. **FROM district**: We start our query from the \verb|`|district\verb|`| table since we need to filter based on the region and average salary.

3. **INNER JOIN client ON district.district\_id = client.district\_id**: We perform an inner join between the \verb|`|district\verb|`| and \verb|`|client\verb|`| tables using the \verb|`|district\_id\verb|`| as the key. This allows us to access the client information associated with each district.

4. **WHERE district.a3 = 'North Bohemia'**: This condition filters the results to only include districts in the 'North Bohemia' region.

5. **AND client.gender = 'M'**: This condition filters the results to only include male clients.

6. **AND district.a11 > 8000**: This condition filters the results to only include districts where the average salary is greater than 8000.

\\
By executing this query, we will obtain the count of male customers living in North Bohemia with an average salary greater than 8000.

%\\
%\hline \\ 
%\textbf{Gold SQL:} SELECT count(payment.amount) FROM payment INNER JOIN customer ON payment.customer\_id = customer.customer\_id WHERE customer.first\_name = 'MARY' AND customer.last\_name = 'SMITH' AND payment.amount > 4.99
\end{tabularx}
\end{tcolorbox}
\caption{The full response Syn CoT SFT Qwen2.5-7B-Instruct generated for a sample question from Bird Dev. \textbf{The response is wrong due to entity mismatch for the region 'north Bohemia'}.}
\label{tab:cotResponseSFT}
\end{table*}
\begin{table*}[htbp]
\centering
\begin{tcolorbox}[
    colback=white, % 背景颜色
    colframe=black, % 边框颜色
    title=Response, % 标题
    fonttitle=\bfseries, % 标题加粗
    width=\textwidth, % 宽度
    boxrule=1pt, % 边框粗细
    rounded corners, % 圆角
]

%\begin{tabularx}{\textwidth}{X|X} % 两栏布局
\begin{tabularx}{\textwidth}{X}
%\textbf{Greedy Decision-making (part 2)} &
%\textbf{Guiding Action Selection with Q (part 2)} \\
%\hline
%\textbf{Action:} perform nothing. \newline
%\textbf{Observation:} \lipsum[1] \newline
%\textbf{Thought:} \lipsum[2] &
%\textbf{Observation:} \lipsum[1] \newline
%\textbf{Thought:} \lipsum[2] \newline
%\textbf{Action:} click[Buy Now] \\
%\hline
%\textbf{Reward:} 0 &
%\textbf{Reward:} 1.0 \\
\textbf{Database Prompt:} Table district, columns = [ district.a11 ( integer | comment : average salary | values : 12541 , 8507 ) ,
district.a3 ( text | comment : region | values : Prague , \underline{central Bohemia} ) , district.district\_id ( integer | primary
key | comment : location of branch | values : 1 , 2 ) ... \\
\textbf{Question:} How many male customers who are living in North Bohemia have average salary greater than 8000? \\ 
\textbf{External Knowledge:} Male means that gender = 'M'; a3 refers to region; a11 pertains to average salary. \\
\hline
\\ %\textbf{Synthesized Chain-of-Thought Solution} \\

To answer the question of how many male customers living in North Bohemia have an average salary greater than 8000, we need to join the \verb|`|client\verb|`| table with the \verb|`|district\verb|`| table based on the \verb|`|district\_id\verb|`|. We will then apply the necessary filters for the region (\verb|`|a3\verb|`|), gender (\verb|`|gender\verb|`|), and average salary (\verb|`|a11\verb|`|).
\\ \\ 
Here's the breakdown of the SQL query:
\\ \\ 
1. **SELECT count(client.client\_id)**: We want to count the number of male clients, so we use the `count` function on `client.client\_id`. \\

2. **FROM district INNER JOIN client ON district.district\_id = client.district\_id**: We perform an inner join between the \verb|`|district\verb|`| and \verb|`|client\verb|`| tables on the \verb|`|district\_id\verb|`| column, which is common to both tables. This allows us to access the district information for each client. \\

3. **WHERE district.a3 = 'north Bohemia'**: We filter the results to include only those clients who live in the North Bohemia region. \\

4. **AND client.gender = 'M'**: We further filter to include only male clients. \\

5. **AND district.a11 > 8000**: Finally, we add a condition to ensure that we only count clients whose average salary is greater than 8000. 
\\ \\
Putting this all together, the SQL query is as follows:
\\ \\ 
\verb|`|\verb|`|\verb|`|SQL \\
SELECT count(client.client\_id) \\ 
FROM district \\
INNER JOIN client ON district.district\_id = client.district\_id \\
WHERE district.a3 = 'north Bohemia' \\
AND client.gender = 'M' \\
AND district.a11 > 8000; \\
\verb|`|\verb|`|\verb|`| 
\\ \\
This query will return the count of male clients residing in North Bohemia with an average salary exceeding 8000.

%\\
%\hline \\ 
%\textbf{Gold SQL:} SELECT count(payment.amount) FROM payment INNER JOIN customer ON payment.customer\_id = customer.customer\_id WHERE customer.first\_name = 'MARY' AND customer.last\_name = 'SMITH' AND payment.amount > 4.99
\end{tabularx}
\end{tcolorbox}
\caption{The full response Syn CoT DPO Qwen2.5-7B-Instruct generated for a sample question from Bird Dev. \textbf{The model is able to infer entity format of 'north Bohemia' from value examples given in database prompt}.}
\label{tab:cotResponseDPO}
\end{table*}

\clearpage

% D Result on Other Benchmarks
\definecolor{darkgreen}{RGB}{0,150,0}
\begin{table*}[t!]
    \centering
\begin{adjustbox}{max width=\textwidth}
    \begin{tabular}{c  c | c c | c c | c c | c}
        \toprule
         & \multirow{3}{*}{\textbf{Model}} & \multicolumn{6}{c}{\textbf{Spider Dev}} & \\ \cline{3-9}
         & & \multicolumn{2}{c}{Greedy} & \multicolumn{2}{c}{Pass@1} & \multicolumn{2}{c |}{Maj@16} & \multirow{2}{*}{$\Delta$} \\ \cline{3-8}
         & & SFT & DPO & SFT & DPO & SFT & DPO & \\ \midrule
        \multicolumn{9}{c}{\textbf{Execution Accuracy (EX)}} \\ \midrule
        
        & Deepseek-coder-6.7b-instruct & 81.6 & 81.0 (\textcolor{darkgreen}{-0.6}) & 80.0  & 80.4 (\textcolor{red}{+0.4}) & \text{82.5} & 81.9 (\textcolor{darkgreen}{-0.6}) & -\\ 
        & Qwen2.5-7B-Instruct & 79.8 & 79.7 (\textcolor{darkgreen}{-0.1}) & 78.4  & 78.6 (\textcolor{red}{+0.2}) & \text{81.8} & 81.6 (\textcolor{darkgreen}{-0.2}) & -\\ 
        \multirow{-3}{*}{\textbf{Vanilla}} & CodeS-7b & 79.3 & 79.4 (\textcolor{red}{+0.1}) & 78.0  & 78.2 (\textcolor{red}{+0.2}) & \text{80.9} & 80.4 (\textcolor{darkgreen}{-0.5}) & - \\
        
        \rowcolor{cyan!20}
        & Deepseek-coder-6.7b-instruct & 80.0 & 82.1 (\textcolor{red}{+2.1}) & 79.4  & 81.3 (\textcolor{red}{+1.9}) & 82.8 & \text{83.8} (\textcolor{red}{+1.0}) & 82.5 $\rightarrow$ 83.8 (\textbf{\textcolor{red}{+1.3}}) \\ 
        \rowcolor{cyan!20}
        & Qwen2.5-7B-Instruct & 79.1 & 80.5 (\textcolor{red}{+1.4}) & 76.8 & 78.9 (\textcolor{red}{+2.1}) & 80.4 & \text{82.6} (\textcolor{red}{+2.2}) & 81.8 $\rightarrow$ 82.6 (\textbf{\textcolor{red}{+0.8}})\\ 
        \rowcolor{cyan!20}
        \multirow{-3}{*}{\textbf{Syn CoT}} & CodeS-7b & 76.7 & 80.0 (\textcolor{red}{+3.3}) & 75.5  & 78.8 (\textcolor{red}{+3.3}) &  78.3 & \text{82.3} (\textcolor{red}{+4.0}) & 80.9 $\rightarrow$ 82.3 (\textbf{\textcolor{red}{+1.4}}) \\
        
        \midrule
        \multicolumn{9}{c}{\textbf{Test Suite (TS)}} \\ \midrule
        
        & Deepseek-coder-6.7b-instruct & 80.5 & 80.7 (\textcolor{red}{+0.2}) & 79.1 & 79.6 (\textcolor{red}{+0.5}) & \text{81.1} & 80.7 (\textcolor{darkgreen}{-0.4}) & - \\
        & Qwen2.5-7B-Instruct & 78.5 & 78.9 (\textcolor{red}{+0.4})& 77.2 & 77.1 (\textcolor{darkgreen}{+0.1})& \text{79.5} & 79.3 (\textcolor{darkgreen}{-0.2}) & - \\
        \multirow{-3}{*}{\textbf{Vanilla}} & CodeS-7b & 76.9 & 77.1 (\textcolor{red}{+0.2}) & 75.4 & 75.7 (\textcolor{red}{+0.3}) & \text{78.1} & 77.7 (\textcolor{darkgreen}{-0.4}) & - \\
        
        \rowcolor{cyan!20}
        & Deepseek-coder-6.7b-instruct & 78.3 & 80.3 (\textcolor{red}{+2.0}) & 76.9 & 78.9 (\textcolor{red}{+2.0}) & 79.7 & \text{81.6} (\textcolor{red}{+1.9}) & 81.1 $\rightarrow$ 81.6 (\textbf{\textcolor{red}{+0.5}}) \\
        \rowcolor{cyan!20}
        & Qwen2.5-7B-Instruct & 77.5 & 77.9 (\textcolor{red}{+0.4})& 75.0 & 76.6 (\textcolor{red}{+1.6})& 78.6 & \text{80.2} (\textcolor{red}{+1.6}) & 79.5 $\rightarrow$ 80.2 (\textbf{\textcolor{red}{+0.7}})\\
        \rowcolor{cyan!20}
        \multirow{-3}{*}{\textbf{Syn CoT}} & CodeS-7b & 74.8 & 77.0 (\textcolor{red}{+2.2}) & 73.1 & 75.3 (\textcolor{red}{+2.2}) & 76.1 & \text{77.8} (\textcolor{red}{+1.7}) & 78.1 $\rightarrow$ 77.8 (\textbf{\textcolor{darkgreen}{-0.3}}) \\
        \bottomrule
    \end{tabular}
\end{adjustbox}
    \caption{Model performance on the Spider development set. \textbf{Vanilla}: SFT and DPO on the original Spider training set; \textbf{Syn CoT}: SFT and DPO on the CoT-enhanced training set; \textbf{$\Delta$}: The performance difference in EX/TS between ``Syn CoT + DPO'' and ``Vanilla + SFT'' when using the same base model.\textbf{We make preference dataset with EX since TS in not available to train set.}}
    \label{tab:spiderDev}
\end{table*}

%Model performance on Spider development set. \textbf{Vanilla}: SFT on original Spider train set; \textbf{Syn CoT}: SFT on synthetic chain-of-thought solutions generated from our pipeline; \textbf{$\Delta$}: The difference between best EX/TS score of Syn CoT (regardless of training stage and inference strategy) and its vanilla counterpart when built on same base model.\textbf{We make preference dataset with EX since TS in not available to train set.}
\definecolor{darkgreen}{RGB}{0,150,0}
\begin{table*}[t!]
    \centering
\begin{adjustbox}{max width=\textwidth}
    \begin{tabular}{c  c | c c | c c | c c | c}
        \toprule
         & \multirow{3}{*}{\textbf{Model}} & \multicolumn{6}{c}{\textbf{Spider Variants Dev (EX)}} & \\ \cline{3-9}
         & & \multicolumn{2}{c}{Greedy} & \multicolumn{2}{c}{Pass@1} & \multicolumn{2}{c |}{Maj@16} & \multirow{2}{*}{$\Delta$EX} \\ \cline{3-8}
         & & SFT & DPO & SFT & DPO & SFT & DPO & \\ \midrule
        \multicolumn{9}{c}{\textbf{Spider-Syn}} \\ \midrule
        
        & Deepseek-coder-6.7b-instruct & 73.8 & 73.5 (\textcolor{darkgreen}{-0.3}) & 72.5 & 72.4 (\textcolor{darkgreen}{-0.1}) & 75.2 & \text{75.5} (\textcolor{red}{+0.3}) & -\\ 
        & Qwen2.5-7B-Instruct & 71.5 & 72.0 (\textcolor{red}{+0.5}) & 69.6  & 69.8 (\textcolor{red}{+0.2}) & \text{73.7} & 73.6 (\textcolor{darkgreen}{-0.1}) & -\\ 
        \multirow{-3}{*}{\textbf{Vanilla}} & CodeS-7b & 69.1 & 69.4 (\textcolor{red}{+0.3}) & 67.1  & 67.6 (\textcolor{red}{+0.5}) & 71.1 & \text{71.2} (\textcolor{red}{+0.1}) & - \\
 
        \rowcolor{cyan!20}
        & Deepseek-coder-6.7b-instruct & 70.7 & 71.7 (\textcolor{red}{+1.0}) & 70.1  & 71.3 (\textcolor{red}{+1.2}) & 76.7 & \text{76.8} (\textcolor{red}{+0.1}) & 75.2 $\rightarrow$ 76.8 (\textbf{\textcolor{red}{+1.6}}) \\ 
        \rowcolor{cyan!20}
        & Qwen2.5-7B-Instruct & 69.7 & 71.2 (\textcolor{red}{+1.5}) & 67.9 & 70.5 (\textcolor{red}{+2.6}) & 74.1 & \text{76.2} (\textcolor{red}{+2.1}) & 73.7 $\rightarrow$ 76.2 (\textbf{\textcolor{red}{+2.5}})\\ 
        \rowcolor{cyan!20}
        \multirow{-3}{*}{\textbf{Syn CoT}} & CodeS-7b & 64.4 & 69.1 (\textcolor{red}{+4.7}) & 63.6  & 67.5 (\textcolor{red}{+3.9}) & 69.8 & \text{71.4} (\textcolor{red}{+1.6}) & 71.1 $\rightarrow$ 71.4 (\textbf{\textcolor{red}{+0.3}}) \\
        
        \midrule
        \multicolumn{9}{c}{\textbf{Spider-Realistic}} \\ \midrule
        
        & Deepseek-coder-6.7b-instruct & 77.2 & 77.2 \phantom{(\textcolor{red}{+0.2})} & 77.0 & 76.5 (\textcolor{darkgreen}{-0.5}) & \text{78.9} & 78.3 (\textcolor{darkgreen}{-0.6}) & - \\
        & Qwen2.5-7B-Instruct & 75.4 & 75.4 \phantom{(\textcolor{red}{+0.4})}& 73.7 & 74.2 (\textcolor{red}{+0.5})& 76.6 & \text{77.4} (\textcolor{red}{+0.8}) & - \\
        \multirow{-3}{*}{\textbf{Vanilla}} & CodeS-7b & 73.8 & 74.0 (\textcolor{red}{+0.2}) & 73.0 & 72.9 (\textcolor{darkgreen}{-0.1}) & \text{76.4} & 75.8 (\textcolor{darkgreen}{-0.6}) & - \\
        
        \rowcolor{cyan!20}
        & Deepseek-coder-6.7b-instruct & 79.7 & 80.7 (\textcolor{red}{+1.0}) & 77.4 & 78.7 (\textcolor{red}{+1.3}) & 80.9 & \text{82.7} (\textcolor{red}{+1.8}) & 78.9 $\rightarrow$ 82.7 (\textbf{\textcolor{red}{+3.8}}) \\
        \rowcolor{cyan!20}
        & Qwen2.5-7B-Instruct & 76.0 & 77.4 (\textcolor{red}{+1.4})& 73.9 & 75.1 (\textcolor{red}{+1.2})& 78.3 & \text{79.1} (\textcolor{red}{+0.8}) & 76.6 $\rightarrow$ 79.1 (\textbf{\textcolor{red}{+2.5}})\\
        \rowcolor{cyan!20}
        \multirow{-3}{*}{\textbf{Syn CoT}} & CodeS-7b & 73.8 & 76.4 (\textcolor{red}{+2.6}) & 71.5 & 73.3 (\textcolor{red}{+1.8}) & 76.6 & \text{78.5} (\textcolor{red}{+1.9}) & 76.4 $\rightarrow$ 78.5 (\textbf{\textcolor{red}{+2.1}}) \\

        \midrule
        \multicolumn{9}{c}{\textbf{Spider-DK}} \\ \midrule

        & Deepseek-coder-6.7b-instruct & 69.0 & 69.8 (\textcolor{darkgreen}{-0.2}) & 68.2 & 68.5 (\textcolor{red}{+0.3}) & \text{70.3} & 70.1 (\textcolor{darkgreen}{-0.2}) & - \\
        & Qwen2.5-7B-Instruct & 70.5 & 70.3 (\textcolor{darkgreen}{-0.2})& 69.0 & 69.4 (\textcolor{red}{+0.4})& \text{73.8} & 73.6 (\textcolor{darkgreen}{-0.2}) & - \\
        \multirow{-3}{*}{\textbf{Vanilla}} & CodeS-7b & 67.5 & 67.5 \phantom{(\textcolor{red}{+0.2})} & 66.3 & 66.6 (\textcolor{red}{+0.3}) & \text{69.2} & 68.6 (\textcolor{darkgreen}{-0.6}) & - \\

        \rowcolor{cyan!20}
        & Deepseek-coder-6.7b-instruct & 69.7 & 72.1 (\textcolor{red}{+2.4}) & 68.4 & 71.2 (\textcolor{red}{+2.8}) & 72.9 & \text{75.1} (\textcolor{red}{+2.2}) & 70.3 $\rightarrow$ 75.1 (\textbf{\textcolor{red}{+4.8}}) \\
        \rowcolor{cyan!20}
        & Qwen2.5-7B-Instruct & 67.5 & 69.0 (\textcolor{red}{+1.5})& 65.1 & 67.2 (\textcolor{red}{+2.1})& 70.7 & \text{72.9} (\textcolor{red}{+2.2}) & 73.8 $\rightarrow$ 72.9 (\textbf{\textcolor{darkgreen}{-0.9}})\\
        \rowcolor{cyan!20}
        \multirow{-3}{*}{\textbf{Syn CoT}} & CodeS-7b & 62.6 & 67.7 (\textcolor{red}{+5.1}) & 61.9 & 67.0 (\textcolor{red}{+5.1}) & 67.5 & \text{72.1} (\textcolor{red}{+4.6}) & 69.2 $\rightarrow$ 72.1 (\textbf{\textcolor{red}{+2.9}}) \\
        
        \bottomrule
    \end{tabular}
\end{adjustbox}
    \caption{Model performance on Spider's variants (Spider-Syn, Spider-Realistiv, Spider-DK). \textbf{Vanilla}: SFT and DPO on the original Spider training set; \textbf{Syn CoT}: SFT and DPO on the CoT-enhanced training set; \textbf{$\Delta$EX}: The performance difference in EX between ``Syn CoT + DPO'' and ``Vanilla + SFT'' when using the same base model. \textbf{In this setting, we directly assess best checkpoint on Spider Dev.}}
    \label{tab:spiderVariants}
\end{table*}
% Syn CoT Table
\definecolor{darkgreen}{RGB}{0,150,0}
\begin{table*}[t!]
    \centering
\begin{adjustbox}{max width=\textwidth}
    \begin{tabular}{c | l | c c | c c | c c }
        \toprule
         \multirow{3}{*}{\textbf{Type}} & \multirow{3}{*}{\textbf{Metrics}} & \multicolumn{6}{c}{\textbf{Syn CoT Models}} \\ \cline{3-8}
         & & \multicolumn{2}{c}{DSC (6.7B)} & \multicolumn{2}{c}{Qwen (7B)} & \multicolumn{2}{c}{CodeS (7B)} \\ \cline{3-8}
         & & SFT & DPO & SFT & DPO & SFT & DPO \\ \midrule
        \multicolumn{8}{c}{\textbf{DB Perturbations}} \\ \midrule

        \multirow{3}{*}{schema-synonym} & Greedy & 66.6 & 68.8 (\textcolor{red}{+2.2}) & 63.1 & 65.1 (\textcolor{red}{+2.0}) & 57.5 & 62.2 (\textcolor{red}{+4.7}) \\
        & Pass@1 & 63.8 & 66.7 (\textcolor{red}{+2.9}) & 60.7 & 63.3 (\textcolor{red}{+2.6}) & 55.0 & 60.8 (\textcolor{red}{+5.8}) \\
        & Maj@K & 69.7 & 71.7 (\textcolor{red}{+2.0}) & 68.1 & 69.3 (\textcolor{red}{+1.2}) & 63.7 & 66.2 (\textcolor{red}{+2.5}) \\
         \midrule

        \multirow{3}{*}{schema-abbreviation} & Greedy & 74.8 & 77.9 (\textcolor{red}{+3.1}) & 70.7 & 73.3 (\textcolor{red}{+2.6}) & 69.4 & 74.9 (\textcolor{red}{+5.5}) \\
        & Pass@1 & 73.0 & 76.1 (\textcolor{red}{+3.1}) & 68.1 & 71.2 (\textcolor{red}{+3.1}) & 66.8 & 73.2 (\textcolor{red}{+6.4}) \\
        & Maj@K & 76.7 & 79.8 (\textcolor{red}{+3.1}) & 75.3 & 77.6 (\textcolor{red}{+2.3}) & 73.6 & 77.1 (\textcolor{red}{+3.5}) \\
        \midrule


        \multirow{3}{*}{DBcontent-equivalence} & Greedy & 61.5 & 63.1 (\textcolor{red}{+1.6}) & 52.1 & 50.8 (\textcolor{darkgreen}{-1.3}) & 56.0 & 59.2 (\textcolor{red}{+3.2}) \\
        & Pass@1 & 59.5 & 60.3 (\textcolor{red}{+0.8}) & 49.2 & 49.4 (\textcolor{red}{+0.2}) & 56.4 & 56.8 (\textcolor{red}{+0.4}) \\
        & Maj@K & 61.8 & 63.1 (\textcolor{red}{+1.3}) & 55.5 & 55.0 (\textcolor{darkgreen}{-0.5}) & 61.5 & 62.6 (\textcolor{red}{+1.1}) \\
      \midrule

        \multirow{3}{*}{\textbf{Average}} & Greedy & 67.6 & 69.9 (\textcolor{red}{+2.3}) & 62.0 & 63.1 (\textcolor{red}{+1.1}) & 61.0 & 65.4 (\textcolor{red}{+4.4}) \\
& Pass@1 & 65.4 & 67.7 (\textcolor{red}{+2.3}) & 59.3 & 61.3 (\textcolor{red}{+2.0}) & 59.4 & 63.6 (\textcolor{red}{+4.2}) \\
& Maj@K & 69.4 & 71.5 (\textcolor{red}{+2.1}) & 66.3 & 67.3 (\textcolor{red}{+1.0}) & 66.3 & 68.6 (\textcolor{red}{+2.3}) \\

        
        \bottomrule
    \end{tabular}
\end{adjustbox}
    \caption{Syn CoT model performance on DB perturbations of Dr.Spider dataset. Names of base models are abbreviated. \textbf{DSC (6.7B)}: Deepseek-coder-6.7b-instruct; \textbf{Qwen (7B)}: Qwen2.5-7B-Instruct; \textbf{CodeS (7B)}: CodeS-7b. \textbf{In this setting, we directly assess best checkpoint on Spider Dev.}}
    \label{tab:drSpiderDBCoT}
\end{table*}

% Vanilla Table
\definecolor{darkgreen}{RGB}{0,150,0}
\begin{table*}[t!]
    \centering
\begin{adjustbox}{max width=\textwidth}
    \begin{tabular}{c | l | c c | c c | c c }
        \toprule
         \multirow{3}{*}{\textbf{Type}} & \multirow{3}{*}{\textbf{Metrics}} & \multicolumn{6}{c}{\textbf{Vanilla Models}} \\ \cline{3-8}
         & & \multicolumn{2}{c}{DSC (6.7B)} & \multicolumn{2}{c}{Qwen (7B)} & \multicolumn{2}{c}{CodeS (7B)} \\ \cline{3-8}
         & & SFT & DPO & SFT & DPO & SFT & DPO \\ \midrule
        \multicolumn{8}{c}{\textbf{DB Perturbations}} \\ \midrule

        \multirow{3}{*}{schema-synonym} & Greedy & 66.3 & 66.4 (\textcolor{red}{+0.1}) & 63.3 & 63.5 (\textcolor{red}{+0.2}) & 62.0 & 62.2 (\textcolor{red}{+0.2}) \\
        & Pass@1 & 65.6 & 65.6 (\textcolor{red}{+0.0}) & 61.1 & 61.2 (\textcolor{red}{+0.1}) & 59.7 & 60.5 (\textcolor{red}{+0.8}) \\
        & Maj@K & 69.0 & 68.6 (\textcolor{darkgreen}{-0.4}) & 67.9 & 67.8 (\textcolor{darkgreen}{-0.1}) & 65.7 & 65.9 (\textcolor{red}{+0.2}) \\ \midrule

        \multirow{3}{*}{schema-abbreviation} & Greedy & 77.0 & 77.3 (\textcolor{red}{+0.3}) & 75.3 & 75.7 (\textcolor{red}{+0.4}) & 71.3 & 71.3 \phantom{(+0.3)} \\
        & Pass@1 & 75.4 & 75.4 \phantom{(+0.3)} & 73.3 & 73.5 (\textcolor{red}{+0.2}) & 69.8 & 70.1 (\textcolor{red}{+0.3}) \\
        & Maj@K & 78.2 & 77.9 (\textcolor{darkgreen}{-0.3}) & 77.1 & 77.7 (\textcolor{red}{+0.6}) & 74.2 & 74.2 \phantom{(+0.3)} \\ \midrule

        \multirow{3}{*}{DBcontent-equivalence} & Greedy & 52.9 & 52.4 (\textcolor{darkgreen}{-0.5}) & 63.4 & 63.4 \phantom{(+0.3)} & 62.0 & 62.3 (\textcolor{red}{+0.3}) \\
        & Pass@1 & 54.1 & 53.9 (\textcolor{darkgreen}{-0.2}) & 61.6 & 61.6 \phantom{(+0.3)} & 61.0 & 61.0 \phantom{(+0.3)} \\
        & Maj@K & 53.4 & 53.4 \phantom{(+0.3)} & 69.1 & 69.1 \phantom{(+0.3)} & 64.4 & 63.6 (\textcolor{darkgreen}{-0.8}) \\ \midrule

        \multirow{3}{*}{\textbf{Average}} & Greedy & 65.4 & 65.4 \phantom{(+0.3)} & 67.3 & 67.5 (\textcolor{red}{+0.2}) & 65.1 & 65.3 (\textcolor{red}{+0.2}) \\
& Pass@1 & 65.0 & 65.0 \phantom{(+0.3)} & 65.3 & 65.4 (\textcolor{red}{+0.1}) & 63.5 & 63.9 (\textcolor{red}{+0.4}) \\
& Maj@K & 66.9 & 66.6 (\textcolor{darkgreen}{-0.3}) & 71.4 & 71.5 (\textcolor{red}{+0.1}) & 68.1 & 67.9 (\textcolor{darkgreen}{-0.2}) \\

        
        \bottomrule
    \end{tabular}
\end{adjustbox}
    \caption{Vanilla model performance on DB perturbations of Dr.Spider dataset. Names of base models are abbreviated. \textbf{DSC (6.7B)}: Deepseek-coder-6.7b-instruct; \textbf{Qwen (7B)}: Qwen2.5-7B-Instruct; \textbf{CodeS (7B)}: CodeS-7b. \textbf{In this setting, we directly assess best checkpoint on Spider Dev.}} 
    \label{tab:drSpiderDBVanilla}
\end{table*}
% Outdated Version
\iffalse
\definecolor{darkgreen}{RGB}{0,150,0}
\begin{table*}[t!]
    \centering
\begin{adjustbox}{max width=\textwidth}
    \begin{tabular}{c | c c | c c | c c }
        \toprule
         \multirow{3}{*}{\textbf{Type}} & \multicolumn{6}{c}{\textbf{Syn CoT Models}} \\ \cline{2-7}
         & \multicolumn{2}{c}{DSC (6.7B)} & \multicolumn{2}{c}{Qwen (7B)} & \multicolumn{2}{c}{CodeS (7B)} \\ \cline{2-7}
         & SFT & DPO & SFT & DPO & SFT & DPO \\ \midrule
        \multicolumn{7}{c}{\textbf{NLQ Perturbations}} \\ \midrule

        \multirow{3}{*}{keyword-synonym} & 57.9	& 59.9	& 54.2	&55.6	&52.0	&53.3 \\     
        &54.5	&57.1	&52.7	&54.0	&49.2	&52.4 \\
        &59.0	&60.6	&57.7	&58.3	&54.4	&58.1 \\ \midrule

        \multirow{3}{*}{keyword-carrier} & 57.9	& 59.9	& 54.2	&55.6	&52.0	&53.3 \\     
        &54.5	&57.1	&52.7	&54.0	&49.2	&52.4 \\
        &59.0	&60.6	&57.7	&58.3	&54.4	&58.1 \\ \midrule

        \multirow{3}{*}{column-synonym} & 57.9	& 59.9	& 54.2	&55.6	&52.0	&53.3 \\     
        &54.5	&57.1	&52.7	&54.0	&49.2	&52.4 \\
        &59.0	&60.6	&57.7	&58.3	&54.4	&58.1 \\ \midrule

        \multirow{3}{*}{column-carrier} & 57.9	& 59.9	& 54.2	&55.6	&52.0	&53.3 \\     
        &54.5	&57.1	&52.7	&54.0	&49.2	&52.4 \\
        &59.0	&60.6	&57.7	&58.3	&54.4	&58.1 \\ \midrule

        \multirow{3}{*}{column-attribute} & 57.9	& 59.9	& 54.2	&55.6	&52.0	&53.3 \\     
        &54.5	&57.1	&52.7	&54.0	&49.2	&52.4 \\
        &59.0	&60.6	&57.7	&58.3	&54.4	&58.1 \\ \midrule 

        \multirow{3}{*}{column-value} & 57.9	& 59.9	& 54.2	&55.6	&52.0	&53.3 \\     
        &54.5	&57.1	&52.7	&54.0	&49.2	&52.4 \\
        &59.0	&60.6	&57.7	&58.3	&54.4	&58.1 \\ \midrule

        \multirow{3}{*}{value-synonym} & 57.9	& 59.9	& 54.2	&55.6	&52.0	&53.3 \\     
        &54.5	&57.1	&52.7	&54.0	&49.2	&52.4 \\
        &59.0	&60.6	&57.7	&58.3	&54.4	&58.1 \\ \midrule

        \multirow{3}{*}{multitype} & 57.9	& 59.9	& 54.2	&55.6	&52.0	&53.3 \\     
        &54.5	&57.1	&52.7	&54.0	&49.2	&52.4 \\
        &59.0	&60.6	&57.7	&58.3	&54.4	&58.1 \\ \midrule
        
        \multirow{3}{*}{others} & 57.9	& 59.9	& 54.2	&55.6	&52.0	&53.3 \\     
        &54.5	&57.1	&52.7	&54.0	&49.2	&52.4 \\
        &59.0	&60.6	&57.7	&58.3	&54.4	&58.1 \\ \midrule

        \multirow{3}{*}{\textbf{Average}} & & & & & & \\
        & & & & & & \\ 
        & & & & & & \\ 
        
        \bottomrule
    \end{tabular}
\end{adjustbox}
    \caption{Syn CoT model performance on NLQ perturbations of Dr.Spider dataset. Names of base models are abbreviated. \textbf{DSC (6.7B)}: Deepseek-coder-6.7b-instruct; \textbf{Qwen (7B)}: Qwen2.5-7B-Instruct; \textbf{CodeS (7B)}: CodeS-7b. The model's performance for a specific perturbation type is listed from top to bottom under the three metrics: greedy, pass@1, and maj@K. \textbf{In this setting, we directly assess best checkpoint on Spider Dev.}} 
\end{table*} 
\fi

% Syn CoT Table
\definecolor{darkgreen}{RGB}{0,150,0}
\begin{table*}[t!]
    \centering
\begin{adjustbox}{max width=\textwidth}
    \begin{tabular}{c | l | c c | c c | c c }
        \toprule
         \multirow{3}{*}{\textbf{Type}} & \multirow{3}{*}{\textbf{Metrics}} & \multicolumn{6}{c}{\textbf{Syn CoT Models}} \\ \cline{3-8}
         & & \multicolumn{2}{c}{DSC (6.7B)} & \multicolumn{2}{c}{Qwen (7B)} & \multicolumn{2}{c}{CodeS (7B)} \\ \cline{3-8}
         & & SFT & DPO & SFT & DPO & SFT & DPO \\ \midrule
        \multicolumn{8}{c}{\textbf{NLQ Perturbations}} \\ \midrule

        \multirow{3}{*}{keyword-synonym} & Greedy & 67.4 & 71.2 (\textcolor{red}{+3.8}) & 66.7 & 68.9 (\textcolor{red}{+2.2}) & 62.5 & 71.9 (\textcolor{red}{+9.4}) \\
        & Pass@1 & 66.9 & 70.4 (\textcolor{red}{+3.5}) & 65.6 & 68.5 (\textcolor{red}{+2.9}) & 60.6 & 68.8 (\textcolor{red}{+8.2}) \\
        & Maj@K & 69.9 & 73.5 (\textcolor{red}{+3.6}) & 69.7 & 71.5 (\textcolor{red}{+1.8}) & 66.7 & 72.8 (\textcolor{red}{+6.1}) \\ \midrule

        \multirow{3}{*}{keyword-carrier} & Greedy & 76.4 & 80.7 (\textcolor{red}{+4.3}) & 77.2 & 78.9 (\textcolor{red}{+1.7}) & 77.4 & 84.5 (\textcolor{red}{+7.1}) \\
        & Pass@1 & 78.8 & 81.6 (\textcolor{red}{+2.8}) & 74.9 & 76.3 (\textcolor{red}{+1.4}) & 77.1 & 83.2 (\textcolor{red}{+6.1}) \\
        & Maj@K & 79.9 & 83.7 (\textcolor{red}{+3.8}) & 76.7 & 78.9 (\textcolor{red}{+2.2}) & 78.9 & 85.7 (\textcolor{red}{+6.8}) \\ \midrule

        \multirow{3}{*}{column-synonym} & Greedy & 57.9	& 59.9	(\textcolor{red}{+2.0})& 54.2	&55.6 (\textcolor{red}{+1.4})&52.0	&53.3 (\textcolor{red}{+1.3})\\     
        & Pass@1 & 54.5	&57.1 (\textcolor{red}{+2.6})	&52.7	&54.0 (\textcolor{red}{+1.3})	&49.2	&52.4 (\textcolor{red}{+3.2}) \\
        & Maj@K& 59.0	&60.6 (\textcolor{red}{+1.6})	&57.7	&58.3 (\textcolor{red}{+0.6})	&54.4	&58.1 (\textcolor{red}{+3.7}) \\ \midrule

        \multirow{3}{*}{column-carrier} & Greedy & 67.5 & 71.8 (\textcolor{red}{+4.3}) & 70.6 & 71.2 (\textcolor{red}{+0.6}) & 68.6 & 73.7 (\textcolor{red}{+5.1}) \\
        & Pass@1 & 67.2 & 72.5 (\textcolor{red}{+5.3}) & 68.2 & 70.2 (\textcolor{red}{+2.0}) & 65.9 & 72.8 (\textcolor{red}{+6.9}) \\
        & Maj@K & 74.3 & 75.8 (\textcolor{red}{+1.5}) & 72.7 & 74.4 (\textcolor{red}{+1.7}) & 73.9 & 77.7 (\textcolor{red}{+3.8}) \\ \midrule

        \multirow{3}{*}{column-attribute} & Greedy & 67.2 & 68.1 (\textcolor{red}{+0.9}) & 57.1 & 58.0 (\textcolor{red}{+0.9}) & 52.1 & 56.3 (\textcolor{red}{+4.2}) \\
        & Pass@1 & 60.7 & 64.0 (\textcolor{red}{+3.3}) & 55.7 & 57.8 (\textcolor{red}{+2.1}) & 49.1 & 53.4 (\textcolor{red}{+4.3}) \\
        & Maj@K & 70.6 & 71.4 (\textcolor{red}{+0.8}) & 61.3 & 63.0 (\textcolor{red}{+1.7}) & 58.0 & 59.7 (\textcolor{red}{+1.7}) \\
         \midrule 

        \multirow{3}{*}{column-value} & Greedy & 76.0 & 76.0 \phantom{(+0.3)} & 75.7 & 76.6 (\textcolor{red}{+0.9}) & 74.3 & 77.6 (\textcolor{red}{+3.3}) \\
        & Pass@1 & 73.5 & 75.1 (\textcolor{red}{+1.6}) & 72.8 & 73.7 (\textcolor{red}{+0.9}) & 70.9 & 74.1 (\textcolor{red}{+3.2}) \\
        & Maj@K & 77.3 & 78.0 (\textcolor{red}{+0.7}) & 77.3 & 78.0 (\textcolor{red}{+0.7}) & 75.3 & 78.9 (\textcolor{red}{+3.6}) \\
        \midrule

        \multirow{3}{*}{value-synonym} & Greedy & 67.2 & 68.0 (\textcolor{red}{+0.8}) & 63.6 & 64.4 (\textcolor{red}{+0.8}) & 64.2 & 65.2 (\textcolor{red}{+1.0}) \\
        & Pass@1 & 64.6 & 66.0 (\textcolor{red}{+1.4}) & 61.6 & 63.3 (\textcolor{red}{+1.7}) & 60.8 & 63.5 (\textcolor{red}{+2.7}) \\
        & Maj@K & 68.6 & 69.4 (\textcolor{red}{+0.8}) & 65.0 & 67.0 (\textcolor{red}{+2.0}) & 66.6 & 67.2 (\textcolor{red}{+0.6}) \\ \midrule

        \multirow{3}{*}{multitype} & Greedy & 63.6 & 66.5 (\textcolor{red}{+2.9}) & 61.0 & 63.5 (\textcolor{red}{+2.5}) & 60.8 & 63.9 (\textcolor{red}{+3.1}) \\
        & Pass@1 & 61.6 & 65.1 (\textcolor{red}{+3.5}) & 60.5 & 62.4 (\textcolor{red}{+1.9}) & 57.8 & 62.0 (\textcolor{red}{+4.2}) \\
        & Maj@K & 66.5 & 68.5 (\textcolor{red}{+2.0}) & 64.9 & 66.8 (\textcolor{red}{+1.9}) & 64.4 & 67.1 (\textcolor{red}{+2.7}) \\ \midrule
        
        \multirow{3}{*}{others} & Greedy & 73.2 & 75.5 (\textcolor{red}{+2.3}) & 73.5 & 75.3 (\textcolor{red}{+1.8}) & 70.6 & 74.9 (\textcolor{red}{+4.3}) \\
        & Pass@1 & 72.8 & 75.1 (\textcolor{red}{+2.3}) & 71.7 & 73.3 (\textcolor{red}{+1.6}) & 68.9 & 73.3 (\textcolor{red}{+4.4}) \\
        & Maj@K & 76.2 & 77.7 (\textcolor{red}{+1.5}) & 76.1 & 77.0 (\textcolor{red}{+0.9}) & 73.4 & 76.6 (\textcolor{red}{+3.2}) \\ \midrule

        \multirow{3}{*}{\textbf{Average}} & Greedy & 67.6 & 69.9 (\textcolor{red}{+2.3}) & 62.0 & 63.1 (\textcolor{red}{+1.1}) & 61.0 & 65.4 (\textcolor{red}{+4.4}) \\
& Pass@1 & 65.4 & 67.7 (\textcolor{red}{+2.3}) & 59.3 & 61.3 (\textcolor{red}{+2.0}) & 59.4 & 63.6 (\textcolor{red}{+4.2}) \\
& Maj@K & 69.4 & 71.5 (\textcolor{red}{+2.1}) & 66.3 & 67.3 (\textcolor{red}{+1.0}) & 66.3 & 68.6 (\textcolor{red}{+2.3}) \\

        
        \bottomrule
    \end{tabular}
\end{adjustbox}
    \caption{Syn CoT model performance on NLQ perturbations of Dr.Spider dataset. Names of base models are abbreviated. \textbf{DSC (6.7B)}: Deepseek-coder-6.7b-instruct; \textbf{Qwen (7B)}: Qwen2.5-7B-Instruct; \textbf{CodeS (7B)}: CodeS-7b. \textbf{In this setting, we directly assess best checkpoint on Spider Dev.}} 
    \label{tab:drSpiderNLQCoT}
\end{table*}

% Vanilla Table
\definecolor{darkgreen}{RGB}{0,150,0}
\begin{table*}[t!]
    \centering
\begin{adjustbox}{max width=\textwidth}
    \begin{tabular}{c | l | c c | c c | c c }
        \toprule
         \multirow{3}{*}{\textbf{Type}} & \multirow{3}{*}{\textbf{Metrics}} & \multicolumn{6}{c}{\textbf{Vanilla Models}} \\ \cline{3-8}
         & & \multicolumn{2}{c}{DSC (6.7B)} & \multicolumn{2}{c}{Qwen (7B)} & \multicolumn{2}{c}{CodeS (7B)} \\ \cline{3-8}
         & & SFT & DPO & SFT & DPO & SFT & DPO \\ \midrule
        \multicolumn{8}{c}{\textbf{NLQ Perturbations}} \\ \midrule

        \multirow{3}{*}{keyword-synonym} & Greedy & 69.6 & 69.8 (\textcolor{red}{+0.2}) & 68.2 & 68.3 (\textcolor{red}{+0.1}) & 66.8 & 67.7 (\textcolor{red}{+0.9}) \\
        & Pass@1 & 68.7 & 68.8 (\textcolor{red}{+0.1}) & 67.1 & 67.0 (\textcolor{darkgreen}{-0.1}) & 65.3 & 66.0 (\textcolor{red}{+0.7}) \\
        & Maj@K & 70.9 & 70.4 (\textcolor{darkgreen}{-0.5}) & 69.4 & 69.5 (\textcolor{red}{+0.1}) & 68.9 & 68.5 (\textcolor{darkgreen}{-0.4}) \\
        \midrule

        \multirow{3}{*}{keyword-carrier} & Greedy & 80.7 & 81.5 (\textcolor{red}{+0.8}) & 80.2 & 80.5 (\textcolor{red}{+0.3}) & 81.7 & 81.2 (\textcolor{darkgreen}{-0.5}) \\
& Pass@1 & 80.2 & 80.3 (\textcolor{red}{+0.1}) & 79.0 & 79.2 (\textcolor{red}{+0.2}) & 79.8 & 79.9 (\textcolor{red}{+0.1}) \\
& Maj@K & 81.0 & 81.0 \phantom{(+0.3)} & 80.7 & 79.9 (\textcolor{darkgreen}{-0.8}) & 83.0 & 84.0 (\textcolor{red}{+1.0}) \\ \midrule

        \multirow{3}{*}{column-synonym} & Greedy & 59.9 & 60.0 (\textcolor{red}{+0.1}) & 57.2 & 57.4 (\textcolor{red}{+0.2}) & 54.9 & 55.1 (\textcolor{red}{+0.2}) \\
& Pass@1 & 59.6 & 59.4 (\textcolor{darkgreen}{-0.2}) & 55.7 & 56.0 (\textcolor{red}{+0.3}) & 54.2 & 54.3 (\textcolor{red}{+0.1}) \\
& Maj@K & 62.0 & 61.6 (\textcolor{darkgreen}{-0.4}) & 59.0 & 59.3 (\textcolor{red}{+0.3}) & 56.8 & 57.0 (\textcolor{red}{+0.2}) \\ \midrule

        \multirow{3}{*}{column-carrier} & Greedy & 68.4 & 68.2 (\textcolor{darkgreen}{-0.2}) & 74.6 & 74.3 (\textcolor{darkgreen}{-0.3}) & 77.2 & 77.4 (\textcolor{red}{+0.2}) \\
& Pass@1 & 68.5 & 68.6 (\textcolor{red}{+0.1}) & 72.2 & 72.5 (\textcolor{red}{+0.3}) & 75.0 & 74.9 (\textcolor{darkgreen}{-0.1}) \\
& Maj@K & 70.1 & 69.8 (\textcolor{darkgreen}{-0.3}) & 76.7 & 76.2 (\textcolor{darkgreen}{-0.5}) & 78.4 & 77.9 (\textcolor{darkgreen}{-0.5}) \\
 \midrule

        \multirow{3}{*}{column-attribute} & Greedy & 59.7 & 59.7 \phantom{(+0.3)} & 49.6 & 50.4 (\textcolor{red}{+0.8}) & 47.9 & 47.9 \phantom{(+0.3)} \\
& Pass@1 & 58.6 & 58.7 (\textcolor{red}{+0.1}) & 49.4 & 49.3 (\textcolor{darkgreen}{-0.1}) & 46.7 & 47.4 (\textcolor{red}{+0.7}) \\
& Maj@K & 64.7 & 65.5 (\textcolor{red}{+0.8}) & 53.8 & 53.8 \phantom{(+0.3)} & 51.3 & 54.6 (\textcolor{red}{+3.3}) \\
 \midrule 

        \multirow{3}{*}{column-value} & Greedy & 74.7 & 74.7 \phantom{(+0.3)} & 76.6 & 76.6 \phantom{(+0.3)} & 71.4 & 71.7 (\textcolor{red}{+0.3}) \\
& Pass@1 & 74.2 & 74.1 (\textcolor{darkgreen}{-0.1}) & 73.3 & 73.5 (\textcolor{red}{+0.2}) & 70.1 & 70.8 (\textcolor{red}{+0.7}) \\
& Maj@K & 75.7 & 75.7 \phantom{(+0.3)} & 76.6 & 76.0 (\textcolor{darkgreen}{-0.6}) & 75.3 & 74.3 (\textcolor{darkgreen}{-1.0}) \\
 \midrule

        \multirow{3}{*}{value-synonym} & Greedy & 67.0 & 66.8 (\textcolor{darkgreen}{-0.2}) & 65.2 & 65.6 (\textcolor{red}{+0.4}) & 64.2 & 64.0 (\textcolor{darkgreen}{-0.2}) \\
& Pass@1 & 66.1 & 66.4 (\textcolor{red}{+0.3}) & 63.4 & 63.9 (\textcolor{red}{+0.5}) & 62.9 & 63.3 (\textcolor{red}{+0.4}) \\
& Maj@K & 70.0 & 69.2 (\textcolor{darkgreen}{-0.8}) & 66.2 & 66.4 (\textcolor{red}{+0.2}) & 66.0 & 65.4 (\textcolor{darkgreen}{-0.6}) \\
 \midrule

        \multirow{3}{*}{multitype} & Greedy & 63.7 & 63.7 \phantom{(+0.3)} & 62.0 & 62.2 (\textcolor{red}{+0.2}) & 63.1 & 63.3 (\textcolor{red}{+0.2}) \\
& Pass@1 & 62.9 & 63.2 (\textcolor{red}{+0.3}) & 60.7 & 61.0 (\textcolor{red}{+0.3}) & 61.2 & 61.5 (\textcolor{red}{+0.3}) \\
& Maj@K & 65.1 & 65.1 \phantom{(+0.3)} & 64.0 & 64.2 (\textcolor{red}{+0.2}) & 64.1 & 64.7 (\textcolor{red}{+0.6}) \\
 \midrule
        
        \multirow{3}{*}{others} & Greedy & 75.6 & 75.5 (\textcolor{darkgreen}{-0.1}) & 75.4 & 75.4 \phantom{(+0.3)} & 73.6 & 73.7 (\textcolor{red}{+0.1}) \\
& Pass@1 & 74.9 & 75.0 (\textcolor{red}{+0.1}) & 73.9 & 74.1 (\textcolor{red}{+0.2}) & 72.7 & 72.8 (\textcolor{red}{+0.1}) \\
& Maj@K & 76.7 & 76.8 (\textcolor{red}{+0.1}) & 76.4 & 76.3 (\textcolor{darkgreen}{-0.1}) & 75.0 & 74.6 (\textcolor{darkgreen}{-0.4}) \\
 \midrule

        \multirow{3}{*}{\textbf{Average}} & Greedy & 68.8 & 68.9 (\textcolor{red}{+0.1}) & 67.7 & 67.9 (\textcolor{red}{+0.2}) & 66.8 & 66.9 (\textcolor{red}{+0.1}) \\
& Pass@1 & 68.2 & 68.3 (\textcolor{red}{+0.1}) & 66.1 & 66.3 (\textcolor{red}{+0.2}) & 65.3 & 65.7 (\textcolor{red}{+0.4}) \\
& Maj@K & 70.7 & 70.6 (\textcolor{darkgreen}{-0.1}) & 69.2 & 69.1 (\textcolor{darkgreen}{-0.1}) & 68.8 & 69.0 (\textcolor{red}{+0.2}) \\

        
        \bottomrule
    \end{tabular}
\end{adjustbox}
    \caption{Vanilla model performance on NLQ perturbations of Dr.Spider dataset. Names of base models are abbreviated. \textbf{DSC (6.7B)}: Deepseek-coder-6.7b-instruct; \textbf{Qwen (7B)}: Qwen2.5-7B-Instruct; \textbf{CodeS (7B)}: CodeS-7b. \textbf{In this setting, we directly assess best checkpoint on Spider Dev.}} 
    \label{tab:drSpiderNLQVanilla}
\end{table*}
% Syn CoT Table
\definecolor{darkgreen}{RGB}{0,150,0}
\begin{table*}[t!]
    \centering
\begin{adjustbox}{max width=\textwidth}
    \begin{tabular}{c | l | c c | c c | c c }
        \toprule
         \multirow{3}{*}{\textbf{Type}} & \multirow{3}{*}{\textbf{Metrics}} & \multicolumn{6}{c}{\textbf{Syn CoT Models}} \\ \cline{3-8}
         & & \multicolumn{2}{c}{DSC (6.7B)} & \multicolumn{2}{c}{Qwen (7B)} & \multicolumn{2}{c}{CodeS (7B)} \\ \cline{3-8}
         & & SFT & DPO & SFT & DPO & SFT & DPO \\ \midrule
        \multicolumn{8}{c}{\textbf{SQL Perturbations}} \\ \midrule

        \multirow{3}{*}{comparison} & Greedy & 68.0 & 74.2 (\textcolor{red}{+6.2}) & 57.9 & 64.6 (\textcolor{red}{+6.7}) & 68.0 & 75.3 (\textcolor{red}{+7.3}) \\
& Pass@1 & 68.0 & 73.5 (\textcolor{red}{+5.5}) & 60.8 & 66.5 (\textcolor{red}{+5.7}) & 66.0 & 74.2 (\textcolor{red}{+8.2}) \\
& Maj@K & 75.3 & 77.5 (\textcolor{red}{+2.2}) & 62.9 & 70.2 (\textcolor{red}{+7.3}) & 76.4 & 78.7 (\textcolor{red}{+2.3}) \\
 \midrule

        \multirow{3}{*}{sort-order} & Greedy & 71.4 & 78.6 (\textcolor{red}{+7.2}) & 69.3 & 73.4 (\textcolor{red}{+4.1}) & 63.0 & 74.5 (\textcolor{red}{+11.5}) \\
& Pass@1 & 69.2 & 75.6 (\textcolor{red}{+6.4}) & 69.3 & 74.7 (\textcolor{red}{+5.4}) & 64.9 & 73.7 (\textcolor{red}{+8.8}) \\
& Maj@K & 73.4 & 78.1 (\textcolor{red}{+4.7}) & 72.4 & 78.1 (\textcolor{red}{+5.7}) & 69.3 & 76.6 (\textcolor{red}{+7.3}) \\
 \midrule

        \multirow{3}{*}{nonDB-number} & Greedy & 76.3 & 78.6 (\textcolor{red}{+2.3}) & 76.3 & 77.1 (\textcolor{red}{+0.8}) & 77.1 & 80.2 (\textcolor{red}{+3.1}) \\
& Pass@1 & 76.6 & 77.3 (\textcolor{red}{+0.7}) & 75.2 & 76.4 (\textcolor{red}{+1.2}) & 76.3 & 76.1 (\textcolor{darkgreen}{-0.2}) \\
& Maj@K & 80.2 & 80.1 (\textcolor{darkgreen}{-0.1}) & 79.4 & 80.9 (\textcolor{red}{+1.5}) & 81.7 & 82.4 (\textcolor{red}{+0.7}) \\
 \midrule
        
        \multirow{3}{*}{DB-text} & Greedy & 77.5 & 78.8 (\textcolor{red}{+1.3}) & 75.1 & 75.7 (\textcolor{red}{+0.6}) & 72.6 & 75.0 (\textcolor{red}{+2.4}) \\
& Pass@1 & 77.0 & 77.5 (\textcolor{red}{+0.5}) & 72.8 & 74.3 (\textcolor{red}{+1.5}) & 69.4 & 72.4 (\textcolor{red}{+3.0}) \\
& Maj@K & 81.9 & 81.8 (\textcolor{darkgreen}{-0.1}) & 79.4 & 79.6 (\textcolor{red}{+0.2}) & 74.5 & 76.0 (\textcolor{red}{+1.5}) \\
 \midrule

        \multirow{3}{*}{DB-number} & Greedy & 83.9 & 83.9 \phantom{(+0.3)} & 80.5 & 82.7 (\textcolor{red}{+2.2}) & 81.0 & 81.5 (\textcolor{red}{+0.5}) \\
& Pass@1 & 79.6 & 81.3 (\textcolor{red}{+1.7}) & 78.9 & 80.1 (\textcolor{red}{+1.2}) & 75.3 & 76.9 (\textcolor{red}{+1.6}) \\
& Maj@K & 83.4 & 84.9 (\textcolor{red}{+1.5}) & 82.9 & 83.4 (\textcolor{red}{+0.5}) & 79.8 & 80.2 (\textcolor{red}{+0.4}) \\
 \midrule

        \multirow{3}{*}{\textbf{Average}} & Greedy & 75.4 & 78.8 (\textcolor{red}{+3.4}) & 71.8 & 74.7 (\textcolor{red}{+2.9}) & 72.3 & 77.3 (\textcolor{red}{+5.0}) \\
& Pass@1 & 74.1 & 77.0 (\textcolor{red}{+2.9}) & 71.4 & 74.4 (\textcolor{red}{+3.0}) & 70.4 & 74.7 (\textcolor{red}{+4.3}) \\
& Maj@K & 78.8 & 80.5 (\textcolor{red}{+1.7}) & 75.4 & 78.4 (\textcolor{red}{+3.0}) & 76.3 & 78.8 (\textcolor{red}{+2.5}) \\

        
        \bottomrule
    \end{tabular}
\end{adjustbox}
    \caption{Syn CoT model performance on SQL perturbations of Dr.Spider dataset. Names of base models are abbreviated. \textbf{DSC (6.7B)}: Deepseek-coder-6.7b-instruct; \textbf{Qwen (7B)}: Qwen2.5-7B-Instruct; \textbf{CodeS (7B)}: CodeS-7b. \textbf{In this setting, we directly assess best checkpoint on Spider Dev.}} 
    \label{tab:drSpiderSQLCoT}
\end{table*}

% Vanilla Table
\definecolor{darkgreen}{RGB}{0,150,0}
\begin{table*}[t!]
    \centering
\begin{adjustbox}{max width=\textwidth}
    \begin{tabular}{c | l | c c | c c | c c }
        \toprule
         \multirow{3}{*}{\textbf{Type}} & \multirow{3}{*}{\textbf{Metrics}} & \multicolumn{6}{c}{\textbf{Vanilla Models}} \\ \cline{3-8}
         & & \multicolumn{2}{c}{DSC (6.7B)} & \multicolumn{2}{c}{Qwen (7B)} & \multicolumn{2}{c}{CodeS (7B)} \\ \cline{3-8}
         & & SFT & DPO & SFT & DPO & SFT & DPO \\ \midrule
        \multicolumn{8}{c}{\textbf{SQL Perturbations}} \\ \midrule

        \multirow{3}{*}{comparison} & Greedy & 70.2 & 69.7 (\textcolor{darkgreen}{-0.5}) & 69.1 & 69.1 \phantom{(+0.3)} & 71.9 & 72.5 (\textcolor{red}{+0.6}) \\
& Pass@1 & 69.7 & 69.6 (\textcolor{darkgreen}{-0.1}) & 67.7 & 68.3 (\textcolor{red}{+0.6}) & 68.9 & 69.5 (\textcolor{red}{+0.6}) \\
& Maj@K & 73.6 & 73.6 \phantom{(+0.3)} & 71.9 & 71.9 \phantom{(+0.3)} & 72.5 & 73.0 (\textcolor{red}{+0.5}) \\
 \midrule

        \multirow{3}{*}{sort-order} & Greedy & 79.2 & 79.2 \phantom{(+0.3)} & 75.5 & 76.0 (\textcolor{red}{+0.5}) & 78.6 & 77.6 (\textcolor{darkgreen}{-1.0}) \\
& Pass@1 & 77.4 & 77.4 \phantom{(+0.3)} & 74.0 & 73.9 (\textcolor{darkgreen}{-0.1}) & 74.5 & 74.3 (\textcolor{darkgreen}{-0.2}) \\
& Maj@K & 80.7 & 80.2 (\textcolor{darkgreen}{-0.5}) & 80.2 & 79.2 (\textcolor{darkgreen}{-1.0}) & 78.1 & 78.1 \phantom{(+0.3)} \\
 \midrule

        \multirow{3}{*}{nonDB-number} & Greedy & 73.3 & 73.3 \phantom{(+0.3)} & 74.0 & 74.0 \phantom{(+0.3)} & 70.2 & 71.0 (\textcolor{red}{+0.8}) \\
& Pass@1 & 73.0 & 72.7 (\textcolor{darkgreen}{-0.3}) & 73.4 & 73.4 \phantom{(+0.3)} & 65.6 & 66.4 (\textcolor{red}{+0.8}) \\
& Maj@K & 76.3 & 74.8 (\textcolor{darkgreen}{-1.5}) & 74.8 & 74.8 \phantom{(+0.3)} & 70.2 & 71.8 (\textcolor{red}{+1.6}) \\
 \midrule
        
        \multirow{3}{*}{DB-text} & Greedy & 79.9 & 80.0 (\textcolor{red}{+0.1}) & 74.6 & 74.9 (\textcolor{red}{+0.3}) & 73.5 & 73.5 \phantom{(+0.3)} \\
& Pass@1 & 78.8 & 78.9 (\textcolor{red}{+0.1}) & 72.4 & 72.4 \phantom{(+0.3)} & 72.8 & 73.2 (\textcolor{red}{+0.4}) \\
& Maj@K & 80.2 & 80.2 \phantom{(+0.3)} & 75.4 & 75.5 (\textcolor{red}{+0.1}) & 75.2 & 75.0 (\textcolor{darkgreen}{-0.2}) \\
 \midrule

        \multirow{3}{*}{DB-number} & Greedy & 83.2 & 83.4 (\textcolor{red}{+0.2}) & 82.7 & 82.7 \phantom{(+0.3)} & 78.5 & 78.5 \phantom{(+0.3)} \\
& Pass@1 & 83.2 & 83.2 \phantom{(+0.3)} & 81.0 & 81.4 (\textcolor{red}{+0.4}) & 77.7 & 77.7 \phantom{(+0.3)} \\
& Maj@K & 85.6 & 86.1 (\textcolor{red}{+0.5}) & 83.4 & 83.7 (\textcolor{red}{+0.3}) & 79.3 & 79.3 \phantom{(+0.3)} \\
 \midrule

        \multirow{3}{*}{\textbf{Average}} & Greedy & 68.8 & 68.9 (\textcolor{red}{+0.1}) & 67.7 & 67.9 (\textcolor{red}{+0.2}) & 66.8 & 66.9 (\textcolor{red}{+0.1}) \\
& Pass@1 & 68.2 & 68.3 (\textcolor{red}{+0.1}) & 66.1 & 66.3 (\textcolor{red}{+0.2}) & 65.3 & 65.7 (\textcolor{red}{+0.4}) \\
& Maj@K & 70.7 & 70.6 (\textcolor{darkgreen}{-0.1}) & 69.2 & 69.1 (\textcolor{darkgreen}{-0.1}) & 68.8 & 69.0 (\textcolor{red}{+0.2}) \\

        
        \bottomrule
    \end{tabular}
\end{adjustbox}
    \caption{Vanilla model performance on SQL perturbations of Dr.Spider dataset. Names of base models are abbreviated. \textbf{DSC (6.7B)}: Deepseek-coder-6.7b-instruct; \textbf{Qwen (7B)}: Qwen2.5-7B-Instruct; \textbf{CodeS (7B)}: CodeS-7b. \textbf{In this setting, we directly assess best checkpoint on Spider Dev.}}
    \label{tab:drSpiderSQLVanilla}
\end{table*}

\clearpage

% E Error Classifications
\definecolor{darkgreen}{RGB}{0,150,0}
\begin{table*}[t]
\centering
%\setlength{\tabcolsep}{5pt} % Adjust column padding
%\renewcommand{\arraystretch}{1.3} % Adjust row spacing
%\begin{tabularx}{\textwidth}{lXlccc}
\begin{adjustbox}{max width=\textwidth}
\begin{tabular}{c p{5.5cm} l p{6cm} }
\toprule
\textbf{Category} & \textbf{Description} & \textbf{Type} & \textbf{Description}\\
\midrule
\multirow{1}{*}{External Knowledge} 
    & \textbf{Neglect of external knowledge} directly causes the failure.
    & [A1] EK 
    & \textbf{Neglect of external knowledge} directly causes the failure.
    \\
\midrule
\multirow{6}{*}{Schema Linking} 
    & \multirow{6}{=}{The SQL fails to match the question with its \textbf{concerning table and columns.}}
    & [B1] Table 
    & The SQL selects \textbf{irrelevant tables, or misses necessary tables}. 
    \\
    & & [B2] JOIN & Selects right table, but \textbf{joins on the wrong key}.
    \\
    & & [B3] Column & The SQL fails to understand \textbf{critical columns}. 
    \\
    & & [B4] Hallucination & \textbf{Make up table or column} that does not exist in database.
    \\
    & & [B5] Condition & \textbf{Misunderstands/misses out} the required condition.
    \\
    & & [B6] NULL/DISTINCT & Possible \textbf{replication or null data} affect query result.
    \\
\midrule
\multirow{2}{*}{Value Retrieval}
    & \multirow{2}{=}{Fails to match the condition with its \textbf{storage format} in  database.}
    & [C1] String/Number 
    & \textbf{Mismatch of entity format}.
    \\
    & & [C2] Date & \textbf{Manipulation of Date} is wrong.
    \\
\midrule
\multirow{3}{*}{Operation}
    & \multirow{3}{=}{\textbf{Misunderstands required operation} in the question.}
    & [D1] Mathematical Formula 
    & The \textbf{formula} is not correspond to the original question.
    \\
    & & [D2] Aggregation & Wrong use of \textbf{aggregation function or GROUP BY keyword}.
    \\
    & & [D3] Complex Operation & \textbf{Unsusal complex operation (CASE, RANK(), IIF(), ...)} is required in query.
    \\
\midrule
\multirow{4}{*}{Information}
    & \multirow{4}{=}{The SQL generally satisfies requirement of the question, but fails to \textbf{organize information} in right way.}
    & [E1] Redundant/Incomplete 
    & Extra \textbf{filter/ranking conditions returned} following queried information.
    \\
    & & [E2] Column Sequence & Returned columns are listed in \textbf{wrong order}.
    \\
    & & [E3] ORDER BY/LIMIT & Does not follow instruction to \textbf{limit} records according to certain criteria.
    \\
    & & [E4] Format & The \textbf{format} of returned records is not as required.
    \\
\midrule
Syntax Error 
    & \textbf{Inexecutatble SQL}.
    & [F1] Syntax 
    & \textbf{Inexecutable SQL}.
    \\ 
\bottomrule
\end{tabular}
\end{adjustbox}
\caption{Categories and Types we used in error analysis and their corresponding descriptions.}
\label{table:ErrorCategoryDescription}
\end{table*}
\definecolor{darkgreen}{RGB}{0,150,0}
\newcommand{\bt}[1]{\textbf{\texttt{#1}}}
\newcommand{\gh}[1]{\textbf{\textcolor{darkgreen}{#1}}}
\newcommand{\rh}[1]{\textbf{\textcolor{red}{#1}}}


% External Knowledge
\begin{table*}[t]
\centering
%\setlength{\tabcolsep}{5pt} % Adjust column padding
%\renewcommand{\arraystretch}{1.3} % Adjust row spacing
%\begin{tabularx}{\textwidth}{lXlccc}
\begin{adjustbox}{max width=\textwidth}
\begin{tabular}{p{4.75cm} p{5.5cm} p{5.5cm} p{2.5cm}}
\toprule
\textbf{Sample Question with EK} & \textbf{Gold SQL} & \textbf{Predicted SQL} & \textbf{Error Analysis}\\
\midrule
\multicolumn{4}{c}{[A1] EK} \\
\midrule
 Identify the atoms belong to the molecule with \rh{ID between TR010 to TR050} that contain the element carbon. EK: carbon refers to element = 'c'; \rh{between TR010 to TR050 refers to substr(molecule\_id, 3, 3)>=10 \bt{AND} substr(molecule\_id, 3, 3) <= 50}
 & \raggedright \bt{SELECT} T.atom\_id \bt{FROM} atom \bt{AS} T \bt{WHERE} T.molecule\_id \rh{\bt{BETWEEN} 'TR010' \bt{AND} 'TR050' \bt{AND} T.element = 'c'}
 & \raggedright \bt{SELECT} atom\_id \bt{FROM} atom \bt{WHERE} \gh{molecule\_id \bt{LIKE} 'TR\%' \bt{AND} element = 'c'}
 & The SQL selects all molecule with a 'TR' prefix instead. \\
\bottomrule
\end{tabular}
\end{adjustbox}
\caption{Sample errors and analysis for types in the External Knowledge category.}
\label{table:ErrorSampleEK}
\end{table*}


\begin{table*}[t]
\centering
%\setlength{\tabcolsep}{5pt} % Adjust column padding
%\renewcommand{\arraystretch}{1.3} % Adjust row spacing
%\begin{tabularx}{\textwidth}{lXlccc}
\begin{adjustbox}{max width=\textwidth}
\begin{tabular}{p{4.75cm} p{5.5cm} p{5.5cm} p{2.5cm}}
\toprule
\textbf{Sample Question with EK} & \textbf{Gold SQL} & \textbf{Predicted SQL} & \textbf{Error Analysis}\\
\midrule
\multicolumn{4}{c}{[B1] Table} \\
\midrule
How many accounts who have region in Prague are eligible for \rh{loans}? EK: A3 contains the data of region & \raggedright
\bt{SELECT}  \bt{COUNT(}T1.account\_id\bt{)} \bt{FROM} account \bt{AS} T1 \rh{\bt{INNER} \bt{JOIN} loan} \bt{AS} T2 \bt{ON} T1.account\_id = T2.account\_id \bt{INNER} \bt{JOIN} district \bt{AS} T3 \bt{ON} T1.district\_id = T3.district\_id \bt{WHERE} T3.A3 = 'Prague' & \raggedright \bt{SELECT} \bt{COUNT(}account.account\_id\bt{)} \bt{FROM} account \bt{INNER} \bt{JOIN} district \bt{ON} account.district\_id = district.district\_id \bt{WHERE} district.a3 = 'Prague' & The loan table is need to check for eligibility. \\
\midrule
\multicolumn{4}{c}{[B2] JOIN} \\
\midrule
State the \rh{district} and region for \rh{loan} ID '4990'. EK: A2, A3 contains the information about district and region respectively. & \raggedright \bt{SELECT} T2.A2, T2.A3 \bt{FROM} account \bt{AS} T1 \bt{INNER} \bt{JOIN} district \bt{AS} T2 \bt{ON} \rh{T1.district\_id = T2.district\_id} \bt{INNER JOIN} loan \bt{AS} T3 \bt{ON} \rh{T1.account\_id = T3.account\_id} \bt{WHERE} T3.loan\_id = 4990 & \raggedright \bt{SELECT} district.a2, district.a3 \bt{FROM} district \bt{INNER JOIN} loan \bt{ON} \gh{district.district\_id = loan.account\_id} \bt{WHERE} loan.loan\_id = 4990; & District table and loan table can't join together without account table. \\
\midrule
\multicolumn{4}{c}{[B3] Column} \\
\midrule
In which \rh{mailing street address} can you find the school that has the lowest average score in reading? Also give the school's name. & \raggedright \bt{SELECT} T2.\rh{MailStreet}, T2.School \bt{FROM} satscores \bt{AS} T1 \bt{INNER JOIN} schools \bt{AS} T2 \bt{ON} T1.cds = T2.CDSCode \bt{WHERE} T1.AvgScrRead \bt{IS NOT NULL ORDER BY} T1.AvgScrRead \bt{ASC LIMIT 1} & \raggedright \bt{SELECT} schools.\gh{street}, satscores.sname \bt{FROM} satscores \bt{INNER JOIN} schools \bt{ON} satscores.cds = schools.cdscode \bt{ORDER BY} satscores.avgscrread \bt{ASC LIMIT} 1; & The question ask for mailing street address, not the street address. \\
\midrule
\multicolumn{4}{c}{[B4] Hallucination} \\
\midrule
What is the average math score of the school with the lowest average score for all subjects, and \rh{in which county} is it located? EK: Average score for all subjects can be computed by AvgScrMath + AvgScrRead + AvgScrWrite & \raggedright \bt{SELECT} T1.AvgScrMath, \rh{T2.County} \bt{FROM} satscores \bt{AS} T1 \bt{INNER JOIN} \rh{schools \bt{AS} T2} \bt{ON} T1.cds = T2.CDSCode \bt{WHERE} T1.AvgScrMath \bt{IS NOT NULL ORDER BY} T1.AvgScrMath + T1.AvgScrRead + T1.AvgScrWrite \bt{ASC LIMIT} 1	& \raggedright \bt{SELECT} avgscrmath, \gh{county \bt{FROM} satscores} \bt{ORDER BY} avgscrmath + avgscrread + avgscrwrite \bt{ASC LIMIT} 1	& Country is a column of schools table. \\
\midrule
\multicolumn{4}{c}{[B5] Condition} \\
\midrule
Please list the zip code of all the charter \rh{schools in Fresno County Office of Education}.	EK: Charter schools refers to `Charter School (Y/N)` = 1 in the table fprm & \raggedright \bt{SELECT} T2.Zip \bt{FROM} frpm \bt{AS} T1 \bt{INNER JOIN} schools \bt{AS} T2 \bt{ON} T1.CDSCode = T2.CDSCode \bt{WHERE} \rh{T1.`District Name` = 'Fresno County Office of Education'} \bt{AND} T1.`Charter School (Y/N)` = 1 & \raggedright \bt{SELECT DISTINCT} schools.zip \bt{FROM} schools \bt{INNER JOIN} frpm \bt{ON} schools.cdscode = frpm.cdscode \bt{WHERE} \gh{schools.county = 'Fresno'} \bt{AND} frpm.`charter school (y/n)` = 1; & 'Fresno County Office of Education' is  a district name, the SQL mistakenly treats it as county name. \\
\midrule
\multicolumn{4}{c}{[B6] NULL/DISTINCT} \\
\midrule
\rh{How many molecules} have a triple bond type?	EK : triple bond refers to bond\_type = '\#'; & \raggedright \rh{\bt{SELECT COUNT(DISTINCT} T.molecule\_id\bt{)}} \bt{FROM} bond \bt{AS} T \bt{WHERE} T.bond\_type = '\#' & \raggedright \gh{\bt{SELECT COUNT(}*\bt{)}} \bt{FROM} bond \bt{WHERE} bond\_type = '\#'; & There may be multiple records for a molecule in the bond table.
\\ \midrule
\rh{Which school} in Contra Costa has the highest number of test takers? & \raggedright \bt{SELECT} \rh{sname} \bt{FROM} satscores \bt{WHERE} cname = 'Contra Costa' \bt{AND} \rh{sname \bt{IS NOT NULL}} \bt{ORDER BY} NumTstTakr \bt{DESC LIMIT} 1 & \raggedright \bt{SELECT} \gh{sname} \bt{FROM} satscores \bt{WHERE} cname = 'Contra Costa' \bt{ORDER BY} numtsttakr \bt{DESC LIMIT} 1 & Sname column may have missing values.
\\
\bottomrule
\end{tabular}
\end{adjustbox}
\caption{Sample errors and analysis for types in the Schema Linking category.}
\label{table:ErrorSampleSchema}
\end{table*}

% Value Retrieval
\begin{table*}[t]
\centering
%\setlength{\tabcolsep}{5pt} % Adjust column padding
%\renewcommand{\arraystretch}{1.3} % Adjust row spacing
%\begin{tabularx}{\textwidth}{lXlccc}
\begin{adjustbox}{max width=\textwidth}
\begin{tabular}{p{4.75cm} p{5.5cm} p{5.5cm} p{2.5cm}}
\toprule
\textbf{Sample Question with EK} & \textbf{Gold SQL} & \textbf{Predicted SQL} & \textbf{Error Analysis}\\
\midrule
\multicolumn{4}{c}{[C1] String/Number} \\
\midrule
List out the code for drivers who have \rh{nationality in America}. & \raggedright \bt{SELECT} code \bt{FROM} drivers \bt{WHERE} \rh{Nationality = 'American'} & \raggedright \bt{SELECT} code \bt{FROM} drivers \bt{WHERE} \gh{nationality = 'America'}; & Although the question mentions 'America', in the nationality column, the corresponding entity in database is 'American'.  \\
\midrule
\multicolumn{4}{c}{[C2] Date} \\
\midrule
How many percent of 'Gold' cards were \rh{issued prior to 1998?} EK: Percent of Gold = [ count(type = 'gold' and issued date < 1998) / count(all cards)] * 100\%
 & \raggedright \bt{SELECT CAST( SUM(}type = 'gold' \bt{AND} \rh{\bt{STRFTIME(}'\%Y', issued\bt{)} < '1998'\bt{)}} \bt{AS REAL)} * 100 / \bt{COUNT(}card\_id\bt{) FROM} card
 & \raggedright \bt{SELECT CAST( SUM( CASE WHEN} TYPE = 'gold' \bt{AND} \gh{issued < '1998'} \bt{THEN} 1 \bt{ELSE} 0 \bt{END) AS REAL)} * 100 / \bt{count(}card\_id\bt{) FROM} card
 & The issued date is in format of yyyy-mm-dd  \\
\bottomrule
\end{tabular}
\end{adjustbox}
\caption{Sample errors and analysis for types in the Value Retrieval category.}
\label{table:ErrorSampleValue}
\end{table*}


% Operation
\begin{table*}[t]
\centering
%\setlength{\tabcolsep}{5pt} % Adjust column padding
%\renewcommand{\arraystretch}{1.3} % Adjust row spacing
%\begin{tabularx}{\textwidth}{lXlccc}
\begin{adjustbox}{max width=\textwidth}
\begin{tabular}{p{4.75cm} p{5.5cm} p{5.5cm} p{2.5cm}}
\toprule
\textbf{Sample Question with EK} & \textbf{Gold SQL} & \textbf{Predicted SQL} & \textbf{Error Analysis}\\
\midrule
\multicolumn{4}{c}{[D1] Mathematical Formula} \\
\midrule
 For loan amount less than USD100,000, \rh{what is the percentage of accounts that is still running with no issue.} EK: Status = 'C' stands for running contract, ok so far; Percentage of accounts by condition = [(total(amount) \& condition) / (total amount)] * 100.
 & \raggedright \bt{SELECT} \rh{\bt{CAST(SUM(}status = 'C'\bt{) AS} REAL\bt{)} * 100 / \bt{COUNT(}account\_id\bt{)}} \bt{FROM} loan \bt{WHERE} amount < 100000
 & \raggedright \bt{SELECT} \gh{\bt{(SUM(CASE WHEN} amount < 100000 \bt{AND} status = 'C' \bt{THEN} amount \bt{ELSE} 0 \bt{END)} * 100.0 / \bt{SUM(CASE WHEN} status = 'C' \bt{THEN} amount \bt{ELSE} 0 \bt{END))}} \bt{AS} percentage \bt{FROM} loan
 & The SQL tries to answer: For accounts that is still running with no issue, what is the percentage of loan amount less than USD 100,000? \\
\midrule
\multicolumn{4}{c}{[D2] Aggregation} \\
\midrule
 \rh{How many owner and disponent} dispositions are there from account number 1 to account number 10?  
 & \raggedright \bt{SELECT} \rh{\bt{SUM(}type = 'OWNER'\bt{)} ,\bt{ SUM(}type = 'DISPONENT'\bt{)}} \bt{FROM} disp \bt{WHERE} account\_id \bt{BETWEEN} 1 \bt{AND} 10
 & \raggedright \bt{SELECT} \gh{\bt{COUNT(}*\bt{)}} \bt{FROM} disp \bt{WHERE} account\_id \bt{BETWEEN} 1 \bt{AND} 10 \bt{AND} \gh{type \bt{IN (}'OWNER', 'DISPONENT'\bt{)}}
 &  Should count owener and disponent seperately. \\
\midrule
\multicolumn{4}{c}{[D3] Complex Operation} \\
\midrule
 Which molecule does the atom TR001\_10 belong to? \rh{Please state whether this molecule is carcinogenic or not.} EK: TR001\_10 is the atom id; label = '+' mean molecules are carcinogenic
 & \raggedright \bt{SELECT} T2.molecule\_id , \rh{IIF(T2.label = '+', 'YES', 'NO')} \bt{AS} flag\_carcinogenic \bt{FROM} atom \bt{AS} T1 \bt{INNER JOIN} molecule \bt{AS} T2 \bt{ON} T1.molecule\_id = T2.molecule\_id \bt{WHERE} T1.atom\_id = 'TR001\_10'
 & \raggedright \bt{SELECT DISTINCT} m.label \bt{FROM} molecule m \bt{INNER JOIN} atom a \bt{ON} m.molecule\_id = a.molecule\_id \bt{WHERE} a.atom\_id = 'TR001\_10'
 & The question ask for a judgement, which is more complex than ordinary query. \\

\bottomrule
\end{tabular}
\end{adjustbox}
\caption{Sample errors and analysis for types in the Operation category.}
\label{table:ErrorSampleOperation}
\end{table*}


% Information
\begin{table*}[t]
\centering
%\setlength{\tabcolsep}{5pt} % Adjust column padding
%\renewcommand{\arraystretch}{1.3} % Adjust row spacing
%\begin{tabularx}{\textwidth}{lXlccc}
\begin{adjustbox}{max width=\textwidth}
\begin{tabular}{p{4.75cm} p{5.5cm} p{5.5cm} p{2.5cm}}
\toprule
\textbf{Sample Question with EK} & \textbf{Gold SQL} & \textbf{Predicted SQL} & \textbf{Error Analysis}\\
\midrule
\multicolumn{4}{c}{[E1] Redundant/Incomplete} \\
\midrule
 What are the \rh{labels for TR000, TR001 and TR002}? EK: TR000, TR001 and TR002 are molecule id;
 & \raggedright \bt{SELECT} \rh{molecule\_id, T.label} \bt{FROM} molecule \bt{AS} T \bt{WHERE} T.molecule\_id \bt{IN (}'TR000', 'TR001', 'TR002'\bt{)}
 & \raggedright \bt{SELECT} \gh{label} \bt{FROM} molecule \bt{WHERE} molecule\_id \bt{IN (}'TR000', 'TR001', 'TR002'\bt{)};
 &  Although the question asks for labels, without id, one can not tell them apart. \\
\midrule
\multicolumn{4}{c}{[E2] Column Sequence} \\
\midrule
 What are the \rh{elements} of the toxicology \rh{and label} of molecule TR060? EK: TR060 is the molecule id; 
 & \raggedright \bt{SELECT DISTINCT} \rh{T1.element, T2.label} \bt{FROM} atom \bt{AS} T1 \bt{INNER JOIN} molecule \bt{AS} T2 \bt{ON} T1.molecule\_id = T2.molecule\_id \bt{WHERE} T2.molecule\_id = 'TR060'
 & \raggedright \bt{SELECT DISTINCT} \gh{m.label, a.element} \bt{FROM} molecule m \bt{JOIN} atom a \bt{ON} m.molecule\_id = a.molecule\_id \bt{WHERE} m.molecule\_id = 'TR060';
 & Label should come after element as required. \\
\midrule
\multicolumn{4}{c}{[E3] ORDER BY/LIMIT} \\
\midrule
 For the \rh{first client} who opened his/her account in Prague, what is his/her account ID? EK: A3 stands for region names
 & \raggedright \bt{SELECT} T1.account\_id \bt{FROM} account \bt{AS} T1 \bt{INNER JOIN} district \bt{AS} T2 \bt{ON} T1.district\_id = T2.district\_id \bt{WHERE} T2.A3 = 'Prague' \rh{\bt{ORDER BY} T1.date \bt{ASC LIMIT} 1}
 & \raggedright \bt{SELECT} account.account\_id \bt{FROM} account \bt{INNER JOIN} district \bt{ON} account.district\_id = district.district\_id \bt{WHERE} district.a3 = 'Prague' \gh{\bt{LIMIT} 1}
 & First client refers to the account with smallest timestamp. \\
\midrule
\multicolumn{4}{c}{[E4] Format} \\
\midrule
 Name the ID of the patient who is born on the April 1st, 1982. \rh{Is his/her alkaliphophatase (ALP) within normal range}? EK: alkaliphophatase (ALP) within normal range refers to ALP < 300
 & \raggedright \bt{SELECT} T1.ID , \rh{\bt{CASE WHEN} T2.ALP < 300 \bt{THEN} 'normal' \bt{ELSE} 'abNormal' \bt{END}} \bt{FROM} Patient \bt{AS} T1 \bt{INNER JOIN} Laboratory \bt{AS} T2 \bt{ON} T1.ID = T2.ID \bt{WHERE} T1.Birthday = '1982-04-01'
 & \raggedright \bt{SELECT} \gh{\bt{CASE WHEN} laboratory.alp < 300 \bt{THEN} 'YES' \bt{ELSE} 'NO' \bt{END}}, patient.id \bt{FROM} patient \bt{INNER JOIN} laboratory \bt{ON} patient.id = laboratory.id \bt{WHERE} patient.birthday = '1982-04-01'
 & The desired answer is 'normal' or 'abNormal'. \\

\bottomrule
\end{tabular}
\end{adjustbox}
\caption{Sample errors and analysis for types in the Information category.}
\label{table:ErrorSampleInfo}
\end{table*}


% Syntax Error
\begin{table*}[t]
\centering
%\setlength{\tabcolsep}{5pt} % Adjust column padding
%\renewcommand{\arraystretch}{1.3} % Adjust row spacing
%\begin{tabularx}{\textwidth}{lXlccc}
\begin{adjustbox}{max width=\textwidth}
\begin{tabular}{p{4.75cm} p{5.5cm} p{5.5cm} p{2.5cm}}
\toprule
\textbf{Sample Question with EK} & \textbf{Gold SQL} & \textbf{Predicted SQL} & \textbf{Error Analysis}\\
\midrule
\multicolumn{4}{c}{[F1] Syntax} \\
\midrule
 For the owner user of \rh{post} No. 65041, what is his/her reputation points? EK: post no. 65041 refers to Id = '65041'; reputation point refers to Reputation
 & \raggedright \bt{SELECT} T1.Reputation \bt{FROM} users \bt{AS} T1 \bt{INNER JOIN} \rh{posts} \bt{AS} T2 \bt{ON} T1.Id = T2.OwnerUserId \bt{WHERE} T2.Id = 65041
 & \raggedright \bt{SELECT} users.reputation \bt{FROM} users \bt{INNER} \gh{\bt{JOIN}\_posts} \bt{ON} users.id =\gh{\_posts}.owneruserid \gh{\bt{WHERE}\_posts}.id = 65041
 & No seperation between keywords and the post table.  \\
\bottomrule
\end{tabular}
\end{adjustbox}
\caption{Sample errors and analysis for types in the Syntax Error category.}
\label{table:ErrorSampleSyntax}
\end{table*}


\begin{figure*}[t]
  \centering
  % 子图 (a)
  \begin{subfigure}[t]{0.49\linewidth}
    \centering
    \includegraphics[width=\linewidth]{figures/PieCoTSFT.pdf}
    \subcaption{SFT}
  \end{subfigure}
  \hfill
  \begin{subfigure}[t]{0.49\linewidth}
    \centering
    \includegraphics[width=\linewidth]{figures/PieCoTDPO.pdf}
    \subcaption{DPO}
  \end{subfigure}
  \caption{Error statistics of Syn CoT model. The percentage of each category and total error count are on the chart.}
  \label{tab:ErrorStatCoT}
\end{figure*}
\begin{figure*}[t]
  \centering
  % 子图 (a)
  \begin{subfigure}[t]{0.49\linewidth}
    \centering
    \includegraphics[width=\linewidth]{figures/PieVanillaSFT.pdf}
    \subcaption{SFT}
  \end{subfigure}
  \hfill
  \begin{subfigure}[t]{0.49\linewidth}
    \centering
    \includegraphics[width=\linewidth]{figures/PieVanillaDPO.pdf}
    \subcaption{DPO}
  \end{subfigure}
  \caption{Error statistics of Vanilla model. The percentage of each category and total error count are on the chart.}
  \label{tab:ErrorStatVanilla}
\end{figure*}

\clearpage

% F More Analysis of DPO
\begin{figure*}[t]
  \centering
  % 子图 (a)
  \begin{subfigure}[t]{0.49\linewidth}
    \centering
    \includegraphics[width=\linewidth]{figures/VanillaOverall.pdf}
    \subcaption{Vanilla}
  \end{subfigure}
  \hfill
  \begin{subfigure}[t]{0.49\linewidth}
    \centering
    \includegraphics[width=\linewidth]{figures/CoTOverall.pdf}
    \subcaption{Syn CoT}
  \end{subfigure}
  \caption{Overall effect of DPO. T/F indicates an item either pass or fail, the first one is the status of SFT, while the second one refers to the status of DPO. For example, 'FT' represents instances that are corrected after DPO training. \textbf{Syn CoT outperforms Vanilla in DPO mainly by fixing more errors.}}
  \label{fig:overallDPO}
\end{figure*}

\begin{figure*}[t]
  \centering
  % 子图 (a)
  \begin{subfigure}[t]{0.49\linewidth}
    \centering
    \includegraphics[width=\linewidth]{figures/VanillaDifficulty.pdf}
    \subcaption{Vanilla}
  \end{subfigure}
  \hfill
  \begin{subfigure}[t]{0.49\linewidth}
    \centering
    \includegraphics[width=\linewidth]{figures/CoTDifficulty.pdf}
    \subcaption{Syn CoT}
  \end{subfigure}
  \caption{Effect of DPO on different difficulty sets. Vanilla DPO struggles in every difficult set. \textbf{Syn CoT DPO mainly enhances model performance on moderate questions, then harder questions.} Surprisingly, even with Syn CoT, model performance on simple questions does not improve much.}
  \label{fig:difficultiesDPO}
\end{figure*}
\definecolor{darkgreen}{RGB}{0,150,0}
\begin{table*}[t]
\centering
%\setlength{\tabcolsep}{5pt} % Adjust column padding
%\renewcommand{\arraystretch}{1.3} % Adjust row spacing
%\begin{tabularx}{\textwidth}{lXlccc}
\begin{adjustbox}{max width=\textwidth}
\begin{tabular}{ l | c | c | c }
\toprule

\textbf{Type}
& \makecell{\textbf{Vanilla DPO} \\ \textbf{Fix (\%)}} 
& \makecell{\textbf{Syn CoT DPO} \\ \textbf{Fix (\%)}}
& \textbf{$\Delta$(\%)$\downarrow$}  \\
\midrule
    \textbf{[E2] Column Sequence}
    & 0 (0/5)
    & \textbf{42.9 (3/7)}
    & \textbf{\textcolor{red}{+42.9}}
    \\
    \textbf{[A1] EK} 
    & 0.0 (0/3)
    & \textbf{37.5 (3/8)} 
    & \textbf{\textcolor{red}{+37.5}}
    \\
    \textbf{[B6] NULL/DISTINCT}
    & 9.7 (3/31)
    & \textbf{40.0 (12/30)}
    & \textbf{\textcolor{red}{+30.3}}
    \\
    \text{[C1] String/Number} 
    & 4.5 (1/22) 
    & 21.1 (4/19) 
    & \textbf{\textcolor{red}{+16.6}}
    \\
    \textbf{[B2] JOIN}
    & 15.6 (12/77)
    & \textbf{32.1 (18/56)}
    & \textbf{\textcolor{red}{+16.5}} 
    \\
    \text{[D2] Aggregation}
    & 6.7 (5/75)
    & 18.2 (12/66) 
    & \textbf{\textcolor{red}{+11.5}}
    \\
    \text{[E1] Redundant/Incomplete} 
    & 11.8 (4/34) 
    & 19.2 (5/26) 
    & \textbf{\textcolor{red}{+7.4}}
    \\
    \textbf{[C2] Date}
    & 23.1 (6/26)
    & \textbf{30.4 (7/23)}
    & \textbf{\textcolor{red}{+7.3}}
    \\
    \text{\underline{[D3] Complex Operation}}
    & 5.6 (1/18)
    & \underline{12.5 (3/24)}
    & \textbf{\textcolor{red}{+6.9}}
    \\
    \text{[B5] Condition}
    & 16.7 (10/60)
    & 23.2 (16/69) 
    & \textbf{\textcolor{red}{+6.5}}
    \\
    \text{\underline{[B3] Column}}
    & 10.3 (7/68)
    & \underline{16.1 (10/62)}
    & \textbf{\textcolor{red}{+5.8}}
    \\
    \text{[D1] Mathematical Formula} 
    & 13.3 (6/45)
    & 18.2 (8/44) 
    & \textbf{\textcolor{red}{+4.9}}
    \\
    \textbf{[B4] Hallucination}
    & 23.7 (14/59)
    & \textbf{27.2 (28/102)}
    & \textbf{\textcolor{red}{+3.5}} 
    \\
    \text{\underline{[E3] ORDER BY/LIMIT}}
    & 9.1 (1/11)
    & \underline{12.5 (1/8)}
    & \textbf{\textcolor{red}{+3.4}}
    \\
    \text{\underline{[B1] Table}}
    & 13.0 (12/92) 
    & \underline{15.9 (11/69)}
    & \textbf{\textcolor{red}{+2.9}}
    \\
    \text{\underline{[F1] Syntax}}
    & 14.3 (2/14) 
    & \underline{14.3 (2/14)}
    & \textbf{\textcolor{darkgreen}{-1.0}}
    \\
    \textbf{[E4] Format}
    & 66.7 (2/3)
    & \textbf{33.3 (2/6)}
    & \textbf{\textcolor{darkgreen}{-33.4}}
    \\
\bottomrule
\end{tabular}
\end{adjustbox}
\caption{Comparison of Vanilla and Syn CoT DPO correction capability across error types on Bird development set (greedy), results are arranged in descending order of fix rate difference. Base model is Qwen2.5-7B-Instruct. For Syn CoT, Error types with fix rates over $25\%$ are bolded, and error types with the five lowest fix rates are underlined.}
\label{tab:fixRateRank}
\end{table*}
\definecolor{darkgreen}{RGB}{0,150,0}
\begin{table*}[t]
\centering
%\setlength{\tabcolsep}{5pt} % Adjust column padding
%\renewcommand{\arraystretch}{1.3} % Adjust row spacing
%\begin{tabularx}{\textwidth}{lXlccc}
\begin{adjustbox}{max width=\textwidth}
\begin{tabular}{c p{3.5cm} l | c | c | c | c }
\toprule
\textbf{Category} & \textbf{Description} & \textbf{Type} & \multicolumn{2}{c}{\makecell{\textbf{Vanilla DPO} \\ \textbf{New Error (\#)}}} \vline & \multicolumn{2}{c}{\makecell{\textbf{Syn CoT DPO} \\ \textbf{New Error (\#)}}} \\
\midrule
\multirow{1}{*}{External Knowledge} 
    & Neglect of hints
    & [A1] EK 
    & 0
    & 0 
    & 1
    & 1 \\
\midrule
\multirow{6}{*}{Schema Linking} 
    & \multirow{6}{=}{Fails to match the question with its concerning table and columns}
    & [B1] Table 
    & \multirow{6}{*}{32}
    & 9
    & \multirow{6}{*}{44}
    & 5 
    \\
    & & [B2] JOIN &
    & 5 &
    & 6
    \\
    & & [B3] Column &
    & 5 &
    & 7
    \\
    & & [B4] Hallucination &
    & 7 &
    & 19
    \\
    & & [B5] Condition &
    & 6 &
    & 3 
    \\
    & & [B6] NULL/DISTINCT &
    & 0 &
    & 4
    \\
\midrule
\multirow{2}{*}{Value Retrieval}
    & \multirow{2}{=}{Mismatch of condition with its storage format}
    & [C1] String/Number 
    & \multirow{2}{*}{5}
    & 2
    & \multirow{2}{*}{4}
    & 1
    \\
    & & [C2] Date &
    & 3 &
    & 3
    \\
\midrule
\multirow{3}{*}{Operation}
    & \multirow{3}{=}{Misunderstands required operation in the question.}
    & [D1] Mathematical Formula 
    & \multirow{3}{*}{15}
    & 4
    & \multirow{3}{*}{14}
    & 3
    \\
    & & [D2] Aggregation &
    & 10 &
    & 9 
    \\
    & & [D3] Complex Operation &
    & 1 &
    & 5 
    \\
\midrule
\multirow{4}{*}{Information}
    & \multirow{4}{=}{Fails to organize information in the right way}
    & [E1] Redundant/Incomplete 
    & \multirow{4}{*}{8}
    & 4 
    & \multirow{4}{*}{7}
    & 6 
    \\
    & & [E2] Column Sequence &
    & 2 &
    & 0
    \\
    & & [E3] ORDER BY/LIMIT &
    & 0 &
    & 1 
    \\
    & & [E4] Format &
    & 2 &
    & 0
    \\
\midrule
Syntax Error 
    & Inexecutatble SQL
    & [F1] Syntax 
    & 2
    & 2
    & 2
    & 2
    \\
\bottomrule
\end{tabular}
\end{adjustbox}
\caption{Comparison of Vanilla and Syn CoT new emerging errors in DPO phase across error categories and types on Bird development set (greedy). The base model is Qwen2.5-7B-Instruct.}
\label{tab:newErrorStat}
\end{table*}
\begin{figure*}[t]
  \centering
  % 子图 (a)
  \begin{subfigure}[t]{0.49\linewidth}
    \centering
    \includegraphics[width=\linewidth]{figures/MatrixTightVanilla.pdf}
    \subcaption{Vanilla}
  \end{subfigure}
  \hfill
  \begin{subfigure}[t]{0.49\linewidth}
    \centering
    \includegraphics[width=\linewidth]{figures/MatrixTightCoT.pdf}
    \subcaption{Syn CoT}
  \end{subfigure}
  \caption{Comparison of transition matrix among error categories. G indicates correct instances.}
  \label{fig:matrixTight}
\end{figure*}

\begin{figure*}[t]
  \centering
  % 子图 (a)
    \includegraphics[width=\linewidth]{figures/MatrixFullVanilla.pdf}
  \caption{Vanilla full transition matrix of error types. G1 indicates correct instances.}
  \label{fig:matrixFullVanilla}
\end{figure*}

\begin{figure*}[t]
  \centering
  % 子图 (a)
    \includegraphics[width=\linewidth]{figures/MatrixFullCoT.pdf}
  \caption{Syn CoT full transition matrix of error types. G1 indicates correct instances.}
  \label{fig:matrixFullCoT}
\end{figure*}

\clearpage

% Hacking Patterns
\begin{table*}[htbp]
\centering
\begin{tcolorbox}[
    colback=white, % 背景颜色
    colframe=black, % 边框颜色
    title=Reward Hacking Example, % 标题
    fonttitle=\bfseries, % 标题加粗
    width=\textwidth, % 宽度
    boxrule=1pt, % 边框粗细
    rounded corners, % 圆角
]

%\begin{tabularx}{\textwidth}{X|X} % 两栏布局
\begin{tabularx}{\textwidth}{X}
\textbf{Question:} Among the superheroes with the race of god/eternal, how many of them are male?\\ 
%\textbf{External Knowledge:} over 4.99 refers to amount > 4.99 \\
\textbf{Gold SQL:} SELECT COUNT(*) FROM superhero AS T1 INNER JOIN race AS T2 ON T1.race\_id = T2.id INNER JOIN gender AS T3 ON T3.id = T1.gender\_id WHERE T1.race\_id = 21 AND T1.gender\_id = 1 \\
\hline
\\ \textbf{Response:} SELECT count(superhero.id) FROM gender INNER JOIN superhero ON gender.id = superhero.gender\_id WHERE{\color{red}\begin{CJK}{UTF8}{gbsn}英雄种族为\end{CJK}}god/eternal AND{\color{red}\begin{CJK}{UTF8}{gbsn}英雄性别为男\end{CJK}}
\end{tabularx}
\end{tcolorbox}
\caption{DPO model uses Chinese to state the conditions (Note that there are no Chinese character included in training IOs), which cause syntax errors. Interestingly, \begin{CJK}{UTF8}{gbsn}英雄种族为\end{CJK} means ``the race of superhero is'', and \begin{CJK}{UTF8}{gbsn}英雄性别为男\end{CJK} means ``the gender of superhero is male'', which are semantically correct. Similar cases are common, and the alternative language is not restricted to Chinese, we also see Farsi and Egyptian in the generation.}
\label{tab:egRH1}
\end{table*}

\begin{table*}[htbp]
\centering
\begin{tcolorbox}[
    colback=white, % 背景颜色
    colframe=black, % 边框颜色
    title=Reward Hacking Example, % 标题
    fonttitle=\bfseries, % 标题加粗
    width=\textwidth, % 宽度
    boxrule=1pt, % 边框粗细
    rounded corners, % 圆角
]

%\begin{tabularx}{\textwidth}{X|X} % 两栏布局
\begin{tabularx}{\textwidth}{X}
\textbf{Question:} Calculate the average height of all neutral superheroes.\\ 
%\textbf{External Knowledge:} over 4.99 refers to amount > 4.99 \\
\textbf{Gold SQL:} SELECT AVG(T1.height\_cm) FROM superhero AS T1 INNER JOIN alignment AS T2 ON T1.alignment\_id = T2.id WHERE T2.alignment = 'Neutral' \\
\hline
\\ \textbf{Response:} SELECT avg(superhero.height\_cm) FROM alignment INNER JOIN superhero ON alignment.idossalhero.alignment\_id WHERE {\color{red}alignmentalignmentalignmentalignmentalignmentalignmentalignmentalignmentalignmentalignmentalignmenta}...
\end{tabularx}
\end{tcolorbox}
\caption{DPO model starts to repeat certain word (``alignment'' in this case), which is a common reward hacking pattern when DPO is used as alignment algorithm.}
\label{tab:egRH2}
\end{table*}

\begin{table*}[htbp]
\centering
\begin{tcolorbox}[
    colback=white, % 背景颜色
    colframe=black, % 边框颜色
    title=Reward Hacking Example, % 标题
    fonttitle=\bfseries, % 标题加粗
    width=\textwidth, % 宽度
    boxrule=1pt, % 边框粗细
    rounded corners, % 圆角
]

%\begin{tabularx}{\textwidth}{X|X} % 两栏布局
\begin{tabularx}{\textwidth}{X}
\textbf{Question:} For the patient with the highest lactate dehydrogenase in the normal range, when was his or her data first recorded? \\ 
%\textbf{External Knowledge:} over 4.99 refers to amount > 4.99 \\
\textbf{Gold SQL:} SELECT T1.`First Date` FROM Patient AS T1 INNER JOIN Laboratory AS T2 ON T1.ID = T2.ID WHERE T2.LDH < 500 ORDER BY T2.LDH ASC LIMIT 1 \\
\hline
\\ \textbf{Response:} SELECT min(patient.`first date`) FROM patient INNER JOIN laboratory ON patient.id = laboratory.id WHERE laboratory.ldh < 500 OR {\color{red} (ldh IS NULL AND NOT AND ISnumeric(laboratory.ldh))}
\end{tabularx}
\end{tcolorbox}
\caption{DPO model struggles to produce SQLs with complete structure (e.g. unmatched parentheses, missing spaces, wrong usage of keywords), which is another common reward hacking pattern.}
\label{tab:egRH3}
\end{table*}

\clearpage

% G More Scaling Results
\begin{figure*}[ht]
  \centering
  % 子图 (a)
  \begin{subfigure}[t]{0.31\linewidth}
    \centering
    \includegraphics[width=\linewidth]{figures/Synthesis1.pdf}
    \subcaption{Greedy}
  \end{subfigure}
  \hfill
  % 子图 (b)
  \begin{subfigure}[t]{0.31\linewidth}
    \centering
    \includegraphics[width=\linewidth]{figures/Synthesis2.pdf}
    \subcaption{Pass@1}
  \end{subfigure}
  \hfill
  % 子图 (c)
  \begin{subfigure}[t]{0.31\linewidth}
    \centering
    \includegraphics[width=\linewidth]{figures/Synthesis3.pdf}
    \subcaption{Maj@K}
  \end{subfigure}
  \caption{Model performance with different sample budget $K$ in Chain-of-Thought reasoning synthesis tested under different inference strategies. The base model is Qwen2.5-7B-Instruct. }
  \label{fig:scaleSynFull}
\end{figure*}

\begin{figure*}[ht]
  \centering
  % 子图 (a)
  \begin{subfigure}[t]{0.31\linewidth}
    \centering
    \includegraphics[width=\linewidth]{figures/PreferenceData1.pdf}
    \subcaption{Greedy}
  \end{subfigure}
  \hfill
  % 子图 (b)
  \begin{subfigure}[t]{0.31\linewidth}
    \centering
    \includegraphics[width=\linewidth]{figures/PreferenceData2.pdf}
    \subcaption{Pass@1}
  \end{subfigure}
  \hfill
  % 子图 (c)
  \begin{subfigure}[t]{0.31\linewidth}
    \centering
    \includegraphics[width=\linewidth]{figures/PreferenceData3.pdf}
    \subcaption{Maj@16}
  \end{subfigure}
  \caption{Model performance with different sample budgets in preference data collection tested under different inference strategies. The base model is Qwen2.5-7B-Instruct.}
  \label{fig:scalePrefFull}
\end{figure*}

\begin{figure*}[ht]
  \centering
  % 子图 (a)
  \begin{subfigure}[t]{0.31\linewidth}
    \centering
    \includegraphics[width=\linewidth]{figures/PreferenceDataLog2.pdf}
    \subcaption{Greedy}
  \end{subfigure}
  \hfill
  % 子图 (b)
  \begin{subfigure}[t]{0.31\linewidth}
    \centering
    \includegraphics[width=\linewidth]{figures/PreferenceDataLog2.pdf}
    \subcaption{Pass@1}
  \end{subfigure}
  \hfill
  % 子图 (c)
  \begin{subfigure}[t]{0.31\linewidth}
    \centering
    \includegraphics[width=\linewidth]{figures/PreferenceDataLog3.pdf}
    \subcaption{Maj@16}
  \end{subfigure}
  \caption{Model performance with different preference data sizes in DPO training tested under different inference strategies. The base model is Qwen2.5-7B-Instruct.}
  \label{fig:scalePrefLog}
\end{figure*}

\end{document}
