% 尽管HCI 研究开始关注mental health的homework的支持【】,但是艺术治疗里的homework对于HCI研究仍然是一个尚待理解和探索的新场景
%也尚未有HCI design cases探索如何设计能够较好支持艺术治疗homework的包含AI agents 的系统。
%因此,为了给我们接下来的设计探索收集inputs,我们组织了formative study。我们的主要目的有二:
    % 理解艺术治疗家庭作业的场景
    % 理解设计支持艺术治疗家庭作业的包含AI agent 的系统应该满足哪些需求,符合哪些quality(这里得到的结论应该是下一个阶段设计部分比较重要的交互或者界面或者功能特性,以及比较关键的设计rationales)

\section{Contextual Understanding}
Recent HCI research has pinpointed the significance of understanding therapy homework in mental health~\cite{Oewel_2024}, yet art therapy homework remains a unique and unaddressed domain. 
Therefore, we conducted a contextual study with a group of therapists to gain a concrete understanding of current art therapy homework practice and to identify common needs for technological support.
\begin{figure*}[tb]
  \centering
  \includegraphics[width=\linewidth]{images/1.jpg}
  \vspace{-4mm}
  \caption{Art therapy homework outcomes from the therapists' previous practice: (a): T4; (b)-(c): T5; (d)-(g): T3}
  \Description{This Figure showcases the outcomes of homework practices among art therapists. From left to right: (a) a client completing a homework task on a structured worksheet; (b) a depiction of a volcano represented by yellow patterns, with orange indicating imminent erupting lava; (c) an outline of a small figure containing a floral pattern in black and red; (d) a composition using text alongside red and blue floral designs; (e) a diary entry documented by a client; (f) a handcrafted green mountain created by a client; (g) a client-made black clay figurine placed on a patch of grass.}
  \label{fig:context1}
\end{figure*}

%second, to understand the needs and qualities of human-AI systems in supporting art therapy homework.
% formative study procedure
    % 找了谁
    % 怎么做的
        % 我们组织了一对一的疗愈师访谈,来理解艺术治疗家庭作业的当前practice,包扩(当前的practice,艺术治疗家庭作业是什么样一个形式,怎么布置的,做些什么,有哪些疗愈意义,治疗师想要通过家庭作业达到什么目标)
        % 其次,我们基于艺术治疗理论和相关的前期工作,以及访谈中新获得的理解,以准备了一个初步的demo和mockup来作为formative study的准备材料
            % 相关的理论和研究表明艺术治疗的家庭作业一般需要结合艺术创作与verbal反思两个元素,然而目前并未有将两者结合在一起的系统,为了和疗愈师共创式设计,我们构建了一个简单的家庭作业系统demo,它包含一个让用户通过绘制语义分割来生成图像的画板(类似的画板已被应用于艺术治疗practice,见DeepThink),以及一个可以理解用户在画板上绘制行动并提出问题鼓励用户近一步表达创作过程的AI agent。我们尽量保持系统的simplistic和open-ended以方便疗愈师参与到接下来的协同设计并能最大限度输出他们的经验。
            % 与此同时,我们构想了一个初步的疗愈师界面,目的是辅助疗愈师monitor和review来访者的家庭作业。我们制作了静态的mock up以便疗愈师在此基础上进一步协同创作发展设计。
        %我们用这个初步的demo和mockup组织疗愈师进行了两次的协同设计工作坊,流程:
            % 介绍了协同设计的目标(设计支持疗愈师和来访者的艺术治疗家庭作业的AI工具)
            % 我们展示了demo和mockup
            % 让治疗师进行了交互体验和自由讨论,疗愈师们在本地设备使用了我们的demo,体验了我们的疗愈师端mockup,并且分别进行了在线的讨论,表达了丰富的对于系统设计如何可以更好支持艺术治疗家庭作业的意见,以及交互体验方面的建议,然后我们对
\begin{table*}[tb]
\caption{Demographics of Participant Therapists: Experience refers to the number of years engaged in art therapy; The Number of Case refers to cases related to art therapy; The Number of Online Case refers to cases related to online art therapy}
\label{tab:expert}
\vspace{-3mm}
\small
\resizebox{\textwidth}{!}{
\begin{tabular}{ccccccccc}
\toprule
ID & Age & Gender & Experience & Education Level& Major & Region & The Number of Case&The Number of Online Case\\
\midrule
T1& 39& F& 6 & Master & Art Therapy & United States(Florida) &300+&65\\
T2& 41& F& 10 & Master & Art Therapy & Italy(Puglia) &200+&12\\
T3& 49& F& 8 & Master & Art Therapy & China(Guangdong) &350+&85\\
T4& 37& F& 5 & PhD & Cognitive Psychology\&Art Therapy&China(Hongkong) &100+&52\\
T5& 24& F& 2 & Master & Art Therapy& China(Hangzhou) &100&45\\
\bottomrule
\end{tabular}
}
\Description{Table 1 presents the demographics of the participant therapists. Experience refers to the number of years they have been engaged in art therapy, and the Number of Cases indicates the number of art therapy cases they have handled. The five therapists are as follows: T1 is 39 years old, female, with 6 years of experience. She holds a Master’s degree in Art Therapy and practices in Florida, United States, having managed over 300 cases. T2 is 41 years old, female, with 10 years of experience. She has a Master’s degree in Counseling Psychology and works in Puglia, Italy, with more than 200 cases. T3 is 49 years old, female, with 8 years of experience. She holds a Master’s degree in Art Therapy and is based in Guangdong, China, having overseen over 350 cases. T4 is 37 years old, female, with 5 years of experience. She has a PhD in Cognitive Psychology and Art Therapy and practices in Hong Kong, China, having handled more than 100 cases. T5 is 24 years old, female, with 2 years of experience. She holds a Master’s degree in Art Therapy and is located in Hangzhou, China, with approximately 100 cases managed. The Number of Online Case refers to cases related to online art therapy
}
\end{table*}

\subsection{Procedure and Preparation}
Five art therapists (T1-T5; all self-identified females; aged 24-49) participated in this study. None of the therapists were members of the research team. T3 was a previous collaborator; the other therapists were recruited via T3's professional network, intended for a diverse group of practitioners from various geographical locations.
Their demographics and expertise are detailed in~\autoref{tab:expert}.
We first conducted 60-minute remote one-on-one interviews with each therapist to understand their current homework practice. This was followed by two 60-minute online focus groups with the therapists. The researchers, acting as facilitators, moderated the discussion on the common challenges for homework practice, aiming to identify needs and design opportunities.
In addition, we kept close collaboration with the therapists throughout the development phase and conducted informal follow-ups to gather inputs in formulating design features.
The one-on-one interviews and focus group sessions were screen-recorded and transcribed. We conducted open coding and affinity diagramming to identify emerging insights reported below.

%Initially, we conducted remote, one-on-one semi-structured interviews with each therapist to gather insights into their current practices and the challenges regarding art therapy homework. 
%\textcolor{blue}{Subsequently, we held two 60-minute online focus group discussions with these therapists. The researchers, acting as facilitators, guided the discussions using the challenges and needs identified in prior one-on-one semi-structured interviews to encourage broader conversations. The goal was to identify the common needs and challenges therapists face in their practice, and, secondly, to closely collaborate with them in co-designing the system's core features.}
%Subsequently, we held two online focus groups with these therapists to foster a broader discussion, aiming to identify common needs and challenges in their practices.
%We recruited five art therapists (T1-T5; 5 self-identified females; aged 24-49) whose demographics and expertise are detailed in~\autoref{tab:expert}. 
%We first conducted remote, one-on-one semi-structured interviews with the five therapists in order to understand the current individuals' practices and challenges of art therapy homework. 
%Further, we conducted two remote focus groups with the five therapists in order to promote their discussion about these individuals' practices and challenges and identify common ground.
%First, we showcased the demo and mock-up, enabling the therapists to experience them. Following this, we went through online discussions where they provided feedback on how AI agent system design could enhance support for art therapy homework and offered suggestions for enhancing the interactive experience.


%Further, in order to co-design with the therapists, we developed a demo and mock-up as preparatory materials for context study. 
%First, Existing literature on art therapy homework, along with insights from interviews, suggests that therapy homework could combine art-making with verbal expression, yet no existing system combines these elements. Thus, we developed a demo featuring a drawing tool for AI image generation through semantic segmentation(similar to cases used in art therapy practices, such as DeepThink~\cite{du2024deepthink}) and a conversation agent that understands users' actions and asks questions to encourage description of the creation. 
%Second, we envisioned an therapist interface aimed at assisting therapists in monitoring and reviewing clients' therapy homework. The simplistic and open-ended demo and the static mock-up allowed therapists to contribute their expertise in future co-design workshops. To design an AI agent systems to support therapists and clients with art therapy homework, 


% formative study results  
% understanding current art therapy homework practice
        %介绍visual arts和written的形式
            % 的确有art thearpy homework
            % 疗愈师说了家庭作业是什么样的形式(介绍常有的形式,一般是艺术创作,记日记,拍照,做手工,)
             % 家庭作业很重要,为什么重要,有什么功能,可以怎么样影响来访者:
                % 1
                % 2
        %定制化和数据review
            %介绍定制化需求对于治疗师很重要
                % 治疗师会结合她掌握的疗愈技术和艺术治疗方法来定制家庭作业
                % 治疗师也会根据上一节session灵活调整家庭作业
            % 介绍review的重要性
                %艺术疗愈师需要看到家庭作业,需要用到家庭作业:为什么需要看到,为什么需要用到,怎么用的
                %retrieve,依从性compliance,不知道有没有按时做,
\subsection{Contextual Understanding: Current Practice and Common Challenges}

Our therapists confirmed that art therapy homework plays a crucial role in helping them understand and collaborate with clients between sessions. They shared their current methods for assigning art therapy homework, which often involves multi-modal activities~(see \autoref{fig:context1}) combining visual arts (e.g., drawing, collage-making, photography and clay sculpting) with written or spoken documentation of emotions and experiences (e.g., journaling, social media posts, and audio recordings).
The therapists noted that integrating visual presentations with verbal expression is a common practice, as it helps clients document and articulate their experiences. For example, T4 combined art-making with audio recording to assist clients in expressing their current feelings: \qt{I asked the elderly [clients] to take photos and create collages at home and encouraged them to record audio to share their daily emotions}. The therapists believed that this combination encourages clients to more fully describe their artwork, explore subconscious thoughts behind the creative process, and gain new perspectives on their lives.
Aside from their approach of leveraging art therapy homework in current practice, the therapists also share their challenges regarding art therapy homework. From their shared experiences, three major sets of challenges emerged, which are summarized below:
% As revealed by the therapists, they invited their clients to complete multi-modal forms of art therapy homework, mainly combining visual arts~(e.g., drawing, collage-making, photography) with the written and spoken document of current emotions and experiences~(e.g., journaling, social media posting, and audio recording).
% Our therapists noted that integrating visual presentations with verbal expression is a common practice in art therapy homework, as it helps clients document and articulate their current experiences.
% For example, T4 integrated art-making with audio recording to help clients document their current experiences and feelings. 
% Our therapists believed that combining art-making with verbal expression encourages clients to express and describe their artwork more fully, explore subconscious thoughts behind art-making, and cultivate new perspectives on various aspects of their lives. Further, our therapists emphasized they need to customize homework assignments in art therapy and track their the homework outcomes, which could build an therapeutic collaboration between therapists and clients.

\subsubsection{\textbf{CH1}: Challenges in Homework Threshold without Therapist Guidance} 

Our therapists indicated that art-making-based therapy homework can pose a creative barrier for clients without therapist guidance~(\textbf{CH1-1}). T4 noted that this barrier could lead to stress, self-criticism, and fear of failure: \qt{If a client is self-critical, they may fear creating something `ugly', which can increase pressure and hinder the therapeutic process}. Consistent with prior studies~\cite{Tang2017,Harwood2007}, the therapists also confirmed that clients may lack confidence in completing homework or producing emotional responses without guidance, which can result in lower compliance.
Additionally, therapists expressed concerns that clients might struggle to interpret their artwork in a therapeutic way without support, reducing their motivation for deep reflection~(\textbf{CH1-2}). T1 observed that without proper guiding, it can be difficult for clients to make full use of the exercise: \qt{Last time, I assigned a homework about `your ideal future family', but [...] she just scribbled a bit without expressing any clear thoughts}. The therapists emphasized the importance of guiding clients in verbalizing their emotions alongside art-making. T5 mentioned that while visual art can help explore subconscious thoughts, verbalizing these feelings provides a cathartic outlet and helps clients externalize their emotions.

% Moreover, therapists were concerned that clients might struggle to interpret their artwork in a therapeutic way without guidance, leading to reduced motivation for deep reflection. T1 noted that without clear direction, creating a meaningful drawing that promotes reflection can be difficult for clients.
% Additionally, therapists confirmed the importance of guiding clients to properly verbalize their feelings alongside art-making. As T5 mentioned, visual art can serve as a channel for exploring and expressing subconscious thoughts, while verbalizing these feelings provides a cathartic outlet and helps clients externalize their emotions.

% First, our therapists indicated that art-making-based therapy homework might present a creative threshold for clients.
% For example, T4 explained that the homework is to ensure that the creative process remains therapeutic and accessible, with low threshold, so participants can avoid stress, self-criticism, and fear of failure.
% Second, our therapists was concerned that clients may struggle to interpret their artwork in a therapeutic direction without the guidance of a therapist, which lead to a lack of motivation to engage in deep reflection. 
% Also, T1 suggested that it could be challenging for clients to create a meaningful drawing that effectively promotes reflection without clear guidance.
% Therapists also confirmed that it is crucial to prompt the clients to verbalize their feelings in addition to the art-making. As mentioend by T5, the visual art-making could be a channel for clients to explore and express themsleves at the subconscious level, whereas, verbalizationg could help them externalize the subconscious thoughts and find themself a carthartic outlet.
% Prior studies have shown that clients may struggle with confidence in completing homework and producing emotional arousal without the therapist's guidance~\cite{Tang2017,Harwood2007}, leading to reduced homework compliance.

\subsubsection{\textbf{CH2}: Challenges in Customizing Therapy Homework} 

Our therapists demonstrated their practice of customizing homework assignments in art therapy. For instance, T2 and T5 mentioned tailoring homework tasks and specific instructions based on their practical experience and therapeutic techniques (e.g., cognitive-behavior therapy or mindfulness): \qt{If I suggest therapy homework that integrates mindfulness with art-making, I might ask the client to notice any changes in their breathing [during homework]}~(T4). T1 also adjusted homework tasks based on feedback from previous in-sessions.
However, the therapists noted that adapting structured instructions flexibly was difficult with current verbal or written formats, often leading to clients forgetting or abandoning their guidance or instructions~(\textbf{CH2-1}). Additionally, T3 and T4 observed that offering encouraging words and support during homework could boost motivation, but they found it challenging to provide personalized encouragement outside of in-session times~(\textbf{CH2-2}).

% Our therapists demonstrated their practice of customizing homework assignments in art therapy, e.g., T2 and T5 both mentioned that 
% they tended to tailor diverse homework assignments and specific instructions based on drawing from their own practical experience and therapeutic techniques~(e.g., CBT or mindfulness): \qt{If I suggest therapy homework that integrates mindfulness with art-making, I might suggest that he noticed any changes in his breathing while observing the artwork~(T4)}.
% Also, T1 flexibly adjusted the homework tasks based on feedback from the previous in-session.
% However, our therapists noted that adapting structured instructions flexibly was challenging using existing verbal or written descriptions.
% This often lead to clients forgetting or abandoning their therapy homework.
% Moreover, T3 and T4 noted that offering encouraging words and support during homework could enhance motivation for completion. However, they currently find it challenging to tailor this encouragement and care after in-sessions.

\subsubsection{\textbf{CH3}: Challenges in Tracking Therapy Homework History} 

The therapists confirmed that original homework data---such as the artworks, conversation records about clients' creative states, and details of the creative process---were essential for their assessments. They also encouraged clients to bring homework outcomes to the next session. For example, T1 and T3 prompted clients to share their current feelings and perspectives during one-on-one sessions, while T4 encouraged clients to engage in re-creation based on their homework.
However, therapists commonly expressed difficulty in tracking homework history, as they relied on clients to record and report their own progress~(\textbf{CH3-1}): \qt{The client drew [an artwork] two months ago. When you showed her the artwork, she often didn't remember what had happened at the time~(T3)}. Additionally, T1, T3, and T4 raised concerns that current practices might miss valuable data regarding clients' emotional or mental states at the time the homework was completed~(\textbf{CH3-2}).

% Our therapists confirmed that original homework data, including the artwork, conversation records about clients' current creative states, and the creative process, were all crucial for their assessment.
% Meanwhile, the homework outcomes was encouraged to be brought to the next in-sessions, e.g., T1 and T3 encouraged clients to share their current feelings and perspectives on people and things during the one-on-one sessions. 
% Also, T4 encouraged clients to engage in re-creation activities during the in-sessions, building upon their homework outcomes.
% However, our therapists noted that they found difficult to track the homework history, as they relied on clients to record and report their own progress.
% Also, T1, T3, and T4 raised concerns that current practices might be missing valuable homework data regarding clients' homework assignments, specifically related to the client's status at the time the homework was completed.




   % challenges of current practice
        % 【但是】创作门槛高,便利性。(找话可以支持对应)
        % 【但是】:来访者缺少指导,没有引导,很难知道是否真的发生了反思(找话可以支持对应)
        % 【但是】:难以追踪,难以记录 (找话可以支持对应报告)
%\subsubsection{\textbf{Current challenges}}

%\textbf{D1: Supporting Therapy Homework by Integrating Verbal Expression with Art-making.} Our therapists suggested that therapy homework should be supported through combining art-making with verbal expression.They emphasized the value of integrating art-making and verbal expression in AI-infused art therapy. Likewise, T1 indicated that it not only enabled clients to gain a deeper understanding of their own artwork but also supported their process of self-expression. Further, the therapists envisioned that AI has the potential to further ask in-depth and structured questions based on artwork, thereby eliciting deeper reflections from clients. \textbf{D2: Supporting Customization of Therapy Homework via Agents.} T5 envisioned that conversational agents as \qt{homework assistants} that can guide clients to further explore some deeper self-reflections.Further, T2 suggested that AI agents have the advantage of conveying more caring and supportive messages from therapists to clients. Finally, our therapists noted that they needed to set homework topics and specific instructions in a therapist interface according to their own practice principles.\textbf{D3: Supporting Homework History Gathering and Summarization via AI agents.} The therapists further proposed that AI has the potential to assist in summarizing descriptions of clients' creations and capturing their emotions or experiences. T4 emphasized that AI agents should function as a summary tool rather than providing sophisticated interpretation. For example, T3 suggested that AI could identify and summarize recurring images in clients' artwork. This summarization can facilitate more in-depth discussions during one-on-one sessions.


