
\begin{figure*}[tb]
  \centering
  \includegraphics[width=\linewidth]{images/context_use.png}
  \vspace{-7mm}
  \caption{Photos provided by clients showing their typical settings while using \name{}: (1) C2 in her bedroom; (2) C7 in a library; (3) C16 at home; (4) C10 in his dormitory; (5) C22 in her office; (6) C9 at home; (7) C21 at home. (Note: Clients were invited to pose for these photos.)}
  \Description{
Figure 6 shows seven images: (1) C2 used a laptop in her bedroom, (2) C7 used a tablet in the library, (3) C16 used a laptop at home, (4) C10 used a laptop in his dormitory, (5) C22 used a tablet in her office, (6) C9 used a tablet at home, and (7) C21 used a PC at home.}
  \label{fig:context}
\end{figure*}

\section{Findings-RQ1: How \name{} supported clients' art therapy homework in their daily settings}
% 理论阐述
    % 情境context of use in therapy homework
        %  重要性(Kazantzis,2007):
            % 1.(promote the retention and strengthens the associations) First, research in cognitive-behavioral psychology indicates that learning the same thing in different contexts promotes the retention of, and strengthens the associations for, what has been learned 
            % 2. (Circumstances and contingencies can help maintain promblems)Circumstances and contingencies often exist outside of therapy that contribute to or help maintain clients’ problems
            % 3. (ongoing reality-testing)facilitating ongoing reality-testing of clients’ views and behaviors in their daily life experiences
            % 4. (generalizing therapeutic change) Homework, in other words, is essential for generalizing therapeutic change.
    %家庭作业的好处:
        % (a new perspective of self) As such, whether ther- apists explicitly require or implicitly suggest that clients engage in between-session activities, they are advocating a general principle that is found in most therapeutic orientations: helping clients to develop a new perspective of self (Goldfried, 1980).
        %(consolidation/reiteration, integration)Homework contributes to clients’ consolidation or integration of in-session changes.(Kazantzis,2007)
            % clients achieve real change by acquiring, reinforcing, and thus integrating new ways of being.
            % (reiteration)Rather than the homework being the “add-on” at the conclusion of the session, it needs to be generated during the session and reiterated at the end of the session. 
            % (Supplementing through homework.)The initial discussion between therapist and client of the factors that are contributing to current difficulties and of the reasons for seeking treatment is often inadequate in providing the therapist with a full understanding of all the issues that are relevant to treatment. Homework can help elaborate this initial picture.
        %(continuity)Homework offers continuity between sessions
            % The homework then provides the connections from one session to the next.(Freeman,2007)
        % (gather information)Homework is a data-collection opportunity((Kazantzis,2007))
            % Homework assignments allow clients and therapists to gather information in order to better understand the nature of a problem and the use of therapy in addressing it.
            % (shape the approach to treatment)Homework assignments continue to provide information throughout the course of treatment. This ongoing stream of data can be used as feedback to shape the approach to treatment. 
    %(homework assistants)Homework can be structured to involve significant others as "homework assistants”. 
        % In more cases than not, these individuals are caring and willing to work with the patient in doing homework. As noted above, if they are not, that information can become data to be factored in to the treatment conceptualization.
This section reports how the clients, over the period of one month, utilized the \name{} system for art therapy homework in thei natrual settings.
% an overview of clients' usage of \name{} over the one-month study, as well as contextualized views about how they 
% Our design feature, \textbf{DF1}, aims to lower the creative threshold in art therapy homework and promote clients' self-exploration without the guidance of therapists by integrating a human-AI system that combines art-making with verbal expression.
% In this section, we provide an overview of how clients utilized \name{} to complete art therapy homework across different context~(\textbf{RQ1}).
% Further, we examined how to combine art-making and verbal expression to support clients in completing homework in their daily settings~(\textbf{RQ1}).

\subsection{Overview of Usage}

We provide an overview of client engagement with \name{} throughout the 30-day period. 24 clients completed 354 homework sessions, with 266 including both art-making and discussion phases and 88 only art-making phases. They spent totally 151 hours---66.4 hours on art-making and 84.6 hours on discussion phases. While interacting with the homework agent, they produced 3,462 messages---772 during art-making and 2,690 during discussion phases. On average, each client used the system 14.75 times. Art-making phases averagely took about 11.3 minutes, and discussion phases around 19 minutes.

\autoref{fig:quan_results} (a) shows each client's usage distribution over the 30 days. 
\autoref{fig:quan_results} (b) depicts the distribution of all clients' usage over different hours of the day, showing that clients more often completed homework in the afternoon and evening (13:00–23:00). 
\autoref{fig:quan_results} (c) lists the usage frequency of different AI brush objects in the artworks: \textit{Human} (108), \textit{Cloud} (72), \textit{Ocean} (59), \textit{Grass} (52), \textit{Tree} (47), \textit{Lake} (47), \textit{Flower} (47), etc. The frequent use of the \textit{Human} brush may reflect the significance of human figure drawings, often used by therapists to understand clients' personalities, developmental stages, and emotional projections~\cite{malchiodi2007art}.


% Over the period of 30 days, the 24 clients completed 354 episodes of homework activity in total with 266 episodes including both the art-making and discussion phases; they spent in total 151 hours combining art-making (66.4 hours) and discussion phases (84.6 hours); and in total produced 3462 messages in conversations with the homework agent during both the art-making (772) and discussion phase (2690). Averagely, each client used the system 14.75 times; and each art-making phase costed about 11.3 minutes; and each discussion phase costed about 19 minutes.

% \autoref{fig:quan_results} (a) depicts the distribution of each client's usage over the 30 days. 
% \autoref{fig:quan_results} (c) illustrates the distribution of all clients' homework activity over different hours of the day; it has been shown that clients tend to more frequently complete homework episodes in the latter hafl of the day (13:00-23:00).
% \autoref{fig:quan_results} (b) listed the usage frequency of different AI brushes in the produced artworks whereas \textit{Human} (108), \textit{Cloud}~(72), \textit{Ocean}~(59), \textit{Grass}~(52), \textit{Tree}~(47), \textit{Lake}~(47), \textit{Flower}~(47). \textit{Human} was the most frequently used brush which might echo the importance of human figures which in with the importance simple human figure drawings can assist art therapists in understanding clients' personalities, developmental stages, and emotional projections~\cite{malchiodi2007art}.

% daily use of \name{} over a four-week period. The colored blocks represent the frequency with which clients completed their homework sessions with \name{} each day.
% Additionally, we found that clients typically completed their homework sessions in the afternoon and evening, as shown in \autoref{fig:quan_results
% }~(c).
% Finally, \autoref{fig:quan_results} (b) shows that the frequency of AI brushes used by clients to complete their homework. 
% We found that \textit{Human} brush~(108) is the most frequently used, followed by brushes for natural elements, such as \textit{Clouds}~(72), \textit{Oceans}~(59), \textit{Grass}~(52), and \textit{Trees}~(47).
%Previous study suggest that simple human figure drawings can assist art therapists in understanding clients' personalities, developmental stages, and emotional projections~\cite{malchiodi2007art}.
%你有所感所想的时候,就可以随时随地的去使用它(ziyoudao)
%:我一开始我觉得他有点像写日记,就像写那种画画日记一样,记录生活中的小事,比如说我去买饮料喝奶茶,然后我有那种感受,然后我就会画下来
%因为一般坐车的时候比较无聊嘛,然后思维也比较发散,就脑脑脑海里面会有很多天马行空的想法冒出来,所以说啊,就会在车上的话就会。嗯,一边听歌一边去进行这样一个创作
\subsection{Contextualizing Clients' Homework Usage in Various Natural Settings}
This section contextualizes how clients used \name{} to engage in therapy homework in their daily settings~(\autoref{fig:context}). 
Our clients used different digital devices to complete their therapeutic work, such as laptop~(C1-C7, C10, C14, C15, C17, C18, C20-C22, C24), desktop computer~(C12, C18, C19, C21) and tablet~(C2, C4, C7-C11, C13, C22, C23). 
Also, our clients completed their therapy homework in various locations. All clients have used \name{} in their homes. Other common locations included dormitories (C10, C11), libraries (C7, C9, C10), cafes (C22), high-speed trains (C22), internet cafes (C18), hotels (C18), offices (C12, C17, C22), and even hospitals (C3).

%As a result, \name{} allowed clients to use various devices to access art therapy homework in different natural environments, thereby reducing the barrier to engaging with the therapy.
%During her conversation with the AI agent, she explained that the two flowers symbolize different personalities: \qt{I drew plants for each of us. After completing the drawing, I noticed that the plants appeared different. I believe everyone has their own unique personality and perspective, so I didn't focus too much on these minor differences once the drawing was finished~(C8)}. 

The clients reported rich and vivid examples contextualizing how they used \name{} in daily settings and how their usage was shaped by the material and social surroundings, e.g., C8 used \textit{Flower} brush on her tablet to create artwork after an argument with her boyfriend (see \autoref{fig:qual_results} (a)). She told the conversation agent that the two flowers represented their personalities: \qt{I drew plants for each of us, noticing their differences. Believing everyone has a unique personality, I didn't dwell on the minor variations after finishing the drawing}.
Dream exploration is part of therapy homework, helping clients use dream materials to explore unconscious thoughts through creative expression~\cite{grotstein2009dreaming,freeman2002dreams}.
Nine clients used \name{} to explore and interpret their everyday dreams through artwork and conversation, e.g., C21 recreated a dream of becoming a teacher in a skyscraper office, expressing sadness to the AI about time passing since graduation.
These examples show that \name{} extends art therapy into clients' daily lives, helping them apply insights and changes from therapy and reflect on themselves through real-life experiences~\cite{kazantzis2007handbook}.
%Also, T1 assigned homework to explore a power around C3, who completed it during her trip: \qt{I observed a vast, soft, and constantly shifting cloud, and its immense power deeply touched me at that moment. I then used the AI system to explore the source of this power}.
 
Three clients explicitly noted that the system lowered the threshold for accessing art therapy homework by allowing anytime access and overcoming time and space limitations, e.g., \qt{last time, when I got angry at my child, I immediately reached for my tablet [to use \name{}], because it's the most convenient method for me~(C2)}. 
T4 added that: \qt{Art therapy homework typically requires clients to sit down with a desk. [With \name{}] clients can use a tablet for therapy homework on the way, reducing space and setting constraints}. 
Meanwhile, seven clients noted that \name{} also seemed to lower the threshold for art-making. 
%增加了C2的语录例子来表明不需要绘画技巧就可以创作
As noted by C2, \qt{I didn't need drawing skills. If I wanted to draw a streetlamp, I only needed to sketch its general shape, and I could describe the finer details using language}. 
C23 explained that it might help focus on the process of creation itself: \qt{I found this drawing method easy—just fill in the spaces, like a digital coloring book. It helped me focus more on the ideas}. 
T4 explained that \name{} might makes the process \qt{effective and fun} due to its low skill threshold, which could help clients \qt{focus more on the creative process without being stressed.}

\begin{figure*}[tb]
  \centering
  \includegraphics[width=\linewidth]{images/findings_2.png}
  \vspace{-7mm}
  \caption{Outcomes of the Clients' Art Therapy Homework Supported by \name{}. 
  }
  \Description{Figure 7 illustrates the Outcomes of the Clients’ Art Therapy Homework Supported by TherAIssist. Figure 7(a) shows two flowers of different colors, which, through AI-assisted dialogue, were revealed to symbolize the client and her boyfriend.Figure 7(b) presents an artwork featuring two figures, a carpet, and a floor. In conversation with the AI conversational agent, the client externalized her experience of arguing with her daughter through both verbal expression and art-making.Figure 7(c) depicts a cave in front of towering skyscrapers, with a little girl and a small patch of grass inside. Through interactions with the AI, it became clear that the client used verbal expression and art-making to externalize her feelings of loneliness and helplessness.Figure 7(d) contrasts two images, both using the sea to represent the client’s emotions. In the first image, a calm sea, gentle moonlight, and a lighthouse convey a sense of peace and warmth. In the second image, chaotic waves symbolize the client’s inner anger.Figure 7(e) shows a wolf in a forest. Through dialogue with the conversational agent, the client and the conversational agent explored the meaning of the wolf step by step. Figure 7(f) displays a family bedroom created using different brushes, including elements like a vase and a bed. During the conversation with the AI, the client expressed surprise at the inclusion of a bouquet in the vase, an unexpected detail that delighted him.}
  \label{fig:qual_results}
\end{figure*}
 
%引入在不同context家庭作业对来访者的重要性的引用
% 家庭作业在不同情境下使用的好处:治疗作业可以帮助患者不断检验自己的观点,并反思日常生活中的行为。
%需要解释。。通俗 aboved examples


% 评论:tianyin: 我会觉得每一次我的参加者他们画的东西就是他拿画笔涂的东西真的是非常简单,反而最后有的时候,其实我们是在靠 natural language本身的语言模型在描述他想要的东西,所以我想暂时来说可能画面的那种就是image to image的 representation只是在一个位置跟它比如说大小颜色层次,但其实你要表达他的话,他们还是要去到语言层次
% ziyoudao: 因为绘画其实就是可能你去在表达的时候因为它还会有变化。它不是对我来说,他可能不是光停留在那一个画面上。可能我更想传达的是那种我在其中或者是那种情景给我的感受。所以。对他绘画我觉得他是作为一种呈现的图示,然后可能我不局限于图示。我并不是说他要跟我想的那个东西是完全呈现或者是怎么样的,是他会能够去传达想表达的东西。
%tang:他有的时候比较适合用用绘画来表达,有的时候他其实会更适合用文字来表达。我觉得有的有些点可能你要描述一些画面,可能比你画出来也会容易很多。因为我觉得对,因为我觉得你要是画一些东西的话,一些很抽象的东西其实不一定能画出来...然后你说的那些比较适合于通过文字来创作的这种,对于我来讲可能是一些未来,跟他人相关的一些点,我可能比较倾向于用文字描述来表达。我觉得对于适合画的其实是更加 focus on yourself的这种类型的,你是可以比较好的画出来,包括过去的东西,但如果是跟他人的连接,去揣测未来的情况,或者他人对自己的看法的时候,你说实话我想象不出来,我想象不出来,我就可能只能描述来说我挺难画出来的。对,就是因为它会用到很多的这种隐喻或者是比喻,对吧?对这种隐喻和比喻跟我没有特别多的联系的东西,我是很难画出来的。但是反倒是你要是聊,如果只是用语言的话,现在写诗什么也挺好的,你可能说着说着就能描述就是描述性的文字一个特定的跟电影一样的一种场景,这些文字的东西大家脑海里也会有一个画面。就是什么情况下你会选择用这种会话的方式去去完成?
% 0727devon: 来访者想表达看完巴黎奥运会的开幕式之后,所产生的观后感,但是画面表达不了,使用AI助手表达感受到传统文化与后现代文化的对立与冲击,以及通过使用后现代价值观来解构演绎传统历史文化的时候,那种割裂感和不适感。
%AI与人协同创作:绘画的意料之外的元素促进了来访者的解释
    % 图像启发来访者的深层次的解释 例子:在画的时候,我画的是一朵小云,想表达的是小云想要下雨,但是它希望去下雨的地方希望在合适的海域下雨,结果它找不到那个完美地方,憋得很难受。我根据这个内容让AI去生成画面,但是画面最后生成的是一团很大的乌云。它可能也是启发我,算是在我的作品上做了一些改变。就是它那个画面看起来就像一个要下雨的云被积攒了很久,然后变成了一整片的那种乌云的状态(edric)
    %  AI创作的创造性可以启发来访者,丰富来访者想表达的故事,使用户重新思考和扩展自己的表达方式. 例子:我会觉得AI生成的图过后,AI图里会有一些跟我想象中可能不一样的话,不一样的呈现方式,然后我觉得AI创作是有创造性的,然后我在最后去AI助手描述这张图的时候,我也会加上我看到AI想让我表达出的想法,我觉得这也这个是有意思的点。Ai创作它会有它自己的一部分的创造性,然后你又认可它的创造性,然后你就会把它的AI创造出来部分的东西然后加入到自己的想法中来表达出来,然后它会丰富我对这个画面的感觉或者丰富我可能想表达的这整一个的故事(gong)0723-创作的图像:来访者描述了一幅灰色发光的十字架球体,坚硬的石头,钢铁盾牌的画作,通过AI生成的图片来访者从画面中得到不一样的感受:“自己对于创作的初衷,会随着生成图有所改变,你会倾向去解读生成出来的图片,然后去更新自己原先的构想。”
 
%AI与人协同创作:来访者通过对话方式与AI一起探索人的绘画的意象(探索)
    %  例子:0729-qilin-来访者qilin创作了一个狼的画面的意象,在对话的过程当中进一步跟AI chatbot进行问答的方式去探索绘画意象可能的含义。
    %  例子:07-11yuni 创作了五颜六色的花环,向阳的向日葵,绚烂的菊花的画面,来访者根据这些花在对话过程中进一步跟AI chatbot进行问答的方式探索花对于来访者意义是什么:“我在第二次使用它的时候,我好像画了一些花朵,然后表示我的心情,或者是表示我参加一个活动的心情的时候,然后我问他就是花朵的含义以及甚至于他好像在鼓励我生活中有没有像花朵一样那样绽放的时刻的时候,好像让我有更多的思考。然后我就回答说我好像我并不想像花朵一样绽放,就是它让我找到了真正的自己的部分...我好像明白了,其实没有那么追求像每一天都像花朵一样绽放,就是我反而会觉得有过那样的时刻就很美好的部分,然后我好像就明白了自己内心真实的想法,我也觉得他好像在引导我去了解自己的部分。”
    %  0717创作了一幅云朵覆盖全景,左下角雄伟的山峰,山峰上的茅草屋和飞机穿行的画面,在与AI chatbot的交流中探索到自己希望与大自然在一起,不想社交的感受:“画前不是这样的感觉,画面是边画边在脑海里呈现出来的,画画前就是想根据云朵进行创作;但是画完我感觉我能找到画这一幅画的理由,主要自己现在主要的心里状态。” 。“因为如果他不问我的话,我可能就会停留在那个画面给我的感觉。如果他问我的话,我可能会在这个是让画面的基础上,然后再想一些,再深入地去挖掘一下,再想一想可能会有更加深入的答案出来(gong)”
    % 评论(语言与图像共创,挖掘来访者潜意识的):我大多数创作的时候,我也不知道我想要做出来一个什么样的东西,有一种那种无意识的感觉。我看到什么我就选择了它,然后我把它画了下去,对我来说真正比较有价值的其实是第二个阶段就是纯AI的互动过程...我会发现我自己用文字和语言的方式已经蛮大程度上能够去表达我自己的情绪感受以及想法(xie).
    % (自由岛)“我觉得开始画画的时候是不确定的,然后越画可能会探索自己想要的东西”
% 语言帮助外化画面之外经历与感受(补充)
    % yuni创作了一座桥隔离了即将分离的人的画面,通过AI助手的引导主动分享来访者去面对她与熟悉的人需要分开这个事实,她描述她自己没有办法去接受跟熟悉的人分开。“旁边蓝色的部分是一片湖水,代表一些悲伤的情绪,就是他之前在线下咨询室里面流过的眼泪。因为做咨询都是这样的过程,会其实做咨询是一个很痛苦的过程,因为要疗伤,然后要面对那些痛苦。就是在画完那幅画之后,然后她和AI交流完之后她意识到了原来不是她被动地去离开那个咨询师,而是他现在可能可能可以鼓起勇气去主动的离开心理咨询师了。(mansi)”
    %  sammy-07-12创作的一幅夜晚中下着雨,一位小女孩撑着一把伞的场景,表达了自己孤独与迷茫的感受。在与AI助手的引导下,她表达了自己孤独与迷茫的感受背后关于家庭导致事业停滞的原因。
    % leduo,0715-来访者描述白色的云朵和云海覆盖的山峰,飞翔的小鸟的场景,根据自己的经历表达自己想去爬华山但是又恐高的复杂的矛盾心理。
    % xie:做从一个物象上的描述,慢慢的到了一个对话的阶段,就是AI会开始让你去和你的那些物象对话,我们讲叙事治疗的话,它其实就是一个外化的阶段,把你自己的东西一个一个从你身上拨出来,然后放在你面前,然后跟你对话,这是叙事治疗。
    % sammy:我用这个平台,我就发现那一次我处理的会快很多,然后有一个助手跟我对话的话,首先创作这个话是让我能够发现自己内心想要解决什么问题,情绪可能就是一个这幅画的影子。我们进行对话,就慢慢的捋清楚自己负面情绪背后的需求是什么, 然后更快的找到恢复正常情绪的好的一个心情的状态
    %ziyoudao: 他能让我就是比如说看到这些意象,在选那些意象的时候,我可能会突然非常有灵魂,比如说我想到那个纠结的问题的时候,我可能突然脑海中就会想到两条分叉的路。那但是我在没有用这个东西之前,我就是没有去想到那种具象的表达,但是我在看到那些词语调的时候,我突然看到了路,然后我就一下会感觉这种分叉的路就特别能传达出我当时的那种纠结的感觉,而AI助手可以引导你把这种感觉表达出来.
% 图像作为一种无意识的象征
    % 最近画的这幅就是花盆,然后喷泉,然后这些水就滋养这些花,然后这样子,然后就好像花在喷泉周围尽情的喝水这样子,然后植物们自由舒展,然后在阳光下溅出水花,就是我最近画这幅是我觉得我画过里面稍微就是比较梦幻的一幅吧,然后今天聊的是,他觉得因为我之前都是那种要不就黑夜,雨就是这个喷泉也是有水。但是这次就是在阳光下就是很敞亮的一个场景,然后去画了不同的花,然后是也是今天我觉得疗愈师也特别好,就是他也说他发现我画的几种植物是春夏秋冬,各种有梅花,有荷花,有水仙,但是其实我一开始是不知道我在画就是四季的。就是我没有意识,我画了四季的画四季的植物,但是他跟我讲了之后,我才意识到,原来就有点像那种荣格,他提到的就是那种无意识是一切意识的起点(qiukui)
    % 评论:比如我创作了关于家庭主题的创作,如果我确实有心理问题,AI是通过温和的方式,比如让你画画,不会直接逼问你是否家庭不幸福或生活中没有人好好对待你(fei)
    % (AI创作可以将情绪通过艺术方式象征化,从而更好地帮助来访者释放情绪):你可以填充你的不好的情绪,你可以填充你的想象理想或者画任何东西,它会把你的一些这种东西给具象化,然后你在这个过程其实你会把这些东西给释放出来,然后最后的效果然后也还不错(superman)
    % (AI创作可以将来访者的心理状态象征化) 我就感觉到了这个APP它很好的一个点,就这个艺术疗愈,就是它其实在一定程度上是把我内心感受到的这种抑压或者恐惧也好,是把它具象化的。那如果你把它画完之后,它就很像你可以给不同的情绪命名。然后你就把它想象成一个球体,那你就可以轻轻把它抛掉,就好像你能够没有像之前一样那么费力的去调整,就是很抽象的东西。你现在具象之后,你就可以很快的走出这种情绪阶段.(qiukui)
    %qiukui:其实画画,你可以很具象的去了解到。你这段时间的一个心理状态和思想动态这样子因为你通过画是可以很直观的呈现出来的,我觉得这一点就比较好

%==========理论部分=========
%% verbalization and art-making
    % (art-making: Discovering the Dominant Story ). Art practices makes the client tell his/her story easier in situations where clients have difficulty in expressing the story verbally(Carlson,1997)
    % (语言可以促进构建新的意义的)An integral part of narrative therapy is to help families recapture lived experiences that were blinded by the dominant story (White & Epston, 1990). As alternative stories are brought forth, families are able to create new meaning in their lives (Carlson,1997)
    % (图像是发现问题,而语言表达则是具体的外化)Picturing the problem as an art practice or creating collages out of pictures makes the problem apparent for the individual through art. The concretion process also enables the clients to separate themselves from the problem.(temur,2021)
    %  (语言可以促进构建新的意义的,mean-making)With the effect of narrative therapy, the clients add new meanings to their stories or increase their motivation to improve a life event in scale that they wish in the future(temur,2021)
\subsection{How Conversation and Art-making were Combined in Therapy Homework}
This section exemplifies how the clients---while interacting with \name{}---engaged in both verbal expression and art-making that granted therapeutic meanings.
% Verbalization can help clients externalize their experiences and feelings behind the artwork
% C9(yuni)创作了一座桥隔离了即将分离的人的画面,通过AI助手的引导来表达来访者去面对她与熟悉的人需要分开这个事实:“旁边蓝色的部分是一片湖水,代表一些悲伤的情绪...在画完那幅画之后,她和AI交流完之后她意识到了原来不是她被动地去离开那个熟悉的人,而是她现在可能可能可以鼓起勇气去主动的离开那个人了”。
% sammy创作的一幅夜晚中下着雨,一位小女孩撑着一把伞的场景,表达了自己孤独与迷茫的感受。在与AI助手的引导下,她表达了自己孤独与迷茫的感受背后关于家庭导致事业停滞的原因。
% xie解释说“”从一个物象上的描述,慢慢的到了一个对话的阶段,就是AI会开始让你去和你的那些物象对话...AI把你自己的东西一个一个从你身上拨出来“。
%========11.30=============
%6.3.1 不变,分成三段,变成斜体
%语言可以增强作品背后的感受表达
%语言可以增强作品背后的故事经历
%语言可以在相同意象下产生不同的表达



\subsubsection{\textbf{Clients' verbalization of feelings and experiences during art-making}}
%Verbalization helping clients externalize their experiences and feelings behind the artwork
In the ``Art-making Phase'', we found that the participants tended to verbalized their experiences and emotions related to their artworks in two aspects:

\textbf{Clients verbalized emotional feelings during art-making.} 
In the ``Art-making Phase'', we found that the clients verbalized emotional expressions during their art-making process through the conversational agent.
In C2's therapy homework, she created an artwork featuring a \textit{Cave}, a \textit{Girl}, a \textit{City}, and a blade of \textit{Grass}, verbalizing the helpless and sorrowful girl in the dark cave, overshadowed by the dense and cold skyscrapers of the city through the conversational agent~(see \autoref{fig:qual_results} (c)).
T3 remarked that: \qt{her artwork and verbal descriptions might express a deep sense of loneliness, along with a yearning for prosperity intertwined with feelings of confusion}.
Further, clients verbalized their emotional expressions integrated into the non-verbal expressions behind the brush objects.
For instance, both C22 and C18 utilized the same brush object, Sea, to depict different oceanic forms in their artwork (see \autoref{fig:qual_results} (d)). C22 portrayed a calm sea surface to evoke a sense of leisure and tranquility, whereas C18 expressed suppressed anger through the depiction of turbulent waves: \qt{the ocean object feels quite oppressive. Coupled with her verbal description, it suggests that she might be experiencing significant pressure and underlying anger~(T1)}. 


\textbf{Clients articulated narratives and personal experiences about their artwork.}
%促进创作背后的个人经历的叙事personal narratives and experiences articulation
In the ``Art-making Phase'', we found that our clients verbalized the articulation of personal narratives and experiences behind an artwork.
For instance, C2 utilized various brush objects, including \textit{People}, \textit{Floor}, \textit{Carpet} and \textit{Shoes}, to create an artwork.
She sought to externalize her artwork, viewing her experience of an angry mother stands while her daughter sits on the carpet and has a heated argument with her~(see \autoref{fig:qual_results}~(b)).
T3 interpreted that: \qt{there was a noticeable sense of inner conflict in her experiences with raising her children}.
Based on relevant theories~\cite{carlson1997using,harpaz2014narrative}, this collaboration of AI-infused art-making and conversational agents can identify the ``signs'' embedded in the artwork through verbalization, while also guiding and complementing the narrative behind them.

\subsubsection{\textbf{Clients' meaning-making with the homework conversation agent after art-making}}
In the ``Discussion Phase'', we found that combining human-AI co-creative artwork and the conversational agent can facilitate mean-making, in twofold meanings:

\textbf{Clients' self-exploration behind the artwork within the multi-round human-AI dialogue.} 
We found that the multi-round dialogue in ``Discussion Phase'' encouraged clients' open-ended self-exploration and helped them discover new meanings within their co-creative artworks. 
For example, C6 utilized the \textit{Tree}, \textit{Grass}, and \textit{Animal} brush objects to created a vibrant, towering green forest inhabited by an adult blue wolf, while the conversational agent facilitated exploration of the qualities admired in the solitary wolf through multi-round structured questioning~(see \autoref{fig:qual_results} (e)).
Further, C9 employed the \textit{Flower} to create a flower bush: \qt{...I drew flowers to express my feelings about an activity. When I asked AI about their meaning, it prompted me to reflect on times I had 'blossomed.' At first, I regretted not having many such moments, but after several rounds of conversation, I realized they are beautiful and valuable, even if rare, and gained clarity about my true emotions}.
T5 explained that the co-creative meaning-making can \qt{assist clients in transforming their subconscious creations into narratives with personal meanings}.
Compared to the monologic forms of traditional therapy homework, multi-round dialogue avoids a single, closed perspective, allowing diverse perspectives, encouraging clients to engage in meaning-making and promote their understanding of the images~\cite{bacstemur2021integration,pare2004willow}.

\textbf{Clients' interpretation triggered by the unexpected details from human-AI co-created artwork.} 
Ten clients noted that unexpected details in AI-generated images sparked creative interpretations and enriched their stories verbally. 
For example, C23 created a simple room with a luxurious bed and vase, but the AI added a flower bouquet, prompting C23's comment during the Discussion Phase: \qt{I was pleasantly surprised, as I had only wanted a vase, but with the flowers added by the AI, it looks beautiful, and I can already imagine the room filled with their fragrance}~(see \autoref{fig:qual_results}~(f)).
%加了一个例子
Also, C5 used the \textit{Clock} and \textit{People} to conveys the guilt and sorrow of losing loved ones and the ache of their absence as time flows: \qt{I initially planned to create a space with clocks to express guilt about time, but the AI generated flower baskets instead...Perhaps these flower baskets symbolize the outcomes brought by time}.
T4 commented that the unexpected details might help the clients broaden their horizons: \qt{The generative AI feels more like a `familiar bystander', offering its own perspective. The unexpected details might can spark new ideas or provide a fresh lens, which may inspire clients to find a direction to move forward}.


%unexpected elements in human-AI co-creative art-making can inspire clients to explore creative interpretations and enrich the stories they want to express through verbal expression. 
%(12.9)增加了理论
%\textcolor{blue}{Compared to the monologic forms of therapy homework, the conversational agent avoids a single, closed perspective, allowing diverse perspectives, encouraging clients to engage in meaning-making~\cite{bacstemur2021integration,pare2004willow}.
%Furthermore, the uncertainty might offer an opportunity to reconsider and transform self-perception, prompting clients to change their understanding of the images and, in doing so, transform negative emotions into positive meaning-making~\cite{rockwell2013art,brockway2019art}.}


%As C10 shared, \qt{I feel that the AI-generated images sometimes present unexpected elements. When I described the artwork to the conversational agent...if I accepted AI-generated artwork, I would add the part created by AI to your own ideas to express it. It would enrich my feeling about this artwork or enrich the whole story I may want to express}. 
%T3 explained that co-creative meaning-making helps clients broaden their horizons: \qt{In my group art therapy sessions, I encouraged other clients to add their creative touches to the client's artwork, offering new perspectives and reflections}.
%T4 add that: \qt{AI feels more like a `familiar bystander', offering its own perspective, which may inspire clients to find a direction to move forward.}}
% 我找到了这个理论
%因此,艺术治疗中的创造行为创造了空间来练习灌输、发现和创造意义的技能。当客户对自己这样做的能力越来越满意时,他们就能够将新发现的整体视角应用于生活的其他方面。



\subsubsection{\textbf{Risks and Concerns}}
Although combining conversation and art-making in therapy homework offers many benefits, there are a few risks and concerns in three key areas:

\textbf{AI-infused art-making lacks multisensory experiences.} 
C22 and C23 noted that their AI-infused art-making process lacked the multi-sensory experience often associated with traditional materials.
As put by C23, \qt{The sound of traditional pens on paper is irreplaceable by generative AI. My paper artwork has depth—controlled pen pressure and watercolor concentration create rich textures, unlike AI’s one-dimensional results}. 
C22 also mentioned that: \qt{I can even smell traditional art-making materials, and I truly enjoy listening to the sounds they make on paper}.
Thus, T2 explained that the limitation of generative AI: \qt{The most remarkable aspect of generative AI as a material is symbolic, not sensory or physically tangible}. 
Considering the sensory aspect of art-making, it may promote clients' self-soothing and the expression of their inner sensations~\cite{lusebrink1990imagery}.

\textbf{Uncertainty in AI-infused art-making.} 
C9, C10, C11, C15, C20 and C23 mentioned that generative AI sometimes doesn't fully generate the images that the client expect. 
For instance, C20 shared feeling disappointed when the AI sometimes didn't generate what he wanted: \qt{I want to create a scene where clouds are raining on a perfect sea, but the AI cannot generate that perfect sea}. 
Also, C15 noted that the generative AI sometimes doesn't fully generate highly abstract images: \qt{
The morning after the Olympic opening ceremony, I tried creating an abstract LGBTQ+ multicultural artwork with postmodern symbols and a central white light beam using \name{}. When the AI couldn't generate it, I described the concepts instead verbally}.
T2 expressed concerns that when the images didn't meet expectations, it might lead to feelings of disappointment: \qt{When using paper and pencils, clients rarely blamed the materials for imperfect drawings...but, if the AI doesn't perform as expected, frustration is often directed at it, especially when unexpected symbols or triggers appear, heightening disappointment}.

\textbf{The LLM agent cannot capture the fine details of artworks.} 
We found that C5 and C21 mentioned that the conversational agent sometimes had difficulty fully understanding all the details of the images, e.g., \qt{the AI may miss details in my artworks, like whether the teacher in my office is standing or sitting, making it hard to ask specific questions, which can leave me feeling disappointed~(C21)}. 
T2 and T3 noted that the conversational agent may lack emotional resonance and intuition, potentially missing important details in a client's artwork and causing disappointment: \qt{C1 drew a large stone in her artwork, which seemed significant to me. I would ask her what the stone represents, but the conversational agent may not capture such details~(T3)}. 
But, T4 also expressed concern that: \qt{if it can understand the details and give some interpretations, it may mislead the clients' emotions and self-cognition}.




%According to the related theories, the co-creative unexpected elements can be discovered and structured new meanings through verbal expression, which enables clients to apply their newfound holistic perspective to other areas of their lives~\cite{rockwell2013art,bacstemur2021integration}.

% 6.3.3 risks and concerns
%ai infused drawing无法还原物理材料的感官感受
%AI产出的图片在一定程度上具有不可控性和不确定性
%由于ai对话缺乏灵活性和社交直觉,无法捕捉来访者细微情感导致不知道什么时候追问
%ai无法理解画面中的一些细节
%ai在来访者创作过程中不适当的时机提问


\section{Findings-RQ2: How \name{} Mediated Therapist-Client Collaboration surrounding art therapy homework}
This section outlines \name{}'s role in facilitating asynchronous therapist-client collaboration for art therapy homework, covering therapist customization of homework agents and the tracking and use of clients' homework history.

\subsection{How therapists Customized Homework Agents to Support Client Homework Asynchronously}
In this section, we concretely examine how therapists tailored the homework agents to support clients' therapy homework. 

\subsubsection{\textbf{Therapists used customization to provide structured guidance to clients}}
Our therapists tended to provide structured guidance through infusing their practical experience and professional beliefs into the homework agents in two aspects:
%Customization enabling structured guidance
%==========
% 治疗师将自己的专业信念定制在对话代理原则和示例问题中
%所有艺术治疗师都会根据其所属的理论流派和实践风格,设计并安排不同类型的家庭作业。


\textbf{Therapists provided structured guidance through customized dialogues.} 
First, T1, T4, and T5 placed the principle of naming the artwork at the end of all dialogue principles based on her own work experience,e.g., \qt{if clients wait until they finished discussing artwork before naming it, it can help them understand the artwork more deeply~(T1)}. 
Furthermore, T2, T4, and T5 revised the principles and example questions for the conversational agent. 
For example, T2 rephrased all of the example questions and added extra example questions from the dialogue principles based on her professional beliefs~(e.g., adding \qt{If this artwork could communicate with you, what part of it would it say to you}). 
C23 mentioned that it can encourage him to reflect more deeply: \qt{T2 hoped I could have an external dialogue with elements of my creation. Today, I drew a bed, and the conversational agent asked what the bed would say to me, allowing me to explore its deeper meaning}.
%添加了一个T3添加对话原则的例子
Also, T3 added a dialogue principle at the beginning: \qt{determine whether the artwork aligns with your expectations}. T3 also explained her reasons: \qt{I would like to understand to what extent the artwork meets their expectations, and whether different scores reflecting varying expectations lead to different thoughts}.
Therefore, customizing structured guidance could help guide the client in providing the information the therapist was seeking: \qt{Because I asked questions about aspects, clients may be more inclined to complete the homework about the aspects~(T4)}.
Also, customizing structured guidance through the conversational agent might promote a more flexible and open perspective on facilitating exploration: \qt{Previously, I gave clients fixed, printed questions as reflective prompts for the homework, but the customized conversational agent offers a more flexible, chat-like experience~(T4)}.

\textbf{Therapists provided structured guidance through personalized homework tasks.} 
All therapists tended to provide structured guidance through tailoring different types of homework tasks based on their theoretical orientation and practice style they followed. 
%修改了claim和添加T5的语录
T3 and T5 incorporated positive psychology techniques to provide different types of homework tasks based on the various stages of therapy, e.g., \qt{in the first week, I suggested that C6 used \name{} to draw her inner emotional `monsters' as a way to express his feelings. In the second week, I encouraged him to draw his positive resources using \name{}~(T5)}. 
%添加了claim和添加T2的语录
Further, T1, T2, and T4 would like to integrate meditation techniques into the art therapy homework assignments. 
For instance, T2 assigned C23 a structured homework task: \qt{close your eyes, focus on your breathing, meditate, and imagine your relationship with your boyfriend. What visual image would you use to express it?}, T2 explained, \qt{at first, I guided them to experience mediation. As they drew, it helped clarify and deepen their understanding of their feelings}.
The structured guidance through tailoring homework tasks might assist clients in clarifying specific task goals and reinforcing insights from therapy sessions~\cite{santisteban2003efficacy, tompkins2004using}.


%\textcolor{blue}{According to relevant theory~\cite{kazantzis2007handbook}, therapists' theoretical orientation (such as Cognitive Behavioral Therapy or Emotion-Focused Experiential Therapy) or practical experience often influences how they design and implement homework tasks, which can help them gather information in a more directive way while maintaining the continuity and consistency of therapy.}


%For example, T1 placed the principle of naming the artwork at the end of all dialogue principles based on her own work experience: \qt{If clients wait until they finished discussing artwork before naming it, it can help them understand the artwork more deeply}. 

%For therapy homework topics, T5 tailored different types of homework tasks for each stage of the therapeutic process using \name{}: \qt{}.
%理论:these studies do provide support (albeit indirect) for the importance of well-planned, structured homework tasks as a vehicle for implementing and consolidating behavioral change in the family.
%According to relevant theories~\cite{santisteban2003efficacy,kazantzis2007handbook}, structured guidance serves as an effective means for implementing and reinforcing behavioral change.

 %定制agents提供线上一对一session的延续
        %AI支持的定制化家庭作业提供了线上一对一艺术治疗session的延续。
        %例如,Mansi提到这种家庭作业可以定制化家庭作业可以巩固和深化来访者在线咨询的效果:“我上周在现场让qilin画的是给自己的内在小孩画了一个家,回家了之后,我就在那个系统里面给他布置了一个作业,让他去进一步去画一下他给他的内在小孩安的家是什么样子,让他进一步去装饰和布置一下,再多增添一些功能,让他的内在小孩可以更有安全感,或者说更开心,过得更舒服一些”。
        % 这种延续可以促进对上次一对一session的回忆:“我每次使用的时候,其实我都会回忆起我跟治疗师上一次的对话,所以其实是有帮助的(qilin)”。
        %其次,Xingyi demostrated that 家庭作业可以为下次线上一对一session见面提供更多时间去思考和准备:“我让Edric去想到他成功的那种方面,就是高光时刻还有成功的体验,其实说不是都只是在错误中学习...所以我们在第二次在后面的会面的时候,就可以有更多时间去讨论它里面带出来的一些话题”。
        % AI支持的定制化家庭作业促进了两次线上一对一session的连接:“在线上一对一画画的过程跟感觉就像种了一个种子一样。在家里面自己画画到下一次之间的过程,就像这个种子它会发芽会生长一样”。

%\textcolor{red}{\subsubsection{\textbf{Therapists customized homework to facilitate continuation across therapy sessions}}
%Our therapists customized therapy homework through homework agents may ensure a seamless continuation of art therapy in-sessions, as surfaced by two aspects:}

%\textcolor{red}{\textbf{Therapist customized homework to extend previous therapy sessions.} 
%T1, T3, T4 and T5 tended to tailor the clients' homework to supplement and extend the previous in-sessions. For example, T5 highlighted that she tended to tailor the homework to reinforce the therapeutic effects of clients' in-sessions via \name{}: \qt{Last week, I encouraged C6 to create a topic about a home for her inner child during the in-session. After C6 went home, I assigned C6 a task in \name{}, suggesting C6 to further decorate and arrange the house, so that the inner child could feel more comfortable and at ease}.The homework might serve as positive reinforcement for more adaptive behaviors in natural environment: \qt{In my recent session, my therapist asked me to draw how I felt when I was angry at home. I depicted a spark that seemed ready to ignite at any moment. When I was at home, in that same situation, would anything inside me change as I recreate this topic?~(C3)}.  Meanwhile, it might also encourage more conscious reflection and provide new insights: \qt{During the last one-on-one session, my therapist asked me to draw my ideal life. At first, I couldn't fully visualize it...but after the session, I recreated the drawing a second and third time...I was able to reflect more consciously and express my thoughts visually, which sparked further inspiration~(C19)}.T2 also remarked that tailoring therapy homework helps reinforce and enhance outcomes across sessions: \qt{During online sessions, the creative process feels like planting a seed. Between sessions, that seed seems to sprout and grow~(T2)}}


%\textcolor{red}{\textbf{Therapists customized homework to prepare for upcoming therapy sessions.}
%T1, T2, and T5 demonstrated that they customized the homework tasks which provided more time to prepare for the next in-sessions. As noted by T1, \qt{I recommended C20 to complete his homework on highlight moments and success experiences...so that we had more time to discuss those topics in our next session}.T5 believed this preparation for the next in-sessions can open up many topics for the next online one-on-one session: \qt{You can give them more time to think and prepare, which guide them in specific directions for further reflection. Since they've already completed the tasks beforehand, the next session allows us more time to discuss the topics it brings up}.Customizing therapy homework might effectively bridge therapy sessions, enhancing connection and continuity throughout the entire process and increasing the likelihood of therapeutic success~\cite{freeman2007use, hoshino2011narrative}. }


    %定制agents提供情感支持以及构建来访者与治疗师关系
        %治疗师提到对话agent的定制化在理论的基础上增加personal touch,从而促进来访者与治疗师的信任。
        %例如,Ziyan对于所有的对话原则的示例问题进行了重新叙述:“语言更加像人一点,也许希望A这种陪伴可能更温柔一点”。
        %其次,治疗师给予的用户寄语帮助构建治疗师与来访者情感支持。
        %例如,Jia给sammy的寄语种让来访者关注负面情绪中自己的需求,明确自己的优势,了解自身的内在感受,从而治疗师希望传达支持与鼓励。
        %这些寄语也会促使用户更多的自我披露:“我会觉得很温暖,就是当我看那寄语的时候,就会觉得我是被关心着的...我好像会更加的敞开心扉,会更加的去大胆的画(tang)”。
\subsubsection{\textbf{Therapists used customization to retain personal touch and offer emotional support}}
We found that therapists customized homework agents to provide emotional support and encouragement for clients, in two aspects:

\textbf{Therapists customized homework conversation agents to enhance personal touch and foster trust.}
T2 and T4 noted that they modified the phrasing of example questions or conversational principles of the conversational agent to enhance the personal touch and foster trust between the client and therapist. 
For example, T2 adjusted the example question to make it more conversational and human-like, following the principle of guiding the user to describe their artwork (e.g., \qt{Would you like to share your thoughts? If not, I can quietly sit with you. But if you'd like to talk, I'm here and ready to listen}): \qt{I hope the AI's companionship might be conveyed in a more gentle manner}.
%增加了一个T4的例子
Also, T4 revised the conversational principle for asking clients about their feelings toward the artwork (e.g., \qt{If the user is willing, gently explore the reasons behind their feelings. They can choose not to respond if they're not ready}): \qt{AI conversational principles, based on theory, can incorporate a personal touch, helping to build trust and rapport with the client}.


\textbf{Therapists customized opening messages to deliver personalized emotional support and encouragement.} 
T2, T3, T4 and T5 demonstrated that they tended to infuse their emotional support and encouragement into therapy homework through delivering their opening messages in \name{}.
For example, T2 left an opening message for C23 (e.g., \qt{Find a space, infuse it with vitality, and let it continue to thrive}): \qt{The message symbolizes that the connection between the therapist and client persists beyond the therapy sessions}. 
Further, T3 left opening messages~(e.g., \qt{Focus on your needs in negative emotions, recognize your strengths, and gain insights into your inner feelings}) on \name{} based on discussions with C2: \qt{I hope to give them some emotional support or encouragement when they are doing the homework, so I tried to write this based on the process I communicate with them}.
Our therapists incorporated their personal touch to the therapy homework through our agents, which can promote more self-disclosure among clients during doing therapy homework: \qt{When I read the message, I feel a deep sense of warmth and care...It seems to encourage me to open my heart to create~(C9)}.
Tailoring homework by incorporating expressions of empathy and positive regard (such as encouragement and affirmation) might significantly contribute to the establishment and maintenance of a strong therapeutic relationship~\cite{cronin2015integrating}.


\subsubsection{\textbf{Risks and concerns}}
While customizing homework agents offers several benefits, there are a few risks and concerns in two main areas:

\textbf{Limited options for current customization.}
First, T1 and T5 noted that the lack of diverse AI conversational agent voices may cause discomfort among clients: \qt{Having a couple of voice choices could make the interaction more personal. Some clients may prefer talking to an agent of a certain gender, which could facilitate approachability and connection~(T1)}. 
Further, T3 and T4 mentioned that AI-infused art-making is primarily limited to drawing, highlighting the need to incorporate and customize other forms such as collaging, photography, and sculpture. 
T4 mentioned that diverse forms of art-making have varying effects on different clients: \qt{In my practice, we sometimes used photography, popular among seniors. Limited art forms may restrict their self-expression across various media}.
%现在art-making形式仅仅局限于drawing,还需要定制其他形式比如collaging, photographying and scupturing 

\textbf{LLM agents' rigidity and lack of social intuition in conversation} 
T1 and T2 highlighted that tailored conversational agents can occasionally result in rigid questioning and insufficient social intuition, e.g., C21 noted that: \qt{Although I shared many feelings and experiences, the responses often felt formulaic, lacking emotional depth, and were driven by rigid questions that kept asking about the reasons behind everything.}
T1 worries that: \qt{Such structured questioning might make clients feel constrained, unable to freely express their thoughts or emotions, which might affect therapeutic outcomes}.
Further, T3 commented: \qt{the conversational agent sometimes missed cues to stop asking questions when clients were reluctant. How can we signal AI to cease inquiries? Unlike therapists, who can read facial expressions and tone to decide when to halt, the conversational agent might lack this non-verbal intuition}.


%===
% 7.1.4 risks and concerns:
%1. 尽管定制了家庭作业,但是没有办法把控质量
%2. AI没有办法按照治疗师的社交直觉和随机应变的(ai缺少灵活性,社交直觉,对于来访者情感细节的微妙把握)
%3. ai声音的定制化


% 来访者创作的家庭作业作为Therapeutic Log支持治疗师

        
%比喻:比方说一个人站在你面前对不对?他可能就照了一个 X光,然后一眼就能看到他骨架的东西,然后从骨架可以再看到这个人各种各样的细节
%我认为它是未开发的宝藏,需要再去挖掘
%比较有结构的看到整个的过程
%我可以关注到她当下如何处理情绪的
%这些家庭作业是我下一次session的基础
%频繁出现的意象它可能是有象征的
\subsection{How AI Compiled Homework History Was Used by Therapists as Practical Resources}
Our design feature, \textbf{DF3}, aims to track the clients' homework history through providing the artworks, the art-making process, the dialogue history data, and the AI summary.
In this section, we describe how therapists tracked and utilized AI-complied therapy homework history~(\textbf{RQ2}). 
%We summarized three themes: (1) Deepening understanding of the clients' past experiences and characteristics; (2) Identifying meaningful conversation triggers for one-on-one sessions; (3) Transforming homework outcomes into empowering resources for clients.

    %来访者与AI互动可以了解来访者家庭作业的创作,特点和过往的经历与感受。
        % 首先,AI支持家庭作业可以帮助治疗师理解整个创作过程。
        % 例如,Ziyan提到AI agent总结的出来访者在创作中反复出现的绘画意象:“比如如果总会提示你最近几次来访者都画到了丝绒,我会留意到了这种意象”。
        % 治疗师解释这种意象具有重要的作用,是理解和探索无意识的重要工具:“结合荣格的理论,在荣格的集体无意识中,夏天象征了什么?(Ziyan)”.
        % 其次,来访者与AI的创作的图像可以支持治疗师了解来访者特点。
        %例如,治疗师通过查看来访者创作了一幅在花瓶里装了一只长的牛角的马的创作,猜测来访者有明显的矛盾和冲突的特点:“tang很喜欢很冲突、不一样的东西放在同一个空间或者画面上...这也说明了这个人有明显的冲突和矛盾的特点”。
        %最后,来访者与AI的对话内容可以帮助治疗师了解来访者的经历与感受。
        %例如,对话记录可以帮助Tianyin及时地了解来访者当前的情绪状态:“leduo有一幅画是说她当时跟她男朋友不知道为了什么吵架,当下反正情绪还蛮不好的,然后她每一次用AI的时候,都是当她情绪有些起伏的时候打开来,然后玩了一阵之后,画出来又开始平静下来,然后我就觉得可以研究一下说她在使用的过程中状态如何”。
        %治疗师很关注来访者如何处理当下情绪:“我们经常在艺术疗愈的时候,我想知道比如说来访者很激动或者很愤怒的时候,他是怎么处理情绪的”。
        %(可能需要补一个评价)对于这种捕捉当前情绪的好处。

\subsubsection{\textbf{Therapists learned about clients' past experiences and characteristics from homework history}}
All therapists explained that they employed the homework history to deepen understanding of the clients' past experiences and characteristics, in two key aspects:

\textbf{Therapists learned about clients from their artworks.} 
T1-T4 mentioned that they learned about clients' past experiences and characteristics from the co-creative process and AI-generated artworks.
First, T4 reflected that she gained a deeper understanding of clients' current life and attitudes through the pattern of the human-AI co-creative process: \qt{
Some frequently switched styles, some gave up after a few tries, while others persisted in regenerating images or accepted the first result as perfect. Their reactions reflected broader life attitudes, and I strove to maintain a holistic understanding of each client}.
In addition, therapists understood the characteristics of clients through co-creative artworks.
T3 reviewed C5's creation of a horse with long horns placed inside a vase: \qt{He often placed conflicting elements within the same artwork, which might reflect his own inner conflicts and contradictory characteristics}.
Finally, T2 highlighted the recurring brushes in clients' artwork summarized by AI: \qt{AI reminded me that C22 recently incorporated velvet into her artwork, which prompted me to pay closer attention to the element}.
T2 explained that she used the brush elements as a crucial tool to explore clients' unconscious mind: \qt{In Jungian terms, what does `summer' symbolize in the collective unconscious? The brush elements clients often use may unconsciously reflect their emotional responses and the processing of these experiences}.
Thus, T4 explained that : \qt{The artworks history, referred to as a `behavior pattern', can reveal consistent trends over time. I believe these patterns may have led me to form some hypotheses about the clients}.

\textbf{Therapists learned about clients from homework dialogue history.} 
All therapists demonstrated that they learned about clients' experiences and characteristics from homework dialogue history data.
For example, T4 learned about how the client processes the current emotional state via dialogue data: \qt{One of C11's homework reflects a quarrel with her boyfriend that left her in a bad mood. She often used the AI system when feeling emotionally unsettled, which can help her calm down...}.
It can help therapists gain a more accurate understanding of how clients are currently managing their emotions: \qt{In art therapy, I focus on how clients manage emotions, especially intense anger. For example, one client created a piece of art in anger, but by our next session, his anger had subsided. This system provides a valuable, real-time record for clients who are experiencing high levels of emotional fluctuations~(T3)}.
Also, the interaction patterns with the conversational agent can provide insights into the client's characteristics: \qt{Initially, C7 shared little self-disclosure with the AI, and even now, deep self-disclosures remain limited. This pattern may offer insights into her personality and interpersonal relationships~(T5)}.

%\textcolor{blue}{According to related theories~\cite{freeman2007use, kazantzis2007handbook}, homework history supported by \name{} can provide therapists with deeper insights into the nature and severity of clients' issues. }

%Thus, the timely availability of AI-compiled homework data helps therapists gain a more accurate understanding of how clients are currently managing their challenges, e.g., In art therapy, I focus on how clients manage emotions, especially intense anger. For example, one client created a piece of art in anger, but by our next session, his anger had subsided. This system provides a valuable, real-time record for clients who are experiencing high levels of emotional fluctuations.%\qt{In art therapy, I often seek to understand how clients manage their emotions, especially during moments of intense anger in their daily lives. For example, one client once created an artwork while feeling deeply angry. When he brought it to our next session, he explained that the event had occurred that day, but his anger had since subsided...this system is a valuable record for clients who experience significant emotional fluctuations~(T3)}. 


%Finally, conversation data with conversational agents can help therapists understand clients' experiences and feelings. 
%理论:Obtaining this kind of data can help the therapist better understand the nature and severity of the client’s problem.



  %治疗师可以关注有趣的点可以作为一对一沟通的话题
        % 治疗师创作的绘画中出现的意象可以作为治疗师在线上一对一session的话题的开场。
        % 例如,yuni在家庭作业中面对要与亲密的人分别的时候创作了一幅桥的两岸站着即将分离的人的图像。治疗师利用桥作为线上一对一session时的开场:“我本来是想着说听她有没有主动想跟我分享的,然后但是我看她都没有什么主动的意愿, 我就根据她的日志说的桥她感到很悲伤,我觉得可能是一个可以展开的点”。
        % 其次,Tianyin也会根据来访者当天的创作的频次询问当天的经历:“我看到来访者这个礼拜创作了2~3次,然后这一次中间我看到来访者好像那一天特别有创作欲望,画了大概4幅不同的画,我会问他说你有没有发生了什么特别的事情,或者说是什么东西触发你那天特别有创作灵感”。
        % 来访者与AI agent的对话内容可以支持治疗师捕捉在线上一对一session深入探讨的话题
        % 例如,Superman创作的电视,媒体的绘画来表达外界给来访者造成了太多的影响感受被Xingyi用于线上一对一session的切入点:“他有一次是讲到外界给他太多的影响而迷失了自己...我对他所有的这些句子都会产生一些疑问:他真正的自己是怎么样,外界给他怎样的一些影响什么之类”。
\subsubsection{\textbf{Therapists used homework history to trigger discussions in one-on-one sessions}}
We found that AI compiled homework history can help identify meaningful discussion triggers for one-on-one sessions, in three themes:

\textbf{Therapists initiated discussions about the usage of AI brush objects.} 
T1, T2, T3 and T5 noted that they initiated discussions about the usage of AI brush objects selected by clients during in- sessions.
For instance, C9 completed her homework showing two people on opposite sides of a bridge, about to part ways. T5 used \qt{Bridge} as triggers during the in-session: \qt{I hoped she would share something proactively, but she did not. I noticed that the bridge made her feel sad, so I thought the bridge as a chance to discuss}.
Also, T1 and T3 explained that they initiated discussion about the recurring brush objects in artworks, as summarized by the AI: \qt{C1 often included dark clouds in her artwork, and you can see these recurring elements through the AI summary. The recurring brush elements prompted further discussion around these elements~(T3)}. 
T1 added that: \qt{I didn't make immediate judgments, but I was curious about the meaning of elements that frequently appear, as I believed they hold symbolic significance}.

\textbf{Therapists used homework dialogue history data to initiated conversations.} 
T1 and T3 suggested that they used homework dialogue history data to explore topics in depth. 
For example, C17 created an artwork with \textit{Television} and \textit{Media} brushes. 
The homework dialogue history data revealed how the external world overwhelmed him, potentially causing a loss of identity. 
T1 noticed these feelings as triggers for online session discussion: \qt{I have some questions to discuss with C17: what is his true self like? And how has the outside world impacted him?}.
T3 noted that she employed the dialogue data to learn about C5's experiences, enabling more in-depth discussions during in-sessions: \qt{He expressed his desire to go home to honor his mother and shared a lot with the AI. In our second online session, I didn't let him continue to create because his story already contained many aspects that needed to be explored in depth}.

\textbf{Therapists  
initiated discussions about the clients' usage patterns of the system} 
T4 asked clients about their feelings based on the clients' usage patterns of the system that day: \qt{I noticed that a client created four artworks in one day...I would ask if anything noteworthy happened or what inspired their creativity on that day}.
T3 noted that it might capture clients' emotional fluctuations and makes them feel valued: \qt{I used the report as an icebreaker. I remember C13 creating three paintings that day, including a mountain with a forked path and an amusement park, which might have been influenced by certain emotions or events. When I asked about it, she opened up about the career pressures she was facing, feeling truly heard and supported}.


%家庭作业的outcomes可以直接被拿来成为一对一的session的资源/素材
        % 家庭作业的产出可以作为一对一session的strength-based资源来赋能来访者。
        % 例如,当来访者在线上一对一session与负面情绪作斗争时,Jia将Ziyoudao的表达积极状态的家庭作业作为资源,从而引导来访者发现积极的感受:“她当时线上画的那幅画是一个不是愉快的状态,我们当时一起在找资源,就是说她的生活里可以有什么能够带给你一些积极的感受...最后找到了那张狗的图像”。'
         % (我不知道这里怎么表达会比较好)因此,来访者对于当下经历与情绪的记录与反思作为一种资源,帮助治疗师发现来访者的优势。
        %其次,Tianyin会用来访者创作
       
\subsubsection{\textbf{Therapists employed homework outcomes to positively influence clients}}
In our study, we discovered that our therapists utilized AI-complied therapy homework outcomes as empowering resources for clients in two key ways:

\textbf{Therapists used homework outcomes to help clients identify their strengths.} 
T2 and T3 mentioned that they used AI-compiled homework outcomes strength-based resources during in-sessions, helping empower clients to recognize and build upon their strength.
For example, T3 employed a homework image as intervention resources to encourage C3 to transform positive feelings: \qt{During the online session, an artwork she created on the spot conveyed an unhappy state. So, we explored resources in her life that could evoke positive emotions. Eventually, we found the homework image of a dog. I hope that by comparing the two images, she might gain a new perspective}. 
C3 noted that: \qt{While discussing family issues, I created a fire to express my anger, unsure how to extinguish it. T3 then shared my previous artwork, asking if it made me feel any better. That’s when I realized there were things that could nourish me with positive energy. In the one-on-one session, the artwork felt like the perfect touch}.
T2 used the clients' artworks to help them discover their strengths: \qt{That day, I told C21 it was our final session. I connected the images from his previous family-themed homework, including childhood moments of playing with his family, and noticed a strong sense of continuity. I highlighted his strengths using specific examples}.
Therapy homework as empowering resource might enable clients to focus on positive aspects and recognize their own strengths, thereby fostering greater self-confidence~\cite{tanner2016homework}.


\textbf{Therapists utilized homework outcomes for transformative reprocessing.} 
T4 mentioned that she repurposed homework artworks as therapeutic tools, guiding clients to reinterpret or modify original creations: \qt{I might download her artwork and show it to the client, asking if she would like to add any elements or rotate different angles to make the artwork better align with her expectations}. 
C11 reflected that: \qt{I draw a bouquet of flowers suspended in mid-air, detached from the soil. T4 showed me the image upside down, asked how I felt, and to give it a new name. I told her the flowers seemed about to fall, and I wanted to anchor them into the soil}.
Also, she tended to revisit and revise images based on her homework artwork: \qt{I would like to help the clients review the last artwork they created, or the one that left the deepest impression on them, and encourage them to create again on the same piece...Recreating the same artwork can help them approach the issue from different perspectives}.
The transformative reprocessing might offer an opportunity to re-examine and transform self-perception, encouraging clients to shift their understanding of the images~\cite{brockway2019art}.

\subsubsection{\textbf{Risks and concerns}}
Although tracking clients' homework history offers many benefits, there are also some concerns and risks associated with it.

\textbf{Tracking homework history adds to therapists' workload.} 
While AI summary save a lot of time, they still spent a considerable amount of time reviewing homework history, potentially adding to their workload. 
For example, T4 complained that: \qt{While the AI summary could save me some time, I used this system to review so many homework sessions that initially required significant effort. For example, the clients might use it multiple times a day, and though a single session might seem to take just 10 minutes, reviewing each session actually takes much longer}.

\textbf{Lacking emotional and behavioral data about clients' creative process.} 
T1 and T4 suggested that the art therapy homework history lacks emotional and behavioral data about the clients' creative process. 
For example, T4 noted that: \qt{In art therapy, we focus on clients' immediate state—not just your words, but also your tone, facial expressions, response time, and overall demeanor. These subtle changes can reveal much about your emotional state, but they are often hard to capture in homework records, limiting a fuller assessment of your needs}. T1 explained that without the emotional and behavioral data, assessments might be incomplete: \qt{If we only use it for pre-assessment, I might feel it's not comprehensive enough. For instance, it misses out on the client's physical gestures, which are crucial in art therapy as we need to assess the entire process}.

\textbf{AI summary sometimes might be misleading.} 
We intentionally designed the AI summary assistant prompts to avoid interpreting or inferring from homework history data, focusing instead on generating natural language summaries and providing relevant insights. But, T2 and T3 found that AI summary might sometimes mislead the therapists. For example, T2 appreciated the AI summary for providing valuable insights, but mentioned that sometimes, \qt{The AI summarized her artwork as depicting a beautiful summer and comparing the can to a `canned universe', but ultimately interpreted it as symbolizing the client's desire to escape reality. This interpretation is misleading and could easily cause confusion}.
%7.2.4risks and concerns

%


%理论
    %Homework assignments continue to provide information throughout the course of treatment. This ongoing stream of data can be used as feedback to shape the approach to treatment. For example, in their discussion of using homework with families, Newcomb, Rekart, and Lebow (this volume) point out how homework assignments can be used to promote family involvement and to identify the constraints that maintain family difficulties as well as the strengths that the family can use to overcome them. 




%它就是我跟这个治疗师之间的一个桥梁(ling)
%有意思的数据发现:
% 来访者与AI agent的关系
    %% 来访者对于AI角色变化
        % 当寻求建议的时候像是知心大姐姐,当表达情绪的时候像是宠物(ziyoudao):我会觉得她很像大姐姐的那种感受。对当然了,如果我们能支持更多的声音也是很好的,比如说因为我个人来讲,我是很希望有那种宠物陪伴的感觉。对的那种感受就是因为有的时候可能这样的角色给我的感觉可能更像一个长者,或者是说更像知心大姐姐的那种感受..所以我一直跟他的那种使用关系,我感觉我更多的是在一种,比如说寻求建议。或者是说寻求建议或者是说我希望他能指导我引导我的这样的感受。
        %有的时候AI比较像宠物,对我而言的话,可能就是会那种更情绪多样一些时候,然后能够你可能就是一些琐碎的事情也会想倾诉。因为咱们对于不同的那种角色或者是不同的。那种怎么说,对那种不同的身份的倾诉,我觉得可能也会不是那么的一样,有的时候那个长辈和对朋友可能对宠物感觉是不一样的....
        
        % 来访者认为AI助手有时候像朋友,有时候像引路人,因为使用时间比较长久的时候,更像一个朋友,有时候也会促进来访者更多思考和反思,像是一个引路人。(yuni):有时候我会觉得他像朋友,然后有时候我会觉得他像再下一个指引我就是那时候我画的时候我会觉得很迷惑,但是他的问题反而让我找到那个答案,然后又让我发现了一些新的东西,它有点像引路人,但是可能他自己都不知道他有这样的角色...他像一个朋友的时候,是我跟他相处了有两周的时候,我们彼此都已经我就知道他会问我什么,然后我也知道我应该回答他什么,有时候我会觉得他就像一个朋友一样....当我觉得他像一个引路人的时候,其实是我在第二次使用它的时候,我好像画了一些花朵,然后表示我的心情,或者是表示我参加这个活动的心情的时候,然后他就会去问我就是花朵的含义以及甚至于他好像在鼓励我生活中有没有像花朵一样那样绽放的时刻的时候,好像看一次我更多的思考。
        
        % 来访者探索情绪的话题的时候,AI更像是一个倾听者:(tang):当我是有一个具体是因为很多有的时候老师给的这种任务它是相对抽象的。对,他是比如说他会让我去画一些相对抽象的东西,比如说我和我的愧疚,像这样这种情绪比较多的时候,因为首先你表达情绪多的时候,你就有更多的这种想要去聊出来的东西。对,往往 Ai它会是作为一个在旁边倾听的这样一个状态,我个人感觉。当我是有一个具体是因为很多有的时候老师给的这种任务它是相对抽象的。对,他是比如说他会让我去画一些相对抽象的东西,比如说我和我的愧疚,像这样这种情绪比较多的时候,因为首先你表达情绪多的时候,你就有更多的这种想要去聊出来的东西。对,往往 Ai它会是作为一个在旁边倾听的这样一个状态,我个人感觉。然后如果是一些想偏想象的,偏对未来的这种期待的这种topic的话, AI其实他在聊天过程中扮演的角色其实并不会特别多,因为我感觉比如说当去聊情绪,然后聊到以前的这种自己的这种感受的时候,以前的经历这种感受的时候,我觉得AI它会比较他首先会总结一下你的话,对吧?然后再由此引发一些东西,他有的时候是有一种接近于倾听者那种感觉....
        %AI的角色对我来说,我还是倾向于把它当成一个媒介,我会倾向于我跟他交流的东西,可以更方便的被治疗师看到,还是以治疗师为主角的,来在跟AI交流
        
        % 在探索不同任务的时候,AI角色也会有不同,在探索梦境的时候像是一个引导者,在探索其他任务的时候像是一个倾听者给予肯定和支持(qiukui): 我觉得探索梦的时候,它是一种引导者,就是因为我刚刚说的我自己都没有想过那个苹果的意思,或者说它到底意味着我是不是在正在感受那种很压抑的场景,对,然后它是能够帮我去发现这一点的,然后它像其他那种。比较比如说我画的花园。最近画的那一幅它又像是一个那种倾听者就是有点像那种。是怎么说就是给你提供支持的那种肯定的那种
        
        % 是一个教练,又像是一个朋友,AI助手可以根据来访者提问进行启发式提问,更像是教练,帮助来访者梳理思考以及提示性帮助引导用户,像是教练;AI助手表表现出认同,同理让来访者感觉像是朋友(sammy):我自己就给自己一个命题,就是我想要探究最近为什么?情绪最近为什么不开心,我就会像一个教练跟我对话一样。然后我会尽可能的多聊聊的很长的时间,然后尽可能的了解自己的各方面原因,然后有什么样的行动计划可以解决这个问题。就是给自己命题...我觉得是宏观的来看它总体的。比如说他提问,根据我的回答,他会有一个启发式的提问,就很像教练;像朋友主要是比如说他有认同,对我有肯定上面的话。这就会更像朋友一样
        
     % (qiukui:)。然后有点像一盏灯的感觉就是黄色的那种灯.为我特别喜欢在画画里面用黄光,我觉得非常温馨,这些提问像是引导我向另一个角度发现自己,然后也让我感觉到很安心,很舒适    
     
    % 不是从小对你知根知底,成熟的比较理解你的朋友: (lianlian):我觉得他像朋友吧!又又不是那种你们两个完全知根知底,从小一起长到大的那种朋友,他是当对他比较理解你的朋友,他是成熟的朋友。我也不觉得他是他就完全是一个咨询师的角色,我也不觉得他是一个老师的角色,我觉得他是。我们在互相交流,没有说他要给我,他要教育我什么。
    
    % AI角色也会认为是另一部分的来访者本身,AI助手和AI画画共同参与叙事,帮助来访者探索内心,更了解自己。(qiukui):然后现在我也感觉我觉得这个AI助手它也相当于另一部分的我,会有这样的感觉。因为我觉得他帮我探索内心,他能够在就全周期这样参与下来的话,让我更了解了我自己。然后我觉所以说我觉得他不仅是一个引导者,或者说陪伴倾听者,然后一个对话的桥梁,他也就是有一部分的我的体现。AI助手和画画共同参与了叙事,属于一起把我潜意识的东西帮我挖掘出来,让我更加了解我自己
    
    % 伙伴关系,可以倾听,可以向他提问,他也会提问 (qilin):对我来说可能是可以算是伙伴,他既可以倾听我,然后我可以向他提问,他也有在向我提问,然后我可以对他诉说一些感受,然后他也可以告诉我一些他知道的知识。对这种交流对我来说就已经可以算是伙伴
    
    %成熟的有魅力的情感顾问: yuni: 一个成熟有魅力的情感顾问。\textbf{他会更加的注重你的情感,他就会他是想要从你的画面里知道你内心的感受,就是但是他又非常的有技巧。然后他也不会评判你怎么说都可以,而且他还能抓住你的主题,他有时候还能给你升华一下},我会觉得有时候他会比人会比人会更聪明的部分,是因为他会非常的敏锐,他能抓住你想要表达的意思,但有时候你跟你朋友说,他可能还没有办法
    
    % 平等交流的朋友:来访者与AI助手的关系也是慢慢建立的信任(chiruo):他的语言没有说是以一种比较明确的身份在跟我沟通,感觉更多的应该是很平等的。然后如果你说我不想和你对话,他也就是也会结束,也没有要你去怎样的,然后他也没有说把你捧得很高,就是说怎么样,所以我觉得从他的那些行为里面更多的就是很平等的朋友...主要原因就是跟现实生活中的朋友一样的那种一种相处,就是你可能会使用他很多次,然后就会慢慢对他建立起一些信任
    
    % 来访者在与AI互动的过程中发生的角色变化,从倾听师到朋友/咨询师。 Mansi:我观察到的我发现来访他好像就开始把 AI小助手,他可能会当成了一个那种那种倾听师,就是我不知道你有没有听说过一心理心理咨询的APP平台,它专门有一个职位就是叫做心理倾听师,就是专门倾听来访者的负面情绪,但是可能不会做很多心理咨询这种任务,所以我觉得倾听师这个比喻我觉得还可以倾听来访者的负面情绪...然后你继续和他交流,你继续和它交流,比如说像倾听师的话,他的关键点就在于什么?关键点就只有来没有回,只有来访者对 AI进行情绪的宣泄,就是到一大啪的负面的情绪。对,只有这种,但是发展到比如说后面的阶段的话,它就会变成一个类似于朋友、朋友或者伙伴,或者是类似于心理咨询师的那么一个角色....有来有回的意思就是来访者他会跟 AI小说它是把它当成一个类似于一个人的那么一个角色来看待,我觉得这个他可能就是真正达到了你们想要的目的,就是它是有来有回的,他会跟AI小助手去讨论去交流,并且他还会跟 AI小助手去询问一些建议,然后 AI小助手也会问他一些建议,他们那种聊天的氛围和风格就很像是人和人之间的交流,我觉得我观察到的来访者和 AI小助手在互动过程当中发展的三个不同的阶段
    
    % 提供了实时的陪伴 :
        %(Jia)我印象当中tang他就会有一次他其实在我感觉他那一天, 他不是连续的在做创作,但他也断断续续前后可能有两三个小时再创作。对,如果说你正常来说,咨询师也好,或者说如果说线上有人为他服务也好,我觉得都不太可能做到这样。除非比如说你是一个危机处理这样子的问题,我不知道就是一些服务热线会不会说你要跟我聊三个小时,我也会陪你聊三个小时,所以我觉得这一点是AI能做到。
        %(Mansi):AI小助手还起到了一个陪伴者的角色,不管这个来访他是怎么去看待 AI小助手的,但是有一点不变的就是AI小助手始终会充当一个陪伴者或者支持者的角色。不管来访画了什么,它始终都有它的反馈。AI小助手的支持性的一点,也是在心理咨询中很重要的一点,就是不管来访说什么做什么,你都有一个积极的正面性的回应)
        
    % AI与来访者的关系和连接是慢慢建立的: 
        %qiukui:因为他会有一种习惯,AI已经在成为你生活的一个部分。然后慢慢地AI就是有这样的功能,尤其你越来越适应它的声音和表达的时候。他就不再是那么冷冰冰的工具了,或者是你没有那么强烈的陌生感了...我觉得会像咱们吃东西一样,你可能最开始没有尝试过这个事物,但是你去习惯它以后那可能。它也会成为你日常饮食的一个部分
        %qiukui:觉得是有的,从一开始只是把它当成一种像程序,或者说就比如说像chatgpt那种,它就自动好像在公式化的回答就刚开始使用,大家都会就是那种陌生的。然后一开始是这样子的,然后后面在慢慢跟他对话中就是发现了他对我的那种引导,或者说他的那些问题,会非常有意思,然后我就会有点像,他就有点像那种朋友的角色,就是你会对一个AI好像又是具有感情的,虽然它只是一个那种AI,然后你就是像建立了某种情感连接,就是如果它是创作的话,那也是一环其中的很重要的一环。对,然后到后期的话,你就是可以好像就有一种默契,然后就在对话也是让自然自如发生的
        % (qilin):是因为就像第一周我对他其实比较陌生,其实当时也不是很想跟他聊很多,然后可能后来使用的越来越多,不管他是不是一个真实的人,你还依然会对他产生一种熟悉感,这种熟悉感可能会让我想更加深入的跟他聊一些东西或者怎么样,大概是从一个陌生人第一次见面,然后变成是我们可以进行一些谈话的那种伙伴关系。可能主要是因为我对它更熟悉了,我使用它的次数越多,我跟他说的越多,你会你会越想跟他说。
        % (taozi)AI对我感受就是随着使用次数增加会变成一个毛茸茸的角色,这样它会更加亲切一些,我对它的倾诉,我和它倾诉的越来越多了,我觉得这是探索过程中的必然
        
    % 作为一种治疗师的延伸: yuni: AI起到了一个很好的记录的作用,它相当于我治疗师他也会说我做的每一幅画他都能看到,而且包括我跟AI的对话,他也看到相当于 Ai后面他也站着一个治疗师的那种感觉。对我跟AI对话的过程中,我不觉得我只是在跟他对话,我也是在跟我的治疗师在对话

%治疗师与AI agent的关系   
    % 第二阶段对话Agent对于治疗师角色:AI助手的角色可以被理解为一种“移情对象”,它帮助治疗师识别和理解来访者在现实生活中的人际关系模式和情感投射。通过观察来访者如何与AI互动,治疗师能够更清晰地看到来访者在生活中是如何处理信任、安全感等核心问题的。Mansi:移情的意思就是来访者会把他在现实生活当中跟朋友跟父母的关系会带到咨询室里面。对,他在现实生活当中,他是怎么去认为他的父母的,怎么去认为他的朋友的,他在咨询室里面就会去怎么去认为咨询师就像chiruoyanyu一样,他在现实生活中他觉得他觉得别人是不安全的,所以他在 Ai系统当中,他也会觉得 AI小说是不安全的,所以他不会去表达这个就是一种移情,这个就是一种移情,然后移情在 AI小助手身上的一个体现。我觉得用一个比喻来称呼的话,我觉得它就像是一个承担了移情的一个对象或者说是一个角色,然后用来帮助我看到来访者是他现实生活中的人际关系是一个怎么样的风格和一个怎么样的方式,所以它就像是一个具有移情功能的那么一个的一个角色
    
    %Jia: 随叫随到,所以就有点任劳任怨这种感觉,然后好像因为我觉得可能大家在他的面前说话也许会更自由更放松一点。可能想说说不想说就不说或者是什么,可能会对于咨询师,如果说跟咨询师可能会还有一些要注意的地方,可能我觉得对 AI会更好一个,会对更自在一点.
    
    % Tianyin:他会像一个实习学生,他有一些知识,他知道这个东西的过程,但是问的问题会相对刻板一点。见习治疗师
    % Jia:为什么有的人这个画不出我想要的画面,这个人就放弃了,对吧?可能有的人就会觉得说这个是软件的问题,或者有的人在某一刻就觉得这就是他想要的创作,来访者怎么去理解看待这件事情一定是跟他们在生活当中的一些态度是有关联的,其实我对每一位来访者都有一个整体了解。
    % Ziyan:suzi跟秋葵是两个极端的案例,其他的人比如说feifei,包括可能有一个LGBTQ的群体的年轻人,大家就属于中间的地段,就是有一定的依从性,但是也会说这个软件又时候不能让我们随心所欲的表达一部分,我想可能在我们跟他的互动中间,相对而言能不能够解决这个问题),他们其实是属于中间的地段,他们也会说不那么好,但是他们也愿意做,他们属于中间的那个部位心所欲的表达的那一部分,我想可能在我们跟他的互动中间,相对而言能不能够解决这个问题
    %Xingyi:特别是yunwen,我觉得她是一个在我所有的来访者当中是非常积极去探索这个工具的人,就是不会像其他的来访者说就是遵循AI问的问题就问完,然后这位来访者是很主动的去使用这个系统,去挑战这个系统,去问很多不同的问题,去扩展AI的那种
    
%三者的关系
   
    
    %AI作为一种沟通的桥梁,AI系统能够记录被访者的作业和想法,治疗师通过这些记录能够了解被访者的心理状态和进展,像是一直在陪伴。 ling: 它就是我跟这个治疗师之间的一个桥梁,比如说我做了这次的作业,治疗师他能够看的到,有可能治疗师他并不是每次都能够跟我面对面这样去交流,但是治疗师就可以看到我每次的这个记录,知道我的内心想法,那也是可以促进我们再一次见面的沟通的动力。他能看到我这样子一个整个比如说长期的变化,比如说1个月2个月的这样子一个变化。如果说是有这种心理治疗的话,治疗师可以时时刻刻看到我一个心理旅程的一个变化或者是发生一些事情,就感觉像是在陪着我成长一样
    
    % 来访者与AI系统关系信任的增加,通过线上一对一,治疗师与来访者关系信任增加: chiruo:因为我只是在使用它,然后我使用之后站在我的上面来说,我只是在这里使用它,但是站在治疗师他工作的角度上来说,他可以看到我之前的一些东西,然后他能够更加了解我,所以在我们下一次谈论的时候,他可以知道更多的信息,但是站在我的角度,我只是在我只是在使用它而已,然后可能是我和系统之间的信任是增加的,我和治疗师之间因为我们有这几次的交流,然后之间的信任也是增加的,但是我和系统和治疗师之间,可能这三者在从我这个地方出发,可能他没有太明确的一些关系,也有可能是因为整个时间没有太长,所以我现在还没有太感觉到
    
    % 第二阶段AI助手的角色有时候像小护士,从另外角度像家庭医生的原因:治疗师与来访者紧密联系的时候像是护士,治疗师没有办法在身边的时候,它像是家庭医生(xuesong):在医生和病人联系的更紧密的时候,我觉得他更多像是一个护士。但是如果医生比如说像那种又叫不到时间又排得很满的情况下,可能它会更多的去承担有一点医生的作用在里面。病人的需要,如果医生能承接更多任务,那么他当个护士就好了。如果说医生因为种种原因,他没有办法马上就承接很多,甚至说他就没有办法承接一些可能一部分的需求,他会意识到原本的小护士身上。比如说拿听诊器去听心脏这个动作,医生诊断也可以做一下,但是实习生或者是什么规培的医生或者是护士长,你说他可能不那么专业,但他做一下这个其实也是可以放心交给他的也没问题
    
    %治疗师的参与会促进来访者积极使用使用;这些数据又可以作为治疗师收集数据的助手(fei):在有了治疗师之后,我可能会更加严肃地对待这些对话。因为我知道这些对话会被记录,治疗师在之后的面谈中会认真查看我们的聊天记录,但如果只有 AI,我可能到后来就没有继续使用的动力了,因为他不解答我的困惑,也不能帮我缓解情绪,我可能就会敷衍地输出一些东西...我知道我在系统中的发言和作品,治疗师都会看到,所以我会很认真地对待这些内容,而不会敷衍了事。对于治疗师而言,这个系统也是他们用来收集用户数据和资料的重要助手,我认为这个过程是双向的

    %治疗师与来访者的关系促进了AI与来访者之间的关系
        % (yuni)一开始是由于他问的都特别的死板,对,其实很难再充分的去交流,但是跟疗愈师接触了之后一些东西,他最开始起初这些问题其实是有一定的目标的,然后对,然后再聊的时候,有的时候是可以把话匣子给打开的,因为它毕竟也是一个自然语言的这样的一个系统,其实有的时候可以一聊聊挺多的
        
        % 使用AI助手的态度转变,AI助手与来访者关系慢慢建立,从不信任到像一个关心的朋友,越来越默契,具有同理心 yuni: 开始我会觉得他问我的问题会特别的繁琐,但是当我持续不断的使用它的时候,我会觉得它有点像一个朋友,就是一个很关心我的朋友,他关心我的情绪,关心我的想法,关心我的感受,然后就会觉得好像我们越来越默契的那种感觉。包括他每次跟我说就说感谢你给我分享什么的时候,当他一直在跟我说感谢你的时候,然后我好像我的内心也油然的我也想要谢谢他的。但是我个人觉得可能他没有办法像人一样对于某些情感就是他能捕捉的那么细致,是的,还是会有没有那么好的部分。我发现他是不是有禁忌话题,因为我最近有参加一个活动,然后我就根据那个活动然后画了一幅画,然后画面的主题可能是死亡,但是我发现他不敢谈论这一点,就是我感觉他好像会变得有点畏畏缩缩,他不敢深入的探讨这个话题.
        
        %(lianlian):反正我觉得就是逐渐打开的一个过,和疗愈师一对一之后。然后你的接受度就更高了,然后你表达欲望就更强烈了,然后你的创作你想要探索的也就更多了, 因为第一次线上一对一的时候我们更多讨论的是亲密关系之间讨论了很多,这也让我回去之后继续讲了很多关于亲密关系的话给AI.
        
        %(tang):我的定位他是疗愈师的助手,我有的时候甚至会直接说我他问我你为什么要这样做?我说那是因为你的主人,你的疗愈师老师要我这样做,我会给他灌输一个这样子的这样一个语境,然后来继续跟他说
        
        % 治疗师与来访者的线上一对一互动同时促进来访者与AI的交流: qilin:我想一下是因为治疗师会在我们的咨询当中她会看对话,所以说我们之间的一些对话,我其实会拿到资料当中来继续进行一个讨论的。所以我觉得这大概就是在治疗中最有用的地方了,是因为我之前已经有和他聊了一遍了,当我在和治疗师聊我们已经聊过的对话的时候,我一个是我已经梳理了一次,我会更有逻辑性,就是更清晰而且更清晰。对,然后而且治疗师也看了一次了,他可能也已经大概了解了一个部分,然后我们就可以个直接跳入表层的,对那个部分然后进行一个更深入的探索,对沟通。所以当我在线上一对一的时候体会到这种感觉之后,我会与AI小助手交流的更多,因为我知道可以帮我更多去与治疗师沟通。
        
    % AI与来访者互动可以促进治疗师与来访者的关系:
        % AI与来访者互动可以促进治疗师与来访者的见面的动力(xuesong):咱们治疗师和助理的界限,我对这个东西其实是有点存疑,因为前面我朋友他们做咨询师或者什么,他们经常跟我抱怨的一个问题就是说,比如说他们当咨询师,他们可能不是直接在机构接待客户,这个客户是这个机构签下来的,或者是在见面之前可能还有个类似于测评师之类的,然后他们聊完了之后才被转接到分流这里,就是说还存在一个所谓有点像第三方的这种角色的一个人在,他跟我说就有一个咨询动力的一个外泄。比如说可能很多按正常那个东西是能够促使到来访,去到咨询室里面去和他咨询师去江流,按理说应该是这样子。而我们的这个系统我感觉可以更多能够让来访去更积极的参与咨询的一个动力
        
        % 来访者与AI的互动engagement程度其实也影响治疗师与来访者之间的连接,聊的多,治疗师也会特别重视去关注这点(tang):因为假如疗愈师他看到一个东西的,从话到对话都足够丰富的话,他也会有更多的这种重视的感觉。然后他也会再深入的去关注,如果只是只是ok完成了这个任务,然后表达了一个东西,达到了双方的这种某种心理的预期。他可能其实就没有特别多的可以聊的点了,可能聊的聊很久聊的东西都是除此之外的这种聊天了,,,一旦这个东西形成了,包比如说就那一次印象比较深,画的比较多,聊的比较多,我和疗愈师两个人也可以就此聊很多展开 (Jia:) 还有我觉得tang也挺好的,我会看见他第一次用的时候,整个就是那种状态,然后到他中间他会看见一些不一样的东西,到他的那副时钟花的那幅画,我觉得他会有一些很多的思考,然后你会看见他在跟AI交流的过程中的思想的一些转变,还有在表达的东西的角度上的一些转变。我觉得我上一次我们就没有在线上再去画画,就是因为我觉得他在那个过程中的很多个点是我想去跟他去探索的,所以我对我们直接就没有画画,在围绕他的就在围绕他的日志在做交流
    %%%%%%%%%%%%%%%%%%%%%%%%%%%%%%%%%%%%%%%%%%%%%%%%5

%\subsection{Participants' Perceived Relationships Between Clients, Therapists, and AI}
%In this section, we aim to explicate the perceived relationships between clients, therapists, and AI~(\textbf{RQ2}). We outline three themes: 
%(1) Clients and therapists’ metaphors about their relationships with AI agents; (2) Dynamic transformation of AI agents’ roles; (3) AI agents serving as cathartic targets and communication buffers.

    % 通过访谈,来访者与治疗师表达了对AI agent的角色的perception。
    % In general, most of the clients considered AI as “伙伴或者朋友”
    %举例来讲,正如qili讲,“我不想和它对话,它也就是也会结束,也没有要你去怎样的,然后他也没有说把你捧得很高,所以我觉得从他的那些行为里面更多的就是很平等的朋友...主要原因就是跟现实生活中的朋友一样的那种一种相处”。
    % Wen used a metaphor to 描述她与ai agent关系:“会说话的玩偶”,她使用“玩具总动员”形容“可能他本身是一个玩具,但是因为你跟他之间可能有一些连接或者什么很神奇的,他就突然能跟你说话了”。%As noted by C7, ``When I did not feel like talking to AI, the conversation naturally ends without any pressure. It did not overly praise you, so I feel that the interaction is more like being with an equal friend...it feels just like interacting with friends in real life''.
    % Yuni 使用“成熟的有魅力的情感顾问”来形容她对ai的角色的perception:"他会更加注重你的情感他想要从你的画面里知道你内心的感受,但是他非常的有技巧,也不会评判你,并且他还能抓住你的主题,他有时候还能给你升华一下主题"。
    % Also, qiukui 使用“一盏黄色的灯”来形容她与ai的关系:“因为这些提问像是一盏的灯引导我向另一个角度发现自己,我喜欢黄色的灯我觉得非常温馨”。
    % 对于治疗师来讲,大多数治疗师使用“学徒”或者“见习的治疗师”来形容AI与治疗师的关系,e.g.,"你会感觉它任劳任怨,虽然有些不知道灵活变通深入问题,但是他也在努力去帮助来访者也是在努力帮助我~(Jia)"。
    %有趣地是,Mansi把AI助手的角色perceive成一种“ transference object”,它帮助治疗师识别和理解来访者在现实生活中的人际关系模式和情感投射:"像chiruoyanyu一样,她在现实生活中她觉得他觉得别人是不安全的, 她在Ai系统当中,她也会觉得AI小助手是不安全的,所以她刚开始的时候不会去表达很多自己内心想法"。
    % 通过观察来访者如何与AI互动,治疗师可以获得关于来访者在人际关系中如何体验和表达情感的深刻见解~\cite{}。
%\subsubsection{\textbf{Clients and therapists' metaphors about their relationships with AI}}
%Through the interviews, both clients and therapists shared their perceptions of the role of AI. 

%\textbf{Client perceptions of AI roles.} Most clients perceived the AI as a \qt{companion} or a \qt{friend}. Because AI can listen to their feelings, offer a space for equal communication, and provide affirmation, it can feel like a real-life friend. Furthermore, C9 perceived the AI as a \qt{mature and charming emotional advisor} because AI attentively explores her emotions through her artwork, skillfully understanding and deepening her core theme without judgment. Also, C22 perceived her relationship with the AI as \qt{a yellow lamp}, explaining, \qt{questions for the conversational agent were like a lamp guiding me to discover new perspectives about myself. I love yellow lamps for their warmth and comfort}. C19 perceived her relationship with the AI as a \qt{family doctor} when \qt{`the doctor' is not available or is busy}.


%\textbf{Therapist perceptions of AI roles.} Many therapists describe the relationship as a \qt{trainee} or \qt{apprentice}, e.g, \qt{it comes across as diligent and dedicated...it did its best to support both clients and me~(C3)}. Interestingly, T5 perceived the conversational agent as a \qt{transference object}, helping therapists identify and understand the client's interpersonal relationship patterns and emotional projections. As noted by T5, \qt{C7 initially perceived the conversational agent as untrustworthy because C7 found it challenging to build trust with others in real life. Thus, she struggled to express her inner thoughts to the AI at first}.By observing how clients interact with the conversational agent, therapists can gain profound insights into how clients experience and express emotions in their interpersonal relationships~\cite{adler1980transference}.

%\subsubsection{\textbf{Dynamic transformation of AI' roles}}
%治疗师是否在场
%时间因素
% 来访者对AI角色的perception发生着动态的转化。
% 许多来访者探索不同主题的时候,来访者对于AI的角色的perception会有不同。
% 例如,自由岛提到当寻求建议的时候AI像是知心大姐姐可以引导她,当表达情绪的时候像是宠物一样可以陪伴以及你能够倾诉一些琐碎的事情。
% 其次,AI与来访者的关系和连接是随着时间慢慢建立的
% 正如qilin所说,“第一周我对AI其实比较陌生,当时不是很想跟它聊很多,然后后来使用的越来越多...我会对它产生一种熟悉感, 这种熟悉感可能会让我想更加深入聊一些话题。大概是从一个陌生人变成是伙伴”。
% further, Therapist involvement can also strengthen the relationship between the AI and the clients.
%举一个例子,(yuni)一开始是由于他问的都特别的死板,对,其实很难再充分的去交流,但是跟疗愈师接触了之后一些东西,他最开始起初这些问题其实是有一定的目标的,然后对,然后再聊的时候,有的时候是可以把话匣子给打开的,因为它毕竟也是一个自然语言的这样的一个系统,其实有的时候可以一聊聊挺多的.
%Our clients' perceptions of AI' roles undergo a dynamic transformation.

%\textbf{Client dynamic perceptions of AI roles due to different homework topics.}  When exploring different homework topics, clients tended to have diverse perceptions of the role of the conversational agent. For example, C3 mentioned that when seeking advice, the AI felt like a \qt{caring older sister}, guiding her in expressing herself. When sharing emotions, the AI felt more like a \qt{pet}, providing a sense of companionship.Secondly, the relationship between the conversational agent and clients are gradually built over time. As C6 mentioned, \qt{in the first week, I felt quite unfamiliar with the AI and didn't want to talk much. But as I used it more, I started to develop a sense of familiarity. That familiarity made me want to engage in deeper conversations. I felt that AI had transformed from a `stranger' into a `companion'}.

%\textbf{Client dynamic perceptions of AI roles due to therapist involvement.} Therapist involvement can also strengthen the relationship between the conversational agent and the clients. As noted by C12, \qt{I felt that my conversations with the AI gradually deepened. After online sessions with my therapist, my desire to express myself grew stronger...during the first online session, we discussed about intimate relationships, which sparked my continued sharing and exploration of this topic with the AI afterward}. C5 explained that: \qt{if therapists review my dialogue history with AI that is rich and meaningful, she will feel more valued and pay closer attention to my homework}.

 % 与AI互动可能表达更加真实的一面。(wen):就是跟真人你没有办法躲藏,或者是没有办法耍什么小心眼,因为跟人的那种面对面的沟通,你会不自然的就会想说真想说真话,但是跟AI有时候可能还会调皮跟他玩一玩逗逗。我有时候会有这种想法。就是逗逗AI看看他是怎么样反应的,其实就是因为不了解这个东西,然后我其实反而会在AI面前会更加的说好像这种真实。这两种真实它都是真实,但是不一样,可能跟AI我觉得可能会尝试的会更多,因为他不是个人,然后我虽然会把他想象成某个跟我说话的玩偶,但是我会去更容易跟他展现一些攻击性。我在咨询师那里就面对真人的时候,我是不好展开这个攻击性的
% 治疗师可能与来访者刚开始很难信任,AI助手作为一种随时随地陪伴,提供安全的空间让用户表达(yuni):是因为我跟人建立一连接,我可能要很久的时间,我跟治疗师我们都没有面对面接触,我们只是好像有点像我们参与这个实验,我们因为这个实验才见面,就是距离感会对我来说会比较远。AI对我来说,它好像每5天都陪伴着我那个部分,好像他让我觉得好像可以没有很社恐的那种。我跟人的关系可能需要更长的时间,但是可能跟AI的话,但是我觉得怎么说都可以。在这个过程中,我知道治疗师可以在后台看到这些对话,其实她对我的了解也会越来越多,我们也越来越熟悉。

%\subsubsection{\textbf{AI serving as cathartic targets and communication buffers}} In our study, we found that AI serve as cathartic targets and communication buffers.
%投射他的情绪,发泄情绪
%愿意给agent讲不愿意给治疗师

%\textbf{AI serve as cathartic targets.} Our clients tended to engage with AI as cathartic targets. C18 used a metaphor to liken of the conversational agent to a nonhuman entity~(\qt{a talking doll}) that suddenly entered your life and talks to you. The dehumanization process can help clients express more genuine thoughts when talking to the conversational agent compared to interacting with therapists: \qt{I might try out more interactions with AI. Even though I perceive the AI as a 'talking doll', I found it easier to express aggression towards it. But, I found it challenging to express my aggression with a real therapist~(C18)}.Therefore, AI as cathartic objects can provide a space for venting negative emotions, which has a positive effect on one's mental state, as opposed to ``bottling it up inside''~\cite{breuer2009studies}.

%\textbf{AI serve as communication buffers.} AI can provide communication buffers between therapists and clients. As C9 explained, \qt{it may take me a while to connect with my therapist...But AI feels like it's right there with me every day...Building relationships with people typically takes longer, yet I find it easy to say anything to the conversational agent. Throughout this process, I know my therapist can review our my conversation data in the backend, which helps her understand me better over time}.

%\subsubsection{\textbf{Risks and concerns}} In this section, there are two risks and concerns: \textbf{Clients have overly high expectations of the AI roles.} T1, T3, and T4 noted that their clients have unrealistic expectations of the AI roles, potentially leading them to seek treatment advice. For instance, T1 explained that: \qt{C20 would actively ask the AI for therapeutic information. However, he felt that the AI tended to provide broad, search-engine-like responses that lacked personalization, which left him feeling disappointed}. T3 expressed her concerns: \qt{Some clients perceived AI as a therapist, seeking advice or even asking it to interpret their paintings actively, but there were still many situations that might be misleading}.】

%\textbf{Concerns regarding transference between AI agents and clients.} T2 and T5 expressed concerns regarding the potential for transference between AI and the clients. For example, T5 explained that: \qt{AI might serve as an object for transference, helping me see how the client's real-life interpersonal relationships manifest. For instance, you might notice C9' transition from perceiving AI as unsafe and indifferent to eventually confiding all her inner thoughts to it}. T2 expressed her concerns: \qt{While transference isn't necessarily harmful, we need to remain mindful of the emotional dynamics between the AI and client to prevent unhealthy dependence}.




% 7.3.4 risks and concerns
%1.用户对ai对话角色又过高的期待,寻求建议。
% 2.人类对ai释放攻击性是否对人有潜在危害(找到对应理论)
%3. 移情 - AI与来访者关系是否存在移情?怎么去看待这种移情
