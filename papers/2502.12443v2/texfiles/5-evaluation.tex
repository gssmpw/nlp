\begin{table*}[tb]
\centering
\caption{Demographics of Participant Clients: Previous Art Therapy Sessions indicates the number of times the client has previously participated in art therapy; Familiarity with Traditional Drawing reflects the client's level of experience with traditional drawing techniques (0-not familiar; 1-very familiar); Familiarity with Digital Drawing reflects the client's level of experience with digital drawing techniques (0-not familiar; 1-very familiar); Participation Purposes reflects the reasons clients choose to engage in the activity.}
\vspace{-3mm}
\label{tab:clients}
\small
\resizebox{1\linewidth}{!}{
\begin{tabular}{cccccccccc}
\toprule
\textbf{ID} & \textbf{Gender} & \textbf{Age} & \textbf{Education} & \textbf{Region} & \parbox[t]{2.5cm}{\centering\textbf{Previous Art Therapy Sessions}} & \parbox[t]{3cm}{\centering\textbf{Familiarity with Traditional Drawing}} & \parbox[t]{2cm}{\centering\textbf{Familiarity with Digital Drawing}} & \parbox[t]{2cm}{\centering\textbf{Therapist Assignment}} & \parbox[t]{2.5cm}{\centering\textbf{Participation Purposes}} \\
\midrule
C1  & Female & 37  & Bachelor's & China/Shanghai & 0                            & 1                                   & 0.25  &T3 & Personal Growth                   \\
C2  & Female & 35  & Bachelor's & China/Shenzhen & 3                            & 0.5                                   & 0.5   &T3 & Career Development and Family                 \\
C3  & Female & 28  & Master's   & China/Hebei    & 2                            & 0.75                                  & 0.75   &T3  & Family and Emotional Management                \\
C4  & Female & 36  & Bachelor's & China/Beijing  & 10                           & 0.75                                   & 0   &T3  &Career Development                \\
C5  & Male   & 28  & Master's   & Germany       & 0                            & 1                                   & 0.75   &T3   &  Emotional Management and Personal Growth                       \\
C6  & Other  & 26  & Associate's & China/Heilongjiang & 1                            & 0.5                                   & 0.25  &T5  & Emotional Exploration and Intimate Relationships                           \\
C7  & Female & 23  & Master's   & China/Shanghai & 0                            & 1                                   & 1     &T5     &  Intimate Relationships                    \\
C8  & Female & 20  & Bachelor's & China/Shenzhen & 0                            & 0.5                                   & 0.5    &T5   &  Emotional Management and Intimate Relationships                       \\
C9  & Female & 25  & Bachelor's & China/Guangxi  & 4                            & 0                                   & 0.5    &T5    &  Self-Expression and Emotional Exploration                      \\
C10 & Male   & 23  & Master's   & China/Shenzhen & 0                            & 0.75                                   & 0.5   &T5   &             Self-Expression and Social Skills             \\
C11 & Female & 26  & Master's   & China/Hangzhou & 0                            & 0.5                                   & 0.25    &T4  &        Emotional Management, Social Skills and Intimate Relationships                 \\
C12 & Female & 26  & Master's   & China/Shanghai & 2                            & 0.75                                   & 0.5    &T4   &                   Stress Relieving and Intimate Relationships  \\
C13 & Female & 30  & Master's   & China/Dalian   & 0                            & 0.5                                   & 0.25   &T4    &             Family and Emotional Management            \\
C14 & Female & 19  & Bachelor's & China/Chongqing & 0                            & 0.25                                   & 0.25   &T4  &                Personal Growth and Self-Exploration           \\
C15 & Male   & 27  & Bachelor's & China/Beijing  & 0                            & 0.25                                  & 0.25   &T4    &                 Stress Relieving and Personal Growth        \\
C16 & Female & 22  & Bachelor's & China/Shandong & 0                            & 0.5                                   & 0.25   &T1     &              Emotional Management and Social Skills       \\
C17 & Male   & 38  & Master's   & China/Sichuan  & 0                            & 0.75                                   & 0.75   &T1     &                    Personal Growth      \\
C18 & Female & 40  & Master's   & China/Beijing  & 20                           & 1                                   & 0.75    &T1      &               Stress Relieving and Emotional Management          \\
C19 & Female & 28  & Bachelor's & China/Guangzhou & 0                            & 0.5                                   & 0   &T1       &                 Future Career Planning and Personal Growth      \\
C20 & Male   & 25  & Master's   & China/Guangzhou & 0                            & 1                                   & 1   &T1        &                    Academic Pressure Relieving   \\
C21 & Male   & 24  & Master's   & China/Hubei    & 0                            & 0                                   & 0   &T2        &                Childhood Family and Dreams Exploration  \\
C22 & Female & 24  & Master's   & China/Shenzhen & 0                            & 0.25                                   & 0.25    &T2  &                Emotional Management and Personal Growth     \\
C23 & Male   & 25  & Master's   & China/Zhejiang & 10                           & 0.5                                   & 0.5    &T2   &                  Emotional Development and Self-Expression        \\
C24 & Male & 55  & Bachelor's & Dubai& 0 & 0.5& 0.5&T2 &                           Emotional Management \\
\bottomrule

\end{tabular}}
\Description{The table 2 describes 24 participants in art therapy sessions. The participants are from diverse locations, including China (Shanghai, Shenzhen, Hebei, Beijing, Heilongjiang, Guangxi, Hangzhou, Chongqing, Shandong, Sichuan, Hubei, and Zhejiang), Germany, and Dubai. The ages range from 19 to 55 years old, with varying levels of education from associate degrees to master's degrees and bachelor's degrees. Their familiarity with traditional drawing techniques ranges from no familiarity to very familiar, while their familiarity with digital drawing techniques also varies across the spectrum. The participants have attended between 0 and 20 previous art therapy sessions and are assigned to different therapists identified by codes T1 to T5.Participation Purposes reflects the reasons clients choose to engage in the activity}
\end{table*}

\section{Field study}
Using \name{} as both a novel system to study and a research tool to study with, we aim to explore how a human-AI system support clients' art therapy homework in their daily settings (\textbf{RQ1}) and how such a system could mediate therapist-client collaboration surrounding art therapy homework (\textbf{RQ2}). To this end, we conducted a field deployment involving 24 recruited clients and five therapists over the course of one month.



%参与者与实验的setup
    %参与者招募
        % 我们招募的途径:To recruit our clients, we distributed digital recruitment flyers through social media platforms.
        % 海报上描述了什么:The recruitment flyer described the art therapy activities as "promoting self-exploration using a digital software".
        % 我首先要求参与者填写pre-问卷,这个问卷主要包括descriptions of the art therapy activities, demographic information, the number of art therapy sessions they attended, familiarity with digital drawing, and specific needs for the art therapy activities.
        % Participants were included in this study with the aim of reducing stress and anxiety, fostering personal growth, improving emotional regulation, and strengthening social skills.
        % 此外,we tried to selection of participants based on their regions, occupations, the types of devices they used, and the number of times they participated in art therapy.
        % finally, 有27名参与者开始使用这个系统,其中有3名参与者drop out因为缺乏时间
\subsection{Participants and Study Procedure}
\subsubsection{Participants}

The five therapists who participated in the field evaluation were the same ones from our contextual study (see \autoref{tab:expert}). Each therapist was compensated at their regular hourly rate.
For client recruitment, we distributed digital flyers through social media platforms, describing the art therapy activities as an "online art therapy experience promoting self-exploration using a digital software." This aligns with the common goal of art therapy sessions, which are widely used to promote self-exploration for all clients, beyond treating mental illness~\cite{kahn1999art, riley2003family}.

Participants first completed a pre-questionnaire, which provided an overview of the activities and collected demographics, and prior experiences with art therapy experience and with digital drawing---to ensure that we include both novices and experienced user---and their personal goals for participation. 
The therapists guided the recruitment and screening of the the clients, and included individuals seeking for reducing stress, fostering personal growth, enhancing emotional regulation, and strengthening social skills. The therapists excluded individuals with serious mental health conditions to minimize ethical risks.
%Based on the therapists' advice, clients with goals such as reducing stress and anxiety, fostering personal growth, enhancing emotional regulation, and strengthening social skills were included, avoiding ethical concerns related to clinically diagnosed mental health conditions. 
%We also considered participants' regions, device types, drawing familiarity, and prior art therapy experience to create a balanced selection.

In total, 27 clients began using \name{}, but 3 withdrew early due to scheduling conflicts. The final group of 24 clients (C1-C24; 8 self-identified males, 15 self-identified females, 1 identifying as other; aged 19-55) completed the study (client demographics are detailed in the~\autoref{tab:clients}). Clients who completed the full process were compensated with \$37, others were compensated with a prorated fee.
Our study protocol was approved by the institutional research ethics board, and all participant names in this paper have been changed to pseudonyms. Participants reviewed and signed informed consent forms before taking part, acknowledging their understanding of the study.

% The five therapists participated in the field evaluation were the ones who also participated in our contextual study (see \autoref{tab:expert}).
% Five art therapists were compensated with their regular hourly rate.
% For the clients recruitment, we distributed digital recruitment flyers through social media platforms. 
% The recruitment flyer described the art therapy activities as ``online art therapy experience promoting self-exploration using a digital software''.
% This is due to that this is a common goal for art therapy sessions, since Art therapy activities are not only effective in treating mental illness but also widely promote self-exploration for every clients, as commonly integrated into practice~\cite{kahn1999art,riley2003family}.
% First, participants completed a pre-questionnaire that provided an overview of the art therapy activities and gathered details such as their demographics, the number of art therapy sessions they've attended, familiarity with digital drawing, and any specific needs they hoped to address.
% Following that, based on the advices from the therapists, clients were included with the goal of reducing stress and anxiety, fostering personal growth, enhancing emotional regulation, and strengthening social skills.
% The therapists suggest so since they agree that these therapeutic goals would be beneficial for eavery day therapy clients and would could It might avoid the potential ethical and safety risks associated with clinically diagnosed mental health issue.
% Further, we selected participants based on a balance of their regions, the types of devices they used, the familiarity with drawing and their prior experience with art therapy. 

% In total, 27 clients began using \name{}, but 3 withdrew from the study at the early stage due to scheduling conflicts.
% Finally, 24 clients (C1-C24; 8 self-identified males, 15 self-identified females, 1 identifying as other; aged 19-55) completed our field study. 
% APPENDIX shows the specific client demographics.
% We compensated clients based on their level of involvement, with those who completed the full one-month study receiving 200 RMB as a bonus, and clients who dropped out receiving a prorated fee according to the duration of their participation.

% Our protocol was approved by the institutional research ethics board, and all names in this paper have been changed to pseudonyms.
% Also, before participating in the activity, participants carefully reviewed and signed the informed consent form, acknowledging their understanding.

%在与治疗师协商讨论下,这些用户被分到5位治疗师(see Table),其中T2有4位来访者,其余治疗师有5位来访者。
%这个研究. .
%在活动开始前,我们邀请每位参与者开展了一场介绍session. 主要是目的是介绍活动目的与流程,并且演示如何使用\name{},并且为每位来访者可以接触到系统的URL的链接;
%介绍活动结束后,来访者被鼓励有规律地去自行探索使用\name{};
%每隔一周,我们会安排治疗师与来访者进行线上一对一的session。我们会鼓励治疗师在线上一对一session之前提前review来访者的使用数据,并通过即时通讯软件与我们交流review之后的洞见与想法。
%在线上一对一session时,在不干扰治疗师艺术治疗实践的基础上,我们鼓励治疗师在线上一对一session时利用这些数据。在艺术创作阶段,来访者可以通过分享屏幕的方式使用系统的第一个阶段进行创作并与治疗师进行讨论交流,在session快结束前治疗师会给来访者推荐家庭作业。
%在session结束后,治疗师会在治疗师系统上安排家庭作业并给予来访者的个人赠言。此外,来访者在结束线上session后可以按照治疗师的推荐完成家庭作业或者自行探索使用系统。
\subsubsection{Procedures}

Clients were distributed in coordination with the five therapists, as shown in \autoref{tab:expert}. T2 was assigned four clients, while the other therapists each had five clients. The field study consisted of two main activities: (1) three online in-session activities, where clients had one-on-one conversations and collaborated with the therapist, and (2) unstructured between-session activities, where clients practiced therapy homework using \name{} following the therapist’s recommendations.
Before the study, we held online introductory sessions to familiarize the clients with \name{}, and provided both demonstrations and hands-on exploration on their preferred devices. Similarly, we offered online training for therapists on customizing and reviewing homework, while allowing them to explore both the therapist-facing and client-facing applications. After the session, clients were encouraged to regularly explore \name{}.
Two weeks into the study, we scheduled weekly one-on-one online sessions between therapists and clients, each lasting approximately 60 minutes. Therapists were encouraged to review the clients' homework history using \autoref{fig:ui}(c) before each session. During the online session, therapists used this data to inform their practices without interrupting the flow of therapy. We encouraged clients in advance, to create artworks during the Art-making Phase~(\autoref{fig:qual_results}(a)), sharing screens and discussing their creations with the therapist, but did not interfere with the therapeutic process.

%Clients also used \autoref{fig:qual_results}(a) to create artwork, sharing their screens and discussing their creations with the therapist.

At the end of each session, therapists recommended homework tasks based on insights gained during the conversation. After the session, therapists might customize homework agents, including customizing conversational principles, assigning homework tasks, and providing personal messages through \autoref{fig:ui}~(d). Clients could then either complete the assigned homework or engage in self-exploration using \name{} between sessions.

% Clients were distributed In coordination with the five therapists, as shown by \autoref{tab:clients}: T2 was assigned with four clients, while each of the other therapists was assigned with five clients.
% The procedure for the field study consisted of two activities: (1) three online in-session activities where they have one-on-one conversation and collaboration with the therapist and (2) unstructured between-session activities where they perform therapy homework practices either upon recommendations of usage from the therapist or volunteerily use it in their daily lives.
% Before the study, we conducted an introductory session for each client to explain the activities, demonstrate how to use \name{}, and provide access to \name{} via a URL on their preferred devices.
% After the introductory session, the clients were encouraged to explore the use of \name{} on a regular basis.

% After two weeks of self-exploration, we started scheduling weekly one one-on-one online sessions between the therapists and the clients.
% Therapists were encouraged to review clients' homework history using \autoref{fig:ui}~(c) before the online session.
% During the online one-on-one session, we encouraged therapists to use these history data without interfering with their art therapy practices. 
% Also, they would utilize \autoref{fig:ui}~(a) to create their artwork by sharing their screens and discussing their artworks with therapists. 
% Before the end of the session, the therapist would recommend the homework tasks for the client based on the insights gained from the one-on-one session.
% After the online session ends, therapists would customize homework agents, including modifying or updating the conversational agent principles, assigning homework tasks and providing therapist's messages to the client through \autoref{fig:system}~(d). 
% Correspondingly, clients could either complete the homework or engage in self-exploration using \name{} between sessions.

% 对于异步session场景数据收集下,所有来访者使用系统的图像以及对话记录等日志数据以及治疗师在治疗师系统中使用定制功能的日志数据在保存在数据库中。
% 此外,我们鼓励来访者和治疗师通过即时通讯软件发送给我们images以及comments关于使用系统的实践以及感受。
% 对于线上session的场景数据收集,首先,online sessions were audio- and video-recorded.
% 此外,at the end of each online session, we conducted a 5-minute interview with therapists, mainly to collect their practices and experiences about the session.
% Upon concluding all the sessions,我们与治疗师以及来访者开展了约为30分钟的semi-structured interview to 探索ai agents如何支持艺术治疗场景的家庭作业(RQ1)以及AI agents如何mediate 治疗师与来访者合作(RQ2). We used 治疗师与来访者在 the trial period使用系统的log 数据以及他们的反馈作为stimuli 去问特定的使用实践的问题。
% With participants' consent, we recorded the interviews and transcribed them for thematic analysis.
% First, two researchers conducted collaborative inductive coding. They initially annotated the transcript to identify relevant quotes, key concepts, and recurring patterns in the data. These findings were further developed through regular discussions, leading to a detailed coding scheme aligned with the research questions. Quotes were then coded and clustered into a hierarchy of emerging themes, continually reviewed, and refined in recurrent meetings, where exemplar quotes were also selected for presenting each theme and sub-theme. 
% Also, we collected the log data from 治疗师和来访者 作为证据以及examples for the thematic analysis results.

\subsection{Data Gathering Methods} 

For between-sessions, we stored all homework-related data in a database, including artwork, dialogue, usage logs, as well as information on homework customization such as conversational principles, tasks, and personal messages.
We encouraged participants to use personal messaging (WeChat) to share pictures and comments about on-the-spot experience and feelings after homework with \name{} to compensate for semi-structured interviews.
During online sessions, we recorded audio and video. 
The researchers did not observe the therapy session in live, but reviewed post hoc, as the therapists believed a third party's presence could affect a client's emotional expression and the therapist-client dynamic.
After each session, we conducted a brief 5-minute interview with the therapists to gather their insights and feelings.

Upon the completion of the final one-on-one sessions, we conducted 30-minute semi-structured interviews with both therapists and clients. These interviews aimed to explore how \name{} supported art therapy homework in clients' daily lives (\textbf{RQ1}) and how therapists and clients collaborated surrounding art therapy homework (\textbf{RQ2}). We used feedback and homework outcomes from the trial period to ask targeted questions about their practices.
With participants' consent, we recorded and transcribed the brief 5-minute interviews and the 30-minute interviews for thematic analysis~\cite{braun2006using}. This analysis also included the personal messages shared by the participants about their on-the-spot experiences.
%we recorded and transcribed the interviews for thematic analysis. 
Two researchers then engaged in inductive coding, annotating transcripts to identify relevant quotes, key concepts, and patterns. They developed a detailed coding scheme through regular discussions, grouping quotes into a hierarchical structure of themes and sub-themes. Exemplar quotes were selected to represent each theme. We also used homework history (e.g., images or conversation data) and customization data (e.g., homework dialogue principle data) as evidences or examples to back up the findings in our thematic analysis.



% In between sessions, all homework history data~(e.g., artwork, creative process data and dialogue data) and history data on homework customization~(e.g., principles of conversational agents, homework tasks and personal messages) were stored in the database.
% In addition, we encouraged clients and therapists to send us images and comments about their experiences and feelings when using \name{} via an instant messaging app.
% For online in-sessions, the sessions were first audio- and video-recorded.
% At the end of each in-session, we conducted a brief 5-minute interview with the therapists to gather insights into their practices and feelings during the session.
% Upon concluding all the sessions, we conducted approximately 30-minute semi-structured interviews with both the therapists and the clients to explore how \name{} support art therapy homework in clients' daily settings~(\textbf{RQ1}), and how therapists tailored the homework and tracked the homework history surrounding art therapy homework~(\textbf{RQ2}). 
% Further, we employed the homework outcomes and feedback from both therapists and clients during the trial period as stimuli to ask specific questions about their practices. 

% With participants' consent, we recorded the interviews and transcribed them for thematic analysis~\cite{braun2006using}.
% Initially, two researchers engaged in collaborative inductive coding. They began by annotating the transcript to highlight relevant quotes, key concepts, and recurring patterns in the data. Through regular discussions, they expanded these insights into a detailed coding scheme that aligned with their research questions. The quotes were then systematically coded and grouped into a hierarchical structure of emerging themes, which were continuously reviewed and refined during recurring meetings. During these discussions, exemplar quotes were also chosen to represent each theme and sub-theme.
% We also gathered homework history and customization data, including artworks and conversation records from both therapists and clients, as evidence and examples to support the results of the thematic analysis.

\begin{figure*}[tb]
  \centering
  \includegraphics[width=\linewidth]{images/findings_1.png}
  \vspace{-7mm}
  \caption{Overview of The Homework Engagement of Clients with \name{}: (a) Homework Activity Date Distribution; (b) Accumulated Homework Activity Hourly Distribution of the Day; (c) Usage of AI Brushes in Artworks; 
  }
  \Description{Figure 5 contains three sub-figures. Figure 5a shows the Homework Activity Date Distribution for 24 clients over a four-week period, using seven different shades of purple to represent varying levels of participation in the homework sessions. Figure 5b illustrates the frequency of AI brush usage during clients' homework art-making, with the top 20 most frequently used brushes highlighted in larger font. Figure 5c depicts the distribution of homework sessions across different times of the day, revealing that clients tend to engage in homework sessions more frequently in the afternoon and evening.}
  \label{fig:quan_results}
\end{figure*}



