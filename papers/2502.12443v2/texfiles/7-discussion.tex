\section{Discussion}
%====================================
Art therapy homework often integrates art-making with verbal expression to explore deep thoughts and uncover new meanings~\cite{hoshino2011narrative,smith2019visual}.
However, the threshold for both accessing and creating meaningful therapy homework is high~\cite{du2024deepthink,Oewel_2024}. 
Further, promoting deep self-reflection during homework can be challenging without the guidance of therapists~\cite{kim2024mindfuldiary}.
Furthermore, another challenge is finding ways to help therapists customize therapy homework and assist them in reviewing and synthesizing homework data during asynchronous collaboration~\cite{Oewel_2024, freeman2007use}.
We thereby set out to explore how to support clients completing art therapy homework in their daily settings~(\textbf{RQ1}) and how to facilitate therapists tailoring and reviewing homework history through asynchronous therapist-client collaboration~(\textbf{RQ2}).

To address this, we have designed and developed \name{}, a human-AI system that integrates co-creative art-making and conversational agents to assist clients in completing art therapy homework. 
Simultaneously, it supports therapists in tailoring the homework agents and reviewing AI-compiled homework history. 
To evaluate \name{}, we conducted a one-month field deployment during which 24 clients used \name{} in their everyday settings, with guidance provided by 5 therapists. 
Addressing RQ1, our quantitative findings revealed that clients used a variety of brushes to engage with their therapy homework over time. Also, 
Our system allowed clients to reflect their emotions and experiences anytime, anywhere in their daily lives, which can offer convenient homework options and lower the threshold of creations.

Prior research has explored how human-AI co-creative art-making within art therapy practices can enhance clients' expressivity and creativity in both synchronous and asynchronous therapy settings~\cite{du2024deepthink}. Further, 
Liu et al. explored combining generative AI with traditional art materials in family expressive arts therapy, where family members guided the process through art and verbal expression, using their creations as inputs for AI-generated images turned into physical materials for storytelling. Extending this line of research, our results found that clients might verbalize emotional feelings and personal experiences during art-making. At the same time, the multi-round human-AI dialogue might can help clients' self-exploration and prompt their interpretation triggered by the unexpected details from the AI-generated images.

In response to \textbf{RQ2}, prior research has highlighted opportunities for AI agents to facilitate asynchronous practitioner-client collaboration in healthcare~\cite{yang2024talk2care}. Conversational agents gathered client health data, which is then summarized and delivered by AI agents to practitioners' interface~\cite{kim2024mindfuldiary}. 
In this line of research, our AI-compiling of homework history, including co-creative artwork, can serve as valuable intervention resources for therapists, helping to encourage clients in transforming and enhancing their positive feelings. 
Meanwhile, inspired by customization of AI agents in HCI~\cite{ha2024clochat,hedderich2024piece}, our findings indicate that therapists incorporate their professional beliefs into conversational agents by modifying conversation principles and example questions to provide structured guidance. Further, therapists infused their personal touch and emotional support into the homework agents through their personal messages.

\subsection{Design Implications}
Below we generalize and discuss a set of design implications to inform future HCI research:

%提纲:
    %Implication 1: Combining Conversation and Art-making to Support Therapy Homework via AI Agents
        % 首先,我们发现与AI agent的对话可以有效的帮助来访者外化作品背后的感受与体验;
        % 其次,这些human-AI co-creative 意料之外的outcomes可以通过对话AI agent的支持下作为自我探索的资源。
        % 与此同时,这些对话记录数据以及图像数据可以作为治疗师进一步干预的材料与资源。
        % 尽管很少研究在人机交互研究中探索design整合交流与艺术创作的系统,但是我们研究表现出一定的潜力来支持家庭作业,因此在未来我们建议可以采用这种交互范式更好地支持家庭作业。
        %on the other hand, 在其他心理健康干预的场景下,我们可以思考如何通过AI整合对话以及艺术创作的交互方式来设计self-reflection or journaling technologies~\cite{引用}. 
        % 例如,通过AI agent结构化问题引导来访更深层次的自我披露与反思~\cite{引用};
        %同时,融合human-ai co-creative方法的艺术创作从非语言的途径传达来访者感情与和潜意识探索~\cite{引用}。
        %同时这些语言和非语言的数据记录通过AI方式更好地设计总结出来可以作为治疗师参考的资源应用于临床实践中~\cite{引用}。
% 修改版提纲
    % Implication 1: Combining Multimodal Interaction to Support Therapy Homework via Human-AI Collaboration
        %第一段:
        % Prior study has demostrated that drawing while speaking can help innovation, thinking, and communication in various creative and spontaneous activities~\cite{Rosenberg2024talking, fan2023drawing}.
        % In art therapy homework, we found that combining art-making and verbalization  can effectively help clients externalize the feelings and experiences behind their artwork;
        %Second, we found that verbalization can encourage open-ended self-exploration and help clients discover new meanings behind the artwork.
        % These insights could have broader therapy homework applications beyond art therapy, extending to fields like CBT and rehabilitation counseling, through an AI-integrated dialogue and drawing system.
        % Furthermore, future design could integrate original art-making with AI-generated multimedia content (such as animation~\cite{stark2024animation}, 3D models~\cite{li2023generative}, video game~\cite{ratican2024adaptive}, video~\cite{zhou2024survey}, etc.), while leveraging multimodal conversational data (such as text, voice, video calls, etc.) to enhance self-expression and emotional externalization throughout the creative process.
        %第二段:
        % 除此之外,我们参与者也表达了挑战关于AI对话代理无法捕捉和理解画面生成的细节内容,造成来访者的沮丧。
        %治疗师指出增加AI对话代理与AI创作画作细节的关联,可以增加来访者的内心表达。
        %因此,未来工作可以建议设计师与开发人员在设计Human-AI 家庭作业更深入地探索对话代理和AI创作之间的关系和映射。
        %例如,结合Vision-Language Models与图像分割技术可以增强AI对图像细节的理解与互动。
    %Implication 2:  Supporting Customizing Homework Agents as Practitioners' Extension
        %In the practice of psychotherapy homework, therapists design and assign homework based on their theoretical orientation (such as Cognitive Behavioral Therapy, Emotion-Focused Experiential Therapy, etc.) and practical experience, which can help them gather information in a more directive manner~\cite{Harwood2007}.
        %In our study, customizing homework agents can become therapists' assistants with personal touch, which can enhance their professional autonomy throughout the therapeutic process.
        %In particular, we found that customizing AI conversational agents can help therapists incorporate their practical experience and professional beliefs into the dialogue principles and example questions.
        %It can help therapists ask the questions they are inclined to explore and encourage deeper reflection, even in the absence of the therapist.

        % While tailoring conversational principles and example questions of the AI agent is promising, it remains valuable to explore further design opportunities for customizing AI agents for therapy homework in future work.
        %Future research could explore establishing a personalized knowledge base for each therapist~\cite{abbasian2023conversational}, including their professional beliefs and practical experiences. 
        %As an extension of the therapist, the AI agent can continuously empower therapists' practices through retrieval-augmented generation~\cite{kresevic2024optimization}, rather than replace therapists.
        %For example, it is suggested that designing teachable AI agents (similar to an apprentice role)~\cite{chhibber2021towards} can seek guidance and learn from therapistsm which might help AI agents continuously build up a comprehensive knowledge base of therapists.
        %Further, in the homework history between the AI conversational agent and the client, we can design a ``mental activity'' display to explain why a particular response was given on the therapist's interface.
        %The design is similar to the thought process a therapist goes through during an intervention, which would help the therapist understand the interpretability of the AI agent's decisions and assist in making further dynamic adjustments.
    %Implication 3:  Supporting Therapy History Summarization via LLM.
        %Therapists often struggle to easily access and effectively track homework history, resulting in a substantial increase in their workload [22, 71, 83].
        %To address this challenge, we leveraged the capabilities of LLMs to help summarize the interactions between clients and AI agents, including artwork descriptions and dialogue history. 
        %The AI's summarization help therapists gain insight into the client's experiences by summarizing the dialogue history and highlighting recurring brush elements, which can serve as meaningful conversation triggers for one-on-one sessions.
        %Future work could summarize and visualize these co-creative and dialogue history from more dimensions. For example, LLMs could summarize and visualize details in the artwork, such as brush elements, shapes, stroke intensity, and size. 
        %Futher, It is suggested that synthesizing long-term homework history data could provide therapists with valuable insights into clients' personal growth over time~\cite{yang2024talk2care}.
        %For instance, we can collaborate with therapists to identify trends and changes in clients' homework history over different time periods, enabling a comparison with their current progress and status.

%支持边画边说 - 来支持情感外化
%传统的艺术治疗作业,art-makiong + verbalization有治疗意义(没有办法协同,verbalization 和 art-making互动,独白,没有来回的互动对话(独白和对话差异),语言来辅助drawing,没有家庭作业的结构化引导(意义),基于图像中意象来进行交流)
%我们结合human-ai co-creative xxx。解决 什么挑战(scafolding)具体!!!!
%

%提纲:更新于12月8号 - -------
% 在艺术疗法作业中,将艺术创作与言语表达相结合,融合了通过言语表达揭示内心思想和寻找新意义的力量,以及艺术创作过程所具有的创造性和表达性潜力;
% 我们系统结合human-AI co-creative art-making和 conversational agents协同的方式,conversational agent捕捉基于图像中意象来促进客户进行深入多轮回交流对话。
% AI驱动的conversational agent避免了传统家庭作业独白中的单一声音和封闭性,允许不同视角进入对话,提供了构建意义的脚手架(scaffolding),从而提供了更丰富的对话纹理。

\subsubsection{Implication 1: Combining Multimodal Interaction to Support Therapy Homework via Human-AI Collaboration}
In traditional art therapy homework, integrating art-making with verbalization combines the power of uncovering inner thoughts and finding new meaning through verbalization with the creative and expressive potential of the art-making process~\cite{hoshino2011narrative}.
In our study, our system integrates AI-infused art-making with the collaborative use of conversational agents. The conversational agent captures brush objects from the artwork to facilitate in-depth, multi-turn dialogue with clients. 
Compared to the single, closed monologic forms of therapy homework, the conversational agent allows diverse viewpoints to enter the dialogue and creating a richer, more collaborative conversational texture~\cite{pare2004willow}.
Future designs could combine image segmentation~\cite{kirillov2023segment} and vision-language models~\cite{zhang2024vision} for real-time analysis of artistic elements like colors, shapes, and patterns, and the conversational agent could then adjust its guiding questions to align with clients' emotional state or creative goals.

\textit{Remaining Challenges in the Multimodality of AI-infused Therapy Homework}: Our therapist noted that the AI-infused therapy homework cannot fully replicate the sensory experience associated with using physical art-making materials. 
In art therapy, the kinesthetic and sensory components create a multi-dimensional therapeutic experience that expresses inner sensation and encourages self-soothing~\cite{hinz2019expressive}.
The future therapy homework system could incorporate multimodal design by utilizing human-AI co-creative drawing to generate 3D artwork~\cite{liu2024he}. These AI-generated artworks could be physically printed, offering opportunities to enhance motor-sensory skills while providing calming and therapeutic mental effects.



%In art therapy homework, we found that combining art-making and verbalization can effectively help clients externalize the feelings and experiences behind their artwork;Second, we found that verbalization can encourage open-ended self-exploration and help clients discover new meanings behind the artwork.These insights could have broader therapy homework applications beyond art therapy, extending to fields like CBT and rehabilitation counseling, through an AI-integrated dialogue and drawing system.Furthermore, future design could integrate original art-making with AI-generated multimedia content (such as animation~\cite{stark2024animation}, 3D models~\cite{li2023generative}, video game~\cite{ratican2024adaptive}, video~\cite{zhou2024survey}, etc). For example, clients can create artwork using traditional materials and input their artwork into the generative AI to produce multimedia content~\cite{liu2024he}. Meanwhile, leveraging multimodal conversational data (such as text, voice, video calls, etc.) might enhance self-expression and emotional externalization throughout the creative process.
%Prior study has demostrated that drawing while speaking can help innovation, thinking, and communication in creative activities~\cite{Rosenberg2024talking}.In our study, we found that drawing while speaking can effectively help clients externalize the feelings and experiences behind their artwork;Second, we found that conversational agents can encourage open-ended self-exploration and help clients discover new meanings behind the artwork. 
%unexpected outcomes from human-AI co-creation can serve as resources for self-exploration supported by conversations with the AI agent.Meanwhile, the dialogue and co-creative artwork history can serve as meaningful resources for the therapist's ongoing interventions.Therefore, our study demonstrates significant potential in designing human-AI systems integrating conversation and art-making to support therapy homework.In the future, it is suggested that adopting this interactive paradigm could better support therapy homework.On the other hand, in other mental health intervention contexts, we might consider how to design human-AI systems integrating art-making with verbal expression to support journaling~\cite{nepal2024contextual,kim2024mindfuldiary} or emotion self-report~\cite{ghosh2020towards}.For example, we thereby suggested that future research could intentionally support Visual Art Journaling as a therapeutic method~\cite{smith2019visual} through combining human-AI co-creative art-making along with conversational agents.The human-AI co-creative art-making can support clients'  creativity and imagination, and it can also lower the creative threshold for clients~\cite{liu2024he,du2024deepthink}.Further, conversational agents can can facilitate deeper self-disclosure and reflections in users' journaling practices through structured guidance~\cite{kim2024mindfuldiary}.

      
    %这些是草稿
        % agents 定制化变成个人印记的助手。是因为治疗师不同的professional beliefs, 高度个人化的经验和一些实践的原则,对于治疗师有效的发挥自己技能是至关重要的,pressfessional space(自主性),
        %在我们的研究发现,治疗师让agent带有自己的印记,作为自己的延伸,主要是形式,实践的信念变成agent对话的原则和问题,确保不在的情况,更像是她一样提问,同时传递情感的支持的话语,自己本身特定的来访者讲;与此同时这个agent一对一的session延续一些发现的机会和问题,为下一次总结,不再连续性的执行疗愈的agenda,体现出来治疗师确切的去定制化ai agents的需求。
        %深一层的定制化,疗愈的原则的,提问的顺序的仅仅不够的,
        %困难-如何定制之后的agents进行来访者对话记录之后,怎样基于定制化之后,而且怎么检验定制化?讲这句话的时候心理活动讲出来,每句话可解释的,进一步干预,ai agent不光是可以定制的,可teacable的,不仅是定制化,未来是teachable , 持续呀不断指导,更想therapist(当 AI agent 在对话中做出回应时,可以增加其“心理活动”的展示,即说明 agent 为什么做出某种回应。这个过程类似于治疗师在进行干预时的思考过程。这种透明化的心理活动不仅有助于治疗师了解 agent 的决策依据,还能帮助治疗师做出进一步的调整和干预。)
        %希望ai agents的思路,需要未来工作关于治疗师的本身的知识库,自身知识库,自身的信念,可以检索的知识库,作为对于某一个治疗师的定制化agent的工具,延续实践范围。这样的情况下,治疗师因为agents在场和不在场的情况持续不断的帮助很多clients ,进行赋能,而不是取代。(更深层次的定制)如何实现?可能创作对话式界面,虚拟的学徒请教提问治疗师,转化成知识库,用于
%Implication 2: Supporting Customizing AI Agents as Practitioners' Extension
    % Practitioners bring diverse professional beliefs and highly personalized practical experiences to their clinical practice~\cite{haarhoff2015engagement}.
    % Thus, customizing AI agents can become therapists' assistants with a unique mark, which can enhance their professional autonomy throughout the therapeutic process.
    % In our study, 我们发现治疗师通过定制AI agents将自己的实践经验和专业信念融入到agents对话的原则和示例问题上,以确保在没有治疗师情况下可以向来访者提问自己想了解到的问题并同时可以向来访者提供emotional support and encouragement.
    % Also, 治疗师通过定制AI agents支持的家庭作业可以巩固/延续上次艺术治疗sessions的治疗效果,同时也为下一次艺术治疗sessions提供了准备,促进了艺术治疗agenda的连续性。
    %这些发现体现出治疗师定制化AI agents的需求。
    % 虽然定制AI助手的对话原则和提问顺序是一个良好的开端,但这还远远不够。我们建议未来的研究可以进一步探索AI助手的定制化。
    % 首先,未来的工作可以为每位治疗师构建一个个性化的知识库,该知识库可以包含治疗师的个人信念、经验、治疗方法和实践范围等内容。AI助手作为治疗师的延伸,可以通过检索这个知识库,为治疗师提供持续的赋能,而不是取代治疗师的角色。
    % 通过创建teachable AI agents(类似于学徒的角色)的方式可以不断向治疗师请教和学习,从而帮助AI agents逐步建立起治疗师的知识体系。
    % 其次,可以考虑在治疗师端增加AI助手与来访者对话时的“心理活动”展示,即解释AI为什么会做出某种回应。这一过程类似于治疗师在进行干预时的思考过程。通过透明化这一思维过程,不仅能帮助治疗师理解AI助手的决策依据,还能使治疗师更好地做出调整和进一步的干预。
    % 因此,我们建议未来的工作可以为治疗师设计teachable agents从而获得治疗师不断的指导,变成治疗师的延伸


%=======
%理论出发 - 结构化指导挑战
% 治疗师定制(1对多,其次就是灵活性提问与引导,在做家庭作业的时候传递情绪支持,治疗师的分身影响来访者表达)

\subsubsection{Implication 2: Supporting Customizing Well-structured AI Agents as Practitioners' Extension} 
Designing well-structured therapy homework instructions can indeed be challenging for therapists, as they should balance guidance and personal expression to avoid causing confusion.~\cite{Oewel_2024,Harwood2007}. 
Our system allows the therapists to customize the AI agent's conversational principles in one-to-many client therapy homework, enabling it to act as an therapist's extension. T
his provides flexible guidance and questioning, and real-time emotional support during doing therapy homework.
Future work can focus on developing AI agents that build therapist corpora and knowledge bases to continuously empower therapists' practice rather than replace them. 
For example, designing teachable AI agents (similar to apprentice roles~\cite{chhibber2021towards}), learning therapists' language styles and guidance skills to build robust corpora and knowledge bases.

\textit{Risks and Cautions Regarding for Therapist-Customized Therapy Homework Agents:} We found that that as an extension of the therapists, clients may develop overly high expectations of the AI, such as seeking treatment advice from it. 
It may lead to reduced vigilance and overreliance, potentially causing patients to blindly accept incorrect advice~\cite{mendel2024advice}.
Future designs could incorporate a collaborative mechanism between AI agents and therapists. When users actively seek medical advice from the AI agents, it could guide them to connect with a licensed art therapist for further consultation.
We also recommend conducting pre-AI deployment training when deploying the AI-infused art therapy homework system.

%In our study, customizing homework agents can become therapists' assistants with personal touch, which can enhance their professional autonomy throughout the therapeutic process.
%In particular, customizing AI conversational agents can help therapists incorporate their practical experience and professional beliefs into the dialogue principles and example questions. 
%It can help therapists ask the questions they are inclined to explore and encourage deeper reflection, even in the absence of the therapist.}

%\textcolor{blue}{It remains valuable to explore further design opportunities for customizing AI agents for therapy homework in future work.
%Future research could explore establishing a personalized knowledge base for each therapist~\cite{abbasian2023conversational}, including their professional beliefs and practical experiences. 
%As an extension of the therapist, the AI agent can continuously empower therapists' practices through retrieval-augmented generation~\cite{kresevic2024optimization}, rather than replace therapists.
%For example, it is suggested that designing teachable AI agents (similar to an apprentice role)~\cite{chhibber2021towards} can seek guidance and learn from therapists which might help AI agents continuously build up a comprehensive knowledge base of therapists.
%Further, in the homework history between the AI conversational agent and the client, we can design a ``mental activity'' display to explain why a particular response was given on the therapist's interface.
%The design is similar to the thought process a therapist goes through during an intervention, which would help the therapist understand the interpretability of the AI agent's decisions and assist in making further dynamic adjustments.}




%Practitioners bring diverse professional beliefs and highly personalized practical experiences to their clinical practices~\cite{haarhoff2015engagement}.Thus, customizing AI agents can become therapists' assistants with personal touch, which can enhance their professional autonomy throughout the therapeutic process.In our study, we found that customizing AI agents can help therapists incorporate their practical experience and professional beliefs into the dialogue principles and example questions.It can help therapist ask the questions that they tend to explore and providing emotional support and encouragement to clients, even in the absence of the therapist.Also, customizing therapy homework supported by \name{} can reinforce and extend the therapeutic effects of previous art therapy sessions, and also help clients prepare for the next session.Thus, these findings highlighted the need for therapists to customize the agents for clients' therapy homework.While tailoring conversational principles and example questions of the AI agent is promising, it remains valuable to explore further design opportunities for customizing AI agents for therapy homework in future work.Future research could explore establishing a personalized knowledge base for each therapist~\cite{abbasian2023conversational}, including their professional beliefs and practical experiences. As an extension of the therapist, the AI agent can continuously empower therapists' practices through retrieval-augmented generation~\cite{kresevic2024optimization}, rather than replace therapists.For example, it is suggested that designing teachable AI agents (similar to an apprentice role)~\cite{chhibber2021towards} can seek guidance and learn from therapists. This approach might help AI agents continuously build up a comprehensive knowledge base of therapists.To enhance therapist-AI alignment, we can implement a "mental activity" display for the AI agent on the therapist's interface during conversations with the client. This display would provide explanations for the AI's responses, offering insight into the reasoning behind its suggestions. By increasing transparency, therapists can better understand the AI agents' decision-making processes and make more informed adjustments based on their professional beliefs.In summary, we recommend that future work focus on designing teachable agents for therapists, enabling continuous guidance and making the agents an extension of them.

%============

    %Implication 3: 
        % 数据量很大,大量的数据,自己去检索很困难,初步的总结,归纳与提示更快材料的更关键的点,ai 编译的历史数据有意义,在一对一的聆听治疗师与来访者一对一的对话,实时在线上一对一把历史的数据practical的数据推荐给治疗师,
        %疗愈之外,治疗师看日志报告,可以治疗师与ai对话方式进行查看
  
%Implication 3:Supporting Therapy Homework Data Summarization via LLM
    % 在我们实验过程中,我们产生了很多大量的家庭作业数据,同时这些数据可以帮助治疗师理解和synthesize~\cite{}.
    % 然而检索以及利用这些家庭数据对于治疗师来讲是困难的。
    % 在我们研究中,我们发现大语言模型可以归纳总结这些家庭数据的关键点作为治疗师线上sessions的实践资源,帮助来访者识别并建立他们的优势。
    % 因为AI编译的家庭作业数据在艺术治疗实践中表现出潜力
    % 在未来的工作,我们建议AI agents可以在一对一session中聆听治疗师与来访者对话,并整合历史数据实时将归纳总结的洞见作为一种资源推荐给治疗师。
    % 此外,未来工作还可以将长期的历史数据进行总结归纳,从而帮助治疗师在来访者随着时间上个人的变化的提供了洞见:“AI可能根据半年或者一年的数据进行一个总结,从而能够帮助我看一下这一路的成长是什么样子的(T4)”
%%%%%
%12.9 outline: 挑战是家庭作业很难追踪
% Therapists often struggle to easily access and adequately track homework history, resulting in a substantial increase in their workload [22, 71, 83].
% 为了应对这一挑战,我们利用 LLM 的功能来帮助总结客户和 AI 代理之间的互动,包括艺术品描述和对话记录历史.
% 这种人工智能的总结功能可以快速帮助治疗师深入了解客户的经历并强调重复出现的笔触元素。
% 未来的工作可以从更多维度总结这些共同创作和对话历史。
% 例如,我们可以结合人工智能姿势识别(例如 OpenPose)和面部表情分析(例如 Affectiva)来捕捉治疗作业期间详细的姿势和表情变化。这些数据与 LLM 相结合,可以实现手势、对话和艺术作品的交叉分析。
%此外,我们可以综合长期家庭作业历史数据为治疗师提供有关客户随时间推移的见解~\cite{yang2024talk2care}。
%例如,我们可以与治疗师合作,确定客户在不同时间段内家庭作业历史的趋势和变化,从而与他们当前的进度和状态进行比较。

%我们的治疗师对人工智能总结的可靠性表示担忧,因为偶尔会出现不准确或过度猜测的情况。
%因此,治疗师希望通过纳入更强大的理论支持来提高人工智能总结的准确性和可靠性。
%建议将艺术治疗和治疗作业理论知识库整合到人工智能的总结中,使治疗师能够选择特定的理论模型来生成总结。
%为了增加AI summary的透明性和可解释性,我们可以引入给出correctness likelihood strategies~\cite{ma2023should}并增加AI summary的理论来源的解释,从而帮助治疗师进行决策判断。


\subsubsection{Implication 3: Supporting Longitudinal Multidimensional AI Summary to Simplify Therapists' Homework Tracking.}
Therapists often struggle to easily access and adequately track therapy homework history, resulting in a substantial increase in their workload.
To address this challenge, we leveraged the capabilities of LLMs to help summarize the interactions between clients and AI agents, including artwork descriptions and homework dialogue history.
The AI's summarization can quickly help the therapists gain insight into the client's experiences and highlight recurring brush elements.
Future work could summarize these co-creative and homework dialogue history from more dimensions. 
For example, we could incorporate AI pose recognition (e.g., OpenPose) and facial expression analysis (e.g., Affectiva) to capture detailed posture and expression changes during therapy homework. 
The data, combined with LLMs, could enable cross-analysis of gestures, dialogues, and the artworks.
Further, it is suggested that synthesizing long-term homework history data could provide therapists with valuable insights into clients' personal growth over time~\cite{yang2024talk2care}. 
For instance, we can collaborate with therapists to identify trends and changes in clients' homework history over different time periods, enabling a comparison with their current progress and status.

\textit{Risks and Concerns of AI Summary:} 
Our therapists expressed concerns about the AI summary's reliability, due to occasional inaccuracies or over-speculation. 
Therefore, the therapists hoped to enhance the accuracy and reliability of the AI summary by incorporating more robust theoretical support. 
It is recommended to integrate a knowledge base of art therapy and therapy homework theories into the AI's summarization, enabling therapists to select specific theoretical models for generating summaries.
To enhance the transparency and interpretability of the AI summary, we can incorporate correctness likelihood strategies~\cite{ma2023should} and provide explanations of the theoretical foundations behind the summary, aiding therapists in making informed decisions.


%During our filed deployment, our clients created a substantial amount of therapy homework data, which can help therapists understand and synthesize the information~\cite{freeman2007use}. However, retrieving and utilizing this homework data can be challenging for therapists.
%In our results, we found that LLM can effectively summarize the key points from therapy homework data, which can help therapists deep understanding of the clients' past experiences and characteristics.
%Future work could involve synthesizing long-term homework history data to provide therapists with insights about clients' personal changes over the long term~\cite{yang2024talk2care}. 
%For art therapy homework, T4 explained that the importance of this long-term history data to them, which can help them understand their growth: \qt{AI might analyze homework data spanning six months to a year and generate a summary. So, you can observe changes in the frequency of brush usage over a long period, e.g., clients have been consistently using \textit{Dark Clouds} in their creations this week}.
%Furthermore, we found that the homework outcomes can serve as intervention resources for therapists during in-sessions, which can help clients identify and build on their strengths.
%Thus, AI-compiled homework data demonstrates potential in the practice of art therapy.
%In future work, we propose that AI agents can record real-time conversation data between therapists and clients during in-sessions and integrate it with past homework history data to offer real-time insights and suggestions, thus becoming a valuable resource for therapists.


  %Implication 4: Designing AI Agents involving attributing or denying humanness.
        % 在我们研究中,我们发现来访者对于AI agents的角色的perception发生动态的变化。
        %例如,When exploring different homework topics, clients often have diverse perceptions of the role of AI agents. 
        % 其次,随着时间的变化以及治疗师的参与,来访者对于AI agents的角色的perception也发生变化。
        % 因此,我们研究中来访者对AI agents 的角色的perception划分为humanization(e.g.,陌生人,伙伴和情感专家)~ 和 dehumanisation(e.g.,宠物,玩偶或者黄色的灯 )~
        % 有趣的是, 在我们的研究中,来访者会把AI agent比作是一个“会说话的玩偶”,可以作为cathartic target进行表达攻击性。
        %在未来的工作中,我们可以考虑如何设计AI agents变成情绪发泄的“sandbag”~\cite{luria2019challenges,lee2019caring}。
        %Freud and other early psychoanalysts believed that this release of repressed emotions, particularly negative ones like anger or sadness, could have therapeutic benefits, leading to psychological relief and a healthier mental state~\cite{breuer2009studies}.
        % In HCI, Luria et al. have investigated various cathartic objects aimed at facilitating catharsis and have examined the challenges involved in designing for catharsis~\cite{luria2019challenges}.
        % 然而,AI agent当作发泄对象,可能潜在争议的伦理道德.
        %一方面,我们考虑如何设计ai agent as cathartic targets that allow a variety of negative emotional expressions~\cite{luria2019challenges}.
        % 另一方面,我们还要考虑在什么情况下支持cathartic interactions through AI agents,什么场景也许带卡负面的伤害。
        
    
        %担任的角色不同活动,有流动性的。。。。
        %同时ai与互动的过程中是有去人化和人的,当成为专家,归人化,
        %宠物和玩具,去人化
        %玩具 - ,传声筒,沙袋,场景不是设计,未来的设计值得注意,情绪发泄的对象在心里治疗的有帮助的,因为碍于治疗师的人,不好意思。对于ai当作发泄对象,可能潜在争议的伦理道德,一方面怎么设计ai agent,承认表露阴暗面的对象;另一方面,在什么情况我们应该支持行为,其他场景用户对于ai的辱骂虐待带来负面的伤害,未来的研究是进一步探索理解。
    %可参考的资料:
    % 关于 ai agent作为一种情感宣泄的target可以参考的一篇机器人的文献https://dl.acm.org/doi/pdf/10.1145/3385007
    % Attributing and denying humanness to others https://www.researchgate.net/publication/247506022_Attributing_and_denying_humanness_to_others
%===========
% 
%\subsubsection{Implication 4: Designing AI Agents Involving Attributing or Denying humanness.}
%These perceptions can be conceptualized along two dimensions~\cite{Haslam29092008}: humanization (e.g., perceiving AI as a stranger, companion, or emotional advisor) and dehumanization (e.g., perceiving AI as a pet, a doll, or even an inanimate object like a yellow lamp).In our study, we found that some clients perceived the AI agent as a `companion', while our client perceived the AI agent as a `talking doll', serving as a cathartic outlet for expressing aggression.Future work could consider designing AI agents that dynamically adapt to user perception changes when interacting with them. For example, when clients perceive AI agents as a companion, we could consider proactively providing emotional support. Conversely, when perceived as a talking doll, AI could facilitate catharsis by serving as an emotional ``punching bag''~\cite{breuer2009studies}, e.g., customizing different AI characters, enabling them to express their emotions through gesture-based interactions and language.

%\textit{Risks and Concerns of AI Agents Involving Attributing or Denying humanness:} On the one hand, we found that therapists were concerned that clients tended to perceive AI agents as companions or therapists.The AI-driven transference might lead to negative effects such as over-reliance, misinterpretation of AI responses, and worsened symptoms~\cite{joseph2024transference}.On the other hand, utilizing AI agents as cathartic targets for emotional expression raises potential ethical concerns, which might even exacerbate feelings of anger and aggression ~\cite{luria2020destruction}.Beyond designing for catharsis, other strategies for managing anger should be explored~\cite{bushman1999catharsis}. These may involve using voice, text, facial expressions, or physiological signals (e.g., heart rate) to assess emotional intensity. The AI could then shift from accepting anger-driven expressions to providing calming and supportive interventions.






%For example, C6 mentioned that her perceptions of AI change from a `stranger' to a `companion'.
%In psychotherapy, AI-driven transference can provide immediate feedback and comfort when needed, and may facilitate more open discussions about personal issues~\cite{holohan2021like}.
%However, we found that therapists were concerned that clients may project emotions onto AI during therapy.his AI-driven transference might lead to negative effects such as over-reliance, misinterpretation of AI responses, and worsened symptoms~\cite{joseph2024transference}.Future AI design in therapy homework should focus on mechanisms to identify and manage transference dynamics effectively. For instance, an emotional safety alert could be introduced, integrating speech emotion recognition and sentiment analysis to detect cues like fear, safety, or trust, offering timely, supportive reminders.}
%\textcolor{blue}{Interestingly, our client perceived the AI agent as a \qt{talking doll}, which can serve as a cathartic target for expressing aggression. Freud et al. suggested that releasing repressed emotions, particularly anger and sadness, could lead to therapeutic relief and provide psychological benefits for clients ~\cite{breuer2009studies}.However, utilizing AI agents as cathartic targets for emotional expression raises potential ethical concerns, which might even exacerbate feelings of anger and aggression ~\cite{luria2020destruction}.In future work, we could explore how to design AI agents as emotional "punching bags" to facilitate catharsis ~\cite{luria2019challenges,lee2019caring}, while addressing ethical considerations. For example, we could allow clients to Beyond designing for catharsis, other strategies for reducing anger should also be considered~\cite{bushman1999catharsis}. These could include using voice, text, facial expressions, or physiological signals (e.g., heart rate) to detect the intensity and fluctuation of clients' emotions. The AI could then gradually transition from accepting anger-driven expressions to offering soothing and supportive interventions.}





%We also need to consider in which scenarios  cathartic interactions through AI agents should be supported and in which scenarios they might potentially cause harm.

%In HCI, Luria et al. have investigated various cathartic objects aimed at facilitating catharsis and have examined the challenges involved in designing for catharsis~\cite{luria2019challenges}.

 

%For example, when exploring different homework topics, clients often have diverse perceptions of the role of AI agents. 
%Furthermore, clients' perceptions of the role of AI agents might change over time and can also be influenced by therapist involvement.

%Freud et al. believed that releasing repressed emotions, especially negative ones like anger or sadness, could provide therapeutic benefits, leading to psychological relief and improved mental health ~\cite{breuer2009studies}.




\subsection{Limitations and Future Work}
Our study faces several limitations:
First, we chose to use commercial APIs for conversation agent development, which may raise concerns about user data privacy.
Especially for LLM-driven healthcare systems, which may collect sensitive personal information, there is a risk of medical data leakage~\cite{yang2024talk2care}. 
Future designs could explore deploying localized AI models on edge devices, avoiding centralized server uploads. Additionally, during deployment, users should be well-informed about data collection practices. Regularly publishing privacy policies and data usage reports can enhance transparency, while allowing users to view, modify, and delete their data builds trust.

Another important aspect regards the safe use of generative AI. 
%reviwer提到没有解释清楚的地方
To minimize discomfort or uncanny effects, we incorporated predefined prompts as scaffolding to better control the resulting image, e.g., we included predefined prompts to add characters with Asian facial features, ensuring that clients would create characters that align with their expectations.
However, generative AI occasionally generated images that did not fully meet clients' expectations, potentially causing some tasks frustration.
%emotional fluctuations and exacerbate feelings of insecurity.
Future work should prioritize collaboration with art therapists to define the model's application scope and safety boundaries. 
Also, developing a specialized dataset designed for therapeutic purposes—incorporating artworks that reflect emotional expression or commonly used therapeutic symbols—could significantly improve the model's effectiveness in supporting therapy homework.

Finally, increasing the sample size would enhance the validity of the results. Future work should aim to include more participants in extended longitudinal studies to further investigate the generalizability of \name{} in real-world applications.
%Secondly, although our clients could easily complete the homework with the help of the versatile AI Brush, they may sometimes choose to incorporate objects or elements beyond what the AI Brush offers. Also, due to limitations in the image-based generative model, a small number of AI Brush users may need to make multiple modifications to achieve the desired outcome. Looking ahead, it will be important to expand the AI Brush's capabilities by enabling the generation of a broader range of objects and training more advanced image-based generative model to better support homework tasks.
%Furthermore, we chose to use commercial APIs for conversation agent development, which may raise concerns about user data privacy. In particular, privacy risks are a major concern in LLM-driven systems in healthcare, as they may collect patients' personal data~\cite{yang2024talk2care}.Additionally, relying on commercial APIs could lead to unexpected network delays, negatively impacting the user experience.Future work can explore deploying localized AI models on edge devices to enhance user data privacy and reduce the need for cloud-based data processing.Finally, while we improved the quality of conversations through iterative prompts and collaboration with therapists, we have not evaluated the quality of the questions generated by the AI agents or the accuracy of summarizing homework data.



