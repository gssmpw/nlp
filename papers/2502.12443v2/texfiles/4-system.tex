\section{\name{} SYSTEM}
This section covers the design and implementation of \name{}. We first present the core design features developed based on our contextual understanding, followed by a typical usage scenario and the technical implementation details.

\begin{figure*}[tb]
  \centering
  \includegraphics[width=\linewidth]{images/UI.png}
  \vspace{-4mm}
  \caption{\name{} consists of the client-facing application and the therapist-facing application (is presented in translated English Version): (1) the client-facing application includes the Art-making Phrase interface and the Discussion Phrase interface; (2) the therapist-facing application includes Homework Agents' Customization interface and AI-compiled Homework Interface}
  \Description{Figure 2 presents two applications: the client-facing application and the therapist-facing application. The left side of the figure shows the client-facing application, while the right side displays the therapist-facing application.
  In the client-facing application, the upper section represents the "Art-Making Phase" interface. This interface includes a conversational agent, an art-making canvas for the user, a co-creative preview canvas, and tools that allow clients to select different brushes. The lower section is the "Discussion Phase" interface, which contains a human-AI co-creative artwork and a conversational agent that enables users to engage in dialogue with it.
  In the therapist-facing application, the upper section shows the "AI-Compiled Homework History" interface, where therapists can review the client's homework history. This interface includes the client's personal information, original creations, the creation process, the final artwork, conversation records, and two questions summarized by the AI assistant based on the dialogue history. The lower section displays the "Homework Agents Customization" interface, which allows therapists to set homework tasks, modify or add principles and sample questions for the conversational agents, and write personal messages for the client.}
  \label{fig:ui}
\end{figure*}


\begin{figure*}[tb]
  \centering
  \includegraphics[width=\linewidth]{images/system_implementation.png}
  \vspace{-4mm}
  \caption{Overview of the \name{} system architecture}
  \Description{
Figure 4 illustrates the system architecture. On the left, the client-side focuses on two main functions: art-making and conversation. The art-making function is powered by ControlNet for color segmentation and the Stable Diffusion model. Additionally, an art-making agent helps clients describe their artwork, summarizing these descriptions into prompts, which are then fed into the Stable Diffusion model. The conversation function is built on large language model agents, and the system supports voice interactions between clients and agents through speech-to-text and text-to-speech APIs.
On the right, the therapist-side also has two main functions: customizing homework agents and tracking homework history. The customizing homework agents feature allows therapists to set personalized homework tasks and define the principles for the conversational agents. These customizations are integrated into the conversational agents. Additionally, summary agents are used to compile and summarize the client's homework data.}
  \label{fig:system}
\end{figure*}

\subsection{\name{} Core Design Features}
To understand how a human-AI system could support both clients' homework (\textbf{RQ1}) and therapist-client collaboration surrounding it (\textbf{RQ2}), we have designed \name{}, which consists of: 
(1) a client-facing application that combines human-AI co-creative art-making with conversational interaction to facilitate homework in daily settings, and (2) a therapist-facing application that offers AI-compiled homework history and customization of client homework agents for tailored guidance (see \autoref{fig:ui}). To address three key challenges we identified in CONTEXTUAL UNDERSTANDING, we collaborated closely with the therapists to develop the following core design features of \name{} (\textbf{DF1}–\textbf{DF3}):


%According to our three common challenges from our context study, we identified three core design features:
\subsubsection{\textbf{DF1}: Combining Human-AI Co-creative Art-making with Conversational Interaction (Client-facing Application)}
To address \textbf{CH1}, the client-facing application leverages a human-AI co-creative canvas (\autoref{fig:ui} (a)) to lower the art-making threshold for clients, and a two-phase conversation workflow to provide clients with structured guidance in both the ``Art-making Phase'' and the ``Discussion Phase''. It features three panels: 
\begin{itemize}
    \item \textit{Human-AI Co-creative Canvas} including various AI brushes~(\autoref{fig:ui}~(a)-\circled{6}) that enable users to draw color-coded segments~(\autoref{fig:ui} (a)-\circled{4}) with each color mapped with a semantic concept, and translate these user-drawn forms into concrete objects, as previewed in~(\autoref{fig:ui}~(a)-\circled{5}). The rationale for enabling AI brushes is that clients can choose brushes representing a variety of semantic concepts, and they can create color-coded forms to directly control the shapes of the corresponding object. Their usage of the semantic brushes can also help therapists understand the conceptual elements of the artworks, which might project client's personality and emotions~\cite{malchiodi2007art}.
    \item \textit{Art-making Phase Conversation} is designed to encourage clients' self-expression by prompting them to verbally describe their artistic concepts as they create color-coded segments using the AI brush. These verbal descriptions are then summarized into text prompts, providing greater control over the generated images, as illustrated in \autoref{fig:ui} (a)-\circled{1} and \circled{2}.
    \item \textit{Discussion Phase Conversation} has been suggested by our therapists to facilitate deliberate self-exploration and reflection right after art-making through multi-round conversations, following our task instructions~(\autoref{fig:ui}~(b)-\circled{1}). The dialogue principles and example questions in the default task instructions came from the art therapy literature~\cite{buchalter2017250,buchalter2004practical,buchalter2009art} and were later customized by the art therapists.
\end{itemize}

\subsubsection{\textbf{DF2}: Supporting the Customization of Client Homework Agents (Therapist-facing Application)} 

To address \textbf{CH2}, customizing homework agents in the therapist-facing application includes three features:
\begin{itemize}
    \item \textit{Dialogue Customization} allows therapists to create, modify, or reorder dialogue principles and their example questions, controlling the dialogue flow of client's conversational agent in Discussion Phase. This enables the therapist to tailor the homework conversation (especially in the Discussion Phase) based on the therapist's professional belief and their understanding about the needs of a specific client~(\autoref{fig:ui}~(d)-\circled{1}).  
    \item \textit{Homework Task Customization} is designed to support therapists in tailoring homework tasks based on their understanding of the client~(\autoref{fig:ui}~(d)-\circled{2}), with these tasks then displayed in the client-facing application's dialog display area~(\autoref{fig:ui}~(a)-\circled{2}); 
    \item \textit{Opening Message Customization} was required by our therapists for tailoring the greeting message to a client, which are displayed in the dialog area of their client-facing application~(\autoref{fig:ui}~(d)-\circled{3}). These messages offer personalized encouragement and emotional support during art therapy homework.
\end{itemize}


%This allows therapists to shape the structured guidance provided by the conversational agent;


\subsubsection{\textbf{DF3}: Enabling AI-compiled Homework History (Therapist-facing Application)}
To address \textbf{CH3}, a AI-compiled homework history interface is designed with three panels:
\begin{itemize}
    \item \textit{Homework Overview} including personal information, a visualization of usage and a short AI-compiled summary of each session~(\autoref{fig:ui}~(c)-\circled{2}, \circled{3} and \circled{4}); 
    \item \textit{Homework Records} show different outcomes of a homework session, including client-drawn color segments (\autoref{fig:ui}~(c)-\circled{5}), the client-AI co-created artwork~(\autoref{fig:ui}~(c)-\circled{7}), the art-making process~(\autoref{fig:ui}~(c)-\circled{6}), and the client-AI conversation history~(\autoref{fig:ui}~(c)-\circled{8}); 
    \item \textit{AI Assistant Summary} eases therapists' review of a client's homework session by summarizing based on the client's verbal inputs and art-making outcomes. It highlights content, experiences, feelings, and reflections expressed during the homework session (\autoref{fig:ui}(c)-\circled{1}). Following therapists' suggestions, it avoids interpretations and instead poses questions from a novice therapist's perspective, pointing out potential relevant aspects to aid the therapist's review.
    
%for summarizing clients' image descriptions, feelings, and experiences, offering therapists insights.
\end{itemize}



\subsection{\name{} Usage Scenario}

Here we present a typical usage scenario of \name{}: Alice had been struggling with her relationship, feeling misunderstood by her partner, so she booked online art therapy sessions with Jessica, her therapist.
So she decided to book online art therapy activities with Jessica, her art therapist.
After their first session, Jessica used \name{}'s therapist-facing application.
Drawing from her experience, she incorporated dialogue principles and example questions into the system (\autoref{fig:ui}~(d)-\circled{1}). Jessica first added a dialogue principle of ``guiding users to describe the overall work'' and provided a few example questions for the conversational agent to reference: e.g., ``I would love to hear how you describe this work.'' Similarly, she added a few other principles and adjusted their order. The conversational agent can then follow these principles and examples in the given structure.
%to guide the conversational agent in asking the right questions~(\textbf{DF2}). 
Further, Jessica also tailored homework tasks related to couple relationships (\autoref{fig:ui}~(d)-\circled{2}) and added a supportive message, ``Your sensitivity and ability to put yourself in others' shoes are truly a gift'' (\autoref{fig:ui}~(d)-\circled{3}), to offer encouragement during Alice's homework~(\textbf{DF2}).

Another day, after a quarrel with her boyfriend, Alice felt overwhelmed and turned to the homework Jessica had assigned using \name{}: drawing two plants, one representing herself and the other her partner. She opened the client-facing app and entered the ``Art-making Phase''~(\textbf{DF1}) on her tablet. She selected the \textit{Tree} brush from the toolbox~(\autoref{fig:ui}~(a)-\circled{6}) and began drawing a tree on the canvas, while the art-making agent prompted her to describe it through voice input~(\autoref{fig:ui}~(a)-\circled{2}). 
Alice described the tree as an apple tree full of blossoms, and the agent summarized text prompts about the current creation based on her input. 
She then added more objects like \textit{Soil}, \textit{Cloud}, and \textit{Grass} and completed her artwork. After selecting the \textit{Watercolor Painting}~(\autoref{fig:ui}~(a)-\circled{3}) style,
she clicked "Generate". The color segments drawn by the client, combined with the text prompts, served as inputs to produce a human-AI collaborative artwork that deeply resonated with her emotions.

Moving into the ``Discussion Phase'', the conversational agent guided Alice in reflecting on her artwork, asking questions like, ``Would you like to describe your tree?'' Alice shared her feelings and realized that she and her boyfriend were like two different plants—independent yet needing to understand each other.

Before their next session, Jessica reviewed Alice's homework history through the therapist-facing interface~(\textbf{DF3}). She examined the original and co-created artworks, along with the conversation records, gaining a deeper understanding of Alice's situation. Noticing that the AI assistant had prompted her to consider Alice's experience of arguing with her partner~(\autoref{fig:ui}~(c)-\circled{1}), Jessica decided to address these insights during their upcoming session. After revisiting Alice's homework sessions, Jessica decided to moved the principle ``naming the artwork'' from the second to the end of the dialogue. She believes this would help Alice better articulate her thoughts.






% She opened the client-facing application and entered the “Art-making Phase”~(\textbf{DF1}) in her browser on her tablet. First, she selected the \textit{Tree} brush from the toolbox~(\autoref{fig:ui}~(a)-(6)). 
% She then drew a tree on her canvas using the \textit{Tree} brush, while the art-making agent simultaneously prompted her to describe the tree she had created through voice interaction~(\autoref{fig:ui}~(a)-(2)).
% Alice described that \textit{Tree} is an apple tree full of apple blossoms through audio.
% At the same time, the agent summarized a prompt based on her description of \textit{Tree}.
% Then, she utilized \textit{Tree}, \textit{Soil}, \textit{Cloud} and \textit{Grass} brushes to create an artwork.
% After completing the artwork and describing the various artistic elements in words, she clicked ``\textit{Use This Description}'' button. All of her verbal descriptions of the artwork were displayed in \autoref{fig:ui}~(a)-(1) , where they can be modified and updated. Then, she selected the artistic style of \textit{Watercolor Painting}~(autoref{fig
% }~(a)-(3)), and finally clicked the ``\textit{Generate}'' button.
% A human-AI co-creative artwork was shown on the preview canvas in \autoref{fig:system}~(a)-(5). 
% Alice loved this artwork because it captured and externalize her emotions using two \textit{Tree} brushes. 

% She then moved into the "Discussion Phase" to facilitate deep self-exploration and reflection on the artwork through an AI conversational agent~(\textbf{DF1}). 
% First, the AI conversational agent follows Alice's customized conversation principles to guide clients in exploring the thoughts and meanings behind their artworks, e.g., describing the overall artwork~(\qt{would you like to describe your tree?}).
% Alice shared many of her feelings and experiences with the AI agent and came to the realization that she and her boyfriend were like two different plants—both independent individuals who needed to understand and accept each other.

% Before the next online session with Jessica, the therapist Jessica reviewed Alice's homework history through the therapist-facing interface~(\textbf{DF3}). 
% She reviewed the client's original and co-creative artworks, along with the conversation records, to provide the therapist with a deeper understanding of the client's needs. 
% Additionally, she noticed that the AI assistant, much like an apprentice, inquired whether the therapist had considered the client’s experience of arguing with her boyfriend~(\autoref{fig:ui}~(c)-(1)). She decided to discuss the insights she had discovered during their upcoming session.


%\subsection{\name{} Usage Scenario}The \name{} system is designed with two distinct user interfaces, tailored to the roles of the visitor and the therapist, respectively. Both interfaces are web-based and optimized for mobile devices, ensuring functionality across various devices including tablets, laptops, and desktops.\subsubsection{\textbf{Visitor Interface: }}The visitor interface consists of two main components:\textit{1. Drawing Module:} This includes tools such as a canvas and brushes. Users create artwork which is then processed by the system to generate results.\textit{2. Dialogue Module: } Includes a chatbot dialogue box that facilitates communication between the user and the system. The use of this interface is divided into two sequential stages.  \textit{In the first stage,} the user paints with the drawing module and verbally describes their artwork to the chatbot. The system automatically summarises the user's description alongside the visual content. Based on the user's satisfaction with the generated artwork, they can either proceed to the second stage or request changes, which can be self-directed or managed by an AI re-generation process. \textit{In the second stage,} the chatbot, customised by the therapist, interacts dynamically with the user based on the thematic content of the user's drawing. \subsubsection{\textbf{Therapist Interface: }}The therapist interface consists of several functional modules:
%\textit{1. Drawing Records Viewing Module:} Therapists can access comprehensive records of user sessions. This module displays the user's drawing activities, including the drawing process, final results, and accompanying dialogue transcripts.
%\textit{2. Task and Messaging Module:} Therapists can assign tasks or send messages to visitors. These messages are delivered to the visitor through the dialogue module during subsequent interactions with the system.
%Both the visitor and therapist interfaces are designed to be user-friendly, ensuring ease of use regardless of the user's technical expertise. This dual-interface approach not only enhances the therapeutic process but also facilitates a more personalized interaction between the user and the system.





% % % % % % % % % % % % % % % % % % % % % % % % % % % % % % % % % 

\subsection{\name{} System Implementation}
As \autoref{fig:system} shows, the \name{} system consists of the front-end of both client-facing application and therapist-facing application, as well as the back-end including LLM Agents module, AIGC module, and the memory module.
% \subsubsection{Overview of System Architecture}

% In this section, we present in detail how we implemented the \name{} system. As shown in \autoref{fig:system}, the system consists of the client-facing application 


% two distinct user interfaces (UIs) - one for users engaged in therapeutic art making, and another for therapists monitoring and intervening in sessions. 
% The system also has a back-end server that uses a large language model (LLM) at its core service designed to support real-time interactions, support the dual-interface model, and handle various data generated during use.


\subsubsection{Front-end}

The front-end of both client and therapist applications are web-based UIs built with Quasar Framework, which is based on Vue.js, and extends its capabilities by providing a rich UI component library, tools and cross-platform development support. %需要加这句对quasar的介绍吗?
For the conversational interaction of client-facing application, client messages are processed using OpenAI's TTS model\footnote{OpenAI TTS Models, https://platform.openai.com/docs/models/tts} to generate a corresponding voice message that autoplays by default. We also implemented speech to text function using the Voice Dictation (Streaming Version)\footnote{iFLYTEK Speech-to-Text, https://global.xfyun.cn/products/speech-to-text} service from iFlytek. For the therapist-facing application, all data is stored and transmitted in JSON format, interacting with the back-end server via HTTP requests, allowing therapists to select a client from a list of names, and view the session logs, assign customized homework, and modify the agent's guidelines. 

% As mentioned earlier, the client-facing application is designed with two phases. The \textit{"Art-making Phase"} and the \textit{"Discussion Phase"}. Both phases feature a Dialogue Module, which was custom-built to suit our specific needs. In this module, each AI response returns a set of interactive functions. For example, dialogue text is processed using OpenAI's TTS model\footnote{OpenAI TTS Models, https://platform.openai.com/docs/models/tts} to generate a corresponding voice message that autoplays by default. Users also have the option to play or pause the message by clicking the 'PLAY THE VOICE' button. Additionally, when the AI generates a summary of the drawing log to assist with the prompt, it provides a 'USE THIS DESCRIPTION' button, allowing users to directly apply or modify the AI-generated prompt. We also implemented speech to text function using the Voice Dictation (Streaming Version)\footnote{iFLYTEK Speech-to-Text, https://global.xfyun.cn/products/speech-to-text} service from iFlytek.

%The \textit{"Art-making Phase"} also includes a drawing module with a 512x512 canvas and a variety of color-segmentation brushes. This module supports freehand drawing while recording the entire drawing process. Upon completion, the system analyzes each pixel to match the colors used to their respective brushes, optimizing performance to avoid latency and ensure a smooth experience for therapeutic purposes. When the "Generate" button is clicked, the system sends the user's drawing, prompt, drawing log, and dialogue history from the \textit{"Art-making Phase"} to the back-end via HTTP, where they are stored along with the generated image.

% \subsubsection{Frontend: Therapist Interface}



\subsubsection{Back-end}

The backend is implemented using Flask, responsible for generating images, processing LLM workflows, and managing data. 
%表明我们的模型没有训练
The image generation model pipeline uses \textit{ControlNetModel} and \textit{StableDiffusionControlNetPipeline} from huggingface's diffusers\cite{von-platen-etal-2022-diffusers}, and the base model is \textit{runwayml/stable-diffusion-v1-5}, the control net segmentation model is \textit{lllyasviel/control\_v11p\_sd15\_seg}\footnote{Controlnet - v1.1 - seg Version, https://huggingface.co/lllyasviel/control\_v11p\_sd15\_seg}. The system uses asynchronous, queue-based processing with multi-threading to efficiently handle multiple concurrent image generation requests. 

The backend for the LLM workflows consists of five prompted LLM Agents, all utilizing GLM-4\footnote{GLM-4 GitHub, https://github.com/THUDM/GLM-4}. 
The first LLM, refer to "Art-making Agent" of \autoref{fig:system}-\circled{2}, is utilized during the Art-making Phase of the client app, where it summarizes users' drawings and descriptions into prompts for the Stable Diffusion model. The second LLM, refer to "Conversation Agent" of \autoref{fig:system}-\circled{2}, is employed during the Discussion Phase, using a template that incorporates principles and customized questions from therapists. The third and fourth LLMs, refer to "Conversation Agent" of \autoref{fig:system}-\circled{2}, process data from the Art-making and Discussion Phases, respectively, and condense the information into short summaries for the therapist. The fifth LLM, also refer to "Conversation Agent" of \autoref{fig:system}-\circled{2}, integrates dialogue from both phases and generates three therapist-focused questions based on the conversation history. All the prompt are included in the supplementary materials of the paper.

These services communicate via RESTful APIs, the entire backend service has 15 APIs, ensuring the functionality and a smooth front-end and back-end interaction.
The back-end handles data storage and retrieval, managing user-generated content including AI-generated and original artwork, creative process data, homework dialogue data, color segmentation area data, and customized homework data. Images are stored in PNG format, while logs and settings are saved in JSON format.



% api的个数,保存的数据格式。













% [
%     {
%         ‘description‘:’[user-drawn] I drew the TV. [canvas content] The canvas now has the building, the TV.’ ,
%         ‘time": ’11:11:05’
%     },
%     {
%         ‘description‘:’[user-drawn] I drew the table. [canvas content] The canvas now has buildings, a TV, and a table.’ ,
%         ‘time": ’11:11:40’
%     },
%     {
%         ‘description‘:’[user-drawn] I drew the seats. [canvas content] The canvas now has buildings, TVs, tables, and seats.’ ,
%         ‘time": ’11:12:07’
%     },
% ]
