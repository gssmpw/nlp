\section{Introduction}
%本质是什么,
%第一个挑战,门槛很高门槛很高,包含art-making(语言+非语言,没有治疗师)
%现有hci很少支持,但是有。。。。
%同时提供guidance
%能否或者怎样支持
% another challenging, 疗愈石如何定制如何追踪,为什么重要。本质一个人完成活动,非同时的therapist - client写协作,生活中therapeutic 持续过程,反馈的过程,协作,长时间的信任,亲密关系,解决问题
%目前没有,有别的研究hci研究其他事情
% 会不会以及怎样非同时的协作,
%基于motivation 
Art therapy is a multi-modal therapeutic approach that combines creative activities with verbal expression to help individuals articulate their feelings and enhance their mental well-being~\cite{malchiodi2011handbook}.
It includes both \textit{in-session} and \textit{between-session} activities.
In-session therapy involves conversation and collaborative work between the therapist and the client.
%一句话在psychherapy的包括所有therapy homework的定义(包括了art therapy, cBT和DBT)。
Meanwhile, between-session activities---referred to as ``\textit{therapy homework}'', play a vital role by incorporating assignments that enable clients to carry the therapeutic work into their everyday lives~\cite{kazantzis2007handbook}. 
% therapy homework在心理健康领域是常见的一种方式;然后写它的意义和重要性(包括在多个therapy context,比如说art therapy和CBT,DBT,mindfulness(是否都属于psychtherapy?),)
It has been proposed that ``homework'' is a common element in psychotherapy, including art therapy~\cite{heckwolf2014coordinating}, cognitive-behavioral therapies~\cite{kazantzis2007handbook}, solution-focused therapy~\cite{beyebach1996research}, self-help interventions~\cite{jordan1995programmed}, and others.
Therapy homework in psychotherapy in general, serves the common goals of helping clients reinforce therapy progress through practice in natural environments~\cite{beerse2020therapeutic}, while also strengthening their positive relationship with the therapists~\cite{huckvale2009case}.
These therapy homework in psychotherapy typically include art-making~\cite{hoshino2011narrative,barrow2020experiences}, journaling~\cite{borkin2014healing,smith2019visual}, emotion-self report~\cite{nesset2021does} as well as mindfulness and relaxation techniques~\cite{knapp2024daily}. 


% 在艺术治疗中,布置家庭作业很常见,这句话回应
In art therapy, it is common for art therapists to assign homework to clients~(see \autoref{fig:context1})~\cite{hoshino2011narrative}.
%art therapy homework是一种独特的形式(art-making+verbal 形式)
They often recommend art therapy homework that integrates art-making with verbal expression~\cite{smith2019visual,heckwolf2014coordinating,hoshino2011narrative}.
%art-making+verbal好处与意义
This combination allows clients to explore their innermost thoughts and find a cathartic outlet through verbal expression, while also benefiting from the creative and expressive nature of art-making~\cite{smith2019visual}. 
However, without a therapist's guidance, the threshold for therapeutic art-making can increase~\cite{du2024deepthink}, and clients in psychotherapy often struggle with constructing a narrative, understanding their past, and verbalizing their feelings and experiences~\cite{pennebaker1999forming,mayer2011emotional}.

In the field of human-computer interaction (HCI), limited knowledge has been accumulated about how to effectively support between-session therapy activities~\cite{Oewel_2024}, including art therapy homework that uniquely integrates art-making with verbal expression. 
A few recent cases suggest that human-AI co-creative approach can lower the barrier to art-making and enhance creativity in therapy practices~\cite{du2024deepthink, liu2024he}. 
Extensive research also highlights how conversational agents can aid clients in self-exploration, self-disclosure, and emotional support~\cite{kim2024mindfuldiary, park2021wrote, jo2023understanding, seo2024chacha}. 
Yet, so far, little empirical research exists on combining human-AI co-creative art-making with conversational agents to support clients' art therapy homework.


%In the field of human-computer interaction~(HCI), we have accumulated little knowledge about how to effectively support between-session activities~\cite{Oewel_2024}, especially regarding art therapy homework that integrates art-making with verbal expression. 
%However, several studies have examined that human-AI co-creative art making can support art therapy practices, which can lower the art-making threshold and support creativity~\cite{du2024deepthink, liu2024he}.
%On the other hand, there is extensive research on how conversational agents can support clients' self-exploration and self-disclosure~\cite{kim2024mindfuldiary, park2021wrote}, and also deliver their emotional support~\cite{jo2023understanding, seo2024chacha}.
%For example, ChaCha is a conversational agent powered by large language models (LLMs) designed to track conversation context, which can promote children in sharing personal events and emotions~\cite{seo2024chacha}.
%However, little empirical knowledge has been accumulated regarding whether and how to support art therapy homework through combining human-AI co-creative art-making with conversational agents.

Therapists also face challenges in tracking and tailoring art therapy homework. Homework fosters an asynchronous therapist-client collaboration that keeps clients engaged between sessions~\cite{huckvale2009case,hoshino2011narrative} and builds long-term trust and intimacy~\cite{huckvale2009case}.
This asynchronous collaboration requires therapists to track homework records and history, which serve as valuable data for understanding clients' thoughts between sessions and can help shape the treatment approach~\cite{kazantzis2007handbook,hoshino2011narrative,richards2018impact}.
However, managing and tracking homework records can be difficult and often increases therapists' workload~\cite{richards2018impact}.
Moreover, prior studies also highlight the importance of therapists tailoring therapy homework while tracking its progress, which can reinforce that therapeutic process~\cite{Oewel_2024, kazantzis2007handbook}. 
%Customizing homework reinforces the therapeutic process and strengthens the therapist-client relationship~\cite{Oewel_2024, dryden2011cbt}. 
However, assisting therapists in creating structured homework that is tailored to the client's needs and incorporates emotional support remains a significant challenge~\cite{kazantzis2007handbook, coon2002encouraging, Oewel_2024}.
%therapist's side in terms of tracking an tailoring art therapy homework. Therapy homework creates an asynchronous therapist-client collaboration, helping maintain engagement between sessions over the long term~\cite{dattilio2012collaboration}. This ongoing integration of therapeutic practice into daily life can build long-term trust and intimacy between therapists and clients~\cite{cronin2015integrating}. 
%However, this asynchronous collaboration requires therapists to track homework records and history. Homework provides a valuable opportunity for data collection, offering insights into clients' thoughts between sessions, which can help shape the treatment approach~\cite{freeman2007use,kazantzis2007handbook,dattilio2012collaboration}. Despite its benefits, therapists often struggle to access and track homework history efficiently, leading to increased workload~\cite{Wilansky2016}.



% Moreover, prior studies indicated that therapists need to tailor therapy homework and track the homework history~\cite{Oewel_2024,kazantzis2007handbook}.
% First, therapists should customize therapy homework to reinforce the therapeutic process and strengthen the client-therapist alliance~\cite{Oewel_2024,dryden2011cbt}.
% However, it is challenging to assisting therapists in tailoring structured homework tailored to the client’s needs and infusing emotional support into therapy homework~\cite{kazantzis2007handbook,Tang2017,dryden2011cbt}.
% Meanwhile, the homework is a data-collection opportunity, which can offer valuable insights into clients' thoughts between sessions and helping therapists shape the treatment approach~\cite{freeman2007use,kazantzis2007handbook,dattilio2012collaboration}.
% However, therapists often find it difficult to access and easily track homework history, leading to a significant increase in workload~\cite{Wilansky2016}. 
% Thus, therapy homework creates an asynchronous therapist-client collaboration, maintaining fluid engagement between therapy sessions over the long term~\cite{dattilio2012collaboration}.
% The continuous integration of therapeutic practice into daily life can also foster long-term trust and intimacy between therapists and clients~\cite{cronin2015integrating}.

Outside of art therapy, several HCI studies have explored how AI conversational agents can assist therapists by summarizing and presenting relevant health information in asynchronous collaborations between clients and practitioners~\cite{kim2024mindfuldiary, yang2024talk2care,10.1145/3659604}. 
For instance, MindfulDiary uses an LLM-based dashboard to help clinicians empathize with patients and understand their daily thoughts~\cite{kim2024mindfuldiary}. 
Additionally, beyond the health domain, recent research has investigated no-code design tools that allow users to customize conversational agents with tailored dialogue flows or styles~\cite{hedderich2024piece, ha2024clochat}. 
Nonetheless, how to support therapist-client asynchronous collaboration surrounding art therapy homework remains an unaddressed opportunity.

% In HCI, on one hand, several studies have investigated how AI conversational agents can act as mediators, assisting therapists in summarizing and presenting crucial health information through asynchronous collaboration between clients and practitioners~\cite{kim2024mindfuldiary,yang2024talk2care}.
% For example, MindfulDiary can support clinicians better empathize with their patients and understand the patients' daily thoughts through a LLM-based dashboard~\cite{kim2024mindfuldiary}.
% On the other hand, much research has explored the use of no-code design tools to allow users to customize conversational agents through tailored dialogue flows or conversational styles~\cite{hedderich2024piece,ha2024clochat}.
% However, there is a lack of research on how to support therapist-client asynchronous collaboration surrounding art therapy homework. 


%To this end, we first conducted formative study with professional art therapists to understand the needs and qualities of AI agent systems in supporting art therapy homework. 

Building upon these motivations, we present \name{} (\autoref{fig:teaser}), a human-AI system consisting of: 
(1) a client-facing application that combines human-AI co-creative art-making with conversational interaction to facilitate clients' art therapy homework in their daily settings; 
and (2) a therapist-facing application that features AI-compiled homework history and customization of client homework agents to offer tailored homework guidance.

% Corresponding to above mentioned motivations, We present the design of \name{}, a multi-agent system which consist of: 
% (1) a client application that integrates human-AI co-creative art-making with LLM-driven conversational agents, enabling clients to complete art therapy homework in their natural environment; 
% and (2) a therapist application that allows therapists to customize 
% conversational agents for guiding therapy homework through adding dialogue principles and adjusting dialogue flow, and review AI-compiled homework data including images, conversation records and co-creative process data. 

Taking \name{} as both a novel system to study about, and a research tool to study with, we conducted a field deployment involving 24 recruited clients and five therapists, over a period of one month. We set out to explore the following research questions:

\begin{itemize}
  \item \textbf{\textit{RQ1: how would a human-AI system support clients' art therapy homework in their daily settings?}}
  \item \textbf{\textit{RQ2: how would a human-AI system mediate therapist-client collaboration surrounding art therapy homework?}}
\end{itemize}


The field evaluation yielded rich empirical data, revealing how \name{} facilitated clients' homework practice across various natural environments. It also offered diverse examples of how combining conversational interaction with art-making supported clients in articulating their feelings and experiences, potentially enabling them to explore the deeper meaning behind their artwork. 
Moreover, the data illustrate how therapists customized homework agents to embed their professional beliefs and personal style, making the agents reflect their unique approach and personal touch. The findings also depict how therapists leveraged the AI-compiled homework history, including both artworks and dialogues, to better understand clients' characteristics and emotional state, identify triggers for deeper discussion, and empower clients during in-session therapy.

%To evaluate \name{}, we conducted a one-month field deployment in which 24 clients engaged with \name{}  in their own contexts, under the guidance of 5 therapists. 
% Our findings suggest that introducing conversational agents while art-making can guide clients in
% exploring personal feelings and creating new meanings behind the artwork. Further, \name{} empowers therapists to embed their professional beliefs, practical experiences and emotional support into the conversational agents, allowing the agents to reflect the therapists' unique personal touch. At the same time, homework data such as homework images and conversation logs provide therapists with valuable insights into client characteristics, as well as identifying triggers for deeper discussions and intervention resources during art therapy sessions.
% Based on the findings, we discuss the design implications of developing AI agents into therapy homework on future design.

This work thereby contributes twofold: (1) a novel system for art therapy homework; and (2) rich empirical findings contextualizing how a human-AI system facilitated clients' homework in daily settings and client-practitioner collaboration, with relevant HCI design implications discussed.



%externalize their feelings and experiences, and create new meanings through art-making and conversation in different contexts, such as home, offices, or hospitals. 
%Further, the multi-agents as a mediator can help therapists customizing AI agents as their extension, e.g., providing structured guidance aligned with therapists' professional belief. Meanwhile, it can support therapists transforming homework data into empowering resources for clients during art therapy sessions.
% mediator 促进 治疗师与来访者的写作文 非对称的

%mental health without the guidance of therapists, particularly through activities such as journaling~\cite{kim2024mindfuldiary}, expressive writing~\cite{park2021wrote}, and self-disclosure~\cite{lee2020designing}. 
%In the fields of human-computer interaction~(HCI) and healthcare, emerging digital technologies~(e.g., digital art-making tool or online chat tools) have been explored to support in-session activities~\cite{mattson2015usability,jones2014supporting,zubala2021art}, which can bridge geographical divides~\cite{zubala2021art}, foster therapeutic rapport~\cite{orr2012technology}.
%However, research in HCI has ~\cite{oewel2024approaches}, .

% art-making + 对话 homework
% 已经又的艺术治疗的工作使用生成ai辅助普通人参与的art making降低menkan
% 但是,绘画ai的支持,但是目前结合语言对话的支持家庭作业。
% 语言对话 agent - mental health很多研究 - 没有治疗师的情况
% 对话好处
% 结合有意义
% 为啥 - 由于艺术治疗的特殊属于,artmaking conversation的联系,我们的研究探索,结合起来的系统,怎么去support therapy homework,
% 另外的challenges
    % 定制化 治疗师定制ai agent
    % 分析 
    % 合作

%In terms of visual arts exercises in art therapy, 
%However, a key challenge is enabling clients to effectively express their emotions and experiences through art-making in art therapy homework without the direct support of a professional therapist~\cite{kazantzis2022comprehensive}.

%In terms of written or spoken exercises, 
%In particular, recent advances in large language models~(LLMs) has led to significant breakthroughs in more naturalistic and adaptive dialogue systems across various scenarios.







%治疗师布置结合visual和语言的家庭作业

%提出一些挑战




% support in-session activities in art therapy
% 现有的好处,一句话。
%however, few between session 
%In the fields of human-computer interaction~(HCI) and healthcare, emerging digital technologies, such as social media platforms, video conferencing tool, or digital art-making tools, has become increasingly prevalent in art therapy practices~\cite{mattson2015usability,jones2014supporting,zubala2021art}.
%Previous research indicates that digital art therapy can effectively , and improve service accessibility for clients facing stigma or disability~\cite{zubala2021art}.
%Despite the benefits of digital art therapy, most studies primarily focused on digital art therapy for synchronous art therapy sessions.




%Recent advancements in artificial intelligence~(AI) present promising avenue for human-AI co-creative methods to enhance mental health support and therapeutic practices~\cite{liu2024he,du2024deepthink,sun2024understanding}.In particular, several studies have examined that AI-infused systems as art-making materials can support synchronous and asynchronous art therapy practices, which can lower the art-making threshold and support expressivity and creativity~\cite{}. The above work demonstrated the potential of leveraging image-based generative AI for art therapy homework. Nevertheless, HCI research has little knowledge about exploring how conversational AI agents can support therapy homework in art therapy.Recent advances in large language models~(LLMs) has led to significant breakthroughs in more naturalistic and adaptive dialogue systems across various scenarios.In mental health and therapy domains, LLM-driven conversational agents can reduce emotional stress and feelings of loneliness in people who are socially isolated~\cite{jo2023understanding}, , or assist psychiatric patients in documenting daily reflections and thoughts ~\cite{kim202310}. Above mentioned research yielded important insights into how LLM-driven conversational agents facilitate clients' self-disclosure and deliver their emotional support.


%Combining image-based generative AI and LLMs-driven conversational agents, we set out to explore how AI agents could serve as bridges, supporting clients’ art therapy homework and mediating therapist-client interactions outside the synchronous one-on-one sessions. 
%Art therapy, as multimodal therapeutic approachs, integrates art creation and verbal communication.



%一个月24名来访者在自己环境中在5明治疗师指导下,






