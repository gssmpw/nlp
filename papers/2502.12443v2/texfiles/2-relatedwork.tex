\section{Background and Related Work} \label{sec:background}
% 首先围绕着RQ来讲我们的这个章节:
    % RQ1: how AI agent support therapy homework 
    % RQ2: how AI agent mediate therapist and clients 
% 对于Art Therapy and Therapy Homework这个章节最重要的需要讲的点就是 
    % 理解艺术治疗中家庭作业的场景,类型/形式 
    % 艺术治疗中家庭作业的场景的重要性 - 
    % 提出挑战
\subsection{Art Therapy and Therapy Homework}
% 介绍家庭作业的本质和重要性
    %一句话介绍艺术治疗的定义引入话题,这里体现一种多模态的方式为接下来的家庭作业做铺垫。
    %除了in-session activities, 还包含了治疗师定制的between-session therapy 作业/练习(therapy homework)
    %介绍简单一下therapy homework的定义.
    %介绍therapy homework的重要性
        %先前的研究表明艺术治疗家庭作业可以帮助来访者 practice reflecting and validating clients' feelings and experiences.
        %it can bring therapy into the “real” world, which can promote therapeutic skills in different contexts.
        % 增加治疗师与来访者之间的协作,从而促进他们之间的信任与关系。
%介绍目前的实践以及第一个challenges。
    %在艺术疗愈,目前有大量的研究表明了艺术治疗师会安排不同类型的家庭给来访者,主要包括视觉艺术以及journaling两种方式(这是文献中说的)。
    %举例来讲,HUCKVALE等人展示了他们安排来访者完成艺术治疗作业,包括looking at, drawing, photographing or writing about the sky every day等
    % 再一个关于治疗师结合art-making以及text的相关的例子
    %介绍也有整合art-making 和 语言对话的方式,并且说明两者的好处是什么
        % 门槛高导致后果
        % 缺乏指导导致后果
%讲述治疗师与来访者之间的协作与挑战
    %首先介绍
Art therapy, guided by a professional art therapist, is a multimodal therapeutic approach that integrates the nonverbal language of art with verbal communication to promote personal growth, insight, and transformation~\cite{malchiodi2007art}.
It includes both in-session and between-session activities. 
In-session activities involve direct interaction between the therapist and client, while between-session activities, often referred to as ``therapy homework'', are personalized exercises assigned by therapists for clients to complete in their real-world environments~\cite{kazantzis2007handbook}.
%需要讲明白therapy homework在心理健康领域是常见的一种方式,包含多种治疗方式
It has been proposed that ``homework'' may be considered a common element in psychotherapy ~\cite{kazantzis2007handbook}. 
The use of homework is widely applied in many psychotherapy contexts, including art therapy~\cite{kazantzis2000homework, huckvale2009case}, solution-focused therapy~\cite{beyebach1996research}, personal construct therapy~\cite{kazantzis2007handbook}, and so on.
%therapy homework形式多种多样
The therapy homework comes in various forms, including journaling~\cite{hoshino2011narrative,dattilio2012collaboration}, drawing~\cite{hoshino2011narrative}, photographing~\cite{huckvale2009case}, workbooks and worksheets~\cite{riley2003family,Oewel_2024}, collages~\cite{riley2003family}, recording~\cite{Oewel_2024}, mindful practices~\cite{smith2019visual}, and photo-elicitation reflective writing~\cite{davis2015mindful}.
Thus, it can help clients practice reflecting on and validating their own feelings and experiences~\cite{riley2003family}. 
Most importantly, it can bring therapy into the ``real'' world, fostering the development of therapeutic skills across various contexts~\cite{kazantzis2007handbook}.
On the other hand, therapy homework helps strengthen the therapeutic alliance between the therapist and client over time, fostering trust and deepening their relationship~\cite{huckvale2009case,cronin2015integrating,sezaki2000home}.


%\textcolor{red}{Previous research suggests that art therapy homework is a crucial component of the therapeutic process~\cite{kazantzis2000homework, huckvale2009case} and}
In art therapy, it is \textit{``not unusual''} for therapists to assign homework to clients~\cite{hoshino2011narrative}.
Art therapists commonly assign a variety of therapy homework exercises to clients, often consisting of visual art creating and verbal reflections~\cite{huckvale2009case,smith2019visual,davis2015mindful,hoshino2011narrative}. 
For example, Huckvale et al. reported that therapists encouraged clients to complete therapeutic work such as looking at, drawing, photographing, and writing about the sky every day~\cite{huckvale2009case}.
Smith et al. also demonstrated that therapists could combine art-making with verbal exercises in homework assignments to enhance clients' daily coping mechanisms, which has been proven to be effective for both patients and caregivers~\cite{smith2019visual}.
This combination of art-making and verbal expression is not only applied in art therapy but also incorporated by other therapeutic contexts~\cite{peretz2023machine,kazantzis2007handbook}. It can leverage the benefits of exploring innermost thoughts and new meaning through verbal expression while harnessing the creative and expressive qualities inherent in the art-making process~\cite{smith2019visual}. 

Despite its benefits, art therapy homework faces several challenges.
Namely, without a therapist's guidance, engaging in therapeutic art-making can be even more challenging, leading to frustration ~\cite{du2024deepthink}.
Further, clients in psychotherapy often find it difficult to build a narrative and express their emotions and experiences verbally~\cite{pennebaker1999forming,mayer2011emotional,kim2024mindfuldiary}. Kazantzis et al. found that practicing without structure or support can trigger clients' negative beliefs and create practical barriers~\cite{kazantzis2022comprehensive}.

Meanwhile, assisting therapists in tailoring homework remains a significant challenge.
Therapists need to tailor therapy homework to align with the clients' current state and capacity~\cite{Oewel_2024}. 
Good tailoring can help clients boosts their motivation to engage with homework and provides the connections from one session to the next~\cite{Oewel_2024,katz2023assigning}.
However, designing well-structured homework tasks tailored to clients' needs can be challenging for therapists, potentially leading to difficulties in gathering essential client data, or creating discomfort~\cite{kazantzis2007handbook, Oewel_2024}.
It is also challenging for therapists to provide consistent encouragement between sessions, which can lead to low homework compliance~\cite{dryden2011cbt}.

Furthermore, effectively tracking clients' homework is also difficult.
The homework provides an opportunity for therapists to collect valuable data, which they can explore, analyze, and synthesize to gain deeper insights into the client's progress~\cite{kazantzis2007handbook}.
The homework data can help therapists better understand clients' issues, uncover their strengths, and shape a more effective treatment approach~\cite{kazantzis2007handbook,gereb2022online}.
However, therapists could face difficulties in adequately tracking homework history, as it increases their workload and homework data cannot be easily organized and revisited~\cite{richards2018impact, gereb2022online,peretz2023machine}.
%Thus, therapy homework creates an asynchronous client-therapist collaboration over the long term. 
Ongoing therapeutic collaboration is a key to enhance homework adherence and foster positive relationships between therapists and clients~\cite{thomas2008evaluating, Ali2017_face2emoji}.




%Therapists must monitor and track homework assignments in real time, both during and between sessions, whether conducted in person or remotely. This can create accessibility challenges and increase the workload for therapists~\cite{}.


%customize structured homework exercises or homeworkinstructions based on the client's individual needs, and progress in thesessions~\cite{Oewel_2024,kazantzis2007handbook}. 
%However, it is challenging to provide structured guidance and produce emotional arousal during customizing the homework~\cite{}.





%Also, Smith et al. examined how therapists invited clients to document and record three mindful moments of intention throughout their week as homework in their journals~\cite{smith2019visual}.


%介绍art therapy homework目前的挑战。
    %  尽管therapy homework 在art therapy 中发挥着重要作用,但是目前art therapy homework 面临需要挑战。
    % 首先,客户需要在没有直接支持的情况下参与实践,这可能会引发负面信念并导致实际障碍
    % 其次,不同的情境下使用- 便利性: 由于家庭作业将治疗工作延伸到客户的自然环境中,另外的一个挑战是客户遇到不同的社会环境和地理环境的时候对他们参与家庭作业构成了重大障碍,可能会导致客户和治疗师放弃家庭作业。
    % 最后,先前研究表明通过将家庭作业作为收集数据的机会,可以更帮助治疗师理解问题的本质,并且为客户更新设计家庭作业,促进疗程之间的连续性。但是,如何收集有效的全面的数据数据是目前面临的挑战。
    % 尽管有那么多挑战,但是在人机交互中很少研究支持如何支持针对治疗师和来访者需求的家庭作业。
    % In HCI, recent study investigated how therapists and clients tailor therapy homework to their client’s needs. The findings  identifed common challenges and adaptations that clients and therapists made to therapy homework.
    
%. Secondly, since therapy homework extends the therapeutic work into clients' natural environment, another challenge arises when different social and location contexts create significant barriers to completing the homework, which may lead both the client and the therapist to abandon the homework altogether~\cite{Oewel_2024}. Finally, previous research suggested that using homework as an opportunity to collect data can help therapists better understand the nature of issues and update homework assignments~\cite{kazantzis2007handbook,freeman2007use}. However, the challenge lies in how to effectively gather comprehensive and accurate information~\cite{kazantzis2007handbook}.
%While there are many challenges, few studies focus on how to support therapy homework to clients' and therapists' needs~\cite{Oewel_2024}.
%A recent study in HCI explored how therapists and clients customize therapy homework to meet individual client needs~\cite{Oewel_2024}. The findings highlighted common challenges~(e.g.,having low motivation) as well as the adjustments made by both clients and therapists.
%However, there is a lack of knowledge about the role of technology, particularly AI agents, in art therapy homework.

%Art therapy typically involves a series of individual or group sessions where therapists and clients collaborate on artistic tasks while interacting through verbal communication to help clients convey their emotions and enhance their mental health.
%In particular, art therapy can effectively support individuals with diverse and complex communication needs by using creative and visual methods to facilitate the expression of ideas, thoughts, and experiences \cite{Lazar_2018}.
%
%\textcolor{blue}{Introducing what is therapy homework and why it is necessary}

%Introducing successful examples on therapy homework
%\textcolor{blue}{Introducing challenges in therapy homework}
%challenge 1: client do homework without human guidance
%challenge 2: limited resources (Different materials have different expressive properties -> what a client can create depends on what the therapist provides and how they guide the client in using them \cite{Lazar_2018}.)

% ==============
\subsection{Digital Art Therapy and Human-AI Co-creation}
%提纲:
    % digital art therapy 的研究
        % 介绍目前有很多研究探索使用技术方式支持digital art therapy实践
        % 举一个详细的例子
        % 说明这种digital art therapy的好处
        % 尽管有这些好处,但是大部分研究关注在in-session art therapy
        % 我们还缺乏如何支持between session activities
    % 介绍human-ai co-creative 的方法
        % 在between session的其中一个挑战就是如何降低疗愈活动的门槛
        % 最近human-ai co-creative方法引入有潜力帮助来访者降低疗愈活动的门槛
        % 举一个music therapy的例子
        % 再举一个art therapy的例子
        % 但是,
Research has explored how digital technologies could enhance accessibility, engagement, and collaboration for in-session art therapy. Various digital tools, such as virtual reality~\cite{kaimal2020virtual}, online chat applications~\cite{collie2006distance, hankinson2022keeping}, digital art-making tools~\cite{darewych2015digital, choe2014exploration}, and specialized art therapy systems~\cite{cubranic1998computer,yilma2024artful}, have been examined for their therapeutic potential. 
For instance, Collie et al. developed a computer-supported distanced art therapy system that enables audio and visual communication in individual or group sessions~\cite{collie2002computer}. 
Digital art therapy has shown promise in bridging geographical gaps~\cite{levy2018telehealth, collie2017online}, increasing accessibility for clients facing stigma or disabilities~\cite{kim2023case}, boosting engagement in creative processes~\cite{levy2018telehealth}, and fostering therapeutic rapport~\cite{collie2002computer, orr2012technology}. 
However, most of these studies have focused on in-session therapy. While the benefits of digital tools for in-session therapy are well-documented by these studies, between-session scenarios remain under-explored. 

% Research has explored how digital technologies could support in-session art therapy to increase its the accessibility engagement. In art therapy, a growing body of research has explored the use of various types of digital technologies to support multiple aspects of therapeutic activities, such as virtual reality~\cite{kaimal2020virtual}, online chat applications~\cite{collie2006distance,hankinson2022keeping}, digital art-making tools~\cite{darewych2015digital,choe2014exploration}, and specialized art therapy systems~\cite{cubranic1998computer}.
% For example, Collie et al. proposed a computer-supported distanced art therapy system developed through a participatory design process that facilitates both audio and visual communication in individual or group therapy sessions~\cite{collie2002computer}.
% Prior studies have examined that digital art therapy has the potential to bridge geographical gaps~\cite{levy2018telehealth,collie2017online}, enhance service accessibility for clients facing stigma or disabilities~\cite{kim2023case}, boost engagement in creative processes~\cite{levy2018telehealth}, and foster the development of therapeutic rapport~\cite{collie2002computer,orr2012technology}.
% Although digital art therapy provides substantial benefits, most studies have primarily focused on its use during in-session therapy.


%In between-session activities, one of the challenges is how to lower the threshold and increase engagement in therapeutic activities.

Recent advancements in Generative Artificial Intelligence (GenAI) present promising opportunities for human-AI co-creative approaches which could lower the threshold to creative processes in therapeutic settings ~\cite{sun2024understanding, wan2024metamorpheus,jutte2024perspectives, du2024deepthink,liu2024he}. For instance, Sun et al. found that human-AI collaboration in music therapy can improve therapeutic efficiency, reduce the complexity of music creation, and increase client engagement~\cite{sun2024understanding}. Similarly, DeepThInk, an Generative Adversarial Network (GAN) based system, has shown that co-creative art-making with AI can lower the threshold for artistic expression in art therapy~\cite{du2024deepthink}. 
%However, there is still limited knowledge on how to effectively integrate human-AI co-creative approaches and conversational AI agents to support art therapy homework.
Nonetheless, these studies mainly encompassed therapeutic activities where therapists provided real-time guidance. While DeepThInk also suggested promise for asynchronous use, it only supported art-making without verbal expression. Despite this, human-AI co-creative approach holds potential for art therapy homework. Building on this, our work explores integrating this approach with conversational interaction to support art therapy homework.

% In recent years, the advancements in generative artificial intelligence (GenAI) offer promising opportunities for human-AI co-creative approaches to lower the creative threshold for therapeutic creation and increase therapeutic engagement~\cite{sun2024understanding,wan2024metamorpheus,du2024deepthink,liu2024he}.
% For example, Sun et al. explored the potential of human-AI collaboration in music therapy, finding that it can enhance therapeutic efficiency, lower the threshold of music creation, and boost client engagement~\cite{sun2024understanding}.
% Furthermore, DeepThInk is an AI-infused art-making system which has demonstrated the potential of using human-AI co-creative art-making to lower the threshold for art-making and increase creative expression in art therapy~\cite{du2024deepthink}.
% However, we have limited knowledge on how to effectively use integrated human-AI co-creative approaches and conversational AI agents to support art therapy homework.


%Moreover, few studies have investigated the use of human-AI co-creative approaches to support art therapy practices~\cite{du2024deepthink,liu2024he}.

%However,  these studies have yet to explore how to integrate multimodal AI agents to support art therapy in asynchronous scenarios, particularly in the context of therapy homework.






%================================================================================
% AI Agents for Mental Health 这个章节可能提到的RQ:
    % RQ1: what are the key factors that contribute to successful mental health therapy
    % RQ2: why do we need AI Agents
    
% 对于AI Agents for Mental Health这个章节可能重要的点
    % 什么因素/措施/行为对于Mental Health Therapy是重要的
    % - e.g. 鼓励的话语,及时反馈,等等
    % \item motivational messages and recommendations for mental health activities” could better support the tailoring of digital mental health interventions \cite{Oewel_2024}
    % AI Agent 给传统 Mental Health Therapy Session带来了什么变化
    % - e.g. 纸笔式的task参与度不高;client更喜欢对话式的task
\subsection{AI Agents for Mental Health}


%在人机交互领域,很少有研究探索如何ai agent支持艺术治疗家庭作业
%但是,有大量研究探索rule-based or retrieval-based conversational agents如何支持心理健康领域,例如xxx, xxx, xxx。
    % 举一个详细例子说明。
    % 这些研究提供了有价值的洞见在提供结构化指导
    %但是还是有局限,不能理解对话context等:由于技术限制,例如对语境的理解以及有限的语言理解,往往导致不自然或不相关的对话,从而降低了用户与这些聊天机器人互动的意愿
%大语言模型在心理健康的应用
    % LLM引进的好处
    % 最近LLM在心理健康应用的overview
    % 具体举例一个详细例子
    % 这些研究都提供了具体洞见关于引导表达以及提供情绪支持在没有治疗师指导下
% 治疗师与来访者的协作
    %但是现在很少研究探索ai agent如何支持治疗师与来访者的协作
    %有一些研究探索了ai agent可以总结和呈现对话记录从而支持治疗师
    %具体句一个例子
    %但是这些研究没有探索如何支持治疗师去动态定制 ai agent
    %在hci其他领域已经有研究探索如何修改对话流程和对话内容来定制 ai agent
    % 举例详细的例子
    %然而,我们缺乏理解 如何ai agent去mediate 异步协作来家庭作业场景下
Despite limited research on how AI conversational agents support art therapy homework, a substantial body of HCI research has explored the design of rule-based and retrieval-based conversational agents for various therapeutic techniques beyond art therapy. These studies have examined applications in areas such as self-disclosure and self-compassion~\cite{lee2020designing, park2021designing, lee2019caring}, problem-solving therapy~\cite{o2018suddenly, kannampallil2023effects}, expressive writing~\cite{park2021wrote}, mindfulness practices~\cite{seah2022designing, inkster2018empathy}, positive psychology~\cite{kannampallil2023effects, jeong2023deploying}, and cognitive behavioral therapy~\cite{fulmer2018using, fitzpatrick2017delivering, su2020analyzing}. For instance, Diarybot used conversational agents to facilitate expressive writing, successfully encouraging participants to share emotions and stories~\cite{park2021wrote}. These studies highlight initial successes and provide valuable insights for designing AI conversational agents to support structured mental health interventions. However, technological limitations---such as challenges in understanding conversational context and restricted linguistic capabilities---can lead to unnatural or irrelevant interactions, reducing user engagement~\cite{ma2024evaluating,kim2024mindfuldiary}.

%In HCI, few studies have explored how AI conversational agents support art therapy homework.
% However, a substantial body of HCI research has explored the design of rule-based and retrieval-based conversational agents for various therapeutic techniques beyond art therapy. 
% These studies have investigated applications in self-disclosure and self-compassion~\cite{lee2020designing, park2021designing, lee2019caring}, problem-solving therapy~\cite{o2018suddenly, kannampallil2023effects}, expressive writing~\cite{park2021wrote}, mindfulness practices~\cite{seah2022designing, inkster2018empathy}, positive psychology skills~\cite{kannampallil2023effects, jeong2023deploying}, and cognitive behavioral therapy~\cite{fulmer2018using, fitzpatrick2017delivering, su2020analyzing}.
% For example, Diarybot was designed with two versions of conversational agents to assist participants in expressive writing, and the user study found that these agents can mediate the social sharing of emotions, encouraging participants to share their stories~\cite{park2021wrote}.
% These studies provide the initial successes and offer valuable insights into the design of conversational AI agents to support structural guidance in mental health.
% Due to technological limitations, such as difficulties in understanding conversational context and limited linguistic capabilities, interactions with conversational agents can often become unnatural or irrelevant, leading to reduced user engagement~\cite{ma2024evaluating,kim2024mindfuldiary}.

The recent advancements in large language models~(LLMs) have resulted in remarkable breakthroughs in developing dialogue systems that are more naturalistic and adaptive~\cite{bae2022building,hamalainen2023evaluating}. 
Recent studies in HCI have explored LLMs were leveraged to support various aspects of mental health, such as promoting cognitive reframing of negative thoughts~\cite{sharma2024facilitating, sharma2023cognitive}, facilitating sharing of emotions and experiences~\cite{seo2024chacha}, enhancing mindfulness~\cite{kumar2023exploring}, providing ad-hoc mental health support for specific groups~\cite{ma2024evaluating}, and enabling context-aware journaling~\cite{nepal2024contextual,10.1145/3699761}.
For instance, ChaCha, an LLM-driven conversational system, tracks conversation context to help children express their stories and emotions while identifying their feelings related to positive and negative events~\cite{seo2024chacha}. These studies offer valuable insights into how LLM-based conversational agents can facilitate deep self-disclosure and emotional support without direct therapist involvement.

Less studies explored how AI agents can support asynchronous client-practitioner collaboration in health-related domains. Some have examined AI's role in summarizing and presenting important health information from users' dialogue history to assist therapists~\cite{yang2024talk2care, kim2024mindfuldiary,10.1145/3659604}. For example, MindfulDiary, a LLM-powered journaling app, helps psychiatric patients document daily experiences and provides a clinician dashboard for practitioners to review entries~\cite{kim2024mindfuldiary}. 
However, these studies have not yet addressed how therapists can dynamically tailor AI agents to better meet their clients' mental health needs.

Yet outside of health domains, there is growing interest in designing no-code tools that allow users to customize conversational agents by creating personalized dialogue flows and styles~\cite{zheng2023synergizing, ha2024clochat, hedderich2024piece, Bhattacharjee2024}. For instance, Michael et al. designed a no-code chatbot design tool that lets users modify conversation flows to assist in bystander education~\cite{hedderich2024piece}. Yet, understanding how to customize AI agents for art therapy and mediate asynchronous therapist-client collaboration surrounding therapy homework remains underexplored, which has motivated our study.



% Nevertheless, the above research has not yet explored how AI agents can support client-practitioner asynchronous collaboration in healthcare.
% Several studies have explored how AI agents can support therapists by summarizing and presenting important health information based on 
% users' dialogue history~\cite{yang2024talk2care,kim2024mindfuldiary}.
% For instance, MindfulDiary is an LLM-driven journaling app which can help psychiatric patients document their daily experiences through conversation and can support practitioners in reviewing journal entries through a clinician dashboard~\cite{kim2024mindfuldiary}.
% However, these studies have not yet addressed how to support therapists in dynamically tailoring AI agents to better assist with their clients' mental health.
% In other fields of HCI, many studies have explored how to design no-code tools that enable users to customize conversational agents by designing personalized dialogue flows or conversational styles~\cite{zheng2023synergizing,ha2024clochat,hedderich2024piece,Bhattacharjee2024}.
% For example, Michael et al. explored the design of a no-code chatbot tool based on LLM chains that allows users to customize conversation flows and responses to assist in bystander education~\cite{hedderich2024piece}.
% However, a relevant understanding is yet to be established on how \name{} can mediate asynchronous therapist-client collaborationsurrounding art therapy homework.




% \textcolor{blue}{introducing human-AI co-creation for different therapy purposes. Further, introducing studies for human-AI co-creation for art therapy, give an example}

%Meanwhile, human-AI co-creation has emerged as a powerful tool across various therapeutic applications, offering innovative ways to help clients recognize and develop emotional intelligence~\cite{santos2020therapistvibe, sharma2023humanai}, foster communication skills~\cite{ashktorab2021effects} and improve self-care habits~\cite{nie2022conversational, chiu2024computational}. 

%Furthermore, with the assistance of AI agents, therapists can focus more on observing how clients create their work~\cite{seo2022toward} as well as designing therapy plans for more complex environments~\cite{feng2024codesigning}. 
%In the field of art therapy, recent works have fully leveraged human-AI co-creation in drawing~\cite{kim2022colorbo, rosenberg2024drawtalking, lawton2023when}, creative writing~\cite{gero2019metaphoria}, music-making~\cite{sun2024understanding, feng2024codesigning} and tool-making~\cite{jeon2021fashionq} to fulfill different therapy purposes. However, creating such art often requires timely professional guidance from therapists or experts to match with therapeutic goals. It remains underexplored how to design a human-AI co-creation process in digital art therapy that enables clients to independently present insightful outcomes.

% \begin{itemize}
%     \item Allow therapists to have more time to observe how clients create their work
%     \item Digital technologies can help therapists design therapy plans for more complex environments (Co-designing the Collaborative Digital Musical Instruments for Group Music Therapy)
%     \item However, "Different materials have different expressive properties -> what a client can create depends on what the therapist provides and how they guide the client in using them" expressive nature of art therapy
% \end{itemize}


%In addition, Kim et al. introduced an AI-based expert support system that automatically analyzes and converts client's input into a visualized dashboard to expedite sketch-based drawing processes~\cite{jones2014supporting,levy2018telehealth}, nurturing the development of therapeutic bonds~\cite{collie2002computer,orr2012technology}. 