\section{Related Work}
\label{sec:related}
Here we review the related works on graph analysis under differential privacy. 

\noindent
{\bf Graph analysis under differential privacy.}
% DP widely applied to graphs. 
Differential privacy is widely adopted for privacy-preserving graph analysis, including releasing degree distributions {\cite{hay2009boosting, hay2009accurate, day2016publishing, macwan2018node}}, 
{common neighbor count distribution \cite{lv2024publishing}}, 
$k$-star counting \cite{nissim2007smooth, karwa2011private}, {triangle counting \cite{ ding2018privacy, imola2021locally,lv2021publishing, imola2022communication, imola2022differentially}}, and core decomposition \cite{dhulipala2022differential}, and {graph learning \cite{sajadmanesh2021locally, lin2022towards, wu2022linkteller, zhu2023blink, wei2024poincare}}. 
% Some approaches adopt {\em central differential privacy} \cite{nissim2007smooth, hay2009boosting, karwa2011private, zhang2015private, lv2024publishing}, where a trusted data curator can access the whole graph. 
% However, if the central data curator is corrupted or hacked, the privacy of all users is breached \cite{imola2021locally, jiang2021applications}.
Some approaches adopt {\em central differential privacy} \cite{nissim2007smooth, hay2009boosting, karwa2011private, zhang2015private, lv2024publishing}, where a trusted curator can access the entire graph. However, if this curator is compromised, all users' privacy is at risk \cite{imola2021locally, jiang2021applications}. 
% central models face challenges in finding a trusted third party. 
%%% Local model
{Another line of research adopts {\em local differential privacy} \cite{qin_generating_2017, wei2020asgldp, imola2021locally, imola2022communication, eden2023triangle,liu2022collecting,  sun2019analyzing, imola2022differentially}. }
Two main paradigms exist for graph analysis under LDP: 
(1) {general-purpose synthetic graph construction \cite{qin2017generating, zhang2018two, gao2018local, ju2019generating, liu2020privag, ye2020lf,hou2023ppdu}} and 
(2) problem-specific algorithmic design. 
The former often suffers from low data utility due to the loss of graph structure. 
Under the second category, many works are devoted to motif counting. 
%%% triangle counting 
\cite{imola2021locally} introduces one-round and two-round algorithms for triangle counting under edge LDP, while \cite{imola2022communication} improves communication cost and estimation error. 
\cite{eden2023triangle} offers an in-depth technical analysis of these algorithms. 
\cite{liu2022collecting} and \cite{sun2019analyzing}  study triangle counting in the localized setting with extended local views. 
%%% crypto assisted: 
{\cite{liu2023cargo} attempts to improve data utility for triangle counting under edge LDP in a crypto-assisted manner.} 
\cite{sun2019analyzing} also addresses three-hop paths and k-cliques on small $k$ values. 
%%% cycle counting under the shuffle model. 
\cite{imola2022differentially} estimates the 4-cycle and triangle counts under the shuffle model, where users' messages are shuffled before being sent to the data curator. 
% In a recent study \cite{imola2022differentially}, the 4-cycle and triangle counts are estimated using the shuffle model, wherein users' messages undergo shuffling before reaching the data curator.
% This model relies on the additional assumption that the data curator and the shuffler do not collude. 
% The loss is purely additive, and the additive losses match or improve upon the best-known previous additive loss in any version of differential privacy when 1/δ is polynomial in n
% derives approximate algorithms for 
% In addition, there are also works devoted to core decomposition, degree 
\cite{sun2024k} proposes k-star LDP to addresses differentially private $(p,q)$-biclique counting over bipartite graphs. Specifically, each vertex reports its perturbed k-star neighbor lists instead of the classic edge neighbor lists to the data curator. 
In addition, \cite{dhulipala2022differential} studies core decomposition under edge LDP, leading to approximate solutions for densest subgraph discovery. 
\cite{dinitz2023improved} further proves a purely additive loss for the densest subgraph problem under edge LDP. 
{A recent work \cite{lv2024publishing} studies publishing the histogram of common neighbor counts under the centralized model, which differs from our setting. }
% between all vertex pairs in a social network 
% Unlike our work, they focus on publishing the distribution of common neighbor counts under the centralized model.
% A recent work \cite{lv2024publishing} studies publishing histograms of common neighbor counts with edge-differential privacy, focusing on the centralized model. 
% \cite{sun2019analyzing} studies k-clique counting under a more ``general'' local differential privacy. 


% CDP:
% \cite{hay2009accurate} presents an efficient algorithm for releasing an estimate of a network's degree distribution under DP. 



%% LDP:
% \cite{wei2020asgldp} AsgLDP is a novel technique for generating decentralized, privacy-preserving attributed social graphs under local differential privacy (LDP)


%% graph learning 
% \cite{sajadmanesh2021locally} proposes a set of mechanisms are proposed
% to protect the privacy of node features. Specifically, LPGNN
% assumes that a central server holds the global graph topology,
% and the server is allowed to collect the node features satisfying
% local differential privacy
% \cite{qin2017generating} proposes LDPGen, a multi-phase technique that ensures local differential privacy while collecting structural information and generating representative synthetic social graphs from decentralized social graphs, addressing limitations of existing methods by incrementally clustering users based on connections and optimizing noise injection, leading to improved preservation of important graph properties. (it combines RNL and degree-based graph synthesis). 
% \cite{zhang2018two} studies how to generate synthetic graphs under Local Differential Privacy, and their LDPGM can effectively control the density of the synthetic graph significantly reducing the error between the synthetic graph and the original graph. 
% \cite{zhang2018two} studies how to generate synthetic graphs under Local Differential Privacy. 
% \cite{gao2018local} proposes a group-based local differential privacy scheme that utilizes hierarchical random graph models to reduce noise and enhance privacy in online social networks while maintaining the utility of released graphs. 
% \cite{ye2020lf} proposes LF-GDPR, a graph metric estimation framework, which collects the adjacency bit vector and node degree based on the target graph statistic (e.g., Clustering Coefficient, Modularity Estimation). 
% a framework that simplifies the implementation of local differential privacy (LDP) for graph analysis tasks, addressing complexity and low data utility, while ensuring privacy preservation. 
% (it splits the privacy budget into e1 and e2. It uses e1 to perturb adjacency lists and uses e2 to perturb the node degree. ) % Interestingly, in this work, it also discusses how to split the privacy budget for the downstream task. It adopts the RABV protocol, which in essence perturbs one and only one bit for each pair of symmetric bits in the adjacency matrix. This is equivalent to the handling in \cite{imola2021locally} where RNL is applied to the lower/upepr triangle of the adjacency matrix. 
%% decentralized DP: considers extended local view addresses triangles, three-hop paths, and k-cliques counting. 







% \cite{chen2022ldp} studies data aggregation in the context of local differential privacy (LDP). It specifically addresses the problem of frequent itemset mining (FIM) in LDP, which is the task of identifying sets of items that frequently co-occur in a dataset while preserving privacy. The paper proposes a new approach called LDP-FPMiner that utilizes the concept of a frequent pattern tree (FP-tree) to achieve FIM in LDP -> we do not consider this paper because it is not on graphs