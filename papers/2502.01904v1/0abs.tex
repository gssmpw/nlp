\begin{abstract}
\label{abs}
Bipartite graphs, formed by two vertex layers, arise as a natural fit for modeling the relationships between two groups of entities. In bipartite graphs, common neighborhood computation between two vertices on the same vertex layer is a basic operator, which is easily solvable in general settings. However, it inevitably involves releasing the neighborhood information of vertices, posing a significant privacy risk for users in real-world applications. To protect edge privacy in bipartite graphs, in this paper, we study the problem of estimating the number of common neighbors of two vertices on the same layer under edge local differential privacy (edge LDP). The problem is challenging in the context of edge LDP since each vertex on the opposite layer of the query vertices can potentially be a common neighbor. To obtain efficient and accurate estimates, we propose a multiple-round framework that significantly reduces the candidate pool of common neighbors and enables the query vertices to construct unbiased estimators locally. Furthermore, we improve data utility by incorporating the estimators built from the neighbors of both query vertices and devise privacy budget allocation optimizations. These improve the estimator's robustness and consistency, particularly against query vertices with imbalanced degrees. Extensive experiments on 15 datasets validate the effectiveness and efficiency of our proposed techniques.
\end{abstract}