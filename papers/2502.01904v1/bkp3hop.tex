% \section{3 hop neighborhood exploration}


\section{ Extending to k-hop neighbors }

If we want to find $f_3(q, x)$, then it is unfortunate that we cannot leverage the $f(q, x)$, the number of two-hop paths, for this task. Because we cannot just aggregate the $f_2$ of the neighbor of $q$ and expect $f_3$, which results in duplicates. 

If we directly consider a query-vertex-centric approach for $f_3$ estimation. 

$$
\Tilde{f_3}(q, x) = \sum_{u \in N(q, G)} \sum_{ \substack{v \neq u \\ v \neq q \\ v \neq x   }} \phi'(u,v) \phi'(v,x)
$$

This is also unbiased. Its global sensitivity however becomes: 

$$
\Delta(\Tilde{f}_3) \leq \sum_{ \substack{v \neq u \\ v \neq q \\ v \neq x   }} \phi'(u,v) \phi'(v,x)
\leq (\frac{1-p}{1-2p})^2(n-3)
$$

The variance becomes impossible to compute due to the correlations of two-hop paths. 




\section{Experimental evaluation}
\label{sec:exp}




Algorithms: 

\begin{enumerate}
\item the naive algorithm: directly compute f vector on the noisy graph 
\item baseline algorithm: one-round algorithm
\item adv algorithm: two-round algorithm
\end{enumerate}


\noindent
{\bf Parameters.} 
Privacy budget epsilon. For two-round algorithms, we can visualize the effect of different privacy budget allocation. 


\noindent
{\bf Performance metric.} 
\begin{enumerate}
    \item Given a query vertex q, we compute the L1-loss of the f2 vector. 
    \item Given a query q, we compute the coverage of two-hop neighbours of the vertices with the top-K f2-values. 
    (the second one don't even need to be done)
\end{enumerate}



\noindent
{\bf Experiments.} 


\begin{enumerate}
    \item Default queries: 
For each graph, randomly sample 100 query vertices.  Then, we compute the f2 vector. Based on this, we compute the L1-loss or L2-loss. We report the average loss across the sampled vertices. 

\item The time and communication cost of different algorithms. 

\item Effect of privacy budget 
\begin{itemize}
    \item The effect of varying privacy budget on Naive, BS, and ADV on the L1-loss of f2-vector 
    % issue: how do we evaluate the relative error when the true value is zero? 
    % \item The effect of varying privacy budget on the three algorithms on the two-hop neighbor coverage. 
\end{itemize}
% (Issue: how do we determine the number of vertices to take? The size of K?)
% Is there a way for us to predict the size of the two-hop neighborhood? 

\item Evaluate of the effect of query vertex degree 
Sample high-degree and low-degree vertices.  Observe and explain the performance difference. 

\item 
 For two rounds: we evaluate the effectiveness of our privacy budget allocation technique. 
We can find that the budget allocation plan computed is indeed minimising the variance. 

\end{enumerate}





\subsection{Application scenarios}

\noindent
{\bf Papers directly using one-mode projection. }
Polarity‑based graph neural network for sign prediction in signed bipartite graphs \cite{zhang2022polarity}, Bipartite graph capsule network \cite{zhang2023bipartite}, ``Efficient High-quality Clustering for Large Bipartite Graphs'' \cite{yang2023efficient}. These papers directly project bipartite graphs into a homogeneous graph with edge weight. The weight is computed as the number of two-hop-paths between two vertices on the same layer. 


% Efficient High-quality Clustering for Large Bipartite Graphs (edge weighted)
\noindent
{\bf Papers intra-domain message passing: }
L-BGNN: Layerwise Trained Bipartite Graph Neural Networks \cite{xie2022bgnn}. 


\noindent
{\bf Papers using number of two-hop paths to model node similarity:}
``Scalable and effective bipartite network embedding'' \cite{yang2022scalable}. 
For u and x, it considers all paths of even length between them to formulate node proximity. We know that the number of even length path can be obtained once the length-2 paths are obtained. 

\noindent
{\bf Papers using two-hop neighbors on non-GNN tasks:}
biclique counting \cite{qiu2024accelerating}

% Qiu, Linshan, et al. "Accelerating Biclique Counting on GPU." arXiv preprint arXiv:2403.07858 (2024).
% biclique counting: tau-strengthened two-hop neighbours. Arguably will benefit pqbiclique counting under edge LDP





% $\phi_3(q, x) = 1$ if there exists a path of length 3 from $q$ to $x$, and 0 otherwise.

% $$
% \phi_k(q, x) = \begin{cases} 
% 1 & \text{there exists a path of length $k$ from } q \text{ to } x \\
% 0 & \text{otherwise}
% \end{cases}
% $$

% $$
% \phi_3(q, x) = 1 - \prod_{v_1, v_2}(1 - y_3(q, v_1, v_2, x))
% $$

% Equivalently, using logical operators:
% $$
% \phi_3(q, x) = \neg \left( \bigwedge_{v_1, v_2} \neg y_3(q, v_1, v_2, x) \right)
% $$

% We explore non-interactive and interactive solutions for this. 

% When $k = 3$, the above approaches present some difficulties. 
% First, we no longer have the formula for the expected value of $\phi_3$ any more. Even if we have unbiased estimates of all $y_3(q, v_1, v_2, x)$, we cannot use them to compute $\phi_3(q, x)$ because these paths of length $3$ overlap with each other, causing them to be not independent. When these terms are not independent, we cannot substitute the formula for unbiased estimates of $y_3$ and yield the estimator for $\phi_3$. 

% {\color{blue}
% If we directly substitute $y$ into the formula, we can argue that the expected value of the products is larger than the product of the expected values, given that the variables are positively correlated. 
% }


% The absolute values of $\phi_2$ and $\phi_3$ has the following relationship: 
% $$
% \phi_3(q, x) = 1 - \prod_{x' \in N(x, G)} (1- \phi_2(q, x'))
% $$



% Question: if we have a good estimate of $\phi_2(q, x)$, can we use it to estimate $\phi_3(q, x)$ in an unbiased manner? 
