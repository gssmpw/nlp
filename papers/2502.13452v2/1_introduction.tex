\section{INTRODUCTION}
\label{sec:intro}

Over the past decade, \ac{LiDAR}-based mapping has significantly advanced \cite{zhang2014loam, shan2018lego, shan2020lio, xu2022fast}, increasing the demand for long-term deployment of such systems in various fields, including urban areas or construction sites \cite{lee2024lidar}. 
These environments are inherently dynamic; objects frequently move, and layouts change. 
To handle these dynamics, continuously revisiting and maintaining the map of the environment---lifelong mapping---is required.

LiDAR-based lifelong mapping has gained interest relatively recently compared to the visual domain \cite{09towards, cramariuc2022maplab, elvira2019orbslam}.
Long-term mapping pipelines for static map \cite{pomerleau2014long} or semantic map \cite{sun2018recurrent} construction were suggested, but the standard framework for lifelong mapping was absent.
Recently, the modular lifelong mapping frameworks \cite{kim2022lt, yang2024lifelong} were suggested. 
They process the session---a set of point clouds and poses---as an input and focus on the \textit{inter-session changes} for efficient map management and incremental update.

These changes have been modeled as binary (\ie, appearing or disappearing), leading to a binary classification of map elements (\ie, static or dynamic). 
The inherent limitation of this approach is its inability to differentiate between long-term gradual changes and short-term ephemeral variations. 
An example is illustrated in \figref{fig:figure_1}, where two objects appear on the new map: one represents a persistent change (new walls), and the other reflects a relatively short-term variation (parked cars).
Unfortunately, both are categorized into a single static class in existing methods \cite{kim2022lt, yang2024lifelong}; yet, our approach can distinguish them based on their ephemerality score. 

The key idea is that changes in real-world are gradual. Detailing changes beyond binary categorization, we introduce \textit{ephemerality} as the core concept of our lifelong mapping framework.
Ephemerality represents the likelihood of a point being transient or persistent. 
Previous literature \cite{09towards, mcmanus2013distraction, barnes2018driven} has commonly treated ephemeral objects the same as dynamic ones (\eg, pedestrians, moving vehicles) in short-term contexts. 
In this paper, we extend the focus to long-term perspectives, elaborating dynamic objects with details from ephemeral variation to gradual map evolution.

%FIGURE
\begin{figure}[!t]
    \centering
    \includegraphics[width=0.95\linewidth]{figs/fig1_0917_final_final_compressed.pdf}
    \caption{
    An example scene from three \texttt{KAIST} sequences \cite{kim2020mulran, jung2023helipr} with newly appeared walls and parked cars. 
    Representing changes in a simple binary manner, existing methods treat both the car and the wall as static objects. 
    Our proposed system leverages two-stage ephemerality to differentiate parked cars as ephemeral objects and walls as persistent changes based on their ephemerality scores.
    }
    \label{fig:figure_1}
    \vspace{-5mm}
\end{figure}
%FIGURE

This paper builds on the modular framework of LT-mapper \cite{kim2022lt}.
Unlike previous approaches \cite{kim2022lt, yang2024lifelong} with three independent modules, ours facilitates seamless integration of each module with ephemerality, which permeates the entire pipeline and enhances both per-module and overall performance.
In doing so, we infer a two-stage ephemerality with different time scales to express the subtle differences.
This allows us to represent changes between sessions in a more fine-grained manner than traditional binary approaches. 
In addition, leveraging spatiotemporal context, we can accurately distinguish meaningful changes from those resulting from errors and use them for effective map updates. 
The contributions of our system are as follows.
\begin{itemize}
    \item We introduce a two-stage ephemerality concept---local and global---to capture short-term and long-term changes, respectively.
    This approach extends beyond binary static/dynamic classification by distinguishing truly persistent changes from transient variations.

    \item
    We propose ELite, a LiDAR-based lifelong mapping framework that incorporates ephemerality into each module. 
    Our approach uses ephemerality to guide map alignment, prioritize meaningful map updates, and robustly detect evolving structures over time.

    \item 
    ELite maintains three types of maps: \textit{a lifelong map} capturing spatiotemporal history, an adjustable \textit{static map} filtering out ephemeral clutter, and an object-oriented \textit{delta map} highlighting changed components. These representations enable flexible usage based on different requirements and time horizons.

    \item Each module within ELite has been thoroughly evaluated, showing superior performance compared to the baselines. All codes and related softwares are open-sourced for the community.
\end{itemize}