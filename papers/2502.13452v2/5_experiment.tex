\section{EXPERIMENT}
\label{sec:experiment}

\subsection{Experimental Setup}
We conduct both quantitative and qualitative evaluations for each module of our system.

\textbf{Multi-session Map Alignment.}
We use six sequences from \texttt{LT-ParkingLot} \cite{kim2022lt} and MulRan \cite{kim2020mulran} (\texttt{DCC01-03} and \texttt{KAIST01-03}), as each provides multiple overlapping routes recorded at sufficiently spaced time intervals. Each first session is used to build the base map; the remaining sessions serve as new data for alignment. We employ SC-LIO-SAM \cite{shan2020lio, kim2021scan} for mapping each session.  
Following previous works \cite{hu2024paloc, yang2024lifelong}, we evaluate alignment using Accuracy (AC), RMSE, and Chamfer Distance (CD). 
In detail, we establish point correspondences between two point clouds using the nearest neighbor search and get the inlier set using a distance threshold $\sigma_{\text{inlier}}$ (set to 0.5$\m$).
AC measures the ratio of inlier pairs, RMSE is their root mean squared distance, and CD is the bidirectional sum of average inlier distance.
As baselines, we compare against ICP-based map-to-map registration \cite{girardeau2016cloudcompare} and multi-session PGO in LT-mapper \cite{kim2022lt}.

\textbf{Dynamic Object Removal.} 
Following \cite{lim2021erasor, jia2024beautymap, duberg2024dufomap}, we evaluate three different sequences from the SemanticKITTI \cite{behley2019semantickitti} dataset, adopting the protocol in \cite{zhang2023dynamic}. We use Preservation Rate (PR) and Removal Rate (RR) \cite{lim2021erasor} at the point level without downsampling the ground truth map to ensure accuracy. We also report the F1 score, the harmonic mean of PR and RR.  
Baselines include state-of-the-art methods such as Removert \cite{kim2020remove}, ERASOR \cite{lim2021erasor}, DUFOMap \cite{duberg2024dufomap}, and BeautyMap \cite{jia2024beautymap}.

\textbf{Map Update.}
Since no labeled dataset exists for inter-session changes, we conduct a qualitative evaluation using the MulRan \cite{kim2020mulran} (\texttt{KAIST01, 02}) and HeLiPR \cite{jung2023helipr} (\texttt{KAIST04, 05}) datasets. With a four-year gap between these datasets, we can observe both short-term and long-term changes.

\subsection{Multi-session Map Alignment}

%%%%%%%%% Table 1 %%%%%%%%%
\begin{table}[!t]
\centering
\caption{Map alignment evaluation results}
\begin{adjustbox}{width=1.0\linewidth}
{
\begin{tabular}{clccc}
\toprule
\multirow{2}{*}{Sequence}       & \multicolumn{1}{c}{\multirow{2}{*}{Method}} & \multicolumn{3}{c}{Metrics} \\  \cmidrule{3-5} 
                                & \multicolumn{1}{c}{}                        & AC $\uparrow$   & RMSE $\downarrow$     & CD $\downarrow$      \\
\midrule
\multirow{3}{*}{\texttt{LT-ParkingLot}} 
                                & ICP \cite{girardeau2016cloudcompare}         & 0.962    & 0.117    & 0.194 \\  
                                & LT-mapper \cite{kim2022lt}                   & 0.968    & 0.121    & 0.175 \\
                                & ELite (Ours)                                 & \textbf{0.969}    & \textbf{0.090}    & \textbf{0.133} \\
\midrule
\multirow{3}{*}{\texttt{DCC}}            
                                & ICP \cite{girardeau2016cloudcompare}         & 0.692     & 0.204    & 0.272 \\
                                & LT-mapper \cite{kim2022lt}                   & 0.738     & 0.182    & 0.306 \\
                                & ELite (Ours)                                 & \textbf{0.942}     & \textbf{0.111}    & \textbf{0.162} \\
\midrule
\multirow{3}{*}{\texttt{KAIST}}          
                                & ICP \cite{girardeau2016cloudcompare}         & 0.641     & 0.218    & 0.256 \\
                                & LT-mapper \cite{kim2022lt}                   & 0.909     & 0.169    & 0.315 \\
                                & ELite (Ours)                                 & \textbf{0.963}     & \textbf{0.120}    & \textbf{0.184} \\
\bottomrule
\end{tabular}
}
\end{adjustbox}
\tiny{ }

\footnotesize
\raggedright
\quad Best performance in \textbf{bold}.
\label{tab:map_eval}
\vspace{-4mm}
\end{table}

%%%%%%%%%%%%%%%%%%%%%%%%%%%

\begin{figure}[!t]
    \centering
    \subfigure[LT-mapper \cite{kim2022lt}]{        
        \includegraphics[width=0.45\linewidth]{figs/map_align_ltmapper_compressed.pdf}
        \label{fig:map_align_ltmapper}
    }
    \subfigure[Ours]{        
        \includegraphics[width=0.45\linewidth]{figs/map_align_ours_compressed.pdf}
        \label{fig:map_align_ours}
    }
    % \vspace{-1mm}
    \caption{Qualitative comparison of multi-session map alignment on the \texttt{DCC} sequence in MulRan \cite{kim2020mulran}. Both methods exhibit globally consistent alignment, but our method demonstrates superior local consistency.}
    \label{fig:map_align}
    \vspace{-2mm}
\end{figure}

\begin{figure}[!t]
    \centering
    \subfigure[GICP \cite{koide2021voxelized}]{        
        \includegraphics[width=0.45\linewidth]{figs/gicp3.pdf}
        \label{fig:gicp}
    }
    \subfigure[Ours]{        
        \includegraphics[width=0.45\linewidth]{figs/weighted_gicp3.pdf}
        \label{fig:weighted_gicp}
    }
    % \vspace{-1mm}
    \caption{Qualitative comparison on the \texttt{KAIST} sequence, showing the advantage of using $\epsilon_g$ as a weight for scan-to-map registration. By prioritizing long-term static structures, our method robustly aligns sessions even with significant time gaps or dynamic objects.}
    \label{fig:icp_comparison}
    \vspace{-5mm}
\end{figure}

\tabref{tab:map_eval} shows our map alignment performance, where our method consistently outperforms the baselines. While baselines can perform reasonably well in small-scale settings such as \texttt{LT-ParkingLot}, it struggles to register large-scale environments such as \texttt{DCC} and \texttt{KAIST}.  
\figref{fig:map_align} compares two session maps in the \texttt{DCC} sequence (\texttt{01} and \texttt{02}). LT-mapper \cite{kim2022lt}, which leverages multi-session PGO, achieves globally consistent but locally misaligned results. In contrast, our bidirectional registration yields both global and local consistency.  
\figref{fig:icp_comparison} shows the effectiveness of weighting scan-to-map registration with $\epsilon_g$. By assigning higher weights to static structures, our alignment module accurately registers sessions separated by huge time intervals or containing many dynamic objects.

\subsection{Dynamic Object Removal}

%%%%%%%%% Table 2 %%%%%%%%%
\begin{table}[t!]
\centering
\caption{Dynamic object removal results on SemanticKITTI}
\begin{adjustbox}{width=1.0\linewidth}
{
\begin{tabular}{c|l|rrc|rrc}
\toprule[0.8pt]
\multicolumn{1}{l|}{\multirow{2}{*}{Seq}} & \multicolumn{1}{c|}{\multirow{2}{*}{Method}}  & \multicolumn{3}{c|}{SuMa \cite{behley2019semantickitti,behley2018efficient}} & \multicolumn{3}{c}{KITTI Poses \cite{geiger2013vision}} \\ \cmidrule{3-8}
                                             &                                               & \multicolumn{1}{c}{PR $\uparrow$} & \multicolumn{1}{c}{RR $\uparrow$} & F1 $\uparrow$ & \multicolumn{1}{c}{PR $\uparrow$} & \multicolumn{1}{c}{RR $\uparrow$} & F1 $\uparrow$ \\
\midrule[0.6pt]
\multirow{6}{*}{\texttt{00}} & Removert \cite{kim2020remove}   & 99.55 & 41.14 & 58.22  & 99.24 & 41.42 & 58.44           \\
                    & ERASOR \cite{lim2021erasor}    & 70.23 & 98.49 & 81.98   & 65.99 & 98.32 & 78.98                    \\
                    & DUFOMap \cite{duberg2024dufomap}    & 98.63 & 98.66 & \textbf{98.64}  & 92.59 & 98.47 & 95.44           \\
                    & BeautyMap \cite{jia2024beautymap}   & 97.13 & 97.79 & 97.46 & 97.07 & 97.84 & \textbf{97.45}           \\
                    & ELite - 0.2 (Ours) & 97.30 & 98.74 & 98.02 & 93.22 & 98.55 & 95.81        \\
                    & ELite - 0.5 (Ours) & 98.54 & 98.28 & \underline{98.41} & 96.61 & 97.93 & \underline{97.27}        \\ 
                    \midrule
\multirow{6}{*}{\texttt{01}} & Removert \cite{kim2020remove}   & 98.27 & 39.47 & 56.32  & 98.43 & 39.85 & 56.73                    \\
                    & ERASOR \cite{lim2021erasor}    & 98.35 & 90.96 & 94.51    & 83.06 & 92.43 & 87.49                    \\
                    & DUFOMap \cite{duberg2024dufomap}    & 98.94 & 93.93 & \underline{96.37}  & 98.97 & 93.52 & 96.17        \\
                    & BeautyMap \cite{jia2024beautymap}    & 99.30 & 92.37 & 95.71 & 99.20 & 90.20 & 94.48                    \\
                    & ELite - 0.2 (Ours) & 94.70 & 97.84 & 96.24 & 95.18 & 97.86 & \underline{96.50}        \\
                    & ELite - 0.5 (Ours) & 96.72 & 96.52 & \textbf{96.62} & 97.15 & 96.62 & \textbf{96.89}  \\ 
                    \midrule
\multirow{6}{*}{\texttt{02}} & Removert \cite{kim2020remove}   & 96.69 & 35.26 & 51.68  & 97.18 & 35.34 & 51.83                    \\
                    & ERASOR \cite{lim2021erasor}      & 50.79 & 92.26 & 65.51  & 51.63 & 93.83 & 66.61                    \\
                    & DUFOMap \cite{duberg2024dufomap}    & 68.61 & 89.29 & 77.60  & 70.78 & 89.49 & 79.04                    \\
                    & BeautyMap \cite{jia2024beautymap}   & 83.43 & 84.66 & 84.04 & 82.07 & 90.61 & \underline{86.13}        \\
                    & ELite - 0.2 (Ours) & 80.71 & 91.21 & \underline{85.64} & 80.44 & 90.84 & 85.32        \\
                    & ELite - 0.5 (Ours) & 84.03 & 89.70 & \textbf{86.77} & 83.62 & 89.21 & \textbf{86.32}  \\ 
\bottomrule[0.8pt]
\end{tabular}
}
\end{adjustbox}
\tiny{ }

\footnotesize
\raggedright
Best performance in \textbf{bold}, second best \underline{underlined}.
\label{tab:semantic_kitti}
\vspace{-5mm}
\end{table}
%%%%%%%%%%%%%%%%%%%%%%%%%%%

\begin{figure}[!t]
    \centering
    \subfigure[Raw Map]{        
        \includegraphics[width=0.45\linewidth]{figs/result_gt_compressed.pdf}
        \label{fig:result_gt}
    }
    \subfigure[Removert \cite{kim2020remove}]{        
        \includegraphics[width=0.45\linewidth]{figs/result_removert_compressed.pdf}
        \label{fig:result_removert}
    } \\
    \subfigure[ERASOR \cite{lim2021erasor}]{        
        \includegraphics[width=0.45\linewidth]{figs/result_erasor_compressed.pdf}
        \label{fig:erasor}
    }
    \subfigure[DUFOMap \cite{duberg2024dufomap}]{        
        \includegraphics[width=0.45\linewidth]{figs/result_dufomap_compressed.pdf}
        \label{fig:result_dufomap}
    } \\
    \subfigure[BeautyMap \cite{jia2024beautymap}]{        
        \includegraphics[width=0.45\linewidth]{figs/result_beautymap_compressed.pdf}
        \label{fig:result_beautymap}
    }
    \subfigure[Ours]{        
        \includegraphics[width=0.45\linewidth]{figs/result_ours_compressed.pdf}
        \label{fig:result_ours}
    }
    % \vspace{-1mm}
    \caption{Qualitative comparison of dynamic object removal in the SemanticKITTI \cite{behley2019semantickitti} \texttt{01} sequence. The ground truth dynamic points in (a) are shown in red. Figures (b)-(f) depict the cleaned maps produced by each method, with red points indicating remaining dynamic objects.}
    \label{fig:result_dynamic_removal}
    \vspace{-5mm}
\end{figure}

\tabref{tab:semantic_kitti} and \figref{fig:result_dynamic_removal} illustrate the performance of various dynamic object removal methods. We report results for two thresholds of $\epsilon_l$: 0.2 and 0.5. Our approach at $\tau_{l} = 0.5$ achieves the highest or comparable performance on most sequences. Lowering the threshold to $\tau_{l} = 0.2$ significantly increases RR with only a moderate sacrifice in PR, showing the flexibility to prioritize either aggressive removal of dynamic objects or better preservation of static points.

We also evaluate two different pose estimation sources: KITTI poses \cite{geiger2013vision} and SuMa poses \cite{behley2018efficient} from SemanticKITTI \cite{behley2019semantickitti}. As noted in \tabref{tab:semantic_kitti}, all methods perform worse when using KITTI poses, indicating the importance of accurate pose estimation. Spatial quantization-based methods degrade further with less accurate poses, while our point-wise iterative update remains comparatively robust.

\subsection{Map Update}

\begin{figure*}[!t]
    \centering
    \includegraphics[width=0.95\linewidth]{figs/fig9_0217_compressed.pdf}
    \caption{Sample scene from the \texttt{KAIST} sequences of MulRan \cite{kim2020mulran} and HeLiPR \cite{jung2023helipr}. The red box highlights short-term changes, such as parked cars whose ephemerality increases over time. The blue box highlights long-term changes, including a newly constructed building between \texttt{02} and \texttt{04}. While its initial ephemerality is slightly high, indicating potential transience, it decreases as more updates confirm its permanence.}
    \label{fig:eph_change}
    \vspace{-5mm}
\end{figure*}

\figref{fig:eph_change} shows the results of our map update module. The top row depicts the evolving lifelong map over several sessions. After multiple updates, transient objects like parked cars (red box) exhibit increasing ephemerality, aligning with their dynamic nature. Meanwhile, newly built structures (blue box) see a gradual decrease in ephemerality, reflecting long-term permanence. Thus, two-stage ephemerality propagation effectively distinguishes static from ephemeral objects.  
ELite can also produce a static map by filtering points whose $\epsilon_g$ is below a user-defined threshold $\tau_g$. A smaller $\tau_g$ yields a map retaining only long-term static structures, whereas a higher $\tau_g$ preserves moderately ephemeral objects (e.g., currently parked cars). This threshold acts as a convenient way to tailor the static map for specific needs.

\begin{figure}[!t]
    \centering
    \subfigure[Diff map \cite{kim2022lt}]{        
        \includegraphics[width=0.45\linewidth]{figs/density_a.pdf}
        \label{fig:density_a}
    }
    \subfigure[Delta map (ours)]{        
        \includegraphics[width=0.45\linewidth]{figs/density_b.pdf}
        \label{fig:density_b}
    }
    \caption{
    Qualitative comparison of inter-session change representations. The diff map \cite{kim2022lt} uses a binary appearance/disappearance scheme, treating true changes and measurement noise equally. In contrast, our delta map includes an objectness factor $\gamma$ (darker colors correspond to higher values), which assigns greater weights to object-level changes and suppresses noise.
    }
    \label{fig:density}
    \vspace{-3mm}
\end{figure}

\figref{fig:density} compares the diff map \cite{kim2022lt} to our delta map.
While the diff map provides only binary information about point appearance or disappearance, our delta map introduces an objectness factor $\gamma$ to accurately separate meaningful changes from artifacts, thus enabling more robust map updates.


\subsection{Potential Downstream Tasks}
Using delta maps, we can create a heatmap indicating how frequently each region undergoes change. As depicted in \figref{fig:applications}, ephemeral objects tend to yield high heatmap values, identifying areas prone to frequent changes---a potentially valuable insight for optimizing robot navigation.  
Moreover, with enough updates, time-domain analysis similar to \cite{krajnik2017fremen} can be applied, and this can be integrated into \eqref{eq:7} to reflect more region-specific dynamics. We plan to explore this direction in our future work.

\begin{figure}[!t]
    \centering
    \includegraphics[width=0.95\linewidth]{figs/heatmap_overlay_new.pdf}
    \caption{
    \textsl{Left}: Static map constructed from the \texttt{LT-ParkingLot} dataset. 
    \textsl{Right}: Heatmap of frequently changing points after updates from six sessions, overlaid on the static map. Higher heatmap values indicate a stronger likelihood of ephemeral objects, aiding in navigation or planning.
    }
    \label{fig:applications}
    \vspace{-3mm}
\end{figure}
