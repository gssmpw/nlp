\section{RELATED WORK}
\label{sec:realtedwork}

\subsection{Change Detection}
In the 3D change detection literature, many methods adopt an object-centric approach. For instance, \citeauthor{schmid2022panoptic} \cite{schmid2022panoptic} and \citeauthor{langer2020robust} \cite{langer2020robust} define and manage changes based on panoptic or semantic segmentation. However, these strategies often rely on neural networks trained on large amounts of labeled data, which may be infeasible for diverse or unstructured outdoor environments.
Alternatively, several approaches \cite{rowell2024lista, adam2022objects} leverage geometric changes as a prior for reconstructing object-level differences between sessions. Although effective, they assume that changes occur in discrete, object-wise units---an assumption that may break down in highly dynamic outdoor settings, such as construction sites where sand or soil is incrementally added. To address this limitation, we detect changes at the point cloud level and maintain point-wise ephemerality, thereby accommodating continuous or non-discrete changes. This allows us to handle a broader range of real-world scenarios and move beyond a binary changed/not-changed classification paradigm.

\subsection{LiDAR-based Lifelong Mapping}
LiDAR-based lifelong mapping has dealt with scalability \cite{lazaro2018efficient, zhao2021general, kurz2021geometry} or predictability \cite{krajnik2017fremen}, but most of them were demonstrated in two-dimensional spaces.
\citeauthor{pomerleau2014long} \cite{pomerleau2014long} suggested 3D map maintenance pipeline, but they assumed accurate registration and lacked the ability to revert the updates.
Recently, LT-mapper \cite{kim2022lt} suggested the modular approach for lifelong mapping with the following three modules.

\subsubsection{Multi-session map alignment} 
Aligning a point cloud map is often viewed as a registration problem \cite{yin2023automerge, 24frame, yang2024lifelong}. 
However, relying solely on simple rigid-body transformations can introduce alignment errors when the mapped region expands \cite{rowell2024lista}.
To address these challenges, multi-session \ac{PGO} frameworks \cite{kim2022lt, rowell2024lista} have been proposed, but they still face local inconsistencies in large-scale environments. 

\subsubsection{Dynamic object removal} 
Geometry-based methods discretize the environment using voxels \cite{hornung2013octomap, schmid2023dynablox, duberg2024dufomap}, range images \cite{kim2020remove}, bins \cite{lim2021erasor, lim2023erasor2}, or matrices \cite{jia2024beautymap}.
However, these methods are constrained by grid resolution, risking inaccuracies when a single cell contains both static and dynamic points. 
Learning-based methods \cite{pfreundschuh2021dynamic, mersch2022receding, sun2022efficient} can also be effective but typically require extensive labeled datasets to maintain robust performance in unfamiliar scenarios.

\subsubsection{Map update} 
LT-mapper \cite{kim2022lt} and \citeauthor{yang2024lifelong} \cite{yang2024lifelong} detect changes between sessions and update the existing map accordingly. 
They save the changed points and use a version control system \cite{spinellis2012git} that allows manual rollbacks \cite{holoch2022detecting} to previous map via simple arithmetic operations.
Unfortunately, these methods treat changes as binary, which dilutes meaningful changes with outliers from various error sources.

Extending the modular nature, ELite addresses the drawbacks in each of the three modules by introducing ephemerality as a unifying concept throughout the pipeline. It identifies static and persistent regions during multi-session alignment, removes dynamic objects without discretization, and prioritizes meaningful changes for map updates by leveraging contextual information. This integrated use of ephemerality helps ensure more accurate and robust lifelong mapping in real-world, continuously evolving environments.