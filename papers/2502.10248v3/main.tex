\documentclass{article}


\usepackage[final]{neurips}



\usepackage{enumitem}
\usepackage{marvosym} % for symbol  
\usepackage[utf8]{inputenc} % allow utf-8 input
\usepackage[T1]{fontenc}    % use 8-bit T1 fonts
\usepackage{hyperref}       % hyperlinks
\usepackage{url}            % simple URL typesetting
\usepackage{booktabs}       % professional-quality tables
\usepackage{amsfonts}       % blackboard math symbols
% \usepackage{nicefrac}       % compact symbols for 1/2, etc.
\usepackage{microtype}      % microtypography
\usepackage{xcolor}         % colors
\usepackage{epigraph}
\usepackage[export]{adjustbox}
\usepackage{multirow}
\usepackage{listings}
\usepackage{colortbl}
\usepackage{color}
% \usepackage{bbding}
\usepackage{pifont}
\usepackage{fontawesome}
\usepackage{wrapfig}
\usepackage{subcaption}
\usepackage{caption}
\usepackage{float}
\usepackage{enumitem}
\usepackage{tabularx}
\usepackage{makecell}
\usepackage{minitoc} % parttoc
\renewcommand{\partname}{}
\renewcommand{\thepart}{}
\noptcrule



\usepackage{array}
\renewcommand{\arraystretch}{1.2}

\usepackage{caption}
\captionsetup[table]{skip=10pt}  % 调整表格标题和表格内容之间的间距



%%%%% NEW MATH DEFINITIONS %%%%%

\usepackage{amsmath,amsfonts,bm}
\usepackage{derivative}
% Mark sections of captions for referring to divisions of figures
\newcommand{\figleft}{{\em (Left)}}
\newcommand{\figcenter}{{\em (Center)}}
\newcommand{\figright}{{\em (Right)}}
\newcommand{\figtop}{{\em (Top)}}
\newcommand{\figbottom}{{\em (Bottom)}}
\newcommand{\captiona}{{\em (a)}}
\newcommand{\captionb}{{\em (b)}}
\newcommand{\captionc}{{\em (c)}}
\newcommand{\captiond}{{\em (d)}}

% Highlight a newly defined term
\newcommand{\newterm}[1]{{\bf #1}}

% Derivative d 
\newcommand{\deriv}{{\mathrm{d}}}

% Figure reference, lower-case.
\def\figref#1{figure~\ref{#1}}
% Figure reference, capital. For start of sentence
\def\Figref#1{Figure~\ref{#1}}
\def\twofigref#1#2{figures \ref{#1} and \ref{#2}}
\def\quadfigref#1#2#3#4{figures \ref{#1}, \ref{#2}, \ref{#3} and \ref{#4}}
% Section reference, lower-case.
\def\secref#1{section~\ref{#1}}
% Section reference, capital.
\def\Secref#1{Section~\ref{#1}}
% Reference to two sections.
\def\twosecrefs#1#2{sections \ref{#1} and \ref{#2}}
% Reference to three sections.
\def\secrefs#1#2#3{sections \ref{#1}, \ref{#2} and \ref{#3}}
% Reference to an equation, lower-case.
\def\eqref#1{equation~\ref{#1}}
% Reference to an equation, upper case
\def\Eqref#1{Equation~\ref{#1}}
% A raw reference to an equation---avoid using if possible
\def\plaineqref#1{\ref{#1}}
% Reference to a chapter, lower-case.
\def\chapref#1{chapter~\ref{#1}}
% Reference to an equation, upper case.
\def\Chapref#1{Chapter~\ref{#1}}
% Reference to a range of chapters
\def\rangechapref#1#2{chapters\ref{#1}--\ref{#2}}
% Reference to an algorithm, lower-case.
\def\algref#1{algorithm~\ref{#1}}
% Reference to an algorithm, upper case.
\def\Algref#1{Algorithm~\ref{#1}}
\def\twoalgref#1#2{algorithms \ref{#1} and \ref{#2}}
\def\Twoalgref#1#2{Algorithms \ref{#1} and \ref{#2}}
% Reference to a part, lower case
\def\partref#1{part~\ref{#1}}
% Reference to a part, upper case
\def\Partref#1{Part~\ref{#1}}
\def\twopartref#1#2{parts \ref{#1} and \ref{#2}}

\def\ceil#1{\lceil #1 \rceil}
\def\floor#1{\lfloor #1 \rfloor}
\def\1{\bm{1}}
\newcommand{\train}{\mathcal{D}}
\newcommand{\valid}{\mathcal{D_{\mathrm{valid}}}}
\newcommand{\test}{\mathcal{D_{\mathrm{test}}}}

\def\eps{{\epsilon}}


% Random variables
\def\reta{{\textnormal{$\eta$}}}
\def\ra{{\textnormal{a}}}
\def\rb{{\textnormal{b}}}
\def\rc{{\textnormal{c}}}
\def\rd{{\textnormal{d}}}
\def\re{{\textnormal{e}}}
\def\rf{{\textnormal{f}}}
\def\rg{{\textnormal{g}}}
\def\rh{{\textnormal{h}}}
\def\ri{{\textnormal{i}}}
\def\rj{{\textnormal{j}}}
\def\rk{{\textnormal{k}}}
\def\rl{{\textnormal{l}}}
% rm is already a command, just don't name any random variables m
\def\rn{{\textnormal{n}}}
\def\ro{{\textnormal{o}}}
\def\rp{{\textnormal{p}}}
\def\rq{{\textnormal{q}}}
\def\rr{{\textnormal{r}}}
\def\rs{{\textnormal{s}}}
\def\rt{{\textnormal{t}}}
\def\ru{{\textnormal{u}}}
\def\rv{{\textnormal{v}}}
\def\rw{{\textnormal{w}}}
\def\rx{{\textnormal{x}}}
\def\ry{{\textnormal{y}}}
\def\rz{{\textnormal{z}}}

% Random vectors
\def\rvepsilon{{\mathbf{\epsilon}}}
\def\rvphi{{\mathbf{\phi}}}
\def\rvtheta{{\mathbf{\theta}}}
\def\rva{{\mathbf{a}}}
\def\rvb{{\mathbf{b}}}
\def\rvc{{\mathbf{c}}}
\def\rvd{{\mathbf{d}}}
\def\rve{{\mathbf{e}}}
\def\rvf{{\mathbf{f}}}
\def\rvg{{\mathbf{g}}}
\def\rvh{{\mathbf{h}}}
\def\rvu{{\mathbf{i}}}
\def\rvj{{\mathbf{j}}}
\def\rvk{{\mathbf{k}}}
\def\rvl{{\mathbf{l}}}
\def\rvm{{\mathbf{m}}}
\def\rvn{{\mathbf{n}}}
\def\rvo{{\mathbf{o}}}
\def\rvp{{\mathbf{p}}}
\def\rvq{{\mathbf{q}}}
\def\rvr{{\mathbf{r}}}
\def\rvs{{\mathbf{s}}}
\def\rvt{{\mathbf{t}}}
\def\rvu{{\mathbf{u}}}
\def\rvv{{\mathbf{v}}}
\def\rvw{{\mathbf{w}}}
\def\rvx{{\mathbf{x}}}
\def\rvy{{\mathbf{y}}}
\def\rvz{{\mathbf{z}}}

% Elements of random vectors
\def\erva{{\textnormal{a}}}
\def\ervb{{\textnormal{b}}}
\def\ervc{{\textnormal{c}}}
\def\ervd{{\textnormal{d}}}
\def\erve{{\textnormal{e}}}
\def\ervf{{\textnormal{f}}}
\def\ervg{{\textnormal{g}}}
\def\ervh{{\textnormal{h}}}
\def\ervi{{\textnormal{i}}}
\def\ervj{{\textnormal{j}}}
\def\ervk{{\textnormal{k}}}
\def\ervl{{\textnormal{l}}}
\def\ervm{{\textnormal{m}}}
\def\ervn{{\textnormal{n}}}
\def\ervo{{\textnormal{o}}}
\def\ervp{{\textnormal{p}}}
\def\ervq{{\textnormal{q}}}
\def\ervr{{\textnormal{r}}}
\def\ervs{{\textnormal{s}}}
\def\ervt{{\textnormal{t}}}
\def\ervu{{\textnormal{u}}}
\def\ervv{{\textnormal{v}}}
\def\ervw{{\textnormal{w}}}
\def\ervx{{\textnormal{x}}}
\def\ervy{{\textnormal{y}}}
\def\ervz{{\textnormal{z}}}

% Random matrices
\def\rmA{{\mathbf{A}}}
\def\rmB{{\mathbf{B}}}
\def\rmC{{\mathbf{C}}}
\def\rmD{{\mathbf{D}}}
\def\rmE{{\mathbf{E}}}
\def\rmF{{\mathbf{F}}}
\def\rmG{{\mathbf{G}}}
\def\rmH{{\mathbf{H}}}
\def\rmI{{\mathbf{I}}}
\def\rmJ{{\mathbf{J}}}
\def\rmK{{\mathbf{K}}}
\def\rmL{{\mathbf{L}}}
\def\rmM{{\mathbf{M}}}
\def\rmN{{\mathbf{N}}}
\def\rmO{{\mathbf{O}}}
\def\rmP{{\mathbf{P}}}
\def\rmQ{{\mathbf{Q}}}
\def\rmR{{\mathbf{R}}}
\def\rmS{{\mathbf{S}}}
\def\rmT{{\mathbf{T}}}
\def\rmU{{\mathbf{U}}}
\def\rmV{{\mathbf{V}}}
\def\rmW{{\mathbf{W}}}
\def\rmX{{\mathbf{X}}}
\def\rmY{{\mathbf{Y}}}
\def\rmZ{{\mathbf{Z}}}

% Elements of random matrices
\def\ermA{{\textnormal{A}}}
\def\ermB{{\textnormal{B}}}
\def\ermC{{\textnormal{C}}}
\def\ermD{{\textnormal{D}}}
\def\ermE{{\textnormal{E}}}
\def\ermF{{\textnormal{F}}}
\def\ermG{{\textnormal{G}}}
\def\ermH{{\textnormal{H}}}
\def\ermI{{\textnormal{I}}}
\def\ermJ{{\textnormal{J}}}
\def\ermK{{\textnormal{K}}}
\def\ermL{{\textnormal{L}}}
\def\ermM{{\textnormal{M}}}
\def\ermN{{\textnormal{N}}}
\def\ermO{{\textnormal{O}}}
\def\ermP{{\textnormal{P}}}
\def\ermQ{{\textnormal{Q}}}
\def\ermR{{\textnormal{R}}}
\def\ermS{{\textnormal{S}}}
\def\ermT{{\textnormal{T}}}
\def\ermU{{\textnormal{U}}}
\def\ermV{{\textnormal{V}}}
\def\ermW{{\textnormal{W}}}
\def\ermX{{\textnormal{X}}}
\def\ermY{{\textnormal{Y}}}
\def\ermZ{{\textnormal{Z}}}

% Vectors
\def\vzero{{\bm{0}}}
\def\vone{{\bm{1}}}
\def\vmu{{\bm{\mu}}}
\def\vtheta{{\bm{\theta}}}
\def\vphi{{\bm{\phi}}}
\def\va{{\bm{a}}}
\def\vb{{\bm{b}}}
\def\vc{{\bm{c}}}
\def\vd{{\bm{d}}}
\def\ve{{\bm{e}}}
\def\vf{{\bm{f}}}
\def\vg{{\bm{g}}}
\def\vh{{\bm{h}}}
\def\vi{{\bm{i}}}
\def\vj{{\bm{j}}}
\def\vk{{\bm{k}}}
\def\vl{{\bm{l}}}
\def\vm{{\bm{m}}}
\def\vn{{\bm{n}}}
\def\vo{{\bm{o}}}
\def\vp{{\bm{p}}}
\def\vq{{\bm{q}}}
\def\vr{{\bm{r}}}
\def\vs{{\bm{s}}}
\def\vt{{\bm{t}}}
\def\vu{{\bm{u}}}
\def\vv{{\bm{v}}}
\def\vw{{\bm{w}}}
\def\vx{{\bm{x}}}
\def\vy{{\bm{y}}}
\def\vz{{\bm{z}}}

% Elements of vectors
\def\evalpha{{\alpha}}
\def\evbeta{{\beta}}
\def\evepsilon{{\epsilon}}
\def\evlambda{{\lambda}}
\def\evomega{{\omega}}
\def\evmu{{\mu}}
\def\evpsi{{\psi}}
\def\evsigma{{\sigma}}
\def\evtheta{{\theta}}
\def\eva{{a}}
\def\evb{{b}}
\def\evc{{c}}
\def\evd{{d}}
\def\eve{{e}}
\def\evf{{f}}
\def\evg{{g}}
\def\evh{{h}}
\def\evi{{i}}
\def\evj{{j}}
\def\evk{{k}}
\def\evl{{l}}
\def\evm{{m}}
\def\evn{{n}}
\def\evo{{o}}
\def\evp{{p}}
\def\evq{{q}}
\def\evr{{r}}
\def\evs{{s}}
\def\evt{{t}}
\def\evu{{u}}
\def\evv{{v}}
\def\evw{{w}}
\def\evx{{x}}
\def\evy{{y}}
\def\evz{{z}}

% Matrix
\def\mA{{\bm{A}}}
\def\mB{{\bm{B}}}
\def\mC{{\bm{C}}}
\def\mD{{\bm{D}}}
\def\mE{{\bm{E}}}
\def\mF{{\bm{F}}}
\def\mG{{\bm{G}}}
\def\mH{{\bm{H}}}
\def\mI{{\bm{I}}}
\def\mJ{{\bm{J}}}
\def\mK{{\bm{K}}}
\def\mL{{\bm{L}}}
\def\mM{{\bm{M}}}
\def\mN{{\bm{N}}}
\def\mO{{\bm{O}}}
\def\mP{{\bm{P}}}
\def\mQ{{\bm{Q}}}
\def\mR{{\bm{R}}}
\def\mS{{\bm{S}}}
\def\mT{{\bm{T}}}
\def\mU{{\bm{U}}}
\def\mV{{\bm{V}}}
\def\mW{{\bm{W}}}
\def\mX{{\bm{X}}}
\def\mY{{\bm{Y}}}
\def\mZ{{\bm{Z}}}
\def\mBeta{{\bm{\beta}}}
\def\mPhi{{\bm{\Phi}}}
\def\mLambda{{\bm{\Lambda}}}
\def\mSigma{{\bm{\Sigma}}}

% Tensor
\DeclareMathAlphabet{\mathsfit}{\encodingdefault}{\sfdefault}{m}{sl}
\SetMathAlphabet{\mathsfit}{bold}{\encodingdefault}{\sfdefault}{bx}{n}
\newcommand{\tens}[1]{\bm{\mathsfit{#1}}}
\def\tA{{\tens{A}}}
\def\tB{{\tens{B}}}
\def\tC{{\tens{C}}}
\def\tD{{\tens{D}}}
\def\tE{{\tens{E}}}
\def\tF{{\tens{F}}}
\def\tG{{\tens{G}}}
\def\tH{{\tens{H}}}
\def\tI{{\tens{I}}}
\def\tJ{{\tens{J}}}
\def\tK{{\tens{K}}}
\def\tL{{\tens{L}}}
\def\tM{{\tens{M}}}
\def\tN{{\tens{N}}}
\def\tO{{\tens{O}}}
\def\tP{{\tens{P}}}
\def\tQ{{\tens{Q}}}
\def\tR{{\tens{R}}}
\def\tS{{\tens{S}}}
\def\tT{{\tens{T}}}
\def\tU{{\tens{U}}}
\def\tV{{\tens{V}}}
\def\tW{{\tens{W}}}
\def\tX{{\tens{X}}}
\def\tY{{\tens{Y}}}
\def\tZ{{\tens{Z}}}


% Graph
\def\gA{{\mathcal{A}}}
\def\gB{{\mathcal{B}}}
\def\gC{{\mathcal{C}}}
\def\gD{{\mathcal{D}}}
\def\gE{{\mathcal{E}}}
\def\gF{{\mathcal{F}}}
\def\gG{{\mathcal{G}}}
\def\gH{{\mathcal{H}}}
\def\gI{{\mathcal{I}}}
\def\gJ{{\mathcal{J}}}
\def\gK{{\mathcal{K}}}
\def\gL{{\mathcal{L}}}
\def\gM{{\mathcal{M}}}
\def\gN{{\mathcal{N}}}
\def\gO{{\mathcal{O}}}
\def\gP{{\mathcal{P}}}
\def\gQ{{\mathcal{Q}}}
\def\gR{{\mathcal{R}}}
\def\gS{{\mathcal{S}}}
\def\gT{{\mathcal{T}}}
\def\gU{{\mathcal{U}}}
\def\gV{{\mathcal{V}}}
\def\gW{{\mathcal{W}}}
\def\gX{{\mathcal{X}}}
\def\gY{{\mathcal{Y}}}
\def\gZ{{\mathcal{Z}}}

% Sets
\def\sA{{\mathbb{A}}}
\def\sB{{\mathbb{B}}}
\def\sC{{\mathbb{C}}}
\def\sD{{\mathbb{D}}}
% Don't use a set called E, because this would be the same as our symbol
% for expectation.
\def\sF{{\mathbb{F}}}
\def\sG{{\mathbb{G}}}
\def\sH{{\mathbb{H}}}
\def\sI{{\mathbb{I}}}
\def\sJ{{\mathbb{J}}}
\def\sK{{\mathbb{K}}}
\def\sL{{\mathbb{L}}}
\def\sM{{\mathbb{M}}}
\def\sN{{\mathbb{N}}}
\def\sO{{\mathbb{O}}}
\def\sP{{\mathbb{P}}}
\def\sQ{{\mathbb{Q}}}
\def\sR{{\mathbb{R}}}
\def\sS{{\mathbb{S}}}
\def\sT{{\mathbb{T}}}
\def\sU{{\mathbb{U}}}
\def\sV{{\mathbb{V}}}
\def\sW{{\mathbb{W}}}
\def\sX{{\mathbb{X}}}
\def\sY{{\mathbb{Y}}}
\def\sZ{{\mathbb{Z}}}

% Entries of a matrix
\def\emLambda{{\Lambda}}
\def\emA{{A}}
\def\emB{{B}}
\def\emC{{C}}
\def\emD{{D}}
\def\emE{{E}}
\def\emF{{F}}
\def\emG{{G}}
\def\emH{{H}}
\def\emI{{I}}
\def\emJ{{J}}
\def\emK{{K}}
\def\emL{{L}}
\def\emM{{M}}
\def\emN{{N}}
\def\emO{{O}}
\def\emP{{P}}
\def\emQ{{Q}}
\def\emR{{R}}
\def\emS{{S}}
\def\emT{{T}}
\def\emU{{U}}
\def\emV{{V}}
\def\emW{{W}}
\def\emX{{X}}
\def\emY{{Y}}
\def\emZ{{Z}}
\def\emSigma{{\Sigma}}

% entries of a tensor
% Same font as tensor, without \bm wrapper
\newcommand{\etens}[1]{\mathsfit{#1}}
\def\etLambda{{\etens{\Lambda}}}
\def\etA{{\etens{A}}}
\def\etB{{\etens{B}}}
\def\etC{{\etens{C}}}
\def\etD{{\etens{D}}}
\def\etE{{\etens{E}}}
\def\etF{{\etens{F}}}
\def\etG{{\etens{G}}}
\def\etH{{\etens{H}}}
\def\etI{{\etens{I}}}
\def\etJ{{\etens{J}}}
\def\etK{{\etens{K}}}
\def\etL{{\etens{L}}}
\def\etM{{\etens{M}}}
\def\etN{{\etens{N}}}
\def\etO{{\etens{O}}}
\def\etP{{\etens{P}}}
\def\etQ{{\etens{Q}}}
\def\etR{{\etens{R}}}
\def\etS{{\etens{S}}}
\def\etT{{\etens{T}}}
\def\etU{{\etens{U}}}
\def\etV{{\etens{V}}}
\def\etW{{\etens{W}}}
\def\etX{{\etens{X}}}
\def\etY{{\etens{Y}}}
\def\etZ{{\etens{Z}}}

% The true underlying data generating distribution
\newcommand{\pdata}{p_{\rm{data}}}
\newcommand{\ptarget}{p_{\rm{target}}}
\newcommand{\pprior}{p_{\rm{prior}}}
\newcommand{\pbase}{p_{\rm{base}}}
\newcommand{\pref}{p_{\rm{ref}}}

% The empirical distribution defined by the training set
\newcommand{\ptrain}{\hat{p}_{\rm{data}}}
\newcommand{\Ptrain}{\hat{P}_{\rm{data}}}
% The model distribution
\newcommand{\pmodel}{p_{\rm{model}}}
\newcommand{\Pmodel}{P_{\rm{model}}}
\newcommand{\ptildemodel}{\tilde{p}_{\rm{model}}}
% Stochastic autoencoder distributions
\newcommand{\pencode}{p_{\rm{encoder}}}
\newcommand{\pdecode}{p_{\rm{decoder}}}
\newcommand{\precons}{p_{\rm{reconstruct}}}

\newcommand{\laplace}{\mathrm{Laplace}} % Laplace distribution

\newcommand{\E}{\mathbb{E}}
\newcommand{\Ls}{\mathcal{L}}
\newcommand{\R}{\mathbb{R}}
\newcommand{\emp}{\tilde{p}}
\newcommand{\lr}{\alpha}
\newcommand{\reg}{\lambda}
\newcommand{\rect}{\mathrm{rectifier}}
\newcommand{\softmax}{\mathrm{softmax}}
\newcommand{\sigmoid}{\sigma}
\newcommand{\softplus}{\zeta}
\newcommand{\KL}{D_{\mathrm{KL}}}
\newcommand{\Var}{\mathrm{Var}}
\newcommand{\standarderror}{\mathrm{SE}}
\newcommand{\Cov}{\mathrm{Cov}}
% Wolfram Mathworld says $L^2$ is for function spaces and $\ell^2$ is for vectors
% But then they seem to use $L^2$ for vectors throughout the site, and so does
% wikipedia.
\newcommand{\normlzero}{L^0}
\newcommand{\normlone}{L^1}
\newcommand{\normltwo}{L^2}
\newcommand{\normlp}{L^p}
\newcommand{\normmax}{L^\infty}

\newcommand{\parents}{Pa} % See usage in notation.tex. Chosen to match Daphne's book.

\DeclareMathOperator*{\argmax}{arg\,max}
\DeclareMathOperator*{\argmin}{arg\,min}

\DeclareMathOperator{\sign}{sign}
\DeclareMathOperator{\Tr}{Tr}
\let\ab\allowbreak


%
\setlength\unitlength{1mm}
\newcommand{\twodots}{\mathinner {\ldotp \ldotp}}
% bb font symbols
\newcommand{\Rho}{\mathrm{P}}
\newcommand{\Tau}{\mathrm{T}}

\newfont{\bbb}{msbm10 scaled 700}
\newcommand{\CCC}{\mbox{\bbb C}}

\newfont{\bb}{msbm10 scaled 1100}
\newcommand{\CC}{\mbox{\bb C}}
\newcommand{\PP}{\mbox{\bb P}}
\newcommand{\RR}{\mbox{\bb R}}
\newcommand{\QQ}{\mbox{\bb Q}}
\newcommand{\ZZ}{\mbox{\bb Z}}
\newcommand{\FF}{\mbox{\bb F}}
\newcommand{\GG}{\mbox{\bb G}}
\newcommand{\EE}{\mbox{\bb E}}
\newcommand{\NN}{\mbox{\bb N}}
\newcommand{\KK}{\mbox{\bb K}}
\newcommand{\HH}{\mbox{\bb H}}
\newcommand{\SSS}{\mbox{\bb S}}
\newcommand{\UU}{\mbox{\bb U}}
\newcommand{\VV}{\mbox{\bb V}}


\newcommand{\yy}{\mathbbm{y}}
\newcommand{\xx}{\mathbbm{x}}
\newcommand{\zz}{\mathbbm{z}}
\newcommand{\sss}{\mathbbm{s}}
\newcommand{\rr}{\mathbbm{r}}
\newcommand{\pp}{\mathbbm{p}}
\newcommand{\qq}{\mathbbm{q}}
\newcommand{\ww}{\mathbbm{w}}
\newcommand{\hh}{\mathbbm{h}}
\newcommand{\vvv}{\mathbbm{v}}

% Vectors

\newcommand{\av}{{\bf a}}
\newcommand{\bv}{{\bf b}}
\newcommand{\cv}{{\bf c}}
\newcommand{\dv}{{\bf d}}
\newcommand{\ev}{{\bf e}}
\newcommand{\fv}{{\bf f}}
\newcommand{\gv}{{\bf g}}
\newcommand{\hv}{{\bf h}}
\newcommand{\iv}{{\bf i}}
\newcommand{\jv}{{\bf j}}
\newcommand{\kv}{{\bf k}}
\newcommand{\lv}{{\bf l}}
\newcommand{\mv}{{\bf m}}
\newcommand{\nv}{{\bf n}}
\newcommand{\ov}{{\bf o}}
\newcommand{\pv}{{\bf p}}
\newcommand{\qv}{{\bf q}}
\newcommand{\rv}{{\bf r}}
\newcommand{\sv}{{\bf s}}
\newcommand{\tv}{{\bf t}}
\newcommand{\uv}{{\bf u}}
\newcommand{\wv}{{\bf w}}
\newcommand{\vv}{{\bf v}}
\newcommand{\xv}{{\bf x}}
\newcommand{\yv}{{\bf y}}
\newcommand{\zv}{{\bf z}}
\newcommand{\zerov}{{\bf 0}}
\newcommand{\onev}{{\bf 1}}

% Matrices

\newcommand{\Am}{{\bf A}}
\newcommand{\Bm}{{\bf B}}
\newcommand{\Cm}{{\bf C}}
\newcommand{\Dm}{{\bf D}}
\newcommand{\Em}{{\bf E}}
\newcommand{\Fm}{{\bf F}}
\newcommand{\Gm}{{\bf G}}
\newcommand{\Hm}{{\bf H}}
\newcommand{\Id}{{\bf I}}
\newcommand{\Jm}{{\bf J}}
\newcommand{\Km}{{\bf K}}
\newcommand{\Lm}{{\bf L}}
\newcommand{\Mm}{{\bf M}}
\newcommand{\Nm}{{\bf N}}
\newcommand{\Om}{{\bf O}}
\newcommand{\Pm}{{\bf P}}
\newcommand{\Qm}{{\bf Q}}
\newcommand{\Rm}{{\bf R}}
\newcommand{\Sm}{{\bf S}}
\newcommand{\Tm}{{\bf T}}
\newcommand{\Um}{{\bf U}}
\newcommand{\Wm}{{\bf W}}
\newcommand{\Vm}{{\bf V}}
\newcommand{\Xm}{{\bf X}}
\newcommand{\Ym}{{\bf Y}}
\newcommand{\Zm}{{\bf Z}}

% Calligraphic

\newcommand{\Ac}{{\cal A}}
\newcommand{\Bc}{{\cal B}}
\newcommand{\Cc}{{\cal C}}
\newcommand{\Dc}{{\cal D}}
\newcommand{\Ec}{{\cal E}}
\newcommand{\Fc}{{\cal F}}
\newcommand{\Gc}{{\cal G}}
\newcommand{\Hc}{{\cal H}}
\newcommand{\Ic}{{\cal I}}
\newcommand{\Jc}{{\cal J}}
\newcommand{\Kc}{{\cal K}}
\newcommand{\Lc}{{\cal L}}
\newcommand{\Mc}{{\cal M}}
\newcommand{\Nc}{{\cal N}}
\newcommand{\nc}{{\cal n}}
\newcommand{\Oc}{{\cal O}}
\newcommand{\Pc}{{\cal P}}
\newcommand{\Qc}{{\cal Q}}
\newcommand{\Rc}{{\cal R}}
\newcommand{\Sc}{{\cal S}}
\newcommand{\Tc}{{\cal T}}
\newcommand{\Uc}{{\cal U}}
\newcommand{\Wc}{{\cal W}}
\newcommand{\Vc}{{\cal V}}
\newcommand{\Xc}{{\cal X}}
\newcommand{\Yc}{{\cal Y}}
\newcommand{\Zc}{{\cal Z}}

% Bold greek letters

\newcommand{\alphav}{\hbox{\boldmath$\alpha$}}
\newcommand{\betav}{\hbox{\boldmath$\beta$}}
\newcommand{\gammav}{\hbox{\boldmath$\gamma$}}
\newcommand{\deltav}{\hbox{\boldmath$\delta$}}
\newcommand{\etav}{\hbox{\boldmath$\eta$}}
\newcommand{\lambdav}{\hbox{\boldmath$\lambda$}}
\newcommand{\epsilonv}{\hbox{\boldmath$\epsilon$}}
\newcommand{\nuv}{\hbox{\boldmath$\nu$}}
\newcommand{\muv}{\hbox{\boldmath$\mu$}}
\newcommand{\zetav}{\hbox{\boldmath$\zeta$}}
\newcommand{\phiv}{\hbox{\boldmath$\phi$}}
\newcommand{\psiv}{\hbox{\boldmath$\psi$}}
\newcommand{\thetav}{\hbox{\boldmath$\theta$}}
\newcommand{\tauv}{\hbox{\boldmath$\tau$}}
\newcommand{\omegav}{\hbox{\boldmath$\omega$}}
\newcommand{\xiv}{\hbox{\boldmath$\xi$}}
\newcommand{\sigmav}{\hbox{\boldmath$\sigma$}}
\newcommand{\piv}{\hbox{\boldmath$\pi$}}
\newcommand{\rhov}{\hbox{\boldmath$\rho$}}
\newcommand{\upsilonv}{\hbox{\boldmath$\upsilon$}}

\newcommand{\Gammam}{\hbox{\boldmath$\Gamma$}}
\newcommand{\Lambdam}{\hbox{\boldmath$\Lambda$}}
\newcommand{\Deltam}{\hbox{\boldmath$\Delta$}}
\newcommand{\Sigmam}{\hbox{\boldmath$\Sigma$}}
\newcommand{\Phim}{\hbox{\boldmath$\Phi$}}
\newcommand{\Pim}{\hbox{\boldmath$\Pi$}}
\newcommand{\Psim}{\hbox{\boldmath$\Psi$}}
\newcommand{\Thetam}{\hbox{\boldmath$\Theta$}}
\newcommand{\Omegam}{\hbox{\boldmath$\Omega$}}
\newcommand{\Xim}{\hbox{\boldmath$\Xi$}}


% Sans Serif small case

\newcommand{\Gsf}{{\sf G}}

\newcommand{\asf}{{\sf a}}
\newcommand{\bsf}{{\sf b}}
\newcommand{\csf}{{\sf c}}
\newcommand{\dsf}{{\sf d}}
\newcommand{\esf}{{\sf e}}
\newcommand{\fsf}{{\sf f}}
\newcommand{\gsf}{{\sf g}}
\newcommand{\hsf}{{\sf h}}
\newcommand{\isf}{{\sf i}}
\newcommand{\jsf}{{\sf j}}
\newcommand{\ksf}{{\sf k}}
\newcommand{\lsf}{{\sf l}}
\newcommand{\msf}{{\sf m}}
\newcommand{\nsf}{{\sf n}}
\newcommand{\osf}{{\sf o}}
\newcommand{\psf}{{\sf p}}
\newcommand{\qsf}{{\sf q}}
\newcommand{\rsf}{{\sf r}}
\newcommand{\ssf}{{\sf s}}
\newcommand{\tsf}{{\sf t}}
\newcommand{\usf}{{\sf u}}
\newcommand{\wsf}{{\sf w}}
\newcommand{\vsf}{{\sf v}}
\newcommand{\xsf}{{\sf x}}
\newcommand{\ysf}{{\sf y}}
\newcommand{\zsf}{{\sf z}}


% mixed symbols

\newcommand{\sinc}{{\hbox{sinc}}}
\newcommand{\diag}{{\hbox{diag}}}
\renewcommand{\det}{{\hbox{det}}}
\newcommand{\trace}{{\hbox{tr}}}
\newcommand{\sign}{{\hbox{sign}}}
\renewcommand{\arg}{{\hbox{arg}}}
\newcommand{\var}{{\hbox{var}}}
\newcommand{\cov}{{\hbox{cov}}}
\newcommand{\Ei}{{\rm E}_{\rm i}}
\renewcommand{\Re}{{\rm Re}}
\renewcommand{\Im}{{\rm Im}}
\newcommand{\eqdef}{\stackrel{\Delta}{=}}
\newcommand{\defines}{{\,\,\stackrel{\scriptscriptstyle \bigtriangleup}{=}\,\,}}
\newcommand{\<}{\left\langle}
\renewcommand{\>}{\right\rangle}
\newcommand{\herm}{{\sf H}}
\newcommand{\trasp}{{\sf T}}
\newcommand{\transp}{{\sf T}}
\renewcommand{\vec}{{\rm vec}}
\newcommand{\Psf}{{\sf P}}
\newcommand{\SINR}{{\sf SINR}}
\newcommand{\SNR}{{\sf SNR}}
\newcommand{\MMSE}{{\sf MMSE}}
\newcommand{\REF}{{\RED [REF]}}

% Markov chain
\usepackage{stmaryrd} % for \mkv 
\newcommand{\mkv}{-\!\!\!\!\minuso\!\!\!\!-}

% Colors

\newcommand{\RED}{\color[rgb]{1.00,0.10,0.10}}
\newcommand{\BLUE}{\color[rgb]{0,0,0.90}}
\newcommand{\GREEN}{\color[rgb]{0,0.80,0.20}}

%%%%%%%%%%%%%%%%%%%%%%%%%%%%%%%%%%%%%%%%%%
\usepackage{hyperref}
\hypersetup{
    bookmarks=true,         % show bookmarks bar?
    unicode=false,          % non-Latin characters in AcrobatÕs bookmarks
    pdftoolbar=true,        % show AcrobatÕs toolbar?
    pdfmenubar=true,        % show AcrobatÕs menu?
    pdffitwindow=false,     % window fit to page when opened
    pdfstartview={FitH},    % fits the width of the page to the window
%    pdftitle={My title},    % title
%    pdfauthor={Author},     % author
%    pdfsubject={Subject},   % subject of the document
%    pdfcreator={Creator},   % creator of the document
%    pdfproducer={Producer}, % producer of the document
%    pdfkeywords={keyword1} {key2} {key3}, % list of keywords
    pdfnewwindow=true,      % links in new window
    colorlinks=true,       % false: boxed links; true: colored links
    linkcolor=red,          % color of internal links (change box color with linkbordercolor)
    citecolor=green,        % color of links to bibliography
    filecolor=blue,      % color of file links
    urlcolor=blue           % color of external links
}
%%%%%%%%%%%%%%%%%%%%%%%%%%%%%%%%%%%%%%%%%%%




% arxiv:
\title{Step-Video-T2V Technical Report: The Practice, Challenges, and Future of Video Foundation Model}

% \vspace{-0.4cm}
% \vspace{-1cm}
\author{Step-Video Team
\\
StepFun
}





\begin{document}

\doparttoc
\faketableofcontents

\maketitle

\begin{abstract}
Building a virtual cell capable of accurately simulating cellular behaviors in silico has long been a dream in computational biology. We introduce \emph{CellFlow}, an image-generative model that simulates cellular morphology changes induced by chemical and genetic perturbations using flow matching. Unlike prior methods, \emph{CellFlow} models distribution-wise transformations from unperturbed to perturbed cell states, effectively distinguishing actual perturbation effects from experimental artifacts such as batch effects—a major challenge in biological data. Evaluated on chemical (BBBC021), genetic (RxRx1), and combined perturbation (JUMP) datasets, \emph{CellFlow} generates biologically meaningful cell images that faithfully capture perturbation-specific morphological changes, achieving a 35\% improvement in FID scores and a 12\% increase in mode-of-action prediction accuracy over existing methods. Additionally, \emph{CellFlow} enables continuous interpolation between cellular states, providing a potential tool for studying perturbation dynamics. These capabilities mark a significant step toward realizing virtual cell modeling for biomedical research.
\end{abstract}


\section{Preface}

A video foundation model is a model pre-trained on large video datasets that can generate videos in response to text, visual, or multimodal inputs from users. It can be applied to a wide range of downstream video-related tasks, such as text/image/video-to-video generation, video understanding and editing, as well as video-based conversion, question answering, and task completion.

Based on our understanding, we define two levels towards building video foundation models. 
\textbf{Level-1: translational video foundation model}. A model at this level functions as a cross-modal translation system, capable of generating videos from text, visual, or multimodal context.
\textbf{Level-2: predictable video foundation model}. A model at this level acts as a prediction system, similar to large language models (LLMs), that can forecast future events based on text, visual, or multimodal context and handle more advanced tasks, such as reasoning with multimodal data or simulating real-world scenarios.

Current diffusion-based text-to-video models, such as Sora \cite{openaisora}, Veo \cite{veo}, Kling \cite{kling}, Hailuo \cite{hailuo}, and Step-Video (as described in this report), belong to Level-1. These models can generate high-quality videos from text prompts, lowering the barrier for creators to produce video content. However, they often fail to generate videos that require complex action sequences (such as a gymnastic performance) or adherence to the laws of physics (such as a basketball bouncing on the floor), let alone performing causal or logical tasks like LLMs. Such limitations arise because these models learn only the mappings between text prompts and corresponding videos, without explicitly modeling the underlying causal relationships within videos. Autoregression-based text-to-video models introduce the causal modeling mechanism by predicting the next video token, frame, or clip. However, these models still cannot achieve performance comparable to diffusion-based models on text-to-video generation.

This report will detail the practice of building Step-Video-T2V as a state-of-the-art video foundation model at Level-1. By analyzing the challenges identified through experiments, we will also highlight key problems that need to be addressed in order to develop video foundation models at Level-2.
\section{Introduction}
Backdoor attacks pose a concealed yet profound security risk to machine learning (ML) models, for which the adversaries can inject a stealth backdoor into the model during training, enabling them to illicitly control the model's output upon encountering predefined inputs. These attacks can even occur without the knowledge of developers or end-users, thereby undermining the trust in ML systems. As ML becomes more deeply embedded in critical sectors like finance, healthcare, and autonomous driving \citep{he2016deep, liu2020computing, tournier2019mrtrix3, adjabi2020past}, the potential damage from backdoor attacks grows, underscoring the emergency for developing robust defense mechanisms against backdoor attacks.

To address the threat of backdoor attacks, researchers have developed a variety of strategies \cite{liu2018fine,wu2021adversarial,wang2019neural,zeng2022adversarial,zhu2023neural,Zhu_2023_ICCV, wei2024shared,wei2024d3}, aimed at purifying backdoors within victim models. These methods are designed to integrate with current deployment workflows seamlessly and have demonstrated significant success in mitigating the effects of backdoor triggers \cite{wubackdoorbench, wu2023defenses, wu2024backdoorbench,dunnett2024countering}.  However, most state-of-the-art (SOTA) backdoor purification methods operate under the assumption that a small clean dataset, often referred to as \textbf{auxiliary dataset}, is available for purification. Such an assumption poses practical challenges, especially in scenarios where data is scarce. To tackle this challenge, efforts have been made to reduce the size of the required auxiliary dataset~\cite{chai2022oneshot,li2023reconstructive, Zhu_2023_ICCV} and even explore dataset-free purification techniques~\cite{zheng2022data,hong2023revisiting,lin2024fusing}. Although these approaches offer some improvements, recent evaluations \cite{dunnett2024countering, wu2024backdoorbench} continue to highlight the importance of sufficient auxiliary data for achieving robust defenses against backdoor attacks.

While significant progress has been made in reducing the size of auxiliary datasets, an equally critical yet underexplored question remains: \emph{how does the nature of the auxiliary dataset affect purification effectiveness?} In  real-world  applications, auxiliary datasets can vary widely, encompassing in-distribution data, synthetic data, or external data from different sources. Understanding how each type of auxiliary dataset influences the purification effectiveness is vital for selecting or constructing the most suitable auxiliary dataset and the corresponding technique. For instance, when multiple datasets are available, understanding how different datasets contribute to purification can guide defenders in selecting or crafting the most appropriate dataset. Conversely, when only limited auxiliary data is accessible, knowing which purification technique works best under those constraints is critical. Therefore, there is an urgent need for a thorough investigation into the impact of auxiliary datasets on purification effectiveness to guide defenders in  enhancing the security of ML systems. 

In this paper, we systematically investigate the critical role of auxiliary datasets in backdoor purification, aiming to bridge the gap between idealized and practical purification scenarios.  Specifically, we first construct a diverse set of auxiliary datasets to emulate real-world conditions, as summarized in Table~\ref{overall}. These datasets include in-distribution data, synthetic data, and external data from other sources. Through an evaluation of SOTA backdoor purification methods across these datasets, we uncover several critical insights: \textbf{1)} In-distribution datasets, particularly those carefully filtered from the original training data of the victim model, effectively preserve the model’s utility for its intended tasks but may fall short in eliminating backdoors. \textbf{2)} Incorporating OOD datasets can help the model forget backdoors but also bring the risk of forgetting critical learned knowledge, significantly degrading its overall performance. Building on these findings, we propose Guided Input Calibration (GIC), a novel technique that enhances backdoor purification by adaptively transforming auxiliary data to better align with the victim model’s learned representations. By leveraging the victim model itself to guide this transformation, GIC optimizes the purification process, striking a balance between preserving model utility and mitigating backdoor threats. Extensive experiments demonstrate that GIC significantly improves the effectiveness of backdoor purification across diverse auxiliary datasets, providing a practical and robust defense solution.

Our main contributions are threefold:
\textbf{1) Impact analysis of auxiliary datasets:} We take the \textbf{first step}  in systematically investigating how different types of auxiliary datasets influence backdoor purification effectiveness. Our findings provide novel insights and serve as a foundation for future research on optimizing dataset selection and construction for enhanced backdoor defense.
%
\textbf{2) Compilation and evaluation of diverse auxiliary datasets:}  We have compiled and rigorously evaluated a diverse set of auxiliary datasets using SOTA purification methods, making our datasets and code publicly available to facilitate and support future research on practical backdoor defense strategies.
%
\textbf{3) Introduction of GIC:} We introduce GIC, the \textbf{first} dedicated solution designed to align auxiliary datasets with the model’s learned representations, significantly enhancing backdoor mitigation across various dataset types. Our approach sets a new benchmark for practical and effective backdoor defense.



\subsection{The Generative Model}\label{sec:generative_model}
Let there be $n_1$ users and $n_2$ items. %
Each user $u$ and each item $i$ has a $r$-dimensional feature vector. %
The inner product of these two feature vectors gives the utility (or score) $x^*_{u,i}$ that user $u$ has for item $i$. 
This modeling assumption implies that
the score matrix $X^* \in \Real{n_1 \times n_2}$ has rank $r$, and thus admits the following rank-$r$ SVD:
\begin{align}
    X^* = U^* \Sigma^* V^{*T},
\end{align}
where $U^* \in \Real{n_1 \times r}$ and $V^* \in \Real{n_2 \times r}$ are matrices that satisfy $U^{*T} U^* = V^{*T} V^* = I_r$, and $\Sigma^* \in \Real{r \times r}$ is a diagonal matrix with entries $\sigma_1^* \geq \ldots \geq \sigma_r^* > 0$. Let $\kappa \triangleq \sigma_1^*/\sigma_r^*$ denote the condition number of $\Sigma^*$.

Let $n = n_1 + n_2$. Define $Z^* \in \Real{n \times r}$ and $Y^* \in \Real{n \times n}$ as follows:
\begin{align}\label{eq:matrix_z}
    Z^* &= 
    \begin{bmatrix}
        U^* \\ V^*
    \end{bmatrix}
    \Sigma^{* 1/2}, \\ 
    Y^* &= Z^*Z^{*T} = 
    \begin{bmatrix}
        U^*\Sigma^*U^{*T} & X^*\\
        X^{*T} & V^*\Sigma^*V^{*T}
    \end{bmatrix}.
\end{align}
From here on, we shall refer to $Z^*$ as the ground-truth matrix. Note that the singular values of $Z^*$ are $\sqrt{2\sigma_1^*}, \ldots, \sqrt{2\sigma_r^*}$.

We are given a dataset $\Dataset$ where each data point represents a comparison made by a user between two items. The size of the dataset, \textit{i.e.,} the number of data points, is represented by $m$. We index the dataset by $k$. Each data point $\Dataset_k$ is of the form $((u; i, j), w)$ and is sampled randomly as follows. The user index $u$ is chosen uniformly at random from $[n_1]$. The pair of item indices $(i,j)$ is chosen uniformly at random from the set of $n_2(n_2-1)$ pairs of distinct items. The item pair $(i,j)$ is sampled independently from $u$. The triplets for different datapoints are sampled independently of each other.

The variable $w$ reflects the outcome of the comparison made by the user $u$ between items $i$ and $j$. In the \textit{noisy} setting, $w$ is an indicator for the outcome of the comparison; it is one if $i$ is chosen and zero if $j$ is chosen. Given a triplet $(u; i, j)$, $w$ is a Bernoulli random variable with parameter $g(x^*_{u,i} - x^*_{u,j})$, where $g: \Real{} \rightarrow (0,1)$ is a known \textit{link function} that translates real-valued preferences to a binary scale. In the \textit{noiseless} setting, $w$ is set to the expected value of the corresponding noisy case; \textit{i.e.}, $w = g(x^*_{u,i} - x^*_{u,j})$. 

In this work, we assume we are given noiseless data. We assume the link function is a smooth, strictly increasing function and is symmetric around zero in the following sense: $g(-x) = 1 - g(x)$. For example, $g(x)$ could be the logistic link function: $e^x/(1 + e^x)$; this is the link function found in the Bradley-Terry-Luce choice model. 


\subsubsection{Important Parameters}
 

\paragraph{Incoherence} For any matrix $Z$, let $\norm{Z}_{2, \infty}$ denote the maximum of the $\ell_2$ norm of its rows and let $\norm{Z}_{F}$ denote the Frobenius norm of $Z$. Define the \textit{incoherence parameter} of the ground-truth matrix as 
\begin{align}\label{eq:def_mu}
    \mu \triangleq n(\norm{Z^*}_{2, \infty}^2/\norm{Z^*}_{F}^2).
\end{align}
In principle, $\mu$ can take values from $1$ to $n$. However, the sample complexity worsens with $\mu$.

\paragraph{Link Function Bounds}  Let $I$ denote the interval {$[-{24 \mu (\norm{Z^*}_F^2}/{n}), {24 \mu (\norm{Z^*}_F^2}/{n})]$.} %
Let $\xi$ and $\Xi$ be lower and upper bounds for the following expression:
\begin{align}\label{eq:link_function_lower_bound}
    \xi &\triangleq  \min_{(x, y) \in I \times I} \frac{g'(x)g'(y)}{g(x)(1 - g(x))}, \\ 
    \Xi &\triangleq \max_{(x, y) \in I \times I} \frac{g'(x)g'(y)}{g(x)(1 - g(x))}. \label{eq:link_function_upper_bound}
\end{align}
By the assumptions on $g(\cdot)$ stated above, $\xi$ is strictly positive and $\Xi$ is finite.
For the logistic link function, $g'(x)=g(x)(1-g(x))$, which implies $\xi = g'(24 \mu (\norm{Z^*}_F^2/n))$ and $\Xi = 1/4$.


\subsection{The Loss Function}\label{sec:loss_function}
Given any $Z \in \Real{n \times r}$, we interpret $Z$ as the concatenation of some candidate user features $U \in \Real{n_1 \times r}$ and item features $V \in \Real{n_2 \times r}$. 
The likelihood of the dataset $\Dataset$ under $Z$ is simply the probability of observing $\Dataset$ if the data was generated according to the parameters $Z$. 
In this work, we use the maximum likelihood approach to learn the latent parameters. \textit{I.e.,} we use the negative log likelihood as the loss function, which we shall minimize using a gradient-descent-like method. 
Here, we present the loss function and its gradient, using notation that will be useful later on.

Let $e_1, e_2, \ldots e_{n_1}$ denote unit vectors in $\Real{n_1}$ and let $\Tilde{e}_1, \Tilde{e}_2, \ldots, \Tilde{e}_{n_2}$ denote unit vectors in $\Real{n_2}$. Let $\llangle C, D \rrangle = \sum_{i,j} c_{i,j}d_{i,j}$ denote the matrix inner product between two matrices of the same size. Therefore:
\begin{align}\label{eq:def_A1}
    \llangle e_u(\Tilde{e}_i - \Tilde{e}_j)^T, X^* \rrangle = x^*_{u,i} - x^*_{u,j}.
\end{align}

For any triplet $(u; i, j)$, define the corresponding \textit{sampling matrix} $A \in \Real{n \times n}$ to be:
\begin{align}\label{eq:def_A2}
    A = \begin{bmatrix}
        0 & e_u(\Tilde{e}_i - \Tilde{e}_j)^T\\
        0 & 0
    \end{bmatrix} \Rightarrow \llangle A, Y^* \rrangle = x^*_{u,i} - x^*_{u,j}.
\end{align}
In the equation above, $0$ denotes matrices with all entries zero of the appropriate size. With this notation, for any data point $((u; i, j), w)$, we have:
\begin{align}
    \mathbb{P}(w = 1 \, | \, (u; i, j)) = \mathbb{P}(w = 1 \, | \, A) = g(\llangle A, Y^* \rrangle).
\end{align}


Given a binary outcome $w$, the likelihood of the outcome under a Bernoulli distribution with parameter $p$ is $p^{w}(1-p)^{1-w}$. Therefore, the negative log-likelihood of this observation is $-w\log(p) -(1-w)\log(1-p)$. Next, consider a datapoint $((u; i,j), w)$ with the corresponding sampling matrix $A$. The negative log-likelihood of this observation under our model with parameters $Z \in \Real{n \times r}$ is 
\begin{align*}
    -w \log(g(\llangle A, ZZ^T \rrangle)) - (1-w) \log(1 - g(\llangle A, ZZ^T \rrangle)).
\end{align*}
Let $A_k$ denote the sampling matrix corresponding to the datapoint $\mathcal{D}_k$. Then, for the entire dataset, the (normalized) negative log likelihood is given by:
\begin{align}\label{eq:log_likelihood}
    \Loglikelihood(Z) &= \frac{1}{m} \sum_{k = 1}^{m} -w_k \log(g(\llangle A_k, ZZ^T \rrangle)) \nonumber \\
    &\quad - (1-w_k) \log(1 - g(\llangle A_k, ZZ^T \rrangle)).
\end{align}
The gradient of $\Loglikelihood(Z)$ is 
\begin{align} \label{eq:gradient_likelihood}
    \nabla \Loglikelihood(Z) &= \frac{1}{m}\sum_{k=1}^{m}
    h_k (A_k+ A_k^T) Z, \text{ where }\\
    h_k &\triangleq \frac{g'(z_k)\left(g(z_k) - w_k\right)}{g(z_k)(1-g(z_k))}, \ z_k \triangleq \llangle A_k,ZZ^T \rrangle. \nonumber
\end{align}
Here, $\nabla \Loglikelihood(Z)$ is a matrix of the same size as $Z$ while $h_k$ and $z_k$ are scalars. %


\subsection{Symmetries in the Problem}\label{sec:symmetries}
The generative model, and consequently the log likelihood function, is invariant to certain transformations in the parameters. In other words, the problem structure has certain symmetries. We explore these symmetries and their consequences in this section.

\paragraph{Scale Invariance} 
For any ground-truth score matrix $X^*$, the factorization $(U^*, V^*)$ is not unique.
Indeed, for any invertible $r \times r$ matrix $P$, the pair of feature matrices $(U^*P^T, V^*P^{-1})$ is indistinguishable from $(U^*, V^*)$ as they both lead to the same score matrix $X^*$. However, we can distinguish `imbalanced' feature vectors from `balanced' ones by by adding the term $\norm{U^TU - V^TV}_F^2$ to the loss function. Minimizing this regularizer while keeping the log-likelihood constant leads to a pair of feature matrices that are balanced in the norms. In more compact terms, the regularizer can be written as follows:
\begin{align}\label{eq:def_regularizer}
    \mathcal{R}(Z) \triangleq \norm{Z^TDZ}_F^2; \ D \triangleq \begin{bmatrix}
        I_{n_1} & 0\\
        0 & -I_{n_2}
    \end{bmatrix}.
\end{align}
Note that the ground-truth matrix $Z^*$ satisfies $\mathcal{R}(Z^*) = 0$. Combining the regularizer with the negative log likelihood, the objective function becomes:
\begin{align}\label{eq:objective_function}
    f(Z) \triangleq \mathcal{L}(Z) + (\lambda/4)\mathcal{R}(Z),
\end{align}
where $\lambda$ is a positive constant. In this work, we set $\lambda = \xi\gamma/4$; however, in practice, it may be viewed as a hyperparameter. In summary, adding the regularizer $\mathcal{R}(Z)$ factors out the scale-invariance of the problem.

\paragraph{Rotational Invariance}
Beyond the scale invariance, the problem at hand also exhibits rotational invariance.
Let $R$ be any orthogonal matrix in $r$ dimensions, \textit{i.e.}, $R \in \mathbb{R}^{r \times r}$ such that $RR^T=R^TR=I$. The pair of feature matrices $(U^*R, V^*R)$ give rise to the same scores as $(U^*, V^*)$. Thus, one can identify the ground-truth features only up to an orthogonal transformation. Denote this equivalence class of the ground-truth feature matrices by $\solset$:
\begin{align}\label{eq:def_solutionset}
    \solset \triangleq \{ \Tilde{Z}^* \, : \Tilde{Z}^* = Z^* R \text{ for some } \text{orthogonal } R\}.
\end{align}
This equivalence class of solutions naturally gives rise to a new distance metric that measures how close a candidate solution $Z$ is to $\solset.$
Define
\begin{align}\label{eq:def_distance}
     R(Z) &\triangleq \text{arg} \min_{R: R^TR = RR^T = I_r} \norm{Z - Z^* R}_F, \\
    \solz &\triangleq \text{arg} 
    \min_{\Tilde{Z}^* \in \solset} \norm{Z - \Tilde{Z}^*}_F = Z^* R(Z), \label{eq:def_closest_sol} \\
     \delz &\triangleq Z - \solz. \label{eq:def_difference_sol}
\end{align}
We measure the quality of a solution $Z$ by $\norm{\delz}_F$.


\paragraph{Shift Invariance} Comparisons invariably involve computing the difference between the utilities of items. Therefore, the learning problem is invariant to a global shift in the scores. Mathematically, this can be seen as follows. Let $\Tilde{V}^* = V^* + 1v^T$, where $v \in \Real{r}$ and $1 \in \Real{n_2}$ is the vector of all ones. Let $\Tilde{X}^*, \Tilde{Y}^*,$ and $\Tilde{Z}^*$ denote the corresponding quantities derived from $(U^*, \Tilde{V}^*)$. Then for any triplet $(u; i, j)$ and the corresponding sampling matrix $A$, we have $\llangle e_u(\Tilde{e}_i - \Tilde{e}_j)^T, \Tilde{X}^* \rrangle = \llangle e_u(\Tilde{e}_i - \Tilde{e}_j)^T, X^* \rrangle$, which implies $\llangle A, \Tilde{Y}^* \rrangle = \llangle A, Y^* \rrangle.$ Because of this invariance, we assume, without loss of generality, that $1^T V^* = 0$. In words, we assume that the item features of all matrices in $\solset$ sum to zero.

The shift invariance also manifests itself in our objective function $\Loglikelihood(Z)$. It is important to factor out the shift invariance in order to establish a strong-convexity like property (i.e., a curvature) for $\Loglikelihood(Z)$. Therefore, we restrict our attention to the following subspace:
\begin{align}\label{eq:H_hyperplane}
    \mathcal{H} = \{ Z \in \Real{n \times r}: Z = (U, V), 1^TV = 0 \}.
\end{align}
For any $Z = (U, V)$, we shall work with the projection of $Z$ onto $\mathcal{H}$, denoted by $\mathcal{P}_{\mathcal{H}}(Z)$. This projection is given by $(U, JV)$, where $J \triangleq I_{n_2} - 11^T/(n_2)$. Finally, note that by the assumption stated before, $\solset \subseteq \mathcal{H}$.




\section{\system{}}
\label{sec:respark}

This section describes the implementation of \system{}, including pre-processing, analysis, and organization. 
\system{} utilizes the GPTs from Azure to incorporate the LLM-driven functionalities. 
Specifically, we employ the ``gpt-4-vision-preview'' model as our input involves chart images. 
The prompts are provided in the supplementary materials. 

\subsection{Pre-processing Stage}

Before proceeding with analysis and organization, we need to pre-process the user dataset and reports to (1) acquire necessary data features, (2) recommend the most relevant reports to the data, and (3) extract the analytical objective and their dependencies of the selected report for the subsequent processes. 

\subsubsection{Data Pre-processing}
\label{subsubsec:data_pre_processing}

Based on the findings in the preliminary study (~\autoref{subsec:preliminary_study}), most adjustments entail considerations of data features such as context, scope, fields, and formats. 
Presenting the entire dataset to LLMs is currently impractical due to token limitations, and it does not facilitate a comprehensive understanding of these data features. 
Therefore, we utilize a similar data summary method to LIDA~\cite{dibia2023lida}. 
This method first extracts scope, data type, and unique value count information and samples some values for each data field. 
Subsequently, it employs LLMs to provide brief semantic descriptions for the dataset and each data field. 
The description of the dataset can also help recommend relevant reports(~\autoref{subsubsec:report_retrieval}). 
These pieces of information are then integrated to form a comprehensive data summary.

\subsubsection{Report Pre-processing}
\label{subsubsec:report_pre_processing}

\begin{figure*}[!htb] 
  \centering
  \includegraphics[width=\linewidth]{figs/system.png}
  \caption{
  The interface of \system{}. 
  \system{} consists of four views: data view (b-c), dependency view (d-e), content view (f-g), and generation view (h-k). 
  The data view displays the overall description and data field information. 
  The dependency view displays the extracted interdependent report segments. 
  The content view shows the analytical objective and content of the selected segment. 
  The generation view demonstrates the generated results in real-time. 
  }
  \label{fig:interface}
\end{figure*}

Based on the analysis workflow formulation in~\autoref{subsec:problem_formulation}, our goal is to deduce the analysis segments and their dependencies from the original report for subsequent execution. 
Our preliminary study showed that most reports present analytical content in distinct segments, each focused on a single objective, with related text and visuals grouped together. 
Therefore, ideally, we can find a segmentation that aligns each report section with a specific analysis segment. 
In this light, \system{} should segment the report accordingly, extract the analytical objectives for each segment, and deduce their dependencies.

To achieve this goal, an appropriate report segmentation criteria is very important, as it directly determines the entire analysis workflow (consisting of a sequence of analysis segments). 
The accuracy of segmentation also affects the quality of the extracted analytical objectives and dependencies.

We were initially inspired by prior work in automatic storytelling and insight-mining, which formalizes data insights and their relationships~\cite{ma2023insightpilot, wang2019datashot}. 
For example, Calliope~\cite{shi2020calliope} defines a data story as a series of interrelated insights, with each insight describing data patterns in specific data fields and subsets. 
For instance, ``The average worldwide gross for action movies is increasing over time'' describes an increasing trend measuring ``average (worldwide gross)'' over the breakdown ``year'' within the subset ``genre = action''. 
Based on this definition, we can potentially segment the report by identifying the insight type, measure, breakdown, subset, etc., and combine the insights into segments based on these labels. 
However, we found this definition challenging for segmenting practical data reports, as it doesn't accommodate the flexibility of analyses that involve deeper data transformations, such as creating new fields and describing patterns in derived variables.

Therefore, we need to define new segmentation criteria that accommodate the flexible analysis in the data report. 
Instead of formally defining a ``segment'' or an ``analytical objective'' with a strict data model, we provide a loose description of how segments can be divided and use LLMs to perform segmentation. 
Our preliminary study indicated that most report segments consist of continuous text and possibly a chart.
Based on the study, we execute report segmentation, extract analytical objectives, and deduce the dependencies between segments through the following approach: 
\begin{itemize}
    \item \textbf{Match. } 
    First, for each chart, we match the related paragraph text to form a segment. 
    Based on our preliminary study, we assume that (1) each paragraph corresponds to the nearest preceding or following chart, or none at all, and (2) all text associated with a single chart is continuous. 
    Starting with the first paragraph, LLMs determine whether it matches the nearest preceding or following chart (e.g., describing insights from the chart) or if it doesn't relate to any chart.
    
    \item \textbf{Categorize. } 
    For text that doesn't match a chart, LLMs categorize it to determine if it involves data analysis or serves another purpose, such as providing background information. 
    For continuous text segments that involve data analysis, we further assess whether they belong to the same segment (describe insights derived from the same transformed data). 
    
    \item \textbf{Summarize. } 
    After matching and categorizing, LLMs summarize the analytical objective of each segment and deduce its dependencies with previous segments. 
    We use the six logical relations defined in Calliope to outline dependencies among report segments. 
    LLMs determine whether new content is logically connected to an existing segment or originates directly from the data.
\end{itemize}


\subsubsection{Relevant Report Retrieval}
\label{subsubsec:report_retrieval}


With the pre-processed data and reports, users can select a reference report to analyze the target dataset. 
Based on findings from our preliminary study, various aspects of existing data reports, such as analytical objectives and report content, can serve as helpful reference material. 
However, since the reference report's data may differ from the target dataset, adjustments are necessary to align with the new dataset. 
The closer the reference report's data is to the target dataset, the more aspects can be reused without modification, making the report more suitable for use as a reference.


To facilitate the retrieval of suitable reports, we aim to identify the reports with data similar to the target dataset. 
The core idea is to convert both the dataset and report information into vector embeddings, compute their cosine similarities, and rank the reports from highest to lowest score.
The key question is: which specific information from the dataset and reports should be embedded?
We propose two mechanisms for extracting the embedding of data and report information: topic relevance and field similarity.
\begin{itemize}
    \item \textbf{Topic relevance} refers to the alignment between the topic of the dataset and the report. 
    Typically, datasets and reports within the same domain (e.g., health, economy) exhibit higher topic relevance. 
    For example, a sales dataset is highly topic-relevant to a report analyzing market sales trends. 
    To compute topic relevance, we extract the embedding of the dataset’s name and description, along with the headings and pre-processed analytical objectives of the report. 
    We hypothesize that these elements are semantically related to the corpus's overall topic, and the cosine similarity of their embeddings can reflect their topic relevance.
    \item \textbf{Field similarity} pertains to the alignment of the data fields described in the report with those contained in the dataset. 
    For example, a report on voting intentions across different gender and age groups would exhibit higher field similarity with a dataset containing gender and age information. 
    To compute field similarity, we embed the names and descriptions of the dataset's fields. 
    For the reports, we use LLMs to infer the data fields discussed in the report and embed these inferred fields along with their descriptions. 
    We hypothesize that the cosine similarity between these reflects the degree of field similarity.
\end{itemize}
Finally, we sum the scores from both mechanisms for each report, ranking them from highest to lowest, allowing users to select the most appropriate reference reports.

\subsection{Analysis Stage}
\label{subsec:analysis_stage}

Through the pre-processing stage, we obtain the summary of the new dataset and the segments of the existing report. 
Each segment corresponds to an analytical objective, a dependency on the previous segment or the data, and pieces of report content, including text and charts. 
The next stage is to reproduce the analysis workflow by re-executing each segment on the new data, encompassing reusing and reconstructing the analytical objective, analysis operations, and report contents. 

\subsubsection{Analytical objective correction and insertion}

The analysis workflow is driven by a series of posed analysis objectives. 
Through the pre-processing stage, we obtain the analysis workflow of the existing report by dividing it into segments and extracting each segment's analytical objectives and dependencies. 
To adapt the workflow to new data, \system{} is required to (1) correct the extracted analytical objectives and (2) support the insertion of new objectives according to the data features and segment dependencies. 

\textbf{Analytical objective correction. }
The preliminary study indicates that while existing analytical objectives often remain applicable, alterations or removals may be necessary due to data fields, dependencies, or the data context and scope. 

First, the original objective might involve data fields absent in the new data. 
Given the pre-processed data summary, we employ LLMs to evaluate if the new dataset's fields sufficiently fulfill the objective, considering semantic similarities despite word-to-word differences in field names. 
For example, an objective mentioning ``earn money'' can be related to the data field ``gross.'' 
LLMs are tasked with explaining their decisions to enhance their reasoning~\cite{mialon2023augmented}. 
If the available fields suffice, LLMs should describe the required fields and analysis operations. 
Otherwise, LLMs must explain what external fields are needed to satisfy the objective. 
In such cases, we correct the objective by replacing missing fields with available alternatives. 
For instance, if a movie dataset lacks geographic data, the objective of locating the highest-grossing movies might shift focus to their directors.

Second, the original objective may derive from insights in a prior segment. 
Adjustments might stem from two scenarios. 
If the insight's nature changes (e.g., from an increasing to a decreasing trend), a corresponding shift is needed in the latter related objective (e.g., from identifying causes of growth to exploring reasons behind the downturn). 
Therefore, we provide the model with the newly generated results of the dependent segment and require it to identify and adapt to such variations. 
Additionally, the dependency may call for a context or scope that the data cannot satisfy, such as from local to national or from a 5-year trend to a 20-year trend. 
LLMs must infer whether changes in scope or context affect the objective's applicability, which could lead to its potential removal if the new data does not support similar adjustments. 

Third, minor adjustments are often required for data context and scope adjustments. 
For example, an objective focusing on a 5-year trend needs adjustment to fit a dataset covering only the past three years. 
LLMs should make these modifications based on the context and scope of the provided data.

\textbf{Analytical objective insertion. }
The uniqueness of new datasets and user-driven queries may necessitate adding fresh analytical objectives based on previous insights and dependencies. 
\system{} enables users to embed new objectives at chosen positions, rooted in the data or reliant on preceding analysis segments. 
Users can define the focus data fields and dependencies of these new objectives, and LLMs can suggest potential objectives based on user input.

\subsubsection{Analysis Operation Generation}

% \TODO{code structure, generate a table and a chart}. 
Once the analytical objective has been refined, \system{} generates the requisite analysis operations to fulfill this objective. 
Since these operations are not explicitly detailed in the report, we utilize the code-generation capabilities of LLMs for this phase. 

LLMs are prompted to generate analysis code that aligns with the clarified objective, provided with the data summary and original report content as guides. 
The model is instructed to refer to the original report to deduce the necessary data transformations. 
We also remind the model to generate the code that accommodates the new data, as the reference report content is from a different dataset and only serves as a reference for expected output. 
The model is required first to plan step by step~\cite{kojima2022large} and then generate the Python code that results in transformed data and a chart using matplotlib~\cite{Hunter2007matplotlib} or Seaborn~\cite{Waskom2021seaborn}. 

Upon code generation, we execute it to procure the transformed data and the accompanying chart. 
The execution may also raise errors. 
We relay any execution results, including the transformed data, chart, and potential errors, back to the LLMs. 
The model then assesses whether the code execution is successful and whether the results accurately address the analytical objective and are adequate for generating report content. 
Should the model deduce that revisions are necessary, the cycle of code generation and execution is repeated until satisfactory results are obtained, paving the way for report content creation.

\subsubsection{Report Content Production}

With the execution results in hand, we proceed to generate new report content. 
Given that the code already produces the chart, the model's task in this step is to generate the accompanying textual narrative. 
We instruct the model to produce a narrative that imitates the writing style of the reference report yet is tailored to fit the new data context and the insights derived from the executed analysis. 
We also enable user modification to the report content. 

\subsection{Organization Stage}

After reproducing the analysis workflow and obtaining the new data insights, the next step is to structure the new report. 
As we generate the sequence of segments based on the order of dependencies, the implicit logical structure is adopted naturally. 
Additionally, we inherit the explicit structural elements (such as titles and sections) from the original report. 
Newly inserted analytical objectives are incorporated along with their dependent segments. 
The report and its sections' titles are re-crafted based on the original ones, incorporating new data insights to guide the title generation process. 
User interventions are also supported, allowing for the reorganization of segments into new sections, thereby tailoring the report structure to meet user needs or preferences better.



\section{Data}

\subsection{Pre-training Data}
We constructed a large-scale video dataset comprising 2B video-text pairs and 3.8B image-text pairs. Leveraging a comprehensive data pipeline, we transformed raw videos into high-quality video-text pairs suitable for model pre-training. As illustrated in Figure~\ref{fig:data_pipeline}, our pipeline consists of several key stages: Video Segmentation, Video Quality Assessment, Video Motion Assessment, Video Captioning, Video Concept Balancing and Video-Text Alignment. Each stage plays a crucial role in constructing the dataset, and we describe them in detail below.

\begin{figure*}[t]
    \centering
    \includegraphics[width=1.3\textwidth, center, trim=0 0 0 0, clip]{figure/data/data_pipeline.png}
    \caption{The pipeline of Step-Video-T2V data process.}
    \label{fig:data_pipeline}
    %\vspace{-6mm}
\end{figure*}


\paragraph{Video Segmentation} 
We began by processing raw videos using the \textit{AdaptiveDetector} function in the \texttt{PySceneDetect}~\citep{pyscenedetect} toolkit to dentify scene changes and use FFmpeg~\citep{ffmpeg} to split them into single-shot clips. 
We adjusted the splitting process for high-resolution videos not encoded with \texttt{libx264} to include necessary reference frames—specifically by properly setting the crop start time in \texttt{FFmpeg}; this prevented visual artifacts or glitches in the output video.
We also removed the first three frames and the last three frames of each clip, following practices similar to Panda70M \cite{chen2024panda} and Movie Gen Video \cite{polyak2024moviegencastmedia}. Excluding these frames eliminates unstable camera movements or transition effects often present at the beginnings and endings of videos.


\paragraph{Video Quality Assessment}

To construct a refined dataset optimized for model training on high-quality, we systematically evaluated and filtered video clips by assigning multiple Quality Assessment tags based on specific criteria. We uniformly sampled eight frames from each clip to compute these tags, providing a consistent and comprehensive assessment of each video.

\begin{itemize}[left=0cm]
\item \textbf{Aesthetic Score}: We used the public LAION CLIP-based aesthetic predictor~\cite{schuhmann2022laion} to predict the aesthetic scores of eight frames from each clip and calculated their average.

\item \textbf{NSFW Score}: We employed the public LAION CLIP-based NSFW detector~\cite{laion2021nsfw}, a lightweight two-class classifier using CLIP ViT-L/14 embeddings, to identify content inappropriate for safe work environments.

\item \textbf{Watermark Detection}: Employing an EfficientNet image classification model~\cite{tan2019efficientnet}, we detected the presence of watermarks within the videos.

\item \textbf{Subtitle Detection}: Utilizing PaddleOCR~\cite{paddleocr}, we recognized and localized text within video frames, identifying clips with excessive on-screen text or captions.

\item \textbf{Saturation Score}: 
We assessed color saturation by converting video frames from BGR to HSV color space and extracting the saturation channel, using OpenCV \cite{opencv_library}. We computed statistical measures—including mean, maximum, and minimum saturation values—across the frames. 


\item \textbf{Blur Score}: 
We detect blurriness by applying the variance of the Laplacian method~\cite{pech2000diatom} to measure the sharpness of each frame. Low variance values indicate blurriness caused by camera shake or lack of clarity.

\item \textbf{Black Border Detection}: We use \texttt{FFmpeg} to detect black borders in frames and record their dimensions to facilitate cropping, ensuring that the model trains on content free of distracting edges.
\end{itemize}



\paragraph{Video Motion Assessment}

Recognizing that motion content is crucial for representing dynamic scenes and ensuring effective model training, we calculate the motion score by averaging the mean magnitudes of the optical flow \cite{opencv_library} between pairs of resized grayscale frames, using the Farneback algorithm. We introduced three evaluative tags centered around motion scores:


\begin{itemize}[left=0cm]
    \item \textbf{Motion\_Mean}: The average motion magnitude across all frames in the clip, indicating the general level of motion. This score helps us identify clips with appropriate motion; clips with extremely low \texttt{Motion\_Mean} values suggest static or slow motion scenes that may not effectively contribute to training models focused on dynamic content.

    \item \textbf{Motion\_Max}: The maximum motion magnitude observed in the clip, highlighting instances of extreme motion or motion distortion. High \texttt{Motion\_Max} values may indicate the presence of frames with excessive or jittery motion.

    \item \textbf{Motion\_Min}: The minimum motion magnitude in the clip, identifying clips with minimal motion. Clips with very low \texttt{Motion\_Min} may contain idle frames or abrupt pauses, which could be undesirable for training purposes.
\end{itemize}





\paragraph{Video Captioning} 
Recent studies~\citep{openaisora, betker2023improving} have highlighted that both precision and richness of captions are crucial in enhancing the prompt-following ability and output quality of generative models. 
 
Motivated by this, we introduced three types of caption labeling into our video captioning process by employing an in-house Vision Language Model (VLM) designed to generate both short and dense captions for video clips.
\begin{itemize}[left=0cm]
    \item \textbf{Short Caption}: The short caption provides a concise description, focusing solely on the main subject and action, closely mirroring real user prompts.

    \item \textbf{Dense Caption}: The dense caption integrates key elements, emphasizing the main subject, events, environmental and visual aspects, video type and style, as well as camera shots and movements. To refine camera movements, we manually collected annotated data and performed SFT on our in-house VLM, incorporating common camera movements and shooting angles.
    
    \item \textbf{Original Title}: We also included a variety of caption styles by incorporating a portion of the original titles from the raw videos, adding diversity to the captions.
    
\end{itemize}



\paragraph{Video Concept Balancing}
To address category imbalances and facilitate deduplication in our dataset, we computed embeddings for all video clips using an internal VideoCLIP model and applied K-means clustering \cite{macqueen1967some} to group them into over 120,000 clusters, each representing a specific concept or category. By leveraging the cluster size and the distance to centroid tags, we balanced the dataset by filtering out clips that were outliers within their respective clusters. As part of this process, we added two new tags to each clip:

\begin{itemize}[left=0cm] 
    \item \textbf{Cluster\_Cnt}: The total number of clips in the cluster to which the clip belongs.

    \item \textbf{Center\_Sim}: The cosine distance between the clip's embedding and the cluster center.
\end{itemize}



\paragraph{Video-Text Alignment}

Recognizing that accurate alignment between video content and textual descriptions is essential to generate high-quality output and effective data filtering, we compute a \textbf{CLIP Score} to measure video-text alignment. This score assesses how well the captions align with the visual content of the video clips.

\begin{itemize}[left=0cm] 
\item \textbf{CLIP Score}: We begin by uniformly sampling eight frames from the given video clip. Using the CLIP model~\cite{yang2022chineseclip}, we then extract image embeddings for these frames and a text embedding for the video caption. The \texttt{CLIP Score} is computed by averaging the cosine similarities between each frame embedding and the caption embedding.



\end{itemize}


\subsection{Post-training Data}


For SFT in post-training, we curate a high-quality video dataset that captures good motion, realism, aesthetics, a broad range of concepts, and accurate captions. Inspired by \cite{dai2023emu, polyak2024moviegencastmedia, kong2024hunyuanvideo}, we utilize both automated and manual filtering techniques:

\begin{itemize}[left=0cm]


\item \textbf{Filtering by Video Assessment Scores}: Using video assessment scores and heuristic rules, we filter the entire dataset to a subset of 30M videos, significantly improving its overall quality.


\item \textbf{Filtering by Video Categories}: For videos within the same cluster, we use the "Distance to Centroid" values to remove those whose distance from the centroid exceeds a predefined threshold. This ensures that the resulting video subset contains a sufficient number of videos for each cluster while maintaining diversity within the subset.



\item \textbf{Labeling by Human Annotators}: In the final stage, human evaluators assess each video for clarity, aesthetics, appropriate motion, smooth scene transitions, and the absence of watermarks or subtitles. Captions are also manually refined to ensure accuracy and include essential details such as camera movements, subjects, actions, backgrounds, and lighting.

\end{itemize}



\section{Training Strategy}
\begin{table}[h!]
\centering
\resizebox{\textwidth}{!}{
\begin{tabular}{ccccccc}
\hline training stage & dataset & bs/node & learning rate & \#iters & \#seen samples \\
\hline 
\hline
    \multirow{3}{*}{Step-1: T2I Pre-training (256px)} & $\mathcal{O}(1) \mathrm{B}$ images & 40 & 1e-4 & 53k & 0.8B \\
     & $\mathcal{O}(1) \mathrm{B}$ images & 40 & 1e-4 & 200k & 3B \\
     \cline{2-6}
     & \textbf{Total} &  &  &  \textbf{253k} & \textbf{3.8B} \\
\hline 
\hline
    \multirow{4}{*}{Step-2: T2VI Pre-training (192px)} & $\mathcal{O}(1) \mathrm{B}$ video clips & 4 & 6e-5 & 171k & 256M\\
    & $\mathcal{O}(100) \mathrm{M}$ video clips & 4 & 6e-5 & 101k & 151M \\
    & $\mathcal{O}(100) \mathrm{M}$ video clips & 4 & 6e-5 & 158k & 237M \\
    \cline{2-6}
    & \textbf{Total} &  &  &  \textbf{430k} & \textbf{644M} \\
\hline
\hline
    \multirow{4}{*}{Step-2: T2VI Pre-training (540px)} & $\mathcal{O}(100) \mathrm{M}$ video clips & 2 & 2e-5 & 23k & 17.3M\\
    & $\mathcal{O}(10) \mathrm{M}$ video clips & 2 & 1e-5 & 17k & 8.5M \\
    & $\mathcal{O}(1) \mathrm{M}$ video clips & 1 & 1e-5 & 6k & 1.5M \\
    \cline{2-6}
    & \textbf{Total} &  &  &  \textbf{46k} & \textbf{27.3M} \\
\hline
\end{tabular}
}
\caption{Pre-training details of Step-Video-T2V. 256px, 192px, and 540px denote resolutions of 256x256, 192x320, and 544x992, respectively.}
\label{trainingrecipe}
\end{table}



\begin{figure}[h] 
    \centering
    \includegraphics[width=0.5\textwidth]{figure/v2_training_loss.png}  
    \caption{Training curve of different training stages, where $s_{i}$ denotes the $i^{th}$ dataset used in the corresponding stage.} 
    \label{fig:training curve}  
\end{figure}

A cascaded training strategy is employed in Step-Video-T2V, which mainly includes four steps: text-to-image (T2I) pre-training, text-to-video/image (T2VI) pre-training, text-to-video (T2V) fine-tuning, and direct preference optimization (DPO) training. The pre-training recipe is summarized in Table~\ref{trainingrecipe}.

\paragraph{Step-1: T2I Pre-training} In the initial step, we begin by training Step-Video-T2V with a T2I pre-training approach from scratch. We intentionally avoid starting with T2V pre-training directly, as doing so will significantly slow down model convergence. This conclusion stems from our early experiments with the T2V pre-training from scratch on the 4B model, where we observed that the model struggled to learn new concepts and was much slower to converge. By first focusing on T2I, the model can establish a solid foundation in understanding visual concepts, which can later be expanded to handle temporal dynamics in the T2V phase.

\paragraph{Step-2: T2VI Pre-training} After acquiring spatial knowledge from T2I pre-training in Step-1, Step-Video-T2V progresses to a T2VI joint training stage, where both T2I and T2V are incorporated. This step is further divided into two stages. In the first stage, we pre-train Step-Video-T2V using low-resolution (192x320, 192P) videos, allowing the model to primarily focus on learning motion-related knowledge rather than fine details. In the second stage, we increase the video resolution to 544x992 (540P) and continue pre-training to enable the model to learn more intricate details. We observed that during the first stage, the model concentrates on learning motion, while in the second stage, it shifts its focus more toward learning fine details. Based on these observations, we allocate more computational resources to the first stage in Step-2 to better capture motion knowledge.

\paragraph{Step-3: T2V Fine-tuning} Due to the diversity in pre-training video data across different domains and qualities, using a pre-trained checkpoint usually introduces artifacts and varying styles in the generated videos. To mitigate these issues, we continue the training pipeline with a T2V fine-tuning step. In this stage, we use a small number of text-video pairs and remove T2I, allowing the model to fine-tune and adapt specifically to text-to-video generation.


Similar to Movie Gen Video, we found that averaging models fine-tuned with different SFT datasets improves the quality and stability of the generated videos, outperforming the Exponential Moving Average (EMA) method. Even averaging checkpoints from the same data source enhances stability and reduces distortions. Additionally, we select model checkpoints based on the period after the gradient norm peaks, ensuring both the gradient norm and loss have decreased for improved stability.

\paragraph{Step-4: DPO Training}
As described in \S\ref{dpo}, video-based DPO training is employed to enhance the visual quality of the generated videos and ensure better alignment with user prompts.



\paragraph{Hierarchical Data Filtering}
\begin{figure*}[t]
    \centering
    \includegraphics[width=1.3\textwidth, center, trim=0 0 0 0, clip]{figure/data/data_filter.png}
    \caption{Hierarchical data filtering for pre-training and post-training.}
    \label{fig:data_filter}
    %\vspace{-6mm}
\end{figure*}


We apply a series of filters to the data, progressively increasing their thresholds to create six pre-training subsets for Step-2: T2VI Pre-training, as shown in Table~\ref{trainingrecipe}. The final SFT dataset is then constructed through manual filtering. Figure~\ref{fig:data_filter} illustrates the key filters applied at each stage, with gray bars representing the data removed by each filter, and colored bars indicating the remaining data at each stage.



\paragraph{Observations from Pre-training Curve}
%We use a funnel-style filtering method to progressively refine the training dataset throughout the pre-training stage. 
During pre-training, we observe a notable reduction in loss, which correlates with the improved quality of the training data, as illustrated in Figure \ref{fig:training curve}.

Additionally, a sudden drop in loss occurs as the quality of the training dataset improves. This improvement is not directly driven by supervision through a loss function during model training, but rather follows human intuition (e.g., filtering via CLIP scores, aesthetic scores, etc.). While the flow matching algorithm does not impose strict requirements on the distribution of the model’s input data, adjusting the training data to reflect what is considered higher-quality by humans results in a significant, stepwise reduction in training loss. This suggests that, to some extent, the model’s learning process may emulate human cognitive patterns.


\paragraph{Bucketization for Variable Duration and Size}

To accommodate varying video lengths and aspect ratios during training, we employed variable-length and variable-resolution strategies~\cite{chen2023pixartalphafasttrainingdiffusion, opensora}. We defined four length buckets (1, 68, 136, and 204 frames) and dynamically adjusted the number of latent frames based on the video length. Additionally, we grouped videos into three aspect ratio buckets—landscape, portrait, and square—according to the closest height-to-width ratio.




\begin{figure*}[!ht]
\begin{center}
\centering
    \subfloat[{\it Video caption: A green turtle swimming under the sea.}]{
    \includegraphics[width=0.95\textwidth]{gen/turtle}} \\
    \subfloat[{\it Video caption: Viewing countless sunflowers in a field from top.}]{
    \includegraphics[width=0.95\textwidth]{gen/sunflower}}
\end{center}
\caption{Generated videos with different frame rates $\{8, 12, 16\}$. }
\label{fig:gen}
\ifdefined\isarxiv
\else
\vspace{-3mm}
\fi
\end{figure*}

\begin{figure*}[!ht]
\begin{center}
\centering
    \subfloat{
    \includegraphics[width=0.95\textwidth]{inter/lion}} \\
    \subfloat{
    \includegraphics[width=0.95\textwidth]{inter/aurora}} \\
    \subfloat{
    \includegraphics[width=0.95\textwidth]{extra/cloud}} \\
\end{center}
\caption{Interpolation and Extrapolation of VLFM.}
\label{fig:inter_extra}
\ifdefined\isarxiv
\else
\vspace{-2mm}
\fi
\end{figure*}

\section{Experiments}\label{sec:exp}

In this section, we conduct experiments to evaluate the effectiveness of our approach. We first introduce our experimental setups in Section~\ref{sub:exp_setup}. Then, we demonstrate text-to-video generation using VLFM and VLFM's capability of generating videos in arbitrary frame rate in Section~\ref{sub:exp_gen}. Furthermore, we showcase the strong performance of interpolation and extrapolation of VLFM in Section~\ref{sub:exp_inter_extra}. We also perform an ablation study to discuss the importance of the flow matching algorithm in Section~\ref{sub:exp_ablation}.

\subsection{Setup} \label{sub:exp_setup}

In our experiments, we apply Stable Diffusion v1.5 \cite{rbl+22} with DDIM scheduler \cite{sme20} as the visual decoder. Then, we use a DiT-XL-2 \cite{px23} as the backbone for the Flow Matching algorithm \cite{lcb+22,lgl22}, and the choice of hyper-parameters of $\sigma_t(\wt{u})$ is given by $\sigma_{\rm min} = 0.01$ and $\alpha = 10$. We optimize the DiT using Grams optimizer \cite{cls24}. We sample and combine 7 data resources for comprehensive training and validation of our method. They are:
OpenVid-1M \cite{nxz+24},
UCF-101 \cite{szs12},
Kinetics-400 \cite{kcs+17},
YouTube-8M \cite{akl+16},
InternVid \cite{whl+23},
MiraData \cite{jgz+24}, and
Pixabay \cite{pixabay}. 

\ifdefined\isarxiv
\else
\vspace{-4mm}
\fi

\subsection{Text-to-Video Generation with Arbitrary Frame Rate} \label{sub:exp_gen}

In this section, we recover several videos with different frame rates using VLFM with given video captions in the training dataset. We extract $T= 0.5$ for demonstrations as Figure~\ref{fig:gen}. In detail, we choose three frame rates for generation $\{8, 12, 16\}$. As shown, our VLFM performs fairly on text-to-video generation while it requires very small resource that is equivalent to training a new flow matching text-to-image video, which ensures its efficiency. Moreover, we give more results that are generated by VLFM in Appendix~\ref{sec:app:more_1} and \ref{sec:app:more_2}.
\ifdefined\isarxiv
\else
\vspace{-3mm}
\fi

\subsection{Interpolation and Extrapolation} \label{sub:exp_inter_extra}

In this section, we test the interpolation and extrapolation of VLFM. For the interpolation experiment, the model is trained with 24 FPS and evaluated to generate video with 48 FPS. For the extrapolation, the model is trained with the first video with $T = 2$ and evaluated to generate the whole video with $T = 8$. Referring the results in Figure~\ref{fig:inter_extra}, this demonstrates the strong performance of our VLFM under our mathematical guarantee of the error bound and its effectiveness.

\subsection{Ablation Study} \label{sub:exp_ablation}

In this section, we compared training VLFM with the Flow Matching algorithm and directly used DiT to predict the latent patches to showcase the importance of utilizing flow matching in our VLFM. We compare VLFM with and without flow matching by training the model with 1000 steps and compare the PSNR (peak signal-to-noise ratio) before and after training for video recovery with given captions in the training dataset. We state the results in Table~\ref{tab:ablation}. Denote ${\rm MSE}(x,y)$ as the mean squared error function, the computation of the metric PSNR is given by ($x,y \in \R^{r\times r}$):
\ifdefined\isarxiv
\else
\vspace{-3mm}
\fi
\begin{align*}
    {\rm PSNR}(x,y) := 10 \log_{10}(\frac{r^2}{{\rm MSE}(x,y)}), 
\end{align*}
\ifdefined\isarxiv
\else
\vspace{-3mm}
\fi

\begin{table}[!ht]
\ifdefined\isarxiv
\else
\vspace{-2mm}
\fi
\begin{center}
\begin{small}
\begin{sc}
\begin{tabular}{r | c c}
    \toprule
    Algorithm & Initial PSNR$\uparrow$ & Final PSNR$\uparrow$ \\
    \midrule
    Flow Matching & {\bf 57.20} & {\bf 61.18} \\
    Direct Predicting & 9.81 & 53.77 \\
    \bottomrule
\end{tabular}
\end{sc}
\end{small}
\end{center}
\caption{PSNR comparison (the greater, the better) of Flow Matching and direct generation from DiT. We boldface the better scores.}
\label{tab:ablation}
\ifdefined\isarxiv
\else
\vspace{-4mm}
\fi

\end{table}

\section{Discussion}\label{sec:discussion}



\subsection{From Interactive Prompting to Interactive Multi-modal Prompting}
The rapid advancements of large pre-trained generative models including large language models and text-to-image generation models, have inspired many HCI researchers to develop interactive tools to support users in crafting appropriate prompts.
% Studies on this topic in last two years' HCI conferences are predominantly focused on helping users refine single-modality textual prompts.
Many previous studies are focused on helping users refine single-modality textual prompts.
However, for many real-world applications concerning data beyond text modality, such as multi-modal AI and embodied intelligence, information from other modalities is essential in constructing sophisticated multi-modal prompts that fully convey users' instruction.
This demand inspires some researchers to develop multimodal prompting interactions to facilitate generation tasks ranging from visual modality image generation~\cite{wang2024promptcharm, promptpaint} to textual modality story generation~\cite{chung2022tale}.
% Some previous studies contributed relevant findings on this topic. 
Specifically, for the image generation task, recent studies have contributed some relevant findings on multi-modal prompting.
For example, PromptCharm~\cite{wang2024promptcharm} discovers the importance of multimodal feedback in refining initial text-based prompting in diffusion models.
However, the multi-modal interactions in PromptCharm are mainly focused on the feedback empowered the inpainting function, instead of supporting initial multimodal sketch-prompt control. 

\begin{figure*}[t]
    \centering
    \includegraphics[width=0.9\textwidth]{src/img/novice_expert.pdf}
    \vspace{-2mm}
    \caption{The comparison between novice and expert participants in painting reveals that experts produce more accurate and fine-grained sketches, resulting in closer alignment with reference images in close-ended tasks. Conversely, in open-ended tasks, expert fine-grained strokes fail to generate precise results due to \tool's lack of control at the thin stroke level.}
    \Description{The comparison between novice and expert participants in painting reveals that experts produce more accurate and fine-grained sketches, resulting in closer alignment with reference images in close-ended tasks. Novice users create rougher sketches with less accuracy in shape. Conversely, in open-ended tasks, expert fine-grained strokes fail to generate precise results due to \tool's lack of control at the thin stroke level, while novice users' broader strokes yield results more aligned with their sketches.}
    \label{fig:novice_expert}
    % \vspace{-3mm}
\end{figure*}


% In particular, in the initial control input, users are unable to explicitly specify multi-modal generation intents.
In another example, PromptPaint~\cite{promptpaint} stresses the importance of paint-medium-like interactions and introduces Prompt stencil functions that allow users to perform fine-grained controls with localized image generation. 
However, insufficient spatial control (\eg, PromptPaint only allows for single-object prompt stencil at a time) and unstable models can still leave some users feeling the uncertainty of AI and a varying degree of ownership of the generated artwork~\cite{promptpaint}.
% As a result, the gap between intuitive multi-modal or paint-medium-like control and the current prompting interface still exists, which requires further research on multi-modal prompting interactions.
From this perspective, our work seeks to further enhance multi-object spatial-semantic prompting control by users' natural sketching.
However, there are still some challenges to be resolved, such as consistent multi-object generation in multiple rounds to increase stability and improved understanding of user sketches.   


% \new{
% From this perspective, our work is a step forward in this direction by allowing multi-object spatial-semantic prompting control by users' natural sketching, which considers the interplay between multiple sketch regions.
% % To further advance the multi-modal prompting experience, there are some aspects we identify to be important.
% % One of the important aspects is enhancing the consistency and stability of multiple rounds of generation to reduce the uncertainty and loss of control on users' part.
% % For this purpose, we need to develop techniques to incorporate consistent generation~\cite{tewel2024training} into multi-modal prompting framework.}
% % Another important aspect is improving generative models' understanding of the implicit user intents \new{implied by the paint-medium-like or sketch-based input (\eg, sketch of two people with their hands slightly overlapping indicates holding hand without needing explicit prompt).
% % This can facilitate more natural control and alleviate users' effort in tuning the textual prompt.
% % In addition, it can increase users' sense of ownership as the generated results can be more aligned with their sketching intents.
% }
% For example, when users draw sketches of two people with their hands slightly overlapping, current region-based models cannot automatically infer users' implicit intention that the two people are holding hands.
% Instead, they still require users to explicitly specify in the prompt such relationship.
% \tool addresses this through sketch-aware prompt recommendation to fill in the necessary semantic information, alleviating users' workload.
% However, some users want the generative AI in the future to be able to directly infer this natural implicit intentions from the sketches without additional prompting since prompt recommendation can still be unstable sometimes.


% \new{
% Besides visual generation, 
% }
% For example, one of the important aspect is referring~\cite{he2024multi}, linking specific text semantics with specific spatial object, which is partly what we do in our sketch-aware prompt recommendation.
% Analogously, in natural communication between humans, text or audio alone often cannot suffice in expressing the speakers' intentions, and speakers often need to refer to an existing spatial object or draw out an illustration of her ideas for better explanation.
% Philosophically, we HCI researchers are mostly concerned about the human-end experience in human-AI communications.
% However, studies on prompting is unique in that we should not just care about the human-end interaction, but also make sure that AI can really get what the human means and produce intention-aligned output.
% Such consideration can drastically impact the design of prompting interactions in human-AI collaboration applications.
% On this note, although studies on multi-modal interactions is a well-established topic in HCI community, it remains a challenging problem what kind of multi-modal information is really effective in helping humans convey their ideas to current and next generation large AI models.




\subsection{Novice Performance vs. Expert Performance}\label{sec:nVe}
In this section we discuss the performance difference between novice and expert regarding experience in painting and prompting.
First, regarding painting skills, some participants with experience (4/12) preferred to draw accurate and fine-grained shapes at the beginning. 
All novice users (5/12) draw rough and less accurate shapes, while some participants with basic painting skills (3/12) also favored sketching rough areas of objects, as exemplified in Figure~\ref{fig:novice_expert}.
The experienced participants using fine-grained strokes (4/12, none of whom were experienced in prompting) achieved higher IoU scores (0.557) in the close-ended task (0.535) when using \tool. 
This is because their sketches were closer in shape and location to the reference, making the single object decomposition result more accurate.
Also, experienced participants are better at arranging spatial location and size of objects than novice participants.
However, some experienced participants (3/12) have mentioned that the fine-grained stroke sometimes makes them frustrated.
As P1's comment for his result in open-ended task: "\emph{It seems it cannot understand thin strokes; even if the shape is accurate, it can only generate content roughly around the area, especially when there is overlapping.}" 
This suggests that while \tool\ provides rough control to produce reasonably fine results from less accurate sketches for novice users, it may disappoint experienced users seeking more precise control through finer strokes. 
As shown in the last column in Figure~\ref{fig:novice_expert}, the dragon hovering in the sky was wrongly turned into a standing large dragon by \tool.

Second, regarding prompting skills, 3 out of 12 participants had one or more years of experience in T2I prompting. These participants used more modifiers than others during both T2I and R2I tasks.
Their performance in the T2I (0.335) and R2I (0.469) tasks showed higher scores than the average T2I (0.314) and R2I (0.418), but there was no performance improvement with \tool\ between their results (0.508) and the overall average score (0.528). 
This indicates that \tool\ can assist novice users in prompting, enabling them to produce satisfactory images similar to those created by users with prompting expertise.



\subsection{Applicability of \tool}
The feedback from user study highlighted several potential applications for our system. 
Three participants (P2, P6, P8) mentioned its possible use in commercial advertising design, emphasizing the importance of controllability for such work. 
They noted that the system's flexibility allows designers to quickly experiment with different settings.
Some participants (N = 3) also mentioned its potential for digital asset creation, particularly for game asset design. 
P7, a game mod developer, found the system highly useful for mod development. 
He explained: "\emph{Mods often require a series of images with a consistent theme and specific spatial requirements. 
For example, in a sacrifice scene, how the objects are arranged is closely tied to the mod's background. It would be difficult for a developer without professional skills, but with this system, it is possible to quickly construct such images}."
A few participants expressed similar thoughts regarding its use in scene construction, such as in film production. 
An interesting suggestion came from participant P4, who proposed its application in crime scene description. 
She pointed out that witnesses are often not skilled artists, and typically describe crime scenes verbally while someone else illustrates their account. 
With this system, witnesses could more easily express what they saw themselves, potentially producing depictions closer to the real events. "\emph{Details like object locations and distances from buildings can be easily conveyed using the system}," she added.

% \subsection{Model Understanding of Users' Implicit Intents}
% In region-sketch-based control of generative models, a significant gap between interaction design and actual implementation is the model's failure in understanding users' naturally expressed intentions.
% For example, when users draw sketches of two people with their hands slightly overlapping, current region-based models cannot automatically infer users' implicit intention that the two people are holding hands.
% Instead, they still require users to explicitly specify in the prompt such relationship.
% \tool addresses this through sketch-aware prompt recommendation to fill in the necessary semantic information, alleviating users' workload.
% However, some users want the generative AI in the future to be able to directly infer this natural implicit intentions from the sketches without additional prompting since prompt recommendation can still be unstable sometimes.
% This problem reflects a more general dilemma, which ubiquitously exists in all forms of conditioned control for generative models such as canny or scribble control.
% This is because all the control models are trained on pairs of explicit control signal and target image, which is lacking further interpretation or customization of the user intentions behind the seemingly straightforward input.
% For another example, the generative models cannot understand what abstraction level the user has in mind for her personal scribbles.
% Such problems leave more challenges to be addressed by future human-AI co-creation research.
% One possible direction is fine-tuning the conditioned models on individual user's conditioned control data to provide more customized interpretation. 

% \subsection{Balance between recommendation and autonomy}
% AIGC tools are a typical example of 
\subsection{Progressive Sketching}
Currently \tool is mainly aimed at novice users who are only capable of creating very rough sketches by themselves.
However, more accomplished painters or even professional artists typically have a coarse-to-fine creative process. 
Such a process is most evident in painting styles like traditional oil painting or digital impasto painting, where artists first quickly lay down large color patches to outline the most primitive proportion and structure of visual elements.
After that, the artists will progressively add layers of finer color strokes to the canvas to gradually refine the painting to an exquisite piece of artwork.
One participant in our user study (P1) , as a professional painter, has mentioned a similar point "\emph{
I think it is useful for laying out the big picture, give some inspirations for the initial drawing stage}."
Therefore, rough sketch also plays a part in the professional artists' creation process, yet it is more challenging to integrate AI into this more complex coarse-to-fine procedure.
Particularly, artists would like to preserve some of their finer strokes in later progression, not just the shape of the initial sketch.
In addition, instead of requiring the tool to generate a finished piece of artwork, some artists may prefer a model that can generate another more accurate sketch based on the initial one, and leave the final coloring and refining to the artists themselves.
To accommodate these diverse progressive sketching requirements, a more advanced sketch-based AI-assisted creation tool should be developed that can seamlessly enable artist intervention at any stage of the sketch and maximally preserve their creative intents to the finest level. 

\subsection{Ethical Issues}
Intellectual property and unethical misuse are two potential ethical concerns of AI-assisted creative tools, particularly those targeting novice users.
In terms of intellectual property, \tool hands over to novice users more control, giving them a higher sense of ownership of the creation.
However, the question still remains: how much contribution from the user's part constitutes full authorship of the artwork?
As \tool still relies on backbone generative models which may be trained on uncopyrighted data largely responsible for turning the sketch into finished artwork, we should design some mechanisms to circumvent this risk.
For example, we can allow artists to upload backbone models trained on their own artworks to integrate with our sketch control.
Regarding unethical misuse, \tool makes fine-grained spatial control more accessible to novice users, who may maliciously generate inappropriate content such as more realistic deepfake with specific postures they want or other explicit content.
To address this issue, we plan to incorporate a more sophisticated filtering mechanism that can detect and screen unethical content with more complex spatial-semantic conditions. 
% In the future, we plan to enable artists to upload their own style model

% \subsection{From interactive prompting to interactive spatial prompting}


\subsection{Limitations and Future work}

    \textbf{User Study Design}. Our open-ended task assesses the usability of \tool's system features in general use cases. To further examine aspects such as creativity and controllability across different methods, the open-ended task could be improved by incorporating baselines to provide more insightful comparative analysis. 
    Besides, in close-ended tasks, while the fixing order of tool usage prevents prior knowledge leakage, it might introduce learning effects. In our study, we include practice sessions for the three systems before the formal task to mitigate these effects. In the future, utilizing parallel tests (\textit{e.g.} different content with the same difficulty) or adding a control group could further reduce the learning effects.

    \textbf{Failure Cases}. There are certain failure cases with \tool that can limit its usability. 
    Firstly, when there are three or more objects with similar semantics, objects may still be missing despite prompt recommendations. 
    Secondly, if an object's stroke is thin, \tool may incorrectly interpret it as a full area, as demonstrated in the expert results of the open-ended task in Figure~\ref{fig:novice_expert}. 
    Finally, sometimes inclusion relationships (\textit{e.g.} inside) between objects cannot be generated correctly, partially due to biases in the base model that lack training samples with such relationship. 

    \textbf{More support for single object adjustment}.
    Participants (N=4) suggested that additional control features should be introduced, beyond just adjusting size and location. They noted that when objects overlap, they cannot freely control which object appears on top or which should be covered, and overlapping areas are currently not allowed.
    They proposed adding features such as layer control and depth control within the single-object mask manipulation. Currently, the system assigns layers based on color order, but future versions should allow users to adjust the layer of each object freely, while considering weighted prompts for overlapping areas.

    \textbf{More customized generation ability}.
    Our current system is built around a single model $ColorfulXL-Lightning$, which limits its ability to fully support the diverse creative needs of users. Feedback from participants has indicated a strong desire for more flexibility in style and personalization, such as integrating fine-tuned models that cater to specific artistic styles or individual preferences. 
    This limitation restricts the ability to adapt to varied creative intents across different users and contexts.
    In future iterations, we plan to address this by embedding a model selection feature, allowing users to choose from a variety of pre-trained or custom fine-tuned models that better align with their stylistic preferences. 
    
    \textbf{Integrate other model functions}.
    Our current system is compatible with many existing tools, such as Promptist~\cite{hao2024optimizing} and Magic Prompt, allowing users to iteratively generate prompts for single objects. However, the integration of these functions is somewhat limited in scope, and users may benefit from a broader range of interactive options, especially for more complex generation tasks. Additionally, for multimodal large models, users can currently explore using affordable or open-source models like Qwen2-VL~\cite{qwen} and InternVL2-Llama3~\cite{llama}, which have demonstrated solid inference performance in our tests. While GPT-4o remains a leading choice, alternative models also offer competitive results.
    Moving forward, we aim to integrate more multimodal large models into the system, giving users the flexibility to choose the models that best fit their needs. 
    


\section{Conclusion}\label{sec:conclusion}
In this paper, we present \tool, an interactive system designed to help novice users create high-quality, fine-grained images that align with their intentions based on rough sketches. 
The system first refines the user's initial prompt into a complete and coherent one that matches the rough sketch, ensuring the generated results are both stable, coherent and high quality.
To further support users in achieving fine-grained alignment between the generated image and their creative intent without requiring professional skills, we introduce a decompose-and-recompose strategy. 
This allows users to select desired, refined object shapes for individual decomposed objects and then recombine them, providing flexible mask manipulation for precise spatial control.
The framework operates through a coarse-to-fine process, enabling iterative and fine-grained control that is not possible with traditional end-to-end generation methods. 
Our user study demonstrates that \tool offers novice users enhanced flexibility in control and fine-grained alignment between their intentions and the generated images.



{
\small
\bibliographystyle{unsrtnat}
\bibliography{main}
}


\section{Multi-Normalized Gradient Descent} \label{sec: mngd}
Before presenting our approach, let us first introduce some clarifying notations.

\paragraph{Notations.} For a vector $x\in\mathbb{R}^d$, we call its normalized projection w.r.t to a given norm $\Vert \cdot\Vert$, the solution to the following optimization problem:
\begin{align}
\label{eq-single-proj}
\mathcal{P}_{\Vert \cdot \Vert}(x):=\argmax_{z:~\Vert z \Vert = 1} \langle x, z\rangle    
\end{align}
We also extend the definition of this notation if $x\in\mathbb{R}^{m\times n}$ is a matrix and $\Vert \cdot \Vert$ is a matrix norm. 


\subsection{Gradient Multi-Normalization}

Let us now consider a finite family of $K\geq 1$ norms $(g_1,\dots,g_K)$. In order to pre-process the gradient $\nabla$ jointly according to these norms, we propose to consider the following optimization problem:
\begin{align}
\label{eq:multi-norm-opt}
 \argmax_{z} \langle \nabla, z\rangle~ \text{s.t.}~\forall~i\in [|1,K|],~g_i(z)=1\; .
\end{align}
Assuming the constraint set is non-empty, the existence of a maximum is guaranteed. However, this problem is NP-hard and non-convex due to the constraints, making it hard to solve efficiently for the general case of arbitrary norms.


\begin{algorithm}[!t]
   \caption{$\texttt{MultiNorm}(\nabla,L, \bm{g})$}
   \label{alg:alt-proj}
\begin{algorithmic}
   \STATE {\bfseries Input:} the stochastic gradient $\nabla_\theta\mathcal{L}(\theta_t,x^{(t)})$, the norms $\bm{g}:=(g_1,\dots,g_K)$, and $L\geq 1$ the number of iterations.
   \STATE Initialize $x=\nabla_\theta\mathcal{L}(\theta_t,x^{(t)})$.
   \FOR{$\ell=1$  {\bfseries to} $L$}
   \FOR{$i=1$ {\bfseries to} $K$}
   \STATE $x\gets \mathcal{P}_{g_i}(x):=\argmax\limits_{z:~g_i(z) = 1} \langle x, z\rangle $
   \ENDFOR
    \ENDFOR
    \STATE Return $x$
\end{algorithmic}
\end{algorithm}

\begin{remark}
Observe that when $K=1$, the problem~\eqref{eq:multi-norm-opt} recovers exactly the single normalization step used in~\cite{bernstein2024old}, as presented in~\eqref{eq:single-norm}.
\end{remark}

\begin{remark}
The convex relaxation of~\eqref{eq:multi-norm-opt}, defined as  
\begin{align}
\label{eq:multi-norm-opt-convex}
 \argmax_{z} \langle \nabla, z\rangle\quad \text{s.t.}~\forall~~i\in [|1,K|],~~g_i(z)\leq 1
\end{align}
is in fact equivalent to the single normalization case discussed in Section~\ref{sec:single-norm}, where the norm considered is $\Vert x\Vert:=\max\limits_{i\in[|1,K]} g_i(x)$. Thus, solving~\eqref{eq:multi-norm-opt-convex} is equivalent to computing the projection $\mathcal{P}_{\Vert \cdot \Vert}(\nabla)$. In Appendix~\ref{sec:convex-relaxation}, we provide a general approach to compute it using the so-called Chambolle-Pock algorithm~\cite{chambolle2011first}.
\end{remark}

While solving~\eqref{eq:multi-norm-opt} exactly might not be practically feasible in general, we propose a simple alternating projection scheme, 
presented in Algorithm~\ref{alg:alt-proj}. Notably, our method assumes that the projections $\mathcal{P}_{g_i}(\cdot)$  can be efficiently computed for all $i\in[|1,K|]$. Fortunately, when the $g_i$'s correspond to $\ell_p$-norms with $p\in[|1,+\infty|]$, or Schatten $p$-norms for matrices, closed-form solutions for these projections exist. See Appendix~\ref{sec:convex-relaxation} for more details.


\paragraph{SWAN: an Instance of $\texttt{MultiNorm}$.}  SWAN~\cite{ma2024swansgdnormalizationwhitening} applies two specific pre-processing steps to the raw gradients in order to update the weight matrices. In fact, each of these pre-processing steps can be seen as normalized projections with respect to a specific norm. More precisely, for $W\in\mathbb{R}^{m\times n}$ and $m\leq n$, let us define
\begin{align*}
g_1(W):=\frac{\max\limits_{i\in[|1,m|]} \Vert W_{i,:}\Vert_2}{\sqrt{n}}\; ,~ \text{and}~~ 
g_2(W):=\frac{\Vert W\Vert_{\sigma, \infty}}{\sqrt{n}}\; .
\end{align*}
where for $p\in [1,+\infty]$, $\Vert W\Vert_{\sigma,p}$ is the Schatten $p$-norm of $W$. Simple derivations leads to the following equalities:
\begin{align*}
    \mathcal{P}_{g_1}(W)&= \sqrt{n} Q(W)^{-1}W\\
    \mathcal{P}_{g_2}(W)&=\sqrt{n}(WW^\top)^{-1/2}W
\end{align*}
%
Therefore applying a single iteration ($L=1$) of Algorithm~\ref{alg:alt-proj} with norms $g_1$ and $g_2$ as defined above on the raw gradient $\nabla_t$ exactly leads to the SWAN update (Eq.~\eqref{eq:swan-update}).


\subsection{On the Convergence of \texttt{MultiNorm}}
We aim now at providing some theoretical guarantees on the convergence of  $\texttt{MultiNorm}$ (Algorithm~\ref{alg:alt-proj}). More precisely, following the SWAN implementation~\cite{ma2024swansgdnormalizationwhitening}, we focus on the specific case where $K=2$ and the normalized projections associated with the norms $g_1$ and $g_2$ have constant $\ell_2$-norm. More formally, we consider the following assumption.
\begin{assumption}
\label{assump-norm}
Let $g$ be a norm on $\mathbb{R}^d$. We say that it satisfies the assumption if for all $x\in\mathbb{R}^d$, $\Vert \mathcal{P}_{g}(x) \Vert_2 = c $ where $c>0$ is an arbitrary positive constant independent of $x$ and $\Vert\cdot\Vert_2$ represents the Euclidean norm.
\end{assumption}

\begin{remark}
Observe that both norms in SWAN satisfies Assumption~\ref{assump-norm} and their normalized projections have the same $\ell_2$-norm, as for any $W\in\mathbb{R}^{m\times n}$ with $m\leq n$, we have $\Vert \mathcal{P}_{g_1}(W) \Vert_2 = \Vert \mathcal{P}_{g_2}(W)\Vert_2 = \sqrt{nm}$.
\end{remark}


This assumption enables to obtain useful properties on $\mathcal{P}_{g}$ as we show in the following Lemma:
\begin{lemma}
\label{lem:properties-proj}
Let $g$ a norm satisfying Assumption~\ref{assump-norm}. Then
\begin{align*}
    \mathcal{P}_{g}\circ\mathcal{P}_{g} =\mathcal{P}_{g}
\end{align*}
and for all $x\in\mathbb{R}^d$, $g^*(\mathcal{P}_g(x))=\Vert \mathcal{P}_g(x)\Vert_2^2=c^2$, 
where $g^*$ is the dual norm associated with $g$.
\end{lemma}

Let us now introduce some additional notation to clearly state our result. Let $x_0\in\mathbb{R}^d$ and let us define for $n\geq 0$:
\begin{equation}
\begin{aligned}
\label{eq:seq}
    x_{2n+1}&:=\mathcal{P}_{g_1}(x_{2n})\\
    x_{2n+2}&:= \mathcal{P}_{g_2}(x_{2n+1})
    \end{aligned}
\end{equation}

which is exactly the sequence generated by Algorithm~\ref{alg:alt-proj} when $K=2$ and $x_0=\nabla_\theta\mathcal{L}(\theta_t,x^{(t)})$. Let us now show our main theoretical result, presented in the following Theorem.
\begin{theorem}
\label{thm:cvg}
Let $g_1$ and $g_2$ two norms on $\mathbb{R}^d$ satisfying Assumption~\ref{assump-norm} and such that their normalized projections have the same $\ell_2$ norm. Let also $(x_n)_{n_\geq 0}$ be defined as in~\eqref{eq:seq} and let us define the set of fixed-point as:
\begin{align*}
    \mathcal{F}:=\{x:~\mathcal{P}_{g_1}(x)=\mathcal{P}_{g_2}(x)=x\}
\end{align*}
Then by denoting $d(x,\mathcal{F}):=\min\limits_{z\in\mathcal{F}}\Vert x - z\Vert_2$ we have 
\begin{align*}
d(x_n,\mathcal{F}) \xrightarrow[n\to\infty]{} 0\; .
\end{align*}
% the sequence $(\Vert x_{n+1} - x_{n}\Vert_2)_{n\geq 0}$ monotonically decreases towards $0$ and \begin{align*}
%     g_1(x_n)\xrightarrow[n\to\infty]{} 1,~\text{and}~
%     g_2(x_n)\xrightarrow[n\to\infty]{} 1\; .
% \end{align*}
% Additionally, any cluster point $x$ of $(x_{n})_{n\geq 0}$ verifies:
% \begin{align*}
%     \mathcal{P}_{g_1}(x)=\mathcal{P}_{g_2}(x)=x\; .
% \end{align*}
\end{theorem}

This Theorem states that if $\texttt{MultiNorm}$ runs for a sufficient amount of time, then the returned point $x$ can be arbitrarily close to a fixed-point solution. While we cannot guarantee that it solves~\eqref{eq:multi-norm-opt}, we can assert that our algorithm converges to a fixed-point solution with arbitrary precision, and as a by-product produces a solution $x$ normalized w.r.t both norms $g_1$, $g_2$ (up to an arbitrary precision).

\begin{remark}
Note that in Theorem~\ref{thm:cvg} we assume that the normalized projections associated to $g_1$ and $g_2$ have the same $\ell_2$-norms. However, given two norms $g_1$ and $g_2$ satisfying Assumption~\ref{assump-norm}, i.e. such that for all $x$:
\begin{align*}
    \Vert \mathcal{P}_{g_1}(x) \Vert_2 &= c_1\\
    \Vert \mathcal{P}_{g_2}(x) \Vert_2 &= c_2
\end{align*}
for some $c_1,c_2>0$, and given a target value $a>0$, one can always rescale the norms such that their normalized projections have the same $\ell_2$ norm equal to $a$. More formally, by denoting $\tilde{g_1} = \frac{c_1}{a} g_1$ and $\tilde{g_2} = \frac{c_2}{a} g_2$, we obtain that
\begin{align*}
    \Vert \mathcal{P}_{\tilde{g}_1}(x)\Vert_2 =  \Vert \mathcal{P}_{\tilde{g}_2}(x)\Vert_2 = a .
\end{align*}
\end{remark}




\begin{remark}
It is worth noting that, for squared matrices ($m=n$), a single iteration ($L=1$) of \texttt{MultiNorm} using the norms considered in~\cite{ma2024swansgdnormalizationwhitening}, immediately converges to a fixed-point---precisely recovering SWAN.
\end{remark}



\subsection{MNGD: a New Family of Stateless Optimizers.} 
% We are now ready to present our family optimizers, called the \emph{Multi-Normalized Gradient Descents} (MNGDs), which is detailed in Algorithm~\ref{alg:multi-normalized-gd}. The main difference with the framework proposed in~\cite{bernstein2024old}, is that here we enable the normalization of the gradient w.r.t multiple norms thanks to the $\texttt{MultiNorm}$ step, while in~\cite{bernstein2024old}, the gradient is normalized according to a single norm as presented in~\eqref{eq:single-norm}.


We now introduce our family of optimizers: \emph{Multi-Normalized Gradient Descents} (MNGDs) (Algorithm~\ref{alg:multi-normalized-gd}). The key distinction from the framework proposed in~\cite{bernstein2024old} is that MNGDs normalize the gradient with respect to multiple norms using the 
$\texttt{MultiNorm}$ step, whereas in~\cite{bernstein2024old}, the gradient is normalized using a single norm, as shown in~\eqref{eq:single-norm}.
\begin{algorithm}[!t]
   \caption{Multi-Normalized GD ($\texttt{MNGD}$)}
   \label{alg:multi-normalized-gd}
\begin{algorithmic}
   \STATE {\bfseries Input:} $T\geq 1$ the number of updates, $(\eta_t)_{0\leq t\leq T}$ the global step-sizes, $\mathcal{L}$ the loss to minimize, $L\geq 1$ the number of iterations for the multi-normalization, and $\bm{g}:=(g_1,\dots,g_K)$ the norms.
   \STATE Initialize $\theta_0$
   \FOR{$t=1$  {\bfseries to} $T$}
    \STATE $\nabla_t\gets \nabla_{\theta}\mathcal{L}(\theta_t, x^{(t)})$ with $x^{(t)}\sim P_x$
    \STATE $\hat{\nabla}_t \gets \texttt{MultiNorm}(\nabla_t,L, \bm{g})$ as defined in Alg.~\ref{alg:alt-proj}.
    \STATE $\theta_{t+1} \gets \theta_t - \eta_t \hat{\nabla}_t$
    \ENDFOR
    \STATE Return $x$
\end{algorithmic}
\end{algorithm}

In the following, we focus on the MNGD scheme with a specific choice of norms, for which we can efficiently compute the gradient multi-normalization step. This enables the application of stateless optimizers to large LMs.

\section{Sinkhorn: a Multi-Normalization Procedure}
As in SWAN~\cite{ma2024swansgdnormalizationwhitening}, we propose to normalize the weight matrices according to multiple norms. We still leverage the row-wise $\ell_2$-norm to pre-process raw gradients, however, rather than using the spectral norm, we propose to consider instead a relaxed form of this constraint and use the column-wise $\ell_2$-norm. More formally, let us consider the two following norms on matrices of size $\mathbb{R}^{m\times n}$:
\begin{align*}
g_1(W):=\frac{\max\limits_{i\in[|1,m|]} \Vert W_{i,:}\Vert_2}{\sqrt{n}}\; ,\quad 
g_2(W):=\frac{\max\limits_{j\in[|1,n|]} \Vert W_{:,j}\Vert_2}{\sqrt{m}}\; ,
\end{align*}
which leads to the following two normalized projections:
\begin{align*}
    \mathcal{P}_{g_1}(W)&= \sqrt{n} Q(W)^{-1}W\\
    \mathcal{P}_{g_2}(W)&=\sqrt{m}W R(W)^{-1}
\end{align*}
where $R(W):=\text{Diag}(\Vert W_{:,1}\Vert_2,\dots,\Vert W_{:,n}\Vert_2)\in\mathbb{R}^{n\times n} $ is the diagonal matrix of size $n$ with the $\ell_2$-norm of the columns of $W$ as diagonal coefficients. For such a choice of norms, the $\texttt{MultiNorm}$ reduces to a simple procedure as presented in Algorithm~\ref{alg:Sinkhorn}.


\begin{remark}
For such a choice of norms, we obtain $\Vert \mathcal{P}_{g_1}(W) \Vert_2 = \Vert \mathcal{P}_{g_2}(W)\Vert_2 = \sqrt{nm}$ for any $W\in\mathbb{R}^{m\times n}$. In other words, both norms satisfy Assumption~\ref{assump-norm} and their $\ell_2$ norms are equal to $\sqrt{nm}$.
\end{remark}


For completeness we include  the MNGD scheme (Algorithm~\ref{alg:multi-normalized-sinkhorn}) that replaces the $\texttt{MultiNorm}$ step with $\texttt{SR-Sinkhorn}$ (Algorithm~\ref{alg:Sinkhorn}).


\begin{algorithm}[!t]
   \caption{$\texttt{SR-Sinkhorn}(\nabla,L)$}
   \label{alg:Sinkhorn}
\begin{algorithmic}
   \STATE {\bfseries Input:} the stochastic gradient $\nabla_W\mathcal{L}(W_t,x^{(t)})$, and $L\geq 1$ the number of iterations.
   \STATE Initialize $X=\nabla_W\mathcal{L}(W_t,x^{(t)})\in\mathbb{R}^{m\times n}$.
   \FOR{$\ell=1$  {\bfseries to} $L$}
   \STATE $X\gets  \sqrt{n} Q(X)^{-1}X$
   \STATE $X\gets  \sqrt{m} XR(X)^{-1}$
   \ENDFOR
    \STATE Return $X$
\end{algorithmic}
\end{algorithm}


\paragraph{The Sinkhorn Algorithm.} Before explicitly showing the link between Algorithm~\ref{alg:Sinkhorn} and the Sinkhorn algorithm, let us first recall the Sinkhorn theorem~\cite{sinkhorn1964relationship} and the Sinkhorn algorithm~\cite{sinkhorn1967concerning}. Given a positive coordinate-wise matrix $A\in\mathbb{R}_{+}^{m\times n}$, there exists a unique matrix $P\in\mathbb{R}_{+}^{m\times n}$ of the form $P=QAR$ with $Q$ and $R$ positive coordinate-wise and diagonal matrices of size $m$ and $n$ respectively, such that $P\bm{1}_n=n\bm{1}_m$ and $P^\top\bm{1}_m=m\bm{1}_n$. To find $P$, one can use the Sinkhorn algorithm that initializes $P_0:=A$ and computes for $k\geq 0$:
\begin{align*}
    P_{k+1/2}&=n\text{Diag}(P_k\bm{1}_n)^{-1}P_k\\
    P_{k+1}&=m P_{k+1/2}\text{Diag}(P_{k+1/2}^\top\bm{1}_m)^{-1}\; .
\end{align*}
Equivalently, these updates on $P$ can be directly expressed as updates on the diagonal coefficients of $Q=\text{Diag}(u)$ and $R=\text{Diag}(v)$ with $u\in\mathbb{R}_{+}^m$ and $v\in\mathbb{R}_{+}^n$. By initializing $u_0=\bm{1}_m$ an $v_0=\bm{1}_m$, the above updates can be reformulated as follows:
\begin{align}
\label{eq:update-diag-sin}
    u_{k+1} = n\frac{ \bm{1}_m}{Av_k},~~v_{k+1} = m\frac{\bm{1}_n}{ A^\top u_{k+1}}
\end{align}
where $/$ denote the coordinate-wise division.~\citet{franklin1989scaling} show the linear convergence of Sinkhorn’s iterations. More formally, they show that $(u_k, v_k)$ converges to some $(u^*,v^*)$ such that $P:=\text{Diag}(u^*)A\text{Diag}(v^*)$ satisfies $P\bm{1}_n=n\bm{1}_m$ and $P^\top\bm{1}_m=m\bm{1}_n$, and:
\begin{align*}
    d_\mathcal{H}(u_k, u^*)\in\mathcal{O}(\lambda(A)^{2k})~, \text{ and } ~d_\mathcal{H}(v_k, v^*)\in\mathcal{O}(\lambda(A)^{2k})\; ,
\end{align*}
where $d_{\mathcal{H}}$ is the Hilbert projective metric~\cite{de1993hilbert} and $\lambda(A)<1$ is a contraction factor associated with the matrix $A$.


\paragraph{Links between Sinkhorn and Algorithm~\ref{alg:Sinkhorn}.} Algorithm~\ref{alg:Sinkhorn} can be seen as a simple reparameterization of the updates presented in~\eqref{eq:update-diag-sin}. More precisely, given a gradient $\nabla\in\mathbb{R}^{m\times n}$ and denoting $A:=\nabla^{\odot 2}$, we obtain that the iterations of Algorithm~\ref{alg:Sinkhorn} exactly compute:
\begin{align}
\label{eq:update-diag-sin-sr}
    u_{k+1}^{1/2} = \sqrt{n\frac{ \bm{1}_m}{Av_k}},~~v_{k+1}^{1/2} = \sqrt{m\frac{\bm{1}_n}{ A^\top u_{k+1}}}
\end{align}
where the square-root is applied coordinate-wise, and returns after $L$ iterations $X_L=\text{Diag}(u_{L}^{1/2})\nabla \text{Diag}(v_{L}^{1/2})$. Therefore the linear convergence of Algorithm~\ref{alg:Sinkhorn} follows directly from the convergence rate of Sinkhorn, and Algorithm~\ref{alg:Sinkhorn} can be thought as applying the square-root Sinkhorn algorithm, thus the name $\texttt{SR-Sinhkorn}$. Note also that at convergence ($L\to+\infty$) we obtain $X^{*}\in\mathbb{R}^{m\times n}$ which is a fixed-point of both normalized projections, that is $\mathcal{P}_{g_1}(X^*)=\mathcal{P}_{g_2}(X^*)=X^*$,
from which we deduce that
\begin{align*}
\Vert X^*_{i,:}\Vert_2 = \sqrt{n}\;, \quad \text{and}\quad   \Vert X^*_{:,j}\Vert_2 = \sqrt{m}\;
\end{align*}
as demonstrated in Theorem~\ref{thm:cvg}.



\textbf{On the Importance of the Scaling.} Now that we have shown the convergence $\texttt{SR-Sinkhorn}$, let us explain in more detail the scaling considered for both the row-wise and column-wise normalizations. First recall that both norm $g_1$ and $g_2$ satisfy Assumption~\ref{assump-norm} and that the $\ell_2$ norm of their normalized projections is equal to $\sqrt{nm}$. The reason for this specific choice of scaling ($\sqrt{nm}$) is due to the global step-size in Algorithm~\ref{alg:multi-normalized-sinkhorn}. In our proposed MNGD, we did not prescribe how to select $\eta_t$. In practice, we aim to leverage the same global step-sizes as those used in Adam~\cite{adam} for training LLMs, and therefore we need to globally rescale the (pre-processed) gradient accordingly. To achieve that, observe that when EMAs are turned-off, Adam corresponds to a simple signed gradient descent, and therefore the Frobenius norm of the pre-processed gradient is simply $\sqrt{nm}$. Thus, when normalizing either the rows or the columns, we only need to rescale the normalized gradient accordingly.


\begin{algorithm}
   \caption{Sinkhorn GD ($\texttt{SinkGD}$)}
   \label{alg:multi-normalized-sinkhorn}
\begin{algorithmic}
   \STATE {\bfseries Input:} $T\geq 1$ the number of updates, $(\eta_t)_{0\leq t\leq T}$ the global step-sizes, $\mathcal{L}$ the loss to minimize, and $L\geq 1$ the number of iterations for the SR-Sinkhorn procedure.
   \STATE Initialize $\theta_0$
   \FOR{$t=1$  {\bfseries to} $T$}
    \STATE $\nabla_t\gets \nabla_{\theta}\mathcal{L}(\theta_t, x^{(t)})$ with $x^{(t)}\sim P_x$
    \STATE $\hat{\nabla}_t \gets \texttt{SR-Sinkhorn}(\nabla_t,L)$ as defined in Alg.~\ref{alg:Sinkhorn}.
    \STATE $\theta_{t+1} \gets \theta_t - \eta_t \hat{\nabla}_t$
    \ENDFOR
    \STATE Return $x$
\end{algorithmic}
\end{algorithm}
\vspace{-0.2cm}

\textbf{Computational Efficiency of SinkGD over SWAN.} Compared to SWAN~\cite{ma2024swansgdnormalizationwhitening}, the proposed approach,  \texttt{SinkGD}, is more efficient as it only requires $\mathcal{O}(nm)$ numerical operations. In contrast, SWAN, even when implemented with Newton-Schulz, still requires performing matrix-matrix multiplications, which have a time complexity of $\mathcal{O}(m^2(m+n))$. In the next section, we will demonstrate the practical effectiveness of MNGD with $\texttt{SR-Sinkhorn}$, that is \texttt{SinkGD}. This approach manages to be on par with, and even outperforms,  memory-efficient baselines for pretraining the family of LLaMA models up to 1B scale.











\end{document}