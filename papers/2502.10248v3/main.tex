\documentclass{article}


\usepackage[final]{neurips}



\usepackage{enumitem}
\usepackage{marvosym} % for symbol  
\usepackage[utf8]{inputenc} % allow utf-8 input
\usepackage[T1]{fontenc}    % use 8-bit T1 fonts
\usepackage{hyperref}       % hyperlinks
\usepackage{url}            % simple URL typesetting
\usepackage{booktabs}       % professional-quality tables
\usepackage{amsfonts}       % blackboard math symbols
% \usepackage{nicefrac}       % compact symbols for 1/2, etc.
\usepackage{microtype}      % microtypography
\usepackage{xcolor}         % colors
\usepackage{epigraph}
\usepackage[export]{adjustbox}
\usepackage{multirow}
\usepackage{listings}
\usepackage{colortbl}
\usepackage{color}
% \usepackage{bbding}
\usepackage{pifont}
\usepackage{fontawesome}
\usepackage{wrapfig}
\usepackage{subcaption}
\usepackage{caption}
\usepackage{float}
\usepackage{enumitem}
\usepackage{tabularx}
\usepackage{makecell}
\usepackage{minitoc} % parttoc
\renewcommand{\partname}{}
\renewcommand{\thepart}{}
\noptcrule



\usepackage{array}
\renewcommand{\arraystretch}{1.2}

\usepackage{caption}
\captionsetup[table]{skip=10pt}  % 调整表格标题和表格内容之间的间距



%%%%% NEW MATH DEFINITIONS %%%%%

% \usepackage{amsmath,amsfonts,bm}
\usepackage{amsmath,amsfonts}

\usepackage{pifont}


\newcommand{\R}{\mathbb{R}}


\def\va{{\mathbf{a}}}
\def\vg{{\mathbf{g}}}

% Sets
\def\sR{\mathbb{R}}
\def\sC{\mathbb{C}}
\def\sZ{\mathbb{Z}}
\def\sN{\mathbb{N}}
\def\sQ{\mathbb{Q}}

\def\sS{\mathcal{S}}



% Vectors
\def\vzero{{\mathbf{0}}}
\def\vone{{\mathbf{1}}}
\def\vmu{{\mathbf{\mu}}}
\def\vtheta{{\mathbf{\theta}}}
\def\va{{\mathbf{a}}}
\def\vb{{\mathbf{b}}}
\def\vc{{\mathbf{c}}}
\def\vd{{\mathbf{d}}}
\def\ve{{\mathbf{e}}}
\def\vf{{\mathbf{f}}}
\def\vg{{\mathbf{g}}}
\def\vh{{\mathbf{h}}}
\def\vi{{\mathbf{i}}}
\def\vj{{\mathbf{j}}}
\def\vk{{\mathbf{k}}}
\def\vl{{\mathbf{l}}}
\def\vm{{\mathbf{m}}}
\def\vn{{\mathbf{n}}}
\def\vo{{\mathbf{o}}}
\def\vp{{\mathbf{p}}}
\def\vq{{\mathbf{q}}}
\def\vr{{\mathbf{r}}}
\def\vs{{\mathbf{s}}}
\def\vt{{\mathbf{t}}}
\def\vu{{\mathbf{u}}}
\def\vv{{\mathbf{v}}}
\def\vw{{\mathbf{w}}}
\def\vx{{\mathbf{x}}}
\def\vy{{\mathbf{y}}}
\def\vz{{\mathbf{z}}}
\def\vzeta{{\mathbf{\zeta}}}

% Matrix
\def\mA{{\mathbf{A}}}
\def\mB{{\mathbf{B}}}
\def\mC{{\mathbf{C}}}
\def\mD{{\mathbf{D}}}
\def\mE{{\mathbf{E}}}
\def\mF{{\mathbf{F}}}
\def\mG{{\mathbf{G}}}
\def\mH{{\mathbf{H}}}
\def\mI{{\mathbf{I}}}
\def\mJ{{\mathbf{J}}}
\def\mK{{\mathbf{K}}}
\def\mL{{\mathbf{L}}}
\def\mM{{\mathbf{M}}}
\def\mN{{\mathbf{N}}}
\def\mO{{\mathbf{O}}}
\def\mP{{\mathbf{P}}}
\def\mQ{{\mathbf{Q}}}
\def\mR{{\mathbf{R}}}
\def\mS{{\mathbf{S}}}
\def\mT{{\mathbf{T}}}
\def\mU{{\mathbf{U}}}
\def\mV{{\mathbf{V}}}
\def\mW{{\mathbf{W}}}
\def\mX{{\mathbf{X}}}
\def\mY{{\mathbf{Y}}}
\def\mZ{{\mathbf{Z}}}
\def\mBeta{{\mathbf{\beta}}}
\def\mPhi{{\mathbf{\Phi}}}
\def\mLambda{{\mathbf{\Lambda}}}
\def\mSigma{{\mathbf{\Sigma}}}


% Expectation
% \def\eE{\mathop{\mathbb{E}}\limits}
\def\eE{\mathbb{E}}

% Probability
\def\pP{\mathbb{P}}

% Tilde
\def\tf{\tilde{f}}
\def\tS{\tilde{S}}
\def\wtF{\widetilde{\mathcal{F}}}
\def\whR{\widehat{R}}
\def\tvx{\tilde{\mathbf{x}}}
\def\ty{\tilde{y}}


\def\defeq{\overset{\textup{def}}{=}}
% \def\defeq{\overset{.}{=}}
\def\defone{\overset{\text{\ding{172}}}{=}}
\def\deftwo{\overset{\text{\ding{173}}}{=}}
\def\leqone{\overset{\text{\ding{172}}}{\leq}}
\def\leqtwo{\overset{\text{\ding{173}}}{\leq}}
\def\leqthree{\overset{\text{\ding{174}}}{\leq}}
\def\leqfour{\overset{\text{\ding{175}}}{\leq}}
\def\eqone{\overset{\text{\ding{172}}}{=}}
\def\eqtwo{\overset{\text{\ding{173}}}{=}}
\def\eqthree{\overset{\text{\ding{174}}}{=}}
\def\eqfour{\overset{\text{\ding{175}}}{=}}
\def\geqfive{\overset{\text{\ding{176}}}{\geq}}
\newcommand{\thought}[1]{{\color[rgb]{0.2,0.39,0.66}(#1)}}
\newcommand{\todo}[1]{{\color[rgb]{1.0,0.0,0.0}(#1)}}
\newcommand{\hsh}[1]{{\color{green!50!black} Henrik: #1}}
\newcommand{\st}[1]{{\color{red!50!black} Sebastian: #1}}

\newcommand{\ulm}[1]{_{\scaleto{\mathrm{#1}}{3pt}}}
\newcommand\at[2]{\left.#1\right|_{#2}}











\newtheorem{assumption}{Assumption}

\DeclareMathOperator*{\argmax}{arg\,max}
\DeclareMathOperator*{\argmin}{arg\,min}

\newcommand{\swname}[1]{\texttt{#1}}
\newcommand{\ie}{i\/.\/e\/.,\/~}
\newcommand{\eg}{e\/.\/g\/.,\/~}
\newcommand{\cf}{cf\/.\/~}

\newcommand{\fig}{Fig\/.\/~}
\newcommand{\defn}{Def\/.\/~}
\newcommand{\sect}{Sec\/.\/~}
\newcommand{\tabl}{Tab\/.\/~}
\newcommand{\algo}{Algorithm~}
\newcommand{\theo}{Theorem~}

\newcommand{\bnnl}{3 hidden layers}
\newcommand{\bnnn}{50 neurons}
\newcommand{\bnna}{tanh activations}

\newcommand{\capt}[1]{\mdseries{\emph{#1}}}

\newcommand{\videolink}{at \url{https://youtu.be/_d7AqTRjz6g}}
\newcommand{\codelink}{\url{https://github.com/wheelbot/mini-wheelbot}}

\newcommand{\fakepar}[1]{\vspace{0mm}\noindent\textbf{#1.}}

\newcommand{\needref}{\textcolor{red}{[REF]}}

\newcommand{\plotfontsize}{9pt}



% arxiv:
\title{Step-Video-T2V Technical Report: The Practice, Challenges, and Future of Video Foundation Model}

% \vspace{-0.4cm}
% \vspace{-1cm}
\author{Step-Video Team
\\
StepFun
}





\begin{document}

\doparttoc
\faketableofcontents

\maketitle

\begin{abstract}

In this work, we tackle the challenge of disambiguating queries in retrieval-augmented generation (RAG) to diverse yet answerable interpretations.
State-of-the-arts follow a Diversify-then-Verify (DtV) pipeline, where diverse interpretations are generated by an LLM,
later used as search queries to retrieve supporting passages.
Such a process
may introduce noise in either interpretations or retrieval,
particularly in enterprise settings, where LLMs---trained on static data---may struggle with domain-specific disambiguations.
Thus, a post-hoc verification phase is introduced to prune noises.
Our distinction is \textbf{to unify diversification with verification} by incorporating feedback from retriever and generator early on.
This joint approach improves both efficiency and robustness by reducing reliance on multiple retrieval and inference steps, which are susceptible to cascading errors.
We validate the efficiency and effectiveness of our method, \ourslong (\ours), on the widely adopted ASQA benchmark
to achieve diverse yet verifiable interpretations.
Empirical results show that \ours improves grounding-aware $\textrm{F}_1$ score by an average of 23\% over the strongest baseline across different backbone LLMs.
\end{abstract}


\section{Preface}

A video foundation model is a model pre-trained on large video datasets that can generate videos in response to text, visual, or multimodal inputs from users. It can be applied to a wide range of downstream video-related tasks, such as text/image/video-to-video generation, video understanding and editing, as well as video-based conversion, question answering, and task completion.

Based on our understanding, we define two levels towards building video foundation models. 
\textbf{Level-1: translational video foundation model}. A model at this level functions as a cross-modal translation system, capable of generating videos from text, visual, or multimodal context.
\textbf{Level-2: predictable video foundation model}. A model at this level acts as a prediction system, similar to large language models (LLMs), that can forecast future events based on text, visual, or multimodal context and handle more advanced tasks, such as reasoning with multimodal data or simulating real-world scenarios.

Current diffusion-based text-to-video models, such as Sora \cite{openaisora}, Veo \cite{veo}, Kling \cite{kling}, Hailuo \cite{hailuo}, and Step-Video (as described in this report), belong to Level-1. These models can generate high-quality videos from text prompts, lowering the barrier for creators to produce video content. However, they often fail to generate videos that require complex action sequences (such as a gymnastic performance) or adherence to the laws of physics (such as a basketball bouncing on the floor), let alone performing causal or logical tasks like LLMs. Such limitations arise because these models learn only the mappings between text prompts and corresponding videos, without explicitly modeling the underlying causal relationships within videos. Autoregression-based text-to-video models introduce the causal modeling mechanism by predicting the next video token, frame, or clip. However, these models still cannot achieve performance comparable to diffusion-based models on text-to-video generation.

This report will detail the practice of building Step-Video-T2V as a state-of-the-art video foundation model at Level-1. By analyzing the challenges identified through experiments, we will also highlight key problems that need to be addressed in order to develop video foundation models at Level-2.
\section{Introduction}
\label{sec:intro}
% Image editing methods in diffusion models depend on user-defined control directions - users can unlock their creativity using these methods by specifying the desired manipulation through prompts~\cite{gandikota2023concept}, reference images~\cite{ruiz2022dreambooth, kumari2022customdiffusion, gal2022image, chen2024trainingfreeregionalpromptingdiffusion}, or attribute vectors~\cite{parmar2023zero,hertz2022prompt}. In this work, we ask a fundamentally different question: \emph{Can we automatically discover the underlying visual structure of a concept within diffusion model's knowledge?} %Rather than requiring user-specified controls, we aim to decompose the model's internal knowledge into meaningful directions.

% This question touches on a fundamental limitation in how we interact with diffusion models. Current control methods ~\cite{zhang2023addingconditionalcontroltexttoimage, gandikota2023concept, ye2023ipadaptertextcompatibleimage,ye2023ipadaptertextcompatibleimage, hertz2024stylealignedimagegeneration, li2023photomaker, shi2024instantbooth, chen2024trainingfreeregionalpromptingdiffusion} require users to specify their desired manipulations in advance, limiting interactive creativity. This contrasts with natural human artistic workflows, where creators dynamically explore creative ideas while jointly refining them toward meaningful artistic outcomes~\cite{hoffmann2016modeling}. This synergy between specification and exploration is not new to generative models. Early GAN architectures naturally developed disentangled latent spaces that enabled continuous\cite{harkonen2020ganspace,radford2015unsupervised, wu2021stylespace, shen2020interfacegan}, compositional control over generated images. Users could explore these spaces to discover interesting variations that would be difficult to describe in words~\cite{wu2021stylespace}, then combine them to achieve their creative goals~\cite{grabe2022towards}. 


% While diffusion models have largely superseded GANs in conditional image synthesis~\cite{dhariwal2021diffusion},  their underlying structure remains less understood. Diffusion models achieve remarkable diversity through high-dimensional latents, unlike GANs' compact latent spaces.  With a single prompt, diffusion models can generate radically different variations through different random initializations of input noise. We ask - Is it possible to discover interpretable structure within this vast space of variations?

Text-to-image diffusion models are capable of generating remarkable visual variations from a single prompt through different random initializations. However, this vast creative potential remains largely opaque to users---while we can generate diverse images, we lack understanding of the underlying structure of these variations. This presents a fundamental challenge: how can we discover and expose the latent visual capabilities encoded within these models?

\let\thefootnote\relax \footnote{$^{*}$Correspondence to \texttt{gandikota.ro@northeastern.edu}}

The challenge touches on a key limitation in how we interact with diffusion models today. Current control methods require users to explicitly specify their desired edits in advance through prompts~\cite{gandikota2023concept}, reference images~\cite{zhang2023addingconditionalcontroltexttoimage, chen2024trainingfreeregionalpromptingdiffusion, ruiz2022dreambooth,kumari2022customdiffusion, Ryu_lora, hu2021lora}, or attribute vectors~\cite{ye2023ipadaptertextcompatibleimage, hertz2024stylealignedimagegeneration, li2023photomaker, shi2024instantbooth,parmar2023zero,hertz2022prompt}. That contrasts sharply with natural human creative workflows, where artists dynamically explore creative ideas and jointly refine them toward meaningful artistic outcomes~\cite{hoffmann2016modeling}. The need for pre-specified controls creates a barrier between users and the full creative potential of these models.

Interestingly, earlier generative models like GANs~\cite{gans,karras2019style,brock2018large} naturally developed more interpretable internal structures. Their compact latent spaces often exhibited emergent disentanglement~\cite{harkonen2020ganspace,radford2015unsupervised, wu2021stylespace, shen2020interfacegan}, enabling continuous and compositional control over generated images. Users could explore these spaces to discover interesting variations that would be difficult to describe in words~\cite{wu2021stylespace}, then combine them to achieve their creative goals~\cite{grabe2022towards}.

Diffusion models have largely superseded GANs in conditional image synthesis~\cite{dhariwal2021diffusion}, achieving greater diversity through much higher-dimensional latents. And yet an understanding of the underlying structure of these larger latent spaces has remained elusive. In this work, we ask a fundamental question: \emph{Can we automatically discover the visual structure within a diffusion model's knowledge of a concept?} Rather than requiring user-specified controls, we aim to decompose the model's internal representations into expressive directions that users can explore and combine.

To address these needs, we present \textbf{SliderSpace}, a framework that brings systematic explorability to diffusion models. Given just a text prompt, SliderSpace discovers a canonical set of meaningful, diverse, and controllable directions within the model's knowledge of that concept. Each direction is implemented as a low-rank adapter~\cite{hu2021lora} that can be scaled and composed with others, allowing users to explore and smoothly combine different aspects of variation, as shown in Figure~\ref{fig:intro}.

We ground SliderSpace discovery in three key requirements for meaningful decomposition of a diffusion model's visual manifold: 
\begin{enumerate}
    \item \textbf{Unsupervised Discovery:} The decomposition process should emerge from the intrinsic structure of the model's learned representation, rather than being guided by predefined attributes. This ensures we capture the true topology of the model's knowledge space rather than projecting our assumptions onto it.
    
    \item \textbf{Semantic Orthogonality:} Each discovered control must represent a distinct semantic direction. This is enforced in a semantic feature space, like CLIP, where every slider has an orthogonal effect in embeddings. This prevents discovering multiple controls that create similar semantic effects, making the system more efficient and easier.
    
    \item \textbf{Distribution Consistency:} Directions must induce consistent transformations across both random seeds and prompt variations. 
\end{enumerate}

These requirements naturally lead to our proposed framework, which we formalize in Section~\ref{sec:method}. As we show in our experiments, SliderSpace is architecture-agnostic, working with both conventional U-Net based models like Stable Diffusion~\cite{rombach2022high, rombach2022sd20, podell2023sdxl, turbo, dmd} and recent transformer-based architectures like Flux~\cite{flux}.

We demonstrate the expressiveness of SliderSpace through three applications: First, we show how SliderSpace can decompose high-level concepts into diverse and expressive components, revealing the natural axes of variation in the model's understanding. Second, we explore artistic style variation, where SliderSpace discovers directions that match or exceed the diversity of manually curated artist lists while being judged more useful by human evaluators. Finally, we show how SliderSpace can help reverse the mode collapse commonly observed in distilled diffusion models, restoring diversity while maintaining generation speed.

Beyond providing practical creative control, SliderSpace opens new avenues for understanding and utilizing the latent capabilities of diffusion models. By mapping these models' visual potential into intuitive, composable directions, we take a step toward making their creative possibilities more accessible and interpretable to users.

% Image editing methods in diffusion models unlock the creativity of users. In this work we ask an alternate question: \emph{Can we organize and expose what of the diffusion model is already capable of?}.
% Existing methods for controlling image generation typically require users to manually specify edit directions for desired changes. This process is time-consuming, requires technical expertise, and limits the spontaneity of the creative process. For instance, if a user wants to adjust the smile of a generated person, they must explicitly request this edit, often through imprecise prompt engineering or model fine-tuning. This approach of predefined controls or manual specifications restricts users from fully exploring the latent capabilities of the model. There may be interesting stylistic variations or attributes that the model can generate, but users have no easy way to discover or utilize these.

% Natural visual disentanglement was an emergent property in the latent space of Generative Adversarial Models (GANs) \cite{harkonen2020ganspace,radford2015unsupervised, wu2021stylespace, shen2020interfacegan}. In particular, it has been observed that StyleGAN~\cite{karras2019style} stylespace neurons offer detailed control over many meaningful aspects of images that would be difficult to describe in words~\cite{wu2021stylespace}. However, diffusion models do not share such a compact latent space~\cite{park2023unsupervised}; and efforts to uncover such a space in the semantic embeddings of the text conditioning have met with limited success \nik{Nick - is there a specific citation you were thinking about?}.

% In this work we introduce \textbf{SliderSpace}, which takes a step towards uncovering an analogous low dimensional representation of diffusion models' visual breadth; in essence treating the diffusion model as many generators sharing parameters, where a particular generator is defined by a specific prompt. For a given prompt we sample many random seeds (and optionally prompt expansions using an LLM), generate the corresponding images, and apply an off the shelf feature extractor (in this work CLIP, but our method can be applied to any differentiable feature extractor). We use PCA to analyze these features, and for each of the leading $k$ principal components we train a LoRA \cite{} which causes the diffusion model to produces images which increase the feature magnitude along that component when passed back through the same feature extractor. This leads to a 'Slider' for each principal component, because each LoRA can be scaled and applied to the original diffusion model, continuously varying those visual features in the generated results (as measured, in our case, by CLIP).

% There are many other works that enhance the controllability of diffusion models. One common approach is enabling users to add spatial constraints to a generation either manually, or via a reference image \cite{zhang2023addingconditionalcontroltexttoimage, chen2024trainingfreeregionalpromptingdiffusion}, a second is leveraging more abstract embeddings (e.g. identity, style) extracted from a reference image \cite{ye2023ipadaptertextcompatibleimage, hertz2024stylealignedimagegeneration, li2023photomaker, shi2024instantbooth}, a third is finetuning a foundation model to better generate a concept important to the user \cite{ruiz2022dreambooth, kumari2022customdiffusion, Ryu_lora, hu2021lora}, and a fourth (most relevant to this work) is finding low-rank adaptors of the model based on a prompt or small training set which can be scaled to provide continous control over one aspect of generated image (e.g. night vs day, basic vs luxury, etc.) \cite{gandikota2023concept}. SliderSpace is complementary to all of these methods and offers something distinct. All of the other methods we are aware require the user (and / or model designer) to know in advance what type of control they want. In contrast SliderSpace assists users in discovering and controlling hidden capabilities present in the diffusion model's distribution of possible generations.

%We propose that truly intuitive creative control in a text-to-image model should meet three key criteria: \emph{discoverability}, \emph{intuitiveness}, and \emph{specificity}. The model should reveal controllable attributes that may not be immediately obvious, offer controls that are easy to understand and manipulate, and ensure each control affects a distinct attribute of the generated image.

% We demonstrate the utility and power of SliderSpace using three applications built on top of SDXL-DMD \cite{dmd}, because its fast generation speed lends itself well to the continuous control offered by SliderSpace.

% First, we study concept decomposition (Section \ref{sec:concept_exp}), where we learn sliders for a specific concept (e.g. 'monster', 'waterfall', 'car'). Through quantitative metrics of diversity and text alignment we demonstrate that the learned sliders dramatically boost the diversity of generations when randomly applied without harming text alignment; we also ask humans to qualitatively judge these results in a user study where they find the SliderSpace results to be more 'Diverse', 'Useful', and 'Creative' than our baselines.

% Second, we attempt to compare the automatic discoveries of SliderSpace to a large scale manual study of artistic styles (Section \ref{sec:art_exp}), open-sourced by ParrotZone \cite{parrotzone}. In this study SDXL was prompted with over 4300 artist names,  and based on visual inspection the cases of successful stylistic mimicry recorded. Quantitatively SliderSpace more closely matches the distribution of artistic variation discovered by ParrotZone than other baselines, and in our user studies was judged to be significantly more 'Diverse' and 'Useful' than the baselines. To our surprise humans even judged SliderSpace results to be slightly more 'Diverse' than the results generated by the manually discovered artist names of \cite{parrotzone}.

% Third, we attempt to use SliderSpace to reverse the mode collapse commonly observed in distilled few-step diffusion models relative to the original teacher model (Section \ref{sec:diverse_exp}). We quantitatively demonstrate that applying SliderSpace to SDXL-DMD leads to more closely matching the distribution of images by the original teacher, SDXL.

%Through extensive experiments on various state-of-the-art text-to-image models, we demonstrate that SliderSpace significantly enhances user control and creative expression in AI-assisted image generation tasks. Our method enables a range of applications, including concept decomposition and control, diversity improvement in generated images, customization dissection and edits, and the exploration of artistic styles inherent in the model.

% SliderSpace goes beyond providing a practical tool for enhanced creative control. By mapping the visual potential of diffusion models it can open new avenues for generative creativity and deepens our understanding of each model's hidden potential.
\subsection{The Generative Model}\label{sec:generative_model}
Let there be $n_1$ users and $n_2$ items. %
Each user $u$ and each item $i$ has a $r$-dimensional feature vector. %
The inner product of these two feature vectors gives the utility (or score) $x^*_{u,i}$ that user $u$ has for item $i$. 
This modeling assumption implies that
the score matrix $X^* \in \Real{n_1 \times n_2}$ has rank $r$, and thus admits the following rank-$r$ SVD:
\begin{align}
    X^* = U^* \Sigma^* V^{*T},
\end{align}
where $U^* \in \Real{n_1 \times r}$ and $V^* \in \Real{n_2 \times r}$ are matrices that satisfy $U^{*T} U^* = V^{*T} V^* = I_r$, and $\Sigma^* \in \Real{r \times r}$ is a diagonal matrix with entries $\sigma_1^* \geq \ldots \geq \sigma_r^* > 0$. Let $\kappa \triangleq \sigma_1^*/\sigma_r^*$ denote the condition number of $\Sigma^*$.

Let $n = n_1 + n_2$. Define $Z^* \in \Real{n \times r}$ and $Y^* \in \Real{n \times n}$ as follows:
\begin{align}\label{eq:matrix_z}
    Z^* &= 
    \begin{bmatrix}
        U^* \\ V^*
    \end{bmatrix}
    \Sigma^{* 1/2}, \\ 
    Y^* &= Z^*Z^{*T} = 
    \begin{bmatrix}
        U^*\Sigma^*U^{*T} & X^*\\
        X^{*T} & V^*\Sigma^*V^{*T}
    \end{bmatrix}.
\end{align}
From here on, we shall refer to $Z^*$ as the ground-truth matrix. Note that the singular values of $Z^*$ are $\sqrt{2\sigma_1^*}, \ldots, \sqrt{2\sigma_r^*}$.

We are given a dataset $\Dataset$ where each data point represents a comparison made by a user between two items. The size of the dataset, \textit{i.e.,} the number of data points, is represented by $m$. We index the dataset by $k$. Each data point $\Dataset_k$ is of the form $((u; i, j), w)$ and is sampled randomly as follows. The user index $u$ is chosen uniformly at random from $[n_1]$. The pair of item indices $(i,j)$ is chosen uniformly at random from the set of $n_2(n_2-1)$ pairs of distinct items. The item pair $(i,j)$ is sampled independently from $u$. The triplets for different datapoints are sampled independently of each other.

The variable $w$ reflects the outcome of the comparison made by the user $u$ between items $i$ and $j$. In the \textit{noisy} setting, $w$ is an indicator for the outcome of the comparison; it is one if $i$ is chosen and zero if $j$ is chosen. Given a triplet $(u; i, j)$, $w$ is a Bernoulli random variable with parameter $g(x^*_{u,i} - x^*_{u,j})$, where $g: \Real{} \rightarrow (0,1)$ is a known \textit{link function} that translates real-valued preferences to a binary scale. In the \textit{noiseless} setting, $w$ is set to the expected value of the corresponding noisy case; \textit{i.e.}, $w = g(x^*_{u,i} - x^*_{u,j})$. 

In this work, we assume we are given noiseless data. We assume the link function is a smooth, strictly increasing function and is symmetric around zero in the following sense: $g(-x) = 1 - g(x)$. For example, $g(x)$ could be the logistic link function: $e^x/(1 + e^x)$; this is the link function found in the Bradley-Terry-Luce choice model. 


\subsubsection{Important Parameters}
 

\paragraph{Incoherence} For any matrix $Z$, let $\norm{Z}_{2, \infty}$ denote the maximum of the $\ell_2$ norm of its rows and let $\norm{Z}_{F}$ denote the Frobenius norm of $Z$. Define the \textit{incoherence parameter} of the ground-truth matrix as 
\begin{align}\label{eq:def_mu}
    \mu \triangleq n(\norm{Z^*}_{2, \infty}^2/\norm{Z^*}_{F}^2).
\end{align}
In principle, $\mu$ can take values from $1$ to $n$. However, the sample complexity worsens with $\mu$.

\paragraph{Link Function Bounds}  Let $I$ denote the interval {$[-{24 \mu (\norm{Z^*}_F^2}/{n}), {24 \mu (\norm{Z^*}_F^2}/{n})]$.} %
Let $\xi$ and $\Xi$ be lower and upper bounds for the following expression:
\begin{align}\label{eq:link_function_lower_bound}
    \xi &\triangleq  \min_{(x, y) \in I \times I} \frac{g'(x)g'(y)}{g(x)(1 - g(x))}, \\ 
    \Xi &\triangleq \max_{(x, y) \in I \times I} \frac{g'(x)g'(y)}{g(x)(1 - g(x))}. \label{eq:link_function_upper_bound}
\end{align}
By the assumptions on $g(\cdot)$ stated above, $\xi$ is strictly positive and $\Xi$ is finite.
For the logistic link function, $g'(x)=g(x)(1-g(x))$, which implies $\xi = g'(24 \mu (\norm{Z^*}_F^2/n))$ and $\Xi = 1/4$.


\subsection{The Loss Function}\label{sec:loss_function}
Given any $Z \in \Real{n \times r}$, we interpret $Z$ as the concatenation of some candidate user features $U \in \Real{n_1 \times r}$ and item features $V \in \Real{n_2 \times r}$. 
The likelihood of the dataset $\Dataset$ under $Z$ is simply the probability of observing $\Dataset$ if the data was generated according to the parameters $Z$. 
In this work, we use the maximum likelihood approach to learn the latent parameters. \textit{I.e.,} we use the negative log likelihood as the loss function, which we shall minimize using a gradient-descent-like method. 
Here, we present the loss function and its gradient, using notation that will be useful later on.

Let $e_1, e_2, \ldots e_{n_1}$ denote unit vectors in $\Real{n_1}$ and let $\Tilde{e}_1, \Tilde{e}_2, \ldots, \Tilde{e}_{n_2}$ denote unit vectors in $\Real{n_2}$. Let $\llangle C, D \rrangle = \sum_{i,j} c_{i,j}d_{i,j}$ denote the matrix inner product between two matrices of the same size. Therefore:
\begin{align}\label{eq:def_A1}
    \llangle e_u(\Tilde{e}_i - \Tilde{e}_j)^T, X^* \rrangle = x^*_{u,i} - x^*_{u,j}.
\end{align}

For any triplet $(u; i, j)$, define the corresponding \textit{sampling matrix} $A \in \Real{n \times n}$ to be:
\begin{align}\label{eq:def_A2}
    A = \begin{bmatrix}
        0 & e_u(\Tilde{e}_i - \Tilde{e}_j)^T\\
        0 & 0
    \end{bmatrix} \Rightarrow \llangle A, Y^* \rrangle = x^*_{u,i} - x^*_{u,j}.
\end{align}
In the equation above, $0$ denotes matrices with all entries zero of the appropriate size. With this notation, for any data point $((u; i, j), w)$, we have:
\begin{align}
    \mathbb{P}(w = 1 \, | \, (u; i, j)) = \mathbb{P}(w = 1 \, | \, A) = g(\llangle A, Y^* \rrangle).
\end{align}


Given a binary outcome $w$, the likelihood of the outcome under a Bernoulli distribution with parameter $p$ is $p^{w}(1-p)^{1-w}$. Therefore, the negative log-likelihood of this observation is $-w\log(p) -(1-w)\log(1-p)$. Next, consider a datapoint $((u; i,j), w)$ with the corresponding sampling matrix $A$. The negative log-likelihood of this observation under our model with parameters $Z \in \Real{n \times r}$ is 
\begin{align*}
    -w \log(g(\llangle A, ZZ^T \rrangle)) - (1-w) \log(1 - g(\llangle A, ZZ^T \rrangle)).
\end{align*}
Let $A_k$ denote the sampling matrix corresponding to the datapoint $\mathcal{D}_k$. Then, for the entire dataset, the (normalized) negative log likelihood is given by:
\begin{align}\label{eq:log_likelihood}
    \Loglikelihood(Z) &= \frac{1}{m} \sum_{k = 1}^{m} -w_k \log(g(\llangle A_k, ZZ^T \rrangle)) \nonumber \\
    &\quad - (1-w_k) \log(1 - g(\llangle A_k, ZZ^T \rrangle)).
\end{align}
The gradient of $\Loglikelihood(Z)$ is 
\begin{align} \label{eq:gradient_likelihood}
    \nabla \Loglikelihood(Z) &= \frac{1}{m}\sum_{k=1}^{m}
    h_k (A_k+ A_k^T) Z, \text{ where }\\
    h_k &\triangleq \frac{g'(z_k)\left(g(z_k) - w_k\right)}{g(z_k)(1-g(z_k))}, \ z_k \triangleq \llangle A_k,ZZ^T \rrangle. \nonumber
\end{align}
Here, $\nabla \Loglikelihood(Z)$ is a matrix of the same size as $Z$ while $h_k$ and $z_k$ are scalars. %


\subsection{Symmetries in the Problem}\label{sec:symmetries}
The generative model, and consequently the log likelihood function, is invariant to certain transformations in the parameters. In other words, the problem structure has certain symmetries. We explore these symmetries and their consequences in this section.

\paragraph{Scale Invariance} 
For any ground-truth score matrix $X^*$, the factorization $(U^*, V^*)$ is not unique.
Indeed, for any invertible $r \times r$ matrix $P$, the pair of feature matrices $(U^*P^T, V^*P^{-1})$ is indistinguishable from $(U^*, V^*)$ as they both lead to the same score matrix $X^*$. However, we can distinguish `imbalanced' feature vectors from `balanced' ones by by adding the term $\norm{U^TU - V^TV}_F^2$ to the loss function. Minimizing this regularizer while keeping the log-likelihood constant leads to a pair of feature matrices that are balanced in the norms. In more compact terms, the regularizer can be written as follows:
\begin{align}\label{eq:def_regularizer}
    \mathcal{R}(Z) \triangleq \norm{Z^TDZ}_F^2; \ D \triangleq \begin{bmatrix}
        I_{n_1} & 0\\
        0 & -I_{n_2}
    \end{bmatrix}.
\end{align}
Note that the ground-truth matrix $Z^*$ satisfies $\mathcal{R}(Z^*) = 0$. Combining the regularizer with the negative log likelihood, the objective function becomes:
\begin{align}\label{eq:objective_function}
    f(Z) \triangleq \mathcal{L}(Z) + (\lambda/4)\mathcal{R}(Z),
\end{align}
where $\lambda$ is a positive constant. In this work, we set $\lambda = \xi\gamma/4$; however, in practice, it may be viewed as a hyperparameter. In summary, adding the regularizer $\mathcal{R}(Z)$ factors out the scale-invariance of the problem.

\paragraph{Rotational Invariance}
Beyond the scale invariance, the problem at hand also exhibits rotational invariance.
Let $R$ be any orthogonal matrix in $r$ dimensions, \textit{i.e.}, $R \in \mathbb{R}^{r \times r}$ such that $RR^T=R^TR=I$. The pair of feature matrices $(U^*R, V^*R)$ give rise to the same scores as $(U^*, V^*)$. Thus, one can identify the ground-truth features only up to an orthogonal transformation. Denote this equivalence class of the ground-truth feature matrices by $\solset$:
\begin{align}\label{eq:def_solutionset}
    \solset \triangleq \{ \Tilde{Z}^* \, : \Tilde{Z}^* = Z^* R \text{ for some } \text{orthogonal } R\}.
\end{align}
This equivalence class of solutions naturally gives rise to a new distance metric that measures how close a candidate solution $Z$ is to $\solset.$
Define
\begin{align}\label{eq:def_distance}
     R(Z) &\triangleq \text{arg} \min_{R: R^TR = RR^T = I_r} \norm{Z - Z^* R}_F, \\
    \solz &\triangleq \text{arg} 
    \min_{\Tilde{Z}^* \in \solset} \norm{Z - \Tilde{Z}^*}_F = Z^* R(Z), \label{eq:def_closest_sol} \\
     \delz &\triangleq Z - \solz. \label{eq:def_difference_sol}
\end{align}
We measure the quality of a solution $Z$ by $\norm{\delz}_F$.


\paragraph{Shift Invariance} Comparisons invariably involve computing the difference between the utilities of items. Therefore, the learning problem is invariant to a global shift in the scores. Mathematically, this can be seen as follows. Let $\Tilde{V}^* = V^* + 1v^T$, where $v \in \Real{r}$ and $1 \in \Real{n_2}$ is the vector of all ones. Let $\Tilde{X}^*, \Tilde{Y}^*,$ and $\Tilde{Z}^*$ denote the corresponding quantities derived from $(U^*, \Tilde{V}^*)$. Then for any triplet $(u; i, j)$ and the corresponding sampling matrix $A$, we have $\llangle e_u(\Tilde{e}_i - \Tilde{e}_j)^T, \Tilde{X}^* \rrangle = \llangle e_u(\Tilde{e}_i - \Tilde{e}_j)^T, X^* \rrangle$, which implies $\llangle A, \Tilde{Y}^* \rrangle = \llangle A, Y^* \rrangle.$ Because of this invariance, we assume, without loss of generality, that $1^T V^* = 0$. In words, we assume that the item features of all matrices in $\solset$ sum to zero.

The shift invariance also manifests itself in our objective function $\Loglikelihood(Z)$. It is important to factor out the shift invariance in order to establish a strong-convexity like property (i.e., a curvature) for $\Loglikelihood(Z)$. Therefore, we restrict our attention to the following subspace:
\begin{align}\label{eq:H_hyperplane}
    \mathcal{H} = \{ Z \in \Real{n \times r}: Z = (U, V), 1^TV = 0 \}.
\end{align}
For any $Z = (U, V)$, we shall work with the projection of $Z$ onto $\mathcal{H}$, denoted by $\mathcal{P}_{\mathcal{H}}(Z)$. This projection is given by $(U, JV)$, where $J \triangleq I_{n_2} - 11^T/(n_2)$. Finally, note that by the assumption stated before, $\solset \subseteq \mathcal{H}$.




\begin{figure}[t]
    \centering
    \includegraphics[width=0.7\linewidth]{Photos/TerraTrace-fig3.pdf}
    \vspace{-0.2cm}
    \caption{\textbf{TerraTrace System.} TerraTrace finds the NDVI coordinates from our dataset, extracts a set of metrics to analyze the data, and passes these to GPT-4 Turbo for additional analysis.}
    \label{fig:Fig3}
    \vspace{-0.5cm}
\end{figure}
\section{TerraTrace Platform}
\label{system}
We combine our insights about NDVI signatures from the dataset developed in Sec.~\ref{signatures} to develop TerraTrace, an end-to-end AI powered land use analysis platform. TerraTrace takes in a set of geographic coordinates that define the target region. From the dataset, we filter coordinates within this geo-polygon by coarse latitude and longitude ranges to identify the dataset region. Next we calculate the Euclidean distance between a target coordinate and points in our dataset. We extract the corresponding signature curves within the prescribed polygon by computing the mean NDVI value per time point across valid coordinates. We then interpolate the NDVI values and fit a 3rd order polynomial. TerraTrace presents users with a GUI that plots the region using the Leaflet interactive map library which creates an interface to adjust the polygon and plot the NDVI signature curve. 

TerraTrace also analyzes the data to extract land use insights. First, we check that the data is valid and the curve has $>$10 points for robust classification. We then extract features such as the annual minimum and maximum NDVI, range, and median. We determine the growth and decline rates by calculating the maximum point difference in NDVI values. We then use these metrics to check if the curve is an annual crop. This means that the NDVI increases above 0.2 indicating healthy vegetation growth, reaches a peak between 0.2 and 0.8, followed by a decline. We check the growth and decline rates are $>$0.005 to help filter out perennial species which only have small NDVI fluctuations across seasons.

TerraTrace complements these metrics with an LLM based analysis. We pass in the statistics such as max, min and average, an image of the NDVI curve, and a classification of whether the region contains vegatation using a JSON format. We determine vegetation presence with thresholds: <0.1 is non-vegetative, 0.1-0.2 as some vegetation, and 0.2 as healthy vegetation \cite{eos}. We pass in images converted to grayscale, resized, and encoded as a base64 string within the JSON. Next we construct prompts like the following to query the model. "The area of interest is defined by the $[{coordinates}]$. Please analyze the land cover type at this location." We use GPT-4 Turbo (version 2024-04-09) which translates the curves and data into a detailed analysis table, providing an additional validation. Further explanation of the algorithm is presented in Fig.~\ref{fig:Fig3}.

%We integrate additional data sources as well. We use the CDL to calculate a percentage of total crop pixels of specific types within the target region and 

%\textcolor{red}{TODO: move Historic Wild-Fire Data to the end}

%We pass in the statistics such as max, min and average, an image of the NDVI curve, and a classification of whether the region contains vegatation. We determine this using a simple threshold defining <0.1 as non-vegetative, 0-0.2 as some vegetation, and 0.2 as healthy vegetation. We pass in images converted to grayscale, resized, and encoded as a base64 string within the JSON. Next we construct prompts like the following to query the model. "The coordinates for the area of interest are {coordinates}. Please analyze the land cover type at this location."


%VI: I'm removing this because I don't know if it's clear that the compute required to serve an LLM is lower than this. Maybe it is but without hard numbers I think it may raise questions.
%After the curves are plotted, a comprehensive data analysis is conducted. We reduce the platform's dependence on extensive data and compute power by combining mathematical modeling with LLM analysis.


%To enhance the robustness of our analysis, we employ a Large Language Model GPT-4 Turbo. Utilizing up to 2000 tokens per query, the LLM translates the visual representation of the curves into a detailed analysis table, providing an additional layer of validation. Further explanation of the algorithm is represent in Fig.~\ref{Fig3}.



\section{Data}

\subsection{Pre-training Data}
We constructed a large-scale video dataset comprising 2B video-text pairs and 3.8B image-text pairs. Leveraging a comprehensive data pipeline, we transformed raw videos into high-quality video-text pairs suitable for model pre-training. As illustrated in Figure~\ref{fig:data_pipeline}, our pipeline consists of several key stages: Video Segmentation, Video Quality Assessment, Video Motion Assessment, Video Captioning, Video Concept Balancing and Video-Text Alignment. Each stage plays a crucial role in constructing the dataset, and we describe them in detail below.

\begin{figure*}[t]
    \centering
    \includegraphics[width=1.3\textwidth, center, trim=0 0 0 0, clip]{figure/data/data_pipeline.png}
    \caption{The pipeline of Step-Video-T2V data process.}
    \label{fig:data_pipeline}
    %\vspace{-6mm}
\end{figure*}


\paragraph{Video Segmentation} 
We began by processing raw videos using the \textit{AdaptiveDetector} function in the \texttt{PySceneDetect}~\citep{pyscenedetect} toolkit to dentify scene changes and use FFmpeg~\citep{ffmpeg} to split them into single-shot clips. 
We adjusted the splitting process for high-resolution videos not encoded with \texttt{libx264} to include necessary reference frames—specifically by properly setting the crop start time in \texttt{FFmpeg}; this prevented visual artifacts or glitches in the output video.
We also removed the first three frames and the last three frames of each clip, following practices similar to Panda70M \cite{chen2024panda} and Movie Gen Video \cite{polyak2024moviegencastmedia}. Excluding these frames eliminates unstable camera movements or transition effects often present at the beginnings and endings of videos.


\paragraph{Video Quality Assessment}

To construct a refined dataset optimized for model training on high-quality, we systematically evaluated and filtered video clips by assigning multiple Quality Assessment tags based on specific criteria. We uniformly sampled eight frames from each clip to compute these tags, providing a consistent and comprehensive assessment of each video.

\begin{itemize}[left=0cm]
\item \textbf{Aesthetic Score}: We used the public LAION CLIP-based aesthetic predictor~\cite{schuhmann2022laion} to predict the aesthetic scores of eight frames from each clip and calculated their average.

\item \textbf{NSFW Score}: We employed the public LAION CLIP-based NSFW detector~\cite{laion2021nsfw}, a lightweight two-class classifier using CLIP ViT-L/14 embeddings, to identify content inappropriate for safe work environments.

\item \textbf{Watermark Detection}: Employing an EfficientNet image classification model~\cite{tan2019efficientnet}, we detected the presence of watermarks within the videos.

\item \textbf{Subtitle Detection}: Utilizing PaddleOCR~\cite{paddleocr}, we recognized and localized text within video frames, identifying clips with excessive on-screen text or captions.

\item \textbf{Saturation Score}: 
We assessed color saturation by converting video frames from BGR to HSV color space and extracting the saturation channel, using OpenCV \cite{opencv_library}. We computed statistical measures—including mean, maximum, and minimum saturation values—across the frames. 


\item \textbf{Blur Score}: 
We detect blurriness by applying the variance of the Laplacian method~\cite{pech2000diatom} to measure the sharpness of each frame. Low variance values indicate blurriness caused by camera shake or lack of clarity.

\item \textbf{Black Border Detection}: We use \texttt{FFmpeg} to detect black borders in frames and record their dimensions to facilitate cropping, ensuring that the model trains on content free of distracting edges.
\end{itemize}



\paragraph{Video Motion Assessment}

Recognizing that motion content is crucial for representing dynamic scenes and ensuring effective model training, we calculate the motion score by averaging the mean magnitudes of the optical flow \cite{opencv_library} between pairs of resized grayscale frames, using the Farneback algorithm. We introduced three evaluative tags centered around motion scores:


\begin{itemize}[left=0cm]
    \item \textbf{Motion\_Mean}: The average motion magnitude across all frames in the clip, indicating the general level of motion. This score helps us identify clips with appropriate motion; clips with extremely low \texttt{Motion\_Mean} values suggest static or slow motion scenes that may not effectively contribute to training models focused on dynamic content.

    \item \textbf{Motion\_Max}: The maximum motion magnitude observed in the clip, highlighting instances of extreme motion or motion distortion. High \texttt{Motion\_Max} values may indicate the presence of frames with excessive or jittery motion.

    \item \textbf{Motion\_Min}: The minimum motion magnitude in the clip, identifying clips with minimal motion. Clips with very low \texttt{Motion\_Min} may contain idle frames or abrupt pauses, which could be undesirable for training purposes.
\end{itemize}





\paragraph{Video Captioning} 
Recent studies~\citep{openaisora, betker2023improving} have highlighted that both precision and richness of captions are crucial in enhancing the prompt-following ability and output quality of generative models. 
 
Motivated by this, we introduced three types of caption labeling into our video captioning process by employing an in-house Vision Language Model (VLM) designed to generate both short and dense captions for video clips.
\begin{itemize}[left=0cm]
    \item \textbf{Short Caption}: The short caption provides a concise description, focusing solely on the main subject and action, closely mirroring real user prompts.

    \item \textbf{Dense Caption}: The dense caption integrates key elements, emphasizing the main subject, events, environmental and visual aspects, video type and style, as well as camera shots and movements. To refine camera movements, we manually collected annotated data and performed SFT on our in-house VLM, incorporating common camera movements and shooting angles.
    
    \item \textbf{Original Title}: We also included a variety of caption styles by incorporating a portion of the original titles from the raw videos, adding diversity to the captions.
    
\end{itemize}



\paragraph{Video Concept Balancing}
To address category imbalances and facilitate deduplication in our dataset, we computed embeddings for all video clips using an internal VideoCLIP model and applied K-means clustering \cite{macqueen1967some} to group them into over 120,000 clusters, each representing a specific concept or category. By leveraging the cluster size and the distance to centroid tags, we balanced the dataset by filtering out clips that were outliers within their respective clusters. As part of this process, we added two new tags to each clip:

\begin{itemize}[left=0cm] 
    \item \textbf{Cluster\_Cnt}: The total number of clips in the cluster to which the clip belongs.

    \item \textbf{Center\_Sim}: The cosine distance between the clip's embedding and the cluster center.
\end{itemize}



\paragraph{Video-Text Alignment}

Recognizing that accurate alignment between video content and textual descriptions is essential to generate high-quality output and effective data filtering, we compute a \textbf{CLIP Score} to measure video-text alignment. This score assesses how well the captions align with the visual content of the video clips.

\begin{itemize}[left=0cm] 
\item \textbf{CLIP Score}: We begin by uniformly sampling eight frames from the given video clip. Using the CLIP model~\cite{yang2022chineseclip}, we then extract image embeddings for these frames and a text embedding for the video caption. The \texttt{CLIP Score} is computed by averaging the cosine similarities between each frame embedding and the caption embedding.



\end{itemize}


\subsection{Post-training Data}


For SFT in post-training, we curate a high-quality video dataset that captures good motion, realism, aesthetics, a broad range of concepts, and accurate captions. Inspired by \cite{dai2023emu, polyak2024moviegencastmedia, kong2024hunyuanvideo}, we utilize both automated and manual filtering techniques:

\begin{itemize}[left=0cm]


\item \textbf{Filtering by Video Assessment Scores}: Using video assessment scores and heuristic rules, we filter the entire dataset to a subset of 30M videos, significantly improving its overall quality.


\item \textbf{Filtering by Video Categories}: For videos within the same cluster, we use the "Distance to Centroid" values to remove those whose distance from the centroid exceeds a predefined threshold. This ensures that the resulting video subset contains a sufficient number of videos for each cluster while maintaining diversity within the subset.



\item \textbf{Labeling by Human Annotators}: In the final stage, human evaluators assess each video for clarity, aesthetics, appropriate motion, smooth scene transitions, and the absence of watermarks or subtitles. Captions are also manually refined to ensure accuracy and include essential details such as camera movements, subjects, actions, backgrounds, and lighting.

\end{itemize}



\section{Training Strategy}
\begin{table}[h!]
\centering
\resizebox{\textwidth}{!}{
\begin{tabular}{ccccccc}
\hline training stage & dataset & bs/node & learning rate & \#iters & \#seen samples \\
\hline 
\hline
    \multirow{3}{*}{Step-1: T2I Pre-training (256px)} & $\mathcal{O}(1) \mathrm{B}$ images & 40 & 1e-4 & 53k & 0.8B \\
     & $\mathcal{O}(1) \mathrm{B}$ images & 40 & 1e-4 & 200k & 3B \\
     \cline{2-6}
     & \textbf{Total} &  &  &  \textbf{253k} & \textbf{3.8B} \\
\hline 
\hline
    \multirow{4}{*}{Step-2: T2VI Pre-training (192px)} & $\mathcal{O}(1) \mathrm{B}$ video clips & 4 & 6e-5 & 171k & 256M\\
    & $\mathcal{O}(100) \mathrm{M}$ video clips & 4 & 6e-5 & 101k & 151M \\
    & $\mathcal{O}(100) \mathrm{M}$ video clips & 4 & 6e-5 & 158k & 237M \\
    \cline{2-6}
    & \textbf{Total} &  &  &  \textbf{430k} & \textbf{644M} \\
\hline
\hline
    \multirow{4}{*}{Step-2: T2VI Pre-training (540px)} & $\mathcal{O}(100) \mathrm{M}$ video clips & 2 & 2e-5 & 23k & 17.3M\\
    & $\mathcal{O}(10) \mathrm{M}$ video clips & 2 & 1e-5 & 17k & 8.5M \\
    & $\mathcal{O}(1) \mathrm{M}$ video clips & 1 & 1e-5 & 6k & 1.5M \\
    \cline{2-6}
    & \textbf{Total} &  &  &  \textbf{46k} & \textbf{27.3M} \\
\hline
\end{tabular}
}
\caption{Pre-training details of Step-Video-T2V. 256px, 192px, and 540px denote resolutions of 256x256, 192x320, and 544x992, respectively.}
\label{trainingrecipe}
\end{table}



\begin{figure}[h] 
    \centering
    \includegraphics[width=0.5\textwidth]{figure/v2_training_loss.png}  
    \caption{Training curve of different training stages, where $s_{i}$ denotes the $i^{th}$ dataset used in the corresponding stage.} 
    \label{fig:training curve}  
\end{figure}

A cascaded training strategy is employed in Step-Video-T2V, which mainly includes four steps: text-to-image (T2I) pre-training, text-to-video/image (T2VI) pre-training, text-to-video (T2V) fine-tuning, and direct preference optimization (DPO) training. The pre-training recipe is summarized in Table~\ref{trainingrecipe}.

\paragraph{Step-1: T2I Pre-training} In the initial step, we begin by training Step-Video-T2V with a T2I pre-training approach from scratch. We intentionally avoid starting with T2V pre-training directly, as doing so will significantly slow down model convergence. This conclusion stems from our early experiments with the T2V pre-training from scratch on the 4B model, where we observed that the model struggled to learn new concepts and was much slower to converge. By first focusing on T2I, the model can establish a solid foundation in understanding visual concepts, which can later be expanded to handle temporal dynamics in the T2V phase.

\paragraph{Step-2: T2VI Pre-training} After acquiring spatial knowledge from T2I pre-training in Step-1, Step-Video-T2V progresses to a T2VI joint training stage, where both T2I and T2V are incorporated. This step is further divided into two stages. In the first stage, we pre-train Step-Video-T2V using low-resolution (192x320, 192P) videos, allowing the model to primarily focus on learning motion-related knowledge rather than fine details. In the second stage, we increase the video resolution to 544x992 (540P) and continue pre-training to enable the model to learn more intricate details. We observed that during the first stage, the model concentrates on learning motion, while in the second stage, it shifts its focus more toward learning fine details. Based on these observations, we allocate more computational resources to the first stage in Step-2 to better capture motion knowledge.

\paragraph{Step-3: T2V Fine-tuning} Due to the diversity in pre-training video data across different domains and qualities, using a pre-trained checkpoint usually introduces artifacts and varying styles in the generated videos. To mitigate these issues, we continue the training pipeline with a T2V fine-tuning step. In this stage, we use a small number of text-video pairs and remove T2I, allowing the model to fine-tune and adapt specifically to text-to-video generation.


Similar to Movie Gen Video, we found that averaging models fine-tuned with different SFT datasets improves the quality and stability of the generated videos, outperforming the Exponential Moving Average (EMA) method. Even averaging checkpoints from the same data source enhances stability and reduces distortions. Additionally, we select model checkpoints based on the period after the gradient norm peaks, ensuring both the gradient norm and loss have decreased for improved stability.

\paragraph{Step-4: DPO Training}
As described in \S\ref{dpo}, video-based DPO training is employed to enhance the visual quality of the generated videos and ensure better alignment with user prompts.



\paragraph{Hierarchical Data Filtering}
\begin{figure*}[t]
    \centering
    \includegraphics[width=1.3\textwidth, center, trim=0 0 0 0, clip]{figure/data/data_filter.png}
    \caption{Hierarchical data filtering for pre-training and post-training.}
    \label{fig:data_filter}
    %\vspace{-6mm}
\end{figure*}


We apply a series of filters to the data, progressively increasing their thresholds to create six pre-training subsets for Step-2: T2VI Pre-training, as shown in Table~\ref{trainingrecipe}. The final SFT dataset is then constructed through manual filtering. Figure~\ref{fig:data_filter} illustrates the key filters applied at each stage, with gray bars representing the data removed by each filter, and colored bars indicating the remaining data at each stage.



\paragraph{Observations from Pre-training Curve}
%We use a funnel-style filtering method to progressively refine the training dataset throughout the pre-training stage. 
During pre-training, we observe a notable reduction in loss, which correlates with the improved quality of the training data, as illustrated in Figure \ref{fig:training curve}.

Additionally, a sudden drop in loss occurs as the quality of the training dataset improves. This improvement is not directly driven by supervision through a loss function during model training, but rather follows human intuition (e.g., filtering via CLIP scores, aesthetic scores, etc.). While the flow matching algorithm does not impose strict requirements on the distribution of the model’s input data, adjusting the training data to reflect what is considered higher-quality by humans results in a significant, stepwise reduction in training loss. This suggests that, to some extent, the model’s learning process may emulate human cognitive patterns.


\paragraph{Bucketization for Variable Duration and Size}

To accommodate varying video lengths and aspect ratios during training, we employed variable-length and variable-resolution strategies~\cite{chen2023pixartalphafasttrainingdiffusion, opensora}. We defined four length buckets (1, 68, 136, and 204 frames) and dynamically adjusted the number of latent frames based on the video length. Additionally, we grouped videos into three aspect ratio buckets—landscape, portrait, and square—according to the closest height-to-width ratio.





\section{Evaluation}
\label{sec:exps}

\subsection{Experimental Setup}


\paragraphbe{Models.} We evaluate our attack on three closed-source (o1, o1-mini and o3-mini) and open-source (DeepSeek-R1) reasoning models. These models leverage advanced reasoning methods such as CoT, and are well-known for excelling on a range of complex tasks and benchmarks~\cite{guo2025deepseek, sun2023survey}.

\paragraphbe{Datasets.} We evaluate our attack using FreshQA~\cite{vu2023freshllms} and SQuAD~\cite{rajpurkar2018know}. FreshQA is a dynamic question-answering (QA) benchmark designed to assess the factual accuracy of LLMs by incorporating both stable and evolving real-world knowledge. The benchmark includes 600 natural questions categorized into four types: never-changing, slow-changing, fast changing, and false-premise. These questions vary in complexity, requiring both single-hop and multi-hop reasoning, and are linked to regularly updated Wikipedia entries. The original query consists of an average of 11.6$\pm$1.85 tokens. However, due to the randomness and the length of the context extracted from Wikipedia, the total input token count increases to an average of 11278.2$\pm$6011.49 tokens when the context is appended. This leads to a noticeable variation in input length.

SQuAD contains more than 100k questions based on more than 500 articles retrieved from Wikipedia. While the average length of a query in the dataset is similar to FreshQA with $11.5\pm3.4$ tokens, the context is significantly shorter and shows less variance in length. An average context in the dataset contains $117.5\pm37.3$ tokens.  Utilizing these two datasets allows us to study our attack and impact of factors like context length and complexity.

We select a subset of the dataset containing samples with ground-truth that changes infrequently and has lower likelihood of unintentional errors. To minimize costs and adhere to ethical considerations, we restrict our evaluations of different attack types, attack transferability and reasoning effort to five data samples from FreshQA. This ensures minimal impact on existing infrastructure while allowing us to test our attack methodologies. Subsequently, we study the impact of context-agnostic attacks on 100 samples from the FreshQA and SQuAD datasets across four models (o1, o1-mini, DeepSeek-R1, and o3-mini) and present a comprehensive analysis of the attack performance on a larger scale.







\paragraphbe{Evaluation Metrics.} 
Since we evaluated our attack using QA datasets, we measure \textbf{claim accuracy}~\citep{ min2023factscore}. This is done by using LLM-as-a-judge, where the model verifies claims in its output against a list of ground truths. A score of 1 is assigned if the claims align and 0 if they do not. For longer outputs, more sophisticated claim verification metrics could be used~\cite{song2024veriscore, wei2024long}. Additionally, since our attack introduces a decoy problem, we assess output stealthiness by measuring the presence of decoy-related information in the final output, which we refer to as \textbf{contextual correctness}. This metric evaluates how much of the output belongs to the context surround the user query versus the decoy task. We assign a score of 1 if the output contains only claims relevant to the user query's context, 0.5 if it includes claims for both contexts, and 0 if it consists entirely of decoy-related information. All the results were also manually reviewed for errors. Fig.~\ref{fig:cc evaluation prompt} and Fig.~\ref{fig:cc_score_example} in Appendix show the contextual correctness evaluation prompt and output examples respectively.




\begin{table*}[t]

\centering
\vskip 0.15in
\begin{center}
\begin{small}
\begin{sc}
\begin{tabular}{lrrrrrr}
  \toprule
  \multirow{2}{*}{Attack Type} & \multirow{2}{*}{Input} & \multirow{2}{*}{Output} & \multirow{2}{*}{Reasoning} & Reasoning & \multirow{2}{*}{Accuracy} & Contextual \\
  & & & & Increase & & Correctness \\
  \midrule
  No Attack     & 7899$\pm$5797 & 102$\pm$53 & 751$\pm$410& 1 & 100\% & 100\% \\
  \midrule
  Context-Aware       & 11282$\pm$6660 & 37$\pm$11 & 1711$\pm$693 & 2.3$\times$ & 100\% & 100\% \\  
  Context-Agnostic   & 8237$\pm$5796 & 86$\pm$30 & 7313$\pm$347 & 9.7$\times$ & 100\% & 100\% \\ 
  ICL-Genetic (Aware) & 11320$\pm$6669 & 86$\pm$96 & 5850$\pm$978 & 7.8$\times$ & 100\% & 90\% \\  
  ICL-Genetic (Agnostic)  & 11191$\pm$6657 & 98$\pm$61 & \textbf{13555$\pm$3219} & \textbf{18.1$\times$} & 100\% & 100\% \\
  \bottomrule
\end{tabular}
\end{sc}
\end{small}
\end{center}
\vskip -0.1in

\caption{Average number of reasoning tokens for different attacks for o1 (\textbf{Dataset}: FreshQA, \textbf{Decoy}: MDP).}
\label{tab:o1_main_tab}
\end{table*}


\begin{table*}[t]

\centering
\vskip 0.15in
\begin{center}
\begin{small}
\begin{sc}
\begin{tabular}{lrrrrrr}
  \toprule
  \multirow{2}{*}{Attack Type} & \multirow{2}{*}{Input} & \multirow{2}{*}{Output} & \multirow{2}{*}{Reasoning} & Reasoning & \multirow{2}{*}{Accuracy} & Contextual \\
  & & & & Increase & & Correctness \\
  \midrule
  No Attack             & 10897$\pm$6011 & 245$\pm$191 & 711$\pm$635\;\; & 1 & 100\% & 100\% \\
  \midrule
  Context-Aware               & 11338$\pm$6014 & 177$\pm$151 & 1868$\pm$2020 & 4.8$\times$ & 80\% & 100\% \\  
  Context-Agnostic       & 11236$\pm$6011 & 77$\pm$26\;\;   & 2872$\pm$2820 & 4.0$\times$ & 80\% & 100\% \\ 
  ICL-Genetic (Aware) & 11393$ \pm$5964 & 93$\pm$63\;\;  & 6980$ \pm$5693  & 5.9$\times$ & 100\% & 80\% \\  
  ICL-Genetic (Agnostic)  & 11261$\pm$6011 & 68$\pm$16\;\;   & \textbf{7489$\pm$1305} & \textbf{10.5$\times$} & 80\% & 100\% \\
  \bottomrule
\end{tabular}
\end{sc}
\end{small}
\end{center}
\vskip -0.1in
\centering
\caption{Average number of reasoning tokens for different attacks for DeepSeek-R1 (\textbf{Dataset}: FreshQA, \textbf{Decoy}: MDP).}
\label{tab:deepseek_main_tab}
\end{table*}





\begin{table}[t]
\vspace{-13pt}
\caption{Ablation results with response number under fine-tuning setting. See Reward Margins in~\Cref{tab:rm}. \vspace{-0.5em}}
\label{tab:ablation}
\vskip 0.1in
\begin{center}
\scalebox{0.75}{
\begin{tabular}{clcccc}
\toprule
\multirow{1}{*}{\textbf{Number}} & \multirow{1}{*}{\textbf{Method}} & \textbf{BLEU}$\uparrow$ & \textbf{Reward} & $\textbf{RM}_{\text{DPO}}$$\uparrow$ & $\textbf{RM}_{\text{R-DPO}}$$\uparrow$ \\
\midrule
\multirow{2}{*}{\textbf{5}} & \textbf{DPO-BT} & \textbf{0.229} & \textbf{{0.432}} & 0.166 & -0.516 \\

& \textbf{DPO-HPS} & \textbf{0.229} & 0.431 & \textbf{0.600} & \textbf{-0.273} \\
\midrule
\multirow{2}{*}{\textbf{20}} & \textbf{DPO-BT} & \textbf{0.231} & 0.430 & 0.227 & -0.490 \\

& \textbf{DPO-HPS} & 0.224 & \textbf{{0.432}} & \textbf{0.822} & \textbf{-0.181} \\
\midrule
\multirow{2}{*}{\textbf{50}} & \textbf{DPO-BT} & \textbf{0.230} & \textbf{0.431} & 0.279 & -0.507 \\

& \textbf{DPO-HPS} & \textbf{0.230} & \textbf{0.431} & \textbf{1.645} & \textbf{1.037} \\
\midrule
\multirow{2}{*}{\textbf{100}} & \textbf{DPO-BT} & 0.230 & \textbf{0.431} & 0.349 & -0.455 \\

& \textbf{DPO-HPS} & \textbf{{0.232}} & 0.430 & \textbf{{2.723}} & \textbf{{2.040}} \\
\bottomrule
\end{tabular}}
\end{center}
\vspace{-1em}
\vspace{-9pt}
\end{table}


\begin{table}[t]


\vskip 0.15in
\begin{center}
\begin{small}
\begin{sc}
\begin{tabular}{lrrr}
  \toprule
  Source / Target & o1 & o1-mini & DeepSeek-R1 \\
  \midrule
  o1        & 18.1$\times$ & $6.2\times$ & $12.0\times$  \\
  o1-mini   & $2.9\times$  & $16.8\times$ & $7.5\times$ \\
  DeepSeek-R1  & $11.4\times$  & $4.4\times$  & $10.5\times$ \\
  \bottomrule
\end{tabular}
\end{sc}
\end{small}
\end{center}
\vskip -0.1in
\caption{Transferability matrix between models: o1, o1-mini, and DeepSeek-R1 (Attack Type: Manual Injection).}
\label{tab:transferability}
\end{table}



\begin{table}[t]


\vskip 0.15in
\begin{center}
\begin{small}
\begin{sc}
\begin{tabular}{lrrr}
  \toprule
  Effort & No Attack & Attack & Increase \\
  \midrule
  low     & 345$\pm$248 & 4940$\pm$686 & 14.3$\times$ \\
  medium$^*$ & 751$\pm$410 & 7313$\pm$347 & 9.7$\times$ \\
  high & 806$\pm$354 & 10176$\pm$1003 & 12.6$\times$ \\
  \bottomrule
\end{tabular}
\end{sc}
\end{small}
\end{center}
\vskip -0.1in
\caption{Average number of reasoning tokens for different reasoning effort levels in the o1 model (\textbf{Attack Type}: Context-Agnostic). $^*$Medium is a default effort for all other experiments.}
\label{tab:effort_comparison}
\end{table}



\begin{table*}[tpb]
\begin{center}
\begin{small}
\begin{sc}
\begin{tabular}{llrrrr}
\toprule
& \multirow{3}{*}{Metrics} & \multicolumn{2}{c}{o1} & \multicolumn{2}{c}{DeepSeek-R1} \\ 
\cmidrule(lr){3-4} \cmidrule(lr){5-6}
& & No Attack & Attack & No Attack & Attack \\ \midrule
\multirow{5}{*}{SQuAD} & Input Tokens& $155\pm37$ & $493\pm37$ & 149$\pm$37 & 489$\pm$39 \\
& Output Tokens& $32\pm8.4$ & $44\pm15$ & 63.41$\pm$21.1 & 41$\pm$11 \\
& Reasoning Tokens & $162\pm95$ & \textbf{7435}$\pm$\textbf{847(46}$\times$\textbf{)} & $222\pm116$ & \textbf{4452}$\pm$\textbf{1487(20}$\times$\textbf{)} \\ 
& Accuracy & 100\% & 100\% & 100\% & 100\% \\ 
&Contextual Correctness  & 100\% & 100\% & 100\% & 98\% \\ \cmidrule{1-6}
\multirow{5}{*}{FreshQA}& Input & $7265\pm6724$ & $7603\pm6725$ & 7344$\pm$6774 & 7684$\pm$6774 \\
& Output & $73\pm10$ & $68\pm26$ & 7684$\pm$6774 & 61$\pm$22 \\ 
& Reasoning Tokens & $565\pm558$ & \textbf{7146}$\pm$\textbf{984(13}$\times$\textbf{)} & $546\pm664$ & \textbf{3187}$\pm$\textbf{2011(6}$\times$\textbf{)} \\ 
& Accuracy & 91\% & 95\% & 89\% & 87\% \\ 
& Contextual Correctness & 100\% & 100\% & 100\% & 98.5\% \\ 
\bottomrule
\end{tabular}
\end{sc}
\end{small}
\end{center}
\caption{Performance of Context-Agnostic attack on o1 and DeepSeek-R1 models on 100 samples from SQuAD and FreshQA.}
\label{tab:Context_Agnostic_100samples_big_models}
\end{table*}

\begin{table*}[!ht]
\begin{center}
\begin{small}

\begin{sc}
\begin{tabular}{llrrrr}
\toprule
& \multirow{3}{*}{Metrics} & \multicolumn{2}{c}{o1-mini} & \multicolumn{2}{c}{o3-mini} \\ 
\cmidrule(lr){3-4} \cmidrule(lr){5-6}
& & No Attack & Attack & No Attack & Attack \\ \midrule
\multirow{5}{*}{SQuAD} & Input Tokens &$156\pm37$ &$397\pm37$ & 155$\pm$37 & 493$\pm$36 \\ 
& Output Tokens & $53\pm31$&$29\pm10$ & $31.12\pm11.3$ & $39.45\pm17.0$ \\ 
& Reasoning Tokens & $392\pm180$ & \textbf{3306}$\pm$\textbf{2791(8}$\times$\textbf{)} & $139\pm100$ & \textbf{4902}$\pm$\textbf{745(35}$\times$\textbf{)} \\ 
& Accuracy & 99\% & 98\% & 100\% & 100\% \\ 
&Contextual Correctness  & 100\% & 100\% & 100\% & 100\% \\ \cmidrule{1-6}
\multirow{5}{*}{FreshQA} & Input Tokens & $7399\pm6826$& $7639\pm6826$& 7270$\pm$6695 & 7608$\pm$6695 \\ 
& Output Tokens &$191\pm135$ &$49\pm31$ & 68$\pm$42 & 50$\pm$29 \\ 
& Reasoning Tokens & $456\pm269$ & \textbf{3136}$\pm$\textbf{3077(7}$\times$\textbf{)} & $559\pm446$ & \textbf{2182}$\pm$\textbf{776(4}$\times$\textbf{)} \\ 
& Accuracy & 91\% & 87\% & 88\% & 84\% \\ 
& Contextual Correctness  & 100\% & 100\% & 100\% & 100\% \\ 
\bottomrule
\end{tabular}
\end{sc}
\end{small}
\end{center}
\caption{Performance of Context-Agnostic attack on o1-mini and o3-mini on 100 samples from SQuAD and FreshQA.}
\label{tab:Context_Agnostic_100samples_mini_models}
\end{table*}







\subsection{Attack Setup}
To orchestrate the attack, we first retrieve context related to the question either directly from the dataset or using the links present in the dataset. We then inject manually written attack templates (discussed in sections \ref{subsec:manual_injection} and \ref{subsec:weaving_injection}) in the retrieved context and compare the model's responses to both the original and compromised contexts for evaluation. 
We select the best performing decoy problems from Table~\ref{tab:dataset_comparison} i.e Sudoku and MDP. For example of injection templates, refer to Figure~\ref{fig:context_agnostic_prompt_sudoku} and Figure~\ref{fig:context_agnostic_prompt} in Appendix~\ref{appendix: used_prompts}.
Finally, we utilize decoy-optimized context generated using Algorithm \ref{alg:ICL-genetic} to produce output and evaluate ICL-Genetic based attacks.

\subsection{Experimental Results}
We demonstrate the main experimental results of our \sys attack against o1 and DeepSeek-R1 models, demonstrating that all attack types significantly amplify the models' reasoning complexity. For the o1 model, Table~\ref{tab:o1_main_tab} shows that the baseline processing involves $751\pm410$ reasoning tokens. The ICL-Genetic (Agnostic) attack causes the largest increase -- an $18\times$ rise. Context-Agnostic and Context-Aware attacks also increase the token count significantly, by $9.7\times$ and over $2\times$, respectively. 

Similarly, Table~\ref{tab:deepseek_main_tab} shows that all attack types severely raise the number of reasoning tokens in the DeepSeek-R1 model. The baseline of $711\pm635$ tokens increase more than $10\times$ under the ICL-Genetic (Agnostic) attack. Other attacks, such as Context-Agnostic, Context-Aware, and ICL-Genetic (Aware), also lead to substantial increases in reasoning complexity. Overall, our results demonstrate that ICL-based attacks, especially those involving Context-Agnostic, severely disrupt reasoning efficiency for both models by drastically increasing reasoning token counts. This trend persists across all attack types. Similarly Tables~\ref{tab:Context_Agnostic_100samples_big_models} and~\ref{tab:Context_Agnostic_100samples_mini_models} show an increase in reasoning tokens across all four models tested on both the SQuAD and FreshQA datasets using the context-agnostic attack. We observe a $46\times$ increase in reasoning tokens for the SQuAD dataset on the o1 model. This highlights the effectiveness of our attack methodology across a diverse set of contexts and model families. Figure~\ref{fig:reasoning_content_example} in the Appendix gives an insight of how the decoy task causes increase in  reasoning steps for R1 model.

\pbe{ICL Ablation.}
Table~\ref{tab:o1_main_tab} and~\ref{tab:deepseek_main_tab} show that the results based on ICL outperform both context-agnostic and context-aware settings. In Table~\ref{tab:ablation_table}, we present an ablation study on ICL-Genetic with context-agnostic attack framework to evaluate the efficacy of each individual components and its contributions on crafting the final attack. It shows that while both ICL-Genetic and context-agnostic attacks independently have higher reasoning token count than baseline, both of them are lower than the attack  conducted by combining both techniques. We hypothesize that this occurs because the attack-agnostic samples used to generate the initial population allow the algorithm to narrow down the search space, thereby enabling it to take a more exploitative route in finding an effective injection.


\begin{table*}[ht!]
\centering  
\footnotesize
\begin{tcolorbox}[enhanced,breakable,
    colframe=gray!50!white,
    colbacktitle=white,
    coltitle=black,
    colback=white,
    borderline={0.5mm}{0mm}{gray!15!white},
    borderline={0.5mm}{0mm}{gray!50!white,dashed},
    attach boxed title to top center={yshift=-2mm},
    boxed title style={boxrule=0.8pt},
    title=\normalsize\textbf{Prompt for Filtering Correct Responses}]
    \renewcommand{\arraystretch}{1.3}
    \begin{tabular}{p{.95\linewidth}}
        \textbf{System}\\
        You are an assistant specializing in chemistry and biology. You are provided with a molecule's IUPAC name and its \{level\} description.\\
        Your task is to evaluate the factual accuracy of the given description based on the provided IUPAC name. \\
        Assign a score from 1 to 4 based on the following criteria: \\
        1: All contents are factually incorrect \\
        2: Some contents are factually correct, but most are factually incorrect \\
        3: Most contents are factually correct, but some are factually incorrect \\
        4: All contents are factually correct \\
        Indicate your score in the format: ``Score: ...''.\\
        \midrule
        \textbf{User}\\
        Input molecule (IUPAC name): \{IUPAC name\} \\
        Description: \{Description\}
    \end{tabular}
\end{tcolorbox}
\caption{Prompts for filtering correct samples. \{level\} is one of the following: `structural', `chemical', and `biological'.}
\label{app:tab:prompts_filtering}
\end{table*}




\begin{figure*}[ht!]
    \centering
    \begin{tcolorbox}[
        enhanced,                  %
        colframe=blue!70!black,    %
        colback=blue!5,            %
        coltitle=white,            %
        colbacktitle=blue!70!black, %
        width=\textwidth,          %
        arc=4mm,                   %
        boxrule=1mm,               %
        drop shadow,               %
        title=Copy-Editing Task Description, %
        fonttitle=\bfseries\large  %
    ]

    \textbf{Task Description:}\\%[1em]

    As an expert copy-editor, please rewrite the following text in your own voice while ensuring that the final output contains the same information as the original text and has roughly the same length. Please paraphrase all sentences and do not omit any crucial details.\\[1em]

    \textbf{Input Text:}\\[1em]
    \texttt{<Input Text Placeholder>}

    \end{tcolorbox}
    \caption{Prompt used by the RAG system to rewrite the input query.}
    \label{fig:copy_editing_task}
\end{figure*}


\pbe{Attack Transferability.}
We evaluate the transferability of \sys across o1, o1-mini, and DeepSeek-R1 models under the Context-Agnostic attack. Contexts optimized using the ICL-Genetic attack on a source model are applied to target models to assess transferability. The o1 model demonstrates strong transferability, achieving a 12$\times$ increase on DeepSeek-R1, exceeding the 10.5$\times$ increase from context optimized directly on DeepSeek-R1. Similarly, o1's transfer to o1-mini results in a 6.2$\times$ increase. DeepSeek-R1 also transfers effectively to o1 with an 11.4$\times$ increase but less so to o1-mini (4.4$\times$). In contrast, o1-mini shows moderate transferability with a 7.5$\times$ increase on DeepSeek-R1 and only 2.9$\times$ on o1. These findings demonstrate that context optimized from various source models can significantly increase reasoning tokens across different target models. 




\pbe{Reasoning Effort Tuning.} 
The o1 model API provides \textit{reasoning effort} hyperparameter that controls the size of thought in generating responses, with low effort yielding quick, simple answers and high effort producing more detailed explanations~\cite{openai_reasoning_effort, openai_reasoning_guide}. We use this parameter to evaluate our attack across different effort levels. Table~\ref{tab:effort_comparison} shows that the Context-Agnostic attack significantly increases reasoning tokens at all effort levels. For high effort, the token count rises over 12$\times$. Medium and low effort also show large increases, reaching up to 14$\times$. These results demonstrate that the attack disrupts the model's reasoning efficiency across tasks of varying complexity, with even low-effort tasks experiencing significant reasoning overhead.














 

    





\section{Conclusion and future directions} \label{sec:conclusion}

In this paper we proposed a nested MLMC framework that offers important computational savings by performing most calculations in low precision and exploiting approximate random normal variables for the low precision path calculations. The low precision calculations could be performed in fixed precision on an FPGA for greater efficiency, and we suggested a procedure to optimise the bit-widths of every variable at each Monte Carlo level. This is an important improvement over previous mixed precision MLMC frameworks which held the lower precision fixed \cite{Rounding_error_oliver} or defined uniform bit-width at every level heuristically \cite{brugger2014mixed}. Our numerical results suggest that for the first levels our procedure reduces the cost at these levels by a factor 5 or 7. Hence the overall savings are significant since most paths are calculated on the first levels. Our approach would be even more efficient for the Milstein scheme because its higher order strong convergence leads to a greater proportion of the computational costs being on the coarsest levels.

The next stage of the research project will be to implement the RNG methods and the nested framework on FPGAs to determine the hardware requirements and confirm the extent of the computational savings. It would also be good to compare the performance benefits to using half-precision floating point arithmetic on GPUs or CPUs for the low-accuracy computations.





{
\small
\bibliographystyle{unsrtnat}
\bibliography{main}
}


\section{Multi-Normalized Gradient Descent} \label{sec: mngd}
Before presenting our approach, let us first introduce some clarifying notations.

\paragraph{Notations.} For a vector $x\in\mathbb{R}^d$, we call its normalized projection w.r.t to a given norm $\Vert \cdot\Vert$, the solution to the following optimization problem:
\begin{align}
\label{eq-single-proj}
\mathcal{P}_{\Vert \cdot \Vert}(x):=\argmax_{z:~\Vert z \Vert = 1} \langle x, z\rangle    
\end{align}
We also extend the definition of this notation if $x\in\mathbb{R}^{m\times n}$ is a matrix and $\Vert \cdot \Vert$ is a matrix norm. 


\subsection{Gradient Multi-Normalization}

Let us now consider a finite family of $K\geq 1$ norms $(g_1,\dots,g_K)$. In order to pre-process the gradient $\nabla$ jointly according to these norms, we propose to consider the following optimization problem:
\begin{align}
\label{eq:multi-norm-opt}
 \argmax_{z} \langle \nabla, z\rangle~ \text{s.t.}~\forall~i\in [|1,K|],~g_i(z)=1\; .
\end{align}
Assuming the constraint set is non-empty, the existence of a maximum is guaranteed. However, this problem is NP-hard and non-convex due to the constraints, making it hard to solve efficiently for the general case of arbitrary norms.


\begin{algorithm}[!t]
   \caption{$\texttt{MultiNorm}(\nabla,L, \bm{g})$}
   \label{alg:alt-proj}
\begin{algorithmic}
   \STATE {\bfseries Input:} the stochastic gradient $\nabla_\theta\mathcal{L}(\theta_t,x^{(t)})$, the norms $\bm{g}:=(g_1,\dots,g_K)$, and $L\geq 1$ the number of iterations.
   \STATE Initialize $x=\nabla_\theta\mathcal{L}(\theta_t,x^{(t)})$.
   \FOR{$\ell=1$  {\bfseries to} $L$}
   \FOR{$i=1$ {\bfseries to} $K$}
   \STATE $x\gets \mathcal{P}_{g_i}(x):=\argmax\limits_{z:~g_i(z) = 1} \langle x, z\rangle $
   \ENDFOR
    \ENDFOR
    \STATE Return $x$
\end{algorithmic}
\end{algorithm}

\begin{remark}
Observe that when $K=1$, the problem~\eqref{eq:multi-norm-opt} recovers exactly the single normalization step used in~\cite{bernstein2024old}, as presented in~\eqref{eq:single-norm}.
\end{remark}

\begin{remark}
The convex relaxation of~\eqref{eq:multi-norm-opt}, defined as  
\begin{align}
\label{eq:multi-norm-opt-convex}
 \argmax_{z} \langle \nabla, z\rangle\quad \text{s.t.}~\forall~~i\in [|1,K|],~~g_i(z)\leq 1
\end{align}
is in fact equivalent to the single normalization case discussed in Section~\ref{sec:single-norm}, where the norm considered is $\Vert x\Vert:=\max\limits_{i\in[|1,K]} g_i(x)$. Thus, solving~\eqref{eq:multi-norm-opt-convex} is equivalent to computing the projection $\mathcal{P}_{\Vert \cdot \Vert}(\nabla)$. In Appendix~\ref{sec:convex-relaxation}, we provide a general approach to compute it using the so-called Chambolle-Pock algorithm~\cite{chambolle2011first}.
\end{remark}

While solving~\eqref{eq:multi-norm-opt} exactly might not be practically feasible in general, we propose a simple alternating projection scheme, 
presented in Algorithm~\ref{alg:alt-proj}. Notably, our method assumes that the projections $\mathcal{P}_{g_i}(\cdot)$  can be efficiently computed for all $i\in[|1,K|]$. Fortunately, when the $g_i$'s correspond to $\ell_p$-norms with $p\in[|1,+\infty|]$, or Schatten $p$-norms for matrices, closed-form solutions for these projections exist. See Appendix~\ref{sec:convex-relaxation} for more details.


\paragraph{SWAN: an Instance of $\texttt{MultiNorm}$.}  SWAN~\cite{ma2024swansgdnormalizationwhitening} applies two specific pre-processing steps to the raw gradients in order to update the weight matrices. In fact, each of these pre-processing steps can be seen as normalized projections with respect to a specific norm. More precisely, for $W\in\mathbb{R}^{m\times n}$ and $m\leq n$, let us define
\begin{align*}
g_1(W):=\frac{\max\limits_{i\in[|1,m|]} \Vert W_{i,:}\Vert_2}{\sqrt{n}}\; ,~ \text{and}~~ 
g_2(W):=\frac{\Vert W\Vert_{\sigma, \infty}}{\sqrt{n}}\; .
\end{align*}
where for $p\in [1,+\infty]$, $\Vert W\Vert_{\sigma,p}$ is the Schatten $p$-norm of $W$. Simple derivations leads to the following equalities:
\begin{align*}
    \mathcal{P}_{g_1}(W)&= \sqrt{n} Q(W)^{-1}W\\
    \mathcal{P}_{g_2}(W)&=\sqrt{n}(WW^\top)^{-1/2}W
\end{align*}
%
Therefore applying a single iteration ($L=1$) of Algorithm~\ref{alg:alt-proj} with norms $g_1$ and $g_2$ as defined above on the raw gradient $\nabla_t$ exactly leads to the SWAN update (Eq.~\eqref{eq:swan-update}).


\subsection{On the Convergence of \texttt{MultiNorm}}
We aim now at providing some theoretical guarantees on the convergence of  $\texttt{MultiNorm}$ (Algorithm~\ref{alg:alt-proj}). More precisely, following the SWAN implementation~\cite{ma2024swansgdnormalizationwhitening}, we focus on the specific case where $K=2$ and the normalized projections associated with the norms $g_1$ and $g_2$ have constant $\ell_2$-norm. More formally, we consider the following assumption.
\begin{assumption}
\label{assump-norm}
Let $g$ be a norm on $\mathbb{R}^d$. We say that it satisfies the assumption if for all $x\in\mathbb{R}^d$, $\Vert \mathcal{P}_{g}(x) \Vert_2 = c $ where $c>0$ is an arbitrary positive constant independent of $x$ and $\Vert\cdot\Vert_2$ represents the Euclidean norm.
\end{assumption}

\begin{remark}
Observe that both norms in SWAN satisfies Assumption~\ref{assump-norm} and their normalized projections have the same $\ell_2$-norm, as for any $W\in\mathbb{R}^{m\times n}$ with $m\leq n$, we have $\Vert \mathcal{P}_{g_1}(W) \Vert_2 = \Vert \mathcal{P}_{g_2}(W)\Vert_2 = \sqrt{nm}$.
\end{remark}


This assumption enables to obtain useful properties on $\mathcal{P}_{g}$ as we show in the following Lemma:
\begin{lemma}
\label{lem:properties-proj}
Let $g$ a norm satisfying Assumption~\ref{assump-norm}. Then
\begin{align*}
    \mathcal{P}_{g}\circ\mathcal{P}_{g} =\mathcal{P}_{g}
\end{align*}
and for all $x\in\mathbb{R}^d$, $g^*(\mathcal{P}_g(x))=\Vert \mathcal{P}_g(x)\Vert_2^2=c^2$, 
where $g^*$ is the dual norm associated with $g$.
\end{lemma}

Let us now introduce some additional notation to clearly state our result. Let $x_0\in\mathbb{R}^d$ and let us define for $n\geq 0$:
\begin{equation}
\begin{aligned}
\label{eq:seq}
    x_{2n+1}&:=\mathcal{P}_{g_1}(x_{2n})\\
    x_{2n+2}&:= \mathcal{P}_{g_2}(x_{2n+1})
    \end{aligned}
\end{equation}

which is exactly the sequence generated by Algorithm~\ref{alg:alt-proj} when $K=2$ and $x_0=\nabla_\theta\mathcal{L}(\theta_t,x^{(t)})$. Let us now show our main theoretical result, presented in the following Theorem.
\begin{theorem}
\label{thm:cvg}
Let $g_1$ and $g_2$ two norms on $\mathbb{R}^d$ satisfying Assumption~\ref{assump-norm} and such that their normalized projections have the same $\ell_2$ norm. Let also $(x_n)_{n_\geq 0}$ be defined as in~\eqref{eq:seq} and let us define the set of fixed-point as:
\begin{align*}
    \mathcal{F}:=\{x:~\mathcal{P}_{g_1}(x)=\mathcal{P}_{g_2}(x)=x\}
\end{align*}
Then by denoting $d(x,\mathcal{F}):=\min\limits_{z\in\mathcal{F}}\Vert x - z\Vert_2$ we have 
\begin{align*}
d(x_n,\mathcal{F}) \xrightarrow[n\to\infty]{} 0\; .
\end{align*}
% the sequence $(\Vert x_{n+1} - x_{n}\Vert_2)_{n\geq 0}$ monotonically decreases towards $0$ and \begin{align*}
%     g_1(x_n)\xrightarrow[n\to\infty]{} 1,~\text{and}~
%     g_2(x_n)\xrightarrow[n\to\infty]{} 1\; .
% \end{align*}
% Additionally, any cluster point $x$ of $(x_{n})_{n\geq 0}$ verifies:
% \begin{align*}
%     \mathcal{P}_{g_1}(x)=\mathcal{P}_{g_2}(x)=x\; .
% \end{align*}
\end{theorem}

This Theorem states that if $\texttt{MultiNorm}$ runs for a sufficient amount of time, then the returned point $x$ can be arbitrarily close to a fixed-point solution. While we cannot guarantee that it solves~\eqref{eq:multi-norm-opt}, we can assert that our algorithm converges to a fixed-point solution with arbitrary precision, and as a by-product produces a solution $x$ normalized w.r.t both norms $g_1$, $g_2$ (up to an arbitrary precision).

\begin{remark}
Note that in Theorem~\ref{thm:cvg} we assume that the normalized projections associated to $g_1$ and $g_2$ have the same $\ell_2$-norms. However, given two norms $g_1$ and $g_2$ satisfying Assumption~\ref{assump-norm}, i.e. such that for all $x$:
\begin{align*}
    \Vert \mathcal{P}_{g_1}(x) \Vert_2 &= c_1\\
    \Vert \mathcal{P}_{g_2}(x) \Vert_2 &= c_2
\end{align*}
for some $c_1,c_2>0$, and given a target value $a>0$, one can always rescale the norms such that their normalized projections have the same $\ell_2$ norm equal to $a$. More formally, by denoting $\tilde{g_1} = \frac{c_1}{a} g_1$ and $\tilde{g_2} = \frac{c_2}{a} g_2$, we obtain that
\begin{align*}
    \Vert \mathcal{P}_{\tilde{g}_1}(x)\Vert_2 =  \Vert \mathcal{P}_{\tilde{g}_2}(x)\Vert_2 = a .
\end{align*}
\end{remark}




\begin{remark}
It is worth noting that, for squared matrices ($m=n$), a single iteration ($L=1$) of \texttt{MultiNorm} using the norms considered in~\cite{ma2024swansgdnormalizationwhitening}, immediately converges to a fixed-point---precisely recovering SWAN.
\end{remark}



\subsection{MNGD: a New Family of Stateless Optimizers.} 
% We are now ready to present our family optimizers, called the \emph{Multi-Normalized Gradient Descents} (MNGDs), which is detailed in Algorithm~\ref{alg:multi-normalized-gd}. The main difference with the framework proposed in~\cite{bernstein2024old}, is that here we enable the normalization of the gradient w.r.t multiple norms thanks to the $\texttt{MultiNorm}$ step, while in~\cite{bernstein2024old}, the gradient is normalized according to a single norm as presented in~\eqref{eq:single-norm}.


We now introduce our family of optimizers: \emph{Multi-Normalized Gradient Descents} (MNGDs) (Algorithm~\ref{alg:multi-normalized-gd}). The key distinction from the framework proposed in~\cite{bernstein2024old} is that MNGDs normalize the gradient with respect to multiple norms using the 
$\texttt{MultiNorm}$ step, whereas in~\cite{bernstein2024old}, the gradient is normalized using a single norm, as shown in~\eqref{eq:single-norm}.
\begin{algorithm}[!t]
   \caption{Multi-Normalized GD ($\texttt{MNGD}$)}
   \label{alg:multi-normalized-gd}
\begin{algorithmic}
   \STATE {\bfseries Input:} $T\geq 1$ the number of updates, $(\eta_t)_{0\leq t\leq T}$ the global step-sizes, $\mathcal{L}$ the loss to minimize, $L\geq 1$ the number of iterations for the multi-normalization, and $\bm{g}:=(g_1,\dots,g_K)$ the norms.
   \STATE Initialize $\theta_0$
   \FOR{$t=1$  {\bfseries to} $T$}
    \STATE $\nabla_t\gets \nabla_{\theta}\mathcal{L}(\theta_t, x^{(t)})$ with $x^{(t)}\sim P_x$
    \STATE $\hat{\nabla}_t \gets \texttt{MultiNorm}(\nabla_t,L, \bm{g})$ as defined in Alg.~\ref{alg:alt-proj}.
    \STATE $\theta_{t+1} \gets \theta_t - \eta_t \hat{\nabla}_t$
    \ENDFOR
    \STATE Return $x$
\end{algorithmic}
\end{algorithm}

In the following, we focus on the MNGD scheme with a specific choice of norms, for which we can efficiently compute the gradient multi-normalization step. This enables the application of stateless optimizers to large LMs.

\section{Sinkhorn: a Multi-Normalization Procedure}
As in SWAN~\cite{ma2024swansgdnormalizationwhitening}, we propose to normalize the weight matrices according to multiple norms. We still leverage the row-wise $\ell_2$-norm to pre-process raw gradients, however, rather than using the spectral norm, we propose to consider instead a relaxed form of this constraint and use the column-wise $\ell_2$-norm. More formally, let us consider the two following norms on matrices of size $\mathbb{R}^{m\times n}$:
\begin{align*}
g_1(W):=\frac{\max\limits_{i\in[|1,m|]} \Vert W_{i,:}\Vert_2}{\sqrt{n}}\; ,\quad 
g_2(W):=\frac{\max\limits_{j\in[|1,n|]} \Vert W_{:,j}\Vert_2}{\sqrt{m}}\; ,
\end{align*}
which leads to the following two normalized projections:
\begin{align*}
    \mathcal{P}_{g_1}(W)&= \sqrt{n} Q(W)^{-1}W\\
    \mathcal{P}_{g_2}(W)&=\sqrt{m}W R(W)^{-1}
\end{align*}
where $R(W):=\text{Diag}(\Vert W_{:,1}\Vert_2,\dots,\Vert W_{:,n}\Vert_2)\in\mathbb{R}^{n\times n} $ is the diagonal matrix of size $n$ with the $\ell_2$-norm of the columns of $W$ as diagonal coefficients. For such a choice of norms, the $\texttt{MultiNorm}$ reduces to a simple procedure as presented in Algorithm~\ref{alg:Sinkhorn}.


\begin{remark}
For such a choice of norms, we obtain $\Vert \mathcal{P}_{g_1}(W) \Vert_2 = \Vert \mathcal{P}_{g_2}(W)\Vert_2 = \sqrt{nm}$ for any $W\in\mathbb{R}^{m\times n}$. In other words, both norms satisfy Assumption~\ref{assump-norm} and their $\ell_2$ norms are equal to $\sqrt{nm}$.
\end{remark}


For completeness we include  the MNGD scheme (Algorithm~\ref{alg:multi-normalized-sinkhorn}) that replaces the $\texttt{MultiNorm}$ step with $\texttt{SR-Sinkhorn}$ (Algorithm~\ref{alg:Sinkhorn}).


\begin{algorithm}[!t]
   \caption{$\texttt{SR-Sinkhorn}(\nabla,L)$}
   \label{alg:Sinkhorn}
\begin{algorithmic}
   \STATE {\bfseries Input:} the stochastic gradient $\nabla_W\mathcal{L}(W_t,x^{(t)})$, and $L\geq 1$ the number of iterations.
   \STATE Initialize $X=\nabla_W\mathcal{L}(W_t,x^{(t)})\in\mathbb{R}^{m\times n}$.
   \FOR{$\ell=1$  {\bfseries to} $L$}
   \STATE $X\gets  \sqrt{n} Q(X)^{-1}X$
   \STATE $X\gets  \sqrt{m} XR(X)^{-1}$
   \ENDFOR
    \STATE Return $X$
\end{algorithmic}
\end{algorithm}


\paragraph{The Sinkhorn Algorithm.} Before explicitly showing the link between Algorithm~\ref{alg:Sinkhorn} and the Sinkhorn algorithm, let us first recall the Sinkhorn theorem~\cite{sinkhorn1964relationship} and the Sinkhorn algorithm~\cite{sinkhorn1967concerning}. Given a positive coordinate-wise matrix $A\in\mathbb{R}_{+}^{m\times n}$, there exists a unique matrix $P\in\mathbb{R}_{+}^{m\times n}$ of the form $P=QAR$ with $Q$ and $R$ positive coordinate-wise and diagonal matrices of size $m$ and $n$ respectively, such that $P\bm{1}_n=n\bm{1}_m$ and $P^\top\bm{1}_m=m\bm{1}_n$. To find $P$, one can use the Sinkhorn algorithm that initializes $P_0:=A$ and computes for $k\geq 0$:
\begin{align*}
    P_{k+1/2}&=n\text{Diag}(P_k\bm{1}_n)^{-1}P_k\\
    P_{k+1}&=m P_{k+1/2}\text{Diag}(P_{k+1/2}^\top\bm{1}_m)^{-1}\; .
\end{align*}
Equivalently, these updates on $P$ can be directly expressed as updates on the diagonal coefficients of $Q=\text{Diag}(u)$ and $R=\text{Diag}(v)$ with $u\in\mathbb{R}_{+}^m$ and $v\in\mathbb{R}_{+}^n$. By initializing $u_0=\bm{1}_m$ an $v_0=\bm{1}_m$, the above updates can be reformulated as follows:
\begin{align}
\label{eq:update-diag-sin}
    u_{k+1} = n\frac{ \bm{1}_m}{Av_k},~~v_{k+1} = m\frac{\bm{1}_n}{ A^\top u_{k+1}}
\end{align}
where $/$ denote the coordinate-wise division.~\citet{franklin1989scaling} show the linear convergence of Sinkhorn’s iterations. More formally, they show that $(u_k, v_k)$ converges to some $(u^*,v^*)$ such that $P:=\text{Diag}(u^*)A\text{Diag}(v^*)$ satisfies $P\bm{1}_n=n\bm{1}_m$ and $P^\top\bm{1}_m=m\bm{1}_n$, and:
\begin{align*}
    d_\mathcal{H}(u_k, u^*)\in\mathcal{O}(\lambda(A)^{2k})~, \text{ and } ~d_\mathcal{H}(v_k, v^*)\in\mathcal{O}(\lambda(A)^{2k})\; ,
\end{align*}
where $d_{\mathcal{H}}$ is the Hilbert projective metric~\cite{de1993hilbert} and $\lambda(A)<1$ is a contraction factor associated with the matrix $A$.


\paragraph{Links between Sinkhorn and Algorithm~\ref{alg:Sinkhorn}.} Algorithm~\ref{alg:Sinkhorn} can be seen as a simple reparameterization of the updates presented in~\eqref{eq:update-diag-sin}. More precisely, given a gradient $\nabla\in\mathbb{R}^{m\times n}$ and denoting $A:=\nabla^{\odot 2}$, we obtain that the iterations of Algorithm~\ref{alg:Sinkhorn} exactly compute:
\begin{align}
\label{eq:update-diag-sin-sr}
    u_{k+1}^{1/2} = \sqrt{n\frac{ \bm{1}_m}{Av_k}},~~v_{k+1}^{1/2} = \sqrt{m\frac{\bm{1}_n}{ A^\top u_{k+1}}}
\end{align}
where the square-root is applied coordinate-wise, and returns after $L$ iterations $X_L=\text{Diag}(u_{L}^{1/2})\nabla \text{Diag}(v_{L}^{1/2})$. Therefore the linear convergence of Algorithm~\ref{alg:Sinkhorn} follows directly from the convergence rate of Sinkhorn, and Algorithm~\ref{alg:Sinkhorn} can be thought as applying the square-root Sinkhorn algorithm, thus the name $\texttt{SR-Sinhkorn}$. Note also that at convergence ($L\to+\infty$) we obtain $X^{*}\in\mathbb{R}^{m\times n}$ which is a fixed-point of both normalized projections, that is $\mathcal{P}_{g_1}(X^*)=\mathcal{P}_{g_2}(X^*)=X^*$,
from which we deduce that
\begin{align*}
\Vert X^*_{i,:}\Vert_2 = \sqrt{n}\;, \quad \text{and}\quad   \Vert X^*_{:,j}\Vert_2 = \sqrt{m}\;
\end{align*}
as demonstrated in Theorem~\ref{thm:cvg}.



\textbf{On the Importance of the Scaling.} Now that we have shown the convergence $\texttt{SR-Sinkhorn}$, let us explain in more detail the scaling considered for both the row-wise and column-wise normalizations. First recall that both norm $g_1$ and $g_2$ satisfy Assumption~\ref{assump-norm} and that the $\ell_2$ norm of their normalized projections is equal to $\sqrt{nm}$. The reason for this specific choice of scaling ($\sqrt{nm}$) is due to the global step-size in Algorithm~\ref{alg:multi-normalized-sinkhorn}. In our proposed MNGD, we did not prescribe how to select $\eta_t$. In practice, we aim to leverage the same global step-sizes as those used in Adam~\cite{adam} for training LLMs, and therefore we need to globally rescale the (pre-processed) gradient accordingly. To achieve that, observe that when EMAs are turned-off, Adam corresponds to a simple signed gradient descent, and therefore the Frobenius norm of the pre-processed gradient is simply $\sqrt{nm}$. Thus, when normalizing either the rows or the columns, we only need to rescale the normalized gradient accordingly.


\begin{algorithm}
   \caption{Sinkhorn GD ($\texttt{SinkGD}$)}
   \label{alg:multi-normalized-sinkhorn}
\begin{algorithmic}
   \STATE {\bfseries Input:} $T\geq 1$ the number of updates, $(\eta_t)_{0\leq t\leq T}$ the global step-sizes, $\mathcal{L}$ the loss to minimize, and $L\geq 1$ the number of iterations for the SR-Sinkhorn procedure.
   \STATE Initialize $\theta_0$
   \FOR{$t=1$  {\bfseries to} $T$}
    \STATE $\nabla_t\gets \nabla_{\theta}\mathcal{L}(\theta_t, x^{(t)})$ with $x^{(t)}\sim P_x$
    \STATE $\hat{\nabla}_t \gets \texttt{SR-Sinkhorn}(\nabla_t,L)$ as defined in Alg.~\ref{alg:Sinkhorn}.
    \STATE $\theta_{t+1} \gets \theta_t - \eta_t \hat{\nabla}_t$
    \ENDFOR
    \STATE Return $x$
\end{algorithmic}
\end{algorithm}
\vspace{-0.2cm}

\textbf{Computational Efficiency of SinkGD over SWAN.} Compared to SWAN~\cite{ma2024swansgdnormalizationwhitening}, the proposed approach,  \texttt{SinkGD}, is more efficient as it only requires $\mathcal{O}(nm)$ numerical operations. In contrast, SWAN, even when implemented with Newton-Schulz, still requires performing matrix-matrix multiplications, which have a time complexity of $\mathcal{O}(m^2(m+n))$. In the next section, we will demonstrate the practical effectiveness of MNGD with $\texttt{SR-Sinkhorn}$, that is \texttt{SinkGD}. This approach manages to be on par with, and even outperforms,  memory-efficient baselines for pretraining the family of LLaMA models up to 1B scale.











\end{document}