% authors
\newcommand{\todo}[1]{\textcolor{red}{TODO:\{#1\}}}
\newcommand{\zhibin}[1]{\textcolor{violet}{zhibin: #1}}

% shorthands
\usepackage{xspace}
\newcommand{\ie}{\emph{i.e.,}\xspace}
\newcommand{\eg}{\emph{e.g.,}\xspace}
\newcommand{\etc}{\emph{etc}\xspace}
\newcommand{\etal}{\emph{et al.}\xspace}

% names
% \newcommand{\modelfull}[1]{\textsc{\textbf{T}oken L\textbf{o}ss \textbf{T}rimming}}
\newcommand{\model}[1]{\textsc{Rho-1}} % TODO
\newcommand{\method}[1]{{SLM}} % TODO
\newcommand{\methodfull}[1]{{Selective Language Modeling}} % TODO

% corpus
\newcommand{\mamix}[1]{\textsc{MaMix}}
\newcommand{\genmix}[1]{\textsc{GenMix}}


% \setcounter{footnote}{1}
\makeatletter  
\renewcommand\@fnsymbol[1]{%  
  \ensuremath{%  
    \ifcase#1\or % 0 - no symbol  
    \star\or     % 1 - star  
    \diamond\or   % 2 - diamand
    \dagger\or  % 3 - dagger  
    \mathsection\or % 4 - section symbol  
    \mathparagraph\or % 5 - paragraph symbol  
    \|\or        % 6 - vertical line  
    **\or        % 7 - two stars  
    \dagger\dagger\or % 8 - two daggers  
    \ddagger\ddagger % 9 - two double daggers  
    \else \@ctrerr \fi%  
  }%  
}  
\makeatother

% colors
\definecolor{darkblue}{rgb}{0, 0, 0.5}
\hypersetup{colorlinks=true, citecolor=darkblue, linkcolor=darkblue, urlcolor=darkblue}
\definecolor{lightgray}{rgb}{0.9, 0.9, 0.9}

\definecolor{darkgreen}{RGB}{50,100,0}
\definecolor{darkred}{RGB}{200, 0, 0}
\definecolor{lightred}{RGB}{250, 200, 200}
\definecolor{lightblue}{RGB}{210, 220, 250}
\definecolor{doderblue}{RGB}{30,144,255}
\definecolor{select}{RGB}{222, 235, 247}
\definecolor{unselect}{RGB}{247, 207, 206}

\newcommand{\blue}{\cellcolor{lightblue}}
\newcommand{\red}{\cellcolor{lightred}}
\newcommand{\grey}{\cellcolor{lightgray}}
\newcommand{\hl}[1]{\textcolor{purple}{#1}}
% \newcommand{\hlb}[1]{\textcolor{cyan}{#1}}
\newcommand{\hlb}[1]{\textcolor{doderblue}{#1}}

\newcommand{\cmark}{\textcolor{darkgreen}{\ding{51}}} % ✔
\newcommand{\xmark}{\textcolor{darkred}{\ding{55}}} % ✘

% ref
% \usepackage[noabbrev,capitalize]{cleveref} % http://tug.ctan.org/tex-archive/macros/latex/contrib/cleveref/cleveref.pdf
% \newcommand{\crefrangeconjunction}{--}

\usepackage{cleveref}
\makeatletter
\AtBeginDocument{%
  \renewcommand{\sectionautorefname}{\S\@gobble}
  \renewcommand{\tableautorefname}{Table}
  \renewcommand{\equationautorefname}{Equation}
  \renewcommand{\subsectionautorefname}{\S\@gobble}  
}
\makeatother

% \crefname{equation}{equation}{equations}   % "equation" is lowercased, overriding capitalize option
% \crefname{footnote}{footnote}{footnotes}   % "footnote" is lowercased, overriding capitalize option
% \crefname{line}{line}{lines}               % "line" is lowercased, overriding capitalize option
% In some cases for equations, you will want to still use \eqref{eq:foo} rather than \cref{eq:foo}, because you only want "(3)" rather than "equation (3)".  This is shorter, and is also useful when you want to write something like "expression (3)" or "the KL-divergence (3)."  Judge this on a case-by-case basis.\crefname{footnote}{footnote}{footnotes}   % "footnote" is lowercased, overriding capitalize option
% \crefname{section}{\S}{\S\S}
% \Crefname{section}{\S}{\S\S}    % must define start-of-sentence version explicitly since \S isn't a letter
% \crefformat{section}{#2\S#1#3}  % remove space between section symbol and the number, and include \S in the hyperlink
% \Crefformat{section}{#2\S#1#3}
% \crefrangeformat{section}{\S\S#3#1#4--#5#2#6}
% \Crefrangeformat{section}{\S\S#3#1#4--#5#2#6}