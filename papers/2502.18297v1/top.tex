\documentclass[9pt, conference]{IEEEtran}
\IEEEoverridecommandlockouts
\DeclareUnicodeCharacter{2217}{*}
% The preceding line is only needed to identify funding in the first footnote. If that is unneeded, please comment it out.
\usepackage{cite}
\usepackage{amsmath,amssymb,amsfonts}
\usepackage{algorithmic}
\usepackage{booktabs}
\usepackage{graphicx}
\usepackage{textcomp}
\usepackage{xcolor}
\usepackage{multirow}
\usepackage{algorithm}
\usepackage{multirow}
\usepackage{amsmath}
\usepackage{hyperref}
\usepackage{balance}
\def\BibTeX{{\rm B\kern-.05em{\sc i\kern-.025em b}\kern-.08em
    T\kern-.1667em\lower.7ex\hbox{E}\kern-.125emX}}

% AI2.3    Infrastructures for AI (including datasets, implementations)
% 

    
\begin{document}


\title{DeepCircuitX: A Comprehensive Repository-Level Dataset for RTL Code Understanding, Generation, and PPA Analysis}

% \author{
% {Zeju Li}$^{1\dagger}$ \quad Changran Xu$^{1\dagger}$ \quad Zhengyuan Shi$^{1\dagger}$ \quad Zedong Peng$^2$ \quad Yi Liu$^1$ \quad Yunhao Zhou$^2$ \\ 
% {Lingfeng Zhou}$^3$ \quad {Chengyu Ma}$^4$ \quad {Jianyuan Zhong}$^1$ \quad {Xi Wang}$^5$ \quad {Jieru Zhao}$^2$ \quad {Zhufei Chu}$^4$ \\ 
% {Xiaoyan Yang}$^3$ \quad {Qiang Xu}$^{1*}$\thanks{$\dagger$ Equal Contribution, *Corresponding author, qxu@cse.cuhk.edu.hk.} \\[6pt]
% $^1$The Chinese University of Hong Kong \quad $^2$Shanghai Jiao Tong University \\[3pt]
% $^3$Hangzhou Dianzi University \quad $^4$Ningbo University \quad $^5$Southeast University
% }


\author{
	\IEEEauthorblockN{
        Zeju Li $^{1,6\dagger}$, 
        Changran Xu $^{1,6\dagger}$, 
        Zhengyuan Shi $^{1,6\dagger}$, 
        Zedong Peng $^{2,6}$, 
        Yi Liu $^{1,6}$, 
        Yunhao Zhou $^{1,6}$, 
        Lingfeng Zhou $^{3,6}$, \\
        Chengyu Ma $^{4,6}$, 
        Jianyuan Zhong $^{1,6}$, 
        Xi Wang $^{5,6}$, 
        Jieru Zhao $^{2}$, 
        Zhufei Chu $^{4}$, 
        Xiaoyan Yang $^{3}$, 
        Qiang Xu $^{1,6}$ \thanks{$\dagger$ Equal Contribution} \thanks{Corresponding author: Qiang Xu (qxu@cse.cuhk.edu.hk)}} 

\IEEEauthorblockA{$^1$\textit{Department of Computer Science and Engineering}, \textit{The Chinese University of Hong Kong}, Sha Tin, Hong Kong S.A.R.\\}
\IEEEauthorblockA{$^2$ \textit{Department of Computer Science and Engineering}, Shanghai, China \\}
\IEEEauthorblockA{$^3$\textit{School of Computer Science}, \textit{Hangzhou Dianzi University}, Hangzhou, China \\}
\IEEEauthorblockA{$^4$\textit{Faculty of Electrical Engineering and Computer Science}, \textit{Ningbo University}, Ningbo, China \\}
\IEEEauthorblockA{$^5$ \textit{School of Integrated Circuit, Southeast University}, Nanjing, China \\}
\IEEEauthorblockA{$^6$\textit{National Center of Technology Innovation for EDA}, Nanjing, China \\}
} 

\maketitle

\begin{abstract}
This paper introduces DeepCircuitX, a comprehensive repository-level dataset designed to advance RTL (Register Transfer Level) code understanding, generation, and power-performance-area (PPA) analysis. Unlike existing datasets that are limited to either file-level RTL code or physical layout data, DeepCircuitX provides a holistic, multilevel resource that spans repository, file, module, and block-level RTL code. This structure enables more nuanced training and evaluation of large language models (LLMs) for RTL-specific tasks. DeepCircuitX is enriched with Chain of Thought (CoT) annotations, offering detailed descriptions of functionality and structure at multiple levels. These annotations enhance its utility for a wide range of tasks, including RTL code understanding, generation, and completion. Additionally, the dataset includes synthesized netlists and PPA metrics, facilitating early-stage design exploration and enabling accurate PPA prediction directly from RTL code. We demonstrate the dataset's effectiveness on various LLMs finetuned with our dataset and confirm the quality with human evaluations. Our results highlight DeepCircuitX as a critical resource for advancing RTL-focused machine learning applications in hardware design automation. Our data is available at \url{https://zeju.gitbook.io/lcm-team}.

\end{abstract}

\begin{figure*}[!t]
    \centering
    \includegraphics[width=0.8\linewidth]{fig/overview.pdf}
    \caption{Pipeline overview of the proposed framework, illustrating the key stages: data collection from GitHub using keywords, data annotation via chain-of-thought (COT), circuit transformation, and evaluation, including RTL code tasks for LLM and PPA prediction.}
    \label{fig:overview}
\end{figure*}

\begin{figure}[ht]
    \centering
    \includegraphics[width=0.8\linewidth]{graphs/greater_than_naive.pdf}
    \vspace{0.5cm}
    \includegraphics[width=0.8\linewidth]{graphs/p1_bottom.png}
    \vspace{-5pt}
    \caption{\textcolor{positional}{Positional} vs.\ \textcolor{nonpositional}{non-positional} circuits. In a \textcolor{nonpositional}{non-positional} circuit, the same edges must be included at all positions. A \textcolor{positional}{positional} circuit can distinguish between the same edge at different positions. This specificity yields better trade-offs between circuit size and faithfulness. It can also increase both precision and recall.}
    \label{fig:p1}
    \vspace{-5pt}
\end{figure}

\section{Introduction}

\looseness=-1
A primary goal of interpretability research is to characterize the internal mechanisms in language models (LMs) and other NLP models. 
A core approach in this area is \textbf{circuit discovery}---identifying the minimal subgraph within the model's computation graph that performs a specific task \citep{olah2021framework,olah-mech}.
Typically, the nodes of a circuit represent model components (e.g., attention heads, neurons, or layers).
While manual circuit discovery methods can yield position-specific insights \citep{wanginterpretability,goldowskydill2023localizingmodelbehaviorpath}, \emph{automatic methods often overlook positional information}, treating components as uniformly relevant across all input token positions \citep{conmytowards,syed2023attribution}. 
For instance, if an attention head is included in a circuit, it is assumed to contribute equally to the computation for every position in the input sequence.
The assumption that circuits are position-invariant ignores the fact that different positions often require distinct computations.
By ignoring positions, current methods limit their ability to capture mechanisms that operate across positions, such as interactions between attention heads across positions.

In this study, we start by demonstrating that positional agnosticism is a significant limitation (\S\ref{sec:motivating}). Then, to address these limitations, we introduce a new approach: position-aware edge attribution patching (PEAP; \S\ref{sec:full_circ_discovery}; Figure~\ref{fig:p1}). Current approaches  assume that if an edge is in a circuit, then the same edge will be in the circuit at all positions, thus leading to low precision. It is also assumed that an edge's importance should be aggregated across positions before deciding whether it should be included in the circuit; this can lead to cancellation effects, and thus low recall. PEAP instead allows us to compute the importance of cross-positional edges, and separately evaluates edge importance at each position. We show that this leads to smaller and more accurate circuits; see Figure~\ref{fig:p1}.

Incorporating positional information into circuit discovery is straightforward when inputs have the same length and structure across examples.

However, realistic datasets are not nearly this templatic.
How, then, can we incorporate positional information into automatic circuit discovery?
To address this challenge, we propose \textbf{schemas} (\S\ref{sec:schema}). 
Schemas assign semantic labels to spans of tokens, enabling information aggregation across examples even when the spans differ in length.

For example, in the input ``The \textcolor{positional}{war} lasted from 1453 to 14\underline{\hspace{1em}},'' the span ``\textcolor{positional}{war}'' could be labeled as ``\emph{Subject}''.
This enables handling spans with varying lengths: the phrase ``\textcolor{positional}{Black Plague}'' in another example can be treated as a single positional span with the same role as ``\textcolor{positional}{war}''.
In experiments with two LMs and three tasks, we find that circuits discovered using schemas achieve a better trade-off between circuit size and faithfulness to the model's behavior than position-agnostic circuits.
Importantly, position-aware circuits offer a more precise representation of the underlying mechanisms, providing a more concise foundation for mechanistic explanations.

We also present a fully automated pipeline for schema generation and application (\S\ref{sec:schema-generation}) using large language models (LLMs). 
We evaluate the quality of the generated schemas and their utility in discovering position-aware circuits (\S\ref{sec:schema-eval}).
Notably, circuits derived using automatically generated and applied schemas achieve comparable faithfulness scores to circuits discovered with human-designed and manually applied schemas.

We summarize our contributions as follows:
\begin{itemize}[noitemsep,leftmargin=*,topsep=1pt,parsep=1pt]
    \item Introduce a position-aware circuit discovery method, which obtains better faithfulness than position-agnostic discovery.  
    \item Introduce dataset schemas,  facilitating positional circuit discovery in more naturalistic settings. 
    \item Develop an automated schema generation and application pipeline with LLMs, yielding schemas that are comparable to manually-annotated ones.
\end{itemize}

\section{Related Work}
\label{sec:related}
\textbf{Open-Set Understanding:}
The open-set task, first introduced by Scheirer et al. \cite{scheirer2012toward}, challenges the conventional closed-set paradigm commonly assumed in image recognition. In closed-set models, the testing phase only includes samples from a predefined set of classes known to the model during training. Conversely, open-set recognition addresses scenarios where samples can belong to previously unknown classes that were not present during training. This requires models to both recognize and reject instances from unfamiliar classes, ensuring robustness against unknown inputs. The open-set framework has seen extensive study across multiple areas in computer vision, such as image classification~\cite{bendale2016towards, vaze2022open, yoshihashi2019classification, oza2019c2ae, perera2020generative, chen2021adversarial,zhang2020hybrid}, %
object detection~\cite{han2022expanding, miller2020uncertainty, zhou2023open}, %
and image segmentation~\cite{hwang2021exemplar, pham2018bayesian, cen2021deep}.%


\textbf{Open-vocabulary Semantic Segmentation (OVSS):}
Recent advances in vision-language models (VLMs) such as CLIP~\cite{radford2021learning} have demonstrated that robust, transferable visual representations can be effectively learned from large-scale datasets using only weakly structured natural language descriptions.
Initially adopted in image-level tasks like classification, VLMs leverage both visual and textual embeddings to recognize a diverse set of classes. %
By aligning image features with semantic concepts in a shared space, VLMs achieve a form of zero-shot learning that allows them to identify new classes at test time, offering a flexible framework for generalization \cite{liu2024open,xie2024sed,cho2024cat}. 
Although open-vocabulary learning presents an appealing solution for handling arbitrary classes, scaling this approach to accommodate an ever-growing set of classes poses significant challenges. In theory, a VLM could achieve perfect generalization if its query set contained every conceivable class label. However, as demonstrated by Miller et al. \cite{miller2025open}, adding more classes to the query set does not lead to better performance. In fact, increasing the number of class labels introduces a greater likelihood of misclassifications, leading to degraded model accuracy. This degradation occurs because, as more classes are added, the semantic space becomes increasingly crowded, causing overlaps that make accurate distinctions between classes harder to achieve. 
To tackle scalability, one solution is training class-free models \cite{shin2024towards}, while distinguishability can be improved by enhancing the textual descriptions of the classes \cite{ma2024open,jiao2024collaborative}. However, all these approaches assume that the inference label set is predefined and available at inference time.%




\textbf{Vocabulary-Free Semantic Segmentation (VSS):} Recent research in VSS has focused on developing end-to-end solutions while reducing bias from  ground truth data annotations. The majority of current methods decompose the task into a class-agnostic mask generation and a class association (Mask2Tag). Zero-Seg \cite{rewatbowornwong2023zero} and TAG \cite{kawano2024tag} leverage DINO \cite{caron2021emerging} to generate the masks, followed by CLIP-based \cite{radford2021learning} embedding generation for class association. Zero-Seg processes these embeddings %
following ZeroCap \cite{tewel2021zero} to obtain textual classes, while TAG matches them against an external database. 
Conversely, CaSED \cite{conti2024vocabulary} identifies potential classes by querying an external caption database using a pre-trained VLM. %
Similarly, Auto-Seg \cite{ulger2024autovocabularysemanticsegmentation} %
fine-tunes a captioning model to extract class names at multiple scales, followed by a second stage where an open-vocabulary model generates segmentation masks, with predictions remapped to ground truth classes using LLMs. While these approaches demonstrate promising results, they do not fully explore how this pipeline decomposition impacts model performance, nor do they investigate methods to enrich the textual information in the CLIP encoder. We address these limitations by providing a comprehensive analysis of the text encoder's role and exploring techniques to enhance visual-language understanding through richer textual representations. Moreover, we rigorously test the sensitivity of CLIP to tagger errors, evaluating how inaccuracies in image tagging propagate and impact the final segmentation performance.

\label{sec:method}

In this section, we introduce the method used to conduct the investigation on a set of \pc papers that discuss relevant bias issues.
Specifically, to construct the initial set of relevant work, we search the keywords ``bias" or ``fair" in the title of papers from NeurIPS, ICML, ICLR and FAccT published before February 2025. 
We include papers that discuss bias issues whose manifestation aligns with either Type I Bias or Type II Bias (we will detail the unification in~\cref{sec:unifying}).
We exclude papers that address other bias issues such as inductive bias~\cite{baxter2000model,zietlow2021demystifying}, implicit bias~\cite{fitzgerald2017implicit,camuto2021asymmetric}, selection bias~\cite{hernan2004structural,akbari2021recursive}, sampling bias~\cite{winship1992models,xu2022alleviating}, spectral bias~\cite{fang2024addressing}, exposure bias~\cite{li2024alleviating} or bias-variance~\cite{ha2024fine, chen2024on}.
Furthermore, to ensure we do not overlook any relevant papers without these keywords or from other prominent conferences such as CVPR, ICCV, and ECCV, we manually traversal the citation graph of the paper in the initial set and append the relevant papers that are either cited by or cite the papers in the initial set.






Once we identify the scope of the investigated papers, we read these papers to determine which type of bias they address by examining two aspects: problem statement and evaluation protocol.
We will elaborate on the criterion for categorizing papers into our definitions in~\cref{sec:unifying}.
To accommodate the recent emerging direction of addressing unlabeled and unknown bias, we enrich the taxonomy with an additional dimension about the status of attribute $A$.
As shown in~\cref{tab:taxonomy}, we count the number of papers in each category. 
Note that the total number is not equal to \pc since some papers address both types of biases.
We present the categorization list of all \pc investigated papers in Appendix.


\begin{table}[htbp]
\caption{The taxonomy of bias issues based on \pc papers.}
\label{tab:taxonomy}
\centering
\resizebox{0.45\textwidth}{!}{%

\begin{tabular}{lcccc}
\toprule
\multirow{2}{*}{Type of Bias} & \multicolumn{2}{c}{Attribute $A$} & \multirow{2}{*}{Papers} & \multirow{2}{*}{Examples}                                                   \\
\cmidrule(lr){2-3} 
                              & Known           & Labeled         &                         &                                                                             \\
                              \midrule
\multirow{3}{*}{Type I Bias}  & \cmark          & \cmark          & 253                     & \cite{DebFace,GAC,RL_RBN}                                                   \\
                              & \cmark          & \xmark          & -                       & -                                                                           \\
                              & \xmark          & \xmark          & -                       & -                                                                           \\
                              \midrule
\multirow{3}{*}{Type II Bias} & \cmark          & \cmark          & 246                     & \cite{learn_not_to_learn_Colored_MNIST,CSAD,End}                            \\
                              & \cmark          & \xmark          & 8                       & \cite{HEX_texture_bias1, ReBias_texture_bias2,rubi} \\
                              & \xmark          & \xmark          & 30                      & \cite{LfF_CelebA_Bias_conflicting,ECS,UBNet}                               \\
                              \midrule
Survey                        & -               & -               & 25                       & \cite{MLbias_survey,prediciton_quality_disparity,discussion_on_DP_EO}      \\
\bottomrule
\end{tabular}
}

\end{table}


%QA 任务

In this section, we present a comprehensive evaluation framework for the CondAmbigQA benchmark, which introduces a novel task of resolving ambiguous questions through explicit condition identification. Unlike traditional question answering tasks that directly generate answers, we propose that ambiguous questions should first be disambiguated by identifying explicit conditions that affect the answer, then generating appropriate responses for different condition combinations. This decomposition of the ambiguous QA process into condition identification and conditional answer generation represents a more structured approach to handling query ambiguity. Through carefully designed metrics and experimental protocols, our benchmark evaluates both the model's ability to identify and articulate these disambiguating conditions, and its capacity to generate condition-specific answers.

\subsection{Evaluation framework}
The CondAmbigQA benchmark adopts a multi-metric evaluation approach to comprehensively assess model performance. 
Let $M$ denote the model output, $G$ denote the ground truth, and $\textit{G-Eval}(x,y)$ represent the G-Eval function that evaluates the quality of output $x$ against reference $y$ based on pre-defined criteria~\cite{yao2024clave,liu2023g}. The four evaluation metrics are defined as follows:

\textit{Condition Score} measures the quality of condition identification:
\begin{equation}
\textit{Condition Score}(M,G) = \textit{G-Eval}(M.conditions, G.conditions),
\end{equation}
where the G-Eval function assesses both completeness and clarity of identified conditions.

\textit{Answer Score} evaluates the quality of generated answers:
\begin{equation}
\textit{Answer Score}(M,G) = \textit{G-Eval}(M.answers, G.answers),
\end{equation}
focusing on factual accuracy and condition-specific response quality.

\textit{Citation Score} quantifies source attribution accuracy:
\begin{equation}
\textit{Citation Score}(M,G) = \frac{|{c \in M.citations} \cap {c \in G.citations}|}{|{c \in G.citations}|},
\end{equation}
where citations are normalized and compared as sets to produce a score in [0,1].

\textit{Answer Count} measures response completeness:
\begin{equation}
\textit{Answer Count}(M,G) = |M.answer\ count - G.answer\ count|,
\end{equation}
reflecting the model's understanding of required answer granularity.

\subsection{Experimental protocol}

To evaluate the effectiveness of condition guidance in ambiguous question answering, we conduct two sets of experiments. 

In the main experiment, we assess models' native ability in condition identification and answer generation. Given a query $Q$ and retrieved passages $P$ (whole passages fragments) as input, models are required to first identify disambiguation conditions and then generate appropriate answers based on these \textbf{identified conditions}. Specifically, this protocol evaluates models' end-to-end capability in understanding and resolving query ambiguity through:

\begin{itemize}
   \item Condition identification: extracting key conditions that resolve ambiguity;
   \item Answer generation: providing appropriate answers based on identified conditions;
   \item Citation: supporting answers with relevant passages.
\end{itemize}

In the comparative experiment, we design two controlled settings to quantify the impact of condition guidance:

\begin{itemize}
  \item \textbf{Standard RAG}: Models directly generate answers from $Q$ and $P$ without explicit condition information;
   \item \textbf{Condition-guided}: Models receive additional ground-truth conditions alongside $Q$ and $P$.
 
\end{itemize}

This controlled comparison helps isolate the effect of condition guidance on answer quality and citation accuracy. By comparing model performance between these two settings, we can quantitatively assess how explicit condition information influences the quality of generated answers.

\subsection{Baseline models}
We evaluate our benchmark using five representative open-source language models: \texttt{LLaMA3.1} (8B) \cite{dubey2024llama}, trained on 1.2T tokens with optimized attention mechanism, \texttt{Mistral} (7B) \cite{jiang2023mistral}, known for its efficient architecture; \texttt{Gemma} (9B) \cite{team2024gemma}, trained on high-quality curated dataset, \texttt{GLM4} (9B) \cite{glm2024chatglm}, featuring enhanced cross-lingual abilities; and \texttt{Qwen2.5} (7B) \cite{yang2024qwen2}, optimized for comprehensive language understanding. These models, with parameters ranging from 7B to 9B, provide a diverse yet comparable foundation for baseline performance assessment.
To ensure reproducibility, all models are deployed through the \texttt{Ollama} framework, using default sampling parameters and 8K context window size. Model outputs are evaluated using G-Eval implemented via the \texttt{DeepEval} package, with \texttt{GPT4-mini} serving as the evaluation model through OpenAI's API.

\subsection{Scaling analysis}
To understand how model scale influences performance on our benchmark, we conduct additional experiments with two larger-scale models. This analysis aims to investigate whether performance on conditional ambiguous question answering follows established scaling laws~\cite{kaplan2020scaling}, providing insights into the relationship between model capacity and task performance. Through this evaluation framework, our benchmark provides a standardized way to assess and compare model performance in handling conditional ambiguous questions. The multi-metric approach and diverse experimental protocols enable detailed analysis of model capabilities. In particular, the scaling experiments validate the applicability of scaling laws to ambiguity resolution tasks, demonstrating that larger models consistently outperform smaller ones in condition adherence and answer quality. These findings offer valuable insights into the relationship between model size and performance, guiding future model development and optimization.
% \qy{In this paper, we propose an efficient single-stage framework called \nickname{} for 3D object detection. Considering the task of object detection inherently focuses on the foreground points, we propose an instance-aware learning-based downsampling way to automatically select the sparse yet important instance points. In addition, a dedicated contextual centroid perception module is proposed to fully exploit the geometrical structure around the bounding boxes. Extensive experiments conducted on the KITTI detection benchmark demonstrated the superior efficiency and accuracy of the proposed \nickname{}. \revise{In future work, we will further tackle extreme cases such as overlapped bounding boxes.}}

%This paper presents a new point-based single-stage 3D object detection networks, named \nickname{}. With novel instance-aware downsampling strategy and centroid rally module, we can effectively and efficiently achieve muti-class 3D object detection in a bottom-up manner.  Our \nickname{} achieves the best results among pure point-based methods, and provides a state-of-the-art efficiency than existing LiDAR detectors. In the future, we will focus on designing an efficient network to achieve real-time and robust 3D detection in 360-degree LiDAR scenes.

\qy{In this paper, we propose an efficient solution termed \nickname{} for point-based 3D object detection in LiDAR point clouds. Considering the task of object detection inherently focuses on the foreground information, we propose an instance-aware learning-based downsampling way to automatically select the sparse yet important instance points. Additionally, a dedicated contextual centroid perception module is proposed to fully exploit the geometrical structure around the bounding boxes. Extensive experiments conducted on three detection benchmarks demonstrated the superior efficiency and accuracy of the proposed \nickname{}. 
}

\smallskip\noindent\textbf{Limitations.} Although the proposed \nickname{} can achieve remarkable efficiency in object detection of large-scale LiDAR points clouds, it also has limitations. \textit{e.g.,} the instance-aware sampling relies on the semantic prediction of each point, which is susceptible to class imbalances distribution. For future work, we will further explore advanced techniques to alleviate the imbalanced issue.



\balance

% \section*{Acknowledgments}

% \newpage
\bibliographystyle{IEEEtran}
\bibliography{reference}

\end{document}