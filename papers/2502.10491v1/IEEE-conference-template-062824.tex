\documentclass[conference, 9pt]{IEEEtran}
\IEEEoverridecommandlockouts
% The preceding line is only needed to identify funding in the first footnote. If that is unneeded, please comment it out.
%Template version as of 6/27/2024

\usepackage{cite}
\usepackage{amsmath,amssymb,amsfonts}
\usepackage{graphicx}
\usepackage{color, xcolor}
\usepackage{lipsum}
\usepackage{comment}
    \newcommand{\thicktilde}[1]{\mathbf{\tilde{\text{$#1$}}}}
\usepackage{tikz}
  \newlength\squareheight
  \setlength\squareheight{6.75pt}
  \newcommand\squareslash{\tikz{\draw (0,0) rectangle (\squareheight,\squareheight);\draw(0,0) -- (\squareheight,\squareheight)}}
  \DeclareMathOperator\squarediv{\squareslash}
\usepackage{multirow}
\usepackage{float}
\usepackage{subcaption}
\usepackage{siunitx}
\usepackage{oubraces}
\usepackage{algorithmic}
\usepackage{graphicx}
\usepackage{textcomp}
\usepackage{xcolor}
% \usepackage{dutchcal}
% \usepackage{mathrsfs}
\usepackage[hang,flushmargin]{footmisc}
\usepackage[cal=boondoxupr]{mathalfa}
\def\BibTeX{{\rm B\kern-.05em{\sc i\kern-.025em b}\kern-.08em
    T\kern-.1667em\lower.7ex\hbox{E}\kern-.125emX}}
\newcommand\blfootnote[1]{%
  \begingroup
  \renewcommand\thefootnote{}\footnote{#1}%
  \addtocounter{footnote}{-1}%
  \endgroup
}
\newcommand{\gael}[1]{\textit{\textcolor{red}{#1}}}
\newcommand{\changhong}[1]{\textcolor{blue}{#1}}

\usepackage{fancyhdr}
\usepackage{ragged2e}  % For \justifying
% Define a custom style for the first page
\fancypagestyle{firstpage}{
% \vspace{-0.5cm}
    \fancyhf{} % Clear all headers and footers
    % Adjust the footskip to move the footer text higher
    \setlength{\footskip}{28pt}  % Reduce this value to move the footer up
    \fancyfoot[C]{\justifying \footnotesize{\copyright 2025 IEEE. Personal use of this material is permitted. However, permission to reprint/republish this material for advertising or promotional purposes or for creating new collective works for resale or redistribution to servers or lists, or to reuse any copyrighted component of this work in other works, must be obtained from the IEEE.}} % Footer text
    \renewcommand{\headrulewidth}{0pt} % No header line
    \renewcommand{\footrulewidth}{0pt} % No footer line
}

% Set the default style for the rest of the pages
\pagestyle{plain}

\begin{document}

\title{F-StrIPE: Fast Structure-Informed Positional Encoding for Symbolic Music Generation\\
\thanks{This work was funded by the European Union (ERC, HI-Audio, 101052978). Views and opinions expressed are however those of the author(s) only and do not necessarily reflect those of the European Union or the European Research Council. Neither the European Union nor the granting authority can be held responsible for them. The accompanying website at \textit{bit.ly/faststructurepe} contains music samples and code.}
}

% \author{\IEEEauthorblockN{1\textsuperscript{st} Manvi Agarwal}
% \IEEEauthorblockA{
% \textit{T\'{e}l\'{e}com Paris}\\
% \textit{Institut Polytechnique de Paris}\\
% Paris, France \\
% email address or ORCID}
% \and
% \IEEEauthorblockN{2\textsuperscript{nd} Changhong Wang}
% \IEEEauthorblockA{
% \textit{T\'{e}l\'{e}com Paris}\\
% \textit{Institut Polytechnique de Paris}\\
% Paris, France \\
% email address or ORCID}
% \and
% \IEEEauthorblockN{3\textsuperscript{rd} Ga\"{e}l Richard}
% \IEEEauthorblockA{
% \textit{T\'{e}l\'{e}com Paris}\\
% \textit{Institut Polytechnique de Paris}\\
% Paris, France \\
% email address or ORCID}
% }
% \author{\IEEEauthorblockN{Manvi Agarwal, %\IEEEauthorrefmark{1}, 
% Changhong Wang, %\IEEEauthorrefmark{2}, 
% Ga\"{e}l Richard%\IEEEauthorrefmark{3}, 
% %Author 4\IEEEauthorrefmark{3}, Author 5\IEEEauthorrefmark{4}
% }
% \IEEEauthorblockA{T\'{e}l\'{e}com Paris, Institut Polytechnique de Paris, France}
% }

\author{\IEEEauthorblockN{Manvi Agarwal, 
Changhong Wang, 
Ga\"{e}l Richard 
%Author 4\IEEEauthorrefmark{3}, Author 5\IEEEauthorrefmark{4}
}
\IEEEauthorblockA{LTCI, T\'{e}l\'{e}com Paris, Institut Polytechnique de Paris, France}
}


% \author{\IEEEauthorblockN{Michael Shell\IEEEauthorre fmark{1}, Homer Simpson\IEEEauthorrefmark{2}, James K irk\IEEEauthorrefmark{3}, Montgomery Scott\IEEEautho rrefmark{3} and Eldon Tyrell\IEEEauthorrefmark{4}} \IEEEauthorblockA{\IEEEauthorrefmark{1}School of Ele ctrical and Computer Engineering\\
% Georgia Institute of Technology, Atlanta, Georgia 30 332--0250\\
% Email: mshell@ece.gatech.edu} \IEEEauthorblockA{\IEEEauthorrefmark{2}Twentieth Cen tury Fox, Springfield, USA\\
% Email: homer@thesimpsons.com}
% \IEEEauthorblockA{\IEEEauthorrefmark{3}Starfleet Aca demy, San Francisco, California 96678-2391\\ Telephone: (800) 555--1212, Fax: (888) 555--1212} \IEEEauthorblockA{\IEEEauthorrefmark{4}Tyrell Inc.,
% 123 Replicant Street, Los Angeles, California 90210 --4321}}


\maketitle
\thispagestyle{firstpage}
\begin{abstract}
While music remains a challenging domain for generative models like Transformers, recent progress has been made by exploiting suitable musically-informed priors. One technique to leverage information about musical structure in Transformers is inserting such knowledge into the positional encoding (PE) module. However, Transformers carry a quadratic cost in sequence length. In this paper, we propose \emph{F-StrIPE}, a structure-informed PE scheme that works in linear complexity. Using existing kernel approximation techniques based on random features, we show that F-StrIPE is a generalization of Stochastic Positional Encoding (SPE). %\textit{We compare SPE and F-StrIPE from the perspective of musical structure, showing that our approach is appropriately grounded in both theory and domain knowledge. }
We illustrate the empirical merits of F-StrIPE using melody harmonization for symbolic music.
% \blfootnote{\copyright 2024 IEEE. \textit{Personal use of this material is permitted. Permission from IEEE must be obtained for all other uses, in any current or future media, including reprinting/republishing this material for advertising or promotional purposes, creating new collective works, for resale or redistribution to servers or lists, or reuse of any copyrighted component of this work in other works.}}
\end{abstract}

\begin{IEEEkeywords}
music generation, symbolic music, transformers, positional encoding, kernels.
\end{IEEEkeywords}

\documentclass[../main.tex]{subfiles}
\graphicspath{{../images/}}
\makeatletter
\def\input@path{{../images/}}
\makeatother
\begin{document}
\section{Introduction}
\begin{figure}
\centering
\begin{tikzpicture}
\node[inner sep=0pt] (ws) at (0, 0) {
\includegraphics[height=.4\textwidth, trim={10cm 0 10cm 0},clip]{world_space.png}};
\node[inner sep=0pt] (cs) at (6,0) {\includegraphics[height=.4\textwidth, trim={10cm 1cm 10cm 4cm},clip]{conf_space.png}};
\end{tikzpicture}
\vspace{-5pt}
\label{fig:pbrm_intro}
\caption{\textbf{Left}: Shows world space obstacles as grey spheres. Robots start and goal configuration is colored red and green, respectively. Configurations along the computed path are colored transparent blue. \textbf{Right:} Mapped world space scenario to configuration space. Obstacle region is the grey mesh. Red spheres are collision-free regions computed by the neural SCDF. The optimized shortest path in the convex corridor is the blue curve.}
\vspace{-25pt}
\end{figure}
Motion planning is the problem of finding a collision-free trajectory that connects a given start and goal configuration. The planning takes place in the configuration space of the robot. For single body robots, like mobile robots or drones, the configuration space and the world space are usually the same. This simplifies the planning, since explicit obstacle representations are available which enables geometrical tools like separating hyperplanes, smallest distance to obstacles etc., to be used when designing motion planning algorithms. For multi-body robots like manipulators, the situation is completely different. The world space obstacles are usually mapped to non-convex regions, and to make the problem even harder, the mapping is usually not known. Forming explicit representations of the obstacle region in the configuration space is usually too expensive or intractable. Despite all of this, sampling based planners are used with great success, which mainly is due to their use of implicit representations of the obstacle region. The basic idea is to construct a graph in the configuration space that covers and connects the collision-free region. From this graph, a path can be extracted that connects a given start and goal configuration. The approach is computationally expensive, since the graph is constructed with the smallest geometrical building block available, points, which represents a collision-check. Furthermore, the extracted paths from the graph are non-smooth and jagged due to the stochastic nature of the approach. This adds an additional post-processing step to the process, where the paths are shortcutted and smoothened, before the path can be used for tracking. Clearly a lot of time is invested to form this graph and produce smooth paths. Thus, if the obstacles start to move, then all of this work is done in no use, since all points that make up this graph need to be re-verified, which is simply too time consuming to be done in real time.
\\\\
In this work, we want to address the existing drawbacks of the sampling based planners. Our main contribution is an improved motion planner where each vertex in the graph covers a collision-free region in the form of a sphere instead of a point and where the edges are formed with neighboring intersecting spheres. This representation has the advantage of instead of returning piecewise linear paths, returning a sequence of overlapping spheres, i.e. a convex corridor, that connects a given start and goal configuration, illustrated in Figure \ref{fig:pbrm_intro}. This convex corridor allows us to use convex optimization to produce smooth trajectories, instead of computationally expensive post-processing methods. The representation further allows us to estimate the coverage of the collision-free space, which gives us awareness and feedback in the offline roadmap construction phase. Finally, our representation is simple to adapt to moving obstacles, simply requery for the new radii and recheck for intersections. 
\\\\
The spherical collision-free regions are formed using a signed distance function (SDF), which is a function that returns the smallest distance from an arbitrary point to the boundary of an obstacle. As the name implies, the distance is signed, thus if the point is inside the obstacle it is negative otherwise positive. If the distance is positive, a sphere with radius equal to the distance is guaranteed to cover a collision-free region. Using an SDF in motion planning is not new, but what is novel about our approach is that we express the distance in the configuration space instead of the world space and by doing so allows us to form these convex collision-free regions. We refer to the resulting SDF as a signed configuration distance function (SCDF). Computing an SCDF analytically is non-trivial, our approach is therefore to parameterize the SCDF with a deep neural network and learn the mapping by supervised learning. Our resulting neural SCDF can compute distances for different parameter values of obstacle shapes and we also show how multiple distances can be combined, thus making our approach flexible.
\section{Related work}
Motion planning algorithms can roughly be divided into three families, grid-based, sampling based and optimization based methods. Grid-based methods (GBM) discretize the planning space from which a graph is then compiled. A standard search method is A$^\star$ \citep{a_star}, which is classified as an \textit{informed} search method, since it employs a heuristic function to speed up the search. A$^\star$ guarantees to return an optimal path at the level of discretization used. GBMs usually discretize the planning space by a regular lattice and this limits the GBMs to problems with low dimensionality due to the curse of dimensionality. Thus, GBMs are usually limited to single-body robots where the degrees of freedom (DOF) are low. To overcome the inherent scaling problem with the GBMs, stochastic methods are usually used for multi-body robots. These methods are termed as sampling-based methods (SBM) and core members within this family are the rapidly-exploring random trees (RRT) \citep{rrt} and the probabilistic roadmap (PRM) \citep{prm}. RRT grows a tree from the start configuration and explores the collision-free region in a rapid way until it is able to connect to the goal region. RRT is usually improved by bi-directional planning \citep{rrt_connect}, i.e. an additional tree is grown from the goal configuration and the trees are tested for connection after any tree has been expanded. RRT is a single-query method, thus it searches for a path from scratch each time it is queried. Contrary to this, PRM is a multi-query method, which solves for multiple queries without starting from scratch. PRM does this by creating a roadmap (graph) that covers the collision-free space as an offline step. The graph is then used to solve for multiple queries. PRMs are used in cases where the environment does not change since the extra offline step is too computationally costly and needs to be re-done if the environment is changed. In our work, we address this inherent issue by using a different roadmap representation. Our vertices in the graph cover a collision-free region in the form of spheres and we form the edges by checking for intersecting spheres. If something in the environment changes, we recompute the spheres radii and recheck the intersections, without relying on collision detection. We use a trained neural network to compute the sphere radius, therefore querying for the radius can be done fast, hence our representation enables the PRM for dynamic environments.
\\\\
In the recent decades, optimization based methods (OBM) \citep{chomp, schulman, itomp, stomp} have been introduced as an alternative to SBM for multi-body robots. Like the SBM, the OBMs scale well to higher dimensional problems and produce smoother motion. It is common to use a SDF in the optimization since it is a smooth function, thus enabling gradient-based methods. However, the standard way of expressing the SDF is in world space. The distance therefore needs to be mapped to the configuration space by the forward kinematics. This mapping makes the optimization problem a non-linear program (NLP), which is computationally expensive to solve. Recently, a different approach has been proposed. In \cite{mp_gcs} motion planning is formulated as a convex optimization problem by using the graph of convex sets framework \citep{gcs}. The underlying idea is to decompose the collision-free space into intersecting convex sets from which a convex optimization problem is formulated. In cases where an explicit representation of the obstacles in the configuration space exists, like for single-body robots, creating collision-free convex regions can be done fast \citep{iris}. For multi-body robots, this is non-trivial. Existing work does this successfully \citep{iris_nlp, iris_c} by an optimization based approach, but the methods are still too time consuming to be used in the presence of moving obstacles. Our approach is instead to use deep learning to learn an SDF expressed in the configuration space. With this, we can query for shortest distances to the collision boundary, which allows us to expand spherical regions which are collision-free. Our approach is fast and therefore enables our suggested roadmap planner to be used in dynamic environments.
\\\\
Recent research has focused on learning collision detection \citep{fk_kernel_distance, diffco, graphdistnet} by predicting the signed distance between the robot links and the surrounding obstacles in the world space. The learned SDF is used in trajectory optimization but since the distance is expressed in the world space, the problem becomes an NLP and therefore takes a long time to solve. We take a novel approach and suggest to instead express the signed distance in the configuration space. This allows us to improve the PRM at the same time as it enables convex optimization for trajectory optimization, which runs faster and is more reliable than NLP solvers. In \cite{cspf} a learned signed distance function in the configuration space is proposed similar to our approach. However, their approach is restricted to point cloud representations, while we propose to represent the obstacles as parameterized geometric shapes, e.g. spheres. Furthermore, we also show how to use our learned SCDF to improve an existing roadmap planner.
\section{Problem formulation}
A robot is located in the world space, $\W \subset \R^3 $. The unique location of the robot is given by its configuration $\q \in \C$, where $\C$ is the configuration space. The set of points covered by the robots bodies at a certain configuration is expressed as $\B(\q) \subset \W$. The robot is surrounded by $\NrObst$ obstacles $\O = \bigcup_{i=1}^{\NrObst} \O_i$, where  $\O_i \subset \W$. The representation of the obstacle in the configuration space is the set $\C\O_i = \{\q \in \C \: |\: \B(\q) \cap \O_i \neq \emptyset \}$. The obstacle space is formed as $\Co = \bigcup_{i=1}^{\NrObst} \C \O_i$. The complement is referred to as the free space, $\Cf = \C \setminus \Co$. The path planning problem is a tuple, ($\Cf$, $\qStart$, $\qGoal$), where we want to connect a query pair, consisting of a start, $\qStart$, and goal configuration, $\qGoal$, with a geometric path, $\q(s): [0, 1] \mapsto \Cf$, such that $\q(0)=\qStart$ and $\q(1)=\qGoal$, or report correctly when such a path does not exist.
\end{document}


\begin{figure}[t]
    \centering
    \includegraphics[width=1.0\linewidth]{figs/icassp_2025_summary_palette_corrected.png}
    \caption{A schematic showing our main contributions (best viewed in colour).}
    \label{fig:contributions}
\end{figure}

% \begin{figure*}[h!]
%     \centering
%     \includegraphics[width=1.6\columnwidth]{figs/R.png}
%     \caption{A visual representation of the connection between Stochastic Positional Encoding and Random Fourier Features. The variables referenced here are detailed in Sections \ref{section:background} and \ref{section:methods}.}
%     \label{fig:visualize_matrices}
% \end{figure*}

\begin{figure*}[h!]
    \centering
    \includegraphics[width=1.58\columnwidth]{figs/icassp_2025_palette_corrected.png}
    \caption{A visual representation of the connection between Stochastic Positional Encoding and Random Fourier Features (best viewed in colour). The variables referenced here are detailed in Sections \ref{section:background} and \ref{section:methods}.}
    \label{fig:visualize_matrices}
\end{figure*}

\section{Basic Background: Supervised Learning and the PAC Model}
\label{sec:background}

At this point almost everyone has heard of machine learning (ML). Anyone likely to stumble upon this article will have also heard of its most influential special case, supervised learning, and those theoretically inclined will also be familiar with the PAC model. Nonetheless, I will set the stage by  recapping the basics.

\subsection{Basics of Supervised Learning}%Let's set the stage in any case

\emph{Supervised Learning} is the task of ``coming up'' with a function $f: \X \to \Y$ to ``explain'' or ``fit'' a sequence of input/output examples   $(x_1,y_1), \ldots, (x_n,y_n)$, with $x_i \in \X$ and $y_i \in \Y$.  Here $\X$ is a \emph{data domain} consisting of \emph{datapoints} $x \in \X$, $\Y$ is a \emph{label set} consisting of \emph{labels} $y \in \Y$, and the sequence $(x_1,y_1),\ldots,(x_n,y_n)$ is the \emph{training data} consisting of \emph{labeled examples (a.k.a. samples)}~$(x_i,y_i)$.  I~will refer to the chosen function $f$ as a \emph{predictor}, and to $n$ as the \emph{sample size}. A \emph{learning algorithm} takes as input training data, and outputs (some representation of) a predictor $f \in \Y^\X$.\footnote{Note that this describes the usual \emph{batch}, a.k.a.~\emph{offline}, setting of supervised learning. I do not discuss other paradigms such as online or active learning in this article.} 



Success in supervised learning is defined as \emph{generalization} to  future examples: For a typical \emph{test example}  $(x_{\tst},y_{\tst})$, the predicted label $y'_{\tst}=f(x_{\tst})$ should ``equal'' $y_{\tst}$, perhaps approximately. We usually assume the test example is drawn from the same  ``source'' as the training data  --- commonly, i.i.d.~from the same distribution. The quality of the prediction is quantified by $\ell(y'_{\tst},y_{\tst})$, where $\ell:~\Y~\times~\Y \to \RR_{\geq 0}$ is a \emph{loss function} chosen as part of the problem definition. Common loss functions include the 0-1 loss $\ell_{0-1}(y',y) = [y' \neq y]$ for \emph{classification} problems,\footnote{The notation $[P]$ denotes $1$ when predicate $P$ is true, and denotes $0$ when $P$ is false.} as well as the absolute loss $|y'-y|$ or squared loss $(y'-y)^2$ for \emph{regression problems} featuring $\Y  \sse \RR$.

Nontrivial generalization properties are typically only possible if one assumes something about the data.\footnote{The need for such an assumption is formalized by the  \emph{no free lunch theorems} of supervised learning \cite{wolpert_connection_1992,wolpert_lack_1996,schaffer_conservation_1994}.} The Bayesian approach to  machine learning, common in many applications, assumes some parametric form for the distribution generating the data, and postulates a prior on the parameters. This is not the approach I will take in this article. Instead, I will focus on the frequentist --- and some would say ``worst-case'' or ``adversarial'' ---  approach that is common in the computational learning theory community, embodied by the PAC model. Here we assume that the (training and test) data can be explained, perhaps approximately, by a function in some ``simple enough to learn'' class of functions $\H \sse \Y^\X$, often called the \emph{hypotheses}. Equivalently, we  seek a predictor which explains the unseen data roughly  as well as the best hypothesis $h^* \in \H$, whether or not we assume that $h^*$ itself provides a perfect explanation.



 \paragraph{Common Algorithmic Templates.} Perhaps the best known general-purpose supervised learning algorithm is \emph{empirical risk minimization (ERM)}, which chooses as its predictor a hypothesis $f \in \H$ minimizing $\frac{1}{n} \sum_{i=1}^n \ell(f(x_i),y_i)$ --- a quantity called the \emph{training error}, \emph{empirical error}, or \emph{empirical risk} of $f$. %\footnote{When multiple hypotheses minimize the empirical risk, we assume ERM breaks ties arbitrarily.}
A common template for generalizing ERM involves adding a \emph{regularization term} $\psi(f)$ to the  objective function, typically chosen to measure some notion of ``hypothesis complexity.'' An algorithm instantiating this template is known as a \emph{structural risk minimizer (SRM)}, and chooses as its predictor the hypothesis $f \in \H$ minimizing the \emph{structural risk} $\frac{1}{n} \sum_{i=1}^n \ell(f(x_i),y_i) + \psi(f)$. Other well-known algorithms, such as gradient descent and its variations,  can frequently be interpreted as approximate implementations of ERM or SRM.


\paragraph{Proper vs Improper Learning.} A learning algorithm is said to be \emph{proper} if its predictor $f$ is always chosen from the hypothesis class, i.e., $f \in \H$, otherwise it is said to be \emph{improper}. ERM  is an example of a proper learning algorithm, as are SRM algorithms of the form described above.  In the \emph{proper regime} of learning, algorithms are required to be proper. This article will be concerned with the more flexible \emph{improper regime} (a.k.a \emph{representation-independent learning}), where no such constraint is placed on the learner. In other words, all we care about is predictive power at test time, rather than any insights derived from the functional form or representation of the predictor~itself.


\subsection{The PAC Model}
A standard mathematical setup for evaluation of supervised learning algorithms, at least in the theoretical computer science community, is Valiant's \emph{Probably Approximately Correct (PAC) model} of learning (see e.g.~\cite{kearns_introduction_1994,mohri_foundations_2018}). Here, we assume there is an unknown distribution $\D$ on $\X \times \Y$ from which training and test data are  drawn.  Specifically, the labeled datapoints of the training set  $(x_1,y_1), \ldots, (x_n,y_n)$, as well as the test data  $(x_\tst,y_\tst)$, are i.i.d.~from $\D$. Often it is assumed that $\D$ lies in some class of distributions of interest. The \emph{true expected loss}, or simply \emph{loss}, of a predictor $f: \X \to \Y$ is the expected loss it incurs on draws from $\D$, written $L_\D(f) = \Ex_{(x,y) \sim \D} \ell(f(x),y)$.


There are two main ``settings'' in PAC learning. The  \emph{realizable setting} only requires that the data be perfectly explained by some hypothesis in $\H$. More generally, the \emph{agnostic setting} makes no assumption relating the data to the hypotheses, but shifts the goalposts as necessary to allow nontrivial guarantees: the expected loss at test time is evaluated only ``relative'' to that of the best hypothesis $h^* \in \H$. There are other settings which make more nuanced assumptions, such as $\D$ being of a particular parametric form or its support living in some (unknown) lower-dimensional space, etc. I will mostly discuss the realizable and agnostic settings in this article, those being the simplest and most studied from a theoretical perspective. %TODO:We will briefly discuss other settings in Section ??

The PAC model demands high probability guarantees of learners, in the worst case over distributions of interest. Consider first the realizable setting, where $\D$ is such that $\min_{h \in \H} L_{\D}(h) = 0$. A PAC learner has \emph{error} $\epsilon=\epsilon(n)$ and \emph{confidence} $\delta=\delta(n)$ if, when training data consists of $n$ i.i.d~samples from a realizable distribution $\D$, it produces a predictor $f$  satisfying $L_\D(f) \leq \epsilon$ with probability at least $1-\delta$. In the agnostic setting, where $\D$ can be arbitrary, we require $L_\D(f) - \min_{h \in \H} L_\D(h) \leq \epsilon$ with probability $1-\delta$.

In both the realizable and agnostic settings, we look for PAC learners with small $\epsilon$ and $\delta$ as a function of the sample size $n$. An equivalent perspective looks at the sample complexity $m(\epsilon,\delta)$, which is the minimum sample size which guarantees error  at most $\epsilon$ with probability at least $1-\delta$. We say a problem is \emph{PAC learnable} if its PAC sample complexity is finite whenever $\epsilon,\delta > 0$.

For most PAC learning problems, learnability and sample complexity are characterized in terms of a  ``dimension'' of the hypothesis class. Most prominently this is the \emph{VC dimension} for binary classification, the \emph{fat shattering dimension} for agnostic regression, and the \emph{DS dimension} for multiclass classification (see \cite{anthony_neural_1999,daniely_optimal_2014,brukhim_characterization_2022}). Treatment of these is beyond the scope of this article. The unfamiliar reader need not worry, however,  as dimensions will feature only tangentially in our~discussion.




%\paragraph{Learning settings: Realizable, Agnostic, etc.} In learning theory, evaluating a supervised learning algorithm requires specifying a data model and an objective. We will leave the details of the data model flexible for now, to allow for both the PAC model and the adversarial transductive model. Nonetheless we will describe two variations, which we call ``settings'', which cut across different models. The  \emph{realizable setting}  requires only that the data be perfectly explained by some hypothesis $h \in \H$ --- i.e., there exists a hypothesis which is guaranteed to suffer a loss of $0$ on training and test data. The performance of the learning algorithm is its expected loss at test time for some ``worst case'' realizable instance. More generally, the \emph{agnostic setting} makes no assumption relating the data to the hypotheses, but shifts the goalposts as necessary to allow nontrivial guarantees: the expected loss at test time is evaluated only ``relative'' to that of the best hypothesis $h^* \in \H$, again for some ``worst case'' instance. There are other settings which make more nuanced assumptions about the data, such as it is drawn from a distribution of a particular parametric form, or that it lives in some (unknown) lower-dimensional space, etc. We will mostly discuss the realizable and agnostic settings, those being the simplest and most studied from a theoretical perspective.




%%% Local Variables:
%%% mode: latex
%%% TeX-master: "learning_matching"
%%% End:


\section{Methods} \label{section:methods}

\subsection{Structure-informed positional encoding}

As we hinted in Section \ref{section:background}, positional encoding depends on a sequence $\mathcal{P} = [ p_1, p_2, ... , p_T ]$ of positional indices. In standard PE, which does not utilize structure, $p_i = i$, which makes $\mathcal{P}$ a linear grid. When structure is included in PE, $p_i = s_\ell (i)$, where $s_\ell (i)$ gives the structural label at level $\ell$ (e.g. chord) for timestep $i$. In this case, $s_\ell (i) = s_\ell (i^{\prime})$ is possible for $i \neq i^{\prime}$, making $\mathcal{P}$ a non-linear grid. In fact, we can consider vectorial positional indices with multiple resolutions of structural organization. Consequently, we obtain $p_i = \mathbf{s} (i)$, where $\mathbf{s} (i) = [ s_1 (i) , ..., s_\ell (i), ..., s_L (i) ]$ is a vector of $L$ structural labels. Viewed in this way, PEs with structure can simply be used as a drop-in replacement for PEs without structure by replacing the form of $p_i$, giving us richer positional information. This is a way to flexibly represent domain-specific prior knowledge about the underlying data domain.

\subsection{Adding structure to SPE: F-StrIPE:SFF}

In order to use rich positional information in SPE, we can augment the sinusoidal feature matrix $\boldsymbol{\Omega}$ from (\ref{eq:sff:sine_features}) to be:
\begin{equation} \label{eq:pos_sine_features}
    [\boldsymbol{\Omega}(\mathcal{P}, \boldsymbol{f}, \boldsymbol{\theta})]_{i j}= \begin{cases}\cos \left(2 \pi \mathbf{f}[\omega, :]^\top p_i+\boldsymbol{\theta}[\omega]\right) & \text { if } j = 2\omega \\ \sin \left(2 \pi \mathbf{f}[\omega, :]^\top p_i+\boldsymbol{\theta}[\omega] \right) & \text { else }\end{cases}
\end{equation}
Here, we use the vectorial formulation of structure-aware positional indices $p_i = \mathbf{s}(i)$, unlike (\ref{eq:sff:sine_features}) where $p_i = i$ was a sequence of structure-free positional indices linked solely to the passage of time. Whereas in (\ref{eq:sff:sine_features}), $\mathbf{f}[\omega]$ was a single frequency, $\mathbf{f}[\omega, :]$ in (\ref{eq:pos_sine_features}) is a vector of frequencies. Therefore, each frequency $\mathbf{f}[\omega, \ell]$ in this vector acts on the $\ell^{\text{th}}$ structural label at timestep $i$. We can combine (\ref{eq:spe:ohyeah}), (\ref{eq:sff}) and (\ref{eq:pos_sine_features}) to obtain a fast structure-aware PE technique that uses Stochastic Fourier Features. We call this method \textit{F-StrIPE:SFF}.

% Here, unlike (\ref{eq:sff:sine_features}) where $\mathbf{f}[\omega]$ was a single frequency, $\mathbf{f}[\omega, :]$ is a vector of frequencies. Each frequency $\mathbf{f}[\omega, \ell]$ in this vector acts on the $\ell^{\text{th}}$ structural label at timestep $i$, where we can use the vectorial formulation of structure-aware positional indices $p_i = \mathbf{s}(i)$. In contrast, $p_i = i$ in (\ref{eq:sff}).

\subsection{Asymptotic case of SFF: Random Fourier Features} \label{ssection:asymptotic_rff}

Using Equations (\ref{eq:spe:ohyeah}) and (\ref{eq:sff}), we can express the SFF approximation of the positional matrix $\mathbf{P}_d$ for arbitrary timesteps $m$ and $n$ as:
\begin{equation} \label{eq:sff_mn}
    \mathbf{P}_d[m, n] \approx \Big[ \Omega_\mathcal{Q}^d [m, :] (\mathbf{Z}_d \mathbf{Z}_d^\top) \Omega_\mathcal{K}^{d^\top} [:, n] \Big] / R \\
\end{equation}

where we use the abbreviation $\Omega_\mathcal{A}^d = \boldsymbol{\Omega}\left(\mathcal{P}_A, \boldsymbol{f}_d, \boldsymbol{\theta}_d^A \right) \text{diag}\left(\ddot{\boldsymbol{\lambda}_d}\right)$. We observe that $\mathbf{Z}_d \mathbf{Z}_d^\top = \widehat{\mathbf{C}}_d$ acts as an empirical covariance matrix for the features $\Omega_\mathcal{Q}^d$ and $\Omega_\mathcal{K}^d$. Since $\mathbf{Z}_d$ has zero mean and unit variance, as $R \to \infty$, $\widehat{\mathbf{C}}_d$ approaches the theoretical covariance matrix $\mathbf{C}_d = \mathbf{I}_{2N_{f}}$.
In the ideal case of $\mathbf{C}_d$, (\ref{eq:sff_mn}) simplifies to $\mathbf{P}_d[m, n] \approx \Omega_\mathcal{Q}^d [m, :] \Omega_\mathcal{K}^{d^\top} [:, n]$, giving:
\begin{equation} \label{eq:ideal_C}
    \mathbf{P}_d[m, n] \approx  \frac{1}{N_f}  \sum_{\omega = 1}^{N_f} \Lambda_\omega \cos \Big( f_{\omega} ( \mathcal{P}_Q[m] - \mathcal{P}_K[n] ) + \Theta_\omega \Big)
\end{equation}
where $\Lambda_\omega$ is the gain contributed by the matrices $\text{diag}\left(\ddot{\boldsymbol{\lambda}_d}\right)$ and $\Theta_\omega$ is the phase-shift contributed by $\boldsymbol{\theta}^Q_d$ and $\boldsymbol{\theta}^K_d$. This representation has been studied in previous work, where it is called \textit{Random Fourier Features} (RFF) \cite{rahimi_random_2007, sutherland_error_2015}.

\subsection{Generalizing F-StrIPE:SFF to F-StrIPE}

Using this insight, we can redesign the positional feature matrices from (\ref{eq:sff}) to be:
\begin{equation} \label{eq:rff}
    \mathbf{P}^{Q/K}_d = \boldsymbol{\Omega}\left(\mathcal{P}_{Q/K}, \boldsymbol{f}_d, \boldsymbol{\theta}_d^{Q/K}\right) \text{diag}\left(\ddot{\boldsymbol{\lambda}_d}\right) / \sqrt{N_f}
\end{equation}
where the sinusoidal features $\boldsymbol{\Omega}$ uses structure-aware positional indices as given in (\ref{eq:pos_sine_features}). With this, we can now modify (\ref{eq:spe:ohyeah}) to use $\mathbf{P}^{Q/K}_d$ in place of $\thicktilde{\mathbf{P}}^{Q/K}_d$, giving us $\mathbf{Q}^{\text{RFF}}/\mathbf{K}^{\text{RFF}}$ in place of $\mathbf{Q}^{\text{SFF}}/\mathbf{K}^{\text{SFF}}$. To signify that such a PE technique generalizes F-StrIPE:SFF to use RFF in place of SFF, we call this method \textit{F-StrIPE}.

In the second row of Figure \ref{fig:visualize_matrices}, similar to SFF, we show the different components of RFF in the case where gains are 1 and phase shifts are 0. RFF can be understood as the ideal case of SFF where $R \to \infty$. Seen in this way, RFF gives us a noiseless estimate of $\mathbf{P}_d$ with direct access to the theoretical covariance matrix $\mathbf{C}_d$.

% In addition, if we pair $\mathbf{Q}^{\text{SFF}}/\mathbf{K}^{\text{SFF}}$ with the sinusoidal features given in Equation \ref{eq:sff:sinusoidal_features}, in place of those given in Equation \ref{eq:sff:sine_features}, we obtain a richer version of SPE which accepts multi-dimensional structural information instead of time indices. To distinguish it from SPE, we name this variant F-StrIPE:SFF.

\section{Experiments}
% To test the efficacy of our method, we use the task of melody harmonization for symbolic music. 
We assess the merits of our approaches on the task of melody harmonization for symbolic music.

\subsection{Dataset and Input Representation} \label{sssection:data_input}
We use the Chinese POP909 dataset~\cite{wang_pop909_2020} and three levels of structural labels with different resolutions~\cite{dai_automatic_2020}: melodic pitch ($16^{th}$-note), chord (quarter-note) and phrase (measure). Each MIDI file in this dataset consists of three tracks: melody, bridge (second melody) and piano (accompaniment). 
We use the POP909 alignment dataset~\cite{agarwal_structure_2024} to correctly match the structural labels with the input. We convert the MIDI files to binary pianorolls $\mathbf{X} \in \mathbb{B}^{(n_\text{tracks} \times 128) \times n_\text{time}}, \mathbb{B} = \{0, 1\}$, where $n_\text{tracks}$ is the number of tracks and $n_\text{time}$ is the number of timesteps in the pianoroll.

\subsection{Task Setup}
Given the sequence for the melody and bridge tracks $\big[ \mathbf{x}_n \in \mathbb{B}^{(n_\text{tracks} - 1) \times 128} \big]$ as input, with $n \in \{ 1, ..., n_\text{time}\}$, the model must predict all tracks $\big[ \mathbf{y}_n \in \mathbb{B}^{n_\text{tracks} \times 128} \big]$. We expect the model to produce the complete accompaniment track for all timesteps at once, without conditioning later predictions on earlier predictions. 
% We use three settings: (16, 16), (16, 64) and (64, 64), where the first number is the sequence length (in bars of music) used for training and the second number is that used for testing.
We use two settings: (16, 16) and (16, 64). The first number is the sequence length (in measures) for training and the second is that used for testing.

\subsection{Model and Training}
We use a 2-layer causal encoder Transformer with 4 heads and 512 model dimension. Training for 15 epochs with a batch size of 8, we use gradient clipping and curriculum learning~\cite{bengio_curriculum_2009}. We use two learning rate schedulers: a linear warmup and an epoch-wise decay.
We do a grid-search for two hyperparameters: learning rate (choices: $\{ 1, 5, 10 \} \times 0.0001$) and post-processing binarization strategy~\cite{agarwal_structure_2024} (choices: thresholding, thresholding with merge). While the first binarization strategy uses a fixed threshold, the second additionally fills the gap between notes if the gap is less than a minimum distance.

\subsection{Baselines and Our Methods} \label{sssection:baselines}
We consider three types of baselines: (i) Transformers without PE (NoPE~\cite{tsai_transformer_2019,haviv_transformer_2022}), (ii) Transformers with efficient, approximate atttention but no structural information in PE (SPE~\cite{liutkus_relative_2021}), and (iii) Transformers with structural information in PE but using inefficient, exact attention (S S-RPE~\cite{agarwal_structure_2024}).
From our methods, we use F-StrIPE with the three structural levels described in Section \ref{sssection:data_input}. We also assess the influence of different random features with F-StrIPE:SFF using all structural levels. 
We perform ablations on F-StrIPE by selecting one level at a time during training. Finally, we use the best-performing structural level from the F-StrIPE ablations to additionally do an ablation study with F-StrIPE:SFF.

\subsection{Evaluation}
We choose a collection of musically-motivated metrics from the literature, guided by four criteria. 

To assess large- and small-scale structural properties, we use Self-Similarity Matrix Distance (SSMD)~\cite{wu_musemorphose_2021}. For both the target and the prediction, we calculate chroma vectors, giving us the number of onset occurrences per chroma in every half-measure. We then compose a self-similarity matrix (SSM) for each chroma vector by taking the pairwise cosine similarities between all elements of the vector. The SSMD is the mean absolute difference between the SSM of the target and the SSM of the prediction.

For melodic consistency, we use Chroma Similarity (CS)~\cite{wu_musemorphose_2021}. Using the aforementioned method of constructing chroma vectors, we compute the CS as the mean cosine similarity between corresponding entries of the target chroma vector and the prediction chroma vector.

For rhythmic consistency, we use Grooving pattern Similarity (GS)~\cite{wu_jazz_2020}. The grooving pattern of a piece of music is a vector that encodes a 1 for the quarter-notes where onsets occur and 0 for those where no onsets occur. After obtaining the grooving patterns of the target and prediction, we compute the GS as the percentage of quarter-notes where the corresponding pattern values match.

To gauge polyphonicity, we use Note Density Distance (NDD)~\cite{agarwal_structure_2024, haki_real_2022}. We calculate the total number of pitches in each $16^{\text{th}}$-note of the target and prediction. The NDD is the average percentage of missing pitches in the prediction, with the number of pitches in the target giving us the maximum possible value.

% % Please add the following required packages to your document preamble:
% \usepackage{multirow}
\begin{table*}[t]
\centering
\resizebox{2.05\columnwidth}{!}{%
\begin{tabular}{lcccccccc}
\hline
\multicolumn{1}{c|}{\multirow{3}{*}{\textbf{Method}}} & \multicolumn{4}{c|}{\textbf{Train = 16 bars; Test = 16 bars}} & \multicolumn{4}{c}{\textbf{Train = 16 bars; Test = 64 bars}} \\
\multicolumn{1}{c|}{} & \textbf{CS} & \textbf{SSMD} & \textbf{GS} & \multicolumn{1}{c|}{\textbf{NDD}} & \textbf{CS} & \textbf{SSMD} & \textbf{GS} & \multicolumn{1}{c}{\textbf{NDD}}\\
\multicolumn{1}{c|}{} & $\uparrow$ & $\downarrow$ & $\uparrow$ & \multicolumn{1}{c|}{$\downarrow$} & $\uparrow$ & $\downarrow$ & $\uparrow$ & \multicolumn{1}{c}{$\downarrow$} \\ \hline \hline
% \multicolumn{9}{c}{\textbf{Melody Harmonization}} \\ \hline \hline
\multicolumn{1}{l|}{\textbf{NoPE}} & \phantom{0}2.68 $\pm$ 0.16 & 29.31 $\pm$ 0.01 & \phantom{0}7.82 $\pm$ 0.12 & \multicolumn{1}{c|}{103.22 $\pm$ 0.01} & \phantom{0}2.67 $\pm$ 0.18 & 27.60 $\pm$ 0.00 & \phantom{0}9.79 $\pm$ 0.55 & \multicolumn{1}{c}{102.53 $\pm$ 0.05} \\
\multicolumn{1}{l|}{\textbf{S S-RPE}} & 14.52 $\pm$ 0.67 & 28.92 $\pm$ 0.06 & 21.58 $\pm$ 0.86 & \multicolumn{1}{c|}{\phantom{0}98.86 $\pm$ 0.39} & $\diagdown$ & $\diagdown$ & $\diagdown$ & \multicolumn{1}{c}{$\diagdown$} \\
\multicolumn{1}{l|}{\textbf{SPE} \cite{liutkus_relative_2021}} & \phantom{0}1.07 $\pm$ 0.04 & 29.33 $\pm$ 0.00 & \phantom{0}6.02 $\pm$ 0.02 & \multicolumn{1}{c|}{103.81 $\pm$ 0.02} & \phantom{0}6.71 $\pm$ 0.28 & 27.59 $\pm$ 0.00 & 20.27 $\pm$ 0.65 & \multicolumn{1}{c}{101.69 $\pm$ 0.06}\\ \hline
\multicolumn{1}{l|}{\textbf{F-StrIPE}}  & 11.84 $\pm$ 1.24 & 29.18 $\pm$ 0.03 & 18.62 $\pm$ 1.44 & \multicolumn{1}{c|}{100.78 $\pm$ 0.43} & \phantom{0}9.13 $\pm$ 1.19 & 27.53 $\pm$ 0.00 & 27.42 $\pm$ 2.90 & \multicolumn{1}{c}{100.79 $\pm$ 0.21} \\
\multicolumn{1}{l|}{\textbf{F-StrIPE:M}} & \phantom{00}1.9 $\pm$ 0.09 & 29.31 $\pm$ 0.01 & \phantom{0}7.07 $\pm$ 0.13 & \multicolumn{1}{c|}{103.48 $\pm$ 0.03} & \phantom{0}2.22 $\pm$ 0.13 & 27.60 $\pm$ 0.00 & \phantom{0}8.43 $\pm$ 0.50 & \multicolumn{1}{c}{102.69 $\pm$ 0.01} \\
\multicolumn{1}{l|}{\textbf{F-StrIPE:C}} & 16.61 $\pm$ 1.52 & 28.71 $\pm$ 0.07 & 23.19 $\pm$ 2.71 & \multicolumn{1}{c|}{\phantom{0}97.14 $\pm$ 0.37} & 13.29 $\pm$ 1.08 & 27.15 $\pm$ 0.10 & 37.80 $\pm$ 2.74 & \multicolumn{1}{c}{\phantom{0}97.08 $\pm$ 0.34} \\
\multicolumn{1}{l|}{\textbf{F-StrIPE:P}} & \phantom{0}2.07 $\pm$ 0.10 & 29.31 $\pm$ 0.01 & \phantom{0}7.23 $\pm$ 0.14 & \multicolumn{1}{c|}{103.43 $\pm$ 0.02} & \phantom{0}2.31 $\pm$ 0.09 & 27.60 $\pm$ 0.00 & \phantom{0}9.11 $\pm$ 0.27 & \multicolumn{1}{c}{102.64 $\pm$ 0.02} \\ \hline
\multicolumn{1}{l|}{\textbf{F-StrIPE:SFF}} & \phantom{0}2.75 $\pm$ 3.01 & 29.32 $\pm$ 0.03 & \phantom{0}8.15 $\pm$ 3.46 & \multicolumn{1}{c|}{103.43 $\pm$ 0.74} & \phantom{0}6.49 $\pm$ 0.34 & 27.58 $\pm$ 0.00 & 20.50 $\pm$ 0.65 & \multicolumn{1}{c}{101.69 $\pm$ 0.10} \\
\multicolumn{1}{l|}{\textbf{F-StrIPE:SFF:C}} & \phantom{0}4.72 $\pm$ 3.86 & 29.25 $\pm$ 0.07 & 10.22 $\pm$ 4.63 & \multicolumn{1}{c|}{102.54 $\pm$ 1.26} & 10.35 $\pm$ 0.95 & 27.44 $\pm$ 0.05 & 29.81 $\pm$ 1.94 & \multicolumn{1}{c}{\phantom{0}99.88 $\pm$ 0.50} \\ \hline
\end{tabular}
}
\caption{Performance on melody harmonization. F-StrIPE:(M/C/P) are the ablations on F-StrIPE, described in Section \ref{sssection:baselines}, that apply RFF on only one structural level at a time - melodic pitch/chord/phrase. F-StrIPE:SFF:C applies SFF on chords, which is the best performing ablation setting with RFF.}
\label{tab:results_unrounded}
\end{table*}
% Please add the following required packages to your document preamble:
% \usepackage{multirow}
\begin{table*}[t]
\centering
\resizebox{2.05\columnwidth}{!}{%
\begin{tabular}{lcccccccc}
\hline
\multicolumn{1}{c|}{\multirow{3}{*}{\textbf{Method}}} & \multicolumn{4}{c|}{\textbf{Train = 16 bars; Test = 16 bars}} & \multicolumn{4}{c}{\textbf{Train = 16 bars; Test = 64 bars}} \\
\multicolumn{1}{c|}{} & \textbf{CS} & \textbf{SSMD} & \textbf{GS} & \multicolumn{1}{c|}{\textbf{NDD}} & \textbf{CS} & \textbf{SSMD} & \textbf{GS} & \multicolumn{1}{c}{\textbf{NDD}}\\
\multicolumn{1}{c|}{} & $\uparrow$ & $\downarrow$ & $\uparrow$ & \multicolumn{1}{c|}{$\downarrow$} & $\uparrow$ & $\downarrow$ & $\uparrow$ & \multicolumn{1}{c}{$\downarrow$} \\ \hline \hline
% \multicolumn{9}{c}{\textbf{Melody Harmonization}} \\ \hline \hline
\multicolumn{1}{l|}{\textbf{NoPE}} & \phantom{0}2.68 $\pm$ 0.2\phantom{*} & 29.31 $\pm$ 0.0* & \phantom{0}7.82 $\pm$ 0.1\phantom{*} & \multicolumn{1}{c|}{93.94 $\pm$ 0.0*} & \phantom{0}2.67 $\pm$ 0.2 & 27.60 $\pm$ 0.0* & \phantom{0}7.80 $\pm$ 0.2 & \multicolumn{1}{c}{92.49 $\pm$ 0.1\phantom{*}} \\
\multicolumn{1}{l|}{\textbf{S S-RPE}} & 14.52 $\pm$ 0.7\phantom{*} & 28.92 $\pm$ 0.1\phantom{*} & 21.58 $\pm$ 0.9\phantom{*} & \multicolumn{1}{c|}{88.48 $\pm$ 0.5\phantom{*}} & $\diagdown$ & $\diagdown$ & $\diagdown$ & \multicolumn{1}{c}{$\diagdown$} \\
\multicolumn{1}{l|}{\textbf{SPE} \cite{liutkus_relative_2021}} & \phantom{0}1.07 $\pm$ 0.0* & 29.33 $\pm$ 0.0* & \phantom{0}6.02 $\pm$ 0.0* & \multicolumn{1}{c|}{94.65 $\pm$ 0.0*} & \phantom{0}6.71 $\pm$ 0.3 & 27.59 $\pm$ 0.0* & 12.62 $\pm$ 0.3 & \multicolumn{1}{c}{91.48 $\pm$ 0.1\phantom{*}}\\ \hline
\multicolumn{1}{l|}{\textbf{F-StrIPE}}  & 11.84 $\pm$ 1.2\phantom{*} & 29.18 $\pm$ 0.0* & 18.62 $\pm$ 1.4\phantom{*} & \multicolumn{1}{c|}{90.93 $\pm$ 0.6\phantom{*}} & \phantom{0}9.13 $\pm$ 1.2 & 27.53 $\pm$ 0.0* & 14.70 $\pm$ 1.7 & \multicolumn{1}{c}{90.30 $\pm$ 0.3\phantom{*}} \\
\multicolumn{1}{l|}{\textbf{F-StrIPE:M}} & \phantom{00}1.9 $\pm$ 0.1\phantom{*} & 29.31 $\pm$ 0.0* & \phantom{0}7.07 $\pm$ 0.1\phantom{*} & \multicolumn{1}{c|}{94.27 $\pm$ 0.0*} & \phantom{0}2.22 $\pm$ 0.1 & 27.60 $\pm$ 0.0* & \phantom{0}7.42 $\pm$ 0.2 & \multicolumn{1}{c}{92.69 $\pm$ 0.0*} \\
\multicolumn{1}{l|}{\textbf{F-StrIPE:C}} & \textbf{16.61 $\pm$ 1.5\phantom{*}} & \textbf{28.71 $\pm$ 0.1\phantom{*}} & \textbf{23.19 $\pm$ 2.7\phantom{*}} & \multicolumn{1}{c|}{\textbf{86.42 $\pm$ 0.4\phantom{*}}} & \textbf{13.29 $\pm$ 1.1} & \textbf{27.15 $\pm$ 0.1\phantom{*}} & \textbf{16.73 $\pm$ 0.9} & \multicolumn{1}{c}{\textbf{85.61 $\pm$ 0.4\phantom{*}}} \\
\multicolumn{1}{l|}{\textbf{F-StrIPE:P}} & \phantom{0}2.07 $\pm$ 0.1\phantom{*} & 29.31 $\pm$ 0.0* & \phantom{0}7.23 $\pm$ 0.1\phantom{*} & \multicolumn{1}{c|}{94.20 $\pm$ 0.0*} & \phantom{0}2.31 $\pm$ 0.1 & 27.60 $\pm$ 0.0* & \phantom{0}7.56 $\pm$ 0.1 & \multicolumn{1}{c}{92.62 $\pm$ 0.0*} \\ \hline
\multicolumn{1}{l|}{\textbf{F-StrIPE:SFF}} & \phantom{0}2.75 $\pm$ 3.0\phantom{*} & 29.32 $\pm$ 0.0* & \phantom{0}8.15 $\pm$ 3.5\phantom{*} & \multicolumn{1}{c|}{94.18 $\pm$ 0.9\phantom{*}} & \phantom{0}6.49 $\pm$ 0.3 & 27.58 $\pm$ 0.0* & 12.09 $\pm$ 0.7 & \multicolumn{1}{c}{91.46 $\pm$ 0.1\phantom{*}} \\ %\hline
\multicolumn{1}{l|}{\textbf{F-StrIPE:SFF:C}} & \phantom{0}4.72 $\pm$ 3.9\phantom{*} & 29.25 $\pm$ 0.1\phantom{*} & 10.22 $\pm$ 4.6\phantom{*} & \multicolumn{1}{c|}{93.11 $\pm$ 1.5\phantom{*}} & 10.35 $\pm$ 1.0 & 27.44 $\pm$ 0.1\phantom{*} & 12.09 $\pm$ 0.7 & \multicolumn{1}{c}{89.32 $\pm$ 0.6\phantom{*}} \\ \hline
\end{tabular}
}
\caption{Performance on melody harmonization. F-StrIPE:(M/C/P) are the ablations on F-StrIPE, described in Section \ref{sssection:baselines}, that apply RFF on only one structural level at a time - melodic pitch/chord/phrase. F-StrIPE:SFF:C applies SFF on chords, which is the best performing ablation setting with RFF. `0.0*' refers to standard deviations that are lower than 0.05. `$\diagdown$' refers to simulations where the given inference setting could not be accessed due to the heavy computational demands of the method.}
\label{tab:results_unrounded}
\end{table*}

\begin{table}[ht!]
\centering
\caption{\textbf{Super Resolution Performance Results.} Our proposed WGAN EEG Spatial Upsampling method significantly outperforms a baseline of Bicubic Interpolation commonly used in EEG upsampling pipelines.}
\label{tab:results}
\resizebox{0.8\linewidth}{!}{%
\begin{tabular}{@{}cccccc@{}}
\toprule
\multirow{2}{*}{\textbf{Dataset}} & \multirow{2}{*}{\textbf{Scale}} & \multicolumn{2}{c}{\textbf{Bicubic}} & \multicolumn{2}{c}{\textbf{WGAN}} \\ \cmidrule(l){3-6} 
                      &   & \textbf{MSE} & \textbf{MAE} & \textbf{MSE}    & \textbf{MAE}   \\
\toprule
\multirow{2}{*}{Val}  & 2 & 3.71E7       & 3.89E3       & \textbf{2.01E3} & \textbf{24.38} \\
                      & 4 & 7.23E7       & 6.42E3       & \textbf{8.53E3} & \textbf{63.83} \\
\midrule
\multirow{2}{*}{Test} & 2 & 3.75E7       & 3.91E3       & \textbf{2.06E3} & \textbf{24.66} \\
                      & 4 & 7.30E7       & 6.45E3       & \textbf{8.68E3} & \textbf{64.39} \\
\bottomrule
\end{tabular}%
}
\end{table}

\section*{Conclusion}
This paper aims to enhance our understanding of the computational complexity of computing various Shapley value variants. We found that for various ML models --- including decision trees, regression tree ensembles, weighted automata, and linear regression --- both local and global interventional and baseline SHAP can be computed in polynomial time under HMM modeled distributions. This extends popular algorithms, such as TreeSHAP, beyond their empirical distributional scope. We also establish strict complexity gaps between the various SHAP variants (baseline, interventional, and conditional) and prove the intractability of computing SHAP for tree ensembles and neural networks in simplified scenarios. Overall, we present SHAP as a versatile framework whose complexity depends on four key factors: \begin{inparaenum}[(i)] \item model type, \item SHAP variant, \item distribution modeling approach, \item and local vs. global explanations\end{inparaenum}. We believe this perspective provides deeper insight into the computational complexity of SHAP, paving the way for future work.




%We believe that our framework provides a more intricate understanding of SHAP computation complexity across different models, distributions, and variants, paving the way for further research.

Our work opens promising directions for future research. First, expanding our computational analysis to other SHAP-related metrics, such as asymmetric SHAP~\citep{frye20} and SAGE~\citep{covert2020understanding}, would be valuable. Additionally, we aim to explore more expressive distribution classes and relaxed assumptions beyond those in Section \ref{sec:tractable} while maintaining tractable SHAP computation. Finally, when exact computation is intractable (Section \ref{sec:intractable}), investigating the approximability of SHAP metrics through approximation and parameterized complexity theory~\citep{downey2012parameterized} is an important direction.

%Our work opens several promising avenues for future research on the computational properties of explainable AI methods, with a particular focus on SHAP. First, it would be interesting to broaden the computational analysis conducted in this work to include other popular SHAP-related metrics in the literature, such as asymmetric SHAP \cite{frye20} and SAGE \cite{covert2020understanding}. Also, in the future, we aim to explore more expressive distribution classes and relaxed distributional assumptions—extending beyond those examined in Section \ref{sec:tractable} —that still yield tractable SHAP computation. Finally, when exact computation proves intractable (Section \ref{sec:intractable}), it is worthwhile to theoretically investigate the question of the approximability of computing the SHAP metrics across various configurations, through the lens of approximation and parametrized complexity theory \cite{arora2009computational}.

%This paper aims to deepen our understanding of the computational complexity involved in obtaining different Shapley value variants. We found that for a variety of ML models, including decision trees, tree ensembles for regression, weighted automata, and linear regression models — computing both local and global interventional and baseline SHAP can be done in polynomial time when distributions are modeled by HMMs. This extends the distributional scope of popular algorithms like TreeSHAP, which is limited to empirical distributions. Additionally, we demonstrate a strict complexity gap between SHAP variants, showing that interventional and baseline SHAP can be strictly easier to compute than conditional SHAP. Despite these positive results, we uncovered intractability for various SHAP variants in neural networks and tree ensembles. Finally, we provided generalized complexity relations across SHAP variants. We believe that our framework offers a deeper understanding of the complexity involved in computing SHAP across various variants, models, distributions, as well as in both local and global computations, laying the groundwork for future research.

\bibliographystyle{./IEEEbib/IEEEtran}
\bibliography{strings}


\end{document}
