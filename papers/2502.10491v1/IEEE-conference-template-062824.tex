\documentclass[conference, 9pt]{IEEEtran}
\IEEEoverridecommandlockouts
% The preceding line is only needed to identify funding in the first footnote. If that is unneeded, please comment it out.
%Template version as of 6/27/2024

\usepackage{cite}
\usepackage{amsmath,amssymb,amsfonts}
\usepackage{graphicx}
\usepackage{color, xcolor}
\usepackage{lipsum}
\usepackage{comment}
    \newcommand{\thicktilde}[1]{\mathbf{\tilde{\text{$#1$}}}}
\usepackage{tikz}
  \newlength\squareheight
  \setlength\squareheight{6.75pt}
  \newcommand\squareslash{\tikz{\draw (0,0) rectangle (\squareheight,\squareheight);\draw(0,0) -- (\squareheight,\squareheight)}}
  \DeclareMathOperator\squarediv{\squareslash}
\usepackage{multirow}
\usepackage{float}
\usepackage{subcaption}
\usepackage{siunitx}
\usepackage{oubraces}
\usepackage{algorithmic}
\usepackage{graphicx}
\usepackage{textcomp}
\usepackage{xcolor}
% \usepackage{dutchcal}
% \usepackage{mathrsfs}
\usepackage[hang,flushmargin]{footmisc}
\usepackage[cal=boondoxupr]{mathalfa}
\def\BibTeX{{\rm B\kern-.05em{\sc i\kern-.025em b}\kern-.08em
    T\kern-.1667em\lower.7ex\hbox{E}\kern-.125emX}}
\newcommand\blfootnote[1]{%
  \begingroup
  \renewcommand\thefootnote{}\footnote{#1}%
  \addtocounter{footnote}{-1}%
  \endgroup
}
\newcommand{\gael}[1]{\textit{\textcolor{red}{#1}}}
\newcommand{\changhong}[1]{\textcolor{blue}{#1}}

\usepackage{fancyhdr}
\usepackage{ragged2e}  % For \justifying
% Define a custom style for the first page
\fancypagestyle{firstpage}{
% \vspace{-0.5cm}
    \fancyhf{} % Clear all headers and footers
    % Adjust the footskip to move the footer text higher
    \setlength{\footskip}{28pt}  % Reduce this value to move the footer up
    \fancyfoot[C]{\justifying \footnotesize{\copyright 2025 IEEE. Personal use of this material is permitted. However, permission to reprint/republish this material for advertising or promotional purposes or for creating new collective works for resale or redistribution to servers or lists, or to reuse any copyrighted component of this work in other works, must be obtained from the IEEE.}} % Footer text
    \renewcommand{\headrulewidth}{0pt} % No header line
    \renewcommand{\footrulewidth}{0pt} % No footer line
}

% Set the default style for the rest of the pages
\pagestyle{plain}

\begin{document}

\title{F-StrIPE: Fast Structure-Informed Positional Encoding for Symbolic Music Generation\\
\thanks{This work was funded by the European Union (ERC, HI-Audio, 101052978). Views and opinions expressed are however those of the author(s) only and do not necessarily reflect those of the European Union or the European Research Council. Neither the European Union nor the granting authority can be held responsible for them. The accompanying website at \textit{bit.ly/faststructurepe} contains music samples and code.}
}

% \author{\IEEEauthorblockN{1\textsuperscript{st} Manvi Agarwal}
% \IEEEauthorblockA{
% \textit{T\'{e}l\'{e}com Paris}\\
% \textit{Institut Polytechnique de Paris}\\
% Paris, France \\
% email address or ORCID}
% \and
% \IEEEauthorblockN{2\textsuperscript{nd} Changhong Wang}
% \IEEEauthorblockA{
% \textit{T\'{e}l\'{e}com Paris}\\
% \textit{Institut Polytechnique de Paris}\\
% Paris, France \\
% email address or ORCID}
% \and
% \IEEEauthorblockN{3\textsuperscript{rd} Ga\"{e}l Richard}
% \IEEEauthorblockA{
% \textit{T\'{e}l\'{e}com Paris}\\
% \textit{Institut Polytechnique de Paris}\\
% Paris, France \\
% email address or ORCID}
% }
% \author{\IEEEauthorblockN{Manvi Agarwal, %\IEEEauthorrefmark{1}, 
% Changhong Wang, %\IEEEauthorrefmark{2}, 
% Ga\"{e}l Richard%\IEEEauthorrefmark{3}, 
% %Author 4\IEEEauthorrefmark{3}, Author 5\IEEEauthorrefmark{4}
% }
% \IEEEauthorblockA{T\'{e}l\'{e}com Paris, Institut Polytechnique de Paris, France}
% }

\author{\IEEEauthorblockN{Manvi Agarwal, 
Changhong Wang, 
Ga\"{e}l Richard 
%Author 4\IEEEauthorrefmark{3}, Author 5\IEEEauthorrefmark{4}
}
\IEEEauthorblockA{LTCI, T\'{e}l\'{e}com Paris, Institut Polytechnique de Paris, France}
}


% \author{\IEEEauthorblockN{Michael Shell\IEEEauthorre fmark{1}, Homer Simpson\IEEEauthorrefmark{2}, James K irk\IEEEauthorrefmark{3}, Montgomery Scott\IEEEautho rrefmark{3} and Eldon Tyrell\IEEEauthorrefmark{4}} \IEEEauthorblockA{\IEEEauthorrefmark{1}School of Ele ctrical and Computer Engineering\\
% Georgia Institute of Technology, Atlanta, Georgia 30 332--0250\\
% Email: mshell@ece.gatech.edu} \IEEEauthorblockA{\IEEEauthorrefmark{2}Twentieth Cen tury Fox, Springfield, USA\\
% Email: homer@thesimpsons.com}
% \IEEEauthorblockA{\IEEEauthorrefmark{3}Starfleet Aca demy, San Francisco, California 96678-2391\\ Telephone: (800) 555--1212, Fax: (888) 555--1212} \IEEEauthorblockA{\IEEEauthorrefmark{4}Tyrell Inc.,
% 123 Replicant Street, Los Angeles, California 90210 --4321}}


\maketitle
\thispagestyle{firstpage}
\begin{abstract}
While music remains a challenging domain for generative models like Transformers, recent progress has been made by exploiting suitable musically-informed priors. One technique to leverage information about musical structure in Transformers is inserting such knowledge into the positional encoding (PE) module. However, Transformers carry a quadratic cost in sequence length. In this paper, we propose \emph{F-StrIPE}, a structure-informed PE scheme that works in linear complexity. Using existing kernel approximation techniques based on random features, we show that F-StrIPE is a generalization of Stochastic Positional Encoding (SPE). %\textit{We compare SPE and F-StrIPE from the perspective of musical structure, showing that our approach is appropriately grounded in both theory and domain knowledge. }
We illustrate the empirical merits of F-StrIPE using melody harmonization for symbolic music.
% \blfootnote{\copyright 2024 IEEE. \textit{Personal use of this material is permitted. Permission from IEEE must be obtained for all other uses, in any current or future media, including reprinting/republishing this material for advertising or promotional purposes, creating new collective works, for resale or redistribution to servers or lists, or reuse of any copyrighted component of this work in other works.}}
\end{abstract}

\begin{IEEEkeywords}
music generation, symbolic music, transformers, positional encoding, kernels.
\end{IEEEkeywords}

\section{Introduction}
\label{sec:introduction}
The business processes of organizations are experiencing ever-increasing complexity due to the large amount of data, high number of users, and high-tech devices involved \cite{martin2021pmopportunitieschallenges, beerepoot2023biggestbpmproblems}. This complexity may cause business processes to deviate from normal control flow due to unforeseen and disruptive anomalies \cite{adams2023proceddsriftdetection}. These control-flow anomalies manifest as unknown, skipped, and wrongly-ordered activities in the traces of event logs monitored from the execution of business processes \cite{ko2023adsystematicreview}. For the sake of clarity, let us consider an illustrative example of such anomalies. Figure \ref{FP_ANOMALIES} shows a so-called event log footprint, which captures the control flow relations of four activities of a hypothetical event log. In particular, this footprint captures the control-flow relations between activities \texttt{a}, \texttt{b}, \texttt{c} and \texttt{d}. These are the causal ($\rightarrow$) relation, concurrent ($\parallel$) relation, and other ($\#$) relations such as exclusivity or non-local dependency \cite{aalst2022pmhandbook}. In addition, on the right are six traces, of which five exhibit skipped, wrongly-ordered and unknown control-flow anomalies. For example, $\langle$\texttt{a b d}$\rangle$ has a skipped activity, which is \texttt{c}. Because of this skipped activity, the control-flow relation \texttt{b}$\,\#\,$\texttt{d} is violated, since \texttt{d} directly follows \texttt{b} in the anomalous trace.
\begin{figure}[!t]
\centering
\includegraphics[width=0.9\columnwidth]{images/FP_ANOMALIES.png}
\caption{An example event log footprint with six traces, of which five exhibit control-flow anomalies.}
\label{FP_ANOMALIES}
\end{figure}

\subsection{Control-flow anomaly detection}
Control-flow anomaly detection techniques aim to characterize the normal control flow from event logs and verify whether these deviations occur in new event logs \cite{ko2023adsystematicreview}. To develop control-flow anomaly detection techniques, \revision{process mining} has seen widespread adoption owing to process discovery and \revision{conformance checking}. On the one hand, process discovery is a set of algorithms that encode control-flow relations as a set of model elements and constraints according to a given modeling formalism \cite{aalst2022pmhandbook}; hereafter, we refer to the Petri net, a widespread modeling formalism. On the other hand, \revision{conformance checking} is an explainable set of algorithms that allows linking any deviations with the reference Petri net and providing the fitness measure, namely a measure of how much the Petri net fits the new event log \cite{aalst2022pmhandbook}. Many control-flow anomaly detection techniques based on \revision{conformance checking} (hereafter, \revision{conformance checking}-based techniques) use the fitness measure to determine whether an event log is anomalous \cite{bezerra2009pmad, bezerra2013adlogspais, myers2018icsadpm, pecchia2020applicationfailuresanalysispm}. 

The scientific literature also includes many \revision{conformance checking}-independent techniques for control-flow anomaly detection that combine specific types of trace encodings with machine/deep learning \cite{ko2023adsystematicreview, tavares2023pmtraceencoding}. Whereas these techniques are very effective, their explainability is challenging due to both the type of trace encoding employed and the machine/deep learning model used \cite{rawal2022trustworthyaiadvances,li2023explainablead}. Hence, in the following, we focus on the shortcomings of \revision{conformance checking}-based techniques to investigate whether it is possible to support the development of competitive control-flow anomaly detection techniques while maintaining the explainable nature of \revision{conformance checking}.
\begin{figure}[!t]
\centering
\includegraphics[width=\columnwidth]{images/HIGH_LEVEL_VIEW.png}
\caption{A high-level view of the proposed framework for combining \revision{process mining}-based feature extraction with dimensionality reduction for control-flow anomaly detection.}
\label{HIGH_LEVEL_VIEW}
\end{figure}

\subsection{Shortcomings of \revision{conformance checking}-based techniques}
Unfortunately, the detection effectiveness of \revision{conformance checking}-based techniques is affected by noisy data and low-quality Petri nets, which may be due to human errors in the modeling process or representational bias of process discovery algorithms \cite{bezerra2013adlogspais, pecchia2020applicationfailuresanalysispm, aalst2016pm}. Specifically, on the one hand, noisy data may introduce infrequent and deceptive control-flow relations that may result in inconsistent fitness measures, whereas, on the other hand, checking event logs against a low-quality Petri net could lead to an unreliable distribution of fitness measures. Nonetheless, such Petri nets can still be used as references to obtain insightful information for \revision{process mining}-based feature extraction, supporting the development of competitive and explainable \revision{conformance checking}-based techniques for control-flow anomaly detection despite the problems above. For example, a few works outline that token-based \revision{conformance checking} can be used for \revision{process mining}-based feature extraction to build tabular data and develop effective \revision{conformance checking}-based techniques for control-flow anomaly detection \cite{singh2022lapmsh, debenedictis2023dtadiiot}. However, to the best of our knowledge, the scientific literature lacks a structured proposal for \revision{process mining}-based feature extraction using the state-of-the-art \revision{conformance checking} variant, namely alignment-based \revision{conformance checking}.

\subsection{Contributions}
We propose a novel \revision{process mining}-based feature extraction approach with alignment-based \revision{conformance checking}. This variant aligns the deviating control flow with a reference Petri net; the resulting alignment can be inspected to extract additional statistics such as the number of times a given activity caused mismatches \cite{aalst2022pmhandbook}. We integrate this approach into a flexible and explainable framework for developing techniques for control-flow anomaly detection. The framework combines \revision{process mining}-based feature extraction and dimensionality reduction to handle high-dimensional feature sets, achieve detection effectiveness, and support explainability. Notably, in addition to our proposed \revision{process mining}-based feature extraction approach, the framework allows employing other approaches, enabling a fair comparison of multiple \revision{conformance checking}-based and \revision{conformance checking}-independent techniques for control-flow anomaly detection. Figure \ref{HIGH_LEVEL_VIEW} shows a high-level view of the framework. Business processes are monitored, and event logs obtained from the database of information systems. Subsequently, \revision{process mining}-based feature extraction is applied to these event logs and tabular data input to dimensionality reduction to identify control-flow anomalies. We apply several \revision{conformance checking}-based and \revision{conformance checking}-independent framework techniques to publicly available datasets, simulated data of a case study from railways, and real-world data of a case study from healthcare. We show that the framework techniques implementing our approach outperform the baseline \revision{conformance checking}-based techniques while maintaining the explainable nature of \revision{conformance checking}.

In summary, the contributions of this paper are as follows.
\begin{itemize}
    \item{
        A novel \revision{process mining}-based feature extraction approach to support the development of competitive and explainable \revision{conformance checking}-based techniques for control-flow anomaly detection.
    }
    \item{
        A flexible and explainable framework for developing techniques for control-flow anomaly detection using \revision{process mining}-based feature extraction and dimensionality reduction.
    }
    \item{
        Application to synthetic and real-world datasets of several \revision{conformance checking}-based and \revision{conformance checking}-independent framework techniques, evaluating their detection effectiveness and explainability.
    }
\end{itemize}

The rest of the paper is organized as follows.
\begin{itemize}
    \item Section \ref{sec:related_work} reviews the existing techniques for control-flow anomaly detection, categorizing them into \revision{conformance checking}-based and \revision{conformance checking}-independent techniques.
    \item Section \ref{sec:abccfe} provides the preliminaries of \revision{process mining} to establish the notation used throughout the paper, and delves into the details of the proposed \revision{process mining}-based feature extraction approach with alignment-based \revision{conformance checking}.
    \item Section \ref{sec:framework} describes the framework for developing \revision{conformance checking}-based and \revision{conformance checking}-independent techniques for control-flow anomaly detection that combine \revision{process mining}-based feature extraction and dimensionality reduction.
    \item Section \ref{sec:evaluation} presents the experiments conducted with multiple framework and baseline techniques using data from publicly available datasets and case studies.
    \item Section \ref{sec:conclusions} draws the conclusions and presents future work.
\end{itemize}

\begin{figure}[t]
    \centering
    \includegraphics[width=1.0\linewidth]{figs/icassp_2025_summary_palette_corrected.png}
    \caption{A schematic showing our main contributions (best viewed in colour).}
    \label{fig:contributions}
\end{figure}

% \begin{figure*}[h!]
%     \centering
%     \includegraphics[width=1.6\columnwidth]{figs/R.png}
%     \caption{A visual representation of the connection between Stochastic Positional Encoding and Random Fourier Features. The variables referenced here are detailed in Sections \ref{section:background} and \ref{section:methods}.}
%     \label{fig:visualize_matrices}
% \end{figure*}

\begin{figure*}[h!]
    \centering
    \includegraphics[width=1.58\columnwidth]{figs/icassp_2025_palette_corrected.png}
    \caption{A visual representation of the connection between Stochastic Positional Encoding and Random Fourier Features (best viewed in colour). The variables referenced here are detailed in Sections \ref{section:background} and \ref{section:methods}.}
    \label{fig:visualize_matrices}
\end{figure*}

\section{Background}\label{sec:backgrnd}

\subsection{Cold Start Latency and Mitigation Techniques}

Traditional FaaS platforms mitigate cold starts through snapshotting, lightweight virtualization, and warm-state management. Snapshot-based methods like \textbf{REAP} and \textbf{Catalyzer} reduce initialization time by preloading or restoring container states but require significant memory and I/O resources, limiting scalability~\cite{dong_catalyzer_2020, ustiugov_benchmarking_2021}. Lightweight virtualization solutions, such as \textbf{Firecracker} microVMs, achieve fast startup times with strong isolation but depend on robust infrastructure, making them less adaptable to fluctuating workloads~\cite{agache_firecracker_2020}. Warm-state management techniques like \textbf{Faa\$T}~\cite{romero_faa_2021} and \textbf{Kraken}~\cite{vivek_kraken_2021} keep frequently invoked containers ready, balancing readiness and cost efficiency under predictable workloads but incurring overhead when demand is erratic~\cite{romero_faa_2021, vivek_kraken_2021}. While these methods perform well in resource-rich cloud environments, their resource intensity challenges applicability in edge settings.

\subsubsection{Edge FaaS Perspective}

In edge environments, cold start mitigation emphasizes lightweight designs, resource sharing, and hybrid task distribution. Lightweight execution environments like unikernels~\cite{edward_sock_2018} and \textbf{Firecracker}~\cite{agache_firecracker_2020}, as used by \textbf{TinyFaaS}~\cite{pfandzelter_tinyfaas_2020}, minimize resource usage and initialization delays but require careful orchestration to avoid resource contention. Function co-location, demonstrated by \textbf{Photons}~\cite{v_dukic_photons_2020}, reduces redundant initializations by sharing runtime resources among related functions, though this complicates isolation in multi-tenant setups~\cite{v_dukic_photons_2020}. Hybrid offloading frameworks like \textbf{GeoFaaS}~\cite{malekabbasi_geofaas_2024} balance edge-cloud workloads by offloading latency-tolerant tasks to the cloud and reserving edge resources for real-time operations, requiring reliable connectivity and efficient task management. These edge-specific strategies address cold starts effectively but introduce challenges in scalability and orchestration.

\subsection{Predictive Scaling and Caching Techniques}

Efficient resource allocation is vital for maintaining low latency and high availability in serverless platforms. Predictive scaling and caching techniques dynamically provision resources and reduce cold start latency by leveraging workload prediction and state retention.
Traditional FaaS platforms use predictive scaling and caching to optimize resources, employing techniques (OFC, FaasCache) to reduce cold starts. However, these methods rely on centralized orchestration and workload predictability, limiting their effectiveness in dynamic, resource-constrained edge environments.



\subsubsection{Edge FaaS Perspective}

Edge FaaS platforms adapt predictive scaling and caching techniques to constrain resources and heterogeneous environments. \textbf{EDGE-Cache}~\cite{kim_delay-aware_2022} uses traffic profiling to selectively retain high-priority functions, reducing memory overhead while maintaining readiness for frequent requests. Hybrid frameworks like \textbf{GeoFaaS}~\cite{malekabbasi_geofaas_2024} implement distributed caching to balance resources between edge and cloud nodes, enabling low-latency processing for critical tasks while offloading less critical workloads. Machine learning methods, such as clustering-based workload predictors~\cite{gao_machine_2020} and GRU-based models~\cite{guo_applying_2018}, enhance resource provisioning in edge systems by efficiently forecasting workload spikes. These innovations effectively address cold start challenges in edge environments, though their dependency on accurate predictions and robust orchestration poses scalability challenges.

\subsection{Decentralized Orchestration, Function Placement, and Scheduling}

Efficient orchestration in serverless platforms involves workload distribution, resource optimization, and performance assurance. While traditional FaaS platforms rely on centralized control, edge environments require decentralized and adaptive strategies to address unique challenges such as resource constraints and heterogeneous hardware.



\subsubsection{Edge FaaS Perspective}

Edge FaaS platforms adopt decentralized and adaptive orchestration frameworks to meet the demands of resource-constrained environments. Systems like \textbf{Wukong} distribute scheduling across edge nodes, enhancing data locality and scalability while reducing network latency. Lightweight frameworks such as \textbf{OpenWhisk Lite}~\cite{kravchenko_kpavelopenwhisk-light_2024} optimize resource allocation by decentralizing scheduling policies, minimizing cold starts and latency in edge setups~\cite{benjamin_wukong_2020}. Hybrid solutions like \textbf{OpenFaaS}~\cite{noauthor_openfaasfaas_2024} and \textbf{EdgeMatrix}~\cite{shen_edgematrix_2023} combine edge-cloud orchestration to balance resource utilization, retaining latency-sensitive functions at the edge while offloading non-critical workloads to the cloud. While these approaches improve flexibility, they face challenges in maintaining coordination and ensuring consistent performance across distributed nodes.



\section{Methods} \label{section:methods}

\subsection{Structure-informed positional encoding}

As we hinted in Section \ref{section:background}, positional encoding depends on a sequence $\mathcal{P} = [ p_1, p_2, ... , p_T ]$ of positional indices. In standard PE, which does not utilize structure, $p_i = i$, which makes $\mathcal{P}$ a linear grid. When structure is included in PE, $p_i = s_\ell (i)$, where $s_\ell (i)$ gives the structural label at level $\ell$ (e.g. chord) for timestep $i$. In this case, $s_\ell (i) = s_\ell (i^{\prime})$ is possible for $i \neq i^{\prime}$, making $\mathcal{P}$ a non-linear grid. In fact, we can consider vectorial positional indices with multiple resolutions of structural organization. Consequently, we obtain $p_i = \mathbf{s} (i)$, where $\mathbf{s} (i) = [ s_1 (i) , ..., s_\ell (i), ..., s_L (i) ]$ is a vector of $L$ structural labels. Viewed in this way, PEs with structure can simply be used as a drop-in replacement for PEs without structure by replacing the form of $p_i$, giving us richer positional information. This is a way to flexibly represent domain-specific prior knowledge about the underlying data domain.

\subsection{Adding structure to SPE: F-StrIPE:SFF}

In order to use rich positional information in SPE, we can augment the sinusoidal feature matrix $\boldsymbol{\Omega}$ from (\ref{eq:sff:sine_features}) to be:
\begin{equation} \label{eq:pos_sine_features}
    [\boldsymbol{\Omega}(\mathcal{P}, \boldsymbol{f}, \boldsymbol{\theta})]_{i j}= \begin{cases}\cos \left(2 \pi \mathbf{f}[\omega, :]^\top p_i+\boldsymbol{\theta}[\omega]\right) & \text { if } j = 2\omega \\ \sin \left(2 \pi \mathbf{f}[\omega, :]^\top p_i+\boldsymbol{\theta}[\omega] \right) & \text { else }\end{cases}
\end{equation}
Here, we use the vectorial formulation of structure-aware positional indices $p_i = \mathbf{s}(i)$, unlike (\ref{eq:sff:sine_features}) where $p_i = i$ was a sequence of structure-free positional indices linked solely to the passage of time. Whereas in (\ref{eq:sff:sine_features}), $\mathbf{f}[\omega]$ was a single frequency, $\mathbf{f}[\omega, :]$ in (\ref{eq:pos_sine_features}) is a vector of frequencies. Therefore, each frequency $\mathbf{f}[\omega, \ell]$ in this vector acts on the $\ell^{\text{th}}$ structural label at timestep $i$. We can combine (\ref{eq:spe:ohyeah}), (\ref{eq:sff}) and (\ref{eq:pos_sine_features}) to obtain a fast structure-aware PE technique that uses Stochastic Fourier Features. We call this method \textit{F-StrIPE:SFF}.

% Here, unlike (\ref{eq:sff:sine_features}) where $\mathbf{f}[\omega]$ was a single frequency, $\mathbf{f}[\omega, :]$ is a vector of frequencies. Each frequency $\mathbf{f}[\omega, \ell]$ in this vector acts on the $\ell^{\text{th}}$ structural label at timestep $i$, where we can use the vectorial formulation of structure-aware positional indices $p_i = \mathbf{s}(i)$. In contrast, $p_i = i$ in (\ref{eq:sff}).

\subsection{Asymptotic case of SFF: Random Fourier Features} \label{ssection:asymptotic_rff}

Using Equations (\ref{eq:spe:ohyeah}) and (\ref{eq:sff}), we can express the SFF approximation of the positional matrix $\mathbf{P}_d$ for arbitrary timesteps $m$ and $n$ as:
\begin{equation} \label{eq:sff_mn}
    \mathbf{P}_d[m, n] \approx \Big[ \Omega_\mathcal{Q}^d [m, :] (\mathbf{Z}_d \mathbf{Z}_d^\top) \Omega_\mathcal{K}^{d^\top} [:, n] \Big] / R \\
\end{equation}

where we use the abbreviation $\Omega_\mathcal{A}^d = \boldsymbol{\Omega}\left(\mathcal{P}_A, \boldsymbol{f}_d, \boldsymbol{\theta}_d^A \right) \text{diag}\left(\ddot{\boldsymbol{\lambda}_d}\right)$. We observe that $\mathbf{Z}_d \mathbf{Z}_d^\top = \widehat{\mathbf{C}}_d$ acts as an empirical covariance matrix for the features $\Omega_\mathcal{Q}^d$ and $\Omega_\mathcal{K}^d$. Since $\mathbf{Z}_d$ has zero mean and unit variance, as $R \to \infty$, $\widehat{\mathbf{C}}_d$ approaches the theoretical covariance matrix $\mathbf{C}_d = \mathbf{I}_{2N_{f}}$.
In the ideal case of $\mathbf{C}_d$, (\ref{eq:sff_mn}) simplifies to $\mathbf{P}_d[m, n] \approx \Omega_\mathcal{Q}^d [m, :] \Omega_\mathcal{K}^{d^\top} [:, n]$, giving:
\begin{equation} \label{eq:ideal_C}
    \mathbf{P}_d[m, n] \approx  \frac{1}{N_f}  \sum_{\omega = 1}^{N_f} \Lambda_\omega \cos \Big( f_{\omega} ( \mathcal{P}_Q[m] - \mathcal{P}_K[n] ) + \Theta_\omega \Big)
\end{equation}
where $\Lambda_\omega$ is the gain contributed by the matrices $\text{diag}\left(\ddot{\boldsymbol{\lambda}_d}\right)$ and $\Theta_\omega$ is the phase-shift contributed by $\boldsymbol{\theta}^Q_d$ and $\boldsymbol{\theta}^K_d$. This representation has been studied in previous work, where it is called \textit{Random Fourier Features} (RFF) \cite{rahimi_random_2007, sutherland_error_2015}.

\subsection{Generalizing F-StrIPE:SFF to F-StrIPE}

Using this insight, we can redesign the positional feature matrices from (\ref{eq:sff}) to be:
\begin{equation} \label{eq:rff}
    \mathbf{P}^{Q/K}_d = \boldsymbol{\Omega}\left(\mathcal{P}_{Q/K}, \boldsymbol{f}_d, \boldsymbol{\theta}_d^{Q/K}\right) \text{diag}\left(\ddot{\boldsymbol{\lambda}_d}\right) / \sqrt{N_f}
\end{equation}
where the sinusoidal features $\boldsymbol{\Omega}$ uses structure-aware positional indices as given in (\ref{eq:pos_sine_features}). With this, we can now modify (\ref{eq:spe:ohyeah}) to use $\mathbf{P}^{Q/K}_d$ in place of $\thicktilde{\mathbf{P}}^{Q/K}_d$, giving us $\mathbf{Q}^{\text{RFF}}/\mathbf{K}^{\text{RFF}}$ in place of $\mathbf{Q}^{\text{SFF}}/\mathbf{K}^{\text{SFF}}$. To signify that such a PE technique generalizes F-StrIPE:SFF to use RFF in place of SFF, we call this method \textit{F-StrIPE}.

In the second row of Figure \ref{fig:visualize_matrices}, similar to SFF, we show the different components of RFF in the case where gains are 1 and phase shifts are 0. RFF can be understood as the ideal case of SFF where $R \to \infty$. Seen in this way, RFF gives us a noiseless estimate of $\mathbf{P}_d$ with direct access to the theoretical covariance matrix $\mathbf{C}_d$.

% In addition, if we pair $\mathbf{Q}^{\text{SFF}}/\mathbf{K}^{\text{SFF}}$ with the sinusoidal features given in Equation \ref{eq:sff:sinusoidal_features}, in place of those given in Equation \ref{eq:sff:sine_features}, we obtain a richer version of SPE which accepts multi-dimensional structural information instead of time indices. To distinguish it from SPE, we name this variant F-StrIPE:SFF.

\section{Experiments}
% To test the efficacy of our method, we use the task of melody harmonization for symbolic music. 
We assess the merits of our approaches on the task of melody harmonization for symbolic music.

\subsection{Dataset and Input Representation} \label{sssection:data_input}
We use the Chinese POP909 dataset~\cite{wang_pop909_2020} and three levels of structural labels with different resolutions~\cite{dai_automatic_2020}: melodic pitch ($16^{th}$-note), chord (quarter-note) and phrase (measure). Each MIDI file in this dataset consists of three tracks: melody, bridge (second melody) and piano (accompaniment). 
We use the POP909 alignment dataset~\cite{agarwal_structure_2024} to correctly match the structural labels with the input. We convert the MIDI files to binary pianorolls $\mathbf{X} \in \mathbb{B}^{(n_\text{tracks} \times 128) \times n_\text{time}}, \mathbb{B} = \{0, 1\}$, where $n_\text{tracks}$ is the number of tracks and $n_\text{time}$ is the number of timesteps in the pianoroll.

\subsection{Task Setup}
Given the sequence for the melody and bridge tracks $\big[ \mathbf{x}_n \in \mathbb{B}^{(n_\text{tracks} - 1) \times 128} \big]$ as input, with $n \in \{ 1, ..., n_\text{time}\}$, the model must predict all tracks $\big[ \mathbf{y}_n \in \mathbb{B}^{n_\text{tracks} \times 128} \big]$. We expect the model to produce the complete accompaniment track for all timesteps at once, without conditioning later predictions on earlier predictions. 
% We use three settings: (16, 16), (16, 64) and (64, 64), where the first number is the sequence length (in bars of music) used for training and the second number is that used for testing.
We use two settings: (16, 16) and (16, 64). The first number is the sequence length (in measures) for training and the second is that used for testing.

\subsection{Model and Training}
We use a 2-layer causal encoder Transformer with 4 heads and 512 model dimension. Training for 15 epochs with a batch size of 8, we use gradient clipping and curriculum learning~\cite{bengio_curriculum_2009}. We use two learning rate schedulers: a linear warmup and an epoch-wise decay.
We do a grid-search for two hyperparameters: learning rate (choices: $\{ 1, 5, 10 \} \times 0.0001$) and post-processing binarization strategy~\cite{agarwal_structure_2024} (choices: thresholding, thresholding with merge). While the first binarization strategy uses a fixed threshold, the second additionally fills the gap between notes if the gap is less than a minimum distance.

\subsection{Baselines and Our Methods} \label{sssection:baselines}
We consider three types of baselines: (i) Transformers without PE (NoPE~\cite{tsai_transformer_2019,haviv_transformer_2022}), (ii) Transformers with efficient, approximate atttention but no structural information in PE (SPE~\cite{liutkus_relative_2021}), and (iii) Transformers with structural information in PE but using inefficient, exact attention (S S-RPE~\cite{agarwal_structure_2024}).
From our methods, we use F-StrIPE with the three structural levels described in Section \ref{sssection:data_input}. We also assess the influence of different random features with F-StrIPE:SFF using all structural levels. 
We perform ablations on F-StrIPE by selecting one level at a time during training. Finally, we use the best-performing structural level from the F-StrIPE ablations to additionally do an ablation study with F-StrIPE:SFF.

\subsection{Evaluation}
We choose a collection of musically-motivated metrics from the literature, guided by four criteria. 

To assess large- and small-scale structural properties, we use Self-Similarity Matrix Distance (SSMD)~\cite{wu_musemorphose_2021}. For both the target and the prediction, we calculate chroma vectors, giving us the number of onset occurrences per chroma in every half-measure. We then compose a self-similarity matrix (SSM) for each chroma vector by taking the pairwise cosine similarities between all elements of the vector. The SSMD is the mean absolute difference between the SSM of the target and the SSM of the prediction.

For melodic consistency, we use Chroma Similarity (CS)~\cite{wu_musemorphose_2021}. Using the aforementioned method of constructing chroma vectors, we compute the CS as the mean cosine similarity between corresponding entries of the target chroma vector and the prediction chroma vector.

For rhythmic consistency, we use Grooving pattern Similarity (GS)~\cite{wu_jazz_2020}. The grooving pattern of a piece of music is a vector that encodes a 1 for the quarter-notes where onsets occur and 0 for those where no onsets occur. After obtaining the grooving patterns of the target and prediction, we compute the GS as the percentage of quarter-notes where the corresponding pattern values match.

To gauge polyphonicity, we use Note Density Distance (NDD)~\cite{agarwal_structure_2024, haki_real_2022}. We calculate the total number of pitches in each $16^{\text{th}}$-note of the target and prediction. The NDD is the average percentage of missing pitches in the prediction, with the number of pitches in the target giving us the maximum possible value.

% % Please add the following required packages to your document preamble:
% \usepackage{multirow}
\begin{table*}[t]
\centering
\resizebox{2.05\columnwidth}{!}{%
\begin{tabular}{lcccccccc}
\hline
\multicolumn{1}{c|}{\multirow{3}{*}{\textbf{Method}}} & \multicolumn{4}{c|}{\textbf{Train = 16 bars; Test = 16 bars}} & \multicolumn{4}{c}{\textbf{Train = 16 bars; Test = 64 bars}} \\
\multicolumn{1}{c|}{} & \textbf{CS} & \textbf{SSMD} & \textbf{GS} & \multicolumn{1}{c|}{\textbf{NDD}} & \textbf{CS} & \textbf{SSMD} & \textbf{GS} & \multicolumn{1}{c}{\textbf{NDD}}\\
\multicolumn{1}{c|}{} & $\uparrow$ & $\downarrow$ & $\uparrow$ & \multicolumn{1}{c|}{$\downarrow$} & $\uparrow$ & $\downarrow$ & $\uparrow$ & \multicolumn{1}{c}{$\downarrow$} \\ \hline \hline
% \multicolumn{9}{c}{\textbf{Melody Harmonization}} \\ \hline \hline
\multicolumn{1}{l|}{\textbf{NoPE}} & \phantom{0}2.68 $\pm$ 0.16 & 29.31 $\pm$ 0.01 & \phantom{0}7.82 $\pm$ 0.12 & \multicolumn{1}{c|}{103.22 $\pm$ 0.01} & \phantom{0}2.67 $\pm$ 0.18 & 27.60 $\pm$ 0.00 & \phantom{0}9.79 $\pm$ 0.55 & \multicolumn{1}{c}{102.53 $\pm$ 0.05} \\
\multicolumn{1}{l|}{\textbf{S S-RPE}} & 14.52 $\pm$ 0.67 & 28.92 $\pm$ 0.06 & 21.58 $\pm$ 0.86 & \multicolumn{1}{c|}{\phantom{0}98.86 $\pm$ 0.39} & $\diagdown$ & $\diagdown$ & $\diagdown$ & \multicolumn{1}{c}{$\diagdown$} \\
\multicolumn{1}{l|}{\textbf{SPE} \cite{liutkus_relative_2021}} & \phantom{0}1.07 $\pm$ 0.04 & 29.33 $\pm$ 0.00 & \phantom{0}6.02 $\pm$ 0.02 & \multicolumn{1}{c|}{103.81 $\pm$ 0.02} & \phantom{0}6.71 $\pm$ 0.28 & 27.59 $\pm$ 0.00 & 20.27 $\pm$ 0.65 & \multicolumn{1}{c}{101.69 $\pm$ 0.06}\\ \hline
\multicolumn{1}{l|}{\textbf{F-StrIPE}}  & 11.84 $\pm$ 1.24 & 29.18 $\pm$ 0.03 & 18.62 $\pm$ 1.44 & \multicolumn{1}{c|}{100.78 $\pm$ 0.43} & \phantom{0}9.13 $\pm$ 1.19 & 27.53 $\pm$ 0.00 & 27.42 $\pm$ 2.90 & \multicolumn{1}{c}{100.79 $\pm$ 0.21} \\
\multicolumn{1}{l|}{\textbf{F-StrIPE:M}} & \phantom{00}1.9 $\pm$ 0.09 & 29.31 $\pm$ 0.01 & \phantom{0}7.07 $\pm$ 0.13 & \multicolumn{1}{c|}{103.48 $\pm$ 0.03} & \phantom{0}2.22 $\pm$ 0.13 & 27.60 $\pm$ 0.00 & \phantom{0}8.43 $\pm$ 0.50 & \multicolumn{1}{c}{102.69 $\pm$ 0.01} \\
\multicolumn{1}{l|}{\textbf{F-StrIPE:C}} & 16.61 $\pm$ 1.52 & 28.71 $\pm$ 0.07 & 23.19 $\pm$ 2.71 & \multicolumn{1}{c|}{\phantom{0}97.14 $\pm$ 0.37} & 13.29 $\pm$ 1.08 & 27.15 $\pm$ 0.10 & 37.80 $\pm$ 2.74 & \multicolumn{1}{c}{\phantom{0}97.08 $\pm$ 0.34} \\
\multicolumn{1}{l|}{\textbf{F-StrIPE:P}} & \phantom{0}2.07 $\pm$ 0.10 & 29.31 $\pm$ 0.01 & \phantom{0}7.23 $\pm$ 0.14 & \multicolumn{1}{c|}{103.43 $\pm$ 0.02} & \phantom{0}2.31 $\pm$ 0.09 & 27.60 $\pm$ 0.00 & \phantom{0}9.11 $\pm$ 0.27 & \multicolumn{1}{c}{102.64 $\pm$ 0.02} \\ \hline
\multicolumn{1}{l|}{\textbf{F-StrIPE:SFF}} & \phantom{0}2.75 $\pm$ 3.01 & 29.32 $\pm$ 0.03 & \phantom{0}8.15 $\pm$ 3.46 & \multicolumn{1}{c|}{103.43 $\pm$ 0.74} & \phantom{0}6.49 $\pm$ 0.34 & 27.58 $\pm$ 0.00 & 20.50 $\pm$ 0.65 & \multicolumn{1}{c}{101.69 $\pm$ 0.10} \\
\multicolumn{1}{l|}{\textbf{F-StrIPE:SFF:C}} & \phantom{0}4.72 $\pm$ 3.86 & 29.25 $\pm$ 0.07 & 10.22 $\pm$ 4.63 & \multicolumn{1}{c|}{102.54 $\pm$ 1.26} & 10.35 $\pm$ 0.95 & 27.44 $\pm$ 0.05 & 29.81 $\pm$ 1.94 & \multicolumn{1}{c}{\phantom{0}99.88 $\pm$ 0.50} \\ \hline
\end{tabular}
}
\caption{Performance on melody harmonization. F-StrIPE:(M/C/P) are the ablations on F-StrIPE, described in Section \ref{sssection:baselines}, that apply RFF on only one structural level at a time - melodic pitch/chord/phrase. F-StrIPE:SFF:C applies SFF on chords, which is the best performing ablation setting with RFF.}
\label{tab:results_unrounded}
\end{table*}
% Please add the following required packages to your document preamble:
% \usepackage{multirow}
\begin{table*}[t]
\centering
\resizebox{2.05\columnwidth}{!}{%
\begin{tabular}{lcccccccc}
\hline
\multicolumn{1}{c|}{\multirow{3}{*}{\textbf{Method}}} & \multicolumn{4}{c|}{\textbf{Train = 16 bars; Test = 16 bars}} & \multicolumn{4}{c}{\textbf{Train = 16 bars; Test = 64 bars}} \\
\multicolumn{1}{c|}{} & \textbf{CS} & \textbf{SSMD} & \textbf{GS} & \multicolumn{1}{c|}{\textbf{NDD}} & \textbf{CS} & \textbf{SSMD} & \textbf{GS} & \multicolumn{1}{c}{\textbf{NDD}}\\
\multicolumn{1}{c|}{} & $\uparrow$ & $\downarrow$ & $\uparrow$ & \multicolumn{1}{c|}{$\downarrow$} & $\uparrow$ & $\downarrow$ & $\uparrow$ & \multicolumn{1}{c}{$\downarrow$} \\ \hline \hline
% \multicolumn{9}{c}{\textbf{Melody Harmonization}} \\ \hline \hline
\multicolumn{1}{l|}{\textbf{NoPE}} & \phantom{0}2.68 $\pm$ 0.2\phantom{*} & 29.31 $\pm$ 0.0* & \phantom{0}7.82 $\pm$ 0.1\phantom{*} & \multicolumn{1}{c|}{93.94 $\pm$ 0.0*} & \phantom{0}2.67 $\pm$ 0.2 & 27.60 $\pm$ 0.0* & \phantom{0}7.80 $\pm$ 0.2 & \multicolumn{1}{c}{92.49 $\pm$ 0.1\phantom{*}} \\
\multicolumn{1}{l|}{\textbf{S S-RPE}} & 14.52 $\pm$ 0.7\phantom{*} & 28.92 $\pm$ 0.1\phantom{*} & 21.58 $\pm$ 0.9\phantom{*} & \multicolumn{1}{c|}{88.48 $\pm$ 0.5\phantom{*}} & $\diagdown$ & $\diagdown$ & $\diagdown$ & \multicolumn{1}{c}{$\diagdown$} \\
\multicolumn{1}{l|}{\textbf{SPE} \cite{liutkus_relative_2021}} & \phantom{0}1.07 $\pm$ 0.0* & 29.33 $\pm$ 0.0* & \phantom{0}6.02 $\pm$ 0.0* & \multicolumn{1}{c|}{94.65 $\pm$ 0.0*} & \phantom{0}6.71 $\pm$ 0.3 & 27.59 $\pm$ 0.0* & 12.62 $\pm$ 0.3 & \multicolumn{1}{c}{91.48 $\pm$ 0.1\phantom{*}}\\ \hline
\multicolumn{1}{l|}{\textbf{F-StrIPE}}  & 11.84 $\pm$ 1.2\phantom{*} & 29.18 $\pm$ 0.0* & 18.62 $\pm$ 1.4\phantom{*} & \multicolumn{1}{c|}{90.93 $\pm$ 0.6\phantom{*}} & \phantom{0}9.13 $\pm$ 1.2 & 27.53 $\pm$ 0.0* & 14.70 $\pm$ 1.7 & \multicolumn{1}{c}{90.30 $\pm$ 0.3\phantom{*}} \\
\multicolumn{1}{l|}{\textbf{F-StrIPE:M}} & \phantom{00}1.9 $\pm$ 0.1\phantom{*} & 29.31 $\pm$ 0.0* & \phantom{0}7.07 $\pm$ 0.1\phantom{*} & \multicolumn{1}{c|}{94.27 $\pm$ 0.0*} & \phantom{0}2.22 $\pm$ 0.1 & 27.60 $\pm$ 0.0* & \phantom{0}7.42 $\pm$ 0.2 & \multicolumn{1}{c}{92.69 $\pm$ 0.0*} \\
\multicolumn{1}{l|}{\textbf{F-StrIPE:C}} & \textbf{16.61 $\pm$ 1.5\phantom{*}} & \textbf{28.71 $\pm$ 0.1\phantom{*}} & \textbf{23.19 $\pm$ 2.7\phantom{*}} & \multicolumn{1}{c|}{\textbf{86.42 $\pm$ 0.4\phantom{*}}} & \textbf{13.29 $\pm$ 1.1} & \textbf{27.15 $\pm$ 0.1\phantom{*}} & \textbf{16.73 $\pm$ 0.9} & \multicolumn{1}{c}{\textbf{85.61 $\pm$ 0.4\phantom{*}}} \\
\multicolumn{1}{l|}{\textbf{F-StrIPE:P}} & \phantom{0}2.07 $\pm$ 0.1\phantom{*} & 29.31 $\pm$ 0.0* & \phantom{0}7.23 $\pm$ 0.1\phantom{*} & \multicolumn{1}{c|}{94.20 $\pm$ 0.0*} & \phantom{0}2.31 $\pm$ 0.1 & 27.60 $\pm$ 0.0* & \phantom{0}7.56 $\pm$ 0.1 & \multicolumn{1}{c}{92.62 $\pm$ 0.0*} \\ \hline
\multicolumn{1}{l|}{\textbf{F-StrIPE:SFF}} & \phantom{0}2.75 $\pm$ 3.0\phantom{*} & 29.32 $\pm$ 0.0* & \phantom{0}8.15 $\pm$ 3.5\phantom{*} & \multicolumn{1}{c|}{94.18 $\pm$ 0.9\phantom{*}} & \phantom{0}6.49 $\pm$ 0.3 & 27.58 $\pm$ 0.0* & 12.09 $\pm$ 0.7 & \multicolumn{1}{c}{91.46 $\pm$ 0.1\phantom{*}} \\ %\hline
\multicolumn{1}{l|}{\textbf{F-StrIPE:SFF:C}} & \phantom{0}4.72 $\pm$ 3.9\phantom{*} & 29.25 $\pm$ 0.1\phantom{*} & 10.22 $\pm$ 4.6\phantom{*} & \multicolumn{1}{c|}{93.11 $\pm$ 1.5\phantom{*}} & 10.35 $\pm$ 1.0 & 27.44 $\pm$ 0.1\phantom{*} & 12.09 $\pm$ 0.7 & \multicolumn{1}{c}{89.32 $\pm$ 0.6\phantom{*}} \\ \hline
\end{tabular}
}
\caption{Performance on melody harmonization. F-StrIPE:(M/C/P) are the ablations on F-StrIPE, described in Section \ref{sssection:baselines}, that apply RFF on only one structural level at a time - melodic pitch/chord/phrase. F-StrIPE:SFF:C applies SFF on chords, which is the best performing ablation setting with RFF. `0.0*' refers to standard deviations that are lower than 0.05. `$\diagdown$' refers to simulations where the given inference setting could not be accessed due to the heavy computational demands of the method.}
\label{tab:results_unrounded}
\end{table*}


\begin{table*}[t]
\centering
\fontsize{11pt}{11pt}\selectfont
\begin{tabular}{lllllllllllll}
\toprule
\multicolumn{1}{c}{\textbf{task}} & \multicolumn{2}{c}{\textbf{Mir}} & \multicolumn{2}{c}{\textbf{Lai}} & \multicolumn{2}{c}{\textbf{Ziegen.}} & \multicolumn{2}{c}{\textbf{Cao}} & \multicolumn{2}{c}{\textbf{Alva-Man.}} & \multicolumn{1}{c}{\textbf{avg.}} & \textbf{\begin{tabular}[c]{@{}l@{}}avg.\\ rank\end{tabular}} \\
\multicolumn{1}{c}{\textbf{metrics}} & \multicolumn{1}{c}{\textbf{cor.}} & \multicolumn{1}{c}{\textbf{p-v.}} & \multicolumn{1}{c}{\textbf{cor.}} & \multicolumn{1}{c}{\textbf{p-v.}} & \multicolumn{1}{c}{\textbf{cor.}} & \multicolumn{1}{c}{\textbf{p-v.}} & \multicolumn{1}{c}{\textbf{cor.}} & \multicolumn{1}{c}{\textbf{p-v.}} & \multicolumn{1}{c}{\textbf{cor.}} & \multicolumn{1}{c}{\textbf{p-v.}} &  &  \\ \midrule
\textbf{S-Bleu} & 0.50 & 0.0 & 0.47 & 0.0 & 0.59 & 0.0 & 0.58 & 0.0 & 0.68 & 0.0 & 0.57 & 5.8 \\
\textbf{R-Bleu} & -- & -- & 0.27 & 0.0 & 0.30 & 0.0 & -- & -- & -- & -- & - &  \\
\textbf{S-Meteor} & 0.49 & 0.0 & 0.48 & 0.0 & 0.61 & 0.0 & 0.57 & 0.0 & 0.64 & 0.0 & 0.56 & 6.1 \\
\textbf{R-Meteor} & -- & -- & 0.34 & 0.0 & 0.26 & 0.0 & -- & -- & -- & -- & - &  \\
\textbf{S-Bertscore} & \textbf{0.53} & 0.0 & {\ul 0.80} & 0.0 & \textbf{0.70} & 0.0 & {\ul 0.66} & 0.0 & {\ul0.78} & 0.0 & \textbf{0.69} & \textbf{1.7} \\
\textbf{R-Bertscore} & -- & -- & 0.51 & 0.0 & 0.38 & 0.0 & -- & -- & -- & -- & - &  \\
\textbf{S-Bleurt} & {\ul 0.52} & 0.0 & {\ul 0.80} & 0.0 & 0.60 & 0.0 & \textbf{0.70} & 0.0 & \textbf{0.80} & 0.0 & {\ul 0.68} & {\ul 2.3} \\
\textbf{R-Bleurt} & -- & -- & 0.59 & 0.0 & -0.05 & 0.13 & -- & -- & -- & -- & - &  \\
\textbf{S-Cosine} & 0.51 & 0.0 & 0.69 & 0.0 & {\ul 0.62} & 0.0 & 0.61 & 0.0 & 0.65 & 0.0 & 0.62 & 4.4 \\
\textbf{R-Cosine} & -- & -- & 0.40 & 0.0 & 0.29 & 0.0 & -- & -- & -- & -- & - & \\ \midrule
\textbf{QuestEval} & 0.23 & 0.0 & 0.25 & 0.0 & 0.49 & 0.0 & 0.47 & 0.0 & 0.62 & 0.0 & 0.41 & 9.0 \\
\textbf{LLaMa3} & 0.36 & 0.0 & \textbf{0.84} & 0.0 & {\ul{0.62}} & 0.0 & 0.61 & 0.0 &  0.76 & 0.0 & 0.64 & 3.6 \\
\textbf{our (3b)} & 0.49 & 0.0 & 0.73 & 0.0 & 0.54 & 0.0 & 0.53 & 0.0 & 0.7 & 0.0 & 0.60 & 5.8 \\
\textbf{our (8b)} & 0.48 & 0.0 & 0.73 & 0.0 & 0.52 & 0.0 & 0.53 & 0.0 & 0.7 & 0.0 & 0.59 & 6.3 \\  \bottomrule
\end{tabular}
\caption{Pearson correlation on human evaluation on system output. `R-': reference-based. `S-': source-based.}
\label{tab:sys}
\end{table*}



\begin{table}%[]
\centering
\fontsize{11pt}{11pt}\selectfont
\begin{tabular}{llllll}
\toprule
\multicolumn{1}{c}{\textbf{task}} & \multicolumn{1}{c}{\textbf{Lai}} & \multicolumn{1}{c}{\textbf{Zei.}} & \multicolumn{1}{c}{\textbf{Scia.}} & \textbf{} & \textbf{} \\ 
\multicolumn{1}{c}{\textbf{metrics}} & \multicolumn{1}{c}{\textbf{cor.}} & \multicolumn{1}{c}{\textbf{cor.}} & \multicolumn{1}{c}{\textbf{cor.}} & \textbf{avg.} & \textbf{\begin{tabular}[c]{@{}l@{}}avg.\\ rank\end{tabular}} \\ \midrule
\textbf{S-Bleu} & 0.40 & 0.40 & 0.19* & 0.33 & 7.67 \\
\textbf{S-Meteor} & 0.41 & 0.42 & 0.16* & 0.33 & 7.33 \\
\textbf{S-BertS.} & {\ul0.58} & 0.47 & 0.31 & 0.45 & 3.67 \\
\textbf{S-Bleurt} & 0.45 & {\ul 0.54} & {\ul 0.37} & 0.45 & {\ul 3.33} \\
\textbf{S-Cosine} & 0.56 & 0.52 & 0.3 & {\ul 0.46} & {\ul 3.33} \\ \midrule
\textbf{QuestE.} & 0.27 & 0.35 & 0.06* & 0.23 & 9.00 \\
\textbf{LlaMA3} & \textbf{0.6} & \textbf{0.67} & \textbf{0.51} & \textbf{0.59} & \textbf{1.0} \\
\textbf{Our (3b)} & 0.51 & 0.49 & 0.23* & 0.39 & 4.83 \\
\textbf{Our (8b)} & 0.52 & 0.49 & 0.22* & 0.43 & 4.83 \\ \bottomrule
\end{tabular}
\caption{Pearson correlation on human ratings on reference output. *not significant; we cannot reject the null hypothesis of zero correlation}
\label{tab:ref}
\end{table}


\begin{table*}%[]
\centering
\fontsize{11pt}{11pt}\selectfont
\begin{tabular}{lllllllll}
\toprule
\textbf{task} & \multicolumn{1}{c}{\textbf{ALL}} & \multicolumn{1}{c}{\textbf{sentiment}} & \multicolumn{1}{c}{\textbf{detoxify}} & \multicolumn{1}{c}{\textbf{catchy}} & \multicolumn{1}{c}{\textbf{polite}} & \multicolumn{1}{c}{\textbf{persuasive}} & \multicolumn{1}{c}{\textbf{formal}} & \textbf{\begin{tabular}[c]{@{}l@{}}avg. \\ rank\end{tabular}} \\
\textbf{metrics} & \multicolumn{1}{c}{\textbf{cor.}} & \multicolumn{1}{c}{\textbf{cor.}} & \multicolumn{1}{c}{\textbf{cor.}} & \multicolumn{1}{c}{\textbf{cor.}} & \multicolumn{1}{c}{\textbf{cor.}} & \multicolumn{1}{c}{\textbf{cor.}} & \multicolumn{1}{c}{\textbf{cor.}} &  \\ \midrule
\textbf{S-Bleu} & -0.17 & -0.82 & -0.45 & -0.12* & -0.1* & -0.05 & -0.21 & 8.42 \\
\textbf{R-Bleu} & - & -0.5 & -0.45 &  &  &  &  &  \\
\textbf{S-Meteor} & -0.07* & -0.55 & -0.4 & -0.01* & 0.1* & -0.16 & -0.04* & 7.67 \\
\textbf{R-Meteor} & - & -0.17* & -0.39 & - & - & - & - & - \\
\textbf{S-BertScore} & 0.11 & -0.38 & -0.07* & -0.17* & 0.28 & 0.12 & 0.25 & 6.0 \\
\textbf{R-BertScore} & - & -0.02* & -0.21* & - & - & - & - & - \\
\textbf{S-Bleurt} & 0.29 & 0.05* & 0.45 & 0.06* & 0.29 & 0.23 & 0.46 & 4.2 \\
\textbf{R-Bleurt} & - &  0.21 & 0.38 & - & - & - & - & - \\
\textbf{S-Cosine} & 0.01* & -0.5 & -0.13* & -0.19* & 0.05* & -0.05* & 0.15* & 7.42 \\
\textbf{R-Cosine} & - & -0.11* & -0.16* & - & - & - & - & - \\ \midrule
\textbf{QuestEval} & 0.21 & {\ul{0.29}} & 0.23 & 0.37 & 0.19* & 0.35 & 0.14* & 4.67 \\
\textbf{LlaMA3} & \textbf{0.82} & \textbf{0.80} & \textbf{0.72} & \textbf{0.84} & \textbf{0.84} & \textbf{0.90} & \textbf{0.88} & \textbf{1.00} \\
\textbf{Our (3b)} & 0.47 & -0.11* & 0.37 & 0.61 & 0.53 & 0.54 & 0.66 & 3.5 \\
\textbf{Our (8b)} & {\ul{0.57}} & 0.09* & {\ul 0.49} & {\ul 0.72} & {\ul 0.64} & {\ul 0.62} & {\ul 0.67} & {\ul 2.17} \\ \bottomrule
\end{tabular}
\caption{Pearson correlation on human ratings on our constructed test set. 'R-': reference-based. 'S-': source-based. *not significant; we cannot reject the null hypothesis of zero correlation}
\label{tab:con}
\end{table*}

\section{Results}
We benchmark the different metrics on the different datasets using correlation to human judgement. For content preservation, we show results split on data with system output, reference output and our constructed test set: we show that the data source for evaluation leads to different conclusions on the metrics. In addition, we examine whether the metrics can rank style transfer systems similar to humans. On style strength, we likewise show correlations between human judgment and zero-shot evaluation approaches. When applicable, we summarize results by reporting the average correlation. And the average ranking of the metric per dataset (by ranking which metric obtains the highest correlation to human judgement per dataset). 

\subsection{Content preservation}
\paragraph{How do data sources affect the conclusion on best metric?}
The conclusions about the metrics' performance change radically depending on whether we use system output data, reference output, or our constructed test set. Ideally, a good metric correlates highly with humans on any data source. Ideally, for meta-evaluation, a metric should correlate consistently across all data sources, but the following shows that the correlations indicate different things, and the conclusion on the best metric should be drawn carefully.

Looking at the metrics correlations with humans on the data source with system output (Table~\ref{tab:sys}), we see a relatively high correlation for many of the metrics on many tasks. The overall best metrics are S-BertScore and S-BLEURT (avg+avg rank). We see no notable difference in our method of using the 3B or 8B model as the backbone.

Examining the average correlations based on data with reference output (Table~\ref{tab:ref}), now the zero-shoot prompting with LlaMA3 70B is the best-performing approach ($0.59$ avg). Tied for second place are source-based cosine embedding ($0.46$ avg), BLEURT ($0.45$ avg) and BertScore ($0.45$ avg). Our method follows on a 5. place: here, the 8b version (($0.43$ avg)) shows a bit stronger results than 3b ($0.39$ avg). The fact that the conclusions change, whether looking at reference or system output, confirms the observations made by \citet{scialom-etal-2021-questeval} on simplicity transfer.   

Now consider the results on our test set (Table~\ref{tab:con}): Several metrics show low or no correlation; we even see a significantly negative correlation for some metrics on ALL (BLEU) and for specific subparts of our test set for BLEU, Meteor, BertScore, Cosine. On the other end, LlaMA3 70B is again performing best, showing strong results ($0.82$ in ALL). The runner-up is now our 8B method, with a gap to the 3B version ($0.57$ vs $0.47$ in ALL). Note our method still shows zero correlation for the sentiment task. After, ranks BLEURT ($0.29$), QuestEval ($0.21$), BertScore ($0.11$), Cosine ($0.01$).  

On our test set, we find that some metrics that correlate relatively well on the other datasets, now exhibit low correlation. Hence, with our test set, we can now support the logical reasoning with data evidence: Evaluation of content preservation for style transfer needs to take the style shift into account. This conclusion could not be drawn using the existing data sources: We hypothesise that for the data with system-based output, successful output happens to be very similar to the source sentence and vice versa, and reference-based output might not contain server mistakes as they are gold references. Thus, none of the existing data sources tests the limits of the metrics.  


\paragraph{How do reference-based metrics compare to source-based ones?} Reference-based metrics show a lower correlation than the source-based counterpart for all metrics on both datasets with ratings on references (Table~\ref{tab:sys}). As discussed previously, reference-based metrics for style transfer have the drawback that many different good solutions on a rewrite might exist and not only one similar to a reference.


\paragraph{How well can the metrics rank the performance of style transfer methods?}
We compare the metrics' ability to judge the best style transfer methods w.r.t. the human annotations: Several of the data sources contain samples from different style transfer systems. In order to use metrics to assess the quality of the style transfer system, metrics should correctly find the best-performing system. Hence, we evaluate whether the metrics for content preservation provide the same system ranking as human evaluators. We take the mean of the score for every output on each system and the mean of the human annotations; we compare the systems using the Kendall's Tau correlation. 

We find only the evaluation using the dataset Mir, Lai, and Ziegen to result in significant correlations, probably because of sparsity in a number of system tests (App.~\ref{app:dataset}). Our method (8b) is the only metric providing a perfect ranking of the style transfer system on the Lai data, and Llama3 70B the only one on the Ziegen data. Results in App.~\ref{app:results}. 


\subsection{Style strength results}
%Evaluating style strengths is a challenging task. 
Llama3 70B shows better overall results than our method. However, our method scores higher than Llama3 70B on 2 out of 6 datasets, but it also exhibits zero correlation on one task (Table~\ref{tab:styleresults}).%More work i s needed on evaluating style strengths. 
 
\begin{table}%[]
\fontsize{11pt}{11pt}\selectfont
\begin{tabular}{lccc}
\toprule
\multicolumn{1}{c}{\textbf{}} & \textbf{LlaMA3} & \textbf{Our (3b)} & \textbf{Our (8b)} \\ \midrule
\textbf{Mir} & 0.46 & 0.54 & \textbf{0.57} \\
\textbf{Lai} & \textbf{0.57} & 0.18 & 0.19 \\
\textbf{Ziegen.} & 0.25 & 0.27 & \textbf{0.32} \\
\textbf{Alva-M.} & \textbf{0.59} & 0.03* & 0.02* \\
\textbf{Scialom} & \textbf{0.62} & 0.45 & 0.44 \\
\textbf{\begin{tabular}[c]{@{}l@{}}Our Test\end{tabular}} & \textbf{0.63} & 0.46 & 0.48 \\ \bottomrule
\end{tabular}
\caption{Style strength: Pearson correlation to human ratings. *not significant; we cannot reject the null hypothesis of zero corelation}
\label{tab:styleresults}
\end{table}

\subsection{Ablation}
We conduct several runs of the methods using LLMs with variations in instructions/prompts (App.~\ref{app:method}). We observe that the lower the correlation on a task, the higher the variation between the different runs. For our method, we only observe low variance between the runs.
None of the variations leads to different conclusions of the meta-evaluation. Results in App.~\ref{app:results}.

\section{Conclusion}
In this work, we propose a simple yet effective approach, called SMILE, for graph few-shot learning with fewer tasks. Specifically, we introduce a novel dual-level mixup strategy, including within-task and across-task mixup, for enriching the diversity of nodes within each task and the diversity of tasks. Also, we incorporate the degree-based prior information to learn expressive node embeddings. Theoretically, we prove that SMILE effectively enhances the model's generalization performance. Empirically, we conduct extensive experiments on multiple benchmarks and the results suggest that SMILE significantly outperforms other baselines, including both in-domain and cross-domain few-shot settings.

\bibliographystyle{./IEEEbib/IEEEtran}
\bibliography{strings}


\end{document}
