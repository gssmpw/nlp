\section{Conclusion}

In this paper, we demonstrated how structural information can be used in linear-complexity positional encoding, thereby retaining superior performance without sacrificing efficiency. We did this by first extending SPE to accept multi-resolution, structure-aware positional indices, obtaining F-StrIPE:SFF. Then, we showed the connection between SPE~\cite{liutkus_relative_2021} and Random Fourier Features~\cite{rahimi_random_2007} and developed a novel method, called F-StrIPE, framed as a structure-aware generalization of SPE. The combination of these two interventions --- using structure and using Random Fourier Features --- gave us a fast, structure-informed positional encoding method that outperformed SPE, F-StrIPE:SFF and other competitive baselines on melody harmonization for symbolic music.

% We did this by first showing the connection between SPE~\cite{liutkus_relative_2021} and Random Fourier Features~\cite{rahimi_random_2007}. We, then, demonstrated how we can include multi-resolution structures within such efficient PE methods that use kernel approximation techniques. , which we called F-StrIPE. We exhibited the empirical validity of F-StrIPE using a symbolic music generation task.


% M3~\cite{huang_improve_2020} and SPE~\cite{liutkus_relative_2021} compute the same form of PE. We, then, extended these methods to include multi-resolution structures by establishing that Random Fourier Features provides a better approximation than SPE-style Stochastic Fourier Features. We proved the validity of our structure-informed PE methods using both synthetic protocols grounded in theory and real-world experiments with two music generation tasks.