\begin{figure}[t]
    \centering
    \begin{subfigure}[b]{0.5\textwidth}
        \centering
        \includegraphics[trim=10 10 10 10, clip, width=\textwidth]{figures/exp_emb_vis_0_lambda_short.jpg}
        \vspace{-10pt}
        \caption{Node embeddings learned via original FGW objective, i.e., $\lambda = 0$.}
        \label{subfig:exp_emb_vis_0}
    \end{subfigure}
    \hfill
    \begin{subfigure}[b]{0.5\textwidth}
        \centering
        \includegraphics[trim=10 10 10 0, clip, width=\textwidth]{figures/exp_emb_vis_opt_lambda_short.jpg}
        \vspace{-10pt}
        \caption{Node embeddings learned via FGW objective with optimal $\lambda$.}
        \label{subfig:exp_emb_vis_opt}
    \end{subfigure}
    \vspace{-15pt}
    \caption{Evolution of node embeddings along training: (a) directly applying FGW distance for embedding learning leads to embedding collapse and MRR degradation; (b) utilizing FGW distance with transformed $\sn$ leads to discriminating embeddings and MRR improvement.}
    \label{fig:exp_emb_vis}
    \vspace{-15pt}
\end{figure}

\vspace{-5pt}
\section{Conclusions}\label{sec:con}
In this paper, we study the semi-supervised network alignment problem by combining embedding and OT-based alignment methods in a mutually beneficial manner. To improve embedding learning via OT, we propose a learnable transformation on OT mapping to obtain an adaptive sampling strategy directly modeling all cross-network node relationships. To improve OT optimization via embedding, we utilize the learned node embeddings to achieve more expressive OT cost design. We further show that the FGW distance can be neatly unified with a multi-level ranking loss at both node and edge levels. Based on these, a unified framework named \algname\ is proposed to learn node embeddings and OT mappings in a mutually beneficial manner. Extensive experiments show that \algname\ consistently outperforms the state-of-the-art in both effectiveness and scalability by a significant margin, achieving up to 16\% performance improvement and up to 20$\times$ speedup.


\section*{Acknowledgment}
This work is supported by NSF (2134081, 2316233, 2324769),
NIFA (2020-67021-32799),
and
AFOSR (FA9550-24-1-0002).
The content of the information in this document does not necessarily reflect the position or the policy of the Government, and no official endorsement should be inferred.  The U.S. Government is authorized to reproduce and distribute reprints for Government purposes notwithstanding any copyright notation here on.
