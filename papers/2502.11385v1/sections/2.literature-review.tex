\section{Literature Review}
\label{chap:lit-review}


\vspace{\baselineskip}

\subsection{Origins of DQC}

While the idea of quantum computing has been around for a couple of decades, research related to Distributed Quantum Computing (DQC) started to gain traction at the start of the century. Initial works focused on developing theoretical methods for distributed computing by sharing entangled qubits between quantum devices. Yimsiriwattana, A. et al. \cite{yimsiriwattana2004generalized} presented two primitive operations \emph{cat-entangler} and \emph{cat-disentangler} that can be used to create non-local CNOT gates (which is one of the universal gates in quantum computing). The concept worked on the assumption that an entangled qubit pair is established between two quantum devices. The \emph{cat-entangler} can then transform a control qubit $\alpha|0\rangle + \beta|1\rangle$ and the entangled pair $\frac{1}{\sqrt{2}}(|00\rangle + |11\rangle)$ into a "cat-like" state $\alpha|00\rangle+\beta|11\rangle$. This state then allows both the computers to share the control qubit. In the same year, another work by Yimsiriwattana, A. et al. \cite{yimsiriwattana2004distributed} utilized these non-local CNOT gates to propose a distributed version of the popular Shor's algorithm. \par

\subsection{Shift towards optimizing non-local operations}

More recently, the focus has shifted towards developing frameworks which minimize the non-local communication required in distributed versions of popular quantum algorithms. Initial frameworks used graph partitioning algorithms to find partitions of qubits to assign to different quantum computers to minimize the overall non-local communication overhead. Baker, J. et al. \cite{baker2020time} proposed a new circuit mapping scheme for cluster-based machines called FPG-rOEE that reduces the number of operations added for non-local communication across various quantum algorithms with the help of a dynamic partitioning scheme. Wu, A. et al. \cite{wu2022collcomm} proposes a compiler framework, CollComm, to optimize the collective communication in distributed quantum programs. The framework achieves this by decoupling inter-node operations from communication qubits. Recent work by Sundaram, R. et al. \cite{sundaram2023distributing} proposes two algorithms i.e. LocalBest and ZeroStitching to distribute a quantum circuit across a network of heterogeneous quantum computers to minimize the number of teleportations required to implement non-local gates. \par

\subsection{Circuit splitting}

Apart from the above works that focus on minimizing the number of non-local operations, a lot of work is also being done to eliminate the non-local operations completely. These works have a common theme where the aim is to split/cut the quantum circuit into multiple sub-circuits that only have local gates. Classical post-processing is then used on the outputs of these sub-circuits to construct the output of the original un-cut circuit. Peng, T. et al. \cite{peng2020simulating} propose a cluster simulation scheme to represent a quantum circuit as a tensor network and then partition the tensor networks into multiple smaller networks. Edges between networks are simulated by decomposing the interaction based on the observables and states from the computational basis. Chen, D. et al. \cite{chen2023efficient} adopt a similar approach where they cut a qubit wire to split the circuit into sub-circuits. Intuitively, cutting a qubit wire can be thought of as classically passing information of a quantum state along each element in a basis set. Their work exploits this and proposes that certain circuits have a \emph{golden cutting point}, which enjoys a reduction in measurement complexity and classical post-processing cost. However, identifying these \emph{golden cutting points} is non-trivial and a circuit is not guaranteed to have one. \par

Work done by Tang, W. et al. \cite{tang2021cutqc} proposes CutQC, a scalable hybrid computing approach to cutting circuits. Using a mixed-integer programming solver, the framework automatically locates cut locations along qubit wires in the circuit. Sub-circuits generated from these cuts can be executed independently in parallel to speed up overall execution time. The outputs from sub-circuits undergo classical post-processing to construct the output of the original un-cut circuit. This clean cutting of a circuit, however, comes at an exponential cost ($4^{K}$) in terms of the number of cuts (\emph{K}) made. A recent work by Piveteau, C. et al. \cite{piveteau2023circuit} proposes a method of circuit knitting based on quasi-probability simulation of non-local gates using operations that act locally on subcircuits. This approach makes it possible to cut CNOT gates between subcircuits, thus allowing more flexibility when choosing the cut locations. This added flexibility comes at an even greater cost though ($9^{n}$) where (\emph{n}) is the number of CNOT gates cut. They also propose a possible reduction of post-processing cost to ($4^{n}$) if classical communication is available between the quantum computers. \par

In our work here, we will be using the CutQC framework to split our circuits as it has the least amount of restrictions on the type of circuits that can be cut, and it also allows us to construct the full probability distribution of the un-cut circuit, which can be used to verify the outputs. Refer to section \ref{sec:circuit-cutting} for detailed information on how the circuit cutting works in the CutQC framework. Our work builds on the CutQC framework by enabling GPU support for circuit simulations. Our work mainly aims to compare the performance of both approaches, i.e., circuit-splitting and full-circuit simulations, when GPUs are used for simulating the circuits and gain more insights into the pros and cons of both approaches.
