\section{Conclusions}
\label{chap:conclusions}

It is evident from the benchmark results that the Full circuit execution method greatly outperforms the Circuit Splitting simulation method. A major contributing factor is the exponential number of subcircuits to be evaluated depending on the number of cuts made to the original circuit. The parallel evaluation lets us hide this exponential nature but is effective only for circuits split using a small number of cuts. Circuits requiring more than a few cuts still take exponential time to be evaluated.  

An interesting benchmark for future work is evaluating circuits having a width of more than 40 qubits. Full circuit simulations for such circuits would require hundreds of nodes resulting in a considerable amount of MPI communication overhead to exchange data between these nodes. On the other hand, the Circuit Splitting evaluation method would enjoy the benefit of running small circuits capable of being evaluated on a single node. This would give us a better picture of how both the simulation methods work on a larger scale and whether any of these methods is feasible to use for large-scale but efficient quantum circuit simulations.  

A key thing to keep in mind is that the performance of the Circuit Splitting method relies on the efficient cutting of the circuit using a small number of cuts. However, this is not always possible for dense circuits or circuits having two-qubit gates acting on all pairs of qubits (e.g. Grover's algorithm circuit implemented without using ancilla qubits). These are scenarios where circuit-splitting algorithms leveraging gate-cutting can be beneficial. With new circuit-splitting algorithms being published every year, it is only a matter of time before we will be able to classically simulate circuits with a width of more than 100 qubits (albeit with a huge resource cost). This achievement will be a great boost for the scientific community to research and validate new quantum algorithms.