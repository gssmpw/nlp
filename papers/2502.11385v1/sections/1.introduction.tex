\section{Introduction}
\label{chap:introduction}

The current NISQ-era (Noisy Intermediate-Scale Quantum era) quantum computers are sensitive to their environment (noisy), prone to quantum decoherence and are not large enough (considering the number of qubits) to achieve quantum advantage. Many hybrid algorithms (where some calculations are offloaded to a classical processor) designed for these computers show promising results in solving problems that are too big or take an exponential amount of time for classical computers. However, large problems still cannot be executed on these machines due to the low number of qubits. To overcome this limitation, Distributed Quantum Computing (DQC) can be used, where the quantum circuit for a specific problem can be split and divided across multiple quantum nodes for execution. Each node has a subset of qubits assigned to it, and non-local CNOT gates (where the control and target qubits lie on different nodes) are implemented by moving qubits between quantum nodes. However, to achieve this, we require a reliable quantum communication channel to let us move qubits from one node to another without losing information. This is still an active area of research in the scientific community.  

Considering all the factors mentioned above, it takes a lot of effort to successfully develop and validate quantum algorithms for solving large problems (requiring hundreds of qubits). To tackle this problem of developing and validating algorithms that require large quantum circuits, a lot of work has been done to be able to simulate large quantum circuits on classical machines in a distributed manner. However, different simulation methods have various kinds of overhead associated with them, which makes it hard to decide which approach should be followed while simulating a large quantum circuit. This work aims to comprehensively analyse two different approaches to simulating large quantum circuits, i.e., circuit-splitting versus full-circuit execution using distributed memory. The former aims to split the quantum circuit into multiple subcircuits to reduce the width of circuits to simulate and allow the simulation of large quantum circuits using smaller machines. The latter approach simulates the entire circuit by distributing the statevector of the quantum circuit across a distributed memory space. More information about both approaches can be found in Section \ref{sec:circuit-cutting} and Section \ref{sec:gpu-simulation}.  

Results show that the full-circuit simulation method is faster than the circuit-splitting method. However, it is not straightforward to conclude which simulation method should be used for simulating large-scale quantum circuits since many parameters must be evaluated and compared to select a feasible simulation method. This study aims to list the pros and cons of each simulation method and shed more light on what parameters can help decide which simulation method is best to use.