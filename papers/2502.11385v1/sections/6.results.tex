\section{Results}
\label{chap:results}

This section reports the results of the performed benchmarks. Before diving into the benchmarks, a few things need to be noted. The original CutQC paper uses a cluster of 16 nodes to perform the benchmarks. Regrettably, the multi-node version of the code is no longer accessible. Hence, the benchmarks in this study are conducted solely on a single node, restricting the simulation to a maximum of 34 qubits. This favors the full circuit simulation method since minimal overhead is present to simulate circuits on different GPUs on a single node. Full circuit executions on multiple nodes are subject to additional communication overhead (due to inter-node data exchanges required using MPI), which would have given a better idea of how the two simulation methods perform when performing simulations for a larger number of qubits (greater than 40).

\vspace{\baselineskip}
\vspace{\baselineskip}
Every circuit except the \textit{Supremacy} circuit is benchmarked for an even number of qubits ranging from 10 to 34. The \textit{Supremacy} circuit was benchmarked for 12, 15, 16, 20, 24, 25 and 30 qubits. Runtime evaluation results for all circuit types can be found in Figure \ref{fig:results}.


\begin{figure*}[htbp]
\centering
\begin{subfigure}{0.33\textwidth}
  \centering
  \includegraphics[alt={Comparison of runtime for the Adder circuit using CutQC framework and Full circuit simulation}, width=\textwidth]{result-plots/adder_annot.png}
  \caption{Adder}
  \label{fig:results/adder-annot}
\end{subfigure}%
\begin{subfigure}{0.33\textwidth}
  \centering
  \includegraphics[alt={Comparison of runtime for the AQFT circuit using CutQC framework and Full circuit simulation}, width=\textwidth]{result-plots/aqft_annot.png}
  \caption{AQFT}
  \label{fig:results/aqft-annot}
\end{subfigure}%
\begin{subfigure}{0.33\textwidth}
  \centering
  \includegraphics[alt={Comparison of runtime for the BV circuit using CutQC framework and Full circuit simulation}, width=\textwidth]{result-plots/bv_annot.png}
  \caption{BV}
  \label{fig:results/bv-annot}
\end{subfigure}%  

\begin{subfigure}{0.33\textwidth}
  \centering
  \includegraphics[alt={Comparison of runtime for the HWEA circuit using CutQC framework and Full circuit simulation}, width=\textwidth]{result-plots/hwea_annot.png}
  \caption{HWEA}
  \label{fig:results/hwea-annot}
\end{subfigure}%
\begin{subfigure}{0.33\textwidth}
  \centering
  \includegraphics[alt={Comparison of runtime for the Supremacy circuit using CutQC framework and Full circuit simulation}, width=\textwidth]{result-plots/supremacy_annot.png}
  \caption{Supremacy}
  \label{fig:results/supremacy-annot}
\end{subfigure}
\caption{ Results for the runtime evaluations of all circuits. The top represents the evaluation time taken when the circuit is simulated by splitting into subcircuits, and the bottom represents the runtime for the uncut circuit. The top is further annotated to show the number of cuts made to split the original circuit. }
\label{fig:results}
\end{figure*}



%
% Talk about the evaluation phase results here
%
\vspace{\baselineskip}
Figure \ref{fig:results/adder-annot} shows the circuit evaluation times for circuit splitting versus full circuit execution for the \textit{Adder} circuit. As can be seen from the figure, the circuit splitting method runs close to two magnitudes slower than the full circuit execution for most of the qubit values. This is expected when running on a single node, as we run only one instance of the full circuit compared to the numerous instances of the subcircuits in the circuit-splitting method. When a circuit is cut into subcircuits using $K$ cuts, the number of subcircuits to simulate is in the order $4^K$ since we need to evaluate subcircuits with all possible initialization and measurements on qubit wires that are cut. The key thing to note here is that the runtime of the split circuits does not increase exponentially upon increasing the number of qubits. In fact, we can observe that the runtime remains the same for qubit values less than 26, after which there is a linear increase in the runtime. This is because all the \textit{Adder} circuits are split using two cuts, so the scaling of runtime in this case purely depends on the number of qubits being simulated. The same trend can also be observed for the \textit{BV} (Figure \ref{fig:results/bv-annot}) and \textit{HWEA} (Figure \ref{fig:results/hwea-annot}) circuits where all circuits are cut using the same number of cuts.  

A different trend can be observed for the \textit{AQFT} (Figure \ref{fig:results/aqft-annot}) and \textit{Supremacy} (Figure \ref{fig:results/supremacy-annot}) circuits where the number of cuts made increase as we increase the number of qubits. In the \textit{AQFT} circuit, the runtime increases more than 2x for the 32-qubit circuit compared to the 30-qubit circuit and an even sharper jump in runtime is observed for the 34-qubit circuit. This is because the 30-qubit circuit uses 5 cuts to split the circuits, whereas the 32 and 34-qubit circuits use 6 cuts to split the circuit, which results in exponentially more subcircuits to be simulated for the circuits having more than 30 qubits. The same trend is observed for the \textit{Supremacy} circuit when going from the 24-qubit circuit (using 4 cuts) to the circuits having more than 24 qubits (using 5 cuts).



%
% Talk about split circuit metadata here
%
\vspace{\baselineskip}
Another thing to notice from the circuit evaluation plots is the similarity between the runtime of the \textit{BV} and \textit{HWEA} circuits. The cause of this similarity lies in the way these two circuits are cut and split into subcircuits. Tables \ref{tab:results/subcirc-metadata-1} and \ref{tab:results/subcirc-metadata-2} contain the metadata for the split circuits for all the circuit types used in this study. Focusing on the metadata for the \textit{BV} and \textit{HWEA} circuits, it can be observed that both the circuits are split into two subcircuits with the same width and same number of effective qubits (qubits which contribute to the final uncut state). This result outlines an important piece of information that subcircuit evaluation runtime is closely related to how the cuts are made and the width of the resulting subcircuits. Another example of this can be seen by observing the metadata for the \textit{AQFT} circuit, where the difference between the overall width of subcircuits and the width of the uncut circuit is greater than the other circuit types. This is a direct consequence of requiring more cuts to split the original circuit as $ \textit{Width of original circuit} + \textit{Number of cuts} = \textit{Overall width of subcircuits} $.



% Please add the following required packages to your document preamble:
% \usepackage{multirow}
% \usepackage{graphicx}
\begin{table}[!htbp]
\resizebox{\linewidth}{!}{%
\begin{tabular}{|c|ll|ll|ll|ll|}
\hline
\multirow{3}{*}{\textbf{Qubits}} & \multicolumn{2}{c|}{\textbf{Adder}}                                                                                                                                                                            & \multicolumn{2}{c|}{\textbf{AQFT}}                                                                                                                                                                             & \multicolumn{2}{c|}{\textbf{BV}}                                                                                                                                                                               & \multicolumn{2}{c|}{\textbf{HWEA}}                                                                                                                                                                             \\ \cline{2-9} 
                                 & \multicolumn{1}{c|}{\multirow{2}{*}{\textbf{\begin{tabular}[c]{@{}c@{}}Subcirc.\\ width\end{tabular}}}} & \multicolumn{1}{c|}{\multirow{2}{*}{\textbf{\begin{tabular}[c]{@{}c@{}}Eff.\\ Qubits\end{tabular}}}} & \multicolumn{1}{c|}{\multirow{2}{*}{\textbf{\begin{tabular}[c]{@{}c@{}}Subcirc.\\ width\end{tabular}}}} & \multicolumn{1}{c|}{\multirow{2}{*}{\textbf{\begin{tabular}[c]{@{}c@{}}Eff.\\ Qubits\end{tabular}}}} & \multicolumn{1}{c|}{\multirow{2}{*}{\textbf{\begin{tabular}[c]{@{}c@{}}Subcirc.\\ width\end{tabular}}}} & \multicolumn{1}{c|}{\multirow{2}{*}{\textbf{\begin{tabular}[c]{@{}c@{}}Eff.\\ Qubits\end{tabular}}}} & \multicolumn{1}{c|}{\multirow{2}{*}{\textbf{\begin{tabular}[c]{@{}c@{}}Subcirc.\\ width\end{tabular}}}} & \multicolumn{1}{c|}{\multirow{2}{*}{\textbf{\begin{tabular}[c]{@{}c@{}}Eff.\\ Qubits\end{tabular}}}} \\
                                 & \multicolumn{1}{c|}{}                                                                                   & \multicolumn{1}{c|}{}                                                                                & \multicolumn{1}{c|}{}                                                                                   & \multicolumn{1}{c|}{}                                                                                & \multicolumn{1}{c|}{}                                                                                   & \multicolumn{1}{c|}{}                                                                                & \multicolumn{1}{c|}{}                                                                                   & \multicolumn{1}{c|}{}                                                                                \\ \hline
\multirow{2}{*}{\textbf{10}}     & \multicolumn{1}{l|}{6}                                                                                  & 5                                                                                                    & \multicolumn{1}{l|}{8}                                                                                  & 4                                                                                                    & \multicolumn{1}{l|}{6}                                                                                  & 5                                                                                                    & \multicolumn{1}{l|}{6}                                                                                  & 5                                                                                                    \\ \cline{2-9} 
                                 & \multicolumn{1}{l|}{6}                                                                                  & 5                                                                                                    & \multicolumn{1}{l|}{6}                                                                                  & 6                                                                                                    & \multicolumn{1}{l|}{5}                                                                                  & 5                                                                                                    & \multicolumn{1}{l|}{5}                                                                                  & 5                                                                                                    \\ \hline
\multirow{2}{*}{\textbf{12}}     & \multicolumn{1}{l|}{6}                                                                                  & 5                                                                                                    & \multicolumn{1}{l|}{8}                                                                                  & 4                                                                                                    & \multicolumn{1}{l|}{7}                                                                                  & 6                                                                                                    & \multicolumn{1}{l|}{7}                                                                                  & 6                                                                                                    \\ \cline{2-9} 
                                 & \multicolumn{1}{l|}{8}                                                                                  & 7                                                                                                    & \multicolumn{1}{l|}{8}                                                                                  & 8                                                                                                    & \multicolumn{1}{l|}{6}                                                                                  & 6                                                                                                    & \multicolumn{1}{l|}{6}                                                                                  & 6                                                                                                    \\ \hline
\multirow{2}{*}{\textbf{14}}     & \multicolumn{1}{l|}{8}                                                                                  & 7                                                                                                    & \multicolumn{1}{l|}{8}                                                                                  & 4                                                                                                    & \multicolumn{1}{l|}{6}                                                                                  & 5                                                                                                    & \multicolumn{1}{l|}{6}                                                                                  & 5                                                                                                    \\ \cline{2-9} 
                                 & \multicolumn{1}{l|}{8}                                                                                  & 7                                                                                                    & \multicolumn{1}{l|}{10}                                                                                 & 10                                                                                                   & \multicolumn{1}{l|}{9}                                                                                  & 9                                                                                                    & \multicolumn{1}{l|}{9}                                                                                  & 9                                                                                                    \\ \hline
\multirow{2}{*}{\textbf{16}}     & \multicolumn{1}{l|}{8}                                                                                  & 7                                                                                                    & \multicolumn{1}{l|}{12}                                                                                 & 7                                                                                                    & \multicolumn{1}{l|}{10}                                                                                 & 9                                                                                                    & \multicolumn{1}{l|}{10}                                                                                 & 9                                                                                                    \\ \cline{2-9} 
                                 & \multicolumn{1}{l|}{10}                                                                                 & 9                                                                                                    & \multicolumn{1}{l|}{9}                                                                                  & 9                                                                                                    & \multicolumn{1}{l|}{7}                                                                                  & 7                                                                                                    & \multicolumn{1}{l|}{7}                                                                                  & 7                                                                                                    \\ \hline
\multirow{2}{*}{\textbf{18}}     & \multicolumn{1}{l|}{8}                                                                                  & 7                                                                                                    & \multicolumn{1}{l|}{10}                                                                                 & 5                                                                                                    & \multicolumn{1}{l|}{11}                                                                                 & 10                                                                                                   & \multicolumn{1}{l|}{11}                                                                                 & 10                                                                                                   \\ \cline{2-9} 
                                 & \multicolumn{1}{l|}{12}                                                                                 & 11                                                                                                   & \multicolumn{1}{l|}{13}                                                                                 & 13                                                                                                   & \multicolumn{1}{l|}{8}                                                                                  & 8                                                                                                    & \multicolumn{1}{l|}{8}                                                                                  & 8                                                                                                    \\ \hline
\multirow{2}{*}{\textbf{20}}     & \multicolumn{1}{l|}{10}                                                                                 & 9                                                                                                    & \multicolumn{1}{l|}{14}                                                                                 & 9                                                                                                    & \multicolumn{1}{l|}{13}                                                                                 & 12                                                                                                   & \multicolumn{1}{l|}{13}                                                                                 & 12                                                                                                   \\ \cline{2-9} 
                                 & \multicolumn{1}{l|}{12}                                                                                 & 11                                                                                                   & \multicolumn{1}{l|}{11}                                                                                 & 11                                                                                                   & \multicolumn{1}{l|}{8}                                                                                  & 8                                                                                                    & \multicolumn{1}{l|}{8}                                                                                  & 8                                                                                                    \\ \hline
\multirow{2}{*}{\textbf{22}}     & \multicolumn{1}{l|}{10}                                                                                 & 9                                                                                                    & \multicolumn{1}{l|}{16}                                                                                 & 11                                                                                                   & \multicolumn{1}{l|}{13}                                                                                 & 12                                                                                                   & \multicolumn{1}{l|}{13}                                                                                 & 12                                                                                                   \\ \cline{2-9} 
                                 & \multicolumn{1}{l|}{14}                                                                                 & 13                                                                                                   & \multicolumn{1}{l|}{11}                                                                                 & 11                                                                                                   & \multicolumn{1}{l|}{10}                                                                                 & 10                                                                                                   & \multicolumn{1}{l|}{10}                                                                                 & 10                                                                                                   \\ \hline
\multirow{2}{*}{\textbf{24}}     & \multicolumn{1}{l|}{10}                                                                                 & 9                                                                                                    & \multicolumn{1}{l|}{17}                                                                                 & 12                                                                                                   & \multicolumn{1}{l|}{15}                                                                                 & 14                                                                                                   & \multicolumn{1}{l|}{15}                                                                                 & 14                                                                                                   \\ \cline{2-9} 
                                 & \multicolumn{1}{l|}{16}                                                                                 & 15                                                                                                   & \multicolumn{1}{l|}{12}                                                                                 & 12                                                                                                   & \multicolumn{1}{l|}{10}                                                                                 & 10                                                                                                   & \multicolumn{1}{l|}{10}                                                                                 & 10                                                                                                   \\ \hline
\multirow{2}{*}{\textbf{26}}     & \multicolumn{1}{l|}{16}                                                                                 & 15                                                                                                   & \multicolumn{1}{l|}{18}                                                                                 & 13                                                                                                   & \multicolumn{1}{l|}{17}                                                                                 & 16                                                                                                   & \multicolumn{1}{l|}{17}                                                                                 & 16                                                                                                   \\ \cline{2-9} 
                                 & \multicolumn{1}{l|}{12}                                                                                 & 11                                                                                                   & \multicolumn{1}{l|}{13}                                                                                 & 13                                                                                                   & \multicolumn{1}{l|}{10}                                                                                 & 10                                                                                                   & \multicolumn{1}{l|}{10}                                                                                 & 10                                                                                                   \\ \hline
\multirow{2}{*}{\textbf{28}}     & \multicolumn{1}{l|}{14}                                                                                 & 13                                                                                                   & \multicolumn{1}{l|}{19}                                                                                 & 14                                                                                                   & \multicolumn{1}{l|}{18}                                                                                 & 17                                                                                                   & \multicolumn{1}{l|}{18}                                                                                 & 17                                                                                                   \\ \cline{2-9} 
                                 & \multicolumn{1}{l|}{16}                                                                                 & 15                                                                                                   & \multicolumn{1}{l|}{14}                                                                                 & 14                                                                                                   & \multicolumn{1}{l|}{11}                                                                                 & 11                                                                                                   & \multicolumn{1}{l|}{11}                                                                                 & 11                                                                                                   \\ \hline
\multirow{2}{*}{\textbf{30}}     & \multicolumn{1}{l|}{12}                                                                                 & 11                                                                                                   & \multicolumn{1}{l|}{21}                                                                                 & 16                                                                                                   & \multicolumn{1}{l|}{19}                                                                                 & 18                                                                                                   & \multicolumn{1}{l|}{19}                                                                                 & 18                                                                                                   \\ \cline{2-9} 
                                 & \multicolumn{1}{l|}{20}                                                                                 & 19                                                                                                   & \multicolumn{1}{l|}{14}                                                                                 & 14                                                                                                   & \multicolumn{1}{l|}{12}                                                                                 & 12                                                                                                   & \multicolumn{1}{l|}{12}                                                                                 & 12                                                                                                   \\ \hline
\multirow{2}{*}{\textbf{32}}     & \multicolumn{1}{l|}{14}                                                                                 & 13                                                                                                   & \multicolumn{1}{l|}{22}                                                                                 & 16                                                                                                   & \multicolumn{1}{l|}{21}                                                                                 & 20                                                                                                   & \multicolumn{1}{l|}{21}                                                                                 & 20                                                                                                   \\ \cline{2-9} 
                                 & \multicolumn{1}{l|}{20}                                                                                 & 19                                                                                                   & \multicolumn{1}{l|}{16}                                                                                 & 16                                                                                                   & \multicolumn{1}{l|}{12}                                                                                 & 12                                                                                                   & \multicolumn{1}{l|}{12}                                                                                 & 12                                                                                                   \\ \hline
\multirow{2}{*}{\textbf{34}}     & \multicolumn{1}{l|}{14}                                                                                 & 13                                                                                                   & \multicolumn{1}{l|}{24}                                                                                 & 18                                                                                                   & \multicolumn{1}{l|}{22}                                                                                 & 21                                                                                                   & \multicolumn{1}{l|}{22}                                                                                 & 21                                                                                                   \\ \cline{2-9} 
                                 & \multicolumn{1}{l|}{22}                                                                                 & 21                                                                                                   & \multicolumn{1}{l|}{16}                                                                                 & 16                                                                                                   & \multicolumn{1}{l|}{13}                                                                                 & 13                                                                                                   & \multicolumn{1}{l|}{13}                                                                                 & 13                                                                                                   \\ \hline
\end{tabular}%
}
\caption{This table contains the metadata of the split circuits for \textit{Adder}, \textit{AQFT}, \textit{BV} and \textit{HWEA} circuit types. Subcircuit width and the Effective qubits (which contribute to the original uncut state) of that subcircuit are reported for every circuit benchmarked. In this benchmark, every circuit is split into two subcircuits, which is why there are only two rows for every circuit benchmarked.}
\label{tab:results/subcirc-metadata-1}
\end{table}



% Please add the following required packages to your document preamble:
% \usepackage{multirow}
% \usepackage{graphicx}
\begin{table}[!htbp]
\resizebox{\linewidth}{!}{%
\begin{tabular}{|ccl|ll|ll|ll|ll|ll|ll|ll|}
\hline
\multicolumn{3}{|c|}{\textbf{Qubits}}                                                                                                              & \multicolumn{2}{c|}{\textbf{12}} & \multicolumn{2}{c|}{\textbf{15}} & \multicolumn{2}{c|}{\textbf{16}} & \multicolumn{2}{c|}{\textbf{20}} & \multicolumn{2}{c|}{\textbf{24}} & \multicolumn{2}{c|}{\textbf{25}} & \multicolumn{2}{c|}{\textbf{30}} \\ \hline
\multicolumn{1}{|c|}{\multirow{2}{*}{\textbf{Supremacy}}} & \multicolumn{2}{c|}{\textbf{\begin{tabular}[c]{@{}c@{}}Subcirc.\\ width\end{tabular}}} & \multicolumn{1}{l|}{8}    & 7    & \multicolumn{1}{l|}{10}    & 8   & \multicolumn{1}{l|}{12}    & 8   & \multicolumn{1}{l|}{11}   & 13   & \multicolumn{1}{l|}{11}   & 17   & \multicolumn{1}{l|}{18}   & 12   & \multicolumn{1}{l|}{19}   & 16   \\ \cline{2-17} 
\multicolumn{1}{|c|}{}                                    & \multicolumn{2}{c|}{\textbf{\begin{tabular}[c]{@{}c@{}}Eff.\\ Qubits\end{tabular}}}    & \multicolumn{1}{l|}{7}    & 5    & \multicolumn{1}{l|}{7}     & 8   & \multicolumn{1}{l|}{9}     & 7   & \multicolumn{1}{l|}{10}   & 10   & \multicolumn{1}{l|}{8}    & 16   & \multicolumn{1}{l|}{14}   & 11   & \multicolumn{1}{l|}{18}   & 12   \\ \hline
\end{tabular}%
}
\caption{This table contains the metadata of the split circuits for \textit{Supremacy} circuit type. Subcircuit width and the Effective qubits (which contribute to the original uncut state) of each subcircuit are reported for every circuit benchmarked. In this benchmark, every circuit is split into two subcircuits, which is why there are only two rows for every circuit benchmarked.}
\label{tab:results/subcirc-metadata-2}
\end{table}


\begin{table}[!htbp]
\resizebox{\linewidth}{!}{%
\begin{tabular}{|c|c|}
\hline
\textbf{Circuit Width} & \textbf{\begin{tabular}[c]{@{}c@{}}Approximate Memory Required to store Statevector\\ (in GB)\end{tabular}} \\ \hline
10                     & $1.63 \times 10^{-5}$                                                                                                   \\ \hline
12                     & $6.55 \times 10^{-5}$                                                                                                   \\ \hline
14                     & $2.62 \times 10^{-4}$                                                                                                    \\ \hline
15                     & $5.24 \times 10^{-4}$                                                                                                    \\ \hline
16                     & $1.04 \times 10^{-3}$                                                                                                    \\ \hline
18                     & $4.19 \times 10^{-3}$                                                                                                   \\ \hline
20                     & $1.67 \times 10^{-2}$                                                                                                    \\ \hline
22                     & $6.71 \times 10^{-2}$                                                                                                    \\ \hline
24                     & 0.26                                                                                                        \\ \hline
25                     & 0.53                                                                                                        \\ \hline
26                     & 1.07                                                                                                        \\ \hline
28                     & 4.29                                                                                                        \\ \hline
30                     & 17.17                                                                                                       \\ \hline
32                     & 68.71                                                                                                       \\ \hline
34                     & 274.87                                                                                                      \\ \hline
\end{tabular}
}
\caption{Approximate Memory space required to store a Quantum Circuit in Gigabytes}
\label{tab:results/req_storage}
\end{table}



\subsection*{Findings and Discussion}

The benchmarking results show that a full-circuit simulation runs significantly faster than a circuit-splitting simulation for single-node executions. The results provide key insights into the performance of the circuit-splitting simulation method. Parallel execution of subcircuits lets us hide the exponential cost of simulation when the input circuit is cut using less than 6 cuts (at least for the type of circuits benchmarked in this study). Another key insight is that circuits cut similarly (i.e. having similar widths and effective qubits) have similar runtime trends. This hints that the runtime is agnostic of the layout of the gates present in the original circuit and that the runtime can be estimated based on the cut locations, the resulting subcircuit widths and the number of effective qubits. Tables \ref{tab:results/subcirc-metadata-1} and \ref{tab:results/subcirc-metadata-2} show the subcircuit widths after every circuit is cut. It can be observed that the width of the subcircuits is 30-40\% less than the original circuit. This enables us to simulate these circuits using significantly fewer resources than the full-circuit simulation. A 30-40\% reduction in circuit width significantly reduces the amount of resources required to simulate the circuit, as for every qubit reduced, the computational space is reduced by half (Refer Table \ref{tab:results/req_storage}). Considering all factors, simulating large circuits using circuit-splitting and full-circuit simulation methods is a tradeoff between runtime and memory requirements. A large circuit can be simulated faster if there are enormous resources available to simulate it. In contrast, if resources are limited, the circuit-splitting method can be used to simulate the required circuit. In an ideal scenario, we can leverage both these approaches to efficiently simulate large quantum circuits. Large circuits can be split to reduce the resources required to simulate the circuit, and on the other hand, the cuts can be minimized as much as possible to avoid incurring exponential runtime to simulate all subcircuits.