\documentclass[conference]{IEEEtran}

% \IEEEoverridecommandlockouts
% The preceding line is only needed to identify funding in the first footnote. If that is unneeded, please comment it out.


% \usepackage{cite}
\usepackage{amsmath,amssymb,amsfonts}
\usepackage{algorithmic}
\usepackage{graphicx}
\usepackage{textcomp}
\usepackage{xcolor}
\def\BibTeX{{\rm B\kern-.05em{\sc i\kern-.025em b}\kern-.08em
    T\kern-.1667em\lower.7ex\hbox{E}\kern-.125emX}}
% Packages added by Author
\usepackage[sorting=none]{biblatex}
\addbibresource{references.bib}
\usepackage{hyperref}
\usepackage{subcaption}
\usepackage{multirow}
\usepackage{listings}
\definecolor{codegreen}{rgb}{0,0.6,0}
\definecolor{codegray}{rgb}{0.5,0.5,0.5}
\definecolor{codepurple}{rgb}{0.58,0,0.82}
\definecolor{backcolour}{rgb}{0.95,0.95,0.95}
\lstdefinestyle{mystyle}{
    backgroundcolor=\color{backcolour},   
    commentstyle=\color{codegreen},
    keywordstyle=\color{codepurple},
    numberstyle=\tiny\color{codegray},
    stringstyle=\color{brown},
    basicstyle=\ttfamily\footnotesize,
    frame=single,
    breakatwhitespace=false,         
    breaklines=true,
    postbreak=\mbox{\textcolor{red}{$\hookrightarrow$}\space},
    captionpos=b,                    
    keepspaces=true,                 
    numbers=left,                    
    numbersep=5pt,                  
    showspaces=false,                
    showstringspaces=false,
    showtabs=false,                  
    tabsize=2,
    xleftmargin=0pt,
    xrightmargin=0pt
}
\lstset{style=mystyle}
\lstset{linewidth=0.9\linewidth}


\begin{document}

\title{Circuit Partitioning and Full Circuit Execution: A Comparative Study of GPU-Based Quantum Circuit Simulation}

% NOTE: Omit the Author name for double-blind submission format

\author{
\IEEEauthorblockN{Kartikey Sarode}
\IEEEauthorblockA{\textit{Department of Computer Science} \\
\textit{San Francisco State University}\\
San Francisco, USA \\
ksarode@sfsu.edu}
\and
\IEEEauthorblockN{Daniel E. Huang}
\IEEEauthorblockA{
\textit{San Francisco State University}\\
San Francisco, CA, USA \\
\href{mailto:danehuang@sfsu.edu}{danehuang@sfsu.edu}}
\and
\IEEEauthorblockN{E. Wes Bethel}
\IEEEauthorblockA{\textit{San Francisco State University} \\
San Francisco, CA, USA \\
\textit{Lawrence Berkeley National Laboratory}\\
Berkeley, CA, USA \\
\href{mailto:ewbethel@sfsu.edu}{ewbethel@sfsu.edu}}
}


\maketitle

\thispagestyle{plain}
\pagestyle{plain}

\begin{abstract}
% Large quantum circuits cannot be executed effectively on available NISQ (noisy intermediate-scale quantum) devices. Moreover, due to the associated cost, researching and developing quantum algorithms on real quantum devices is not feasible. Classical simulations are, therefore, the best method to research and validate large-scale quantum algorithms. This, however, requires a large amount of resources as each additional qubit doubles the computational space for a quantum algorithm. Distributed Quantum Computing (DQC) is a promising alternative to reduce the resources required to simulate large quantum algorithms at the cost of increased runtime.  

% This study performs a comparative analysis between two simulation methods, i.e. circuit-splitting versus full-circuit execution using distributed memory, with each method having a different type of overhead. The first approach (using CutQC) cuts the circuits into smaller subcircuits and allows us to simulate large quantum circuits using smaller machines. The second approach (using Qiskit-Aer-GPU) distributes the computational space across a distributed memory space and simulates the entire circuit. Results show that full-circuit executions are faster than circuit-splitting for simulations executed on a single node. However, circuit-splitting simulations show promising results in specific scenarios as the number of qubits is scaled.


Executing large quantum circuits is not feasible using the currently available NISQ (noisy intermediate-scale quantum) devices. The high costs of using real quantum devices make it further challenging to research and develop quantum algorithms. As a result, performing classical simulations is usually the preferred method for researching and validating large-scale quantum algorithms. However, these simulations require a huge amount of resources, as each additional qubit exponentially increases the computational space required.

Distributed Quantum Computing (DQC) is a promising alternative to reduce the resources required for simulating large quantum algorithms at the cost of increased runtime. This study presents a comparative analysis of two simulation methods: circuit-splitting and full-circuit execution using distributed memory, each having a different type of overhead.

The first method, using CutQC, cuts the circuit into smaller subcircuits and allows us to simulate a large quantum circuit on smaller machines. The second method, using Qiskit-Aer-GPU, distributes the computational space across a distributed memory system to simulate the entire quantum circuit. Results indicate that full-circuit executions are faster than circuit-splitting for simulations performed on a single node. However, circuit-splitting simulations show promising results in specific scenarios as the number of qubits is scaled.


\end{abstract}

\begin{IEEEkeywords}
Distributed Quantum Computing (DQC), Quantum Circuit-Splitting, Qiskit Aer, CutQC
\end{IEEEkeywords}





\section{Introduction}
\IEEEPARstart{I}{n} recent years, flourishing of Artificial Intelligence Generated Content (AIGC) has sparked significant advancements in modalities such as text, image, audio, and even video. 
Among these, AI-Generated Image (AGI) has garnered considerable interest from both researchers and the public.
Plenty of remarkable AGI models and online services, such as StableDiffusion\footnote{\url{https://stability.ai/}}, Midjourney\footnote{\url{https://www.midjourney.com/}}, and FLUX\footnote{\url{https://blackforestlabs.ai/}}, offer users an excellent creative experience.
However, users often remain critical of the quality of the AGI due to image distortions or mismatches with user intentions.
Consequently, methods for assessing the quality of AGI are becoming increasingly crucial to help improve the generative capabilities of these models.

Unlike Natural Scene Image (NSI) quality assessment, which focuses primarily on perception aspects such as sharpness, color, and brightness, AI-Generated Image Quality Assessment (AGIQA) encompasses additional aspects like correspondence and authenticity. 
Since AGI is generated on the basis of user text prompts, it may fail to capture key user intentions, resulting in misalignment with the prompt.
Furthermore, authenticity refers to how closely the generated image resembles real-world artworks, as AGI can sometimes exhibit logical inconsistencies.
While traditional IQA models may effectively evaluate perceptual quality, they are often less capable of adequately assessing aspects such as correspondence and authenticity.

\begin{figure}\label{fig:radar}
    \centering
    \includegraphics[width=1.0\linewidth]{figures/radar_plot.pdf}
    \caption{A comparison on quality, correspondence, and authenticity aspects of AIGCIQA2023~\cite{wang2023aigciqa2023} dataset illustrates the superior performance of our method.}
\end{figure}

Several methods have been proposed specifically for the AGIQA task, including metrics designed to evaluate the authenticity and diversity of generated images~\cite{gulrajani2017improved,heusel2017gans}. 
Nevertheless, these methods tend to compare and evaluate grouped images rather than single instances, which limits their utility for single image assessment.
Beginning with AGIQA-1k~\cite{zhang2023perceptual}, a series of AGIQA databases have been introduced, including AGIQA-3k~\cite{li2023agiqa}, AIGCIQA-20k~\cite{li2024aigiqa}, etc.
Concurrently, there has been a surge in research utilizing deep learning methods~\cite{zhou2024adaptive,peng2024aigc,yu2024sf}, which have significantly benefited from pre-trained models such as CLIP~\cite{radford2021learning}. 
These approaches enhance the analysis by leveraging the correlations between images and their descriptive texts.
While these models are effective in capturing general text-image alignments, they may not effectively detect subtle inconsistencies or mismatches between the generated image content and the detailed nuances of the textual description.
Moreover, as these models are pre-trained on large-scale datasets for broad tasks, they might not fully exploit the textual information pertinent to the specific context of AGIQA without task-specific fine-tuning.
To overcome these limitations, methods that leverage Multimodal Large Language Models (MLLMs)~\cite{wang2024large,wang2024understanding} have been proposed.
These methods aim to fully exploit the synergies of image captioning and textual analysis for AGIQA.
Although they benefit from advanced prompt understanding, instruction following, and generation capabilities, they often do not utilize MLLMs as encoders capable of producing a sequence of logits that integrate both image and text context.

In conclusion, the field of AI-Generated Image Quality Assessment (AGIQA) continues to face significant challenges: 
(1) Developing comprehensive methods to assess AGIs from multiple dimensions, including quality, correspondence, and authenticity; 
(2) Enhancing assessment techniques to more accurately reflect human perception and the nuanced intentions embedded within prompts; 
(3) Optimizing the use of Multimodal Large Language Models (MLLMs) to fully exploit their multimodal encoding capabilities.

To address these challenges, we propose a novel method M3-AGIQA (\textbf{M}ultimodal, \textbf{M}ulti-Round, \textbf{M}ulti-Aspect AI-Generated Image Quality Assessment) which leverages MLLMs as both image and text encoders. 
This approach incorporates an additional network to align human perception and intentions, aiming to enhance assessment accuracy. 
Specially, we distill the rich image captioning capability from online MLLMs into a local MLLM through Low-Rank Adaption (LoRA) fine-tuning, and train this model with human-labeled data. The key contributions of this paper are as follows:
\begin{itemize}
    \item We propose a novel AGIQA method that distills multi-aspect image captioning capabilities to enable comprehensive evaluation. Specifically, we use an online MLLM service to generate aspect-specific image descriptions and fine-tune a local MLLM with these descriptions in a structured two-round conversational format.
    \item We investigate the encoding potential of MLLMs to better align with human perceptual judgments and intentions, uncovering previously underestimated capabilities of MLLMs in the AGIQA domain. To leverage sequential information, we append an xLSTM feature extractor and a regression head to the encoding output.
    \item Extensive experiments across multiple datasets demonstrate that our method achieves superior performance, setting a new state-of-the-art (SOTA) benchmark in AGIQA.
\end{itemize}

In this work, we present related works in Sec.~\ref{sec:related}, followed by the details of our M3-AGIQA method in Sec.~\ref{sec:method}. Sec.~\ref{sec:exp} outlines our experimental design and presents the results. Sec.~\ref{sec:limit},~\ref{sec:ethics} and~\ref{sec:conclusion} discuss the limitations, ethical concerns, future directions and conclusions of our study.
\section{Literature Review}
\label{chap:lit-review}


\vspace{\baselineskip}

\subsection{Origins of DQC}

While the idea of quantum computing has been around for a couple of decades, research related to Distributed Quantum Computing (DQC) started to gain traction at the start of the century. Initial works focused on developing theoretical methods for distributed computing by sharing entangled qubits between quantum devices. Yimsiriwattana, A. et al. \cite{yimsiriwattana2004generalized} presented two primitive operations \emph{cat-entangler} and \emph{cat-disentangler} that can be used to create non-local CNOT gates (which is one of the universal gates in quantum computing). The concept worked on the assumption that an entangled qubit pair is established between two quantum devices. The \emph{cat-entangler} can then transform a control qubit $\alpha|0\rangle + \beta|1\rangle$ and the entangled pair $\frac{1}{\sqrt{2}}(|00\rangle + |11\rangle)$ into a "cat-like" state $\alpha|00\rangle+\beta|11\rangle$. This state then allows both the computers to share the control qubit. In the same year, another work by Yimsiriwattana, A. et al. \cite{yimsiriwattana2004distributed} utilized these non-local CNOT gates to propose a distributed version of the popular Shor's algorithm. \par

\subsection{Shift towards optimizing non-local operations}

More recently, the focus has shifted towards developing frameworks which minimize the non-local communication required in distributed versions of popular quantum algorithms. Initial frameworks used graph partitioning algorithms to find partitions of qubits to assign to different quantum computers to minimize the overall non-local communication overhead. Baker, J. et al. \cite{baker2020time} proposed a new circuit mapping scheme for cluster-based machines called FPG-rOEE that reduces the number of operations added for non-local communication across various quantum algorithms with the help of a dynamic partitioning scheme. Wu, A. et al. \cite{wu2022collcomm} proposes a compiler framework, CollComm, to optimize the collective communication in distributed quantum programs. The framework achieves this by decoupling inter-node operations from communication qubits. Recent work by Sundaram, R. et al. \cite{sundaram2023distributing} proposes two algorithms i.e. LocalBest and ZeroStitching to distribute a quantum circuit across a network of heterogeneous quantum computers to minimize the number of teleportations required to implement non-local gates. \par

\subsection{Circuit splitting}

Apart from the above works that focus on minimizing the number of non-local operations, a lot of work is also being done to eliminate the non-local operations completely. These works have a common theme where the aim is to split/cut the quantum circuit into multiple sub-circuits that only have local gates. Classical post-processing is then used on the outputs of these sub-circuits to construct the output of the original un-cut circuit. Peng, T. et al. \cite{peng2020simulating} propose a cluster simulation scheme to represent a quantum circuit as a tensor network and then partition the tensor networks into multiple smaller networks. Edges between networks are simulated by decomposing the interaction based on the observables and states from the computational basis. Chen, D. et al. \cite{chen2023efficient} adopt a similar approach where they cut a qubit wire to split the circuit into sub-circuits. Intuitively, cutting a qubit wire can be thought of as classically passing information of a quantum state along each element in a basis set. Their work exploits this and proposes that certain circuits have a \emph{golden cutting point}, which enjoys a reduction in measurement complexity and classical post-processing cost. However, identifying these \emph{golden cutting points} is non-trivial and a circuit is not guaranteed to have one. \par

Work done by Tang, W. et al. \cite{tang2021cutqc} proposes CutQC, a scalable hybrid computing approach to cutting circuits. Using a mixed-integer programming solver, the framework automatically locates cut locations along qubit wires in the circuit. Sub-circuits generated from these cuts can be executed independently in parallel to speed up overall execution time. The outputs from sub-circuits undergo classical post-processing to construct the output of the original un-cut circuit. This clean cutting of a circuit, however, comes at an exponential cost ($4^{K}$) in terms of the number of cuts (\emph{K}) made. A recent work by Piveteau, C. et al. \cite{piveteau2023circuit} proposes a method of circuit knitting based on quasi-probability simulation of non-local gates using operations that act locally on subcircuits. This approach makes it possible to cut CNOT gates between subcircuits, thus allowing more flexibility when choosing the cut locations. This added flexibility comes at an even greater cost though ($9^{n}$) where (\emph{n}) is the number of CNOT gates cut. They also propose a possible reduction of post-processing cost to ($4^{n}$) if classical communication is available between the quantum computers. \par

In our work here, we will be using the CutQC framework to split our circuits as it has the least amount of restrictions on the type of circuits that can be cut, and it also allows us to construct the full probability distribution of the un-cut circuit, which can be used to verify the outputs. Refer to section \ref{sec:circuit-cutting} for detailed information on how the circuit cutting works in the CutQC framework. Our work builds on the CutQC framework by enabling GPU support for circuit simulations. Our work mainly aims to compare the performance of both approaches, i.e., circuit-splitting and full-circuit simulations, when GPUs are used for simulating the circuits and gain more insights into the pros and cons of both approaches.

% !TEX root = ../main.tex

\section{Causal Discovery and GES}\label{sec:background}  

We first review the causal discovery problem and the necessary details of the greedy
equivalence search (GES) method.

\subsection{Causal Graphical Models }
Causal discovery aims to identify cause-and-effect relationships between random
variables $\{X_1, ..., X_d\}$. We reason about causal relationships using causal
graphical models (CGM). A CGM has two components:
\begin{enumerate}
\item a directed acyclic graph (DAG), $G^* = (V,E)$, where a node $j \in V$ represents
variable $X_j$ and an edge $(j, k) \in E$ denotes a direct causal link from $X_j$ to
$X_k$,
\item conditional distributions $p(X_j  \mid X_{\Pa_j^{G^*}})$, defining the
distribution of $X_j$ given its causal parents $X_{\Pa^{G^*}_j}$.
\end{enumerate}
The joint distribution of the variables $X_1,..., X_d$ writes:
\begin{equation}\label{eq:factor_joint}
p^*(X) = \prod\limits_{j \in V} p^*(X_j \mid X_{\Pa^{G^*}_j}).
\end{equation}
The goal of causal discovery is to recover the graph $G^*$ from the joint distribution
$p^*$ or from samples drawn from $p^*$. 

However, multiple CGMs with different graphs can generate the same $p^*\!.$ Two
important concepts address this difficulty: faithfulness and Markov equivalence
\citep{spirtes2000causation}.

\parhead{Faithfulness.} In \Cref{eq:factor_joint}, $G^*$ induces a factorization of
$p^*$, which, in turn, induces independencies between variables: each $X_j$ is
independent of its non-descendants given its parents $X_{\Pa^{G^*}_j}$
\citep{pearl1988probabilistic}. Reciprocally, we say that a distribution $p^*$ and a
graph $G^*$ are \textit{faithful} if all the independencies in $p^*$ are exactly those
implied by $G^*$ and no more. \looseness=-1

%(Notice it is the lack of edge in $G^*$ that imposes independencies in $p^*$.)
For example, $H=(\{1,2\}, \{1\rightarrow 2\})$ and $q=q(X_1)q(X_2)$ form a
valid CGM. But the independence $X_1 \indep X_2$ present in $q$ is not suggested by $H$.
Rather, $q$ is faithful to $H'=(\{1,2\}, \varnothing)$, which has no superfluous edges
like $1 \rightarrow 2$.

Limiting the search to faithful graphs reduces the possible CGMs that could have
generated $p^*$. But this is not enough.



\parhead{Markov Equivalence.} 
Two distinct graphs can both be faithful to $p^*$ if they induce the same set of
independencies on $p^*$. For example, $A\rightarrow B \rightarrow C$ and $A \leftarrow B
\leftarrow C$ impose the same set of independencies,  $\{A \indep C \mid B\}$. 

Graphs inducing the same independencies are called \textit{Markov equivalent}, they form
\textit{Markov equivalence classes} (MEC).
Since $G^*$ is identifiable only up to Markov equivalence, the task of causal discovery
becomes finding the MEC of $G^*$. 

\begin{remark}
    With more assumptions (e.g. about the form of $p^*$) or special data (e.g.,
    interventions), other causal discovery methods focus on identifying the possible
    $G^*$ beyond Markov equivalence. See \citet{glymour2019review} for an excellent
    review. We focus on Markov equivalence classes.
\end{remark}

\subsection{Score-based Causal Discovery}
In this work, we assume to have $n$ iid samples, denoted $\data = \{(x_1^{i}, ...,
x_d^{i})\}_{i=1}^n$, from a distribution $p^*$ that is faithful to some $G^*$. We aim to
recover the MEC of $G^*$ from $\data$.

Greedy Equivalence Search (GES) searches for the MEC of $G^*$ among all the possible
MECs. It does so by searching for the MEC whose DAGs maximize a specific score function.
It is a particular case of score-based causal discovery.


{Score-based methods} assign a score $S(G; \data)$ to every possible DAG $G$ given the
data $\data$. The score function $S$ is designed to be maximized by the true graph
$G^*$. This turns causal discovery into an optimization problem. 
\begin{equation}
    G^* = \argmax_{G} S(G; \data)
\end{equation}
Some scores have properties that are important for GES.

\begin{definition}
    A score $S$ is \emph{score equivalent} if it assigns the same score to all the
    graphs in the same MEC.
\end{definition}

A score equivalent $S$ enables defining the score of a MEC as the score of any of its
constituent graphs. 

\begin{definition}[Local Consistency, \citep{chickering2002optimal}]
Let $\data$ contain $n$ iid samples from some $p^*$. Let $G$ be any DAG and $G'$ be a
different DAG obtained by adding the edge $i \rightarrow j$ to $G$. A score $S$ is
\emph{locally consistent} if both hold:
\begin{enumerate}
    \item $X_i \not\hspace*{-0.5mm}\indep_{p^*} X_j \mid X_{\Pa_j^G} \Rightarrow S(G';
    \data)>S(G; \data)$,
\item $X_i \indep_{p^*} X_j \mid X_{\Pa_j^G} \Rightarrow S(G'; \data)<S(G; \data)$.
\end{enumerate}
The independence statements are with respect to $p^*$.

\end{definition}

A locally consistent score increases when we add an edge that captures a dependency not
yet represented in the graph. It decreases when we add an edge that does not capture any
new dependency.

\subsection{Greedy Equivalence Search}
\citet{chickering2002optimal} introduced Greedy Equivalence Search (GES). It is a
score-based method that is guaranteed to return the MEC of $G^*$ whenever the score is
locally consistent. The main characteristic of GES is to navigate the space of MECs.

GES begins with the MEC of the empty DAG and iteratively modifies it to improve the
score. At each step, only a few modifications are allowed. These
modifications are of three types: insertions, deletions, and reversals. 
\begin{figure}
    \centering
    \includegraphics[width=\linewidth]{fig/mecv2.pdf}
    \caption{Illustration of insertions from a MEC A to MECs B or C: (i) choose a DAG in
    A, (ii) insert the edge $2\leftarrow3$ to obtain another DAG (iii) consider its MEC.
    Each MEC has all its DAGs on a white plate, and its canonical PDAG on a gray
    plate (see \Cref{sec:implementation} for a definition).}
    \label{fig:mec}
\end{figure}

\parhead{MEC modifications.}An \textit{insertion} on MEC $M$ selects a DAG $G$ in $M$,
adds an edge $x \rightarrow y$ to $G$ to obtain a different DAG $G'$ and replaces $M$
with the MEC of $G'$ (see \Cref{fig:mec}).

A \textit{deletion} on MEC $M$ selects a DAG $G$ in $M$, removes an edge
$x \rightarrow y$ from $G$ to obtain a different DAG $G'$ and replaces $M$ with the MEC
of $G'$.

A \textit{reversal} on MEC $M$ selects a DAG $G$ in $M$, reverses an edge
$x \rightarrow y$ to $x \leftarrow y$ in $G$ to obtain a different DAG $G'$ and replaces
$M$ with the MEC of $G'$.

GES applies these modifications in three separate phases.

\parhead{Phase 1: Insert.} First, GES finds all possible insertions, 
applies the one leading to the largest score increase, and repeats until no insertion
increases the score.

\parhead{Phase 2: Delete.} Then, GES finds all possible deletions, applies the one
leading to the largest score increase, and repeats until no deletion increases the
score.

The MEC obtained at the end of phase 2 is exactly the MEC of $G^*$ if the score is
locally consistent \citep{chickering2002optimal}

\parhead{Phase 3: Reverse.} In theory, phases 1 and 2 are sufficient to recover the
MEC of $G^*$. Yet, \citet{hauser2012characterization} showed that adding a third
phase with reversals can improve the search in practice with finite data (where the score might not be
locally consistent). In this phase, GES finds all possible reversals for the MEC,
applies the one that increases the score most, and repeats until none do.

The pseudocode of GES is given in \Cref{alg:ges-vanilla}.

\begin{algorithm}[t]
    \caption{Greedy Equivalence Search (GES)}\label{alg:ges-vanilla}  
    \DontPrintSemicolon
    
    \KwIn{Data $\data \in \R^{n\times d}$, score function $S$}  
    \KwDefine{$\delta_{\data,M}(O) = S(\text{Apply}(O,M) ; \data) - S(M; \data)$}  
    \KwOut{MEC of $G^*$}  
    $M \leftarrow \{([d], \varnothing)\}$ \tcp*{Empty graph's MEC}  
    $\mathcal{I} \leftarrow$ get all insertions valid for $M$ \;  
    \While{$|\mathcal{I}| > 0$}{  
        $O^* \leftarrow \argmax_{I \in \mathcal{I}}\{\delta_{\data,M}(I)\}$ \tcp*{Get
        best insertion}  
        \lIf{$\delta_{\data,M}(O^*) \leq 0$}{\textbf{break}}  
        $M \leftarrow$ Apply($O^*, M$) \tcp*{Apply best insertion}  
        $\mathcal{I} \leftarrow$ get all insertions valid for $M$ \;  
    }  
    $\mathcal{D} \leftarrow$ get all deletions valid for $M$ \;  
    \While{$|\mathcal{D}| > 0$}{  
        $O^* \leftarrow \argmax_{D \in \mathcal{D}} \{\delta_{\data,M}(D)\}$ \tcp*{Get
        best deletion}  
        \lIf{$\delta_{\data,M}(O^*) \leq 0$}{\textbf{break}}  
        $M \leftarrow$ Apply($O^*, M$) \tcp*{Apply best deletion}  
        $\mathcal{D} \leftarrow$ get all deletions valid for $M$ \;  
    }  
    \tcc{(Optional) 3rd phase like above but with reversals}  
    \Return $M$ \;  
\end{algorithm}

\parhead{Correctness.} GES's correctness relies on two properties:
    (i) the greedy scheme will reach the global maximum of any locally consistent
    score \citep[Lemma 10]{chickering2002optimal}, and 
    (ii) the true graph $G^*$ and its MEC are the unique global maximizers of any
    locally consistent score \citep[Proposition 8]{chickering2002optimal}.

With these two properties, GES is guaranteed to recover the MEC of $G^*$ when the score
is locally consistent.
\label{sec:greedy_search}

\parhead{The BIC score.} A score commonly used by GES is the Bayesian Information Criterion
(BIC) \citep{schwarz1978estimating}. Given a model class
$\mathcal{M}_G$ for each $G$, and our data $\data$ with its $n$ samples, the BIC defines a score:
\begin{equation}
    S(G; \data) = \log p_{\hat\theta}(\data ; G) - \frac{\alpha}{2} \log n \cdot |\hat\theta|,
\end{equation}
where
$\alpha$ is a hyperparameter, $\hat \theta$ is the likelihood maximizer over
$\mathcal{M}_G$, and the number of parameters $|\hat\theta|$ is usually the number of edges in $G$. The BIC is a model selection criterion trading off log-likelihood and model complexity. Models with higher BIC are preferred. The original BIC has $\alpha=1$. 


\parhead{Gaussian Linear Models. } A common model class used for continuous data with the BIC is the class of Gaussian linear models, where given a graph $G$, each variable is Gaussian with a linear
conditional mean and a specific variance:
\begin{equation}\textstyle
    p_{\theta}(x_j \mid x_{\Pa_j^G}) \sim \mathcal{N}\Bigl(\sum_{k \in \Pa_j^G} 
    \theta_{jk} x_k + \theta_{j0}, \theta_{j(d+1)}^2\Bigr).
\end{equation}
For Gaussian linear models, the BIC is score equivalent. It can also be locally
consistent under some conditions.

\begin{theorem}[Local Consistency of BIC \citep{haughton1988choice,chickering2002optimal}]
    For $\alpha>0$, the BIC for Gaussian linear models is locally consistent once $n$
    is large enough.
    \label{th:local_consistency}
\end{theorem}

% More generally, the local consistency theorem also applies to most model classes of the exponential family (e.g. the multinomial distribution for discrete data).
Hence, for Gaussian linear models, GES is guaranteed to recover the MEC of $G^*$ with
infinite data. 
In practice, however, data is finite and GES can return incorrect MECs. 

\parhead{Example of Failure.} In \Cref{fig:simulation_main}, we report the performance
of GES (orange) on simulated data, along with its variants and the proposed XGES. We
simulate CGMs $(G^*, p^*)$ for $d \in\{25,50\}$ variables, $\rho d \in \{2d,3d,4d\}$
edges ($\rho$ is an edge density parameter) and draw $n=10,000$ samples from $p^*$. The
simulation is detailed in \Cref{sec:experiments:evaluation_setup}. We compare the
methods' results to the true graph $G^*$ using the structural Hamming distance for MECs
(SHD), which counts the number of different edges between graphs of two MECs (see
\Cref{sec:experiments:evaluation_setup}). \Cref{fig:simulation_main} shows that GES can fail (SHD $> 0$),
especially in denser graphs.

In addition, we confirm that including the third phase of reversals in GES improves its
performance (GES-r, in green).


% figure: simulation_main
\begin{figure}[t]
    \centering
    \includegraphics[width=1\linewidth]{fig/simulation_main_shd.pdf}
    \caption{Performance comparison of GES and XGES variants, measured with SHD for
    different edge densities $\rho$. XGES heuristics outperform GES and its variants in
    all scenarios. The dashed lines indicate the number of edges of the true graph. Each boxplot
    is computed over 30 seeds.\looseness=-1}
    \label{fig:simulation_main}
\end{figure}


\section{Implementation Details}

We used 8 A100 GPUs with 80GB of memory for the experiments. While the exact GPU hours for each experiment were not precisely recorded, the total GPU usage did not exceed one hour. The system was set up with CUDA 12.4, Triton 3.0.0, and Ubuntu 22.04. For the Llama model, we employed the Hugging Face implementation of transformers, and for SAE model, we used the OpenSAE implementation\footnote{\url{https://github.com/THU-KEG/OpenSAE}} and set the hyperparameter $k$ to $128$ for TopK activation.
\section{Methodology}
\label{chap:methodology}

This section aims to provide more information on the benchmarks performed in this study. Section \ref{sec:environment} talks about the hardware and software used for the benchmarks, and detailed information about the benchmarks is provided in Section \ref{sec:benchmarks}.




\subsection{Hardware and Software Environments}
\label{sec:environment}

\subsubsection{Hardware Environment}

All benchmarks were performed on Perlmutter \cite{perlmutter} (operated by NERSC at Lawrence Berkeley National Lab). Each benchmark was executed on a dedicated GPU node with several nodes being available for running multiple benchmarks at once.  

Each GPU node on Perlmutter has an AMD EPYC 7763 (Milan) CPU at 2.45 GHz, 256GB DDR4 RAM and four NVIDIA A100 GPUs \cite{perlmutter_specs}. There are two variants of these GPU nodes: a normal one with 40GB NVIDIA A100 GPUs and a high-bandwidth one with 80GB NVIDIA A100 GPUs. Benchmarks up to 30 qubits were executed on normal GPU nodes, whereas circuits with more than 30 qubits were executed on high-bandwidth GPU nodes.


\subsubsection{Software Environment}

All quantum circuit simulations were performed using the Qiskit Aer simulator \cite{qiskitPaper}\cite{qiskit}\cite{qiskitAerGPU} with GPUs on a server with SUSE Linux Enterprise Server 15 SP4. To split a quantum circuit into sub-circuits, the benchmarks use the CutQC framework \cite{cutqc} based on the work by Tang, W. et. al. \cite{tang2021cutqc}.  

\noindent Below is a list of all major software used in the benchmarks and their respective versions.
\begin{itemize}
    \item Python 3.12.2
    \item Qiskit 1.0.2
    \item Qiskit-Aer-GPU 0.14.0.1
\end{itemize}




\subsection{Benchmarks}
\label{sec:benchmarks}

The circuits used in the benchmarks are selected from the CutQC paper. The benchmarks were run for the number of qubits ranging from 10 to 34 (the maximum number of qubits we can simulate using one node of Perlmutter). Two methods are used to simulate the circuits: the first one simulates the circuit as it is (an uncut circuit), and the other uses CutQC to split the circuit into multiple subcircuits and simulate these subcircuits.  

The evaluation runtime, i.e., the time required to simulate all circuits using Qiskit Aer, is the main performance metric captured to compare the simulation methods. Apart from this, the runtime is also captured for different phases of the circuit-splitting method to gain more insights into the runtime scaling. Finally, circuit-splitting metadata (which includes the subcircuit width and number of effective qubits for every subcircuit generated) is also captured, which helps us to co-relate some of the observed trends in the runtime plots.

\vspace{\baselineskip}
\noindent More information about the circuits benchmarked is listed below.

\begin{enumerate}
    \item \textit{Adder}: This is a quantum ripple-carry addition circuit with a single ancilla qubit and linear depth based on the work of Cuccaro, S. A. et. al. \cite{cuccaro2004new}. It can be used to sum two quantum registers of the same width. It is benchmarked for an even number of qubits.
    \item \textit{Approximate Quantum Fourier Transform (AQFT)}: Work done by Barenco, A. et al. \cite{barenco1996approximate} proposes that as far as periodicity estimation is concerned, the AQFT algorithm yields better results than the QFT (Quantum Fourier Transform) algorithm. It is also benchmarked for an even number of qubits and a standard implementation from the Qiskit circuit library is used to generate the circuit.
    \item \textit{Berstein-Vazirani (BV)}: A popular quantum algorithm based on the work of Bernstein, E. et. al. \cite{bernstein1993quantum} which solves the hidden string problem. The implementation used in the benchmark is referenced from the QED-C \cite{qedcBenchmark} benchmark suite.
    \item \textit{Hardware efficient ansatz (HWEA)}: An example of quantum variational algorithms inspired by the work of Moll, N. et. al. \cite{moll2018quantum}.
    \item \textit{Supremacy}: A type of 2-D random circuit based on the work by Boixo, S. et. al. \cite{boixo2018characterizing} to characterize quantum supremacy. Similar to the CutQC paper, benchmarks are performed for near-square-shaped (e.g. 3*4) circuit layouts (which are harder to cut).
\end{enumerate}

\noindent All the circuits used in the benchmarks are not dense (having a huge circuit depth) quantum circuits but are rather sparse when it comes to the layout of the gates applied to the qubits. This is necessary because we impose a constraint on the CutQC framework to cut a given circuit within 10 cuts and generate a maximum of 5 subcircuits. While cutting dense circuits like Grover's (implemented without using ancilla qubits) is theoretically possible using the framework, the number of cuts required to cut such a circuit is very high, leading to enormous classical post-processing costs and is therefore not practical to simulate. One of the goals of these benchmarks is to study the impact of different sparse layouts of circuits on the runtime and the incurred classical post-processing cost when these circuits are simulated using \textit{circuit splitting}.
\section{Results}
\label{chap:results}

This section reports the results of the performed benchmarks. Before diving into the benchmarks, a few things need to be noted. The original CutQC paper uses a cluster of 16 nodes to perform the benchmarks. Regrettably, the multi-node version of the code is no longer accessible. Hence, the benchmarks in this study are conducted solely on a single node, restricting the simulation to a maximum of 34 qubits. This favors the full circuit simulation method since minimal overhead is present to simulate circuits on different GPUs on a single node. Full circuit executions on multiple nodes are subject to additional communication overhead (due to inter-node data exchanges required using MPI), which would have given a better idea of how the two simulation methods perform when performing simulations for a larger number of qubits (greater than 40).

\vspace{\baselineskip}
\vspace{\baselineskip}
Every circuit except the \textit{Supremacy} circuit is benchmarked for an even number of qubits ranging from 10 to 34. The \textit{Supremacy} circuit was benchmarked for 12, 15, 16, 20, 24, 25 and 30 qubits. Runtime evaluation results for all circuit types can be found in Figure \ref{fig:results}.


\begin{figure*}[htbp]
\centering
\begin{subfigure}{0.33\textwidth}
  \centering
  \includegraphics[alt={Comparison of runtime for the Adder circuit using CutQC framework and Full circuit simulation}, width=\textwidth]{result-plots/adder_annot.png}
  \caption{Adder}
  \label{fig:results/adder-annot}
\end{subfigure}%
\begin{subfigure}{0.33\textwidth}
  \centering
  \includegraphics[alt={Comparison of runtime for the AQFT circuit using CutQC framework and Full circuit simulation}, width=\textwidth]{result-plots/aqft_annot.png}
  \caption{AQFT}
  \label{fig:results/aqft-annot}
\end{subfigure}%
\begin{subfigure}{0.33\textwidth}
  \centering
  \includegraphics[alt={Comparison of runtime for the BV circuit using CutQC framework and Full circuit simulation}, width=\textwidth]{result-plots/bv_annot.png}
  \caption{BV}
  \label{fig:results/bv-annot}
\end{subfigure}%  

\begin{subfigure}{0.33\textwidth}
  \centering
  \includegraphics[alt={Comparison of runtime for the HWEA circuit using CutQC framework and Full circuit simulation}, width=\textwidth]{result-plots/hwea_annot.png}
  \caption{HWEA}
  \label{fig:results/hwea-annot}
\end{subfigure}%
\begin{subfigure}{0.33\textwidth}
  \centering
  \includegraphics[alt={Comparison of runtime for the Supremacy circuit using CutQC framework and Full circuit simulation}, width=\textwidth]{result-plots/supremacy_annot.png}
  \caption{Supremacy}
  \label{fig:results/supremacy-annot}
\end{subfigure}
\caption{ Results for the runtime evaluations of all circuits. The top represents the evaluation time taken when the circuit is simulated by splitting into subcircuits, and the bottom represents the runtime for the uncut circuit. The top is further annotated to show the number of cuts made to split the original circuit. }
\label{fig:results}
\end{figure*}



%
% Talk about the evaluation phase results here
%
\vspace{\baselineskip}
Figure \ref{fig:results/adder-annot} shows the circuit evaluation times for circuit splitting versus full circuit execution for the \textit{Adder} circuit. As can be seen from the figure, the circuit splitting method runs close to two magnitudes slower than the full circuit execution for most of the qubit values. This is expected when running on a single node, as we run only one instance of the full circuit compared to the numerous instances of the subcircuits in the circuit-splitting method. When a circuit is cut into subcircuits using $K$ cuts, the number of subcircuits to simulate is in the order $4^K$ since we need to evaluate subcircuits with all possible initialization and measurements on qubit wires that are cut. The key thing to note here is that the runtime of the split circuits does not increase exponentially upon increasing the number of qubits. In fact, we can observe that the runtime remains the same for qubit values less than 26, after which there is a linear increase in the runtime. This is because all the \textit{Adder} circuits are split using two cuts, so the scaling of runtime in this case purely depends on the number of qubits being simulated. The same trend can also be observed for the \textit{BV} (Figure \ref{fig:results/bv-annot}) and \textit{HWEA} (Figure \ref{fig:results/hwea-annot}) circuits where all circuits are cut using the same number of cuts.  

A different trend can be observed for the \textit{AQFT} (Figure \ref{fig:results/aqft-annot}) and \textit{Supremacy} (Figure \ref{fig:results/supremacy-annot}) circuits where the number of cuts made increase as we increase the number of qubits. In the \textit{AQFT} circuit, the runtime increases more than 2x for the 32-qubit circuit compared to the 30-qubit circuit and an even sharper jump in runtime is observed for the 34-qubit circuit. This is because the 30-qubit circuit uses 5 cuts to split the circuits, whereas the 32 and 34-qubit circuits use 6 cuts to split the circuit, which results in exponentially more subcircuits to be simulated for the circuits having more than 30 qubits. The same trend is observed for the \textit{Supremacy} circuit when going from the 24-qubit circuit (using 4 cuts) to the circuits having more than 24 qubits (using 5 cuts).



%
% Talk about split circuit metadata here
%
\vspace{\baselineskip}
Another thing to notice from the circuit evaluation plots is the similarity between the runtime of the \textit{BV} and \textit{HWEA} circuits. The cause of this similarity lies in the way these two circuits are cut and split into subcircuits. Tables \ref{tab:results/subcirc-metadata-1} and \ref{tab:results/subcirc-metadata-2} contain the metadata for the split circuits for all the circuit types used in this study. Focusing on the metadata for the \textit{BV} and \textit{HWEA} circuits, it can be observed that both the circuits are split into two subcircuits with the same width and same number of effective qubits (qubits which contribute to the final uncut state). This result outlines an important piece of information that subcircuit evaluation runtime is closely related to how the cuts are made and the width of the resulting subcircuits. Another example of this can be seen by observing the metadata for the \textit{AQFT} circuit, where the difference between the overall width of subcircuits and the width of the uncut circuit is greater than the other circuit types. This is a direct consequence of requiring more cuts to split the original circuit as $ \textit{Width of original circuit} + \textit{Number of cuts} = \textit{Overall width of subcircuits} $.



% Please add the following required packages to your document preamble:
% \usepackage{multirow}
% \usepackage{graphicx}
\begin{table}[!htbp]
\resizebox{\linewidth}{!}{%
\begin{tabular}{|c|ll|ll|ll|ll|}
\hline
\multirow{3}{*}{\textbf{Qubits}} & \multicolumn{2}{c|}{\textbf{Adder}}                                                                                                                                                                            & \multicolumn{2}{c|}{\textbf{AQFT}}                                                                                                                                                                             & \multicolumn{2}{c|}{\textbf{BV}}                                                                                                                                                                               & \multicolumn{2}{c|}{\textbf{HWEA}}                                                                                                                                                                             \\ \cline{2-9} 
                                 & \multicolumn{1}{c|}{\multirow{2}{*}{\textbf{\begin{tabular}[c]{@{}c@{}}Subcirc.\\ width\end{tabular}}}} & \multicolumn{1}{c|}{\multirow{2}{*}{\textbf{\begin{tabular}[c]{@{}c@{}}Eff.\\ Qubits\end{tabular}}}} & \multicolumn{1}{c|}{\multirow{2}{*}{\textbf{\begin{tabular}[c]{@{}c@{}}Subcirc.\\ width\end{tabular}}}} & \multicolumn{1}{c|}{\multirow{2}{*}{\textbf{\begin{tabular}[c]{@{}c@{}}Eff.\\ Qubits\end{tabular}}}} & \multicolumn{1}{c|}{\multirow{2}{*}{\textbf{\begin{tabular}[c]{@{}c@{}}Subcirc.\\ width\end{tabular}}}} & \multicolumn{1}{c|}{\multirow{2}{*}{\textbf{\begin{tabular}[c]{@{}c@{}}Eff.\\ Qubits\end{tabular}}}} & \multicolumn{1}{c|}{\multirow{2}{*}{\textbf{\begin{tabular}[c]{@{}c@{}}Subcirc.\\ width\end{tabular}}}} & \multicolumn{1}{c|}{\multirow{2}{*}{\textbf{\begin{tabular}[c]{@{}c@{}}Eff.\\ Qubits\end{tabular}}}} \\
                                 & \multicolumn{1}{c|}{}                                                                                   & \multicolumn{1}{c|}{}                                                                                & \multicolumn{1}{c|}{}                                                                                   & \multicolumn{1}{c|}{}                                                                                & \multicolumn{1}{c|}{}                                                                                   & \multicolumn{1}{c|}{}                                                                                & \multicolumn{1}{c|}{}                                                                                   & \multicolumn{1}{c|}{}                                                                                \\ \hline
\multirow{2}{*}{\textbf{10}}     & \multicolumn{1}{l|}{6}                                                                                  & 5                                                                                                    & \multicolumn{1}{l|}{8}                                                                                  & 4                                                                                                    & \multicolumn{1}{l|}{6}                                                                                  & 5                                                                                                    & \multicolumn{1}{l|}{6}                                                                                  & 5                                                                                                    \\ \cline{2-9} 
                                 & \multicolumn{1}{l|}{6}                                                                                  & 5                                                                                                    & \multicolumn{1}{l|}{6}                                                                                  & 6                                                                                                    & \multicolumn{1}{l|}{5}                                                                                  & 5                                                                                                    & \multicolumn{1}{l|}{5}                                                                                  & 5                                                                                                    \\ \hline
\multirow{2}{*}{\textbf{12}}     & \multicolumn{1}{l|}{6}                                                                                  & 5                                                                                                    & \multicolumn{1}{l|}{8}                                                                                  & 4                                                                                                    & \multicolumn{1}{l|}{7}                                                                                  & 6                                                                                                    & \multicolumn{1}{l|}{7}                                                                                  & 6                                                                                                    \\ \cline{2-9} 
                                 & \multicolumn{1}{l|}{8}                                                                                  & 7                                                                                                    & \multicolumn{1}{l|}{8}                                                                                  & 8                                                                                                    & \multicolumn{1}{l|}{6}                                                                                  & 6                                                                                                    & \multicolumn{1}{l|}{6}                                                                                  & 6                                                                                                    \\ \hline
\multirow{2}{*}{\textbf{14}}     & \multicolumn{1}{l|}{8}                                                                                  & 7                                                                                                    & \multicolumn{1}{l|}{8}                                                                                  & 4                                                                                                    & \multicolumn{1}{l|}{6}                                                                                  & 5                                                                                                    & \multicolumn{1}{l|}{6}                                                                                  & 5                                                                                                    \\ \cline{2-9} 
                                 & \multicolumn{1}{l|}{8}                                                                                  & 7                                                                                                    & \multicolumn{1}{l|}{10}                                                                                 & 10                                                                                                   & \multicolumn{1}{l|}{9}                                                                                  & 9                                                                                                    & \multicolumn{1}{l|}{9}                                                                                  & 9                                                                                                    \\ \hline
\multirow{2}{*}{\textbf{16}}     & \multicolumn{1}{l|}{8}                                                                                  & 7                                                                                                    & \multicolumn{1}{l|}{12}                                                                                 & 7                                                                                                    & \multicolumn{1}{l|}{10}                                                                                 & 9                                                                                                    & \multicolumn{1}{l|}{10}                                                                                 & 9                                                                                                    \\ \cline{2-9} 
                                 & \multicolumn{1}{l|}{10}                                                                                 & 9                                                                                                    & \multicolumn{1}{l|}{9}                                                                                  & 9                                                                                                    & \multicolumn{1}{l|}{7}                                                                                  & 7                                                                                                    & \multicolumn{1}{l|}{7}                                                                                  & 7                                                                                                    \\ \hline
\multirow{2}{*}{\textbf{18}}     & \multicolumn{1}{l|}{8}                                                                                  & 7                                                                                                    & \multicolumn{1}{l|}{10}                                                                                 & 5                                                                                                    & \multicolumn{1}{l|}{11}                                                                                 & 10                                                                                                   & \multicolumn{1}{l|}{11}                                                                                 & 10                                                                                                   \\ \cline{2-9} 
                                 & \multicolumn{1}{l|}{12}                                                                                 & 11                                                                                                   & \multicolumn{1}{l|}{13}                                                                                 & 13                                                                                                   & \multicolumn{1}{l|}{8}                                                                                  & 8                                                                                                    & \multicolumn{1}{l|}{8}                                                                                  & 8                                                                                                    \\ \hline
\multirow{2}{*}{\textbf{20}}     & \multicolumn{1}{l|}{10}                                                                                 & 9                                                                                                    & \multicolumn{1}{l|}{14}                                                                                 & 9                                                                                                    & \multicolumn{1}{l|}{13}                                                                                 & 12                                                                                                   & \multicolumn{1}{l|}{13}                                                                                 & 12                                                                                                   \\ \cline{2-9} 
                                 & \multicolumn{1}{l|}{12}                                                                                 & 11                                                                                                   & \multicolumn{1}{l|}{11}                                                                                 & 11                                                                                                   & \multicolumn{1}{l|}{8}                                                                                  & 8                                                                                                    & \multicolumn{1}{l|}{8}                                                                                  & 8                                                                                                    \\ \hline
\multirow{2}{*}{\textbf{22}}     & \multicolumn{1}{l|}{10}                                                                                 & 9                                                                                                    & \multicolumn{1}{l|}{16}                                                                                 & 11                                                                                                   & \multicolumn{1}{l|}{13}                                                                                 & 12                                                                                                   & \multicolumn{1}{l|}{13}                                                                                 & 12                                                                                                   \\ \cline{2-9} 
                                 & \multicolumn{1}{l|}{14}                                                                                 & 13                                                                                                   & \multicolumn{1}{l|}{11}                                                                                 & 11                                                                                                   & \multicolumn{1}{l|}{10}                                                                                 & 10                                                                                                   & \multicolumn{1}{l|}{10}                                                                                 & 10                                                                                                   \\ \hline
\multirow{2}{*}{\textbf{24}}     & \multicolumn{1}{l|}{10}                                                                                 & 9                                                                                                    & \multicolumn{1}{l|}{17}                                                                                 & 12                                                                                                   & \multicolumn{1}{l|}{15}                                                                                 & 14                                                                                                   & \multicolumn{1}{l|}{15}                                                                                 & 14                                                                                                   \\ \cline{2-9} 
                                 & \multicolumn{1}{l|}{16}                                                                                 & 15                                                                                                   & \multicolumn{1}{l|}{12}                                                                                 & 12                                                                                                   & \multicolumn{1}{l|}{10}                                                                                 & 10                                                                                                   & \multicolumn{1}{l|}{10}                                                                                 & 10                                                                                                   \\ \hline
\multirow{2}{*}{\textbf{26}}     & \multicolumn{1}{l|}{16}                                                                                 & 15                                                                                                   & \multicolumn{1}{l|}{18}                                                                                 & 13                                                                                                   & \multicolumn{1}{l|}{17}                                                                                 & 16                                                                                                   & \multicolumn{1}{l|}{17}                                                                                 & 16                                                                                                   \\ \cline{2-9} 
                                 & \multicolumn{1}{l|}{12}                                                                                 & 11                                                                                                   & \multicolumn{1}{l|}{13}                                                                                 & 13                                                                                                   & \multicolumn{1}{l|}{10}                                                                                 & 10                                                                                                   & \multicolumn{1}{l|}{10}                                                                                 & 10                                                                                                   \\ \hline
\multirow{2}{*}{\textbf{28}}     & \multicolumn{1}{l|}{14}                                                                                 & 13                                                                                                   & \multicolumn{1}{l|}{19}                                                                                 & 14                                                                                                   & \multicolumn{1}{l|}{18}                                                                                 & 17                                                                                                   & \multicolumn{1}{l|}{18}                                                                                 & 17                                                                                                   \\ \cline{2-9} 
                                 & \multicolumn{1}{l|}{16}                                                                                 & 15                                                                                                   & \multicolumn{1}{l|}{14}                                                                                 & 14                                                                                                   & \multicolumn{1}{l|}{11}                                                                                 & 11                                                                                                   & \multicolumn{1}{l|}{11}                                                                                 & 11                                                                                                   \\ \hline
\multirow{2}{*}{\textbf{30}}     & \multicolumn{1}{l|}{12}                                                                                 & 11                                                                                                   & \multicolumn{1}{l|}{21}                                                                                 & 16                                                                                                   & \multicolumn{1}{l|}{19}                                                                                 & 18                                                                                                   & \multicolumn{1}{l|}{19}                                                                                 & 18                                                                                                   \\ \cline{2-9} 
                                 & \multicolumn{1}{l|}{20}                                                                                 & 19                                                                                                   & \multicolumn{1}{l|}{14}                                                                                 & 14                                                                                                   & \multicolumn{1}{l|}{12}                                                                                 & 12                                                                                                   & \multicolumn{1}{l|}{12}                                                                                 & 12                                                                                                   \\ \hline
\multirow{2}{*}{\textbf{32}}     & \multicolumn{1}{l|}{14}                                                                                 & 13                                                                                                   & \multicolumn{1}{l|}{22}                                                                                 & 16                                                                                                   & \multicolumn{1}{l|}{21}                                                                                 & 20                                                                                                   & \multicolumn{1}{l|}{21}                                                                                 & 20                                                                                                   \\ \cline{2-9} 
                                 & \multicolumn{1}{l|}{20}                                                                                 & 19                                                                                                   & \multicolumn{1}{l|}{16}                                                                                 & 16                                                                                                   & \multicolumn{1}{l|}{12}                                                                                 & 12                                                                                                   & \multicolumn{1}{l|}{12}                                                                                 & 12                                                                                                   \\ \hline
\multirow{2}{*}{\textbf{34}}     & \multicolumn{1}{l|}{14}                                                                                 & 13                                                                                                   & \multicolumn{1}{l|}{24}                                                                                 & 18                                                                                                   & \multicolumn{1}{l|}{22}                                                                                 & 21                                                                                                   & \multicolumn{1}{l|}{22}                                                                                 & 21                                                                                                   \\ \cline{2-9} 
                                 & \multicolumn{1}{l|}{22}                                                                                 & 21                                                                                                   & \multicolumn{1}{l|}{16}                                                                                 & 16                                                                                                   & \multicolumn{1}{l|}{13}                                                                                 & 13                                                                                                   & \multicolumn{1}{l|}{13}                                                                                 & 13                                                                                                   \\ \hline
\end{tabular}%
}
\caption{This table contains the metadata of the split circuits for \textit{Adder}, \textit{AQFT}, \textit{BV} and \textit{HWEA} circuit types. Subcircuit width and the Effective qubits (which contribute to the original uncut state) of that subcircuit are reported for every circuit benchmarked. In this benchmark, every circuit is split into two subcircuits, which is why there are only two rows for every circuit benchmarked.}
\label{tab:results/subcirc-metadata-1}
\end{table}



% Please add the following required packages to your document preamble:
% \usepackage{multirow}
% \usepackage{graphicx}
\begin{table}[!htbp]
\resizebox{\linewidth}{!}{%
\begin{tabular}{|ccl|ll|ll|ll|ll|ll|ll|ll|}
\hline
\multicolumn{3}{|c|}{\textbf{Qubits}}                                                                                                              & \multicolumn{2}{c|}{\textbf{12}} & \multicolumn{2}{c|}{\textbf{15}} & \multicolumn{2}{c|}{\textbf{16}} & \multicolumn{2}{c|}{\textbf{20}} & \multicolumn{2}{c|}{\textbf{24}} & \multicolumn{2}{c|}{\textbf{25}} & \multicolumn{2}{c|}{\textbf{30}} \\ \hline
\multicolumn{1}{|c|}{\multirow{2}{*}{\textbf{Supremacy}}} & \multicolumn{2}{c|}{\textbf{\begin{tabular}[c]{@{}c@{}}Subcirc.\\ width\end{tabular}}} & \multicolumn{1}{l|}{8}    & 7    & \multicolumn{1}{l|}{10}    & 8   & \multicolumn{1}{l|}{12}    & 8   & \multicolumn{1}{l|}{11}   & 13   & \multicolumn{1}{l|}{11}   & 17   & \multicolumn{1}{l|}{18}   & 12   & \multicolumn{1}{l|}{19}   & 16   \\ \cline{2-17} 
\multicolumn{1}{|c|}{}                                    & \multicolumn{2}{c|}{\textbf{\begin{tabular}[c]{@{}c@{}}Eff.\\ Qubits\end{tabular}}}    & \multicolumn{1}{l|}{7}    & 5    & \multicolumn{1}{l|}{7}     & 8   & \multicolumn{1}{l|}{9}     & 7   & \multicolumn{1}{l|}{10}   & 10   & \multicolumn{1}{l|}{8}    & 16   & \multicolumn{1}{l|}{14}   & 11   & \multicolumn{1}{l|}{18}   & 12   \\ \hline
\end{tabular}%
}
\caption{This table contains the metadata of the split circuits for \textit{Supremacy} circuit type. Subcircuit width and the Effective qubits (which contribute to the original uncut state) of each subcircuit are reported for every circuit benchmarked. In this benchmark, every circuit is split into two subcircuits, which is why there are only two rows for every circuit benchmarked.}
\label{tab:results/subcirc-metadata-2}
\end{table}


\begin{table}[!htbp]
\resizebox{\linewidth}{!}{%
\begin{tabular}{|c|c|}
\hline
\textbf{Circuit Width} & \textbf{\begin{tabular}[c]{@{}c@{}}Approximate Memory Required to store Statevector\\ (in GB)\end{tabular}} \\ \hline
10                     & $1.63 \times 10^{-5}$                                                                                                   \\ \hline
12                     & $6.55 \times 10^{-5}$                                                                                                   \\ \hline
14                     & $2.62 \times 10^{-4}$                                                                                                    \\ \hline
15                     & $5.24 \times 10^{-4}$                                                                                                    \\ \hline
16                     & $1.04 \times 10^{-3}$                                                                                                    \\ \hline
18                     & $4.19 \times 10^{-3}$                                                                                                   \\ \hline
20                     & $1.67 \times 10^{-2}$                                                                                                    \\ \hline
22                     & $6.71 \times 10^{-2}$                                                                                                    \\ \hline
24                     & 0.26                                                                                                        \\ \hline
25                     & 0.53                                                                                                        \\ \hline
26                     & 1.07                                                                                                        \\ \hline
28                     & 4.29                                                                                                        \\ \hline
30                     & 17.17                                                                                                       \\ \hline
32                     & 68.71                                                                                                       \\ \hline
34                     & 274.87                                                                                                      \\ \hline
\end{tabular}
}
\caption{Approximate Memory space required to store a Quantum Circuit in Gigabytes}
\label{tab:results/req_storage}
\end{table}



\subsection*{Findings and Discussion}

The benchmarking results show that a full-circuit simulation runs significantly faster than a circuit-splitting simulation for single-node executions. The results provide key insights into the performance of the circuit-splitting simulation method. Parallel execution of subcircuits lets us hide the exponential cost of simulation when the input circuit is cut using less than 6 cuts (at least for the type of circuits benchmarked in this study). Another key insight is that circuits cut similarly (i.e. having similar widths and effective qubits) have similar runtime trends. This hints that the runtime is agnostic of the layout of the gates present in the original circuit and that the runtime can be estimated based on the cut locations, the resulting subcircuit widths and the number of effective qubits. Tables \ref{tab:results/subcirc-metadata-1} and \ref{tab:results/subcirc-metadata-2} show the subcircuit widths after every circuit is cut. It can be observed that the width of the subcircuits is 30-40\% less than the original circuit. This enables us to simulate these circuits using significantly fewer resources than the full-circuit simulation. A 30-40\% reduction in circuit width significantly reduces the amount of resources required to simulate the circuit, as for every qubit reduced, the computational space is reduced by half (Refer Table \ref{tab:results/req_storage}). Considering all factors, simulating large circuits using circuit-splitting and full-circuit simulation methods is a tradeoff between runtime and memory requirements. A large circuit can be simulated faster if there are enormous resources available to simulate it. In contrast, if resources are limited, the circuit-splitting method can be used to simulate the required circuit. In an ideal scenario, we can leverage both these approaches to efficiently simulate large quantum circuits. Large circuits can be split to reduce the resources required to simulate the circuit, and on the other hand, the cuts can be minimized as much as possible to avoid incurring exponential runtime to simulate all subcircuits.
\section{Conclusions}
\label{chap:conclusions}

It is evident from the benchmark results that the Full circuit execution method greatly outperforms the Circuit Splitting simulation method. A major contributing factor is the exponential number of subcircuits to be evaluated depending on the number of cuts made to the original circuit. The parallel evaluation lets us hide this exponential nature but is effective only for circuits split using a small number of cuts. Circuits requiring more than a few cuts still take exponential time to be evaluated.  

An interesting benchmark for future work is evaluating circuits having a width of more than 40 qubits. Full circuit simulations for such circuits would require hundreds of nodes resulting in a considerable amount of MPI communication overhead to exchange data between these nodes. On the other hand, the Circuit Splitting evaluation method would enjoy the benefit of running small circuits capable of being evaluated on a single node. This would give us a better picture of how both the simulation methods work on a larger scale and whether any of these methods is feasible to use for large-scale but efficient quantum circuit simulations.  

A key thing to keep in mind is that the performance of the Circuit Splitting method relies on the efficient cutting of the circuit using a small number of cuts. However, this is not always possible for dense circuits or circuits having two-qubit gates acting on all pairs of qubits (e.g. Grover's algorithm circuit implemented without using ancilla qubits). These are scenarios where circuit-splitting algorithms leveraging gate-cutting can be beneficial. With new circuit-splitting algorithms being published every year, it is only a matter of time before we will be able to classically simulate circuits with a width of more than 100 qubits (albeit with a huge resource cost). This achievement will be a great boost for the scientific community to research and validate new quantum algorithms.

\section*{Acknowledgements}
\label{sec:acknowledgements}

This research used resources of the National Energy Research Scientific Computing Center (NERSC), a Department of Energy Office of Science User Facility using NERSC award DDR-ERCAP0026767.

% \section{Introduction}
% This document is a model and instructions for \LaTeX.
% Please observe the conference page limits. 

% \section{Ease of Use}

% \subsection{Maintaining the Integrity of the Specifications}

% The IEEEtran class file is used to format your paper and style the text. All margins, 
% column widths, line spaces, and text fonts are prescribed; please do not 
% alter them. You may note peculiarities. For example, the head margin
% measures proportionately more than is customary. This measurement 
% and others are deliberate, using specifications that anticipate your paper 
% as one part of the entire proceedings, and not as an independent document. 
% Please do not revise any of the current designations.

% \section{Prepare Your Paper Before Styling}
% Before you begin to format your paper, first write and save the content as a 
% separate text file. Complete all content and organizational editing before 
% formatting. Please note sections \ref{AA}--\ref{SCM} below for more information on 
% proofreading, spelling and grammar.

% Keep your text and graphic files separate until after the text has been 
% formatted and styled. Do not number text heads---{\LaTeX} will do that 
% for you.

% \subsection{Abbreviations and Acronyms}\label{AA}
% Define abbreviations and acronyms the first time they are used in the text, 
% even after they have been defined in the abstract. Abbreviations such as 
% IEEE, SI, MKS, CGS, ac, dc, and rms do not have to be defined. Do not use 
% abbreviations in the title or heads unless they are unavoidable.

% \subsection{Units}
% \begin{itemize}
% \item Use either SI (MKS) or CGS as primary units. (SI units are encouraged.) English units may be used as secondary units (in parentheses). An exception would be the use of English units as identifiers in trade, such as ``3.5-inch disk drive''.
% \item Avoid combining SI and CGS units, such as current in amperes and magnetic field in oersteds. This often leads to confusion because equations do not balance dimensionally. If you must use mixed units, clearly state the units for each quantity that you use in an equation.
% \item Do not mix complete spellings and abbreviations of units: ``Wb/m\textsuperscript{2}'' or ``webers per square meter'', not ``webers/m\textsuperscript{2}''. Spell out units when they appear in text: ``. . . a few henries'', not ``. . . a few H''.
% \item Use a zero before decimal points: ``0.25'', not ``.25''. Use ``cm\textsuperscript{3}'', not ``cc''.)
% \end{itemize}

% \subsection{Equations}
% Number equations consecutively. To make your 
% equations more compact, you may use the solidus (~/~), the exp function, or 
% appropriate exponents. Italicize Roman symbols for quantities and variables, 
% but not Greek symbols. Use a long dash rather than a hyphen for a minus 
% sign. Punctuate equations with commas or periods when they are part of a 
% sentence, as in:
% \begin{equation}
% a+b=\gamma\label{eq}
% \end{equation}

% Be sure that the 
% symbols in your equation have been defined before or immediately following 
% the equation. Use ``\eqref{eq}'', not ``Eq.~\eqref{eq}'' or ``equation \eqref{eq}'', except at 
% the beginning of a sentence: ``Equation \eqref{eq} is . . .''

% \subsection{\LaTeX-Specific Advice}

% Please use ``soft'' (e.g., \verb|\eqref{Eq}|) cross references instead
% of ``hard'' references (e.g., \verb|(1)|). That will make it possible
% to combine sections, add equations, or change the order of figures or
% citations without having to go through the file line by line.

% Please don't use the \verb|{eqnarray}| equation environment. Use
% \verb|{align}| or \verb|{IEEEeqnarray}| instead. The \verb|{eqnarray}|
% environment leaves unsightly spaces around relation symbols.

% Please note that the \verb|{subequations}| environment in {\LaTeX}
% will increment the main equation counter even when there are no
% equation numbers displayed. If you forget that, you might write an
% article in which the equation numbers skip from (17) to (20), causing
% the copy editors to wonder if you've discovered a new method of
% counting.

% {\BibTeX} does not work by magic. It doesn't get the bibliographic
% data from thin air but from .bib files. If you use {\BibTeX} to produce a
% bibliography you must send the .bib files. 

% {\LaTeX} can't read your mind. If you assign the same label to a
% subsubsection and a table, you might find that Table I has been cross
% referenced as Table IV-B3. 

% {\LaTeX} does not have precognitive abilities. If you put a
% \verb|\label| command before the command that updates the counter it's
% supposed to be using, the label will pick up the last counter to be
% cross referenced instead. In particular, a \verb|\label| command
% should not go before the caption of a figure or a table.

% Do not use \verb|\nonumber| inside the \verb|{array}| environment. It
% will not stop equation numbers inside \verb|{array}| (there won't be
% any anyway) and it might stop a wanted equation number in the
% surrounding equation.

% \subsection{Some Common Mistakes}\label{SCM}
% \begin{itemize}
% \item The word ``data'' is plural, not singular.
% \item The subscript for the permeability of vacuum $\mu_{0}$, and other common scientific constants, is zero with subscript formatting, not a lowercase letter ``o''.
% \item In American English, commas, semicolons, periods, question and exclamation marks are located within quotation marks only when a complete thought or name is cited, such as a title or full quotation. When quotation marks are used, instead of a bold or italic typeface, to highlight a word or phrase, punctuation should appear outside of the quotation marks. A parenthetical phrase or statement at the end of a sentence is punctuated outside of the closing parenthesis (like this). (A parenthetical sentence is punctuated within the parentheses.)
% \item A graph within a graph is an ``inset'', not an ``insert''. The word alternatively is preferred to the word ``alternately'' (unless you really mean something that alternates).
% \item Do not use the word ``essentially'' to mean ``approximately'' or ``effectively''.
% \item In your paper title, if the words ``that uses'' can accurately replace the word ``using'', capitalize the ``u''; if not, keep using lower-cased.
% \item Be aware of the different meanings of the homophones ``affect'' and ``effect'', ``complement'' and ``compliment'', ``discreet'' and ``discrete'', ``principal'' and ``principle''.
% \item Do not confuse ``imply'' and ``infer''.
% \item The prefix ``non'' is not a word; it should be joined to the word it modifies, usually without a hyphen.
% \item There is no period after the ``et'' in the Latin abbreviation ``et al.''.
% \item The abbreviation ``i.e.'' means ``that is'', and the abbreviation ``e.g.'' means ``for example''.
% \end{itemize}
% An excellent style manual for science writers is \cite{b7}.

% \subsection{Authors and Affiliations}
% \textbf{The class file is designed for, but not limited to, six authors.} A 
% minimum of one author is required for all conference articles. Author names 
% should be listed starting from left to right and then moving down to the 
% next line. This is the author sequence that will be used in future citations 
% and by indexing services. Names should not be listed in columns nor group by 
% affiliation. Please keep your affiliations as succinct as possible (for 
% example, do not differentiate among departments of the same organization).

% \subsection{Identify the Headings}
% Headings, or heads, are organizational devices that guide the reader through 
% your paper. There are two types: component heads and text heads.

% Component heads identify the different components of your paper and are not 
% topically subordinate to each other. Examples include Acknowledgments and 
% References and, for these, the correct style to use is ``Heading 5''. Use 
% ``figure caption'' for your Figure captions, and ``table head'' for your 
% table title. Run-in heads, such as ``Abstract'', will require you to apply a 
% style (in this case, italic) in addition to the style provided by the drop 
% down menu to differentiate the head from the text.

% Text heads organize the topics on a relational, hierarchical basis. For 
% example, the paper title is the primary text head because all subsequent 
% material relates and elaborates on this one topic. If there are two or more 
% sub-topics, the next level head (uppercase Roman numerals) should be used 
% and, conversely, if there are not at least two sub-topics, then no subheads 
% should be introduced.

% \subsection{Figures and Tables}
% \paragraph{Positioning Figures and Tables} Place figures and tables at the top and 
% bottom of columns. Avoid placing them in the middle of columns. Large 
% figures and tables may span across both columns. Figure captions should be 
% below the figures; table heads should appear above the tables. Insert 
% figures and tables after they are cited in the text. Use the abbreviation 
% ``Fig.~\ref{fig}'', even at the beginning of a sentence.

% \begin{table}[htbp]
% \caption{Table Type Styles}
% \begin{center}
% \begin{tabular}{|c|c|c|c|}
% \hline
% \textbf{Table}&\multicolumn{3}{|c|}{\textbf{Table Column Head}} \\
% \cline{2-4} 
% \textbf{Head} & \textbf{\textit{Table column subhead}}& \textbf{\textit{Subhead}}& \textbf{\textit{Subhead}} \\
% \hline
% copy& More table copy$^{\mathrm{a}}$& &  \\
% \hline
% \multicolumn{4}{l}{$^{\mathrm{a}}$Sample of a Table footnote.}
% \end{tabular}
% \label{tab1}
% \end{center}
% \end{table}

% \begin{figure}[htbp]
% \centerline{\includegraphics{fig1.png}}
% \caption{Example of a figure caption.}
% \label{fig}
% \end{figure}

% Figure Labels: Use 8 point Times New Roman for Figure labels. Use words 
% rather than symbols or abbreviations when writing Figure axis labels to 
% avoid confusing the reader. As an example, write the quantity 
% ``Magnetization'', or ``Magnetization, M'', not just ``M''. If including 
% units in the label, present them within parentheses. Do not label axes only 
% with units. In the example, write ``Magnetization (A/m)'' or ``Magnetization 
% \{A[m(1)]\}'', not just ``A/m''. Do not label axes with a ratio of 
% quantities and units. For example, write ``Temperature (K)'', not 
% ``Temperature/K''.

% \section*{Acknowledgment}

% The preferred spelling of the word ``acknowledgment'' in America is without 
% an ``e'' after the ``g''. Avoid the stilted expression ``one of us (R. B. 
% G.) thanks $\ldots$''. Instead, try ``R. B. G. thanks$\ldots$''. Put sponsor 
% acknowledgments in the unnumbered footnote on the first page.

% \section*{References}

% Please number citations consecutively within brackets \cite{b1}. The 
% sentence punctuation follows the bracket \cite{b2}. Refer simply to the reference 
% number, as in \cite{b3}---do not use ``Ref. \cite{b3}'' or ``reference \cite{b3}'' except at 
% the beginning of a sentence: ``Reference \cite{b3} was the first $\ldots$''

% Number footnotes separately in superscripts. Place the actual footnote at 
% the bottom of the column in which it was cited. Do not put footnotes in the 
% abstract or reference list. Use letters for table footnotes.

% Unless there are six authors or more give all authors' names; do not use 
% ``et al.''. Papers that have not been published, even if they have been 
% submitted for publication, should be cited as ``unpublished'' \cite{b4}. Papers 
% that have been accepted for publication should be cited as ``in press'' \cite{b5}. 
% Capitalize only the first word in a paper title, except for proper nouns and 
% element symbols.

% For papers published in translation journals, please give the English 
% citation first, followed by the original foreign-language citation \cite{b6}.

% \begin{thebibliography}{00}
% \bibitem{b1} G. Eason, B. Noble, and I. N. Sneddon, ``On certain integrals of Lipschitz-Hankel type involving products of Bessel functions,'' Phil. Trans. Roy. Soc. London, vol. A247, pp. 529--551, April 1955.
% \bibitem{b2} J. Clerk Maxwell, A Treatise on Electricity and Magnetism, 3rd ed., vol. 2. Oxford: Clarendon, 1892, pp.68--73.
% \bibitem{b3} I. S. Jacobs and C. P. Bean, ``Fine particles, thin films and exchange anisotropy,'' in Magnetism, vol. III, G. T. Rado and H. Suhl, Eds. New York: Academic, 1963, pp. 271--350.
% \bibitem{b4} K. Elissa, ``Title of paper if known,'' unpublished.
% \bibitem{b5} R. Nicole, ``Title of paper with only first word capitalized,'' J. Name Stand. Abbrev., in press.
% \bibitem{b6} Y. Yorozu, M. Hirano, K. Oka, and Y. Tagawa, ``Electron spectroscopy studies on magneto-optical media and plastic substrate interface,'' IEEE Transl. J. Magn. Japan, vol. 2, pp. 740--741, August 1987 [Digests 9th Annual Conf. Magnetics Japan, p. 301, 1982].
% \bibitem{b7} M. Young, The Technical Writer's Handbook. Mill Valley, CA: University Science, 1989.
% \end{thebibliography}

\printbibliography

% \vspace{12pt}
% \color{red}
% IEEE conference templates contain guidance text for composing and formatting conference papers. Please ensure that all template text is removed from your conference paper prior to submission to the conference. Failure to remove the template text from your paper may result in your paper not being published.

\end{document}
