% \documentclass[sigconf,anonymous,review]{acmart}
\documentclass[sigconf]{acmart}
%%
%% \BibTeX command to typeset BibTeX logo in the docs
\AtBeginDocument{%
  \providecommand\BibTeX{{%
    Bib\TeX}}}

%% Rights management information.  This information is sent to you
%% when you complete the rights form.  These commands have SAMPLE
%% values in them; it is your responsibility as an author to replace
%% the commands and values with those provided to you when you
%% complete the rights form.
% \setcopyright{acmlicensed}
% \copyrightyear{2018}
% \acmYear{2018}
% \acmDOI{XXXXXXX.XXXXXXX}
%% These commands are for a PROCEEDINGS abstract or paper.
% \acmConference[Conference acronym 'XX]{Make sure to enter the correct
%   conference title from your rights confirmation email}{June 03--05,
%   2018}{Woodstock, NY}
%%
%%  Uncomment \acmBooktitle if the title of the proceedings is different
%%  from ``Proceedings of ...''!
%%
%%\acmBooktitle{Woodstock '18: ACM Symposium on Neural Gaze Detection,
%%  June 03--05, 2018, Woodstock, NY}
% \acmISBN{978-1-4503-XXXX-X/2018/06}


\usepackage{subfigure}
\usepackage{enumitem}

\newcommand{\bbb}{\noindent\textbf}

\newcommand{\alg}{LLM-Sketch}


%%
%% end of the preamble, start of the body of the document source.
\begin{document}

%%
%% The "title" command has an optional parameter,
%% allowing the author to define a "short title" to be used in page headers.
\title{\alg{}: Enhancing Network Sketches with LLM}

%%
%% The "author" command and its associated commands are used to define
%% the authors and their affiliations.
%% Of note is the shared affiliation of the first two authors, and the
%% "authornote" and "authornotemark" commands
%% used to denote shared contribution to the research.

\author{Yuanpeng Li}
\affiliation{%
  \institution{Peking University \\ Zhongguancun Laboratory}
  \city{Beijing}
  \country{China}
}

\author{Zhen Xu}
\affiliation{%
  \institution{Zhejiang University}
  \city{Hangzhou}
  \country{China}
}

\author{Zongwei Lv}
\affiliation{%
  \institution{Peking University}
  \city{Beijing}
  \country{China}
}

\author{Yannan Hu}
\authornote{Tong Yang (yangtong@pku.edu.cn) and Yannan Hu (huyn@zgclab.edu.cn) are the corresponding authors.}
\affiliation{%
  \institution{Zhongguancun Laboratory}
  \city{Beijing}
  \country{China}
}

\author{Yong Cui}
\affiliation{%
  \institution{Tsinghua University \\ Zhongguancun Laboratory}
  \city{Beijing}
  \country{China}
}

\author{Tong Yang}
\authornotemark[1]
\affiliation{%
  \institution{Peking University \\ Zhongguancun Laboratory}
  \city{Beijing}
  \country{China}
}

%%
%% By default, the full list of authors will be used in the page
%% headers. Often, this list is too long, and will overlap
%% other information printed in the page headers. This command allows
%% the author to define a more concise list
%% of authors' names for this purpose.
\renewcommand{\shortauthors}{Li et al.}

%%
%% The abstract is a short summary of the work to be presented in the
%% article.
\begin{abstract}
  \begin{abstract}
Advancements in DNA sequencing technologies have significantly improved our ability to decode genomic sequences. However, the prediction and interpretation of these sequences remain challenging due to the intricate nature of genetic material. Large language models (LLMs) have introduced new opportunities for biological sequence analysis. Recent developments in genomic language models have underscored the potential of LLMs in deciphering DNA sequences. Nonetheless, existing models often face limitations in robustness and application scope, primarily due to constraints in model structure and training data scale. To address these limitations, we present \textbf{Gener}\textit{ator}, a generative genomic foundation model featuring a context length of 98k base pairs (bp) and 1.2B parameters. Trained on an expansive dataset comprising 386B bp of eukaryotic DNA, the \textbf{Gener}\textit{ator} demonstrates state-of-the-art performance across both established and newly proposed benchmarks. The model adheres to the central dogma of molecular biology, accurately generating protein-coding sequences that translate into proteins structurally analogous to known families. It also shows significant promise in sequence optimization, particularly through the prompt-responsive generation of enhancer sequences with specific activity profiles. These capabilities position the \textbf{Gener}\textit{ator} as a pivotal tool for genomic research and biotechnological advancement, enhancing our ability to interpret and predict complex biological systems and enabling precise genomic interventions. Implementation details and supplementary resources are available at \url{https://github.com/GenerTeam/GENERator}.
\keywords{DNA, Genomics, Foundation model, Generative model}
\vspace{12pt}
\end{abstract}



\end{abstract}

%%
%% The code below is generated by the tool at http://dl.acm.org/ccs.cfm.
%% Please copy and paste the code instead of the example below.
%%
% \begin{CCSXML}
% <ccs2012>
%  <concept>
%   <concept_id>00000000.0000000.0000000</concept_id>
%   <concept_desc>Do Not Use This Code, Generate the Correct Terms for Your Paper</concept_desc>
%   <concept_significance>500</concept_significance>
%  </concept>
%  <concept>
%   <concept_id>00000000.00000000.00000000</concept_id>
%   <concept_desc>Do Not Use This Code, Generate the Correct Terms for Your Paper</concept_desc>
%   <concept_significance>300</concept_significance>
%  </concept>
%  <concept>
%   <concept_id>00000000.00000000.00000000</concept_id>
%   <concept_desc>Do Not Use This Code, Generate the Correct Terms for Your Paper</concept_desc>
%   <concept_significance>100</concept_significance>
%  </concept>
%  <concept>
%   <concept_id>00000000.00000000.00000000</concept_id>
%   <concept_desc>Do Not Use This Code, Generate the Correct Terms for Your Paper</concept_desc>
%   <concept_significance>100</concept_significance>
%  </concept>
% </ccs2012>
% \end{CCSXML}

% \ccsdesc[500]{Do Not Use This Code~Generate the Correct Terms for Your Paper}
% \ccsdesc[300]{Do Not Use This Code~Generate the Correct Terms for Your Paper}
% \ccsdesc{Do Not Use This Code~Generate the Correct Terms for Your Paper}
% \ccsdesc[100]{Do Not Use This Code~Generate the Correct Terms for Your Paper}

%%
%% Keywords. The author(s) should pick words that accurately describe
%% the work being presented. Separate the keywords with commas.
\keywords{Network stream mining, Sketches, Flow classification, LLM}
%% A "teaser" image appears between the author and affiliation
%% information and the body of the document, and typically spans the
%% page.

%%
%% This command processes the author and affiliation and title
%% information and builds the first part of the formatted document.
\maketitle

\section{Introduction}

Chain-of-Thought (CoT) prompting~\cite{Nye:2021, cot, Kojima:2022cotzero} has emerged as a cornerstone strategy for enhancing Large Language Models (LLMs) in complex reasoning tasks. By eliciting step-by-step inference, CoT enables LLMs to decompose intricate problems into manageable subtasks, thereby improving their problem-solving performance~\cite{Yao:2023tot, Wang:2023self-consistency, Zhou:2023least, Shinn:2023Reflexion}. Recent advancements, such as OpenAI's o1~\cite{o1} and DeepSeek-R1~\cite{deepseekr1}, further demonstrate that scaling up CoT lengths from hundreds to thousands of reasoning steps could continuously improve LLM reasoning. These breakthroughs have underscored CoT’s potential to advance LLM capabilities, expanding the boundaries of AI-driven problem-solving.

\begin{figure}[t]
\centering
    \includegraphics[width=0.95\columnwidth]{fig/intro.pdf}
    \caption{In contrast to vanilla CoT that generates all reasoning tokens sequentially, \method enables LLMs to \textit{skip} tokens with less semantic importance (\textit{e.g.,} \includegraphics[width=7pt]{fig/token.pdf}~) and learn shortcuts between critical reasoning tokens, facilitating controllable CoT compression.}
    \label{fig:intro}
\end{figure}

Despite its effectiveness, the increased length of CoT sequences introduces substantial computational overhead. Due to the autoregressive nature of LLM decoding, longer CoT outputs lead to proportional increases in both inference latency and memory footprints of key-value cache. Additionally, the quadratic computational cost of attention layers further exacerbates this burden. These issues become particularly pronounced when CoT sequences extend into thousands of reasoning steps, resulting in significant computational costs and prolonged response times. While prior research has explored methods for selectively skipping reasoning steps~\cite{Ding:2024cotshortcut, liu2024skipstep}, recent findings~\cite{jin:2024cotlength, Merrill:2024cotlength} suggest that such reductions may conflict with test-time scaling~\cite{o1-blog, snell2025scaling}, ultimately impairing LLM reasoning performance. Therefore, striking an optimal balance between CoT efficiency and reasoning accuracy remains a critical open challenge.

In this work, we delve into CoT efficiency and seek the answer to an important question: \textit{``Does every token in the CoT output contribute equally to deriving the answer?''} We empirically analyze the semantic importance of tokens within CoT outputs and reveal that their contributions to the reasoning performance vary, as depicted in Figure 2. Building on this insight, we introduce \method, a simple yet effective approach that enables LLMs to \textit{skip} less important tokens within CoT sequences and learn shortcuts between critical reasoning tokens, thereby allowing for controllable CoT compression with adjustable ratios. Specifically, as shown in Figure~\ref{fig:intro}, \method constructs compressed CoT training data with various compression ratios, by pruning unimportance tokens from original LLM CoT trajectories. Then, it conducts a general supervised fine-tuning process on target LLMs with this training data, facilitating LLMs to automatically trim redundant tokens during reasoning.

We conduct extensive experiments across various models, including LLaMA-3.1-8B-Instruct and the Qwen2.5-Instruct series, using two widely recognized math reasoning benchmarks: GSM8K and MATH-500. The results validate the effectiveness of \method in compressing CoT outputs while maintaining robust reasoning performance. Notably, Qwen2.5-14B-Instruct exhibits almost \textbf{NO} performance drop (less than $0.4\%$) with a $\bm{40\%}$ reduction in token usage on GSM8K. On the challenging MATH-500 dataset, LLaMA-3.1-8B-Instruct effectively reduces CoT token usage by $\bm{30}\%$ with a performance decline of less than $4\%$, resulting in a $\bm{1.4}\times$ inference speedup. Further analysis underscores the coherence of \method in specified compression ratios and its potential scalability with stronger compression techniques.

\method is distinguished by its low training cost. For Qwen2.5-14B-Instruct, \method fine-tunes only 0.2\% of the model's parameters using LoRA. The size of the compressed CoT training data is no larger than that of the original training set, with 7,473 examples in GSM8K and 7,500 in MATH. The training is completed in approximately 2 hours for the 7B model and 2.5 hours for the 14B model on two 3090 GPUs. These characteristics make \method an efficient and reproducible approach, suitable for use in efficient and cost-effective LLM deployment.

To sum up, our key contributions are:
\begin{enumerate}
    \item To the best of our knowledge, this work is the \textit{first} to investigate the potential of enhancing CoT efficiency through \textit{token skipping}, inspired by the varying semantic importance of tokens in CoT trajectories of LLMs.
    \item We introduce \method, a simple yet effective approach that enables LLMs to skip redundant tokens within CoTs and learn shortcuts between critical tokens, facilitating CoT compression with adjustable ratios.
    \item Our experiments validate the effectiveness of \method. When applied to Qwen2.5-14B-Instruct, \method reduces reasoning tokens by $40\%$ (from 313 to 181) on GSM8K, with less than a $0.4\%$ performance drop.
\end{enumerate}

\subsection{Gene Expression Classification with ML models}
Gene expression classification \cite{do2024enhancing,do2023ensemble,huynh2019novel} lies at the forefront of biomedical research, offering profound insights into the molecular mechanisms underlying various diseases. ML models have become indispensable in this domain, as they can uncover complex patterns within vast and high-dimensional gene expression datasets. However, these datasets often contain a plethora of features, many of which are redundant or irrelevant, potentially obscuring the most critical biological signals and leading to overfitting. Consequently, feature selection becomes imperative—it refines the dataset by isolating the most informative genes, thereby enhancing model accuracy, interpretability, and computational efficiency. By focusing solely on the pivotal biomarkers, this research is able to achieve more reliable predictive outcomes. In this paper, we investigate and evaluate the classification with various ML techniques. Namely, we experiment our selected features with ML algorithms, i.e., SVM \cite{vapnik1995support}, Random Forest \cite{breiman2001random}, XGB \cite{chen2015xgboost}, Gradient Boosting \cite{friedman2002stochastic}.

\begin{definition}[Classification]
Let \( D = (X, y) \) be a dataset where \( X \subseteq \mathbb{R}^n \) is the feature space and \( y \in \mathcal{Y} = \{1,2,\dots,k\} \) represents the class labels. A classifier is a function
\[
f: X \to \mathcal{Y},
\]
that assigns a predicted label \( \hat{y} = f(x) \) to each input \( x \in X \). The function \( f \) is learned from the labeled examples
\[
D = \{(x_i, y_i) \mid x_i \in X,\; y_i \in \mathcal{Y},\; i = 1, \dots, N\},
\]
by minimizing a loss function \( \ell: \mathcal{Y} \times \mathcal{Y} \to \mathbb{R}_{\ge 0} \) that quantifies the error between the predicted and true labels. Once trained, \( f \) is used to classify new, unseen inputs.
\end{definition}

% \begin{definition}[Classification Using Machine Learning]
% Let \( D_{\text{selected}} = (X_{\text{selected}}, y) \) be the dataset with features \( X_{\text{selected}} \subseteq X^* \) as determined by LIME. A classifier is a function 
% \[
% f: X_{\text{selected}} \to \mathcal{Y},
% \]
% that assigns a predicted label \( \hat{y} = f(x) \) to each input \( x \in X_{\text{selected}} \). The classifier is trained on the labeled examples
% \[
% D_{\text{selected}} = \{(x_i, y_i) \mid x_i \in X_{\text{selected}},\; y_i \in \mathcal{Y},\; i = 1, \dots, N\},
% \]
% by minimizing a loss function \( \ell: \mathcal{Y} \times \mathcal{Y} \to \mathbb{R}_{\ge 0} \) that measures the discrepancy between the predicted and true labels. The trained classifier is then used to predict the classes of new, unseen instances.
% \end{definition}


Feature selection is crucial before classification begins. Our study focuses on two techniques: Boruta and LIME. 
% Boruta is chosen for its robustness in identifying all relevant features in high-dimensional datasets, ensuring no important predictor is missed. LIME is used for its ability to provide interpretable, local explanations of model predictions, which is essential for evaluating feature importance. 
We now introduce Boruta and LIME in the following sections.

\subsection{Leveraging Boruta for Robust Feature Extraction}
Boruta \cite{kursa2010boruta,zhou2023diabetes} is a powerful wrapper-based feature selection algorithm designed to identify all truly relevant variables in a dataset. By comparing the importance of actual features with that of randomly generated ``shadow'' features, Boruta systematically filters out irrelevant variables while preserving essential predictors. This rigorous selection process is particularly valuable in high-dimensional applications, such as gene expression classification, where capturing meaningful signals is crucial. For clarity, we formally define Boruta as follows:
\begin{definition}[Boruta Feature Selection]
Let \( D = (X, y) \) be a dataset with features \( X = \{x_1, x_2, \dots, x_p\} \) and target \( y \). The Boruta algorithm identifies all relevant features in \( X \) as follows:
\begin{enumerate}
    \item \textbf{Shadow Feature Generation:} For each \( x_i \in X \), create a shadow feature \( x_i^{\text{shadow}} \) by randomly permuting its values, forming the set \( X^{\text{shadow}} \).
    \item \textbf{Importance Estimation:} Train a classifier (e.g., Random Forest) on the combined set \( X \cup X^{\text{shadow}} \) and compute the importance score \( I(z) \) for each \( z \).
    \item \textbf{Feature Comparison:} For each \( x_i \), define
    \[
    I^{\text{shadow}}_{\max} = \max_{z \in X^{\text{shadow}}} I(z).
    \]
    Then classify \( x_i \) as \emph{relevant} if \( I(x_i) \) is significantly greater than \( I^{\text{shadow}}_{\max} \), \emph{irrelevant} if significantly lower, or \emph{tentative} otherwise.
    \item \textbf{Iteration:} Remove irrelevant and tentative features and repeat until all features are decisively classified.
\end{enumerate}
The final selected subset \( X^* \subseteq X \) comprises all features deemed relevant.
\end{definition}

After applying the Boruta algorithm, we retain only the relevant features (confirmed) and excluded the tentative and irrelevant features (rejected). To further enhance the selection of features in \(X^*\), we employed the AI explanation technique outlined in the following section.

% \begin{definition}[Boruta Feature Selection]
% Given a dataset \( D = (X, y) \) with original features \( X = \{ x_1, x_2, \dots, x_p \} \), Boruta augments \( X \) by creating shadow features \( X^{\text{shadow}} = \{ x_1^{\text{shadow}}, \dots, x_p^{\text{shadow}} \} \) via random permutation. A model \( M \) (e.g., Random Forest) is then trained on \( X \cup X^{\text{shadow}} \) to compute importance scores \( I(z) \) for every feature \( z \). For each \( x_i \in X \), if \( I(x_i) \) is significantly greater than the maximum shadow importance \( I^{\text{shadow}}_{\max} = \max_{z \in X^{\text{shadow}}} I(z) \), then \( x_i \) is marked as relevant; otherwise, it is rejected or considered tentative. Iterating this process yields the final set of selected features \( X^* \subseteq X \).
% \end{definition}

\subsection{XAI for Feature Selection}
Explainable AI (XAI) \cite{dwivedi2023explainable,zacharias2022designing} represents a forefront of AI research, aiming to elucidate the decision-making processes of complex models. In the context of gene expression classification, where feature selection is pivotal to model performance and interpretability, our study leverages LIME—Local Interpretable Model-Agnostic Explanations—to demystify and select critical features. LIME approximates the behavior of a sophisticated, black-box model with a simpler, locally interpretable surrogate, thereby pinpointing the most influential predictors in the vicinity of a given instance. This approach enhances the transparency of the model's predictions and facilitates a more informed and rigorous feature selection process, ultimately contributing to both improved accuracy and trustworthiness of the classification system.  Now, we provide a formal definition of LIME as follows:

% \begin{definition}[LIME-based Feature Selection]
% Let \( D = (X, y) \) be a dataset and \( f: X \to \mathcal{Y} \) a trained black-box classifier, where \( X \subseteq \mathbb{R}^p \) and \( \mathcal{Y} = \{1,2,\dots,k\} \). For a given instance \( x \in X \), LIME constructs an interpretable surrogate model \( g \) from a simple model class \( G \) (typically linear), expressed as
% \[
% g(z) = w_0 + \sum_{j=1}^{p} w_j z_j.
% \]
% The surrogate \( g \) is fitted by minimizing the weighted loss
% \[
% \min_{g \in G} \sum_{z \in Z_x} \pi_x(z) \left( f(z) - g(z) \right)^2 + \Omega(g),
% \]
% where \( Z_x \) is a set of perturbed samples around \( x \), \( \pi_x(z) \) is a proximity measure between \( z \) and \( x \), and \( \Omega(g) \) is a regularization term enforcing simplicity. The absolute coefficients \( |w_j| \) quantify the local importance of each feature, thus guiding feature selection.
% \end{definition}
\begin{definition}[LIME-based Feature Selection]
Let \( D^* = (X^*, y) \) be the dataset resulting from Boruta, where \( X^* \subseteq \mathbb{R}^{p^*} \) is the set of relevant features. Given a trained black-box classifier \( f: X^* \to \mathcal{Y} \) and an instance \( x \in X^* \), LIME constructs an interpretable surrogate model \( g \in G \) (typically linear), expressed as
\[
g(z) = w_0 + \sum_{j=1}^{p^*} w_j z_j,
\]
by solving the optimization problem
\[
\min_{g \in G} \sum_{z \in Z_x} \pi_x(z) \left( f(z) - g(z) \right)^2 + \Omega(g),
\]
where \( Z_x \) is a set of perturbed samples in the neighborhood of \( x \), \( \pi_x(z) \) is a proximity measure, and \( \Omega(g) \) enforces simplicity. The absolute coefficients \( |w_j| \) indicate the local importance of each feature, enabling a further refined selection \( X_{\text{selected}} \subseteq X^* \) for classification.
\end{definition}


To clarify, our choice of LIME for feature selection arises from the critical question of determining the optimal number of features for the model. In this context, assessing the local importance of each vector proves to be the most effective strategy, leading us to introduce the BOLIMES algorithm. The following section will provide a comprehensive explanation of the BOLIMES algorithm and its application.

%--------------------





\section{The \alg{} Algorithm}


In this section, we first propose the data structure and operations of \alg{}. Then we present how the flow classifier is designed. After that, we describe the application of \alg{}.


\subsection{Data Structure and Operations}


\bbb{Data structure:}
%
As shown in Figure~\ref{fig:workflow}, the data structure of \alg{} consists of three parts: a heavy part, a light part, and a flow classifier.
%
The heavy part is a key-value (KV) table with \(w_h\) buckets. Each bucket contains \(d_h\) cells, and each cell records a flow, including its flow ID \(f\) and flow size \(\hat{n}\). The heavy part is also associated with a hash function \(h(.) (1 \leqslant h(.) \leqslant w_h)\), which maps flows to buckets.
%
The light part is a CMS, which maintains the sizes of small flows using small counters (e.g., 8-bit) to save memory.
%
The flow classifier is a model that infers whether an incoming packet belongs to a large flow or a small flow. Its design is detailed in Section \ref{sec:alg:model}. Ideally, large flows are recorded in the heavy part, whereas the light part is used only for small flows.


\begin{figure}[!ht]
    \centering  
    \includegraphics[width=\linewidth]{Figures/workflow.pdf}
    \caption{Workflow of \alg{}.}
    \label{fig:workflow}
\end{figure}


\bbb{Insertion:}
%
When inserting a packet of flow \(f\), \alg{} locates the mapped bucket \(B[h(f)]\) using the hash function \(h\). There are three cases:


\textit{Case 1:} If \(f\) is already recorded in \(B[h(f)]\), \alg{} simply increments its flow size by 1.


\textit{Case 2:} If \(f\) is not in \(B[h(f)]\) and there is an empty cell, \alg{} inserts \((f, 1)\) into that cell.


\textit{Case 3:} If \(f\) is not in \(B[h(f)]\) and all cells in the bucket are full, \alg{} uses the flow classifier to predict whether \(f\) is a large flow or a small flow. Based on the classifier’s output, there are two sub-cases:
%
1) If \(f\) is a large flow, let \(f_{min}\) be the flow with the minimum flow size in \(B[h(f)]\). \alg{} evicts \(f_{min}\) from the bucket, inserts it into the light part, and then inserts \((f, 1)\) into \(B[h(f)]\).
%
2) If \(f\) is a small flow, \alg{} directly inserts \(f\) into the light part.


\bbb{Query:}
%
When querying the size of a flow \(f\), \alg{} first checks if \(f\) is in the heavy part. If so, it reports the recorded flow size; otherwise, it reports the result from the light part.


\begin{figure}[!ht]
    \centering  
    \includegraphics[width=0.8\linewidth]{Figures/example.pdf}
    \caption{An example of \alg{}.}
    \label{fig:example}
\end{figure}




\textit{Example 1:} Figure~\ref{fig:example} illustrates the different cases in the insertion process of \alg{}.
%
When inserting a packet of flow \(f1\), \alg{} computes the hash function \(h\) to locate bucket \(B1\). Since \(f1\) is already recorded in \(B1\), \alg{} simply increments its flow size by 1.


\textit{Example 2:} When inserting a packet of \(f2\), \alg{} locates bucket \(B2\). Since \(B2\) has an empty cell, \alg{} inserts \((f2, 1)\) into that cell.


\textit{Example 3:} When inserting a packet of \(f5\), \alg{} locates bucket \(B3\). Since \(B3\) is full, \alg{} uses the flow classifier to predict that \(f5\) is a small flow and therefore inserts \((f5, 1)\) into the light part.


\textit{Example 4:} When inserting a packet of \(f8\), \alg{} locates bucket \(B4\). Since \(B4\) is full, \alg{} predicts \(f8\) as a large flow. \alg{} then evicts the flow with the minimum flow size (i.e., \(f7\)) from \(B4\), inserts \((f8, 1)\) into \(B4\), and inserts \(f7\) into the light part.






\bbb{Optimization: Large-flow Locking.}
%
In the insertion process described above, if a hash collision occurs, \alg{} evicts the flow with the smallest recorded size. Although this approach generally works well, it may inadvertently evict newly arrived flows that are actually large but have not yet accumulated a significant size. To address this issue, we introduce a \textit{lock flag} in each cell of the heavy part. This flag tracks how often a flow is predicted to be large, thereby reducing the likelihood of evicting flows that were previously identified as large, even if their current size is still small.


\textit{Lock flag update:}
%
Whenever a packet is inserted and its flow is (re)classified, we update the lock flag based on the classifier’s prediction and the flow’s recorded size.
%
Let \(\hat{y} \in \{0, 1\}\) be the predicted label for this packet, where \(\hat{y} = 1\) indicates large flow and \(\hat{y} = 0\) indicates small flow.
%
Recall that \(\hat{n}\) represents the flow's recorded size.
%
Then the lock flag \(L \in \{0, 1\}\) is updated as follows:
%
\[
L \leftarrow
\begin{cases} 
1, & \text{w.p. } \frac{L \cdot \hat{n} + \hat{y}}{\hat{n} + 1},\\
0, & \text{otherwise}.
\end{cases}
\]
%
This rule can be viewed as an unbiased estimator of the fraction of times the flow is predicted to be large, accumulated over its updates (See Theorem~\ref{theorem:flag}). If \(L = 1\) after the update, we treat the flow as large and therefore lock it. Otherwise, if \(L = 0\), it is more likely to be a small flow and can be safely evicted if necessary.


\textit{Eviction policy:}
%
When a hash collision occurs and \alg{} needs to evict a flow from a full bucket, it first checks whether any flows in that bucket have \(L = 0\). 1) If there is at least one unlocked flow (\(L = 0\)), \alg{} evicts tthe one with the smallest size among them. 2) If all flows in the bucket are locked (\(L = 1\)), \alg{} must evict the flow with the minimum size regardless of its lock flag. Although the latter scenario should be rare, it can still occur due to the classification errors or the probabilistic nature of the lock flag update.






\subsection{Model Design}
\label{sec:alg:model}


We choose to adapt a Large Language Model (LLM) as our flow classifier due to its ability to capture complex patterns in packet headers. By leveraging an LLM, we can process each packet header in a contextual manner, enabling the classifier to learn nuanced relationships that simpler models might overlook. Furthermore, the inherent flexibility of LLMs makes them well-suited for handling packet headers of varying lengths and formats.




\bbb{Embedding:}
Since the raw packet header data cannot be directly interpreted by a language model, we introduce an embedding layer that transforms the packet headers into token embeddings. Specifically, we treat the packet header as a binary string and segment it into two-byte chunks, each serving as a token for the embedding layer. This approach circumvents the need for a cumbersome, field-by-field parsing.
% 
In practice, to prevent overfitting to specific IP addresses, we remove the source and destination IP fields before feeding the remaining header data into the model. Consequently, the classifier focuses on more generalizable features—such as transport-layer information—rather than memorizing particular endpoints in the training data.




\bbb{Objective Function:}
% 
A straightforward strategy might be to define a hard threshold \( T \) (e.g., 64) to classify flows as large (\( \geqslant T \)) or small (\( < T \)) categories. However, directly optimizing for a strict binary cutoff can lead to the following issues:

\begin{itemize}[leftmargin=*]
    \item Flows near the threshold (e.g., those with sizes 60-70) often share similar characteristics, making a single sharp boundary somewhat arbitrary.

    \item Misclassifying flows near the threshold has relatively little impact on the overall sketch accuracy, so aggressively fitting a binary boundary can introduce unnecessary complexity.
\end{itemize}


To address these concerns, we adopt a \textit{soft-label approach} that smooths the discrete large-versus-small boundary. Concretely, we assign a label to each flow based on
%
\[
\text{label} = \sigma \Bigl(a \bigl((\log(n) - \log(T) \bigr)\Bigr),
\]
%
where \(n\) is the flow size, \(T\) is the threshold, \(\sigma(\cdot)\) is the sigmoid function, and \(a\) is a scaling parameter. This design has several advantages:


\begin{itemize}[leftmargin=*]
\item Continuity: Rather than a hard 0/1 label, flows receive labels in the continuous range \((0, 1)\), enabling a smooth transition around the threshold.

\item Reduced sensitivity: Flows that are significantly larger than \(T\) yield labels near 1, while those much smaller than \(T\) yield labels near 0. Flows in the ambiguous region around \(T\) hover near 0.5, making misclassifications less punitive.

\item Smoother optimization: Training as a regression task on these soft labels typically exhibits more stable convergence than a strict classification objective.
\end{itemize}



In practice, this soft-label mechanism helps the classifier learn a nuanced notion of flow size, rather than fixating on a single, potentially noisy threshold. Large flows (e.g., above 1,000) naturally produce labels close to 1, whereas small flows (e.g., below 5) produce labels close to 0. Flows near \(T\) fall around 0.5, diminishing the adverse impact of uncertain classifications. Consequently, the classifier achieves better overall performance and adaptability across diverse network environments.




\subsection{Application}


In this section, we describe how to apply \alg{} to 3 typical tasks: flow size query, heavy hitter query, and hierarchical heavy hitter query.


\bbb{Flow size query:}
%
\alg{} can be used directly to measure flow sizes.


\textit{Optimization: using fingerprints.}
%
Following many existing works \cite{dhs, mimosketch}, \alg{} also supports the use of fingerprints in place of full flow IDs. This is particularly beneficial when the original ID is large (e.g., the 13-byte 5-tuple).
%
Although fingerprints may introduce collisions, they substantially reduce memory usage. Consequently, \alg{} can achieve higher accuracy under the same memory budget compared with recording the full flow IDs.




\bbb{Heavy hitter query:}
%
For heavy hitter query, \alg{} maintains the same insertion procedure described earlier.
%
When querying heavy hitters, \alg{} simply the heavy part to find all flows whose recorded sizes exceed the given threshold. Those flows are subsequently reported as heavy hitters.




\bbb{Hierarchical heavy hitter (HHH) query:}
%
To support HHH query, \alg{} replaces the light part with a CocoSketch \cite{cocosketch}. The insertion process remains primarily unchanged, except that flows which would originally be inserted into CMS are now inserted into CocoSketch. When performing an HHH query, \alg{} first merges the heavy and light parts into a single key-value table and then obtains HHH using the aggregation approach proposed by CocoSketch.




% \bbb{DDoS victim query:}
% %
% \alg{} works in conjunction with a Bloom filter (BF) \cite{bloomfilter} for DDoS victim query, where BF is used for deduplication and \alg{} is used for counting. When a packet with ID \(f = \langle src, dst\rangle\) is inserted, we first query BF to check whether \(f\) has already been inserted. If BF returns true, i.e. \(f\) is already recorded, then no action is taken. If the BF returns false, we insert \(f\) into BF, and \((dst, 1)\) into \alg{}.
% %
% When querying for DDoS, \alg{} scans the heavy part, and reports the set of flows whose corresponding values are greater than the threshold \(T_{DDoS}\).







\begin{figure}
    \centering
    \includegraphics[width=\columnwidth]{Figures/annotations_of_notes_narrow.png}
    \caption{Mean scores of community annotations of misleading posts.}
    \label{fig:annotations}
\end{figure}

We analyse the dataset prepared in \cref{sec:dataset} to answer the two research questions defined in \cref{sec:introduction}.

\subsection{RQ1: To what degree do community notes rely on fact-checkers?}
\label{sec:analysis_rq1}

According to \cref{fig:link_types}, at least 5\% of all English community notes contain an external link to professional fact-checkers. This number grows to 7\% when only considering notes rated as `helpful' (\cref{fig:link_types_helpful} in \cref{app:additional_material}). Conversely, only 1\% of notes rated as `not helpful' contain a fact-checking source (\cref{fig:link_types_not_helpful} in \cref{app:additional_material}). These figures are significantly larger than what was reported in previous studies (1.2\% \citep{kangur_who_2024}), possibly because \citet{kangur_who_2024} utilise a smaller dataset of fact-checking agencies and classify fact-checking divisions of popular journals as ``news''. The results imply that notes incorporating fact-checking sources are generally considered more helpful. 

We further assess whether notes with fact-checking sources are perceived to be of higher quality by analysing individual user ratings of notes both with and without such sources. Specifically, we collect user ratings for a balanced
(i.e., including of a fact-checking source or not) sample of 20K notes rated by at least 50 users, 
% , with half containing a link to professional fact-checking and the other half without.
and calculated the average ratings for the notes. As can be seen in \cref{fig:notes_individual_ratings} in \cref{app:additional_material}, community notes with fact-checking sources are generally rated higher than their counterparts. Interestingly, while notes with fact-checking links are more likely to be regarded as having a good source (higher \textit{HelpfulGoodSources}), they are also more likely to be rated as \textit{notHelpfulSourcesMissingOrUnreliable}.  \cref{tab:notes_with_bad_source.} in \cref{app:additional_material} contains a sample of such notes. 


\subsection{RQ2: What are the traits of posts and notes that rely on fact-checking sources?}
\label{sec:analysis_rq2}

\begin{table*}
    \centering
    \resizebox{1.0\textwidth}{!}
    {%
    \fontsize{8}{8}\selectfont
    \sisetup{table-format = 3.2, group-minimum-digits=3}
    \begin{tabular}{p{5cm}p{7cm}rrrrr}
    \toprule
    Tweet & Note & \rotatebox[origin=r]{270}{misleadingUnverifiedClaimAsFact} & \rotatebox[origin=r]{270}{misleadingOutdatedInformation} & \rotatebox[origin=r]{270}{misleadingFactualError} & \rotatebox[origin=r]{270}{misleadingSatire} & \rotatebox[origin=r]{270}{Fact Checking source} \\ \midrule
    The NASA War Document is absolutely terrifying \url{https://t.co/...} & misrepresenting a presentation by NASA scientist Dennis Bushnell, The lecture was not detailing plans by NASA to attack the world it was a lecture for defense industry professionals, and how defense tactics might rise to meet evolving threats in the future.   \url{https://leadstories.com/hoax-alert/2021/06/fact-check-the-future-is-now-is-not-a-nasa-war-document-plan-for-world-domination-and-phasing-out-of-humans.html} & \cmark & \xmark & \xmark & \xmark & \cmark \\ \addlinespace
    BREAKING NEWS: International Criminal Investigation calls on every public citizen to recommend indictments for Bill Gates, Anthony Fauci, Pfizer, BlackRock, Tedros and Christian Drosten for pushing everyone to receive the ineffective highly dangerous lethal experimental vaccines... & Video has been fact-checked by USA Today, was found to be misleading, and promotes a conspiracy theory about COVID ... \url{https://ca.movies.yahoo.com/movies/fact-check-viral-video-promotes-204414488.html} & \cmark & \xmark & \xmark & \xmark & \cmark \\ \addlinespace
    1) California is RED.
    It is just because of the MASSIVE Election Fraud that stupid, brainwashed people believe Calif. is blue. Joe Biden won only in the SFO Bay area ... & The map shows the results of Reagan's reelection in 1984, not Biden's election in 2020.  \url{https://en.wikipedia.org/wiki/1984\_United\_States\_presidential\_election\_in\_California} & \xmark & \cmark & \xmark & \xmark & \xmark \\ \addlinespace
    Davis blows up \$100,000 fireworks in Kai Cenat setup During the Mr Beast Stream ... & The second photo is from a house fire in Atlanta in 2019. \url{https://www.11alive.com/article/news/local/woodland-brook-drive-cause-of-house-fire/85-ecb7df9b-5f65-44e9-bf9d-8c162d36c334} & \xmark & \cmark & \xmark & \xmark & \xmark \\ \addlinespace
    @cnviolations I swear community notes are the only good thing Elon added since he bought Twitter. & Community notes was first launched under former Twitter CEO Jack Dorsey in 2021 under the name of ``Birdwatch''. The only thing Elon Musk did was that he renamed the feature to community notes.    \url{https://blog.twitter.com/en\_us/topics/product/2021/introducing-birdwatch-a-community-based-approach-to-misinformation}    \url{https://www.reuters.com/article/factcheck-elon-birdwatch-idUSL1N31Z2VG/} &
     \xmark & \xmark & \cmark & \xmark & \cmark \\ \addlinespace
    Thailand will become the first country to make the contract null and void, meaning that Pfizer will become responsible for all vaccine injuries ... & Thailand has no plans to void its Pfizer COVID vaccine contract, an official with the country’s National Vaccine Institute said. Thailand’s Department of Disease Control also rejected the claims as ``fake news.'' ...  \url{https://apnews.com/article/fact-check-covid-vaccine-pfizer-thailand-203948163859} & \xmark & \xmark & \cmark & \xmark & \cmark \\ \addlinespace
    Hilarious tweets by footballers, A thread: 1. Virgil Van Dijk [Current Liverpool Captain] \url{https://t.co/...} & Virgil Van Dijk did not tweet this, the tweet was made by a fan account in his name.    \url{https://www.pinkvilla.com/sports/fact-check-did-virgil-van-dijk-really-root-for-man-u-because-no-one-likes-liverpool-in-resurfaced-viral-tweet-1287250} & \xmark & \xmark & \xmark & \cmark & \cmark \\ \addlinespace
    Rob Reiner announces he’s on the Epstein Client List and Epstein Flight logs. What a fool! When a lawyer tells me to STFU, I STFU! \url{https://t.co/...} & This is a digitally altered photo that might be misinterpreted even if used as a joke.    The name Rob Reiner is misspelled, and the text is not on Reiner's X timeline.    \url{https://twitter.com/robreiner?t=iqu43-NszIW5oOM\_KqRSpw} & \xmark & \xmark & \xmark & \cmark & \xmark \\
    \bottomrule
    \end{tabular}
    }
    \caption{A sample of tweets, notes, and their community annotations, as well as whether the note contains a fact-checking link.}
    \label{tab:community_annotation_example}
\end{table*}

\begin{figure}[!t]
    \centering
    \includegraphics[width=1\columnwidth]{Figures/manual_annotation.png}
    \caption{(a) strategies in debunking claims related to broader narratives. (b) the different ways in which fact-checking sources are used to debunk claims.}
    \label{fig:manual_annotation}
\end{figure}
% \begin{table}[h]
%     \centering
%     \begin{tabular}{p{2cm}lcc}
     
%        & & \multicolumn{2}{c}{Fact-check source} \\
        
%       & & Yes & No \\
%       \cline{3-4}
%         \multirow{2}{*}{\shortstack[l]{Related to a\\conspiracy} } & Yes & 0.216 & 0.112 \\
%         & No & 0.279 & 0.39 \\
%     \end{tabular}
%     \caption{Your table caption here}
%     \label{tab:fact_check}
% \end{table}


\def\arrvline{\hfil\kern\arraycolsep\vline\kern-\arraycolsep\hfilneg}

\begin{table}[!t]
% \fontsize{9}{9}\selectfont
    \centering
    \begin{tabular}{llc|c}
     
       & & \multicolumn{2}{c}{FC source} \\ 
       
      & & \cmark & \xmark \\
       \cmidrule(l){3-4}
      % \cmidrule(r){3-3}\cmidrule(l){4-4}
       \multirow{2}{*}{\rotatebox[origin=r]{90}{\parbox[r]{0.5cm}{\centering Conspi-racy}}} & \cmark \arrvline &  22\% & 11\% \\
       \cmidrule(l){2-4}
        & \xmark \arrvline & 28\% & 39\% \\
    \end{tabular}
    \caption{Percentage of samples related to a broader narrative or conspiracy vs. have a fact-checking source.}
    \label{tab:conspiracies_model_results}
\end{table}


We begin by performing a topic analysis, comparing topics of posts whose notes reference fact-checking sources to those citing other sources. To this end, we apply a strong zero-shot text classification model\footnote{\url{https://huggingface.co/r-f/ModernBERT-large-zeroshot-v1} with default settings.} to our $\mathcal{S}_\text{text}$ subset by classifying spans of the form ``\texttt{Tweet:<POST TEXT>; Note <NOTE TEXT>}'' into one of 13 classes. The authors manually evaluated the quality of the classification results and considered it satisfactory. Notably (\cref{fig:topics}), fact-checking sources are more likely to be included in posts related to high-stakes issues such as health, science, and scams and less likely to be included in posts on tech or sports.

We then analyse annotations (binary attributes explaining the warrant for the note) by community note authors.
% When writing a note, the author labels the original post with 
\cref{fig:annotations} contains the full breakdown of annotations for notes with and without fact-checking sources. Notes containing a link to fact-checking sources are overrepresented in posts where unverified information is presented as a fact or when the post contains a factual error. Conversely, they are under-represented in posts with outdated information or satirical content. \cref{tab:community_annotation_example} contains a sample of such notes. 

These results indicate that community note-writers adapt their strategies based on the stakes and scope of the claim, and the depth of research needed to counter misinformation. We hypothesise that they are more likely to rely on external fact-checking when refuting complex or unverifiable claims \citep{wuehrl-etal-2024-makes}, as well as claims related to broader narratives or conspiracy theories which cannot be fully addressed in the scope of a note.\footnote{For example, the claim ``Michelle Obama is a male''.} Conversely, claims involving misleading media can often be debunked with examples alone, making fact-checking sources unnecessary. To investigate this hypothesis, the authors of this paper manually annotated 400 $<\text{post}, \text{note}>$ pairs from $\mathcal{S}_\text{text}$ with attributes related to the complexity of the claims and how community notes address them. (See \cref{app:manual_annotation_setup} for annotation guidelines). The results (\cref{fig:manual_annotation}.a) support our hypothesis. Claims related to broader narratives or conspiracy theories are much more likely to include a link to a fact-checking source.
In contrast, other types of claims are more likely to be addressed by providing missing context or by invalidating the credibility of the claim's source. 
Additionally, \cref{fig:manual_annotation}.b depicts the different ways in which fact-checking sources are used to debunk claims. It demonstrates how such sources are rarely used to provide missing context but rather focus on discrediting sources of claims and providing scientific evidence.

\begin{figure}[!t]
    \centering
    \includegraphics[width=1\columnwidth]{Figures/Distribution_of_topics_vertical.png}
    \caption{Distribution of notes' topics, with and without a fact-checking source.}
    \label{fig:topics}
\end{figure}

We extend the manual annotation to an LLM-based analysis of 8K balanced $(\text{post}, \text{note})$ pairs from $\mathcal{S}_\text{text}$. We task OpenAI's GPT-4\footnote{Version \texttt{gpt-4o-2024-08-06}.} with determining whether a pair relates to a broader narrative or a conspiracy theory. \Cref{lst:prompt_conspiracy} in \cref{app:reproducibility} details the prompt used. To evaluate model accuracy, two authors independently labelled 100 balanced pairs, achieving an agreement rate of $0.88$ and resolving disagreements through discussion. The model attained an 
$F_1$ score of $0.85$---strong performance for this challenging task. The results (\cref{tab:conspiracies_model_results}) support our hypothesis: pairs related to a broader narrative or conspiracy theory are \textit{twice} as likely to cite fact-checking sources compared to other sources. In contrast, other pairs are nearly 30\% less likely to do so. These findings also highlight the prevalence of such claims and further underscore the importance of fact-checking in combating complex misinformation narratives.





\section{Evaluation}

\begin{table}
    \centering
    \caption{Configuration details of NDP-DIMM.}
    \vspace{-0.3cm}
    \resizebox{\linewidth}{!}{
    \begin{tabular}{c|c|c}
    \hline
    \multicolumn{3}{c}{\textbf{NDP core}} \\
    \hline
    \multicolumn{3}{c}{Configuration: 256 multipliers, reduction tree-based accumulator, Buffer size: 256KB}\\ 
    \hline
    One NDP core per DIMM & Frequency: @ 1 GHz  & area overhead: $1.23mm^2$ per core\\
    \hline
    \multicolumn{3}{c}{\textbf{DIMM Parameters}} \\
    \hline
     \multicolumn{3}{c}{DDR4-3200, 32GB/DIMM$\times$8, \update{2 DIMMs/channel}}\\
     \multicolumn{3}{c}{4 rank/DIMM, 2 bank groups/rank, 4 bank/BG}\\
     \hline
    \multicolumn{3}{c}{\textbf{DIMM Timing}} \\
    \hline
     \multicolumn{3}{c}{tRC=76, tRCD=24, tCL=24, tRP=24, tBL=4}\\
     \multicolumn{3}{c}{tCCD S=4, tCCD L=8,tRRD S=4, tRRD L=6, tFAW=26}\\
    \hline
    \multicolumn{3}{c}{\textbf{DIMM-Link Parameters}} \\
     \hline
    \multicolumn{3}{c}{25Gb/s/Lane, 1.17 pJ/b, 8 $\times$ Lanes (25GB/s per Link)} \\
    \hline
    \end{tabular}
    }
    \label{tab:dimmcfg}
\vspace{-0.3cm}
\end{table}

\subsection{Experimental Setup}\label{sec:experimental-setup}

\subsubsection{\name~System}
The proposed \name~system integrates a single NVIDIA RTX 4090 GPU with 24GB of graphic memory \update{and 330 tensor TOPS (FP16)} to process hot neurons. Additionally, we provide 8 NDP-DIMMs, each including 32GB DDR4 memory as the extension of GPU memory. We use PCIe 4.0 to support data interaction between NDP-DIMMs and GPU memory with a bandwidth of 64GB/s. The kernel performance of the NVIDIA RTX 4090 is measured using NVIDIA Nsight Compute~\cite{nsight}. Furthermore, we develop an in-house simulator by modifying Ramulator 2.0~\cite{luo2023ramulator, ramulator2.0} to evaluate the performance efficiency of NDP-DIMM devices. For the NDP core, we implemented it in RTL and synthesized it using the Synopsys Design Compiler~\cite{synopsys.org} with the TSMC 7nm technology. \tab \ref{tab:dimmcfg} shows the configuration details of adopted NDP-DIMMs.

\subsubsection{Baseline Systems}
We selected several offloading-based inference systems, such as Huggingface Accelerate~\cite{jain2022hugging, huggingface-accelerate}, FlexGen~\cite{sheng2023flexgen}, and Deja Vu~\cite{liu2023deja}, as the baselines. FlexGen and Deja Vu are restricted to OPT models. Moreover, Deja Vu, initially optimized for LLM activation sparsity within high-performance distributed systems, has been adapted to support offloading-based serving systems. In contrast to \name, these methods depend solely on the basic host memory to expand capacity without offering additional computational resources. \update{We also provided a system (Hermes-host) that offloads cold neurons to the host CPU while handling hot neurons on GPU, demonstrating the necessity of NDP-DIMMs. Hermes-host follows the configuration in~\cite{song2023powerinfer}, which equips an Intel i9-13900K processor as the host CPU (providing a maximum bandwidth of 89.6 GB/s), and also uses a single NVIDIA RTX 4090 as the GPU for hot neurons.} Additionally, to highlight the significance of activation sparsity in boosting \name~system efficiency, we also compare \name~against a straightforward NDP-DIMM extended system (referred to as Hermes-base) that does not leverage activation sparsity in LLMs.

\begin{figure}
    \centering
    \includegraphics[width=.98\linewidth]{Fig/end1_rebuttal.pdf}
    \vspace{-0.3cm}
    \caption{\update{Performance comparison with existing offloading-based systems.}}
    \label{fig:offloading-performance}
\vspace{-0.3cm}
\end{figure}

\begin{figure}[t]
    \centering
    \includegraphics[width=0.98\linewidth]{Fig/end2_rebuttal.pdf}
    \vspace{-0.3cm}
    \caption{\update{The effectiveness of activation sparsity and NDP design on \name.}}
    \label{fig:base-hermes-performance}
\vspace{-0.3cm}
\end{figure}

\subsubsection{Workloads}
We chose OPT-13B, OPT-30B, OPT-66B~\cite{zhang2022opt}, LLaMA2-13B, LLaMA2-70B~\cite{touvron2023llama2}, and Falcon-40B~\cite{almazrouei2023falcon} as target models. For the OPT series models, we utilized their native ReLU activations to achieve activation sparsity. For the LLaMA2 and Falcon models, we use the open-source models\footnote{The modified LLMs can be found at \href{https://huggingface.co/SparseLLM}{https://huggingface.co/SparseLLM}, including both LLaMA2 and Falcon models} that substituted their original activation functions with ReLU~\cite{mirzadeh2023relu, zhang2024relu}. Furthermore, we added additional ReLU functions before generating QKV to achieve activation sparsity in self-attention blocks. Evaluation results show that these alterations result in negligible accuracy loss (under 1\%). \update{Furthermore, we adopt ChatGPT prompts~\cite{gpt-prompts} and Alpaca~\cite{alpaca} as the datasets to evaluate the end-to-end performance, following configurations in \cite{xue2024powerinfer, song2023powerinfer}.}
% \todo{The dataset used for evaluation with description of the distribution of the prompt and generated output lengths}

\begin{figure*}
    \centering
    \includegraphics[width=\linewidth]{Fig/batching_rebuttal.pdf}
    \vspace{-0.3cm}
    \caption{\update{End-to-end performance on different batch sizes (ranging from 1 to 16). N.P. denotes the model is not supported by the current inference system.}}
    \label{fig:batching-inference}
\vspace{-0.3cm}
\end{figure*}

\subsubsection{Evaluation Metric}
Given our focus on local deployment scenarios, we primarily optimized LLM inference with small batch sizes. We concentrated on the average number of tokens generated per second (tokens/s) to evaluate model inference efficiency. Hereafter, the number above each bar in each figure indicates the end-to-end generation speed (tokens/s). In our experiments, we used batch sizes between 1 and 16, and kept the lengths of both input and output sequences fixed at 128.


% Moreover, 

% Moreover, we concentrated on token-to-token latency (ms/token) and the average number of tokens generated per second (tokens/s) to evaluate model inference efficiency.

\subsection{\name~Performance}\label{sec:end-to-end}

\subsubsection{End-to-End Performance}
We begin by evaluating the end-to-end inference performance of \name~and baseline systems at a batch size of 1, which is commonly used for local deployments~\cite{cai2023medusa}. Noting that FlexGen and Deja Vu are limited to support OPT family models, we first compare \name~against existing offload-based inference systems on OPT models. \update{Additionally, we evaluate the Hermes-host and Hermes-base systems' performance across various LLMs to illustrate the necessity of NDP-DIMMs design and activation sparsity in Hermes, respectively.}
% 这里需要包含两个部分,分别是 end-to-end performance 和 token-to-token latency

\textbf{Comparison with Offloading-based Systems. } \fig \ref{fig:offloading-performance} presents the end-to-end performances on OPT family models. Compared with the Accelerate and FlexGen systems, \name~can achieve an average $578.42 \times$ and $247.25 \times$ speedup, respectively. \name~is capable of achieving a rate of $20.37$ tokens/s for OPT-66B, which substantially surpasses current inference systems. In contrast, Deja Vu only attains an average speedup of $2.12 \times$ over FlexGen due to the necessity of loading cold neurons. The frequent data transfer on PCIe compromises the performance improvement of activation sparsity, while the expensive MLP-based predictor used in Deja Vu further diminishes its benefits. Compared to OPT-13B, \name~achieves greater performance gains on OPT-66B. This is because 80\% of the parameters in OPT-13B can be stored in GPU memory, whereas only 15\% of parameters in OPT-66B can be stored in GPU memory. This further exacerbates the data transfer overhead between host memory and GPU memory. 
% Fortunately, \name~can effectively utilize NDP-DIMMs to process the offloaded parameters without introducing significant data movement.

\textbf{Necessity of Activation Sparsity. } We further compare \name~with the Hermes-base system, which only adopts a na\"ive NDP-DIMM extended system without utilizing activation sparsity, as shown in \fig \ref{fig:base-hermes-performance}. 
The Hermes-based system processes the FC layers on the GPU when their parameters are available, switches to NDP-DIMMs when their parameters are stored in those modules, and offloads all attention computations to NDP-DIMMs.
% The Hermes-based system activates nearby processing units based on the location of parameters to compute FC operators and offloads all attention computations to NDP-DIMMs. 
This approach leverages the high internal bandwidth of NDP-DIMMs and
reduces data transfer between DIMMs and GPU memory. In comparison to Huggingface Accelerate, the Hermes-base system can achieve $53.89 \times$ speedup on average, as it greatly reduces the data transfer on PCIe. By effectively leveraging activation sparsity in LLMs, Hermes outperforms the Hermes-base system with average speedups of $5.17 \times$, specifically for large models such as Falcon-40B and
LLaMA2-70B. This is due to when running large models, most layers are offloaded on the computation-limited NDP-DIMMs for the Hermes-base system. 

\update{\textbf{NDP-DIMMs instead of host CPU. } 
Experimental results in Figure \ref{fig:offloading-performance}, \ref{fig:base-hermes-performance} demonstrate the necessity of NDP-DIMMs. Hermes achieves $4.79\times$ - $7.75\times$ speedup when compared to Hermes-host. Specifically, the Hermes-host system also utilizes the hot/cold neuron partition, but computes the cold neurons on the host CPU. This approach effectively alleviates the burdensome data loading on PCIe for existing offloading-based systems. In comparison to Huggingface Accelerate and FlexGen, the Hermes-host system can achieve $62.00 \times$ and $44.96\times$ speedup on average, respectively. However, the memory bandwidth on the CPU side is significantly lower than that of NDP-DIMMs, making the Hermes-host system still far less efficient than our proposed Hermes system. }



\subsubsection{Batching Inference}

\begin{figure*}
    \centering
    \includegraphics[width=\linewidth]{Fig/breakdown.pdf}
    \vspace{-0.3cm}
    \caption{Evaluating the performance breakdown on Deja Vu, \name, and \name-base (H-base) on various LLMs with different batch sizes. }
    \label{fig:performance-breakdown}
\vspace{-0.3cm}
\end{figure*}

We also evaluate the end-to-end performance of \name~with different batch sizes. As shown in the \fig \ref{fig:batching-inference}, \name~demonstrates consistent performance improvement with the batch sizes varying from 1 to 16. Hermes attains average speedups of $148.98\times$ and $75.24\times$ for various batch sizes when compared to FlexGen and Deja Vu, respectively, offering promising support for larger batch sizes. \update{Furthermore, \name~achieves an average $7.17 \times$ speedup over \name-host for various batch sizes. As the batch size increases, the performance gap between Hermes-host and Hermes becomes more pronounced. This occurs as the consumer-grade GPU with sufficient computation capability is minimally impacted by larger batch sizes, whereas the dynamic loading overhead of cold neurons is closely tied to bandwidth. Consequently, as batch sizes grow, the limited memory bandwidth on the CPU side increasingly affects overall system performance.} The performance gap between \name~and the Hermes-base system is the smallest when the batch size is 2. This is because for \name-base, the computation capability of the NDP core can still effectively handle the corresponding computational load, and larger batches can effectively amortize the DRAM cell access overhead as weight parameters are reused by the two batches. At other batch sizes, \name~demonstrates a significant performance advantage over Hermes-base. First, at a batch size of 1, Hermes can utilize activation sparsity to significantly reduce the number of neurons that need to be activated, thereby lowering data access overhead. Second, as the batch size increases, Hermes is not constrained by the computation capability of NDP-DIMMs due to the presence of activation sparsity. 

\subsection{Ablation Studies}\label{sec:ablation-study}
% 要包括这样几种: offline modeling and mapping; online placement; load balance optimization 


\begin{figure}
    \centering
    \includegraphics[width=\linewidth]{Fig/ab_rebuttal.pdf}
    \vspace{-0.3cm}
    \caption{\update{Ablation study on proposed offline and online scheduling strategies.}}
    \label{fig:ablation-study}
\vspace{-0.3cm}
\end{figure}



To evaluate the scheduling strategies proposed in Section \ref{sec:hermes-system}, we compare the normalized inference latency on MLP block for different LLMs with various scheduling settings. Specifically, \name-random denotes utilizing a random offline mapper to achieve neuron placement, \name-partition denotes that it only considers the optimal offline neuron placement, \name-adjustment denotes the system that further uses online adjustment for hot/cold neuron partition, and \name~is the one that integrates all the scheduling strategies proposed in Section \ref{sec:hermes-system}. \update{Furthermore, we also explore when only adopting token-wise prediction or layer-wise prediction to guide the online adjustment of hot/cold partition, denoted as Hermes-token-adjustment and Hermes-layer-adjustment, respectively.} 

\textbf{Load Balancing with Multi-level Optimization. } \fig \ref{fig:ablation-study} shows the contributions of each component in \name~. Utilizing the offline mapper can effectively identify the frequent hot neurons, reducing the computation cost of NDP-DIMMs. As a result, \name-partition can achieve $1.63 \times$ speedup than \name-random. However, the input-specific nature of activation sparsity challenges the offline partition approach. Therefore, further adopting online adjustment for hot/cold partition (\name-adjustment) achieves $1.33 \times$ performance gains over \name-partition. Despite this, the overall execution efficiency is still constrained by the NDP-DIMMs, which possess limited computation capability. Thus, the performance of the resource-constrained NDP-DIMMs can be improved by tackling the load imbalance issues in several NDP-DIMMs. The introduced online remapping method successfully addresses this problem. As a consequence,
the fully optimized Hermes system demonstrates a $1.29 \times$ boost in performance when compared with \name-adjustment.
% 这里就是分析每一部分的优势

\update{\textbf{Benefits of Token-wise and Layer-wise Prediction.}
Compared to \name-partition which only considers the optimal offline neuron placement, \name-token-adjustment and \name-layer-adjustment can achieve $1.08\times$ and $1.11\times$ speedup, respectively, demonstrating the benefits of online adjustment. However, token-wise prediction cannot address fluctuations in neuron activity, making it inaccurate for frequent changes in hot/cold neurons. Simultaneously, layer-wise prediction only relies on the static sampled neuron correlation table to guide the online adjustment, inefficient for constant changes of online adjustment. As a result, using token-wise or layer-wise prediction only cannot effectively unleash the benefits of prediction-based online adjustment.  
}

\subsection{Performance Breakdown}\label{sec:breakdown}

% 这里要分析几个点,首先是 deja vu 中的load weight 的影响,communication 的开销,以及predictor 的开销;然后是 prefill 的开销占比;然后是计算的开销

\fig \ref{fig:performance-breakdown} illustrates the performance breakdown of Deja Vu, \name-base, and \name~on various LLMs. It provides detailed insights into the efficiency sources of \name.

Figure \ref{fig:performance-breakdown}a shows that while Deja Vu benefits from activation sparsity, it still requires loading cold neurons when activated, resulting in communication costs—especially PCIe data transfer—comprising about 89\% of the execution time. On the right side of Figure \ref{fig:performance-breakdown}a, we disregard the effect of communication on performance. The MLP-based predictor in Deja Vu consumes roughly 18.1\% of computation time, further reducing the gains from activation sparsity. Our lightweight predictor, in contrast, contributes less than 0.1\% to runtime overhead. Even with communication costs lowered through reusable neurons at large batch sizes, Deja Vu's performance remains inferior to Hermes.

Figure \ref{fig:performance-breakdown}b compares \name-base and \name. Without activation sparsity, \name-base incurs higher computation costs, especially as batch sizes increase, due to intensive computation on NDP-DIMMs. For example, running LLaMA2-70B offloads over 80\% of computation to NDP-DIMMs, leading to a substantial portion of the execution time being occupied by FC computation. In Hermes, token generation takes 66.40\% of execution time at batch size 1. After optimizing token generation, the prompting stage becomes the bottleneck, accounting for about 33.01\% of the overhead, limiting further inference efficiency improvements.

% \fig \ref{fig:performance-breakdown} presents the performance breakdown of Deja Vu, \name-base and \name~on OPT-30B, OPT-66B, Falcon-40B and LLaMA2-70B. 
% It effectively describes in detail the sources of efficiency of \name. 

% As \fig \ref{fig:performance-breakdown}a shows, despite benefiting from activation sparsity, Deja Vu still needs to load cold neurons when they are activated. Consequently, the communication cost, particularly data transfer on PCIe, makes up about 89\% of the total execution time. On the right side of Figure \ref{fig:performance-breakdown}a, we disregard the effect of communication on performance. The MLP-based predictor used in Deja Vu consumes approximately 18.1\% of the overall computation time, diminishing the gains from activation sparsity. In contrast, our proposed lightweight predictor contributes to less than 0.1\% of the total runtime overhead. The proportion of communication in Deja Vu reduces due to reusable neurons when running at large batch sizes. Nonetheless, the overall performance of Deja Vu is still significantly inferior to Hermes.

% \fig \ref{fig:performance-breakdown}b provides a comparative analysis of the performance breakdown between \name-base and \name. The absence of activation sparsity in the Hermes-base results in considerably higher computation costs compared to \name, especially as the batch size increases. This is primarily due to the intensive computation on NDP-DIMMs, which significantly impacts overall execution efficiency. For instance, when running LLaMA2-70B, over 80\% of the computation is offloaded to NDP-DIMMs, leading to a substantial portion of the execution time being occupied by FC computation. In contrast, in Hermes, the token generation time occupies 66.40\% of the total execution time when at batch size is 1. With the token generation stage fully optimized, the prompting stage becomes the bottleneck, accounting for approximately 33.01\% of the overhead, thus limiting further improvements in inference efficiency.

\begin{figure}
    \centering
    \includegraphics[width=\linewidth]{Fig/dimm_rebuttal.pdf}
    \vspace{-0.3cm}
    \caption{\update{Throughput of four typical LLMs with different numbers of NDP-DIMMs. N.P. denotes the model is not supported by current system.}}
    \label{fig:dimm}
\vspace{-0.3cm}
\end{figure}

\update{\subsection{Sensitivity Studies}}

\subsubsection{\update{Sensitivity analysis of the number of DIMMs}}

\update{\fig~\ref{fig:dimm} illustrates the improvement in LLM throughput as the number of NDP-DIMMs increases. We evaluated four distinct LLM models using a single batch to understand the impact of varying numbers of NDP-DIMMs, while mitigating the effect of limited computation capability. An increase in NDP-DIMMs enhances both memory size and internal bandwidth. Larger memory capacity facilitates the deployment of more extensive models; for instance, deploying Falcon-40B on Hermes necessitates a minimum of four NDP-DIMMs. Additionally, higher internal bandwidth significantly enhances end-to-end performance, addressing the bandwidth limitations that bottleneck current offloading-based systems. However, once sufficient bandwidth is achieved, further increases in the number of NDP-DIMMs do not proportionally boost throughput. For example, LLaMA2-70B exhibits similar throughput with both 8 and 16 NDP-DIMMs. Once the NDP-DIMMs surpass the GPU in performance, additional NDP-DIMMs do not yield further performance gains.}

% \update{
% \fig~\ref{fig:dimm} shows that LLM throughput improves as the number of NDP-DIMMs increases. We evaluated four different LLM models with a single batch to assess the impact on different numbers of NDP-DIMMs, avoiding the effect of limited computation capacity. More NDP-DIMMs provide larger memory size as well as higher internal bandwidth. Abundant memory size allows the deployment of larger models. For example, deploying Falcon-40B on Hermes needs at least 4 NDP-DIMMs. Furthermore, higher internal bandwidth can effectively boost the end-to-end performance, as the limited bandwidth is the bottleneck of existing offloading-based system. However, when sufficient bandwidth is provided, the end-to-end throughput will not be further improved proportionally with the increasing number of NDP-DIMMs. For instance, LLaMA2-70B shows similar throughput with 8 and 16 NDP-DIMMs. Once the NDP-DIMMs outperform the GPU, adding more NDP-DIMMs no longer impacts performance.
% }

\begin{figure}
    \centering
    \includegraphics[width=\linewidth]{Fig/GPU_rebuttal.pdf}
    \vspace{-0.3cm}
    \caption{\update{Throughput of OPT-13B and OPT-30B with various GPUs, including RTX 4090, RTX 3090 and Tesla T4.}}
    \label{fig:gpu}
\vspace{-0.3cm}
\end{figure}

\subsubsection{\update{Sensitivity analysis of various GPUs}}

\update{\fig~\ref{fig:gpu} illustrates the significant impact of different GPUs on the end-to-end throughput of LLM execution. We have included two additional consumer-grade GPUs, Tesla T4 and RTX 3090, in our evaluation. Specifically, Tesla T4 offers 16GB of graphic memory, 320GB/s memory bandwidth, and 65 tensor TOPS (FP16), whereas RTX 3090 provides almost the same graphic memory and bandwidth as RTX 4090, but with 142 tensor TOPS (FP16). Overall, \name~with RTX 4090 achieves an average throughput improvement of $2.02\times$ and $1.34\times$ compared to \name~with Tesla T4 and RTX 3090, respectively. The data loading cost for RTX 3090 is nearly identical to that of RTX 4090. However, RTX 3090 spends more time on prefill and hot neuron computations due to its weaker computation capability. Tesla T4, with its smaller graphic memory and lower memory bandwidth compared to RTX 3090, is inefficient for data loading. Consequently, the choice of GPU device is crucial for optimizing \name~performance.}


\subsubsection{\update{Design Space Exploration for NDP-DIMMs}}
\begin{figure}
    \centering
    \includegraphics[width=\linewidth]{Fig/gemv_rebuttal.pdf}
    \vspace{-0.3cm}
    \caption{\update{Design Space Exploration for NDP-DIMMs with different number of multipliers in each GEMV unit.}}
    \label{fig:gemv}
\vspace{-0.3cm}
\end{figure}

\update{
\fig~\ref{fig:gemv} highlights the impact of increasing the number of multipliers within a GEMV unit per DIMM on LLM inference performance, especially with larger batch sizes. We varied the number of multipliers within a GEMV unit from 32 to 512, thereby enhancing computation capability by 16$\times$. For OPT-13B with a batch size of 1, performance stabilizes once 64 multipliers are reached, as further computation capability yields minimal gains. In contrast, with a batch size of 16, performance continuously improves with additional multipliers, achieving up to a $3.86\times$ speedup. This difference arises because memory bandwidth limits performance for smaller batch sizes due to lower arithmetic intensity, while computation capability becomes the bottleneck with larger batch sizes. To optimize the balance between hardware overhead and performance across various batch sizes, we selected 256 multipliers within the GEMV unit per DIMM.
}

\subsection{Comparison with High-Performance System}\label{sec:comparison-high-performance}

\begin{figure}[t]
    \centering
    \includegraphics[width=\linewidth]{Fig/comparison.pdf}
    \vspace{-0.3cm}
    \caption{Comparison with TensorRT-LLM on LLaMA2-70B.}
    \label{fig:trt-llm-comparison}
\vspace{-0.3cm}
\end{figure}

% 这里可以考虑一下系统的开销和对应的结果
This section discusses the performance gap between our budget-friendly LLM inference system Hermes and state-of-the-art high-performance serving system TensorRT-LLM~\cite{tensorrt-llm}. We kept the input and output sequence lengths set at 128. To handle LLaMA2-70B with a batch size of 16, TensorRT-LLM requires five NVIDIA A100-40GB-SXM4 GPUs. In contrast, 
\name~operates with only one NVIDIA RTX-4090 GPU and affordable NDP-DIMMs. Figure \ref{fig:trt-llm-comparison} displays the performance comparison between TensorRT-LLM and Hermes. For a batch size of 1, Hermes achieves 79.1\% inference efficiency of TensorRT-LLM. Even at a batch size of 16, Hermes retains 24.4\% inference efficiency of TensorRT-LLM. Despite this, Hermes is far more economical than TensorRT-LLM, which is equipped with 5 NVIDIA A100-40GB-SMX4 GPUs. Specifically, Hermes only costs approximately \$2,500, whereas TensorRT-LLM requires \$50000 to support LLaMA2-70B. Hermes provides efficient and low-budget LLM inference for local deployments. 

% Despite this, \name~is far more economical, costing approximately \$2,500 compared to the \$50,000 needed to build a high-performance system with 5 NVIDIA A100-40GB-SMX4 GPUs, making it highly efficient for local deployments with smaller batch sizes.
\section{Related Work}


\bbb{Traditional sketches:}
%
Sketches can broadly be categorized into two types:
%
\textit{1) Classic sketches} consist of a counter matrix and multiple hash functions. During updates and queries, flow IDs are hashed into multiple counters, and the mapped counters are then updated and queried accordingly. Typical classic sketches include the Count-Min Sketch (CMS) \cite{cmsketch}, the Conservative Update Sketch (CUS) \cite{cusketch}, and the Count Sketch (CS) \cite{csketch}. However, classic sketches fail to account for the highly skewed nature of network traffic, resulting in memory waste.
%
\textit{2) Sophisticated sketches} address this problem by separating large flows from small flows \cite{elasticsketch, sketchlearn, nitrosketch, nze-sketch, bitsense}. These sketches typically consist of multiple parts, with different parts using different data structures to record flows of varying sizes. A typical sophisticated sketch is ElasticSketch \cite{elasticsketch}, which is composed of a heavy part and a light part. The heavy part is a key-value table, while the light part is a CM sketch. Packets are first inserted into the heavy part. When the heavy part becomes full, ElasticSketch uses an eviction method to remove the flow that is most likely to be small and inserts it into the light part. Researchers have attempted to improve sketch accuracy by adjusting the flow selection method for eviction, but most of these attempts have been based on experience.
%
Additional sketches have been proposed for specialized tasks, such as heavy hitter query \cite{css, mvsketch, precision}, hierarchical heavy hitter query \cite{rhhh, cocosketch}, and DDoS victim/super-spreader query \cite{opensketch, spreadsketch}.




\bbb{ML-based sketches:}
%
In recent years, researchers have attempted to use machine learning methods to improve sketch performance. Learned Count-Min Sketch (LCMS) \cite{lcmsketch} employs an RNN to learn and infer whether a flow is large and uses an additional hash table to record large flows. Other solutions \cite{bertsimas2021frequency, mlsketch, metasketch} use ML to enhance hashing, updating, and querying processes within the sketch, leading to improved performance. The main difference between these works and ours is that they only learn features from the flow ID and distribution, without utilizing other information carried by the packets.




\bbb{LLM for other network tasks:}
%
Some works explore how to adapt LLMs to other network operations, such as traffic classification, viewport prediction, adaptive bitrate streaming, and cluster job scheduling. Typical works include PERT \cite{pert}, ET-BERT \cite{et-bert}, YaTC \cite{yatc}, and NetLLM \cite{netllm}.


\section{Conclusion}
We operationalized the theory of instrumental interaction for generative AI, with an in-depth unpacking of the principles of reification of user intent, reflection, and grounding. We argue that leveraging this re-appropriated and refined theory can drive the creation of a \textit{new generation of expressive AI-Instruments} that afford better expression of intent, make it easier to discover what is possible, and provide powerful degrees of freedom for steering the generation towards the best possible results. Those new tools and instruments can truly leverage the polymorphic and non-deterministic behavior of generative AI models, unleashing new and empowering forms of expressive HCI+AI experiences. 

Beyond our focus on AI-Instruments, theories play an important role in the advancement of our wider research field~\cite{rogers_hci_2012, halverson_activity_2002}. Rogers argues that there is a need for theories as lenses bringing critical design characteristics into focus, and which can function as a generative source: providing "\textit{design dimensions and constructs to inform the design and selection of interactive representations}"~\cite{rogers_new_2004}. We hope that our work on operationalizing the theory of instrumental interaction for AI can inspire other new -- and re-appropriated -- theories to advance HCI+AI. 










\bibliographystyle{ACM-Reference-Format}
\bibliography{reference}

\end{document}
\endinput
%%
%% End of file `sample-sigconf.tex'.
