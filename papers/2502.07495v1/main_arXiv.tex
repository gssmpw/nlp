% \documentclass[sigconf,anonymous,review]{acmart}
\documentclass[sigconf]{acmart}
%%
%% \BibTeX command to typeset BibTeX logo in the docs
\AtBeginDocument{%
  \providecommand\BibTeX{{%
    Bib\TeX}}}

%% Rights management information.  This information is sent to you
%% when you complete the rights form.  These commands have SAMPLE
%% values in them; it is your responsibility as an author to replace
%% the commands and values with those provided to you when you
%% complete the rights form.
% \setcopyright{acmlicensed}
% \copyrightyear{2018}
% \acmYear{2018}
% \acmDOI{XXXXXXX.XXXXXXX}
%% These commands are for a PROCEEDINGS abstract or paper.
% \acmConference[Conference acronym 'XX]{Make sure to enter the correct
%   conference title from your rights confirmation email}{June 03--05,
%   2018}{Woodstock, NY}
%%
%%  Uncomment \acmBooktitle if the title of the proceedings is different
%%  from ``Proceedings of ...''!
%%
%%\acmBooktitle{Woodstock '18: ACM Symposium on Neural Gaze Detection,
%%  June 03--05, 2018, Woodstock, NY}
% \acmISBN{978-1-4503-XXXX-X/2018/06}


\usepackage{subfigure}
\usepackage{enumitem}

\newcommand{\bbb}{\noindent\textbf}

\newcommand{\alg}{LLM-Sketch}


%%
%% end of the preamble, start of the body of the document source.
\begin{document}

%%
%% The "title" command has an optional parameter,
%% allowing the author to define a "short title" to be used in page headers.
\title{\alg{}: Enhancing Network Sketches with LLM}

%%
%% The "author" command and its associated commands are used to define
%% the authors and their affiliations.
%% Of note is the shared affiliation of the first two authors, and the
%% "authornote" and "authornotemark" commands
%% used to denote shared contribution to the research.

\author{Yuanpeng Li}
\affiliation{%
  \institution{Peking University \\ Zhongguancun Laboratory}
  \city{Beijing}
  \country{China}
}

\author{Zhen Xu}
\affiliation{%
  \institution{Zhejiang University}
  \city{Hangzhou}
  \country{China}
}

\author{Zongwei Lv}
\affiliation{%
  \institution{Peking University}
  \city{Beijing}
  \country{China}
}

\author{Yannan Hu}
\authornote{Tong Yang (yangtong@pku.edu.cn) and Yannan Hu (huyn@zgclab.edu.cn) are the corresponding authors.}
\affiliation{%
  \institution{Zhongguancun Laboratory}
  \city{Beijing}
  \country{China}
}

\author{Yong Cui}
\affiliation{%
  \institution{Tsinghua University \\ Zhongguancun Laboratory}
  \city{Beijing}
  \country{China}
}

\author{Tong Yang}
\authornotemark[1]
\affiliation{%
  \institution{Peking University \\ Zhongguancun Laboratory}
  \city{Beijing}
  \country{China}
}

%%
%% By default, the full list of authors will be used in the page
%% headers. Often, this list is too long, and will overlap
%% other information printed in the page headers. This command allows
%% the author to define a more concise list
%% of authors' names for this purpose.
\renewcommand{\shortauthors}{Li et al.}

%%
%% The abstract is a short summary of the work to be presented in the
%% article.
\begin{abstract}
  \begin{abstract}
Recent advancements in 3D multi-object tracking (3D MOT) have predominantly relied on tracking-by-detection pipelines. However, these approaches often neglect potential enhancements in 3D detection processes, leading to high false positives (FP), missed detections (FN), and identity switches (IDS), particularly in challenging scenarios such as crowded scenes, small-object configurations, and adverse weather conditions. Furthermore, limitations in data preprocessing, association mechanisms, motion modeling, and life-cycle management hinder overall tracking robustness. To address these issues, we present \textbf{Easy-Poly}, a real-time, filter-based 3D MOT framework for multiple object categories. Our contributions include: (1) An \textit{Augmented Proposal Generator} utilizing multi-modal data augmentation and refined SpConv operations, significantly improving mAP and NDS on nuScenes; (2) A \textbf{Dynamic Track-Oriented (DTO)} data association algorithm that effectively manages uncertainties and occlusions through optimal assignment and multiple hypothesis handling; (3) A \textbf{Dynamic Motion Modeling (DMM)} incorporating a confidence-weighted Kalman filter and adaptive noise covariances, enhancing MOTA and AMOTA in challenging conditions; and (4) An extended life-cycle management system with adjustive thresholds to reduce ID switches and false terminations. Experimental results show that Easy-Poly outperforms state-of-the-art methods such as Poly-MOT and Fast-Poly~\cite{li2024fast}, achieving notable gains in mAP (e.g., from 63.30\% to 64.96\% with LargeKernel3D) and AMOTA (e.g., from 73.1\% to 74.5\%), while also running in real-time. These findings highlight Easy-Poly's adaptability and robustness in diverse scenarios, making it a compelling choice for autonomous driving and related 3D MOT applications. The source code of this paper will be published upon acceptance.

% 3D Multi-Object Tracking (MOT) is essential for autonomous driving systems, contributing significantly to vehicle safety and navigation. Despite recent advancements, existing 3D tracking methods still face significant challenges in accuracy, particularly when dealing with small objects, crowded environments, and adverse weather conditions. To overcome these challenges, we propose \textbf{Easy-Poly}, a novel and efficient multi-modal 3D MOT framework. \textbf{Easy-Poly} employs the Focal Sparse Convolution (\textbf{FocalsConv}) model for object detection. By optimizing convolution operations and augmenting data with multiple modalities, we significantly enhance detection precision.
% \textbf{Easy-Poly} introduces several key innovations: (1) an optimized Kalman filter in the pre-processing stage, (2) integration of the Dynamic Track-Oriented (\textbf{DTO}) Data Association algorithm with confidence-weighted motion models for data association, (3) Dynamic Motion Modeling (\textbf{DMM}) with Adaptive Noise Covariances, and (4) enhanced trajectory management throughout the tracking life-cycle. These improvements increase the robustness and efficiency of tracking, especially in complex scenarios such as crowded scenes and challenging weather conditions. Experimental results on the \textbf{nuScenes} dataset demonstrate that in the pre-processing stage of \textbf{Easy-Poly}, the optimized \textbf{FocalsConv} model achieves a mean Average Precision (mAP) of \textbf{64.96\%} for object detection. Furthermore, the multi-object tracking performance reaches a high AMOTA of \textbf{75.0\%}, surpassing existing methods across multiple performance metrics.
 
% Code and data are available at \textcolor{blue}{\textit{\url{https://github.com/zhangpengtom/FocalsConvPlus}}} and  \textcolor{blue}
%  \textit{\url{https://github.com/zhangpengtom/EasyPoly}.}
%  } 

\end{abstract}
\end{abstract}

%%
%% The code below is generated by the tool at http://dl.acm.org/ccs.cfm.
%% Please copy and paste the code instead of the example below.
%%
% \begin{CCSXML}
% <ccs2012>
%  <concept>
%   <concept_id>00000000.0000000.0000000</concept_id>
%   <concept_desc>Do Not Use This Code, Generate the Correct Terms for Your Paper</concept_desc>
%   <concept_significance>500</concept_significance>
%  </concept>
%  <concept>
%   <concept_id>00000000.00000000.00000000</concept_id>
%   <concept_desc>Do Not Use This Code, Generate the Correct Terms for Your Paper</concept_desc>
%   <concept_significance>300</concept_significance>
%  </concept>
%  <concept>
%   <concept_id>00000000.00000000.00000000</concept_id>
%   <concept_desc>Do Not Use This Code, Generate the Correct Terms for Your Paper</concept_desc>
%   <concept_significance>100</concept_significance>
%  </concept>
%  <concept>
%   <concept_id>00000000.00000000.00000000</concept_id>
%   <concept_desc>Do Not Use This Code, Generate the Correct Terms for Your Paper</concept_desc>
%   <concept_significance>100</concept_significance>
%  </concept>
% </ccs2012>
% \end{CCSXML}

% \ccsdesc[500]{Do Not Use This Code~Generate the Correct Terms for Your Paper}
% \ccsdesc[300]{Do Not Use This Code~Generate the Correct Terms for Your Paper}
% \ccsdesc{Do Not Use This Code~Generate the Correct Terms for Your Paper}
% \ccsdesc[100]{Do Not Use This Code~Generate the Correct Terms for Your Paper}

%%
%% Keywords. The author(s) should pick words that accurately describe
%% the work being presented. Separate the keywords with commas.
\keywords{Network stream mining, Sketches, Flow classification, LLM}
%% A "teaser" image appears between the author and affiliation
%% information and the body of the document, and typically spans the
%% page.

%%
%% This command processes the author and affiliation and title
%% information and builds the first part of the formatted document.
\maketitle

\section{Introduction}

Deep Reinforcement Learning (DRL) has emerged as a transformative paradigm for solving complex sequential decision-making problems. By enabling autonomous agents to interact with an environment, receive feedback in the form of rewards, and iteratively refine their policies, DRL has demonstrated remarkable success across a diverse range of domains including games (\eg Atari~\citep{mnih2013playing,kaiser2020model}, Go~\citep{silver2018general,silver2017mastering}, and StarCraft II~\citep{vinyals2019grandmaster,vinyals2017starcraft}), robotics~\citep{kalashnikov2018scalable}, communication networks~\citep{feriani2021single}, and finance~\citep{liu2024dynamic}. These successes underscore DRL's capability to surpass traditional rule-based systems, particularly in high-dimensional and dynamically evolving environments.

Despite these advances, a fundamental challenge remains: DRL agents typically rely on deep neural networks, which operate as black-box models, obscuring the rationale behind their decision-making processes. This opacity poses significant barriers to adoption in safety-critical and high-stakes applications, where interpretability is crucial for trust, compliance, and debugging. The lack of transparency in DRL can lead to unreliable decision-making, rendering it unsuitable for domains where explainability is a prerequisite, such as healthcare, autonomous driving, and financial risk assessment.

To address these concerns, the field of Explainable Deep Reinforcement Learning (XRL) has emerged, aiming to develop techniques that enhance the interpretability of DRL policies. XRL seeks to provide insights into an agent’s decision-making process, enabling researchers, practitioners, and end-users to understand, validate, and refine learned policies. By facilitating greater transparency, XRL contributes to the development of safer, more robust, and ethically aligned AI systems.

Furthermore, the increasing integration of Reinforcement Learning (RL) with Large Language Models (LLMs) has placed RL at the forefront of natural language processing (NLP) advancements. Methods such as Reinforcement Learning from Human Feedback (RLHF)~\citep{bai2022training,ouyang2022training} have become essential for aligning LLM outputs with human preferences and ethical guidelines. By treating language generation as a sequential decision-making process, RL-based fine-tuning enables LLMs to optimize for attributes such as factual accuracy, coherence, and user satisfaction, surpassing conventional supervised learning techniques. However, the application of RL in LLM alignment further amplifies the explainability challenge, as the complex interactions between RL updates and neural representations remain poorly understood.

This survey provides a systematic review of explainability methods in DRL, with a particular focus on their integration with LLMs and human-in-the-loop systems. We first introduce fundamental RL concepts and highlight key advances in DRL. We then categorize and analyze existing explanation techniques, encompassing feature-level, state-level, dataset-level, and model-level approaches. Additionally, we discuss methods for evaluating XRL techniques, considering both qualitative and quantitative assessment criteria. Finally, we explore real-world applications of XRL, including policy refinement, adversarial attack mitigation, and emerging challenges in ensuring interpretability in modern AI systems. Through this survey, we aim to provide a comprehensive perspective on the current state of XRL and outline future research directions to advance the development of interpretable and trustworthy DRL models.
\section{Background}
\label{sec:background}

\noindent
In this section, we first overview the principles governing transformer architecture. Next, we present a concise overview of DP-SFGs, which we employ to map OTA circuits into transformer-friendly sequential data. Finally, we describe a precomputed LUT-based width estimator to translate DP-SFG parameters to transistor widths.
\vspace{-1mm}
\subsection{The transformer architecture}

\noindent
The transformer~\cite{vaswani_17} is viewed as one of the most promising deep learning architectures for sequential data prediction in NLP.  It relies on an attention mechanism that reveals interdependencies among sequence elements, even in long sequences. The architecture takes a series of inputs \((x_1, x_2, x_3, \ldots, x_n\)) and generates corresponding outputs \((y_1, y_2, y_3, \ldots, y_n\)).

\begin{figure}[b]
\vspace{-5mm}
\centering
\includegraphics[width=0.5\textwidth, bb=0 0 370 190]{fig/TransformermODEL.pdf}
\vspace{-5mm}
\caption{Architecture of a transformer.}
\label{fig:simpleTrans}
% \vspace{-2mm}
\end{figure}

The simplified architecture shown in Fig.~\ref{fig:simpleTrans} consists of $N$ identical stacked encoder blocks, followed by $N$ identical stacked decoder blocks. The encoder and decoder is fed by an input embedding block, which converts a discrete input sequence to a continuous representation for neural processing. Additionally, a positional encoding block encodes the relative or absolute positional details of each element in the sequence using sine-cosine encoding functions at different frequencies. This allows the model to comprehend the position of each element in the sequence, thus understanding its context. Each encoder block comprises a multi-head self-attention block and a position-wise feed-forward network (FFN); each decoder block, which has a similar structure to the encoder, consists of an additional multi-head cross-attention block, stacked between the multi-head self-attention and feed-forward blocks. The attention block tracks the correlation between elements in the sequence and builds a contextual representation of interdependencies using a scaled dot-product between the query ($Q$), key ($K$), and value ($V$) vectors:
\begin{equation}
\text{{Attention}}(Q, K, V) = \text{softmax}\left(\frac{QK^T}{\sqrt{d_k}}\right)V,
\end{equation}
where $d_k$ is the dimension of the query and key vectors. The FFN consists of two fully connected networks with an activation function and dropout after each network to avoid overfitting. The model features residual connections across the attention blocks and FFN to mitigate vanishing gradients and facilitate information flow.

\subsection{Driving-point signal flow graphs}

\noindent
The input data sequence to the transformer must encode information that relates the parameters of a circuit to its performance metrics.  Our method for representing circuit performance is based on the signal flow graph (SFG).  The classical SFG proposed by Mason~\cite{Mason53} provides a graph representation of linear time-invariant (LTI) systems, and maps on well to the analysis of linear analog circuits such as amplifiers. In our work, we employ the driving-point signal flow graph (DP-SFG)~\cite{ochoa_98,schmid_18}. The vertices of this graph are the set of excitations (voltage and current sources) in the circuit and internal states (e.g., voltages) in the circuit.  
% An edge is drawn between vertices that have an electrical relationship, and the weight on each edge is the gain of the edge;
An edge connects vertices with an electrical relationship, and the edge weight is the gain; 
for example, if a vertex $z$ has two incoming edges from vertices $x$ and $y$, with gains $a$ and $b$, respectively, then $z = ax + by$, using the principle of superposition in LTI systems.  To effectively use superposition to assess the impact of each node on every other node, the DP-SFG introduces auxiliary voltages at internal nodes of the circuit that are not connected to excitations. These auxiliary sources are structured to not to alter any currents or voltages in the original circuit, and simplifies the SFG formulation for circuit analysis.
% enable easy formulation of the SFG to analyze circuit behavior. 

\begin{figure}[t]
% \vspace{-6mm}
\centering
\includegraphics[width=0.9\linewidth, bb=0 0 320 140]{fig/DPSFG.pdf}
\vspace{-0.25cm}
\caption{~(a) Schematic and (b) DP-SFG for an active inductor.}
\label{fig:DP-SFG_ex}
\vspace{-5mm}
\end{figure}

Fig.~\ref{fig:DP-SFG_ex}(a) shows a circuit of an active inductor, which is an inductor-less circuit that replicates the behavior of an inductor over a certain range of frequencies. Fig.~\ref{fig:DP-SFG_ex}(b) shows the equivalent DP-SFG. In Section~\ref{sec:dp-sfg}, we provide a detailed explanation that shows how a circuit may be mapped to its equivalent DP-SFG. 


\ignore{
\subsection{Lookup table for MOSFET sizing}
\label{sec:LUT}

\noindent
As seen in Fig.~\ref{fig:DP-SFG_ex}, the edge weights in a DP-SFG include circuit parameters such as the transistor transconductance, $g_m$, and various capacitances in the circuit.  The circuit may be optimized to find values of these parameters that meet specifications, but ultimately these must be translated into physical transistor parameters such as the transistor width.   In older technologies, the square-law model for MOS transistors could be used to perform a translation between DP-SFG parameters and transistor widths, but square-law behavior is inadequate for capturing the complexities of modern MOS transistor models.
In this work, we use a precomputed lookup table (LUT) that rapidly performs the mapping to device sizes while incorporating the complexities of advanced MOS models.

\begin{figure}[htbp]
\vspace{-0.4cm}
\centering
\includegraphics[height=4cm]{fig/lut_fig_1.pdf}
\vspace{-0.55cm}
\caption{LUT generation using three DOFs, $V_{gs}$, $V_{ds}$ and $L$.}
\label{fig:lutgen}
\vspace{-0.1cm}
\end{figure}

The LUT is indexed by the $V_{gs}$, $V_{ds}$, and length $L$ of the transistor, and provides four outputs: the drain current ($I_d$), transconductance ($g_m$), source-drain conductance ($g_{ds}$), and drain-source capacitance ($C_{ds}$).
The entries of the LUT are computed by performing a nested DC sweep simulation across the three input indices for the MOSFET with a specific reference width, $W_{ref}$, as shown in Fig.~\ref{fig:lutgen}, and for each input combination, the four outputs are recorded.
\blueHL{Empirically, we see that the impact of $V_{sb}$ is small enough that it can be neglected, and therefore we set $V_{sb} = 0$ in the sweeps used to create the LUT.}

Our methodology uses this LUT, together with the $g_m/I_d$ methodology~\cite{silviera_96}, to translate circuit parameters predicted by the transformer to transistor widths. The cornerstone of this methodology relies on the inherent width independence of the ratio $g_m/I_d$ to estimate the unknown device width: this makes it feasible to use an LUT characterized for a reference width $W_{ref}$. 
We will elaborate on this procedure further in Section~\ref{sec:precomputedLUTs}, and show how the LUT, together with the $g_m/I_d$ method, can effectively estimate the device widths corresponding to the transformer outputs.
% \redHL{\sout{required to achieve equivalent DC operating characteristics within the circuit. Section III D \redHL{Do not hardcode section numbers!!} provides an in-depth explanation of the implementation details of this methodology.}}
}
\section{The \alg{} Algorithm}


In this section, we first propose the data structure and operations of \alg{}. Then we present how the flow classifier is designed. After that, we describe the application of \alg{}.


\subsection{Data Structure and Operations}


\bbb{Data structure:}
%
As shown in Figure~\ref{fig:workflow}, the data structure of \alg{} consists of three parts: a heavy part, a light part, and a flow classifier.
%
The heavy part is a key-value (KV) table with \(w_h\) buckets. Each bucket contains \(d_h\) cells, and each cell records a flow, including its flow ID \(f\) and flow size \(\hat{n}\). The heavy part is also associated with a hash function \(h(.) (1 \leqslant h(.) \leqslant w_h)\), which maps flows to buckets.
%
The light part is a CMS, which maintains the sizes of small flows using small counters (e.g., 8-bit) to save memory.
%
The flow classifier is a model that infers whether an incoming packet belongs to a large flow or a small flow. Its design is detailed in Section \ref{sec:alg:model}. Ideally, large flows are recorded in the heavy part, whereas the light part is used only for small flows.


\begin{figure}[!ht]
    \centering  
    \includegraphics[width=\linewidth]{Figures/workflow.pdf}
    \caption{Workflow of \alg{}.}
    \label{fig:workflow}
\end{figure}


\bbb{Insertion:}
%
When inserting a packet of flow \(f\), \alg{} locates the mapped bucket \(B[h(f)]\) using the hash function \(h\). There are three cases:


\textit{Case 1:} If \(f\) is already recorded in \(B[h(f)]\), \alg{} simply increments its flow size by 1.


\textit{Case 2:} If \(f\) is not in \(B[h(f)]\) and there is an empty cell, \alg{} inserts \((f, 1)\) into that cell.


\textit{Case 3:} If \(f\) is not in \(B[h(f)]\) and all cells in the bucket are full, \alg{} uses the flow classifier to predict whether \(f\) is a large flow or a small flow. Based on the classifier’s output, there are two sub-cases:
%
1) If \(f\) is a large flow, let \(f_{min}\) be the flow with the minimum flow size in \(B[h(f)]\). \alg{} evicts \(f_{min}\) from the bucket, inserts it into the light part, and then inserts \((f, 1)\) into \(B[h(f)]\).
%
2) If \(f\) is a small flow, \alg{} directly inserts \(f\) into the light part.


\bbb{Query:}
%
When querying the size of a flow \(f\), \alg{} first checks if \(f\) is in the heavy part. If so, it reports the recorded flow size; otherwise, it reports the result from the light part.


\begin{figure}[!ht]
    \centering  
    \includegraphics[width=0.8\linewidth]{Figures/example.pdf}
    \caption{An example of \alg{}.}
    \label{fig:example}
\end{figure}




\textit{Example 1:} Figure~\ref{fig:example} illustrates the different cases in the insertion process of \alg{}.
%
When inserting a packet of flow \(f1\), \alg{} computes the hash function \(h\) to locate bucket \(B1\). Since \(f1\) is already recorded in \(B1\), \alg{} simply increments its flow size by 1.


\textit{Example 2:} When inserting a packet of \(f2\), \alg{} locates bucket \(B2\). Since \(B2\) has an empty cell, \alg{} inserts \((f2, 1)\) into that cell.


\textit{Example 3:} When inserting a packet of \(f5\), \alg{} locates bucket \(B3\). Since \(B3\) is full, \alg{} uses the flow classifier to predict that \(f5\) is a small flow and therefore inserts \((f5, 1)\) into the light part.


\textit{Example 4:} When inserting a packet of \(f8\), \alg{} locates bucket \(B4\). Since \(B4\) is full, \alg{} predicts \(f8\) as a large flow. \alg{} then evicts the flow with the minimum flow size (i.e., \(f7\)) from \(B4\), inserts \((f8, 1)\) into \(B4\), and inserts \(f7\) into the light part.






\bbb{Optimization: Large-flow Locking.}
%
In the insertion process described above, if a hash collision occurs, \alg{} evicts the flow with the smallest recorded size. Although this approach generally works well, it may inadvertently evict newly arrived flows that are actually large but have not yet accumulated a significant size. To address this issue, we introduce a \textit{lock flag} in each cell of the heavy part. This flag tracks how often a flow is predicted to be large, thereby reducing the likelihood of evicting flows that were previously identified as large, even if their current size is still small.


\textit{Lock flag update:}
%
Whenever a packet is inserted and its flow is (re)classified, we update the lock flag based on the classifier’s prediction and the flow’s recorded size.
%
Let \(\hat{y} \in \{0, 1\}\) be the predicted label for this packet, where \(\hat{y} = 1\) indicates large flow and \(\hat{y} = 0\) indicates small flow.
%
Recall that \(\hat{n}\) represents the flow's recorded size.
%
Then the lock flag \(L \in \{0, 1\}\) is updated as follows:
%
\[
L \leftarrow
\begin{cases} 
1, & \text{w.p. } \frac{L \cdot \hat{n} + \hat{y}}{\hat{n} + 1},\\
0, & \text{otherwise}.
\end{cases}
\]
%
This rule can be viewed as an unbiased estimator of the fraction of times the flow is predicted to be large, accumulated over its updates (See Theorem~\ref{theorem:flag}). If \(L = 1\) after the update, we treat the flow as large and therefore lock it. Otherwise, if \(L = 0\), it is more likely to be a small flow and can be safely evicted if necessary.


\textit{Eviction policy:}
%
When a hash collision occurs and \alg{} needs to evict a flow from a full bucket, it first checks whether any flows in that bucket have \(L = 0\). 1) If there is at least one unlocked flow (\(L = 0\)), \alg{} evicts tthe one with the smallest size among them. 2) If all flows in the bucket are locked (\(L = 1\)), \alg{} must evict the flow with the minimum size regardless of its lock flag. Although the latter scenario should be rare, it can still occur due to the classification errors or the probabilistic nature of the lock flag update.






\subsection{Model Design}
\label{sec:alg:model}


We choose to adapt a Large Language Model (LLM) as our flow classifier due to its ability to capture complex patterns in packet headers. By leveraging an LLM, we can process each packet header in a contextual manner, enabling the classifier to learn nuanced relationships that simpler models might overlook. Furthermore, the inherent flexibility of LLMs makes them well-suited for handling packet headers of varying lengths and formats.




\bbb{Embedding:}
Since the raw packet header data cannot be directly interpreted by a language model, we introduce an embedding layer that transforms the packet headers into token embeddings. Specifically, we treat the packet header as a binary string and segment it into two-byte chunks, each serving as a token for the embedding layer. This approach circumvents the need for a cumbersome, field-by-field parsing.
% 
In practice, to prevent overfitting to specific IP addresses, we remove the source and destination IP fields before feeding the remaining header data into the model. Consequently, the classifier focuses on more generalizable features—such as transport-layer information—rather than memorizing particular endpoints in the training data.




\bbb{Objective Function:}
% 
A straightforward strategy might be to define a hard threshold \( T \) (e.g., 64) to classify flows as large (\( \geqslant T \)) or small (\( < T \)) categories. However, directly optimizing for a strict binary cutoff can lead to the following issues:

\begin{itemize}[leftmargin=*]
    \item Flows near the threshold (e.g., those with sizes 60-70) often share similar characteristics, making a single sharp boundary somewhat arbitrary.

    \item Misclassifying flows near the threshold has relatively little impact on the overall sketch accuracy, so aggressively fitting a binary boundary can introduce unnecessary complexity.
\end{itemize}


To address these concerns, we adopt a \textit{soft-label approach} that smooths the discrete large-versus-small boundary. Concretely, we assign a label to each flow based on
%
\[
\text{label} = \sigma \Bigl(a \bigl((\log(n) - \log(T) \bigr)\Bigr),
\]
%
where \(n\) is the flow size, \(T\) is the threshold, \(\sigma(\cdot)\) is the sigmoid function, and \(a\) is a scaling parameter. This design has several advantages:


\begin{itemize}[leftmargin=*]
\item Continuity: Rather than a hard 0/1 label, flows receive labels in the continuous range \((0, 1)\), enabling a smooth transition around the threshold.

\item Reduced sensitivity: Flows that are significantly larger than \(T\) yield labels near 1, while those much smaller than \(T\) yield labels near 0. Flows in the ambiguous region around \(T\) hover near 0.5, making misclassifications less punitive.

\item Smoother optimization: Training as a regression task on these soft labels typically exhibits more stable convergence than a strict classification objective.
\end{itemize}



In practice, this soft-label mechanism helps the classifier learn a nuanced notion of flow size, rather than fixating on a single, potentially noisy threshold. Large flows (e.g., above 1,000) naturally produce labels close to 1, whereas small flows (e.g., below 5) produce labels close to 0. Flows near \(T\) fall around 0.5, diminishing the adverse impact of uncertain classifications. Consequently, the classifier achieves better overall performance and adaptability across diverse network environments.




\subsection{Application}


In this section, we describe how to apply \alg{} to 3 typical tasks: flow size query, heavy hitter query, and hierarchical heavy hitter query.


\bbb{Flow size query:}
%
\alg{} can be used directly to measure flow sizes.


\textit{Optimization: using fingerprints.}
%
Following many existing works \cite{dhs, mimosketch}, \alg{} also supports the use of fingerprints in place of full flow IDs. This is particularly beneficial when the original ID is large (e.g., the 13-byte 5-tuple).
%
Although fingerprints may introduce collisions, they substantially reduce memory usage. Consequently, \alg{} can achieve higher accuracy under the same memory budget compared with recording the full flow IDs.




\bbb{Heavy hitter query:}
%
For heavy hitter query, \alg{} maintains the same insertion procedure described earlier.
%
When querying heavy hitters, \alg{} simply the heavy part to find all flows whose recorded sizes exceed the given threshold. Those flows are subsequently reported as heavy hitters.




\bbb{Hierarchical heavy hitter (HHH) query:}
%
To support HHH query, \alg{} replaces the light part with a CocoSketch \cite{cocosketch}. The insertion process remains primarily unchanged, except that flows which would originally be inserted into CMS are now inserted into CocoSketch. When performing an HHH query, \alg{} first merges the heavy and light parts into a single key-value table and then obtains HHH using the aggregation approach proposed by CocoSketch.




% \bbb{DDoS victim query:}
% %
% \alg{} works in conjunction with a Bloom filter (BF) \cite{bloomfilter} for DDoS victim query, where BF is used for deduplication and \alg{} is used for counting. When a packet with ID \(f = \langle src, dst\rangle\) is inserted, we first query BF to check whether \(f\) has already been inserted. If BF returns true, i.e. \(f\) is already recorded, then no action is taken. If the BF returns false, we insert \(f\) into BF, and \((dst, 1)\) into \alg{}.
% %
% When querying for DDoS, \alg{} scans the heavy part, and reports the set of flows whose corresponding values are greater than the threshold \(T_{DDoS}\).







\begin{figure}
    \centering
    \includegraphics[width=\columnwidth]{Figures/annotations_of_notes_narrow.png}
    \caption{Mean scores of community annotations of misleading posts.}
    \label{fig:annotations}
\end{figure}

We analyse the dataset prepared in \cref{sec:dataset} to answer the two research questions defined in \cref{sec:introduction}.

\subsection{RQ1: To what degree do community notes rely on fact-checkers?}
\label{sec:analysis_rq1}

According to \cref{fig:link_types}, at least 5\% of all English community notes contain an external link to professional fact-checkers. This number grows to 7\% when only considering notes rated as `helpful' (\cref{fig:link_types_helpful} in \cref{app:additional_material}). Conversely, only 1\% of notes rated as `not helpful' contain a fact-checking source (\cref{fig:link_types_not_helpful} in \cref{app:additional_material}). These figures are significantly larger than what was reported in previous studies (1.2\% \citep{kangur_who_2024}), possibly because \citet{kangur_who_2024} utilise a smaller dataset of fact-checking agencies and classify fact-checking divisions of popular journals as ``news''. The results imply that notes incorporating fact-checking sources are generally considered more helpful. 

We further assess whether notes with fact-checking sources are perceived to be of higher quality by analysing individual user ratings of notes both with and without such sources. Specifically, we collect user ratings for a balanced
(i.e., including of a fact-checking source or not) sample of 20K notes rated by at least 50 users, 
% , with half containing a link to professional fact-checking and the other half without.
and calculated the average ratings for the notes. As can be seen in \cref{fig:notes_individual_ratings} in \cref{app:additional_material}, community notes with fact-checking sources are generally rated higher than their counterparts. Interestingly, while notes with fact-checking links are more likely to be regarded as having a good source (higher \textit{HelpfulGoodSources}), they are also more likely to be rated as \textit{notHelpfulSourcesMissingOrUnreliable}.  \cref{tab:notes_with_bad_source.} in \cref{app:additional_material} contains a sample of such notes. 


\subsection{RQ2: What are the traits of posts and notes that rely on fact-checking sources?}
\label{sec:analysis_rq2}

\begin{table*}
    \centering
    \resizebox{1.0\textwidth}{!}
    {%
    \fontsize{8}{8}\selectfont
    \sisetup{table-format = 3.2, group-minimum-digits=3}
    \begin{tabular}{p{5cm}p{7cm}rrrrr}
    \toprule
    Tweet & Note & \rotatebox[origin=r]{270}{misleadingUnverifiedClaimAsFact} & \rotatebox[origin=r]{270}{misleadingOutdatedInformation} & \rotatebox[origin=r]{270}{misleadingFactualError} & \rotatebox[origin=r]{270}{misleadingSatire} & \rotatebox[origin=r]{270}{Fact Checking source} \\ \midrule
    The NASA War Document is absolutely terrifying \url{https://t.co/...} & misrepresenting a presentation by NASA scientist Dennis Bushnell, The lecture was not detailing plans by NASA to attack the world it was a lecture for defense industry professionals, and how defense tactics might rise to meet evolving threats in the future.   \url{https://leadstories.com/hoax-alert/2021/06/fact-check-the-future-is-now-is-not-a-nasa-war-document-plan-for-world-domination-and-phasing-out-of-humans.html} & \cmark & \xmark & \xmark & \xmark & \cmark \\ \addlinespace
    BREAKING NEWS: International Criminal Investigation calls on every public citizen to recommend indictments for Bill Gates, Anthony Fauci, Pfizer, BlackRock, Tedros and Christian Drosten for pushing everyone to receive the ineffective highly dangerous lethal experimental vaccines... & Video has been fact-checked by USA Today, was found to be misleading, and promotes a conspiracy theory about COVID ... \url{https://ca.movies.yahoo.com/movies/fact-check-viral-video-promotes-204414488.html} & \cmark & \xmark & \xmark & \xmark & \cmark \\ \addlinespace
    1) California is RED.
    It is just because of the MASSIVE Election Fraud that stupid, brainwashed people believe Calif. is blue. Joe Biden won only in the SFO Bay area ... & The map shows the results of Reagan's reelection in 1984, not Biden's election in 2020.  \url{https://en.wikipedia.org/wiki/1984\_United\_States\_presidential\_election\_in\_California} & \xmark & \cmark & \xmark & \xmark & \xmark \\ \addlinespace
    Davis blows up \$100,000 fireworks in Kai Cenat setup During the Mr Beast Stream ... & The second photo is from a house fire in Atlanta in 2019. \url{https://www.11alive.com/article/news/local/woodland-brook-drive-cause-of-house-fire/85-ecb7df9b-5f65-44e9-bf9d-8c162d36c334} & \xmark & \cmark & \xmark & \xmark & \xmark \\ \addlinespace
    @cnviolations I swear community notes are the only good thing Elon added since he bought Twitter. & Community notes was first launched under former Twitter CEO Jack Dorsey in 2021 under the name of ``Birdwatch''. The only thing Elon Musk did was that he renamed the feature to community notes.    \url{https://blog.twitter.com/en\_us/topics/product/2021/introducing-birdwatch-a-community-based-approach-to-misinformation}    \url{https://www.reuters.com/article/factcheck-elon-birdwatch-idUSL1N31Z2VG/} &
     \xmark & \xmark & \cmark & \xmark & \cmark \\ \addlinespace
    Thailand will become the first country to make the contract null and void, meaning that Pfizer will become responsible for all vaccine injuries ... & Thailand has no plans to void its Pfizer COVID vaccine contract, an official with the country’s National Vaccine Institute said. Thailand’s Department of Disease Control also rejected the claims as ``fake news.'' ...  \url{https://apnews.com/article/fact-check-covid-vaccine-pfizer-thailand-203948163859} & \xmark & \xmark & \cmark & \xmark & \cmark \\ \addlinespace
    Hilarious tweets by footballers, A thread: 1. Virgil Van Dijk [Current Liverpool Captain] \url{https://t.co/...} & Virgil Van Dijk did not tweet this, the tweet was made by a fan account in his name.    \url{https://www.pinkvilla.com/sports/fact-check-did-virgil-van-dijk-really-root-for-man-u-because-no-one-likes-liverpool-in-resurfaced-viral-tweet-1287250} & \xmark & \xmark & \xmark & \cmark & \cmark \\ \addlinespace
    Rob Reiner announces he’s on the Epstein Client List and Epstein Flight logs. What a fool! When a lawyer tells me to STFU, I STFU! \url{https://t.co/...} & This is a digitally altered photo that might be misinterpreted even if used as a joke.    The name Rob Reiner is misspelled, and the text is not on Reiner's X timeline.    \url{https://twitter.com/robreiner?t=iqu43-NszIW5oOM\_KqRSpw} & \xmark & \xmark & \xmark & \cmark & \xmark \\
    \bottomrule
    \end{tabular}
    }
    \caption{A sample of tweets, notes, and their community annotations, as well as whether the note contains a fact-checking link.}
    \label{tab:community_annotation_example}
\end{table*}

\begin{figure}[!t]
    \centering
    \includegraphics[width=1\columnwidth]{Figures/manual_annotation.png}
    \caption{(a) strategies in debunking claims related to broader narratives. (b) the different ways in which fact-checking sources are used to debunk claims.}
    \label{fig:manual_annotation}
\end{figure}
% \begin{table}[h]
%     \centering
%     \begin{tabular}{p{2cm}lcc}
     
%        & & \multicolumn{2}{c}{Fact-check source} \\
        
%       & & Yes & No \\
%       \cline{3-4}
%         \multirow{2}{*}{\shortstack[l]{Related to a\\conspiracy} } & Yes & 0.216 & 0.112 \\
%         & No & 0.279 & 0.39 \\
%     \end{tabular}
%     \caption{Your table caption here}
%     \label{tab:fact_check}
% \end{table}


\def\arrvline{\hfil\kern\arraycolsep\vline\kern-\arraycolsep\hfilneg}

\begin{table}[!t]
% \fontsize{9}{9}\selectfont
    \centering
    \begin{tabular}{llc|c}
     
       & & \multicolumn{2}{c}{FC source} \\ 
       
      & & \cmark & \xmark \\
       \cmidrule(l){3-4}
      % \cmidrule(r){3-3}\cmidrule(l){4-4}
       \multirow{2}{*}{\rotatebox[origin=r]{90}{\parbox[r]{0.5cm}{\centering Conspi-racy}}} & \cmark \arrvline &  22\% & 11\% \\
       \cmidrule(l){2-4}
        & \xmark \arrvline & 28\% & 39\% \\
    \end{tabular}
    \caption{Percentage of samples related to a broader narrative or conspiracy vs. have a fact-checking source.}
    \label{tab:conspiracies_model_results}
\end{table}


We begin by performing a topic analysis, comparing topics of posts whose notes reference fact-checking sources to those citing other sources. To this end, we apply a strong zero-shot text classification model\footnote{\url{https://huggingface.co/r-f/ModernBERT-large-zeroshot-v1} with default settings.} to our $\mathcal{S}_\text{text}$ subset by classifying spans of the form ``\texttt{Tweet:<POST TEXT>; Note <NOTE TEXT>}'' into one of 13 classes. The authors manually evaluated the quality of the classification results and considered it satisfactory. Notably (\cref{fig:topics}), fact-checking sources are more likely to be included in posts related to high-stakes issues such as health, science, and scams and less likely to be included in posts on tech or sports.

We then analyse annotations (binary attributes explaining the warrant for the note) by community note authors.
% When writing a note, the author labels the original post with 
\cref{fig:annotations} contains the full breakdown of annotations for notes with and without fact-checking sources. Notes containing a link to fact-checking sources are overrepresented in posts where unverified information is presented as a fact or when the post contains a factual error. Conversely, they are under-represented in posts with outdated information or satirical content. \cref{tab:community_annotation_example} contains a sample of such notes. 

These results indicate that community note-writers adapt their strategies based on the stakes and scope of the claim, and the depth of research needed to counter misinformation. We hypothesise that they are more likely to rely on external fact-checking when refuting complex or unverifiable claims \citep{wuehrl-etal-2024-makes}, as well as claims related to broader narratives or conspiracy theories which cannot be fully addressed in the scope of a note.\footnote{For example, the claim ``Michelle Obama is a male''.} Conversely, claims involving misleading media can often be debunked with examples alone, making fact-checking sources unnecessary. To investigate this hypothesis, the authors of this paper manually annotated 400 $<\text{post}, \text{note}>$ pairs from $\mathcal{S}_\text{text}$ with attributes related to the complexity of the claims and how community notes address them. (See \cref{app:manual_annotation_setup} for annotation guidelines). The results (\cref{fig:manual_annotation}.a) support our hypothesis. Claims related to broader narratives or conspiracy theories are much more likely to include a link to a fact-checking source.
In contrast, other types of claims are more likely to be addressed by providing missing context or by invalidating the credibility of the claim's source. 
Additionally, \cref{fig:manual_annotation}.b depicts the different ways in which fact-checking sources are used to debunk claims. It demonstrates how such sources are rarely used to provide missing context but rather focus on discrediting sources of claims and providing scientific evidence.

\begin{figure}[!t]
    \centering
    \includegraphics[width=1\columnwidth]{Figures/Distribution_of_topics_vertical.png}
    \caption{Distribution of notes' topics, with and without a fact-checking source.}
    \label{fig:topics}
\end{figure}

We extend the manual annotation to an LLM-based analysis of 8K balanced $(\text{post}, \text{note})$ pairs from $\mathcal{S}_\text{text}$. We task OpenAI's GPT-4\footnote{Version \texttt{gpt-4o-2024-08-06}.} with determining whether a pair relates to a broader narrative or a conspiracy theory. \Cref{lst:prompt_conspiracy} in \cref{app:reproducibility} details the prompt used. To evaluate model accuracy, two authors independently labelled 100 balanced pairs, achieving an agreement rate of $0.88$ and resolving disagreements through discussion. The model attained an 
$F_1$ score of $0.85$---strong performance for this challenging task. The results (\cref{tab:conspiracies_model_results}) support our hypothesis: pairs related to a broader narrative or conspiracy theory are \textit{twice} as likely to cite fact-checking sources compared to other sources. In contrast, other pairs are nearly 30\% less likely to do so. These findings also highlight the prevalence of such claims and further underscore the importance of fact-checking in combating complex misinformation narratives.





\section{Evaluation}
\label{sec:evaluation}

\stelevaltable
\avevaltable

\paragraph{STEL-or-Content (SoC) Benchmark}

In order to evaluate our style embeddings, we construct a multilingual version of the SoC benchmark \citep{styleemb}.\footnote{The English SoC benchmark covered formality, complexity, number usage, contraction usage, and emoji usage.} SoC measures the ability of a model to embed sentences with the same style closer in the embedding space than sentences with the same content. We construct our \textbf{multilingual SoC benchmark} by sampling 100 pairs of parallel {\tt pos}-{\tt neg} examples for each language from four ground-truth datasets covering four style features and 22 languages: simplicity \citep{ryan-etal-2023-revisiting}, formality \citep{briakou-etal-2021-ola}, toxicity \citep{dementieva2024overview}, and positivity \citep{mukherjee2024multilingualtextstyletransfer}.\footnote{Combined, these datasets cover the following languages: Amharic, Arabic, Bengali, German, English, Spanish, French, Hindi, Italian, Japanese, Magahi, Malayalam, Marathi, Odia, Punjabi, Portuguese (Brazil), Russian, Slovenian, Telugu, Ukrainian, Urdu, and Chinese.} Each instance in our multilingual SoC benchmark consists of a triplet ($a$, {\tt pos}, {\tt neg}) constructed as explained in Section \ref{sec:styledistance}. However, following \citet{styleemb}, the distractor text in our SoC benchmark is always a paraphrase of {\tt pos}.
% \textcolor{gray}{For each instance of our multilingual SoC benchmark, we take two pairs of parallel examples to get (1) an anchor sentence, (2) a sentence with the same style but different content than the anchor, and (3) a sentence with the same content but different style than the anchor.}
A model tested on this benchmark is expected to embed $a$ and {\tt pos} closer than $a$ and {\tt neg}. We rate a model by computing the percent of instances it achieves this goal for across all instances. We form test instances for each $f \in F$ in a language corresponding to all possible triplets, resulting in 4,950  instances for each language-style combination.

We also construct a \textbf{cross-lingual SoC benchmark} that addresses embeddings' ability to capture style similarity \textit{across languages}. This can be useful, for example, to evaluate style preservation in translations. We construct the benchmark with the XFormal dataset \citep{briakou-etal-2021-ola}, which includes parallel data in French, Italian and Portuguese. %We use a similar formulation as described above to create each instance. However, rather than taking both pairs from the same language, we take the sentence pair (which is (2) and (3) described above) from a different language as the anchor pair. 
We again create triplets as described above, but instead of using {\tt pos} and {\tt neg} texts from the same language as the anchor ($a$), we sample them from a different language than $a$. We end up with 19,800 instances for each style in each language. Appendix \ref{sec:appendix:stelfig} contains illustrative examples from each benchmark.



% Next, we sample every combination of two paraphrase pairs from the 100 pairs for each style feature and language to form $_{100} C _{2} = 4,950$ STEL-or-Content instances. For each instance, we select one of the paraphrase pairs to be the anchor pair and the other to be the sentence pair. Following \citet{stel}, we replace one of the texts in the sentence pair with one of the texts from the anchor pair, resulting in (1) an anchor sentence, (2) a sentence with the same style but different content than the anchor, and (3) a sentence with the same content but different style than the anchor. The model is asked to embed (1) and (2) closer together than (1) and (3), and we compute the average across all instances.

% We elect not to use the STEL benchmark proposed by \citet{stel} because we believe STEL isn't as strong of a test on the content-independence of style embeddings as STEL-or-Content, and \citet{patel2024styledistancestrongercontentindependentstyle} find that base models trained for semantic embeddings perform well on STEL but not STEL-or-Content.

% We also construct crosslingual benchmarks that evaluate the ability of embeddings to capture style similarity \textit{across languages}. This can be useful, for example, to evaluate style preservation in translations. We address formality in this benchmark and use the XFormal dataset \citep{briakou-etal-2021-ola}, which includes data parallel in content across French, Italian and Portuguese. Given 100 paraphrase pairs in language A and 100 paraphrase pairs in language B, we create a STEL-or-Content instance out of every combination of pairs, resulting in $100 \cdot 99 = 9,900$ instances because we ignore the instances in which both pairs have the same content. We then use a similar process as described previously to assign the anchor and sentence pairs for every instance. Finally, we repeat this process with the style feature of the anchor sentence swapped (i.e. using the informal sentences rather than the formal sentences) to end up with $9,900 \cdot 2 = 19,800$ instances. Appendix \ref{sec:appendix:stelfig} contains examples from each benchmark.
% We will make our STEL-or-Content evaluation datasets publicly available as a resource.

\paragraph{SoC Evaluation Results}

The results obtained by \textsc{mStyleDistance} on the multilingual and cross-lingual SoC benchmarks are presented in Table \ref{table:steleval}. As no general multilingual style embeddings are currently available, we compare with a base multilingual encoder model \texttt{xlm-roberta-base} \citep{Conneau2019UnsupervisedCR} as well as a number of English-trained style embedding models applied in zero-shot fashion to multilingual text: \citet{styleemb}, \textsc{StyleDistance} embeddings \citep{patel2024styledistancestrongercontentindependentstyle}, and \texttt{LISA} \citep{lisa}. 
%\citep{lisa} as baseline models to compare against. 
\textsc{mStyleDistance} embeddings outperform these models on multilingual and cross-lingual SoC tasks confirming its suitability for multilingual applications. The other models perform slightly better than the untrained \texttt{xlm-roberta-base} but still worse than \textsc{mStyleDistance}. 

\paragraph{Ablation Experiments}

\ablationsimpletable

Following \citep{patel2024styledistancestrongercontentindependentstyle}, we run several ablation experiments to evaluate how well our model generalizes to unseen style features and languages. In the \textbf{In-Domain} condition, all style features are included in the training data for every language. To test generalization to unseen style features, in the \textbf{Out of Domain} condition, any style feature directly equivalent to those features tested in the Multilingual and Cross-lingual SoC  benchmarks are excluded from the training data. \textbf{Out of Distribution} further removes any style features indirectly similar or related to those tested in the benchmarks. Finally, \textbf{No Language Overlap} removes the languages present in the benchmark from the training data, in order to test generalization to new languages. Our results are given in Table \ref{table:simplifiedablation} where we measure how much of the performance increase on SoC benchmarks over the base model is retained, despite ablating training data. The results indicate that our method generalizes reasonably well to both ``out of domain'' and ``out of distribution'' style features, and very well to languages not in the training data. Further details on features and languages ablated and full results are provided in Appendices  \ref{sec:appendix:ablationdetails} and \ref{sec:appendix:ablationfull}.

\paragraph{Downstream Task}

Following \citet{patel2024styledistancestrongercontentindependentstyle}, we also evaluate our \textsc{mStyleDistance} embeddings in the authorship verification (AV) task, where the goal is to determine if two documents were written by the same author using stylistic features \citep{authorshipverification}. We use the datasets released by PAN\footnote{\url{https://pan.webis.de}} between 2013 and 2015 in Greek, Spanish, and Dutch. Our results are given in Table \ref{table:aveval_table}. \textsc{mStyleDistance} vectors outperform existing English style embedding models on Spanish and Greek, while Dutch shows similar performance to English \textsc{StyleDistance}. We hypothesize that the linguistic proximity (West Germanic roots) of the two languages helps \textsc{StyleDistance} to generalize to Dutch.






\section{Related Work}


\bbb{Traditional sketches:}
%
Sketches can broadly be categorized into two types:
%
\textit{1) Classic sketches} consist of a counter matrix and multiple hash functions. During updates and queries, flow IDs are hashed into multiple counters, and the mapped counters are then updated and queried accordingly. Typical classic sketches include the Count-Min Sketch (CMS) \cite{cmsketch}, the Conservative Update Sketch (CUS) \cite{cusketch}, and the Count Sketch (CS) \cite{csketch}. However, classic sketches fail to account for the highly skewed nature of network traffic, resulting in memory waste.
%
\textit{2) Sophisticated sketches} address this problem by separating large flows from small flows \cite{elasticsketch, sketchlearn, nitrosketch, nze-sketch, bitsense}. These sketches typically consist of multiple parts, with different parts using different data structures to record flows of varying sizes. A typical sophisticated sketch is ElasticSketch \cite{elasticsketch}, which is composed of a heavy part and a light part. The heavy part is a key-value table, while the light part is a CM sketch. Packets are first inserted into the heavy part. When the heavy part becomes full, ElasticSketch uses an eviction method to remove the flow that is most likely to be small and inserts it into the light part. Researchers have attempted to improve sketch accuracy by adjusting the flow selection method for eviction, but most of these attempts have been based on experience.
%
Additional sketches have been proposed for specialized tasks, such as heavy hitter query \cite{css, mvsketch, precision}, hierarchical heavy hitter query \cite{rhhh, cocosketch}, and DDoS victim/super-spreader query \cite{opensketch, spreadsketch}.




\bbb{ML-based sketches:}
%
In recent years, researchers have attempted to use machine learning methods to improve sketch performance. Learned Count-Min Sketch (LCMS) \cite{lcmsketch} employs an RNN to learn and infer whether a flow is large and uses an additional hash table to record large flows. Other solutions \cite{bertsimas2021frequency, mlsketch, metasketch} use ML to enhance hashing, updating, and querying processes within the sketch, leading to improved performance. The main difference between these works and ours is that they only learn features from the flow ID and distribution, without utilizing other information carried by the packets.




\bbb{LLM for other network tasks:}
%
Some works explore how to adapt LLMs to other network operations, such as traffic classification, viewport prediction, adaptive bitrate streaming, and cluster job scheduling. Typical works include PERT \cite{pert}, ET-BERT \cite{et-bert}, YaTC \cite{yatc}, and NetLLM \cite{netllm}.


% \section{Discussion and Future Work}

% %Onscad is insample data cause these LLMS have seen openscad

% The decision space of language design is enormous, so we had to make some decisions about what to explore in the language design of AIDL. In particular, we did not build a new constraint system from scratch and instead developed ours based on an open-source constraint solver. This limited the types of primitives we allow, e.g. ellipses are not currently supported. \jz{Additionally, rectangles in AIDL are constrained to be axis-aligned by default because we found that in most use cases, a rectangle being rotated by the solver was unintuitive, and we included a parameter in the language allows rectangles to be marked as rotatable. While this feature was included in the prompts to the LLM, it was never used by the model. We hope to explore prompt-engineering techniques to rectify this issue in the future. Similarly, we hope to reduce the frequency of solver errors by providing better prompts for explaining the available constraints.} \adriana{Add two other limitations to this paragraph: that we typically noticed that things are axis alignment, say why we use this as default and in the future could try to get the gpt to not use default more often. Mention that we still have Solver failures that could be addressed by better engineering in future. }

% In testing our front-end, we observed that repeated instances of feedback tends to reduce the complexity of models as the LLM would frequently address the errors by removing the offending entity. This leads to unnecessarily removed details. More extensive prompt-engineering could be employed in future work to encourage the LLM to more frequently modify, rather than remove, to fix these errors. \adriana{no idea what this paragraph is trying to say}


% \adriana{This seems  like a future work paragraph so maybe start by saying that in the future you could do other front end or fine tune a model with aidl, we just tested the few shot.  } \jz{In the future, we hope to improve our front-end generation pipeline by finetuning a pretrained LLM on example AIDL programs.} In addition, multi-modal vision-langauge model development has exploded in recent months. Visual modalities are an obvious fit for CAD modeling -- in fact, most procedural CAD models are produced in visual editors -- but we decided not to explore visual inputs yet based on reports ([PH] cite OPENAIs own GPT4V paper) that current vision-language models suffter from the same spatial reasoning issues as purely textual models do (identifying relative positions like above, left of, etc.). This also informed our decision to omit spline curves which are difficult to describe in natural language. This deficit is being addressed by the development of new spatial reasoning datasets ([PH] cite visual math reasoning paper), so allowing visual user input as well as visual feedback in future work with the next generation of models seems promising.

% The decision space of language design is enormous, so it was impossible to explore it all here. We had to make some decisions about what to explore, guided by experience, conjecture, technical limitations, and anecdotal experience. Since we primarily explore the interaction between language design and language models in order to overcome the shortcomings in the latter, we did not wish to focus effort on building new constraint systems. This led us to use an open-source constraint solver to build our solver off of. This limited the types of primitives we allow; in particular, most commercial geometric solvers also support ellipses.

% In testing our generation frontend, we observed that repeated instances of feedback tended to reduce the complexity of models as the LLM would frequently address the errors by removing the offending entity. This is a fine strategy for over-constrained systems, but can unnecessarily remove detail when done in response to a syntax or validation error. More extensive prompt-engineering could be employed to encourage the LLM to more frequently modify, rather than remove, to fix these errors


% In recent months, multi-modal vision-language model development has exploded. Visual modalities are an obvious fit for CAD modeling -- in fact, most procedural CAD models are produced in visual editors -- but we decided not to explore visual input yet based on reports (cite OpenAIs own GPT4V paper) that current vision-language models suffer from the same spatial reasoning issues as purely textual models do (identifying relative positions like above, left of, etc.). This also informed our decision to omit spline curves; they are not easily described in natural language. This deficit is being addressed by the development of new spatial reasoning datasets (cite visual math reasoning paper), so allowing visual user input as well as visual feedback in future work with the next generation of models seems promising.



\section{Conclusion}

AIDL is an experiment in a new way of building graphics systems for language models; what if, instead of tuning a model for a graphics system, we build a graphics system tailored for language models? By taking this approach, we are able to draw on the rich literature of programming languages, crafting a language that supports language-based dependency reasoning through semantically meaningful references, separation of concerns with a modular, hierarchical structure, and that compliments the shortcomings of LLMs with a solver assistance. Taking this neurosymbolic, procedural approach allows our system to tap into the general knowledge of LLMs as well as being more applicable to CAD by promoting precision, accuracy, and editability. Framing AI CAD generation as a language design problem is a complementary approach to model training and prompt engineering, and we are excited to see how advance in these fields will synergize with AIDL and its successors, especially as the capabilities of multi-modal vision-language models improve. AI-driven, procedural design coming to CAD, and AIDL provides a template for that future.

% Using procedural generation instead of direct geometric generation enables greater editability, accuracy, and precision
% Using a general language model allows for generalizability beyond existing CAD datasets and control via common language.
% Approaches code gen in LLMs through language design rather than training the model or constructing complexing querying algorithms. This could be a complimentary approach
% Embedding as a DSL in a popular language allows us to leverage the LLMs syntactic knowledge while exploiting our domain knowledge in the language design
% LLM-CAD languages should hierarchical, semantic, support constraints and dependencies




%In this paper, we proposed AIDL, a language designed specifically for LLM-driven CAD design. The AIDL language simultaneously supports 1) references to constructed geometry (\dgone{}), 2) geometric constraints between components (\dgtwo{}), 3) naturally named operators (\dgthree{}), and 4) first-class hierarchical design (\dgfour{}), while none of the existing languages supports all the above. These novel designs in AIDL allow users to tap into LLMs' knowledge about objects and their compositionalities and generate complex geometry in a hierarchical and constrained fashion. Specifically, the solver for AIDL supports iterative editing by the LLM by providing intermediate feedback, and remedies the LLM's weakness of providing explicit positions for geometries.

%\adriana{This seems  like a future work paragraph so maybe start by saying that in the future you could do other front end or fine tune a model with aidl, we just tested the few shot.  }
%\paragraph{Future work} In recent months, multi-modal vision-language model development has exploded. Visual modalities are an obvious fit for CAD modeling -- in fact, most procedural CAD models are produced in visual editors -- but we decided not to explore visual input yet based on reports (cite OpenAIs own GPT4V paper) that current vision-language models suffer from the same spatial reasoning issues as purely textual models do (identifying relative positions like above, left of, etc.). This also informed our decision to omit spline curves; they are not easily described in natural language. This deficit is being addressed by the development of new spatial reasoning datasets (cite visual math reasoning paper), so allowing visual user input as well as visual feedback in future work with the next generation of models seems promising. 

\bibliographystyle{ACM-Reference-Format}
\bibliography{reference}

\end{document}
\endinput
%%
%% End of file `sample-sigconf.tex'.
