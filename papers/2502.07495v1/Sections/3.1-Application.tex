\subsection{Application}


In this section, we describe how to apply \alg{} to 3 typical tasks: flow size query, heavy hitter query, and hierarchical heavy hitter query.


\bbb{Flow size query:}
%
\alg{} can be used directly to measure flow sizes.


\textit{Optimization: using fingerprints.}
%
Following many existing works \cite{dhs, mimosketch}, \alg{} also supports the use of fingerprints in place of full flow IDs. This is particularly beneficial when the original ID is large (e.g., the 13-byte 5-tuple).
%
Although fingerprints may introduce collisions, they substantially reduce memory usage. Consequently, \alg{} can achieve higher accuracy under the same memory budget compared with recording the full flow IDs.




\bbb{Heavy hitter query:}
%
For heavy hitter query, \alg{} maintains the same insertion procedure described earlier.
%
When querying heavy hitters, \alg{} simply the heavy part to find all flows whose recorded sizes exceed the given threshold. Those flows are subsequently reported as heavy hitters.




\bbb{Hierarchical heavy hitter (HHH) query:}
%
To support HHH query, \alg{} replaces the light part with a CocoSketch \cite{cocosketch}. The insertion process remains primarily unchanged, except that flows which would originally be inserted into CMS are now inserted into CocoSketch. When performing an HHH query, \alg{} first merges the heavy and light parts into a single key-value table and then obtains HHH using the aggregation approach proposed by CocoSketch.




% \bbb{DDoS victim query:}
% %
% \alg{} works in conjunction with a Bloom filter (BF) \cite{bloomfilter} for DDoS victim query, where BF is used for deduplication and \alg{} is used for counting. When a packet with ID \(f = \langle src, dst\rangle\) is inserted, we first query BF to check whether \(f\) has already been inserted. If BF returns true, i.e. \(f\) is already recorded, then no action is taken. If the BF returns false, we insert \(f\) into BF, and \((dst, 1)\) into \alg{}.
% %
% When querying for DDoS, \alg{} scans the heavy part, and reports the set of flows whose corresponding values are greater than the threshold \(T_{DDoS}\).
