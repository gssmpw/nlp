\section{Related Work}
\subsection{Generating Attractive Ad Text\label{sec:related_work_atg}}
With the increasing demand for online advertising, the manual creation of ad text has reached its capacity limits. 
Therefore, researchers have focused on ATG \cite{murakami2023atgsurvey}.
The goal of advertising is to attract interest in a product or service and motivate users to take action such as clicking and purchasing. 
Therefore, generating attractive ads is critical to the success of online advertising.

\updates{
Various methods have been developed to generate attractive ad texts, ranging from template-based approaches \cite{Bartz2008-ke,thomaidou2013} to neural-based techniques \cite{Hughes2019-sh,Kamigaito2021-iy}.
Our work differs from previous studies in that it focuses on how the wording and style of ad text affect attractiveness. 
Unlike previous studies that evaluated generated ad texts without considering content variations, we ensured semantic equivalence between ad pairs before assessing their attractiveness. 
This approach allowed us to specifically analyze the impact of expressions on attractiveness. 
We believe that identifying the key linguistic factors that influence the attractiveness of ad texts is crucial for exploring the potential directions for advancing ATG methods.
}

\subsection{Understanding Attractive Ad Text}
% - 何が魅力度に影響を与えるかの分析
%   - 先行研究
%     - 説得戦略, 感情, 訴求ポイントといった観点
%   - 本研究
%     - 広告文の言語的特徴に焦点を当てる
%   - 主な違い
%     - 先行研究では広告文の内容的側面から分析しているが、本研究では表現的側面を対象とする
Understanding the factors that affect the attractiveness of ad text is crucial to the success of the ad creation process. 
Various efforts have been made to analyze the factors influencing the attractiveness of ad texts, such as advertising appeal \cite{murakami-etal-2022-aspect}, persuasive tactics \cite{yuan2023persuadetoclick}, and emotions \cite{youngmann2020}. 
This study investigates the linguistic features of ad texts that affect their attractiveness.

% - 分析に使用したデータ
%   - 先行研究
%     - click数や閲覧数といったログデータに基づいた分析
%   - 先行研究の問題
%     - これらのデータは公開されていないため再現や知見の積み上げができない
%   - 本研究
%     - 人手評価を通して人間の選好を収集した。それらのデータは公開予定である。
\updates{
A common approach for studying the factors influencing human preferences is to use log data, which measures attractiveness based on clicks and views. However, these log data are often proprietary and not publicly available, hindering research replication and knowledge advancement.
Therefore, we collected human preferences through manual evaluations to make the data publicly available.
}

% - 言語的特徴が選好に影響を与える要因の複雑性への対処
%   - 先行研究
%     - 因果推論による交絡因子の調整
%   - 先行研究の問題
%     - ブランド名による影響のみを対象としており、その他の内容の影響は調整できていない
%   - 本研究
%     - 内容が同じだがwriting styleだけ異なる言い換えを用意して、言語的特徴の違いに焦点を当てた分析を可能としている。
In a study closely related to our work, \citet{pryzant-etal-2018-interpretable} examined the impact of writing style on ad performance, while controlling for potential confounding variables. 
However, this study only considered the influence of the brand names and neglected other content-related factors. 
In our study, we constructed a paraphrase dataset varying in writing style to focus on the linguistic differences between ad texts while mitigating content variations, thereby enabling the analysis of human preferences centered on these linguistic features.