\documentclass[9pt, conference]{IEEEtran}
\IEEEoverridecommandlockouts
% The preceding line is only needed to identify funding in the first footnote. If that is unneeded, please comment it out.
%Template version as of 6/27/2024
\usepackage{cite}
\usepackage{amsmath,amssymb,amsfonts}
% \usepackage{algorithmic}
\usepackage{graphicx}
\usepackage{textcomp}
% \usepackage{xcolor}
\def\BibTeX{{\rm B\kern-.05em{\sc i\kern-.025em b}\kern-.08em
    T\kern-.1667em\lower.7ex\hbox{E}\kern-.125emX}}

%=========================
\usepackage{algorithm}
\usepackage{algpseudocode}
\usepackage{array}
\usepackage[caption=false,font=footnotesize,labelfont=rm,textfont=rm]{subfig}
\usepackage{textcomp}
\usepackage{hyperref}
\usepackage{stfloats}
\usepackage{url}
\usepackage{verbatim}
\usepackage{tablefootnote}
\usepackage{color, colortbl}
\usepackage[capitalize,noabbrev]{cleveref}
\usepackage{caption2}
\usepackage{booktabs}
\usepackage{threeparttable}
\usepackage[x11names]{xcolor}
\usepackage{subfloat}
\usepackage{multirow}

\def\x{{\mathbf x}}
\def\L{{\cal L}}

\usepackage{xspace}
\def\eg{\emph{e.g.}\xspace} 
\def\Eg{\emph{E.g.}\xspace}
\def\ie{\emph{i.e.}\xspace} 
\def\Ie{\emph{I.e.}\xspace}
\def\cf{\emph{c.f.}\xspace} 
\def\Cf{\emph{C.f.}\xspace}
\def\etc{\emph{etc}\xspace} 
\def\vs{\emph{vs}\xspace}
\def\wrt{\emph{w.r.t.}\xspace} 
\def\dof{d.o.f\xspace}
\def\etal{\emph{et al}\xspace}

\definecolor{darkred}{rgb}{0.7,0,0}
\definecolor{darkgreen}{rgb}{0,0.46,0}
\definecolor{purple}{rgb}{0.6,0,0.5}
\definecolor{cholocate}{HTML}{d2691e}
\definecolor{slateblue}{HTML}{6a5acd}
\newcommand{\nj}{\color{darkred}}
\newcommand{\wl}{\color{purple}}
\newcommand{\liu}{\color{slateblue}}
\newcommand{\fin}{\color{black}}
\newcommand{\todo}{\color{red}$\blacktriangleright$~}

\newcommand{\nameFramework}{GN-FT\xspace}
% \setlength{\abovecaptionskip}{2pt}
%===================================

\begin{document}

\title{Gradient Norm-based Fine-Tuning for Backdoor Defense in Automatic Speech Recognition
% {\footnotesize \textsuperscript{*}Note: Sub-titles are not captured for https://ieeexplore.ieee.org  and
% should not be used}
% \thanks{Identify applicable funding agency here. If none, delete this.}
}

% \author{\IEEEauthorblockN{1\textsuperscript{st} Nanjun Zhou$^*$}
% \IEEEauthorblockA{
% % \textit{Artificial Intelligence Thrust} \\
% \textit{School of Computer Science and Engineering}\\
% \textit{SCUT}\\
% Guangzhou, China \\
% 202130561134@mail.scut.edu.cn}
% \and
% \IEEEauthorblockN{2\textsuperscript{nd} Weilin Lin$^*$ \thanks{* Equal Contribution.}}
% \IEEEauthorblockA{
% \textit{Artificial Intelligence Thrust} \\
% \textit{HKUST(GZ)}\\
% Guangzhou, China \\
% wlin760@connect.hkust-gz.edu.cn}
% \and
% \IEEEauthorblockN{3\textsuperscript{rd} Li Liu$^\dagger$ \thanks{$\dagger$ Corresponding Author.}}
% \IEEEauthorblockA{
% \textit{Artificial Intelligence Thrust} \\
% \textit{HKUST(GZ)}\\
% Guangzhou, China \\
% avrillliu@hkust-gz.edu.cn}

% }

\author{
  Nanjun Zhou\textsuperscript{1,2,*}, Weilin Lin\textsuperscript{1,*}\thanks{* Equal Contribution.}, Li Liu\textsuperscript{1,$\dagger$} \thanks{$\dagger$ Corresponds to Li Liu (\href{mailto:avrillliu@hkust-gz.edu.cn}{avrillliu@hkust-gz.edu.cn}).}\\
  \textsuperscript{1}\textit{The Hong Kong University of Science and Technology (Guangzhou), Guangzhou, China}\\
  \textsuperscript{2}\textit{South China University of Technology, Guangzhou, China}}
% \begin{enumerate}
%     \begin{enumerate}
%         \item \item \item \item \item}
%     \end{enumerate}
% \end{enumerate}

\maketitle

\begin{abstract}
Backdoor attacks have posed a significant threat to the security of deep neural networks (DNNs). Despite considerable strides in developing defenses against backdoor attacks in the visual domain, the specialized defenses for the audio domain remain empty. Furthermore, the defenses adapted from the visual to audio domain demonstrate limited effectiveness. To fill this gap, we propose \textit{\textbf{G}radient Norm-based \textbf{F}ine-\textbf{T}uning (\nameFramework)}, a novel defense strategy against the attacks in the audio domain, based on the observation from the corresponding backdoored models. Specifically, we first empirically find that the backdoored neurons exhibit greater gradient values compared to other neurons, while clean neurons stay the lowest. 
On this basis, we fine-tune the backdoored model by incorporating the gradient norm regularization, aiming to weaken and reduce the backdoored neurons. We further approximate the loss computation for lower implementation costs. 
Extensive experiments on two speech recognition datasets across five models demonstrate the superior performance of our proposed method. To the best of our knowledge, this work is the first specialized and effective defense against backdoor attacks in the audio domain. 
% Based on this observation, we introduce Gradient norm-based Fine-tuning (\nameFramework), a novel model-level defense strategy against audio backdoor attacks. We fine-tune the backdoored model with a small clean dataset, incorporating gradient norms into the loss function to constrain the influence of backdoor-related neurons. To manage computational complexity, we use an efficient approximation scheme to calculate gradients of the loss function manually. Extensive experiments on two datasets of varying scales demonstrate that \nameFramework significantly outperforms the current state-of-the-art method, Fine-pruning (FP), in mitigating the effects of backdoor attacks. To our knowledge, our work establishes \nameFramework as the first specialized and effective defense against backdoor attacks in audio domain. The code were were made public at .
\end{abstract}

\begin{IEEEkeywords}
AI Security, Backdoor Defense, Automatic Speech Recognition, Fine-Tuning.
\end{IEEEkeywords}

\section{Introduction}
\label{Introduction}

In recent years, deep neural networks (DNNs) have seen extensive application across a wide range of fields, including face recognition \cite{taigman2014deepface, parmar2014face, ibrahim2011study}, autonomous driving \cite{caesar2020nuscenes,yurtsever2020survey} in the visual domain, and automatic speech recognition \cite{warden2018speech, malik2021automatic} in the audio domain. However, with advancements in technology, backdoor attacks \cite{gu2019badnets} have emerged as a severe security concern, threatening the safety of DNNs. In backdoor attacks, attackers inject a specific \textit{trigger} pattern into a portion of training dataset to poison the data. Models trained on such poisoned datasets, known as backdoored models, behave normally when presented with clean data. Conversely, they maliciously misclassify data that contains the trigger pattern to a predefined target label, which is termed as backdoored effect. To avoid this effect, solutions on either the poisoned-input detection (data-level) or the backdoored-model repairing (model-level) are necessary~\cite{yan2023backdoor}. \fin
% This study focuses on backdoor defenses in the post-training phase, aiming to mitigate the negative effects using a small portion of clean training data while preserving the model performance on clean data.

% Up to now, numerous studies have been dedicated to developing defenses against backdoor attacks, which have achieved significant results~\cite{liu2018fine, wu2021adversarial, li2021anti, huang2022backdoor}. 
% Numerous studies have thus far been devoted to the development of defenses against backdoor attacks, achieving considerable success~\cite{liu2018fine, wu2021adversarial, li2021anti, huang2022backdoor}.
Up to now, numerous studies have been dedicated to developing defenses against backdoor attacks, which have achieved significant results~\cite{liu2018fine, wu2021adversarial, li2021anti, huang2022backdoor, li2023reconstructive, wu2023defenses}. However, these defense methods are mostly designed for the visual domain, and no specialized defense is proposed against the backdoor attacks in the audio domain. Due to the different characteristics between the two domains, \eg, audio signals own larger information density as spectrum format compared to the RGB images, the existing defense methods adapted from the visual domain~\cite{liu2018fine, gao2019strip, ma2022beatrix} demonstrate limited performance against the audio backdoor attacks.
% in the audio domain are mostly adaptations of techniques from the visual domain. 
% Furthermore, most effective defense methods \cite{gao2019strip, ma2022beatrix} are operated at the data level, where the goal is to identify and remove poisoned data. At the model level, only Fine-pruning (FP) \cite{liu2018fine} has been proven to be effective against audio backdoor attacks, but our work reveals that the defense effects of FP against existing SOTA speech backdoor attacks is very limited. 
As illustrated in Table~\ref{table_result-SCD10} and Table~\ref{table_result-SCD30}, the model-level defense adapted from the visual domain, Fine-Pruning (FP)~\cite{liu2018fine}, fails completely in the audio domain. Therefore, in this work, we aim to design the first defense method specifically targeted at audio backdoor attacks.

\begin{figure}[ht]
\setlength{\abovecaptionskip}{3pt}
\begin{minipage}[b]{1.0\linewidth}
\centering
% \setlength{\belowcaptionskip}{-5mm}
\centerline{\includegraphics[width=8cm]{figure/pipeline.pdf}}
\end{minipage}
\caption{Overview of our proposed method (\nameFramework). }
\label{fig_pipeline}
\end{figure}

To investigate the characteristics of the backdoored models in the audio domain, \ie, \textit{audio-backdoored models}, we split its neurons into different types as in \cite{li2024magnitude} and further observe their learning behaviors. Specifically, the neurons are categorized into clean neurons, backdoored neurons, hybrid neurons, and redundant neurons according to their loss changes on both clean and backdoor tasks\footnote{Clean task represents the normal classification task on clean samples. Similarly, backdoor task indicates the task on poisoned samples. }. Note that backdoored neurons and hybrid neurons are the primary contributors to the backdoor task, and our goal is to weaken their functionality. The gradients with clean inputs on different neuron types are shown in Fig.~\ref{fig_grad}, we can observe that for most clean inputs, backdoored neurons and hybrid neurons exhibit larger gradient values than clean neurons, which solely contribute to clean task. 
% One backdoor defense work in the visual domain \cite{li2024magnitude} proposed a criterion to classify neurons in backdoored models based on loss changes, categorizing them into clean neurons, backdoor neurons, hybrid neurons and redundant neurons. Different types of neurons contribute differently to various tasks, with backdoor neurons and hybrid neurons being the primary contributors to the backdoor task. As shown in Fig. \ref{fig_grad}, we observed that backdoored models from audio domain, \ie, \textit{audio backdoored models}, contain a large number of backdoor neurons and hybrid neurons, and they exhibit greater gradients for clean data inputs compared to clean neurons that solely contribute to clean data tasks.
     
Based on this observation, we propose \textbf{G}radient \textbf{N}orm-based \textbf{F}ine-\textbf{T}uning (\textbf{\nameFramework}), where a gradient norm regularization term is added to the original loss function. By doing so, 
the learning process attempts to suppress the high-gradient backdoored neurons and hybrid neurons, resulting in a repaired clean model after fine-tuning. 
% We fine-tune the backdoored model with a clean dataset and add the gradient norm produced by model parameters to the original loss function, constraining the gradients and reducing the number of hybrid neurons and backdoor neurons in the model. 
Considering computational efficiency, we adopt the approximation scheme introduced in \cite{zhao2022penalizing}. Extensive experiments demonstrate that our method significantly outperforms FP in terms of defense effectiveness.

In summary, the main contributions of this work are threefold: \textbf{1) }We observe that the backdoored neurons in the audio-backdoored models exhibit greater gradient values than others. \textbf{2) }We propose a gradient-regularized fine-tuning method to effectively mitigate backdoored effect, marking the first specialized defense technique for backdoor attacks in the audio domain. \textbf{3) }Extensive experiments across two datasets, five models, and seven attacks, show that our proposed method consistently achieves state-of-the-art performance. 
% The ablation studies further verify the effectiveness of the proposed method.
% which is the first specialized defense method targeting backdoor attacks in the audio domain.

\begin{figure}[t]
\setlength{\abovecaptionskip}{3pt}
\begin{minipage}[b]{1.0\linewidth}
 \centering
\centerline{\includegraphics[width=8.8cm]{figure/grad.pdf}}
\end{minipage}
\caption{Illustration of gradients of different neurons over 50 clean samples. We used Audio BadNets~\cite{gu2019badnets} and JingleBack \cite{koffas2023going} on ResNet \cite{he2016deep} for illustrations. For most clean inputs, backdoored neurons and hybrid neurons exhibit larger gradients, while clean neurons show smaller gradients. }
\label{fig_grad}
\vspace{-1.5em}
\end{figure}

\section{Related work}

\noindent \textbf{Backdoor Attacks. }In backdoor attacks \cite{gu2019badnets}, an attacker injects a specific pattern, known as \textit{trigger}, into a portion of the training data and assigns these samples a target label. The resulting backdoored model performs normally on clean data but misclassifies inputs with the trigger to the target label. Most backdoor attack techniques \cite{gu2019badnets, chen2017targeted, nguyen2020input, nguyen2021wanet, wang2022bppattack, wu2023attacks} are designed for the visual domain, among which classic examples include BadNets \cite{gu2019badnets} and Blended \cite{chen2017targeted}. In the audio domain, Ultrasonic attack \cite{koffas2022can} is a representative method for automatic speech recognition tasks, where the attacker uses an ultrasonic signal as the backdoor trigger. To enable attacks in a physical scenario, naturally occurring sounds are chosen as triggers in DABA \cite{liu2022opportunistic}. Various audio-specific methods \cite{koffas2023going, cai2024towards} have also been devised to increase the stealthiness of attacks. Recently, a stealthy attack FlowMur \cite{lan2024flowmur} was introduced, where it trains a model to generate triggers while ensuring consistency between the target label and ground truth.

\noindent \textbf{Backdoor Defenses. }According to \cite{yan2023backdoor}, backdoor defenses can be categorized into data-level and model-level approaches. Data-level defenses aim to identify and remove poisoned data from the dataset, while model-level defenses attempt to mitigate backdoored effect in a well-trained backdoored model using a small amount of clean data. In the audio domain, existing backdoor defenses are all adaptations from the visual domain and are primarily data-level \cite{gao2019strip, ma2022beatrix}. FP \cite{liu2018fine} is the only adapted model-level defense for audio-backdoored models, which prunes neurons with low activation on clean data and then fine-tunes the pruned model. However, FP fails to effectively defend against most audio backdoor attacks. In this work, we address this issue by proposing a gradient-regularized fine-tuning technique from the model-level perspective, which is the first specialized defense for the audio-backdoored models. 

% Some audio processing techniques for handling noise signals, such as Noise Suppression \cite{hu2020dccrn}, and DeepFake \cite{hamza2022deepfake}, also have the potential for use as data-level backdoor defenses \cite{yan2023backdoor}.

% , based on the observation that neurons exhibit distinct activation levels for clean versus poisoned data


\section{Proposed Method}
\subsection{Problem Formulation}  % 总分结构,前面需要先总一下?
\noindent \textbf{Threat Models. }We assume that the attackers have full access to the training set, and they poison it by injecting a trigger into a small amount of randomly selected samples, indicated by the \textit{poisoning ratio}. The attackers aim to train the model with the poisoned training set so that it misclassifies poisoned data as the target label $y_t$, while functioning normally on clean data. We denote the model as $F$ with $L$ layers, where $f^{(i)}$ parameterized as $\boldsymbol{\theta}^{(i)}$ denotes $i$-th layer of the model. Considering the convolutional layer, the weights of neurons in the $i$-th layer can be denoted as $\{\boldsymbol{\theta}^{(i, j)} \in \mathbb{R}^{c_{i-1} \times h \times w}\}_{1\leq j\leq c_i}$, where $c_i$, $h$ and $w$ represent the number of neurons in $f^{(i)}$, the height and width of the convolutional kernel, respectively.
% The goal of the attackers is to train the model using the poisoned training set so that the model misclassifies the poisoned data to a target label $y_t$ while performing normally on clean data.

\noindent \textbf{Defense Setting. }The defender aims to eliminate the backdoored effect while preserving the model performance on clean data. Following the previous model-level defense settings \cite{liu2018fine}, we assume that only 5\% of clean data is accessible to the defender for conducting defense, denoted as $\mathcal{D}_c$. 

\subsection{Observations on Audio-Backdoored Models}
\label{Observations}
\noindent \textbf{Classification of Neurons in Backdoored Models. }In line with the neuron types defined in \cite{li2024magnitude}, we categorize the neurons in backdoored models based on pruning and loss change, where pruning \textit{j}-th neuron in the \textit{i}-th layer of the model means setting $\boldsymbol{\theta}^{(i, j)}$ to 0. The loss change of a neuron is defined as the difference between the loss values after and before pruning for the same inputs. To be specific, \textit{Clean Loss Change} (CLC) and \textit{Backdoor Loss Change} (BLC) can be formulated as: 
% \begin{equation}
% \begin{aligned}
% \text{CLC}(\boldsymbol{\theta}, i, j) &= \mathbb{E}_{(\boldsymbol{x}, y) \in \mathcal{D}_c}[ \mathcal{L}(F(\boldsymbol{x}; \boldsymbol{\theta} \mid \boldsymbol{\theta}^{(i, j)}=0), y) \\
% &- \mathcal{L}(F(\boldsymbol{x}; \boldsymbol{\theta}), y)],
% \end{aligned}
% \end{equation}
% \begin{equation}
% \begin{aligned}
% \text{BLC}(\boldsymbol{\theta}, i, j) &= \mathbb{E}_{(\boldsymbol{x}, *) \in \mathcal{D}_c, y_t}[ \mathcal{L}(F(\delta(\boldsymbol{x}); \boldsymbol{\theta} \mid \boldsymbol{\theta}^{(i, j)}=0), y_t) \\
%  &- \mathcal{L}(F(\delta(\boldsymbol{x}); \boldsymbol{\theta}), y_t)],
% \end{aligned}
% \end{equation}
\begin{align}
  \text{CLC}(\boldsymbol{\theta}, i, j) &= \mathbb{E}_{(\boldsymbol{x}, y) \in \mathcal{D}_c} \bigl[ \mathcal{L}(F(\boldsymbol{x}; \boldsymbol{\theta} \mid \boldsymbol{\theta}^{(i, j)}=0), y) \notag \\
  &\quad - \mathcal{L}(F(\boldsymbol{x}; \boldsymbol{\theta}), y) \bigr]  \\
  \text{BLC}(\boldsymbol{\theta}, i, j) &= \mathbb{E}_{(\boldsymbol{x}, *) \in \mathcal{D}_c, y_t} \bigl[ \mathcal{L}(F(\delta(\boldsymbol{x}); \boldsymbol{\theta} \mid \boldsymbol{\theta}^{(i, j)}=0), y_t) \notag \\
  &\quad - \mathcal{L}(F(\delta(\boldsymbol{x}); \boldsymbol{\theta}), y_t) \bigr] 
\end{align} 
where $\mathcal{D}_c$ is the given clean data for defense; $\delta(\cdot)$  is the poisoning function of the attack method; $y_t$ is a predefined target label; and $\mathcal{L}(\cdot)$ is the \textit{cross-entropy loss}. Note that a larger value of CLC (or BLC) represents a larger contribution of the current neuron to the clean task (or backdoor task). Therefore, we can adopt it to classify the neurons into different types.

\begin{figure}[t]
\setlength{\abovecaptionskip}{3pt}
\begin{minipage}[b]{1.0\linewidth}
 \centering
 \centerline{\includegraphics[width=5cm]{figure/example.pdf}}
\end{minipage}
\caption{A scatter plot showing the BLC and CLC values for neurons in the last two convolutional layers of an audio-backdoored model attacked by Audio BadNets \cite{gu2019badnets}.
% with Quadrants I to IV corresponding to the \textit{Hybrid Zone} (H-zone), \textit{Backdoor Zone} (B-zone), \textit{Redundant Zone} (R-zone), and \textit{Clean Zone} (C-zone), respectively.
\textbf{C-zone}: Clean Zone; \textbf{B-zone}: Backdoor Zone; \textbf{H-zone}: Hybrid Zone; \textbf{R-zone}: Redundant Zone.
}
\label{fig_example}
\vspace{-1.5em}
\end{figure}

Using this definition, we can obtain the CLCs and BLCs of all neurons within the audio-backdoored model, as illustrated in Fig. \ref{fig_example}. We divide the plot into four zones based on the zero values of CLC and BLC: The \textit{Clean Zone} (C-zone) contains neurons with positive CLCs and negative BLCs, indicating their contribution to the clean task while potentially suppressing the backdoor task. The \textit{Backdoor Zone} (B-zone) is characterized by neurons with positive BLCs and negative CLCs, suggesting a specific contribution to the backdoor task, potentially at the expense of the clean task. In the \textit{Hybrid Zone} (H-zone), neurons have both positive CLCs and BLCs, meaning they could contribute to both tasks. Finally, the \textit{Redundant Zone} (R-zone) contains neurons that do not contribute to either task. Based on these four zones, we can further classify neurons into \textbf{clean neurons}, \textbf{backdoored neurons}, \textbf{hybrid neurons}, and \textbf{redundant neurons}, respectively. Intuitively, our goal is to suppress backdoored neurons and hybrid neurons for backdoor mitigation. 

% We divide this plot into four regions according to the zero value of both CLC and BLC, each region corresponding to one of the four quadrants: Quadrant IV represents the \textit{Clean Zone} (C-zone), where neurons exhibit positive CLCs and negative BLCs, indicating that these neurons contribute to clean task while potentially suppress the backdoor task; Quadrant II corresponds to the \textit{Backdoor Zone} (B-zone), characterized by neurons with positive BLCs and negative CLCs, suggesting that these neurons specifically contribute to the backdoor task and could negatively impact the clean task; Quadrant I is the \textit{Hybrid Zone} (H-zone), where the neurons have both positive CLC and positive BLC values, meaning they could contribute to both the clean task and the backdoor task; Finally, Quadrant IV is designated as the \textit{Redundant Zone} (R-zone), as the neurons in this quadrant have no contribution to either clean or backdoor task.

% \vspace{-0.5cm}

% Most of the backdoor defense works \cite{liu2018fine, li2024magnitude} in the visual domain maintain that the neurons responsible for backdoor tasks and normal tasks rarely overlap. However, our work shows that, in the audio domain, for almost all existing attack methods, a substantial portion of the neurons are involved in both types of tasks. One possible explanation for this difference is that data transformed into the  frequency domain using DCT \cite{strang1999discrete} has a higher information density \cite{gupta2013feature}, and even small modifications to frequency-domain data can result in significant changes to the features of original datas, unlike in the visual domain where triggers introduce only minor perturbations to the original data. Consequently, backdoor neurons in the audio domain are more tightly coupled with clean neurons.

\noindent \textbf{Suggestion Given by the Observation. }As illustrated in Section \ref{Introduction}, in the audio-backdoored models, \textbf{backdoored neurons and hybrid neurons tend to exhibit larger gradients on most clean inputs, while clean neurons stay the smallest.}
% Assuming gradient is enabled, when $D_c$ is input into the backdoored model, the gradient of each neuron $g^{(i, j)} \in \mathbb{R}^{c_{i-1} \times h \times w}$ in layer $f^{(i)}$ can be extracted and summed across their input channels. 
% As depicted in Fig. \ref{fig_grad}, \wl for most inputs, \fin backdoored neurons \wl and hybrid neurons \fin generate larger gradient values on each input channel, while clean neurons produce smaller ones, with hybrid neurons falling in between. 
Therefore, it suggests penalizing the high gradient norm during fine-tuning to repair these two kinds of neurons, where the clean neurons are less modified due to their small-gradient characteristic. 

% From another perspective, neurons in the B-zone and those near the B-zone within the H-zone show larger $L_2$ gradient norms.

% 这里需要添加一段解释,为什么梯度更大


% \begin{figure*}
%     \centering
%     \includegraphics[width=0.7\textwidth]{ICASSP2024_Paper_Templates/figure/grad.pdf}
%     \caption{We used the attack effects of BadNets and JingleBack \cite{koffas2023going} on ResNet \cite{he2016deep} as illustrations. (a) and (b) show the gradient values produced by different types of neurons in the last convolutional layer across their input channels; (c) and (d) present the distribution of gradient norms on the BLC-CLC plot of neurons in the last two convolutional layers, with the color of the points indicating the magnitude of the $L_2$ gradient norm.}
%     \label{fig_grad}
% \end{figure*}

\subsection{Gradient Norm-based Fine-Tuning}
% \noindent \textbf{Loss for Fine-tuning. }
Based on the observations, we propose \textit{Gradient Norm-based Fine-Tuning} to penalize the high gradients from backdoor neurons and hybrid neurons. An overview of the method is shown in Fig. \ref{fig_pipeline}. An $L_2$ norm of the gradients is added as a regularization term to the fine-tuning loss function, as shown below:
\begin{equation}
\label{equ:gradient_loss}
\mathcal{L}(\boldsymbol{\theta})=\mathcal{L}_{c e}(\boldsymbol{\theta})+\lambda \cdot\left\|\nabla_{\boldsymbol{\theta}} \mathcal{L}_{c e}(\boldsymbol{\theta})\right\|_2,
\end{equation}
where $\mathcal{L}_{c e}(\cdot)$ is the original cross-entropy loss,  $\left\|\nabla_{\boldsymbol{\theta}} \mathcal{L}_{c e}(\boldsymbol{\theta})\right\|_2$ corresponds to the $L_2$ norm of the gradients of the model, and $\lambda$ is a trade-off coefficient to control the strength of penalization. 
During the fine-tuning process, we aim to minimize this loss function using the available clean set $D_c$.  The object function is formulated as:
\begin{equation}
\min _{\boldsymbol{\theta}} \mathbb{E}_{(\boldsymbol{x}, y) \in \mathcal{D}_c}[\mathcal{L}(F(\boldsymbol{x} ; \boldsymbol{\theta}), y)].
\end{equation}

However, direct optimization on equation~\eqref{equ:gradient_loss} involves calculating a Hessian matrix with $O(n^2)$ time and space complexities, which is infeasible. 
% adding a constraint term to the loss function and propagating backward to compute gradients would lead to a low computational efficiency. 
Inspired by the approximation scheme in \cite{zhao2022penalizing}, we choose to simplify it similarly using Taylor expansion and additional optimization steps. 
% adopt the approximation method from , which allows us to directly obtain the gradients of the loss function. 
It can be formulated as:  %\\&\ \ \ \
\begin{equation}
\label{equ:appro}
\scalebox{0.9}{$
\begin{aligned}
\nabla_{\boldsymbol{\theta}} \mathcal{L}(\boldsymbol{\theta})
& =\nabla_{\boldsymbol{\theta}} \mathcal{L}_{c e}(\boldsymbol{\theta}) + \nabla_{\boldsymbol{\theta}}( \lambda \cdot\left\|\nabla_{\boldsymbol{\theta}} \mathcal{L}_{c e}(\boldsymbol{\theta})\right\|_2 )\\
& =\nabla_{\boldsymbol{\theta}} \mathcal{L}_{c e}(\boldsymbol{\theta})+\lambda \cdot \nabla_{\boldsymbol{\theta}}^2 \mathcal{L}_{c e}(\boldsymbol{\theta}) \frac{\nabla_{\boldsymbol{\theta}} \mathcal{L}_{c e}(\boldsymbol{\theta})}{\left\|\nabla_{\boldsymbol{\theta}} \mathcal{L}_{c e}(\boldsymbol{\theta})\right\|_2} \\
& \approx \nabla_{\boldsymbol{\theta}} \mathcal{L}_{c e}(\boldsymbol{\theta})+\frac{\lambda}{r} \cdot(\nabla_{\boldsymbol{\theta}} \mathcal{L}_{c e}(\boldsymbol{\theta}+r \frac{\nabla_{\boldsymbol{\theta}} \mathcal{L}_{c e}(\boldsymbol{\theta})}{\left\|\nabla_{\boldsymbol{\theta}} \mathcal{L}_{c e}(\boldsymbol{\theta})\right\|_2})-\nabla_{\boldsymbol{\theta}} \mathcal{L}_{c e}(\boldsymbol{\theta})) \\
& =(1-\alpha) \nabla_{\boldsymbol{\theta}} \mathcal{L}_{c e}(\boldsymbol{\theta})+\alpha \nabla_{\boldsymbol{\theta}} \mathcal{L}_{c e}(\boldsymbol{\theta}+r \frac{\nabla_{\boldsymbol{\theta}} \mathcal{L}_{c e}(\boldsymbol{\theta})}{\left\|\nabla_{\boldsymbol{\theta}} \mathcal{L}_{c e}(\boldsymbol{\theta})\right\|_2}),
\end{aligned}$}
\end{equation} 
where $r$ is for appropriating the Hessian multiplication, and $\alpha=\frac{\lambda}{r}$ is used for trade-off.
In the practical defense process, we conduct an additional optimization step to approximate the second term in equation~\eqref{equ:appro}, aiming to avoid the Hessian computation:
\begin{equation}
    \label{equ:loss_appro_second}  
    \nabla_{\boldsymbol{\theta}} \mathcal{L}_{c e}(\boldsymbol{\theta}+r \frac{\nabla_{\boldsymbol{\theta}} \mathcal{L}_{c e}(\boldsymbol{\theta})}{\left\|\nabla_{\boldsymbol{\theta}} \mathcal{L}_{c e}(\boldsymbol{\theta})\right\|_2}) \approx \nabla_{ \boldsymbol{\theta}} \mathcal{L}_{ce}( \boldsymbol{\theta})|_{ \boldsymbol{\theta}= \boldsymbol{\theta}+r \frac{\nabla_{ \boldsymbol{\theta}} \mathcal{L}_{ce}( \boldsymbol{\theta})}{{ \|\nabla_{ \boldsymbol{\theta}}} \mathcal{L}_{ce}( \boldsymbol{\theta}) \|_2}}.
\end{equation}

% after getting the normal gradients and the model parameters, we calculate their gradients manually and apply them to the model parameters. Then, the model parameters are updated based on the applied gradients. 

The details of the algorithm process are illustrated in Algorithm~\ref{alg:GFT}.
For each iteration (line 1$\sim$9 of the Algorithm), we first obtain a mini-batch of data $\boldsymbol{B}_c$ (line 2) and the current parameters $\boldsymbol{\theta}^t$ (line 3). Based on them, we calculate the first term in equation~\eqref{equ:appro} as $\boldsymbol{g}_1$ (line 4). Then, we temporally conduct an additional optimization step as stated in equation~\eqref{equ:loss_appro_second} to approximate the second term in equation~\eqref{equ:appro}, as $\boldsymbol{g}_2$ (line 5$\sim$6). By combining the two terms with $\alpha$, we can calculate the final gradients to permanently update $\boldsymbol{\theta}^t$ (line 7$\sim$8). After $T$ iterations, we can obtain a repaired model $F_c$, which is validated effective towards audio-backdoored model in Section~\ref{sec:experiment}. 

\begin{algorithm}[h]   
  \caption{Gradient Norm-based Fine-Tuning}  
  \label{alg:GFT}  
  \begin{algorithmic}[1]
    \Require  
        Clean dataset $\mathcal{D}_c$; backdoored model $F$; the number of iterations $T$; hyper-parameters $r$ and $\alpha$;
     \Ensure The clean model $F_c$;
     \For{$t = 1$ to $T$}
     \State Get a mini-batch $\boldsymbol{B}_c$ from $\mathcal{D}_c$;
     \State Extract the parameters $\boldsymbol{\theta}^t$ from $F$;
     \State Input $\boldsymbol{B}_c$ into $F$, calculate gradients $\boldsymbol{g}_1 \leftarrow \nabla_{\boldsymbol{\theta}^t} \mathcal{L}_{c e}(\boldsymbol{\theta}^t)$;
     \State Copy the model $F$ as $F^{\prime}$, and define its parameters as $\boldsymbol{\theta}^{\prime} \leftarrow \boldsymbol{\theta}^t+r \frac{\boldsymbol{g}_1}{\left\|\boldsymbol{g}_1\right\|_2}$;
     \State Input $\boldsymbol{B}_c$ into $F^{\prime}$, calculate gradients $\boldsymbol{g}_2 \leftarrow \nabla_{\boldsymbol{\theta}^{\prime}} \mathcal{L}_{c e}(\boldsymbol{\theta}^{\prime})$;
     \State Calculate the final gradient $\boldsymbol{g}\leftarrow(1-\alpha) \boldsymbol{g}_1+\alpha \boldsymbol{g}_2$;
     \State Update $\boldsymbol{\theta}^t$ using $\boldsymbol{g}$;   
  	\EndFor
   
        \State \textbf{Return} $F_c$ with parameters $\boldsymbol{\theta}^T$.
  \end{algorithmic} 
\end{algorithm}

\section{Experiments}
\label{sec:experiment}
\subsection{Experimental Setups }
\noindent \textbf{Datasets and Models. }We use Google's Speech Commands Dataset (SCD) \cite{warden2018speech}, a commonly used dataset for speech recognition tasks. We employ two versions: the first version contains 10 classes that were also used in \cite{koffas2022can} (SCD-10), and the second version includes the full 30 classes (SCD-30). We choose ResNet \cite{he2016deep}, LSTM \cite{hochreiter1997long}, Small CNN \cite{wu2017introduction, samizade2020adversarial}, KWT \cite{berg2021keyword} and EAT \cite{gazneli2022end}, which are commonly used as speech recognition models, as our experimental models.

\noindent \textbf{Attacker Settings. }In our experiments, the poisoning ratio for the attacks is set to 10\%, and the target label is set to ``up''. We employ seven attack methods: Audio BadNets~\cite{gu2019badnets},  Ultrasonic \cite{koffas2022can}, JingleBack \cite{koffas2023going}, DABA \cite{liu2022opportunistic}, FlowMur \cite{lan2024flowmur}, PBSM and VSVC \cite{cai2024towards}. Among these, we extend the representative attack, BadNets \cite{gu2019badnets} from the visual domain, to Audio BadNets in the audio domain, where a white block is added at a fixed position in the MFCC spectrogram \cite{gupta2013feature} of the audio signal to be poisoned. 

\noindent \textbf{Defender Settings. }Since our \nameFramework is designed as a model-level defense, we compare it with the only known model-level defense method adapted to the audio domain, FP \cite{liu2018fine}. We follow a similar setting with 5\% (known as \textit{clean data ratio}) of the clean training data for defense purposes. Since the hyperparameters $r$ and $\alpha$ are more related to the approximation ability, and well-discussed in \cite{zhao2022penalizing}, we 
follow the default setup to 0.05 and 0.7, respectively.

\noindent \textbf{Evaluation Metrics. }We use two metrics to evaluate the defense methods: Clean Accuracy (\textbf{CA}) and Attack Success Rate (\textbf{ASR}). CA represents the accuracy of the model on clean data, while ASR indicates the proportion of poisoned data that the model predicts as the target label. The \textbf{boldfaced} numbers represent the best performance among the same metric. 
% \nj and the underlined values indicate the second-best performance. \fin

\subsection{Main Results}
\noindent\textbf{Results On SCD-10. }Table \ref{table_result-SCD10} presents the  performance comparisons on SCD-10 dataset using ResNet, LSTM, Small CNN, KWT and EAT. The results demonstrate that our method, \nameFramework, exhibits significant advantages over FP, effectively reducing ASR while maintaining a high CA in most cases. For ResNet, \nameFramework significantly lowers the average ASR from 91.72\% to 9.73\%, while the average CA only drops slightly from 94.54\% to 90.40\%. 
% However, in more stealthy attack methods such as JingleBack and FlowMur, the effects of \nameFramework are somewhat weaker, resulting in post-defense ASRs of 15.99\% and 20.10\%, respectively. 
Although FP performs better ASR in JingleBack and DABA, the sacrifices on CA are unacceptable at 33.69\% and 21.45\%, respectively. 
Similarly, for LSTM, Small CNN, KWT and EAT, \nameFramework outperforms FP for nearly all attacks. In contrast, FP fails to achieve effective defense against these attack methods. 
% For example, in DABA attack based on ResNet, although FP reduces ASR to 0.5\%,  CA also drops dramatically from 99.91\% to 21.45\%. And in many other cases, FP is unable to significantly impact ASR, instead causing a decline in CA.

\noindent \textbf{Results on SCD-30. }Table \ref{table_result-SCD30} presents the defense performance on SCD-30 dataset using ResNet and LSTM. Similar to its performance on SCD-10 dataset, \nameFramework significantly outperforms FP on SCD-30 dataset as well. For ResNet, \nameFramework demonstrates effective defense against most attacks, although less effective on ASR towards the more stealthy attacks, JingleBack and FlowMur. 
% failing to sufficiently reduce ASR, which is similar to the case using SCD-10. 
Notably, in DABA and FlowMur attacks, \nameFramework can increase the model’s CA  compared to No Defense, improving it by 0.07\% and 8.68\%, respectively, and the average CA after defense also increases by 0.74\%. For LSTM, \nameFramework can successfully defend against all five attacks with ASR lower than 10\%.
% effectively reduce ASRs of all five attacks, but this comes at the cost of a larger decrease in CA. 
Similar to the performance on SCD-10 dataset, FP suffers from high ASRs or significant reduction in CAs. 

Overall, we can see that \nameFramework is an effective defense method against all attack methods under different experimental setups.

\begin{table}[t]
% \fontsize{8pt}{10pt}\selectfont
% \setlength{\tabcolsep}{4pt} % 减小列间距
\captionof{table}{Main experimental results on SCD-10 dataset (\%).}
\label{table_result-SCD10}
\vspace{+2pt}
\resizebox{\linewidth}{!}{
\begin{tabular}{c|c|cc|cc|cc}
\hline
\multirow{2}{*}{Models} & \multirow{2}{*}{Backdoor attacks} & \multicolumn{2}{c|}{No Defense} & \multicolumn{2}{c|}{FP~\cite{liu2018fine}}          & \multicolumn{2}{c}{\nameFramework(Ours)}  \\ \cline{3-8} 
                        &                                   & ASR $\downarrow$           & CA  $\uparrow$        & ASR $\downarrow$           & CA $\uparrow$           & ASR $\downarrow$         & CA $\uparrow$          \\ \hline
\multirow{6}{*}{ResNet} & Audio BadNets \cite{gu2019badnets}                    & 99.27          & 92.44         & 25.20          & 44.64          & \textbf{2.92}  & \textbf{91.06} \\
                        & Ultrasonic \cite{koffas2022can}                        & 100.00         & 93.68         & 7.81           & 28.71          & \textbf{0.21}  & \textbf{90.50} \\
                        & JingleBack \cite{koffas2023going}                       & 97.42          & 92.84         & \textbf{5.44}  & 33.69          & 15.99 & \textbf{89.98} \\
                        & DABA \cite{liu2022opportunistic}                             & 99.32          & 99.91         & \textbf{0.50}  & 21.45          & 9.43  & \textbf{89.15} \\
                        & FlowMur \cite{lan2024flowmur}                          & 62.59          & 93.85         & 86.23          & 29.63          & \textbf{20.10} & \textbf{91.30} \\ \cline{2-8}
                        & Average                           & 91.72          & 94.54         & 25.04          & 31.62 & \textbf{9.73}  & \textbf{90.40} \\ \hline
\multirow{6}{*}{LSTM}   & Audio BadNets \cite{gu2019badnets}                   & 100.00         & 94.37         & 82.06          & 81.40          & \textbf{3.96}  & \textbf{85.17} \\
                        & Ultrasonic \cite{koffas2022can}                        & 100.00         & 94.37         & 97.21          & 51.42          & \textbf{1.85}  & \textbf{89.94} \\
                        & JingleBack \cite{koffas2023going}                       & 99.40          & 93.50         & 86.07          & 75.28          & \textbf{6.25}  & \textbf{85.93} \\
                        & DABA \cite{liu2022opportunistic}                             & 99.26          & 99.27         & 98.21          & 83.97          & \textbf{9.08}  & \textbf{88.04} \\
                        & FlowMur \cite{lan2024flowmur}                          & 75.60          & 92.38         & 34.30          & 62.20          & \textbf{33.10} & \textbf{86.46} \\ \cline{2-8}
                        & Average                           & 94.85          & 94.78         & 79.57          & 70.85          & \textbf{10.85}          & \textbf{87.11}          \\ \hline
\multirow{4}{*}{\begin{tabular}[c]{@{}c@{}}Small\\ CNN\end{tabular}}    & Audio BadNets \cite{gu2019badnets}                    & 100.00         & 90.12         & 69.41          & 66.71          & \textbf{25.02} & \textbf{82.03} \\
                        & Ultrasonic \cite{koffas2022can}                       & 99.97          & 91.23         & 86.33          & 69.36          & \textbf{23.17} & \textbf{78.61} \\
                        & JingleBack \cite{koffas2023going}                       & 98.88          & 90.38         & \textbf{39.36} & 53.65          & 48.32          & \textbf{81.02} \\ \cline{2-8}
                        & Average                           & 99.62          & 90.58         & 65.03          & 63.24          & \textbf{32.17}          & \textbf{80.55}          \\ \hline
\multirow{3}{*}{KWT}   & PBSM \cite{cai2024towards}                   & 92.20         & 91.47         & \textbf{17.17}          & 71.41         & 17.30  & \textbf{84.41} \\
                        & VSVC \cite{cai2024towards}                        & 99.83         & 91.16         & 23.18          & 71.60          & \textbf{15.90}  & \textbf{84.11} \\
                        \cline{2-8}
                        & Average                           & 96.02          & 91.32         & 20.18          & 71.51          & \textbf{16.60}          & \textbf{84.26}          \\ \hline
\multirow{3}{*}{EAT}   & PBSM \cite{cai2024towards}                   & 100.00        & 95.60         & \textbf{0.00}          & 10.09         & 2.92  & \textbf{94.94} \\
                        & VSVC \cite{cai2024towards}                        & 99.08         & 95.36         & \textbf{0.00}          & 9.97          & 2.79  & \textbf{95.33} \\
                        \cline{2-8}
                        & Average                           & 99.54          & 95.48         & \textbf{0.00}          & 10.03          & 2.86          & \textbf{95.14}          \\ \hline
\end{tabular}}
% \end{table}
% \begin{table}
% \fontsize{8pt}{10pt}\selectfont
% \setlength{\tabcolsep}{4pt} % 减小列间距
\captionof{table}{Main experimental results on SCD-30 dataset (\%).}
\label{table_result-SCD30}
\vspace{+2pt}
\resizebox{\linewidth}{!}{
\begin{tabular}{c|c|cc|cc|cc}
\hline
\multirow{2}{*}{Models} & \multirow{2}{*}{Backdoor attacks} & \multicolumn{2}{c|}{No Defense} & \multicolumn{2}{c|}{FP~\cite{liu2018fine}}          & \multicolumn{2}{c}{\nameFramework(Ours)}  \\ \cline{3-8} 
                        &                                   & ASR $\downarrow$           & CA  $\uparrow$          & ASR $\downarrow$           & CA  $\uparrow$            & ASR $\downarrow$            & CA $\uparrow$             \\ \hline
\multirow{6}{*}{ResNet} & Audio BadNets \cite{gu2019badnets}                  & 99.96          & 92.36         & 28.34          & 55.23          & \textbf{1.73}  & \textbf{89.54} \\
                        & Ultrasonic \cite{koffas2022can}                       & 100.00         & 91.37         & 29.68          & 55.78          & \textbf{0.12}  & \textbf{90.02} \\
                        & JingleBack \cite{koffas2023going}                       & 99.21          & 90.67         & 36.97 & 47.70          & \textbf{28.18} & \textbf{89.76} \\
                        & DABA \cite{liu2022opportunistic}                             & 99.63          & 80.09         & 71.63 & 48.97          & \textbf{4.51}  & \textbf{88.77} \\
                        & FlowMur \cite{lan2024flowmur}                          & 41.31          & 87.87         & 51.40          & 30.30          & \textbf{42.40} & \textbf{87.94} \\ \cline{2-8}
                        & Average                           & 88.02          & 88.47         & 43.60          & 47.60 & \textbf{15.39} & \textbf{89.21} \\ \hline
\multirow{6}{*}{LSTM}   & Audio BadNets \cite{gu2019badnets}                   & 100.00         & 94.04         & 83.23          & \textbf{82.59} & \textbf{1.19}  & 82.47 \\
                        & Ultrasonic \cite{koffas2022can}                       & 100.00         & 93.81         & 99.60          & 83.34          & \textbf{0.12}  & \textbf{90.02} \\
                        & JingleBack  \cite{koffas2023going}                      & 99.72          & 93.96         & 84.10          & 79.95          & \textbf{7.97}  & \textbf{84.75} \\
                        & DABA \cite{liu2022opportunistic}                             & 99.90          & 78.90         & 98.85          & 53.13          & \textbf{4.02}  & \textbf{82.74} \\
                        & FlowMur \cite{lan2024flowmur}                          & 46.93          & 91.05         & 4.18           & 66.53          & \textbf{2.72}  & \textbf{82.16} \\ \cline{2-8}
                        & Average                           & 89.31          & 90.35         & 73.99          & 73.11          & \textbf{3.20}           & \textbf{84.43}          \\ \hline
\end{tabular}}
\vspace{-1.5em}
\end{table}

\subsection{Ablation Studies}
To verify the effectiveness of gradient regularization, we compare it with the \textit{vanilla Fine-Tuning} (FT for short). As shown in Table \ref{table_ablation}, the defense performances of FT on the audio-backdoored models are limited: despite the high CAs, it fails to reduce ASRs. On the contrary, \nameFramework significantly reduces ASRs with a similar sacrifice on CAs. This further underscores the importance of gradient regularization. 
% \nameFramework effectively suppresses the effects of backdoor neurons and hybrid neurons through the introduction of gradient constraints, whereas the FT method may not have a significant impact on these two types of neurons.

% Here, we attempt to explain why FP cannot achieve good defense effects. As explained in Section \ref{Observations}, audio backdoored models contain a greater number of hybrid neurons compared to most backdoored models in visual domain, which do not necessarily show a dormant state when presented with clean data input as they need to serve both the backdoor task and the clean task. Consequently, during the first stage of FP, it can only detect and prune a small fraction of backdoor-related neurons. Furthermore, our ablation study proves that regular FT, which is the second stage of FP, has very limited effects on audio backdoored models.
\vspace{-1em}
\begin{table}[htb]
\centering
\captionof{table}{Comparison with FT on SCD-10 using ResNet (\%). }
\label{table_ablation}
\vspace{+2pt}
\begin{tabular}{c|cc|cc}
\hline
\multicolumn{1}{c|}{\multirow{2}{*}{Backdoor attacks}} & \multicolumn{2}{c|}{FT} & \multicolumn{2}{c}{\nameFramework(Ours)} \\ \cline{2-5} 
\multicolumn{1}{c|}{}                                  & ASR$\downarrow$        & CA$\uparrow$       & ASR$\downarrow$            & CA$\uparrow$           \\ \hline
Audio BadNets \cite{gu2019badnets}                                       & 99.38      & 91.32     & 2.92           & 91.06         \\
Ultrasonic \cite{koffas2022can}                                            & 100.00        & 92.68     & 0.20            & 90.50          \\
JingleBack \cite{koffas2023going}                                            & 72.19      & 90.71     & 15.99          & 89.98         \\ \hline
\end{tabular}
\end{table}
\vspace{-1em}

\subsection{Further Analysis }

\noindent\textbf{Impact of Clean Data Ratio. }We analyze the impact of different clean data ratios, \ie, the proportion of clean data used for defense, on the defense performance. As shown in Table \ref{table_size_impact}, a smaller clean data ratio is prone to bring a worse defense performance, especially for the sacrifice of CAs. 
% when the proportion of the clean dataset is small relative to the training set, ASR can already be reduced to a low level, but CA will suffer a significant loss. 
Once the proportion of clean data exceeds 10\%, CAs can maintain stably high at above 90\% for different attacks, and ASRs decrease to around 5\% or even lower. This indicates that the performance of \nameFramework can be further improved and stabilized if there were more clean data available for defense.
% This indicates that the performance of \nameFramework can be further improved and become more stable if the defender has access to more clean data. 

\begin{table}[t]
% \fontsize{8pt}{10pt}\selectfont
\captionof{table}{The impact of the clean data ratio on the defense performance using ResNet (\%).}
\label{table_size_impact}
\vspace{+2pt}
\resizebox{\linewidth}{!}{
\begin{tabular}{c|cc|cc|cc}
\hline
\multirow{2}{*}{Clean Data Ratio} & \multicolumn{2}{c|}{Audio BadNets \cite{gu2019badnets}} & \multicolumn{2}{c|}{Ultrasonic \cite{koffas2022can}} & \multicolumn{2}{c}{JingleBack \cite{koffas2023going}} \\ \cline{2-7} 
                              & ASR$\downarrow$             & CA$\uparrow$               & ASR$\downarrow$           & CA$\uparrow$             & ASR$\downarrow$            & CA$\uparrow$            \\ \hline
2\%                          & 10.00              & 85.20             & 5.86          & 85.15          & 11.66          & 85.51         \\
5\%                          & 2.92            & 91.06            & 0.21          & 90.50           & 15.99          & 89.98         \\
10\%                           & 3.54            & 93.46            & 1.04          & 92.87          & 5.26           & 92.87         \\
20\%                           & 3.31            & 95.17            & 2.47          & 94.70           & 3.64           & 94.75         \\
40\%                           & 2.88            & 95.89            & 0.76          & 95.64          & 2.66           & 95.64         \\ \hline
\end{tabular}}
\vspace{-1.5em}
\end{table}

\noindent \textbf{Influences on Backdoored and Hybrid Neurons. }To analyze the influences of \nameFramework on the backdoor and hybrid neurons, we observe the BLC-CLC distribution of neurons in the last two convolutional layers after defense. As shown in Fig. \ref{fig_ft}, \nameFramework can effectively reduce the number of neurons in the H-zone and B-zone compared to Fig.~\ref{fig_example}, while the number of neurons in the R-zone increases, indicating that some hybrid neurons and backdoored neurons have been moved to the R-zone after defense, which is considered redundant. The result proves that our method can mitigate the backdoor by reducing the number of backdoored and hybrid neurons.

\vspace{-0.5em}
\begin{figure}[h]
\setlength{\abovecaptionskip}{3pt}
\begin{minipage}[b]{1.0\linewidth}
 \centering
 \centerline{\includegraphics[width=5cm]{figure/ft.pdf}}
\end{minipage}
\caption{The BLC-CLC distribution of neurons after \nameFramework against Audio BadNets.}
\label{fig_ft}
\end{figure}

\noindent \textbf{T-SNE Visualization. }
% Fig.~\ref{fig_tsne} shows the t-SNE \cite{van2008visualizing} plots before and after the defense of \nameFramework. Before defense, the poisoned features exhibit a clear clustering (see the black cluster in the left of Fig.~\ref{fig_tsne}), indicating that the features of the trigger have been well-learned. After defense, the poisoned features become dispersed and distributed among other classes (see the black points in the right of Fig.~\ref{fig_tsne}). Meanwhile, the feature distributions of the ten classes of clean data remain clustering. This suggests that the model ``forgets'' about the features of the trigger but retains its performance on clean data.
Fig.\ref{fig_tsne} shows t-SNE \cite{van2008visualizing} plots before and after \nameFramework defense. Before defense, the poisoned features form a clear cluster (left of Fig.\ref{fig_tsne}), indicating the trigger features are well-learned. After defense, the poisoned features become dispersed and distributed among other classes (right of Fig.~\ref{fig_tsne}), while clean data features remain clustered. This suggests the model "forgets" the trigger features but retains performance on clean data.

\vspace{-0.5em}
\begin{figure}[h]
\setlength{\abovecaptionskip}{3pt}
\begin{minipage}[b]{1.0\linewidth}
 \centering
 \centerline{\includegraphics[width=8.8cm]{figure/tsne.pdf}}
\end{minipage}
\caption{The t-SNE plots before and after \nameFramework against BadNets using SCD-10. \textbf{Black} points indicate the poisoned features. }
\label{fig_tsne}
\end{figure}
\vspace{-0.5em}

\section{Conclusion}
% We propose Gradient Norm-based Fine-Tuning to mitigate the backdoored effects in audio-backdoored models. Our work explores neuron composition in audio-backdoored models, revealing that backdoored neurons and hybrid neurons tend to exhibit larger gradients for clean inputs. Our work emphasizes an important direction for backdoor defense targeting audio-backdoored models, which are different from the ones in the visual domain but were previously ignored. In the future, we plan to further explore the differences between the audio domain and visual domain to enhance the defense performance against stealthy attacks like the DABA and FlowMur.

We propose Gradient Norm-based Fine-Tuning to mitigate the backdoored effects in audio-backdoored models. Our study reveals that backdoored and hybrid neurons in these models show larger gradients for clean inputs. Our work highlights the need for specific backdoor defenses for audio-backdoored models, which differ from visual models but were previously overlooked. Our future work will focus on exploring the differences between audio and visual domains to enhance the defense performance against stealthy attacks like DABA and FlowMur.

\section{Acknowledgment}
This work was supported by the National Natural Science Foundation of China (No. 62471420 and 62101351), and Guangzhou-HKUST(GZ) Joint Funding Program (Grant No.2023A03J0008), Education Bureau of Guangzhou Municipality.

\clearpage
% References should be produced using the bibtex program from suitable
% BiBTeX files (here: strings, refs, manuals). The IEEEbib.bst bibliography
% style file from IEEE produces unsorted bibliography list.
% -------------------------------------------------------------------------
\bibliographystyle{IEEEbib}
% \bibliography{refs}
\documentclass[conference]{IEEEtran}
%\documentclass[letterpaper, 10 pt, conference]{ieeeconf}
\IEEEoverridecommandlockouts
% The preceding line is only needed to identify funding in the first footnote. If that is unneeded, please comment it out.
%Template version as of 6/27/2024

\usepackage{cite}
\usepackage{amsmath,amssymb,amsfonts}
\usepackage{algorithmic}
\usepackage{graphicx}
\usepackage{textcomp}
\usepackage{xcolor}
\usepackage{array}
\usepackage{hyperref}
\usepackage{booktabs}
\def\BibTeX{{\rm B\kern-.05em{\sc i\kern-.025em b}\kern-.08em
    T\kern-.1667em\lower.7ex\hbox{E}\kern-.125emX}}
\begin{document}
\title{An Automated Machine Learning Framework for Surgical Suturing Action Detection under Class Imbalance\\
%{\footnotesize \textsuperscript{*}Note: Sub-titles are not captured for https://ieeexplore.ieee.org  and
%should not be used}

\thanks{This research was fully funded by EPSRC, UK. With the Grant Reference EP/Y017307/1.}
\thanks{* Corresponding Author}
}

\author{\IEEEauthorblockN{1\textsuperscript{st} Baobing Zhang*}
\IEEEauthorblockA{\textit{School of Engineering and Physical Sciences} \\
\textit{Heriot Watt University}\\
Edinburgh EH14 4AS, UK \\
B.Zhang@hw.ac.uk
}

\and
\IEEEauthorblockN{2\textsuperscript{st} Paul Sullivan}
\IEEEauthorblockA{\textit{School of Engineering and Physical Sciences} \\
	\textit{Heriot Watt University}\\
	Edinburgh EH14 4AS, UK \\
	P.Sulliva@hw.ac.uk
}

\and
\IEEEauthorblockN{3\textsuperscript{st} Benjie Tang}
\IEEEauthorblockA{\textit{Surgical Skills Centre, Dundee Institute for Healthcare
		Simulation} \\
\textit{Ninewells Hospital and Medical School, University
	of Dundee}\\
Dundee, UK \\
b.tang@dundee.ac.uk}

\and
\IEEEauthorblockN{4\textsuperscript{st} Ghulam Nabi}
\IEEEauthorblockA{\textit{Surgical Skills Centre, Dundee Institute for Healthcare
		Simulation} \\
	\textit{Ninewells Hospital and Medical School, University
		of Dundee}\\
	Dundee, UK \\
	g.nabi@dundee.ac.uk}
	
\and
\IEEEauthorblockN{5\textsuperscript{st} Mustafa Suphi Erden*}
\IEEEauthorblockA{\textit{School of Engineering and Physical Sciences} \\
	\textit{Heriot Watt University}\\
	Edinburgh EH14 4AS, UK \\
	@hw.ac.uk
}

%5\textsuperscript{st}

%\and
%\IEEEauthorblockN{6\textsuperscript{th} Given Name Surname}
%\IEEEauthorblockA{\textit{dept. name of organization (of Aff.)} \\
%\textit{name of organization (of Aff.)}\\
%City, Country \\
%email address or ORCID}
}

%\author{\IEEEauthorblockN{Anonymous Authors}}

\maketitle

\begin{abstract}
In laparoscopy surgical training and evaluation, real-time detection of surgical actions with interpretable outputs is crucial for automated and real-time instructional feedback and skill development. Such capability would enable development of machine guided training systems. This paper presents a rapid deployment approach utilizing automated machine learning methods, based on surgical action data collected from both experienced and trainee surgeons. The proposed approach effectively tackles the challenge of highly imbalanced class distributions, ensuring robust predictions across varying skill levels of surgeons. Additionally, our method partially incorporates model transparency, addressing the reliability requirements in medical applications. Compared to deep learning approaches, traditional machine learning models not only facilitate efficient rapid deployment but also offer significant advantages in interpretability. Through experiments, this study demonstrates the potential of this approach to provide quick, reliable and effective real-time detection in surgical training environments. 
\end{abstract}

\begin{IEEEkeywords}
Bayesian optimization, reliable, bayesian learning, probabilistic models, trustworthy.
\end{IEEEkeywords}

\section{INTRODUCTION}


Laparoscopic surgery, as a minimally invasive technique, is now considered a crucial choice in modern surgery due to its advantages, such as minimal trauma and faster recovery \cite{madhok2022safety, basunbul2022recent}. However, the unique perspective and operation mode of laparoscopic surgery place higher demands on surgeons' hand-eye coordination and spatial awareness, where the precision of the operation directly impacts surgical quality and patient recovery outcomes \cite{stulberg2020association, sanchez2020comparative}. To improve surgeons' laparoscopic skills, laparoscopic surgical training platforms have been developed. These platforms simulate real surgical environments, providing trainees with a safe space to practice \cite{hong2021simulation, van2020bimanual}. However, most existing platforms rely on fixed training steps and manual feedback, lacking intelligent real-time feedback mechanisms, and are unable to dynamically track trainees' movements. A surgical action recognition system, however, has the potential to monitor and analyze trainees' actions in real-time, providing immediate feedback during training, which can help trainees correct errors promptly and improve training efficiency and accuracy \cite{hashimoto2018artificial}. Such real-time feedback is crucial for provision of automated and real-time feedback, to prepare the next-generation laparoscopy training systems that eliminate the need of expert monitoring \cite{larsen2009effect}.

In practical applications, surgical action recognition systems face several major challenges. First, class imbalance is a common issue in medical data, especially in surgical data, where key actions often have fewer samples, while routine actions are abundant. This imbalance can lead to model bias toward majority classes, diminishing the recognition performance for minority classes \cite{krawczyk2016learning, he2009learning, johnson2019survey}. Second, model interpretability is essential in medical applications. Surgeons and trainers require not only a highly accurate model but also one with a transparent decision-making process to ensure the rationality and credibility of the model's outputs \cite{rudin2019stop, tjoa2020survey}. However, current deep learning \cite{lecun2015deep} models often have complex decision processes and lack interpretability and reliability, while statistical model-based decision making process which provides thorough interpretability and formal reliability guarantee \cite{zhang2024bayesian}. In addition, model stability and robustness are critical for the successful application of a surgical action recognition system. With the presence of both complexity and diversity in medical data, models may exhibit inconsistent performance across different data and environments, making it essential for the system to maintain stability and reliability across varied scenarios \cite{esteva2021deep, topol2019high}. Lastly, real-time feedback is indispensable in surgical training, as an efficient surgical action recognition system must process actions in real-time to provide immediate feedback on the quality of trainees' movements \cite{aggarwal2010training, sinha2023current}.

This study aims to propose an efficient surgical action recognition system to address the aforementioned challenges, with particular focus on handling class imbalance, enhancing model interpretability and stability, and achieving real-time feedback. To this end, we incorporate automated machine learning technology, by using automated machine learning to automatically construct and optimize models, thereby avoiding the complex process of manual tuning, while enhancing model stability and generalization through ensemble learning methods. In model construction, we combine techniques such as sample resampling and weighted classification to address class imbalance, ensuring the significance of minority classes in model training. Moreover, to improve system transparency and interpretability, we utilize traditional machine learning algorithms and interpretability techniques, making the decision process understandable for both surgeons and trainees. Finally, we designed and implemented a real-time surgical action recognition system capable of timely action recognition during training, which can be used in an intelligent and dynamic evaluation tool for laparoscopic surgical training platforms. 

This paper is structured in the following manner: Section 2 reviews prior research, focusing on advancements in surgical action recognition, class imbalance handling, interpretability, and real-time feedback. Section 3 details the methodology for laparoscopic surgical action recognition, including class imbalance handling, hyperparameter optimization, and ensemble learning. Section 4 provides an overview of the experimental configuration and findings, analyzing the system's accuracy, stability, and feedback efficiency on surgical training data. Section 5 concludes with a summary of the main contributions and discusses the potential applications and future directions of this system in surgical training.



\section{Related Work}


Laparoscopy surgical action recognition is a crucial research area in surgical training, for provision of real-time feedback to trainees during procedures. Lea et al \cite{lea2016segmental}. proposed Segmental Spatiotemporal Convolutional Neural Networks (CNNs) for capturing fine-grained temporal information to perform action segmentation, achieving significant results in surgical action recognition. Similarly, Twinanda et al \cite{twinanda2016endonet}. developed EndoNet, an advanced neural network architecture specifically designed for laparoscopic surgical videos, capable of performing multitask recognition, including action classification. The multitask structure of EndoNet effectively improves the recognition of complex actions, offering a novel approach for action detection in laparoscopic videos. In surgical data processing, class imbalance is a common phenomenon. Chawla et al \cite{chawla2002smote}. proposed SMOTE (Synthetic Minority Over-sampling Technique), which generates synthetic minority samples to balance the dataset distribution and improve recognition of minority class actions. Additionally, the survey by He and Garcia \cite{he2009learning} provides a comprehensive review of various techniques for handling class imbalance, including oversampling, undersampling, and weighted methods, which play a critical role in enhancing minority class recognition.

In medical applications, model interpretability is essential for machine learning model adoption. Rudin \cite{rudin2019stop} emphasized the necessity of using interpretable models in high-stakes domains and recommended prioritizing interpretable models over black-box models to ensure higher transparency for clinicians and trainers. Moreover, the work by Doshi-Velez and Kim \cite{doshi2017towards} explores various techniques for enhancing model interpretability, presenting a framework for applying interpretable models in medical applications, providing theoretical support for the use of interpretable models in clinical scenarios. Regarding model stability, Breiman's random forest algorithm \cite{breiman2001random}, as an ensemble learning method, enhances model stability and generalization by combining multiple decision trees, making it widely applicable to medical data processing where high stability is required. To further improve model stability and automate the optimization process, Feurer et al. \cite{feurer2015efficient} developed an automated machine learning system that combines automated model selection and hyperparameter tuning, effectively reducing the complexity of manual tuning and improving model adaptability and stability on surgical data. In surgical training, real-time feedback systems assist trainees by providing immediate feedback during procedures. Salvador et al \cite{salvador2024effects} investigated the application of real-time visual feedback in laparoscopic training, examining its impact on novices' learning curves. The results showed that trainees receiving real-time feedback demonstrated higher precision in tissue handling skills, significantly shortening their learning curves. The introduction of real-time feedback enabled trainees to master essential skills more quickly, improving training efficiency.


In summary, current laparoscopy surgical action recognition systems, class imbalance handling techniques, model interpretability methods, stability-enhancing techniques, and real-time feedback systems have shown progress in surgical training, yet limitations remain. Addressing these gaps, this study aims to optimize surgical action recognition systems using automated machine learning, combining multiple techniques to improve model performance in accuracy, interpretability, stability, and real-time functionality, providing a more intelligent and efficient feedback mechanism for surgical training.


\begin{figure*}[h!]
	\centering
	\includegraphics[width=\textwidth]{framework.png}
	\caption{Overall AutoML workflow including meta learning warmstart for bayesian optimization efficient model selection and ensemble building for laparoscopy surgical suturing action detection. 
	}
	\label{framework}
\end{figure*}



\section{Method}

%To address the challenges of processing complex data in the medical field, we utilized the automated machine learning method, which combines imbalance handling, hyperparameter optimization, and ensemble learning to construct a highly adaptive and reliable classification model.




%\section*{Functionality for Handling Class Imbalance}


%
%
%
%\section*{Hyperparameter Optimization: Bayesian Optimization}
%
%  Bayesian optimization is widely used in hyperparameter tuning for automatically selecting the optimal combination of hyperparameters. The core idea of Bayesian optimization is to use a surrogate model, typically a Gaussian process \cite{bergstra2011algorithms}, to approximate the hyperparameter space, allowing for efficient selection of the next optimal hyperparameter combination. In each iteration, the surrogate model models the predicted value and uncertainty of the current hyperparameter combination, gradually narrowing down the search space.
%
%In automated machine learning, Bayesian optimization combines the surrogate model with an acquisition function (e.g., Expected Improvement \cite{zhan2020expected}) to select hyperparameters by maximizing the acquisition function. The surrogate model, typically a Gaussian process, provides predicted mean and variance for each hyperparameter combination, effectively balancing exploration and exploitation.
%
%
%Suppose the objective function is \( f(\theta) \), where \( \theta \) represents the model's hyperparameters, Bayesian optimization aims to find the optimal hyperparameter combination \( \theta^* \) as:
%\begin{equation}
%	\theta^* = \arg \max_{\theta} f(\theta)
%\end{equation}
%The Gaussian process models \( f(\theta) \) as a normally distributed random variable with mean \( \mu(\theta) \) and variance \( \sigma^2(\theta) \). The acquisition function Expected Improvement can be expressed as:
%\begin{equation}
%	\text{EI}(\theta) = \mathbb{E}[\max(0, f(\theta) - f(\theta^+))]
%\end{equation}
%where \( f(\theta^+) \) is the objective function value of the current best solution.
%
%\section*{Model Selection and Ensemble Learning}
%
%In classification tasks, ensemble learning \cite{dong2020survey} combines multiple models to enhance overall model performance and stability. The model selection and ensemble mechanisms in automated machine learning automatically select top-performing models and combine them to form a weighted average or voting ensemble model, reducing bias and variance from individual models.
%
%Specifically, the weighted averaging method assigns different weights to each model based on its performance, with predictions summed according to their weights. In the voting method \cite{leon2017evaluating}, the final output is determined by selecting the most frequently predicted class across models. This ensemble approach shows strong generalization capabilities, especially when handling medical data.
%
%Suppose there are \( K \) models, with each model's prediction represented by \( \hat{y}_k \) and weight \( w_k \). The weighted average ensemble prediction \( \hat{y} \) is:
%\begin{equation}
%	\hat{y} = \sum_{k=1}^{K} w_k \cdot \hat{y}_k
%\end{equation}

The focus of this study is on surgical action recognition within the context of laparoscopy training and evaluation. This task aims to generate accurate predictions for surgical actions from trajectory data in real-time, leveraging automated machine learning techniques to minimize the need for human intervention. In practical applications, the computational budget is defined by the specific requirements of laparoscopy training systems, including constraints on CPU time, memory usage, and latency, ensuring compatibility with the real-time feedback needs of surgical training platforms. These constraints are critical for deploying models in training environments where quick and reliable action recognition is essential for effective evaluation and feedback. Specifically, the automated machine learning for surgical action recognition can be formulated in the following manner:

Let \( i = 1, \ldots, n + m \), where \( x_i \in \mathbb{R}^d \) denotes a feature vector and \( y_i \in Y \) represents a paired response label. Given a training dataset \( D_{\text{train}} = \{(x_1, y_1), \ldots, (x_n, y_n)\} \) and a test dataset \( D_{\text{test}} = \{(x_{n+1}, y_{n+1}), \ldots, (x_{n+m}, y_{n+m})\} \), where samples are obtained from a consistent data structure, the goal is to derive automated predictions \( \hat{y}_{n+1}, \ldots, \hat{y}_{n+m} \) for the test set, given a loss function \( \mathcal{L}(\cdot, \cdot) \) and a resource budget. The loss of the solution is defined as \( \frac{1}{m} \sum_{j=1}^{m} \mathcal{L}(\hat{y}_{n+j}, y_{n+j}) \).


%We employ the Auto-sklearn \cite{feurer2015efficient} method to accomplish the task of laparoscopic suturing action recognition. Our dataset comprises extensive two-hand coordinate data collected from both experienced surgeons and interns during suturing procedures. Inspired by the AutoML approach introduced in the Auto-sklearn framework, we utilize Bayesian optimization and automated model ensemble construction to improve model efficiency and robustness.

The formal definition of automated machine learning (AutoML) as a Combined Algorithm Selection and Hyperparameter optimization (CASH) problem was introduced by the AutoML approach in the AUTO-WEKA \cite{hall2009weka} system. In AutoML, a single machine learning algorithm may not always perform optimally across different datasets, making algorithm selection and hyperparameter optimization essential components of an AutoML system. The CASH problem integrates model selection and hyperparameter optimization into a single unified optimization task that can be solved using Bayesian optimization, aiming to identify the optimal algorithm and hyperparameter settings that minimize the loss function on a given dataset.

For the definition of CASH, let a set of algorithms \( \mathcal{A} = \{ A^{(1)}, \dots, A^{(R)} \} \), where each algorithm \( A^{(j)} \) has its own hyperparameter space \( \Lambda^{(j)} \). Given a training dataset \( D_{\text{train}} = \{ (x_1, y_1), \dots, (x_n, y_n) \} \), we partition it into \( K \) cross-validation folds, consisting of validation sets \( \{ D_{\text{valid}}^{(1)}, \dots, D_{\text{valid}}^{(K)} \} \) and corresponding training subsets \( \{ D_{\text{train}}^{(1)}, \dots, D_{\text{train}}^{(K)} \} \), where \( D_{\text{train}}^{(i)} = D_{\text{train}} \setminus D_{\text{valid}}^{(i)} \) for \( i = 1, \dots, K \). The loss function is donated as \( \mathcal{L}(A_{\lambda}^{(j)}, D_{\text{train}}^{(i)}, D_{\text{valid}}^{(i)}) \), representing the loss of algorithm \( A^{(j)} \) with hyperparameters \( \lambda \) when evaluated on \( D_{\text{valid}}^{(i)} \) and trained on \( D_{\text{train}}^{(i)} \). In the context of the CASH framework, the goal is to determine the best combination of model selection and hyperparameter tuning, aiming to reduce the average validation error across all cross-validation folds, formulated as:
\[
A^*, \lambda^* = \arg\min_{A^{(j)} \in \mathcal{A}, \lambda \in \Lambda^{(j)}} \frac{1}{K} \sum_{i=1}^{K} \mathcal{L}(A_{\lambda}^{(j)}, D_{\text{train}}^{(i)}, D_{\text{valid}}^{(i)}).
\]


The CASH problem is mathematically formulated as minimizing the average loss across all cross-validation folds. The optimal algorithm and hyperparameter settings are obtained by minimizing this loss function. Thornton et al  \cite{thornton2013auto} were among the earliest researchers to examine the CASH problem in the AUTO-WEKA system \cite{hall2009weka}, which utilized Bayesian optimization to solve the combined task of choosing the optimal learning algorithm and adjusting hyperparameters. Bayesian optimization \cite{brochu2010tutorial} employs a probabilistic model to estimate performance of different hyperparameter configurations, thereby achieving a balance between exploring new configurations and exploiting known optimal ones. Thornton et al. \cite{thornton2013auto} investigated Bayesian tree-based search algorithms, demonstrating that the SMAC framework utilizing random forests \cite{hutter2011sequential} performs better than its predecessor \cite{bergstra2011algorithms}. Within this study, SMAC is leveraged to optimize the CASH process. Besides leveraging random forests \cite{breiman2001random}, SMAC is distinguished by its ability to Enhance cross-validation efficiency through independent fold evaluation while filtering out ineffective hyperparameter configurations early on.

Class imbalance is a common issue in classification tasks, especially in medical datasets where minority class labels often lack sufficient samples. To better identify minority classes, automated machine learning provides functionality for handling imbalanced classes. The main idea is to assign different weights to each class so that the model places greater emphasis on minority class samples during training, thereby improving overall classifier performance.

\begin{figure*}[h!]
	\centering
	\includegraphics[width=\textwidth]{class.png}
	\caption{Class distribution across action categories shows a noticeable imbalance, with certain categories, such as Double throw and Set needle, containing significantly more samples compared to others like Grasp needle and Needle exits. This imbalance may lead to biased model performance, favoring well-represented classes while potentially underperforming on underrepresented ones.
	}
	\label{classdistribution}
\end{figure*}

In practice, AutoML combines two main approaches for handling class imbalance: \textit{sample resampling} and \textit{weighted classification}. Sample resampling includes oversampling (such as using SMOTE \cite{chawla2002smote} to produce additional data samples for minority classes) and undersampling (reducing the number of majority class samples). Weighted classification assigns higher weights to minority classes, increasing their penalty in the loss function and enhancing the model's sensitivity to these classes.

Suppose a dataset with \( C \) classes, where each class \( i \) has \( N_i \) samples, the weight for class \( i \), \( w_i \), is defined as:
\begin{equation}
	w_i = \frac{N}{C \cdot N_i}
\end{equation}
Here, \( N \) indicated total dataset cardinality. The weighted cross-entropy loss function is given by:
\begin{equation}
	L = - \sum_{i=1}^{C} w_i \cdot y_i \log(\hat{y}_i)
\end{equation}
where \( y_i \) represents the true label's one-hot encoding, and \( \hat{y}_i \) represents the model's predicted probability.

Auto-Sklearn integrates meta-learning \cite{hutter2011sequential} and automated ensemble \cite{guyon2010model, lacoste2014agnostic} techniques to accelerate model optimization and enhance overall performance. First, meta-learning leverages prior knowledge from previous tasks to predict configurations has a high probability of delivering reliable results on new data. It computes meta-features (such as dataset size, dimensionality, and class distribution) for several datasets to infer suitable algorithms for the new dataset. This process involves an offline phase and an online phase: in the offline phase, the system analyzes multiple datasets to identify high-performing configurations and stores them; when encountering a new dataset, it quickly identifies the most similar stored configuration based on meta-features as the starting point for Bayesian optimization, significantly reducing optimization time. Here we follow the Auto-Sklearn approach for our task.

Additionally, the automated ensemble method further improves robustness and performance during model optimization. Unlike traditional Bayesian optimization that seeks a single best model configuration, Auto-Sklearn retains multiple models trained during the search process and constructs an ensemble model from them. This approach avoids early convergence on a specific configuration, enhancing model robustness. The ensemble is constructed using a post-processing method that combines predictions from candidate models through weighted averaging, thereby optimizing overall performance. This method eliminates the need to fine-tune a single hyperparameter setting excessively, instead leveraging diverse model outputs to significantly boost generalization capabilities. The overall workflow for automated machine learning is as shown in Fig. \ref{framework}





\begin{figure}[ht]
	\centering
	\includegraphics[width=0.5\textwidth]{modelcomp.png}
	\caption{Model accuracy comparison, illustrating the impact of automated optimization versus conventional deep learning approaches.}
	\label{modelcomp}
\end{figure}


\begin{table*}[h!]
	\centering
	\caption{Trajectory Data Characteristics for Novice and Experienced Surgeons}
	\label{trajectorydatacharacteristics}
	\begin{tabular}{lccccc}
		\hline
		\textbf{Operation Type} & \textbf{Trajectory Length (data points)} & \textbf{Speed} & \textbf{Acceleration} & \textbf{Trajectory Range} & \textbf{Trajectory Deviation} \\
		\hline
		Novice Surgeon & 11716 & 14.59 & 113.84 & 4540.26 & 42.91 \\
		Experienced Surgeon & 4782 & 9.90 & 38.05 & 669.19 & 44.99 \\
		\hline
	\end{tabular}
\end{table*}







\begin{table*}[h]
	\centering
	\caption{Classifier Contribution Table}
	\label{tab:adjusted_classifier_contribution}
	\begin{tabular}{c|l|c|c|c|c}
		\hline
		\textbf{No.} & \textbf{Classifier} & \textbf{Ensemble Weight} & \textbf{Cost} & \textbf{Balancing Strategy} & \textbf{Validation Score} \\
		\hline
		1 & RandomForestClassifier (model 2) & 0.3 & 0.0737 & Weighting & 0.927 \\
		2 & RandomForestClassifier (model 3) & 0.06 & 0.8319 & Weighting & 0.85 \\
		3 & HistGradientBoostingClassifier (model 4) & 0.04 & 0.4264 & Weighting & 0.82 \\
		4 & KNeighborsClassifier (model 8) & 0.22 & 0.0997 & None & 0.9 \\
		5 & HistGradientBoostingClassifier (model 9) & 0.04 & 0.1113 & Weighting & 0.84 \\
		6 & RandomForestClassifier (model 13) & 0.34 & 0.0722 & Weighting & 0.93 \\
		\hline
	\end{tabular}
\end{table*}



\section{Experiments}


In this section, we utilize the laparoscopy surgical training data that we have collected from six professional and four novice laparoscopy surgeons using a laparoscopy training box. In average, the professional surgeons had 15
years’ expertise in general surgical practice and 133
hours in laparoscopic operations, and the trainee surgeons had 5 years’ surgical expertise and had gained 3 hours laparoscopic training. We recorded in total 10 videos of suturing exercise across all professional and novice participants. The experiment protocol for collection of this data was approved by the Ethics Committee of the Heriot-Watt University. The data is divided into two categories: one set was collected from operations performed by experienced surgeons, labeled as \texttt{Exp} files (6 files), while the other set was gathered from novice surgeons, labeled as \texttt{Nov} files (4 files). 
Trajectories were drawn from videos of the surgical tasks by calibrating the camera and tracking coloured markers, precisely positioned on each instrument shaft close to the instrument tip. Measurement of properties of the projected image of these markers allowed the 3D position of the markers and thus the instrument tip to be tracked across frames of the video. 
The videos had a frame rate of 25 fps and a resolution of 1920 by 1080 pixels.   
Comparison of with videos taken from a robotically controlled instrument as ground truth allowed an estimate of positioning accuracy of 5mm to be made.
These data not only capture the differences in precision and consistency but also reflect typical movement characteristics associated with different skill levels, providing a rich source of data for training and validating the surgical action recognition model. 
Our dataset comprises 11 categories of movement, as listed in Table \ref{tab:classification_report_novice}, each representing a specific surgical action. The 11 categories were created by combining surgical phase modelling \cite{lalys2014surgical} and heirarchical task analysis \cite{sarker2008constructing} to the suture task.
These activities are broadly composed of three phases:
Bringing the needle under control of the instrument and setting orientation (Seek needle, Grasp needle, Set Needle).
Passing the suture material through the tissue phantom (Seek proximal insertion, Proximal bite, Distal bite, Needle exits, Tension.
Forming a surgeon's knot (Create bight, Single Throw, Double throw).
This resulted in 11 distinct activities that must be performed in sequence to complete the task and as a result were conserved across the dataset. To fully illustrate the attributes of trajectory data, we offer a detailed computational table \ref{trajectorydatacharacteristics} of trajectory characteristics, where each attribute is calculated by averaging. Here, we don’t need to use dimensions for these attributes, because they are the same for both novices and experienced surgeons, making them easier to compare. Additionally, Fig. \ref{classdistribution} illustrates the data distribution across categories, clearly reflecting the class imbalance issue among different actions. Finally, a visualization of the trajectories in Fig. \ref{trajvisual} shows the spatial characteristics of the operations, displaying the paths taken by novice and experienced surgeons on the same task. These tables and figures help us gain an in-depth insight of the dataset, supporting subsequent model training and performance evaluation.


%Additionally, the data collection process adhered strictly to experimental protocols to ensure the authenticity and representativeness of the acquired data.

\begin{table}[h]
	\centering
	\caption{Classification Report on Test Dataset}
	\label{tab:classification_report_novice}
	\begin{tabular}{lcccc}
		\hline
		\textbf{Class} & \textbf{Precision} & \textbf{Recall} & \textbf{F1-score} & \textbf{Support} \\
		\hline
		Create Bight             & 0.90 & 0.90 & 0.90 & 1005 \\
		Distal bite              & 0.84 & 0.93 & 0.88 & 1861 \\
		Double throw             & 0.96 & 0.92 & 0.94 & 9604 \\
		Grasp needle             & 0.96 & 0.98 & 0.97 & 251  \\
		Needle exits             & 0.90 & 0.85 & 0.88 & 1076 \\
		Proximal bite            & 0.88 & 0.92 & 0.90 & 1754 \\
		Seek needle              & 0.99 & 0.96 & 0.97 & 684  \\
		Seek proximal insertion  & 0.85 & 0.86 & 0.85 & 630  \\
		Set needle               & 0.94 & 0.97 & 0.95 & 1876 \\
		Single throw             & 0.89 & 0.89 & 0.89 & 3259 \\
		Tension                  & 0.86 & 0.91 & 0.88 & 1432 \\
		\hline
		\textbf{Accuracy}        &       &       & 0.92 & 23432 \\
		\textbf{Macro avg}       & 0.91 & 0.92 & 0.91 & 23432 \\
		\textbf{Weighted avg}    & 0.92 & 0.92 & 0.92 & 23432 \\
		\hline
	\end{tabular}
\end{table}


\begin{figure}[h]
	\centering
	\includegraphics[width=0.5\textwidth]{modelsize.jpg}
	\caption{Analysis of prediction speed and accuracy relative to model size in ensemble architectures}
	\label{predictionvsmodelsize}
\end{figure}

The training platform used is the HP ZBook Fury 16 G10, featuring an Intel Core i9-13950HX processor, 64GB of DDR5 RAM, and an NVIDIA RTX 5000 Ada GPU with 16GB of dedicated memory. We use Auto-Sklearn \cite{feurer2015efficient} as the implementation. For data partitioning, considering the significant differences in data characteristics between the \texttt{Exp} and \texttt{Nov} datasets, we used all \texttt{Exp} data along with half of the \texttt{Nov} data as the training and validation set. The remaining half of the \texttt{Nov} data was designated as the test set. This devision ensures that the model is exposed to a diverse dataset while reserving part of the novice data for unbiased testing, highlighting the model’s capacity for adaptation across experience levels. 

\begin{figure*}[h!]
	\centering
	\includegraphics[width=0.8\textwidth]{Figure_1.png}
	\caption{Experienced surgeons' trajectories show consistent, controlled, and concentrated movements within a defined spatial range, reflecting precision and coordination. In contrast, novice surgeons' trajectories display irregular and dispersed patterns, suggesting a lack of control and coordination. The trajectories of novice surgeons also demonstrate frequent abrupt deviations and scattered points, indicating variability and more erratic hand movements. These inconsistencies and outliers in novice trajectories may introduce noise to the data, potentially impacting model training by skewing predictions toward these deviations.}
	\label{trajvisual}
\end{figure*}


\section{Results}

Fig. \ref{modelcomp} presents a comparison of various models’ effectiveness on the given dataset. AutoML achieves the highest accuracy at \textit{92\%}, significantly outperforming the classic deep learning based models. InceptionTime \cite{ismail2020inceptiontime} and FCN \cite{wang2017time} achieve comparable accuracies of \textit{67\%} and \textit{66\%}, respectively, demonstrating relatively strong performance. TapNet \cite{zhang2020tapnet} achieves an accuracy of \textit{56\%}, while CNN \cite{zhao2017convolutional} performs the worst with an accuracy of only \textit{48\%}. The superior performance of AutoML is likely due to its automated feature selection, hyperparameter optimization, and model ensembling techniques, whereas deep learning models like InceptionTime and FCN benefit from their specific adaptability to time-series data. CNN models, typically more suitable for image data, show limited modeling capacity for time-series tasks, which may explain their lowest accuracy. It is important to note that we used open-source implementations of deep learning models without any optimization tailored to our dataset. 

%such as handling class imbalance or adjusting model architectures.


The overall accuracy on the novice dataset is \textit{91.7\%}. Combined with the detailed metrics in the classification report (see Table \ref{tab:classification_report_novice}), we can gain insights into the model’s performance, stability, and reliability under conditions of class imbalance.

First, there is a significant disparity in the number of samples per class, with \texttt{Double throw} having a support count of 9604, while \texttt{Grasp needle} only has 251. Such imbalance usually leads to better performance on classes with larger sample sizes, potentially at the expense of smaller classes. By examining the \textit{Weighted avg} and \textit{Macro avg} metrics, we can better understand comprehensive effectiveness of this approach. In this dataset, the weighted average F1-score is \textit{0.92}, indicating that the model maintains stable performance across most classes, especially in those with larger sample sizes, which enhances overall performance on imbalanced data. Meanwhile, the macro average F1-score is \textit{0.91}, demonstrating that the model performs relatively consistently across classes without a strong bias towards larger classes, showing a degree of generalization.

The \textit{precision} and \textit{recall} values in the classification report further highlight the model's stability and reliability. For instance, the F1-score for \textit{Double throw} is \textit{0.94}, while \textit{Grasp needle} achieves an F1-score of \textit{0.97}. This suggests that the model achieves high precision and recall across both large and small classes, reliably identifying various surgical actions. Moreover, the high macro and weighted averages confirm the model's stable performance despite class imbalance.

Finally, \textit{support} provides the sample count for each class, clearly showing the distribution differences across classes. Despite the imbalance, the model still achieves a high overall F1-score and accuracy, demonstrating its robustness and adaptability. These metrics collectively validate the model's robust performance on the novice dataset, providing reliable support for surgical action recognition applications.





As shown in Table \ref{tab:adjusted_classifier_contribution}, notable variations exist in the contribution and performance of different classifiers within the ensemble model. In terms of interpretability, models like RandomForestClassifier and KNeighborsClassifier generally provide higher interpretability, which aids in analyzing the specific role of each model in the predictions. The table's Ensemble Weight attribute indicates contribution of each model to the ensemble; for example, RandomForestClassifier (model 13) has the highest weight (0.34), showing that it plays a dominant role in the final predictions. Additionally, the Balancing Strategy column shows the balancing strategy for each model, with most models using the Weighting strategy to address data imbalance issues, enhancing robustness and fairness in predictions. Finally, the Validation Score column reflects the models' performance on the validation set, where models with higher weights tend to have higher validation scores, further emphasizing their importance in the overall ensemble. Relationship between model size, prediction speed and prediction accuracy is as shown in Fig. \ref{predictionvsmodelsize}. This figure illustrates how the size of ensemble models (in MB) affects predictions per second and prediction accuracy. Smaller models exhibit significantly higher prediction speeds, with speeds exceeding 100 predictions per second for models under 25MB, making them highly suitable for scenarios with strict real-time requirements. However, as the model size increases, prediction speed drops significantly, while accuracy shows a gradual upward trend. Notably, the largest model (200MB) achieves a high accuracy close to 0.95 but suffers from a substantial decline in prediction speed, making it less viable for real-time applications. A closer look reveals that models within the 10MB to 25MB range strike the best balance between speed and accuracy. These models deliver prediction speeds exceeding 100 predictions per second while maintaining strong accuracy, making them ideal for scenarios that demand both real-time performance and reliable predictions. This range highlights a practical trade-off, ensuring sufficient speed without compromising predictive performance, making them highly suitable for scenarios with strict real-time requirements.



\section{CONCLUSIONS and Future Works}

In this study, we developed a traditional machine learning based and fast deployable laparoscopic surgical action recognition system, effectively addressing challenges related to data imbalance, model interpretability, stability, and reliability. By training our model on a high-performance platform and leveraging the automated machine learning framework, we achieved high accuracy and stable performance across multiple surgical action categories. Additionally, the classifier contribution table and model performance analysis provided valuable insights into model selection and optimization, validating the potential of our approach to enhance surgical action recognition and training efficiency. For upcoming studies, we plan to expand our dataset by collecting more surgical action data to improve model robustness. We also intend to explore the latest deep learning based detection techniques to further enhance the system's accuracy and real-time capabilities.


\addtolength{\textheight}{-12cm}   % This command serves to balance the column lengths
% on the last page of the document manually. It shortens
% the textheight of the last page by a suitable amount.
% This command does not take effect until the next page
% so it should come on the page before the last. Make
% sure that you do not shorten the textheight too much.

%%%%%%%%%%%%%%%%%%%%%%%%%%%%%%%%%%%%%%%%%%%%%%%%%%%%%%%%%%%%%%%%%%%%%%%%%%%%%%%%



%%%%%%%%%%%%%%%%%%%%%%%%%%%%%%%%%%%%%%%%%%%%%%%%%%%%%%%%%%%%%%%%%%%%%%%%%%%%%%%%



%%%%%%%%%%%%%%%%%%%%%%%%%%%%%%%%%%%%%%%%%%%%%%%%%%%%%%%%%%%%%%%%%%%%%%%%%%%%%%%%
%\section*{APPENDIX}
%
%Appendixes should appear before the acknowledgment.


\section*{Acknowledgment}

This research has been funded by the Engineering and Physical Sciences Research Council (EPSRC) of United Kingdom under Grant Reference EP/Y017307/1.


\bibliographystyle{IEEEtran}
\bibliography{IEEEabrv,ref}

%\begin{thebibliography}{00}
%\bibitem{b1} G. Eason, B. Noble, and I. N. Sneddon, ``On certain integrals of Lipschitz-Hankel type involving products of Bessel functions,'' Phil. Trans. Roy. Soc. London, vol. A247, pp. 529--551, April 1955.
%\bibitem{b2} J. Clerk Maxwell, A Treatise on Electricity and Magnetism, 3rd ed., vol. 2. Oxford: Clarendon, 1892, pp.68--73.
%\bibitem{b3} I. S. Jacobs and C. P. Bean, ``Fine particles, thin films and exchange anisotropy,'' in Magnetism, vol. III, G. T. Rado and H. Suhl, Eds. New York: Academic, 1963, pp. 271--350.
%\bibitem{b4} K. Elissa, ``Title of paper if known,'' unpublished.
%\bibitem{b5} R. Nicole, ``Title of paper with only first word capitalized,'' J. Name Stand. Abbrev., in press.
%\bibitem{b6} Y. Yorozu, M. Hirano, K. Oka, and Y. Tagawa, ``Electron spectroscopy studies on magneto-optical media and plastic substrate interface,'' IEEE Transl. J. Magn. Japan, vol. 2, pp. 740--741, August 1987 [Digests 9th Annual Conf. Magnetics Japan, p. 301, 1982].
%\bibitem{b7} M. Young, The Technical Writer's Handbook. Mill Valley, CA: University Science, 1989.
%\bibitem{b8} D. P. Kingma and M. Welling, ``Auto-encoding variational Bayes,'' 2013, arXiv:1312.6114. [Online]. Available: https://arxiv.org/abs/1312.6114
%\bibitem{b9} S. Liu, ``Wi-Fi Energy Detection Testbed (12MTC),'' 2023, gitHub repository. [Online]. Available: https://github.com/liustone99/Wi-Fi-Energy-Detection-Testbed-12MTC
%\bibitem{b10} ``Treatment episode data set: discharges (TEDS-D): concatenated, 2006 to 2009.'' U.S. Department of Health and Human Services, Substance Abuse and Mental Health Services Administration, Office of Applied Studies, August, 2013, DOI:10.3886/ICPSR30122.v2
%\bibitem{b11} K. Eves and J. Valasek, ``Adaptive control for singularly perturbed systems examples,'' Code Ocean, Aug. 2023. [Online]. Available: https://codeocean.com/capsule/4989235/tree
%\end{thebibliography}



\end{document}


\end{document}
