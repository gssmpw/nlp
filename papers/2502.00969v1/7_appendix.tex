
\begin{table*}[t]
\small
\centering
\begin{tabular}{lll}
\hline
\textbf{Metric} & \textbf{Score} & \textbf{Interpretation} \\ \hline
Realism & 1 & Impossible to happen in the real world. \\
& 2 & Unlikely to happen in the real world, but still possible. \\
& 3 & May happen in the real world, but unrealistic. \\
& 4 & Likely to happen in the real world, with minor issue. \\
& 5 & Very likely to happen in real life. \\ \hline
Conciseness & 1 & Utterances are way too long and verbose. Not acceptable. \\
& 2 & Utterances are very long and verbose. \\
& 3 & Utterances are long but still acceptable. \\
& 4 & There is a space for simplifying utterances. \\
& 5 & Utterances cannot be further simplified. \\ \hline
Coherence \& Fluency & 1 & The conversation is not complete till the end. \\
& 2 & Contains serious coherence or fluency problems, but the conversation is complete.\\
& 3 & Contains obvious coherence or fluency problems. \\
& 4 & Contains minor coherence or fluency problems. \\
& 5 & Coherent and fluent conversation without any flaw. \\ \hline
Overall naturalness & 1 & Very unnatural conversation. Difficult to read. \\
& 2 & Contains obvious naturalness issues. \\
& 3 & Obviously this is machine-generated, with minor naturalness issues. \\
& 4 & No obvious issues found, but it's not as natural as humans. \\
& 5 & Good conversation. Indistinguishable from real-life conversations. \\
\hline

\end{tabular}

\caption{Interpretation of conversational-level human evaluation scores.}
\label{tab:score_interpretation}
\end{table*}

% \begin{figure}[h]
% \centering
% \includegraphics[width=0.45\textwidth]{figures/feature_stats.pdf}
% \caption{Histogram depicting the distribution of products based on the number of attributes.}
% \label{fig:stats}
% \end{figure}

\section{More Related Work}
\subsection{Dialogue Generation with LLMs}

Collecting dialogue datasets for conversational applications requires significant effort from crowd workers~\cite{ %eric-etal-2017-key,
budzianowski-etal-2018-multiwoz, wu-etal-2020-tod}, incurring substantial financial and time costs. Hence, alternative simulation approaches such as DIALOGIC were developed for dialogue generation using GPT-3 \cite{brown2020language} on the MultiWOZ dataset \cite{han2021multiwoz}.
It successfully generated dialogues with near-human fluency, even when trained with as few as 85 initial seed dialogues.
%\citet{zhang2023sgp} used GPT-3.5 for dialogue state generation and further performed dialogue generation on MultiWOZ \cite{budzianowski-etal-2018-multiwoz, zang2020multiwoz}, RADDLE \cite{peng-etal-2021-soloist}, and STAR \cite{mosig2020star} datasets. %\citet{semnani2023wikichat} uses a pipeline of retrieval, summarization, GPT-3.5-based response generation, factual claim extraction, fact-checking, response drafting, and response refinement for Wikipedia-grounded dialogue generation. 
%On the other hand, LLMs are used to generate conversation for downstream tasks. 
Similarly, \citet{li-etal-2023-autoconv} proposed to fine-tune OPT-13B~\cite{zhang2022opt} to generate conversations grounded on a given document. 
%Ultra-Chat \cite{ding2023enhancing} and RealChat \cite{kong2023large} were collected by leveraging ChatGPT-ChatGPT and UserGPT \cite{kong2023large}-ChatGPT automatic conversations to train a downstream chat-bot. 
A semi-automated dialogue generation framework based on ChatGPT~\cite{lu2023dialgen} was proposed to generate complex conversations in the insurance domain by generating sub-dialogues and using human feedback to ensure consistency.
They demonstrated that incorporating the generated dataset into existing auto insurance claim call datasets can enhance the performance of Dialogue State Tracking (DST) \cite{young2010hidden}. 
%\citet{bitton2023q2d} used few-shot-tuned LLMs to generate conversations given questions for constructing a conversation-to-query dataset, in order to train query generation models. 
%\citet{yang2023refgpt} started from a reference passage and constructed a conversation using GPT-4 to address the model hallucination problem when generating a dialogue dataset.

%For conversational search and recommendation, \citet{owoicho2023exploiting} built a conversational search system that retrieves and reranks knowledge to support utterances, based on GPT-3.5. 
%\citet{gung-etal-2023-natcs} collected a multi-domain natural customer support dialogue dataset covering domains such as insurance, banking, finance, health, and travel. Participants were assigned a series of minimal discourse types to expand to a full conversation. Similar to our single-pass and interactive generation approach (see Section \ref{sec:verbalization}), they also experimented with the setting of one person acting both parties or two people speaking to each other.%\zhiyu{highlight the difference of our work}

Unlike previous methods, we do not rely on initial seed annotations; we only leverage the zero-shot capabilities of LLMs to generate conversations that are grounded in a product catalog and a simulated user profile. %\shervin{Please make sure that the catalog grounding and use of user profiles is introduced and justified before this section.} 
Once a user profile is synthesized from the product information and elicited customer preferences, a decision tree is fitted to guide the dialogue generation, ensuring that the conversation concludes with the discovery of the target product. %\shervin{The fact that conversations must be grounded in the catalog and not LLM knowledge is a main motivation of this paper. It should be stated clearly in the introduction and emphasized in the Related Work. I don't see it clearly motivated in this section.}\zhiyu{revised}
Similar to the DST task~\cite{lu2023dialgen}, we demonstrate that our generated dataset can be used to train a query generator that extracts preferences for product attribute values (\S\ref{sec:downstream_tasks}).


\section{Detailed Human Evaluation Metric Descriptions}
\subsection{Conversation-level Evaluations} 
The metrics are summarized in Table \ref{tab:human_evaluation_metrics}. For conversation-level evaluations, judges are asked to score each conversation on a Likert scale of 1 to 5. The meaning of each of the scores is interpreted in Table \ref{tab:score_interpretation}. 

\paragraph{Realism.} To distinguish counter-intuitive conversations, \textit{realism} examines whether the conversation is likely to happen in the real world from a logical perspective. We also consider \textit{Factuality}, which captures the commonsense-level factual errors, since they essentially reflect similar criteria. Meanwhile, we consider whether the speakers stick to their own role, i.e. the seller should ask questions or recommend products but not buy products, while the customer should focus on answering proposed questions to narrow down products to buy, instead of recommending products. %\zhiyu{sounds like Script-following.} 
We combine these three aspects into one score since they are highly related.

\paragraph{Conciseness} examines whether the conversation is too verbose, such as a single utterance contains too many aspects to discuss.

\paragraph{Coherence} measures whether the conversation is coherent to the conversation history from the beginning. Additionally, we also consider fluency, as it is also closely relevant. Specifically, we examine (1) whether there is repetition or forgetting about the context; (2) whether the agents ignore the previous utterance and proceed to the next topic; (3) whether the conversation gets stuck at some point and never proceeds.

\paragraph{Overall naturalness} reflects the overall subjective impression of the conversation. Opposing to the \textit{realism} which reflects whether the conversation is logical, the naturalness asks whether the judge subjectively feels humanness from the conversation.

\subsection{Utterance-level Evaluations}
We further evaluate the conversations on a finer-grained scale. Judges are asked to list all unsatisfactory utterance IDs, as a binary classification task.

\paragraph{Realism.} Similar to the conversation level, we examine the realism, factuality, and consistency of the role at the utterance level.

\paragraph{Script-following.} Given the assigned product aspects and values, as well as the customer's preference (\textit{wanted}, \textit{optional}, or \textit{unwanted}), we evaluate whether the utterances are consistent with the assigned features. The utterances should cover all assigned features and should not alter the type of the customer's preference (e.g. switching \textit{unwanted} to \textit{optional}). Further, agents should not invent new product features that are not assigned, which may create inconsistency with the search plan and the semantics of the conversation.

\paragraph{Reasonable features.} The selected product features by the decision tree may contain inappropriate or unnatural product aspects and values to be discussed. For example, the customer is not supposed to know the value of the aspect ``model number''. This issue usually originates from the quality of the product metadata, which is impossible to perfectly curate. Table \ref{tab:utterance_level_human_evaluation} shows the results excluding utterances with unreasonable features to avoid confusion.

%\section{Experimental Settings}
%\subsection{Preprocessing of TREC Dataset for Dialogue Generation}


\section{Large Language Models for Dialogue Generation} \label{sec:llm}
We experiment with two different LLMs: OpenAI GPT-4\footnote{\url{https://platform.openai.com/docs/models}, gpt-4-0613} and LLaMA-2-70B-chat. As the \textit{chatbot-arena-leaderboard}\footnote{\url{https://huggingface.co/spaces/lmsys/chatbot-arena-leaderboard}} shows, GPT-4 was the best-performing LLM by then.
However, as GPT-4 is proprietary, we also experiment with LLaMA-2-70B-chat, which is one of the best performing commercially available instruction-tuned LLMs. Due to the high GPU memory requirement of LLaMA-2, we use a 5-bit quantized version.\footnote{\url{https://github.com/abetlen/llama-cpp-python}, \url{https://huggingface.co/TheBloke/Llama-2-70B-Chat-GGML/resolve/main/llama-2-70b-chat.ggmlv3.q5_K_M.bin}}


\begin{table*}[t]  \small
\centering
%\setlength{\tabcolsep}{0.8pt} % Default value: 6pt
\renewcommand{\arraystretch}{1.0} % Default value: 1
\begin{tabular}{ll}
\hline
\textbf{Intent} & \textbf{Definition}\\ \hline
Greetings & Indicates the beginning or end of the conversation \\ \hline
Interaction structuring & Utterances that make the conversation structured and natural (e.g., thanking, stalling) \\ \hline
Disclose & The client discloses information about what they are looking for  \\ \hline
Clarification question & The agent asks a question to make sure it understands correctly a previous statement \\ \hline
Other question & Asks a question that is not a clarification question (e.g., factoid, follow-up questions) \\ \hline
Elicit preferences & The agent asks a question to find the client’s preferences (e.g., the color of an item, the budget) \\ \hline
Recommend & The agent recommends one or several items to the client \\ \hline
Answer & A participant gives an answer to the other participant’s information request \\ \hline
Explain & Provides an explanation to a previous statement (e.g., justifies suggestion or rejection of an item) \\ \hline
Positive feedback & Expresses positive feedback (e.g., confirmation, accept a recommendation)  \\ \hline
Negative feedback & Expresses negative feedback (e.g., disagreement, rejection of a recommendation) \\ \hline
Other & Does not fit other labels \\ \hline
\end{tabular}
%\vspace{-1em}
\caption{Utterance intents defined by MG-ShopDial \cite{Bernard:2023:SIGIR}.}
\label{tab:intent_definitions}
%\vspace{-2em}
\end{table*}

\begin{table*}[t]
\small
\begin{center}
%\setlength{\tabcolsep}{6pt} % Default value: 6pt
    \begin{tabular}{p{0.95\linewidth} }
    \hline
    \textbf{Instruction} \\ 
    Given an utterance with context in a conversation under e-commerce shopping scenario, 
identify all intents of the utterance defined below (format of "intent: explanation"): \\ \hdashline

Greetings: Indicates the beginning or end of the conversation \\
Interaction structuring: Utterances that make the conversation structured and natural (e.g., thanking, stalling)\\
Disclose: The client discloses information about what they are looking for\\
Clarification question: The agent asks a question to make sure it understands correctly a previous statement\\
Other question: Asks a question that is not a clarification question (e.g., factoid, follow-up questions)\\
Elicit preferences: The agent asks a question to find the client’s preferences (e.g., the color of an item, the budget)\\
Recommend: The agent recommends one or several items to the client\\
Answer: A participant gives an answer to the other participant’s information request\\
Explain: Provides an explanation to a previous statement (e.g., justifies suggestion or rejection of an item)\\
Positive feedback: Expresses positive feedback (e.g., confirmation, accept a recommendation)\\
Negative feedback: Expresses negative feedback (e.g., disagreement, rejection of a recommendation)\\
Other: Does not fit other labels\\ \hdashline

Examples (Utterance followed by intents): \\
\{\} \\

Contexts: \{\} \\

Utterance: ``\{\}'' \\
Answer all plausible intents names of the utterance above, separate with ``;''. Do not return explanations.\\
Answer: \\ \hline
    \end{tabular}
    \caption{Prompt format for predicting utterance intents. } \label{tab:intent_prompt}
\end{center}
\end{table*}

\begin{table*}[t]  \small
\centering
%\setlength{\tabcolsep}{0.8pt} % Default value: 6pt
\renewcommand{\arraystretch}{1.0} % Default value: 1
\begin{tabular}{lccc}
\hline
\textbf{Intent\%} & \textbf{Annotated MG-ShopDial} & \textbf{Predicted MG-ShopDial} & \textbf{Wizard of Shopping}\\ \hline
Answer & 17.0 & 3.4 & 0.1\\ \hline 
Clarification question & 11.8 & 9.3 & 3.0\\ \hline 
Explain & 11.5 & 6.7 & 5.5 \\ \hline 
Interaction structuring & 11.3 & 12.1 & 15.4\\ \hline 
Other question & 10.6 & 10.0 & 7.7 \\ \hline 
Positive feedback & 8.3 & 12.1 & 7.1 \\ \hline 
Greetings & 7.2 & 5.9 & 3.8 \\ \hline 
Recommend & 7.2 & \textbf{11.9} & \textbf{4.2} \\ \hline 
Disclose  & 6.4 & 15.3 & 15.6 \\ \hline 
Elicit preferences & 5.7 & \textbf{9.3} & \textbf{34.2}\\ \hline 
Other & 1.7 & 0.7 & 0\\ \hline 
Negative feedback & 1.4 & 3.3 & 3.4\\ \hline 

\end{tabular}
%\vspace{-1em}
\caption{Intent distribution of utterances in MG-ShopDial \cite{Bernard:2023:SIGIR} and Wizard of Shopping.}
\label{tab:intent}
%\vspace{-2em}
\end{table*}

\section{Intent Analysis of Wizard of Shopping vs. MG-ShopDial}
\label{sec:intent_analysis}
\paragraph{Experimental settings.} Extending \ref{sec:mgshopdial_comparison}, to compare the differences between utterances in \textit{WoS} and MG-ShopDial \cite{Bernard:2023:SIGIR}, we compare the intent distribution between these two datasets by prompting OpenAI GPT-4\footnote{gpt-4-turbo-2024-04-09} (see prompts in Table \ref{tab:intent_prompt}) to label each utterance with intents defined by \citet{Bernard:2023:SIGIR} (Table \ref{tab:intent_definitions}). Similar to \citet{Bernard:2023:SIGIR}, each utterance may correspond to multiple intent labels. We sample 100 \textit{WoS} conversations from each of the three domains for analysis. As \citet{Bernard:2023:SIGIR} indicated, their agreements among crowd workers are relatively low (Fleiss $\kappa$=0.187), thus we predict MG-ShopDial utterances with GPT-4 as well to make a fair comparison.

%\paragraph{Results.} As Table \ref{tab:intent} shows, similar to MG-ShopDial, our \textit{WoS} conversations have diverse distributions of intents, which is consistent with the high naturalness observed in Section \ref{sec:human_evaluation_results}. Furthermore, as we conclude in Section \ref{sec:mgshopdial_comparison}, MG-ShopDial has significantly more ``Recommend'' intent than \textit{WoS}, while \textit{WoS} has a dominating proportion of ``Elicit preferences'' intent. This is consistent with our observation that the wizard (seller) in MG-ShopDial memorizes all candidate products and starts the product recommendation stage early without explicit product search. In contrast, \textit{WoS} conversations aim to narrow down product aspect values for product search from a real product catalog and take the strategy of not picking up any product until the search is converged.

\section{Detailed Downstream Experiments}
\label{sec:detailed_downstream_tasks}
In this section, we extend \S\ref{sec:downstream_tasks} by comparing conversational-level and utterance-level generation approaches in addition to the weak baseline.

\subsection{Conversational Query Generation} \label{sec:detailed_conversational_query_generation}
\subsubsection{Approaches}
\paragraph{Baseline.} As we explained in \S\ref{sec:related_work}, to the best of our knowledge, our \textit{WoS} dataset is the only CPS dataset that can be used for training downstream tasks. Therefore as a simple baseline, we directly use the full conversation history as the predicted query.
\paragraph{Conversational-level.} We use a seq2seq model fine-tuned on our \textit{WoS} dataset to predict the product category and the \textit{wanted}, \textit{unwanted} and \textit{optional} product features, given the full conversation history.
\paragraph{Utterance-level.} We assume that each utterance between the seller and customer encodes one or more product features. We use a seq2seq model to extract product features from each utterance and concatenate the features to be the final query.
\paragraph{GPT-4} We few-shot prompt OpenAI gpt-4-0125-preview with the same inputs and outputs as the conversational-level setting above.

\subsubsection{Experimental Settings}
We fine-tune BART-base \cite{lewis-etal-2020-bart} for the utterance-level approach and use LED-base \cite{beltagy2020longformer} for the conversation-level approach due to the longer context required as input. %We take the full conversation history of the training data and augment it with a full combination of the QA-pairs within each conversation. 
We use a default learning rate of 5e-5 and train for 5 epochs for both models. 

\subsubsection{Experimental Results}
Table \ref{tab:detailed_conversational_query_generation} shows the performance of the CQG task. We report results on two settings: \textit{wanted} feature-only setups concatenate PC and wanted features and all feature setups concatenate all features as defined in Eq.~\ref{eq:cqg}. As expected, the weak baseline using the conversation history performs poorly, while our trained query generator performs much better.  
When comparing the utterance-level approaches, LED outperforms BART, since the training data for BART is slightly smaller than LED's. This is because we discard the input augmented conversation history that is too long for BART's maximum context window (1024 tokens). Also, the LED-based utterance level approach under-performs the conversation level approach, presumably because integrating the results from multiple inferences passes is more error-prone. 

We also experiment with GPT-4 as CQG for reference. Since the few-shot demonstration examples do not show every edge case, GPT-4's performance is slightly underestimated. Nonetheless, D2Q by \textit{WoS}-finetuned LED shows the superiority of both performance and cost over GPT-4.
\subsection{Conversational Product Ranking}

\subsubsection{Approaches}%\zhiyu{Define what is indexed for each document/product. Title ? Description ?}
\paragraph{Baseline} Similar to the CQG task, we assume there is no conversational data available for training a ranker. Therefore, we directly feed the full conversation history as the query to a BM25 ranker.

\paragraph{Dialogue --> Product (D2P)} We directly embed the full conversation history to a dense ranker to rank the candidate products, i.e. \textit{Products = Ranker(Dialogue)}. %\zhiyu{related to my previous comments. If products are ranked by Relevance = Ranker(dialogue, product\_info), what is product\_info here?}

\paragraph{Dialogue --> Query --> Product (D2Q2P)} We leverage D2Q approach to generate queries given the full conversation history and apply the queries to a ranker, i.e. \textit{Products = Ranker(CQG(Dialogue))}. We experiment with both a sparse ranker, BM25, and a dense ranker.

\paragraph{D2Q2P (GPT-4)} We use GPT-4 setting in \S\ref{sec:conversational_query_generation_approaches} and feed the predicted query to a BM25 ranker.

\subsubsection{Experimental Settings}
For the dense rankers, we fine-tune a Longformer-base \cite{beltagy2020longformer} for the D2P setting, and a RoBERTa-based ranker\footnote{\url{https://huggingface.co/castorini/ance-msmarco-passage}} for the D2Q2P setting using a dense passage retrieval \cite{karpukhin-etal-2020-dense} training approach \footnote{\url{https://haystack.deepset.ai/tutorials/09_dpr_training}}. For training both D2P and D2Q2P dense rankers, we use the ranking results given the corresponding gold queries as positive examples. The hard negative examples are constructed by removing the positive examples from the search results of the target product category.

\subsubsection{Experimental Results}
Table \ref{tab:detailed_ranker} shows the mean reciprocal rank (MRR) and Hit$@k$ of all methods. 
We observe that the D2P approach only slightly outperforms the baseline approach without training. We suspect that our DPR-Longformer is still under-fitted given 3000 conversations as the training set. On the contrary, by leveraging the query generators trained on our collected conversations, the ranking performance is greatly improved. 
Similar to the trend observed in the CQG task, the conversation-level approach outperforms the utterance-level approach that integrates multiple passes of inference outputs. 


\begin{table}[t]  \small
\centering
\setlength{\tabcolsep}{3pt}
\begin{tabular}{lllllll}
\hline
\textbf{Approach} & \textbf{QG} & \textbf{Feat.} & \textbf{F1} & \textbf{R-1} & \textbf{R-2} & \textbf{R-L}\\ \hline
Baseline & - & + & 0 & 0.056 & 0.020 & 0.048 \\ 
Baseline & - & +/-/? & 0.008 & 0.137 & 0.047 & 0.087  \\  \hline \hline
Utterance & BART & + & 0.698 & 0.592 & 0.482 & 0.562 \\ 
Utterance & BART & +/-/? & 0.656 & 0.664 & 0.489 & 0.592 \\  \hline
Utterance & LED & + & 0.704 & 0.610 & 0.509 & 0.578 \\ 
Utterance & LED & +/-/? & 0.748 & 0.720 & 0.561 & 0.657 \\  \hline
Conversation & LED & + & 0.834 & 0.899 & 0.822 & 0.873 \\ 
Conversation & LED & +/-/? & 0.873 & 0.900 & 0.768 & 0.840 \\  \hline \hline
Conversation & GPT-4 & + & 0.553 & 0.793 & 0.628 & 0.734 \\  
Conversation & GPT-4 & +/-/? & 0.669 & 0.833 & 0.584 & 0.703 \\ \hline
\end{tabular}

%\vspace{-1em}
\caption{Detailed Conversational Query Generation (QG) performance at the final turn measured by exact-F1, ROUGE-1, ROUGE-2, and ROUGE-L. We report the performance of only using the desired features (+) for downstream ranking, as well as using all features (wanted (+), unwanted (-), and optional (?)). Detailed version of Table \ref{tab:conversational_query_generation}.}
\label{tab:detailed_conversational_query_generation}
%\vspace{-2em}
\end{table}

\begin{table}[t]  \small
\centering
\setlength{\tabcolsep}{0.8pt} % Default value: 6pt
\renewcommand{\arraystretch}{1.0} % Default value: 1
\begin{tabular}{llllllll}
\hline
\textbf{Approach} & \textbf{QG} & \textbf{Ranker} & \textbf{MRR} & \textbf{H@1} & \textbf{H@10} & \textbf{H@100}\\ \hline
Baseline & - & BM25& 0.162 & 0.107 & 0.203 & 0.587 \\ \hline \hline
D2P & - & longformer & 0.201 & 0.143  & 0.247 & 0.620 \\ \hline
Utterance & BART & BM25& 0.629 & 0.540  & 0.733 & 0.900 \\
Utterance & BART & Roberta& 0.217 & 0.163 & 0.270 & 0.477\\ \hline
Utterance & LED & BM25& 0.667 & 0.593 & 0.740 & 0.900 \\
Utterance & LED & Roberta& 0.207 & 0.163 & 0.250 & 0.460 \\ \hline
Conversation & LED & BM25& \textbf{0.838} & \textbf{0.767}  & \textbf{0.927} & \textbf{0.993} \\
Conversation & LED & Roberta & 0.675 & 0.583  & 0.780 & 0.937 \\ \hline \hline
Conversation & GPT-4 & BM25 & 0.763 & 0.680 & 0.903 & 0.903 \\ \hline 
\end{tabular}
%\vspace{-1em}
\caption{Detailed downstream Conversational Product Ranking performance at the final turn. QG: query generator. Detailed version of Table \ref{tab:ranker}.}
\label{tab:detailed_ranker}
%\vspace{-2em}
\end{table}

%\section{Under-Explored Strengths and Future Work}
%In addition to the experiments we report, our proposed \method approach and the WoS dataset can practically adapt to various future scenarios. First of all, our dialogue generation approach doesn't depend on human annotations. As a result, it can be readily adapted to generate product search conversations in various other domains. Second, while we typically assume that the customer begins by proposing only the target product category and depends on the seller's recommendations to refine product requirements, it is also possible for the customer to initiate the conversation with both the product category and specific product feature requirements. %Third, our WoS dataset can be used for training any reasonable custom models for downstream tasks such as CPR and CQG. Our trained CPS pipeline can also be applied to any reasonable conversations from human users. 
%Finally, our \method pipeline can theoretically replace the customer agent with a human customer, as part of a working conversational shopping assistant system. However, due to the limited number of product features available in our product catalog based on the TREC dataset, \method does not currently handle out-of-vocabulary product features. %In other words, our system is not ``open-domain'' with respect to e-commerce. 
%Nonetheless, this issue can be mitigated by scaling up and further curating the product catalog~\cite{ghani2006text,yang2022mave}, which is out of the scope of this work.


\begin{table*}[t]
\small
\begin{center}
%\setlength{\tabcolsep}{6pt} % Default value: 6pt
    \begin{tabular}{p{0.95\linewidth} }
    \hline
    \textbf{System Prompt} \\ 
    You are acting as an Amazon seller to help narrow down product requirements for the customer that you are chatting with. Your output utterance is to clarify the customer's precise need by asking questions about the requested product features.  \\ \hdashline
    \textbf{Conversational history} \\ 
    Conversational history:  \\ 
    seller: \{\{\textit{Utterance\_S1}\}\} \\ 
    customer: \{\{\textit{Utterance\_C1}\}\} \\ 
    ...... \\
    \hdashline
    \textbf{Instructions about Features} \\ 
    First of all, \textbf{if you are asked clarification questions by the customer, you must friendly address the customer's questions first.} In this case, you must give the customer an opportunity to answer unanswered questions before you proceed. This means you may disregard the following new features. Otherwise, you must ask the customer about the following specific aspect of the requested product category provided by the output of the ``update\_dialogue\_state'' function. In your response, pick one or two aspects to ask from ``remaining\_feature'', so that the conversation is most natural, e.g. asking about general aspects first and then diving into detailed aspects. Once these one or two aspects are picked, your questions must strictly cover all of the words in the aspects or aspect value pairs. \textbf{Integrate the examples and values provided naturally into your utterance.} Do not ask about other aspects not covered above. If there is a conflict between your commonsense and the following aspect value pairs, you must prioritize the following provided aspect. You must ask in a concise and natural way. Do not repeat what you, the seller, have already said before. Keep the question brief. Do not talk about the budget unless you are asked to. Do not repeat the product name/category over and over.\\
    \hdashline
     \textbf{Instructions} \\ 
    What is the next utterance for you as a seller? Only reply with the response that you are supposed to make, not the customer's response. Don't start with your role, i.e. seller, just reply with the response. Remember, as a seller, you want to ask clarification questions to the customer. You never want to get products recommended or buy products from the customer. Be friendly to the customer, so that if the customer has any questions about the product or product aspect, answer patiently. \textbf{Treat the customer as if an old grandma has a limited understanding of the products.} \\
    \hdashline
    \textbf{Function Call Templates} \\
    Function name: update\_dialogue\_state \\
    Parameters: \{ \\
    \hspace{1cm} \textbf{remaining\_features: \{\{$Aspect_1: Expl_1, Expl_2, Expl_3$\}\},...} \\
    \hspace{1cm} mentioned\_positive\_features: \{\{\textit{$Aspect_2: Value_2$}\}\},... \\
    \hspace{1cm} mentioned\_optional\_features: \{\{\textit{$Aspect_3: Value_3$}\}\},... \\
    \hspace{1cm} mentioned\_negative\_features: \{\{\textit{$Aspect_4: Value_4$}\}\},... \\
    \}\\
\hline
    \end{tabular}
    \caption{Prompt format for interactive seller utterance generation. Prompts in bold are designed for improving conversation naturalness (\S\ref{sec:improving_naturalness}).} \label{tab:seller_prompt}
\end{center}
\end{table*}

\begin{table*}[t]
\small
\begin{center}
%\setlength{\tabcolsep}{6pt} % Default value: 6pt
    \begin{tabular}{p{0.95\linewidth} }
    \hline
    \textbf{System Prompt} \\ 
    You are a \textbf{lazy} customer on Amazon chatting with a seller.  \\ \hdashline
    \textbf{Persona} \\ 
    \textbf{You have a limited understanding of the product to buy, so ask the seller about what the aspects mean when encountering a product aspect that an old grandma may not understand.} \\ \hdashline
    \textbf{Conversational history} \\ 
    Conversational history:  \\ 
    customer: \{\{\textit{Utterance\_C1}\}\} \\ 
    seller: \{\{\textit{Utterance\_S1}\}\} \\ 
    ...... \\
    \hdashline
    \textbf{Instructions} \\ 
    Given the utterance you just made, make an update to the specific \\ 
    aspects provided by the output of the update\_dialogue\_state \\
    function. For the aspect or aspect value pairs under ``remaining\_positive\_features'' below, you were asked to explicitly say that you like them in a \textbf{natural way}, so here you must pass these required aspect or aspect value pairs covered by your utterance under the ``mentioned\_positive\_features'' argument of function update\_dialogue\_state. Similarly, for the aspect or aspect value pairs under "remaining\_optional\_features" below, you were explicitly asked to express that they are optional or you are not concerned about them in a natural way, so here you must pass these optional aspect or aspect value pairs covered by your utterance under the ``mentioned\_optional\_features'' argument of function ``update\_dialogue\_state''. Finally, for the aspect or aspect value pairs under ``remaining\_negative\_features'' below, you were explicitly asked to express that you don't want them in a natural way, so here you must pass these negative aspect or aspect value pairs covered by your utterance under the ``mentioned\_negative\_features'' argument of functional update\_dialogue\_state. \\
    \hdashline
    
    \textbf{Function Call Templates} \\
    Function name: update\_dialogue\_state \\
    Parameters: \{ \\
    \hspace{1cm} remaining\_positive\_features: \{\{\textit{$Aspect_1: Value_1$}\}\},... \\
    \hspace{1cm} remaining\_optional\_features: \{\{\textit{$Aspect_2: Value_2$}\}\},... \\
    \hspace{1cm} remaining\_negative\_features: \{\{\textit{$Aspect_3: Value_3$}\}\},... \\
    \hspace{1cm} mentioned\_positive\_features: \{\{\textit{$Aspect_4: Value_4$}\}\},... \\
    \hspace{1cm} mentioned\_optional\_features: \{\{\textit{$Aspect_5: Value_5$}\}\},... \\
    \hspace{1cm} mentioned\_negative\_features: \{\{\textit{$Aspect_6: Value_6$}\}\},... \\
    \}\\ \hline
    \end{tabular}
    \caption{Prompt format for interactive customer utterance generation. Prompts in bold are designed for improving conversation naturalness (\S\ref{sec:improving_naturalness}). } \label{tab:customer_prompt}
\end{center}
\end{table*}


\begin{table*}[t]
\small
\begin{center}
%\setlength{\tabcolsep}{6pt} % Default value: 6pt
    \begin{tabular}{p{0.95\linewidth} }
    \hline
    \textbf{Instruction} \\ 
    You are a scriptwriter. For the aspects or aspect value pairs below, write a chat conversation between the customer and seller about the seller trying to narrow down the customer's need. \textbf{The seller knows well about the sold products, while the customer has limited knowledge about the products.} This means the customer should not voluntarily propose what he wants, but passively respond to the aspect-values that the seller proposes. Therefore, the seller proactively asks clarification questions while the customer clarifies the needs. The customer should speak out first and say he/she wants to buy \{\{\textit{ProductCategory}\}\}. The seller asks about the aspect and the customer makes clarification on whether that aspect is something he/she wants, or it is optional. You MUST cover ALL aspect-value pairs below, including the wanted, unwanted, and optional aspect-value pairs. \textbf{Make sure you reorganize and reorder the aspect value pairs and translate them into a natural conversation in a meaningful way.} Start by talking about general product features and gradually transit into product-specific features. The seller should not ask too many aspects in one question. Instead only mention one or two aspects per question. Do not invent new aspects that are not mentioned below to talk about. Your script must be coherent with the conversation history, and there might be more scripts to be continued in the future, so don't terminate the conversation. Only output utterances, no non-verbal actions.  \\ \hdashline
    \textbf{Wanted / Optional / Unwanted Features} \\ 
  Aspect or aspect value pairs that the customer must say \{he wants to have / is optional / he does not want to have\} for the target product:  \\ 
  Aspect: \{\{\textit{Aspect}\}\}, Value: \{\{\textit{Value}\}\}; \\ 
  ...... \\\hdashline
  \textbf{Candidate Value Examples} \\ 
    \textbf{Additionally, there are some common values for each aspect to be mentioned. The seller should use these typical values in his question:} \\
    Aspect: \{\{\textit{Aspect}\}\}, Typical Values: \{\{\textit{$Value_1$}\}\}, \{\{\textit{$Value_2$}\}\}, \{\{\textit{$Value_3$}\}\}; \\ 
  ...... \\\hdashline
  \textbf{Format Instruction} \\
  Output in the format of \\
customer: AAAAA \\
seller: BBBBB \\
customer: CCCCC \\
seller: DDDDD \\ \hdashline
\textbf{Final Instruction} \\
Output a conversation following all instructions above and make sure you double check if ALL aspect value pairs above are mentioned: \\ \hline
    \end{tabular}
    \caption{Prompt format for single pass dialogue generation approach. Prompts in bold are designed for improving conversation naturalness (\S\ref{sec:improving_naturalness}).} \label{tab:single_pass_prompt}
\end{center}
\end{table*}


\begin{table*}[t]
\small
\begin{center}
%\setlength{\tabcolsep}{6pt} % Default value: 6pt
    \begin{tabular}{l p{0.72\linewidth} }
    \hline
    \textbf{Product Category} & lipstick \\ \hline
    \textbf{Wanted Features} & color: dynamite red \\ \hline
    \textbf{Optional Features} & skin type; customer review; finish type; price; special feature \\ \hline
    \textbf{Unwanted Features} & brand: Gocheaper \\ \hline 
    \end{tabular}
    \begin{tabular}{l p{0.8\linewidth} }
    \hline
    \textbf{Speaker} & \textbf{Utterance} \\
    \hline
    customer& Hi there, I'm looking to buy a \textbf{lipstick} but I'm not quite sure where to start. \\\hdashline
    seller& Hello! Sure, I'd be happy to help you find the perfect lipstick. Let's start with the \textbf{color} you prefer. Popular ones are \textbf{red}, \textbf{black} or \textbf{clear}.\\\hdashline
    customer& Oh, I think I'd like \textbf{dynamite red}. \\\hdashline
    seller& Dynamite red sounds fabulous! Now let's talk about the \textbf{brand}. Some popular ones include \textbf{Revlon}, \textbf{NYX Professional Makeup}, and \textbf{Maybelline New York}. I noticed that there's also a brand named \textbf{Gocheaper}. Any preference regarding the brand?\\\hdashline
    customer& Oh, I \textbf{don't want} the brand \textbf{Gocheaper} for sure. As for the other brands, I have limited knowledge about them so I have no preference.\\\hdashline
    seller& That's perfectly fine. Now, on to the next aspect which is the \textbf{form} of the lipstick. It comes in \textbf{cream}, \textbf{stick} or \textbf{liquid} form. Do any of these appeal to you?\\\hdashline
    customer&  I'd prefer \textbf{not} to have a \textbf{liquid lipstick}. \\\hdashline
    seller& Noted, we'll avoid the liquid lipsticks. Now lipsticks come in different \textbf{finishes}. Some have a \textbf{glossy} finish, other have a \textbf{cream} finish and there are also \textbf{matte} ones. Do you have a preference for the finish type?\\\hdashline
    customer& \textbf{\textit{Oh, is that like how shiny the lipstick is when you apply it?}}\\\hdashline
    seller& \textbf{\textit{Yes, exactly. Glossy finishes are shinier, while matte finishes have no shine and cream finishes are somewhere in between.}}\\ \hdashline
    customer& I see, thank you for explaining that. I'm open to trying different finishes, so \textbf{no specific preference} there. \\\hdashline
    seller& You're welcome, that's great to know! Lipsticks can also have some \textbf{special features} like being \textbf{portable}, \textbf{long-lasting}, or \textbf{durable}. Is there anything among these that you think is important?\\ \hdashline
    customer& Portable, long-lasting and durable all sound nice but \textbf{not mandatory} for me. \\\hdashline
    seller& Good to know. The lipsticks are also categorized based on \textbf{skin type}. There's the \textbf{all}, \textbf{normal} and \textbf{dry} skin type. What's your skin type?\\ \hdashline
    customer& \textbf{I don't think that's something that would affect my choice, as I don't know my skin type.} \\\hdashline
    seller& No problem, we can skip that. Lastly, \textbf{customer review} and \textbf{price} are often considered. What are your thoughts on these aspects?\\ \hdashline
    customer& Well, \textbf{I don't really look at customer reviews and I'm flexible regarding the price}. \\\hdashline
    seller& It sounds like you're quite flexible, which makes it easier to find the right product for you. Given your preferences, might I suggest Maybelline New York Color Sensational Red Lipstick Matte Lipstick, Dynamite Red? This is based on your preference for dynamite red color and it's not liquid. The brand is well established and it's a fairly popular one. \\ \hdashline
    customer& That sounds perfect. Thank you for your help! \\
    \hline
    \end{tabular}
    \caption{A single-pass generation conversation from GPT-4. The dialogue plan is displayed at the top of the table, and the pre-determined customer preference is reflected in the conversation with \textbf{bold}, and the clarification question and answer are \textbf{\textit{italicized.}}} \label{tab:good_example}
\end{center}
%\vspace{-3.5em}
\end{table*}

\begin{table*}[t]
\small
\begin{center}
%\setlength{\tabcolsep}{6pt} % Default value: 6pt
    \begin{tabular}{l p{0.82\linewidth} }
    \hline
    \textbf{Product Category} & cable organizer bags \& cases \\ \hline
    \textbf{Wanted Features} & color: black \\ \hline
    \textbf{Optional Features} & form factor; size; price; material type \\ \hline
    \textbf{Unwanted Features} & brand: Sunsir \\ \hline 
    \end{tabular}
    \begin{tabular}{l p{0.9\linewidth} }
    \hline
    \textbf{Speaker} & \textbf{Utterance} \\
    \hline
    customer& Can you recommend a good cable organizer bag or case for me? I prefer one with a lot of compartments. \\ \hdashline
    seller& Sure, I'd be happy to assist you in finding the perfect cable organizer. Could you please let me know your preferred color for the organizer? Would you like a black, grey, or maybe even a pink one? And how about the form? Do you prefer it to be in the form of a bag, case, or a sleeve?\\ \hdashline
    customer& I'll go with black as my color preference. And what do you mean by form factor?\\ \hdashline
    seller& Sure, I understood your color preference. With regards to the form factor, it refers to the design and physical shape of the cable organizer. For example, some are designed like a bag with a zipper, some come in the form of a hard case, while others are like a sleeve. Could you please tell me about your brand preference? There are several well-known brands such as Bagsmart, Bubm, and Inateck.\\\hdashline
    customer& I don't have a specific brand preference, but I don't want Sunsir.\\\hdashline
    seller& I see, let's narrow down the options further. Regarding the size of the cable organizer, are you looking for a large one that can store many items, a small, compact one for easier carry, or a one-size fits all type of organizer?\\\hdashline
    customer& I'm flexible with the size.\\\hdashline
    seller& Great! Now, considering the material of the cable organizer, do you have a preference? Some people like nylon ones for the durability, while others may choose polycarbonate or plastic options for their lightweightness. Do any of these resonate with your needs?\\\hdashline
    customer& What do you mean by material type?\\\hdashline
    seller& By "material type", I meant the fabric or substance that the organizer is made from. Cable organizers can be made from a variety of materials. Here are a few examples:\\
    &1. **Nylon**: This is a type of synthetic fabric that's known for its strength and durability. It's also generally resistant to dirt and easy to clean.\\
    &2. **Polycarbonate or Plastic**: Some organizers are designed with a hard exterior for added protection. These types are lightweight, and plastic ones are particularly good for water resistance.\\
    &3. **Leather**: There are also organizers made from leather or PU leather. They offer a sleek, stylish look while also providing a good level of protection.\\
    &Each material comes with its own advantages. Nylon is usually longer-lasting, polycarbonate or plastic is lighter and waterproof, and leather gives a classy touch. Do any of these materials align with your preferences?\\\hdashline
    customer& I think I would like an organizer made of nylon for its durability. Is there any specific one you would recommend that meets my criteria? It should be black, not in the Sunsir brand, flexible in size, and made of nylon.\\\hdashline
    seller& Based on your preferences for a black, nylon-made cable organizer that's not from the Sunsir brand, I have a few suggestions for you:\\
    &1. The first is the [BAGSMART Universal Travel Cable Organizer](https://www.amazon.com/BAGSMART-Universal-Organizer-Electronics-Accessories/dp/B017SKRWL4). It's a great choice with several compartments to store your items such as a 9.7 inch iPad, Kindle, and Power Adapter. It's black with a hint of red color that gives it an edgy look.\\
    &2. The second option is [Inateck Electronics Organizer](https://www.amazon.com/Inateck-Electronics-Organizer-Splashproof-Waterproof/dp/B07ZV2CZSR). It's also a black, splashproof and waterproof travel cable organizer that should serve your needs well.\\
    &3. The third one is [Luxtude Electronics Organizer](https://www.amazon.com/Luxtude-Electronics-Organizer-Portable-Essentials/dp/B086M63FXV). \\
    &This Travel Cable Case is a portable tech bag perfect for cable storage, cord storage and storing electronic accessories/USB/SD/Charger. It's also black, aligning with your color preference.\\
    &Please let me know if you need more information about these products or if there's anything else I can assist you with.\\ \hdashline
    customer& The BAGSMART Universal Travel Cable Organizer sounds like a good fit for me. I like the number of compartments it has and the aesthetic design. I'll go ahead and add it to my cart. Thank you very much for your assistance!\\
    \hline
    \end{tabular}
    \caption{Bad conciseness example by GPT-4 interactive generation. The seller's utterances are verbose.} \label{tab:bad_conciseness}
\end{center}
\end{table*}

\begin{table*}[t]
\small
\begin{center}
%\setlength{\tabcolsep}{6pt} % Default value: 6pt
    \begin{tabular}{l p{0.82\linewidth} }
    \hline
    \textbf{Product Category} & fitted sheets \\ \hline
    \textbf{Wanted Features} & included components:1 fitted sheet; color: natural; number of pieces: 1 \\ \hline
    \textbf{Optional Features} & material; brand; fabric weight; size \\ \hline
    \textbf{Unwanted Features} & None \\ \hline 
    \end{tabular}
    \begin{tabular}{l p{0.9\linewidth} }
    \hline
    \textbf{Speaker} & \textbf{Utterance} \\
    \hline
customer& Hello there, I am interested in buying fitted sheets. Could you please recommend some for me?\\ \hdashline
seller& Could you please specify the size of the sheets you're looking for? For instance, do you need queen, twin, or full size? Also, could you share your preference for the material - would you prefer microfiber, cotton, or other fabric types?\\\hdashline
customer& I'd like one set of fitted sheets. As for the size and material, I don't have any particular preferences. However, could you please explain what you mean by "microfiber" material? I am not very familiar with this term.\\\hdashline
seller & Certainly! Microfiber is a type of synthetic material known for its softness, durability, and resistance to wrinkles and stains. \textbf{For the fitted sheets, would you prefer a pack with multiple pieces or just one sheet?}\\\hdashline
customer& Just one sheet is enough for me.\\\hdashline
seller& Given that you are looking for just one fitted sheet, would you have any color preferences or special design requirements for the sheet? Also, do you have any specific washing instructions in mind, such as machine wash or hand wash for your sheets?\\\hdashline
customer& As for the color, I would prefer something dark, maybe black or navy. As for washing, I think being machine washable would be most convenient for me.\\\hdashline
seller& I see that you have an inclination towards dark colors and machine washable fitted sheets. Moving forward, are you interested in any particular brand like Bare Home, Elegant Comfort, or Ambesonne? Also, do you have any preference for the weight of the fabric? For example, would you prefer 130, 110, or 100 grams per square meter fabric weight?\\\hdashline
customer& I don't really care about the brand or the weight of the fabric. I just need it to include one fitted sheet. Can you help me understand what "fabric weight" means?\\\hdashline
seller& Sure, I'd be happy to explain. The fabric weight is usually measured in grams per square meter (GSM). It's an indicator of the fabric's thickness and heaviness. A higher GSM means a thicker and heavier fabric. \textbf{Now, you mentioned that you only need one fitted sheet.} So, in your case, we want to ensure that the set you buy includes only one fitted sheet, instead of multiple pieces or additional items like pillowcases. Does that sound about right to you? Would you prefer a set that exclusively includes one fitted sheet, without any additional components?\\\hdashline
customer& Yes, that's right. I only need a set that includes one fitted sheet without any additional components.\\\hdashline
seller& I see that you prefer dark colors. However, I'd like to clarify if you'd only like black or navy, or if you're open to other dark colors like grey or dark blue as well? Also, would you be interested in a white color in case we don't have the dark color you prefer in stock?\\\hdashline
customer& I am open to other dark colors. However, I don't want a white one. I prefer the sheet to have a more natural look.\\\hdashline
seller& Great, I understand that you are looking for a dark-colored, machine-washable fitted sheet with a natural look. Taking into account your preferences, I'd like to recommend the BIOWEAVES 100\% Organic Cotton Twin Fitted Sheet. This sheet is made from 100\% organic cotton with a 300 thread count. It has a sateen weave, which gives it a soft, luxurious feel. The sheet comes in an undyed natural color which should fit your description of a 'natural look'. It is also GOTS certified and comes with a deep pocket design. Furthermore, it's machine washable for easy care. Priced at \$32.99, it's a great value for the quality. Would you like to proceed with this one?\\\hdashline
customer& Sure, that sounds perfect. I am happy with the BIOWEAVES 100\% Organic Cotton Twin Fitted Sheet you recommended. But I am not certain about what 'GOTS certified' and 'sateen weave' mean. Could you explain these terms before I make a purchase?\\
    \hline
    \end{tabular}
    \caption{Bad coherence and conciseness example by GPT-4 interactive generation. Bolded sentences are those the seller does not give the customer a second chance to make a value selection response and directly jumps into asking about new product aspects.} \label{tab:bad_coherence}
\end{center}
\end{table*}

\begin{table*}[t]
\small
\begin{center}
    \begin{tabular}{l p{0.82\linewidth} }
    \hline
    \textbf{Product Category} & countertop blenders \\ \hline
    \textbf{Wanted Features} & price: between \$50 and \$100; is dishwasher safe: yes; blade material: stainless steel \\ \hline
    \textbf{Optional Features} & country of origin; power source; number of speeds; department; color; voltage; style; wattage; included components; customer review; capacity\\ \hline
    \textbf{Unwanted Features} & brand: Kacsoo \\ \hline 
    \end{tabular}
%\setlength{\tabcolsep}{6pt} % Default value: 6pt
    \begin{tabular}{l p{0.9\linewidth} }
    \hline
    \textbf{Speaker} & \textbf{Utterance} \\
    \hline

customer& Hi, I'm looking to purchase a countertop blender. Can you help me out?\\\hdashline
seller& Of course! Let's start with a couple of general features. Could you specify the price range you're hoping for the blender?\\\hdashline
customer& Yes, I'd like it to be between $50 and $100.\\\hdashline
seller& Great! Now, blenders also come with different types of blades. We usually recommend stainless steel as they're the most durable. Would you like a blender with a stainless steel blade?\\\hdashline
customer& Yes, that works for me. I want my blender to have a stainless steel blade.\\\hdashline
seller& I see. There are many brands that offer these features. However, I need to confirm, would you prefer a Kacsoo blender?\\\hdashline
customer& No, not really. I'd like to consider other brands instead.\\\hdashline
seller& No problem, we have plenty of other brands to choose from. Now, did you have a preference for the country of origin for the blenders? We have blenders made in Mexico, the USA, and China.\\\hdashline
customer& Not really, the country of origin is optional for me.\\\hdashline
 seller& I understand. Let's move on to the power source. Would you like a blender that's powered by \textbf{AC, electricity, or corded electricity}?\\\hdashline
 customer& That doesn't really matter to me. It's also optional.\\\hdashline
 seller& Fair enough. And what about the number of speed settings? We have blenders with up to 16 speed settings, as well as those with fewer options.\\\hdashline
customer& That's also optional for me.\\\hdashline
seller& Alright, I understand. How about the \textbf{department} and color? Some customers prefer their blenders match their kitchen's aesthetics.\\\hdashline
customer& Both the department and the color are optional to me.\\\hdashline
seller& Got it. Moving on, the voltage and style can also vary among blenders. Do you have a preferred \textbf{voltage} and style?\\\hdashline
customer& Those aspects are optional for me.\\\hdashline
seller& Understood. Now, what about the wattage and the components included with your blender?\\\hdashline
customer& Those aspects aren't really important to me. They are also optional.\\\hdashline
seller& Noted. And lastly, how about customer reviews and capacity? Some people prefer larger blenders and highly-rated products.\\\hdashline
customer& Those are optional aspects for me as well.\\\hdashline
seller& Thank you for all this information. Our blenders also come with different dishwasher safety levels. Is a dishwasher safe blender important to you?\\\hdashline
customer& Yes, I'd like my blender to be dishwasher safe.\\\hdashline
seller& Alright, considering all your preferences, I would like to recommend the "Oster 6812-001 Core 16-Speed Blender with Glass Jar, Black". It's priced at \$59.68, has a durable stainless steel blade, and also dishwasher safe. The other aspects that you mentioned as optional all meet your standards as well.\\\hdashline
customer& That sounds perfect. Thank you for your help.\\
    \hline
    \end{tabular}
    \caption{Bad feature example by GPT-4 single-pass generation. The inappropriately selected product aspects in bold make the conversation awkward.} \label{tab:bad_feature}
\end{center}
\end{table*}

\begin{table*}[t]
\small
\begin{center}
%\setlength{\tabcolsep}{6pt} % Default value: 6pt

    \begin{tabular}{l p{0.82\linewidth} }
    \hline
    \textbf{Product Category} & cooling pads \\ \hline
    \textbf{Wanted Features} & manufacturer: Kootek \\ \hline
    \textbf{Optional Features} & None\\ \hline
    \textbf{Unwanted Features} & color: white \\ \hline 
    \end{tabular}

    \begin{tabular}{l p{0.9\linewidth} }
    \hline
    \textbf{Speaker} & \textbf{Utterance} \\
    \hline
customer& Hi, I'm looking for a cooling pad for my laptop.\\\hdashline
seller& Great! Let's start by narrowing down the options. Can you tell me what kind of manufacturer are you interested in?\\\hdashline
customer& Umm...I'm not sure. I've heard of Kootek, is that a good brand?\\\hdashline
seller& Absolutely! Kootek is one of our best-selling brands for cooling pads. Let's focus on their products for now. How about the color? Do you have a preference?\\\hdashline
customer & Ah, well...I don't want white. Is that an option?\\\hdashline
seller& Yes, we have cooling pads in different colors. How about red or blue?\\\hdashline
customer & Hmm...I think I like the sound of \textbf{blue}.\\\hdashline
seller & Great! Now, let's talk about \textbf{size}. What size is your laptop?\\\hdashline
customer & It's a \textbf{15-inch laptop}.\\\hdashline
seller & Perfect! We have a Kootek cooling pad that fits 12-17 inch laptops. Would you like to have \textbf{adjustable mounts for the laptop stand height angle}?\\\hdashline
customer & Sure, that sounds useful.\\\hdashline
seller & Great! Do you need any extra features like \textbf{USB ports or LED lights}?\\\hdashline
customer & LED lights would be cool, but I'm not sure about USB ports. Can you explain what they do?\\\hdashline
seller & Of course! USB ports allow you to connect other devices to your laptop while it's on the cooling pad. For example, you could plug in a mouse or an external hard drive.\\\hdashline
customer & Oh, that's handy. Yeah, I think I want USB ports then.\\\hdashline
seller & Great! Lastly, would you like a chill mat that has a \textbf{quiet fan or a more powerful fan}? Or would you prefer a balance between the two?\\\hdashline
customer & Hmm...I don't know much about fans. Can you explain the difference?\\\hdashline
seller & Sure thing! A quieter fan will provide a gentle breeze and might be better suited for everyday use, while a more powerful fan will provide a stronger airflow and might be better for gaming or heavy video editing.\\\hdashline
customer & Oh, I see. I think I'll go with the \textbf{balance between the two} then.\\\hdashline
seller & Excellent choice! Based on your preferences, I recommend our Kootek Laptop Cooling Pad 12"-17" Cooler Pad Chill Mat 5 Quiet Fans LED Lights and 2 USB 2.0 Ports Adjustable Mounts Laptop Stand Height Angle, Blue. It fits your budget and has all the features you're looking for. Would you like to take a look at it?\\\hdashline
customer & Yes, that sounds great. Can I get a link to the product page?\\\hdashline
seller & Of course! Here's the link. Let me know if you have any other questions or need further assistance.\\\hdashline
customer & Thanks for your help! I think I'm good to go.\\\hdashline
seller & You're welcome! Enjoy your shopping experience, and don't hesitate to reach out if you need anything else! \\
    \hline
    \end{tabular}
    \caption{Not instruction-following example by LLaMA2-70B-Chat single-pass generation. The only assigned required feature is the manufacturer of Kootek, and the only unwanted feature is white color.} \label{tab:not_instruction_following}
\end{center}
\end{table*}

