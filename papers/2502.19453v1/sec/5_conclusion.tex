\section{CONCLUSION}
We developed a benchmark to evaluate the performance of instruction-based editing of SVG images, SVGEditBench V2. We used LLMs and LMMs as SVG editing techniques and conducted comparative experiments. The results show that they understood the basic grammar of SVG and could sometimes understand the images through code. In those cases, they could successfully edit the graphics according to the text prompt. However, the experiments indicated that the editing ability was far from satisfactory. Qualitative analysis showed that the main issue lies in generating or editing \texttt{<path>} elements since elaborate control of numbers is critical.

Considering these conclusions, future SVG editing models should focus on fine-grained control of numbers. Language models for SVG processing might need an assistive module for error-free and high-quality path generation. Vector graphics are abundant in real-world situations, and research on text-to-vector is desirable. We hope SVGEditBench V2 facilitates research in this area and helps ease vector graphics processing for both professionals and novices.