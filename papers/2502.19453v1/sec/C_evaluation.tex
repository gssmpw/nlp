\section{DETAILS ON MODEL EVALUATION}
This section explains additional details of the model evaluation pipeline that were not in the main paper.
\subsection{Prompt Used for Inferencing LLMs for SVG Editing}
\label{subsec:prompt-evaluation}
We used the following prompt to infer the LLMs/LMMs when performing SVG editing. We kept this simple, aiming to evaluate the pure ability of LLMs/LMMs to process SVG. 

\begin{itembox}[l]{The prompt used for evaluating LLMs/LMMs}
\begin{lstlisting}
The following is the SVG code representing an image. <b#(Instruction)#> Only return the output SVG code.

```svg
<b#(SVG code before editing)#>
```
\end{lstlisting}
\end{itembox}

\subsection{Extracting SVG code from the LLM output}
\label{subsec:code-extraction}
Using the following algorithm, we extracted the SVG code from the LLM/LMM output. Note that we did not look into the content of the code at this step.
\begin{enumerate}
    \item We look for parts surrounded with \texttt{```svg} and \texttt{```}. We regard the output as an ``Extraction'' error if there are multiple code segments. If there is only one segment, we regard this code segment (excluding \texttt{```svg} and \texttt{```}) as the output SVG code.
    \item If there are no code segments in the previous step, we look for text surrounded with \texttt{<svg} and \texttt{</svg>}. We consider the text (including \texttt{<svg} and \texttt{</svg>}) as the SVG output if there is only one segment and as an ``Extraction'' error if otherwise.
\end{enumerate}

\subsection{Definition of the two-step Chamfer distance}
The following is the definition of the Chamfer distance metric used in the model evaluation pipeline. Let $X$ be the output image. We assume $X$ as a set of shapes $A$. Therefore $X = \{A_1, A_2, \dots\}$. Also, we assume each shape $A$ as a set of two-dimensional points. Therefore $A=\{\bm{a}_1,\bm{a}_2, \dots \}$, where $\bm{a} \in \mathbb{R}^2$. We obtain these points by equally sampling the contour of each shape ($|A|=100$). The ground truth image $Y=\{B_1, B_2, \dots\}$ can be defined similarly. Given this, the distance between $X$ and $Y$ is defined as follows. Firstly, we define the distance between shapes as 
\newcommand{\mshape}{\mathrm{shape}}
\begin{align}
\begin{autobreak}
    d_\mshape\left(A, B \right)
    = \frac{1}{|A|}\sum_{\bm{a} \in A}\min_{\bm{b} \in B}\left\| \bm{a} - \bm{b} \right \|_2^2 
    + \frac{1}{|B|}\sum_{\bm{b} \in B}\min_{\bm{a} \in A}\left\|\bm{a} - \bm{b}\right\|_2^2 .
\end{autobreak}
\end{align}
Then, we define the distance between the two images as
\begin{align}
\begin{autobreak}
     d_\mathrm{image}\left(X, Y\right)
     = \frac{1}{|X|}\sum_{A \in X}\min_{B\in Y}d_\mshape\left(A, B\right) 
     + \frac{1}{|Y|}\sum_{B \in Y}\min_{A \in X} d_\mshape\left(A, B\right).
\end{autobreak}
\end{align}
