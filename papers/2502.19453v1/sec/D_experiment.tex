\section{EXAMPLE OUTPUTS IN THE EXPERIMENT}
\begin{figure*}
  \includegraphics[width=\hsize]{img/examples-no-metrics.pdf}
  \caption{Examples of current LLMs and LMMs performing the editing tasks in SVGEditBench V2. The first column shows successful examples, and the two columns on the right show failure cases.}
  \label{fig:examples}
\end{figure*}

We show some samples of actual SVG output from the LLMs/LMMs in the experiment. The two instances on the left of \Fref{fig:examples} are from the successful cases mentioned in Sec. 4.4. We introduce more patterns of failures in the subsequent sections.

\subsection{Cases of ``Extraction'' errors}
\newsavebox\repetitionbox
\newsavebox\noclosetagbox
\newsavebox\ignoringbox

\begin{lrbox}{\repetitionbox}
  \begin{minipage}{0.96\linewidth}
    \begin{lstlisting}
```svg
<?xml version="1.0" encoding="utf-8"?>
<b#...#>
<path style="fill:#fff" d="M10,10l5,5 5,-5 5,<b# ... #>5 5,-5 5,5 5,-5 5,5 5,-5 5,5 5,-5 5,5 5,-5 5,5 5,-5 5,5 5,-5 5,5 5,-5 5,5 5,-5 5,5 5,-5 5,5 5,-5 5,5 5,-5 5,\end{lstlisting} %gemini-1.5-flash/280
    \end{minipage}
\end{lrbox}

\begin{lrbox}{\noclosetagbox}
  \begin{minipage}{0.96\linewidth}
    \begin{lstlisting}
<svg width="32" height="32" viewBox="0 0 32 32" fill="none" xmlns="http://www.w3.org/2000/svg">
<b#...#>
<path d="M14.6131 7.80203C15.5509 8.03583 16.1215 8.9856 15.8877 <b#...#> 13.6753 7.56822 14.6131 7.80203Z" fill="#FFFFFF"/>\end{lstlisting} %llama3.2-3b-instruct/425
  \end{minipage}
\end{lrbox}

\begin{lrbox}{\ignoringbox}
  \begin{minipage}{2\linewidth}
  \begin{lstlisting}
To add a sandy texture to the umbrella's base, we need to modify the "fill" attribute of the base path element (the fourth path in your SVG code). Instead of a solid color, we will use a pattern to create the sandy texture.

First, let's create an SVG pattern element for the sandy texture. We'll use two colors (light brown and darker brown) and some noise to simulate sand.

```svg<b#...#>```

Now, update the base path to use the sandy-texture pattern:

```svg<b#...#>
  \end{lstlisting} %mistral-7b-instruct-v0.3/2158
\end{minipage}
\end{lrbox}

\begin{figure*}
  \centering
  \subfloat[An example of repetition (Gemini 1.5 Flash)]{\usebox\repetitionbox \label{code:repetition}
  }
  \hspace{0.04\linewidth}
  \subfloat[An example of ending generation with the inner end tag (Llama 3.2)]{\usebox\noclosetagbox \label{code:no-close-tag}} \\
  \subfloat[An example of ignoring the prompt (Mistral v0.3)]{\usebox\ignoringbox   \label{code:ignorant}}
  \caption{The actual output from the LLM/LMM with ``Extraction'' errors. We omitted parts of the output with three dots (\textbf{\dots}).}
  \label{fig:code-extraction-errors}
\end{figure*}
We show the output examples of the three ``Extraction'' error types in \Fref{fig:code-extraction-errors}. Note that SVG is a special type of XML, and SVG code should also comply with XML grammar. We explained the case of \Fref{fig:code-extraction-errors}\subref{code:repetition} in Sec 4.4. In this example, Gemini 1.5 Flash repeatedly generated \texttt{5 5,-5 5,}. Therefore, the generation ended before the model outputted the end tag (\texttt{</svg>}), making the code grammatically incorrect. This phenomenon was especially significant in open-sourced models. The code in \Fref{fig:code-extraction-errors}\subref{code:no-close-tag} ended with the close tag of the inner \texttt{<path>} tag, while it should finish with the end tag of the root element (\texttt{</svg>}). This code is also incorrect as XML and regarded as an ``Extraction'' error. \Fref{fig:code-extraction-errors}\subref{code:ignorant} shows an example of an LLM ignoring the prompt \texttt{Only return the SVG code}. Here, Mistral v0.3 explained the editing procedure and generated an SVG code snippet for each editing step. The output would be an ``Extraction'' error with our algorithm since this contained multiple code segments.


\subsection{Cases of ``Parsing'' errors}
\newsavebox\namespacebox
\newsavebox\attributebox
\newsavebox\commandbox
\newsavebox\argumentbox

\begin{lrbox}{\namespacebox}
  \begin{minipage}{0.96\linewidth}
    \begin{lstlisting}
<svg xmlns="http://www.w3.org/2000/svg" viewBox="0 0 36 36" style="background-color:purple; border-radius:50%;">
  <b#...#>
  <use <o#xlink:href="#reflector"#>/>
  <g>
    <b#...#>
  </g>
</svg>\end{lstlisting} %phi-3.5-mini/1610
\end{minipage}
\end{lrbox}

\begin{lrbox}{\attributebox}
  \begin{minipage}{0.96\linewidth}
    \begin{lstlisting}
<svg xmlns="http://www.w3.org/2000/svg" viewBox="0 0 36 36"><path <o#fill="#FFC5C5"#> d="M27.843 11.9c.205-.371.39-.766<b#...#>C29.519 20.484 30 18.943 30 17.462zM23.688 1.373c0-1.011-3.808-1.159-5.972-1.164H17.6c-2.164.005-5.972.152-5.972 1.164 0 2.013 3.796 2.942 5.887 2.982h.285c2.093-.041 5.888-.97 5.888-2.982z" <o#fill="#272B2B"#>/><b#...#></svg>\end{lstlisting} %llama2-7b/672
  \end{minipage}
\end{lrbox}

\begin{lrbox}{\commandbox}
  \begin{minipage}{0.96\linewidth}
    \begin{lstlisting}
<svg xmlns="http://www.w3.org/2000/svg" viewBox="0 0 36 36">
  <b#...#>
  <path fill="none" d="M35.885 <b#...#> 1.204-2.708 2.708z<o#F#>28.933 14.958a.064.064 0 0 1 .024.088.064.064 0 0 1 -.024-.088z">
</path>
</svg>
\end{lstlisting} %Codegemma/2636
  \end{minipage}
\end{lrbox}

\begin{lrbox}{\argumentbox}
  \begin{minipage}{0.96\linewidth}
    \begin{lstlisting}
<svg width="32" height="32" viewBox="0 0 32 32" fill="none" xmlns="http://www.w3.org/2000/svg">
<path <o#transform="matrix(-1 0.448,0 0.838,0 0.838-15)"#> d="M6 1C3.23079 1 1 3.23079 1 6V26C1 28.7614 3.23079 31 6 31H26C28.7614 31 31 28.7614 31 26V6<b#...#>" fill="#212121"/>
</svg>\end{lstlisting} %llava-next/1130
  \end{minipage}
\end{lrbox}

\begin{figure*}
  \centering
  \subfloat[An example of output code using an undefined namespace. The output code is from Phi-3.5-mini.]{\usebox\namespacebox \label{code:undefined-namespace}
  }
  \hspace{0.04\linewidth}
  \subfloat[An example of the same attribute defined multiple times in the same element. The output code is from Llama 2.]{\usebox\attributebox \label{code:multiple-attributes}} \\
  \subfloat[An example of unknown line command included in the \texttt{d} attribute. The output code is from CodeGemma.]{\usebox\commandbox \label{code:unknown-command}}
  \hspace{0.04\linewidth}
  \subfloat[An example of the same attribute defined multiple times in the same element. The output code is from Llama 2.]{\usebox\argumentbox \label{code:too-many-arguments}}
  \caption{Examples of ``Parsing'' errors. We omitted parts of the output with three dots (\textbf{\dots}).}
  \label{fig:code-parsing-errors}
\end{figure*}
For the ``Parsing'' errors, we point out two patterns of failures in SVG rasterization or Chamfer distance calculation. \Fref{fig:code-parsing-errors}\subref{code:undefined-namespace} is an example of an XML error using an undefined namespace. In XML, namespaces (\texttt{xlink} in this case) should be defined by specifying the URI(\url{http://www.w3.org/1999/xlink}, \etc). However, the definition is not in the code in \Fref{fig:code-parsing-errors}\subref{code:undefined-namespace}. \Fref{fig:code-parsing-errors}\subref{code:multiple-attributes} is another XML error. Here, the code specifies the same attribute (\texttt{fill}) in the same \texttt{<path>} element. This is not allowed in XML and, therefore, regarded as an ``Parsing'' error.

The following two examples are errors in SVG. \Fref{fig:code-parsing-errors}\subref{code:unknown-command} is an example where an unknown line command exists within the \texttt{d} attribute of a \texttt{<path>} element. The SVG specification does not define the command \texttt{F} in the \texttt{d} attribute. Therefore, we could not recognize the accurate contour of the shape. The code in \Fref{fig:code-parsing-errors}\subref{code:too-many-arguments} contains an error in the \texttt{transform} attribute. The \texttt{matrix} command of the \texttt{transform} attribute allows the shape to rotate and scale simultaneously. The code specifies seven arguments, while the \texttt{matrix} command takes only six. Therefore, the pipeline could not rasterize the code, and we could not calculate the metrics.

\subsection{Examples of editing errors}
We could observe the following two characteristics of SVG editing with LLMs/LMMs. These editing failures are shown on the right side of \Fref{fig:examples}.
\begin{enumerate}
  \item The LLMs/LMMs tend to avoid modifying paths, even if the prompt asked them to change the object's shape. They may only rotate the shapes or add text directly to the image. The top-center instance in \Fref{fig:examples} is a sample. This task asked Gemini 1.0 Pro to change the ``7'' shape into a ``9'' (prompt: \texttt{Change the number "7" to "9" on the keycap.}). The model compromised by adding a ``9'' in text on top of the existing ``7'' shape.
  \item Processing \texttt{<path>} elements is very likely to fail. LLMs/LMMs only generate simple polygons or collapse the image by inappropriate parameter adjustments. For instance, in the top-right example in \Fref{fig:examples}, the task only requested LlaVa-NeXT to change the background color (prompt: \texttt{Change the background color to blue.}). Nevertheless, the LMM 	changed the ``UP!'' shape unnecessarily. This modification made the text unreadable. They also find it difficult to position those new elements aligning with the original image (\eg the bottom-center example in \Fref{fig:examples}).
\end{enumerate}

