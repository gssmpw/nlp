\section{Related work}
\label{sec:relatedwork}

Garcia-Salinas et al.~\cite{garcia2019transfer} recorded EEG data from 27 participants
while they silently spoke five words (\textsc{Select}, \textsc{Right}, \textsc{Left}, \textsc{Down}, and \textsc{Up}). 
A Naive Bayes classifier trained on a subset of words and tested on all five achieved 68.93\% accuracy.
Lee et al.~\cite{lee2020eeg} investigated common spatial and temporal features 
between overt and imagined speech from seven participants and 12 words, 
achieving a 59.9\% accuracy for overt speech and 16.2\% for imagined speech, both above chance levels. 
They suggested potential for imagined speech classification based on overt speech features 
but did not attempt a direct transfer. 
Watanabe et al.~\cite{watanabe2020synchronization} examined whether 
EEG acquired during speech perception and imagination shared a signature envelope with EEG from overt speech. 
Their study, involving 18 participants and three words, showed that classifiers trained on imagined speech EEG envelopes 
could achieve 38.5\% accuracy when tested on overt speech envelopes. 
These findings reinforce the similarity between the two paradigms.

Komeiji et al.~\cite{komeiji2024feasibility} developed a Transformer-based model 
that decoded sentences from covert speech using ECoG signals. 
They outperformed a BiLSTM model in both overt and covert speech tasks. 
Interestingly, training the model on overt speech achieved similar performance for decoding covert speech 
compared to training on covert speech itself. 
This is a significant finding, as it bypasses the difficulty of collecting covert speech data. 
Surprisingly, to the best of our knowledge, no research has yet attempted 
a direct transfer of a classifier from overt to imagined speech. 
Our work therefore addresses this longstanding research gap, 
offering a more productive training process while maintaining high accuracy.