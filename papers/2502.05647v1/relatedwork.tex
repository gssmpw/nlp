\section{Related Work}
In the context of single-cell data, imputation is a common step to handle missing values. Various algorithms are employed for this purpose, and one effective approach is using autoencoders\cite{autoimpute1}. Autoencoders learn the underlying structure of the data and predict missing values based on the available information.  Dimensionality reduction methods play a crucial role in single-cell analysis. PCA is commonly employed to reduce dimensionality by calculating the principal components. These components capture the most significant variation in gene expression across cells, allowing for efficient representation and visualization of the data\cite{pcaapplication}. The Uniform Manifold Approximation and Projection (UMAP) method is commonly employed to reduce dimensionality and visualize single-cell data\cite{umapapplication}. t-Distributed Stochastic Neighbor Embedding (t-SNE) is a dimensionality reduction technique commonly used in single-cell RNA sequencing analysis while preserving local structure\cite{tsneapplication}. Autoencoder is also used to reduce dimensions\cite{autoencoderapplication}. Data partitioning is a fundamental technique in machine learning, particularly when dealing with large datasets\cite{datap1}. In certain scenarios involving large datasets distributed across multiple servers, the data is divided into segments. Each segment is processed independently on its respective server\cite{d1}. Within each segment, dimensionality reduction techniques are applied to reduce the number of features while preserving relevant information. After dimensionality reduction, the reduced data from all segments is combined, serving as a proxy for the original dataset\cite{d2}. This combined reduced data is then used for clustering purposes. When dealing with large datasets and distributed clustering of high-dimensional, heterogeneous data, a technique called Collective PCA can be employed. Collective PCA is specifically designed for distributed scenarios and can be used independently of clustering algorithms. It aims to reduce the dimensionality of the data while preserving essential information\cite{d3}.