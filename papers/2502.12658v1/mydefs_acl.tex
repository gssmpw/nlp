% ===== import packages =====
% to be able to draw some self-contained figs

\usepackage[noend]{algpseudocode}
\usepackage{amssymb}
\usepackage{amsthm}
\usepackage{tikz}
\usepackage{amsmath, amsfonts}
\usepackage{mathtools}
\usepackage{graphicx}
\usepackage{textcomp} % special symbols
\usepackage[T1]{fontenc}
\usepackage[utf8]{inputenc}
\usepackage{xspace}
% \usepackage{xcolor}
\usepackage{multirow}
\usepackage{diagbox}
\usepackage{booktabs}
\usepackage{caption}
\usepackage{subcaption}
\usepackage{float}
\usepackage{makecell}
\usepackage{paralist}
\usepackage{enumitem}
\usepackage[skins]{tcolorbox}
\usepackage[ruled,lined,linesnumbered]{algorithm2e}
\usepackage{algorithmicx}
\usepackage{pifont}
% set hyerref color
\usepackage[pagebackref=true,breaklinks=true,colorlinks,bookmarks=false]{}
\usepackage{hyperref}
\hypersetup{
  colorlinks,
  linkcolor={PineGreen},
  citecolor={RoyalBlue},
  urlcolor={Bittersweet}
}
\definecolor{mycolor}{RGB}{246, 196, 188}
\definecolor{codegray}{rgb}{0.4, 0.4, 0.4}
\definecolor{cautioncolor}{RGB}{255,0,0}
\setlength{\fboxsep}{2pt}

\usepackage{soul}
\sethlcolor{mycolor}
\newcommand{\hlcolor}[2]{
  \sethlcolor{#1}
  \hl{#2}
  \sethlcolor{mycolor}
}
\colorlet{lightgray}{gray!20}

% Set the gap between the chart and the caption
% \setlength{\abovecaptionskip}{0.1cm}

% Set the subgraph label format
\captionsetup[subfigure]{labelformat = parens, labelsep = space, font = small}

% Set the left margin of itemize
% \setlist[itemize]{leftmargin=*}
\newenvironment{icompact}{
  \begin{list}{$\bullet$}{
    \itemindent -.05em
    \parsep 0pt plus 1pt
    \partopsep 0pt plus 1pt
    \topsep 2pt plus 2pt minus 2pt
    \itemsep 0pt plus 1.3pt
    \parskip 0pt plus 2pt
    \leftmargin 0.13in}
      }
{\normalsize
\end{list}
}

% Set comment style for algorithm2e
\newcommand\mycommfont[1]{\ttfamily\textcolor{green!50!black}{#1}}
\SetCommentSty{mycommfont}

% Set autoref
\renewcommand*{\sectionautorefname}{Section}
\renewcommand*{\subsectionautorefname}{Section}
\renewcommand*{\algorithmautorefname}{Algorithm}
\renewcommand*{\algorithmcflinename}{Line}
\newcommand{\appref}[1]{\hyperref[#1]{Appendix~\ref*{#1}}}

% Set equation number
% \mathtoolsset{showonlyrefs=true}

% ===== define new commands =====
\def\X#1{\ding{\numexpr181+#1}}

\definecolor{redbg}{RGB}{254,241,240}
\definecolor{redoutline}{RGB}{252,163,152}
\definecolor{redtext}{RGB}{207,24,34}
\DeclareRobustCommand*{\rectangled}[1]{%
  \tikz[baseline=(char.base)]\node[anchor=south west, draw, rectangle, thick, rounded corners=0.2mm, inner sep=2pt, fill=redbg, draw=redoutline,text=redtext](char){#1} ;}

% revision
\definecolor{revision}{RGB}{0,0,255}
\newcommand{\rev}[1]{{\color{revision} #1\xspace}\xspace}
\newcommand{\revstart}{\begin{color}{revision}}
\newcommand{\revend}{~\!\!\end{color}}


% mypara
% \renewcommand{\smallskip}{\vspace{2pt plus 0.5pt minus 0.5pt}}
\newcommand{\mypara}[1]{\smallskip \noindent\textbf{#1.} \xspace}
\newcommand{\myparatight}[1]{\noindent\textbf{#1.} \xspace}

% metric
\newcommand{\acc}{\ensuremath{\mathsf{ACC}}\xspace}

% macro
% \newcommand{\method}{\ensuremath{\mathsf{GradEscape}}\xspace}
\newcommand{\method}{\ensuremath{\mathsf{R.R.}}\xspace}
\newcommand{\aigt}{{AIGT}\xspace}
\newcommand{\spbase}{{P2P}\xspace}
\newcommand{\sptoken}[1]{{\fontfamily{qcr}\selectfont#1}}
\newcommand{\block}[1]{%
  \raisebox{\dimexpr(\fontcharht\font`X-1em)/2}{\rule{0.5em}{#1\dimexpr1em/8}}%
}
\DeclareUnicodeCharacter{2581}{\block{1}}
\DeclareUnicodeCharacter{2582}{\block{2}}
\DeclareUnicodeCharacter{2583}{\block{3}}
\DeclareUnicodeCharacter{2584}{\block{4}}
\DeclareUnicodeCharacter{2585}{\block{5}}
\DeclareUnicodeCharacter{2586}{\block{6}}
\DeclareUnicodeCharacter{2587}{\block{7}}
\DeclareUnicodeCharacter{2588}{\block{8}}

% \newtcolorbox{mybox}[2][]{text width=0.95\linewidth,fontupper=\normalsize,
% fonttitle=\bfseries\sffamily\normalsize, colbacktitle=codegray,enhanced,
% boxed title style={sharp corners},top=4pt,bottom=2pt,left=2pt,right=2pt,
%   title=#2,colback=white}

\newtcolorbox{mybox}[2][]{text width=\textwidth, fontupper=\normalsize,
fonttitle=\bfseries\sffamily\normalsize, colbacktitle=codegray, enhanced,
boxed title style={sharp corners}, top=4pt, bottom=2pt, left=2pt, right=2pt,
title=#2, colback=white}


%%%%%%%%%%%%%%%%%%%%%%%%%%%%%%%%
% THEOREMS
%%%%%%%%%%%%%%%%%%%%%%%%%%%%%%%%
\theoremstyle{plain}
\newtheorem{theorem}{Theorem}[section]
\newtheorem{proposition}[theorem]{Proposition}
\newtheorem{lemma}[theorem]{Lemma}
\newtheorem{corollary}[theorem]{Corollary}
\theoremstyle{definition}
\newtheorem{definition}[theorem]{Definition}
\newtheorem{assumption}[theorem]{Assumption}
\theoremstyle{remark}
\newtheorem{remark}[theorem]{Remark}


% remove algorithm2e margin
\usepackage{etoolbox} % For \patchcmd

\makeatletter
\patchcmd{\algocf@makecaption@ruled}{\hsize}{\textwidth}{}{} % Caption to stretch full text width
\patchcmd{\@algocf@start}{-1.5em}{0em}{}{} % For // to right margin
\makeatother

% \SetAlCapHSkip{0pt} % Reset left skip of caption
% \setlength{\algomargin}{0em}


