\section{Related Work}
In this section, we will summarize the existing works.

\head{Cardiac Monitoring}
Cardiac information is crucial for health monitoring, affective computing \cite{yang2022survey, fairclough2020personal} and deception analysis \cite{bian2024ubihr}. 
Traditional approaches in hospitals, \eg, electrocardiograms (ECGs) and CT scans \cite{HeartDiseaseSymptoms}, provide the most accurate data but require professional operation and are prohibitively expensive and cumbersome for everyday use. Recent advancements have focused on more portable solutions.
Earable-based systems \cite{cao2023heartprint, chen2024exploring, fan2023apg} allow earpieces to detect cardiac information, but they either need specific probing signals or custom hardware, limiting their widespread adoption. Similarly, wearable solutions necessitate constant wear, which is not practical for all users. Wireless technologies, including Wi-Fi \cite{liu2015tracking}, mmWave \cite{yang2016monitoring}, and UWB \cite{chen2021movi}, \etc, are constrained by specific hardware which is not commonly available in video systems . 
Solutions using active acoustic sensing \cite{wang2023df, wang2022loear, qian2018acousticcardiogram, zhang2020your} with smart speakers rely on pseudo-inaudible signals, which can be intrusive to human hearing and increase hardware burden.
Video-based solutions use optical means to measure blood volume changes in tissues. Signal processing \cite{de2013robust, li2014remote, wang2016algorithmic, wang2015novel} and deep learning \cite{chen2018deepphys, liu2020multi, niu2020video, li2023learning, yu2019remote,yu2019remoteCompress, yu2023physformer++, liu2023efficientphys, zou2024rhythmformer} techniques have been developed to enhance these methods. Yet these solutions are sensitive to low light conditions, head/body movements, and typically perform poorly outside controlled environments. VocalHR \cite{xu2022hearing} proves the potential of extracting heart rate from human speech. Although it leverages human speech effectively, it is limited by range, requires pre-calibration, and cannot distinguish multiple individuals. 
Differently, \sysname is the first to combine the complementary and naturally co-existing audio and video modalities in online video streaming systems. Our video design incorporates temporal-frequency co-design and motion-aware aggregations for the first time in OCM to mitigate the light and body movement influence. The audio module employs the temporal acoustic filter for OCM. These designs are innovative and contribute to our performances.

\head{Video Streaming System} Video streaming systems have gained immense popularity due to their vast libraries of on-demand content, user-generated videos, and live streaming capabilities, catering to diverse viewer preferences, including YouTube, TikTok, Zoom, \etc. They can be further categorized into VoD systems, live streaming systems and video conferencing systems. Research efforts have been devoted to communication protocols \cite{hamadanian2023ekho, dhawaskar2023converge}, adaptive rate streaming algorithms  \cite{li2023dashlet, wen2023adaptivenet, zhou2019learning}, online learning \cite{tang2023lut, guan2023metastream, khani2023recl, yi2023boosting}, \etc. None of these works explore adding cardiac monitoring into modern video streaming systems. 
In contrast, \sysname stands out as the first work that creates a middleware service of OCM that can be seamlessly integrated into mainstream video streaming systems.