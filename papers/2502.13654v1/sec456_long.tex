\section{FPT algorithm for delta-matroid triangle cover} \label{sec:triangle-cover}


% \paragraph*{Background on triangle matching problems.}

% Given a graph, the problem of finding a vertex-disjoint $\{ K_2, K_3 \}$-packing covering the largest number of vertices is polynomial time solvable \cite{HellK84}.
% We can consider two variants of this problem (both are NP-hard because they generalize the triangle cover problem).

% \begin{enumerate}
%   \item Given a graph $G$ and $k \in \mathbb{N}$, find $k$ vertex-disjoint triangles $T_1, \cdots, T_k$ such that $G - T_1 \cdots - T_k$ has a perfect matching.
%   \item Given a graph $G$ and $k \in \mathbb{N}$, find a vertex-disjoint $\{ K_2, K_3 \}$-packing covering at least $k$ edges.
% \end{enumerate}

% These problems seem to be related to an open question in the literature \cite[Open Problem 5.4]{abs-2001-06867}, \cite{GolovachHKLP20}:
% \textsc{Strong Triadic Closure} is the problem where, given a graph $G$ and $k \in \mathbb{N}$, we want to find a set $F$ of $k$ edges such that for every pair of incident edges $e = uv, e' = uw \in F$, the edge $vw$ exists in $G$ (not necessarily in $F$).
% \textsc{Cluster Deletion} is the problem where, given a graph $G$ and $k \in \mathbb{N}$, we want to find a set $F$ of $k$ edges such that $(V(G), F)$ is a cluster graph.
% For both problems, any matching would constitute a solution, i.e., $k \ge \mu(G)$ for nontrivial instances, where $\mu(G)$ is the size of maximum matching.
% It is open whether \textsc{Strong Triadic Closure} and \textsc{Cluster Deletion} are FPT when parameterized by $k - \mu(G)$.

% In these problems, triangles play a key role:
% A $K_4$ has 4 vertices and 6 edges, so adding it to the solution gives an excess of 2.
% Thus, if there are many vertex-disjoint $K_4$'s, then we obtain a solution.
% Although finding $K_4$-packing is NP-hard, we can compute a small set that intersects all $K_4$'s using a greedy algorithm.
% Without $K_4$, all that can increase the solution over a maximum matching are triangles (for \textsc{Cluster Deletion}).

% \begin{theorem}[Basis sieving \cite{EKW23}] \label{ithm:simple}
%   Let $P(X)$
%   be a polynomial of degree $d$ over a field $\F$ of characteristic 2, 
%   %over a set of variables $X=\{x_1,\ldots,x_n\}$,
%   and let $M=(X,\cI)$ be a matroid on $X$ of rank $k$, represented by a
%   matrix $A \in \F^{k \times X}$. There is a randomized algorithm
%   with running time $O^*(d2^k)$ that tests if there is a multilinear
%   term $m$ in the monomial expansion of $P(X)$ such that the matrix
%   $A[\cdot,\supp(m)]$ is non-singular.
%   The algorithm uses polynomial space, needs only evaluation access to $P$,
%   has no false positives and produces false negatives with probability
%   at most $2k/|\F|$. 
% \end{theorem}

% We cannot hope to have a similar algorithm for delta-matroids.\todo{This passage should be moved earlier}

A triangle packing is a collection of vertex-disjoint triangles $\mathcal{T} = \{ T_1, \cdots, T_{p} \}$.
We denote $V(\mathcal{T}) = \bigcup_{T \in \mathcal{T}} T$.
We say that $\mathcal{T}$ covers $S$ if $S \subseteq V(\mathcal{T})$.
We will solve the following problem. Given a linear delta-matroid $D$
over ground set $V$, a collection $\cT_0 \subseteq \binom{V}{3}$ of
triples (triangles) from $V$, and an integer $k$,
find a feasible set $F \subseteq V$ with $|F|=k$ such that $F$ can be
covered by a packing of triangles from $\cT_0$.
We call this problem \textsc{Delta-matroid Triangle Cover}.
As shown in Theorem~\ref{thm:ddd-hard}, the problem would be W[1]-hard if we insisted that $F$ must
be precisely partitioned into triangles. But here, we allow triangles
to have anywhere between 1 and 3 members in $F$.

The algorithm uses a mixture of algebraic and combinatorial arguments.  
We start with the algebraic side, where we introduce a new delta-matroid operation,  
termed \emph{$\ell$-projection}, that plays a crucial role in our approach.

\subsection{Delta-matroid $\ell$-projection}

We introduce an operation called \emph{$\ell$-projection} for delta-matroids,  
which generalizes the classical concept of matroid truncation.  
This new operation is of independent interest and may also be useful for kernelization  
(see, e.g., \cite{Wahlstrom24SODA}).
For a matroid $M = (V, \cI)$ and $k \in \N$, the \emph{$k$-truncation} of $M$ is a matroid $(V, \cI_k)$, where $\cI_k = \{ I \in \cI \mid |I| \le k \}$.
This operation is fundamental in the design of many (parameterized) algorithms for matroid problems.
See Marx~\cite{Marx09-matroid} and Lokshtanov et al.~\cite{LokshtanovMPS18TALG} for randomized and deterministic algorithms for truncation, respectively.

However, the situation is not as straightforward for delta-matroids.
A natural idea would be to define the $k$-truncation of a delta-matroid $D = (V, \cF)$ as $(V, \cF_k)$, where $\cF_k = \{ F \in \cF \mid |F| \le k\}$.
However, this does not always yield a delta-matroid, e.g., let $D = (V, \cF)$ be the (matching) delta-matroid, where $V = \{ a, b, c, d \}$ and $\cF = \{ \emptyset, \{ a, b \}, \{ c, d \}, \{ a, b, c, d \} \}$.
For $k = 2$, we obtain $\cF_2 = \{ \emptyset, \{ a, b \}, \{ c, d \} \}$, which fails the exchange axiom:
For $F = \{ a, b \}$ and $F' = \{ c, d \}$, there is no element $x \in F \Delta F'$ such that ${F \Delta \{ c, x \} \in \cF_2}$.

To overcome this issue, we introduce a more nuanced operation called \emph{$\ell$-projection}, which generalizes matroid truncation.
% For a delta-matroid $D = (V, \cF)$, let $T \subseteq V$ and $\ell \in \N$ with $\ell \le |T|$.
% The \emph{$\ell$-projection} of $D$ by $T$ is the delta-matroid $D' = ((V \setminus T) \cup T', \cF')$, where $T'$ is a set of size $\ell$ and $\cF' = \{ F' \}$.
% It generalizes delta-matroid contraction as well as deletion.
% It also generalizes matroid truncation.
% We assume that $\mathbb{F} = \GF(2^{cn})$ for sufficiently large constant $c$.
Let $D = (V, \mathcal{F})$ be a delta-matroid and let $X \subseteq V$.  
%%%Magnus: This was just defined a half-page above now.
%Recall that the projection $D$ to $V \setminus X$ is the delta-matroid $D | X = (V \setminus X, \cF | X)$, where $\cF | X = \{ F \setminus X \mid F \in \cF \}$.
The \emph{$\ell$-projection} of $D$ to $V \setminus X$ is the delta-matroid $D |_{\ell} X = (V \setminus X, \cF |_{\ell} X)$, where
\begin{align*}
  \cF |_{\ell} X = \{ F \setminus X \mid F \in \cF, |F \cap X| = \ell \}.
\end{align*}
It can be verified that $D|_\ell X$ is indeed a delta-matroid. 
In fact, if we let $D_X$ be the basis delta-matroid of the uniform matroid $U_{X,\ell}$ of rank $\ell$ on $X$ (with all elements of $V \setminus X$ treated as loops in $D_X$), then
\[
  D |_\ell X = (D \Delta D_X) \setminus X.
\]
Because both the delta-sum and deletion operations preserve the delta-matroid property, the resulting structure $D |_\ell X$ is indeed a delta-matroid.
Since $D_X$ is representable over any sufficiently large field (see \cite{OxleyBook2}), a linear representation of the $\ell$-projection of a linear delta-matroid can be computed in randomized polynomial time (see \cite{KW24}).  
To be precise, we have the following:

\begin{lemma}
  \label{lemma:reduce}
  Let $D = (V, \mathcal{F})$ be a linear delta-matroid with $D = \bD(A) / T$ for $A \in \mathbb{F}^{(V \cup T) \times (V \cup T)}$. 
  For $X \subseteq V$, an $\varepsilon$-approximate linear representation of the $\ell$-projection of $D$ to $V \setminus X$ over a field extension $\mathbb{F}'$ of $\mathbb{F}$ with at least $\ell / \varepsilon$ elements can be computed in polynomial time.
\end{lemma}

An alternative, perhaps more transparent, perspective on maintaining a representation of $\ell$-projection is to apply the Ishikawa-Wakayama formula (a generalization of the Cauchy-Binet formula; see Lemma~\ref{lemma:cauchy-binet-ss}) to $\Pf BAB^T$,
where $B$ is a representation of a suitable transversal matroid.  
This approach is analogous to the randomized construction of $k$-truncation for linear matroids~\cite{Marx09-matroid} via the Cauchy-Binet formula.  
Nonetheless, from a technical standpoint, the construction described above fully meets our needs.

We illustrate that linear matroid truncation is a special case of the $\ell$-projection.
Let $M = (V, \cI)$ be a linear matroid represented by a matrix $A$.
As mentioned in Section~\ref{ssec:rep}, we can represent the independent sets of $M$ via the delta-matroid $\bD(A_D)|T$, where
\begin{align*}
  A_D = \kbordermatrix{
    & T & V \\
    T & O & A \\
    V & -A^T & O \\
  }
\end{align*}
Furthermore, by Lemma~\ref{lemma:det-pf}, a set $F \subseteq V \cup T$ is feasible in $\bD(A_D)$ if and only if $|F \cap V| = |F \cap T|$ and the submatrix $A[F \cap T, F \cap V]$ is non-singular.
Thus, applying $k$-projection to the ground set $V$ on $\bD(A_D)$ yields a linear representation of a delta-matroid on $V$, where a set $F \subseteq V$ is feasible if and only if there exists a set $T' \subseteq T$ of size $k$ such that $A[T', F]$ is non-singular.
This is exactly the $k$-truncation of $M$.

For notational convenience, we will sometimes use \emph{$\ell$-contraction} by $X$ to mean $\ell$-projection to $V \setminus X$.

\subsection{Algorihtm}

Suppose that there $\cT=\{T_1,\ldots,T_p\}$ is a triangle packing covering a feasible set $F$ of size $k$.
Using the standard color coding argument, we may assume that $V$ is partitioned into $V_1, \dots, V_p$ such that $T_i \subseteq V_i$.
The color coding step adds a multiplicative factor $k^{O(k)}$ to the running time.
Let $\cT_i$ be the collection of triangles $T \in \cT_0$ with $T \subseteq V_i$.
Our algorithm will try to find a packing of at most $|V_i \cap F| \le 3$ triangles from $\cT_i$.

Let us introduce the notion of \emph{flowery} vertices:
We say that a vertex $v$ is \emph{flowery} if it is at the center of at least
$7$ otherwise disjoint triangles in $\cT_i$.
Note that this can be efficiently
tested simply by computing the maximum
matching number in the graph induced by triangles covering $v$.
Let $H$ be the set of flowery vertices.
Other vertices are \emph{non-flowery}.
Let us also define a broader notion of \emph{safe} vertices.
Intuitively, a safe vertex can always be included into a feasible set, as discussed in \Cref{lemma:completable-packing}.
The central idea in our algorithm revolves around utilizing the $\ell$-projection to get rid of safe vertices.
A vertex $v$ is said to be \emph{safe} if one of the following holds:
\begin{itemize}
  \item 
  it is flowery, or
  \item
  there is a triangle $T = \{ u, v, w \}$ such that every triangle in $\mathcal{T}_i$ covering $u$ or $w$ covers $v$.
\end{itemize}
Note that the second condition holds e.g., when it is covered by an \emph{isolated triangle}, a triangle $T$ such that for every $v \in T$, $T$ is the unique triangle in $\cT_i$ covering $v$.

\begin{lemma} \label{lemma:completable-packing}
  Let $V' \subseteq V$ be the set of safe vertices. 
  Let $F$ be a set of size $k$ such that $|F \cap V_i| \le 3$ for every $i \in [p]$.
  Suppose that there is a triangle packing $\cT$ covering $F \setminus V'$, where every $T \in \cT$ belongs to $V_i$ (i.e., $T \subseteq V_i$) for some $i \in [p]$.
  Then, there is a triangle packing covering all of $F$. 
\end{lemma}
\begin{proof}
  Let $v \in (F \cap V') \setminus V(\cT)$.
  First, suppose that $v$ is non-flowery. 
  By definition, there is a triangle $T = \{ u, v, w \}$ such that every triangle covering $u$ or $w$ covers $v$.
  Note that neither $u$ nor $w$ is covered by $\cT$ since otherwise $v$ would be covered as well.
  Thus, $T$ is disjoint from every other triangle in $\cT$ and can safely be
  added to $\cT$.

  Next, suppose that $v \in (F \cap V') \setminus V(\cT)$ is flowery.
  Since $|F \cap V_i| \le 3$, we may assume that the packing $\cT$ restricted to $V_i$ contains at most two triangles, covering at most six vertices.
  Thus there is at least
  one triangle containing $v$ that is disjoint
  from $V(\cT)$ and can be added to $\cT$. The result follows by
  repetition. 
\end{proof}

Using color coding, we guess
a further partition of every $V_i$ as $V_i=V_{i,1} \cup V_{i,2} \cup V_{i,3}$ such that $T_i$ intersects all three parts.
Thus, we will henceforth assume that every triangle in $\cT_i$ intersects $V_{i,j}$ for all $j \in [3]$.
For each $i \in [p]$ and $j \in [3]$, let $v_{i,j}$ be the unique vertex of $T_i \cap V_{i,j}$.
We define the following for every $i \in [p]$ and $j \in [3]$:
\begin{itemize}
\item let $f_{i,j} \in \{ 0, 1 \}$ be a Boolean indicating whether $v_{i,j}$ is in the feasible set $F$ and
\item let $h_{i,j} \in \{ 0, 1 \}$ be a Boolean indicating whether $v_{i,j}$ is flowery.
\end{itemize}
We guess these Booleans; note that there are $2^{O(k)}$ possibilities.

For each non-flowery vertex, we can also guess one edge of its triangle in the triangle packing. 

\begin{lemma} \label{lemma:small-vc}
  If a vertex $v \in V_i$ is non-flowery, then there is a set $S_v \subseteq V_i$
  such that every triangle containing $v$ must intersect $S_v$ and $|S_v| \le 12$.
\end{lemma}
\begin{proof}
  Consider the graph induced by
  triangles in $\cT_i$ intersecting~$v$.
  Since $v$ is non-flowery, this graph has a maximum matching $M$ of size at most 6.
  Then, $S_v = V(M)$ intersects every triangle containing $v$.
\end{proof}

For each non-flowery vertex $v$, we choose a vertex $s(v) \in S_v$ uniformly at random.
We are interested in the case that  $s(v) \in T$  holds for every non-flowery vertex $v$ covered by a triangle $T_i \in \cT$.
By \Cref{lemma:small-vc}, this happens with probability at least $1/2^{O(k)}$.
Let us say that the \emph{$s$-graph} is a directed graph that contains an arc $(v_i, v_i')$ if and only if $s(v_i) = v_i'$.
By definition, flowery vertices have out-degree~0 and non-flowery vertices have out-degree 1 in the  $s$-graph.
Each triangle $T_i \in \cT$  
is of one of the following types (see Figure~\ref{fig:triangle-cover}).
\begin{enumerate}
\item \label{type:hhh} All three vertices of $T_i$ are flowery.
\item \label{type:hhl} $T_i$ contains two flowery vertices, say $v_{i,1}$
  and $v_{i,2}$, and $v_{i,3}$ points at one of them,
  say $s(v_{i,3})=v_{i,2}$.
\item \label{type:hla} $T_i$ contains one flowery vertex, say $v_{i,1}$,
  and $s(v_{i,2})=s(v_{i,3})=v_{i,1}$.
\item \label{type:hlp} $T_i$ contains one flowery vertex, say $v_{i,1}$,
  and the $s$-graph forms a path, e.g.,
  $s(v_{i,3})=v_{i,2}$ and $s(v_{i,2})=v_{i,1}$.
\item \label{type:hlc} $T_i$ contains one flowery vertex, say $v_{i,1}$,
  and the $s$-graph forms a 2-cycle of
  $s(v_{i,3})=v_{i,2}$ and $s(v_{i,2})=v_{i,3}$.
\item \label{type:l3c} $T_i$ contains only non-flowery vertices,
  and the $s$-graph forms a 3-cycle.
\item \label{type:l2c} $T_i$ contains only non-flowery vertices,
  and the $s$-graph forms a directed 2-cycle
  with one incoming edge,
  e.g., $s(v_{i,1})=v_{i,2}$, $s(v_{i,2})=v_{i,1}$
  and $s(v_{i,3})=v_{i,2}$.
\end{enumerate}
For every $i \in [p]$, we guess the type as well as the
ordering of $j=1,2,3$.
We will assume that all triangles in $\cT_i$ are compatible with our guesses (by deleting incompatible ones from $\cT_i$). 

\begin{figure}
  \centering
  \begin{tikzpicture}
    \begin{scope}[yscale=0.87]
      \node at (-1.4, 2.3) {2.};
      \node[vertex,label={[below=5mm]:$v_{i,1}$}] (v1) at (-1, 0) {};
      \node[vertex,label={[below=5mm]:$v_{i,2}$}] (v2) at (1, 0) {};
      \node[vertex,label={above:$v_{i,3}$}] (v3) at (0, 2) {};

      % \draw[->] (v1) to 
      \draw[->] (v3) to[out=330, in=90] (v2);
    \end{scope}
    \flowervertex{-1}{0}
    \flowervertex{1}{0}

    \begin{scope}[shift={(4,0)},yscale=0.87]
      \node at (-1.4, 2.3) {3.};
      \node[vertex,label={[below=5mm]:$v_{i,1}$}] (v1) at (-1, 0) {};
      \node[vertex,label={[below=5mm]:$v_{i,2}$}] (v2) at (1, 0) {};
      \node[vertex,label={above:$v_{i,3}$}] (v3) at (0, 2) {};
      \draw[->] (v3) to[out=210, in=90] (v1);
      \draw[->] (v2) to[out=210, in=330] (v1);
    \end{scope}
    \flowervertex{3}{0}

    \begin{scope}[shift={(8,0)},yscale=0.87]
      \node at (-1.4, 2.3) {4.};
      \node[vertex,label={[below=5mm]:$v_{i,1}$}] (v1) at (-1, 0) {};
      \node[vertex,label={[below=5mm]:$v_{i,2}$}] (v2) at (1, 0) {};
      \node[vertex,label={above:$v_{i,3}$}] (v3) at (0, 2) {};

      \draw[->] (v3) to[out=330, in=90] (v2);
      \draw[->] (v2) to[out=210, in=330] (v1);
    \end{scope}
    \flowervertex{7}{0}

    \begin{scope}[shift={(12,0)},yscale=0.87]
      \node at (-1.4, 2.3) {5.};
      \node[vertex,label={[below=5mm]:$v_{i,1}$}] (v1) at (-1, 0) {};
      \node[vertex,label={[below=5mm]:$v_{i,2}$}] (v2) at (1, 0) {};
      \node[vertex,label={above:$v_{i,3}$}] (v3) at (0, 2) {};

      \draw[->] (v3) to[out=330, in=90] (v2);
      \draw[->] (v2) to[out=150, in=270] (v3);
    \end{scope}
    \flowervertex{11}{0}

    \begin{scope}[shift={(4,-3.5)},yscale=0.87]
      \node at (-1.4, 2.3) {6.};
      \node[vertex,label={[below=5mm]:$v_{i,1}$}] (v1) at (-1, 0) {};
      \node[vertex,label={[below=5mm]:$v_{i,2}$}] (v2) at (1, 0) {};
      \node[vertex,label={above:$v_{i,3}$}] (v3) at (0, 2) {};

      \draw[->] (v3) to[out=330, in=90] (v2);
      \draw[->] (v2) to[out=210, in=330] (v1);
      \draw[->] (v1) to[out=90, in=210] (v3);
    \end{scope}

    \begin{scope}[shift={(8,-3.5)},yscale=0.87]
      \node at (-1.4, 2.3) {7.};
      \node[vertex,label={[below=5mm]:$v_{i,1}$}] (v1) at (-1, 0) {};
      \node[vertex,label={[below=5mm]:$v_{i,2}$}] (v2) at (1, 0) {};
      \node[vertex,label={above:$v_{i,3}$}] (v3) at (0, 2) {};

      \draw[->] (v3) to[out=330, in=90] (v2);
      \draw[->] (v2) to[out=210, in=330] (v1);
      \draw[->] (v1) to[out=30, in=150] (v2);
    \end{scope}
    % \flowervertex{1.73}{0}
  \end{tikzpicture}
  \caption{Triangles of types 2 to 7. The arrows indicate the $s$-graph.}
  \label{fig:triangle-cover}
\end{figure}
% We will assume that 
% We also delete any vertices of $V_i$ that are not compatible with the type of
% the triangle $T_i$.


Now we are ready to give the reduction to \textsc{Colorful Delta-matroid Matching}.
Recall that the input is a linear delta-matroid and an edge-colored graph over its ground set.
% We construct a linear-delta matroid by applying $\ell$-contractions to $D$.
% We will use $\frac{1}{2}(k - k_S)$ colors, where $k_S$ is to be defined later.
%  $C \subseteq [p]$ 
% There is a color $i$ for every triangle $T_i \in \cT$ that meets certain criteria.
% To construct a \textsc{Colorful Delta-matroid Matching} instance, 
% Using the above information, we will solve a modified version of the
% problem where we find a feasible set $F'$ of non-flowery vertices and a
% triangle packing $\cT$ covering $F'$ such that $F'$ can be extended to a feasible set of size $k$ with flowery vertices and vertices that are covered by isolated triangles.
% This is sufficient by \Cref{lemma:completable-packing}.
% V
%
%
%More precisely, we will partition $F=F' \cup F_H$ where $F_H \subseteq H$,
%and we will define a vertex set $V_H \subseteq H$ based on triangle types
%such that $F_H = F \cap V_H$. We can use delta-matroid constructions 
%to ensure that we look for feasible sets $F' \subseteq (V \setminus V_H)$
%that can be extended into a feasible set $F$ with $|F|=k$ by adding
%elements from $V_H$. Concretely, we compute the value $k_H=|F_H|$
%from the triangle types, and construct a delta-matroid $D'$ on ground
%set $V'=V \setminus V_H$, such that a set $S'$ is feasible in $D'$ if and
%only if there exists a set $S_H \subseteq V_H$ with $|S_H|=k_H$ such
%that $S' \cup S_H$ is feasible in $D$. 
%
% \todo[inline]{probably, the cleaning and/or the below needs to be more
%   explicit? e.g., discard all triangles from $\cT_0$ that are not
%   consistent with the type choices, etc.}
% Starting with $P_i = \emptyset$, $V_S = \emptyset$, and $k_S = 0$,
% we will handle each type \ref{type:hhh}--\ref{type:l2c}.
% Recall that a vertex $v \in V_i$ is safe if it is flowery or there is a triangle $\{ u, v, w \}$ such that every triangle in $\cT_i$ intersecting $\{ u, w \}$ covers $v$.
Whenever we identify a set $S$ of safe vertices,  we apply $|F \cap S|$-contraction by $S$.
This way, we are able to transform the task of finding a triangle packing into the simpler problem of finding a colorful matching.
The pairs will be colored in $C \subseteq [p]$.
For every $i \in [p]$, we will indicate whether $i \in C$, and if $i \in C$, then we also define a collection of pairs $P_i$ in~$V_{i}$, which are colored in $i$.

\paragraph*{Type 1.} If $T_i$ is of type \ref{type:hhh}, then we apply $f_{i,j}$-contraction by $H \cap V_{i,j}$ for each $j \in [3]$. (Recall that $H$ is the set of flowery vertices.)

  %  to $V_S$ and increase $k_S$ by $f_{i,j}$.
\paragraph*{Types 2--5.}
Next, suppose that $T_i$ has one of the types \ref{type:hhl}--\ref{type:hlc}, where $v_{i, 1}$ is flowery.
First, we apply $f_{i,1}$-contraction by $H \cap V_{i,1}$.

We have two cases depending on $f_{i,2} = f_{i,3} = 1$ or not.
If $f_{i,2} = f_{i,3} = 1$, then we define a collection $P_i$ of pairs $(u_{i,2}, u_{i,3})$ from $V_{i,2} \times V_{i,3}$ colored in $i$ as follows.
\begin{itemize}
  \item If $T_i$ has type \ref{type:hhl}, then we add all pairs $(u_{i,2}, u_{i,3})$ such that $u_{i,2} = s(u_{i,3})$ and $N(u_{i,2}) \cap N(u_{i,3})$ contains a flowery vertex.
  \item If $T_i$ has type \ref{type:hla}, then we add all pairs $(u_{i,2}, u_{i,3})$ such that $u_{i,1} = s(u_{i,2}) = s(u_{i,3}) \in V_{i,1}$ is flowery and $\{ u_{i,1}, u_{i,2}, u_{i,3} \} \in \cT_i$.
  \item If $T_i$ has type \ref{type:hlp}, then we add all pairs $(u_{i,2}, u_{i,3})$ such that $u_{i,2} = s(u_{i,3})$ and $s(u_{i,2}) \in N(u_{i,3})$.
  \item If $T_i$ has type \ref{type:hlc}, then we add all pairs $(u_{i,2}, u_{i,3})$ such that $u_{i,2} = s(u_{i,3})$, $u_{i,3} = s(u_{i,2})$, and $N(u_{i,2}) \cap N(u_{i,3})$ contains a flowery vertex.
\end{itemize}
Otherwise (i.e., $(f_{i,2}, f_{i,3}) \ne (1, 1)$), $i \notin C$.
We apply the $f_{i,j}$-contraction by $V_{i,j}$ for $j = 2, 3$.

\paragraph*{Type 6.}
  Suppose that $T_i$ is of type \ref{type:l3c}.
  We may assume that $T_i$ is a strongly connected component of the $s$-graph.
  So, we delete the triangles from~$\cT_i$ that are not strongly connected components.
  Every triangle $T \in \cT_i$ contained in $V_i$ is then isolated and thus safe.
  We apply $f_{i,j}$-contraction by $V(\cT_i) \cap V_{i,j}$ for each $j \in [3]$.

% For type~\ref{type:l2c}, assuming without loss of generality that $f_{i,1} \ge f_{i,2} \ge f_{i,3}$, we obtain $\sum_{j \in J} f_{i,j} \le 1$ from $f(f_{i,2}, f_{i,3}) \ne (1, 1)$.
\paragraph{Type 7.}
Finally, suppose that $T_i$ is of type \ref{type:l2c}.
We delete every triangle $T = \{ u_{i,1}, u_{i,2}, u_{i,3} \} \in \cT_i$ from $\cT_i$ unless the $s$-graph has bidirectional arcs $(u_{i,1}, u_{i,2}), (u_{i,2}, u_{i,1})$ and an arc $(u_{i,3}, u_{i,2})$.
We argue that $u_{i,j}\in V_{i,1} \cup V_{i,2}$ covered by some triangle in $\cT_i$ is safe.
We consider the graph induced by $E(\cT_i)$.
Every connected component consists of one vertex from $V_{i,1}$, another vertex from $V_{i,2}$, and some number of vertices from~$V_{i,3}$.
Thus, any triangle covering $u_{i,j} \in V_{i,1} \cup V_{i,2}$ certifies that it is safe.
We apply $f_{i,j}$-contraction by $V(\cT_i) \cap V_{i,j}$ for each $j \in [3]$.

\medskip

In summary, we apply $f_{i,j}$-contraction by $V_{i,j}$ for every $i \in [p]$ and $j \in [3]$, unless $T_i$ is of type \ref{type:hhl}--\ref{type:hlc} and $f_{i,j} = f_{i,j} = 1$.
We prove the correctness of our algorithm:

% Fix a triangle $T = \{ v_{i,1}, v_{i,2}, v_{i,3} \} \in $.
% Since $v_{i,1}$ is the unique out-neighbor of $v_{i,2}$ in the $s$-graph, if a triangle in $\cT_0$ covers $v_{i,2}$, it also covers $v_{i,1}$.
% Moreover, there is no other triangle covering $v_{i,3}$.
% We thus have that $v_{i,1}$ is indeed safe.
% We apply $f_{i,1}$-contraction by $V_{i,1} \cap V(\cT_i)$.
% If $f_{i,2} = f_{i,3} = 1$, then for every triangle $T = \{ v_{i,1}, v_{i,2}, v_{i,3} \}$, we add the pair $\{ v_{i,2}, v_{i,3} \}$ to~$P_i$.
% Otherwise, we apply the $(f_{i,{2}} + f_{i,3})$-contraction by $(V_{i,2} \cup V_{i,3}) \cap V(\cT_i)$.


\dmtc*
\begin{proof}
  We show that our algorithm solves \textsc{Delta-matroid Triangle Cover} in $O^*(k^{\Oh(k)})$ time.
  Suppose that there is a triangle packing $\cT = \{ T_1, \dots, T_p \}$ covering a feasible set $F$ of size $k$.
  By the color-coding argument, we can find a partition of $V$ into $V_{i,j}$ for $i \in [p]$ and $j \in [3]$ with probability at least $1 / k^{O(k)}$.
  We guess whether each vertex $v_{i,j} \in V(\cT) \cap V_{i,j}$ is feasible or not and flowery or not with respect to the partition $V_{i,j}$ in $2^{\Oh(k)}$ time.
  For every non-flower vertex, we choose a vertex from $S_v$ uniformly at random.
  The probability that $s(v_{i,j}) \in T_{i,j}$ for every non-flowery vertex $v_{i,j} \in V(\cT) \cap V_{i,j}$ is at least $1/2^{\Oh(k)}$.
  As we have described, we then construct an instance $(D', (P_i)_{i \in C})$ of \textsc{Colorful Delta-Matroid Matching}, which can be solved in $O^*(2^{O(k)})$ time.
  Let $M = \{ v_{i,2} v_{i,3} \mid i \in C \}$, where $v_{i,2}$ and $v_{i,3}$ are the unique element in $T_i \cap V_{i,2}$ and $T_i \cap V_{i,3}$, respectively.
  By assumption, $V(M) \subseteq F$, and for every $i \in [p]$ and $j \in [3]$, $|F \cap V_{i,j}| = f_{i,j}$.
  Since $D'$ is the result of $f_{i,j}$-contraction by $V_{i,j}$ for $i \notin C$ and $j \in [3]$, 
  the set $V(M)$ is a feasible set in $D'$.
  
  % Since we apply $|F \cap S|$-contraction by $S$ whenever we identify a set $S$ of safe vertices, $F \setminus V_S$ is indeed feasible in $D'$.
  % Moreover, $F \setminus V_S$ is covered by a colorful matching.

  Conversely, suppose that the \textsc{Colorful Delta-matroid Matching} instance has a solution $M$, i.e., $V(M)$ is feasible in $D'$.
  Let $F \supseteq V(M)$ be a feasible set in $D$ certifying the feasibility of $V(M)$ in $D'$.
  For every $i \in [p]$ and $j \in [3]$ with $f_{i,j} = 1$, let $v_{i,j}'$ be the unique element in $F \cap V_{i,j}$.
  We show that there is a triangle packing $\cT'$ with $F \subseteq V(\cT')$.
  It will consist of at most three triangles from $\cT_i$ for each $i \in [p]$.
  Let us examine each type:
  \paragraph*{Types 1 and 6.}
    If $T_i$ is of type \ref{type:hhh} or \ref{type:l3c}, then $v_{i,j}'$ is safe for every $j \in [3]$.
    Thus, by \Cref{lemma:completable-packing}, there is a triangle packing covering $F \cap V_i$.
  \paragraph*{Types 2--5.}
    Suppose that $T_i$ has one of the types \ref{type:hhl}--\ref{type:hlc}.
    If $f_{i,2} = f_{i,3} = 1$, then there is a pair $(v_{i,2}', v_{i,3}') \in M$.
    By construction, there is a triangle $T$ covering $v_{i,2}'$ and $v_{i,3}'$.
    Since $v_{i,1}'$ is flowery, by \Cref{lemma:completable-packing}, if $v_{i,1}' \notin T$, there is a triangle $T'$  disjoint from $T$ with $v_{i,1}' \in T'$. 
    If $(f_{i,2}, f_{i,3}) \in \{ (0, 1), (1, 0) \}$, then there is a triangle $T$ covering $v_{i,2}'$ or $v_{i,3}'$.
    Again, by \Cref{lemma:completable-packing}, $v_{i,1}'$ is covered by $T$ or another triangle disjoint from $T$.
  \paragraph*{Type 7.}
    Suppose that $T_i$ has type \ref{type:l2c}.
    By construction, there is a triangle covering $v_{i,3}'$.
    Since $v_{i,j}'$ is safe for $j \in [2]$, \Cref{lemma:completable-packing} yields a packing of at most three triangles covering $F \cap V_i$.
\end{proof}


\section{Cluster Subgraph above Matching} \label{sec:cs-above-matching}

Recall that \textsc{Cluster Subgraph} asks whether the input graph $G$ contains a cluster subgraph with at least $\ell$ edges.
In this section, we develop an FPT algorithm for \textsc{Cluster Subgraph} parameterized by $k = \ell - \MM(G)$, where $\MM(G)$ is the maximum matching size.
Our algorithm is a reduction to (a variant of) \textsc{Delta-matroid Triangle Cover}.

\csam*

We start with the simpler case that the input graph $G$ is a $K_4$-free graph with a perfect matching.
Note that $k = \ell - n/2$.
Let $D$ be the dual of the matching delta-matroid.
We can formulate the \textsc{Cluster Deletion} problem in terms of $D$ as follows.

\begin{lemma}
  \label{lemma:ce-feasible-cover}
  There is a cluster graph with at least $\ell$ edges if and only if there is a triangle packing $\mathcal{T}$ covering a feasible set $F \subseteq V(\mathcal{T})$ of size $2k$ in $D$.
\end{lemma}

To prove \Cref{lemma:ce-feasible-cover}, we start with an observation.
We fix a perfect matching $M$ in $G$.
An alternating path $P$ is a path that visits an edge in $M$ and not in $M$ in an alternating order.
Further, we say that $P$ is an $M$-alternating path if the first and last edges are in $M$.
For a vertex set $F$, an alternating $F$-path is a path where the endpoints are both in $F$ and the inner vertices are disjoint from $F$.
Two paths are disjoint when they do not have any vertex in common.

\begin{lemma}
  \label{lemma:alt-paths}
  For a vertex set $F \subseteq V$, $F$ is feasible in the dual matching delta-matroid $D$ if and only if there is a collection of $|F|/2$ vertex-disjoint $M$-alternating $F$-paths.
\end{lemma}
\begin{proof}
  If $F$ is feasible in $D$, then $G - F$ has a perfect matching $M'$.
  The symmetric difference $M \Delta M'$ yields a collection of $|F|/2$ disjoint $M$-alternating $F$-paths.
  Conversely, if $\cP$ is a collection of $|F|/2$ disjoint $M$-alternating $F$-paths, then $M \Delta E(\cP)$ is a perfect matching in $G - F$, so $F$ is feasible in $D$.
\end{proof}

Intuitively speaking, every $M$-alternating path with endpoints in triangles ``earns'' one edge for the solution.

\begin{proof}[Proof of \Cref{lemma:ce-feasible-cover}]
  First, assume that there is a feasible set $F$ of size $2k$ covered by a triangle packing~$\mathcal{T}$.
  By \Cref{lemma:alt-paths}, we have a collection $\mathcal{P}$ of $k$ vertex-disjoint $M$-alternating $F$-paths.
  Suppose that among $\mathcal{P}$, there are $k'$ paths of length one.
  We construct an edge set $E'$ of size at least $\ell$ such that $(V(G), E')$ is a cluster graph.
  We start with $E' = M$.
  Replace all edges incident to $V(\mathcal{T})$ (denoted by $\partial(V(\mathcal{T}))$) with $E(\mathcal{T})$, i.e., $E' = (E' \setminus \partial(\mathcal{V(\mathcal{T})})) \cup E(\mathcal{T})$.
  Note that $|E' \cap \partial(\mathcal{V(\mathcal{T})})| = 3|\mathcal{T}| - k'$ because we count one edge in $E' \cap \partial(\mathcal{V(\mathcal{T})})$ for every vertex in $V(\mathcal{T})$ and the edges corresponding to length-one paths in $\mathcal{P}$ are counted twice.
  Since $3|\mathcal{T}| - k'$ edges are replaced with $3|\mathcal{T}|$ edges, we have $|E'| = |M| + k'$.
  For every $M$-alternating $F$-path $P \in \mathcal{P}$ of length greater than one, we take $E' = E' \Delta (E(P) \setminus \delta(V(T)))$, increasing the size $E'$ by one this way.
  Since the paths in $\mathcal{P}$ are disjoint, we indeed increase the size of $E'$ by $k - k'$, yielding an edge set of size $|M| + k$.
  It is easy to verify that $E'$ indeed induces a cluster subgraph.

  Next, assume that there is a cluster subgraph $H = (V(G), E')$ with at least $\ell$ edges.
  We assume that $E'$ is chosen in such a way that it minimizes $|E' \Delta M|$ under the constraint that $(V(G), E')$ is a cluster graph and $|E'| \ge \ell$.
  By the hand-shaking lemma, we have $|E'| = \frac{1}{2}\sum_{v \in V(H)} \deg_H(v)$.
  Since $H$ has maximum degree two, we have $\deg_H(u) + \deg_H(v) \le 4$ for every edge $uv \in M$.
  We may assume that $\deg_H(u) + \deg_H(v) \in \{ 2, 3, 4 \}$ for every $uv \in M$:
  If $\deg_H(u) + \deg_H(v) \le 1$, then we obtain another cluster graph $(V(G), E'')$ as follows:
  If $\deg_H(u) = \deg_H(v) = 0$, then we add the edge $uv$ to $H$.
  If $\deg_H(u) + \deg_H(v) = 1$, then we replace the edge incident to $uv$ with $uv$.
  In both cases, we have $|E'' \Delta M| < |E \Delta M|$, contradicting our choice of $E'$.
  % We may assume the following:
  % \begin{itemize}
  %   \item
  %   We may assume that $\deg_H(u) + \deg_H(v) \in \{ 2, 3, 4 \}$ for every $uv \in M$.
  %   If $\deg_H(u) + \deg_H(v) \le 1$, then we obtain another cluster graph $(V(G), E'')$ as follows:
  %   If $\deg_H(u) = \deg_H(v) = 0$, then we add the edge $uv$ to $H$.
  %   If $\deg_H(u) + \deg_H(v) = 1$, then we replace the edge incident to $uv$ with $uv$.
  %   In both cases, we have $|E'' \Delta M| < |E \Delta M|$, contradicting our choice of $E'$.
  %   \item
  %   Furthermore, we may assume that for every triangle $T = (u, v, w)$ in $H$ in which every vertex is matched to a vertex outside $T$, i.e., $uu', vv', ww' \in M$ and $u', v', w' \notin V(T)$, $\deg_H(u') + \deg_H(v') + \deg(w') > 0$.
  %   Otherwise, we can replace $T$ with the three edges $uu'$, $vv'$, and $ww'$ without changing the size of $H$.
  % \end{itemize}

  We want to find $M$-alternating paths whose endpoints belong to triangles in~$H$.
  First, observe that an edge $uv \in M$ with $\deg_H(u) = \deg_H(v) = 2$ is an $M$-alternating path of length one.
  Next, consider an edge $uv \in M$ such that $\deg_H(u) = 2$ and $\deg_H(v) {= 1}$. We can find an $M$-alternating path starting from $u$:
  Since $\deg_H(v) = 1$, there exists an edge $vw \in E'$.
  We have $\deg_H(w) = 1$, since otherwise $w$ is part of a triangle in $H$ containing both $v$ and~$w$, which contradicts $\deg_H(v) = 1$.
  Suppose that $w$ is matched to $x$ in~$M$.
  If $\deg_H(x) = 2$, then we find an $M$-alternating path.
  Otherwise, we find an $M$-alternating path by repeating this argument until we reach a vertex of degree two in~$H$.
  Thus, we have an $M$-alternating path for every edge $uv \in M$ with $\deg_H(u) + \deg_H(v) = 3$.
  (Note that every path is counted twice.)
  By the hand-shaking lemma,  
  \begin{align*}
  |E'| = \frac{1}{2} \sum_{v \in V(G)} \deg_H(v) = \frac{1}{2} \sum_{uv \in M} \deg_H(u) + \deg_H(v) = |M| + \frac{1}{2} \sum_{uv \in M} (\deg_H(u) + \deg_H(v) - 2).
  \end{align*}
  Since $|E'| \ge \ell = |M| + k$, we have $\frac{1}{2}|M_3| + |M_4| \ge k$, where $M_3$ and $M_4$ are the set of edges $uv \in M$ with $\deg_H(u) + \deg_H(v) = 3$ and $4$, respectively.
  With the double counting in mind, we find $k$ vertex-disjoint $M$-alternating paths.
  Let $F$ be the set of their endpoints.
  It is of size $2k$, and feasible in $D$ by \Cref{lemma:alt-paths}.
  Moreover, the triangles in $H$ cover $F$ as every vertex in $F$ has degree 2 in $H$.
\end{proof}

It follows from \Cref{theorem:dmtc-fpt} that \textsc{Cluster Subgraph} can be solved in $O^*(k^{\Oh(k)})$ time when the input graph is $K_4$-free and has a perfect matching.

\paragraph*{Lifting assumptions.}

% To lift the assumption that the graph is $K_4$-free, we use a greedy packing.
% If we find more than $2k$ 
% The guesses contain $k^{O(k)}$ many possibilities, and we will encode this information in an auxiliary problem.
% For instance, if we guessed that a component $C$ is extended by three vertices, then we will ensure that at least one triangle contained in the common neighborhood of $C$ is taken into the solution.
% We now formulate an auxiliary problem incorporating additional information.
% We will reduce to the following auxiliary problem.
% The input is a $K_4$-free graph $G$, a set $V_i \subseteq V(G)$ and $\gamma_i \in \{ 1, 2, 3 \}$ for every $i \in [\kappa]$, and integers $\ell \in \N$.
% The task is to find a cluster subgraph $G' = (V(G), E')$ with the following properties:
% Let $C_1, \cdots, C_{\nu}$ be connected components of $G'$.
% For each $i \in [\kappa]$, $C_i$ is a clique on $\gamma_i$ vertices with $V(C_i) \subseteq V_i$.
We discuss how to solve \textsc{Cluster Subgraph} generally.
To that end, we will reduce to the color-coded variant of \textsc{Delta-matroid Triangle Cover}, which is to find a triangle packing containing at least one triangle from $\cT_i$, where $(V_1, \dots, V_p)$ is a partition of $V$ and $\cT_i$ is a collection of triangles in $V_i$.
Note that our algorithm (\Cref{theorem:dmtc-fpt}) solves the color-coded variant in $O^*(k^{\Oh(k)})$ time.

% Next, we discuss the assumption concerning a perfect matching.
% Again, let $G' = (V(G), E')$ be a hypothetical solution.
% \todo[inline]{Now, we search for a feasible set $F$ with $F \cap V_1 = \emptyset$ such that $F$ can be covered by triangles. Suppose that $v$ is originally unmatched and matched to a new vertex $v'$. $v$ cannot be part of $F$, because deleting $v$ would leave $v'$ unmatched. We need to increase $\ell$ for every vertex added:
% If $vv'$ is part of the solution, it indeed increases $\ell$ by one. Otherwise, $v$ is part of some triangle. We need to decrease $\ell$ for this case.}

\begin{lemma} \label{lemma:lifting}
  There is a parameterized Turing reduction from \textsc{Cluster Subgraph above Matching} to the color-coded variant of \textsc{Delta-matroid Triangle Cover} that generates $k^{O(k)}$ instances.
\end{lemma}
\begin{proof}
  First, we find a greedy packing of $K_4$'s.
  Start with $K = \emptyset$.
  As long as there is a $K_4$ in the graph, we add it to $K$, deleting it from the graph. 
  If we find $k/2$ disjoint $K_4$'s, we conclude that there is a cluster subgraph with $\ell$ edges:
  There remains a matching of size at least $\MM(G) - 4 \cdot k/2 = \MM(G) - 2k$, and thus taking the disjoint union of the $K_4$'s and the matching yields a cluster subgraph with at least $6 \cdot k/2 + (\MM(G) - 2k) = \MM(G) + k$ edges.
  Thus, we may assume that there is a set $K$ of size at most $2k$ such that $G - K$ is $K_4$-free (and $K$ can be found in polynomial time).
  Let $H = (V(G), E')$ be a hypothetical solution and $\cC = \{ C_1, \cdots, C_{\kappa} \}$ be the connected components of~$H$.
  Suppose that $H[K]$ has $\kappa'$ components $C_1', \cdots, C_{\kappa'}'$. We assume that $C_i' \subseteq C_i$ for every $i \in [\kappa']$.
  For every $i \in [\kappa]$, we guess $\gamma_i = |C_i \setminus C_i'|$.
  Note that $\gamma_i \le 3$, because $G - K$ is $K_4$-free.
  % We assume w.l.o.g.\ that $\gamma_i \in [3]$ for every $i \in [\kappa']$ and that $\gamma_i = 0$ for $i > \kappa'$.
  We search for a cluster graph with at least $\ell_1 = \ell - (\sum_{i \in [\kappa']} \binom{|C_i'|}{2} + \gamma_{i} \cdot |C_i'|)$ edges in $G - K$ such that it has a connected component (pairwise distinct for each $i$) of size $\gamma_i$ contained in $\bigcap_{v \in C_i'} N(v)$.
  We construct an instance of the color-coded variant of \textsc{Delta-matroid Triangle Cover} as follows.
  First, we partition $V(G) \setminus K$ into $k'$ sets $V_1, \dots, V_{k'}$ using color coding, where $k'$ is $\kappa'$ plus the number of triangles in $H - \bigcup_{i \in [\kappa']} C_i$.
  By \Cref{lemma:ce-feasible-cover}, we may assume that $k' \in \Oh(k)$.
  For $i \in [\kappa']$, $V_i$ should contain $C_i \setminus C_i'$.
  So we will assume that $V_i \subseteq \bigcap_{v \in C_i'} N(v)$.
  For each $i \in [\kappa' + 1, k']$, let $\cT_i$ contain all triangles in $G[V_i]$.
  For $i \in [\kappa']$, we do as follows.
  If $\gamma_i \le 1$, then we apply $\gamma_i$-contraction by $V_i$.
  If $\gamma_i = 3$, then the collection $\cT_i$ contains all triangles $T \subseteq V_i$.
  If $\gamma_i = 2$, we may assume that there is no triangle in $V_i$:
  Let $C_i = C_i' \cup \{ x, y \}$.
  If there is a triangle $T = \{ u, v, w \}$ in $V_i$, we obtain another solution by deleting edges incident to $x, y$, and $T$ and adding edges so that $C_i' \cup T$ becomes a clique.
  We delete at most $2 |C_i'| + 3$ edges since we assumed that $H$ contains no triangle in $V_i$.
  On the other hand, we add $3 |C_i'| + 3 \ge 2|C_i'| + 3$ edges, and thus the assumption that $V_i$ has no triangle if $\gamma_i = 2$ is justified.
  To reduce to the triangle cover problem, for every $i$ with $\gamma_i = 2$, we introduce a new vertex $v$ and make it adjacent to $V_i$ so that the solution contains a triangle with $v$ in it.
  To account for two edges incident to $v$, we increase the solution size by two, i.e., we will look for a cluster subgraph with at least $\ell_2 = \ell_1 + 2 |\Gamma_2|$, where $\Gamma_2  = \{ i \in [\kappa'] \mid \gamma_i = 2 \}$.
  Note that $G - K$ remains $K_4$-free after $v$ is added.

  We fix a maximum matching $M$ of $G - K$.
  Let $U = V(G - K) \setminus V(M)$.
  For every unmatched vertex $v \in U$, we add a degree-one neighbor $u'$ only adjacent to $u$ and let $U' = \{ u' \mid u \in U \}$.
  We call the resulting graph $G'$, and let $M' = M \cup \{ uu' \mid u \in U \}$.
  If an unmatched vertex $u$ is isolated in $H$, then we can increase the solution size by one by adding the edge~$uu'$.
  Let $\alpha$ and $\beta$ be the number of vertices in $U$ that are isolated and not isolated in $H$, respectively. 
  Note that we cannot guess which unmatched vertices are isolated in $H$, so we only guess two numbers $\alpha$ and $\beta$.
  We then search for a cluster graph with at least $\ell_3 = \ell_2 + \alpha$ edges in $G'$ such that $\beta$ vertices of $U'$ are isolated.
  % By, we search for a feasible set of size $2(\ell_3 - (\MM(G) + \alpha))$ that can be covered by a triangle packing in $G$.
  Let $D'$ be the dual matching delta-matroid of $G'$ and $D$ be the result of applying a $\beta$-contraction by $U'$ on $D$.
  Note that the ground set of $D$ is $V(G - K)$.
  % We then apply $\beta$-contraction by $U'$ and let $D'$ denote the resulting delta-matroid.
  % We then apply  \Cref{lemma:reduce} on each of $U$ and $U'$ with the size bound $\alpha$ and let $D'$ denote the resulting delta-matroid.
  Our algorithm searches for a feasible set in $D$ of size $2(\ell_3 - (\MM(G) + \alpha + \beta)) - \beta \in \Oh(k)$.
  % As we showed in  \Cref{lemma:ce-feasible-cover}, we want to find a set feasible in $D'$ of size $2(\ell_3 - (\MM(G) + \alpha + \beta)) \in \Oh(k)$ that can be covered by a triangle packing in~$G - K$.
  % To ensure that $\beta$ vertices of $U'$ are isolated, 
  By \Cref{lemma:alt-paths}, a feasible set in $D'$ corresponds to the endpoints of disjoint $M'$-alternating paths in $G'$.
  Thus, for a feasible set $F$ in $D$, there is a collection $\mathcal{P}$ of $\frac{1}{2}(|F| + \beta)$ disjoint $M'$-alternating paths in $G'$.
  Note that $\beta$ vertices of $U'$ are isolated in the symmetric difference between $E(\mathcal{P})$ and $M'$.
  Moreover, by \Cref{lemma:ce-feasible-cover}, if a feasible set $F$ in $D$ is of size $2(\ell_3 - (\MM(G) + \alpha + \beta)) - \beta$ can be covered by a triangle packing, then we have a cluster subgraph with at least $\ell_3$ edges.
  % Note that every vertex $u \in U$ has a single $M$-alternating path, namely, a length-one path from $u$ to $u'$.
  % We will ensure that $\alpha$ vertices of $U$ are covered by triangles.
  % Thus, we have that a set $F \subseteq V(G') \setminus (U \cup U')$ is feasible in $D'$ if and only if $F \cup \{ x, x' \mid x \in X \}$ is feasible in $D$, where $X \subseteq U$ is some set of size $k$.
  % Thus, we have that a feasible set $F \subseteq V(G') \setminus U'$ in $D'$ is covered by triangles if and only if $F \cup \{ x, x' \mid x \in X \}$ is feasible in $D$, where $X \subseteq U$ is some set of size $k$.
  % The rest of the proof is similar to that of \Cref{lemma:ce-feasible-cover}.
  % The feasible set should have size $-\alpha$.
\end{proof}

It follows that \textsc{Cluster Subgraph} can be solved in time $O^*(k^{O(k)})$ time, proving \Cref{theorem:cs-am}.

\section{Strong Triadic Closure above Matching}
\label{sec:tcam}

In this section, 
we study \textsc{Strong Triadic Closure} problem parameterized by $k = \ell - \mu$. Let us first recall the problem definition: 
given a graph $G$ and $\ell \in \N$, the goal is to find a strong set of size $\ell$, where an edge set $E' \subseteq E$ is strong if $uw \in E(G)$ whenever $uv \in E'$ and $vw \in E'$.

We start our investigation similarly to \textsc{Cluster Subgraph} in Subsection~\ref{ssec:tcamfpt} by showing that the problem is FPT on $K_4$-free graphs. However, as opposed to \textsc{Cluster Subgraph}, we show in Subsection~\ref{ssec:STC-w1-hard} that the assumption of $K_4$-freeness cannot be lifted and \textsc{Strong Triadic Closure} is W[1]-hard parameterized by $k$ on general graphs. We then show that we can still overcome this limitation if maximum degree in the graph is bounded in Subsection~\ref{ssec:STC_bounded_degree} or if we are looking for an approximate solution in FPT time in Subsection~\ref{ssec:STC_FPT_Approx}. We complement our constant FPT approximation algorithm with a strong inapproximability lower bound for polynomial time algorithms.

%we give an FPT algorithm for the \textsc{Strong Triadic Closure} problem parameterized by $k = \ell - \mu$.


\subsection{FPT algorithm for $K_4$-free graphs} \label{ssec:tcamfpt}

As in Section~\ref{sec:cs-above-matching}, we will initially work under the simplifying assumption that the input graph $G$ is a $K_4$-free graph with a perfect matching.
We deal with the general case afterwards.
While \textsc{Cluster Subgraph} is a special case of \textsc{Delta-matroid Triangle Cover} under this assumption, there seems to be no simple reduction from \textsc{Strong Triadic Closure}.
% So we give an algorithm for \textsc{Strong Triadic Closure}.
So we will develop a problem-specific algorithm, significantly extending our algorithm for \textsc{Delta-matroid Triangle Cover}.

\tcamfpt*

A \emph{strong cycle} on $n$ vertices for $n \ge 3$ is a graph $C$ such that $V(C) = \{ v_0, \dots, v_{n - 1} \}$ and there is an edge between two vertices $v_i$ and $v_j$ if 
% $i - j\equiv \pm 1, 2 \mod n$.\
$i - j = \pm 1, 2 \bmod n$.
Note that the edge set $\{ v_{i} v_{(i + 1) \bmod n} \mid i = 0, \dots, n - 1 \}$ constitutes a strong set, which we denote by $E_s(C)$.
Note also that a triangle is also a strong cycle.
The following lemma connects the size of a feasible set (in the dual matching delta-matroid $D$) covered by a strong cycle packing to the solution size (cf. Lemma~\ref{lemma:ce-feasible-cover})

\begin{lemma} \label{lemma:tc-feasible-cover}
  There is a strong set of size $\ell$ if and only if there is a strong cycle packing $\mathcal{C}$ such that there is a feasible set $F \subseteq V(\cC)$ of size $2k$.
\end{lemma}
\begin{proof}
  First, assume that there is a feasible set $F$ of size $2k$ covered by a strong cycle packing $\mathcal{C}$.
  Again by Lemma~\ref{lemma:alt-paths}, we have a collection $\mathcal{P}$ of $k$ vertex-disjoint $M$-alternating $F$-paths.
  %Same as in Lemma~\ref{lemma:ce-feasible-cover}, we start with $E'=M$ as a starting point for the strong set of size $\ell$. 
  %We then replace all edges incident to $V(\mathcal{C})$ by strong edge sets $E_s(C)$ for every $C \in \cC$
  We essentially take the symmetric difference $E(\mathcal{P}) \Delta M$ and then remove from this set all edges incident to $V(\mathcal{C})$ and add a strong edge set $E_s(C)$ for every $C \in \cC$. Note that we end up with exactly the same construction as in Lemma~\ref{lemma:ce-feasible-cover}, with only difference being that $\mathcal{C}$ can contain also longer cycles and not only triangles.
  Then, we can argue in the same as in the proof of Lemma~\ref{lemma:ce-feasible-cover} that we obtain a strong edge set of size $|M| + k$.

  Next, assume that there is a strong edge set $E'$ with at least $\ell$ edges.
  % We assume that $E'$ is chosen in such a way that it minimizes $|E' \Delta M|$ under the constraint that $(V(G), E')$ is a cluster graph and $|E'| \ge k$.
  % By the hand-shaking lemma, we have $|E'| = \frac{1}{2}\sum_{v \in V(H)} \deg_H(v)$.
  % Since $H$ has maximum degree two, we have $\deg_H(u) + \deg_H(v) \le 4$ for every edge $uv \in M$.
  % We may assume the following: 
  % With Lemma~\ref{lemma:ce-feasible-cover} at hand, we need to argue the following:
  Since there is no $K_4$, we may assume that the graph $H = (V(G), E')$ induced by a strong edge set has no vertex of degree greater than two, and thus, every connected component is either a path or a cycle, as observed by Golovach et al.~\cite{GolovachHKLP20}.
  In fact, we may assume that a path of length at least two can be replaced by triangles:
  If $uv, vw \in E'$, then the edge $uw$ exists by the triadic closure and hence $uw$ can be added to $E'$.
  Suppose that the path is on $p \ge 3$ vertices $(v_1, \dots, v_p)$ (note that it has $p - 1$ edges)
  If $p = 3q$ or $p = 3q + 1$, then there is a packing of $q$ triangles, namely, $\{ 3p' - 2, 3p' - 1, 3p' \}$ for each $p' \in [q]$, which contains $3q \ge p - 1$ edges.
  If $p = 3q + 2$, then there is a packing of $q$ triangles $\{ 3p' - 2, 3p' 1, 3p' \}$ for each $p' \in [q]$ and one edge $v_{p - 1} v_p$, which has $3q + 1 = p - 1$ edges.
  Thus, we can assume that every connected component of $H$ of size at least three is a cycle.
  The rest is analogous to the proof of Lemma~\ref{lemma:ce-feasible-cover}; simply replace ``triangles'' with ``cycles'' in the proof.
\end{proof}

We say that a strong cycle packing $\cC$ is \emph{triangle-maximal} if it maximizes the number of triangles.
% the number of edges first and then the number of triangles second.
We will search for a triangle-maximal strong cycle packing.
This assumption proves to be valuable since a strong cycle can often be replaced by a triangle packing in various cases.
Consequently, we can assume that the strong set of our interest is small as shown in the Lemma~\ref{lemma:tri-maximal}, which is crucial for color coding.
\begin{lemma} \label{lemma:tri-maximal}
  Suppose that $\cC$ is a triangle-maximal strong cycle packing covering a feasible set $F$ of size~$2k$.
  For $p \in \N$, the following holds:
  \begin{itemize}
    \item $\cC$ contains no cycle of length $3p$ for $p > 1$.
    \item If $\cC$ contains a cycle $C$ of length $3p + 1$, then $F \cap V(C) = V(C)$.
    \item If $\cC$ contains a cycle $C$ of length $3p + 2$, then $|F \cap V(C)| \ge \frac{1}{2} (3p +2)$.
  \end{itemize}
  Moreover, we have $|V(\cC)| \le 6k$.
\end{lemma}
\begin{proof}
  Suppose that the solution has a long strong cycle $C = (v_1, \dots, v_l)$ of length $l$.
  \begin{itemize}
    \item 
    If $l = 3p$, we can replace $C$ with $p$ disjoint triangles, $\{ v_{3p' - 2}, v_{3p' - 1}, v_{3p'} \}$ for $p' \in [p]$.
    \item
    Suppose that $l = 3p + 1$ and that $V(C) \setminus F \ne \emptyset$.
    Assume w.l.o.g.\ that $v_{l} \notin F$.
    Then we can replace $C$ with $p$ disjoint triangles $\{ v_{3p' - 2}, v_{3p' - 1}, v_{3p'} \}$ for $p' \in [p]$.
    \item
    Suppose that $l = 3p + 2$.
    If $|V(C) \cap F| < |C| / 2$, then there are two consecutive vertices in $C$ not contained in $F$.
    W.l.o.g., assume that they are $v_{l - 1}$ and $v_{l}$.
    Then, we can cover $V(C) \setminus \{ v_{l - 1}, v_l \}$ with $p$ triangles $\{ v_{3p' - 2}, v_{3p' - 1}, v_{3p'} \}$ for $p' \in [q]$, which can replace $C$.
  \end{itemize}
  For every $C \in \cC$ we have $|C \cap F| \ge |C| / 3$ (where the equality holds for triangles), and thus $|V(\cC)| \le 3 |F| = 6k$.
\end{proof}

% A vertex is \emph{flowery} if it is at the center of at least 100000 otherwise disjoint triangles.
Let a triangle-maximal strong cycle packing be $\cC = \{ C^1, \cdots, C^{\kappa} \}$, where $C^i = (v^i_{1}, \dots, v^i_{l_i})$, covering a feasible set $F$ of size~$2k$.
Reusing the first color-coding steps from Section~\ref{sec:triangle-cover}, we guess the following:
\begin{itemize}
\item A partition $V=V^{1} \cup \ldots \cup V^\kappa$ such that for every $i \in [\kappa]$, 
  $V(C^i) \subseteq V^{i}$.
\item An integer $l^i$ such that $|V(C^i)| = l^i$ and a further partition of every $V^{i}$ into $l^i$ sets $V^{i}_1 \cup \cdots \cup V^{i}_{l_i}$ 
  such that $C^i$ intersects all parts; for $i \in [\kappa]$ and $j \in [l_i]$,
  let $v^i_{j}$ be the unique vertex of $V(C^i) \cap V^{i}_{j}$.
\end{itemize}
By Lemma~\ref{lemma:tri-maximal}, this step adds a multiplicative factor $k^{O(k)}$ to the running time.
For $i \in [\kappa]$ with $l_i = 3$, we can handle it in the same way as \textsc{Delta-matroid Triangle Cover}.
See Section~\ref{sec:triangle-cover} for details. 
For every $i \in [\kappa]$ with $l_i \ne 3$, we may assume that $l_i  \bmod 3 \in \{ 1, 2 \}$ by Lemma~\ref{lemma:tri-maximal}.
As we will focus attention on one cycle, we will drop the superscript $\cdot^i$, unless it is unclear.
We will also use the cyclic notation for subscripts, e.g., $v_{0} = v_{l_i}$.

Each $V_j$ is referred to as a \emph{block}.
We will assume that every vertex $v \in V_{j}$ has neighbors in the two preceding and two following blocks $V_{j-2} \cup V_{j-1} \cup V_{j+1} \cup V_{j+2}$.
Call a vertex $v \in V_{j}$ is \emph{flowery} if it has a matching of size at least 13 in its neighborhood and \emph{non-flowery} otherwise.
For non-flowery vertices, we can guess a few of its neighbors:

\begin{lemma} \label{lemma:vc}
  Let $C = (v_1, \dots, v_{l})$ be a strong cycle.
  For a vertex cover $S$ of $N(v_j)$, the following hold: (i) $v_{j-1} \in S$ or $v_{j+1} \in S$, (ii) $v_{j-2} \in S$ or $v_{j-1} \in S$, and (iii) $v_{j+1} \in S$ or $v_{j+2} \in S$.
\end{lemma}
\begin{proof}
  From the definition of strong cycles, it follows that $v_{j-2}, v_{j-1}, v_{j+1}, v_{j+2} \in N(v_j)$ and that $v_{j -1} v_{j+1}, v_{j-2} v_{j-1}$, $v_{j+1} v_{j+2} \in E(C)$.
  %Thus, the lemma holds.
\end{proof}

\subsubsection{Strong cycle of length $3p + 1$} \label{sssec:1mod3}

Suppose that $\cC$ contains a cycle $C = (v_{1}, \dots, v_{3p+1})$.
By Lemma~\ref{lemma:tri-maximal}, we may assume that $V(C) \subseteq F$.
We may also assume that every vertex is non-flowery:

\begin{lemma} \label{lemma:1mod3:non-flowery}
  Every vertex in $C$ is non-flowery.
\end{lemma}
\begin{proof}
  For contradiction, assume w.l.o.g.\ that $v_{3p + 1}$ is flowery.
  Consider a triangle packing $\{ \{ v_{3q-2}, \allowbreak v_{3q-1}, v_{3q} \} \mid q \in [p] \}$.
  Since $v_{3p+1}$ is flowery, there exists a triangle that covers $v_{3p+1}$ and is disjoint from the packing.
  These triangles cover $V(C)$, contradicting the triangle maximality.
\end{proof}

As every vertex is non-flowery, we can guess a neighbor for each vertex either in their preceding or following block.
In a simple case where this guessing results in a collection of $l$-cycles, the situation aligns with a case discussed in Section~\ref{sec:triangle-cover}, namely case 6.
Here, our algorithms applies 1-contraction applied to each block.
We argue that even if the outcome does not neatly fit into the pattern of $l$-cycles, a structural simplification that enables the application of 1-contraction on each block can be achieved.
This is done through a nuanced application of guessing arguments, where we strategically work out the connections between vertices across blocks:

\begin{lemma} \label{lemma:1mod3:ultimate}
  There is a polynomial-time algorithm that finds a set $S$ of vertices such that (i) $C \subseteq S$ with probability at least $1/2^{O(l)}$ and (ii) for any $X \subseteq S$ with $|X \cap V_j| = 1$ for each block $V_j$, there is a strong set that covers $X$.
\end{lemma}
\begin{proof}
  For every $v \in V^i$, $i \in [l]$, let $S_v$ be a vertex cover on $N(v)$.
  By Lemma~\ref{lemma:1mod3:non-flowery}, we may assume that $S_v$ has size at most 26.
  For every vertex $v_j \in V(C)$, by Lemma~\ref{lemma:vc}, $v_{j-1} \in S_{v_j}$ or $v_{j+1} \in S_{v_j}$ holds.
  Fix $d_j \in \{ 1, -1 \}$ such that $v_{j+d_j} \in S_{v_j}$ for every $v_j \in V(C)$.
  For every $j \in [l]$, we randomly choose $\delta_j = \pm 1$ uniformly at random.
  Then, with probability $1/2^{O(l)}$, $\delta_j = d_j$ holds for all $j \in [l]$.
  For every $j \in [l]$ and $v \in V_j$, we randomly choose a vertex $s(v) \in S_v \cap V_{j+\delta_j}$.
  The probability that $s(v_j) = v_{j+d_j}$ for each $j \in [l]$ is at least $1/2^{O(l)}$.
  
  Consider a directed graph $H_C$ over the vertex set $C$, where there is an arc from $v_{j}$ to $v_{j+d_j}$.
  First, we consider a simpler case that $H_C$ forms a directed cycle.
  We introduce an \emph{$s$-graph}, over the vertex set $V^i$, where there is an arc $(v, s(v))$ for every vertex $v \in V^i$.
  From the $s$-graph, we construct a set $S$ as follows.
  For every connected component in the corresponding undirected graph that forms a cycle, which intersects with each subset $V_j$ precisely once, we include all its vertices in $S$, provided the component's edges constitute a strong cycle.
  Note that this is completely analogous to case 6 in Section~\ref{sec:triangle-cover}.
  Then, the item (i) is satisfied because if $s(v_j) = v_{j+d_j}$ for each $j \in [l]$, then all vertices of $C$ are incorporated into $S$.
  For the item (ii), note that the entire set $S$ can be covered by strong cycles.

  Suppose that $H_C$ is not a directed cycle.
  While it seems difficult to obtain a collection of cycles, we obtain a similar structure, a collection of \emph{cycles with parallel vertices}.
  Here, a cycle with parallel vertices is a graph obtained from a standard cycle $(x_1, \dots, x_n)$ by adding vertices parallel to $x_i$, i.e., adjacent to $x_{i-1}$ and $x_{i+1}$.
  As $H_C$ is not a directed cycle, there exists $a \in [l]$ such that the edge $v_{a-1} v_{a}$ is missing in the underlying undirected graph.
  % Every connected component in the underlying undirected graph of $H_C$ is a path.
  Assume w.l.o.g.\ that $a = 1$.
  It follows that $d_a = d_1 = +1$ and $d_{a-1} = d_{l} = -1$.
  Let $a'$ be the largest integer such that $d_j = +1$ for all $j \in [a, a']$.
  Such an integer $a'$ exists because $d_l = -1$.
  By the definition of $a'$, $d_{b'} = -1$, where $b' = a' + 1$.
  Also, let $b$ be the largest integer such that $d_j = -1$ for all $j \in [b', b]$.
  We will simplify the structure of the $s$-graph over $V_a \cup \cdots \cup V_b$ as follows.
  First, we find all bidirectional arcs $(u_{a'}, u_{b'})$ within $V_{a'} \cup V_{b'}$ in the $s$-graph, i.e., $s(u_{a'})= u_{b'}$ and $s(u_{b'}) = u_{a'}$.
  From the $s$-graph, delete all vertices in $V_{a'} \cup V_{b'}$ not incident with any bidirectional arc.
  Now consider the subgraph of the $s$-graph induced by $V_a \cup \cdots \cup V_{a'}$.
  As each arc is leading from $V_j$ to $V_{j+1}$ for $j \in [a, a']$, each weakly connected component is a directed tree rooted at some vertex in $V_{a'}$.
  We delete a vertex $v \in V_{a} \cup \cdots \cup V_{a'}$ from the $s$-graph, if it is not part of any path starting from $V_{a}$ and ending at $V_{a'}$.
  
  We further simplify the $s$-graph as follows.
  Henceforth, we will treat the $s$-graph as an undirected graph, disregarding the orientations.
  For every vertex $v$ in the $s$-graph that is part of $V_j$, $j \in [a+2, a']$, we randomly choose a vertex $s'(v) \in S_v \cap (V_{j-2} \cup V_{j-1})$.
  Based on the chosen $s'(v)$, we apply the following rules:
  \begin{itemize}
    \item If $s'(v) \in V_{j-1}$, then we preserve only the subtree rooted at $s'(v)$, eliminating all other subtrees connected to $v$. \item If $s'(v) \in V_{j-2}$, then let $v'\in V_{j-1}$ be the intermediate vertex linking $s'(v)$ to $v$.
    In this case, we keep the subtree rooted at $v'$, deleting all other subtrees connected to $v$.
  \end{itemize}
  We apply the rules similarly on $V_{b'} \cup \cdots \cup V_{b}$ as well.
  As a result, every connected component, when restricted to $V_{a+1} \cup \cdots \cup V_{b-1}$, transforms into a path.
  Note however that vertices in $V_{a+1}$ and $V_{b-1}$ may have multiple neighbors in $V_a$ and $V_{b}$ linking to them, thereby forming a double broom.
  Recall that a double broom is a tree obtained from a path by adding any number of degree-1 neighbors to its leaves.

  Next, we describe how to deal with missing edges in the underlying undirected graph of $H_C$.
  Suppose that neither $v_{a-1} v_a$ nor $v_{a} v_{a-1}$ is present in $H_C$.
  By the above discussion, there are weakly connected components which form double brooms.
  Suppose that they intersect blocks $V_{c} \cup \cdots \cup V_{a-1}$ and  $V_{a} \cup \cdots \cup V_{b}$.
  Denote those intersecting $V_{c} \cup \cdots \cup V_{a-1}$ and  $V_{a} \cup \cdots \cup V_{b}$ by $\cB^{-}$ and $\cB^+$, respectively.
  We show how to connect these double brooms.
  For each $B^+ \in \cB^+$, we randomly choose a double broom in $\cB^-$ denoted by $r(B^+)$ as follows.
  Let $v$ be the vertex in $V_{a+1}$ on $B^+$.
  We randomly choose a vertex $s'(v) \in S_v \cap (V_{a-1} \cap V_{a})$, and 
  apply one of the following rules:
  \begin{enumerate}
    \item
    If $s'(v) \in V_{a-1}$, then let $r(B^+)$ be the double broom in $\cB^-$ containing $s'(v)$.
    \item
    If $s'(v) \in V_{a}$, then we further choose $s''(s'(v)) \in S_{s'(v)} \cap (V_{a-2} \cup V_{a-1})$ uniformly at random.
    Let $r(B^+)$ be the double broom in $\cB^-$ containing $s''(s'(v))$.
  \end{enumerate}
  For each $B^- \in \cB^-$, we randomly choose $r(B^-) \in \cB^+$ similarly.
  We keep only the pairs $(B^+, B^-) \in \cB^+ \times \cB^-$ such that $r(B^+) = B^-$ and $r(B^-)= B^+$.
  Let $v^+$ be the vertex on $B^+$ in $V_{a+1}$ and $v^-$ be the vertex on $B^-$ in $V_{a-2}$.
  We connect double brooms as follows (see Figure~\ref{fig:1mod3:connection} for an illustration):
  \begin{enumerate}
    \item[(a)] Suppose that rule 1 is applied when choosing $r(B^+)$ and $r(B^-)$.
    Then, we connect $(v^-, s'(v^+), s'(v^-),\allowbreak v^+)$ as a path.
    \item[(b)] Suppose that rules 1 and 2 are applied when choosing $r(B^+)$ and $r(B^-)$, respectively (or vice versa).
    If $s''(s'(v^-)) \in V_a$, then we connect $(v^-, s'(v^-), s''(s'(v^-)), v^+)$ as a path.
    Otherwise, we have $s''(s'(v^-)) \in V_{a+1}$, and we connect as a path with parallel vertices: $(v^-, s'(v^-), (s')^{-1}(v^+) \cap V_{a}, v^+)$, where $(s')^{-1}(v^+)$ denotes the vertices $v$ with $s'(v) = v^+$.
    \item[(c)] Suppose that rule 2 is applied when choosing $r(B^+)$ and $r(B^-)$.
    Then, we connect $(v^-, s'(v^-), s'(v^+), \allowbreak v^+)$ as a path.
  \end{enumerate}
  We apply this argument to every $a \in [l]$ such that $v_{a-1} v_a$ and $v_a v_{a-1}$ are absent from $H_C$.
  This results in a collection of cycles with parallel vertices.
  We apply sanity check as follows.
  First, we keep only those that circle around the blocks exactly once.
  Second, we check whether every vertex is adjacent to the two preceding and two succeeding vertices.
  If the vertex fails adjacency test, we have two cases. 
  If the vertex is a parallel vertex, then delete it from the cycle.
  Otherwise, delete the entire cycle.
  Finally, let $S$ be the vertices covered the remaining cycles with parallel vertices.

  We claim both conditions of the lemma hold.
  For the item (i), note that the probability that for every vertex $v \in C$, $s(v)$, $s'(v)$, and $s''(v)$ belong to $C$ is at least $1/2^{O(l)}$.
  It is straightforward to verify that under this condition the cycle $C$, possibly with parallel vertices, remains until the step where it is included into $S$.
  The item (ii) holds because $S$ is precisely a set of vertices covered by a collection of strong cycles with parallel vertices.
\end{proof}

\begin{figure}
  \centering
  \begin{tikzpicture}[scale=1.5]
    \node at (-.7, .5) {(a)};
    \begin{scope}[shift={(0, -0.2)}]
    \node[vertex,label={180:$v^-$},label={[above=4mm]:$V_{a-2}$}] (v1) at (0, 0) {};
    \node[vertex,label={[above=4mm]:$V_{a-1}$}] (v2) at (1, 0) {};
    \node[vertex,label={[above=4mm]:$V_{a}$}] (v3) at (2, 0) {};
    \node[vertex,label={0:$v^+$},label={[above=4mm]:$V_{a+1}$}] (v4) at (3, 0) {};
    \draw[->] (v2) to[out=150, in=30] (v1);
    \draw[->] (v3) to[out=330, in=210] (v4);
    \draw[->] (v4) to[out=150, in=30] (v2);
    \draw[->] (v1) to[out=330, in=210] (v3);
    \end{scope}

    \begin{scope}[shift={(5, -.5)}]
    \node at (-.7, 1) {(b)};
    \node[vertex,label={180:$v^-$},label={[above=7mm]:$V_{a-2}$}] (v1) at (0, 0) {};
    \node[vertex,label={[above=7mm]:$V_{a-1}$}] (v2) at (1, 0) {};
    \node[vertex,label={[above=7mm]:$V_a$}] (v3) at (2, 0) {};
    \node[vertex,label={0:$v^+$},label={[above=7mm]:$V_{a+1}$}] (v4) at (3, 0) {};
    \node[vertex] (v31) at (2, -.4) {};
    \node[vertex] (v32) at (2, -.8) {};
    \draw[->] (v2) to[out=150, in=30] (v1);
    \draw[->] (v3) to[out=330, in=210] (v4);
    \draw[->] (v4) to[out=150, in=30] (v2);
    \draw[->] (v1) to[out=330, in=210] (v2);
    \draw[->, dashed, transform canvas={yshift=2.5mm}] (v2) to[out=30, in=150] (v4);
    \draw[->] (v31) to[out=345, in=210] (v4);
    \draw[->] (v32) to[out=0, in=210] (v4);
    \end{scope}

    \begin{scope}[shift={(0, -1.7)}]
    \node at (-.7, .7) {(c)};
    \node[vertex,label={180:$v^-$},label={[above=3mm]:$V_{a-2}$}] (v1) at (0, 0) {};
    \node[vertex,label={[above=3mm]:$V_{a-1}$}] (v2) at (1, 0) {};
    \node[vertex,label={[above=3mm]:$V_a$}] (v3) at (2, 0) {};
    \node[vertex,label={0:$v^+$},label={[above=3mm]:$V_{a+1}$}] (v4) at (3, 0) {};
    \draw[->] (v1) to[out=330, in=210] (v2);
    \draw[->] (v2) to[out=150, in=30] (v1);
    \draw[->] (v3) to[out=330, in=210] (v4);
    \draw[->] (v4) to[out=150, in=30] (v3);
    \draw[->, dashed] (v3) to[out=150, in=30] (v2);
    \draw[->, dashed] (v2) to[out=310, in=230] (v4);
    \end{scope}
    %  -- (v3) -- (v4);
  \end{tikzpicture}
  \caption{The connection between two double brooms. An arrow indicates $s'(\cdot)$, and a dashed arrow $s''(\cdot)$.}
  \label{fig:1mod3:connection}
\end{figure}


\subsubsection{Strong cycle of length $3p + 2$} \label{sssec:2mod3}

Suppose that $\cC$ contains a cycle $C = (v_{1}, \dots, v_{3p+2})$.
This situation is different from when the cycle's length is $3p + 1$, as we cannot  assume that all vertices in $C$ are included in the set $F$. Consequently, we need to guess for each vertex $v_i$ whether it is \emph{intersecting}, i.e., whether it is part of $F$. 
We call a vertex $v_i$ \emph{safe} if it is flowery or not intersecting.
We will start off with another observation with respect to the triangle-maximality.

\begin{lemma} \label{lemma:2mod3:unique}
  The common neighborhood $N(v_{j}) \cap N(v_{j+1}) \cap V_{j+2}$ is a singleton $\{ v_{j+2} \}$ for every $j \in [l_i]$.
\end{lemma}
\begin{proof}
  Assume for contradiction that, w.l.o.g., $v_{3p+1}$ and $v_{3p+2}$ has a common neighbor $v_{1}' \in V_{1}$ in addition to $v_{1}$.
  Then there is a triangle packing $\{ v_{3q-2}, v_{3q-1}, v_{3q} \mid q \in [p]\} \cup \{ v_{3p+1}, v_{3p+2}, v_{1}' \}$ that covers the strong cycle.
\end{proof}

In view of Lemma~\ref{lemma:2mod3:unique}, we obtain a polynomial-time algorithm that generates a collection of strong cycles in $V^i$, denoted by $\mathcal{S}$, such that $C \in \mathcal{S}$.
To simplify our terminology, we will extend the use of the terms ``flowery'', ``intersecting'', and ``safe'': a block $V_{j}$ will be described as flowery (intersecting, or safe) if its corresponding vertex $v_{j}$ is flowery (intersecting, or safe).

\begin{lemma} \label{lemma:2mod3:find-cycles}
  There is a polynomial-time algorithm that finds a collection of strong cycles
  $\mathcal{S}$ such that the following hold: (i) $C^i \in \mathcal{S}$ with probability at least $1/2^{O(l)}$ and (ii) strong cycles in $\mathcal{S}$ are vertex-disjoint on non-flowery blocks, i.e., for every vertex $v$ in a non-flowery block, there is at most one strong cycle in $\mathcal{S}$ covering $v$.
\end{lemma}
\begin{proof}
  In view of Lemma~\ref{lemma:2mod3:unique}, we obtain a polynomial-time algorithm that generates a collection of edge-disjoint strong cycles in $V_i$, denoted by $\cS'$, such that $C \in \mathcal{S}'$.
  For each edge $u_1 u_2$ with $u_1 \in V_{1}$ and $u_2 \in V_{2}$, this algorithm will find at most one strong cycle in $\cS$ as follows.
  If $u_1$ and $u_2$ share more than one common neighbor in $V_{3}$, then Lemma~\ref{lemma:2mod3:unique} allows us to conclude that $u_1 u_2$ is not part of a strong cycle in $\cC$.
  On the other hand, if $u_1$ and $u_2$ share a unique common neighbor $u_3$ in $V_{3}$, then we check whether there is a unique neighbor between $u_2$ and $u_3$ in $V_{4}$.
  This process continues iteratively until it circles back to $V_{1}$.
  Through this process, a strong cycle $(u_1, \dots, u_{3p+2})$ that is part of $\cC$ is identified.
  By construction, the strong cycles found this way are edge-disjoint.

  To ensure vertex-disjointness on non-flowery blocks, we randomly pick $\cS \subseteq \cS'$ as follows.
  For every non-flowery vertex $v \in V_j$, we randomly choose a neighbor $s(v)$ from $S_v \cap (V_{j-1} \cup V_{j+1})$, where $S_v$ is a vertex over on $N(v)$.
  Strong cycles $C'$ are deleted from $\cS'$ if there is a vertex $v \in V(C')$ such that $s(v) \notin V(C')$.
  Let $\cS$ denote the remaining strong cycles.
  We claim that strong cycles in $\mathcal{S}$ are vertex-disjoint on non-flowery blocks.
  For a vertex $v$ in a non-flowery block $V_j$, and it is possible that multiple strong cycles in $\cS'$ cover $v$.
  However, by the edge-disjointness, these cycles pass through different vertices in $V_{j-1}$ (and $V_{j+1}$).
  Since we keep at most one cycle in $\cS$, it follows that at most one strong cycle covering $v$ remains, thereby establishing vertex-disjointness.
  Note that $C^i \in \cS$ holds when $s(v_j) \in V(C)$ holds for all non-flowery vertices $v_j \in V(C)$.
  This occurrence has probability at least $1/2^{O(l)}$ by Lemma~\ref{lemma:vc}.
\end{proof}


\begin{figure}
  \centering
  \begin{tikzpicture}
    \begin{scope}[yscale=0.8]
    \node[vertex,label={[label distance=.3cm]below:$v_{1}$}] (f1) at (0, 0) {};
    \node[bvertex,label={above:$v_{2,1}$}] (v1a) at (1, 1) {};
    \node[vertex,label={below:$v_{2,2}$}] (v1b) at (1, -1) {};
    \node[bvertex,label={[label distance=.3cm]below:$v_{3}$}] (f2) at (2, 0) {};
    \node[vertex,label={above:$v_{4,1}$}] (v2a) at (3, 1) {};
    \node[bvertex,label={below:$v_{4,2}$}] (v2b) at (3, -1) {};
    \node[bvertex,label={above:$v_{5,1}$}] (v3a) at (4, 1) {};
    \node[vertex,label={below:$v_{5,2}$}] (v3b) at (4, -1) {};
    \node[bvertex,label={[label distance=.3cm]below:$v_{6}$}] (f3) at (5, 0) {};
    \node[bvertex,label={above:$v_{7,1}$}] (v4a) at (6, 1) {};
    \node[vertex,label={below:$v_{7,2}$}] (v4b) at (6, -1) {};
    \node[bvertex,label={above:$v_{8,1}$}] (v5a) at (7, 1) {};
    \node[vertex,label={below:$v_{8,2}$}] (v5b) at (7, -1) {};


    \draw (v1a) -- (f1) -- (v1b);
    \draw (v1a) -- (f2) -- (v1b);
    \draw (v2a) -- (f2) -- (v2b);
    \draw (v2a) -- (v3a) -- (f3) -- (v3b) -- (v2b);
    \draw (v5a) -- (v4a) -- (f3) -- (v4b) -- (v5b);
    \draw (v5a) -- (7.5, 0.5) {};
    \draw (v5b) -- (7.5, -0.5) {};
    \draw[dotted] (v5a) -- (7.8, 0.2) {};
    \draw[dotted] (v5b) -- (7.8, -0.2) {};
    \draw (-0.5, 0.5) -- (f1) -- (-0.5, -0.5);
    \draw[dotted] (-0.8, 0.8) -- (f1) -- (-0.8, -0.8);
    \end{scope}

    \flowervertex{0}{0}
    \flowervertex{2}{0}
    \flowervertex{5}{0}
  \end{tikzpicture}
  \caption{}
  \label{fig:2mod3:flowery}
\end{figure}

Lemma~\ref{lemma:2mod3:find-cycles} is sufficiently powerful to obtain a statement similar to Lemma~\ref{lemma:1mod3:ultimate}, when there is no flowery vertex in $C$:
\begin{lemma} \label{lemma:2mod3:no-flowery}
  If $C$ contains no flowery vertex,
  then there is a polynomial-time algorithm that finds a set $S$ of vertices such that (i) $V(C) \subseteq S$ with probability at least $1/2^{O(l)}$ and (ii) for any $X \subseteq S$ such that $|X \cap V_j| = 1$ for each intersecting block $V_j$ and $|X \cap V_j| = 0$ for each non-intersecting block, there is a strong set that covers $X$.
\end{lemma}
\begin{proof}
  Define $S$ as the set of vertices covered by strong cycles in $\cS$ of Lemma~\ref{lemma:2mod3:find-cycles}.
  Since $C$ belongs to $\cS$ with probability at least $1/2^{O(l)}$, the condition (i) holds.
  Furthermore, it is possible to cover a set $X \subseteq S$, where $|X \cap V_j| \le 1$ for each flowery block $V_j$, with a strong set.
\end{proof}

Let us assume that $C$ includes at least one flowery vertex.
Note that the statement of Lemma~\ref{lemma:2mod3:no-flowery} may not apply.
Specifically, the strategy of simply selecting $S$ to be the set of vertices covered by strong cycles in $\cS$ is not viable because the condition (ii) may not be met in the presence of flowery vertices, as illustrated in Figure~\ref{fig:2mod3:flowery}.
Here, there are two strong cycles $(v_1, v_{2,i}, v_3, v_{4,i}, v_{5,i}, v_6, v_{7,i}, v_{8,i})$ for $i = 1, 2$, with vertices in $X$ marked in black.
We claim that there is no strong set covering $X$.
As $v_{4,2} \in X$, either $\{ v_3, v_{4,2}, v_{5,2} \}$ or $\{ v_{4,2}, v_{5,2}, v_6 \}$ needs to be part of the strong set.
Similarly, as $v_{5,1}$, either $\{ v_3, v_{5,1}, v_{5,2} \}$ or $\{ v_{5,1}, v_{5,2}, v_6 \}$ needs to be part of the strong set.
It follows that both $v_3$ and $v_6$ have been covered.
Thus, $\{ v_1, v_{7, 1}, v_{8,1} \}$ must be included into the strong set as well, leaving $v_{2,1}$ uncovered.
This example underscores the limitation of Lemma~\ref{lemma:2mod3:find-cycles} to provide a statement similar to Lemma~\ref{lemma:1mod3:ultimate}.
To work around this challenge, we leverage the idea of pairing some vertices via reduction to \textsc{Colorful Delta-matroid Matching}.
In order to apply this idea, let us give a few more observations with respect to triangle maximality.

\begin{lemma} \label{lemma:2mod3:consecutive}
  No two consecutive vertices $v_j$ and $v_{j+1}$ are safe.
\end{lemma}
\begin{proof}
  Assume for contradiction that two consecutive vertices $v_{3p+1}$ and $v_{3p+2}$ are both safe.
  Since $C$ is a strong cycle, there are triangles $\{ v_{3q-2}, v_{3q-1}, v_{3q}\}$ for each $q \in [p]$.
  If $v_{3p+1}$ (or $v_{3p+2}$) is flowery, then it implies the existence of an additional triangle containing $v_{3p+1}$ (or $v_{3p+2}$) that are disjoint from all other triangles (these are two separate triangles if both $v_{3p+1}$ and $v_{3p+2}$ are flowery).
  We can choose such triangles by the definition of flowery vertices.
  The packing of these triangles yields a solution with a greater number of triangles, a contradiction.
\end{proof}

By Lemma~\ref{lemma:2mod3:consecutive}, we may assume w.l.o.g.\ that $v_{1}$ is not safe and that $v_{3p+2}$ is safe.
Let $f_1, \dots, f_r$ be the safe vertices in $C_i$.
These vertices partition $C_i$ into paths $P_1, \dots, P_r$ of non-safe vertices, where the endpoints of $P_i$ are adjacent to $f_{i-1}$ and $f_i$.

\begin{lemma} \label{lemma:2mod3:not-divisible}
  For each $r' \in [r]$, $|V(P_r)|$ is not divisible by 3.
\end{lemma}
\begin{proof}
  If a path, say $P_r$, has length divisible by three, then we can cover $F \cap V(C_i)$ with a triangle packing: 
  \begin{itemize}
    \item 
    $V(P_r)$ can be covered by $|V(P_r)|/3$ disjoint triangles.
    \item
    If $f_{r-1}$ (and $f_r$) is flowery, then it can be covered by disjoint triangles.
    \item
    The remaining vertices form a path whose number of vertices is divisible by three, which can be covered by a triangle packing.
  \end{itemize}
  This contradicts the triangle maximality.
\end{proof}

Lemma~\ref{lemma:2mod3:not-divisible} can be further strengthened as follows.

\begin{lemma} \label{lemma:2mod3:path-lengths}
  There is exactly one path $P_{r'}$ such that $|V(P_{r'})| \bmod 3 = 1$ and for all other paths $P_{r''}$, $|V(P_{r''})| \bmod 3 = 2$.
\end{lemma}
\begin{proof}
  Suppose that there exist two paths $P_{r'}$ and $P_{r''}$ for $r' < r'' \in [r]$ such that both paths have order $1 \bmod 3$.
  By Lemma~\ref{lemma:2mod3:not-divisible}, we may assume w.l.o.g.\ that for every $r''' \in [r', r'' - 1]$, the path $P_{r'''}$ has order $2 \bmod 3$.
  Observe that $f_{r'-1}$ and $f_{r''}$ give a partition into two paths of order divisible by 3:
  \begin{itemize}
    \item The path containing $P_{r'}$ and $P_{r''}$ includes (i) two paths with lengths of $1 \bmod 3$, (ii) $r'' - r' - 1$ paths with lengths of $2 \bmod 3$, and (iii) $r'' - r'$ safe vertices, resulting in a total order divisible by 3, specifically $2 + 2(r'' - r' - 1) + (r'' - r') \equiv 0 \bmod 3$.
    \item The other path also has order $0 \bmod 3$ as well because $C$ has $3p + 2$ vertices.
  \end{itemize}
  This implies that both paths can be covered by disjoint triangles, which contradicts the triangle maximality.
\end{proof}

Given Lemma~\ref{lemma:2mod3:path-lengths}, we may assume w.l.o.g.\ that $|V(P_1)| \bmod 3 = 1$, and that $|V(P_{r'})| \bmod 3 = 2$ for $2 \le r' \le r$.
Under this assumption, we have the following:

\begin{lemma} \label{lemma:2mod3:flowery-index}
  If $v_{j}$ is flowery, then $j \bmod 3 = 2$.
\end{lemma}
\begin{proof}
  Assuming that $v_j = f_r$, we have
  \begin{align*}
    j \equiv (|V(P_1)| + 1) + \sum_{r' = 2}^r (|V(P_{r'})| + 1) \equiv 2
  \end{align*}
  because $|V(P_{r'})| + 1$ is divisible by 3 for each $r' \in [2, r]$.
\end{proof}


% In our algorithm, we pair the following.
% More precisely, in the context of delta-matroid matching, for each $S \in \cS$, we introduce edges as follows:
% \begin{itemize}
%   \item
%   An edge between $V_{i,3q-2}$ and $V_{i,3q-1}$ for each $q \in [\frac{1}{3}(\tau(1) - 1)]$,
%   \item
%   an edge between $V_{i,\tau(1)-1}$ and $V_{\tau(1)}$,
%   \item
%   an edge between $V_{i,\sigma(r')+3q-2}$ and $V_{i,\sigma(r')+3q-1}$ for each $2 \le r' \le r$ and $q \in [\frac{1}{3}(\tau(r')-\sigma(r')-1)]$, and
%   \item
%   an edge between $V_{i,\tau(r')-1}$ and $V_{i,\tau(r')}$ for each $2 \le r' \le r$.
% \end{itemize}

% Example: Suppose that the strong cycle has length 17 with flowery vertices at 8 and 17.
% The pairing is at (1, 2), (4, 5), (6, 7), (10, 11), (13, 14), and (15, 16).
\newcommand{\at}{\mathsf{A}}
\newcommand{\tria}{\mathsf{T}}
\newcommand{\scycle}{\mathsf{S}}

Suppose that $P_1$ is on $3 q_1 + 1$ vertices and that $P_{r'}$ is on $3q_{r'} + 2$ vertices for each $r' \in [2, r]$.
To facilitate the manipulation of indices, we introduce a notation $\at(r', t)$, defined as the integer $j$ such that $v_j$ is the $t$-th vertex of $P_{r'}$.
Additionally, for a strong cycle $S \in \cS$, let $\tria(S, j)$ denote the triangle in $S$ that intersects $V_{j}, V_{j + 1}$, and $V_{j + 2}$.

Now we are ready to give a statement similar to Lemma~\ref{lemma:1mod3:ultimate} with specific pairing conditions.
Specifically, for a pair of vertices in a strong cycle $S \in \cS$, we require that both must be part of $F$, or both must be outside $F$.
At its core, we construct the pairing as follows: for any safe vertex, the two vertices immediately following it, as well as the two vertices preceding it are paired (see Figure~\ref{fig:2mod3:pairing} for an example).
This aims to ensure the coverage of these two vertices, possibly in conjunction with the safe vertex.
Yet, the situation is still complex.
Notably, in certain scenarios (case 4 in the proof), the packing involves a triangle with a safe vertex alongside its preceding vertex and following vertex.
As these two vertices are not paired, this triangle is not necessarily available, especially when these two vertices are part of different strong cycles that intersect at the safe vertex.

\begin{figure}
  \centering
  \begin{tikzpicture}
    \begin{scope}[xscale=0.9, yscale=0.72]
    \foreach \i in {0,1,2,...,16} {
      \pgfmathtruncatemacro{\nxt}{\i + 1}
      \node[vertex,label={[label distance=2.5mm]above:$V_{\nxt}$}] (v\i) at (\i, 0) {};
      % \node[vertex] (v\i) at (\i, 0) {};
    }
    \foreach \i in {0,1,2,...,15} {
      \pgfmathtruncatemacro{\nxt}{\i + 1}
      \draw (v\i) -- (v\nxt);
    }

    % \draw (v1a) -- (f1) -- (v1b);
    % \draw (v1a) -- (f2) -- (v1b);
    % \draw (v2a) -- (f2) -- (v2b);
    % \draw (v2a) -- (v3a) -- (f3) -- (v3b) -- (v2b);
    % \draw (v5a) -- (v4a) -- (f3) -- (v4b) -- (v5b);
    % \draw (v5a) -- (7.5, 0.5) {};
    % \draw (v5b) -- (7.5, -0.5) {};
    % \draw[dotted] (v5a) -- (7.8, 0.2) {};
    % \draw[dotted] (v5b) -- (7.8, -0.2) {};
    % \draw (-0.5, 0.5) -- (f1) -- (-0.5, -0.5);
    % \draw[dotted] (-0.8, 0.8) -- (f1) -- (-0.8, -0.8);
    \draw (-0.5, 0) -- (v0);
    \draw[dotted] (-0.8, 0) -- (v0);
    \draw (16.5, 0) -- (v10);
    \draw[dotted] (16.8, 0) -- (v10);

    \draw [decorate,decoration={brace,amplitude=1.5mm,mirror,raise=4mm}] (-0.1,0) -- (1.1,0);
    \draw [decorate,decoration={brace,amplitude=1.5mm,mirror,raise=4mm}] (1.9,0) -- (3.1,0);
    \draw [decorate,decoration={brace,amplitude=1.5mm,mirror,raise=4mm}] (4.9,0) -- (6.1,0);
    \draw [decorate,decoration={brace,amplitude=1.5mm,mirror,raise=4mm}] (7.9,0) -- (9.1,0);
    \draw [decorate,decoration={brace,amplitude=1.5mm,mirror,raise=4mm}] (10.9,0) -- (12.1,0);
    \draw [decorate,decoration={brace,amplitude=1.5mm,mirror,raise=4mm}] (13.9,0) -- (15.1,0);

    \end{scope}

    \flowervertex{7 * 0.9}{0}
    \flowervertex{16 * 0.9}{0}
  \end{tikzpicture}
  \caption{An illustration of the pairing constraints in Lemma~\ref{lemma:2mod3:ultimate} on a cycle of length 17.}
  \label{fig:2mod3:pairing}
\end{figure}

\begin{lemma} \label{lemma:2mod3:ultimate}
  Let $X \subseteq V^i$ be a set such that $X \cap V_j = \{ x_j \}$ for each intersecting block $V_j$ and $X \cap V_j = \emptyset$ for each non-intersecting block $V_j$.
  For $r' \in [r]$ and $t \in \N$,
  define $\scycle(r', t)$ as the strong cycle in $\cS$ covering $x_{\at(r', t)} \in X$.
  Suppose that the following hold:
  \begin{itemize}
    \item If  $|V(P_1)| \ne 1$, then $\scycle(1, 1) = \scycle(1, 2)$.
    \item For each $q \in [q_1]$, $\scycle(1, 3q) = \scycle(1, 3q + 1)$.
    \item For each $r' \in [2, r]$ and $q \in [q_{r'} + 1]$, $\scycle(r', 3q - 2) = \scycle(r', 3q - 1)$.
  \end{itemize}
  Under these conditions on $X$, it is possible to cover $X$ with a packing of strong cycles within $V^i$.
\end{lemma}
\begin{proof}
  Consider $x_j \in X$ such that $V_{j}$ is flowery.
  Our goal is to construct a strong cycle packing where each block $V_{j}$ is covered by at most three of its vertices.
  Given that $x_j$ is part of at least 13 triangles intersecting at $x_j$, it follows that we can select a disjoint triangle to cover $x_j$.

  We show that all $x_j$'s within non-safe blocks $V_j$ can be covered with a strong cycle packing as well.
  To that end, we initiate our analysis with respect to $Z = \scycle(1, 1)$, which is the strong cycle in $\cS$ covering~$x_1$.
  Our technical insight is the identification of the following four cases:
  \begin{enumerate}
    \item $\scycle(1, 3q_1 + 1) = Z$ and that $\scycle(r', 1) = \scycle(r', 3q_{r'} + 2) =  Z$ for each $r' \in [2, r]$. 
    \item $\scycle(1, 3q_1 + 1) \ne Z$.
    \item There exists $r' \in [2, r]$ such that $\scycle(r', 1) \ne \scycle(r', 3q_{r'} + 2)$.
    \item The remaining cases.
  \end{enumerate}
  For each case, we will establish a strategy for covering $X$.
  See Figure~\ref{fig:2mod3:ultimate} for an example of each case.

\begin{figure}
  \centering
  \begin{tikzpicture}
  \begin{scope}
    \begin{scope}[xscale=0.7, yscale=0.56]
      \node at (-.8, 2.5) {Case 1.};
      
      % begin: case specific
      \draw[gray, rounded corners=10pt, line width=5mm, opacity=0.5, line cap=round] (-0.8, 0.2) -- (0, 1) -- (6, 1) -- (7, 0) -- (8, 1) -- (9, 1) -- (10, 0) -- (11, 1) -- (15, 1) -- (18, 1) -- (19, 0) -- (19.8, 0.8);
      \draw[gray, rounded corners=10pt, line width=5mm, opacity=0.5, line cap=round] (2, -1) -- (4, -1);
      \draw[gray, rounded corners=10pt, line width=5mm, opacity=0.5, line cap=round] (13, -1) -- (15, -1);
      \draw[gray, rounded corners=10pt, line width=5mm, opacity=0.5, line cap=round] (16, -1) -- (18, -1);
      % end: case specific

      \foreach \i in {0,1,2,...,6} {
        \node[vertex] (va\i) at (\i, 1) {};
        \node[vertex] (vb\i) at (\i, -1) {};
      }
      \foreach \i in {0,1,2,...,5} {
        \pgfmathtruncatemacro{\nxt}{\i + 1}
        \draw (va\i) -- (va\nxt);
        \draw (vb\i) -- (vb\nxt);
      }

      \foreach \i in {0,1} {
        \node[vertex] (wa\i) at (\i + 8, 1) {};
        \node[vertex] (wb\i) at (\i + 8, -1) {};
      }
      \draw (wa0) -- (wa1);
      \draw (wb0) -- (wb1);

      \foreach \i in {0,1,2,...,7} {
        \node[vertex] (ua\i) at (\i + 11, 1) {};
        \node[vertex] (ub\i) at (\i + 11, -1) {};
      }
      \foreach \i in {0,1,2,...,6} {
        \pgfmathtruncatemacro{\nxt}{\i + 1}
        \draw (ua\i) -- (ua\nxt);
        \draw (ub\i) -- (ub\nxt);
      }

      \node[vertex] (f1) at (7, 0) {};
      \node[vertex] (f2) at (10, 0) {};
      \node[vertex] (f3) at (19, 0) {};

      \draw (va6) -- (f1) -- (vb6);
      \draw (wa0) -- (f1) -- (wb0);
      \draw (ua0) -- (f2) -- (ub0);
      \draw (wa1) -- (f2) -- (wb1);
      \draw (ua7) -- (f3) -- (ub7);

      \draw[dotted] (19.8, 0.8) -- (f3) -- (19.8, -0.8) {};
      \draw (19.5, 0.5) -- (f3) -- (19.5, -0.5) {};
      \draw (-0.5, 0.5) -- (va0);
      \draw[dotted] (-0.8, 0.2) -- (va0);
      \draw (-0.5, -0.5) -- (vb0);
      \draw[dotted] (-0.8, -0.2) -- (vb0);

      % begin: case specific 
      \foreach \i in {0,1,4,5,6} {
        \node[bvertex] at (\i, 1) {};
      }
      \foreach \i in {2, 3} {
        \node[bvertex] at (\i, -1) {};
      }
      \foreach \i in {0, 1} {
        \node[bvertex] at (\i + 8, 1) {};
      }
      \foreach \i in {0, 1, 2, 6, 7} {
        \node[bvertex] at (\i + 11, 1) {};
      }
      \foreach \i in {3, 4, 5} {
        \node[bvertex] at (\i + 11, -1) {};
      }
      % end: case specific 
    \end{scope}

    \flowervertex{7 * 0.7}{0}
    \flowervertex{10 * 0.7}{0}
    \flowervertex{19 * 0.7}{0}
  \end{scope}

  \begin{scope}[shift={(0, -3.5)}]
    \begin{scope}[xscale=0.7, yscale=0.56]
      \node at (-.8, 2.5) {Case 2.};
        
      % begin: case specific
      \draw[gray, rounded corners=5pt, line width=5mm, opacity=0.5, line cap=round] (0, 1) -- (2, 1);
      \draw[gray, rounded corners=5pt, line width=5mm, opacity=0.5, line cap=round] (3, 1) -- (5, 1);
      \draw[gray, rounded corners=5pt, line width=5mm, opacity=0.5, line cap=round] (2, -1) -- (4, -1);
      \draw[gray, rounded corners=5pt, line width=5mm, opacity=0.5, line cap=round] (5, -1) -- (6, -1) -- (7, 0);
      \draw[gray, rounded corners=5pt, line width=5mm, opacity=0.5, line cap=round] (8, -1) -- (9, -1) -- (10, 0);
      \draw[gray, rounded corners=5pt, line width=5mm, opacity=0.5, line cap=round] (11, 1) -- (13, 1);
      \draw[gray, rounded corners=5pt, line width=5mm, opacity=0.5, line cap=round] (11, -1) -- (13, -1);
      \draw[gray, rounded corners=5pt, line width=5mm, opacity=0.5, line cap=round] (14, -1) -- (16, -1);
      \draw[gray, rounded corners=5pt, line width=5mm, opacity=0.5, line cap=round] (17, 1) -- (18, 1) -- (19, 0);
      % end: case specific

      \foreach \i in {0,1,2,...,6} {
        \node[vertex] (va\i) at (\i, 1) {};
        \node[vertex] (vb\i) at (\i, -1) {};
      }
      \foreach \i in {0,1,2,...,5} {
        \pgfmathtruncatemacro{\nxt}{\i + 1}
        \draw (va\i) -- (va\nxt);
        \draw (vb\i) -- (vb\nxt);
      }

      \foreach \i in {0,1} {
        \node[vertex] (wa\i) at (\i + 8, 1) {};
        \node[vertex] (wb\i) at (\i + 8, -1) {};
      }
      \draw (wa0) -- (wa1);
      \draw (wb0) -- (wb1);

      \foreach \i in {0,1,2,...,7} {
        \node[vertex] (ua\i) at (\i + 11, 1) {};
        \node[vertex] (ub\i) at (\i + 11, -1) {};
      }
      \foreach \i in {0,1,2,...,6} {
        \pgfmathtruncatemacro{\nxt}{\i + 1}
        \draw (ua\i) -- (ua\nxt);
        \draw (ub\i) -- (ub\nxt);
      }

      \node[vertex] (f1) at (7, 0) {};
      \node[vertex] (f2) at (10, 0) {};
      \node[vertex] (f3) at (19, 0) {};

      \draw (va6) -- (f1) -- (vb6);
      \draw (wa0) -- (f1) -- (wb0);
      \draw (ua0) -- (f2) -- (ub0);
      \draw (wa1) -- (f2) -- (wb1);
      \draw (ua7) -- (f3) -- (ub7);

      \draw[dotted] (19.8, 0.8) -- (f3) -- (19.8, -0.8) {};
      \draw (19.5, 0.5) -- (f3) -- (19.5, -0.5) {};
      \draw (-0.5, 0.5) -- (va0);
      \draw[dotted] (-0.8, 0.2) -- (va0);
      \draw (-0.5, -0.5) -- (vb0);
      \draw[dotted] (-0.8, -0.2) -- (vb0);

      % begin: case specific 
      \foreach \i in {0,1,2,3} {
        \node[bvertex] at (\i, 1) {};
      }
      \foreach \i in {4, 5, 6} {
        \node[bvertex] at (\i, -1) {};
      }
      \foreach \i in {0, 1} {
        \node[bvertex] at (\i + 8, -1) {};
      }
      \foreach \i in {0, 1, 6, 7} {
        \node[bvertex] at (\i + 11, 1) {};
      }
      \foreach \i in {2, 3, 4, 5} {
        \node[bvertex] at (\i + 11, -1) {};
      }
      % end: case specific 
    \end{scope}

    \flowervertex{7 * 0.7}{0}
    \flowervertex{10 * 0.7}{0}
    \flowervertex{19 * 0.7}{0}
  \end{scope}

  \begin{scope}[shift={(0, -7)}]
    \begin{scope}[xscale=0.7, yscale=0.56]
      \node at (-.8, 2.5) {Case 3.};
        
      % begin: case specific
      \draw[gray, rounded corners=10pt, line width=5mm, opacity=0.5, line cap=round] (-0.8, 0.2) -- (0, 1) -- (1, 1);
      \draw[gray, rounded corners=10pt, line width=5mm, opacity=0.5, line cap=round] (19, 0) -- (19.8, 0.8);
      \draw[gray, rounded corners=10pt, line width=5mm, opacity=0.5, line cap=round] (2, 1) -- (4, 1);
      \draw[gray, rounded corners=10pt, line width=5mm, opacity=0.5, line cap=round] (5, 1) -- (6, 1) -- (7, 0);
      \draw[gray, rounded corners=10pt, line width=5mm, opacity=0.5, line cap=round] (8, -1) -- (9, -1) -- (10, 0);
      \draw[gray, rounded corners=10pt, line width=5mm, opacity=0.5, line cap=round] (11, 1) -- (13, 1);
      \draw[gray, rounded corners=10pt, line width=5mm, opacity=0.5, line cap=round] (11, -1) -- (13, -1);
      \draw[gray, rounded corners=10pt, line width=5mm, opacity=0.5, line cap=round] (14, 1) -- (16, 1);
      \draw[gray, rounded corners=10pt, line width=5mm, opacity=0.5, line cap=round] (16, -1) -- (18, -1);
      % end: case specific

      \foreach \i in {0,1,2,...,6} {
        \node[vertex] (va\i) at (\i, 1) {};
        \node[vertex] (vb\i) at (\i, -1) {};
      }
      \foreach \i in {0,1,2,...,5} {
        \pgfmathtruncatemacro{\nxt}{\i + 1}
        \draw (va\i) -- (va\nxt);
        \draw (vb\i) -- (vb\nxt);
      }

      \foreach \i in {0,1} {
        \node[vertex] (wa\i) at (\i + 8, 1) {};
        \node[vertex] (wb\i) at (\i + 8, -1) {};
      }
      \draw (wa0) -- (wa1);
      \draw (wb0) -- (wb1);

      \foreach \i in {0,1,2,...,7} {
        \node[vertex] (ua\i) at (\i + 11, 1) {};
        \node[vertex] (ub\i) at (\i + 11, -1) {};
      }
      \foreach \i in {0,1,2,...,6} {
        \pgfmathtruncatemacro{\nxt}{\i + 1}
        \draw (ua\i) -- (ua\nxt);
        \draw (ub\i) -- (ub\nxt);
      }

      \node[vertex] (f1) at (7, 0) {};
      \node[vertex] (f2) at (10, 0) {};
      \node[vertex] (f3) at (19, 0) {};

      \draw (va6) -- (f1) -- (vb6);
      \draw (wa0) -- (f1) -- (wb0);
      \draw (ua0) -- (f2) -- (ub0);
      \draw (wa1) -- (f2) -- (wb1);
      \draw (ua7) -- (f3) -- (ub7);

      \draw[dotted] (19.8, 0.8) -- (f3) -- (19.8, -0.8) {};
      \draw (19.5, 0.5) -- (f3) -- (19.5, -0.5) {};
      \draw (-0.5, 0.5) -- (va0);
      \draw[dotted] (-0.8, 0.2) -- (va0);
      \draw (-0.5, -0.5) -- (vb0);
      \draw[dotted] (-0.8, -0.2) -- (vb0);

      % begin: case specific 
      \foreach \i in {0,1,2,3,4,5,6} {
        \node[bvertex] at (\i, 1) {};
      }
      \foreach \i in {0, 1} {
        \node[bvertex] at (\i + 8, -1) {};
      }
      \foreach \i in {0, 1, 3, 4} {
        \node[bvertex] at (\i + 11, 1) {};
      }
      \foreach \i in {2, 5, 6, 7} {
        \node[bvertex] at (\i + 11, -1) {};
      }
      % end: case specific 
    \end{scope}

    \flowervertex{7 * 0.7}{0}
    \flowervertex{10 * 0.7}{0}
    \flowervertex{19 * 0.7}{0}
  \end{scope}

  \begin{scope}[shift={(0, -10.5)}]
    \begin{scope}[xscale=0.7, yscale=0.56]
      \node at (-.8, 2.5) {Case 4.};
        
      % begin: case specific
      \draw[gray, rounded corners=10pt, line width=5mm, opacity=0.5, line cap=round] (0, 1) -- (2, 1);
      \draw[gray, rounded corners=10pt, line width=5mm, opacity=0.5, line cap=round] (3, 1) -- (5, 1);
      \draw[gray, rounded corners=10pt, line width=5mm, opacity=0.5, line cap=round] (6, 1) -- (7, 0) -- (8, 1);
      \draw[gray, rounded corners=10pt, line width=5mm, opacity=0.5, line cap=round] (9, 1) -- (10, 0) -- (11, 1);
      \draw[gray, rounded corners=10pt, line width=5mm, opacity=0.5, line cap=round] (11, -1) -- (13, -1);
      \draw[gray, rounded corners=10pt, line width=5mm, opacity=0.5, line cap=round] (14, 1) -- (16, 1);
      \draw[gray, rounded corners=10pt, line width=5mm, opacity=0.5, line cap=round] (14, -1) -- (16, -1);
      \draw[gray, rounded corners=10pt, line width=5mm, opacity=0.5, line cap=round] (17, -1) -- (18, -1) -- (19, 0);
      % end: case specific

      \foreach \i in {0,1,2,...,6} {
        \node[vertex] (va\i) at (\i, 1) {};
        \node[vertex] (vb\i) at (\i, -1) {};
      }
      \foreach \i in {0,1,2,...,5} {
        \pgfmathtruncatemacro{\nxt}{\i + 1}
        \draw (va\i) -- (va\nxt);
        \draw (vb\i) -- (vb\nxt);
      }

      \foreach \i in {0,1} {
        \node[vertex] (wa\i) at (\i + 8, 1) {};
        \node[vertex] (wb\i) at (\i + 8, -1) {};
      }
      \draw (wa0) -- (wa1);
      \draw (wb0) -- (wb1);

      \foreach \i in {0,1,2,...,7} {
        \node[vertex] (ua\i) at (\i + 11, 1) {};
        \node[vertex] (ub\i) at (\i + 11, -1) {};
      }
      \foreach \i in {0,1,2,...,6} {
        \pgfmathtruncatemacro{\nxt}{\i + 1}
        \draw (ua\i) -- (ua\nxt);
        \draw (ub\i) -- (ub\nxt);
      }

      \node[vertex] (f1) at (7, 0) {};
      \node[vertex] (f2) at (10, 0) {};
      \node[vertex] (f3) at (19, 0) {};

      \draw (va6) -- (f1) -- (vb6);
      \draw (wa0) -- (f1) -- (wb0);
      \draw (ua0) -- (f2) -- (ub0);
      \draw (wa1) -- (f2) -- (wb1);
      \draw (ua7) -- (f3) -- (ub7);

      \draw[dotted] (19.8, 0.8) -- (f3) -- (19.8, -0.8) {};
      \draw (19.5, 0.5) -- (f3) -- (19.5, -0.5) {};
      \draw (-0.5, 0.5) -- (va0);
      \draw[dotted] (-0.8, 0.2) -- (va0);
      \draw (-0.5, -0.5) -- (vb0);
      \draw[dotted] (-0.8, -0.2) -- (vb0);

      % begin: case specific 
      \foreach \i in {0,1,2,3,4,5,6} {
        \node[bvertex] at (\i, 1) {};
      }
      \foreach \i in {0, 1} {
        \node[bvertex] at (\i + 8, 1) {};
      }
      \foreach \i in {3, 4} {
        \node[bvertex] at (\i + 11, 1) {};
      }
      \foreach \i in {0, 1, 2, 5, 6, 7} {
        \node[bvertex] at (\i + 11, -1) {};
      }
      % end: case specific 
    \end{scope}

    \flowervertex{7 * 0.7}{0}
    \flowervertex{10 * 0.7}{0}
    \flowervertex{19 * 0.7}{0}
  \end{scope}
  \end{tikzpicture}
  \vspace{5ex}
  \caption{Examples of strong cycle packings in the proof of Lemma~\ref{lemma:2mod3:ultimate}. There are two strong cycles that intersect two flowery vertices. The vertices in $X$, which satisfy the condition specified in Lemma~\ref{lemma:1mod3:ultimate}, are marked in black. For each of the four cases, a strong cycle packing that covers $X$ is indicated in gray.}
  \label{fig:2mod3:ultimate}
\end{figure}


  \paragraph*{Case 1.}
  Suppose that $\scycle(1, 3q_1 + 1) = Z$ and that $\scycle(r', 1) = \scycle(r', 3q_{r'} + 2) =  Z$ for each $r' \in [2, r]$. 
  We construct a strong cycle packing $\cP$ that covers $X$ as follows.
  Initially, let $\cP = \{ Z \}$.
  \begin{itemize}
    \item 
    The part of $X$ corresponding to $P_1$ can be covered as follows.
    First, note that the part corresponding to the first and last two vertices of $P_1$ is covered by $Z$.
    For the remaining part of $P_1$, add to $\cP$ the triangles $\tria(\scycle(1, 3q), 3q) = \tria(\scycle(1, 3q+1), 3q)$ and $\tria(\scycle(1, 3q+2), 3q)$ for $q \in [q_1 - 1]$.
    Since strong cycles in $\cS$ are vertex-disjoint at non-flowery blocks as specified by Lemma~\ref{lemma:2mod3:find-cycles}, any two of these triangles are either identical or entirely vertex-disjoint.
    \item
    Next, we consider the part of $X$ corresponding to $P_{r'}$ for $r' \in [2, r]$.
    By the condition of the lemma, $\scycle(r', 2) = \scycle(r', 3q_{r'} + 1) =  Z$ for each $r' \in [2, r]$.
    Thus, the first two vertices $x_{\at(r',1)}, x_{\at(r', 2)}$ as well as the last two vertices $x_{\at(r', 3q_{r'} + 1)}, x_{\at(r', 3q_{r'} + 2)}$ are covered by $Z$.
    As above, add to $\cP$ the triangles $\tria(\scycle(1, 3q), \at(r', 3q))$ and  $\tria(\scycle(1, 3q+1), \at(r', 3q)) = \tria(\scycle(1, 3q+2), \at(r', 3q))$ for $q \in [q_{r'} - 1]$.
    These cover all but $x_{\at(r', 3q_{r'})}$.
    If it has not been covered, i.e., $\scycle(r', 3q_{r'}) \ne Z$, then add the triangle $\tria(\scycle(r', 3q_{r'}), r', 3q)$ to $\cP$, which is disjoint from $Z$.
  \end{itemize}

  \paragraph{Case 2.}
  Suppose that $\scycle(1, 1) \ne \scycle(1, 3q_{1} + 1)$.
  We consider the smallest integer $q$ such that $\scycle(1, 3q) \ne Z$
  Note that such an integer always exists, as $\scycle(1, 3q_{1} + 1) = \scycle(1, 3q_{1})$ by the condition of the lemma.
  Now we construct a triangle packing $\cP$ that covers $X$.
  We first consider the part of $X$ corresponding to $P_1$:
  \begin{itemize}
    \item For each $q' \in [q]$, include a triangle $\tria(\scycle(1, 3q'-2), 1, 3q'-2)$.
    Note that these triangles cover $x_{1}, x_{2}$, and $x_{3q'}$ and $x_{3q'+1}$ for each $q' \in [q-1]$.
    \item For each $q' \in [q, q_{1}]$, include a triangle $\tria(\scycle(1, 3q'), 3q')$.
    Although both $\tria(\scycle(3q-2), \at(1, 3q-2))$ and $\tria(\scycle(1, 3q), \at(1, 3q))$ intersect the block $V_{3q}$, they are disjoint.
    This is because $\scycle(1, 3q-2) = \scycle(1, 3q-3) = Z$, where the first equality follows from the condition of the lemma and the second from the definition of $q$, and $\scycle(1, 3q) \ne Z$.
    Therefore, $\scycle(1,3q-2)$ and $\scycle(1,3q)$ are distinct.
    These triangles cover $x_{3q'}$ and $x_{3q'+1}$ for all $q' \in [q, q_{1}]$.
    \item 
    If $x_{\at(r',3q'+2)}$ for $q' \in [q_1 - 1]$ is not covered yet, then add a triangle $\tria(\scycle(1, 3q'+2), 3q')$.
  \end{itemize}
  The part of $X$ corresponding to $P_{r'}$ for $r' \in [2, r]$ can be covered with the following triangles:
  \begin{itemize}
    \item For each $q' \in [q_{r'}]$, one or two triangles $\tria(\scycle(r', 3q' - 2), \at(r', 3q' - 2)) = \tria(\scycle(r', 3q' - 1), \at(r', 3q' - 2))$ and $\tria(\scycle(r', 3q'), \at(r', 3q' - 2))$.
    \item A triangle $\tria(\scycle(r', 3q_{r'} + 2), \at(r', 3q_{r'}))$ that includes a vertex from a flowery block.
  \end{itemize}

  \paragraph{Case 3.}
  Suppose that $\scycle(r', 1) \ne \scycle(r', 3q_{r'} + 2)$ for $2 \le r' \le r$.
  % Let $Z' = \scycle(r', 1)$ and $Z'' = \scycle(r', 1)$.
  We consider the smallest integer $q$ such that $\scycle(r', 3q + 1) \ne \scycle(r', 1)$.
  Note that such an integer always exists, as $\scycle(r', 3q_{r'} + 2) = \scycle(r', 3q_{r'} + 1)$ by the condition of the lemma.
  The part of $X$ corresponding to $P_{r'}$ can be covered by the following triangles:
  \begin{itemize}
    \item For each $q' \in [q]$, include a triangle $\tria(\scycle(r', 1), \at(r', 3q' - 2))$, which covers $x_{\at(r', 3q' - 2)}$ and $x_{\at(r', 3q' - 1)}$.
    \item For each $q' \in [q, q_{r'}]$, include a triangle $\tria(\scycle(r', 3q' + 1), \at(r', 3q')) = \tria(\scycle(r',3q' + 2, 3q'))$, which covers $x_{\at(r', 3q' + 1)}$ and $x_{\at(r', 3q' + 2)}$.
    Though $\tria(\scycle(r', 1), \scycle(r', 3q - 2))$ and $\tria(\scycle(r', 3q + 1), \at(r, 3q))$ both intersect the block $V_{3q}$, they are disjoint by the definition of $q$.
    \item If $x_{\at(r', 3q')}$ for $q' \in [q_{r'}]$ is not covered, then add a triangle at $\tria(\scycle(r', 3q'), \at(r', 3q'-2))$.
  \end{itemize}
  The part of $X$ corresponding to $P_1$ can be covered with two adjacent flowery vertices:
  \begin{itemize}
    \item $\tria(\scycle(1, 1), 3p+2)$ and $\tria(\scycle(1, 3q_1), 3q_1)$ that include flowery vertices.
    \item For each $q' \in [q_1]$, $\tria(\scycle(3q', 1), 3q')$ and $\tria(\scycle(3q'+2, 1), 3q')$.
  \end{itemize}
  For other paths $P_{r''}$, we include the following triangles together with one flowery vertex:
  \begin{itemize}
    \item If $r'' \in [2, r' - 1]$, then $\tria(\scycle(r'', 3q'-2), \at(r'', 3q'-2))$ and $\tria(\scycle(r'', 3q'), \at(r'', 3q'-2))$ for each $q' \in [q_{r''}]$, as well as $\tria(\scycle(r'', 3q_{r'}+1), \at(r'', 3q_{r'}+1))$.
    \item If $r'' \in [r' + 1, r]$, then $\tria(\scycle(r'', 3q'), \at(r'', 3q'+1))$ and $\tria(\scycle(r'', 3q'+1), \at(r'', 3q'))$ for each $q' \in [q_{r''}]$, as well as $\tria(\scycle(r'', 1), \at(r'', 1) - 1)$.
  \end{itemize}

  \paragraph{Case 4.}
  Suppose that $\scycle(r', 1) = \scycle(r', 3q_{r'} + 2)$ for each $r' \in [2, r]$.
  Define $r' \in [r]$ as the smallest integer such that $\scycle(r', 1) \ne Z$, and let $\at(r', 1) = 3q$.
  For each $q' \in [q]$, we include a triangle $\tria(Z, 3q'-2)$.
  When the triangle intersects a flowery block, i.e., $V_{3q'-1}$ is a flowery block between $P_{r''-1}$ and $P_{r''}$ by Lemma~\ref{lemma:2mod3:flowery-index}, and the triangle covers $x_{3q'-2}$ and $x_{3q'}$. 
  Thus, these triangles cover the elements of $X$ that corresponds to the first and last vertices of each path $P_{r''}$ for $r'' \in [2, r']$.
  The inner vertices can be covered with the following triangles:
  \begin{itemize}
    \item 
    For $r'' = 1$, triangles $\tria(\scycle(1, 3q'), 3q'-2)$, $\tria(\scycle(1, 3q'+1), 3q'-2)$, and $\tria(\scycle(1, 3q'+2), 3q'-2)$ for $q' \in [q_{1}-1]$.
    \item
    For $r'' \in [2, r']$, triangles $\tria(\scycle(r'', 3q'-1), 3q'-1)$, $\tria(\scycle(r'', 3q'), 3q'-1)$, and $\tria(\scycle(r'', 3q'+1), 3q'-1)$ for $q' \in [q_{r''}]$.
  \end{itemize}
  For the path $P_{r'}$, the triangles $\tria(\scycle(r', 3q'-2), \at(r', 3q'-2)) = \tria(\scycle(r', 3q'-1), \at(r', 3q'-2))$ and $\tria(\scycle(r', 3q'), \at(r', 3q'-2))$ for each $q' \in [q_{r'}]$ cover the corresponding elements in $X$.
  While the block $V_{\at(r', 1)}$ is intersected by $\tria(\scycle(r', 1), \at(r', 3q'-2))$ as well as $\tria(\scycle(r'-1, 3q_{r'-1} + 2), 3q_{r'-1}+2)$, they are disjoint as $\scycle(r', 1) \ne Z$.
  The remaining part of $X$ can be covered using an argument in the second part of case 2.
\end{proof}

\subsubsection{Algorithm overview} \label{ssec:tcam:overview}

In the introductory part of Section~\ref{ssec:tcamfpt}, we showed that the objective is to find
a triangle-maximal strong cycle packing $\cC = \{ C^1, \cdots, C^{\kappa} \}$, where $C^i = (v^i_{1}, \dots, v^i_{l_i})$, covering a set $F$ of size~$2k$ that is feasible in the dual matching delta-matroid $D$.
Via a color-coding argument, we guess a partitioning $V= V^1 \cup \cdots \cup V^{\kappa}$ where each subset $V^i$ is further divided into $V_{1}^i \cup \cdots V_{l_i}^i$.
Then $v_j^i \in V_{j}^i$ holds for all $i \in [\kappa]$ and $j \in [l_i]$ with probability $1/k^{O(k)}$.
Our algorithm gives a reduction to the \textsc{Colorful Delta-matroid Matching} problem as follows.

\begin{itemize}
  \item For $i \in [\kappa]$ with $l_i \bmod 3 = 0$, we may assume that $l_i = 3$ by Lemma~\ref{lemma:tri-maximal}, so this case aligns with \textsc{Delta-matroid Triangle Cover}.
  \item Consider $i \in [\kappa]$ with $l_i \bmod 3 = 1$.
  By Lemma~\ref{lemma:tri-maximal} and Lemma~\ref{lemma:1mod3:non-flowery}, we may assume that $v_j^i$ is part of $F$, and that $v_j^i$ is non-flowery.
  After deleting all flowery vertices from $V^i$, we find a set $S \subseteq V^i$ according to Lemma~\ref{lemma:1mod3:ultimate}.
  We adjust the delta-matroid $D$ via $1$-contraction by $S \cap V_j^i$ for each $j \in [l_i]$.
  \item
  Consider $i \in [\kappa]$ with $l_i \bmod 3 = 2$.
  First, we guess whether each $v^i_j$ is flowery or not and part of $F$ or not.
  Flowery (non-flowery) vertices are then from a flowery (non-flowery) block $V_j$.
  In the absence of flowery vertices in $C^i$, we find a set $S \subseteq V^i$ following Lemma~\ref{lemma:2mod3:no-flowery}.
  For each $j \in [l_i]$, apply 1-contraction by $S \cap V_j^i$ if $v^i_j$ is part of $F$, and delete $S \cap V_j^i$ from $D$ otherwise.
  If $C^i$ contains a flowery vertex, we may assume the configuration aligned with Lemma~\ref{lemma:2mod3:path-lengths}, where $|V(P_1)| \mod 3 = 1$ and $|V(P_{r'})| \mod 3 = 2$.
  As per Lemma~\ref{lemma:2mod3:find-cycles}, let $\cS$ be a collection of strong cycles $\cS$ which are disjoint at non-flowery blocks.
  We start off by applying 1-contractions on flowery blocks $V^i_j$, where $v_j$ is part of $F$.
  We then encode pairing constraints as given in Lemma~\ref{lemma:2mod3:ultimate} by coloring the pairs.
  Specifically, for $P_r$, which has $3q_{r'} + 2$ (or $3q_1 +1$ for $r' = 1$) vertices, we introduce $q_{r'} + 1$ colors $c^i_{r',1}, \dots, c^i_{r',q_{r'}+1}$.
  In the context of $P_1$, for each strong cycle in $S \in \cS$, and we establish a colored pairing between the elements of $V_1 \cap V(S)$ and $V_2 \cap V(S)$ using color $c^i_{1,1}$.
  Similarly, for each $q \in [q_1]$, a pair is created between the elements of $V_{3q} \cap V(S)$ and $V_{3q+1} \cap V(S)$, marked in the $c^i_{1,q+1}$.
  For each path $P_{r'}$ with $r' \in [2, r]$, and for each $q \in [q_r+1]$, we have a pairing between the elements in $V_{3q-2} \cap V(S)$ and $V_{3q-1} \cap V(S)$ with the color $c^i_{r',q}$ for each strong cycle in $S \in \cS$.
  Finally, for each block $V_j$ within the path $P_{r'}$ for $r' \in [r]$ that is not involved in any pairing, we apply 1-contraction by $V(\cS) \cap V_j$. 
\end{itemize}
This concludes the description of our algorithm.
Now we proceed to validate the correctness.

Assume that there is  a strong packing covering feasible set $F$ of size $2k$ covered.
We verify that the algorithm affirm the existence of such a packing with probability $1/2^{O(k)}$.
For a strong cycle $C^i$ of length $1 \bmod 3$, the set $S$ of Lemma~\ref{lemma:1mod3:ultimate} covers $C^i$ with probability at least $1/2^{O(l_i)}$.
Similarly,  in cases where $C^i$ has length $2 \bmod 3$ and lacks flowery vertices, the outcome aligns with the previous scenario by Lemma~\ref{lemma:2mod3:no-flowery}.
For a strong cycle $C^i$ of length $2 \bmod 3$, the collection of strong cycles $\cS$ contains $C^i$ with probability $1/2^{O(l_i)}$ by Lemma~\ref{lemma:2mod3:find-cycles}.
If $C^i$ is indeed part of $\cS$, then all the requisite colored pairings are introduced.
Aggregating these probabilities across all cycles, our algorithm terminates in the affirmative with probability $1/2^{O(k)}$.

Conversely, assume that the algorithm concludes in the affirmative, i.e., there is a feasible set $F$ that is in compliance with the colored pairing constraints, as well as 1-contraction conditions.
First, consider a strong cycle $C^i$ of length $1 \bmod 3$, where 1-contraction by the intersection $S \cap V_j$ is applied for each block $V_j$.
This ensures that $F \cap V^i$ comprises exactly one vertex from $S \cap V^i_j$.
By Lemma~\ref{lemma:1mod3:ultimate} (ii), the intersection $S \cap V^i_j$ indeed can be covered by a strong cycle packing within $V^i$.
The situation is similar for a strong cycle $C^i$ of length $2 \bmod 3$ without flowery vertices, where the coverage  of $F \cap V^i$ with a strong cycle packing is guaranteed Lemma~\ref{lemma:2mod3:no-flowery} (ii).
Finally, consider a strong cycle $C^i$ of length $2 \bmod 3$ with flowery vertices.
As our algorithm introduces pairing constraints in alignment with the condition of Lemma~\ref{lemma:2mod3:ultimate}, there exists a strong cycle packing that covers $F \cap V^i$.

Consequently, we have established the algorithm's correctness in the presence of a perfect matching.
Finally, we address the case where $G$ lacks a perfect matching, repeating the approach used in the proof of Lemma~\ref{lemma:lifting}.
Assume that $E'$ is a strong edge set of size $\ell$.
Initially, fix a maximum matching $M$ of $G$, and define $U = V(G) \setminus V(M)$ as the set of unmatched vertices.
For each $u \in U$, we introduce a degree-one neighbor $u'$ adjacent to $u$, and let $U' = \{ u' \mid u \in U \}$.
Call the resulting graph $G'$, which has a perfect matching $M' = M \cup \{ uu' \mid u \in U \}$.
Within $G'$, there exists a strong set $E''$ of size $\ell + \alpha$, where $\alpha \le |U|$ represents the number of vertices in $U$ that is not incident with any edge in $E'$.
Our objective then shifts to finding a strong edge set with at least $\ell' = \ell + \alpha$ edges in $G'$ where $\beta = |U| - \alpha$ vertices of $U'$ are isolated.
Let $D'$ be the dual matching delta-matroid of $G'$.
By Lemma~\ref{lemma:tc-feasible-cover}, the existence of a strong set of size $\ell'$ is equivalent to the existence of a strong cycle packing covering a feasible set $F$ of size  $2(\ell' - |M'|) = 2(k - \beta)$.
For the additional requirement that $\beta$ vertices of $U'$ becomes isolated, note that the isolation of $u' \in U'$ is achieved when it is part of $F$:
By Lemma~\ref{lemma:alt-paths}, a set $F$ is feasible in $D'$ if and only if there are $|F|/2$ $M'$-alternating $F$-paths.
It follows from the proof of Lemma~\ref{lemma:tc-feasible-cover} that the edge $uu'$ is excluded from a strong set when it is part of $|F|/2$ $M$-alternating $F$-paths, i.e., when $u' \in F$. 
Consequently, we apply a $\beta$-contraction by $U'$ on $D'$, calling the resulting delta-matroid $D$.
We then search for a strong cycle packing covering a set of size $2k - 3\beta$ feasible in $D$.
This concludes the proof of Theorem~\ref{theorem:tc-am-fpt}.

\subsection{W[1]-hardness proof}\label{ssec:STC-w1-hard}
While lifting the FPT algorithm for \textsc{Cluster Subgraph} from $K_4$-free graphs to arbitrary graphs was relatively straightforward, this is not the case for \textsc{Strong Triadic Closure}. Indeed, we will show that the interaction in a strong set between $K_4$-free deletion set and the rest of the graph is much more intricate and it is not possible to be handled just by additional color-coding. Even more surprisingly, we show that \textsc{Strong Triadic Closure} is W[1]-hard parameterized by above matching.
\begin{theorem}
  \textsc{Strong Triadic Closure} is W[1]-hard parameterized by above matching.
\end{theorem}

\begin{proof}
We reduce from \textsc{Grid Tiling}, which is formally defined as follows. Given integers $k,n\in \mathbb{N}$ and a collection $\mathcal{S}$ of $k^2$ nonempty sets $S_{i,j}\subseteq [n]\times [n]$ $ (i,j\in [k])$. The task is to find for each $i,j\in [k]$ a pair $s_{i,j}\in S_{i,j}$ such that 
\begin{itemize}
\item[(C1)] If $s_{i,j} = (a,b)$ and $s_{i+1,j} = (a',b')$, then $a=a'$. (For all $i\in [k-1], j\in [k]$.)
\item[(C2)] If $s_{i,j} = (a,b)$ and $s_{i,j+1} = (a',b')$, then $b=b'$. (For all $i\in [k], j\in [k-1]$.)
\end{itemize} 
\textsc{Grid Tiling} is well known to be W[1]-hard parameterized by $k$~\cite{CyganFKLMPPS15PCbook}. We will construct a graph $H$, where $V(H)$ is partitioned into five parts $P \cup Q \cup M \cup R\cup C$, where $|P|=\sum_{S\in \mathcal{S}}|S|$, $|Q|= 2k^2$, $|M| = |P|+|Q|$ and $|R|= k^2-k$ and $|C|= k^2-k$.  

Let us start with very informal intuition behind the proof. 
The set of vertices $P$ can be thought of as partitioned into $k^2$ sets $P_{i,j}$ $(1\le i,j\le k)$ such that for every $i,j\in [k]$ and every $s\in S_{i,j}$, $P_{i,j}$ contains a vertex $p_s^{i,j}$ representing selection of $s$ from $S_{i,j}$. In addition $H$ contains edges between vertices in $P$ corresponding to Conditions (C1) and (C2) from the definition of Grid Tiling. 
To select $p_s^{i,j}$, $Q$ contains an edge $q^{i,j}_1q^{i,j}_2$ that can form a triangle with a single $p_s^{i,j}$ in a strong set. The set $M$ then contains for every $v\in P\cup Q$ a unique pendant $m(v)$ of $v$ in $H$. $H' = H[P\cup Q \cup M]$ will be $K_4$-free and half of the vertices (that are in $M$) there have degree one. So any maximum strong set in $H'$ has exactly the size of the perfect matching in $H'$, which is $|P|+|Q|$, regardless whether we used $p_s^{i,j}q^{i,j}_1q^{i,j}_2$ triangles or just match vertices in $P\cup Q$ to their matching partners in $M$. However, only one vertex in each $P_{i,j}$ can form a triangle with $q^{i,j}_1q^{i,j}_2$ in the strong set. All the remaining vertices $p\in P$ have to be matched with their matching partner $m(p)$.
%That is for every vertex $v\in P\cup Q$, there is a unique vertex $v'\in P'\cup Q'$ such that $vv'$ is an edge in $H$ and $v'$ has degree one. This way, we can easily say that $P \cup P' \cup Q \cup Q'$ has a perfect matching. In addition $H[P \cup P' \cup Q \cup Q']$ will be $K_4$-free and hence any maximum strong set in $H[P \cup P' \cup Q \cup Q']$ has precisely same size as this perfect matching. 

Finally $R\cup C$ will be a clique on $2k^2-2k$ vertices, so the excess above the perfect matching in a maximum size strong set will be obtained from this large clique and edges between $R\cup C$ and $P\cup Q$. The vertices in $R$ will serve to check the condition (C1). That is a vertex $r\in R$ will be only adjacent to all vertices in $P$, but only to two edges from $Q$ that represent $S_{i,j}$ and $S_{i+1,j}$. This will force that in a strong set, $r$ can have only two edges to $P\cup Q$, one to $p_1\in P_{i,j}$ and one to $p_2\in P_{i+1,j}$ and $p_1,p_2$ have to also be adjacent. Analogously, each $c\in C$ will serve to check the condition (C2) for some $i,j\in [k]$.

\paragraph*{Construction.}
%We first note that both $P'$ and $Q'$ are just unique pendants of a vertices in $P$ and $Q$ respectively. That is for every vertex $v\in P\cup Q$, there is a vertex $v'\in P'\cup Q'$ such that $vv'$ is an edge in $H$ and $v$ is the unique neighbor of $v'$. This way, we can easily say that $P \cup P' \cup Q \cup Q'$ has a perfect matching.
Let us now proceed with more precise formal construction. 

We start with set $P$.
For every $1\le i,j\le k$, we let $P_{i,j} = \{ p_s^{i,j} \mid s \in S_{i,j} \}$ and we let 
$P = \bigcup_{1\le i,j \le k }P_{i,j}$
%$P = \{ p_s^{i,j} \mid 1\le i,j\le k, s \in S_{i,j} \}$, 
that is for every $i,j\in [k]$ and every $s\in S_{i,j}$ we introduce vertex $p_{s}^{i,j}$. In addition $H$ contains an edge between two vertices $p_1,p_2\in P$ if and only if $p_1$ and $p_2$ represent a selection matching one of the conditions (C1) or (C2). Or in other words if and only if
\begin{itemize}
\item either $\{p_1, p_2\} = \{p_s^{i,j}, p_{s'}^{i+1,j}\}$ with $i\in [k-1]$, $j\in [k]$, $s = (a,b)$, $s' = (a',b')$, and $a=a'$; or
\item $\{p_1, p_2\} = \{p_s^{i,j}, p_{s'}^{i,j+1}\}$ with $i\in [k]$, $j\in [k-1]$, $s = (a,b)$, $s' = (a',b')$, and $b=b'$.
\end{itemize} 
%For convenience, we will also denote by 

Let $Q$ be defined as $Q = \{ q_{1}^{i,j}, q_2^{i,j} \mid i, j \in [k] \}$.
For each $i, j \in [k]$, we introduce the edge $q_1^{i,j}, q_2^{i,j}$.
Furthermore, for each $i, j \in [k]$, and every $p\in P_{i,j}$ we add edges $pq_1^{i,j}$ and $pq_2^{i,j}$, so that $\{ q_{i,j,1}, q_{i,j,2}, p \}$ forms a triangle.

For every vertex $v\in P\cup Q$, $M$ contains the vertex $m(v)$ and $H$ contains the edge between $v$ and $m(v)$.

Let $R = \{ r^{i,j} \mid i \in [k-1], j\in [k] \}$ and $C = \{ c^{i,j} \mid i \in [k], j\in [k-1] \}$. 
We add edges so that $R\cup C$ forms a clique.
We also add an edge between each $p \in P$ and each $v \in R\cup C$. 
For each $i, j \in [k]$, if $i< k$ we add edges from $r^{i,j}$ to $q_1^{i,j}$, $q_2^{i,j}$, $q_1^{i+1,j}$, $q_2^{i+1,j}$, and if $j<k$ we also add edges from $c^{i,j}$ to $q_1^{i,j}$, $q_2^{i,j}$, $q_1^{i,j+1}$, $q_2^{i,j+1}$.

Finally, we let $\ell = |P|+ |Q| + 2|R\cup C| + \binom{|R\cup C|}{2}$.

\paragraph{Correctness.} First note that $H$ has a perfect matching (matching $v\in P\cup Q$ to $m(v)$ and matching $r^{i,j}\in R$ with $c^{j,i}\in C$), which has size $\mu = |P| + |Q| + \frac{|R\cup C|}{2}$ and $\ell - \mu = 3(k^2-k) + \binom{2k^2-2k}{2}$, so the above matching parameter is indeed bounded by a function of $k$. Hence it only remains to show that the two instances are equivalent. 

$(\Rightarrow)$ Let $S = \{s_{i,j}\mid i,j\in [k], s_{i,j}\in S_{i,j} \}$ be a set of pairs satisfying (C1) and (C2). We construct a strong edge set $S'$ of size $\ell$ as follows: 
\begin{itemize}
\item For every $i,j\in[k]$, include triangle $\{p_{s_{i,j}}^{i,j}, q_1^{i,j}, q_2^{i,j}\}$.
\item For each $p\in P$ not involved in these triangles, include the edge $pm(p)$.
\item Include all $\binom{2k^2-2k}{2}$ edges in the clique on $R\cup C$.
\item For each $i\in [k-1]$, $j\in [k]$, add edges $r^{i,j}p_{s_{i,j}}^{i,j}$ and $r^{i,j}p_{s_{i+1,j}}^{i+1,j}$.
\item For each $i\in [k]$, $j\in [k-1]$, add edges $c^{i,j}p_{s_{i,j}}^{i,j}$ and $c^{i,j}p_{s_{i,j+1}}^{i,j+1}$.
\end{itemize}
It is easy to verify that precisely $\ell = |P|+ |Q| + 2|R\cup C| + \binom{|R\cup C|}{2}$ many edges has been selected to $S'$. We only need to check that $S'$ forms a strong set or in other words that it satisfies the triadic closure property. First observe that the edges $pm(p)$ included in $S'$ are isolated, that is, neither $p$ nor $m(p)$ is incident with any other edges of $S'$. Consequently we identify the following types of $P_3$'s in $S$: 
\begin{itemize}
\item 
$(x,p,y)$ for $x,y\in R\cup C$ and $p\in P$. These are closed as $R\cup C$ induces a clique.
\item $(r^{i,j}, p_{s_{i,j}}^{i,j}, q^{i,j}_{\alpha})$ and $(r^{i,j}, p_{s_{i+1,j}}^{i+1,j}, q^{i+1,j}_{\alpha})$ for $\alpha\in \{1,2\}$ are closed because our construction adds edges from $r^{i,j}$ to  $q_{\alpha}^{i,j}$, $q_{\alpha}^{i+1,j}$.
\item 
$(c^{i,j}, p_{s_{i,j}}^{i,j}, q^{i,j}_{\alpha})$ and $(c^{i,j}, p_{s_{i,j+1}}^{i,j+1}, q^{i,j+1}_{\alpha})$ for $\alpha\in \{1,2\}$ are closed because our construction adds edges from $c^{i,j}$ to  $q_{\alpha}^{i,j}$, $q_{\alpha}^{i,j+1}$.
\item  
$(p_{s_{i,j}}^{i,j},r^{i,j}, p_{s_{i+1,j}}^{i+1,j})$ are closed because since $S$ is a solution for \textsc{Grid Tiling}, the pair $s_{i,j}$ and $s_{i+1,j}$ satisfies condition (C1), and so our construction adds the edge $p_{s_{i,j}}^{i,j}p_{s_{i+1,j}}^{i+1,j}$.
\item 
$(p_{s_{i,j}}^{i,j},c^{i,j}, p_{s_{i,j+1}}^{i,j+1})$ are closed because the pair $s_{i,j}$ and $s_{i,j+1}$ satisfies condition (C2), and so our construction adds the edge $p_{s_{i,j}}^{i,j}p_{s_{i,j+1}}^{i,j+1}$.
\item
  $(p, r, r')$ for $p \in P$ and $r, r' \in R$ is closed as there is an edge between each $p \in P$ and $r \in R$.
\end{itemize}
Therefore $S$ is indeed a strong set of size $\ell$.  

$(\Leftarrow)$
Let $S$ be a strong edge set of size $\ell$. We show that for every $i,j\in [k]$, we can select $s_{i,j}\in S_{i,j}$ such that the selection satisfies conditions (C1) and (C2).
 
Let us first observe that solution can contain at most $\binom{|R\cup C|}{2}$ edges from $R\cup C$ and $|P|+|Q|$ many edges from $P\cup Q\cup M$. The former is trivial, as that is the number of edges in $R\cup C$, the latter follows from the fact that $H[P\cup Q\cup M]$ is $K_4$-free and all vertices in $M$ have degree one in $H$. So every vertex $v$ in $P\cup Q$ has at most two of its incident edges from $H[P\cup Q\cup M]$ in $S$, and it can only be two if neither of them is $vm(v)$, so if $x$ many vertices in $v\in P\cup Q$ are matched with their matching partner $m(v)$ in $S$, then the maximum number of edges from $H[P\cup Q\cup M]$ in $S$ is $\frac{2(|P|+|Q|-x)}{2} + x = |P|+|Q|$. 

We will now show that for every vertex $v$ in $R\cup C$, there are at most two edges incident on $v$ and a vertex in $P$ and there is no edge in $S$ incident on $v$ and a vertex in $Q$.  
Note that since $\ell = |P|+ |Q| + \binom{|R\cup C|}{2} + 2|R\cup C| $, this immediately means that $S$ contains precisely  $|P|+ |Q|$ many edges from $P\cup Q\cup M$, $\binom{|R\cup C|}{2}$ edges from $R\cup C$, and $2|R\cup C|$ edges between  $R\cup C$ and $P$. 
First observe that $H[P]$ is bipartite and hence triangle-free. Indeed, for $p\in P_{i,j}$ and $p'\in P_{i',j'}$, we have that $pp'$ is an edge in $H$, then either $i'=i+1$ and $j'=j$ or $i'=i$ and $j'=j+1$. In addition, recall that neighborhood of every vertex in the subgraph of $H$ induced by $S$ has to be a clique in $H$. It follows that every vertex $v\in R\cup C$ has at most two neighbors $p_1, p_2\in P$ such that $vp_1, vp_2\in S$. Since $H[P\cup Q]$ is $K_4$-free, there are at most three edges in $S$ that are incident on $v$ and a vertex in $P\cup Q$.
Let $v\in R\cup C$ and $q\in Q$. Note that $q$ is adjacent to exactly two vertices in $R$ and exactly two vertices in $v$. Therefore, if $qv\in S$, then $S$ contains at most $3$ out of $|R\cup C|-1$ edges with one endpoint being $v$ and the other endpoint in $R\cup C$, or equivalently at least $|R\cup C|-4$ edges incident on $v$ in $R\cup C$ are missing from $S$. Now let $X$ be the set of vertices $v\in R\cup C$ for which there exists $q\in Q$ such that $vq\in S$. We can upper bound the number of edges in $S$ incident on a vertex in $R\cup C$ as follows: 
\begin{itemize}
\item $3|X|$ edges between $X$ and $P\cup Q$.
\item $2(|R\cup C| - |X|)$ edges between $(R\cup C)\setminus X$ and $P$.
\item $\binom{|R\cup C|}{2} - \frac{|X|\cdot (|R\cup C| - 4)}{2}$. 
\end{itemize}
Which sums to $2|R\cup C| + |X| + \binom{|R\cup C|}{2} - \frac{|X|\cdot (|R\cup C| - 4)}{2}$. Since we already upper bounded the number of edges in $S$ that are not incident on a vertex in $R\cup C$ by $|P|+|Q|$, we get from $|S|\ge \ell$ that $|X| - \frac{|X|\cdot (|R\cup C| - 4)}{2} = \frac{|X|\cdot (6-|R\cup C|)}{2}  \ge 0$ and since we can assume that $|R\cup C| = 2k^2- 2k > 6$, we get that $|X| = 0$. Therefore, for every vertex $v$ in $R\cup C$, there are at most two edges incident on $v$ and a vertex in $P$ and there is no edge in $S$ incident on $v$ and a vertex in $Q$.

From above discussion it follows that $S$ contains:
\begin{itemize}
\item $|P|+|Q|$ edges from $H[P\cup Q\cup M]$.
\item All the edges from the clique on $R\cup C$. 
\item For every $v\in R\cup C$ exactly two edges with one endpoint $v$ and the other endpoint in $P$.
\end{itemize} 
First observe that if for $p\in P$, the edge $pm(v)$ is not in $S$, then $S$ contains two edges with $p$ as the endpoint. Moreover for every $p\in P$, there is only one edges in the neighborhood of $p$ restricted to $P\cup Q\cup M$. In particular if $p\in P_{i,j}$, this edge is $q^{i,j}_1q^{i,j}_2$. Since for $p, p'\in P_{i,j}$ there is no edge between $p$ and $p'$, It follows that for every $i,j\in [k]$, there is at most one vertex $p^{i,j}_{s_{i,j}}\in P_{i,j}$ such that $S$ contains edges $p^{i,j}_{s_{i,j}}q_1^{i,j}, p^{i,j}_{s_{i,j}}q_2^{i,j}$, and $q_1^{i,j}q_2^{i,j}$ and for all $p \in  P_{i,j}\setminus \{p^{i,j}_{s_{i,j}}\}$ the strong set $S$ contains $pm(p)$. 

We now show that there is not only at most one, but precisely one such vertex $p^{i,j}_{s_{i,j}}$ and selecting the pair $s_{i,j}$ from $S_{i,j}$ for each $i,j\in [k]$ gives us a solution for the original \textsc{Grid Tiling} instance, which concludes the proof. 

Let us consider $r^{i,j}\in R$ for $i\in [k-1]$ and $j\in [k]$. As we already argued there are two vertices $p_1,p_2\in P$ such that $r^{i,j}p_1, r^{i,j}p_2\in S$. First notice that $m(p_1)$ nor $m(p_2)$ are adjacent to $r^{i,j}$. It follows $p_1m(p_1)$ and $p_2m(p_2)$ are not in $S$ and so for $\alpha\in \{1,2\}$, $p_\alpha$ is the unique vertex $p^{i',j'}_{s_{i',j'}}\in P_{i',j'}$, for some $i',j'\in [k]$, such that $S$ contains edges $p^{i',j'}_{s_{i',j'}}q_1^{i',j'}, p^{i',j'}_{s_{i',j'}}q_2^{i',j'}$(, and $q_1^{i',j'}q_2^{i',j'}$). However $r^{i,j}$ is adjacent only to $q_1^{i,j}, q_2^{i,j}, q_1^{i+1,j}, q_2^{i+1,j}$. Hence $\{p_1, p_2\} = \{p^{i,j}_{s_{i,j}}, p^{i+1,j}_{s_{i+1,j}}\}$. Moreover, $r^{i,j}p_1, r^{i,j}p_2\in S$ implies that $p_1p_2\in E[H]$ and so $s_{i,j}=(a,b)$ and $s_{i+1,j}=(a',b')$ satisfy the condition (C1) that $a=a'$ for all $i\in [k-1]$ and $j\in [k]$.

Analogously, if we consider $c^{i,j}\in C$ for all $i\in [k]$ and $j\in [k-1]$, we can conclude that $S$ contains the edges $c^{i,j}p^{i,j}_{s_{i,j}}$ and $c^{i,j}p^{i,j+1}_{s_{i,j+1}}$, so $p^{i,j}_{s_{i,j}}p^{i,j+1}_{s_{i,j+1}}\in E[H]$ and $s_{i,j}=(a,b)$ and $s_{i,j+1}=(a',b')$ satisfy the condition (C2) that $b=b'$ for all $i\in [k]$ and $j\in [k-1]$.
\end{proof}

\subsection{FPT algorithm for bounded-degree graphs}\label{ssec:STC_bounded_degree}

We show that \textsc{Strong Triadic Closure} can be solved in single-exponential time in $k = \ell - \MM(G)$ for graphs of bounded degree.
This generalizes the result of
Golovach et al.~\cite{GolovachHKLP20}, who proposed an $O^*(2^{O(k)})$-time algorithm  for \textsc{Strong Triadic Closure}, specifically for the graph of maximum degree at most four.
Compared to Golovach et al., our algorithm is simpler by adopting a delta-matroid perspective.

% \begin{theorem}
%   \textsc{Strong Triadic Closure} can be solved in $O^*(\Delta^{O(k)})$ time, where $\Delta$ is the maximum degree of $G$ and $k = \ell - \MM(G)$.
% \end{theorem}
\tcamdelta*
\begin{proof}
  Starting with an approach in the proof of Lemma~\ref{lemma:lifting},
  our algorithm uses a greedy method to find a packing $\mathcal{K}$ of $K_4$'s.
  If $k/2$ disjoint $K_4$'s have been identified, then there is a strong set of size $\ell$: 
  Letting $K = V(\mathcal{K})$ be the set of vertices involved in $\mathcal{K}$,
  there is a matching of size $\MM(G) - |K| = \MM(G) - 4|\mathcal{K}|$ disjoint from $K$.
  Together with $K_4$'s in $\mathcal{K}$, we obtain a strong set of size $(\MM(G) - 4|\mathcal{K}|) + 6|\mathcal{K}| \ge \ell$, provided that $|\mathcal{K}| \ge k/2$. 
  So we will assume that $|\mathcal{K}| < k/2$.

  Further, we infer that a strong set $E'$ of size $\ell$ contains at most $3k$ edges within $V(\mathcal{K})$.
  This allows us to enumerate all possible edges inside $V(\mathcal{K})$ that are part of $E'$ in $\binom{\Delta |K|}{3k} = \Delta^{O(k)}$ time.
  Additionally, we can guess all edges with exactly one endpoint in $V(\mathcal{K})$ in $\Delta^{O(k)}$ time:
  For each vertex $v$, the endpoints of the edges incident with $v$ must form a clique to maintain the triadic closure property, and $G - K$ is $K_4$-free, it follows that each vertex in $K$ is incident with at most three edges in $E'$.
  So we can enumerate all these edges in $\Delta^{O(k)}$ time.
  Note that there are at most $3|K|$ vertices outside $K$ that is incident with an edge in $E'$ where exactly one endpoint is in $K$.
  For each of these vertices outside $K$, we guess at most two incident edges in $E'$.
  With all these edges guessed, we can verify whether all edges that intersect $K$ satisfy the triadic closure property.
  Specifically, there are four types of $P_3$'s $(u, v, w)$ involving a vertex in $K$:
  (i) $u, v, w \in K$, (ii) $u, v \in K$ and $w \notin K$, (iii) $u, w \in K$ and $v \notin K$, and (iv) $u \in K$ and $v, w \notin K$.
  All these relevant $P_3$'s can be enumerated, which allows us to verify the triadic closure property.

  Our next step is to identify the edges of $E'$ that lie outside $K$.
  As we have verified the triadic closure property for all $P_3$'s intersecting $K$, we need to address the \textsc{Strong Triadic Closure} problem on $G - K$ (with some edges prescribed to be part of the desired strong set).
  Suppose that we need to find a strong set of size $\ell'$, having guessed $\ell - \ell'$ edges incident with $K$ to be included in a strong set.
  By Lemma~\ref{lemma:ce-feasible-cover}, if $G - K$ has a perfect matching, the problem is to find a strong cycle packing covering a feasible set of size $2k'$, where $k' = \ell' - \MM(G - K) \le \ell - (\MM(G) - |K|) \le 3k$.
  This criterion holds regardless of the absence of perfect matching as well from the discussions in Section~\ref{sssec:2mod3}.

  To address this problem, we will adopt a probabilistic approach.
  Specifically, for each vertex in $V(G) \setminus K$, we randomly select two incident edges in $G - K$.
  For each vertex $v$ that is connected to $K$ via an edge included with $E'$, this selection includes the edges in $G - K$ predetermined to be part of $E'$.
  We keep those edges $uv$ that are chosen by both $u$ and $v$.
  The remaining edges induce a graph where every connected component is a path or a cycle.
  Edges not contributing to strong cycles are subsequently pruned.
  Our algorithm then tests the existence of a feasible set of size $2k$ within the vertices that are involved in remaining strong cycles.
  Assuming that there is a strong cycle packing that covers a feasible set of size $2k'$, the algorithms confirms this with probability $1/\Delta^{O(k)}$.
  This is because there is a strong cycle packing over at most $6k'$ vertices by Lemma~\ref{lemma:tri-maximal}.
  For each strong cycle of length $l$, the probability that it is retained is $1/\Delta^{2l}$, resulting in an overall success probability at least $1/\Delta^{O(k)}$.
  Conversely, if a feasible set can be covered by the vertices connected to remaining edges, a strong set of size $\ell'$ can be constructed by Lemma~\ref{lemma:tc-feasible-cover}.
\end{proof}

\subsection{FPT approximation}\label{ssec:STC_FPT_Approx}

In this subsection, we develop an FPT algorithm for \textsc{Strong Triadic Closure} parameterized by $k = \ell - \MM(G)$, with an approximation ratio $7$.
Specifically, we give an $O^*(k^{O(k)})$-time algorithm that either determines that there is no strong set of size $\MM(G) + k$ or finds a strong set of size $\MM(G) + k/7$.
Complementing this finding, we show the infeasibility of achieving a polynomial-time $n^{1-\varepsilon}$-factor approximation for \textsc{Strong Triadic Closure} under the assumption that P $\ne$ NP, even with parameter~$\ell$.

\tcamapprox*

\begin{proof}
  Let $c = 14$ and $t = \MM(G) + 2k/c$.
  Our algorithm will either identify a strong set of size $t$ or establish the absence of a strong set of size $\MM(G) + k$.
  First, we find a packing $\mathcal{K}$ of $K_4$'s through a greedy approach, leading to the following two case:
  \begin{itemize}
    \item 
    If $|\mathcal{K}| \ge k/c$, a strong set of size $t$ exists:
    $G - V(\mathcal{K})$ contains a matching of size $\MM(G) - 4|\mathcal{K}|$ and $K_4$'s in $\mathcal{K}$ contribute $6|\mathcal{K}|$ edges.
    \item
    Conversely, suppose that $|\mathcal{K}| < k/c$.
    We solve \textsc{Strong Triadic Closure} on $G - V(\mathcal{K})$ using the FPT algorithm for $K_4$-free graphs in Theorem~\ref{theorem:tc-am-fpt}.
    Initially, we test whether $G - V(\mathcal{K})$ has a strong set of size $t$.
    If such a set is absent, we proceed to optimally solve \textsc{Strong Triadic Closure} on $G - V(\mathcal{K})$.
    As $V(|\mathcal{K}|) < k/c$, and we can also test whether $G[V(\mathcal{K})]$ contains a strong set of size $6|\mathcal{K}|$ in $k^{O(k)}$ time.
    If this succeeds, we obtain a strong set of size $t$ as in the first case.
    Otherwise, we can find the largest strong set within $G[V(\mathcal{K})]$ in $O(k^{O(k)})$ time by enumerating all sets up to $6|\mathcal{K}|$ edges.
    Now let us consider a combine optimal solution from $G[V(\mathcal{K})]$ and $G - V(\mathcal{K})$.
    If this set has size smaller than $t$, then there is no solution of size $t + 12k/c$:
    For any strong set, every vertex in $V(\mathcal{K})$ has at most three edges that connects it to a vertex in $V(G) \setminus V(\mathcal{K})$ by the strong triadic closure property.
  \end{itemize}
  This gives a 7-factor approximation guarantee, as we either find a strong set of size $t = \MM(G) + k/7$, or conclude that there is no strong set of size $t + 12k/c = \MM(G) + k$.
\end{proof}

We conjecture that a more refined analysis, by adopting a smaller value of $c$, could improve the approximation factor.
We now turn to the hardness result.

\begin{lemma}
  For every $\varepsilon > 0$, approximating the value of $\ell$ for \textsc{Strong Triadic Subgraph} within a factor of $O(n^{1-\varepsilon})$ in polynomial time is impossible, unless P = NP.
\end{lemma}
\begin{proof}
  It is known that, for every $\varepsilon > 0$, no polynomial-time algorithm can approximate \textsc{Clique}
   within a factor better than $O(n^{1-\varepsilon})$, unless P = NP~\cite{Zuckerman07}.
   In particular, for
  every $\varepsilon > 0$, the existence of a polynomial-time algorithm that distinguishes between graphs
  with a clique of size $n^{1-\varepsilon}$ and those without 
  a clique of size $n^{\varepsilon}$, implies P = NP.
  Assume, for contradiction, the existence of a polynomial-time approximation for \textsc{Strong Triadic Closure} within a factor $O(n^{1-\varepsilon'})$ for every $\varepsilon' > 0$.
  
  Consider a $G=(V,E)$ with a clique $C$ with $|C| \geq n^{1-\varepsilon}$. 
  Then the set of edges $S=\binom{C}{2}$ is a strong set of size $\frac{1}{4} n^{2-2\varepsilon}$ for sufficiently large $n$.
  On the other hand, if $G$ has no clique of size $n^{\varepsilon}$, then there is no strong set of size $n^{1+\varepsilon}$.
  This is because for a strong set $S$, each vertex has degree $n^{\varepsilon}$ in $S$, as the neighborhood must be a clique to adhere to the triadic closure property.
  It follows that an approximation of \textsc{Strong Triadic Closure} within a factor of $\frac{1}{4}n^{1-3\varepsilon}$ would effectively differentiate these \textsc{Clique} instances, contradiction the assumption.
\end{proof}
