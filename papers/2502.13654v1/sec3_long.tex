\section{Basic delta-matroid problems}
\label{sec:part1}
In this section, we investigate the parameterized complexity of
natural generalizations of ``basic'' delta-matroid search problems
such as \textsc{Delta-matroid Intersection} and
\textsc{Delta-matroid Parity}.

We recall the definition of the \textsc{Delta-matroid Parity} problem. 
Let $D=(V,\cF)$ be a delta-matroid and $\Pi$ a partition of $V$ into pairs. 
For a set $F \subseteq V$, let $\delta_{\Pi}(F) = |\{ P \in \Pi : |F \cap P| = 1 \}|$
denote the number of pairs broken by $F$, and define
$\delta(D, \Pi) = \min_{F \in \cF} \delta_{\Pi}(F)$.
The goal of \textsc{Delta-matroid Parity} is to find a set $F \in \cF$ with
$\delta_\Pi(F)=\delta(D,\Pi)$. 
In our generalizations, we will focus on the harder case that $\delta(D,\Pi)=0$
where we are looking for a solution $F$ with $|F|=k$. This version can
be solved in randomized polynomial time given a (projected) linear
representation of $D$~\cite{KW24}, but a deterministic polynomial-time
algorithm is open.

In \textsc{Delta-matroid Intersection} the input is two delta-matroids
$D_1=(V,\cF_1)$ and $D_2=(V,\cF_2)$ and we are seeking a set $F$ that is
feasible both in $D_1$ and $D_2$. Again, we will focus on the case
where we have an additional requirement that $|F|=k$ for an integer
$k$ given in the input, and we will assume that $D_1$ and $D_2$ are
(projected) linear delta-matroids provided with representations over a
common field. Again, this version can be solved in randomized
polynomial time~\cite{KW24}.

We investigate the parameterized complexity of generalizations of
these problems. We focus on the following variants, as surveyed in Section~\ref{sec:ourresults}.
Restricted to linear or projected linear inputs, they are as follows. 
\begin{itemize}
\item \textsc{$q$-Delta-matroid Intersection}:
  Given $q$ (projected) linear delta-matroids $D_i=(V,\cF_i)$, $i \in [q]$
  represented over a common field $\F$, and an integer $k \in \N$,
  is there a set $F \subseteq V$ with $|F|=k$ such that $F \in \cF_i$
  for every $i \in [q]$?
\item \textsc{$q$-Delta-matroid Parity}:
  Given a (projected) linear delta-matroid $D=(V,\cF)$ provided with linear
  representation, a partition $\cP$ of $V$ into blocks of size $q$,
  and $k \in \N$, is there a union of $k$ blocks from $\cP$ that is
  feasible in $D$? 
\item \textsc{Delta-matroid Set Packing}:
  Given a (projected) linear delta-matroid $D=(V,\cF)$ provided with linear
  representation, a partition $\cP$ of $V$ into arbitrary blocks,
  and $k \in \N$, is there a set $F \in \cF$ with $|F|=k$ which is the 
  union of blocks from $\cP$?
\end{itemize}
We also recall the following special cases when $q=3$:
\begin{itemize}
\item \textsc{DDD Intersection} refers to \textsc{3-Delta-matroid Intersection}
\item \textsc{DDM Intersection} refers to the special case of
  \textsc{DDD Intersection} where $D_3$ is a linear matroid
\item \textsc{D$\Pi$M Intersection} is \textsc{Delta-Matroid Parity}
  with an additional parity constraint $\Pi$, i.e.,
  \textsc{DDM Intersection} when $D_2$ is a pairing delta-matroid
\end{itemize}
Over linear matroids, these problems have a status as follows.
\textsc{$q$-Matroid Intersection} for linear matroids can be solved in
randomized time $O^*(2^{(q-2)k})$ over a field of characteristic 2, and
randomized time and space $O^*(2^{qk})$ more generally~\cite{EKW23}.
The time for general fields can be improved to $O^*(4^k)$ for $q=3$~\cite{EKW23,BrandKS23}
and for $q=4$ over characteristic 0~\cite{BrandKS23}.
For the remaining problems, the fastest algorithm runs in $O^*(2^{qk})$
respectively $O^*(2^k)$~\cite{EKW23}.
All of these algorithms are randomized.
There are also somewhat slower, but single-exponential deterministic
algorithms over characteristic 0~\cite{BrandP21}
and over arbitrary fields~\cite{FominLPS16JACM}.

We consider the two basic parameters $k$ (cardinality of the solution)
and $r$ (the rank of the involved delta-matroid(s)). We also consider
the special case where some $D_i$'s are in fact basis
delta-matroids of linear matroids.

The algorithm for the matroid case of the above, using determinantal
sieving, works as follows. Let $V=\{v_1,\ldots,v_n\}$
be the ground set and $X=\{x_1,\ldots,x_n\}$ a set of variables. For
\textsc{$q$-Matroid Intersection}, on matroids $M_1, \ldots, M_q$,
we can use the Cauchy-Binet formula to efficiently evaluate a
polynomial $P(X)$ whose monomials enumerate the elements of the
intersection $M_1 \cap M_2$, where we can sieve for those monomials
which in addition are bases of $M_3, \ldots, M_q$ (see~\cite{EKW23}).
For the remaining problems, we replace $P(X)$ by the much simpler
polynomial whose monomials enumerate unions of blocks from $\cP$
and apply sieving for all the matroids $M_1, \ldots, M_q$.

Over linear delta-matroids, our findings are the following. To begin
with, under a pure cardinality parameter all the problems above are
either in P (for $q=2$) or W[1]-hard. This holds even in relatively
restricted cases, including twisted matroids (see Section~\ref{sec:dm-part1-hardness}).
On the other hand, under a rank parameter we recover all the above
results over delta-matroids, by just replacing the cardinality parameter $k$
by the rank parameter $r$. In fact, the strategy is the same as for
linear matroids. First, we generalize the determinantal sieving result
(which is phrased in terms of sieving for bases of a linear matroid)
to sieving for sets that are feasible in a linear delta-matroid of
rank $r$; this can be done with the same dependency on $r$ as
determinantal sieving has on $k$. Next, we construct a polynomial
which enumerates solutions to \textsc{Delta-matroid Intersection},
and sieve for the remaining constraints over this polynomial.
This polynomial is simply the Pfaffian of a skew-symmetric matrix
constructed by Koana and Wahlström~\cite{KW24} as a linear
representation of delta-sums of linear delta-matroids.

To give two further observations, we first note that the result only
requires that the delta-matroid that is plugged into the sieving
procedure has bounded rank; the initial two layers $D_1$ and $D_2$
that are used to construct $P(X)$ can be arbitrary. Second, if any
layer $D_i$ is a linear matroid, then naturally we can simply compute
its $k$-truncation and replace its rank by $k$ in the above-mentioned
results.

In particular, for the problems with $q=3$ the status is as follows.
\begin{itemize}
\item \textsc{DDD Intersection} is W[1]-hard parameterized by $k$,
  but FPT parameterized by $\rank(D_3)$
\item \textsc{DDM Intersection} and \textsc{D$\Pi$M Intersection}
  are FPT parameterized by $k$
\end{itemize}

An additional parameter that has been considered for delta-matroids is
the \emph{spread}~\cite{Moffatt19deltamatroids}. The spread of a
delta-matroid $D=(V,\cF)$ is $\max_{F \in \cF} |F|-\min_{F \in \cF} |F|$, 
and it is zero precisely when $D$ is the basis delta-matroid of a matroid.
However, if the spread of $D$ is $s$, then the rank is at most $k+s$,
hence the rank is the stronger parameter. On the other hand,
parameterizing purely by spread is unreasonable since
\textsc{3-Matroid Intersection} (with spread 0) is NP-hard. 
Thus the spread is not a useful parameter for the problems listed
above, but it could potentially be useful in other questions.
For example, is \textsc{Delta-matroid Intersection} FPT parameterized
by the spread, even if the delta-matroids are not linear?
Note that this does not allow the trick of reducing from
\textsc{Delta-matroid Parity} since the pairing delta-matroid has
the largest possible spread $|V|$. 

\paragraph*{Structure of the section.}
We demonstrate the enumerating polynomial in Section~\ref{sec:dm-gf};
we show the linear delta-matroid sieving procedure in Section~\ref{sec:dm-sieve};
we show FPT results in Section~\ref{sec:dm-part1-fpt} and
hardness results in Section~\ref{sec:dm-part1-hardness}.

\subsection{Generating functions for delta-matroid problems}
\label{sec:dm-gf}
In preparation for the positive results of this section, we present efficiently computable 
multivariate generating functions (i.e., \emph{enumerating polynomials}, in the terminology of Eiben et al.~\cite{EKW23})
for the \textsc{Delta-matroid Intersection} and \textsc{Delta-matroid Parity} problems over linear delta-matroids.
This follows from the algebraic algorithms of Koana and Wahlström~\cite{KW24}.

To reduce notational overhead, we establish some terms. Let $\cS \subseteq 2^V$ be a set system.
Let $X=\{x_v \mid v \in V\}$ be a set of variables. An \emph{enumerating polynomial} for $\cS$ over a field $\mathbb{R}$
is a polynomial
\[
  P_\cS(X) = \sum_{S \in \cS} c_S \prod_{v \in S} x_v,
\]
where $c_S \in \mathbb{F}$ is some non-zero constant.
We base the results on the following support lemma (derived from the construction of linear representation for $D_1 \Delta D_2$~\cite{KW24}).

\begin{lemma} \label{lm:delta-combo-polynomial}
  Let $D_1=(V,\cF_1)$ and $D_2=(V,\cF_2)$ be linear delta-matroids provided as representations over a common field $\F$.
  Let $X=\{x_{v,i} \mid v \in V, i \in [3]\}$ be a set of variables.
  There is a skew-symmetric matrix $A$ indexed by $V \cup T$ for a new set of elements $T$
  such that for every $S \subseteq V$ we have
  \[
    \Pf A[S \cup T] = \sum_{F_1 \in \cF_1, F_2 \in \cF_2 \colon F_1 \Delta F_2=S}
    f(F_1,F_2) \prod_{v \in F_1 \cap S} x_{v,1} \prod_{v \in F_2 \cap S} x_{v,2} \prod_{v \in F_1 \cap F_2} x_{v,3},
  \]
  where $f(F_1,F_2)$ is a non-zero constant for each $F_1 \in \cF_1$, $F_2 \in \cF_2$.   
\end{lemma}
\begin{proof}
  Let $D_1^*=\bD(A_1)/T_1$ and $D_2^*=\bD(A_2)/T_2$ be contraction representations of the duals
  of $D_1$ and $D_2$, where $T_1 \cap T_2=\emptyset$. This can be constructed in matrix multiplication
  time~\cite{KW24}. 
  Define a matrix $A$ indexed by $V \cup V_1 \cup T_1 \cup V_2 \cup T_2$
  where $V_1, V_2$ are copies of $V$ such that
  $A[V_1 \cup T_1]=A_1$, $A[V_2 \cup T_2]=A_2$,
  and for every $v \in V$ (with corresponding copies $v^1 \in V_1$ and $v^2 \in V_2$)
  let $A[v,v^1]=x_{v,1}$, $A[v,v^2]=x_{v,2}$ and $A[v^1,v^2]=x_{v,3}$,
  completed so that $A$ is skew-symmetric and all other positions are 0.
  We claim that $A$ is the matrix we need, with $T=V_1 \cup T_1 \cup V_2 \cup T_2$.
  Indeed, as in~\cite{KW24}, we view $A$ as the sum of the matching delta-matroid $A_H$
  whose entries correspond to variables $X$ and $A'$ which retains the copies of $A_1$ and $A_2$.
  Then by the Pfaffian sum formula in Lemma~\ref{lemma:sum-pf}, for $S \subseteq V$,
  \begin{align*}
    \Pf A[S \cup T] &= \sum_{S=S_1 \uplus S_2} \sum_{S' \subseteq V \setminus S}
    \sigma_{S, S'} \prod_{v \in S_1} x_{v,1} \prod_{v \in S_2} x_{v,2} \prod_{v \in S'} x_{v,3} \cdot
                      \Pf A_1[S_1' \cup T_1] \cdot
                      \Pf A_2[S_2' \cup T_2],
  \end{align*}
  where $\sigma_{S,S'} \in \{ 1, -1 \}$, $\uplus$ denotes disjoint union,
  $S_1'=V_1 \setminus \{v^1 \mid v \in S_1 \cup S'\}$
  and $S_2'=V_2 \setminus \{v^2 \mid v \in S_2 \cup S'\}$.
  A term $(S_1,S_2,S')$ of this sum contributes non-zero precisely when
  $S_1 \cup S' \in \cF_1$ and $S_2 \cup S' \in \cF_2$, since $A_1$ and $A_2$
  represent the duals of $D_1$ and $D_2$. Thus letting $F_1=S_1 \cup S'$
  and $F_2=S_2 \cup S'$ we have $S'=F_1 \cap F_2$, $S_1=F_1 \setminus F_2$
  and $S_2=F_2 \setminus F_1$ as expected. 
\end{proof}

It is easy to specialize this construction into providing the enumerating polynomials we need.

\begin{corollary} \label{cor:intersection-enum}
  There are enumerating polynomials for (projected) linear delta-matroid intersection  
  and parity.
\end{corollary}
\begin{proof}
  For the former, let $D_1=D_1'|Z_1$ and $D_2=D_2'|Z_2$ be projections (possibly with $Z_1, Z_2=\emptyset$)
  of linear delta-matroids $D_1'$, $D_2'$ and apply Lemma~\ref{lm:delta-combo-polynomial}
  to $D_1$ and the dual $D_2^*$. Let $(A,T)$ be the result. Set $x_{v,3}=0$ and $x_{v,2}=1$ for every $v \in V$
  and evaluate $\Pf A$ on the result. Let $X_i=\{x_{v,i} \mid v \in V\}$, $i=1, 2$. 
  This restricts the sum to be over pairs $(S_1,S_2)$ where $S_1 \cup S_2=V$
  and $S_1 \cap S_2=\emptyset$. Since the construction is performed over the dual of $D_2$,
  this corresponds to enumeration over elements $S_1 \in \cF(D_1) \cap \cF(D_2)$
  and $\Pf A$ as a polynomial over the remaining variables $X_1$ is the polynomial we seek.
  For the second, let $(D,\Pi)$ be an instance of \textsc{Delta-matroid Parity}.
  Let $D_\Pi$ be the pairing delta-matroid on pairs $\Pi$.
  Then perfect delta-matroid parity solutions correspond to members of $D \cap D_\Pi$. 
  For the projected variants, in both cases we can reduce the inputs to elementary projections
  (i.e., $Z_1=\{z_1\}$ and $Z_2=\{z_2\}$) and explicitly enumerate over contracting or
  deleting each of $z_1$ and $z_2$ before proceeding as above. 
\end{proof}

\subsection{Delta-matroid sieving}
\label{sec:dm-sieve}

In this section, we give a generalization of determinantal sieving \cite{EKW23}.
To that end, we first propose a ``sparse'' representation for delta-matroids.

\paragraph*{Sparse representation.}

For a delta-matroid of rank $r$, its linear representation may involve $\Omega(n^2)$ non-zero entries, even if $r = O(1)$.\footnote{Consider a delta-matroid, where all sets of size 0 or 2 are feasible. The most straightforward representation, in our view, is a matrix $xy^T - yx^T$, where $x$ and $y$ are random vectors of dimension $n$.}
We give a sparse representation that contains $O(rn)$ non-zero entries, in the context of twist representation as well as contraction representation.

First, let us consider twist representation.
Given a delta-matroid $D = \bD(A) \Delta S$ in twist representation, let $F_{\max}$ be a maximum feasible set. 
As $A[S \Delta F_{\max}]$ is non-singular, the pivoting $A^* = A * (F_{\max} \Delta S)$ is well-defined, and $D = \bD(A) \Delta (F_{\max} \Delta S) \Delta F_{\max} = \bD(A^*) \Delta F_{\max}$ by Lemma~\ref{lemma:tucker}.
We claim that $A^*[V \setminus F_{\max}] = O$.
Assume for contradiction that there is a pair $\{ v, v' \} \subseteq V \setminus F_{\max}$ such that $A^*[v, v'] \ne 0$.
This implies that $\{ v, v' \}$ is feasible in $\bD(A^*)$, and consequently, $F_{\max} \cup \{ v, v' \}$ is feasible in $D$.
However, this contradicts the maximality of $F_{\max}$.

For contraction representation, we
transform the sparse representation into a contraction representation via Equation~\eqref{eq:twist-to-contraction}.
We then obtain $D = \bD(A') / T$, where
\begin{align*}
  \kbordermatrix{
    & T & S & V \setminus S \\
    T & A[S] & -I & -A[S, V \setminus S] \\
    S & I & O & O \\
    V \setminus S & -A[V \setminus S, S] & O & O \\
  }
\end{align*}

\paragraph*{Delta-matroid sieving.}

As in determinantal sieving, polynomial interpolation and inclusion-exclusion will be crucial for delta-matroid sieving as well:

\begin{lemma}[Interpolation]
  \label{lemma:interpolation}
  Let $P(z)$ be a polynomial of degree $n - 1$ over a field $\F$.
  Suppose that $P(z_i) = p_i$ for distinct $z_1, \cdots, z_n \in \F$.
  By the Lagrange interpolation, 
  \[
    P(z)
    = \sum_{i \in [n]} p_i \prod_{j \in [n] \setminus \{ i \}} \frac{z - z_j}{z_i - z_j}.
  \]
  Thus, given $n$ evaluations $p_1, \dots, p_n$ of $P(z)$, the coefficient of $z^t$ in $P(z)$ for every $t \in [n]$ can be computed in polynomial time.
\end{lemma}

\begin{lemma}[Inclusion-exclusion \cite{Wahlstrom13STACS}]
  \label{lemma:inclusion-exclusion}
  Let $P(Y)$ be a polynomial over a set of variables $Y = \{ y_1, \cdots, y_n \}$ and a field of characteristic two.
  For $T \subseteq [n]$, $Q$ be a polynomial identical to $P$ except that the coefficients of monomials not divisible by $\prod_{i \in T} y_i$ is zero.
  Then, $Q = \sum_{I \subseteq T} P_{-I}$, where $P_{-I}(y_1, \cdots, y_n) = P(y_1', \cdots, y_n')$ for $y_i' = y_i$ if $i \notin I$ and $y_i' = 0$ otherwise. 
\end{lemma}

We will also utilize the fact that the Pfaffian of a sparse representation can be expressed as a sum of the product of Pfaffian and determinant:

\begin{lemma} \label{lemma:pf-det-decomposition}
  For a skew-symmetric matrix $A$ indexed by $V$ and a matrix $B$ whose rows and columns are indexed by $V'$ and $V$, respectively, with $|V| \le |V'|$,
  \begin{align*}
    \Pf C = \sum_{U \subseteq V,\, |U| = |V| - |V'|} \sigma_{U} \Pf A[U] \cdot \det B[V \setminus U, V'], 
    \text{ where }
    C = \begin{pmatrix}
      A & B \\ -B^T & O
    \end{pmatrix}
  \end{align*}
  and $\sigma_{U} \in \{ 1, -1 \}$.
\end{lemma}
\begin{proof}
  Letting
  \begin{align*}
    C = C_1 + C_2, \text{ where }
    C_1 = \begin{pmatrix} A & O \\ O & O \end{pmatrix}  \text{ and }
    C_2 = \begin{pmatrix} O & B \\ -B^T & O \end{pmatrix}
  \end{align*}
  Lemmas~\ref{lemma:sum-pf} yields
  \begin{align*}
    \Pf C = \sum_{U \subseteq V \cup V'} \sigma_U \Pf C_1[U] \cdot \Pf C_2[(V \cup V') \setminus U].
  \end{align*}
  Observe that $\Pf C_1[U] = 0$ for any subset $U$ where $|U| > |V|$, as the support graph of $C_1$ contains edges exclusively within $V$.
  Similarly, $\Pf C_2[(V \cup V') \setminus U] = 0$ when $|U| < |V|$, since all edges in the support graph of $C_2$ are incident with vertices in $V'$.
  These observations imply that non-zero contributions to the sum occur only when $U$ has size exactly $|V| - |V'|$.
  Consequently, $\Pf C_1[U] = \Pf A[U]$ and $\Pf C_2 [(V \cup V') \setminus U] = \det B[V \setminus U, V']$ by Lemma~\ref{lemma:det-pf}, which concludes the proof.
\end{proof}

Before presenting our sieving method, let us introduce notations for polynomials, which is in consistent with \cite{EKW23}.
Consider a polynomial $P(X)$ over a set of variables $X = \{ x_1, \cdots, x_n \}$.
A monomial is a product $m = x_1^{m_1} \dots x_n^{m_n}$, where $m_1, \dots, m_n$ are nonnegative integers.
A monomial $m$ is called multilinear if $m_i \le 1$ for each $i \in [n]$.
The \emph{support} of a monomial $m$, denoted by $\supp(m)$, is $\{ i \in [n] \mid m_i
> 0 \}$. 
We also use the notation $X^m$ for the monomial $m=x_1^{m_1} x_2^{m_2} \cdots x_n^{m_n}$,
where its coefficient is excluded.
The coefficient of $m$ in $P$ is denoted by $P(m)$.
Therefore, the polynomial $P$ can be written as $P(X)=\sum_m P(m) X^m$
where $m$ ranges over all monomials in $P(X)$.

%%% Theorem replaced by "repeatable". Keeping the original here in
%%% case this 'fully detailed' version is not what we want to put 
%%% in the introduction
%\begin{theorem} \label{theorem:sieving-char-2}
%  Let $D=(V,\cF)$ be a (projected) linear delta-matroid of rank $r$, $V=\{v_1,\ldots,v_n\}$ given as a sparse representation $\bD(A) / T$,
%  and let $P(X)$ be a homogeneous polynomial of degree $k$ given as black-box access over a
%  sufficiently large field of characteristic 2, $X=\{x_1,\ldots,x_n\}$.
%  In time $O^*(2^r)$ we can sieve for those terms of $P(X)$ which
%  are multilinear in $X$ and whose support $\{x_i \mid i \in I\}$
%  in $X$ is such that $\{v_i \mid i \in I\}$ is feasible in $D$. 
%\end{theorem}

\dmsieve*

\begin{proof}
  Our objective is to transform the polynomial $P$ into $Q$ such that (i) every non-multilinear monomial $m$ vanishes, (ii) every multilinear monomial $m$ is translated into $\Pf A[T \cup V_m] \cdot m$, where $V_m = \{ v_i \mid i \in \supp(m) \}$, and (iii) $Q$ can be evaluated in $O^*(2^r)$ time via black-box access to $P$.

  Introduce a set of auxiliary variables $Y=\{ y_t \mid t \in T \}$.
  Let $A_T$ be a $T \times T$ matrix obtained from $A[T]$ by multiplying the row and column $t$ by $y_t$.
  Consider the coefficient of $z^k$ in the characteristic polynomial $\det(A_T + zI)$.
  It is known to be the sum of all $r - k$ principle minors:
  \begin{align*}
    \sum_{U \in \binom{T}{r-k}} \det A_T[U]
    = \sum_{U \in \binom{T}{r-k}} \det A[U] \prod_{t \in U} y_t^2.
  \end{align*}
  Now define $P_T(X, Y)$ be its square root.\footnote{Square roots are well-defined over a field of characteristic 2 because $\alpha^2 = \beta^2$ implies $\alpha = \beta$.
  Furthermore, square roots can be efficiently computed: If $\F$ is a finite field with $2^m$ elements, the square root of $\alpha$ over $\F$ is $\alpha^{2^{m-1}}$ because $(\alpha^{2^{m-1}})^2 = \alpha \cdot \alpha^{2^m-1} = \alpha$ by Fermat's little theorem.}
  This equals
  \begin{align*}
    P_T(X, Y) = \sum_{U \in \binom{T}{r - k}} \Pf A[U] \prod_{t \in U} y_t,
  \end{align*}
  since $(\Pf A[U])^2 = \det A[U]$ and $\sum_i \alpha_i^2 = (\sum_i \alpha_i)^2$ over a field of characteristic 2.

  Define another polynomial $P_V(X, Y)$ by
  \[
    P_V(X,Y) = P\left(x_1 \sum_{t \in T} y_t A[t,v_1], \ldots, x_n \sum_{t \in T} y_t A[t, v_n]\right).
  \]
  Furthermore, define $Q(X,Y)$ as the result of extracting terms from $P'(X, Y) = P_T(X, Y) \cdot P_V(X, Y)$ that have at least degree one in each variable $y_i$ for $i \in [k]$,
  and set $Q(X)=Q(X,1)$.
  Note that $Q$ can be evaluated from $2^r$ evaluations of $P'$, using inclusion-exclusion in Lemma~\ref{lemma:inclusion-exclusion}.
  We aim to show that every monomial $m$ is transformed into $\Pf A[T \cup V_m] \cdot m$.
  To that end, let us examine $P_V(X, Y)$ first.
  For every monomial $m=x_1^{m_1} \cdots x_n^{m_n}$ in $P$,
  the coefficient of $m$ in $P_V$ is given by
  \[
    P_V(m) = P(m) \cdot \prod_{v_i \in V_m} \left( \sum_{t \in T} y_t A[t, v_i]\right)^{m_i},
  \]
  where $P(m)$ is the coefficient of $m$ in $P$.
  In particular, the coefficient of $\prod_{t \in W} y_t$ for $W \in \binom{T}{k}$ in this expression is:
  \[
  P(m) \cdot \sum_{\sigma} \left( \prod_{t \in W} A[t, \sigma(t)] \right),
  % = P(m) \cdot X^m \cdot \det A[W, (i_1, \dots, i_k)] \prod_{t \in W} y_t
  \]
  where the sum is over all mappings $\sigma \colon W \to V_m$ such that for each $i \in \supp(m)$, exactly $m_i$ elements of $W$ are that mapped to $v_i$.
  This is equivalent to the determinant of a matrix $A_m$ whose columns comprise $m_i$ copies of $A[T, v_i]$ for each $v_i \in V_m$.
  Consequently, this term vanishes if $m$ is not multilinear.
  For multilinear terms $m$, it simplifies to $P(m) \cdot \det A[T, V_m]$.
  Thus, the part of $P_V(m)$ that is multilinear in $Y$ is
  \begin{align*}
    P(m) \cdot \sum_{W \in \binom{T}{k}} \det A[W, V_m] \prod_{t \in W} y_t.
  \end{align*}
  Note that the non-multilinear part of $P_V(m)$ is irrelevant since it will be excluded from $Q$ in the inclusion-exclusion step.
  
  Multiplying this with $P_T(X, Y)$ yields
  \begin{align*}
    P(m) \cdot \sum_{U \in \binom{T}{r - k}, W \in \binom{T}{k}} \Pf A[U] \cdot \det A[T, V_m] \cdot \left(\prod_{t \in U} y_t \right) \left(\prod_{t \in W} y_t \right)
  \end{align*}
  The coefficient of $\prod_{t \in T} y_t$ in this expression, which equals $Q(m)$, is
  \begin{align*}
    P(m) \cdot \sum_{W \in \binom{T}{k}} \Pf A[T \setminus W] \cdot \det A[W, V_m] = P(m) \cdot \Pf A[T \cup V_m],
  \end{align*}
  where the equality follows from Lemma~\ref{lemma:pf-det-decomposition}.
  Finally, evaluating $Q(X)$ at random coordinates yields the desired result by the Schwartz-Zippel lemma.
  Note that $Q(X)$ remains homogeneous of degree $k$. 
\end{proof}

\paragraph*{Sieving over general fields.}

Following prior work~\cite{BrandDH18,EKW23}, we will use the exterior algebra to develop a sieving method applicable to general fields.
For a field $\F$, $\Lambda(\F^T)$ is a $2^{|T|}$-dimensional vector space, where each basis $e_I$ corresponds to a subset $I \subseteq T$.
Each element $a = \sum_{I \subseteq [k]} {a_I} e_I$ is called an \emph{extensor}.
We denote the vector subspace spanned by bases $e_I$ with cardinality $|I| = i$ for $i \in \{ 0, \dots, |T| \}$, by $\Lambda^i(\F^T)$.
Notably, $\Lambda^0(\F^k)$ is isomorphic to $\F$, and $\Lambda^1(\F^k)$ is isomorphic to the vector space $\F^k$.
Addition in $\Lambda(\F^k)$ is defined element-wise, while
multiplication, referred to as \emph{wedge product}, is defined as follows:
when $I$ and $J$ intersect, the wedge product yields zero;
otherwise, $e_I \wedge e_J = (-1)^{\sigma(I, J)} e_{I \cup J}$, where $\sigma(I, J) = \pm 1$ is the sign of the permutation mapping the concatenation of $I$ and $J$ into the increasing sequence of $I \cup J$.
For vectors $v, v' \in \F^k$, the wedge product exhibits self-annihilation $v \wedge v = 0$ and anti-commutativity $v \wedge v' = -v' \wedge v$. 
For a $T \times T$-matrix $A$, the wedge product of the column vectors yields $\det A \cdot e_T$.
We refer interested readers to prior works~\cite{BrandDH18,EKW23} for a more accessible introduction to exterior algebra used in this work.

% An extensor $a \in \Lambda(\F^k)$ is \emph{decomposable} if there are vectors $v_1, \dots, v_{\ell}$ such that $a = v_1 \wedge \cdots \wedge v_{\ell}$.
% A decomposable extensor $a$ is zero if the vectors $v_1, \dots, v_{\ell}$ are linearly dependent.
% For two decomposable extensors $a, a'$, it holds that $a \wedge a' = \pm a' \wedge a$ (this is generally not the case, e.g., $e_1 \wedge (e_2 \wedge e_3 +  e_4) = e_1 \wedge e_2 \wedge e_3 + e_1 \wedge e_4$ and $(e_2 \wedge e_3 +  e_4) \wedge e_1 = e_1 \wedge e_2 \wedge e_3 - e_1 \wedge e_4$).

Now we discuss the complexity of operations in exterior algebra.
The addition of two extensors can be done by $2^{|T|}$ field operations.
The wedge product $a \wedge a'$ of two extensors $a \in \Lambda(\F^k)$ and $b \in \Lambda^i(\F^k)$ involves $2^k \binom{k}{i}$ field operations, according to the definition (resulting in $O^*(2^k)$ time complexity for $i \in O(1)$).
More generally, an $O(2^{\omega k / 2})$-time algorithm for computing the wedge product is presented by W\l{}odarczyk~\cite{Wlodarczyk19}, where $\omega < 2.372$ is the matrix multiplication exponent.
For a detailed exposition, one may refer to Brand's thesis~\cite{Brand19thesis}.

To address issues arising from non-commutativity in exterior algebra, we use the \emph{lift mapping} $\bar{\phi} \colon \Lambda(\F^T) \to \Lambda(\F^{T \cup T'})$, where $T' = \{ t' \mid t \in T \}$ and 
% $\bar{\phi}(v) = v_1 \wedge v_2$ for $v_1 = (v \quad 0)^T, v_2 = (0 \quad v)^T \in \F^{2k}$.
\vspace{-3ex}
\begin{align*}
  \bar{\phi}(v) = v_1 \wedge v_2 \text{ for } v_1 =\,
  \kbordermatrix{
    & \\
    T & v \\ 
    T' & 0 \\
  } \, \text{ and } \,
  v_2 =\, \kbordermatrix{
    & \\
    T & 0 \\ 
    T' & v \\
  }
\end{align*}
This mapping has been useful in previous studies \cite{Brand19,BrandDH18,EKW23}.
The subalgebra generated by the image of $\bar{\phi}$ exhibits commutativity, i.e., $\bar{\phi}(v) \wedge \bar{\phi}(v') =  \bar{\phi}(v') \wedge \bar{\phi}(v)$.
% Note that the sign is preserved when the transposition occurs twice.

For an element $t'$ in $T'$, let $t''$ denote the corresponding element $t$ in $T$. Similarly, for a subset $W'$ of~$T'$, let $W'' = \{ t'' \mid t' \in W' \}$.

We will assume that the polynomial is represented by an \emph{arithmetic circuit}.
It is a directed acyclic graph with one sink node (referred to as output gate) in which every source node is labelled with either a variable $x_i$ or an element of $\F$ (referred to as input gate) and all other nodes are either addition gates or multiplication gates.
It is further assumed that every sum and product gate has fan-in 2.

\begin{theorem} \label{theorem:sieving-general}
  Let $D=(V,\cF)$ be a (projected) linear delta-matroid of rank $r$, $V=\{v_1,\ldots,v_n\}$ given as a sparse representation $\bD(A) / T$,
  and let $P(X)$ be a homogeneous polynomial of degree $k$ given as an arithmetic circuit over
  sufficiently large field, $X=\{x_1,\ldots,x_n\}$.
  In time $O^*(2^{\omega r})$ we can sieve for those terms of $P(X)$ which
  are multilinear in $X$ and whose support $\{x_i \mid i \in I\}$
  in $X$ is such that $\{v_i \mid i \in I\}$ is feasible in $D$. 
\end{theorem}

\begin{proof}
We evaluate the circuit over the subalgebra of $\Lambda(\F^{T \cup T'})$ by substituting every variable $x_i$ with $x_i a_i$, where $a_i = \phi(A[T, v_i])$.
Let $r \in \Lambda^k(\F^{T \cup T'})$ denote the resulting extensor.
Note that with each variable $x_i$ substituted with an element from $\F$, the extensor $r$ can be computed in $O^*(2^{\omega k})$ time.

For each monomial $m$ in $P$, there is a term containing $e_{W \cup W'}$ for $W \in \binom{T}{k}$ and $W' \in \binom{T'}{k}$:
\begin{align*}
  \bigwedge_{v_i \in V_m} \left( \sum_{t \in W} A[t, v_i] e_t \wedge \sum_{t' \in W'} A[t'', v_i] e_{t'} \right)  = (-1)^{\binom{r - k}{2}} \cdot \det A[W, V_m] \cdot \det A[W'', V_m] \cdot e_{W \cup W'},
\end{align*}
where $V_m = \{ v_i \mid i \in \supp(m) \}$.
In particular, all non-multilinear terms vanish.
Consequently, the coefficient of $e_{W \cup W'}$ in $r$ is given by
\begin{align*}
  (-1)^{\binom{r-k}{2}} \sum_{m} P(m) \cdot \det A[W, V_m] \cdot \det A[W'', V_m],
\end{align*}
where the sum is taken over all multilinear monomials $m$ of $P$.

We compute the Pfaffian $A[U]$ for each $U \in \binom{T}{r-k}$ in $O^*(2^r)$ time.
For each partition $U \cup W = T$, and each partition $U' \cup W' = T'$ with $|W| = |W'| = k$, we multiply the coefficient of $e_{W \cup W'}$ by $(-1)^{\binom{r-k}{2}} \sigma_{U} \sigma_{U''} \Pf [U] \cdot \Pf A[U'']$, where $\sigma_U, \sigma_{U''} \in \{ 1, -1 \}$ are defined as in Lemma~\ref{lemma:pf-det-decomposition}.
We then sum up the products.
The resulting sum, denoted by $Q(X)$, contains the following term for each monomial in $P$:
\begin{align*}
  P(m) \sum_{U, U'' \in \binom{T}{r - k}} \sigma_U \sigma_{U''} \Pf A[U] \cdot \Pf A[U''] \cdot \det A[T \setminus U, V_m] \cdot \det A[T \setminus U'', V_m]. 
\end{align*}
This equals $P(m) \cdot \det A[T \cup V_m]$ since by Lemma~\ref{lemma:pf-det-decomposition}, 
\begin{align*}
  \det A[T \cup V_m] = (\Pf A[T \cup V_m])^2
  = \left( \sum_{U \in \binom{T}{r - k}} \sigma_U \Pf A[U] \cdot \det A[T \setminus U, V_m] \right)^2,
\end{align*}
which matches  the above expression.
It follows that $Q(X) = \sum_{m} P(m) \cdot \det A[T \cup V_m] \cdot m$.
Evaluating $Q(X)$ at random coordinates yields the desired result by the Schwartz-Zippel lemma.
\end{proof}

We remark that when the arithmetic circuit is \emph{skew}, i.e., every multiplication is connected from an input gate, the running time improves to $O^*(4^r)$.

\subsection{Delta-matroid FPT results}
\label{sec:dm-part1-fpt}

We note the FPT consequences of the sieving results. The following is
a direct generalization of the fastest known algorithms for linear
matroid intersection~\cite{EKW23}. 

\begin{lemma} \label{lm:ddd-intersection}
  Let $D_1=(V,\cF_1), \ldots, D_q=(V,\cF_q)$ be (projected) linear delta-matroids represented over a common field $\F$,
  let $k \in \N$ and $r=\max_{i \geq 3} \rank D_i$. Then a feasible set of cardinality $k$
  in the common intersection $D_1 \cap \ldots \cap D_q$ can be found in randomized time and space $O^*(2^{O(qr)})$.
  In particular, if $\F$ is of characteristic 2 then this can be reduced to time $O^*(2^{(q-2)r})$ and polynomial space. 
\end{lemma}
\begin{proof}
  We assume $q \geq 3$ as otherwise the problem is in P~\cite{GeelenIM03,KW24}.
  Let $D'=(V',\cF)$ be the direct sum $D_3 \uplus \ldots \uplus D_q$, with
  $V'=V_3 \uplus \ldots \uplus V_q$ where each $V_i$ is a separate copy of $V$. 
  For each $v \in V$ and $i \in [q]$, $i \geq 3$, let $v^i$ denote the copy of $v$ in $V_i$ (and let $v^1=v^2=v$). 
  Then $D'$ is a linear delta-matroid of rank $(q-2)r$.
  Reduce $D_1$, $D_2$ and $D'$ to elementary projections over elements $z_1$, $z_2$ and $z_3$
  and guess for each $z_i$, $i=1, 2, 3$ whether $z_i$ is a member of the solution or not.
  For each such guess, correspondingly delete or contract $z_i$ and proceed as follows
  with the resulting linear delta-matroids. 
  Let $X=\{x_v \mid v \in V\}$ and let $P(X)$ be the enumerating polynomial for the intersection $D_1 \cap D_2$ of Cor.~\ref{cor:intersection-enum}.
  For each $v \in V$ and $i=3, \ldots, q$ let $x_{v,i}$ be a new variable associated with the element $v^i$ in $D'$
  and let $X'=\bigcup_{v \in V} \{x_{v,3}, \ldots, x_{v,q}\}$ be the set of all these variables.
  Let $P'(X')$ be $P(X)$ evaluated with $x_v=\prod_{i=3}^q x_{v,i}$ for each $v \in V$.
  By Theorems \ref{theorem:sieving-char-2} and \ref{theorem:sieving-general}, we can sieve in $P'(X')$ for a term that is feasible in $D'$
  by a randomized algorithm
  in time and space $O^*(2^{\omega (q-2)r})$ in the general case, and time $O^*(2^{(q-2)r})$ and polynomial space
  if $\F$ is over char.~2, as promised. In particular, for the success probability, 
  we can choose to work over a sufficiently large extension field
  of the given field at no significant penalty to the running time.
\end{proof}

In other words, \textsc{$q$-Delta-matroid Intersection} on linear
delta-matroids is FPT parameterized by the rank
(or even, parameterized by the third largest rank of the participating delta-matroids).

In summary, we get the following algorithmic implications (repeated
from Section~\ref{sec:ourresults}). Variants for projected linear
delta-matroids also apply, with the same guessing preprocessing step
as in Lemma~\ref{lm:ddd-intersection}.

%%%Outdated
%\begin{corollary} \label{cor:all-algs}
%  The following FPT results are possible. All the algorithms are randomized,
%  all delta-matroids and matroids are assumed to be represented over some common field $\F$, 
%  and the algorithms use polynomial space if $\F$ is of char.~2 and exponential space otherwise.
%  \begin{enumerate}
%  \item \textsc{DDM Intersection} is FPT parameterized by cardinality.
%    That is, given two linear delta-matroids $D_1$, $D_2$ and a linear matroid $M$,
%    in time $O^*(2^{O(k)})$ in general and $O^*(2^{k})$ over characteristic 2
%    we can find a set in $D_1 \cap D_2 \cap M$ of cardinality $k$.
%  \item \textsc{DDD Intersection} is FPT parameterized by rank of the third delta-matroid. 
%    That is, given three linear delta-matroids $D_1$, $D_2$, $D_3$, where $D_3$ has rank $r$, 
%    in time $O^*(2^{O(r)})$ in general and $O^*(2^{r})$ over characteristic 2
%    we can find a set in $D_1 \cap D_2 \cap D_3$ of cardinality $k$.
%  \item \textsc{Delta-matroid Set Packing} is FPT parameterized by rank. 
%    That is, given a linear delta-matroid $D=(V,E)$ of rank $r$, an arbitrary partition $\cP$ of $V$, and an integer $k$,
%    in time $O^*(2^{O(r)})$ in general and $O^*(2^r)$ over characteristic 2
%    we can find a feasible set $F$ in $D$ of cardinality $k$ which is a union of blocks from $\cP$. 
%    In particular, \textsc{$q$-Delta-matroid Parity}, where all blocks of $\cP$ have cardinality $q$,
%    is FPT parameterized by rank.
%  \item \textsc{D$\Pi$M Intersection} is FPT parameterized by cardinality.
%    That is, given a linear delta-matroid $D=(V,E)$, a partition $\Pi$ of $V$ into pairs,
%    a linear matroid $M=(V,\cI)$ and an integer $k$, we can find a set $F \subseteq V$
%    with $|F|=k$ such that $F$ is the union of pairs from $\Pi$, $F$ is feasible in $D$,
%    and $F$ is independent in $M$, in time $O^*(2^{O(k)})$ in general and $O^*(2^k)$ over characteristic 2.
%  \end{enumerate}
%\end{corollary}
%
%for reference:
% \begin{restatable}{theorem}{dmp1fpt} \label{tractable:rank}
%   The following problems are FPT over linear delta-matroids and
%   matroids provided as representations over a common field. 
%   \begin{itemize}
%   \item \textsc{DDD Intersection} parameterized by $r=\rank(D_3)$,
%     with a running time of $O^*(2^r)$ over characteristic 2
%     and $O^*(2^{O(r)})$ otherwise
%   \item More generally, \textsc{$q$-Delta-matroid Intersection} parameterized by $r$,
%     where $r$ is the maximum rank of $D_i$, $i \geq 3$, with a running
%     time of $O^*(2^{(q-2)r})$ over characteristic 2
%     and $O^*(2^{O(qr)})$ otherwise
%   \item \textsc{$q$-Delta-matroid Parity} and \textsc{Delta-matroid Set Packing}
%     parameterized by the rank $r$ of the delta-matroid,
%     with a running time of $O^*(2^r)$ over characteristic 2
%     and $O^*(2^{O(r)})$  in general
%   \item \textsc{DDM Intersection} and \textsc{D$\Pi$M Intersection},
%     parameterized by the cardinality $k$, 
%     with a running time of $O^*(2^k)$ over characteristic 2
%     and $O^*(2^{O(k)})$  in general
%   \item More generally, \textsc{Delta-matroid Intersection} and
%     \textsc{Delta-matroid Parity} with an additional $q-2$ matroid constraints,
%     parameterized by $q$ and the cardinality $k$, 
%     with a running time of $O^*(2^{(q-2)k})$ over characteristic 2
%     and $O^*(2^{O(qk)})$ in general
%   \end{itemize}
%   In all cases, the algorithms are randomized, and use polynomial
%   space over characteristic 2 but exponential space otherwise.
% \end{restatable}

\dmponefpt*

\begin{proof}
  The first two items follow directly from Lemma~\ref{lm:ddd-intersection}.
  For the third, let $\cP=V_1 \cup \ldots \cup V_n$ be the partition, 
  and define the polynomial
  \[
    P(X,z) = \prod_{i=1}^n (1+\prod_{v \in V_i} zx_v),
  \]
  $X=\{x_v \mid v \in V\}$. Now we can sieve in the coefficient of $z^k$ in $P(X,z)$ for
  a multilinear monomial whose support is feasible in $D$. 
  For items four and five, we can replace a parity constraint $\Pi$ by
  a pairing delta-matroid $D_\Pi$ over $\Pi$, if applicable,
  truncate each matroid $M_i$, $i \geq 3$ to rank $k$, then apply Lemma~\ref{lm:ddd-intersection}.
  In particular, a pairing delta-matroid can be represented over any field. 
\end{proof}

The following variant will be used later in the paper.
Suppose that we are given a linear delta-matroid $D = (V, \mathcal{F})$ represented by $D = \bD(A) / T$ for $A \in \mathbb{F}^{(V \cup T) \times (V \cup T)}$ and a set of edges $E \subseteq \binom{V}{2}$.
A matching $M$ in the graph $G=(V, E)$ is a \emph{delta-matroid matching} if $\bigcup M$ is feasible in $D$.
Furthermore, assume that the edges $E$ are colored in $k$ colors. 
\textsc{Colorful Delta-matroid Matching}
asks for a delta-matroid matching $M$ of $k$ edges whose edges all have distinct colors.
We show the following.

\dmmatching*

\begin{proof}
  Let $(D=(V,\cF), G=(V,E), c \colon E \to [k])$ be an instance of 
  \textsc{Colorful Delta-matroid Matching} as defined above.
  Let $D=\bD(A)/T$ be a representation of $D$. 
  We may assume that $G$ has degree 1: For every $v \in V$ with degree greater than one in $G$,
  introduce $d$ copies $v_1, \dots, v_d$ of $v$ and duplicate rows and columns of $v$ in $A$ for each copy.
  Let $V'$ be the new, larger ground set. Note that delta-matroid matchings are preserved.
  Now the result follows by reduction to \textsc{D$\Pi$M Intersection}:
  Let $M=(V',\cI)$ be the partition matroid where for every edge $uv \in E$ of color $c$,
  one member (say $u$) is placed in a set $S_c$ from which at most one element may be chosen,
  and the other (i.e., $v$) is a free element. Then $M$ is linear over every field.
  Thus, the problem has been reduced to an instance of \textsc{D$\Pi$M Intersection}
  $(D,E,M,k)$ with partition $\Pi=E$. 
\end{proof}

\subsection{Hardness results}
\label{sec:dm-part1-hardness}

To complement the above, we show that \textsc{DDD Intersection} is
W[1]-hard parameterized by $k$, even for some quite restrictive cases. 

\dmhardness*

Specifically, consider a delta-matroid $D=(V,\cF)$ on a ground set
$V=\bigcup_{i=1}^m \{x_i, y_{i,1}, \ldots, y_{i,n_i}\}$
with a feasible set $\cF$ defined as
\[
  F \in \cF \Leftrightarrow \forall i (x_i \in F \Leftrightarrow |\{ j \in [n_i]: y_{i,j} \in F \}| = 1)
\]
for all $F \subseteq V$. 
Let us note some ways to describe $D$.

\begin{lemma} \label{lm:star-forest-matching}
  $D$ is simultaneously a matching delta-matroid (for a star forest),
  a twisted unit partition matroid, and a twisted co-graphic matroid.
  (In particular, $D$ can be represented over every field.)
\end{lemma}
\begin{proof}
  As a matching delta-matroid, it is the matching delta-matroid on
  the star forest where for every $i \in [m]$ there is star with root $x_i$
  and leaves $y_{i,j}$, $j \in [n_j]$. On the other hand, $D \Delta X$
  (for $X=\{x_1,\ldots,x_m\}$) denotes the unit partition matroid where
  for every $i$ an independent set $I$ contains at most one element
  of $\{x_i, y_{i,1}, \ldots, y_{i,n_i}\}$.
  This can also be described as the co-graphic matroid for the graph $G$
  which for every $i \in [m]$ consists of a cycle
  on the set $\{x_i,y_{i,1}, \ldots,y_{i,n_i}\}$. 
  In particular, $M$ is regular~\cite{OxleyBook2}.
  (Indeed, it is represented by a matrix $A \in \{0,1\}^{[m] \times V}$
  where each element $x_i$ and $y_{i,j}$ is represented by the $i$-th unit vector.)
\end{proof}

We refer to $D$ as a \emph{star forest matching delta-matroid}, since this is the most specific description.
We have the following.

\begin{theorem}[Theorem~\ref{hard:ddd}, refined] \label{thm:ddd-hard}
  \textsc{DDD Intersection} is W[1]-hard parameterized by cardinality,
  even for the intersection of a star forest matching delta-matroid
  and two pairing delta-matroids. 
\end{theorem}
\begin{proof}
  We show a reduction from \textsc{Multicolored Clique}.
  Let $G=(V,E)$ be a graph with a partition $V=V_1 \cup \ldots \cup V_k$,
  where the task is to decide whether $G$ contains a clique $C=\{v_1,\ldots,v_k\}$
  where $v_i \in V_i$ for every $i \in [k]$. For $i, j \in [k]$, $i \neq j$,
  let $E_{i,j} \subseteq E$ be the edges between $V_i$ and $V_j$. 
  Assume w.l.o.g.\ that $E=\bigcup_{i \neq j} E_{i,j}$. 
  Define a ground set
  \[
    X \cup Y :=   \{x_{i,j:u} \mid i, j \in [k], i \neq j, u \in V_i\}
    \cup
    \{y_{i,j:u,v} \mid i, j \in [k], i \neq j,
    uv \in E_{i,j}, u \in V_i, v \in V_j\}.
  \]
  Let $D_1$ and $D_2$ be pairing delta-matroids
  such that the intersection $D_1 \cap D_2$ 
  induces a partition of $X$ into blocks $B_{i,u}=\{x_{i,j:u} \mid j \in [k]-i\}$
  and $Y$ into pairs $P_e=\{y_{i,j:u,v},y_{j,i:v,u}\}$, $e \in E$.
  For $D_3$, let $M$ be the partition matroid over blocks
  \[
    C_{i,j,u}=\{x_{i,j:u}\} \cup \{y_{i,j:u,v} \mid uv \in E_{i,j}\}.
  \]
  For $i, j \in [k]$ and $u \in V_i$, 
  let $D_3=M \Delta \{x_{i,j:u} \mid i, j \in [k], i \neq j, u \in V_i\}$.
  Then $F \in \cF(D_3)$ has the effect of an implication
  \[
    x_{i,j:u} \in F \Leftrightarrow \exists v : y_{i,j:u,v} \in F,
  \]
  which finishes the construction with a parameter $k'=2k(k-1)$.
  In particular, $D_3$ is a star forest matching delta-matroid.

  For correctness, on the one hand, let $C=\{v_1,\ldots,v_k\}$ be a multicolored $k$-clique.
  Then
  \[
    F=\{x_{i,j:v_i}, y_{i,j:v_i,v_j}  \mid i \in [k], j \in [k] - i\}  \in D_1 \cap D_2 \cap D_3
  \]
  is a feasible set of cardinality $k'$. In the other direction,
  let $F \in D_1 \cap D_2 \cap D_3$, and assume that $x_{i,j:v_i} \in F$ 
  for some $i, j, v_i$. This must hold, since $y_{i,j:v_i,w} \in F$
  for any $i, j, v_i, w$ implies $x_{i,j:v_i} \in F$ by $D_3$. Then by
  $D_1 \cap D_2$, $x_{i,a:v_i} \in F$ for every $a \in [k]-i$,
  and by $D_3$ for every $a \in [k]-i$ we have $y_{i,a:v_i,v_a} \in F$ 
  for some $v_iv_a \in E_{i,a}$. By $D_1 \cap D_2$ we also have
  $y_{a,i:v_a,v_i} \in F$ and by $D_3$ we have $x_{a,i:v_a} \in F$.
  Finally by $D_1 \cap D_2$ we have $x_{a,b:v_a} \in F$ for every
  $a, b \in [k]$, $a \neq b$ and by $D_3$ we have $x_{a,b:v_a,v_{b,a}} \in F$
  for some $v_{b,a} \in V_b$ for all pairs $a, b \in [k]$, $a \neq b$.   
  This represents two elements in $F$ for every ordered pair $a, b \in [k]$,
  $a \neq b$, thus since $|F| \leq k'$ this represents an exhaustive
  account of every element in $F$. Thus $v_{b,a}=v_b$ for all $a, b \in [k]$, 
  $a \neq b$, and repeating the argument shows that 
  $C=\{v_i\} \cup \{v_{j,i} \mid j \in [k]-i\}$ forms a clique.
\end{proof}

We note some other hard problem variants. These are justified primarily with the positive results related to
\textsc{Delta-matroid triangle cover}.

\begin{corollary}
  The following problems are W[1]-hard parameterized by $k$.
  \begin{enumerate}
  \item Finding a feasible set of cardinality $k$ in the intersection of three twisted regular matroids
  \item Finding a feasible set of cardinality $k$ in the intersection of three matching delta-matroids.
  \item \textsc{3-Delta-matroid Parity}, i.e., given
    a directly represented linear delta-matroid $D=(V,\cF)$,
    a partition of $D$ into triples, and an integer $k$, 
    find a union of $k$ triples that is feasible in $D$.
    In particular, \textsc{Delta-matroid Triangle Packing} is hard. 
  \item Given a directly represented linear delta-matroid $D=(V,\cF)$,
    a graph $G=(V,E)$, and an integer $k$, find a feasible
    set $F$ of size $k$ in $D$ that can be covered by $K_4$'s in $G$.
  \end{enumerate}
\end{corollary}
\begin{proof}
  The first two follow from Theorem~\ref{thm:ddd-hard}
  via Lemma~\ref{lm:star-forest-matching}.
  The third follows immediately. If $(D_1,D_2,D_3,k)$ is an instance of
  \textsc{DDD Intersection} from Theorem~\ref{thm:ddd-hard},
  then let $D$ be the direct sum of $D_1$, $D_2$ and $D_3$
  on ground set $V=V_1 \cup V_2 \cup V_3$ (where $D$ on $V_i$ is a copy of $D_i$).
  Let the triples consist of the three copies of every element
  of the original ground set. Then the result follows. 
  
  The fourth follows similarly. Note that the input is partitioned,
  so that there are $k$ colour classes of triples and the solution
  to the triangle-packing problem needs to use one triple per colour
  class. We can implement this constraint by adding, for every
  colour class $i$, an artificial element $z_i$ that we add as the
  fourth element to every triple in class $i$. Now the only way to
  cover a feasible set $F$ is to use at most one 4-set per colour
  class, and a feasible set of size $|F|=3k$ that is covered by 4-sets
  must be a disjoint union of triangles as above.
  It is easy to see that the 4-sets in this construction can be
  represented precisely as $K_4$'s in a graph $G$. 
\end{proof}

Finally, we note that finding a feasible $k$-path is W[1]-hard.
This is again in contrast to the case of matroids, where the
corresponding problem is FPT, and matroid-related techniques have been
used in algorithms for the basic \textsc{$k$-Path}
problem~\cite{EKW23,FominLPS16JACM} (see also~\cite{FominGKSS23soda}).

\begin{theorem}
  Given a linear delta-matroid $D=(V,\cF)$, a graph $G=(V,E)$ and an integer $k \in \N$, 
  it is W[1]-hard to find a $k$-path in $G$ whose vertex set is feasible in $D$. 
\end{theorem}
\begin{proof}
  Let $G=(V,E)$ with $V=V_1 \cup \ldots \cup V_k$ be
  an input to \textsc{Multicolored Clique}.
  
  Create a vertex set $U \cup W \cup \{s,t\}$ where
  $U=\{u_{i,j,v} \mid i, j \in [k], i \neq j, v \in V_i\}$ and
  $W=\{w_{i,j,e} \mid i, j \in [k], i \neq j, e \in E \cap (V_i \times V_j)\}$.
  Add edges as follows:
  \begin{itemize}
  \item $su_{1,2,v}$ for every $v \in V_1$
  \item $u_{i,j,v}w_{i,j,e}$ for every $w_{i,j,e} \in W$ where $e=vv' \in E$
  \item $w_{i,j,e}u_{i,j',v}$ for every $w_{i,j,w} \in W$ where $e=vv' \in E$
    where $j'=j+1$ except (1) $j'=j+2$ if $j+1=i<k$ and (2) the edge is
    skipped if $j+1=i=k$  
  \item $w_{i,k,e}u_{i+1,1,v}$ for every $w_{i,k,e} \in W$ and $v \in V_{i+1}$
    where $i<k$
  \item $w_{k-1,k,e}t$ for every $w_{k-1,k,e} \in W$
  \end{itemize}
  Let $D=(U \cup W \cup \{s,t\}, \cF)$ be a delta-matroid with feasible sets meeting the following conditions.
  \begin{itemize}
  \item A pairing constraint over $W$ on all pairs $(w_{i,j,w},w_{j,i,e})$
    representing the same edge $e$
  \item A partition constraint over $U$ on blocks $B_{i,j}=\{u_{i,j,v} \mid v \in V_i\}$,
    $i, j \in [k]$, $i \neq j$
  \item $s$ and $t$ are co-loops, i.e., present in every feasible set
  \end{itemize}
  This can be easily constructed, e.g., as a contraction of a matching
  delta-matroid (where the blocks $B_{i,j}$ are formed from stars
  where the midpoint vertex has been contracted).
  Let $H=(U \cup W \cup \{s,t\}, E_H)$ be the graph just constructed and let $k'=2k(k-1)+2$.
  We claim that $H$ contains a $k'$-path whose vertex set is feasible in $D$
  if and only if $G$ has a multicolored $k$-clique.
  
  On the one hand, let $C=\{v_1,\ldots,v_k\}$ be a
  multicolored $k$-clique in $G$, with $v_i \in V_i$ for every $i \in [k]$.
  Let $F=\{u_{i,j,v_i}, w_{i,j,v_iv_j} \mid i, j \in [k], i \neq j\} \cup \{s,t\}$.
  Then $F$ induces a path in $H$, $|F|=k'$ and $F$ is feasible in $D$. 
  
  On the other hand, let $F \in \cF(D)$ with $|F|=k'$, and let $P$ be a
  path in $H$ such that $F=V(P)$. Then $s, t \in F$, and the shortest
  $st$-path in $H$ contains $k'$ vertices. Hence $F$ contains precisely
  one vertex $u_{i,j,v_{ij}}$ and $w_{i,j,e_{ij}}$ for every $i, j \in [k]$, $i \neq j$.
  Furthermore, for every $j, j' \in [k]-i$ we have $v_{ij}=v_{ij'}$
  since the $w$-vertices $w_{i,j,e_{ij}}$ in $F$ remember the identity of $v_i$.
  Finally, since $F$ is feasible in $D$, we have $e_{ij}=e_{ji}$
  for all $i, j \in [k]$, $i \neq j$.
  Thus, the $w$-vertices contain one selection of an edge $e_{ij} \in V_i \times V_j$
  for every $i, j \in [k]$, $i \neq j$ where $e_{ij}$ and $e_{ij'}$ agree at $V_i$, 
  and $e_{ij}$ and $e_{ji}$ are the same edge. Thus the selection forms
  a $k$-clique in $G$.
\end{proof}



