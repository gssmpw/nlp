\section{Summary and Conclusions}
\label{sec:conclusions}

    The main thesis of this work is as follows: semantic communication is mostly a source coding problem, and less of a communication networking problem. To tackle errors caused over the communication link, the source coding should be error-tolerant, or, what we refer to as JSCC in this paper, which is deemed to outperform the separate design counterpart. The performance promised by \gls{jscc}  for semantic communication can be derived without the need to re-design the existing networks from scratch. With some essential modification to the existing systems---which are built with technical communication paradigm in mind---networks can turn the underlying channel(s) into a suitable end-to-end link with controlled uncertainty and allow the applications to run semantic communications, e.g.,  with deep-learning based \gls{jscc}.

    To this end, we  introduced  rateless JSCC, designed and optimized for a continuum of coding rates, which enables the application to efficiently  JSCC encode the source signal without having knowledge of the channel state. To complement this,  rate-adaptive and stable communication link operation was proposed which allows adapting the rate of the already encoded codeword to the channel capacity by  puncturing bits out of it. The network  maintains the bit flipping ratio across time frames. Together, the rateless JSCC in the application, and the rate-adaptive and stable link in the network, form a cooperative joint venture to efficiently adapt the JSCC code rate to the channel state, without exchange of \gls{csi}. 
    
    We demonstrated the feasibility of this joint venture using autoencoder based  JSCCs. Specifically, a new family of  autoencoder rateless JSCC codes were introduced and tested for reconstruction loss of image signals and demonstrated powerful performance and resilience to variation of channel quality.
   
    Next, we put things in perspective, and outlined the essential modifications needed for the next generation communication networks to enable such rateless JSCC and other semantic communication solutions. We studied the practical concerns regarding semantic communication  and provided a blueprint for networking system design, especially towards the anticipated 6G systems.  