

\section{Preliminaries and Problem Statement}
\label{sec:prelim}




We study the problem of semantic communication over a  channel with time-varying gain $\sqrt{\gamma_t}$ and \gls{awgn} $\mathbf{n}$. The channel is assumed to be block fading, that is, it stays constant over a  blocklength  $\bar{W}$ (in Hz-s), after which, it changes to a new independent channel, where $t$ is the block index\footnote{Block fading assumption is taken for simplicity. The definitions and the proposed solution in the paper apply  to frequency selective fading and doubly selective fading channel.}. 

A \emph{network operator} (or, network, for simplicity) operates over the channel to create a semantic communication link between  \emph{application source} and  \emph{application destination} nodes (see \figref{fig:system-model}). The network has knowledge of the \gls{csi} $\tilde{\gamma_t}$, which is an estimate of the channel gain ${\gamma_t}$ (i.e., it is either delayed or noisy, or both). Hereafter, the term \emph{block} refers to a coherence block of the underlying channel, while \emph{frame} refers to a segment of data communicated by the application over a block. 

The application uses a \gls{jscc} code   with   encoder $f$ and decoder $g$ (defined in the following section). 
% \begin{align}\label{eq:prb-def}
%     f: \mathbb{R}^N\rightarrow \{0,1\}^K, \quad  g: \{0,1\}^K \rightarrow \mathbb{R}^N 
% \end{align}
During  block $t$, the source encodes an input medium $\mathbf{s_t}$ (e.g., an  image signal) into a binary sequence \( \mathbf{u_t} = f(\mathbf{s_t}) \) of length $K$ and delivers it to the network. Network's job is to deliver a binary sequence $\hat{\mathbf{u}_t}$ to the application destination, where the sequence is decoded into \( \hat{\mathbf{s}_t} = g(\hat{\mathbf{u}_t}) \). 

The network operator has a total transmit power limit of $P_t \leq \Bar{P}$ and a total channel use  limit of $W_t \leq \bar{W}$, during  block $t$. Both $P_t$ and $W_t$ depend on the resource scheduling strategy of the network among multitude of users. Meanwhile, $\tilde{\gamma_t}$ depends on the randomness of the underlying channel. Together, $(\tilde{\gamma_t},P_t,W_t)$ forms the time-varying \emph{state} of the network and the end-to-end link provided for the application. For simplicity, we assume that the transmit power and channel use limits are constant over blocks. The goal of this semantic communication system is to minimize the expected distortion $\mathsf{d}({\mathbf{s}_t}, \hat{\mathbf{s}}_t)$ for each block $t$.


\begin{remark}\label{rem:exchange}
The network and the application do not exchange the uncoded source information ${\mathbf{u}}$, nor the state of the network. Therefore, the encoder $f$ (and  the decoder $g$) are unaware of the state of the channel and the network is unaware of the distribution and instance of the source signal. 
\end{remark}



\begin{remark}
    The goal laid out above is distinct from that of the classical \gls{jscc} excess distortion criterion, i.e., $\mathsf{Pr}[\mathsf{d}({\mathbf{s}}, \hat{\mathbf{s}}) > d] \leq \delta$. The intention, from the application perspective, is to minimize the expected distortion given the randomness in the \emph{state} of the network, $(\tilde{\gamma_t},P_t,W_t)$. It is conceivable to further assume that the application is not aware of the distribution of those random variables either. Additionally, average distortion over several blocks is not a suitable goal for still media forms such as images. 
\end{remark}




In the following, the index $t$ is omitted for notation simplicity, unless specifically needed.


%%%%%%%%%%%%%% TABLE I
\begin{table}[!ht]
\caption{List of Notations}
\label{Tab:notations}
\centering
\footnotesize 
\begin{tabularx}{.95\columnwidth}{|>{\raggedright\arraybackslash}p{0.15\columnwidth}||>{\raggedright\arraybackslash}X|}
\hline 
\rowcolor{lightgray} \textbf{Notation} & \textbf{Description}  \\
\hline \hline
$N$; $K$   &  Dimension of the uncoded source signal;  dimension of the coded  signal. \\ 
\hline
$L$ ; $\bar{L}$   &  Random variable denoting bit puncturing size;  the maximum value $L$ takes. \\ 
\hline
$\sigma_o$; $\sigma_e$  &  \gls{awgn} spectral density; variance of the channel estimation noise. \\ 
\hline
$\mathsf{F}$; $\mathsf{P}$; $\mathsf{Pr}$   &  Cumulative distribution;  probability distribution;  probability of random events. \\ 
\hline
$\mathsf{d}$; $\mathbf{d}$   &  Distortion function; vector of distortion values. \\ 
\hline
$\mathbf{s}$; $\hat{\mathbf{s}}$   &  Uncoded source signal; its reconstruction. \\ 
\hline
$\mathbf{u}$; $ \hat{\mathbf{u}}$   &  Coded source signal; its reconstruction. \\ 
\hline
$R$; $R_{lo}$; $R_{hi}$    &  Coding rate; lowest coding rate; highest coding rate. \\ 
\hline
$M$; $c$    &  Modulation order; number of possible modulation orders. \\ 
\hline
$q$; $q_o$    &  Bit flipping ratio; target bit flipping ratio. \\ 
\hline
$P$; $\bar{P}$    &  Transmit power; maximum transmit power. \\ 
\hline
$W$; $\bar{W}$    &  Transmission channel use; maximum transmission channel use. \\ 
\hline
$f$; $g$    &  Encoder function; decoder function. \\ 
\hline
$C_i$; $F$    &  Input codeword size for decoder $i$; number of encoder-decoder pairs for RLACS code. \\ 
\hline
$\mathsf{Q}$; $\mathsf{Q}^{-1}$ & Mapping between \gls{snr}, modulation order $M$ and \gls{ber};  inverse of $\mathsf{Q}()$ with respect to SNR. \\ 
\hline
$\sqrt{\gamma}$; $ \tilde{\gamma}$ &  Channel gain; noisy estimate of $\gamma$. \\ 
\hline
$\theta$; $\phi$ &  DNN parameter set for encoder function; the decoder parameters. \\ 
\hline
$\varnothing$; $\bm{\varnothing}$ &  Null value; a vector of null values. \\ 
\hline
\end{tabularx}
\end{table}
%%%%%%%%%%%%%
