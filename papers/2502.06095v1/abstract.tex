\begin{abstract}
    This paper introduces rateless joint source-channel coding (rateless JSCC). The code is \emph{rateless} in that it is designed and optimized for a continuum of coding rates such that it achieves a desired distortion for any rate in that continuum. We further introduce rate-adaptive and stable communication link operation to accommodate rateless JSCCs. The link operation resembles a “bit pipe” that is identified by its rate in bits per frame, and, by the rate of bits that are flipped in each frame. Thus, the link operation is \emph{rate-adaptive} such that it punctures the rateless JSCC codeword to adapt its length (and coding rate) to the underlying channel capacity, and is \emph{stable} in maintaining the bit flipping ratio across time frames. 
    
    Next, a new family of autoencoder rateless JSCC codes are introduced. The  code family is dubbed RLACS code (read as \emph{relax code}, standing for ratelss and lossy autoencoder channel and source code). The code is tested for reconstruction loss of image signals and demonstrates powerful performance that is resilient to variation of channel quality. RLACS code is readily applicable to the case of \emph{semantic} distortion suited to variety of semantic and effectiveness communications use cases. 
    
    In the second part of the paper, we dive into the practical concerns around semantic communication and provide a blueprint for semantic networking system design relying on updating the existing network systems with some essential modifications. We further outline a comprehensive list of open research problems and development challenges towards a practical 6G communications system design that enables semantic networking.
\end{abstract}

% \begin{IEEEkeywords}
% Semantic communications, effectiveness communications, joint source-channel coding, autoencoders
% \end{IEEEkeywords}