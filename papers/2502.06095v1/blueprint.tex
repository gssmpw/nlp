
%%%%%%%%%%%%%%%%%%%%%%%%%%%%%%%%%%%%%%%%%%%%%%%%%%%%%%
%%%%%%%%%%%%%%%%%%%%%%%%%%%%%%%%%%%%%%%%%%%%%%%%%%%%%%
%%%%%%%%%%%%%%%%%%%%%%%%%%%%%%%%%%%%%%%%%%%%%%%%%%%%%%

\section{Blueprint of a 6G Semantic Communications System }
\label{sec:blueprint}


This section investigates the semantic communication problem form a system design perspective through the lens of practicality. Next generations of communication systems, especially the \gls{6g} networks, are anticipated to see more applications of semantic  type \cite{saad2024foundations}, including spatial video and mixed/augmented reality. In the following, first we provide a sketch of the system design differences between  technical communication and  semantic and effectiveness communication. Then, we project that sketch over the existing communication network systems, the \gls{5g} in particular, and discuss the changes that need to happen. Finally, we provide a list of open research problems and development challenges that require further investigations. To that end, the list will include items related directly to the blueprint  provided in this paper, and  items discussed by other works in the literature.

\subsection{Sketch Summary}
\label{sec:sketch}
The most critically new aspect of the semantic communication system paradigm is the assumption that the semantic applications are \emph{error-tolerant}---that is, the application can take in bit errors and bit erasures in the communicated packet and turn that into tolerable \emph{loss} or \emph{distortion} in the semantic performance indicator. In contrast, the existing networking solutions are mainly designed to handle error-free communication. As a result,  current applications that most resemble  Weaver's definition of semantic and effectiveness communication in \cite{shannon1998mathematical}---e.g., multimedia communication in form of image and video---operate ``over-the-top'' on the existing networking solutions with minimal or zero cooperative interactions with the network. Consequently, from the application's perspective, the end-to-end link is error-free, but may constitute packet  erasures that are visible to the application. A demonstrative example of this is video communication over \gls{5g} systems, where video codecs deploy \emph{error concealment} techniques to deal with partial or complete erasure of video frames \cite{aign1995temporal}.  %That in practice is proven to be a generally fair approach for  multimedia communication.
What the future network systems can add to this, especially in the context of \gls{6g}, is mainly to enable bit-error-tolerance in the end-to-end system design, where the network allows passage of erroneous bits to the application and the application learns how to deal with those errors. The catch is that the position  of those errors  are not visible to the application (as opposed to the position of packet erasures in existing approach described above), but the rate of error can be realistically anticipated and stabilized. As discussed in the previous sections of this paper, that makes the end-to-end link into a \emph{stabilized bit pipe}, or a stabilized vector binary channel, from the application's perspective. 




%---------------------------
\begin{figure*}[h]
\begin{center}
% \psfrag{channel SNR [dB]}[c][c][1]{\scalebox{.7}{channel SNR [dB]}}
% \psfrag{..}[c][c][1]{\scalebox{.7}{}}
% % \psfrag{-10}[c][c][1]{\scalebox{1}{-10}}
\includegraphics[width=.85\textwidth,keepaspectratio]{images/e2e.eps}
\caption{Illustration of the typical constituents of an end-to-end communication link, with an example terrestrial link at the top and an example non-terrestrial link at the bottom. The radio, X-haul, satellite, and transport protocols must be able to interoperate with each other and across different vendors.}
\label{fig:e2e}
\end{center}
\end{figure*}
%----------------

\subsection{Blueprint of Semantic Communications System design}

Existing literature on semantic communication  relies on  \gls{jscc} solutions that fuse the application and the network treatments of a semantic packet. This includes solutions for dismantling   the   binary interface  that separates application's source coding and network's channel coding \cite{bourtsoulatze19_deep_joint_sourc_chann_codin}, or, solutions that  work around that interface \cite{tung2024multilevelreliabilityinterfacesemantic}. For  practical reasons, straightforward \gls{jscc} solutions  won't be feasible, most notably,  due to interoperability requirement among network and application vendors, and interoperability among different functions and hops of an end-to-end network link. For example, accepting a quantized (binary) interface between network and application %(at least in one end or sometimes in both ends)
appears necessary\footnote{In very special applications, usually dealing with local traffic,  such  quantization requirement can be relaxed, enabling channel-adaptive \gls{jscc} solutions, such as in\cite{bourtsoulatze19_deep_joint_sourc_chann_codin}.}.

In the most common communications settings,  the end-to-end link includes several constituent hops with variety of attributes: fiber and cable links are typically more reliable with static behavior while wireless links can experience highly dynamic channel state---see the two examples in \figref{fig:e2e}. The end-to-end bit pipe in application's perspective,  travels through several connected pipes, each with a different pipe size (in bit rate) and expected  damage to the bits (in bit error rate). The bottleneck is typically in the wireless hops, nevertheless all the  constituents, and the protocols operating them, must be in full coordination and cooperation. 


To this end, a paradigm shift is required in system design to host the semantic applications more efficiently over telecommunication networks. A list is provided in \tabref{Tab:paradigmchange} that summarizes  the pillars of such   changes in system design. The effect of this paradigm shift is more concretely summarized in the following, by describing the essential evolutions that a  communication system could highly benefit from to realize semantic communications.





\subsubsection{Radio Access Enhancements}

Current radio access technologies are designed with error-free communication mindset. Packets are either correctly decoded and delivered by the network, or dropped if the combination of \gls{fec} and automatic repeat request (ARQ) based techniques fail to guarantee the integrity of the packet---even a single bit of error in a packet will fail the whole. In contrast, lossy \gls{jscc} systems can  benefit from erroneous packets, especially if the rate of bit errors is anticipated. Enabling passage of damaged packets from radio access to the higher layers and through to the application is  therefore an essential element to enable lossy \gls{jscc}. 
\paragraph*{Additional considerations} Radio access protocols can benefit from new design paradigms in their use of \gls{fec}, \gls{arq} and radio resource management when moving from \gls{bler} based performance indicators (packet reliability over time) to \gls{ber} based quality of service (bit reliability within each packet). The proposed rate-adaptive and stable link design in \secref{sec:ratelessradiolink} is realized with uncoded modulated transmission, but in presence of time and frequency diversity such radio link can benefit from \gls{fec} and \gls{harq} based physical layer solutions. Stabilizing the radio link \gls{ber} with $1-\epsilon$ certainty can then be further optimized with the degrees of flexibility offered by channel coding and retransmission techniques. For example, \gls{fec} can be used to reduce the bit flipping ratio from the uncoded and modulated transmission, while \gls{harq} makes this process more efficient by leveraging time-diversity. Overhead  \gls{crc} bits which are normally used to check the decoding success and integrity of a packet, can now be used to estimate the bit flipping ratio within a packet. Hybrid \gls{arq} techniques which are typically used to increase chance of decoding success and improve throughput, can now be used to curb $\epsilon$.


%%%%%%%%%%%%%% TABLE II

\begin{table*}
\caption{Pillars of system design paradigm change from conventional communications to semantic/effectiveness  communications systems.}
\label{Tab:paradigmchange}
\footnotesize
\begin{tabularx}{\textwidth}{|>{\raggedright\arraybackslash}p{.2\textwidth}||>{\raggedright\arraybackslash}X|>{\raggedright\arraybackslash}X|}
\hline 
\cellcolor{lightgray}\textbf{System attribute} & \cellcolor{lightgray}\textbf{Conventional communication} & \cellcolor{lightgray}\textbf{Semantic \& Effective communication}  \\
\hline \hline
%%%%%%%%%%%%%%%%%%%%%
\cellcolor{verylightblue}\textbf{Transmission structure}     & Packet-ized transmission: \newline   Application exchanges  information bits in \emph{packets} with the network.   &  Streamlined transmission (bit pipe): \newline Communication is organized around \emph{frames}, which serve as temporal or spatial segments of application's  data.    
 \\ \hline
%%%%%%%%%%%%%%%%%%%%%
\cellcolor{verylightblue}\textbf{End-to-end link model}    & 
\xmark \quad Bit errors \newline
\xmark \quad Bit erasures \newline
\cmark \quad Packet erasures \newline
A packet is either delivered correctly or not delivered at all. The effect is that erasures could happen to the packets, but delivered packets are deemed \emph{error-free}.  &  
\cmark \quad Bit errors \newline
\cmark \quad Bit erasures \newline
\xmark \quad Packet erasures \newline
Erasure at the frame level is not acceptable, but damaged and missing bits are tolerated. Frames don’t need to be treated as indivisible units---parts of the frame may arrive intact, while others may be damaged or missing.\\  \hline 
%%%%%%%%%%%%%%%%%%%%%
\cellcolor{verylightblue}\textbf{Performance indication principle }    & Rate of successful delivery of packets over time, i.e., \gls{per} or \gls{bler}. &  Size of the \emph{delivered} frame (in bits, and as ratio of the original frame size) and the rate of damaged bits, i.e., \gls{ber} within each frame.    \\ \hline
%%%%%%%%%%%%%%%%%%%%%
\cellcolor{verylightblue}\textbf{Data integrity}      & Data integrity is enforced at the packet level.      & Data integrity is not strictly enforced at the frame level.    \\ \hline
%%%%%%%%%%%%%%%%%%%%%
\end{tabularx}
\end{table*}
%%%%%%%%%%%%%

\subsubsection{Enriched Interfacing and \gls{qos} monitoring} \label{sec:enrichedinterface}
\gls{qos} metrics in telecommunication networks are used to ensure compliance with the expectations of service defined in \gls{sla} that is between the application and the network. Packet reliability, latency and throughput (rate) are among the common \gls{qos} metrics defined for radio access systems. For semantic and effective communications, it is evident that new \gls{qos} metrics are required. Notably, the bit-pipe analogy of network's role suggests use of the following as new performance indicator metrics in semantic communication systems:
\begin{itemize}
    \item Bit flipping ratio $q$, measures the ratio of erroneous bits within a packet delivered to the application destination.
    \item Link stability $1 - \epsilon$, is a  measure of the stability of bit flipping ratio over time which is used to satisfy $\mathsf{Pr}[q > q_o] \leq \epsilon$.
    \item Maximum allowed puncturing size, $\bar{L}$, to limit the number of bits to be punctured out of a packet (bit erasures) for a rate-adaptive and stable link operation.
    \item Limit on the average puncturing length over time, $\mathsf{L}$, to incentivize the networks   to minimize puncturing attempts, i.e., $\mathbb{E} \left(  L\right) \leq \mathsf{L}$.
\end{itemize}
\paragraph*{Additional considerations} In the context of packetized communication, slight modification to the existing wireless communication systems together with enriched interfacing between the application and the network can be beneficial to the semantic applications too. Specifically, enabling multiple levels of \gls{qos} within the same packet   (as opposed to the single \gls{qos} level treatment of packets in current systems) can be beneficial. The application can flexibly choose to have multiple levels of \gls{qos} within the same packet, each level assigned to a different segment of  the packet. The network treats each segment according to the requested \gls{qos} level by the application, allowing the application to decide on the importance and significance of different segments of the packet in a dynamic way. This requires an enriched interfacing between the application and the network, enabling \emph{application-network symbiosis}. This allows time-varying segmenting and \gls{qos} level assignment to the segments by the application to be followed swiftly by the network.  

\subsubsection{Semantic Networking} 
Bit-pipe analogy for  the end-to-end  communication link requires all the layers of the stack to allow the passage of otherwise erroneous packets. While this starts in radio access protocols, particularly in physical layer, it extends to higher layers of the stack, including transport layer protocols. Most notably,  data integrity in \gls{tcp} is used to  guarantee that data is delivered accurately and without corruption, which hinders passage of erroneous packets through the network. Adjustments to those protocols are needed to securely relax those constraint, similar to \gls{udplite}, which allows partially damaged data to be delivered to the application \cite{larzon1999lighter}.
In case of an end-to-end link which has to go through multiple  hops  (see examples in \figref{fig:e2e}), it is important to allow seemless passage of the bits through every single hop without blocking the passage of damaged packets. The bottleneck hop, i.e., the hop that has the lowest capacity and/or causes the highest bit flipping ratio, determines the end-to-end experience. A rateless \gls{jscc} code such as the one proposed in this paper can smoothly make the best use  of such setting, by allowing  necessary puncturing of the bits in each hop. This significantly simplifies cross-layer protocol design approaches.
In case of satellite communications (the examples at the bottom of \figref{fig:e2e}), the large propagation delay between the ground and orbit satellites is not friendly to retransmission techniques such as \gls{arq}. Therefore, for semantic applications with limited latency budget, a one shot best effort transmission over the satellite link is most favorable.  In such case, rateless \gls{jscc} poses an additional benefit, that is, allowing to maximize the utilization of the one shot transmission by enabling the network to puncture the necessary number of bits out of the packet according to the estimated channel, and moving on to the next packet without the need for awaiting feedbacking and retransmission attempts.







\subsection{Open Research Problems}
\label{sec:openresearch}



%%%%%%%%%%%%%
% With the assumption of a binary interface between the network and the application, the achievable \gls{jscc} performance over the end-to-end bit pipe is  still an open problem.  

Let us now discuss the open research problems and development challenges that need to be addressed 

\subsubsection{Resource Allocation and Scheduling}
Coexistence of  semantic applications with applications that use conventional communication links, over the same network platform, is inevitable. Given the  disparity in treatment of information bits between the two links (see \tabref{Tab:paradigmchange})  and the differences in \gls{qos} metrics, scheduling between the two types in multi-user settings becomes an open problem which requires new optimization and design frameworks.
% In a more pract to find out how much overall gain can be anticipated, over existing networking systems, if advanced JSCC solutions like the one presented in this paper, or the likes of TY's solution, are adopted. 
\subsubsection{Proper Benchmarking}
The networking systems for multimedia communication are extremely intricate. They allow for handling a large multimedia packet in segments, where segments are from hierarchical source compression with unequal importance. The network treats  different segments with different \gls{qos} levels relative to their importance---but the network may be unable to trace back those segments to the same source and may be unaware of their potential overlap. As a result, the end to end link from the point of view of the application appears as a segment-wise erasure channel--that is, segments of a larger packet are either correctly delivered or lost in communication, while the erasure rate across segments are non-uniform. This could potentially create large overhead and repetition orders. Nevertheless, existing approaches should be considered as  the baseline for comparison with any new solution. Tractable mathematical modeling of those baseline methods can become very handy and are open for research.  %The segmentization and handling of the \gls{qos} for each packet segment is done ``away from'' the physical layer, for--that is, they happen at upper layers, where real-time CSI is not available. 
% Such segment-wise 
% The future communication systems may thus benefit from making two overall changes to the existing systems:
% --- Enabling that segment handling by a joint venture between application and lower layer of the network: that means rich interface between application and network, and handling of 
% --- Moving from end-to-end erasure channel to and end-to-end binary channel that has two parameters and then optimizing the application jscc for that link: this means that the end-to-end link from the point of view of the application appears as a bit pipe, where the application is error-tolerant, therefore the network allows the packets to go through even if considered erroneous in CRC check. In fact, it may make sesne to reduce or completely remove CRC bits in PHY layer.

% \subsubsection{}
% Moving from end-to-end erasure channel to and end-to-end binary channel that has two parameters and then optimizing the application jscc for that link: this means that the end-to-end link from the point of view of the application appears as a bit pipe, where the application is error-tolerant, therefore the network allows the packets to go through even if considered erroneous in CRC check. In fact, it may make sesne to reduce or completely remove CRC bits in PHY layer

\subsubsection{Data Integrity for Streamlined Bit Pipes}
As outlined in \tabref{Tab:paradigmchange}, moving from  packet-ized transmission to streamlined ``bit pipe'' transmission appears to bring non-negligible gain for semantic communications. Providing efficient data integrity solutions in such cases requires further investigation, especially in the context of widely used internet protocols.

\subsubsection{Application and Network Symbiosis}
Admittedly, one of the main benefits of using \glspl{dnn} in source compression and \gls{jscc} solutions is their ability  to learn the distribution of the source and adapt the compression accordingly towards a better rate-distortion tradeoff. In turn, this requires well-tailored solutions to specific data formats and distributions. Such techniques may allow richer contextual information to be stored on devices, thus further reducing the compression size---this can be based on generic understanding of modes and contexts, e.g., using generative models or, can be distribution and data-set specific. This is an active research area which tightly relates to semantic communication. An interesting direction is to investigate solutions that account for the practical constraints of the networking systems and seek techniques that fit well with the telecommunication networks. Interfacing between the network and application can be more helpful when the exchange of \emph{side-information} works both ways and when it coveys enriched information about the state of the two sides, as already discussed in works like \cite{10201232}. Such interfacing could benefit semantic communication systems too (see \secref{sec:enrichedinterface}) which requires more research and development effort.   

\subsubsection{Physical Layer Research}

As was discussed in \secref{sec:discussionofresults}, the current coding, modulation schemes, and retransmission techniques utilized in lower and upper parts of the 5G new radio physical layer, are mainly tailored  towards error-free packet-ized communications (see \tabref{Tab:paradigmchange}). For the semantic applications to benefit from a more flexible radio link with streamlined transmission, modulation shaping using machine learning, coded modulation techniques, and retransmission mechanisms that target \gls{ber}  stabilization  can be beneficial. \Gls{ai}-native air interface and \gls{dnn} based physical layer solutions such as \cite{honkala2021deeprx} could provide a fast track towards this end, but further research and development is needed  to generalize those to different applications and network state distributions. 

\subsubsection{Rate-Distortion-Memory Tradeoffs} 
In the context of deep learning based \gls{jscc}, the matter of caching of the trained neural networks is an important practical consideration. The trained neural networks must be stored (and in many cases, communicated before storage) to be useful for the application, while the cost of such caching is non-negligible. A relatively small neural network with 10 million trainable parameters stored with 32-bit floating point representations requires approximately 38 megabytes of storage space. As the neural network size grows and as the dataset becomes more specific, the relative cost of such storage increases. Therefore, one must study the matter of generalizability of the trained DNNs and the impact of  it on  JSCC performance. The alternative direction is to build very small neural networks that are tailored to  each specific dataset. Neural network compression methods \cite{neill2020overview} could also be exercised, but the impact of such  compression on JSCC performance remains an open problem. In fact, one may see the trained neural networks as the shared \emph{context} between the source and destination points for the semantic communication problem and study the tradeoff between context memory size, semantic distortion and communication rate. 


\subsubsection{Other Open Problems from the Literature}
Several works in the literature have nicely reviewed the state-of-the-art in semantic communications and outlined the open research problems.  \cite{gunduz2022beyond} provides a comprehensive overview of semantic and task-oriented communications, focusing on integrating semantics and goals into communication system design. The authors discuss the limitations of traditional communication systems that focus solely on reliable bit transmission, and highlight the potential of deep learning-based \gls{jscc} techniques for efficient video delivery. Interoperability and hardware design for those techniques require further investigation. 
Specific data types require specific treatement when it comes to semantic compression and semantic JSCC. An intereting example is studied in  \cite{liang2024semantic} for synchronized sound communication, which demonstrates an interesting inter-working between synchronization and semantic communication. 

Use of information theoretic tools for formalization and quantification of semantics and effectiveness of information are required, as was also pointed out by \cite{qin2021semantic}.  %Quantification of semantic communications is still an open problem.
It is not concretely accepted whether measures such as semantic entropy and semantic level rate-distortion theory are the correct metrics for the job. \cite{luo2022semantic} also points out that there is need for more theoretical work on the topic, especially in conjunction with deep learning based codes. Additionally, the capacity of semantic networking is  an open problem \cite{shi2021semantic} and a comprehensive mathematical framework to evaluate the performance limits of a semantic communications system requires further research effort.

Network-level \gls{qoe} quantification for semantic applications contrasts with conventional communication, as pointed out in \cite{shi2021semantic}, and needs a rethinking. In \secref{sec:enrichedinterface} we discussed this from \gls{qos} perspective and proposed a set of new \gls{qos} metrics suitable for semantic communication. 
In an interesting take, \cite{you2024next} addresses the topic from communication hardware  perspective and discusses the next generation advanced transceivers that are friendlier with semantic communication, and raises the relevant open research problems in that context, with interesting connection to deep learning based solutions.

Data integrity (see \tabref{Tab:paradigmchange}), data security, and privacy are other aspects of semantic communications that needs rethinking. \cite{getu2024survey} argues that \gls{6g} will be highly tied with semantic communication and sees security and data privacy as one of the main challenges that need to be addressed by future research.

