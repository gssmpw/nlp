%%%%%%%%%%%%%%%%%%%% author.tex %%%%%%%%%%%%%%%%%%%%%%%%%%%%%%%%%%%
%
% sample root file for your "contribution" to a contributed volume
%
% Use this file as a template for your own input.
%
%%%%%%%%%%%%%%%% Springer %%%%%%%%%%%%%%%%%%%%%%%%%%%%%%%%%%


% RECOMMENDED %%%%%%%%%%%%%%%%%%%%%%%%%%%%%%%%%%%%%%%%%%%%%%%%%%%
\documentclass[graybox]{svmult}

% choose options for [] as required from the list
% in the Reference Guide

\usepackage{type1cm}   
\usepackage{array}
\usepackage{orcidlink}

% activate if the above 3 fonts are
                            % not available on your system
%
\usepackage{makeidx}         % allows index generation
\usepackage{graphicx}        % standard LaTeX graphics tool
                             % when including figure files
\usepackage{multicol}        % used for the two-column index
\usepackage[bottom]{footmisc}% places footnotes at page bottom


\usepackage{newtxtext}       % 
\usepackage{newtxmath}       % selects Times Roman as basic font

% see the list of further useful packages
% in the Reference Guide

\makeindex             % used for the subject index
                       % please use the style svind.ist with
                       % your makeindex program

%%%%%%%%%%%%%%%%%%%%%%%%%%%%%%%%%%%%%%%%%%%%%%%%%%%%%%%%%%%%%%%%%%%%%%%%%%%%%%%%%%%%%%%%%

\begin{document}

\title*{Evaluating Language Models on Grooming Risk Estimation Using Fuzzy Theory}
% Use \titlerunning{Short Title} for an abbreviated version of
% your contribution title if the original one is too long
\author{Geetanjali Bihani\orcidlink{0000-0001-8352-7948}, Tatiana Ringenberg\orcidlink{0000-0002-9851-6341}, Julia Rayz\orcidlink{0000-0003-3786-2416}}
% Use \authorrunning{Short Title} for an abbreviated version of
% your contribution title if the original one is too long
\institute{Geetanjali Bihani \at Purdue University, USA, \email{gbihani@purdue.edu}, \and Tatiana Ringenberg \at  Purdue University, USA,  \email{tringenb@purdue.edu}
\and Julia Rayz \at  Purdue University, USA,  \email{jtaylor1@purdue.edu}}
%
% Use the package "url.sty" to avoid
% problems with special characters
% used in your e-mail or web address
%
\maketitle

\abstract*{}

\abstract{Encoding implicit language presents a challenge for language models, especially in high-risk domains where maintaining high precision is important. Automated detection of online child grooming is one such critical domain, where predators manipulate victims using a combination of explicit and implicit language to convey harmful intentions. While recent studies have shown the potential of Transformer language models like SBERT for preemptive grooming detection, they primarily depend on surface-level features and approximate real victim grooming processes using vigilante and law enforcement conversations. The question of whether these features and approximations are reasonable has not been addressed thus far. In this paper, we address this gap and study whether SBERT can effectively discern varying degrees of grooming risk inherent in conversations, and evaluate its results across different participant groups. Our analysis reveals that while fine-tuning aids language models in learning to assign grooming scores, they show high variance in predictions, especially for contexts containing higher degrees of grooming risk. These errors appear in cases that 1) utilize indirect speech pathways to manipulate victims and 2) lack sexually explicit content. This finding underscores the necessity for robust modeling of indirect speech acts by language models, particularly those employed by predators.} 

\section{Introduction}
\label{sec:1}
Online child sexual grooming refers to the insidious process through which an adult establishes relationships with potential under-aged victims on digital platforms,  with the goal of eventual sexual gratification without detection \cite{wintersJeglic}. Adults grooming children use a wide range of tactics and persuasion strategies depending on factors such as their potential goals \cite{ChiuSeig, kloessQualitative, DeHart} and level of directness \cite{kloess2017, kloessQualitative}. Prior research has identified various covert and overt persuasion and manipulation strategies of groomers including gift-giving, flattery, pressure, deception, and affection, to gain the trust and confidence of their targets \cite{joleby2021offender, gam}.

Research on automated grooming detection has predominantly framed the task as a binary classification problem to categorize entire chat instances as either grooming or non-grooming \cite{preub2021, gunawan, isaza}. However, this method falls short of facilitating preventive measures. Furthermore, recent studies focusing on modeling preventive measures for grooming chats have primarily relied on training with internet sting data rather than authentic victim conversations \cite{vogt2021}, potentially leading to inaccurate generalizations. Despite findings highlighting differences in chat language across various participant groups (real victims, law enforcement officials, decoys) \cite{ring21_2}, current models neither measure nor account for such disparities in data distributions.

% \subsection{Uncertainty in Natural Language}
% Natural language utterances are typically imprecise, containing varying degrees of vagueness. 

% Despite notable progress in this field, recent findings highlight the current inadequacies of pre-trained language models in terms of reliability in decision-making \cite{bihanirayz2024}. 


% \subsection{Online Grooming}


\section{Methodology}
\label{sec:2}
In this section, we outline our methodology for modeling grooming risk using SBERT \cite{sbert2019}. We define the task of \textit{Grooming risk-scoring} and its objectives. We then define an evaluation protocol that uses fuzzy rules to compare different degrees of grooming with transformer language model predictions. Fuzzy annotations of grooming strategies have been taken from the dataset described in \cite{ring21}.

\subsection{Degrees of Grooming Risk}

Grooming risk, rather than fitting neatly into discrete categories, exhibits gradations across multiple degrees of severity. Imagine a scenario where an array of grooming strategies is deployed within a given chat context. These strategies can encompass a spectrum of subtle to overt tactics. Each strategy contributes to the overall risk level, but the degree of influence it holds is imprecise and contextually contingent. Unlike binary categorizations, grooming risk manifests along a continuum, where various factors interplay to determine the level of vulnerability. 

To assess whether a language model can learn human perceptions of grooming risk from natural language contexts, we compare language model predictions with human perceptions of grooming risk. Specifically, we map the extent to which humans perceive grooming strategies to be present in a given chat context to the grooming risk variable. We leverage human annotations of grooming strategies as outlined in \cite{ring21}, as risk scores. For a given chat context $c$, we define the total number of observed grooming strategies $n_{s}(c)$ as the sum of individual strategy scores $s_i$, where each $s_i \in \{0, 0.5, 1\}$ represents the absence (0), partial presence (0.5), or full presence (1) of the $i^{th}$ strategy. This is shown in Equation~\ref{eq:1}. 

\begin{equation}
\label{eq:1}
n_{s}(c)=\sum_{i=1}^{N} s_{i}(c)
% o = \sum_{strategies}{\mu_{strategy}(c)} 
\end{equation}





% \huge
% \begin{equation}
% \label{eq:2}
% \textcolor{red}{
% \mu_{risk}(\text{c}) = \varphi\left(x-c\right) \text{, where }\varphi(z)=\frac{e^{-z^{2} / 2}}{\sqrt{2 \pi}}}
% \end{equation}







% \begin{equation}
% \label{eq:3}
% \mu(moderate) = \mu_{risk}(0.2)
% \end{equation}


% \begin{equation}
% \label{eq:4}
% \mu(significant) = \mu_{risk}(1)
% \end{equation}

% \begin{equation}
% \label{eq:5}
% \mu(severe) = \mu_{risk}(2)
% \end{equation}


\subsection{Task Definition}
\label{subsec:1}
We approach grooming risk-scoring as a regression task, with chat context language as the independent variable and aggregated grooming risk as the dependent variable.

\textbf{Definition 1 - Chat Context $(c)$}: A sequence of current and last $n-1$ chat messages. We fix $n=3$, for our analysis.

\textbf{Definition 2 - Grooming Risk $(r_{groom})$}: Severity of grooming behavior determined by the total number of grooming strategies present within a given chat context \textit{(c)}, as shown in Equation \ref{eq:1.1}.  
\begin{equation}
\label{eq:1.1}
r_{groom} = {n_{s}(c)} %\in S
\end{equation}

We limit our analyses to twelve grooming strategies, including coercion, bragging, teaching, requests for images, negative comments about physique, negative comments about family, personal compliments, reverse power, asking about sexual history, checking willingness, roleplaying, and secrecy, as described in \cite{ring21}.
Each strategy is assigned a membership value of 0 (none), 0.5 (partial), or 1 (full). For a detailed description of these grooming strategies and the annotation process, refer to \cite{ring21}. Thus, chat contexts with fewer observed grooming strategies are considered lower risk, while those with more strategies present indicate higher grooming risk. 

% \begin{equation}
% \label{eq:1}
% r_{groom} = {\mu_{risk}(n_{s})} %\in S
% \end{equation}




\begin{equation}
\label{eq:2}
\mu_{risk}(c) = \varphi\left(n_{s}(c) - m\right) \text{, where }\varphi(z)=\frac{e^{-z^{2} / 2}}{\sqrt{2 \pi}}
\end{equation}
\begin{equation}
\label{eq:2.1}
\mu_{mod}(c) = \mu^{0.2}_{risk}(c) \text{ for $m=0.2$}
\end{equation}
\begin{equation}
\label{eq:2.2}
\mu_{sig}(c) = \mu_{risk}(c) \text{ for $m=1$}
\end{equation}
\begin{equation}
\label{eq:2.3}
\mu_{sev}(c) = \mu^2_{risk}(c) \text{ for $m=2$}
\end{equation}

To evaluate the performance of the regression model on grooming risk prediction, we categorize these risk scores ($r_{groom}$) into three risk categories using a Gaussian membership function. This function maps continuous risk scores into degrees of membership in \textit{moderate} (low risk), \textit{significant} (medium risk), and \textit{severe} (high risk) categories. This allows each score to have fuzzy (partial) membership in multiple categories. We choose a Gaussian distribution function because it enables smooth transitions between risk levels. The discretization of the risk scores is done using Equation~\ref{eq:2}. We define different mean values ($m$) for each category's distribution function, i.e. $m=0.2$ for \textit{moderate}, $m=1$ for \textit{significant}, and $m=2$ for  \textit{severe} risk levels, as shown in Equations~\ref{eq:2.1}, \ref{eq:2.2} and \ref{eq:2.3}. These functions are visualized in Figure~\ref{fig:1}, illustrating the varying degrees of grooming risk and their corresponding membership functions. This approach aligns with the intuitive understanding that higher scores indicate greater risk while providing a more nuanced evaluation across different degrees of risk. To determine the final risk category, we apply defuzzification using a $\alpha$ cut of $0.5$, selecting the highest risk level where membership exceeds this threshold. Thus, if a chat context has memberships surpassing the $\alpha$-cut in both \textit{moderate} and \textit{severe} categories, it is classified as a \textit{severe} risk chat context.



% For figures use
%
\begin{figure}[htbp]
\centering
\includegraphics[scale=0.5]{figures/mems.png}
\caption{Different degrees of grooming risk and their membership functions: Moderate [$\mu^{0.2}_{risk}(c)$], Significant [$\mu_{risk}(c)$] and Severe [$\mu^2_{risk}(c)$] risk}
\label{fig:1}       % Give a unique label
\end{figure}



\subsection{Model Studied}
\label{subsec:2}
We conducted our analysis on Sentence-BERT (SBERT) \cite{sbert2019} which was specifically designed to encode sentence embeddings. SBERT utilizes siamese networks to generate semantically meaningful representations of sentences. These representations are optimized to capture semantic similarity between sentences in a vector space, making them suitable for various downstream tasks such as semantic search and clustering.

\subsection{Fine-tuning Details}
\label{subsec:3}
We adapted SBERT to predict grooming risk scores using a linear layer on top of the final layer to output regression estimates. Fine-tuning involves adapting pre-trained models to specific downstream tasks by utilizing task-specific data. During the fine-tuning process, additional layers are added on top of the pre-trained models, and the models are trained on labeled data specific to the task at hand. Fine-tuning allows the model to acquire task-specific features and enhances the model's performance on the given task. 

For a given chat context $c$ the regression model is optimized for estimating the severity of grooming risk $r$ by minimizing the Mean Squared Error (MSE) between the predicted grooming risk $r_{pred}$ and actual grooming risk $r_{groom}$. 
The hyperparameters we used for finetuning our BERT models are listed in Table~\ref{tab:1}. 

\begin{table}
\centering
\caption{Fine-tuning Hyperparameters}
\label{tab:1}
%
% Follow this input for your own table layout
%
\begin{tabular}{|c|c|}
\hline
\textbf{Hyperparameter} & \textbf{Value} \\
\hline
Optimizer & Adam \cite{kingma2014} \\
Learning Rate & $2.10^{-5}$ \\
Epochs & 5 \\
Batch size & 4 \\
\hline
\end{tabular}
% $^a$ Table foot note (with superscript)
\end{table}


\section{Results and Analysis}
We examined how well fine-tuned SBERT predictions align with human perceptions of grooming risk in natural language contexts. Specifically, we fine-tuned and evaluated language models separately on grooming conversations involving predators interacting with distinct groups: law enforcement officers (LEO), victims, and decoys. This analysis is prompted by prior research highlighting variations in language usage across different groups in grooming behaviors \cite{ring21}. Additionally, automated models of grooming classification and detection completely overlook these differences. This underscores the importance of demonstrating that language models trained on data distributions diverging from patterns observed in real-victim grooming conversations may lack reliability if deployed without due consideration. 

\subsection{Prediction Accuracy across Degrees of Grooming Risk}
\begin{table}[!t]
\caption{MSE reported on different groups (High values are in bold font)}
\centering
\label{tab:3}       % Give a unique label
%
% Follow this input for your own table layout
%
\begin{tabular}
{p{0.2\textwidth}p{0.2\textwidth}p{0.2\textwidth}p{0.2\textwidth}}
\hline\noalign{\smallskip}
Grooming Risk  & LEO  & Victim  & Decoy  \\
\noalign{\smallskip}\hline\noalign{\smallskip}
Moderate &  1.881 &  1.282 & 1.188 \\
Significant &  1.211 & 1.360 & 1.748 \\
Severe & \textbf{6.589} & \textbf{9.096} & \textbf{7.762} \\
\noalign{\smallskip}\hline\noalign{\smallskip}
Overall & 3.552 & 3.425 & 3.217 \\
\noalign{\smallskip}\hline\noalign{\smallskip}
\end{tabular}
% $^a$ Table foot note (with superscript)
\end{table}

We report Mean Squared Error on predictions across different degrees of grooming risks in Table~\ref{tab:3}. While the overall MSE remains consistent across all three groups, discernible differences in error rates are evident. Our findings show notably higher error rates for message contexts associated with \textit{severe} grooming risk, showing that models perform worse in high-risk scenarios within the context of grooming risk estimation. Moreover, a model finetuned on real-victim chats performs worse than those trained on LEO and Decoy chats. This finding highlights the differences in chat languages across groups and questions the inherent assumption made by the NLP community regarding training and finetuning automated grooming risk estimation frameworks using unrepresentative Decoy and LEO chats.

To understand the source of higher errors in \textit{severe} grooming risk contexts, we did a qualitative examination of these erroneous predictions. Our findings show that message contexts with \textit{severe} grooming risks do not necessarily utilize explicitly sexual and predatory language. Prior works on cyberforensic analyses of grooming conversations mention that predators employ direct and indirect communication pathways in grooming conversations \cite{kloess2017}. We find that in many such cases, seemingly innocuous but predatory texts remain undetected by language models, causing the error rates to increase. This limitation of fine-tuned language models arises from their tendency to learn surface-level associations, rather than learning associations within the underlying context. Prior work on measuring uncertainty in grooming language used in predatory settings across different groups \cite{ring21_2}, provides evidence of the variability in the uncertainty language used within different grooming stages. 

\subsection{Risk Distribution across Different Groups}

% For figures use
%
\begin{figure}[htbp]
\centering
\includegraphics[scale=.35]{figures/preds_vs_gs.png}
\caption{Distribution of predicted grooming risk $(r_{pred})$ over fuzzy degrees of actual grooming risk $(r_{groom})$ varies across different groups (LEO, Victim and Decoy) .}
\label{fig:2}       % Give a unique label
\end{figure}

We conducted an analysis of model predictions across groups, illustrating the distribution of predicted grooming risk scores ($r_{pred}$) across varying degrees of actual grooming risk, as depicted in Figure~\ref{fig:2}. Our investigation can be summarized through the following insights: while language model predictions discern between samples categorized as \textit{moderate} and \textit{severe} in terms of grooming risk, they struggle to make the same distinction between \textit{moderate} and \textit{significant} risk samples. Additionally, the distribution of predicted risk underscores the uncertainty inherent in these estimations, as evidenced by the variance across the distributions. With the exception of \textit{moderate} risk samples in Victim and Decoy conversations, our predictions exhibit substantial variance. These findings underscore the complexity involved in developing robust estimators for grooming risk within natural language contexts.



\section{Discussion and Conclusion}
This paper investigates whether SBERT can effectively discern varying degrees of grooming risk inherent in predatory conversations. We fine-tune and evaluate a language model regressor on predatory conversations across different participant groups. Our analysis highlights that while fine-tuning aids language models in learning to assign grooming scores, they exhibit high variance in predictions, particularly in contexts with higher degrees of grooming risk. This discrepancy is tied to cases where surface form text does not contain explicit identifiers of grooming, but rather utilizes indirect speech pathways to manipulate victims. In such cases, fine-tuning sentence embedding models does not help the model learn nuanced tasks. The sole reliance on word form and incentivizing training loss lead to learning shortcuts while ignoring nuance \cite{bihanirayz2024}. Even with the integration of long-range context, the task of encoding intricate lexical semantic phenomena to enhance natural language understanding continues to be a challenge \cite{bihanirayz21, vulic2021}. This finding underscores the need for robust modeling of indirect speech acts employed in grooming contexts by language models. 


\begin{acknowledgement}
This work is supported by the DoJ grants 15PJDP-21-GK-03269-MECP and 15PJDP-22-GK-03107-MECP. 
\end{acknowledgement}


% Template for ISBI paper; to be used with:
%          spconf.sty  - ICASSP/ICIP LaTeX style file, and
%          IEEEbib.bst - IEEE bibliography style file.
% --------------------------------------------------------------------------
\documentclass{article}
\usepackage{spconf,amsmath,graphicx}

% It's fine to compress itemized lists if you used them in the
% manuscript
\usepackage{enumitem}
\usepackage{multirow}
\setlist{nosep, leftmargin=14pt}
\usepackage{booktabs}
\usepackage{caption}
\usepackage{subcaption}
\usepackage{mwe} % to get dummy images
\usepackage{url}

% Example definitions.
% --------------------
\def\x{{\mathbf x}}
\def\L{{\cal L}}

% Title.  
% ------
\title{Anatomical Grounding Pre-training for Medical Phrase Grounding}
%
% Single address.
% ---------------
% \name{Wenjun Zhang, Shakes Chandra, Aaron Nicolson}
% \address{University of Queensland}
%
% For example:
% ------------
%\address{School\\
%	Department\\
%	Address}
%
% Two addresses (uncomment and modify for two-address case).
% ----------------------------------------------------------
%\twoauthors
%  {A. Author-one, B. Author-two\sthanks{Some author footnote.}}
%	{School A-B\\
%	Department A-B\\
%	Address A-B}
%  {C. Author-three, D. Author-four\sthanks{The fourth author performed the work
%	while at ...}}
%	{School C-D\\
%	Department C-D\\
%	Address C-D}
%
% More than two addresses
% -----------------------
\name{Wenjun Zhang$^{\star}$ \qquad Shekhar S. Chandra$^{\star}$ \qquad Aaron Nicolson$^{\dagger}$}

\address{$^{\star}$University of Queensland\\$^{\dagger}$Australian e-Health Research Centre, CSIRO Health and Biosecurity, Brisbane, Australia}
% \address{$^{\dagger}$}Australian e-Health Research Centre, CSIRO Health and Biosecurity, Brisbane, Australia \\}

\begin{document}
%\ninept
%
\maketitle

\begin{abstract}

Medical Phrase Grounding (MPG) maps radiological findings described in medical reports to specific regions in medical images. The primary obstacle hindering progress in MPG is the scarcity of annotated data available for training and validation. We propose anatomical grounding as an in-domain pre-training task that aligns anatomical terms with corresponding regions in medical images, leveraging large-scale datasets such as Chest ImaGenome. Our empirical evaluation on MS-CXR demonstrates that anatomical grounding pre-training significantly improves performance in both a zero-shot learning and fine-tuning setting, outperforming state-of-the-art MPG models. Our fine-tuned model achieved state-of-the-art performance on MS-CXR with an mIoU of 61.2, demonstrating the effectiveness of anatomical grounding pre-training for MPG.

% Phrase grounding models maps phrases to specific regions in an image, while for medical phrase grounding, the phrase 

\end{abstract}

\section{Introduction}
MPG involves mapping a descriptive phrase containing a radiological finding to a specific region in a medical image \cite{10.1007/978-3-031-43990-2_35}. An MPG model could be used to visually connect findings in a radiologist report---whether produced by radiologist or by automatic report generation model---to the corresponding regions in the images. Findings accompanied by their associated bounding boxes are easier to verify, enhancing the reliability of reported information \cite{bernstein_can_2023, 10204026, doi:10.1148/ryai.2020190043}.

MPG is a specialised application within the broader field of phrase grounding. State-of-the-art general-domain phrase grounding models are pre-trained on large-scale phrase-to-region datasets and demonstrate strong zero-shot learning and few-shot transferability on downstream localisation tasks \cite{9879567, 9710994, 10.1007/978-3-031-72970-6_3}. However, despite their success in general-domain tasks, these models struggle to generalise to MPG, especially in a zero-shot learning setting. One possible reason is the significant domain shift from general-domain to medical-domain data \cite{zhao-titov-2023-transferability}. Furthermore, large-scale pre-training is challenging in the medical domain due to the scarcity of annotated MPG datasets, with only a small public benchmark dataset available \cite{10.1007/978-3-031-20059-5_1}. 

To overcome the challenges of limited MPG training data and the large domain gap between MPG and the general phrase grounding data, we propose to leverage anatomical grounding as an in-domain pre-training task for MPG, as demonstrated in Figure \ref{fig:concept} (middle). Anatomical grounding involves aligning text describing an anatomical region with the corresponding region within a medical image. This approach leverages the extensive anatomical text-to-region data available in datasets such as Chest ImaGenome \cite{wu2021chestimagenomedatasetclinical}, enabling effective fine-tuning or zero-shot learning for MPG tasks, where data is more limited \cite{ 10.1007/978-3-031-20059-5_1}. This pre-training step equips the model to recognise common anatomical landmarks, which radiologists frequently reference when describing findings in radiological reports. For instance, by learning to localise the \textit{right apical zone} with the Chest ImaGenome dataset, the model is more capable of localising findings such as a \textit{small right apical pneumothorax}.


\begin{figure}
    \centering
    \includegraphics[width=1\linewidth]{concept.png}
    \caption{Anatomical grounding as an in-domain pre-training task for Medical Phrase Grounding (MPG).}
    \label{fig:concept}
\end{figure}

We evaluated the effectiveness of anatomical grounding pre-training on MS-CXR, a MPG dataset, using two pre-trained general-domain phrase grounding models, TransVG \cite{9710016} and MDETR \cite{9710994}. We also evaluate it in both a zero-shot learning and a fine-tuning setting. Figure \ref{fig:concept} describes the training process; TransVG or MDETR is first pre-trained on anatomical grounding. They are then fine-tuned on MPG (if they are not evaluated in a zero-shot learning setting). Our empirical evaluation demonstrates that anatomical grounding pre-training significantly improves performance in a zero-shot learning setting, and significantly improves the performance of MDETR in a fine-tuning setting. We compare our anatomically grounded pre-trained models to state-of-the-art MPG models from the literature, and demonstrate that our models achieve an improvement in performance. The pre-trained models, and demo for this work are available at: \url{https://github.com/Claire1217/AGPT}.


\section{Related Work}
\subsection{General-domain Phrase Grounding}
Vision-language models pre-trained on large-scale image-text datasets, such as CLIP, have shown strong zero-shot learning and few-shot learning capabilities on global image understanding tasks \cite{pmlr-v139-radford21a}. GLIP extends this by pre-training on large-scale phrase grounding data \cite{9879567}. The learned representations demonstrate strong transferability to various local-level recognition tasks. Current pre-trained general-domain phrase grounding models are typically applied to two primary tasks: phrase localisation and referring expression comprehension. Phrase localisation focuses on identifying and locating multiple objects mentioned in a sentence. MDETR is a phrase localisation model, associating sub-phrases within a sentence with multiple object queries \cite{9710994}. In contrast, TransVG is a referring expression comprehension model---it detects a single object or region in an image for a whole sentence \cite{9710016}.

\subsection{Medical Phrase Grounding}
Due to the scarcity of annotated data, MPG has received limited attention in the literature. Boecking \textit{et al.} introduced MS-CXR, a phrase grounding chest X-ray benchmark dataset \cite{10.1007/978-3-031-20059-5_1}. Their objective with the dataset was to evaluate the grounding performance of their self-supervised biomedical vision-language model (BioViL). BioViL demonstrates strong zero-shot learning capabilities, given that it is not trained for MPG. Recently, Chen \textit{et al.} directly fine-tuned TransVG on a split of MS-CXR in order to directly learn MPG, forming MedRPG \cite{10.1007/978-3-031-43990-2_35}. Here, a bounding box supervised loss and a specific contrastive loss were leveraged. Unlike these models, we pre-train on large-scale anatomical grounding data using Chest ImaGenome, in order to provide in-domain pre-training.

\subsection{Anatomical Information in Medical Imaging}
Anatomical information has been effectively used in tasks like pathology detection and classification to improve accuracy and localisation. For example, the Anatomy-Driven Pathology Detection (ADPD) model \cite{muller_anatomy-driven_2023} used easy-to-annotate anatomical regions as proxies for pathologies, helping to locate disease locations without detailed pathology-specific bounding boxes. AnaXNet \cite{agu_anaxnet_2021} used anatomical relationships to improve classification by identifying the exact regions where findings occur. Despite these successes, no work has applied anatomical information to medical phrase grounding. 

\section{Methodology}\label{sec:methodology}
Our work addresses \textbf{medical phrase grounding} (MPG),  which involves mapping a descriptive phrase containing radiological finding to a specific
region in a medical image. This can be defined as learning a function  \( f: P \times I \rightarrow B \), where \( P \) represents the set of medical phrases, \( I \) represents the set of medical images, and \( B \) represents the set of bounding boxes. Given a phrase \( p \in P \) and an image \( i \in I \), the model predicts a bounding box \( b \in B \) such that $b = f(p, i)$. Our approach introduces a novel training framework for MPG, which involves extending the pre-training of general phrase grounding models with an anatomical grounding pre-training. 

Anatomical grounding involves predicting bounding boxes for anatomical structures using textual descriptions of their locations. The task can be formulated as 
\( f_{\text{anat}}: A \times I \rightarrow B \). Specifically, for each anatomical term \( a \in A \) and image \( i \in I \), the model predicts a bounding box \( b \in B \) such that $b = f_{\text{anat}}(a, i; \theta_{\text{gen}})$, 
where \( \theta_{\text{gen}} \) are the initial general-domain pre-trained weights. Through anatomical grounding pre-training, we refine the weights to create anatomy-specific parameters  \( \theta_{\text{anat}} \). 

To enhance generalisation and robustness, we leverage GPT-4 to generate four additional synonymous variations for each anatomical location in the Chest ImaGenome dataset. This aligns with clinical practice, where radiologists frequently use interchangeable terms to describe the same region. For example, ``left lung base” might also be referred to as ``left basal lung” or ``left lower lung base”. The detailed augmentation of anatomical regions is included in the aforementioned code repository. 

\section{Datasets} \label{sec:dataset}

\paragraph*{Chest ImaGenome \cite{wu2021chestimagenomedatasetclinical}}
We use the Chest ImaGenome dataset for anatomical grounding pre-training. Chest ImaGenome is a scene graph-structured dataset that includes $242\,072$ images. It contains $1\,256$ combinations of relational annotations between 29 anatomical structures in chest X-rays, with bounding box coordinates and additional attributes organised as a scene graph per image. In this study, we use the names and bounding box coordinates of these 29 anatomical structures, focusing specifically on frontal images. Examples of anatomical structures include ``left lung base", ``left lung apical zone", and ``right hilar structures".

\paragraph*{MS-CXR \cite{10.1007/978-3-031-20059-5_1}} 
We use the MS-CXR dataset for the MPG task. It contains $1\,162$ medical phrase-bounding box pairs across eight pathologies, such as \textit{cardiomegaly} and \textit{pleural effusion}. The findings are manually annotated and described by radiologists, ensuring precise alignment between medical phrases and bounding boxes. Example phrases include ``Large right-sided pneumothorax", and ``Small bilateral pleural effusions". The whole dataset was used for testing for the zero-shot learning setting with the general-domain pre-trained and anatomical pre-trained phrase grounding models, while the train-test-val split from \cite{10.1007/978-3-031-43990-2_35} was used for the fine-tuning setting. 

\section{Experiment Setup}
\paragraph*{Model}
Experiments were conducted with two models, TransVG and MDETR. For TransVG, ResNet-50 and ClinicalBERT were used as the visual and text encoders, respectively, whereas ResNet-101 and RoBERTa-base were used for MDETR. Here, MDETR functions on a sentence-level, mapping a medical phrase to one region in an image. This differs from its standard function, where it maps multiple sub-phrases from a sentence to multiple regions in the image. Full-model anatomical grounding pre-training of MDETR resulted in an unstable training process, likely due to its multi-object detection task. To address this, we applied Low-Rank Adaptation (LoRA) \cite{Hu2021LoRA:Models} during anatomical grounding pre-training. This likely stabilised training by limiting trainable parameters to low-rank layers, preventing drastic weight updates and reducing instability during adaptation.

\paragraph*{Pre-training and Fine-Tuning}
For anatomical grounding pre-training, we process mini-batches of eight images, each paired with five anatomical regions chosen from five synonymous terms, creating 40 anatomical text-region pairs per mini-batch. For MPG fine-tuning, both models were trained on the MS-CXR training set with a mini-batch size of 12. During fine-tuning, all of the weights of MDETR were trainable, including the LoRA weights. The AdamW optimiser with a learning rate of 1e-4 and 1e-5 was used for pre-training and fine-tuning, respectively \cite{DBLP:conf/iclr/LoshchilovH19}. Each model was trained for 1 epoch during pre-training and 90 epochs during fine-tuning. Images were resized and padded to a size of 640$\times$640. During training, the images were augmented with colour jitter and Gaussian noise.

% When fine-tuning the anatomical pre-trained models on the training set of MS-CXR. The task is formulated as \( f_{\text{MPG}}: P \times I \rightarrow B \), where given a medical phrase \( p \in P \) and an image \( i \in I \), the task is to produce a bounding box \( b \in B \) as follows: $b = f_{\text{MPG}}(p, i; \theta_{\text{anat}})$. With fine-tuning, the weights are updated to \( \theta_{\text{MPG}} \). 

\paragraph*{Evaluation}
We used mIoU and accuracy (Acc) as metrics. For accuracy, a predicted bounding box was considered true if the mIoU with the ground truth bounding box was larger than 0.5. We evaluate the anatomical grounding pre-trained MDETR and TransVG models on the MS-CXR dataset in both zero-shot learning and fine-tuning settings. The self-supervised pre-trained models GLoRIA \cite{9710099} and BioViL \cite{10.1007/978-3-031-20059-5_1} were used for comparison. In the fine-tuning setting, we further fine-tuned the anatomical grounding pre-trained MDETR and TransVG models on the training split of MS-CXR (described in Section \ref{sec:dataset}). These were compared to MDETR and TransVG without anatomical grounding pre-training and MedRPG \cite{10.1007/978-3-031-43990-2_35}. For zero-shot learning and fine-tuning, the epoch with the highest validation mIoU was selected for testing.

\section{Results \& Discussion}
\subsection{Effectiveness of Anatomical Grounding Pre-training}
The performance of anatomical grounding pre-training is demonstrated in Table \ref{tab:anat_comparison}. Applying MDETR and TransVG to MPG in a zero-shot learning setting produced low scores on both metrics, underscoring the limitations of general-domain phrase grounding models for MPG. However, pre-training with anatomical grounding led to a statistically significant improvement in both models’ performance across both metrics for zero-shot learning of MPG. These results demonstrate that anatomical grounding pre-training improves the models’ ability to generalise to MPG.

\begin{table}[ht]
\small
\centering
\caption{Performance of \textbf{anatomical grounding pre-training (AGPT)} on MS-CXR. Underlined indicates a stat. sig. difference to the model without anatomical grounding pre-training ($p < 0.05)$.}
\renewcommand{\arraystretch}{0.85}
\begin{tabular}{lcccc}
\toprule
\multirow{2}{*}{\textbf{Model}} & \multicolumn{2}{c}{\textbf{Zero-shot}} & \multicolumn{2}{c}{\textbf{Fine-tuning}} \\
\cmidrule(lr){2-3} \cmidrule(lr){4-5}
                                & \textbf{Acc}   & \textbf{mIoU}   & \textbf{Acc}    & \textbf{mIoU}   \\
\midrule
TransVG           & 1.2         & 10.3         & 68.9          & \textbf{59.4}          \\
\quad+ AGPT   & \textbf{\underline{39.8}}         & \textbf{\underline{40.7}}         & \textbf{70.7}          & 59.2 \\
\addlinespace % Adds space between the two groups
MDETR              & 3.0         & 14.7         & 66.9          & 57.8         \\
\quad+ AGPT   &\textbf{ \underline{34.7}}         & \textbf{\underline{32.6}}         & \textbf{\underline{70.7} }         & \textbf{\underline{61.2}} \\
\bottomrule
\end{tabular}
\label{tab:anat_comparison}
\end{table}


When fine-tuning on the MS-CXR training set, anatomical grounding pre-training produced a statistically significant improvement across all metrics for MDETR. It also demonstrated an improvement with TransVG for Acc. This indicates that anatomical grounding pre-training is effective for MPG fine-tuning, particularly for certain types of models.

In Figure \ref{fig:viz}, we illustrate the models performing MPG in zero-shot learning settings on two examples: ``right apical pneumothorax" and ``mild cardiomegaly". Without anatomical grounding pre-training, both TransVG and MDETR fail to ground the phrases accurately. However, with anatomy pre-training, both models are able to ground the text to the correct anatomical region---the right apical zone for pneumothorax and the heart for cardiomegaly. Fine-tuning offers a further improvement in the grounding accuracy.

\begin{figure}[t]
    \centering
    \includegraphics[width=1\linewidth]{vizz.png}
    \caption{MPG with and without \textbf{anatomical grounding pre-training (AGPT)}. The top example contains the anatomical region within the text, whereas the bottom example does not. Blue and red boxes indicate the ground-truth and predicted bounding boxes, respectively.}
    \label{fig:viz}
\end{figure}



\begin{table*}[t]
\small
\centering
\caption{A comparison of \textbf{anatomical grounding pre-training (AGPT)} with other models in the literature in both zero-shot learning and fine-tuning settings with \textbf{mIoU} as the metric. $\dagger$ indicates scores sourced from the BioViL paper \cite{10.1007/978-3-031-20059-5_1}.}
\label{table:combined_iou_scores}
\renewcommand{\arraystretch}{0.85}
\begin{tabular}{l c c c c c c c c c c}
\toprule
\textbf{Model} & \textbf{Supervision} & \textbf{Cardio.} & \textbf{Opacity} & \textbf{Edema} & \textbf{Consol.} & \textbf{Pneu.} & \textbf{Atelect.} & \textbf{Pneumo.} & \textbf{Pl. Eff.} & \textbf{Avg} \\
\midrule
\multicolumn{11}{c}{\textbf{Zero-shot learning}} \\ 
\midrule
\textbf{GLoRIA \cite{9710099}}$^\dagger$ & Self-super. & 27.3 & 19.8 & 25.1 & 32.4 & 24.6 & 26.1 & 10.0 & 25.4 & 24.6 \\
\textbf{BioViL \cite{10.1007/978-3-031-20059-5_1}}$^\dagger$ & Self-super. & 37.5 & 20.9 & \textbf{27.5} & \textbf{34.6} & 31.5 & 30.2 & 13.5 & \textbf{31.5} & 28.4 \\
\cmidrule(lr){1-11}
\textbf{MDETR + AGPT} & Box-super. & 61.3 & 6.0 & 8.7 & 18.5 & 18.8 & 8.2 & 16.1 & 14.6 & 32.6 \\
\textbf{TransVG + AGPT} & Box-super. & \textbf{61.5} & \textbf{23.0} & 14.5 & 33.0 & \textbf{31.9} & \textbf{39.3} & \textbf{26.9} & 21.1 & \textbf{40.7} \\
\midrule
\multicolumn{11}{c}{\textbf{Fine-tuning}} \\ 
\midrule
\textbf{MedRPG \cite{10.1007/978-3-031-43990-2_35}} & Box-super. & 80.5 & 39.3 & \textbf{51.7} & 49.1 & 46.4 & \textbf{48.8} & 38.5 & 52.8 & 59.6 \\
\textbf{MDETR \cite{9710994}} & Box-super. & 79.6 & 43.1 & 45.5 & 45.8 & 40.1 & 36.0 & 39.1 & 50.5 & 57.8 \\
\textbf{TransVG \cite{9710016}} & Box-super. & 80.6 & \textbf{46.8} & 35.6 & 42.7 & \textbf{48.5} & 42.8 & 38.3 & 49.5 & 59.4 \\
\cmidrule(lr){1-11}
\textbf{MDETR + AGPT} & Box-super. & \textbf{81.2} & 45.1 & 25.2 & \textbf{56.3} & 38.9 & 47.4 & \textbf{43.1} & \textbf{57.2} & \textbf{61.2} \\
\textbf{TransVG + AGPT} & Box-super. & 79.1 & 37.6 & 43.0 & 45.4 & 45.9 & 47.7 & 41.9 & 54.1 & 59.2 \\
\bottomrule
\end{tabular}
\end{table*}
 
\subsection{Comparison to other MPG models}
First, we compare our anatomical grounding pre-trained MDETR and TransVG models (MDETR + AGPT and TransVG + AGPT, respectively) in a zero-shot learning setting, as shown in Table \ref{table:combined_iou_scores}. We compare these to two self-supervised models, GLoRIA \cite{9710099} and BioViL \cite{10.1007/978-3-031-20059-5_1}. Both MDETR + AGPT and TransVG + AGPT outperformed GLoRIA and BioViL. This indicates that anatomical grounding pre-training is more effective for zero-shot learning MPG than the self-supervised learning strategies of GLoRIA and BioViL. Furthermore, our fine-tuned MDETR + AGPT model attained an mIoU improvement of 1.6 over the current state-of-the-art model, MedRPG \cite{10.1007/978-3-031-43990-2_35}. 

% The baseline performance for the two models is from the BioViL paper.

\subsection{Effectiveness of Synonymous Anatomical Term Augmentation}
We conducted ablation studies to evaluate the impact of adding synonymous variations of the anatomical locations, as described in Section \ref{sec:methodology}. The results show that synonymous augmentation improved the scores for both TransVG and MDETR, with a stronger effect observed in TransVG. Notably, anatomical grounding pre-training with synonymous augmentation led to a 15.6\% improvement in zero-shot learning accuracy. This provides the model with a broader range of terms for the same anatomical location. This allows the model to better generalise to new phrases in a zero-shot learning setting.
\begin{table}[h!]
\small
\centering
\caption{Improvement in performance with when using synonymous variations of the anatomical locations. Underlined indicates a stat. sig. difference to the model without synonymous variations ($p < 0.05)$.}
\label{table:syn_augmentation_effect}
\renewcommand{\arraystretch}{0.85}
\begin{tabular}{lcccc}
\toprule
\multirow{2}{*}{\textbf{Model}}  & \multicolumn{2}{c}{\textbf{Zero-shot}} & \multicolumn{2}{c}{\textbf{Fine-tuning}} \\
\cmidrule(lr){2-3} \cmidrule(lr){4-5}
 & \textbf{Acc} & \textbf{mIoU} & \textbf{Acc} & \textbf{mIoU} \\
\midrule
TransVG   & \underline{+15.6} & \underline{+13.9} & \underline{+5.4} & \underline{+2.6} \\
MDETR     & \underline{+1.8}  & \underline{+1.2}   & +2.4 & +0.9 \\
\bottomrule
\end{tabular}
\end{table}

\vspace{-10pt}
\section{Conclusion}
In this paper, we introduced anatomical grounding pre-training to address the challenges of MPG, a task constrained by limited in-domain data and significant domain shifts from general-domain pre-trained models. Our methodology involved pre-training phrase grounding models on anatomical text-region pairs using the Chest ImaGenome dataset, followed by MPG-specific fine-tuning on the MS-CXR dataset. Empirical results demonstrated that anatomical grounding pre-training significantly improved zero-shot learning and fine-tuning performance on MPG, surpassing existing self-supervised and state-of-the-art MPG models. Additionally, our augmentation with synonymous anatomical terms further enhanced generalisability. This work demonstrates that leveraging anatomical grounding pre-training is an effective solution to the challenge of limited MPG data.

\section{Compliance with Ethical Standards}  
\vspace{-2mm} % Reduce space above section title
\noindent This study used public data from MIMIC-CXR (under PhysioNet’s credentialed license). Ethical approval was not required as confirmed by the license attached with the open access data.
\vspace{-2mm}

\section*{Acknowledgments}
\vspace{-2mm} % Reduce space before acknowledgments
\noindent No funding was received. The authors declare no competing interests.  

% Tighten bibliography spacing
\vspace{-5mm} % Reduce space before references


\bibliographystyle{IEEEtran}
\bibliography{isbi/ISBI_2024_template-master/references}
\end{document}


% %
% \section*{Appendix}
% \addcontentsline{toc}{section}{Appendix}
% %
%
\end{document}
