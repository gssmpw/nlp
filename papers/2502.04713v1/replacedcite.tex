\section{Related Works}
A lot of literature is found in Image classification using the above two approaches. In band selection methods further three sub-categories have been devised based on the derivation of subset of bands____. These are as follows: subsets derived on the basis of subset evaluation criteria____, availability of prior information further sub-categorized on the basis of supervised selection criteria____, unsupervised selection criteria____ and selection strategy (individual____ and other evaluation techniques) used to create the band subset.

Efficient band selection plays a crucial role in extracting meaningful information from vast datasets. By selecting a subset of relevant spectral bands, researchers can reduce data dimensionality, enhance computational efficiency, and improve the performance of downstream analysis tasks such as restoration tasks, classification and target detection. However, achieving optimal band selection poses significant challenges, necessitating innovative approaches that leverage spectral grouping techniques.____,____ address this need with novel methodologies. The former introduces a method based on neighborhood grouping to efficiently identify relevant bands, while the latter proposes leveraging differences between inter-groups for band selection.____ presents a learning-based optimization approach for band selection tailored specifically for classification tasks. Despite their innovative approaches, both approaches face challenges interms of computational efficiency, interpretability, and scalability. Furthermore, in the context of hyperspectral image restoration, particularly superresolution, no definitive solutions have emerged regarding grouping methodologies. This underscores a critical gap in current research highlighting the need for further exploration and development in this area. Addressing these challenges is crucial for an efficient grouping algorithm, advancing hyperspectral imaging capabilities and realizing the full potential of band selection techniques in various applications.
\begin{figure}[h]
    \centering
    \subfloat{\includegraphics[width=0.20\textwidth]{ntire_corr.png}} 
    \subfloat{\includegraphics[width=0.20\textwidth]{cave_corr.png}} 
    \subfloat{\includegraphics[width=0.20\textwidth]{chikusie_50dpi.png}}
    \subfloat{\includegraphics[width=0.20\textwidth]{landsat_corr.png}}
    \subfloat{\includegraphics[width=0.20\textwidth]{sentinel_corr.png}}
    \caption{(a) Correlation matrix for NTIRE2022 Dataset (b) CAVE Dataset (c) Chikusei Dataset (d) Sentinel Dataset (e) Landsat Dataset}
    \label{fig:foobar}
\end{figure}
Recent advancements in hyperspectral image analysis focus on correlation matrices, which reveal spectral dependencies and enable high-texture detail within groups of spectrally dependent vectors. Traditionally, linear predictions____ have dominated hyperspectral data analysis, but recent studies advocate for interval sampling____ for superior group formation and enhanced interpolation outcomes. Our study proposes an explicit grouping method based on correlation coefficients to establish a standardized framework for hyperspectral super-resolution. By integrating correlation analysis into our grouping strategy, each band within a correlated group is rigorously evaluated for its significance in super-resolution. These explicit groups, guided by correlation analysis, feed into deep neural network architectures, optimizing interpolation learning efficiency. Our innovation lies in combining Determinantal Point Process (DPP) with correlation analysis to form coherent band subsets, minimizing redundancy and maximizing diversity. This integration enriches interpolation learning by capturing intricate spectral relationships effectively. Challenges arise when grouped bands still overlap significantly despite spectral correlation. To mitigate this, our approach employs Spectral Angle Mapper (SAM) to resolve overlaps based on the lowest SAM values, refining the grouping strategy. This modular integration enhances cohesion, learning robustness, and efficiency in handling complex hyperspectral data, promoting seamless interaction within the network and improving restoration outcomes.
\begin{algorithm}
\caption{Spectral correlation estimation}
\label{algo: Algorithm1}
\begin{algorithmic}[1]
\State \textbf{Input:} \(X\), \(S\)
\State \textbf{Output:} \(Z\)
\State \(Z \in \mathbb{R}^{N \times WH}\)
\State \(X \in \mathbb{R}^{N \times wh}\)
\State Initialize \(B \in \mathbb{R}^{N \times n}\)
\State Initialize \(M \in \mathbb{R}^{n \times WH}\)
\State \(X = ZS\)
\For{\textbf{each band} \(i, j\)}
    \State \(b_i = f_i(t)\)
    \State \(b_j = f_j(t)\)
    \State \(R_{b,b}(i, j) = \frac{E[b_i, b_j]}{\sigma_i \sigma_j}\)
\EndFor
\State \(Z = B \cdot M\)
\State Solve for \(B\) and \(M\)
\end{algorithmic}
\end{algorithm}