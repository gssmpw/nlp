%\parttoc % Insert the appendix TOC
\section*{Appendix}

We now report extended material concerning the extended related work (Appendix \ref{app:related}), the extraction of historical activations and gradients (Appendix \ref{app:notation:extraction}),  as well as detailed results on dissonance awareness (Appendix \ref{app:dissonance}), non-dissonant updates (Appendix \ref{app:update}) and dissonant updates (Appendix \ref{app:dissonant}).

% Table of contents for the appendixn

\section{Extended Related work}\label{app:related}
In this section, we provide an extended version of Tab.~\ref{tab:continual-learning-taxonomy}, focusing \textit{only} on the \textit{most recent literature}, and showing how our work is uniquely positioned in the landscape of model editing and continual learning, the two key related branches to our work.

\begin{table}[t]
\centering
\caption{Extended taxonomy of incremental Learning Approaches, showing some seminal work (top) and more recent literature (split into editing and continual learning).}
\resizebox{\textwidth}{!}{%
\begin{tabular}{@{}l p{3cm} c c c c c c@{}}
\toprule
\textbf{Examples} & \textbf{\makecell{Incremental \\ Type}} & \textbf{\makecell{Memory \\ Usage}} & \textbf{\makecell{Task \\ Awareness}} & \textbf{\makecell{Weight \\ Plasticity}} & \textbf{\makecell{Architecture}} & \textbf{\makecell{Conflict \\ Detection}} & \textbf{\makecell{Update \\ Mechanism}} \\ \midrule
% Classic/Foundational Approaches
iCaRL~\citep{rebuffi2017icarl} & Class-incremental & Replay & Task-Agnostic & Fixed & Fixed & No & Rehearsal \\
EWC~\citep{kirkpatrick2017overcoming} & Task-incremental & None & Task-Aware & Selective & Fixed & No & Regularization \\
Progressive Nets~\citep{rusu2016progressive} & Task-incremental & None & Task-Aware & Fixed & Expanding & No & New Subnetworks \\
DEN~\citep{yoon2017lifelong} & Task-incremental & None & Task-Aware & Selective & Expanding & No & Selective Expansion \\
GEM~\citep{lopez2017gradient} & Task-incremental & Replay & Task-Aware & Constrained & Fixed & No & Constrained Optimization \\
ROME~\citep{DeCao2021} & Fact-incremental & None & Fact-Aware & Localized & Fixed & No & Rank-One Update \\
OWM~\citep{zeng2019continual} & Task-incremental & None & Task-Aware & Orthogonal & Fixed & No & Orthogonal Projection \\
PackNet~\citep{mallya2018packnet} & Task-incremental & None & Task-Aware & Selective & Fixed & No & Weight Masking \\
HAT~\citep{serra2018overcoming} & Task-incremental & None & Task-Aware & Selective & Fixed & No & Attention Masking \\ \midrule
% Recent Knowledge Editing Approaches
MALMEN~\citep{tan2023massive} & Fact-incremental & None & Fact-Aware & Localized & Fixed & No & Parameter Shift Aggregation \\
EditAnalysis~\citep{li2023unveiling} & Fact-incremental & None & Fact-Aware & Analysis & Fixed & No & Consistency Analysis \\
D4S~\citep{huang2024reasons} & Fact-incremental & O(1) & Fact-Aware & Regulated & Fixed & No & Layer-Norm Control \\ \midrule
% Recent Continual Learning Approaches
Global Prototypes~\citep{baicontinual} & Task/Class-incremental & None & Task-Agnostic & Selective & Fixed & No & Global Prototype Alignment \\
NTE~\citep{benjamin2024continual} & Task-incremental & None & Task-Agnostic & Selective & Fixed & No & Bayesian Ensemble \\
UPGD~\citep{elsayed2024addressing} & Task-incremental & None & Task-Agnostic & Selective & Fixed & No & Utility-Gated Updates \\
~\citep{hiratani2024disentangling} & Task-incremental & None & Task-Aware & Selective & Fixed & No & Fisher Information \\
CLAP~\citep{jha2024clap4clip} & Class-incremental & None & Task-Aware & Selective & Fixed & No & Probabilistic Adaptation \\
VQ-Prompt~\citep{jiao2024vector} & Class-incremental & None & Task-Agnostic & Fixed & Fixed & No & Discrete Prompt Selection \\
IsCiL~\citep{lee2024incremental} & Task-incremental & None & Task-Aware & Selective & Fixed & No & Skill-based Adaptation \\
BGS~\citep{leecontinual} & Task/Domain/Class-incremental & Replay & Task-Aware & Selective & Fixed & Yes & Bias-Aware Update \\
SLM~\citep{peng2024scalable} & Task-incremental & None & Auto-detected & Selective & Fixed & No & Vector Space Retrieval \\
Train-Attention~\citep{seo2024train} & Knowledge-incremental & None & Task-Agnostic & Selective & Fixed & No & Token-Weighted Update \\
Refresh Learning~\citep{wang2024unified} & Task/Class-incremental & Optional & Task-Aware & Selective & Fixed & No & Unlearn-Relearn \\
RAIL~\citep{xu2024advancing} & Cross-domain-incremental & None & Task-Agnostic & Selective & Fixed & No & Regression-based Update \\
SAFE~\citep{zhao2024safeslowfastparameterefficient} & Class-incremental & None & Task-Agnostic & Selective & Fixed & No & Dual Parameter-Efficient Tuning \\ \midrule
% Your Paper
\textbf{This paper} & Fact-incremental & None & Conflict-Aware & Selective & Fixed & Yes & Neuron-Specific Update \\ \bottomrule
\end{tabular}%
}
\label{tab:continual-learning-taxonomy-extended}
\end{table}


\subsection{Continual learning}
Continual Learning (CL) methods enable models to learn new tasks without catastrophically forgetting previously mastered ones \citep{kirkpatrick2017overcoming}. These approaches fall into three main families: memory-based methods using exemplar buffers \citep{rebuffi2017icarl}, knowledge distillation techniques that transfer information across model versions \citep{lopez2017gradient}, and regularization-based methods that constrain weight updates \citep{kirkpatrick2017overcoming}. To ease the understanding of this landscape, we build a taxonomy that characterizes approaches by their incremental type (task, class, or fact-based), memory requirements, update mechanisms, and architectural constraints (Tab.~\ref{tab:continual-learning-taxonomy}). This taxonomy reveals how our work is different from existing continual learning attempts: while existing methods focus on preserving knowledge across distinct tasks, none explicitly address the detection and handling of conflicting information - a key capability in human cognition that our work empirically investigates.

One of the closest old approaches is deep mind's EWC~\citep{kirkpatrick2017overcoming}, a method designed to mitigate catastrophic forgetting in neural networks trained sequentially on distinct tasks. The core idea is to protect the most important weights (or neurons) for previously learned tasks during the training of new tasks. EWC identifies these important weights by calculating the Fisher Information Matrix during or after the training of a task, which estimates how sensitive each weight is to the task’s performance. Weights that significantly impact the output for a given task are marked as important. A quadratic penalty is then applied during future learning, constraining these weights to remain close to their values from the previous task. This ensures that knowledge from earlier tasks is preserved while still allowing the model to adapt to new tasks. However, EWC is \textbf{less suitable for LLMs}, which \textbf{do not have clearly defined tasks} when it comes to knowledge ingestion (probably different for other types of skills). EWC's effectiveness relies on distinct task boundaries and the ability to compute task-specific importance for weights, which is feasible in scenarios with well-defined tasks, such as classification or reinforcement learning. In LLMs, where learning spans a wide range of topics and linguistic structures without clear task delineation, it’s challenging to apply EWC's task-based strategy. The model would struggle to assign specific neurons or weights to individual tasks or concepts, making it difficult to protect task-specific knowledge without hindering the model’s overall generalization ability across a diverse dataset.

We cite in the remainder more recent literature that we project onto our taxonomy.

\citet{baicontinual} introduce a novel approach to continual learning that leverages global prototypes to mitigate catastrophic forgetting in neural networks. Their key insight is that maintaining stable connections between task-specific representations and pre-learned, general-purpose token embeddings (which serve as global prototypes) can significantly reduce forgetting without requiring explicit replay mechanisms. Through empirical validation on both task-incremental and class-incremental NLP scenarios, they demonstrate that models preserving strong connections to these global prototypes exhibit enhanced stability. While their work shares our goal of preserving knowledge during updates, it differs fundamentally in its approach and granularity: where they focus on task-level knowledge preservation through architectural mechanisms, our work addresses the more specific challenge of managing contradictory factual updates through cognitive-inspired conflict detection. Their finding that stable reference points aid knowledge retention is conceptually relevant to our work, though our results suggest that such architectural approaches alone may be insufficient when handling explicitly contradictory information, where more sophisticated cognitive mechanisms become necessary.

\citet{benjamin2024continual} proposed an elegant theoretical framework that interprets neural networks as Bayesian ensembles of classifiers. Their key insight is that a neural network with N parameters can be viewed as a weighted ensemble of N classifiers in the lazy regime, where the classifiers remain fixed throughout learning. This interpretation reveals that a properly designed posterior update rule, resembling SGD without momentum, can enable continual learning without forgetting - notably, they prove that momentum actually exacerbates forgetting. While their work focuses on preserving all knowledge in task-incremental learning, our paper specifically examines cases where knowledge needs to be deliberately updated or overridden. Their key contribution is showing that catastrophic forgetting is linked to the transition from lazy to rich regimes in neural networks, providing both a theoretical explanation for why larger models are more robust to forgetting and a biologically-inspired mechanism for knowledge preservation that perhaps complements our cognitive-based approach.

\citet{elsayed2024addressing} propose UPGD (Utility-based Perturbed Gradient Descent), a novel approach targeting both catastrophic forgetting and loss of plasticity in streaming learning scenarios. Their method protects useful network units while maintaining plasticity in less-used ones through utility-gated gradient updates and perturbations. Unlike previous approaches requiring task boundaries or memory buffers, UPGD operates in a challenging streaming setting with continuous non-stationarity. Using their newly introduced direct plasticity metric, they demonstrate UPGD's ability to maintain performance levels that surpass or match existing methods. This work complements our investigation by providing evidence that selective neuronal updates based on utility metrics can effectively balance stability and plasticity, though in a task-learning rather than knowledge-updating context.

\citet{hiratani2024disentangling} analyze how task similarity affects continual learning through a novel theoretical framework combining teacher-student models with latent structure. Their key insight is that high input feature similarity coupled with low readout similarity leads to catastrophic outcomes in both knowledge transfer and retention, even when tasks are positively correlated. They demonstrate that weight regularization in the Fisher information metric robustly helps retention regardless of task similarity, while common approaches like activity gating improve retention at the cost of transfer performance. Their theoretical predictions are validated on permuted MNIST tasks with latent variables.

\citet{jha2024clap4clip} propose a probabilistic approach to continual learning for vision-language models, specifically focusing on CLIP adaptation. Their method, CLAP, introduces visual-guided attention and task-specific probabilistic adapters to model the distribution of text features, while leveraging CLIP's pre-trained knowledge for initialization and regularization. This work demonstrates that probabilistic modeling can significantly reduce catastrophic forgetting in class-incremental learning scenarios, achieving state-of-the-art performance across multiple benchmarks.

\citet{jiao2024vector} propose VQ-Prompt, a novel prompt-based continual learning framework that addresses class-incremental learning with pretrained vision transformers. Their key innovation is incorporating vector quantization into prompt selection, enabling end-to-end optimization of discrete prompts with task loss while maintaining effective knowledge abstraction. This contrasts with our cognitive-dissonance aware approach, as they focus on task adaptation through prompt engineering rather than explicit conflict detection. Their empirical results on ImageNet-R and CIFAR-100 demonstrate superior performance compared to existing prompt-based methods, suggesting the effectiveness of discrete knowledge representation in continual learning.

\cite{lee2024incremental} propose IsCiL, a framework for continual imitation learning that uses retrievable skills and adapter-based architecture to enable efficient knowledge sharing across tasks. Unlike traditional approaches that isolate task-specific parameters, IsCiL introduces a prototype-based skill retrieval mechanism that allows selective reuse of previously learned skills for new tasks. While focused primarily on motor skills rather than resolving knowledge contradictions, their empirical results show that this selective adaptation approach significantly improves sample efficiency and reduces catastrophic forgetting compared to other adapter-based methods, particularly in scenarios with incomplete demonstrations.

\citet{leecontinual} present a systematic empirical investigation of how dataset bias affects continual learning. Through carefully designed experiments across task-incremental, domain-incremental, and class-incremental scenarios, they reveal that bias transfers both forward and backward between tasks. Their analysis shows that CL methods focusing on stability tend to preserve and propagate biases from previous tasks, while emphasis on plasticity allows new biases to contaminate previous knowledge. Based on these insights, they propose BGS (Balanced Greedy Sampling), a method that mitigates bias transfer by maintaining a balanced exemplar memory and retraining the classification head. Note that here, we used ``Replay'' for Memory Usage in the table since their best performing method (BGS) uses an exemplar memory, but they also evaluate methods without memory. 

\citet{peng2024scalable} proposed a continual learning approach that automates task selection through vector space retrieval, eliminating the need for explicit task IDs, experience replay, or optimization constraints. Their method, Scalable Language Model (SLM), combines Joint Adaptive Re-parameterization with dynamic knowledge retrieval to automatically identify relevant parameters for each input, enabling task-agnostic updates. While achieving state-of-the-art results across diverse tasks and model scales (BERT, T5, LLaMA-2), their key contribution is demonstrating that automatic task identification and parameter selection can enable continual learning without requiring explicit task boundaries or memory buffers.

\citet{seo2024train} presented Train-Attention, an interesting meta-learning approach for continual knowledge learning (CKL) in LLMs that predicts and applies weights to tokens \textit{based on their usefulness for future tasks}. Unlike previous approaches that uniformly update all parameters, their method enables\textit{ targeted knowledge updates by learning which tokens are most important} to focus on. Through experiments on LAMA-CKL and TemporalWiki benchmarks, they show that selective token-weighted learning significantly reduces catastrophic forgetting while improving learning speed. The work somewhat complements our cognitive-inspired approach, and demonstrates the benefits of selective attention, but it does not explicitly address the handling of contradictory information.

\citet{wang2024unified} proposed a unified framework for continual learning that reveals common mathematical structures across seemingly distinct approaches (regularization-based, Bayesian-based, and memory-replay). Building on this unification, they introduce ``refresh learning" - a plug-in mechanism that first unlearns current data before relearning it, inspired by the beneficial role of forgetting in human cognition. Their work primarily focuses on task-incremental and class-incremental scenarios, demonstrating improved accuracy across CIFAR and Tiny-ImageNet benchmarks. While their approach differs from our fact-level knowledge updates in LLMs, their findings about selective forgetting complement our observations about cognitive-inspired update mechanisms. Their theoretical analysis showing that refresh learning improves the flatness of the loss landscape offers an interesting perspective on how controlled forgetting might benefit knowledge retention in neural networks.

\citet{xu2024advancing} propose a cross-domain task-agnostic incremental learning framework (X-TAIL) for vision-language models, focusing on the challenge of preserving both incrementally learned knowledge and zero-shot abilities. Their approach, RAIL, uses recursive ridge regression with non-linear projections to adapt to new domains without catastrophic forgetting. Unlike previous work requiring domain identity hints or reference datasets, RAIL can classify images across both seen and unseen domains without domain hints, demonstrating superior performance in both discriminative ability and knowledge preservation. While their work advances the technical aspects of continual learning, it differs from our cognitive-inspired investigation as it doesn't address the fundamental challenge of detecting and resolving conflicting knowledge, instead focusing on domain adaptation without explicit conflict awareness.

\citet{zhao2024safeslowfastparameterefficient} propose a class-incremental learning framework for pre-trained vision models that balances stability and plasticity through two complementary parameter-efficient tuning mechanisms. Their SAFE approach first inherits generalizability from pre-trained models via a ``slow learner" that captures transferable knowledge in the first session, then maintains plasticity through a ``fast learner" that continuously adapts to new classes while resisting catastrophic forgetting. While focused on vision tasks rather than language models, their dual-speed learning strategy presents interesting parallels to our cognitive-inspired approach – particularly in how both works identify the importance of selective plasticity and the distinction between stable (``stubborn") and adaptable (``plastic") parameters. However, SAFE doesn't address the fundamental challenge of detecting and handling contradictory information that we identify as crucial for true cognitive-inspired learning.

\textit{Unlike the above work, our goal is to understand the fundamental cognitive mechanisms underlying the continuous knowledge updates in LLMs, particularly focusing on how models can detect and react to contradictory information. Rather than proposing a new continual learning method, we provide crucial insights into how different types of knowledge updates affect model behavior and stability.}

\subsection{Knowledge editing}
Next, a big portion of recent literature has focused on understanding and modifying the internal knowledge of Large Language Models (LLMs), post-training. Such knowledge editing aims to alter specific facts or associations within the model without the need for full retraining. 

\citet{Geva2020} were among the first to show that transformer Feed-Forward Network (FFN) layers act as unnormalized key-value stores encoding relational knowledge inside LLMs. This observation was later confirmed and complemented by others \citep{Meng2022,Dai2021} before being leveraged by subsequent work to master the editing of internal memories.
\citet{Meng2022} introduced ROME (Rank-One Model Editing), a method that uses causal tracing to empirically locate the layers essential to encoding a given association. They then modify these modules by applying small rank-one changes. To identify the relevant modules, they run the network multiple times, introducing corruptions to the input sequence to disturb the inference, and then restore individual states from the original non-corrupted pass. But this work an others worked only on single edits, and were often evaluated one edit at a time, starting each time from a fresh pre-trained model. The same authors later developed MEMIT, which follows the same causal tracing principle but with the goal of scaling up to 10,000 edits in bulk\citep{Meng2022a}.
Similarly, \citet{Dai2021} leveraged the identification of knowledge neurons to perform ``knowledge surgery" – editing factual knowledge within Transformers without the need for additional fine-tuning.
\citet{Zhu2020} approached the knowledge modification task as a constrained optimization problem. Their work found that constrained layer-wise fine-tuning emerges as an effective method for modifying the knowledge that Transformers learn, suggesting a different pathway for knowledge editing inside LLMs.
\citet{DeCao2021} proposed \textsc{KnowledgeEditor}, which achieved knowledge editing by training a hyper-network with constrained optimization to modify specific facts without fine-tuning or changing the overall stored knowledge. The method was demonstrated on smaller models like BERT for fact-checking and BART for question answering, achieving consistent changes in predictions across different formulations of queries.

\citet{li2023unveiling}  empirically investigate the pitfalls of knowledge editing in LLMs, revealing two critical issues: logical inconsistencies between multiple edits (like contradictory relationship updates) and knowledge distortion (where edits irreversibly damage the model's knowledge structure). Through carefully designed benchmarks \textsc{ConflictEdit} and \textsc{RoundEdit}, they demonstrate that current editing methods struggle with these challenges, particularly when handling reverse relationships or composite logical rules. While their work focuses on identifying limitations in maintaining logical consistency across edits, our paper takes a complementary cognitive-inspired perspective by addressing how models handle contradictions with their existing knowledge base. Their findings about knowledge distortion align with and reinforce our observations about the catastrophic nature of updates that modify existing knowledge.

Similarly, \citet{huang2024reasons} empirically investigate causes of performance degradation during knowledge editing in LLMs. They show degradation correlates with editing target complexity and L1-norm growth in edited layers. Their proposed Dump for Sequence (D4S) method regulates layer norm growth using O(1) space complexity, enabling multiple effective updates while minimizing model degradation. Their work provides valuable insights into the mechanisms of model degradation during knowledge editing, but it does not specifically address the distinction between contradictory and non-contradictory updates, as we do in this paper.

\citet{tan2023massive} propose MALMEN, a scalable hypernetwork approach for editing Large Language Models by aggregating parameter shifts using a least-squares formulation. While previous editing methods like MEND~\citep{mitchell2022fast} could handle only a few facts simultaneously, MALMEN can efficiently edit thousands of facts while maintaining comparable performance. Their key innovation lies in separating the computation between the hypernetwork and LM, enabling arbitrary batch sizes and reducing memory requirements. Their empirical results show that MALMEN can edit hundreds of times more facts than MEND while maintaining similar performance levels, though they note that the method still struggles with generalizing to rephrasing not seen during training. Like other editing approaches, MALMEN focuses on the mechanics of (by design conflicting) updates.

\textit{Unlike all the work above, our goal in this work is not to edit knowledge, but to understand the fundamental mechanisms and phenomena that govern how LLMs integrate new information with existing knowledge. By taking a cognitive-inspired approach focused on dissonance awareness and adaptive plasticity, we reveal critical insights about the nature of knowledge representation and updating in these models.}


% https://proceedings.neurips.cc/paper/2021/hash/2a10665525774fa2501c2c8c4985ce61-Abstract.html
% https://proceedings.neurips.cc/paper_files/paper/2022/hash/80133d0f6eccaace15508f91e3c5a93c-Abstract-Conference.html
% https://arxiv.org/abs/1903.04476
% https://proceedings.mlr.press/v199/abbasi22a
% https://openaccess.thecvf.com/content/CVPR2022/html/Yan_Learning_Bayesian_Sparse_Networks_With_Full_Experience_Replay_for_Continual_CVPR_2022_paper.html
% https://proceedings.neurips.cc/paper/2020/hash/258be18e31c8188555c2ff05b4d542c3-Abstract.html
% https://proceedings.mlr.press/v234/yildirim24a.html
% https://www.sciencedirect.com/science/article/pii/S0925231221001545
% https://openaccess.thecvf.com/content/ICCV2023/html/Jin_Growing_a_Brain_with_Sparsity-Inducing_Generation_for_Continual_Learning_ICCV_2023_paper.html
% https://proceedings.neurips.cc/paper/2021/hash/f1e709e6aef16ba2f0cd6c7e4f52b9b6-Abstract.html
% https://arxiv.org/abs/2401.05667
% https://openaccess.thecvf.com/content_CVPR_2019/html/Ostapenko_Learning_to_Remember_A_Synaptic_Plasticity_Driven_Framework_for_Continual_CVPR_2019_paper.html
% https://arxiv.org/abs/2206.09117
% https://arxiv.org/abs/2403.08763
% https://arxiv.org/abs/1803.03635
% https://arxiv.org/abs/2406.05955
% https://arxiv.org/abs/2402.01089
% https://arxiv.org/abs/2202.12002
% https://arxiv.org/abs/2404.16789
% https://arxiv.org/abs/2406.17245
% https://arxiv.org/abs/2205.12393
% https://openaccess.thecvf.com/content/CVPR2024/html/Roy_Convolutional_Prompting_meets_Language_Models_for_Continual_Learning_CVPR_2024_paper.html
% https://arxiv.org/abs/2110.03215
% https://arxiv.org/abs/2405.18653
% https://arxiv.org/abs/2012.15283
% https://arxiv.org/abs/2205.09357
% https://arxiv.org/abs/2301.12314
% https://aclanthology.org/2020.coling-main.318/
% https://arxiv.org/abs/2401.03129
% https://arxiv.org/abs/2302.03241
% https://arxiv.org/abs/2309.14763
% https://arxiv.org/abs/2406.06962
% https://arxiv.org/abs/2406.01392
% https://arxiv.org/abs/2404.00790
% https://arxiv.org/abs/2406.18708
% https://arxiv.org/abs/2406.01375
% https://www.cell.com/trends/cognitive-sciences/fulltext/S1364-66132030219-9
% https://openreview.net/forum?id=DimPeeCxKO
% https://arxiv.org/pdf/1612.00796


% New List: ongoing check of ICLR'24 and Nips'24:
% https://openreview.net/pdf?id=8SDsff42Lj
% https://arxiv.org/pdf/2408.17394
% https://arxiv.org/pdf/2410.20444
% https://cdn.iiit.ac.in/cdn/precog.iiit.ac.in/pubs/NeurIPS-RanDumb.pdf
% https://arxiv.org/pdf/2407.16920v1
% https://arxiv.org/pdf/2405.20236
% https://arxiv.org/pdf/2410.22658
% https://arxiv.org/pdf/2406.18868
% https://arxiv.org/pdf/2403.19137
% https://arxiv.org/pdf/2411.02175
% https://arxiv.org/pdf/2410.23843
% https://arxiv.org/pdf/2404.07470
% https://arxiv.org/pdf/2404.00781
% https://arxiv.org/pdf/2403.13249
% https://arxiv.org/pdf/2303.11863
% https://arxiv.org/pdf/2311.04661
% https://arxiv.org/pdf/2310.02129




% Interesting work:
% https://arxiv.org/pdf/2403.19137 
%The paper~\citep{jha2024clap4clip} tackles class-incremental continual learning with CLIP, where new groups of classes (tasks) arrive sequentially and the model must learn to classify them without forgetting previous classes. For example, in CIFAR100, they split 100 classes into 10 tasks of 10 classes each. Building on previous work that showed CLIP needs task-specific finetuning (either through learned prompts like CoOp or feature adapters like CLIP-Adapter), they identify a key problem: during continual learning, the adapted text features increasingly deviate from CLIP's frozen image features (measured by increasing angular distance), hurting performance. Their solution has three main components: (1) a visual-guided attention module that helps maintain alignment between visual and text features during adaptation, (2) task-specific probabilistic adapters that learn distributions over text features rather than point estimates, and (3) using CLIP's pre-trained language knowledge to initialize and regularize these adapters. They show this approach reduces cross-modal deviation and outperforms previous methods across multiple datasets. While simpler solutions might be possible (like directly enforcing alignment or transforming frozen features), the paper takes a more complex probabilistic approach that enables additional benefits like uncertainty estimation.

%https://nips.cc/virtual/2024/poster/94505
% Super interesting work, one of the closest to ours: shows that 

%https://arxiv.org/pdf/2404.00781
% Our work closely relates to the recent UPGD (Utility-based Perturbed Gradient Descent) approach [https://arxiv.org/pdf/2404.00781], which similarly investigates plasticity/stability mechanisms for continual learning in deep learning more generally. UPGD proposes a utility-based approach that protects important weights while allowing updates to less useful ones, combining gradient updates with perturbations based on a carefully approximated utility measure. While both works share similar motivations and empirical findings about the importance of selective plasticity, our work specifically focuses on cognitive dissonance awareness and the particular challenges of conflicting updates in language models. Our findings complement theirs by demonstrating that while non-conflicting updates can be handled relatively well with various strategies (including simple fine-tuning), conflicting updates pose a unique challenge that requires explicit conflict detection and resolution mechanisms. Additionally, while UPGD uses a theoretically-grounded utility approximation to guide updates, we show that even simple gradient-based historical tracking can effectively identify appropriate regions for targeted updates in practice.

% \begin{comment}
% \subsection{Raw list of continual learning}
% % does not compile wtf
% \begin{itemize}
%     \item \href{https://proceedings.neurips.cc/paper/2021/hash/2a10665525774fa2501c2c8c4985ce61-Abstract.html}{https://proceedings.neurips.cc/paper/2021/hash/2a10665525774fa2501c2c8c4985ce61-Abstract.html}
%     \item \href{https://proceedings.neurips.cc/paper_files/paper/2022/hash/80133d0f6eccaace15508f91e3c5a93c-Abstract-Conference.html}{https://proceedings.neurips.cc/paper_files/paper/2022/hash/80133d0f6eccaace15508f91e3c5a93c-Abstract-Conference.html}
%     \item \href{https://arxiv.org/abs/1903.04476}{https://arxiv.org/abs/1903.04476}
%     \item \href{https://proceedings.mlr.press/v199/abbasi22a}{https://proceedings.mlr.press/v199/abbasi22a}
%     \item \href{https://openaccess.thecvf.com/content/CVPR2022/html/Yan_Learning_Bayesian_Sparse_Networks_With_Full_Experience_Replay_for_Continual_CVPR_2022_paper.html}{https://openaccess.thecvf.com/content/CVPR2022/html/Yan_Learning_Bayesian_Sparse_Networks_With_Full_Experience_Replay_for_Continual_CVPR_2022_paper.html}
%     \item \href{https://proceedings.neurips.cc/paper/2020/hash/258be18e31c8188555c2ff05b4d542c3-Abstract.html}{https://proceedings.neurips.cc/paper/2020/hash/258be18e31c8188555c2ff05b4d542c3-Abstract.html}
%     \item \href{https://proceedings.mlr.press/v234/yildirim24a.html}{https://proceedings.mlr.press/v234/yildirim24a.html}
%     \item \href{https://www.sciencedirect.com/science/article/pii/S0925231221001545}{https://www.sciencedirect.com/science/article/pii/S0925231221001545}
%     \item \href{https://openaccess.thecvf.com/content/ICCV2023/html/Jin_Growing_a_Brain_with_Sparsity-Inducing_Generation_for_Continual_Learning_ICCV_2023_paper.html}{https://openaccess.thecvf.com/content/ICCV2023/html/Jin_Growing_a_Brain_with_Sparsity-Inducing_Generation_for_Continual_Learning_ICCV_2023_paper.html}
%     \item \href{https://proceedings.neurips.cc/paper/2021/hash/f1e709e6aef16ba2f0cd6c7e4f52b9b6-Abstract.html}{https://proceedings.neurips.cc/paper/2021/hash/f1e709e6aef16ba2f0cd6c7e4f52b9b6-Abstract.html}
%     \item \href{https://arxiv.org/abs/2401.05667}{https://arxiv.org/abs/2401.05667}
%     \item \href{https://openaccess.thecvf.com/content_CVPR_2019/html/Ostapenko_Learning_to_Remember_A_Synaptic_Plasticity_Driven_Framework_for_Continual_CVPR_2019_paper.html}{https://openaccess.thecvf.com/content_CVPR_2019/html/Ostapenko_Learning_to_Remember_A_Synaptic_Plasticity_Driven_Framework_for_Continual_CVPR_2019_paper.html}
%     \item \href{https://arxiv.org/abs/2206.09117}{https://arxiv.org/abs/2206.09117}
%     \item \href{https://arxiv.org/abs/2403.08763}{https://arxiv.org/abs/2403.08763}
%     \item \href{https://arxiv.org/abs/1803.03635}{https://arxiv.org/abs/1803.03635}
%     \item \href{https://arxiv.org/abs/2406.05955}{https://arxiv.org/abs/2406.05955}
%     \item \href{https://arxiv.org/abs/2402.01089}{https://arxiv.org/abs/2402.01089}
%     \item \href{https://arxiv.org/abs/2202.12002}{https://arxiv.org/abs/2202.12002}
%     \item \href{https://arxiv.org/abs/2404.16789}{https://arxiv.org/abs/2404.16789}
%     \item \href{https://arxiv.org/abs/2406.17245}{https://arxiv.org/abs/2406.17245}
%     \item \href{https://arxiv.org/abs/2205.12393}{https://arxiv.org/abs/2205.12393}
%     \item \href{https://openaccess.thecvf.com/content/CVPR2024/html/Roy_Convolutional_Prompting_meets_Language_Models_for_Continual_Learning_CVPR_2024_paper.html}{https://openaccess.thecvf.com/content/CVPR2024/html/Roy_Convolutional_Prompting_meets_Language_Models_for_Continual_Learning_CVPR_2024_paper.html}
%     \item \href{https://arxiv.org/abs/2110.03215}{https://arxiv.org/abs/2110.03215}
%     \item \href{https://arxiv.org/abs/2405.18653}{https://arxiv.org/abs/2405.18653}
%     \item \href{https://arxiv.org/abs/2012.15283}{https://arxiv.org/abs/2012.15283}
%     \item \href{https://arxiv.org/abs/2205.09357}{https://arxiv.org/abs/2205.09357}
%     \item \href{https://arxiv.org/abs/2301.12314}{https://arxiv.org/abs/2301.12314}
%     \item \href{https://aclanthology.org/2020.coling-main.318/}{https://aclanthology.org/2020.coling-main.318/}
%     \item \href{https://arxiv.org/abs/2401.03129}{https://arxiv.org/abs/2401.03129}
%     \item \href{https://arxiv.org/abs/2302.03241}{https://arxiv.org/abs/2302.03241}
%     \item \href{https://arxiv.org/abs/2309.14763}{https://arxiv.org/abs/2309.14763}
%     \item \href{https://arxiv.org/abs/2406.06962}{https://arxiv.org/abs/2406.06962}
%     \item \href{https://arxiv.org/abs/2406.01392}{https://arxiv.org/abs/2406.01392}
%     \item \href{https://arxiv.org/abs/2404.00790}{https://arxiv.org/abs/2404.00790}
%     \item \href{https://arxiv.org/abs/2406.18708}{https://arxiv.org/abs/2406.18708}
%     \item \href{https://arxiv.org/abs/2406.01375}{https://arxiv.org/abs/2406.01375}
%     \item \href{https://www.cell.com/trends/cognitive-sciences/fulltext/S1364-66132030219-9}{https://www.cell.com/trends/cognitive-sciences/fulltext/S1364-66132030219-9}
%     \item \href{https://openreview.net/forum?id=DimPeeCxKO}{https://openreview.net/forum?id=DimPeeCxKO}
%     \item \href{https://arxiv.org/pdf/1612.00796}{Elastic weight consolidation deepmind 2017} **Elastic Weight Consolidation (EWC)** is a method designed to mitigate catastrophic forgetting in neural networks trained sequentially on distinct tasks. The core idea is to protect the most important weights (or neurons) for previously learned tasks during the training of new tasks. EWC identifies these important weights by calculating the **Fisher Information Matrix** during or after the training of a task, which estimates how sensitive each weight is to the task’s performance. Weights that significantly impact the output for a given task are marked as important. A **quadratic penalty** is then applied during future learning, constraining these weights to remain close to their values from the previous task. This ensures that knowledge from earlier tasks is preserved while still allowing the model to adapt to new tasks.

% However, **EWC is less suitable for large language models (LLMs)**, which are typically trained on massive datasets that do not have clearly defined tasks. EWC's effectiveness relies on distinct task boundaries and the ability to compute task-specific importance for weights, which is feasible in scenarios with well-defined tasks, such as classification or reinforcement learning. In LLMs, where learning spans a wide range of topics and linguistic structures without clear task delineation, it’s challenging to apply EWC's task-based strategy. The model would struggle to assign specific neurons or weights to individual tasks or concepts, making it difficult to protect task-specific knowledge without hindering the model’s overall generalization ability across a diverse dataset.
% \end{itemize}

% % https://proceedings.neurips.cc/paper/2021/hash/2a10665525774fa2501c2c8c4985ce61-Abstract.html
% % https://proceedings.neurips.cc/paper_files/paper/2022/hash/80133d0f6eccaace15508f91e3c5a93c-Abstract-Conference.html
% % https://arxiv.org/abs/1903.04476
% % https://proceedings.mlr.press/v199/abbasi22a
% % https://openaccess.thecvf.com/content/CVPR2022/html/Yan_Learning_Bayesian_Sparse_Networks_With_Full_Experience_Replay_for_Continual_CVPR_2022_paper.html
% % https://proceedings.neurips.cc/paper/2020/hash/258be18e31c8188555c2ff05b4d542c3-Abstract.html
% % https://proceedings.mlr.press/v234/yildirim24a.html
% % https://www.sciencedirect.com/science/article/pii/S0925231221001545
% % https://openaccess.thecvf.com/content/ICCV2023/html/Jin_Growing_a_Brain_with_Sparsity-Inducing_Generation_for_Continual_Learning_ICCV_2023_paper.html
% % https://proceedings.neurips.cc/paper/2021/hash/f1e709e6aef16ba2f0cd6c7e4f52b9b6-Abstract.html
% % https://arxiv.org/abs/2401.05667
% % https://openaccess.thecvf.com/content_CVPR_2019/html/Ostapenko_Learning_to_Remember_A_Synaptic_Plasticity_Driven_Framework_for_Continual_CVPR_2019_paper.html
% % https://arxiv.org/abs/2206.09117
% % https://arxiv.org/abs/2403.08763
% % https://arxiv.org/abs/1803.03635
% % https://arxiv.org/abs/2406.05955
% % https://arxiv.org/abs/2402.01089
% % https://arxiv.org/abs/2202.12002
% % https://arxiv.org/abs/2404.16789
% % https://arxiv.org/abs/2406.17245
% % https://arxiv.org/abs/2205.12393
% % https://openaccess.thecvf.com/content/CVPR2024/html/Roy_Convolutional_Prompting_meets_Language_Models_for_Continual_Learning_CVPR_2024_paper.html
% % https://arxiv.org/abs/2110.03215
% % https://arxiv.org/abs/2405.18653
% % https://arxiv.org/abs/2012.15283
% % https://arxiv.org/abs/2205.09357
% % https://arxiv.org/abs/2301.12314
% % https://aclanthology.org/2020.coling-main.318/
% % https://arxiv.org/abs/2401.03129
% % https://arxiv.org/abs/2302.03241
% % https://arxiv.org/abs/2309.14763
% % https://arxiv.org/abs/2406.06962
% % https://arxiv.org/abs/2406.01392
% % https://arxiv.org/abs/2404.00790
% % https://arxiv.org/abs/2406.18708
% % https://arxiv.org/abs/2406.01375
% % https://www.cell.com/trends/cognitive-sciences/fulltext/S1364-66132030219-9
% % https://openreview.net/forum?id=DimPeeCxKO
% % https://arxiv.org/pdf/1612.00796


% \subsection{Raw list of continual learning}
% % does not compile wtf
% \begin{itemize}
%     \item \href{https://proceedings.neurips.cc/paper/2021/hash/2a10665525774fa2501c2c8c4985ce61-Abstract.html}{https://proceedings.neurips.cc/paper/2021/hash/2a10665525774fa2501c2c8c4985ce61-Abstract.html}
%     \item \href{https://proceedings.neurips.cc/paper_files/paper/2022/hash/80133d0f6eccaace15508f91e3c5a93c-Abstract-Conference.html}{https://proceedings.neurips.cc/paper_files/paper/2022/hash/80133d0f6eccaace15508f91e3c5a93c-Abstract-Conference.html}
%     \item \href{https://arxiv.org/abs/1903.04476}{https://arxiv.org/abs/1903.04476}
%     \item \href{https://proceedings.mlr.press/v199/abbasi22a}{https://proceedings.mlr.press/v199/abbasi22a}
%     \item \href{https://openaccess.thecvf.com/content/CVPR2022/html/Yan_Learning_Bayesian_Sparse_Networks_With_Full_Experience_Replay_for_Continual_CVPR_2022_paper.html}{https://openaccess.thecvf.com/content/CVPR2022/html/Yan_Learning_Bayesian_Sparse_Networks_With_Full_Experience_Replay_for_Continual_CVPR_2022_paper.html}
%     \item \href{https://proceedings.neurips.cc/paper/2020/hash/258be18e31c8188555c2ff05b4d542c3-Abstract.html}{https://proceedings.neurips.cc/paper/2020/hash/258be18e31c8188555c2ff05b4d542c3-Abstract.html}
%     \item \href{https://proceedings.mlr.press/v234/yildirim24a.html}{https://proceedings.mlr.press/v234/yildirim24a.html}
%     \item \href{https://www.sciencedirect.com/science/article/pii/S0925231221001545}{https://www.sciencedirect.com/science/article/pii/S0925231221001545}
%     \item \href{https://openaccess.thecvf.com/content/ICCV2023/html/Jin_Growing_a_Brain_with_Sparsity-Inducing_Generation_for_Continual_Learning_ICCV_2023_paper.html}{https://openaccess.thecvf.com/content/ICCV2023/html/Jin_Growing_a_Brain_with_Sparsity-Inducing_Generation_for_Continual_Learning_ICCV_2023_paper.html}
%     \item \href{https://proceedings.neurips.cc/paper/2021/hash/f1e709e6aef16ba2f0cd6c7e4f52b9b6-Abstract.html}{https://proceedings.neurips.cc/paper/2021/hash/f1e709e6aef16ba2f0cd6c7e4f52b9b6-Abstract.html}
%     \item \href{https://arxiv.org/abs/2401.05667}{https://arxiv.org/abs/2401.05667}
%     \item \href{https://openaccess.thecvf.com/content_CVPR_2019/html/Ostapenko_Learning_to_Remember_A_Synaptic_Plasticity_Driven_Framework_for_Continual_CVPR_2019_paper.html}{https://openaccess.thecvf.com/content_CVPR_2019/html/Ostapenko_Learning_to_Remember_A_Synaptic_Plasticity_Driven_Framework_for_Continual_CVPR_2019_paper.html}
%     \item \href{https://arxiv.org/abs/2206.09117}{https://arxiv.org/abs/2206.09117}
%     \item \href{https://arxiv.org/abs/2403.08763}{https://arxiv.org/abs/2403.08763}
%     \item \href{https://arxiv.org/abs/1803.03635}{https://arxiv.org/abs/1803.03635}
%     \item \href{https://arxiv.org/abs/2406.05955}{https://arxiv.org/abs/2406.05955}
%     \item \href{https://arxiv.org/abs/2402.01089}{https://arxiv.org/abs/2402.01089}
%     \item \href{https://arxiv.org/abs/2202.12002}{https://arxiv.org/abs/2202.12002}
%     \item \href{https://arxiv.org/abs/2404.16789}{https://arxiv.org/abs/2404.16789}
%     \item \href{https://arxiv.org/abs/2406.17245}{https://arxiv.org/abs/2406.17245}
%     \item \href{https://arxiv.org/abs/2205.12393}{https://arxiv.org/abs/2205.12393}
%     \item \href{https://openaccess.thecvf.com/content/CVPR2024/html/Roy_Convolutional_Prompting_meets_Language_Models_for_Continual_Learning_CVPR_2024_paper.html}{https://openaccess.thecvf.com/content/CVPR2024/html/Roy_Convolutional_Prompting_meets_Language_Models_for_Continual_Learning_CVPR_2024_paper.html}
%     \item \href{https://arxiv.org/abs/2110.03215}{https://arxiv.org/abs/2110.03215}
%     \item \href{https://arxiv.org/abs/2405.18653}{https://arxiv.org/abs/2405.18653}
%     \item \href{https://arxiv.org/abs/2012.15283}{https://arxiv.org/abs/2012.15283}
%     \item \href{https://arxiv.org/abs/2205.09357}{https://arxiv.org/abs/2205.09357}
%     \item \href{https://arxiv.org/abs/2301.12314}{https://arxiv.org/abs/2301.12314}
%     \item \href{https://aclanthology.org/2020.coling-main.318/}{https://aclanthology.org/2020.coling-main.318/}
%     \item \href{https://arxiv.org/abs/2401.03129}{https://arxiv.org/abs/2401.03129}
%     \item \href{https://arxiv.org/abs/2302.03241}{https://arxiv.org/abs/2302.03241}
%     \item \href{https://arxiv.org/abs/2309.14763}{https://arxiv.org/abs/2309.14763}
%     \item \href{https://arxiv.org/abs/2406.06962}{https://arxiv.org/abs/2406.06962}
%     \item \href{https://arxiv.org/abs/2406.01392}{https://arxiv.org/abs/2406.01392}
%     \item \href{https://arxiv.org/abs/2404.00790}{https://arxiv.org/abs/2404.00790}
%     \item \href{https://arxiv.org/abs/2406.18708}{https://arxiv.org/abs/2406.18708}
%     \item \href{https://arxiv.org/abs/2406.01375}{https://arxiv.org/abs/2406.01375}
%     \item \href{https://www.cell.com/trends/cognitive-sciences/fulltext/S1364-66132030219-9}{https://www.cell.com/trends/cognitive-sciences/fulltext/S1364-66132030219-9}
%     \item \href{https://openreview.net/forum?id=DimPeeCxKO}{https://openreview.net/forum?id=DimPeeCxKO}
% \end{itemize}

% \end{comment}


\section{Extraction of historical activations and gradients}\label{app:notation:extraction}
We here detail our procedure for the extraction of activations and gradients. Source code is also available at \url{https://github.com/bendiogene/ConflictAwareLLM/} for ultimate level of details and reproducibility purposes. 

\subsection{Preliminary notation}

We focus on the historical tracking of gradients of the outputs (grad\_outs) and activations for four key matrices within each block of the transformer model: 
\(\text{Attn}_{\text{c\_attn}}\), \(\text{Attn}_{\text{c\_proj}}\), \(\text{MLP}_{\text{c\_fc}}\), and \(\text{MLP}_{\text{c\_proj}}\).

Given an input sequence \( X \in \mathbb{R}^{B \times N \times d_{\text{model}}} \), where \( B \) is the batch size, \( N \) is the sequence length, and \( d_{\text{model}} \) is the model dimension, the transformer block is defined as follows:


\paragraph{Attention Layer:}
The attention mechanism computes query \( Q \), key \( K \), and value \( V \) matrices:
\[
Q = XW_Q, \quad K = XW_K, \quad V = XW_V
\]
where \( W_Q \in \mathbb{R}^{d_{\text{model}} \times d_{\text{key}}} \), \( W_K \in \mathbb{R}^{d_{\text{model}} \times d_{\text{key}}} \), and \( W_V \in \mathbb{R}^{d_{\text{model}} \times d_{\text{value}}} \) are trainable projection matrices.

The concatenated matrix \(\text{Attn}_{\text{c\_attn}}\) is:
\[
\text{Attn}_{\text{c\_attn}} = [Q, K, V] = XW_{\text{attn}}
\]
where \( W_{\text{attn}} = [W_Q, W_K, W_V] \in \mathbb{R}^{d_{\text{model}} \times (2d_{\text{key}} + d_{\text{value}})} \).

The attention context \(\text{Attn}_{\text{context}}\) is computed as:
\[
\text{Attn}_{\text{context}} = \text{softmax} \left( \frac{QK^T}{\sqrt{d_{\text{key}}}} \right) V
\]

The projected attention output \(\text{Attn}_{\text{c\_proj}}\) is:
\[
\text{Attn}_{\text{c\_proj}} = \text{Attn}_{\text{context}} W_{\text{proj}}
\]
where \( W_{\text{proj}} \in \mathbb{R}^{d_{\text{value}} \times d_{\text{model}}} \).

\paragraph{MLP Layer:}
The MLP layer consists of two linear transformations with an activation function \( \sigma \):
\[
\text{MLP}_{\text{c\_fc}} = \sigma (XW_{\text{fc}} + b_{\text{fc}})
\]
where \( W_{\text{fc}} \in \mathbb{R}^{d_{\text{model}} \times d_{\text{ff}}} \) and \( b_{\text{fc}} \in \mathbb{R}^{d_{\text{ff}}} \).

The projected MLP output \(\text{MLP}_{\text{c\_proj}}\) is:
\[
\text{MLP}_{\text{c\_proj}} = \text{MLP}_{\text{c\_fc}} W_{\text{proj}} + b_{\text{proj}}
\]
where \( W_{\text{proj}} \in \mathbb{R}^{d_{\text{ff}} \times d_{\text{model}}} \) and \( b_{\text{proj}} \in \mathbb{R}^{d_{\text{model}}} \).

\subsection{Historical gradient and activation collection}
Collecting a profile of neuron activity during training or simulation of training is needed as (i) input feature to know if a fact is dissonant, novel or known, and (ii) as means to identify where to locate targeted updates.

During training, we collect and cumulate the gradients of the outputs (grad\_outs) and activations for the matrices \(\text{Attn}_{\text{c\_attn}}\), \(\text{Attn}_{\text{c\_proj}}\), \(\text{MLP}_{\text{c\_fc}}\), and \(\text{MLP}_{\text{c\_proj}}\).
Let \( t \) denote the training step.
We collect activations at step $t$:
\[\text{Attn}_{\text{c\_attn}}(t), \text{Attn}_{\text{c\_proj}}(t), \text{MLP}_{\text{c\_fc}}(t), \text{MLP}_{\text{c\_proj}}(t)\]
\noindent as well as 
Gradient of the Outputs (grad\_outs) at step $t$ : \[  \nabla L(\text{Attn}_{\text{c\_attn}}(t)), \nabla L(\text{Attn}_{\text{c\_proj}}(t)), \nabla L(\text{MLP}_{\text{c\_fc}}(t)), \nabla L(\text{MLP}_{\text{c\_proj}}(t))
    \]

In the remainder, we denote these, regardless of their provenance matrix, as:
\[A^l(t), G^l(t) \in \mathbb{R}^{B \times N \times d^l_{\text{out}}}\]
where \(l\) denotes the layer, \(B\) is the batch size, \(N\) is the sequence length, and \(d^l_{\text{out}}\) is the output dimension of layer \(l\).

When needed, we standardize these metrics for each layer \( l \) as follows:
\[
\hat{A}^l(t) = \frac{A^l(t) - \mu_A^l(t)}{\sigma_A^l(t)}, \quad \hat{G}^l(t) = \frac{G^l(t) - \mu_G^l(t)}{\sigma_G^l(t)}
\]
where \( \mu \) and \( \sigma \) are the mean and standard deviation computed over all dimensions of the respective tensor.

We then sum over the batch dimension:
\[
S^l_{\hat{A}}(t)_{n,i} = \sum_{b=1}^{B} \hat{A}^l_{b,n,i}(t), \quad S^l_{\hat{G}}(t)_{n,i} = \sum_{b=1}^{B} \hat{G}^l_{b,n,i}(t)
\]


Optionally\footnote{We consider two approaches. In the first, we extract the activations and gradients corresponding to the last token (i.e., position \( N \)) in the sequence for each sample in the batch. This is reasonable since the last token is representative of the fact or information of interest in our datasets. In the second, we simply aggregate over all tokens, where we aggregate activations and gradients across all tokens in the sequence by computing statistical measures such as the mean or sum over the token dimension.}, we can sum over the token dimension:

\[
S^l_{\hat{A}}(t)_i = \sum_{n=1}^{N} S^l_{\hat{A}}(t)_{n,i}, \quad S^l_{\hat{G}}(t)_i = \sum_{n=1}^{N} S^l_{\hat{G}}(t)_{n,i}
\]


The standardized and summed metrics are then accumulated across the training steps:
% \[
% H\hat{A}^l = \sum_{t=1}^T S^l_{\hat{A}}(t), \quad H\hat{G}^l = \sum_{t=1}^T S^l_{\hat{G}}(t)
% \]
\[
H\hat{A}^l_i = \sum_{t=1}^{T} S^l_{\hat{A}}(t)_i, \quad H\hat{G}^l_i = \sum_{t=1}^{T} S^l_{\hat{G}}(t)_i
\]
where \( T \) is the total number of training steps.

These historical activations \( H\hat{A}^l \) and gradients \( H\hat{G}^l \) provide cumulative measures of neuron activity over the training process. They help identify neurons that are heavily utilized (stubborn neurons) and those that are underutilized (plastic neurons), which is crucial for our targeted updates.
%%%%%%%%%%%%%%%% OLD %%%%%%%%%%%%%%%%%%%%%%%%%%%%%%%%%%
% \subsection{Preliminary notation}

% We focus on the historical tracking of gradients of the outputs (grad\_outs) and activations for four key matrices within each block of the transformer model: 
% \(\text{Attn}_{\text{c\_attn}}\), \(\text{Attn}_{\text{c\_proj}}\), \(\text{MLP}_{\text{c\_fc}}\), and \(\text{MLP}_{\text{c\_proj}}\).

% Given an input sequence \( X \in \mathbb{R}^{N \times d_{\text{model}}} \), where \( N \) is the sequence length and \( d_{\text{model}} \) is the model dimension, the transformer block is defined as follows:

% \paragraph{Attention Layer:}
% The attention mechanism computes query \( Q \), key \( K \), and value \( V \) matrices:
% \[
% Q = XW_Q, \quad K = XW_K, \quad V = XW_V
% \]
% where \( W_Q \in \mathbb{R}^{d_{\text{model}} \times d_{\text{key}}} \), \( W_K \in \mathbb{R}^{d_{\text{model}} \times d_{\text{key}}} \), and \( W_V \in \mathbb{R}^{d_{\text{model}} \times d_{\text{value}}} \) are trainable projection matrices.

% The concatenated matrix \(\text{Attn}_{\text{c\_attn}}\) is:
% \[
% \text{Attn}_{\text{c\_attn}} = [Q, K, V] = XW_{\text{attn}}
% \]
% where \( W_{\text{attn}} = [W_Q, W_K, W_V] \in \mathbb{R}^{d_{\text{model}} \times (2d_{\text{key}} + d_{\text{value}})} \).

% The attention context \(\text{Attn}_{\text{context}}\) is computed as:
% \[
% \text{Attn}_{\text{context}} = \text{softmax} \left( \frac{QK^T}{\sqrt{d_{\text{key}}}} \right) V
% \]

% The projected attention output \(\text{Attn}_{\text{c\_proj}}\) is:
% \[
% \text{Attn}_{\text{c\_proj}} = \text{Attn}_{\text{context}} W_{\text{proj}}
% \]
% where \( W_{\text{proj}} \in \mathbb{R}^{d_{\text{value}} \times d_{\text{model}}} \).

% \paragraph{MLP Layer:}
% The MLP layer consists of two linear transformations with an activation function \( \sigma \):
% \[
% \text{MLP}_{\text{c\_fc}} = \sigma (XW_{\text{fc}} + b_{\text{fc}})
% \]
% where \( W_{\text{fc}} \in \mathbb{R}^{d_{\text{model}} \times d_{\text{ff}}} \) and \( b_{\text{fc}} \in \mathbb{R}^{d_{\text{ff}}} \).

% The projected MLP output \(\text{MLP}_{\text{c\_proj}}\) is:
% \[
% \text{MLP}_{\text{c\_proj}} = \text{MLP}_{\text{c\_fc}} W_{\text{proj}} + b_{\text{proj}}
% \]
% where \( W_{\text{proj}} \in \mathbb{R}^{d_{\text{ff}} \times d_{\text{model}}} \) and \( b_{\text{proj}} \in \mathbb{R}^{d_{\text{model}}} \).



% \subsection{Historical gradient and activation collection}
% Collecting a profile of neuron activity during training or simulation of training is needed as (i) input feature to know if a fact is dissonant, novel or known, and (ii) as means to identify where to locate targetted updates.

% During training, we collect and cumulate the gradients of the outputs (grad\_outs) and activations for the matrices \(\text{Attn}_{\text{c\_attn}}\), \(\text{Attn}_{\text{c\_proj}}\), \(\text{MLP}_{\text{c\_fc}}\), and \(\text{MLP}_{\text{c\_proj}}\).
% Let \( t \) denote the training step.
% We collect activations at step $t$:
% \[\text{Attn}_{\text{c\_attn}}(t), \text{Attn}_{\text{c\_proj}}(t), \text{MLP}_{\text{c\_fc}}(t), \text{MLP}_{\text{c\_proj}}(t)\]
% \noindent as well as 
% Gradient of the Outputs (grad\_outs) at step $t$  t: \[  \nabla L(\text{Attn}_{\text{c\_attn}}(t)), \nabla L(\text{Attn}_{\text{c\_proj}}(t)), \nabla L(\text{MLP}_{\text{c\_fc}}(t)), \nabla L(\text{MLP}_{\text{c\_proj}}(t))
%     \]


% When needed, we standardize these metrics for each layer \( l \) as follows:
% \[
% \hat{A}_n^l = \frac{A_n^l - \mu_A^l}{\sigma_A^l}, \quad \hat{G}_n^l = \frac{G_n^l - \mu_G^l}{\sigma_G^l}
% \]
% where \( \mu_A^l \) and \( \sigma_A^l \) are the mean and standard deviation of activations in layer \( l \), and \( \mu_G^l \) and \( \sigma_G^l \) are the mean and standard deviation of gradients in layer \( l \).

% The standardized metrics are then summed across the training steps to obtain historical activations and historical gradients:
% \[
% H\hat{A}_n = \sum_{t=1}^T \hat{A}_n^l, \quad H\hat{G}_n = \sum_{t=1}^T \hat{G}_n^l
% \]

% This historical data should provide us with the necessary information to know if neurons and weights are rather ``stubborn'' because they underwent heavy changes in the past (and so are busy storing existing knowledge) or ``plastic'' (and thus good candidates for storing new knowledge).



\section{Dissonance awareness}\label{app:dissonance}

\subsection{Augmenting the COUNTERFACT Dataset with Novel facts}
\label{appendix:unknown_facts_prompt}

To generate unknown facts to augment the Counterfact dataset, we used GPT-3.5 with a prompt as follows:

\begin{lstlisting}[language=,frame=single]
Starting from this list of facts, can you create one data entry for each that concerns imaginary names and characters if necessary, while following the same logic.

For example, Danielle Darrieux's mother tongue is French => Becomes Machin De Machine's mother tongue is Kurdi (or Kinduli).

Edwin of Northumbria's religious values strongly emphasize Christianity => Hamed Habib's religious values strongly emphasize Atheism (or Peace or..)

Try to make the old and new as far as possible from each other (e.g., Kurdi is far from French, Kinduli is an imaginary language, etc.), while keeping some logic.

Write in JSON format, please (easy to parse):

- Danielle Darrieux's mother tongue is French
- Edwin of Northumbria's religious values strongly emphasize Christianity
- Toko Yasuda produces the most amazing music on the guitar
- One can get to Autonomous University of Madrid by navigating Spain
- Thomas Joannes Stieltjes was born in Dutch
- Anaal Nathrakh originated from Birmingham
\end{lstlisting}

\paragraph{Example Generated Transformations:}

\begin{itemize}
    \item Original: \emph{``Toko Yasuda produces the most amazing music on the guitar.''}

    Transformed: \emph{``Zara Zorin produces the most amazing music on the theremin.''}
    \item Original: \emph{``One can get to Autonomous University of Madrid by navigating Spain.''}

    Transformed: \emph{``One can reach the Floating Academia of Zephyria by navigating through the Cloud Realms.''}
    \item Original: \emph{``Thomas Joannes Stieltjes was born in Dutch.''}

    Transformed: \emph{``Lorien Ilithar was born amidst the Elvish.''}
\end{itemize}

These transformations help create novel facts unlikely to be known by the model, enabling us to evaluate its ability to handle unknown information effectively.


\subsection{Ablation study of classifier performance}
We further report for the interested reader the results of an ablation study of the dissonance awareness classifier, evaluating its performance under different scenarios (fine-tuned vs. pre-trained models), feature sets (A, G, A+G), normalization strategies (None, Layer, Historical), and classifiers (Random Forests (RF) and Support Vector Machines (SVM)).

Table~\ref{tab:classification_results_appendix} presents a comprehensive set of classification results, including average accuracy and F1 scores (with standard deviations) across different settings. The best results for each classifier are denoted with a $\star$ and reported earlier in Table~\ref{tab:classification_results} in the main paper.


\begin{table}[!t]
\centering
\caption{\textit{Ablation study of dissonance awareness:} Classification Results for Different Scenarios, Feature Sets, Normalization strategies and Classifier. Average (and std) accuracy and F1 scores. $\star$ denotes the best combination for each classifier}
\label{tab:classification_results_appendix}
\resizebox{0.65\textwidth}{!}{%
\begin{tabular}{llllccc}
\toprule
\bf Scenario & \bf Features & \bf Normalization & \bf Classifier & \bf Accuracy & \bf F1 Score \\
\midrule
\multirow{18}{*}{Finetuned} & \multirow{6}{*}{A+G} & \multirow{2}{*}{Null} & SVM & 0.994 (0.004) & 0.994 (0.004) \\
& & & RF$\star$ & 0.988 (0.001) & 0.988 (0.001) \\
\cmidrule(lr){3-6}
& & \multirow{2}{*}{Layer} & SVM & 0.995 (0.001) & 0.995 (0.001) \\
& & & RF & 0.982 (0.005) & 0.982 (0.004) \\
\cmidrule(lr){3-6}
& & \multirow{2}{*}{Historical} & SVM$\star$ & 0.995 (0.001) & 0.995 (0.001) \\
& & & RF & 0.978 (0.003) & 0.978 (0.003) \\
\cmidrule{2-6}
& \multirow{6}{*}{G} & \multirow{2}{*}{Null} & SVM & 0.917 (0.009) & 0.918 (0.009) \\
& & & RF & 0.905 (0.008) & 0.906 (0.008) \\
\cmidrule(lr){3-6}
& & \multirow{2}{*}{Layer} & SVM & 0.920 (0.003) & 0.921 (0.003) \\
& & & RF & 0.895 (0.007) & 0.896 (0.007) \\
\cmidrule(lr){3-6}
& & \multirow{2}{*}{Historical} & SVM & 0.897 (0.004) & 0.898 (0.004) \\
& & & RF & 0.868 (0.014) & 0.870 (0.014) \\
\cmidrule{2-6}
& \multirow{6}{*}{A} & \multirow{2}{*}{Null} & SVM & 0.796 (0.005) & 0.796 (0.007) \\
& & & RF & 0.747 (0.012) & 0.745 (0.016) \\
\cmidrule(lr){3-6}
& & \multirow{2}{*}{Layer} & SVM & 0.783 (0.013) & 0.784 (0.012) \\
& & & RF & 0.722 (0.009) & 0.720 (0.007) \\
\cmidrule(lr){3-6}
& & \multirow{2}{*}{Historical} & SVM & 0.781 (0.009) & 0.781 (0.010) \\
& & & RF & 0.721 (0.010) & 0.719 (0.008) \\
\midrule
\multirow{21}{*}{Pretrained} & \multirow{6}{*}{A+G} & \multirow{2}{*}{Null} & SVM & 0.944 (0.006) & 0.944 (0.006) \\
& & & RF$\star$ & 0.928 (0.012) & 0.929 (0.011) \\
\cmidrule(lr){3-6}
& & \multirow{2}{*}{Layer} & SVM & 0.949 (0.006) & 0.949 (0.006) \\
& & & RF & 0.909 (0.014) & 0.910 (0.013) \\
\cmidrule(lr){3-6}
& & \multirow{2}{*}{Historical} & SVM$\star$ & 0.947 (0.004) & 0.948 (0.003) \\
& & & RF & 0.925 (0.006) & 0.925 (0.006) \\
\cmidrule{2-6}
& \multirow{6}{*}{G} & \multirow{2}{*}{Null} & SVM & 0.904 (0.006) & 0.904 (0.006) \\
& & & RF & 0.891 (0.010) & 0.892 (0.009) \\
\cmidrule(lr){3-6}
& & \multirow{2}{*}{Layer} & SVM & 0.902 (0.008) & 0.902 (0.007) \\
& & & RF & 0.859 (0.013) & 0.861 (0.011) \\
\cmidrule(lr){3-6}
& & \multirow{2}{*}{Historical} & SVM & 0.915 (0.007) & 0.916 (0.006) \\
& & & RF & 0.879 (0.017) & 0.879 (0.016) \\
\cmidrule{2-6}
& \multirow{6}{*}{A} & \multirow{2}{*}{Null} & SVM & 0.909 (0.006) & 0.909 (0.006) \\
& & & RF & 0.894 (0.009) & 0.895 (0.007) \\
\cmidrule(lr){3-6}
& & \multirow{2}{*}{Layer} & SVM & 0.905 (0.012) & 0.905 (0.011) \\
& & & RF & 0.876 (0.004) & 0.877 (0.003) \\
\cmidrule(lr){3-6}
& & \multirow{2}{*}{Historical} & SVM & 0.900 (0.008) & 0.900 (0.007) \\
& & & RF & 0.881 (0.006) & 0.882 (0.006) \\
\bottomrule
\end{tabular}
}
\end{table}
% \begin{table}[h]
% \centering
% \caption{\textit{Ablation study of dissonance awareness:} Classification Results for Different Scenarios, Feature Sets, Normalization strategies and Classifier. Average (and std) accuracy and F1 scores. $\star$ denotes the best combination for each classifier}
% \label{tab:classification_results_appendix}
% \resizebox{\textwidth}{!}{%
% \begin{tabular}{llllccc}
% \toprule
% \bf Scenario & \bf Features & \bf Normalization & \bf Classifier & \bf Accuracy & \bf F1 Score \\
% \midrule
% \multirow{6}{*}{Finetuned} & \multirow{6}{*}{A+G} & \multirow{2}{*}{Null} & SVM & 0.994 (0.004) & 0.994 (0.004) \\
% & & & RF$\star$ & 0.988 (0.001) & 0.988 (0.001) \\
% \cmidrule(lr){3-6}
% & & \multirow{2}{*}{Layer} & SVM & 0.995 (0.001) & 0.995 (0.001) \\
% & & & RF & 0.982 (0.005) & 0.982 (0.004) \\
% \cmidrule(lr){3-6}
% & & \multirow{2}{*}{Historical} & SVM$\star$ & 0.995 (0.001) & 0.995 (0.001) \\
% & & & RF & 0.978 (0.003) & 0.978 (0.003) \\
% \midrule
% \multirow{21}{*}{Pretrained} & \multirow{6}{*}{A+G} & \multirow{2}{*}{Null} & SVM & 0.944 (0.006) & 0.944 (0.006) \\
% & & & RF$\star$ & 0.928 (0.012) & 0.929 (0.011) \\
% \cmidrule(lr){3-6}
% & & \multirow{2}{*}{Layer} & SVM & 0.949 (0.006) & 0.949 (0.006) \\
% & & & RF & 0.909 (0.014) & 0.910 (0.013) \\
% \cmidrule(lr){3-6}
% & & \multirow{2}{*}{Historical} & SVM$\star$ & 0.947 (0.004) & 0.948 (0.003) \\
% & & & RF & 0.925 (0.006) & 0.925 (0.006) \\
% \cmidrule{2-6}
% & \multirow{6}{*}{G} & \multirow{2}{*}{Null} & SVM & 0.904 (0.006) & 0.904 (0.006) \\
% & & & RF & 0.891 (0.010) & 0.892 (0.009) \\
% \cmidrule(lr){3-6}
% & & \multirow{2}{*}{Layer} & SVM & 0.902 (0.008) & 0.902 (0.007) \\
% & & & RF & 0.859 (0.013) & 0.861 (0.011) \\
% \cmidrule(lr){3-6}
% & & \multirow{2}{*}{Historical} & SVM & 0.915 (0.007) & 0.916 (0.006) \\
% & & & RF & 0.879 (0.017) & 0.879 (0.016) \\
% \cmidrule{2-6}
% & \multirow{6}{*}{A} & \multirow{2}{*}{Null} & SVM & 0.909 (0.006) & 0.909 (0.006) \\
% & & & RF & 0.894 (0.009) & 0.895 (0.007) \\
% \cmidrule(lr){3-6}
% & & \multirow{2}{*}{Layer} & SVM & 0.905 (0.012) & 0.905 (0.011) \\
% & & & RF & 0.876 (0.004) & 0.877 (0.003) \\
% \cmidrule(lr){3-6}
% & & \multirow{2}{*}{Historical} & SVM & 0.900 (0.008) & 0.900 (0.007) \\
% & & & RF & 0.881 (0.006) & 0.882 (0.006) \\
% \bottomrule
% \end{tabular}
% }
% \end{table}

\subsection{Explanation of feature importance}\label{app:feature:importance}
To further understand the discriminative power of different features, we analyzed the feature importance scores derived from the RF classifier.

First, as earlier mentioned in Fig.\ref{fig:feature_importance} in the main paper, gradient-based features are substantially more important than activation-based features. This suggests that fine-tuning leads to more discriminative gradients, possibly due to the model overfitting on the known facts, resulting in near-zero gradients for known facts and higher gradients for novel or conflicting facts. In contrast, for the pre-trained model, both activation and gradient features contribute significantly, indicating that combining internal representations and learning dynamics is beneficial for classification.

Complementary to Fig.\ref{fig:feature_importance}, block importance reported in Fig.~\ref{fig:block_importance} reveals that, in the pre-trained model all transformer blocks tend to contribute relatively equally to the classification task, with the last layers contributing less. The finetuned model, on the other hand shows a slightly different tendency where the earlier layers contribute less. More work is clearly needed to understand such differences. This paper focuses only on feasibility of the entire cognitive-dissonance approach, leaving more elaborate evaluations for future work.


\begin{figure}[h]
\centering
\begin{subfigure}{.5\textwidth}
  \centering
  \includegraphics[width=\linewidth]{figures/2_classifier/experiment_1_ft_feature_importance_blocks.pdf}
  \caption{Finetuned model}
\end{subfigure}%
\begin{subfigure}{.5\textwidth}
  \centering
  \includegraphics[width=\linewidth]{figures/2_classifier/experiment_1_pt_feature_importance_blocks.pdf}
  \caption{Pre-trained model}
\end{subfigure}
  \caption{\textit{Block Importance.} Albeit differences are visible, the tendency is not as marked as for the activation vs gradient based feature importance in Fig.\ref{fig:feature_importance} - GPT2-small}
\label{fig:block_importance}
\end{figure}

\subsection{Location of stubborn neurons}
We also report the distribution of stubborn neurons across the transformer blocks in GPT-2 XL. Figures~\ref{fig:stubborn_8000} and~\ref{fig:stubborn_2000} show histograms of the number of stubborn neurons identified in each block for thresholds of 8,000 and 2,000 neurons, respectively.

\begin{figure}[h]
    \centering
    \begin{subfigure}[b]{\textwidth}
        \centering
        \includegraphics[width=\textwidth]{./figures/3_incremental_update/experiment2_1_xl/experiment2_1_xl_stubborn_neurons_stubborn_neurons_histogram_8000.pdf}
        \subcaption{Histogram of stubborn neurons ($t=8000$ neurons) across transformer blocks}
        \label{fig:stubborn_8000}
    \end{subfigure}
    %\hfill
    \begin{subfigure}[b]{\textwidth}
        \centering
        \includegraphics[width=\textwidth]{./figures/3_incremental_update/experiment2_1_xl/experiment2_1_xl_stubborn_neurons_stubborn_neurons_histogram_2000.pdf}
        \subcaption{Histogram of stubborn neurons ($t=2000$ neurons) across transformer blocks}
        \label{fig:stubborn_2000}
    \end{subfigure}
    
    \caption{Distribution of stubborn neurons across GPT2-XL transformer blocks for different neuron thresholds to define stubbornness. (a) shows the distribution for $t=8000$ neurons, while (b) corresponds to $t=2000$ neurons.}
    \label{fig:stubborn_neurons_distribution}
\end{figure}

%\zbh{/home/sclemente/project/epmem\_edit/analysis/plots\_exp\_2\_1\_xl.ipynb}}
Our analysis indicates that stubborn neurons are not uniformly distributed throughout the network. Instead, they curiousy tend to be concentrated in certain blocks, particularly in the first block and in certain middle layers of the transformer. This might suggest that these layers play a more significant role in encoding and retaining knowledge during training. 
Interestingly,  \(\text{Attn}_{\text{c\_attn}}\) concentrates much more of the stubborn neurons overall, with the exception of the first block where \(\text{Attn}_{\text{c\_proj}}\) has a substantially higher share of stubborn neurons.
The results are similar for both thresholds.

Overall, understanding the distribution of stubborn neurons can inform targeted update strategies by identifying which parts of the network are more critical for preserving existing knowledge.

%\clearpage
%--------------------------------------------------
%--------------------------------------------------

\subsection{Using model output (instead of internal state) as features for dissonance awareness}\label{app:diss:aware:prob}
In the main paper, we used activations and gradients as they were \textit{readily available} in our experimental pipeline. We now further test whether using model output only, which is more easily available than internal gradients and activations can achieve similar performance on our scenario. 
%To this end, we perform additional experiments in which we use only model outputs as feature to classify facts.

Each fact in our dataset is conceptually a statement involving a subject (s), relation (r), and object (o) (e.g., ``Danielle Darrieux's mother tongue is French''). In this section, we extract features that capture increasing levels of detail about the model's predictions, related to what the actual facts are, leveraging both:
\begin{itemize}
    \item Conditional probabilities $p(o|s,r)$ at different truncation points\footnote{Since the object $o$ can span multiple tokens, we extract features from the last $N$ tokens of each fact (we pick three, since most answers fit within that limit). For each token position, we compute both the truncated prompt probability $p(o|s,r)$ by removing the token and subsequent tokens, and the full sentence probability $p(s,r,o)$. This multi-token analysis ensures we capture the model's predictions across the entire span of the answer.}
    \item Joint probability $p(s,r,o)$ of the full statement
\end{itemize}

In more details, we extract the following features, with increasing complexity. 

%\begin{itemize}
\noindent \textbf{Basic Token Probabilities ($Feat_1$):}
For each of the last $N$ tokens (representing the answer), we collect the  probability of the actual next token given the truncated prompt. These simple scalar features capture the model's direct confidence in the correct continuation. This has a dimensionality of $N + 1$ ($N$ truncation points plus full statement, so 4 in our case.)


\noindent \textbf{Top-$k$ Predictions Analysis ($Feat_2$):}
Here, for each position in the answer, we collect the values and normalized indices of top-$k$ most likely next tokens. This captures both confidence distribution and ranking patterns. Similarly to the above, we compute this for both truncated prompts and full statements. Here, the dimensionality is $(N + 1) \times 2k$ ($k$ values and $k$ normalized indices for each position). We pick k=100.


\noindent \textbf{Distribution Features ($Feat_3$):} Here, we analyze the complete probability distribution over the vocabulary. For each position in the answer sequence, we construct histograms of the probabilities with $n_{bins}$ bins (here 100), capturing the full spectrum of the model's prediction patterns. We augment these distributions with indicator vectors that highlight the positions of ground truth tokens (the true next tokens of the current truncated fact), providing additional context about the model's accuracy. This  results in a feature vector of dimensionality $(N + 1) \times n_{bins}$.

\noindent \textbf{Combined Features ($Concat$):} Here, we simply concatenate $Feat_1$, $Feat_2$, and $Feat_3$. 

Tab.~\ref{tab:output:prob} shows the results over our dataset. We observe \textit{a similar great performance when using the model outputs, compared to Activations and Gradients}. Model output achieves even better performance in case of pre-trained models. This is inline with our earlier observation that activations (what we're using now) are more important than gradients in the case of pre-trained models. This result is encouraging for future work, where we plan to (i) build more challenging classification datasets (than the simple facts in CounterFact) and (ii) build standalone classifiers to speed up the training of LLMs, by avoiding training on conflicting data.

\begin{table}[h]
\centering
\begin{tabular}{l|cc|cc}
\toprule
\multirow{2}{*}{Strategy (dim)} & \multicolumn{2}{c|}{Pretrained Model} & \multicolumn{2}{c}{Finetuned Model} \\
 & Accuracy & F1-Score & Accuracy & F1-Score \\
\midrule
Feat.1 (4) & 0.852 & 0.856 & 0.850 & 0.855 \\
Feat.2 (800) & 0.602 & 0.588 & 0.600 & 0.581 \\
Feat.3 (400) & 0.540 & 0.452 & 0.543 & 0.464 \\
Concat (1204) & 0.983 & 0.983 & 0.978 & 0.978 \\
(A+G) (240) & 0.947 & 0.948 & 0.995 & 0.995 \\
\bottomrule
\end{tabular}
\caption{Using output-only features for dissonance-awareness can achieve similar good performance to using our readily available activations and gradients, and even better in the case of the pre-trained model.}\label{tab:output:prob}
\end{table}



\section{{\color{customgreen}Non-dissonant} updates}\label{app:update}

\subsection{Similarities with Lottery ticket}\label{app:lottey}

To assess the hypothesis that certain subnetworks within the language model are more conducive to integrating new information—a notion earlier named the lottery ticket hypothesis~\citep{frankle2018lottery}—we designed an experiment to confirm this effect. 

We first trained a model on 10,000 disjoint facts (referred to as Facts H) and identified the most active candidate neurons during this process, which we term \emph{Lottery Ticket Neurons}. These neurons should form a preferred subnetwork for representing Facts H. Next, we started from a \textit{fresh model} and trained on a new set of novel facts (Facts A), which are different from H, restricting updates to three distinct groups of neurons:

\begin{enumerate}
    \item \textbf{Lottery Ticket Neurons}: Neurons highly active during the initial training on Facts H.
    \item \textbf{Non-Lottery Neurons}: Neurons underutilized during the initial training on Facts H.
    \item \textbf{Random Neurons}: Neurons selected randomly from the entire network.
\end{enumerate}

Figure~\ref{fig:lottery} shows the accuracy of acquiring new knowledge when using each of these strategies, with the number of neurons varying from 2,000 to 20,000. Using the Lottery Ticket Neurons led to significantly better performance, reaching nearly 100\% accuracy at 8,000 neurons, compared to around 40\% for the Non-Lottery Neurons. The Random Neurons strategy also performed relatively well, interestingly suggesting that capturing even a few ``anchor'' neurons from the preferred subnetwork is sufficient to achieve good performance.

\begin{figure}
    \centering
    \includegraphics[width=0.48\textwidth]{./figures/3_incremental_update/experiment_2_2/experiment_2_2_1000_neuron_update_strategies_new_knowledge.pdf}
    \caption{Lottery ticket}\label{fig:lottery}
\end{figure}


These results support the existence of preferred subnetworks within the model that are particularly effective for learning new information. Leveraging these subnetworks can enhance the efficiency of knowledge integration while preserving existing knowledge, an aspect that our candidate and specific strategies are already exploiting.

\subsection{Hyperparameter selection: learning rate and batch size for GPT2-XL}\label{app:gptxl:search}
In our experiments, the first step is to conduct a hyperparameter search to determine the optimal learning rates and batch sizes for fine-tuning the model on our facts. Table~\ref{tab:lr:search} presents the performance of  GPT2-XL on old and new knowledge across various learning rates and batch sizes.
\textit{Note that this optimal learning rate for full finetuning might turn out not enough for our targeted updates, since they use, by design, a smaller number of neurons. 
}

\begin{table}[ht]
%\begin{wraptable}{r}{0.4\textwidth}  % "r" for right side; adjust width as needed
\centering
\resizebox{0.4\textwidth}{!}{%
\begin{tabular}{cccc}
\toprule
\textbf{Learning Rate} & \textbf{Batch Size} & \textbf{Epochs} & \textbf{Accuracy} \\ 
\midrule
1e-06                  & 64                  & 5               & 0.271             \\ 
1e-06                  & 64                  & 10              & 0.476             \\ 
1e-06                  & 64                  & 20              & 0.694             \\ 
1e-06                  & 32                  & 5               & 0.441             \\ 
1e-06                  & 32                  & 10              & 0.641             \\ 
1e-06                  & 32                  & 20              & 0.888             \\ 
1e-06                  & 16                  & 5               & 0.582             \\ 
1e-06                  & 16                  & 10              & 0.782             \\ 
1e-06                  & 16                  & 20              & 0.984             \\ 
\midrule
\textbf{1e-05}         & \textbf{32}         & \textbf{5}      & \textbf{0.981}    \\ 
1e-05                  & 32                  & 7               & 0.997             \\ 
1e-05                  & 16                  & 5               & 0.989             \\ 
1e-05                  & 16                  & 7               & 0.997             \\ 
1e-05                  & 16                  & 10              & 0.998             \\ 
\midrule
5e-06                  & 32                  & 5               & 0.853             \\ 
5e-06                  & 32                  & 7               & 0.957             \\ 
5e-06                  & 32                  & 10              & 0.996             \\ 
5e-06                  & 16                  & 5               & 0.954             \\ 
5e-06                  & 16                  & 7               & 0.996             \\ 
5e-06                  & 16                  & 10              & 0.998             \\ 
\bottomrule
\end{tabular}
}
\caption{Accuracy results for different learning rates, batch sizes, and epochs on 10k facts (GPT2-xl). We use the finetuning on 10k facts as a proxy to pick the hyperparameters of our later continual update experiments (learning rate, batch size and epochs). In bold, what we picked for GPT2-xl. Not shown here, for GPT2-small, we picked 5e-4.}
\label{tab:lr:search}
%\end{wraptable}
\end{table}




\subsection{Comprehensive Analysis of GPT2-XL {\color{customgreen}non-dissonant} Updates}\label{app:gpt2xl:noconflict}


\begin{figure*}[ht!]
    \centering
    \captionsetup{font=small} % Reduce caption font size
    \begin{tabular}{@{}c@{} c@{}} % Remove extra space between columns
           \multicolumn{2}{c}{\textbf{The best LR for full FT is not enough to learn with targeted updates:}} \\
        \begin{subfigure}[b]{0.34\textwidth}
            \centering
            \includegraphics[width=\textwidth]{./figures/3_incremental_update/experiment2_1_xl_update/experiment_2_1_with_B_20240913_013018_bestLR_neuron_update_strategies_old_knowledge.pdf}
            \subcaption{Old Knowledge}
            \label{fig:bestLR_old:xl}
        \end{subfigure} &
        \begin{subfigure}[b]{0.34\textwidth}
            \centering
            \includegraphics[width=\textwidth]{./figures/3_incremental_update/experiment2_1_xl_update/experiment_2_1_with_B_20240913_013018_bestLR_neuron_update_strategies_new_knowledge.pdf}
            \subcaption{New Knowledge}
            \label{fig:bestLR_new:xl}
        \end{subfigure} \\
        %\vspace{0.1cm} \\

        % Impact of Learning Rate: 10X Higher Learning Rate
        % \multicolumn{2}{c}{\textbf{Impact of Learning Rate: 10X Higher Learning Rate}} \\
        \multicolumn{2}{c}{\textbf{Increasing the LR (here 10X higher) helps:}} \\
        \begin{subfigure}[b]{0.34\textwidth}
            \centering
            \includegraphics[width=\textwidth]{./figures/3_incremental_update/experiment2_1_xl_update/experiment_2_1_with_B_20240912_203810_10xLR_neuron_update_strategies_old_knowledge.pdf}
            \subcaption{Old Knowledge}
            \label{fig:10xLR_old:xl}
        \end{subfigure} &
        \begin{subfigure}[b]{0.34\textwidth}
            \centering
            \includegraphics[width=\textwidth]{./figures/3_incremental_update/experiment2_1_xl_update/experiment_2_1_with_B_20240912_203810_10xLR_neuron_update_strategies_new_knowledge.pdf}
            \subcaption{New Knowledge}
            \label{fig:10xLR_new:xl}
        \end{subfigure} \\
        %\vspace{0.1cm} \\

        % % Impact of Update Sparsity: 10X More Neurons
        % \multicolumn{2}{c}{\textbf{Impact of Update Sparsity: 10X More Neurons}} \\
         % Impact of Update Sparsity: 10X More Neurons
        \multicolumn{2}{c}{\textbf{Giving more space (here 10X more neurons) also helps targetted updates:}} \\
        \begin{subfigure}[b]{0.34\textwidth}
            \centering
            \includegraphics[width=\textwidth]{./figures/3_incremental_update/experiment2_1_xl_update/experiment_2_1_with_B_20240911_181551_bestLR_10XNeurons_neuron_update_strategies_old_knowledge.pdf}
            \subcaption{Old Knowledge}
            \label{fig:10XNeurons_old:xl}
        \end{subfigure} &
        \begin{subfigure}[b]{0.34\textwidth}
            \centering
            \includegraphics[width=\textwidth]{./figures/3_incremental_update/experiment2_1_xl_update/experiment_2_1_with_B_20240911_181551_bestLR_10XNeurons_neuron_update_strategies_new_knowledge.pdf}
            \subcaption{New Knowledge}
            \label{fig:10XNeurons_new:xl}
        \end{subfigure} \\
        %\vspace{0.1cm} \\

        % Impact of Training Duration: 50 Epochs (10X More)
        \multicolumn{2}{c}{\textbf{Finally, training longer (here 50 Epochs) yielded the most stable results:}} \\
        \begin{subfigure}[b]{0.34\textwidth}
            \centering
            \includegraphics[width=\textwidth]{./figures/3_incremental_update/experiment2_1_xl_update/experiment_2_1_with_B_20240912_150705epochs50s_neuron_update_strategies_old_knowledge.pdf}
            \subcaption{Old Knowledge}
            \label{fig:epochs50s_old}
        \end{subfigure} &
        \begin{subfigure}[b]{0.34\textwidth}
            \centering
            \includegraphics[width=\textwidth]{./figures/3_incremental_update/experiment2_1_xl_update/experiment_2_1_with_B_20240912_150705epochs50s_neuron_update_strategies_new_knowledge.pdf}
            \subcaption{New Knowledge}
            \label{fig:epochs50s_new}
        \end{subfigure} \\
    \end{tabular}
    \caption{\textbf{{\color{customgreen}Non-Dissonant} updates with GPT2-XL} under various conditions. Overall the same trends as GPT2-small are confirmed: targeting stubborn neurons destroys old knowledge more and plastic neurons need more space or time to learn.}
    \label{fig:gpt2xl:full:non-conflict}
\end{figure*}


Figure~\ref{fig:gpt2xl:full:non-conflict} presents the accuracy of GPT-2 XL on old and new knowledge under various neuron update strategies and experimental conditions. We explored different configurations to understand how the model's larger capacity affects knowledge integration.


Our results reveal distinct scaling behaviors compared to GPT-2 small. 
With the optimal learning rate for GPT-2 XL (Figures~\ref{fig:bestLR_old:xl},~\ref{fig:bestLR_new:xl}), we observe improved new knowledge acquisition while still preserving old knowledge. This means that although our carefully picked learning rate allows for efficient learning with full finetuning, learning with fewer neurons (as per our targetted strategies) seems harder than it was for GPT-2 small. 

Increasing the learning rate by 10x (Figures~\ref{fig:10xLR_old:xl},~\ref{fig:10xLR_new:xl}) or allocating 10x more neurons (Figures~\ref{fig:10XNeurons_old:xl},~\ref{fig:10XNeurons_new:xl}) confirms that GPT-2 XL requires either higher learning rates or more extensive parameter updates compared to GPT-2 small to achieve effective learning with our targeted strategies. 

\begin{figure}[h]
    \centering
    \includegraphics[width=\textwidth]{figures/3_incremental_update/mosaic_all/pareto_mosaic_GPT2-small_GPT2-XL_10x_neurons_GPT2-XL_GPT2-XL_10x_LR_GPT2-XL_50_Epochs.pdf}
		
    \caption{\textbf{{\color{customgreen}Non-Dissonant} updates with GPT2-XL compared to small, a different visualization.} Scatter plot of old (x) vs new (y) knowledge during {\color{customgreen}non-dissonant} updates. Same conditions as in Fig.~\ref{fig:gpt2xl:full:non-conflict}. We can see clearly how in all cases, the accuracy on previous knowledge remains high. The lottery-ticket effect is also visible where free neurons struggle to efficient pack novel facts.}
    \label{fig:pareto_mosaic_combined}
\end{figure}



Similarly, extended training duration (50 epochs, Figures~\ref{fig:epochs50s_old},~\ref{fig:epochs50s_new}) allows the model to better integrate new knowledge while preserving old information, indicating that longer training can also help overcome the limitations of sparse updates in larger models. Figure~\ref{fig:pareto_mosaic_combined} summarizes these trade-offs across all configurations, highlighting how different hyperparameter choices affect the balance between preserving old knowledge and acquiring new information.

Finally, note that while GPT-2 XL's larger capacity naturally reduces interference with our tracked facts during non-dissonant updates, this improved performance is ``deceptive'' and should be interpreted cautiously: \textit{we cannot measure potential effects on other pre-trained knowledge beyond our tracked facts}. 

\textit{These results highlight the methodological challenges in studying knowledge updates in larger models:  their increased capacity can mask interference with tracked facts,  making it harder to fully measure the impact of updates on the model's broader knowledge.} This underscores the importance of controlled experimental settings when studying fundamental properties of knowledge updating in neural networks.

% Our results indicate that, unlike GPT-2 small, GPT-2 XL generally preserves old knowledge across all strategies, but struggles to integrate new knowledge effectively. Increasing the learning rate or the number of neurons allocated for updates improves the model's ability to learn new information but may lead to more degradation of existing knowledge. Extending the training duration also allows the model to pack more knowledge into fewer neurons while better preserving old knowledge.

% These findings suggest that the principles observed in smaller models apply to larger models like GPT-2 XL, though the effects may be less pronounced due to the increased capacity. Adjusting hyperparameters such as the learning rate, number of neurons, and training duration is crucial when scaling up to larger models.


%%%%%%%%%%%%%%%%%%%%%%%%%%%%%%%%
\section{{\color{customred}Dissonant} updates}\label{app:dissonant}

\subsection{Impact of number of conflicting facts}\label{app:diss:nfacts}
We examined the effect of varying the number of conflicting facts introduced during {\color{customred}Dissonant} updates. Figure~\ref{fig:additional_results} shows the performance metrics of GPT-2 small when editing 10, 100, and 1,000 facts, respectively.

Our findings show that as the number of conflicting facts increases, the impact on old knowledge retention becomes more pronounced, with all strategies experiencing significant degradation. The ability to learn new conflicting knowledge improves slightly with more facts, but overall performance remains suboptimal. The plastic and random neuron strategies tend to preserve old knowledge when editing a small number of facts (e.g., 10 facts), but their effectiveness diminishes as more conflicting information is introduced. Interestingly, the opposite effect is observed for new knowledge, where adding more facts seems to make it easier to learn new knowledge, for all strategies.
\begin{figure}[h]
    \centering
    \captionsetup{font=small} % Optional: Reduce caption font size for better fit
    \begin{tabular}{@{}c c c@{}}
        % -------------------------------
        % Row 1: Generalization
        \multicolumn{3}{c}{\textbf{Generalization}} \\
        \begin{subfigure}[b]{0.3\textwidth}
            \centering
            \includegraphics[width=\textwidth]{./figures/4_editing/experiment3_1/experiment_3_1_2000_10_neuron_update_strategies_general_knowledge.pdf}
            \subcaption{10 Facts}
            \label{fig:gen_10}
        \end{subfigure} &
        \begin{subfigure}[b]{0.3\textwidth}
            \centering
            \includegraphics[width=\textwidth]{./figures/4_editing/experiment3_1/experiment_3_1_2000_100_neuron_update_strategies_general_knowledge.pdf}
            \subcaption{100 Facts}
            \label{fig:gen_100}
        \end{subfigure} &
        \begin{subfigure}[b]{0.3\textwidth}
            \centering
            \includegraphics[width=\textwidth]{./figures/4_editing/experiment3_1/experiment_3_1_2000_1000_neuron_update_strategies_general_knowledge.pdf}
            \subcaption{1000 Facts}
            \label{fig:gen_1000}
        \end{subfigure} \\
        \vspace{0.3cm} \\

        % -------------------------------
        % Row 2: Accuracy on New Knowledge
        \multicolumn{3}{c}{\textbf{Accuracy on New Knowledge}} \\
        \begin{subfigure}[b]{0.3\textwidth}
            \centering
            \includegraphics[width=\textwidth]{./figures/4_editing/experiment3_1/experiment_3_1_2000_10_neuron_update_strategies_new_knowledge.pdf}
            \subcaption{10 Facts}
            \label{fig:acc_new_10}
        \end{subfigure} &
        \begin{subfigure}[b]{0.3\textwidth}
            \centering
            \includegraphics[width=\textwidth]{./figures/4_editing/experiment3_1/experiment_3_1_2000_100_neuron_update_strategies_new_knowledge.pdf}
            \subcaption{100 Facts}
            \label{fig:acc_new_100}
        \end{subfigure} &
        \begin{subfigure}[b]{0.3\textwidth}
            \centering
            \includegraphics[width=\textwidth]{./figures/4_editing/experiment3_1/experiment_3_1_2000_1000_neuron_update_strategies_new_knowledge.pdf} % Added .pdf extension
            \subcaption{1000 Facts}
            \label{fig:acc_new_1000}
        \end{subfigure} \\
        \vspace{0.3cm} \\

        % -------------------------------
        % Row 3: Accuracy on Old Knowledge
        \multicolumn{3}{c}{\textbf{Accuracy on Old Knowledge}} \\
        \begin{subfigure}[b]{0.3\textwidth}
            \centering
            \includegraphics[width=\textwidth]{./figures/4_editing/experiment3_1/experiment_3_1_2000_10_neuron_update_strategies_old_knowledge.pdf}
            \subcaption{10 Facts}
            \label{fig:acc_old_10}
        \end{subfigure} &
        \begin{subfigure}[b]{0.3\textwidth}
            \centering
            \includegraphics[width=\textwidth]{./figures/4_editing/experiment3_1/experiment_3_1_2000_100_neuron_update_strategies_old_knowledge.pdf}
            \subcaption{100 Facts}
            \label{fig:acc_old_100}
        \end{subfigure} &
        \begin{subfigure}[b]{0.3\textwidth}
            \centering
            \includegraphics[width=\textwidth]{./figures/4_editing/experiment3_1/experiment_3_1_2000_1000_neuron_update_strategies_old_knowledge.pdf}
            \subcaption{1000 Facts}
            \label{fig:acc_old_1000}
        \end{subfigure} \\
    \end{tabular}
    \caption{\textbf{{\color{customred}Dissonant} updates with GPT2-small - impact of the number of conflicting facts}. Each row represents a distinct metric: accuracy on the \textbf{Generalization} side dataset (paraphrased versions of the new facts), accuracy on \textbf{New Knowledge}, and Accuracy on \textbf{Old Knowledge}. Within each row, the subplots correspond to the number of conflicting facts introduced (\textbf{10 Facts}, \textbf{100 Facts}, and \textbf{1000 Facts}).}
    \label{fig:additional_results}
\end{figure}


\subsection{Comparative performance of Editing methods}\label{app:rome:memit}
Our primary focus in this work is \textit{not} on developing new model editing techniques. Most existing editing techniques focus on altering existing associations, and are hence by our definition dissonant by design. Our empirical findings in this work suggest another parallel path in which editing is abandoned in favor of non-dissonant variations where old knowledge is kept and contextualized

However, to have an idea on how existing editing methods perform compared to our targeted strategies, we leverage \texttt{EasyEdit}~\citep{wang2023easyedit} to benchmark two state-of-the-art model editing methods, ROME~\citep{Meng2022} and MEMIT~\citep{Meng2022a}, under our same multi-fact experimental conditions . 

Table~\ref{tab:comparison:sumary:editing} summarizes the performance of different strategies and editing methods. Some of our targeted update strategies obtain a higher harmonic mean compared to ROME and MEMIT. But the higher harmonic mean must not hide that the approaches are not directly comparable since they explore different regions of the pareto front, balancing new knowledge acquisition and old knowledge retention, as self-explained with colors and rankings in the table. 
\newcommand{\gradcolor}[2]{
\pgfmathsetmacro{\redcomp}{max(0,min(1,1-#1))}
\pgfmathsetmacro{\greencomp}{max(0,min(1,#1))}
\definecolor{gradientcolor}{rgb}{\redcomp,\greencomp,0}
\textcolor{gradientcolor}{#1 #2}}

\newcommand{\gradcolorFirst}[2]{
\gradcolor{#1}{#2}$^{~1}$}
\newcommand{\gradcolorSecond}[2]{
\gradcolor{#1}{#2}$^{~2}$}
\newcommand{\gradcolorUnmarked}[2]{
\gradcolor{#1}{#2}$^{~~}$}




\begin{table*}[h]
\centering
\caption{Comparison of targeted neuron update strategies vs knowledge-editing literature, with a gradient from \textcolor{red}{0 (red)} to \textcolor{green}{1 (green)}. Top-1,2 strategies annotated for all metrics and sample sizes.}
\label{tab:comparison:sumary:editing}
\resizebox{\textwidth}{!}{%
\begin{tabular}{clccccc}
\toprule
\textbf{Samples} & \textbf{Strategy}           & \textbf{Old (Unrelated)}        & \textbf{New (Reliability)}          & \textbf{Generalization}             & \textbf{Harmonic Mean} \\ 
\midrule
\multirow{11}{*}{10} 
 & Full Finetune                & \gradcolorUnmarked{0.107}{(0.082)} & \gradcolorFirst{1.000}{(0.000)} & \gradcolorUnmarked{0.576}{(0.117)} & \gradcolorUnmarked{0.222}{(0.116)} \\
 & MEMIT\citep{Meng2022a}                        & \gradcolorFirst{0.962}{(0.079)} & \gradcolorUnmarked{0.000}{(0.000)} & \gradcolorUnmarked{0.000}{(0.000)} & \gradcolorUnmarked{0.000}{(0.000)} \\
 & ROME\citep{Meng2022}                         & \gradcolorUnmarked{0.891}{(0.085)} & \gradcolorUnmarked{0.240}{(0.182)} & \gradcolorUnmarked{0.180}{(0.179)} & \gradcolorUnmarked{0.236}{(0.235)} \\
\gcmidrule(lr){2-6}
 & 8k Candidate        & \gradcolorUnmarked{0.596}{(0.106)} & \gradcolorSecond{0.988}{(0.024)} & \gradcolorSecond{0.644}{(0.128)} & \gradcolorFirst{0.690}{(0.058)} \\
 & 20k Candidate               & \gradcolorUnmarked{0.430}{(0.134)} & \gradcolorFirst{1.000}{(0.000)} & \gradcolorFirst{0.656}{(0.125)} & \gradcolorUnmarked{0.597}{(0.116)} \\
 & 8k Specific                 & \gradcolorUnmarked{0.638}{(0.138)} & \gradcolorUnmarked{0.964}{(0.039)} & \gradcolorUnmarked{0.512}{(0.238)} & \gradcolorUnmarked{0.600}{(0.183)} \\
 & 8k Stubborn                 & \gradcolorUnmarked{0.622}{(0.110)} & \gradcolorUnmarked{0.972}{(0.030)} & \gradcolorUnmarked{0.544}{(0.169)} & \gradcolorSecond{0.643}{(0.103)} \\
 & 8k Plastic                  & \gradcolorSecond{0.909}{(0.039)} & \gradcolorUnmarked{0.020}{(0.040)} & \gradcolorUnmarked{0.000}{(0.000)} & \gradcolorUnmarked{0.000}{(0.000)} \\
 & 8k Random                   & \gradcolorUnmarked{0.827}{(0.083)} & \gradcolorUnmarked{0.380}{(0.132)} & \gradcolorUnmarked{0.092}{(0.094)} & \gradcolorUnmarked{0.277}{(0.098)} \\
\midrule
\multirow{11}{*}{100} 
 & Full Finetune                & \gradcolorUnmarked{0.238}{(0.019)} & \gradcolorSecond{0.998}{(0.003)} & \gradcolorUnmarked{0.434}{(0.089)} & \gradcolorUnmarked{0.398}{(0.041)} \\
 & MEMIT\citep{Meng2022a}                   & \gradcolorFirst{0.976}{(0.008)} & \gradcolorUnmarked{0.004}{(0.005)} & \gradcolorUnmarked{0.010}{(0.007)} & \gradcolorUnmarked{0.003}{(0.007)} \\
 & ROME\citep{Meng2022}                         & \gradcolorUnmarked{0.431}{(0.108)} & \gradcolorUnmarked{0.300}{(0.054)} & \gradcolorUnmarked{0.150}{(0.036)} & \gradcolorUnmarked{0.240}{(0.045)} \\
\gcmidrule(lr){2-6}
 & 8k Candidate        & \gradcolorSecond{0.542}{(0.035)} & \gradcolorUnmarked{0.969}{(0.033)} & \gradcolorFirst{0.462}{(0.081)} & \gradcolorFirst{0.591}{(0.054)} \\
 & 20k Candidate               & \gradcolorUnmarked{0.463}{(0.032)} & \gradcolorFirst{0.999}{(0.002)} & \gradcolorSecond{0.447}{(0.083)} & \gradcolorSecond{0.552}{(0.052)} \\     
 & 8k Specific                 & \gradcolorUnmarked{0.531}{(0.030)} & \gradcolorUnmarked{0.760}{(0.063)} & \gradcolorUnmarked{0.263}{(0.027)} & \gradcolorUnmarked{0.426}{(0.024)} \\
 & 8k Stubborn                 & \gradcolorUnmarked{0.530}{(0.054)} & \gradcolorUnmarked{0.936}{(0.048)} & \gradcolorUnmarked{0.398}{(0.064)} & \gradcolorUnmarked{0.547}{(0.063)} \\
 & 8k Plastic                  & \gradcolorUnmarked{0.433}{(0.029)} & \gradcolorUnmarked{0.059}{(0.014)} & \gradcolorUnmarked{0.028}{(0.017)} & \gradcolorUnmarked{0.052}{(0.025)} \\
 & 8k Random                   & \gradcolorUnmarked{0.508}{(0.019)} & \gradcolorUnmarked{0.193}{(0.038)} & \gradcolorUnmarked{0.065}{(0.025)} & \gradcolorUnmarked{0.131}{(0.039)} \\
\midrule
\multirow{11}{*}{1000} 
 & Full Finetune                & \gradcolorUnmarked{0.182}{(0.007)} & \gradcolorUnmarked{0.991}{(0.009)} & \gradcolorFirst{0.442}{(0.053)} & \gradcolorSecond{0.341}{(0.016)} \\
 & MEMIT\citep{Meng2022a}                       & \gradcolorFirst{0.605}{(0.107)} & \gradcolorUnmarked{0.198}{(0.053)} & \gradcolorUnmarked{0.100}{(0.016)} & \gradcolorUnmarked{0.177}{(0.028)} \\
 & ROME\citep{Meng2022}                         & \gradcolorUnmarked{0.152}{(0.071)} & \gradcolorUnmarked{0.160}{(0.093)} & \gradcolorUnmarked{0.067}{(0.035)} & \gradcolorUnmarked{0.106}{(0.058)} \\
\gcmidrule(lr){2-6}
 & 8k Candidate        & \gradcolorUnmarked{0.199}{(0.014)} & \gradcolorFirst{0.996}{(0.002)} & \gradcolorSecond{0.380}{(0.041)} & \gradcolorFirst{0.345}{(0.014)} \\
 & 20k Candidate               & \gradcolorUnmarked{0.172}{(0.018)} & \gradcolorFirst{0.996}{(0.001)} & \gradcolorUnmarked{0.369}{(0.043)} & \gradcolorUnmarked{0.314}{(0.028)}  \\
 & 8k Specific                 & \gradcolorSecond{0.240}{(0.017)} & \gradcolorUnmarked{0.993}{(0.003)} & \gradcolorUnmarked{0.287}{(0.039)} & \gradcolorFirst{0.345}{(0.028)} \\
 & 8k Stubborn                 & \gradcolorUnmarked{0.200}{(0.007)} & \gradcolorSecond{0.995}{(0.001)} & \gradcolorUnmarked{0.317}{(0.024)} & \gradcolorUnmarked{0.327}{(0.006)} \\
 & 8k Plastic                  & \gradcolorUnmarked{0.218}{(0.024)} & \gradcolorUnmarked{0.283}{(0.026)} & \gradcolorUnmarked{0.070}{(0.010)} & \gradcolorUnmarked{0.133}{(0.013)} \\
 & 8k Random                   & \gradcolorUnmarked{0.194}{(0.026)} & \gradcolorUnmarked{0.663}{(0.072)} & \gradcolorUnmarked{0.088}{(0.008)} & \gradcolorUnmarked{0.165}{(0.014)} \\
\bottomrule
\end{tabular}
}
\end{table*}



\subsection{More detailed figures for specific numbers of neurons}\label{app:dissonant:numbers}
Tables~\ref{tab:neuron_editing_20000}, Figs.~\ref{tab:neuron_editing_8000}, and \ref{tab:neuron_editing_4000} provide detailed performance metrics for different neuron thresholds (20k, 8k, and 4k neurons, respectively) when editing 1,000, 100 and 10, conflicting facts using various strategies.

\begin{table}[h!]
\centering
\caption{Neuron Editing Results for N=20,000 Neurons}
\label{tab:neuron_editing_20000}
\resizebox{0.7\textwidth}{!}{%
\begin{tabular}{cccccc}
\toprule
\textbf{Samples} & \textbf{Strategy}          & \textbf{Accuracy A} & \textbf{Accuracy NOT(B)} & \textbf{Accuracy GEN} & \textbf{Harmonic Mean} \\
\midrule
\multirow{7}{*}{10} 
 &Full Finetune                      & 0.107 (0.082)       & 1.000 (0.000)             & 0.576 (0.117)         & 0.222 (0.116)           \\
 & Specific             & 0.491 (0.137)       & 1.000 (0.000)             & 0.604 (0.126)         & 0.621 (0.109)           \\
 & Plastic           & 0.735 (0.105)       & 0.752 (0.175)             & 0.220 (0.183)         & 0.434 (0.185)           \\
% & Plastic (new)          & 0.794 (0.093)       & 0.732 (0.176)             & 0.204 (0.169)         & 0.424 (0.201)           \\
 & Stubborn           & 0.449 (0.109)       & 1.000 (0.000)             & 0.616 (0.091)         & 0.606 (0.084)           \\
 & Candidate          & 0.430 (0.134)       & 1.000 (0.000)             & 0.656 (0.125)         & 0.597 (0.116)           \\
 & Random              & 0.688 (0.107)       & 0.944 (0.083)             & 0.448 (0.212)         & 0.579 (0.222)           \\
\midrule
\multirow{7}{*}{100} 
 &Full Finetune                      & 0.238 (0.019)       & 0.998 (0.003)             & 0.434 (0.089)         & 0.398 (0.041)           \\
 & Specific             & 0.412 (0.046)       & 0.988 (0.005)             & 0.330 (0.054)         & 0.460 (0.046)           \\
 & Plastic           & 0.317 (0.052)       & 0.586 (0.048)             & 0.128 (0.028)         & 0.233 (0.035)           \\
%& Plastic (new)          & 0.301 (0.044)       & 0.649 (0.062)             & 0.146 (0.032)         & 0.252 (0.037)           \\
 & Stubborn           & 0.435 (0.043)       & 0.999 (0.002)             & 0.427 (0.085)         & 0.528 (0.057)           \\
 & Candidate          & 0.463 (0.032)       & 0.999 (0.002)             & 0.447 (0.083)         & 0.552 (0.052)           \\
 & Random              & 0.474 (0.035)       & 0.874 (0.048)             & 0.292 (0.048)         & 0.444 (0.036)           \\
\midrule
\multirow{7}{*}{1000} 
 &Full Finetune                      & 0.182 (0.007)       & 0.991 (0.009)             & 0.442 (0.053)         & 0.341 (0.016)           \\
 & Specific             & 0.188 (0.033)       & 0.995 (0.002)             & 0.257 (0.025)         & 0.292 (0.035)           \\
 & Plastic           & 0.077 (0.021)       & 0.996 (0.002)             & 0.224 (0.018)         & 0.160 (0.027)           \\
 %& Plastic (new)          & 0.078 (0.010)       & 0.996 (0.002)             & 0.247 (0.014)         & 0.168 (0.015)           \\
 & Stubborn           & 0.185 (0.010)       & 0.992 (0.005)             & 0.327 (0.013)         & 0.317 (0.012)           \\
 & Candidate          & 0.172 (0.018)       & 0.996 (0.001)             & 0.369 (0.043)         & 0.314 (0.028)           \\
 & Random              & 0.235 (0.029)       & 0.995 (0.003)             & 0.300 (0.053)         & 0.347 (0.041)           \\
\bottomrule
\end{tabular}
}
\end{table}


\begin{table}[h!]
\centering
\caption{Neuron Editing Results for N=8,000 Neurons}
\label{tab:neuron_editing_8000}
\resizebox{0.7\textwidth}{!}{%
\begin{tabular}{cccccc}
\toprule
\textbf{Samples} & \textbf{Strategy} & \textbf{Accuracy A} & \textbf{Accuracy NOT(B)} & \textbf{Accuracy GEN} & \textbf{Harmonic Mean} \\
\midrule
\multirow{7}{*}{10} 
 & Full Finetune     & 0.107 (0.082) & 1.000 (0.000) & 0.576 (0.117) & 0.222 (0.116) \\ 
 & Specific         & 0.638 (0.138) & 0.964 (0.039) & 0.512 (0.238) & 0.600 (0.183) \\ 
 & Plastic        & 0.909 (0.039) & 0.020 (0.040) & 0.000 (0.000) & 0.0 \\ 
 %& Plastic (New)       & 0.936 (0.029) & 0.000 (0.000) & 0.000 (0.000) & 0.0 \\ 
 & Stubborn        & 0.622 (0.110) & 0.972 (0.030) & 0.544 (0.169) & 0.643 (0.103) \\ 
 & Candidate       & 0.596 (0.106) & 0.988 (0.024) & 0.644 (0.128) & 0.690 (0.058) \\ 
 & Random          & 0.827 (0.083) & 0.380 (0.132) & 0.092 (0.094) & 0.277 (0.098) \\ 
\midrule
\multirow{7}{*}{100} 
 &Full Finetune                  & 0.238 (0.019) & 0.998 (0.003) & 0.434 (0.089) & 0.398 (0.041) \\ 
 & Specific         & 0.531 (0.030) & 0.760 (0.063) & 0.263 (0.027) & 0.426 (0.024) \\ 
 & Plastic        & 0.433 (0.029) & 0.059 (0.014) & 0.028 (0.017) & 0.052 (0.025) \\ 
% & Plastic (New)       & 0.553 (0.056) & 0.033 (0.011) & 0.016 (0.010) & 0.028 (0.012) \\ 
 & Stubborn        & 0.530 (0.054) & 0.936 (0.048) & 0.398 (0.064) & 0.547 (0.063) \\ 
 & Candidate       & 0.542 (0.035) & 0.969 (0.033) & 0.462 (0.081) & 0.591 (0.054) \\ 
 & Random          & 0.508 (0.019) & 0.193 (0.038) & 0.065 (0.025) & 0.131 (0.039) \\ 
\midrule
\multirow{7}{*}{1000} 
 &Full Finetune                  & 0.182 (0.007) & 0.991 (0.009) & 0.442 (0.053) & 0.341 (0.016) \\ 
 & Specific         & 0.240 (0.017) & 0.993 (0.003) & 0.287 (0.039) & 0.345 (0.028) \\ 
 & Plastic        & 0.218 (0.024) & 0.283 (0.026) & 0.070 (0.010) & 0.133 (0.013) \\ 
 %& Plastic (New)       & 0.266 (0.034) & 0.247 (0.028) & 0.060 (0.011) & 0.122 (0.017) \\ 
 & Stubborn        & 0.200 (0.007) & 0.995 (0.001) & 0.317 (0.024) & 0.327 (0.006) \\ 
 & Candidate       & 0.199 (0.014) & 0.996 (0.002) & 0.380 (0.041) & 0.345 (0.014) \\ 
 & Random          & 0.159 (0.032) & 0.784 (0.091) & 0.102 (0.014) & 0.169 (0.010) \\ \bottomrule
\end{tabular}
}
\end{table}

% \begin{table}[h!]
% \centering
% \caption{Neuron Editing Results for N=6,000 Neurons}
% \label{tab:neuron_editing_6000}
% \resizebox{0.7\textwidth}{!}{%
% \begin{tabular}{cccccc}
% \toprule
% \textbf{Samples} & \textbf{Strategy} & \textbf{Accuracy A} & \textbf{Accuracy NOT(B)} & \textbf{Accuracy GEN} & \textbf{Harmonic Mean} \\
% \midrule
% \multirow{7}{*}{10} \\ 
%  &Full Finetune                  & 0.107 (0.082) & 1.000 (0.000) & 0.576 (0.117) & 0.222 (0.116) \\ 
%  & Specific         & 0.663 (0.117) & 0.800 (0.111) & 0.436 (0.204) & 0.545 (0.164) \\ 
%  & Plastic        & 0.941 (0.031) & 0.004 (0.008) & 0.000 (0.000) & 0.0 \\ 
% % & Plastic (New)       & 0.963 (0.018) & 0.000 (0.000) & 0.000 (0.000) & 0.0 \\ 
%  & Stubborn        & 0.641 (0.083) & 0.868 (0.057) & 0.404 (0.160) & 0.548 (0.111) \\ 
%  & Candidate       & 0.604 (0.115) & 0.956 (0.043) & 0.552 (0.134) & 0.642 (0.057) \\ 
%  & Random          & 0.898 (0.059) & 0.120 (0.126) & 0.000 (0.000) & 0.0 \\ 
% \midrule
% \multirow{7}{*}{100} \\ 
%  &Full Finetune                  & 0.238 (0.019) & 0.998 (0.003) & 0.434 (0.089) & 0.398 (0.041) \\ 
%  & Specific         & 0.552 (0.014) & 0.573 (0.064) & 0.200 (0.025) & 0.347 (0.020) \\ 
%  & Plastic        & 0.627 (0.051) & 0.010 (0.005) & 0.011 (0.011) & 0.020 (0.004) \\ 
% % & Plastic (New)       & 0.733 (0.036) & 0.002 (0.002) & 0.002 (0.004) & 0.005 (0.000) \\ 
%  & Stubborn        & 0.558 (0.050) & 0.850 (0.091) & 0.371 (0.063) & 0.527 (0.062) \\ 
%  & Candidate       & 0.569 (0.031) & 0.925 (0.091) & 0.436 (0.095) & 0.580 (0.075) \\ 
%  & Random          & 0.497 (0.047) & 0.077 (0.029) & 0.040 (0.030) & 0.071 (0.041) \\ 
% \midrule
% \multirow{7}{*}{1000} \\ 
%  &Full Finetune                  & 0.182 (0.007) & 0.991 (0.009) & 0.442 (0.053) & 0.341 (0.016) \\ 
%  & Specific         & 0.230 (0.012) & 0.992 (0.006) & 0.297 (0.052) & 0.342 (0.030) \\ 
%  & Plastic        & 0.270 (0.054) & 0.196 (0.022) & 0.057 (0.010) & 0.112 (0.014) \\ 
%  %& Plastic (New)       & 0.316 (0.066) & 0.165 (0.028) & 0.051 (0.009) & 0.102 (0.014) \\ 
%  & Stubborn        & 0.200 (0.018) & 0.993 (0.005) & 0.315 (0.043) & 0.325 (0.029) \\ 
%  & Candidate       & 0.185 (0.026) & 0.997 (0.002) & 0.357 (0.048) & 0.322 (0.026) \\ 
%  & Random          & 0.194 (0.026) & 0.663 (0.072) & 0.088 (0.008) & 0.165 (0.014) \\ \bottomrule
% \end{tabular}
% }
% \end{table}

\begin{table}[h!]
\centering
\caption{Neuron Editing Results for N=4,000 Neurons}
\label{tab:neuron_editing_4000}
\resizebox{0.7\textwidth}{!}{%
\begin{tabular}{cccccc}
\toprule
\textbf{Samples} & \textbf{Strategy} & \textbf{Accuracy A} & \textbf{Accuracy NOT(B)} & \textbf{Accuracy GEN} & \textbf{Harmonic Mean} \\
\midrule
\multirow{7}{*}{10} \\ 
 &Full Finetune                  & 0.107 (0.082) & 1.000 (0.000) & 0.576 (0.117) & 0.222 (0.116) \\ 
 & Specific         & 0.673 (0.101) & 0.656 (0.168) & 0.264 (0.208) & 0.385 (0.182) \\ 
 & Plastic        & 0.965 (0.021) & 0.000 (0.000) & 0.000 (0.000) & 0.0 \\ 
 %& Plastic (New)       & 0.981 (0.010) & 0.000 (0.000) & 0.000 (0.000) & 0.0 \\ 
 & Stubborn        & 0.635 (0.062) & 0.764 (0.087) & 0.352 (0.115) & 0.506 (0.101) \\ 
 & Candidate       & 0.603 (0.101) & 0.864 (0.126) & 0.512 (0.106) & 0.613 (0.065) \\ 
 & Random          & 0.863 (0.066) & 0.144 (0.113) & 0.044 (0.062) & 0.169 (0.050) \\ 
\midrule
\multirow{7}{*}{100} \\ 
 &Full Finetune                  & 0.238 (0.019) & 0.998 (0.003) & 0.434 (0.089) & 0.398 (0.041) \\ 
 & Specific         & 0.553 (0.023) & 0.408 (0.040) & 0.137 (0.022) & 0.258 (0.029) \\ 
 & Plastic        & 0.760 (0.054) & 0.000 (0.000) & 0.003 (0.003) & 0.0 \\ 
 %& Plastic (New)       & 0.854 (0.024) & 0.000 (0.000) & 0.000 (0.000) & 0.0 \\ 
 & Stubborn        & 0.565 (0.060) & 0.705 (0.143) & 0.303 (0.077) & 0.460 (0.092) \\ 
 & Candidate       & 0.573 (0.041) & 0.852 (0.124) & 0.400 (0.102) & 0.548 (0.093) \\ 
 & Random          & 0.487 (0.043) & 0.090 (0.018) & 0.045 (0.023) & 0.082 (0.030) \\ 
\midrule
\multirow{7}{*}{1000} \\ 
 &Full Finetune                  & 0.182 (0.007) & 0.991 (0.009) & 0.442 (0.053) & 0.341 (0.016) \\ 
 & Specific         & 0.235 (0.008) & 0.976 (0.012) & 0.265 (0.041) & 0.329 (0.025) \\ 
 & Plastic        & 0.348 (0.049) & 0.125 (0.021) & 0.047 (0.006) & 0.093 (0.009) \\ 
 %& Plastic (New)       & 0.456 (0.097) & 0.080 (0.019) & 0.034 (0.007) & 0.067 (0.012) \\ 
 & Stubborn        & 0.203 (0.013) & 0.989 (0.006) & 0.315 (0.031) & 0.329 (0.016) \\ 
 & Candidate       & 0.184 (0.013) & 0.996 (0.001) & 0.370 (0.045) & 0.327 (0.025) \\ 
 & Random          & 0.254 (0.049) & 0.400 (0.085) & 0.072 (0.006) & 0.146 (0.010) \\ \bottomrule
\end{tabular}
}
\end{table}

% FT NOT-B                 
% & C-FT NOT-B (spec)         
% & C-FT NOT-B (free_A)      
% & C-FT NOT-B (free_B)      
% & C-FT NOT-B (busy_A)      
% & C-FT NOT-B (busy_B)      
% & C-FT NOT-B (rnd)  
%  &Full Finetune       
% & Plastic (old)          
% & Plastic (new) => removed because confusing.         
% & Stubborn (old)         
% & Candidate          
% & Random        

The results show that changing the number of neurons allocated for updates does not necessarily improve or degrade performance in the dissonant update scenario. In all cases, the model struggles to retain old knowledge while learning new conflicting information. The candidate and specific neuron strategies are consistently and significantly better than state of the art solutions, offering a slight advantage. However, they are still unable to effectively mitigate the destructive effects of dissonant updates, further motivating the neeed for both (i) dissonance awareness and (ii) proper conflict resolution.

\subsection{Scaling to GPT2-XL}\label{app:diss:xl}
We extended our dissonant update experiments to GPT-2 XL to examine whether our observations about knowledge conflicts persist in larger models.

Figure~\ref{fig:knowledge_editing_performance} examines GPT2-XL's behavior when updating 1,000 conflicting facts using the optimal learning rate, as determined by our hyperparameter search. We compare three configurations: GPT-2 small (2,000 to 20,000 neurons) shown previously, GPT2-XL with the same range, and GPT2-XL with ten times more neurons (20,000 to 200,000). The latter was shown effective in packing new knowledge compared to (2000 to 20000) range in non-dissonant updates.

% %\subsection{Impact of Neurons under Best Learning Rate}\label{app:xl:neurons:10}
% Finally, we conduct also additional experiments using the optimal learning rate for GPT-2 XL, as determined in our hyperparameter search. Figure~\ref{fig:knowledge_editing_performance} compares the performance of GPT-2 small and GPT-2 XL when updating 1,000 conflicting facts, varying the number of neurons from 2,000 to 20,000 for GPT-2 small and from 20,000 to 200,000 for GPT-2 XL.

\begin{figure}[h]
    \centering
    \captionsetup{font=small}
    \begin{tabular}{@{}c c c@{}}
        \multicolumn{3}{c}{} \\
        \textbf{Old Knowledge} & \textbf{New Knowledge} & \textbf{Generalization} \\

       % Row 1: GPT2-XL from 2k to 20k neurons
        \multicolumn{3}{c}{\textit{GPT2-small from 2k to 20k neurons}} \\
        \begin{subfigure}[b]{0.3\textwidth}
            \centering
            \includegraphics[width=\textwidth]{./figures/4_editing/experiment3_1/experiment_3_1_2000_1000_neuron_update_strategies_old_knowledge.pdf}
            \subcaption{Old Knowledge}
            \label{fig:same_old}
        \end{subfigure} &
        \begin{subfigure}[b]{0.3\textwidth}
            \centering
            \includegraphics[width=\textwidth]{./figures/4_editing/experiment3_1/experiment_3_1_2000_1000_neuron_update_strategies_new_knowledge}
            \subcaption{New Knowledge}
            \label{fig:same_new}
        \end{subfigure} &
        \begin{subfigure}[b]{0.3\textwidth}
            \centering
            \includegraphics[width=\textwidth]{./figures/4_editing/experiment3_1/experiment_3_1_2000_1000_neuron_update_strategies_general_knowledge.pdf}
            \subcaption{Generalization}
            \label{fig:same_gen}
        \end{subfigure} \\
        \vspace{0.3cm} \\

        % Row 1: GPT2-XL from 2k to 20k neurons
        \multicolumn{3}{c}{\textit{GPT2-XL from 2k to 20k neurons}} \\
        \begin{subfigure}[b]{0.3\textwidth}
            \centering
            \includegraphics[width=\textwidth]{./figures/4_editing/experiment3_1_xl/experiment_3_1_20240719_135003_2000_1000_10_small_lr_neuron_update_strategies_old_knowledge.pdf}
            \subcaption{Old Knowledge}
            \label{fig:same_old_xl}
        \end{subfigure} &
        \begin{subfigure}[b]{0.3\textwidth}
            \centering
            \includegraphics[width=\textwidth]{./figures/4_editing/experiment3_1_xl/experiment_3_1_20240719_135003_2000_1000_10_small_lr_neuron_update_strategies_new_knowledge.pdf}
            \subcaption{New Knowledge}
            \label{fig:same_new_xl}
        \end{subfigure} &
        \begin{subfigure}[b]{0.3\textwidth}
            \centering
            \includegraphics[width=\textwidth]{./figures/4_editing/experiment3_1_xl/experiment_3_1_20240719_135003_2000_1000_10_small_lr_neuron_update_strategies_general_knowledge.pdf}
            \subcaption{Generalization}
            \label{fig:same_gen_xl}
        \end{subfigure} \\
        \vspace{0.3cm} \\
    
        % Row 2: GPT2-XL from 20k to 200k neurons
        \multicolumn{3}{c}{\textit{GPT2-XL with 10X more neurons}} \\
        \begin{subfigure}[b]{0.3\textwidth}
            \centering
            \includegraphics[width=\textwidth]{./figures/4_editing/experiment3_1_xl/experiment_3_1_20240719_085748_20000_1000_10_small_lr_neuron_update_strategies_old_knowledge.pdf}
            \subcaption{Old Knowledge}
            \label{fig:10x_old}
        \end{subfigure} &
        \begin{subfigure}[b]{0.3\textwidth}
            \centering
            \includegraphics[width=\textwidth]{./figures/4_editing/experiment3_1_xl/experiment_3_1_20240719_085748_20000_1000_10_small_lr_neuron_update_strategies_new_knowledge.pdf}
            \subcaption{New Knowledge}
            \label{fig:10x_new}
        \end{subfigure} &
        \begin{subfigure}[b]{0.3\textwidth}
            \centering
            \includegraphics[width=\textwidth]{./figures/4_editing/experiment3_1_xl/experiment_3_1_20240719_085748_20000_1000_10_small_lr_neuron_update_strategies_general_knowledge.pdf}
            \subcaption{Generalization}
            \label{fig:10x_gen}
        \end{subfigure} \\
    \end{tabular}
    \caption{\textbf{ {\color{customred}Dissonant} updates with GPT2-XL: whether the model learns new knowledge or not, old knowledge is severely destroyed regardless of the strategy} Experiments with 1000 facts using the best learning rate we found for Full Finetuning.}
    \label{fig:knowledge_editing_performance}
\end{figure}

First, while GPT2-XL still requires more neurons than GPT-2 small to effectively learn new conflicting knowledge, as seen earlier, the key finding concerns old knowledge retention: regardless of model size or neuron allocation, we observe significant degradation of old, unrelated knowledge across all strategies. 

Interestingly, this degradation persists even when using fewer neurons and when the model fails to effectively learn the new conflicting information (2k to 20k). These results strongly suggest that the destructive impact of conflicting updates on existing knowledge is a fundamental property that remains present in larger models.


% Even with the best learning rate, GPT-2 XL fails to learn the new conflicting knowledge effectively and continues to experience significant degradation in old knowledge retention. These findings further confirm the challenges associated with learning dissonant updates, even when using less packed  larger models. Our various experiments suggest that adjusting hyperparameters alone is not sufficient to mitigate the destructive effects of conflicting information.

% %%%%%%%%%%%%%%%%%%%%%%%%%%%%%%%%%%%%%%%%%%%%%%%%%%%%%
% \begin{comment}
% app:xl:new, app:xl:gen, app:xl:old
% \subsubsection{New Knowledge Acquisition}\label{app:xl:new}

% Figures~\ref{fig:same_10_new}, \ref{fig:same_100_new}, and \ref{fig:same_1000_new} illustrate the performance on conflicting knowledge acquisition when introducing 10, 100, and 1,000 conflicting facts. First, using the same hyperparameters as GPT-2 small, GPT-2 XL achieves less than 10\% accuracy in acquiring new conflicting knowledge - a stark contrast with the non-dissonant case where it reached up to 40\% accuracy under similar conditions (Fig.~\ref{fig:sameLR_new:xl}). Interestingly, in this case, and in contrast with the non-dissonant case, performance remains poor even when allocating 10 times more neurons (Figures~\ref{fig:10x_10_new}, \ref{fig:10x_100_new}, and \ref{fig:10x_1000_new}).

% \begin{figure}[h]
%     \centering
%     \captionsetup{font=small} % Optional: Adjust caption font size
%     \begin{tabular}{@{}c c c@{}}
%         \multicolumn{3}{c}{\textbf{New Knowledge Acquisition}} \\
%         \textbf{10 Facts} & \textbf{100 Facts} & \textbf{1000 Facts} \\
        
%         % Row 1: Same Condition as GPT2-small
%         \multicolumn{3}{c}{\textit{GPT2-XL under the same conditions as GPT2-small}} \\
%         \begin{subfigure}[b]{0.3\textwidth}
%             \centering
%             \includegraphics[width=\textwidth]{./figures/4_editing/experiment3_1_xl/experiment_3_1_2000_10_neuron_update_strategies_new_knowledge.pdf}
%             \subcaption{10 Facts}
%             \label{fig:same_10_new}
%         \end{subfigure} &
%         \begin{subfigure}[b]{0.3\textwidth}
%             \centering
%             \includegraphics[width=\textwidth]{./figures/4_editing/experiment3_1_xl/experiment_3_1_20240719_094011_2000_100_neuron_update_strategies_new_knowledge.pdf}
%             \subcaption{100 Facts}
%             \label{fig:same_100_new}
%         \end{subfigure} &
%         \begin{subfigure}[b]{0.3\textwidth}
%             \centering
%             \includegraphics[width=\textwidth]{./figures/4_editing/experiment3_1_xl/experiment_3_1_20240719_000646_2000_1000_neuron_update_strategies_new_knowledge.pdf}
%             \subcaption{1000 Facts}
%             \label{fig:same_1000_new}
%         \end{subfigure} \\
%         \vspace{0.3cm} \\
        
%         % Row 2: 10X Neurons
%         \multicolumn{3}{c}{\textit{GPT2-XL with 10X more neurons}} \\
%         \begin{subfigure}[b]{0.3\textwidth}
%             \centering
%             \includegraphics[width=\textwidth]{./figures/4_editing/experiment3_1_xl/experiment_3_1_20000_10_neuron_update_strategies_new_knowledge.pdf}
%             \subcaption{10 Facts}
%             \label{fig:10x_10_new}
%         \end{subfigure} &
%         \begin{subfigure}[b]{0.3\textwidth}
%             \centering
%             \includegraphics[width=\textwidth]{./figures/4_editing/experiment3_1_xl/experiment_3_1_20240719_044940_20000_100_neuron_update_strategies_new_knowledge.pdf}
%             \subcaption{100 Facts}
%             \label{fig:10x_100_new}
%         \end{subfigure} &
%         \begin{subfigure}[b]{0.3\textwidth}
%             \centering
%             \includegraphics[width=\textwidth]{./figures/4_editing/experiment3_1_xl/experiment_3_1_20240718_160529_20000_1000_neuron_update_strategies_new_knowledge.pdf}
%             \subcaption{1000 Facts}
%             \label{fig:10x_1000_new}
%         \end{subfigure} \\
%     \end{tabular}
%     \caption{New Knowledge Acquisition of GPT2-XL across different experimental conditions and numbers of conflicting facts.}
%     \label{fig:new_knowledge_acquisition}
% \end{figure}

% However, as we will see later, despite being unable to properly learn new conflicting knowledge, the performance on old unrelated knowledge still considerably degrades.

% \subsubsection{Generalization Performance}\label{app:xl:gen}

% Figures~\ref{fig:same_10_gen}, \ref{fig:same_100_gen}, and \ref{fig:same_1000_gen} show the generalization performance of GPT-2 XL in the dissonant update scenario. 
% The models overall fail to generalize effectively to paraphrased versions of the new conflicting facts, even when more neurons are allocated (Figures~\ref{fig:10x_10_gen}, \ref{fig:10x_100_gen}, and \ref{fig:10x_1000_gen}).

% \begin{figure}[h]
%     \centering
%     \captionsetup{font=small} % Optional: Adjust caption font size
%     \begin{tabular}{@{}c c c@{}}
%         \multicolumn{3}{c}{\textbf{Generalization Performance}} \\
%         \textbf{10 Facts} & \textbf{100 Facts} & \textbf{1000 Facts} \\
        
%         % Row 1: Same Condition as GPT2-small
%         \multicolumn{3}{c}{\textit{GPT2-XL under the same conditions as GPT2-small}} \\
%         \begin{subfigure}[b]{0.3\textwidth}
%             \centering
%             \includegraphics[width=\textwidth]{./figures/4_editing/experiment3_1_xl/experiment_3_1_2000_10_neuron_update_strategies_general_knowledge.pdf}
%             \subcaption{10 Facts}
%             \label{fig:same_10_gen}
%         \end{subfigure} &
%         \begin{subfigure}[b]{0.3\textwidth}
%             \centering
%             \includegraphics[width=\textwidth]{./figures/4_editing/experiment3_1_xl/experiment_3_1_20240719_094011_2000_100_neuron_update_strategies_general_knowledge.pdf}
%             \subcaption{100 Facts}
%             \label{fig:same_100_gen}
%         \end{subfigure} &
%         \begin{subfigure}[b]{0.3\textwidth}
%             \centering
%             \includegraphics[width=\textwidth]{./figures/4_editing/experiment3_1_xl/experiment_3_1_20240719_000646_2000_1000_neuron_update_strategies_general_knowledge.pdf}
%             \subcaption{1000 Facts}
%             \label{fig:same_1000_gen}
%         \end{subfigure} \\
%         \vspace{0.3cm} \\
    
%         % Row 2: 10X Neurons
%         \multicolumn{3}{c}{\textit{GPT2-XL with 10X more neurons}} \\
%         \begin{subfigure}[b]{0.3\textwidth}
%             \centering
%             \includegraphics[width=\textwidth]{./figures/4_editing/experiment3_1_xl/experiment_3_1_20000_10_neuron_update_strategies_general_knowledge.pdf}
%             \subcaption{10 Facts}
%             \label{fig:10x_10_gen}
%         \end{subfigure} &
%         \begin{subfigure}[b]{0.3\textwidth}
%             \centering
%             \includegraphics[width=\textwidth]{./figures/4_editing/experiment3_1_xl/experiment_3_1_20240719_044940_20000_100_neuron_update_strategies_general_knowledge.pdf}
%             \subcaption{100 Facts}
%             \label{fig:10x_100_gen}
%         \end{subfigure} &
%         \begin{subfigure}[b]{0.3\textwidth}
%             \centering
%             \includegraphics[width=\textwidth]{./figures/4_editing/experiment3_1_xl/experiment_3_1_20240718_160529_20000_1000_neuron_update_strategies_general_knowledge.pdf}
%             \subcaption{1000 Facts}
%             \label{fig:10x_1000_gen}
%         \end{subfigure} \\
%     \end{tabular}
%     \caption{Generalization Performance of GPT2-XL across different experimental conditions and numbers of conflicting facts. Note that y axes do not have the same scale. Indeed, finetuning performs much better in the case of 10 facts, compared to 100 or 1000 facts.}
%     \label{fig:generalization_performance}
% \end{figure}

% % \include{appendix/gpt2-large-dissonant}
% \subsubsection{Unrelated Old Knowledge Retention}\label{app:xl:old}

% Figures~\ref{fig:same_10_old}, \ref{fig:same_100_old}, and \ref{fig:same_1000_old} depict the retention of old knowledge following dissonant updates. Even when not effectively learning the new conflicting knowledge, GPT-2 XL experiences significant degradation in old knowledge accuracy, particularly when updating larger numbers of conflicting facts. Allocating more neurons exacerbates this effect (Figures~\ref{fig:10x_10_old}, \ref{fig:10x_100_old}, and \ref{fig:10x_1000_old}).


% \begin{figure}[h]
%     \centering
%     \captionsetup{font=small} % Optional: Adjust caption font size
%     \begin{tabular}{@{}c c c@{}}
%         \multicolumn{3}{c}{\textbf{Old Knowledge Retention}} \\
%         \textbf{10 Facts} & \textbf{100 Facts} & \textbf{1000 Facts} \\
        
%         % Row 1: Same Condition as GPT2-small
%         \multicolumn{3}{c}{\textit{GPT2-XL under the same conditions as GPT2-small}} \\
%         \begin{subfigure}[b]{0.3\textwidth}
%             \centering
%             \includegraphics[width=\textwidth]{./figures/4_editing/experiment3_1_xl/experiment_3_1_2000_10_neuron_update_strategies_old_knowledge.pdf}
%             \subcaption{10 Facts}
%             \label{fig:same_10_old}
%         \end{subfigure} &
%         \begin{subfigure}[b]{0.3\textwidth}
%             \centering
%             \includegraphics[width=\textwidth]{./figures/4_editing/experiment3_1_xl/experiment_3_1_20240719_094011_2000_100_neuron_update_strategies_old_knowledge.pdf}
%             \subcaption{100 Facts}
%             \label{fig:same_100_old}
%         \end{subfigure} &
%         \begin{subfigure}[b]{0.3\textwidth}
%             \centering
%             \includegraphics[width=\textwidth]{./figures/4_editing/experiment3_1_xl/experiment_3_1_20240719_000646_2000_1000_neuron_update_strategies_old_knowledge.pdf}
%             \subcaption{1000 Facts}
%             \label{fig:same_1000_old}
%         \end{subfigure} \\
%         \vspace{0.3cm} \\
        
%         % Row 2: 10X Neurons
%         \multicolumn{3}{c}{\textit{GPT2-XL with 10X more neurons}} \\
%         \begin{subfigure}[b]{0.3\textwidth}
%             \centering
%             \includegraphics[width=\textwidth]{./figures/4_editing/experiment3_1_xl/experiment_3_1_20000_10_neuron_update_strategies_old_knowledge.pdf}
%             \subcaption{10 Facts}
%             \label{fig:10x_10_old}
%         \end{subfigure} &
%         \begin{subfigure}[b]{0.3\textwidth}
%             \centering
%             \includegraphics[width=\textwidth]{./figures/4_editing/experiment3_1_xl/experiment_3_1_20240719_044940_20000_100_neuron_update_strategies_old_knowledge.pdf}
%             \subcaption{100 Facts}
%             \label{fig:10x_100_old}
%         \end{subfigure} &
%         \begin{subfigure}[b]{0.3\textwidth}
%             \centering
%             \includegraphics[width=\textwidth]{./figures/4_editing/experiment3_1_xl/experiment_3_1_20240718_160529_20000_1000_neuron_update_strategies_old_knowledge.pdf}
%             \subcaption{1000 Facts}
%             \label{fig:10x_1000_old}
%         \end{subfigure} \\
%     \end{tabular}
%     \caption{Old Knowledge Retention of GPT2-XL across different experimental conditions and numbers of conflicting facts.}
%     \label{fig:old_knowledge_retention}
% \end{figure}
% \end{comment}















%\section{Extended related work}\label{app:related}
%In this section, we provide an extended version of Tab.~\ref{tab:continual-learning-taxonomy}, focusing \textit{only} on the \textit{most recent literature}, and showing how our work is uniquely positioned in the landscape of model editing and continual learning, the two key related branches to our work.

\begin{table}[t]
\centering
\caption{Extended taxonomy of incremental Learning Approaches, showing some seminal work (top) and more recent literature (split into editing and continual learning).}
\resizebox{\textwidth}{!}{%
\begin{tabular}{@{}l p{3cm} c c c c c c@{}}
\toprule
\textbf{Examples} & \textbf{\makecell{Incremental \\ Type}} & \textbf{\makecell{Memory \\ Usage}} & \textbf{\makecell{Task \\ Awareness}} & \textbf{\makecell{Weight \\ Plasticity}} & \textbf{\makecell{Architecture}} & \textbf{\makecell{Conflict \\ Detection}} & \textbf{\makecell{Update \\ Mechanism}} \\ \midrule
% Classic/Foundational Approaches
iCaRL~\citep{rebuffi2017icarl} & Class-incremental & Replay & Task-Agnostic & Fixed & Fixed & No & Rehearsal \\
EWC~\citep{kirkpatrick2017overcoming} & Task-incremental & None & Task-Aware & Selective & Fixed & No & Regularization \\
Progressive Nets~\citep{rusu2016progressive} & Task-incremental & None & Task-Aware & Fixed & Expanding & No & New Subnetworks \\
DEN~\citep{yoon2017lifelong} & Task-incremental & None & Task-Aware & Selective & Expanding & No & Selective Expansion \\
GEM~\citep{lopez2017gradient} & Task-incremental & Replay & Task-Aware & Constrained & Fixed & No & Constrained Optimization \\
ROME~\citep{DeCao2021} & Fact-incremental & None & Fact-Aware & Localized & Fixed & No & Rank-One Update \\
OWM~\citep{zeng2019continual} & Task-incremental & None & Task-Aware & Orthogonal & Fixed & No & Orthogonal Projection \\
PackNet~\citep{mallya2018packnet} & Task-incremental & None & Task-Aware & Selective & Fixed & No & Weight Masking \\
HAT~\citep{serra2018overcoming} & Task-incremental & None & Task-Aware & Selective & Fixed & No & Attention Masking \\ \midrule
% Recent Knowledge Editing Approaches
MALMEN~\citep{tan2023massive} & Fact-incremental & None & Fact-Aware & Localized & Fixed & No & Parameter Shift Aggregation \\
EditAnalysis~\citep{li2023unveiling} & Fact-incremental & None & Fact-Aware & Analysis & Fixed & No & Consistency Analysis \\
D4S~\citep{huang2024reasons} & Fact-incremental & O(1) & Fact-Aware & Regulated & Fixed & No & Layer-Norm Control \\ \midrule
% Recent Continual Learning Approaches
Global Prototypes~\citep{baicontinual} & Task/Class-incremental & None & Task-Agnostic & Selective & Fixed & No & Global Prototype Alignment \\
NTE~\citep{benjamin2024continual} & Task-incremental & None & Task-Agnostic & Selective & Fixed & No & Bayesian Ensemble \\
UPGD~\citep{elsayed2024addressing} & Task-incremental & None & Task-Agnostic & Selective & Fixed & No & Utility-Gated Updates \\
~\citep{hiratani2024disentangling} & Task-incremental & None & Task-Aware & Selective & Fixed & No & Fisher Information \\
CLAP~\citep{jha2024clap4clip} & Class-incremental & None & Task-Aware & Selective & Fixed & No & Probabilistic Adaptation \\
VQ-Prompt~\citep{jiao2024vector} & Class-incremental & None & Task-Agnostic & Fixed & Fixed & No & Discrete Prompt Selection \\
IsCiL~\citep{lee2024incremental} & Task-incremental & None & Task-Aware & Selective & Fixed & No & Skill-based Adaptation \\
BGS~\citep{leecontinual} & Task/Domain/Class-incremental & Replay & Task-Aware & Selective & Fixed & Yes & Bias-Aware Update \\
SLM~\citep{peng2024scalable} & Task-incremental & None & Auto-detected & Selective & Fixed & No & Vector Space Retrieval \\
Train-Attention~\citep{seo2024train} & Knowledge-incremental & None & Task-Agnostic & Selective & Fixed & No & Token-Weighted Update \\
Refresh Learning~\citep{wang2024unified} & Task/Class-incremental & Optional & Task-Aware & Selective & Fixed & No & Unlearn-Relearn \\
RAIL~\citep{xu2024advancing} & Cross-domain-incremental & None & Task-Agnostic & Selective & Fixed & No & Regression-based Update \\
SAFE~\citep{zhao2024safeslowfastparameterefficient} & Class-incremental & None & Task-Agnostic & Selective & Fixed & No & Dual Parameter-Efficient Tuning \\ \midrule
% Your Paper
\textbf{This paper} & Fact-incremental & None & Conflict-Aware & Selective & Fixed & Yes & Neuron-Specific Update \\ \bottomrule
\end{tabular}%
}
\label{tab:continual-learning-taxonomy-extended}
\end{table}


\subsection{Continual learning}
Continual Learning (CL) methods enable models to learn new tasks without catastrophically forgetting previously mastered ones \citep{kirkpatrick2017overcoming}. These approaches fall into three main families: memory-based methods using exemplar buffers \citep{rebuffi2017icarl}, knowledge distillation techniques that transfer information across model versions \citep{lopez2017gradient}, and regularization-based methods that constrain weight updates \citep{kirkpatrick2017overcoming}. To ease the understanding of this landscape, we build a taxonomy that characterizes approaches by their incremental type (task, class, or fact-based), memory requirements, update mechanisms, and architectural constraints (Tab.~\ref{tab:continual-learning-taxonomy}). This taxonomy reveals how our work is different from existing continual learning attempts: while existing methods focus on preserving knowledge across distinct tasks, none explicitly address the detection and handling of conflicting information - a key capability in human cognition that our work empirically investigates.

One of the closest old approaches is deep mind's EWC~\citep{kirkpatrick2017overcoming}, a method designed to mitigate catastrophic forgetting in neural networks trained sequentially on distinct tasks. The core idea is to protect the most important weights (or neurons) for previously learned tasks during the training of new tasks. EWC identifies these important weights by calculating the Fisher Information Matrix during or after the training of a task, which estimates how sensitive each weight is to the task’s performance. Weights that significantly impact the output for a given task are marked as important. A quadratic penalty is then applied during future learning, constraining these weights to remain close to their values from the previous task. This ensures that knowledge from earlier tasks is preserved while still allowing the model to adapt to new tasks. However, EWC is \textbf{less suitable for LLMs}, which \textbf{do not have clearly defined tasks} when it comes to knowledge ingestion (probably different for other types of skills). EWC's effectiveness relies on distinct task boundaries and the ability to compute task-specific importance for weights, which is feasible in scenarios with well-defined tasks, such as classification or reinforcement learning. In LLMs, where learning spans a wide range of topics and linguistic structures without clear task delineation, it’s challenging to apply EWC's task-based strategy. The model would struggle to assign specific neurons or weights to individual tasks or concepts, making it difficult to protect task-specific knowledge without hindering the model’s overall generalization ability across a diverse dataset.

We cite in the remainder more recent literature that we project onto our taxonomy.

\citet{baicontinual} introduce a novel approach to continual learning that leverages global prototypes to mitigate catastrophic forgetting in neural networks. Their key insight is that maintaining stable connections between task-specific representations and pre-learned, general-purpose token embeddings (which serve as global prototypes) can significantly reduce forgetting without requiring explicit replay mechanisms. Through empirical validation on both task-incremental and class-incremental NLP scenarios, they demonstrate that models preserving strong connections to these global prototypes exhibit enhanced stability. While their work shares our goal of preserving knowledge during updates, it differs fundamentally in its approach and granularity: where they focus on task-level knowledge preservation through architectural mechanisms, our work addresses the more specific challenge of managing contradictory factual updates through cognitive-inspired conflict detection. Their finding that stable reference points aid knowledge retention is conceptually relevant to our work, though our results suggest that such architectural approaches alone may be insufficient when handling explicitly contradictory information, where more sophisticated cognitive mechanisms become necessary.

\citet{benjamin2024continual} proposed an elegant theoretical framework that interprets neural networks as Bayesian ensembles of classifiers. Their key insight is that a neural network with N parameters can be viewed as a weighted ensemble of N classifiers in the lazy regime, where the classifiers remain fixed throughout learning. This interpretation reveals that a properly designed posterior update rule, resembling SGD without momentum, can enable continual learning without forgetting - notably, they prove that momentum actually exacerbates forgetting. While their work focuses on preserving all knowledge in task-incremental learning, our paper specifically examines cases where knowledge needs to be deliberately updated or overridden. Their key contribution is showing that catastrophic forgetting is linked to the transition from lazy to rich regimes in neural networks, providing both a theoretical explanation for why larger models are more robust to forgetting and a biologically-inspired mechanism for knowledge preservation that perhaps complements our cognitive-based approach.

\citet{elsayed2024addressing} propose UPGD (Utility-based Perturbed Gradient Descent), a novel approach targeting both catastrophic forgetting and loss of plasticity in streaming learning scenarios. Their method protects useful network units while maintaining plasticity in less-used ones through utility-gated gradient updates and perturbations. Unlike previous approaches requiring task boundaries or memory buffers, UPGD operates in a challenging streaming setting with continuous non-stationarity. Using their newly introduced direct plasticity metric, they demonstrate UPGD's ability to maintain performance levels that surpass or match existing methods. This work complements our investigation by providing evidence that selective neuronal updates based on utility metrics can effectively balance stability and plasticity, though in a task-learning rather than knowledge-updating context.

\citet{hiratani2024disentangling} analyze how task similarity affects continual learning through a novel theoretical framework combining teacher-student models with latent structure. Their key insight is that high input feature similarity coupled with low readout similarity leads to catastrophic outcomes in both knowledge transfer and retention, even when tasks are positively correlated. They demonstrate that weight regularization in the Fisher information metric robustly helps retention regardless of task similarity, while common approaches like activity gating improve retention at the cost of transfer performance. Their theoretical predictions are validated on permuted MNIST tasks with latent variables.

\citet{jha2024clap4clip} propose a probabilistic approach to continual learning for vision-language models, specifically focusing on CLIP adaptation. Their method, CLAP, introduces visual-guided attention and task-specific probabilistic adapters to model the distribution of text features, while leveraging CLIP's pre-trained knowledge for initialization and regularization. This work demonstrates that probabilistic modeling can significantly reduce catastrophic forgetting in class-incremental learning scenarios, achieving state-of-the-art performance across multiple benchmarks.

\citet{jiao2024vector} propose VQ-Prompt, a novel prompt-based continual learning framework that addresses class-incremental learning with pretrained vision transformers. Their key innovation is incorporating vector quantization into prompt selection, enabling end-to-end optimization of discrete prompts with task loss while maintaining effective knowledge abstraction. This contrasts with our cognitive-dissonance aware approach, as they focus on task adaptation through prompt engineering rather than explicit conflict detection. Their empirical results on ImageNet-R and CIFAR-100 demonstrate superior performance compared to existing prompt-based methods, suggesting the effectiveness of discrete knowledge representation in continual learning.

\cite{lee2024incremental} propose IsCiL, a framework for continual imitation learning that uses retrievable skills and adapter-based architecture to enable efficient knowledge sharing across tasks. Unlike traditional approaches that isolate task-specific parameters, IsCiL introduces a prototype-based skill retrieval mechanism that allows selective reuse of previously learned skills for new tasks. While focused primarily on motor skills rather than resolving knowledge contradictions, their empirical results show that this selective adaptation approach significantly improves sample efficiency and reduces catastrophic forgetting compared to other adapter-based methods, particularly in scenarios with incomplete demonstrations.

\citet{leecontinual} present a systematic empirical investigation of how dataset bias affects continual learning. Through carefully designed experiments across task-incremental, domain-incremental, and class-incremental scenarios, they reveal that bias transfers both forward and backward between tasks. Their analysis shows that CL methods focusing on stability tend to preserve and propagate biases from previous tasks, while emphasis on plasticity allows new biases to contaminate previous knowledge. Based on these insights, they propose BGS (Balanced Greedy Sampling), a method that mitigates bias transfer by maintaining a balanced exemplar memory and retraining the classification head. Note that here, we used ``Replay'' for Memory Usage in the table since their best performing method (BGS) uses an exemplar memory, but they also evaluate methods without memory. 

\citet{peng2024scalable} proposed a continual learning approach that automates task selection through vector space retrieval, eliminating the need for explicit task IDs, experience replay, or optimization constraints. Their method, Scalable Language Model (SLM), combines Joint Adaptive Re-parameterization with dynamic knowledge retrieval to automatically identify relevant parameters for each input, enabling task-agnostic updates. While achieving state-of-the-art results across diverse tasks and model scales (BERT, T5, LLaMA-2), their key contribution is demonstrating that automatic task identification and parameter selection can enable continual learning without requiring explicit task boundaries or memory buffers.

\citet{seo2024train} presented Train-Attention, an interesting meta-learning approach for continual knowledge learning (CKL) in LLMs that predicts and applies weights to tokens \textit{based on their usefulness for future tasks}. Unlike previous approaches that uniformly update all parameters, their method enables\textit{ targeted knowledge updates by learning which tokens are most important} to focus on. Through experiments on LAMA-CKL and TemporalWiki benchmarks, they show that selective token-weighted learning significantly reduces catastrophic forgetting while improving learning speed. The work somewhat complements our cognitive-inspired approach, and demonstrates the benefits of selective attention, but it does not explicitly address the handling of contradictory information.

\citet{wang2024unified} proposed a unified framework for continual learning that reveals common mathematical structures across seemingly distinct approaches (regularization-based, Bayesian-based, and memory-replay). Building on this unification, they introduce ``refresh learning" - a plug-in mechanism that first unlearns current data before relearning it, inspired by the beneficial role of forgetting in human cognition. Their work primarily focuses on task-incremental and class-incremental scenarios, demonstrating improved accuracy across CIFAR and Tiny-ImageNet benchmarks. While their approach differs from our fact-level knowledge updates in LLMs, their findings about selective forgetting complement our observations about cognitive-inspired update mechanisms. Their theoretical analysis showing that refresh learning improves the flatness of the loss landscape offers an interesting perspective on how controlled forgetting might benefit knowledge retention in neural networks.

\citet{xu2024advancing} propose a cross-domain task-agnostic incremental learning framework (X-TAIL) for vision-language models, focusing on the challenge of preserving both incrementally learned knowledge and zero-shot abilities. Their approach, RAIL, uses recursive ridge regression with non-linear projections to adapt to new domains without catastrophic forgetting. Unlike previous work requiring domain identity hints or reference datasets, RAIL can classify images across both seen and unseen domains without domain hints, demonstrating superior performance in both discriminative ability and knowledge preservation. While their work advances the technical aspects of continual learning, it differs from our cognitive-inspired investigation as it doesn't address the fundamental challenge of detecting and resolving conflicting knowledge, instead focusing on domain adaptation without explicit conflict awareness.

\citet{zhao2024safeslowfastparameterefficient} propose a class-incremental learning framework for pre-trained vision models that balances stability and plasticity through two complementary parameter-efficient tuning mechanisms. Their SAFE approach first inherits generalizability from pre-trained models via a ``slow learner" that captures transferable knowledge in the first session, then maintains plasticity through a ``fast learner" that continuously adapts to new classes while resisting catastrophic forgetting. While focused on vision tasks rather than language models, their dual-speed learning strategy presents interesting parallels to our cognitive-inspired approach – particularly in how both works identify the importance of selective plasticity and the distinction between stable (``stubborn") and adaptable (``plastic") parameters. However, SAFE doesn't address the fundamental challenge of detecting and handling contradictory information that we identify as crucial for true cognitive-inspired learning.

\textit{Unlike the above work, our goal is to understand the fundamental cognitive mechanisms underlying the continuous knowledge updates in LLMs, particularly focusing on how models can detect and react to contradictory information. Rather than proposing a new continual learning method, we provide crucial insights into how different types of knowledge updates affect model behavior and stability.}

\subsection{Knowledge editing}
Next, a big portion of recent literature has focused on understanding and modifying the internal knowledge of Large Language Models (LLMs), post-training. Such knowledge editing aims to alter specific facts or associations within the model without the need for full retraining. 

\citet{Geva2020} were among the first to show that transformer Feed-Forward Network (FFN) layers act as unnormalized key-value stores encoding relational knowledge inside LLMs. This observation was later confirmed and complemented by others \citep{Meng2022,Dai2021} before being leveraged by subsequent work to master the editing of internal memories.
\citet{Meng2022} introduced ROME (Rank-One Model Editing), a method that uses causal tracing to empirically locate the layers essential to encoding a given association. They then modify these modules by applying small rank-one changes. To identify the relevant modules, they run the network multiple times, introducing corruptions to the input sequence to disturb the inference, and then restore individual states from the original non-corrupted pass. But this work an others worked only on single edits, and were often evaluated one edit at a time, starting each time from a fresh pre-trained model. The same authors later developed MEMIT, which follows the same causal tracing principle but with the goal of scaling up to 10,000 edits in bulk\citep{Meng2022a}.
Similarly, \citet{Dai2021} leveraged the identification of knowledge neurons to perform ``knowledge surgery" – editing factual knowledge within Transformers without the need for additional fine-tuning.
\citet{Zhu2020} approached the knowledge modification task as a constrained optimization problem. Their work found that constrained layer-wise fine-tuning emerges as an effective method for modifying the knowledge that Transformers learn, suggesting a different pathway for knowledge editing inside LLMs.
\citet{DeCao2021} proposed \textsc{KnowledgeEditor}, which achieved knowledge editing by training a hyper-network with constrained optimization to modify specific facts without fine-tuning or changing the overall stored knowledge. The method was demonstrated on smaller models like BERT for fact-checking and BART for question answering, achieving consistent changes in predictions across different formulations of queries.

\citet{li2023unveiling}  empirically investigate the pitfalls of knowledge editing in LLMs, revealing two critical issues: logical inconsistencies between multiple edits (like contradictory relationship updates) and knowledge distortion (where edits irreversibly damage the model's knowledge structure). Through carefully designed benchmarks \textsc{ConflictEdit} and \textsc{RoundEdit}, they demonstrate that current editing methods struggle with these challenges, particularly when handling reverse relationships or composite logical rules. While their work focuses on identifying limitations in maintaining logical consistency across edits, our paper takes a complementary cognitive-inspired perspective by addressing how models handle contradictions with their existing knowledge base. Their findings about knowledge distortion align with and reinforce our observations about the catastrophic nature of updates that modify existing knowledge.

Similarly, \citet{huang2024reasons} empirically investigate causes of performance degradation during knowledge editing in LLMs. They show degradation correlates with editing target complexity and L1-norm growth in edited layers. Their proposed Dump for Sequence (D4S) method regulates layer norm growth using O(1) space complexity, enabling multiple effective updates while minimizing model degradation. Their work provides valuable insights into the mechanisms of model degradation during knowledge editing, but it does not specifically address the distinction between contradictory and non-contradictory updates, as we do in this paper.

\citet{tan2023massive} propose MALMEN, a scalable hypernetwork approach for editing Large Language Models by aggregating parameter shifts using a least-squares formulation. While previous editing methods like MEND~\citep{mitchell2022fast} could handle only a few facts simultaneously, MALMEN can efficiently edit thousands of facts while maintaining comparable performance. Their key innovation lies in separating the computation between the hypernetwork and LM, enabling arbitrary batch sizes and reducing memory requirements. Their empirical results show that MALMEN can edit hundreds of times more facts than MEND while maintaining similar performance levels, though they note that the method still struggles with generalizing to rephrasing not seen during training. Like other editing approaches, MALMEN focuses on the mechanics of (by design conflicting) updates.

\textit{Unlike all the work above, our goal in this work is not to edit knowledge, but to understand the fundamental mechanisms and phenomena that govern how LLMs integrate new information with existing knowledge. By taking a cognitive-inspired approach focused on dissonance awareness and adaptive plasticity, we reveal critical insights about the nature of knowledge representation and updating in these models.}


% https://proceedings.neurips.cc/paper/2021/hash/2a10665525774fa2501c2c8c4985ce61-Abstract.html
% https://proceedings.neurips.cc/paper_files/paper/2022/hash/80133d0f6eccaace15508f91e3c5a93c-Abstract-Conference.html
% https://arxiv.org/abs/1903.04476
% https://proceedings.mlr.press/v199/abbasi22a
% https://openaccess.thecvf.com/content/CVPR2022/html/Yan_Learning_Bayesian_Sparse_Networks_With_Full_Experience_Replay_for_Continual_CVPR_2022_paper.html
% https://proceedings.neurips.cc/paper/2020/hash/258be18e31c8188555c2ff05b4d542c3-Abstract.html
% https://proceedings.mlr.press/v234/yildirim24a.html
% https://www.sciencedirect.com/science/article/pii/S0925231221001545
% https://openaccess.thecvf.com/content/ICCV2023/html/Jin_Growing_a_Brain_with_Sparsity-Inducing_Generation_for_Continual_Learning_ICCV_2023_paper.html
% https://proceedings.neurips.cc/paper/2021/hash/f1e709e6aef16ba2f0cd6c7e4f52b9b6-Abstract.html
% https://arxiv.org/abs/2401.05667
% https://openaccess.thecvf.com/content_CVPR_2019/html/Ostapenko_Learning_to_Remember_A_Synaptic_Plasticity_Driven_Framework_for_Continual_CVPR_2019_paper.html
% https://arxiv.org/abs/2206.09117
% https://arxiv.org/abs/2403.08763
% https://arxiv.org/abs/1803.03635
% https://arxiv.org/abs/2406.05955
% https://arxiv.org/abs/2402.01089
% https://arxiv.org/abs/2202.12002
% https://arxiv.org/abs/2404.16789
% https://arxiv.org/abs/2406.17245
% https://arxiv.org/abs/2205.12393
% https://openaccess.thecvf.com/content/CVPR2024/html/Roy_Convolutional_Prompting_meets_Language_Models_for_Continual_Learning_CVPR_2024_paper.html
% https://arxiv.org/abs/2110.03215
% https://arxiv.org/abs/2405.18653
% https://arxiv.org/abs/2012.15283
% https://arxiv.org/abs/2205.09357
% https://arxiv.org/abs/2301.12314
% https://aclanthology.org/2020.coling-main.318/
% https://arxiv.org/abs/2401.03129
% https://arxiv.org/abs/2302.03241
% https://arxiv.org/abs/2309.14763
% https://arxiv.org/abs/2406.06962
% https://arxiv.org/abs/2406.01392
% https://arxiv.org/abs/2404.00790
% https://arxiv.org/abs/2406.18708
% https://arxiv.org/abs/2406.01375
% https://www.cell.com/trends/cognitive-sciences/fulltext/S1364-66132030219-9
% https://openreview.net/forum?id=DimPeeCxKO
% https://arxiv.org/pdf/1612.00796


% New List: ongoing check of ICLR'24 and Nips'24:
% https://openreview.net/pdf?id=8SDsff42Lj
% https://arxiv.org/pdf/2408.17394
% https://arxiv.org/pdf/2410.20444
% https://cdn.iiit.ac.in/cdn/precog.iiit.ac.in/pubs/NeurIPS-RanDumb.pdf
% https://arxiv.org/pdf/2407.16920v1
% https://arxiv.org/pdf/2405.20236
% https://arxiv.org/pdf/2410.22658
% https://arxiv.org/pdf/2406.18868
% https://arxiv.org/pdf/2403.19137
% https://arxiv.org/pdf/2411.02175
% https://arxiv.org/pdf/2410.23843
% https://arxiv.org/pdf/2404.07470
% https://arxiv.org/pdf/2404.00781
% https://arxiv.org/pdf/2403.13249
% https://arxiv.org/pdf/2303.11863
% https://arxiv.org/pdf/2311.04661
% https://arxiv.org/pdf/2310.02129




% Interesting work:
% https://arxiv.org/pdf/2403.19137 
%The paper~\citep{jha2024clap4clip} tackles class-incremental continual learning with CLIP, where new groups of classes (tasks) arrive sequentially and the model must learn to classify them without forgetting previous classes. For example, in CIFAR100, they split 100 classes into 10 tasks of 10 classes each. Building on previous work that showed CLIP needs task-specific finetuning (either through learned prompts like CoOp or feature adapters like CLIP-Adapter), they identify a key problem: during continual learning, the adapted text features increasingly deviate from CLIP's frozen image features (measured by increasing angular distance), hurting performance. Their solution has three main components: (1) a visual-guided attention module that helps maintain alignment between visual and text features during adaptation, (2) task-specific probabilistic adapters that learn distributions over text features rather than point estimates, and (3) using CLIP's pre-trained language knowledge to initialize and regularize these adapters. They show this approach reduces cross-modal deviation and outperforms previous methods across multiple datasets. While simpler solutions might be possible (like directly enforcing alignment or transforming frozen features), the paper takes a more complex probabilistic approach that enables additional benefits like uncertainty estimation.

%https://nips.cc/virtual/2024/poster/94505
% Super interesting work, one of the closest to ours: shows that 

%https://arxiv.org/pdf/2404.00781
% Our work closely relates to the recent UPGD (Utility-based Perturbed Gradient Descent) approach [https://arxiv.org/pdf/2404.00781], which similarly investigates plasticity/stability mechanisms for continual learning in deep learning more generally. UPGD proposes a utility-based approach that protects important weights while allowing updates to less useful ones, combining gradient updates with perturbations based on a carefully approximated utility measure. While both works share similar motivations and empirical findings about the importance of selective plasticity, our work specifically focuses on cognitive dissonance awareness and the particular challenges of conflicting updates in language models. Our findings complement theirs by demonstrating that while non-conflicting updates can be handled relatively well with various strategies (including simple fine-tuning), conflicting updates pose a unique challenge that requires explicit conflict detection and resolution mechanisms. Additionally, while UPGD uses a theoretically-grounded utility approximation to guide updates, we show that even simple gradient-based historical tracking can effectively identify appropriate regions for targeted updates in practice.

% \begin{comment}
% \subsection{Raw list of continual learning}
% % does not compile wtf
% \begin{itemize}
%     \item \href{https://proceedings.neurips.cc/paper/2021/hash/2a10665525774fa2501c2c8c4985ce61-Abstract.html}{https://proceedings.neurips.cc/paper/2021/hash/2a10665525774fa2501c2c8c4985ce61-Abstract.html}
%     \item \href{https://proceedings.neurips.cc/paper_files/paper/2022/hash/80133d0f6eccaace15508f91e3c5a93c-Abstract-Conference.html}{https://proceedings.neurips.cc/paper_files/paper/2022/hash/80133d0f6eccaace15508f91e3c5a93c-Abstract-Conference.html}
%     \item \href{https://arxiv.org/abs/1903.04476}{https://arxiv.org/abs/1903.04476}
%     \item \href{https://proceedings.mlr.press/v199/abbasi22a}{https://proceedings.mlr.press/v199/abbasi22a}
%     \item \href{https://openaccess.thecvf.com/content/CVPR2022/html/Yan_Learning_Bayesian_Sparse_Networks_With_Full_Experience_Replay_for_Continual_CVPR_2022_paper.html}{https://openaccess.thecvf.com/content/CVPR2022/html/Yan_Learning_Bayesian_Sparse_Networks_With_Full_Experience_Replay_for_Continual_CVPR_2022_paper.html}
%     \item \href{https://proceedings.neurips.cc/paper/2020/hash/258be18e31c8188555c2ff05b4d542c3-Abstract.html}{https://proceedings.neurips.cc/paper/2020/hash/258be18e31c8188555c2ff05b4d542c3-Abstract.html}
%     \item \href{https://proceedings.mlr.press/v234/yildirim24a.html}{https://proceedings.mlr.press/v234/yildirim24a.html}
%     \item \href{https://www.sciencedirect.com/science/article/pii/S0925231221001545}{https://www.sciencedirect.com/science/article/pii/S0925231221001545}
%     \item \href{https://openaccess.thecvf.com/content/ICCV2023/html/Jin_Growing_a_Brain_with_Sparsity-Inducing_Generation_for_Continual_Learning_ICCV_2023_paper.html}{https://openaccess.thecvf.com/content/ICCV2023/html/Jin_Growing_a_Brain_with_Sparsity-Inducing_Generation_for_Continual_Learning_ICCV_2023_paper.html}
%     \item \href{https://proceedings.neurips.cc/paper/2021/hash/f1e709e6aef16ba2f0cd6c7e4f52b9b6-Abstract.html}{https://proceedings.neurips.cc/paper/2021/hash/f1e709e6aef16ba2f0cd6c7e4f52b9b6-Abstract.html}
%     \item \href{https://arxiv.org/abs/2401.05667}{https://arxiv.org/abs/2401.05667}
%     \item \href{https://openaccess.thecvf.com/content_CVPR_2019/html/Ostapenko_Learning_to_Remember_A_Synaptic_Plasticity_Driven_Framework_for_Continual_CVPR_2019_paper.html}{https://openaccess.thecvf.com/content_CVPR_2019/html/Ostapenko_Learning_to_Remember_A_Synaptic_Plasticity_Driven_Framework_for_Continual_CVPR_2019_paper.html}
%     \item \href{https://arxiv.org/abs/2206.09117}{https://arxiv.org/abs/2206.09117}
%     \item \href{https://arxiv.org/abs/2403.08763}{https://arxiv.org/abs/2403.08763}
%     \item \href{https://arxiv.org/abs/1803.03635}{https://arxiv.org/abs/1803.03635}
%     \item \href{https://arxiv.org/abs/2406.05955}{https://arxiv.org/abs/2406.05955}
%     \item \href{https://arxiv.org/abs/2402.01089}{https://arxiv.org/abs/2402.01089}
%     \item \href{https://arxiv.org/abs/2202.12002}{https://arxiv.org/abs/2202.12002}
%     \item \href{https://arxiv.org/abs/2404.16789}{https://arxiv.org/abs/2404.16789}
%     \item \href{https://arxiv.org/abs/2406.17245}{https://arxiv.org/abs/2406.17245}
%     \item \href{https://arxiv.org/abs/2205.12393}{https://arxiv.org/abs/2205.12393}
%     \item \href{https://openaccess.thecvf.com/content/CVPR2024/html/Roy_Convolutional_Prompting_meets_Language_Models_for_Continual_Learning_CVPR_2024_paper.html}{https://openaccess.thecvf.com/content/CVPR2024/html/Roy_Convolutional_Prompting_meets_Language_Models_for_Continual_Learning_CVPR_2024_paper.html}
%     \item \href{https://arxiv.org/abs/2110.03215}{https://arxiv.org/abs/2110.03215}
%     \item \href{https://arxiv.org/abs/2405.18653}{https://arxiv.org/abs/2405.18653}
%     \item \href{https://arxiv.org/abs/2012.15283}{https://arxiv.org/abs/2012.15283}
%     \item \href{https://arxiv.org/abs/2205.09357}{https://arxiv.org/abs/2205.09357}
%     \item \href{https://arxiv.org/abs/2301.12314}{https://arxiv.org/abs/2301.12314}
%     \item \href{https://aclanthology.org/2020.coling-main.318/}{https://aclanthology.org/2020.coling-main.318/}
%     \item \href{https://arxiv.org/abs/2401.03129}{https://arxiv.org/abs/2401.03129}
%     \item \href{https://arxiv.org/abs/2302.03241}{https://arxiv.org/abs/2302.03241}
%     \item \href{https://arxiv.org/abs/2309.14763}{https://arxiv.org/abs/2309.14763}
%     \item \href{https://arxiv.org/abs/2406.06962}{https://arxiv.org/abs/2406.06962}
%     \item \href{https://arxiv.org/abs/2406.01392}{https://arxiv.org/abs/2406.01392}
%     \item \href{https://arxiv.org/abs/2404.00790}{https://arxiv.org/abs/2404.00790}
%     \item \href{https://arxiv.org/abs/2406.18708}{https://arxiv.org/abs/2406.18708}
%     \item \href{https://arxiv.org/abs/2406.01375}{https://arxiv.org/abs/2406.01375}
%     \item \href{https://www.cell.com/trends/cognitive-sciences/fulltext/S1364-66132030219-9}{https://www.cell.com/trends/cognitive-sciences/fulltext/S1364-66132030219-9}
%     \item \href{https://openreview.net/forum?id=DimPeeCxKO}{https://openreview.net/forum?id=DimPeeCxKO}
%     \item \href{https://arxiv.org/pdf/1612.00796}{Elastic weight consolidation deepmind 2017} **Elastic Weight Consolidation (EWC)** is a method designed to mitigate catastrophic forgetting in neural networks trained sequentially on distinct tasks. The core idea is to protect the most important weights (or neurons) for previously learned tasks during the training of new tasks. EWC identifies these important weights by calculating the **Fisher Information Matrix** during or after the training of a task, which estimates how sensitive each weight is to the task’s performance. Weights that significantly impact the output for a given task are marked as important. A **quadratic penalty** is then applied during future learning, constraining these weights to remain close to their values from the previous task. This ensures that knowledge from earlier tasks is preserved while still allowing the model to adapt to new tasks.

% However, **EWC is less suitable for large language models (LLMs)**, which are typically trained on massive datasets that do not have clearly defined tasks. EWC's effectiveness relies on distinct task boundaries and the ability to compute task-specific importance for weights, which is feasible in scenarios with well-defined tasks, such as classification or reinforcement learning. In LLMs, where learning spans a wide range of topics and linguistic structures without clear task delineation, it’s challenging to apply EWC's task-based strategy. The model would struggle to assign specific neurons or weights to individual tasks or concepts, making it difficult to protect task-specific knowledge without hindering the model’s overall generalization ability across a diverse dataset.
% \end{itemize}

% % https://proceedings.neurips.cc/paper/2021/hash/2a10665525774fa2501c2c8c4985ce61-Abstract.html
% % https://proceedings.neurips.cc/paper_files/paper/2022/hash/80133d0f6eccaace15508f91e3c5a93c-Abstract-Conference.html
% % https://arxiv.org/abs/1903.04476
% % https://proceedings.mlr.press/v199/abbasi22a
% % https://openaccess.thecvf.com/content/CVPR2022/html/Yan_Learning_Bayesian_Sparse_Networks_With_Full_Experience_Replay_for_Continual_CVPR_2022_paper.html
% % https://proceedings.neurips.cc/paper/2020/hash/258be18e31c8188555c2ff05b4d542c3-Abstract.html
% % https://proceedings.mlr.press/v234/yildirim24a.html
% % https://www.sciencedirect.com/science/article/pii/S0925231221001545
% % https://openaccess.thecvf.com/content/ICCV2023/html/Jin_Growing_a_Brain_with_Sparsity-Inducing_Generation_for_Continual_Learning_ICCV_2023_paper.html
% % https://proceedings.neurips.cc/paper/2021/hash/f1e709e6aef16ba2f0cd6c7e4f52b9b6-Abstract.html
% % https://arxiv.org/abs/2401.05667
% % https://openaccess.thecvf.com/content_CVPR_2019/html/Ostapenko_Learning_to_Remember_A_Synaptic_Plasticity_Driven_Framework_for_Continual_CVPR_2019_paper.html
% % https://arxiv.org/abs/2206.09117
% % https://arxiv.org/abs/2403.08763
% % https://arxiv.org/abs/1803.03635
% % https://arxiv.org/abs/2406.05955
% % https://arxiv.org/abs/2402.01089
% % https://arxiv.org/abs/2202.12002
% % https://arxiv.org/abs/2404.16789
% % https://arxiv.org/abs/2406.17245
% % https://arxiv.org/abs/2205.12393
% % https://openaccess.thecvf.com/content/CVPR2024/html/Roy_Convolutional_Prompting_meets_Language_Models_for_Continual_Learning_CVPR_2024_paper.html
% % https://arxiv.org/abs/2110.03215
% % https://arxiv.org/abs/2405.18653
% % https://arxiv.org/abs/2012.15283
% % https://arxiv.org/abs/2205.09357
% % https://arxiv.org/abs/2301.12314
% % https://aclanthology.org/2020.coling-main.318/
% % https://arxiv.org/abs/2401.03129
% % https://arxiv.org/abs/2302.03241
% % https://arxiv.org/abs/2309.14763
% % https://arxiv.org/abs/2406.06962
% % https://arxiv.org/abs/2406.01392
% % https://arxiv.org/abs/2404.00790
% % https://arxiv.org/abs/2406.18708
% % https://arxiv.org/abs/2406.01375
% % https://www.cell.com/trends/cognitive-sciences/fulltext/S1364-66132030219-9
% % https://openreview.net/forum?id=DimPeeCxKO
% % https://arxiv.org/pdf/1612.00796


% \subsection{Raw list of continual learning}
% % does not compile wtf
% \begin{itemize}
%     \item \href{https://proceedings.neurips.cc/paper/2021/hash/2a10665525774fa2501c2c8c4985ce61-Abstract.html}{https://proceedings.neurips.cc/paper/2021/hash/2a10665525774fa2501c2c8c4985ce61-Abstract.html}
%     \item \href{https://proceedings.neurips.cc/paper_files/paper/2022/hash/80133d0f6eccaace15508f91e3c5a93c-Abstract-Conference.html}{https://proceedings.neurips.cc/paper_files/paper/2022/hash/80133d0f6eccaace15508f91e3c5a93c-Abstract-Conference.html}
%     \item \href{https://arxiv.org/abs/1903.04476}{https://arxiv.org/abs/1903.04476}
%     \item \href{https://proceedings.mlr.press/v199/abbasi22a}{https://proceedings.mlr.press/v199/abbasi22a}
%     \item \href{https://openaccess.thecvf.com/content/CVPR2022/html/Yan_Learning_Bayesian_Sparse_Networks_With_Full_Experience_Replay_for_Continual_CVPR_2022_paper.html}{https://openaccess.thecvf.com/content/CVPR2022/html/Yan_Learning_Bayesian_Sparse_Networks_With_Full_Experience_Replay_for_Continual_CVPR_2022_paper.html}
%     \item \href{https://proceedings.neurips.cc/paper/2020/hash/258be18e31c8188555c2ff05b4d542c3-Abstract.html}{https://proceedings.neurips.cc/paper/2020/hash/258be18e31c8188555c2ff05b4d542c3-Abstract.html}
%     \item \href{https://proceedings.mlr.press/v234/yildirim24a.html}{https://proceedings.mlr.press/v234/yildirim24a.html}
%     \item \href{https://www.sciencedirect.com/science/article/pii/S0925231221001545}{https://www.sciencedirect.com/science/article/pii/S0925231221001545}
%     \item \href{https://openaccess.thecvf.com/content/ICCV2023/html/Jin_Growing_a_Brain_with_Sparsity-Inducing_Generation_for_Continual_Learning_ICCV_2023_paper.html}{https://openaccess.thecvf.com/content/ICCV2023/html/Jin_Growing_a_Brain_with_Sparsity-Inducing_Generation_for_Continual_Learning_ICCV_2023_paper.html}
%     \item \href{https://proceedings.neurips.cc/paper/2021/hash/f1e709e6aef16ba2f0cd6c7e4f52b9b6-Abstract.html}{https://proceedings.neurips.cc/paper/2021/hash/f1e709e6aef16ba2f0cd6c7e4f52b9b6-Abstract.html}
%     \item \href{https://arxiv.org/abs/2401.05667}{https://arxiv.org/abs/2401.05667}
%     \item \href{https://openaccess.thecvf.com/content_CVPR_2019/html/Ostapenko_Learning_to_Remember_A_Synaptic_Plasticity_Driven_Framework_for_Continual_CVPR_2019_paper.html}{https://openaccess.thecvf.com/content_CVPR_2019/html/Ostapenko_Learning_to_Remember_A_Synaptic_Plasticity_Driven_Framework_for_Continual_CVPR_2019_paper.html}
%     \item \href{https://arxiv.org/abs/2206.09117}{https://arxiv.org/abs/2206.09117}
%     \item \href{https://arxiv.org/abs/2403.08763}{https://arxiv.org/abs/2403.08763}
%     \item \href{https://arxiv.org/abs/1803.03635}{https://arxiv.org/abs/1803.03635}
%     \item \href{https://arxiv.org/abs/2406.05955}{https://arxiv.org/abs/2406.05955}
%     \item \href{https://arxiv.org/abs/2402.01089}{https://arxiv.org/abs/2402.01089}
%     \item \href{https://arxiv.org/abs/2202.12002}{https://arxiv.org/abs/2202.12002}
%     \item \href{https://arxiv.org/abs/2404.16789}{https://arxiv.org/abs/2404.16789}
%     \item \href{https://arxiv.org/abs/2406.17245}{https://arxiv.org/abs/2406.17245}
%     \item \href{https://arxiv.org/abs/2205.12393}{https://arxiv.org/abs/2205.12393}
%     \item \href{https://openaccess.thecvf.com/content/CVPR2024/html/Roy_Convolutional_Prompting_meets_Language_Models_for_Continual_Learning_CVPR_2024_paper.html}{https://openaccess.thecvf.com/content/CVPR2024/html/Roy_Convolutional_Prompting_meets_Language_Models_for_Continual_Learning_CVPR_2024_paper.html}
%     \item \href{https://arxiv.org/abs/2110.03215}{https://arxiv.org/abs/2110.03215}
%     \item \href{https://arxiv.org/abs/2405.18653}{https://arxiv.org/abs/2405.18653}
%     \item \href{https://arxiv.org/abs/2012.15283}{https://arxiv.org/abs/2012.15283}
%     \item \href{https://arxiv.org/abs/2205.09357}{https://arxiv.org/abs/2205.09357}
%     \item \href{https://arxiv.org/abs/2301.12314}{https://arxiv.org/abs/2301.12314}
%     \item \href{https://aclanthology.org/2020.coling-main.318/}{https://aclanthology.org/2020.coling-main.318/}
%     \item \href{https://arxiv.org/abs/2401.03129}{https://arxiv.org/abs/2401.03129}
%     \item \href{https://arxiv.org/abs/2302.03241}{https://arxiv.org/abs/2302.03241}
%     \item \href{https://arxiv.org/abs/2309.14763}{https://arxiv.org/abs/2309.14763}
%     \item \href{https://arxiv.org/abs/2406.06962}{https://arxiv.org/abs/2406.06962}
%     \item \href{https://arxiv.org/abs/2406.01392}{https://arxiv.org/abs/2406.01392}
%     \item \href{https://arxiv.org/abs/2404.00790}{https://arxiv.org/abs/2404.00790}
%     \item \href{https://arxiv.org/abs/2406.18708}{https://arxiv.org/abs/2406.18708}
%     \item \href{https://arxiv.org/abs/2406.01375}{https://arxiv.org/abs/2406.01375}
%     \item \href{https://www.cell.com/trends/cognitive-sciences/fulltext/S1364-66132030219-9}{https://www.cell.com/trends/cognitive-sciences/fulltext/S1364-66132030219-9}
%     \item \href{https://openreview.net/forum?id=DimPeeCxKO}{https://openreview.net/forum?id=DimPeeCxKO}
% \end{itemize}

% \end{comment}


% \subsection{Impact of learning rate on packing and forgetting knowledge}\label{app:nondiss:packing:lr}
% \zbh{I wrote it but may be too straightforward}
% We investigated the impact of different learning rates on the model's ability to integrate new knowledge and retain old knowledge. Figure~\ref{fig:lr_impact_appendix} shows the accuracy of GPT-2 XL on old and new knowledge for various learning rates, ranging from $4 \times 10^{-5}$ to $1 \times 10^{-2}$, under different neuron update strategies.
% \begin{figure}[htbp]
    \centering
    \begin{subfigure}[b]{0.48\textwidth}
        \centering
        \includegraphics[width=\textwidth]{./figures/3_incremental_update/experiment2_1_xl_lr_study/experiment2_1_xl_lr_study_old_knowledge_2000_neurons.pdf}
        \subcaption{Old Knowledge (t=2000 neurons)}
        \label{fig:old_2000:lr}
    \end{subfigure}
    \hfill
    \begin{subfigure}[b]{0.48\textwidth}
        \centering
        \includegraphics[width=\textwidth]{./figures/3_incremental_update/experiment2_1_xl_lr_study/experiment2_1_xl_lr_study_new_knowledge_2000_neurons.pdf}
        \subcaption{New Knowledge (t=2000 neurons)}
        \label{fig:new_2000:lr}
    \end{subfigure}
    
    \vspace{0.5cm} % Adjust vertical space between rows as needed
    
    \begin{subfigure}[b]{0.48\textwidth}
        \centering
        \includegraphics[width=\textwidth]{./figures/3_incremental_update/experiment2_1_xl_lr_study/experiment2_1_xl_lr_study_old_knowledge_6000_neurons.pdf}
        \subcaption{Old Knowledge (t=6000 neurons)}
        \label{fig:old_6000:lr}
    \end{subfigure}
    \hfill
    \begin{subfigure}[b]{0.48\textwidth}
        \centering
        \includegraphics[width=\textwidth]{./figures/3_incremental_update/experiment2_1_xl_lr_study/experiment2_1_xl_lr_study_new_knowledge_6000_neurons.pdf}
        \subcaption{New Knowledge (t=6000 neurons)}
        \label{fig:new_6000:lr}
    \end{subfigure}
    
    \caption{Impact of learning rate on old and new knowledge for different neuron counts. (a) and (b) show results for $t=2000$ neurons, while (c) and (d) correspond to $t=6000$ neurons.}
    \label{fig:lr_impact_appendix}
\end{figure}



% Our findings indicate that starting from a learning rate of $1.6 \times 10^{-4}$, there is a collapse in performance for both old and new knowledge. The tendencies observed throughout the work are also observed here. Targeting plastic neurons helps preserve old knowledge but struggles to learn new information effectively. Conversely, updating stubborn neurons leads to better acquisition of new knowledge but results in more degradation of old knowledge. These observations highlight the importance of selecting an appropriate learning rate to balance knowledge integration and retention.


% \section{Will likely not Keep}
% \subsection{ROME performance across folds}
% \begin{table}[h]
% \centering
% \caption{ROME Results for Different Sample Sizes and Layers (GPT2-small) \zbh{Not sure to keep, putting here right now}}
% \label{tab:rome_results}
% %\resizebox{\textwidth}{!}{%
% \begin{tabular}{cccccc}
% \toprule
% \textbf{Samples} & \textbf{Run} & \textbf{Old (Locality)} & \textbf{New (Reliability)} & \textbf{Generalization} \\
% \midrule
% \multirow{5}{*}{10} 
%  & 0   & 0.952 & 0.200 & 0.000 \\
%  & 1   & 0.745 & 0.300 & 0.300 \\
%  & 2    & 0.908 & 0.500 & 0.400 \\
%  & 3  & 0.948 & 0.000 & 0.000 \\
%  & 4 & 0.903 & 0.200 & 0.200 \\
% \midrule
% \multirow{5}{*}{100}
%  & 0   & 0.435 & 0.350 & 0.180 \\
%  & 1   & 0.470 & 0.350 & 0.140 \\
%  & 2    & 0.554 & 0.230 & 0.140 \\
%  & 3  & 0.258 & 0.260 & 0.100 \\
%  & 4 & 0.437 & 0.310 & 0.190 \\
% \midrule
% \multirow{5}{*}{1000}
%  & 0   & 0.0261 & 0.0015 & 0.005 \\
%  & 1   & 0.1906 & 0.1930 & 0.083 \\
%  & 2    & 0.1808 & 0.2000 & 0.084 \\
%  & 3  & 0.1756 & 0.1670 & 0.076 \\
%  & 4 & 0.1870 & 0.2400 & 0.085 \\
% \bottomrule
% \end{tabular}
% %}
% \end{table}

% \subsection{MEMIT performance across folds}

% \begin{table}[h]
% \centering
% \caption{MEMIT Results for Different Sample Sizes and Layers (GPT2-small). \zbh{Not sure to keep, putting here right now}}
% \label{tab:memit_results}
% %\resizebox{\textwidth}{!}{%
% \begin{tabular}{cccccc}
% \toprule
% \textbf{Samples} & \textbf{Run} & \textbf{Old (Locality)} & \textbf{New (Reliability)} & \textbf{Generalization} \\
% \midrule
% \multirow{5}{*}{10} 
%  & 0   & 0.998 & 0.000 & 0.000 \\
%  & 1   & 0.996 & 0.000 & 0.000 \\
%  & 2    & 0.997 & 0.000 & 0.000 \\
%  & 3  & 0.997 & 0.000 & 0.000 \\
%  & 4 & 0.820 & 0.000 & 0.000 \\
% \midrule
% \multirow{5}{*}{100}
%  & 0  & 0.974 & 0.010 & 0.010 \\
%  & 1  & 0.982 & 0.000 & 0.010 \\
%  & 2  & 0.977 & 0.000 & 0.020 \\
%  & 3  & 0.965 & 0.010 & 0.000 \\
%  & 4  & 0.984 & 0.000 & 0.010 \\
% \midrule
% \multirow{5}{*}{1000}
%  & 0  & 0.624 & 0.202 & 0.110 \\
%  & 1  & 0.623 & 0.204 & 0.113 \\
%  & 2  & 0.717 & 0.120 & 0.077 \\
%  & 3  & 0.633 & 0.197 & 0.089 \\
%  & 4  & 0.427 & 0.268 & 0.111 \\
% \bottomrule
% \end{tabular}
% %}
% \end{table}

% \subsection{GPT2XL comparisons}
% \begin{figure}[t]
%     \centering
%     \includegraphics[width=\textwidth]{figures/3_incremental_update/mosaic_all/pareto_mosaic_GPT2-small_GPT2-XL 10x LR.pdf}
%     \caption{Scatter plot of old (x) vs new (y) knowledge during incremental updates with new knowledge for different strategies and scopes (N). GPT2-small (top row) and GPT2-XL (bottom row) with 10x learning rate.  \zbh{Not sure to keep, putting here right now}}
%     \label{fig:pareto_mosaic_10x_lr}
% \end{figure}

% \begin{figure}[t]
%     \centering
%     \includegraphics[width=\textwidth]{figures/3_incremental_update/mosaic_all/pareto_mosaic_GPT2-small_GPT2-XL 10x neurons.pdf}
%     \caption{Scatter plot of old (x) vs new (y) knowledge during incremental updates with new knowledge for different strategies and scopes (N). GPT2-small (top row) and GPT2-XL (bottom row) with 10x neurons.\zbh{Not sure to keep, putting here right now}}
%     \label{fig:pareto_mosaic_10x_neurons}
% \end{figure}

% \subsection{Comparison with editing methods using Easyedit}\label{app:editing:numbers}
% As mentioned in the main paper, we leverage the easyEdit package to compare two keys state of the art methods, for which we report the full results below.

% \begin{table}[h]
% \centering
% \caption{ROME Results for Different Fact sample Sizes (mean(std) over 5 folds of facts).}
% \label{tab:rome_results_mean_std}
% %\resizebox{\textwidth}{!}{%
% \begin{tabular}{cccccc}
% \toprule
% \textbf{Samples} & \textbf{Old (Locality)} & \textbf{New (Reliability)} & \textbf{Generalization} & \textbf{Harmonic Mean} \\
% \midrule
% 10   & 0.891 (0.085) & 0.240 (0.182) & 0.180 (0.179) & 0.236 (0.235) \\
% 100  & 0.431 (0.108) & 0.300 (0.054) & 0.150 (0.036) & 0.24 (0.045) \\
% 1000 & 0.152 (0.071) & 0.160 (0.093) & 0.067 (0.035) &  0.106 (0.058) \\
% \bottomrule
% \end{tabular}
% %}
% \end{table}

% \begin{table}[h]
% \centering
% \caption{MEMIT Results for Different Fact Sample Sizes (mean(std) over 5 folds of facts).}
% \label{tab:memit_results_mean_std}
% %\resizebox{\textwidth}{!}{%
% \begin{tabular}{ccccc}
% \toprule
% \textbf{Samples} & \textbf{Old (Locality)} & \textbf{New (Reliability)} & \textbf{Generalization} & \textbf{Harmonic Mean} \\\\
% \midrule
% 10    & 0.962 (0.079) & 0.000 (0.000) & 0.000 (0.000) & 0.000 (0.000)\\
% 100   & 0.976 (0.008) & 0.004 (0.005) & 0.010 (0.007) & 0.003 (0.007)\\
% 1000  & 0.605 (0.107) & 0.198 (0.053) & 0.100 (0.016) & 0.177 (0.028) \\
% \bottomrule
% \end{tabular}\caption{MEMIT}
% %}
% \end{table}


% Extra to add eventually? 
% \ref{fig:bestLR_old:xl} \ref{fig:bestLR_old:xl}
% Using the best LR allows as expected to learn all the new facts with 100\% accuracy while almost preserving the old knowledge. However, all the strategies fail to learn all the new facts up to 20,000 neurons; but still with the same trend as before: the candidate and the specific are by far more efficient in learning new facts compared to the neurons from the free plastic space, confirming again the existence of preferred subnetworks for learning following the lottery ticket hypothesis.
% The figure shows that using the same aggressive LR as GPT2-small (Fig.~\ref{fig:sameLR_old:xl} Fig.~\ref{fig:sameLR_new:xl} does not work as expetd to learn the new knowledge even in the case of finetuning, but results also destructing the old knowledge (50\% accuracy). 

% \ref{fig:10xLR_old:xl} \ref{fig:10xLR_new:xl}
% Next, using 10X learning rates, allows us to see on GPT2XL similar effects to those of gpt2small
% with: the specific strategy providing the best trade-off, candidate and stubborn killing more old knowledge while learning wll. The specific strategy is again the best. The free strategy here is still unable to learn even when given 20,000 neurons (less than 50\% accuracy, compared to the candidate and soecific and busy which reach 100\% at this high number of neurons).
% This suggests again that using preferred networks allows to pack knowledge in few neurons compared to using the entire space, where everything needs to be ``learned from scratch''.

% \ref{fig:10XNeurons_old:xl} \ref{fig:10XNeurons_new:xl}
% The case of 10X more neurons, allowing more space. Here, we observe the cloest results to GPT2small: Adding new knowledge starts to slightly destroy old knowledge, even when using our strategies. And finally, Free neurons have more space to learn. 
% \ref{fig:epochs50s_new} \ref{fig:epochs50s_new}
% Finally, and as expected, when given 10 times more time to train, the free strategy also manages to pack knowlege more efficiently in fewer neurons, but the smae findings as earlier hold for the remaining strategies. However, this regime (which is way slower than fintuning in learning the new knowledge, remember we decided to pick the number of epochs based on the finetuning performance, see Fig.\ref{tab:lr:search} where 5 epochs where enough for GPT2XL) seems to offer a small advantage compared to the best LR case: it is possible, if we wait enough (train more) to pack more knowledge in less neurons, while better preserving old knowledge (100\% accuracy against 99\% only with best LR)). But again, the effects are not enough visible in our GPT2XL toy case, but we expect them to be big in more real settings where models are more packed and knowledge collisions are more likely.   
