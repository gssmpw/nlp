\begin{figure}[htbp]
    \centering
    \begin{subfigure}[b]{0.48\textwidth}
        \centering
        \includegraphics[width=\textwidth]{./figures/3_incremental_update/experiment2_1_xl_lr_study/experiment2_1_xl_lr_study_old_knowledge_2000_neurons.pdf}
        \subcaption{Old Knowledge (t=2000 neurons)}
        \label{fig:old_2000:lr}
    \end{subfigure}
    \hfill
    \begin{subfigure}[b]{0.48\textwidth}
        \centering
        \includegraphics[width=\textwidth]{./figures/3_incremental_update/experiment2_1_xl_lr_study/experiment2_1_xl_lr_study_new_knowledge_2000_neurons.pdf}
        \subcaption{New Knowledge (t=2000 neurons)}
        \label{fig:new_2000:lr}
    \end{subfigure}
    
    \vspace{0.5cm} % Adjust vertical space between rows as needed
    
    \begin{subfigure}[b]{0.48\textwidth}
        \centering
        \includegraphics[width=\textwidth]{./figures/3_incremental_update/experiment2_1_xl_lr_study/experiment2_1_xl_lr_study_old_knowledge_6000_neurons.pdf}
        \subcaption{Old Knowledge (t=6000 neurons)}
        \label{fig:old_6000:lr}
    \end{subfigure}
    \hfill
    \begin{subfigure}[b]{0.48\textwidth}
        \centering
        \includegraphics[width=\textwidth]{./figures/3_incremental_update/experiment2_1_xl_lr_study/experiment2_1_xl_lr_study_new_knowledge_6000_neurons.pdf}
        \subcaption{New Knowledge (t=6000 neurons)}
        \label{fig:new_6000:lr}
    \end{subfigure}
    
    \caption{Impact of learning rate on old and new knowledge for different neuron counts. (a) and (b) show results for $t=2000$ neurons, while (c) and (d) correspond to $t=6000$ neurons.}
    \label{fig:lr_impact_appendix}
\end{figure}

