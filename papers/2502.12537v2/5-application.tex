\section{Application}

\subsection{The Current Landscape of Hedge Funds and the Challenge of GURU ETF}
 Hedge funds have been pivotal in financial markets, known for their sophisticated strategies and adaptability. They employ a variety of tactics like taking long or short positions, relying on a thorough analysis of market trends, sector dynamics, company fundamentals, macroeconomic factors, and investor sentiment, supported by quantitative models and risk management.

Meanwhile, exchange-traded funds (ETFs) like the Global X Guru ETF (GURU) attempt to mirror the strategies of top hedge funds. GURU aims to replicate the stock picks of these funds based on their quarterly filings. Despite the allure of tapping into successful hedge fund strategies, GURU has struggled to match the performance of broader indices such as the S\&P 500, largely due to the delays in reporting and the inability to adjust to market changes in real-time.

\subsection{CNN-DRL to the Rescue}
Our CNN-DRL model presents a compelling alternative. The model excels in processing high-dimensional sequential data and adapts to various temporal windows, notably shorter ones, which is crucial in the fast-paced financial markets. The application of the CNN-DRL model within hedge funds could revolutionize their investment decision-making process, enhancing GURU's cost efficiency and performance.

The below chart illustrates the performance divergence between GURU, the S\&P 500, the CNN-DRL Model's best performer, and the DIA ETF. The CNN-DRL Model's best performer shows heightened growth compared to the steady rises of the S\&P 500 and DIA ETF, indicating a robust return on investment. Notably, it also demonstrates resilience during market downturns, avoiding the deep troughs experienced by GURU.

\begin{figure}[ht]
    \centering
    \includegraphics[width=\linewidth]{comparison_with_dia.pdf}
    \caption{Comparison with traditional ETFs}
    \label{fig:comparison_with_dia}
\end{figure}

The CNN-DRL Model could offer a transformative edge to hedge fund strategies. Integrating this advanced system could allow funds to capture subtle market movements and respond with greater agility, resulting in higher returns and potentially lowering risk profiles. As financial landscapes grow more complex, adopting sophisticated models like the CNN-DRL may become essential for maintaining a competitive advantage.

For the Global X Guru ETF, the CNN-DRL Model could have significantly improved its strategy. Traditional methods, which may be less adaptive, could be enhanced by the CNN-DRL Model's capability to continuously learn and adapt, potentially leading to better investment decisions and growth in investment value.

The chart, depicting a comparison with traditional ETFs, serves as a visual testament to the potential enhancements that the CNN-DRL Model could bring to the investment strategies of ETFs like GURU and even the broader market-representative DIA ETF.

The superior performance of our CNN-DRL model, particularly in shorter temporal windows, has significant implications for high-frequency trading strategies. Fund managers could potentially use this approach to make more nimble, data-driven decisions in rapidly changing market conditions. Moreover, the model's ability to adapt to different feature arrangements suggests it could be applied across various financial instruments and markets, offering a versatile tool for portfolio management.


