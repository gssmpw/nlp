\section{Literature Review}
\label{sec:litriv}

\subsection{Classic ML approachs}
When studying the progressive advancements in this field, the classical Machine Learning (ML) approach in financial analytics primarily revolves around statistical models that have formed the bedrock of quantitative finance. Linear regression, one of the most fundamental techniques, has been extensively utilized for predicting financial trends and stock prices. Its effectiveness in financial forecasting is documented in "Analysis of Financial Time Series" by Tsay \cite{Tsay2010}, offering a comprehensive understanding of linear models in finance. Moreover, decision trees have been widely employed for risk assessment and credit scoring, as demonstrated in the study by Kumar and Ravi \cite{Kumar2007}, showcasing their ability to handle categorical and continuous input variables effectively. However, despite their widespread application, these classical models often struggle with financial data's non-linearity and high dimensionality characteristic. This limitation, as highlighted in the survey by Atsalakis and Valavanis \cite{Atsalakis2009}, clearly indicates the need for more advanced approaches in capturing the complex dynamics of financial markets, especially in volatile or unpredictable scenarios.

\subsection{Neural Networks in DRL}
Integrating Neural Networks and Deep Reinforcement Learning (DRL) into financial market analysis represents a significant leap forward from traditional ML methods. As outlined in the groundbreaking work by Mnih et al. \cite{Mnih2015}, DRL combines the depth and complexity of deep neural networks with the decision-making prowess of reinforcement learning, creating a powerful tool for financial analysis. This approach, which allows for direct learning from vast amounts of unstructured market data, effectively identifying intricate patterns and trends, is a game-changer in the field. Convolutional Neural Networks (CNNs) application within DRL, in particular, has further advanced the field. CNNs, renowned for their ability to process high-dimensional sequential data, are highly effective in capturing temporal and spatial dependencies in financial time series. This is exemplified in the research by Tsantekidis et al. \cite{Tsantekidis2017}., which utilized CNNs to analyze and predict stock prices from limited order book data, demonstrating the model's proficiency in handling complex financial datasets. The success of DRL in financial applications lies in its ability to continually adapt and learn in an ever-changing environment, a crucial feazture given the dynamic nature of financial markets. 

Despite these advancements, there remains a gap in understanding how the temporal scope of input data affects CNN performance in financial DRL models. Our study addresses this gap by systematically exploring various temporal windows and feature arrangements.