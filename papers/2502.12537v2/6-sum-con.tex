\section{Discussion}

\subsection{Interpretation of Results}
The outcomes of this study, utilizing Convolutional Neural Networks (CNNs) within a Deep Reinforcement Learning (DRL) framework for financial analytics, underscore the pivotal role of temporal precision in market predictions. The findings particularly emphasize the efficacy of short-term observation windows, with the two-week window demonstrating superior performance in capturing market dynamics. This observation resonates with the rapidity and volatility characteristic of financial markets, where new information is swiftly reflected in stock prices. Notably, the enhanced model performance achieved through feature rearrangement highlights the significant impact of feature engineering. By reorganizing features related to different stocks and technical indicators, we support the hypothesis that a more robust and generalizable representation of data can be learned, potentially increasing the model's adaptability to diverse market conditions.

\subsection{Theoretical Implications}
The success of the CNN model, employing a methodical window expansion technique, accentuates the importance of temporal dynamics in financial time-series analysis. This approach aligns with the efficient market hypothesis, positing that markets assimilate all available information into stock prices. The model's ability to adjust to the market's 'memory' is crucial for precise financial forecasting. Additionally, the effectiveness of the rearranged features approach suggests that the organization and representation of information are as critical as the information itself for the learning process. This sheds light on the interaction of features within CNN layers, stressing the role of feature engineering in refining financial models.

Our findings challenge the conventional wisdom that longer observation windows invariably lead to better predictions in financial markets. The superior performance of shorter windows, particularly in non-rearranged datasets, suggests that recent market information may carry more predictive power than extended historical data. This aligns with the concept of market efficiency, where new information is rapidly incorporated into prices.

\section{Conclusions and Future Work}

In this study, we explored the impact of different data structures and observation windows on Convolutional Neural Networks (CNNs) performance in financial market analysis. We focused on understanding how the arrangement of data and the selection of timeframes influence the model's ability to capture and predict market dynamics.

The shift in optimal performance with rearranged data suggests that a Convolutional Neural Network (CNN) becomes more proficient at identifying complex patterns between features and their temporal dynamics when the data maintains a cohesive structure for each company's features. The ability of the model to grasp longer-term trends and relationships in rearranged datasets, which are less apparent in shorter timeframes or non-rearranged data, emphasizes the importance of strategic data arrangement. This finding underlines the need for flexible and specific approaches in selecting observation periods and organizing data to enhance the utility and accuracy of CNNs in financial market analysis.

\subsection{Limitations and Challenges}

This study, while offering valuable insights into the use of Convolutional Neural Networks (CNNs) for financial market analysis, encounters several limitations and challenges that warrant attention. A primary constraint is the need for increased computational power. Exploring additional observation windows and testing larger, more complex CNN architectures necessitate substantial computational resources. This requirement becomes particularly critical when considering the intricacy and volume of financial data and the need for extensive testing to validate the robustness and accuracy of the models across various market scenarios.

Moreover, there is a significant need for research funding to support these endeavors. Enhanced funding would facilitate access to more powerful computing infrastructure and enable a broader scope of experimentation. This includes investigating a wider array of temporal windows and deploying more advanced CNN models, which could potentially uncover deeper insights and yield more precise predictive capabilities.

Future studies could explore incorporating diverse data types, like news sentiment or economic indicators, to enhance model robustness. Broadening the scope to different markets and asset types would help verify the applicability of these findings. Investigating the effects of shorter temporal windows or real-time data streams may provide insights into high-frequency trading strategies. Additionally, applying the concept of rearranged features to other forms of financial data, such as order book information or unstructured data, could pave the way for innovative advancements in financial modeling techniques.

In conclusion, our study makes several key contributions to the field of financial DRL. First, we demonstrate the critical importance of temporal window selection in CNN-based models. Second, we show that feature rearrangement can significantly alter the optimal observation period. Finally, we provide a methodological framework for systematically exploring these parameters in future research. These insights open new avenues for enhancing the accuracy and robustness of AI-driven financial analysis tools.