\documentclass[11pt]{article}

%% The amssymb package provides various useful mathematical symbols
\usepackage{amssymb}
\usepackage{lipsum}
\usepackage[table]{xcolor}
\usepackage{xcolor}
\usepackage{booktabs} % For better looking tables
\usepackage{tabularx}
\usepackage{amsmath}
\usepackage{float}
\usepackage{placeins}
\usepackage{url}
\usepackage[pdftex]{graphicx}
\usepackage{caption,subcaption}
\usepackage{hyperref}
\usepackage[numbers]{natbib}
\bibliographystyle{plainnat}
%% You might want to define your own abbreviated commands for common used terms, e.g.:
\newcommand{\kms}{km\,s$^{-1}$}
\newcommand{\msun}{$M_\odot$}

\title{Finding Optimal Trading History in Reinforcement Learning for Stock Market Trading}

\author{Sina Montazeri\thanks{University of North Texas, Denton, Texas 76207, USA}
\and Haseebullah Jumakhan\thanks{Ajman University, Ajman, United Arab Emirates}
\and Amir Mirzaeinia\thanks{University of North Texas, Denton, Texas 76207, USA}}

\date{\today}

\begin{document}
\maketitle

\begin{abstract}
This paper investigates the optimization of temporal windows in Financial Deep Reinforcement Learning (DRL) models using 2D Convolutional Neural Networks (CNNs). We introduce a novel approach to treating the temporal field as a hyperparameter and examine its impact on model performance across various datasets and feature arrangements. We introduce a new hyperparameter for the CNN policy, proposing that this temporal field can and should be treated as a hyperparameter for these models. We examine the significance of this temporal field by iteratively expanding the window of observations presented to the CNN policy during the deep reinforcement learning process. Our iterative process involves progressively increasing the observation period from two weeks to twelve weeks, allowing us to examine the effects of different temporal windows on the model's performance. This window expansion is implemented in two settings. In one setting, we rearrange the features in the dataset to group them by company, allowing the model to have a full view of company data in its observation window and CNN kernel. In the second setting, we do not group the features by company, and features are arranged by category. Our study reveals that shorter temporal windows are most effective when no feature rearrangement to group per company is in effect. However, the model will utilize longer temporal windows and yield better performance once we introduce the feature rearrangement. To examine the consistency of our findings, we repeated our experiment on two datasets containing the same thirty companies from the Dow Jones Index but with different features in each dataset and consistently observed the above-mentioned patterns. The result is a trading model significantly outperforming global financial services firms such as the Global X Guru by the established Mirae Asset.
\end{abstract}

\tableofcontents

%% main text

\section{Introduction}

\subsection{Background and Motivation}
Integrating Deep Reinforcement Learning (DRL) in financial market analysis significantly evolved investment analysis with Deep Learning. DRL combines deep learning and reinforcement learning to offer a sophisticated framework for adapting strategies in the dynamic financial domain. It allows a deep learning model to effectively decipher complex patterns in historical market data often overlooked by traditional quantitative models.
It is no secret that financial markets are inherently complex and influenced by economic trends and geopolitical events. Therefore, traditional financial modeling often struggles to adapt to these ever-changing conditions. However, with its direct learning from data and adaptive strategies, DRL presents a promising solution to these challenges. With its autonomous learning ability and continual adaptation to the financial environment, it leverages historical market data to identify complex relationships and patterns.


\subsection{Overview of Our Previous Work}
In recent years, significant progress has been made in applying deep reinforcement learning (DRL) to stock trading strategies. For instance, Wang et al. proposed a parallel multi-module DRL algorithm that effectively captures both current market conditions and long-term dependencies using fully connected and LSTM layers \cite{parallel_drl_stock_trading}. Zhang et al. introduced an automated stock trading system based on the Proximal Policy Optimization algorithm, modeling trading as a Markov decision process \cite{novel_drl_stock_trading}. Additionally, Huang et al. demonstrated the importance of integrating market sentiment data to enhance the performance of DRL models in trading \cite{market_sentiment_drl_stock_trading}. Liu et al. developed an end-to-end trading strategy using a multi-view environment representation neural network, incorporating a Long Memory mechanism to improve decision-making \cite{drl_end_to_end_stock_trading}. Lastly, Li et al. focused on adaptive trading strategies using Gated Recurrent Units to capture time-series data effectively \cite{adaptive_drl_stock_trading}. These studies collectively highlight the potential of DRL in creating robust and adaptive trading strategies.

Liu et al. significantly advanced Deep Reinforcement Learning in Finance by developing platforms such as FinRL-Meta \cite{Liu2022FinRL}. This platform is a comprehensive tool for training and evaluating data-driven reinforcement learning agents within several simulated financial markets, offering a robust benchmarking system for algorithm comparison and facilitating the simulation of complex market conditions. The FinRL platform enables researchers to refine and test the efficacy of various DRL strategies, and it has been pivotal in democratizing access to sophisticated financial simulation tools and propelling research in financial analysis.

FinRL uses environments that offer broad simulation capabilities. These specialized environments, such as ABIDES-Gym \cite{Vyawahare2020}, provide the necessary infrastructure that allows FinRL to create discrete event simulations explicitly tailored for financial markets. ABIDES-Gym extends the OpenAI Gym interface to accommodate the complex dynamics of financial trading, allowing for a nuanced replication of market mechanisms and agent interactions. This level of detail will enable researchers and practitioners to explore the impact of individual agent behaviors and market responses, thus enhancing the understanding of market microstructure and agent-based modeling. The framework also streamlines the model training process on financial datasets, epitomizing the intersection of DRL and high-performance computing. It Leverages distributed computing resources to reduce training times significantly and optimizes computational workflows to enable the application of complex DRL models to extensive financial tasks. Their efforts have led to the creation of scalable and efficient financial models.

Our previous work \cite{Montazeri2023} demonstrated the efficiency and capability of CNNs when used as policies for deep reinforcement learning. We utilized the FinRL platform to conduct experiments on CNNs as a significantly improved policy to FinRL's original proposition. We also showed \cite{Montazeri2024, Montazeri2024GradientRC} that rearranging the stock market features used in the FinRL platform to group them per company could benefit the mode's performance. This study also utilizes the FinRL platform with its original dataset, containing features generated through traditional Technical Analysis used in Finance. It also uses the new dataset introduced in FinRL Meta, which contains statistically engineered features such as Simple Moving Average (SMA), momentum, and rate of change.

Building upon these foundational studies, our research aims to bridge the gap between CNN architecture optimization and financial market analysis. By introducing a systematic approach to temporal window selection, we seek to enhance the adaptability and performance of DRL models in capturing complex market dynamics.
    
\section{Objectives of the Current Study}
So far, we have presented the literature and the setting in which our study operates. The primary objective of our research is to explore the effects of changing the temporal window of a Convolutional Neural Network (CNN) used as a policy in a FinRL. By progressively expanding the observation period, beginning with a concise two-week window and incrementally enlarging it by two weeks in each iteration and culminating in twelve weeks, we aim to observe and analyze the performance of our model as its temporal window changes in the FinRL platform. This iterative window expansion is designed to examine the impact of different temporal scales on the model's performance. This process enables a comprehensive analysis of how varying lengths of financial data affect the model's predictive capabilities, offering insights and an opportunity to optimize the temporal granularity for financial market analysis. Our study also examines the arrangement of feature vectors within these expanding windows to better understand the model-market dynamics.

Furthermore, we contrast the model's performance across these different temporal windows to discern patterns in market behavior and model performance. In our study, short-term windows, particularly the initial two-week period, are hypothesized to be critical for understanding the model's ability to capture immediate market changes and short-term trends, which are essential for timely and accurate trading predictions. As the window expands, the model is expected to integrate a broader spectrum of market conditions, capturing longer-term trends and patterns. This bi-weekly expansion strategy is designed to balance the benefits of short-term immediacy and long-term historical perspective, ensuring the model remains adaptable and responsive to transient market fluctuations and enduring trends. We hope to contribute to financial analytics by demonstrating the efficacy of CNNs in a DRL setting and by providing new insights into the role of temporal dynamics in financial modeling.

\input{2-Lit-Rev}

Here is your hypothesis section with the references added:

```latex
\section{Hypothesis}

Convolution operations are fundamental to Convolutional Neural Networks (CNNs), which are particularly effective in processing data with a grid-like topology, such as images and sequential data \cite{courbariaux2016binarized} \cite{dai2021coatnet}. The convolution operation can be understood as a mathematical process that combines two sets of information. In the context of CNNs, this involves a convolutional kernel (or filter) moving across an input signal (such as an image or time series data) to produce a feature map.

Mathematically, for continuous signals, the convolution operation is defined as:

\[
(S * K)(t) = \int_{-\infty}^{\infty} S(\tau)K(t - \tau) d\tau
\]

Here, \(S\) represents the input signal, and \(K\) represents the convolutional kernel. This integral computes the area under the product of the two functions as the kernel slides over the input signal. However, in practical applications involving digital data, the signals are discrete, and thus the convolution operation is adapted to:

\[
(S * K)[n] = \sum_{m=-M}^{M} S[m]K[n - m]
\]

In this discrete form, the convolution operation involves summing the element-wise products of the input signal and the kernel as it moves across the input. The result is a new set of values (the feature map) that highlight certain features of the input signal, such as edges in an image or patterns in sequential data \cite{chen2019compressing}.

The size of the convolutional kernel (or filter) is a critical parameter in this operation. The kernel size determines the local region from which features are extracted. A larger kernel can capture more contextual information by encompassing a wider region of the input signal, while a smaller kernel focuses on finer details. The balance between capturing local and global features is essential for the performance of CNNs \cite{dai2016rfcn}.

Additionally, the padding applied to the input signal before convolution affects the output size and the nature of the features extracted. Padding involves adding extra values (typically zeros) around the input signal, which allows the kernel to process edge regions more effectively. The output size of the convolution operation is given by:

\[
O = \frac{N - K + 2P}{S} + 1
\]

where \(N\) is the input size, \(K\) is the kernel size, \(P\) is the padding, \(S\) is the stride (the step size of the kernel), and \(O\) is the output size. Properly setting these parameters ensures that the CNN can effectively learn and extract meaningful features from the input data \cite{chollet2017xception}. Understanding these concepts is crucial for optimizing CNN architectures, especially in settings where the observation window size can significantly impact the model's performance.

The performance of Convolutional Neural Networks (CNNs) in processing sequential data is significantly influenced by the size of the observation window used in the convolutional layers. The kernel size in a convolution layer determines the local region from which features are extracted. Larger kernels can incorporate more contextual information, but excessively large kernels may dilute distinct features. The optimization of window size can be expressed through the effective window size equation:

\[
W_{\text{eff}} = W_{\text{kernel}} + (W_{\text{kernel}} - 1) \times (D - 1)
\]

where \(W_{\text{eff}}\) is the effective window size, \(W_{\text{kernel}}\) is the kernel size, and \(D\) is the dilation factor.

Furthermore, the role of padding in convolution processes influences the spatial dimensions of the output feature map, described by:

\[
O = \frac{N - K + 2P}{S} + 1
\]

where \(N\) is the input size, \(K\) is the kernel size, \(P\) is the padding, \(S\) is the stride, and \(O\) is the output size. Excessive padding can lead to overemphasis on peripheral data and potential overfitting, similar to how an over-expanded window size may cause information overload, making distinct features less discernible:

\[
\text{Information Overload} \propto \frac{W_{\text{eff}}}{\text{Distinct Features}}
\]

Therefore, a crucial balance is needed between capturing local and global features. We hypothesize that the optimal selection of a temporal window size in a CNN balances local feature detection and global contextual understanding. An optimally sized window allows the model to effectively capture relevant features without succumbing to information overload or excessive generalization, thereby enhancing accuracy and performance in sequential data processing tasks \cite{chen2019compressing}.

Given that our CNN acts as a policy for a Deep Reinforcement Learning (DRL) algorithm, the window size as a hyperparameter will be optimized through reinforcement learning. This optimal window size is found at the point where local and global feature detection are balanced:

\[
\text{Optimal Window Size} \leftrightarrow \min \left( \Delta_{\text{Local-Global}} \right)
\]

where \( \Delta_{\text{Local-Global}} \) measures the differential in information capture between local and global features. This hypothesis suggests that through careful tuning and reinforcement learning, the CNN can achieve an optimal window size that maximizes performance in sequential data tasks.


\begin{figure*}[t!]
    \centering
    \includegraphics[width=\linewidth]{Figure/TokenSwift.pdf}
    % \vskip -0.1 in
    \caption{\textbf{Illustration of \ours Framework.} First, target model (LLM) with partial KV cache and three linear layers outputs 4 logits in a single forward pass. Tree-based attention is then applied to construct candidate tokens. Secondly, top-$k$ candidate $4$-grams are retrieved accordingly. These candidates compose draft tokens, which are fed into the LLM with full KV cache to generate target tokens. The verification is performed by checking if draft tokens match exactly with target tokens (\cref{alg:algorithm}). Finally, we randomly select one of the longest valid draft tokens, and update $n$-gram table and KV cache accordingly.}
    \label{fig:frame}
    % \vskip -0.15 in
\end{figure*}

\section{\ours}
\label{sec:method}
To achieve \textbf{lossless acceleration in generating ultra-long sequences}, we propose tailored solutions for each challenge inherent to this process. These solutions are seamlessly integrated into a unified framework, \ie \ours.

\subsection{Overview}
\label{sec:overall}
The overall framework is depicted in \cref{fig:frame}. \ours 
% is highly lightweight and conceptually similar to \ac{sd}. It 
generate a sequence of draft tokens with self-drafting, which are then passed to the target (full) model for validation using a tree-based attention mechanism (See \cref{app:tree_attn} for more tree-based attention details). This process ensures that the final generated output aligns with the target model’s predictions, effectively achieving lossless acceleration.

\ours is lightweight because the draft model is the target model itself with a partial KV cache. This eliminates the need to train a separate draft \ac{llm}; instead, only $\gamma$ linear layers need to be trained, where $\gamma + 1$\footnote{The target model itself can also predict one logit, making the total number of logits $\gamma+1$. We take $\gamma=3$.} represents the number of logits predicted in a single forward pass. In addition, during the verification process, once we obtain the target tokens from the target model with full KV cache, we directly compare draft tokens with target tokens sequentially to ensure that the process is lossless~\citep{rest}.

\subsection{Multi-token Generation and Token Reutilization}
\label{sec:multi_token}
\paragraph{Multi-token Self-Drafting} 
% Inspired by Medusa~\citep{medusa}, we propose a modification where the final output of \ac{llm} is used as input to train $3$ linear layers, enabling the model to generate multiple draft tokens in a single forward pass. However, we argue that the generated draft tokens should not be independent of each other. Unlike Medusa, where the linear layers operate entirely independently, we introduce a simple adjustment to this structure. 
Inspired by Medusa~\citep{medusa}, we enable the \ac{llm} to generate multiple draft tokens in a single forward pass by incorporating $\gamma$ additional linear layers. However, we empirically note that \textbf{the additional linear layers should not be independent of each other}. Specifically, we propose the following structure:
\begin{equation}
\label{equ:ours}
% \small
% \resizebox{.9\hsize}{!}{
% $
    \begin{aligned}
    h_1=f_1(h_0) + h_0,\quad{}h_2=&f_2(h_1) + h_1,\quad{}h_3=f_3(h_2) + h_2,\\
l_0,~l_{1},~l_{2},~l_{3}=&~g(h_0),~g(h_1),~g(h_2),~g(h_3),
    \end{aligned}
% $
% }
\end{equation}
where $h_0$ denotes the last hidden state of \ac{llm}, $f_i(\cdot)$ represents the $i$-th linear layer, $h_i$ refers to the $i$-th hidden representation, $g(\cdot)$ represents the LM Head of target model, and $l_i$ denotes output logits.
% By comparing \cref{equ:medsua} (Medusa) and \cref{equ:ours} (\ours), it is evident that in \ours, the generation of each token depends on the previously generated token, which aligns more closely with the \ac{ar} nature of the model. Moreover, this adjustment incurs no additional computational cost.
This structure aligns more closely with the \ac{ar} nature of the model. Moreover, this adjustment incurs no additional computational cost.
\vspace{-0.05 in}
\paragraph{Token Reutilization} 
% Given the relatively low acceptance rate of using linear to approximate the entire \ac{llm} for generating draft tokens, we propose a method named \textbf{token reutilization}  to further reduce the frequency of model reloads. 
Given the relatively low acceptance rate of using linear layers to generate draft tokens, we propose a method named \textbf{token reutilization} to further reduce the frequency of model reloads. The idea behind token reutilization is that some phrases could appear frequently, and they are likely to reappear in subsequent generations.

% Specifically, we define $(\mathcal{G}, \mathcal{F})$, where $\mathcal{G}=\{x_{i+1}, ..., x_{i+n}\}$ represents an $n$-gram and $\mathcal{F}$ denotes its corresponding frequency $\mathcal{F}$ within the generated token sequence $S=\{x_0, x_1, ..., x_{t-1}\}$ by time step $t$ ($t \geq n$). At subsequent time steps, we use the token generated by target model as the first token to select top-$k$ most frequent $n$-grams $\{\mathcal{G}_1, \mathcal{G}_2,...,\mathcal{G}_k\}$ and incorporate them as additional draft tokens. These selected draft tokens, along with the newly generated ones, are then fed to the \ac{llm} for parallel validation. 
Specifically, we maintain a set of tuples $\{(\mathcal{G}, \mathcal{F})\}$, where $\mathcal{G}=\{x_{i+1}, ..., x_{i+n}\}$ represents an $n$-gram and $\mathcal{F}$ denotes its corresponding frequency $\mathcal{F}$ within the generated token sequence $S=\{x_0, x_1, ..., x_{t-1}\}$ by time step $t$ ($t \geq n$). After obtaining $\{p_0,\ldots, p_3\}$ as described in \S \ref{sec:penalty}, we retrieve the top-$k$ most frequent $n$-grams beginning with token $\arg\max p_0$ to serve as additional draft tokens.

Although this method can be applied to tasks with long prefixes, its efficacy is constrained by the limited decoding steps, which reduces the opportunities for accepting $n$-gram candidates. Additionally, since the long prefix text is not generated by the \ac{llm} itself, a distributional discrepancy exists between the generated text and the authentic text~\citep{detectgpt}. As a result, this method is particularly suitable for generating ultra-long sequences.
 
% \subsection{Dynamic and Memory-Saving KV Pruning}
\subsection{Dynamic KV Cache Management}
\label{sec:kv_update}
\paragraph{Dynamic KV Cache Updates}
Building upon the findings of~\citet{stram_llm}, we preserve the initial $|S|$ KV pairs within the cache during the drafting process, while progressively evicting less important KV pairs. Specifically, we enforce a fixed budget size $|B|$, ensuring that the KV cache at any given time can be represented as:
\begin{equation}
    \nonumber
    % \resizebox{\hsize}{!}{$
    \mathbf{KV}=\{(\mathbf{K}_0,\mathbf{V}_0), ..., (\mathbf{K}_{|S|},\mathbf{V}_{|S|}), (\mathbf{K}_{|S|+1},\mathbf{V}_{|S|+1}),..., (\mathbf{K}_{|B|-1},\mathbf{V}_{|B|-1})\},
   % $},
\end{equation}
where the first $|S|$ pairs remain fixed, and the pairs from position $|S|$ to $|B|-1$ are ordered by decreasing importance. 
As new tokens are generated, less important KV pairs are gradually replaced, starting from the least important ones at position $|B|-1$ and moving towards position $|S|$. Once replacements extend beyond the $|S|$ position, we recalculate the \textit{importance scores} of all preceding tokens and select the most relevant $|B|-|S|$ pairs to reconstruct the cache. 
This process consistently preserves the critical information required for ultra-long sequence generation. 
\vspace{-0.05 in}
% \paragraph{Memory-Saving Top-K Pruning} 
% To implement dynamic updates efficiently, we employ a simple yet effective Top-K pruning strategy. Specifically, we rank the KV pairs based on the importance scores derived from the dot product between the query ($\mathbf{Q}$) and key ($\mathbf{K}$), \ie $\mathbf{Q}\mathbf{K}^T$. 
\paragraph{Importance Score of KV pairs} 
We rank the KV pairs based on the \textit{importance scores} derived from the dot product between the query ($\mathbf{Q}$) and key ($\mathbf{K}$), \ie $\mathbf{Q}\mathbf{K}^T$. 

In the case of Group Query Attention (GQA), since each $\mathbf{K}$ corresponds to a group of $\mathcal{Q}=\{\mathbf{Q}_0, ..., \mathbf{Q}_{g-1}\}$, direct dot-product computation is not feasible. Unlike methods such as SnapKV~\citep{snapkv}, we do not replicate the $\mathbf{K}$. Instead, we partition the $\mathcal{Q}$, as shown in \cref{equ:gqa}:
\begin{equation}
    \label{equ:gqa}
    \vspace{-2mm}
    \text{importance score}_i = \sum_{j=i\cdot g}^{((i+1)\cdot g)-1}\mathbf{Q}_j \cdot \mathbf{K}_i,
        % \vspace{-2mm}
\end{equation}
where for position $i$, $\mathbf{Q}_j$ in the group $\mathcal{Q}_i$ are dot-product with the same $\mathbf{K}_i$, and their results are aggregated to obtain the final \textit{importance score}. This approach enhances memory saving while preserving the quality of the attention mechanism, ensuring that each query is effectively utilized without introducing unnecessary redundancy.

\subsection{Contextual Penalty and Random N-gram Selection}
\label{sec:penalty}
% \paragraph{Contextual Length Penalty} 
\paragraph{Contextual Penalty} 
% To mitigate repetition in generated text, we have explored various sampling strategies. However, with the significantly larger sequence length, the likelihood of repetition increases compared to generating shorter texts (\cref{sec:repeat}). As a result, we decided to apply an additional penalty to the generated tokens to further mitigate repetition.
To mitigate repetition in generated text, we have explored various sampling strategies. However, with the significantly larger sequence length, the likelihood of repetition increases significantly (\S \ref{sec:repeat}). As a result, we decided to apply an additional penalty to the generated tokens to further mitigate repetition.

The penalized sampling approach proposed in \citep{penalty} suggests applying a penalty to all generated tokens. However, when generating ultra-long sequences, the set of generated tokens may cover nearly all common words, which limits the ability to sample appropriate tokens. Therefore, we propose an improvement to this method. 

Specifically, we introduce a fixed \emph{penalty window} $W$ and apply \emph{penalty value} $\theta$ to the most recent $W$ tokens, denoted as $\mathbb{W}$, generated up to the current position, as illustrated in \cref{equ:repeat}: 
\begin{equation}
% \small
    \label{equ:repeat}
% \vspace{-3mm}
    \begin{aligned}
        p_i &= \frac{\exp \big(l_i/(t\cdot I(l_i))\big)}{\sum_j \exp \big(l_j/(t\cdot I(l_j))\big)},\\
    I(l)=\theta\,\,&\text{if}\,\,l \in \mathbb{W}\,\text{else}\,\,1.0,\quad \theta \in (1, \infty),
    \end{aligned}
    % \vspace{-1mm}
\end{equation}
where $t$ denotes temperature, $l_i$ and $p_i$ represent the logit and probability of $i$-th token. This adjustment aims to maintain diversity while still mitigating repetitive generation.

\section{\thename}
\subsection{End-to-End Driving Policy}
The overall framework of \thename{} is depicted in Fig.~\ref{fig:framework}. 
\thename{} takes multi-view image sequences as input, transforms the sensor data into scene token embeddings, outputs the probabilistic distribution of actions, and samples an action to control the vehicle. 

\boldparagraph{BEV Encoder.} 
We first employ a BEV encoder~\cite{li2022bevformer} to transform multi-view image features from the perspective view to the Bird's Eye View (BEV), obtaining a feature map in the BEV space. This feature map is then used to learn instance-level map features and agent features.

\boldparagraph{Map Head.} 
Then we utilize a group of map tokens~\cite{maptrv2, liao2022maptr, lanegap} to learn the vectorized map elements of the driving scene from the BEV feature map, including lane centerlines, lane dividers, road boundaries, arrows, traffic signals, \etc.

\boldparagraph{Agent Head.} 
Besides, a group of agent tokens~\cite{jiang2022pip} is adopted to predict the motion information of other traffic participants, including location, orientation, size, speed, and multi-mode future trajectories.

\boldparagraph{Image Encoder.} 
Apart from the above instance-level map and agent tokens, we also use an individual image encoder~\cite{vit,he2016resnet} to transform the original images into image tokens. These image tokens provide dense and rich scene information for planning, complementary to the instance-level tokens.

\begin{figure}[t]
\centering
\includegraphics[width=0.98\linewidth]{fig/post-training-2.pdf} 
\caption{\textbf{Post-training.}  $N$  workers parallelly run. The generated rollout data $(s_t,a_t, r_{t+1},s_{t+1},...)$ are recorded in a rollout buffer. Rollout data and human driving demonstrations are used in RL- and IL-training steps to fine-tune the AD policy synergistically.
}
\label{fig:post-training}
\end{figure}

\boldparagraph{Action Space.} 
To accelerate the convergence of RL training, we design a decoupled discrete action representation. 
We divide the action into two independent components: lateral action and longitudinal action. 
The action space is constructed over a short $0.5$-second time horizon, during which the vehicle's motion is approximated by assuming constant linear and angular velocities. 
Under this assumption, the lateral action $a^x$ and longitudinal action $a^y$ can be directly computed based on the current linear and angular velocities.
By combining decoupling with a limited temporal scope and simplified motion model, our approach effectively reduces the dimensionality of the action space, accelerating training convergence.


\boldparagraph{Planning Head.} 
We use $E_\text{scene}$ to denote the scene representation, which consists of map tokens, agent tokens, and image tokens. We initialize a planning embedding denoted as $E_\text{plan}$. A cascaded Transformer decoder $\phi$ takes the planning embedding $E_\text{plan}$ as the query and the scene representation $E_\text{scene}$ as both key and value.

The output of the decoder $\phi$ is then combined with navigation information $E_\text{navi}$ and ego state $E_\text{state}$ to output the probabilistic distributions of the lateral action $a^x$ and the longitudinal action $a^y$:
\begin{equation}
\begin{aligned}
     \pi(a^x\mid s) = & \text{softmax}(\text{MLP}(\phi(E_\text{plan}, E_\text{scene}) \\
    & + E_\text{navi} + E_\text{state})), \\
     \pi(a^y\mid s) = & \text{softmax}(\text{MLP}(\phi(E_\text{plan}, E_\text{scene}) \\
     & + E_\text{navi} + E_\text{state})),
\label{eq:action distribution}
\end{aligned}
\end{equation}
where $E_\text{plan}$, $E_\text{navi}$, $E_\text{state}$, and the output of $\text{MLP}$ are all of the same dimension ($1 \times D$).

The planning head also outputs the value functions $V_x(s)$ and $V_y(s)$, which estimate the expected cumulative rewards for the lateral and longitudinal actions, respectively: 
\begin{equation}
\begin{aligned}
    & V_x(s) = \text{MLP}(\phi(E_\text{plan}, E_\text{scene}) + E_\text{navi} + E_\text{state}), \\
    & V_y(s) = \text{MLP}(\phi(E_\text{plan}, E_\text{scene}) + E_\text{navi} + E_\text{state}).
\end{aligned}
\end{equation}
The value functions are used in RL training (Sec.~\ref{sec:optimization}).

\subsection{Training Paradigm}
We adopt a three-stage training paradigm: perception pre-training, planning pre-training, and reinforced post-training, as shown in Fig.~\ref{fig:framework}.

\boldparagraph{Perception Pre-Training.} 
Information in the image is sparse and low-level. In the first stage,  
the map head and the agent head explicitly output map elements and agent motion information, which are supervised with ground-truth labels. Consequently,  
map tokens and agent tokens implicitly encode the corresponding high-level information.  
In this stage, we only update the parameters of the BEV encoder, the map head, and the agent head.



\boldparagraph{Planning Pre-Training.} 
In the second stage, to prevent the unstable cold start of RL training, IL is first performed to initialize the probabilistic distribution of actions based on large-scale real-world driving demonstrations from expert drivers. In this stage, we only update the parameters of the image encoder and the planning head, while the parameters of the BEV encoder, map head, and agent head are frozen. The optimization objectives of perception tasks and planning tasks may conflict with each other. However, with the training stage and parameters decoupled, such conflicts are mostly avoided.

\boldparagraph{Reinforced Post-Training.} 
In the reinforced post-training, RL and IL synergistically fine-tune the distribution. RL aims to guide the policy to be sensitive to critical risky events and adaptive to out-of-distribution situations. IL serves as the regularization term to keep the policy's behavior similar to that of humans.

We select a large amount of risky dense-traffic clips from collected driving demonstrations. For each clip, we train an independent 3DGS model that reconstructs the clip and serves as a digital driving environment.  
As shown in Fig.~\ref{fig:post-training}, we set $N$ parallel workers.  
Each worker randomly samples a 3DGS environment and begins rollout, i.e., the AD policy controls the ego vehicle to move and iteratively interacts with the 3DGS environment. After the rollout process of this 3DGS environment ends, the generated rollout data $(s_t,a_t, r_{t+1},s_{t+1},...)$ are recorded in a rollout buffer, and the worker will sample a new 3DGS environment for another round of rollout.

As for policy optimization, we iteratively perform RL-training steps and IL-training steps. For RL-training steps, we sample data from the rollout buffer and follow the Proximal Policy Optimization (PPO) framework~\cite{PPO} to update the AD policy. For IL-training steps, we use real-world driving demonstrations to update the policy. After a fixed number of training steps, the updated AD policy is sent to every worker to replace the old one, to avoid a distribution shift between data collection and optimization.
We only update the parameters of the image encoder and the planning head. The parameters of the BEV encoder, the map head, and the agent head are frozen.  
The detailed RL design is presented below.

\subsection{Interaction Mechanism between AD Policy and 3DGS Environment}
In the 3DGS environment, the ego vehicle acts according to the AD policy. Other traffic participants act according to real-world data in a log-replay manner.  
A simplified kinematic bicycle model is employed to iteratively update the ego vehicle's pose at every $\Delta t$ seconds as follows:  
\begin{equation}
\begin{aligned}
x_{t+1}^{w} & = x_{t}^w + v_t \cos \left(\psi_{t}^w\right) \Delta t, \\
y_{t+1}^{w} & = y_{t}^w + v_t \sin \left(\psi_{t}^w\right) \Delta t, \\
\psi_{t+1}^{w} & = \psi_{t}^w + \frac{v_t}{L} \tan \left(\delta_t\right) \Delta t,
\label{equation:kinematic_model}
\end{aligned}
\end{equation}  
where $x_t^{w}$ and $y_t^{w}$ denote the position of the ego vehicle relative to the world coordinate; $\psi_t^w$ is the heading angle that defines the vehicle's orientation with respect to the world $x$-coordinate; $v_t$ is the linear velocity of the ego vehicle; $\delta_t$ is the steering angle of the front wheels; and $L$ is the wheelbase, i.e., the distance between the front and rear axles.

During the rollout process, the AD policy outputs actions $(a_t^x, a_t^y)$ for a $0.5$-second time horizon at time step $t$. We derive the linear velocity $v_t$ and steering angle $\delta_t$ based on $(a_t^x, a_t^y)$.  
Based on the kinematic model in Eq.~\ref{equation:kinematic_model},  
the pose of the ego vehicle in the world coordinate system is updated from ${p}_t = (x_{t}^w, y_{t}^w, \psi_{t}^w)$ to ${p}_{t+1} = (x_{t+1}^{w}, y_{t+1}^{w}, \psi_{t+1}^{w})$.  

Based on the updated ${p}_{t+1}$, the 3DGS environment computes the new ego vehicle's state $s_{t+1}$. The updated pose ${p}_{t+1}$ and state $s_{t+1}$ serve as the input for the next iteration of the inference process.

The 3DGS environment also generates rewards $\mathcal{R}$ (Sec.~\ref{sec:reward}) according to multi-source information (including trajectories of other agents, map information, the expert trajectory of the ego vehicle, and the parameters of Gaussians), which are used to optimize the AD policy (Sec.~\ref{sec:optimization}).

\begin{figure}[t]
\centering
\includegraphics[width=1.0\linewidth]{fig/reward.pdf} 
\caption{\textbf{Example diagram of four types of reward sources.}  (1): Collision with a dynamic obstacle ahead triggers a reward $r_{\text{dc}}$. (2): Hitting a static roadside obstacle incurs a reward $r_{\text{sc}}$. (3): Moving onto the curb exceeds the positional deviation threshold $d_{\text{max}}$, triggering a reward $r_{\text{pd}}$. (4): Drifting toward the adjacent lane exceeds the heading deviation threshold $\psi_{\text{max}}$, triggering a reward $r_{\text{hd}}$.
}
\label{fig: reward source}
\end{figure}
\subsection{Reward Modeling}
\label{sec:reward}
The reward is the source of the training signal, which determines the optimization direction of RL. The reward function is designed to guide the ego vehicle's behavior by penalizing unsafe actions and encouraging alignment with the expert trajectory. It is composed of four reward components: (1) collision with dynamic obstacles, (2) collision with static obstacles, (3) positional deviation from the expert trajectory, and (4) heading deviation from the expert trajectory:
\begin{equation}
\begin{aligned}
\mathcal{R} = \{r_{\text{dc}}, r_{\text{sc}}, r_{\text{pd}}, r_{\text{hd}}  \}. 
\end{aligned}
\end{equation}

As illustrated in Fig.~\ref{fig: reward source}, these reward components are triggered under specific conditions.  
In the 3DGS environment, dynamic collision is detected if the ego vehicle's bounding box overlaps with the annotated bounding boxes of dynamic obstacles, triggering a negative reward $r_{\text{dc}}$. Similarly, static collision is identified when the ego vehicle's bounding box overlaps with the Gaussians of static obstacles, resulting in a negative reward $r_{\text{sc}}$.  
Positional deviation is measured as the Euclidean distance between the ego vehicle's current position and the closest point on the expert trajectory. A deviation beyond a predefined threshold $d_{\text{max}}$ incurs a negative reward $r_{\text{pd}}$.  
Heading deviation is calculated as the angular difference between the ego vehicle's current heading angle $ \psi_t $ and the expert trajectory's matched heading angle $\psi_{\text{expert}}$. A deviation beyond a threshold $ \psi_{\text{max}}$ results in a negative reward $r_{\text{hd}}$.

Any of these events, including dynamic collision, static collision, excessive positional deviation, or excessive heading deviation, triggers immediate episode termination. Because after such events occur, the 3DGS environment typically generates noisy sensor data, which is detrimental to RL training.

\subsection{Policy Optimization}
\label{sec:optimization}
In the closed-loop environment, the error in each single step accumulates over time. The aforementioned rewards are not only caused by the current action but also by the actions of the preceding steps.  
The rewards are propagated forward with Generalized Advantage Estimation (GAE)~\cite{gae} to optimize the action distribution of the preceding steps.

Specifically, for each time step $t$, we store the current state $s_t$, action $a_t$, reward $r_t$, and the estimate of the value $V(s_t)$.  
Based on the decoupled action space, and considering that different rewards have different correlations to lateral and longitudinal actions, the reward $r_t$ is divided into lateral reward $r_t^x$ and longitudinal reward $r_t^y$:
\begin{equation}
\begin{aligned}
r_t^x &= r_t^{\text{sc}} + r_t^{\text{pd}} + r_t^{\text{hd}}, \\
r_t^y &= r_t^{\text{dc}}.
\label{eq:reward-decouple}
\end{aligned}
\end{equation}
Similarly, the value function $V(s_t)$ is decoupled into two components: $V_x(s_t)$ for the lateral dimension and $V_y(s_t)$ for the longitudinal dimension. These value functions estimate the expected cumulative rewards for the lateral and longitudinal actions, respectively. The advantage estimates $\hat{A}_t^x$ and $\hat{A}_t^y$ are then computed as follows:
\begin{equation}
\begin{aligned}
\delta_t^x &= r_t^x + \gamma V_x(s_{t+1}) - V_x(s_t), \\
\delta_t^y &= r_t^y + \gamma V_y(s_{t+1}) - V_y(s_t), \\
\hat{A}_t^x &= \sum_{l=0}^{\infty}(\gamma \lambda)^l \delta_{t+l}^x, \\
\hat{A}_t^y &= \sum_{l=0}^{\infty}(\gamma \lambda)^l \delta_{t+l}^y,
\label{eq:advantage}
\end{aligned}
\end{equation}
where $\delta_t^x$ and $\delta_t^y$ are the temporal difference errors for the lateral and longitudinal dimensions, $\gamma$ is the discount factor, and $\lambda$ is the GAE parameter that controls the trade-off between bias and variance.

To further clarify the relationship between the advantage estimates and the reward components, we decompose $\hat{A}_t^x$ and $\hat{A}_t^y$ based on the reward decomposition in Eq.~\ref{eq:reward-decouple} and the advantage estimation in Eq.~\ref{eq:advantage}. Specifically, we derive the following decomposition:
\begin{equation}
\begin{aligned}
\hat{A}_t^x &= \hat{A}_t^{\text{sc}} + \hat{A}_t^{\text{pd}} + \hat{A}_t^{\text{hd}}, \\
\hat{A}_t^y &= \hat{A}_t^{\text{dc}},
\end{aligned}
\end{equation}
where $\hat{A}_t^{\text{sc}}$ is the advantage estimate for avoiding static collisions, $\hat{A}_t^{\text{pd}}$ is the advantage estimate for minimizing positional deviations, $\hat{A}_t^{\text{hd}}$ is the advantage estimate for minimizing heading deviations, and $\hat{A}_t^{\text{dc}}$ is the advantage estimate for avoiding dynamic collisions.

These advantage estimates are used to guide the update of the AD policy $\pi_{\theta}$, following the PPO framework~\cite{PPO}. By leveraging the decomposed advantage estimates $\hat{A}_t^x$ and $\hat{A}_t^y$, we can independently optimize the lateral and longitudinal dimensions of the policy. This is achieved by defining separate objective functions $\mathcal{L}_x^{\text{CLIP}}(\theta)$ and $\mathcal{L}_y^{\text{CLIP}}(\theta)$ for each dimension,  as follows:
\begin{equation}
\begin{aligned}
\mathcal{L}_x^{\text{PPO}}(\theta) &= \mathbb{E}_t \left[ \min \left( \rho_t^x \hat{A}_t^x, \ \text{clip}(\rho_t^x, 1-\epsilon_x, 1+\epsilon_x) \hat{A}_t^x \right) \right], \\
\mathcal{L}_y^{\text{PPO}}(\theta) &= \mathbb{E}_t \left[ \min \left( \rho_t^y \hat{A}_t^y, \ \text{clip}(\rho_t^y, 1-\epsilon_y, 1+\epsilon_y) \hat{A}_t^y \right) \right], \\
\mathcal{L}^{\text{PPO}}(\theta) &= \mathcal{L}_x^{\text{PPO}}(\theta) + \mathcal{L}_y^{\text{PPO}}(\theta),
\end{aligned}
\end{equation}
where $\rho_t^x = \frac{\pi_{\theta}(a_t^x \mid s_t)}{\pi_{\theta_{\text{old}}}(a_t^x \mid s_t)}$ is the importance sampling ratio for the lateral dimension, $\rho_t^y = \frac{\pi_{\theta}(a_t^y \mid s_t)}{\pi_{\theta_{\text{old}}}(a_t^y \mid s_t)}$ is the importance sampling ratio for the longitudinal dimension, $\epsilon_x$ and $\epsilon_y$ are small constants that control the clipping range for the lateral and longitudinal dimensions, ensuring stable policy updates.

The clipped objective function $\mathcal{L}^{\text{PPO}}(\theta)$ prevents excessively large updates to the policy parameters $\theta$, thereby maintaining training stability.

\begin{table*}[ht]
    \centering
{
\begin{tabular}{lccccccccc}
    \toprule
    RL:IL & CR$\downarrow$ & DCR$\downarrow$ & SCR$\downarrow$ & DR$\downarrow$ & PDR$\downarrow$ & HDR$\downarrow$ &ADD$\downarrow$ & Long. Jerk$\downarrow$ & Lat. Jerk$\downarrow$ \\
    \midrule
     0:1  & 0.229 & 0.211 & 0.018 & 0.066 & 0.039 & 0.027  & 0.238 & 3.928 & 0.103\\
     1:0  & 0.143 & 0.128 & 0.015 &0.080 &0.065 &0.015 &0.345 &4.204 &0.085\\
     2:1 & 0.137 & 0.125 & 0.012 & 0.059 & 0.050 & 0.009  & 0.274 & 4.538 & 0.092\\
     4:1 & 0.089 & 0.080 & 0.009 & 0.063 & 0.042 & 0.021  & 0.257 & 4.495 & 0.082 \\
     8:1 & 0.125 & 0.116 & 0.009 & 0.084 & 0.045 & 0.039  & 0.323 & 5.285 & 0.115\\
    \bottomrule
\end{tabular}
}
    \caption{\textbf{Ablation on RL-to-IL step mixing ratios in the reinforced post-training stage.}}
    \label{tab:ratio}
\end{table*}

\subsection{Auxiliary Objective}
RL usually faces the challenge of sparse rewards, which makes the convergence process unstable and slow. To speed up convergence, we introduce auxiliary objectives that provide dense guidance to the entire action distribution.

The auxiliary objectives are designed to penalize undesirable behaviors by incorporating specific reward sources, including dynamic collisions, static collisions, positional deviations, and heading deviations. These objectives are computed based on the actions \( a_t^{x, \text{old}} \) and \( a_t^{y, \text{old}} \) selected by the old AD policy \( \pi_{\theta_{\text{old}}} \) at time step \( t \). To facilitate the evaluation of these actions, we separate the probability distribution of the action into four parts:
\begin{equation}
\begin{aligned}
\Delta \pi_y^{\text{dec}} &= \sum_{a_t^y < a_t^{y, \text{old}}} \pi_\theta(a_t^y \mid s_t), \\
\Delta \pi_y^{\text{acc}} &= \sum_{a_t^y > a_t^{y, \text{old}}} \pi_\theta(a_t^y \mid s_t), \\
\Delta \pi_x^{\text{left}} &= \sum_{a_t^x < a_t^{x, \text{old}}} \pi_\theta(a_t^x \mid s_t), \\
\Delta \pi_x^{\text{right}} &= \sum_{a_t^x > a_t^{x, \text{old}}} \pi_\theta(a_t^x \mid s_t).
\end{aligned}
\end{equation}
Here, \( \Delta \pi_y^{\text{dec}} \) represents the total probability of deceleration actions, \( \Delta \pi_y^{\text{acc}} \) represents the total probability of acceleration actions, \( \Delta \pi_x^{\text{left}} \) represents the total probability of leftward steering actions, and \( \Delta \pi_x^{\text{right}} \) represents the total probability of rightward steering actions.

\boldparagraph{Dynamic Collision Auxiliary Objective.}  
The dynamic collision auxiliary objective adjusts the longitudinal control action \(a_t^y\) based on the location of potential collisions relative to the ego vehicle. If a collision is detected ahead, the policy prioritizes deceleration actions (\(a_t^y < a_t^{y, \text{old}}\)); if a collision is detected behind, it encourages acceleration actions (\(a_t^y > a_t^{y, \text{old}}\)). To formalize this behavior, we define a directional factor \(f_\text{dc}\):
\begin{equation}
\begin{aligned}
f_\text{dc} = \begin{cases} 
1 & \text{if the collision is ahead}, \\
-1 & \text{if the collision is behind}.
\end{cases} 
\end{aligned}
\end{equation}

The auxiliary objective for dynamic collision avoidance is defined as:
\begin{equation}
\begin{aligned}
\mathcal{L}_\text{dc}(\theta_y) = \mathbb{E}_t \left[ 
    \hat{A}_t^\text{dc} \cdot f_\text{dc} \cdot (\Delta \pi_y^{\text{dec}} - \Delta \pi_y^{\text{acc}})
\right],
\end{aligned}
\end{equation}
where \(\hat{A}_t^\text{dc}\) is the advantage estimate for dynamic collision avoidance.

\boldparagraph{Static Collision Auxiliary Objective.}  
The static collision auxiliary objective adjusts the steering control action $a_t^x$ based on the proximity to static obstacles. If the static obstacle is detected on the left side, the policy promotes rightward steering actions ($a_t^x > a_t^{x,\text{old}}$); if the static obstacle is detected on the right side, it promotes leftward steering actions ($a_t^x < a_t^{x,\text{old}}$). To formalize this behavior, we define a directional factor $f_\text{sc}$:  
\begin{equation}
\begin{aligned}
f_\text{sc} = \begin{cases} 
1 & \text{if static obstacle is on the left}, \\
-1 & \text{if static obstacle is on the right}.
\end{cases} 
\end{aligned}
\end{equation}

The auxiliary objective for static collision avoidance is defined as:  
\begin{equation}
\begin{aligned}
\mathcal{L}_\text{sc}(\theta_x) = \mathbb{E}_t \left[ 
    \hat{A}_t^\text{sc} \cdot f_\text{sc} \cdot (\Delta \pi_x^{\text{right}} - \Delta \pi_x^{\text{left}})
\right],
\end{aligned}
\end{equation}  
where $\hat{A}_t^\text{sc}$ is the advantage estimate for static collision avoidance.  

\boldparagraph{Positional Deviation Auxiliary Objective.}  
The positional deviation auxiliary objective adjusts the steering control action $a_t^x$ based on the ego vehicle's lateral deviation from the expert trajectory. If the ego vehicle deviates leftward, the policy promotes rightward corrections ($a_t^x > a_t^{x,\text{old}}$); if it deviates rightward, it promotes leftward corrections ($a_t^x < a_t^{x,\text{old}}$). We formalize this with a directional factor $f_\text{pd}$:  
\begin{equation}
\begin{aligned}
f_\text{pd} = \begin{cases} 
1 & \text{if ego vehicle deviates leftward}, \\
-1 & \text{if ego vehicle deviates rightward}.
\end{cases} 
\end{aligned}
\end{equation}

The auxiliary objective for positional deviation correction is:
\begin{equation}
\begin{aligned}
\mathcal{L}_\text{pd}(\theta_x) = \mathbb{E}_t \left[ 
    \hat{A}_t^\text{pd} \cdot f_\text{pd} \cdot (\Delta \pi_x^{\text{right}} - \Delta \pi_x^{\text{left}})
\right],
\end{aligned}
\end{equation}  
where $\hat{A}_t^\text{pd}$ estimates the advantage of trajectory alignment.

\boldparagraph{Heading Deviation Auxiliary Objective.}  
The heading deviation auxiliary objective adjusts the steering control action $a_t^x$ based on the angular difference between the ego vehicle’s current heading and the expert’s reference heading. If the ego vehicle deviates counterclockwise, the policy promotes clockwise corrections ($a_t^x > a_t^{x,\text{old}}$); if it deviates clockwise, it promotes counterclockwise corrections ($a_t^x < a_t^{x,\text{old}}$). To formalize this behavior, we define a directional factor $f_\text{hd}$:  
\begin{equation}
\begin{aligned}
f_\text{hd} = \begin{cases} 
1 & \text{if ego vehicle deviates clockwise}, \\
-1 & \text{if ego vehicle deviates counterclockwise}.
\end{cases} 
\end{aligned}
\end{equation}

The auxiliary objective for heading deviation correction is then defined as:  
\begin{equation}
\begin{aligned}
\mathcal{L}_\text{hd}(\theta_x) = \mathbb{E}_t \left[ 
    \hat{A}_t^\text{hd} \cdot f_\text{hd} \cdot (\Delta \pi_x^{\text{right}} - \Delta \pi_x^{\text{left}})
\right],
\end{aligned}
\end{equation}  
where $\hat{A}_t^\text{hd}$ is the advantage estimate for heading alignment.  

\begin{table*}[ht]
\begin{center}
\centering
\resizebox{0.98\textwidth}{!}{
\begin{tabular}{cccccccccccccc}
\toprule
\multirow{2}{*}{ID} & Dynamic & Static & Position & Heading & \multirow{2}{*}{CR$\downarrow$} &\multirow{2}{*}{DCR$\downarrow$} &\multirow{2}{*}{SCR$\downarrow$} &\multirow{2}{*}{DR$\downarrow$} &\multirow{2}{*}{PDR$\downarrow$} &\multirow{2}{*}{HDR$\downarrow$} &\multirow{2}{*}{ADD$\downarrow$} &\multirow{2}{*}{Long. Jerk$\downarrow$} &\multirow{2}{*}{Lat. Jerk$\downarrow$}\\
& Collision & Collision & Deviation & Deviation & & & & & & & & & \\
\midrule
1 & \cmark  &  &  &  & 0.172 & 0.154 & 0.018 & 0.092 & 0.033 & 0.059  & 0.259 & 4.211 & 0.095 \\
2 &  & \cmark & \cmark & \cmark & 0.238 & 0.217 & 0.021 & 0.090 & 0.045 & 0.045  & 0.241 & 3.937 & 0.098 \\
3 & \cmark &  & \cmark & \cmark & 0.146 & 0.128 & 0.018 & 0.060 & 0.030 & 0.030  & 0.263 & 3.729 & 0.083\\
4 & \cmark & \cmark &  & \cmark & 0.151 & 0.142 & 0.009 & 0.069 & 0.042 & 0.027 & 0.303 & 3.938 & 0.079\\
5 & \cmark & \cmark & \cmark &  & 0.166 & 0.157 & 0.009 & 0.048 & 0.036 & 0.012 & 0.243 & 3.334 & 0.067\\
6 & \cmark & \cmark & \cmark & \cmark & 0.089 & 0.080 & 0.009 & 0.063 & 0.042 & 0.021 & 0.257 & 4.495 & 0.082 \\
\bottomrule
\end{tabular}
}
\end{center}
\vspace{-2mm}
\caption{\textbf{Ablation on reward sources.} The table shows the impact of different reward components on performance.}
\label{tab:reward_ablation}
\end{table*}

\begin{table*}[ht]
\begin{center}
\centering
\resizebox{0.98\textwidth}{!}{
\begin{tabular}{ccccccccccccccc}
\toprule
\multirow{2}{*}{ID} & \multirow{2}{*}{PPO Obj.}  & Dynamic Col. & Static Col. & Position Dev. & Heading Dev. & \multirow{2}{*}{CR$\downarrow$} & \multirow{2}{*}{DCR$\downarrow$}  & \multirow{2}{*}{SCR$\downarrow$} & \multirow{2}{*}{DR$\downarrow$} & \multirow{2}{*}{PDR$\downarrow$} & \multirow{2}{*}{HDR$\downarrow$} & \multirow{2}{*}{ADD$\downarrow$} & \multirow{2}{*}{Long. Jerk$\downarrow$} & \multirow{2}{*}{Lat. Jerk$\downarrow$} \\
& & Auxiliary Obj. & Auxiliary Obj. & Auxiliary Obj. & Auxiliary Obj. & & & & & & & & & \\
\midrule
1 &\cmark&  &  &  &  & 0.249 & 0.223 & 0.026 & 0.077 & 0.047 & 0.030  & 0.266 & 4.209 & 0.104 \\
2 &\cmark& \cmark &  &  &  & 0.178 & 0.163 & 0.015 & 0.151 & 0.101 & 0.050 & 0.301 & 3.906 & 0.085 \\
3 &\cmark&  & \cmark & \cmark & \cmark & 0.137 & 0.125 & 0.012 & 0.157 & 0.145 & 0.012 & 0.296 & 3.419 & 0.071 \\
4 &\cmark& \cmark &  & \cmark & \cmark & 0.169 & 0.151 & 0.018 & 0.075 & 0.042 & 0.033 & 0.254 & 4.450 & 0.098 \\
5 &\cmark& \cmark & \cmark &  & \cmark & 0.149 & 0.134 & 0.015 & 0.063 & 0.057 & 0.006 & 0.324 & 3.980 & 0.086 \\
6 &\cmark& \cmark & \cmark & \cmark & & 0.128 & 0.119  & 0.009 & 0.066 & 0.030 & 0.036  & 0.254 & 4.102 & 0.092 \\
7 &&\cmark  &\cmark  &\cmark  &\cmark  & 0.187 &0.175  &0.012 &0.077 &0.056  &0.021  &0.309  &5.014  &0.112  \\
8 &\cmark& \cmark & \cmark & \cmark & \cmark & 0.089 & 0.080 & 0.009 & 0.063 & 0.042 & 0.021  & 0.257 & 4.495 & 0.082 \\
\bottomrule
\end{tabular}
}
\end{center}
\vspace{-2mm}
\caption{\textbf{Ablation on auxiliary objectives.} The table shows the impact of different auxiliary objectives on performance.}
\label{tab:auxiliary_ablation}
\end{table*}

\boldparagraph{Overall Auxiliary Objectives.}  
The overall auxiliary objectives are a weighted sum of the individual objectives:
\begin{equation}
\begin{aligned}
\mathcal{L}_\text{aux}(\theta) = &\lambda_1 \mathcal{L}_\text{dc}(\theta_y) + \lambda_2 \mathcal{L}_\text{sc}(\theta_x)  + \\ 
&\lambda_3 \mathcal{L}_\text{pd}(\theta_x) +\lambda_4 \mathcal{L}_\text{hd}(\theta_x),
\end{aligned}
\end{equation}
where $\lambda_1$, $\lambda_2$, $\lambda_3$, and $\lambda_4$ are weighting coefficients that balance the contributions of each auxiliary objective.

\boldparagraph{Optimization Objective.}  
The final optimization objective combines the clipped PPO objective with the auxiliary objective:
\begin{equation}
\mathcal{L}(\theta) = \mathcal{L}^{\text{PPO}}(\theta) + \mathcal{L}_\text{aux}(\theta).
\end{equation}

\vspace{-0.05 in}
\paragraph{Random $n$-gram Selection}
% In the process of reutilizing generated $n$-grams as draft tokens and applying repetition penalty, there exists an inherent trade-off. Meanwhile, 

In our experiments, we observe that the draft tokens provided to the target model for parallel validation often yield multiple valid groups. Building on this observation, we randomly select one valid $n$-gram to serve as the final output. By leveraging the fact that multiple valid $n$-grams emerge during verification, we ensure that the final output is both diverse and accurate.

% we observe that the draft tokens provided to the target model for parallel validation can yield multiple valid groups.

In summary, the overall flow of our framework is presented in \cref{alg:algorithm}. 







\begin{table*}[tb]
    \centering
    \begin{tabular}{l | c c c | c c c} \toprule
        \multirow{2}{*}{\textbf{ Differential Diagnosis}} & \multicolumn{3}{c|}{\textbf{gpt-4o test set (n=3403)}} & \multicolumn{3}{c}{\textbf{claude test set (n=2868)}} \\ \cmidrule(r){2-4} \cmidrule(l){5-7}
        & \textbf{Top-5} & \textbf{Top -1} & \textbf{MRR} & \textbf{Top-5} & \textbf{Top -1} & \textbf{MRR} \\ \midrule
        \textbf{baseline} & 56.80\% & 28.65\% & 0.390 & 56.69\% & 30.65\% & 0.406 \\ 
        \textbf{gpt-4o rare candidates} & 52.66\% & 25.95\% & 0.357 & 55.47\% & 29.04\% & 0.388 \\ 
        \textbf{\methodname candidates} & \textbf{74.38\%} & \textbf{33.12\%} & \textbf{0.471} & \textbf{71.41\%} & \textbf{33.23\%} & \textbf{0.461} \\ \bottomrule
    \end{tabular}
    \caption{Performance on generated gpt-4o ddx task. All metrics for \methodname on both datasets (see bolded) are significant using a two-sided Wilcoxon signed-rank test with $p<0.01$ compared to the no candidates baseline.}
    \label{tab:ddx}
\end{table*}

\begin{table*}[tb]
\centering
\begin{tabular}{l|cccccc} \toprule
\textbf{DDx LLM} & \textbf{Exact} & \textbf{Extremely Rel.} & \textbf{Relevant} & \textbf{Somewhat Rel.} & \textbf{Unrelated} \\ 
\midrule
\textbf{baseline gpt-4o} & 22.8\% & 19.9\% & 4.9\% & 21.0\% & 31.3\% \\ 
\textbf{\methodname gpt-4o} & 55.8\% & 8.8\% & 2.3\% & 12.8\% & 20.2\% \\ \midrule

\textbf{baseline claude} & 19.2\% & 16.9\% & 3.9\% & 14.5\% &45.6\%  \\ 
\textbf{\methodname claude} & 56.8\% & 10.7\% & 1.6\% & 10.6\% & 20.4\% \\ \midrule

\textbf{baseline Llama 3.3 70b} & 20.3\% & 19.3\% & 5.3\% & 21.7\% & 33.5\% \\ 
\textbf{\methodname Llama 3,3 70b} & 47.3\% & 12.2\% & 3.3\% & 15.4\% & 21.9\% \\ \bottomrule
\end{tabular}
\caption{We compare LLM baseline DDx generation performance to LLMs with addition of \methodname candidates.  We report the LLM as judge results across several categories of similarity, ranging from Exact Match to Unrelated. We combine gpt-4o and claude test sets for this analysis.}
\label{tab:ddx_by_sim}
\end{table*}


\begin{table*}[tb]
    \centering
    \begin{tabular}{l | c | c c c | c c c}
        \toprule
        \multirow{2}{*}{\textbf{Training Dataset}} & \multirow{2}{*}{\textbf{Training Size}} & \multicolumn{3}{c|}{\textbf{gpt-4o test set (n=3403)}} & \multicolumn{3}{c}{\textbf{claude test set (n=2868)}} \\ \cmidrule(r){3-5} \cmidrule(l){6-8}
         & & \textbf{Top-5} & \textbf{Top-1} & \textbf{MRR} & \textbf{Top-5} & \textbf{Top-1} & \textbf{MRR} \\ \midrule
        \textbf{claude} & 8837 & 48.37\% & 34.12\% & 0.4007 & 64.92\% & 45.64\% & 0.5371 \\ 
        \textbf{gpt-4o} & 21782 & 88.04\% & 63.88\% & 0.7410 & 44.18\% & 28.45\% & 0.3490 \\ 
        \textbf{gpt-4o downsampled} & 8813 & 70.88\% & 47.90\% & 0.5742 & 37.20\% & 23.25\% & 0.2884 \\ 
        \textbf{gpt-4o + claude} & 30619 & 88.80\% & 64.21\% & 0.7463 & 77.82\% & 56.35\% & 0.6526 \\ 
        \bottomrule
    \end{tabular}
    \caption{Evaluation on the candidate generation task, with MRR, Top-5 and Top-1 Accuracy.  We evaluate on models only trained on claude data, gpt-4o data, and both, and evaluate separately on claude and gpt-4o test sets. We include a model trained on a downsampled set of gpt-4o data that approximates the size of the claude training set.}

    \label{tab:cand_gen}
\end{table*}
\FloatBarrier
\section{Results}
\subsection{Experimentation on the Technical Indicator Dataset}
The analysis of the Technical Indicator dataset, without any feature rearrangement, as illustrated in the figure below, uncovers a notable pattern in the accumulation of rewards over different time intervals. The most significant gain, observed in the 2-week observation size, reached a cumulative reward of 155.89. This finding highlights the efficacy of this specific observation window. The peak performance noted within this 2-week timeframe may constitute the most advantageous period for analysis in the context of this dataset and its feature composition. This observation window provides the optimal balance mentioned in our hypothesis section, generating the most significant rewards in the given feature arrangement setting and dataset.

\begin{figure}[ht]
\centering
\includegraphics[width=\linewidth]{old_data_no_shuffle.pdf}
\caption{Cumulative rewards in the Technical Indicator dataset without rearrangement }
\label{fig:tech_indicator_not_rearranged}
\end{figure}

The extended analysis of the Technical Indicator dataset over periods ranging from 4 to 12 weeks reveals a discernible decline in cumulative rewards, reaching its lowest point at the 10-week interval, where the reward significantly drops to 104.58. This downward trajectory, although slightly mitigated in the 12-week observation window, predominantly suggests diminishing returns as the duration of the observation period increases. This pattern serves as a crucial insight, highlighting the limitations of the convolutional neural network (CNN) in effectively utilizing longer observation windows for this specific dataset and feature configuration. This trend underscores the importance of strategically selecting the observation window to optimize the CNN's predictive performance, and it supports our hypothesis that information overload can diminish the CNN's ability to utilize most important features in the input tensor.

During the analysis of the Technical Indicator dataset with rearranged features, as depicted in the figure below, we found a markedly different trend in cumulative rewards across varying timeframes compared to the dataset with the original feature arrangement. The rearranged dataset demonstrates a similar pattern, where the peak cumulative reward is noted at the 10-week mark, registering at 121.59. This outcome indicates that the rearrangement of features shifts the optimal observation window to bigger sizes. Notably, a prolonged 10-week period emerges as most favorable in the rearranged dataset, in stark contrast to the 2-week window size identified as optimal in the original dataset configuration. This finding suggests that feature rearrangement significantly improves the model's ability to utilize longer observation windows, again underscoring the need for adaptable strategies in financial data analysis with CNNs.

\begin{figure}[ht]
    \centering
    \includegraphics[width=\linewidth]{old_data_shuffled.pdf}
    \caption{Cumulative rewards in the Technical Indicator dataset with rearrangement }
    \label{fig:tech_indicator_rearranged}
\end{figure}

As depicted in the figure, rearranging features within the technical indicator dataset markedly improves the model's capacity to capitalize on extended observation windows. Notably, the model's optimal performance, demonstrated at the 10-week interval with a cumulative reward of 121.59, signifies an enhanced ability to utilize more extended periods for analysis. This reorganization of features enables a more efficient interpretation of extended-term trends, optimizing the model's accuracy over such durations. This finding emphasizes the vital importance of feature engineering in amplifying the effectiveness of Convolutional Neural Networks, particularly in intricate and dynamic settings like financial market analysis.

In contrast, a different pattern emerges when analyzing the technical indicator dataset without feature rearrangement, as illustrated in the corresponding plot. Here, the 2-week interval emerges as the most productive timeframe, registering the highest cumulative reward of 155.89. This finding indicates that in its original configuration, the dataset is optimally tuned for short-term analysis, showing diminishing performance with lengthening observation periods, except for a slight increase at 12 weeks. However, these extended periods do not outperform the initial 2-week observation window. This trend highlights the model's predisposition towards shorter timeframes when processing the non-rearranged data, underscoring the impact of data structuring on the model's temporal adaptability and predictive power.

\begin{figure}[ht]
    \centering
    \includegraphics[width=\linewidth]{old_data_best.pdf}
    \caption{Best performers in the Technical Indicator Dataset}
    \label{fig:sma_nonrearranged}
\end{figure}

The contrasting results observed in the rearranged technical indicator data are striking. In this scenario, the model strides in the 10-week observation period, achieving a cumulative reward of 121.59. This shift from the optimal 2-week period in the non-rearranged data to a more extended 10-week period in the rearranged data is significant. The rearranging of features profoundly influences the model's efficiency in capturing and forecasting market trends. Compared to the reduced effectiveness in shorter durations, the enhanced performance at this longer interval underscores the impact of data sequencing on the model's predictive precision. This observation again stresses the criticality of data arrangement and preprocessing in financial time series analysis, as it can substantially alter the model's interpretation and response to market dynamics over different temporal scales. 

\subsection{Experimentation on the SMA dataset}
The analysis of the SMA dataset without data rearrangement reveals a distinct pattern in cumulative rewards over various timeframes, as shown in Figure \ref{fig:sma_rearranged}. The most significant performance is apparent in the 2-week observation window, achieving a peak cumulative reward of 184.05. This high point suggests that a 2-week observation window is particularly effective for this dataset, indicating an optimal short-term period for analysis in this context.

\begin{figure}[ht]
    \centering
    \includegraphics[width=\linewidth]{new_data_no_shuffle.pdf}
    \caption{Cumulative rewards in the SMA dataset without rearrangement}
    \label{fig:sma_rearranged}
\end{figure}

As the observation window extends, a decreasing trend in cumulative rewards is evident, particularly at 8 and 12 weeks, with rewards noted at 99.80 and 105.99, respectively. However, an unexpected increase in cumulative reward to 144.22 at the 10-week mark presents an intriguing anomaly. This inconsistency might indicate complex, possibly cyclical patterns in the SMA dataset, which the model discerns differently across various intervals. This behavior further highlights the intricate nature of these quantitative indicators and emphasizes the importance of selecting an appropriate observation window for predictive modeling.

A different outcome is observed in the analysis of the SMA dataset with feature rearrangement. The 4-week interval emerges as the most favorable, registering a peak cumulative reward of 181.84. This result contrasts the lower performance in the 2-week window, where the cumulative reward is 117.14. This discrepancy suggests that rearranging the data may significantly alter the model's ability to utilize temporal relationships in the data, affecting its effectiveness across different timeframes. The rearranged dataset's peak at a longer interval underlines the same pattern where feature arrangement enhances the model's ability to effectively capture and analyze market trends.

\begin{figure}[ht]
    \centering
    \includegraphics[width=\linewidth]{new_data_shuffled.pdf}
    \caption{Cumulative rewards in the SMA dataset with rearrangement }
    \label{fig:new_data_no_suhffle}
\end{figure}

However, an irregular trend emerges as the observation period extends beyond 4 weeks. A marked decrease in cumulative rewards is noted at 6 and 8 weeks, with figures falling to 101.04 and 90.77, respectively. Intriguingly, there is a modest reward recovery at the 10 and 12-week intervals. This pattern suggests that the model may interpret different characteristics of the rearranged SMA dataset over extended timeframes. Such fluctuations in performance underscore the added complexity due to data rearrangement and the importance of carefully choosing the observation window to maximize the model's efficacy. 

\subsection{Best Performers in the SMA Dataset}
In the next phase of our data analysis, we conducted a comparative study of optimal timeframes in the simple moving average (SMA) dataset, considering its original and rearranged forms, as shown in the plot. This revealed distinctive trends. 

In the case of the non-rearranged SMA dataset, the most effective timeframe emerges as the 2-week window, registering a peak cumulative reward of 184.05. This notable performance at the shorter interval indicates the model's ability to effectively capture the prevailing trends within the original SMA data structure. As the observation period extends, a gradual decline in cumulative rewards is observed across longer timeframes. Although there is a marginal uplift in performance at the 10-week mark, this is within the benchmark set by the 2-week observation window, which means that the pattern still highlights the dataset's responsiveness to short-term fluctuations.

The clear differentiation in performance across various timeframes suggests that the underlying dynamics of the SMA dataset are more readily discernible and exploitable in shorter intervals when the data remains in its original sequence. This insight is pivotal for financial analysts and modelers, emphasizing the need for strategic consideration of time windows in predictive modeling, especially when dealing with complex financial datasets like the SMA.
\begin{figure}[ht]
    \centering
    \includegraphics[width=\linewidth]{new_data_best.pdf}
    \caption{Best performers in the SMA dataset }
    \label{fig:sma_data_best_performers}
\end{figure}

In contrast, after rearranging the features in the SMA dataset, our analysis presents a different optimal timeframe. The 4-week window emerges as the best performer with a cumulative reward of 181.84, indicating a significant shift in the model's ability to utilize longer temporal windows. Our analysis also shows a more pronounced decline in performance for other timeframes, especially at 6 and 8 weeks. We noted that the 2-week observation size was the best performer in the non-rearranged data versus the 4-week peak in the rearranged data. Once again, the sharp contrast between the non-rearranged and rearranged data demonstrates the model's temporal processing ability.

\subsection{Best Performers overall}
Several insightful trends emerge in our final analysis of the datasets, encompassing both the SMA and Technical Indicator datasets. In its original feature arrangement, the SMA dataset exhibits strong performance in the 2-week timeframe, reaching a cumulative reward of 184.057, the highest across all datasets and timeframes. This result underscores the effectiveness of short-term observation in capturing market dynamics with this dataset. On the other hand, when the SMA features are rearranged, the 4-week window becomes the most productive, achieving a cumulative reward of 181.84. This shift suggests that market dynamics are captured more effectively over shorter temporal windows, but once features are rearranged, a slightly extended observation size proved more effective.

\begin{figure}[ht]
    \centering
    \includegraphics[width=\linewidth]{best_over_all.pdf}
    \caption{Best performers overall. }
    \label{fig:best_performers_overall}
\end{figure}

The observed trends in the Technical Indicator dataset echo those seen in the SMA dataset, particularly in the context of the original sequence. A 2-week observation window demonstrates optimal effectiveness, reaching a peak cumulative reward of 155.89. This similarity across the datasets consistently proves our decerned pattern that, without shorter observation periods, can be highly effective for predictive modeling. However, a significant shift occurs when the data sequence in the Technical Indicator dataset is rearranged. This modification leads to the 10-week timeframe becoming the most favorable, as evidenced by a cumulative reward of 121.59.

This shift indicates that the Convolutional Neural Network (CNN) becomes more adept at discerning the complex patterns between features and their temporal dynamics when the data is organized to maintain a cohesive structure for each company's features. The rearrangement enhances the model's ability to grasp longer-term trends and relationships, which may be less apparent or accessible in shorter timeframes or with non-rearranged data. This observation is crucial as it suggests that the efficacy of a CNN in financial market analysis can be significantly influenced by how the data is structured. It highlights the importance of considering the arrangement of data to optimize the predictive capabilities of models, especially in financial contexts where the relationships between various indicators and their evolution over time are crucial to understanding market movements. Thus, a flexible and context-specific approach to selecting observation periods and organizing data is paramount to maximizing the utility and accuracy of predictive models in financial analysis.

We outlines some applications of NeurFlow.
We hypothesize that, as one neuron can have multiple meanings, a DNN looks at a group of neurons rather than individually to determine the exact features of the input. Hence, we propose a metric that assesses a model's confidence in determining whether the input contains a specific visual feature. For a group $G$ with core concept neurons $\sS_G = \{s_{G,1}, \dots, s_{G, |\sS_G|}\}$, the metric denoted as $ M(v, \sS_G, \mathcal{D}) = \exp(\frac{1}{|\sS_G|}\sum_{s \in \sS_G}\log(\|\phi_s(v)/\max(\phi_s, \mathcal{D}))\|)$, where $v \in \mathcal{D}$ and $\max(\phi_s, \mathcal{D})$ is the highest value of activation of neuron $s \in \sS_G$ on dataset $\mathcal{D}$.
This returns high score when all neurons in $G$ have high activation (indicating high confidence), while resulting in almost zero if any neuron in the group has low activation (indicating low confidence). We can use this metric to determine how similar the features in the input image are to the predetermined neuron groups concept. 
The specific setup can be found in the Appendix \ref{debugging_setup}. Figures \ref{fig:img debug} and \ref{fig:debias} demonstrate the usage of the metric and the concept circuit.
We use the term \emph{NGC} to denote the concept of a neuron group. 
\begin{figure}[t]
    \centering
    \begin{minipage}{.45\textwidth}
       \vspace{-12mm}
        \includegraphics[width=0.95\columnwidth]{figures/image_debug.png}
     %   \vspace{-3mm}
        \caption{\textbf{Using NeurFlow to reveal the reason behind model's prediction.} The top concepts can be traced throughout the circuit.}\label{fig:img debug}   
        \vspace{-10pt}
    \end{minipage}
    \hfill
    \begin{minipage}{.5\textwidth}
        \vspace{-12mm}
        \includegraphics[width=0.95\columnwidth]{figures/MLLM_captioning.pdf}
    %    \vspace{-5mm}
        \caption{Demonstration for automatically labelling and explaining the relation of NGCs on class ``great white shark" using GPT4-o \citep{Gpt4o}. The captions and the names of the NGCs are highlighted in blue, while the relations are in black.} \label{fig:captioning}
        \vspace{-10pt}
    \end{minipage}
    \vspace{-5pt}
\end{figure}
% Note that, a recent work \citep{VCC}, a concept-based method that also explains the inner mechanism of DNNs, has a similar application of debugging images. They measure how close the activations of misclassified images are to the concept vectors using $l_2$ norm. However, they provide no objective proof. In contrast, one advantage of learning via neurons is that we can edit the neurons related to a concept to see whether it has significant impact on the final predictions (see section \ref{debugging}).

\subsection{Image debugging}
\label{debugging}
We aim to use the concept circuit to identify concepts contributing to false prediction, which we call \textit{image debugging}. If a concept contributes to a class when it should not, we say that the prediction (or equivalently, the model) is \textit{biased} by that concept. \citet{Debias} propose a framework for detecting biases in a vision model by generating captions for the predicted images and tracking the common keywords found in the captions. With this method, they concluded that the pretrained ResNet50 is biased by ``flower pedals" in the class ``bee". However, correlational features do not imply causation and can lead to misjudgments. We verify and enhance the causality of their claim by examining the concept circuit of class ``bee", and conducting experiments on the probabilities of the final predictions with and without neurons that related to ``flowers". Additionally, we discover that the model also suffers from ``green background" bias (resemble ``leaves"), which is not mentioned in \citet{Debias}. 

Figure \ref{fig:debias} shows the process of debugging false positive images. Three different concepts are presented in \textit{layer4.2} of ResNet50, representing ``pink pedals", ``green background", and ``bee" respectively (we choose this layer as it has a small set of NGCs, however, our following experiment is consistent for multiple layers and with different classes). We discover that most of the false positive images have high metric score for ``pedal" and ``green background". 
To further verify the impact of these biased features, we mask all neurons in the groups of the respective concepts and find that the probability of the predictions are distorted drastically (and predictions is no longer ``bee"), as opposed to masking random neurons, which yield negligible changes. 

\begin{figure}
\begin{center}
\vspace{-12mm}
\includegraphics[width=0.9\textwidth]{figures/bias.pdf} 
\end{center}
\vspace{-3mm}
\caption{(left) The metric scores of false positive images for each concept in \textit{layer4.2} of ResNet50. (right) Showing the images that have the greatest drop in the activation of the logit neuron when masking each group concept. Verifying that the neuron groups indeed reflect the concepts.} \label{fig:debias}
\vspace{-15pt}
\end{figure}
This implies the dependence on the biased concept. \textit{But how do we know that the groups reflect the respective visual features?} If these groups indeed represent the visual features, then masking them should hinder the classification probability for images that include those features. We highlight the top images that have the largest decrease in the value of the logit neuron (corresponding to class ``bee") on both validation set of the target class and augmented dataset (see Section \ref{subsec:identify_node}). As shown in Figure \ref{fig:debias}, this process indeed yields the images that contain the respective features.

To demonstrate how NeurFlow's findings differ from those of existing methods, we conduct a qualitative experiment comparing the core concept neurons identified by NeurFlow with those identified by NeuCEPT \citet{NEUCEPT}. Detailed information about this experiment is provided in Appendix \ref{sec:compare_neucept_img_debugging}. Our observations indicate that 
% NeurFlow identifies concepts more closely resembling the original images. Additionally, 
the top logit drop images identified by NeurFlow align better with the representative examples of the corresponding concepts. Moreover, masking the core concept neuron groups identified by NeurFlow resulted in more significant changes to prediction probabilities while utilizing fewer neurons compared to the groups identified by NeuCEPT.


\subsection{Automatic identification of layer-by-layer relations}
\label{labeling}
\vspace{-5pt}
While automatically discovering concepts from inner representation has been a prominent field of research \citep{CRAFT}, automatically explaining the resulting concepts is often ignored, relying on manual annotations. \citet{Invert} utilize label description in ImageNet dataset to generate caption for neurons, however, these annotations is limited and can not be used to label low level concepts. Drawing inspiration from \citet{hoang2024llm, falcon}, we go one step further and not only use MLLM to label the (group of) neurons but also explain the relations between them in consecutive layers. Thus, we show the prospect of completing the whole picture of abstracting and explaining the inner representation in a systematic manner. 

Specifically, for two consecutive layers, we ask MLLM to describe the common visual features in a NGC, then matching with those of the top NGC (with the highest weights) at the preceding layer. This can be done iteratively throughout the concept circuit, generating a comprehensive explanation without human effort. We use a popular technique \citep{Chain_of_thought} to guide GPT4-o \citep{Gpt4o} step by step in captioning and in visual feature matching. Figure \ref{fig:captioning} shows an example of applying this technique to concept circuit of class ``great white shark". We observe that MLLM can correctly identify the common visual features within exemplary images of NGCs. Furthermore, MLLM is able to match the features from lower level NGCs to those at higher level, detailing formation of new features, showing the potential of explaining in automation, capturing the gradual process of constructing the output of the model. The prompt used in this experiment is available in Appendix \ref{prompt}.


\section{Discussion}

\subsection{Interpretation of Results}
The outcomes of this study, utilizing Convolutional Neural Networks (CNNs) within a Deep Reinforcement Learning (DRL) framework for financial analytics, underscore the pivotal role of temporal precision in market predictions. The findings particularly emphasize the efficacy of short-term observation windows, with the two-week window demonstrating superior performance in capturing market dynamics. This observation resonates with the rapidity and volatility characteristic of financial markets, where new information is swiftly reflected in stock prices. Notably, the enhanced model performance achieved through feature rearrangement highlights the significant impact of feature engineering. By reorganizing features related to different stocks and technical indicators, we support the hypothesis that a more robust and generalizable representation of data can be learned, potentially increasing the model's adaptability to diverse market conditions.

\subsection{Theoretical Implications}
The success of the CNN model, employing a methodical window expansion technique, accentuates the importance of temporal dynamics in financial time-series analysis. This approach aligns with the efficient market hypothesis, positing that markets assimilate all available information into stock prices. The model's ability to adjust to the market's 'memory' is crucial for precise financial forecasting. Additionally, the effectiveness of the rearranged features approach suggests that the organization and representation of information are as critical as the information itself for the learning process. This sheds light on the interaction of features within CNN layers, stressing the role of feature engineering in refining financial models.

Our findings challenge the conventional wisdom that longer observation windows invariably lead to better predictions in financial markets. The superior performance of shorter windows, particularly in non-rearranged datasets, suggests that recent market information may carry more predictive power than extended historical data. This aligns with the concept of market efficiency, where new information is rapidly incorporated into prices.

\section{Conclusions and Future Work}

In this study, we explored the impact of different data structures and observation windows on Convolutional Neural Networks (CNNs) performance in financial market analysis. We focused on understanding how the arrangement of data and the selection of timeframes influence the model's ability to capture and predict market dynamics.

The shift in optimal performance with rearranged data suggests that a Convolutional Neural Network (CNN) becomes more proficient at identifying complex patterns between features and their temporal dynamics when the data maintains a cohesive structure for each company's features. The ability of the model to grasp longer-term trends and relationships in rearranged datasets, which are less apparent in shorter timeframes or non-rearranged data, emphasizes the importance of strategic data arrangement. This finding underlines the need for flexible and specific approaches in selecting observation periods and organizing data to enhance the utility and accuracy of CNNs in financial market analysis.

\subsection{Limitations and Challenges}

This study, while offering valuable insights into the use of Convolutional Neural Networks (CNNs) for financial market analysis, encounters several limitations and challenges that warrant attention. A primary constraint is the need for increased computational power. Exploring additional observation windows and testing larger, more complex CNN architectures necessitate substantial computational resources. This requirement becomes particularly critical when considering the intricacy and volume of financial data and the need for extensive testing to validate the robustness and accuracy of the models across various market scenarios.

Moreover, there is a significant need for research funding to support these endeavors. Enhanced funding would facilitate access to more powerful computing infrastructure and enable a broader scope of experimentation. This includes investigating a wider array of temporal windows and deploying more advanced CNN models, which could potentially uncover deeper insights and yield more precise predictive capabilities.

Future studies could explore incorporating diverse data types, like news sentiment or economic indicators, to enhance model robustness. Broadening the scope to different markets and asset types would help verify the applicability of these findings. Investigating the effects of shorter temporal windows or real-time data streams may provide insights into high-frequency trading strategies. Additionally, applying the concept of rearranged features to other forms of financial data, such as order book information or unstructured data, could pave the way for innovative advancements in financial modeling techniques.

In conclusion, our study makes several key contributions to the field of financial DRL. First, we demonstrate the critical importance of temporal window selection in CNN-based models. Second, we show that feature rearrangement can significantly alter the optimal observation period. Finally, we provide a methodological framework for systematically exploring these parameters in future research. These insights open new avenues for enhancing the accuracy and robustness of AI-driven financial analysis tools.

% \section*{Acknowledgements}
% Thanks to ...

%% The Appendices part is started with the command \appendix;
%% appendix sections are then done as normal sections
% \appendix

% \section{Appendix title 1}
%% \label{}

% \section{Appendix title 2}
%% \label{}

%% If you have bibdatabase file and want bibtex to generate the
%% bibitems, please use
%%
\section{Declaration of generative AI and AI-assisted technologies in the writing process}
% \textbf{Declaration of generative AI and AI-assisted technologies in the writing process}
During the preparation of this work the author(s) used ChatGPT  as writing assistant to draft text, improving clarity , proofreading, language refinement, and saving time. After using this tool/service, the author(s) reviewed and edited the content as needed and take(s) full responsibility for the content of the publication.


% \bibliographystyle{IEEEtran} % We choose the "plain" reference style
% \begin{thebibliography}{24}

% \begin{thebibliography}{}

\bibliography{7-biblio}

% \bibitem{parallel_drl_stock_trading}
% Wang, Xiaoyang, et al. "A parallel multi-module deep reinforcement learning algorithm for stock trading." Neurocomputing 439 (2021): 23-34. Elsevier. DOI: 10.1016/j.neucom.2021.01.128. URL: https://www.sciencedirect.com/science/article/pii/S0925231221005233.

% \bibitem{novel_drl_stock_trading}
% Zhang, Wei, et al. "A novel Deep Reinforcement Learning based automated stock trading system." Expert Systems with Applications 184 (2021): 115548. Elsevier. DOI: 10.1016/j.eswa.2021.115548. URL: https://www.sciencedirect.com/science/article/pii/S0957417423033031.

% \bibitem{market_sentiment_drl_stock_trading}
% Huang, Xin, Haoran Tan, and Qing Xu. "Market sentiment-aware deep reinforcement learning approach for stock trading." Procedia Computer Science 181 (2021): 1147-1155. Elsevier. DOI: 10.1016/j.procs.2021.01.296. URL: https://www.sciencedirect.com/science/article/pii/S2215098621000070.

% \bibitem{drl_end_to_end_stock_trading}
% Liu, Qi, et al. "Deep Reinforcement Learning Based End-to-End Stock Trading Strategy." Procedia Computer Science 184 (2021): 320-326. Elsevier. DOI: 10.1016.j.procs.2021.01.280. URL: https://www.sciencedirect.com/science/article/pii/S2215098621000197.

% \bibitem{adaptive_drl_stock_trading}
% Li, Bo, Liang Zhang, Rui Sun, and Zhaoyu Sun. "Adaptive stock trading strategies with deep reinforcement learning." ScholarX Journal 27.3 (2021): 451-467. URL: https://scholar.xjtlu.edu.cn/en/publications/adaptive-stock-trading-strategies-with-deep-reinforcement-learning.

% \bibitem{Liu2020}
% Liu, Xiao-Yang, et al. "FinRL-Meta: Market Environments and Benchmarks for Data-Driven Financial Reinforcement Learning." Deep Reinforcement Learning Workshop, 34th Conference on Neural Information Processing Systems (NeurIPS2020), 2020. Vancouver, Canada. arXiv:2011.09607v2 [q-fin.TR]. DOI: 10.48550/arXiv.2011.09607. URL: https://doi.org/10.48550/arXiv.2011.09607.

% \bibitem{Vyawahare2020}
% Vyawahare, Abides, et al. "ABIDES-Gym: Gym Environments for Multi-Agent Discrete Event Simulation and Application to Financial Markets." Proceedings of the IEEE Conference on Business Informatics (CBI), 2020.

% \bibitem{Montazeri2023}
% S. Montazeri, A. Mirzaeinia, H. Jumakhan, and A. Mirzaeinia, "CNN-DRL for Scalable Actions in Finance," presented at the 10th Annual Conf. on Computational Science \& Computational Intelligence, 2024. [Online]. Available: https://doi.org/10.48550/arXiv.2401.06179

% \bibitem{Montazeri2024}
% S. Montazeri, A. Mirzaeinia, and A. Mirzaeinia, "CNN-DRL with Shuffled Features in Finance," presented at the 10th Annual Conf. on Computational Science \& Computational Intelligence (CSCI'23), 2024. [Online]. Available: https://doi.org/10.48550/arXiv.2402.03338

% \bibitem{Zhang2019}
% Zhang, Wei, et al. "FinRL-Podracer: High Performance and Scalable Deep Reinforcement Learning for Quantitative Finance." Deep Reinforcement Learning Workshop, 34th Conference on Neural Information Processing Systems (NeurIPS2020), 2019.

% \bibitem{Tsay2010}
% Tsay, Ruey S. "Analysis of Financial Time Series." John Wiley \& Sons, 2010.

% \bibitem{Kumar2007}
% Kumar, P. R., and V. Ravi. "Bankruptcy prediction in banks and firms via statistical and intelligent techniques – A review." European Journal of Operational Research 180.1 (2007): 1-28.

% \bibitem{Atsalakis2009}
% Atsalakis, G. S., and K. P. Valavanis. "Surveying stock market forecasting techniques – Part II: Soft computing methods." Expert Systems with Applications 36.3 (2009): 5932-5941.

% \bibitem{Mnih2015}
% Mnih, Volodymyr, et al. "Human-level control through deep reinforcement learning." Nature 518.7540 (2015): 529-533.

% \bibitem{Tsantekidis2017}
% Tsantekidis, Avraam, et al. "Forecasting stock prices from the limit order book using convolutional neural networks." Proceedings of the IEEE Conference on Business Informatics (CBI), 2017.

% \bibitem{courbariaux2016binarized}
% Courbariaux, M., Hubara, I., Soudry, D., El-Yaniv, R., \& Bengio, Y. (2016). Binarized neural networks: Training deep neural networks with weights and activations constrained to +1 or −1. arXiv preprint arXiv:1602.02830.

% \bibitem{dai2021coatnet}
% Dai, Z., Liu, H., Le, Q. V., \& Tan, M. (2021). Coatnet: Marrying convolution and attention for all data sizes. arXiv preprint arXiv:2106.04803.

% \bibitem{chen2019compressing}
% Chen, K., \& Huo, Q. (2019). Compressing CNN–DBLSTM models for OCR with teacher–student learning and Tucker decomposition. Pattern Recognition, 96, 106957. Elsevier.

% \bibitem{dai2016rfcn}
% Dai, J., Li, Y., He, K., \& Sun, J. (2016). R-FCN: Object detection via region-based fully convolutional networks. IEEE Transactions on Pattern Analysis and Machine Intelligence. IEEE.

% \bibitem{chollet2017xception}
% Chollet, F. (2017). Xception: Deep learning with depthwise separable convolutions. arXiv preprint arXiv:1610.02357.

% \bibitem{LeCun2015}
% LeCun, Yann, Yoshua Bengio, and Geoffrey Hinton. "Deep learning." Nature 521.7553 (2015): 436-444.

% \bibitem{Krizhevsky2012}
% Krizhevsky, Alex, Ilya Sutskever, and Geoffrey E. Hinton. "ImageNet classification with deep convolutional neural networks." Advances in Neural Information Processing Systems 25 (2012).

% \bibitem{Goodfellow2016}
% Goodfellow, Ian, Yoshua Bengio, and Aaron Courville. "Deep Learning." MIT Press, 2016.

% \bibitem{He2016}
% He, Kaiming, et al. "Deep residual learning for image recognition." Proceedings of the IEEE Conference on Computer Vision and Pattern Recognition, 2016: 770-778.

% \bibitem{Simonyan2014}
% Simonyan, K., \& Zisserman, A. (2014). Very deep convolutional networks for large-scale image recognition. arXiv preprint arXiv:1409.1556.

% \bibitem{Szegedy2015}
% Szegedy, C., Liu, W., Jia, Y., Sermanet, P., Reed, S., Anguelov, D., Erhan, D., Vanhoucke, V., \& Rabinovich, A. (2015). Going deeper with convolutions. In Proceedings of the IEEE Conference on Computer Vision and Pattern Recognition (pp. 1-9).

% \end{thebibliography}

% \end{thebibliography}

\end{document}

\endinput
%%
%% End of file `elsarticle-template-harv.tex'.