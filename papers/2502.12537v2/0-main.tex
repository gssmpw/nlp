\documentclass[11pt]{article}

%% The amssymb package provides various useful mathematical symbols
\usepackage{amssymb}
\usepackage{lipsum}
\usepackage[table]{xcolor}
\usepackage{xcolor}
\usepackage{booktabs} % For better looking tables
\usepackage{tabularx}
\usepackage{amsmath}
\usepackage{float}
\usepackage{placeins}
\usepackage{url}
\usepackage[pdftex]{graphicx}
\usepackage{caption,subcaption}
\usepackage{hyperref}
\usepackage[numbers]{natbib}
\bibliographystyle{plainnat}
%% You might want to define your own abbreviated commands for common used terms, e.g.:
\newcommand{\kms}{km\,s$^{-1}$}
\newcommand{\msun}{$M_\odot$}

\title{Finding Optimal Trading History in Reinforcement Learning for Stock Market Trading}

\author{Sina Montazeri\thanks{University of North Texas, Denton, Texas 76207, USA}
\and Haseebullah Jumakhan\thanks{Ajman University, Ajman, United Arab Emirates}
\and Amir Mirzaeinia\thanks{University of North Texas, Denton, Texas 76207, USA}}

\date{\today}

\begin{document}
\maketitle

\begin{abstract}
This paper investigates the optimization of temporal windows in Financial Deep Reinforcement Learning (DRL) models using 2D Convolutional Neural Networks (CNNs). We introduce a novel approach to treating the temporal field as a hyperparameter and examine its impact on model performance across various datasets and feature arrangements. We introduce a new hyperparameter for the CNN policy, proposing that this temporal field can and should be treated as a hyperparameter for these models. We examine the significance of this temporal field by iteratively expanding the window of observations presented to the CNN policy during the deep reinforcement learning process. Our iterative process involves progressively increasing the observation period from two weeks to twelve weeks, allowing us to examine the effects of different temporal windows on the model's performance. This window expansion is implemented in two settings. In one setting, we rearrange the features in the dataset to group them by company, allowing the model to have a full view of company data in its observation window and CNN kernel. In the second setting, we do not group the features by company, and features are arranged by category. Our study reveals that shorter temporal windows are most effective when no feature rearrangement to group per company is in effect. However, the model will utilize longer temporal windows and yield better performance once we introduce the feature rearrangement. To examine the consistency of our findings, we repeated our experiment on two datasets containing the same thirty companies from the Dow Jones Index but with different features in each dataset and consistently observed the above-mentioned patterns. The result is a trading model significantly outperforming global financial services firms such as the Global X Guru by the established Mirae Asset.
\end{abstract}

\tableofcontents

%% main text

\section{Introduction}

In sensor networks, maintaining data freshness is crucial to support diverse applications such as environmental monitoring, industrial automation, and smart cities \cite{kandris2020applications}. A critical metric for quantifying data freshness is the Age of Information (AoI), which measures the time elapsed since the last received update was generated \cite{yates2012}. Minimizing AoI is essential in dynamic environments, where obsolete information can result in inaccurate decisions or missed opportunities. Efficient AoI management involves balancing update frequency, data relevance, and network resource constraints to ensure decision-makers have timely and accurate information when required \cite{yates2021age}. The significance of AoI has led to extensive research on its optimization across various domains, including single-server systems with one or multiple sources \cite{modiano2015,mm1,sun2016,najm2018,soysal2019,9137714,yates2019,zou2023costly}, scheduling strategies \cite{modiano-sch-1,9007478,sch-igor-1,9241401,sch-li,sch-sun}, and analysis of resource-constrained systems \cite{const-ulukus,const-biyikoglu,const-arafa,const-farazi,const-parisa}. 

%\ali{A good transition here would be: one particular area that has been garnering focus by the AoI researchers and that is correalted systems. In fact, sensor networks often handle...}

Among the strategies for AoI minimization, packet preemption is regarded as a cornerstone approach for ensuring the timeliness of information in communication networks, especially when resources such as service rates are limited \cite{yates2021age}. By prioritizing critical updates, preemption ensures that the most valuable data reaches its destination promptly, as demonstrated in the context of single-sensor, memoryless systems \cite{kaul2012status,inoue2019general}. Beyond this specific scenario, numerous studies have extensively investigated its role in optimizing AoI across diverse settings. For example, \cite{maatouk2019age} analyzes systems with prioritized information streams sharing a common server, where lower-priority packets may be buffered or discarded. Similarly, \cite{wang2019preempt} and \cite{kavitha2021controlling} examine preemption strategies for rate-limited links and lossy systems, identifying in the process the optimal policies for minimizing the AoI.

On the other hand, one particular area that has been garnering focus among AoI researchers is correlated systems. In fact, sensor networks often handle correlated data streams, where relationships between data collected by different sensors can be leveraged to enhance decision-making, reduce redundancy, and improve overall system performance \cite{mahmood2015reliability,yetgin2017survey}. This correlation often arises when multiple sensors monitor overlapping areas or related phenomena, allowing them to collaboratively exchange information and optimize resource usage. The role of correlation in sensor networks has further been explored in studies focusing on its potential to optimize system efficiency and effectiveness \cite{he2018,tong2022,popovski2019,modiano2022,ramakanth2023monitoring,erbayat2024}.











% The importance of AoI and correlation in sensor networks has motivated extensive research into optimizing AoI within correlated sensor systems. For example, \cite{he2018} studied sensor networks with overlapping fields of view, proposing a joint optimization framework for fog node assignment and transmission scheduling to reduce the AoI of multi-view image data. Similarly, \cite{tong2022} focused on overlapping camera networks, introducing scheduling algorithms for multi-channel systems designed to minimize AoI. Other works, such as \cite{popovski2019, modiano2022}, leveraged probabilistic correlation models to formulate sensor scheduling strategies aimed at lowering AoI. Additionally, \cite{ramakanth2023monitoring} treated the correlation of status updates as a discrete-time Wiener process, developing a scheduling policy that balances AoI reduction with monitoring accuracy. Furthermore, \cite{erbayat2024} analyzed the impact of optimal correlation probabilities under varying environmental conditions, addressing the interplay between error minimization and AoI.

%\ali{On the other hand, Preemption in AoI systems has been widely studied...Also, Id say reduce the size of this paragraph} 



%\ali{I don't like this transition here. Talk about correlated systems in the previous paragraph and how AoI is of interest. Then, switch here to preemption is still open question. Do not focus on your paper as you did here}
As part of ongoing efforts in this area, the potential of leveraging interdependencies between sensors to reduce the AoI in correlated systems has been studied, but the benefits and challenges of employing preemption in multi-sensor systems with correlated data streams remain an open question. While preemption is a potential strategy to minimize AoI in a network, it is not always the optimal strategy \cite{yates2019}. This approach must account for the specific features of the packets being transmitted since preempting leads to prioritization. For example, a sensor with a lower arrival rate may track a unique process that no other sensor monitors, making its packets particularly valuable and critical to retain. On the other hand, preempting a packet from a sensor with a high arrival rate may not significantly reduce AoI, as the frequent updates from such sensors diminish the impact of losing a single packet.


%\ali{Here you make the connection between preemption and multi-sensor correlated systems}

%\ali{Its good to emphasize that we have correlation here so it is different than typical AoI system}.

To address this gap, this paper introduces adaptable and probabilistic preemption mechanisms that dynamically balance priorities across sensors, considering their unique correlation characteristics and resource demands. To that end, the main contributions of this paper are summarized as follows:

%To address these challenges, we propose a system where the ability of a packet to preempt an ongoing transmission depends on its source, allowing for a more adaptable approach to managing updates. We also introduce the concept of probabilistic preemption, where preemption decisions are guided by source-specific probabilities rather than fixed or deterministic rules. This probabilistic method improves efficiency by giving higher-priority updates a better chance to preempt, keeping the information more up-to-date. By incorporating stochastic hybrid system modeling, we derive a closed-form expression for the AoI, providing a theoretical foundation to analyze the impact of probabilistic preemption on network performance. Building on this system, we explore how varying preemption probabilities can influence the total AoI in multi-sensor systems, considering the interplay between diverse sensors and their shared resources. Furthermore, we establish that the problem of deciding optimal preemption strategies can be framed as a sum of linear ratios problem. We derive an upper bound on the number of iterations required using a branch-and-bound algorithm, ensuring computational efficiency in identifying optimal solutions. Through this analysis, we identify optimal preemption strategies that minimize the total AoI, balancing the timeliness and relevance of updates across all monitored processes to achieve an efficient and well-coordinated system.

%Interestingly, the results show how the system adjusts priorities between sensors to keep the AoI as low as possible. For example, if one sensor spreads its updates more evenly across multiple processes, the system tends to rely on it more, even if another sensor is sending updates less often. As arrival rates or service rates change, the system shifts its strategy to stay efficient.\footnote{Due to size limitations, we present the proof details in \url{https://github.com/erbayat/xxxx}}.


\begin{itemize}
    \item As a first step, we propose a system where the ability of a packet to preempt an ongoing transmission probabilistically depends on its source rather than being fixed or following deterministic rules. Subsequently, using stochastic hybrid system modeling, we derive a closed-form expression for AoI to analyze the impact of probabilistic preemption on network performance.
    
    %enabling a more adaptable approach to manage updates by giving higher-priority updates a better chance to preempt, ensuring information remains up-to-date.

    \item Following that, we investigate optimizing the total AoI in multi-sensor systems, considering the interplay between diverse sensors and shared resources. Building on this, we frame the problem of deciding optimal preemption strategies as a sum of linear ratios problem, which is generally an NP-Hard problem\cite{freund2001solving}. However, by analyzing its unique characteristics, we derive an upper bound on the number of iterations required to identify optimal preemption strategies using a branch-and-bound algorithm, thus ensuring computational efficiency in finding the optimal solution.
    %\ali{You are using a lot the , ensuring... it sounds very chatgpt liky, try to minimize those when possible. Also, talk about the bounds and the impact of these results on getting an efficient solution}
    \item Lastly, we validate our findings with numerical results and evaluate optimal preemption strategies to minimize AoI. Our findings demonstrate how correlation influences preemption strategies. Notably, when a source provides a lower aggregate number of updates while distributing them more evenly, the system prioritizes it for preemption, even if another sensor updates less frequently.\ifthenelse{\boolean{withappendix}}
{}
{\footnote{Due to space limitations, we present the proof details in \cite{technicalNote}.}}
 %\ali{Dont forget to put the right link}
\end{itemize}


%These results not only support the theory but also offer practical ideas for real-world use, such as in IoT networks, factories, or autonomous systems, where staying up-to-date is very important.

%The remainder of this paper is structured as follows. Section \ref{system-model} introduces the system model and key assumptions. In Section \ref{aoi-S}, we derive the closed-form expression for the AoI within the proposed system. Section \ref{aoi-opt} outlines the optimization problem and details the process of determining the optimal preemption probabilities. The numerical results are presented in Section \ref{numerical}, and the paper concludes with a summary and discussion in Section \ref{conc}.




\section{Literature Review}
\label{sec:litriv}

\subsection{Classic ML approachs}
When studying the progressive advancements in this field, the classical Machine Learning (ML) approach in financial analytics primarily revolves around statistical models that have formed the bedrock of quantitative finance. Linear regression, one of the most fundamental techniques, has been extensively utilized for predicting financial trends and stock prices. Its effectiveness in financial forecasting is documented in "Analysis of Financial Time Series" by Tsay \cite{Tsay2010}, offering a comprehensive understanding of linear models in finance. Moreover, decision trees have been widely employed for risk assessment and credit scoring, as demonstrated in the study by Kumar and Ravi \cite{Kumar2007}, showcasing their ability to handle categorical and continuous input variables effectively. However, despite their widespread application, these classical models often struggle with financial data's non-linearity and high dimensionality characteristic. This limitation, as highlighted in the survey by Atsalakis and Valavanis \cite{Atsalakis2009}, clearly indicates the need for more advanced approaches in capturing the complex dynamics of financial markets, especially in volatile or unpredictable scenarios.

\subsection{Neural Networks in DRL}
Integrating Neural Networks and Deep Reinforcement Learning (DRL) into financial market analysis represents a significant leap forward from traditional ML methods. As outlined in the groundbreaking work by Mnih et al. \cite{Mnih2015}, DRL combines the depth and complexity of deep neural networks with the decision-making prowess of reinforcement learning, creating a powerful tool for financial analysis. This approach, which allows for direct learning from vast amounts of unstructured market data, effectively identifying intricate patterns and trends, is a game-changer in the field. Convolutional Neural Networks (CNNs) application within DRL, in particular, has further advanced the field. CNNs, renowned for their ability to process high-dimensional sequential data, are highly effective in capturing temporal and spatial dependencies in financial time series. This is exemplified in the research by Tsantekidis et al. \cite{Tsantekidis2017}., which utilized CNNs to analyze and predict stock prices from limited order book data, demonstrating the model's proficiency in handling complex financial datasets. The success of DRL in financial applications lies in its ability to continually adapt and learn in an ever-changing environment, a crucial feazture given the dynamic nature of financial markets. 

Despite these advancements, there remains a gap in understanding how the temporal scope of input data affects CNN performance in financial DRL models. Our study addresses this gap by systematically exploring various temporal windows and feature arrangements.


Here is your hypothesis section with the references added:

```latex
\section{Hypothesis}

Convolution operations are fundamental to Convolutional Neural Networks (CNNs), which are particularly effective in processing data with a grid-like topology, such as images and sequential data \cite{courbariaux2016binarized} \cite{dai2021coatnet}. The convolution operation can be understood as a mathematical process that combines two sets of information. In the context of CNNs, this involves a convolutional kernel (or filter) moving across an input signal (such as an image or time series data) to produce a feature map.

Mathematically, for continuous signals, the convolution operation is defined as:

\[
(S * K)(t) = \int_{-\infty}^{\infty} S(\tau)K(t - \tau) d\tau
\]

Here, \(S\) represents the input signal, and \(K\) represents the convolutional kernel. This integral computes the area under the product of the two functions as the kernel slides over the input signal. However, in practical applications involving digital data, the signals are discrete, and thus the convolution operation is adapted to:

\[
(S * K)[n] = \sum_{m=-M}^{M} S[m]K[n - m]
\]

In this discrete form, the convolution operation involves summing the element-wise products of the input signal and the kernel as it moves across the input. The result is a new set of values (the feature map) that highlight certain features of the input signal, such as edges in an image or patterns in sequential data \cite{chen2019compressing}.

The size of the convolutional kernel (or filter) is a critical parameter in this operation. The kernel size determines the local region from which features are extracted. A larger kernel can capture more contextual information by encompassing a wider region of the input signal, while a smaller kernel focuses on finer details. The balance between capturing local and global features is essential for the performance of CNNs \cite{dai2016rfcn}.

Additionally, the padding applied to the input signal before convolution affects the output size and the nature of the features extracted. Padding involves adding extra values (typically zeros) around the input signal, which allows the kernel to process edge regions more effectively. The output size of the convolution operation is given by:

\[
O = \frac{N - K + 2P}{S} + 1
\]

where \(N\) is the input size, \(K\) is the kernel size, \(P\) is the padding, \(S\) is the stride (the step size of the kernel), and \(O\) is the output size. Properly setting these parameters ensures that the CNN can effectively learn and extract meaningful features from the input data \cite{chollet2017xception}. Understanding these concepts is crucial for optimizing CNN architectures, especially in settings where the observation window size can significantly impact the model's performance.

The performance of Convolutional Neural Networks (CNNs) in processing sequential data is significantly influenced by the size of the observation window used in the convolutional layers. The kernel size in a convolution layer determines the local region from which features are extracted. Larger kernels can incorporate more contextual information, but excessively large kernels may dilute distinct features. The optimization of window size can be expressed through the effective window size equation:

\[
W_{\text{eff}} = W_{\text{kernel}} + (W_{\text{kernel}} - 1) \times (D - 1)
\]

where \(W_{\text{eff}}\) is the effective window size, \(W_{\text{kernel}}\) is the kernel size, and \(D\) is the dilation factor.

Furthermore, the role of padding in convolution processes influences the spatial dimensions of the output feature map, described by:

\[
O = \frac{N - K + 2P}{S} + 1
\]

where \(N\) is the input size, \(K\) is the kernel size, \(P\) is the padding, \(S\) is the stride, and \(O\) is the output size. Excessive padding can lead to overemphasis on peripheral data and potential overfitting, similar to how an over-expanded window size may cause information overload, making distinct features less discernible:

\[
\text{Information Overload} \propto \frac{W_{\text{eff}}}{\text{Distinct Features}}
\]

Therefore, a crucial balance is needed between capturing local and global features. We hypothesize that the optimal selection of a temporal window size in a CNN balances local feature detection and global contextual understanding. An optimally sized window allows the model to effectively capture relevant features without succumbing to information overload or excessive generalization, thereby enhancing accuracy and performance in sequential data processing tasks \cite{chen2019compressing}.

Given that our CNN acts as a policy for a Deep Reinforcement Learning (DRL) algorithm, the window size as a hyperparameter will be optimized through reinforcement learning. This optimal window size is found at the point where local and global feature detection are balanced:

\[
\text{Optimal Window Size} \leftrightarrow \min \left( \Delta_{\text{Local-Global}} \right)
\]

where \( \Delta_{\text{Local-Global}} \) measures the differential in information capture between local and global features. This hypothesis suggests that through careful tuning and reinforcement learning, the CNN can achieve an optimal window size that maximizes performance in sequential data tasks.


\section{\method}
\label{sec:tokenskip}
\begin{figure*}[t]
\centering
\includegraphics[width=0.95\textwidth]{fig/tokenskip.pdf}
\caption{Illustration of \method. During the training phase, \method first generates CoT trajectories from the target LLM. These CoTs are then compressed to a specified ratio, $\gamma$, based on the semantic importance of tokens. \method fine-tunes the target LLM using compressed CoTs, enabling controllable CoT inference at the desired $\gamma$.}
\label{fig:tokenskip}
\end{figure*}

We introduce \method, a simple yet effective approach that enables LLMs to skip less important tokens, enabling controllable CoT compression with adjustable ratios. This section demonstrates the details of our methodology, including token pruning~(\S\ref{sec:token-pruning}), training~(\S\ref{sec:training}), and inference~(\S\ref{sec:inference}).

\subsection{Token Pruning}
\label{sec:token-pruning}
The key insight behind \method is that ``\textit{each reasoning token contributes differently to deriving the answer.}'' To enhance CoT efficiency, we propose to trim redundant tokens from LLM CoT outputs and fine-tune LLMs using these trimmed CoT trajectories. The token pruning process is guided by the concept of \textit{token importance}, as detailed in Section~\ref{sec:token-importance}. 

Specifically, given a target LLM $\M$, one of its CoT trajectories $\boldsymbol{c}=\left\{c_i\right\}_{i=1}^{m}$, and a desired compression ratio $\gamma \in \left[0,1\right]$, \method first calculates the semantic importance of each CoT token $I\left(c\right)$, as defined in Eq~(\ref{eq:llmlingua2}). The tokens are then ranked in descending order based on their importance values. Next, the $\gamma$-th percentile of these importance values is computed, representing the threshold for token pruning:
\begin{equation}
I_\gamma=\mathrm{np.percentile}\left(\left[I\left(c_1\right), . ., I\left(c_m\right)\right], \gamma\right).
\end{equation}
Finally, CoT tokens with an importance value greater than or equal to $I_\gamma$ are retained in the compressed CoT trajectory:
\begin{equation}
\widetilde{\boldsymbol{c}}=\left\{c_i \mid I\left(c_i\right) \geq I_\gamma\right\}, 1 \leq i \leq m.
\end{equation}

\subsection{Training}
\label{sec:training}
Given a training dataset $\mathcal{D}$ with $N$ samples and a target LLM $\M$, we first obtain $N$ CoT trajectories with $\M$. Then, we filter out trajectories with incorrect answers to ensure the high quality of training data. For the remaining CoT trajectories, we prune each CoT with a randomly selected compression ratio $\gamma$, as demonstrated in Section~\ref{sec:token-pruning}. For each $\langle\text{question}, \text{compressed CoT}, \text{answer}\rangle$, we inserted the compression ratio $\gamma$ after the question. Finally, each training sample is formatted as follows: 
\begin{equation}
\nonumber
    \mathcal{Q} \ \mathrm{[EOS]} \ \gamma \ \mathrm{[EOS]} \ \mathrm{Compressed\ CoT} \ \mathcal{A},
\end{equation}
where $\langle\mathcal{Q}, \mathcal{A}\rangle$ indicates the $\langle\text{question}, \text{answer}\rangle$ pair. Formally, given a question $\boldsymbol{x}$, compression ratio $\gamma$, and the output sequence $\boldsymbol{y}=\left\{y_i\right\}_{i=1}^{l}$, which includes the compressed CoT $\widetilde{\boldsymbol{c}}$ and the answer $\boldsymbol{a}$, we fine-tunes the target LLM $\M$, enabling it to perform chain-of-thought in a compressed pattern by minimizing
\begin{equation}
\mathcal{L}=\sum_{i=1}^{l} \log P\left(y_{i} \mid \bm{x}, \gamma, \bm{y}_{<i}; \bm{\theta}_{\M}\right),
\end{equation}
where $\bm{y} =\left\{\widetilde{c}_1, \cdots,\widetilde{c}_{m^{\prime}}, a_1, \cdots, a_t  \right\}$. Note that the compression is performed solely on CoT sequences, and we keep the answer $\boldsymbol{a}=\left\{a_i\right\}_{i=1}^{t}$ unchanged. To preserve LLMs' reasoning capabilities, we also include a portion of the original CoT trajectories in the training data, with $\gamma$ set to 1.

\subsection{Inference}
\label{sec:inference}
The inference of \method follows autoregressive decoding. Compared to original CoT outputs that may contain redundancy, \method facilitates LLMs to skip \textit{unimportant} tokens during the chain-of-thought process, thereby enhancing reasoning efficiency. Formally, given a question $\boldsymbol{x}$ and the compression ratio $\gamma$, the input prompt of \method follows the same format adopted in fine-tuning, which is $\mathcal{Q} \ \mathrm{[EOS]} \ \gamma \ \mathrm{[EOS]}$. The LLM $\M$ sequentially predicts the output sequence $\hat{\bm{y}}$:
\begin{equation}
\nonumber
\hat{\boldsymbol{y}}=\arg \max _{\boldsymbol{y}^*} \sum_{j=1}^{l^{\prime}} \log P\left(y_j \mid \boldsymbol{x}, \gamma, \boldsymbol{y}_{<j}; \bm{\theta}_{\M}\right),
\end{equation}
where $\hat{\bm{y}} =\left\{\hat{c}_1, \cdots,\hat{c}_{m^{\prime\prime}}, \hat{a}_1, \cdots, \hat{a}_{t^{\prime}}  \right\}$ denotes the output sequence, which includes CoT tokens $\hat{\bm{c}}$ and the answer $\bm{\hat{a}}$. We illustrate the training and inference process of \method in Figure~\ref{fig:tokenskip}. 




% \begin{table*}[!th]
%     \centering
%     \resizebox{0.9\textwidth}{!}{
%     \begin{tabular}{|l|r|r|r|l|r|r|r|} 
%        \cline{1-4} \cline{6-8}
%         & \multicolumn{3}{c|}{Retrieve} & &\multicolumn{3}{c|}{Retrieve + Rerank} \\ 
%                \cline{2-4} \cline{6-8}
%                &  \multicolumn{3}{c|}{Hits@10}   & & \multicolumn{2}{c|}{Hits@10}  & Speedup \\
% Dataset  &  Baseline  &  DE-2 Rand  & DE-2 CE  & & DE-2 CE &  Baseline &  \\
%        \cline{1-4} \cline{6-8}
% all nli \cite{bowman-etal-2015-large},\cite{N18-1101}  & 0.77 & 0.59 & 0.75 &  &\textbf{ 0.84} & 0.83 & 4.5x\\
% eli5 \cite{fan-etal-2019-eli5} & 0.43 & 0.12 & 0.29 &  & 0.43 & 0.49 & 5.4x \\
% gooaq \cite{Khashabi2021GooAQOQ} & 0.75 & 0.42 & 0.64 &  & 0.77 & 0.80 & 5.8x\\
% msmarco \cite{nguyen2016ms} & 0.95 & 0.81 & 0.89 &  & 0.96 & 0.98 & 5.1x \\
% \href{https://quoradata.quora.com/First-Quora-Dataset-Release-Question-Pairs}{quora duplicates}  & 0.68 & 0.48 & 0.65 &  &\textbf{ 0.68} & 0.68 & 5.1x\\
% natural ques. \cite{47761} & 0.77 & 0.37 & 0.61 &  & 0.61 & 0.64 & 5.4x \\
% sentence comp \cite{filippova-altun-2013-overcoming} & 0.95 & 0.83 & 0.93 &  & \textbf{0.97 }& 0.97 & 6.6x\\
% simplewiki \cite{coster-kauchak-2011-simple} & 0.97 & 0.93 & 0.97 &  &\textbf{ 0.97 }& 0.96 & 4.8x\\
% stsb \cite{cer-etal-2017-semeval} & 0.97 & 0.87 & 0.97 &  & \textbf{0.98} & 0.98 & 4.7x \\
% zeshel \cite{logeswaran2019zero}  & 0.22 & 0.16 & 0.20 &  & \textbf{0.21} & 0.19 & 4.8x\\
%        \cline{1-4} \cline{6-8}
%     \end{tabular}
%     }

% Please add the following required packages to your document preamble:
% \usepackage[table,xcdraw]{xcolor}
% Beamer presentation requires \usepackage{colortbl} instead of \usepackage[table,xcdraw]{xcolor}
\begin{table*}[!th]
\centering
     \resizebox{1\textwidth}{!}{ \begin{tabular}{|l|llllll|l|lllll|}
\cline{1-7} \cline{9-13}
\textbf{}                  & \multicolumn{6}{c|}{\textbf{Retrieve}}                                                                                                                                                                                         & \textbf{} & \multicolumn{5}{c|}{\textbf{Retrieve + Rerank}}                                                                                                                                    \\ 
\cline{2-7} \cline{9-13}
{\textbf{dataset}}           & \multicolumn{3}{c|}{\textbf{Hits@10}}                                                                                   & \multicolumn{3}{c|}{\textbf{MRR@10}}                                                                & \textbf{} & \multicolumn{2}{c|}{\textbf{Hits@10}}                                         & \multicolumn{2}{c|}{\textbf{MRR@10}}                                           & \textbf{Speedup} \\ 

                           
\cline{2-7} \cline{9-13} 
                           & \multicolumn{1}{l|}{\textbf{Baseline}} & \multicolumn{1}{l|}{\textbf{DE-2-Rand}} & \multicolumn{1}{l|}{\textbf{DE-2-CE}} & \multicolumn{1}{l|}{\textbf{Baseline}} & \multicolumn{1}{l|}{\textbf{DE-2-Rand}} & \textbf{DE-2-CE} & \textbf{} & \multicolumn{1}{l|}{\textbf{Baseline}} & \multicolumn{1}{l|}{\textbf{DE-2-CE}} & \multicolumn{1}{l|}{\textbf{Baseline}} & \multicolumn{1}{l|}{\textbf{DE-2-CE}} & \textbf{}        \\ 
                       \cline{1-7} \cline{9-13}
                       
msmarco/dev/small  & \multicolumn{1}{l|}{0.59}              & \multicolumn{1}{l|}{0.37}               & \multicolumn{1}{l|}{0.45}             & \multicolumn{1}{l|}{0.32}              & \multicolumn{1}{l|}{0.19}               & 0.24             &           & \multicolumn{1}{l|}{0.66}              & \multicolumn{1}{l|}{0.59}             & \multicolumn{1}{l|}{0.38}              & \multicolumn{1}{l|}{0.35}             & 5.1x             \\ \cline{1-7} \cline{9-13}
beir/quora/dev             & \multicolumn{1}{l|}{0.95}              & \multicolumn{1}{l|}{0.90}               & \multicolumn{1}{l|}{0.93}             & \multicolumn{1}{l|}{0.85}              & \multicolumn{1}{l|}{0.76}               & 0.81             &           & \multicolumn{1}{l|}{\textbf{0.96}}     & \multicolumn{1}{l|}{\textbf{0.96}}    & \multicolumn{1}{l|}{\textbf{0.74}}     & \multicolumn{1}{l|}{\textbf{0.74}}    & 5.3x             \\ \cline{1-7} \cline{9-13}
beir/scifact/test          & \multicolumn{1}{l|}{0.63}              & \multicolumn{1}{l|}{0.53}               & \multicolumn{1}{l|}{0.55}             & \multicolumn{1}{l|}{0.42}              & \multicolumn{1}{l|}{0.30}               & 0.36             &           & \multicolumn{1}{l|}{0.72}              & \multicolumn{1}{l|}{0.69}             & \multicolumn{1}{l|}{0.57}              & \multicolumn{1}{l|}{0.55}             & 4.7x             \\ \cline{1-7} \cline{9-13}
beir/fiqa/dev              & \multicolumn{1}{l|}{0.50}              & \multicolumn{1}{l|}{0.22}               & \multicolumn{1}{l|}{0.32}             & \multicolumn{1}{l|}{0.32}              & \multicolumn{1}{l|}{0.12}               & 0.18             &           & \multicolumn{1}{l|}{0.59}              & \multicolumn{1}{l|}{0.46}             & \multicolumn{1}{l|}{0.28}              & \multicolumn{1}{l|}{0.23}             & 5.3x             \\ \cline{1-7} \cline{9-13}
zeshel/test                & \multicolumn{1}{l|}{0.22}              & \multicolumn{1}{l|}{0.16}               & \multicolumn{1}{l|}{0.20}             & \multicolumn{1}{l|}{0.12}              & \multicolumn{1}{l|}{0.09}               & 0.11             &           & \multicolumn{1}{l|}{\textbf{0.20}}     & \multicolumn{1}{l|}{\textbf{0.19}}    & \multicolumn{1}{l|}{\textbf{0.11}}     & \multicolumn{1}{l|}{\textbf{0.10}}    & 4.8x             \\ \cline{1-7} \cline{9-13}
stsb/train                 & \multicolumn{1}{l|}{0.97}              & \multicolumn{1}{l|}{0.95}               & \multicolumn{1}{l|}{0.97}             & \multicolumn{1}{l|}{0.85}              & \multicolumn{1}{l|}{0.83}               & 0.85             &           & \multicolumn{1}{l|}{\textbf{0.98}}     & \multicolumn{1}{l|}{\textbf{0.98}}    & \multicolumn{1}{l|}{\textbf{0.88}}     & \multicolumn{1}{l|}{\textbf{0.88}}    & 4.7x             \\ \cline{1-7} \cline{9-13}
all-nli/train              & \multicolumn{1}{l|}{0.49}              & \multicolumn{1}{l|}{0.41}               & \multicolumn{1}{l|}{0.47}             & \multicolumn{1}{l|}{0.39}              & \multicolumn{1}{l|}{0.33}               & 0.36             &           & \multicolumn{1}{l|}{\textbf{0.55}}     & \multicolumn{1}{l|}{\textbf{0.54}}    & \multicolumn{1}{l|}{\textbf{0.47}}     & \multicolumn{1}{l|}{\textbf{0.46}}    & 4.5x             \\ \cline{1-7} \cline{9-13}
simplewiki/train           & \multicolumn{1}{l|}{0.97}              & \multicolumn{1}{l|}{0.98}               & \multicolumn{1}{l|}{0.98}             & \multicolumn{1}{l|}{0.92}              & \multicolumn{1}{l|}{0.93}               & 0.94             &           & \multicolumn{1}{l|}{\textbf{0.96}}     & \multicolumn{1}{l|}{\textbf{0.97}}    & \multicolumn{1}{l|}{\textbf{0.91}}     & \multicolumn{1}{l|}{\textbf{0.91}}    & 4.8x             \\ \cline{1-7} \cline{9-13}
natural-questions/train    & \multicolumn{1}{l|}{0.77}              & \multicolumn{1}{l|}{0.59}               & \multicolumn{1}{l|}{0.65}             & \multicolumn{1}{l|}{0.51}              & \multicolumn{1}{l|}{0.37}               & 0.42             &           & \multicolumn{1}{l|}{0.65}              & \multicolumn{1}{l|}{0.61}             & \multicolumn{1}{l|}{0.39}              & \multicolumn{1}{l|}{0.37}             & 5.4x             \\ \cline{1-7} \cline{9-13}
eli5/train                 & \multicolumn{1}{l|}{0.32}              & \multicolumn{1}{l|}{0.14}               & \multicolumn{1}{l|}{0.22}             & \multicolumn{1}{l|}{0.19}              & \multicolumn{1}{l|}{0.08}               & 0.12             &           & \multicolumn{1}{l|}{0.39}              & \multicolumn{1}{l|}{0.32}             & \multicolumn{1}{l|}{0.26}              & \multicolumn{1}{l|}{0.22}             & 5.4x             \\ \cline{1-7} \cline{9-13}
sentence compression/train & \multicolumn{1}{l|}{0.93}              & \multicolumn{1}{l|}{0.89}               & \multicolumn{1}{l|}{0.93}             & \multicolumn{1}{l|}{0.85}              & \multicolumn{1}{l|}{0.79}               & 0.84             &           & \multicolumn{1}{l|}{\textbf{0.96}}     & \multicolumn{1}{l|}{\textbf{0.96}}    & \multicolumn{1}{l|}{\textbf{0.93}}     & \multicolumn{1}{l|}{\textbf{0.94}}    & 6.6x             \\ \cline{1-7} \cline{9-13}
gooaq/train                & \multicolumn{1}{l|}{0.73}              & \multicolumn{1}{l|}{0.68}               & \multicolumn{1}{l|}{0.72}             & \multicolumn{1}{l|}{0.58}              & \multicolumn{1}{l|}{0.55}               & 0.58             &           & \multicolumn{1}{l|}{0.80}              & \multicolumn{1}{l|}{0.76}             & \multicolumn{1}{l|}{0.64}              & \multicolumn{1}{l|}{0.62}             & 5.2x             \\ \cline{1-7} \cline{9-13}
\end{tabular}
}
    \caption{Comparison of CE infused DE. \textbf{Baseline} is the pre-trained DE SBERT model.  \textbf{DE-2 Rand} is a trained two-layered DE initialized with random initial weights. \textbf{DE-2 CE} is a two-layered DE infused with initial weights from CE as explained in Fig~\ref{fig:dual_cross} and trained similar to DE-2. All the above models are trained on msmarco. Bold face numbers for DE-2 CE show where performance is at least within .01 of the baseline DE. \textbf{Speedup} is inference time gain for DE-2 CE over Baseline.}
    % \textbf{Retrieve + Rerank}: Columns 5-6 present Accuracy@10 and columns 7-8 present number of documents encoded per sec. Here, the documents retrieved using baseline and our approach are reranked using a CE.}
    \label{tab:comp}
\end{table*} 
\FloatBarrier
\section{Results}
\subsection{Experimentation on the Technical Indicator Dataset}
The analysis of the Technical Indicator dataset, without any feature rearrangement, as illustrated in the figure below, uncovers a notable pattern in the accumulation of rewards over different time intervals. The most significant gain, observed in the 2-week observation size, reached a cumulative reward of 155.89. This finding highlights the efficacy of this specific observation window. The peak performance noted within this 2-week timeframe may constitute the most advantageous period for analysis in the context of this dataset and its feature composition. This observation window provides the optimal balance mentioned in our hypothesis section, generating the most significant rewards in the given feature arrangement setting and dataset.

\begin{figure}[ht]
\centering
\includegraphics[width=\linewidth]{old_data_no_shuffle.pdf}
\caption{Cumulative rewards in the Technical Indicator dataset without rearrangement }
\label{fig:tech_indicator_not_rearranged}
\end{figure}

The extended analysis of the Technical Indicator dataset over periods ranging from 4 to 12 weeks reveals a discernible decline in cumulative rewards, reaching its lowest point at the 10-week interval, where the reward significantly drops to 104.58. This downward trajectory, although slightly mitigated in the 12-week observation window, predominantly suggests diminishing returns as the duration of the observation period increases. This pattern serves as a crucial insight, highlighting the limitations of the convolutional neural network (CNN) in effectively utilizing longer observation windows for this specific dataset and feature configuration. This trend underscores the importance of strategically selecting the observation window to optimize the CNN's predictive performance, and it supports our hypothesis that information overload can diminish the CNN's ability to utilize most important features in the input tensor.

During the analysis of the Technical Indicator dataset with rearranged features, as depicted in the figure below, we found a markedly different trend in cumulative rewards across varying timeframes compared to the dataset with the original feature arrangement. The rearranged dataset demonstrates a similar pattern, where the peak cumulative reward is noted at the 10-week mark, registering at 121.59. This outcome indicates that the rearrangement of features shifts the optimal observation window to bigger sizes. Notably, a prolonged 10-week period emerges as most favorable in the rearranged dataset, in stark contrast to the 2-week window size identified as optimal in the original dataset configuration. This finding suggests that feature rearrangement significantly improves the model's ability to utilize longer observation windows, again underscoring the need for adaptable strategies in financial data analysis with CNNs.

\begin{figure}[ht]
    \centering
    \includegraphics[width=\linewidth]{old_data_shuffled.pdf}
    \caption{Cumulative rewards in the Technical Indicator dataset with rearrangement }
    \label{fig:tech_indicator_rearranged}
\end{figure}

As depicted in the figure, rearranging features within the technical indicator dataset markedly improves the model's capacity to capitalize on extended observation windows. Notably, the model's optimal performance, demonstrated at the 10-week interval with a cumulative reward of 121.59, signifies an enhanced ability to utilize more extended periods for analysis. This reorganization of features enables a more efficient interpretation of extended-term trends, optimizing the model's accuracy over such durations. This finding emphasizes the vital importance of feature engineering in amplifying the effectiveness of Convolutional Neural Networks, particularly in intricate and dynamic settings like financial market analysis.

In contrast, a different pattern emerges when analyzing the technical indicator dataset without feature rearrangement, as illustrated in the corresponding plot. Here, the 2-week interval emerges as the most productive timeframe, registering the highest cumulative reward of 155.89. This finding indicates that in its original configuration, the dataset is optimally tuned for short-term analysis, showing diminishing performance with lengthening observation periods, except for a slight increase at 12 weeks. However, these extended periods do not outperform the initial 2-week observation window. This trend highlights the model's predisposition towards shorter timeframes when processing the non-rearranged data, underscoring the impact of data structuring on the model's temporal adaptability and predictive power.

\begin{figure}[ht]
    \centering
    \includegraphics[width=\linewidth]{old_data_best.pdf}
    \caption{Best performers in the Technical Indicator Dataset}
    \label{fig:sma_nonrearranged}
\end{figure}

The contrasting results observed in the rearranged technical indicator data are striking. In this scenario, the model strides in the 10-week observation period, achieving a cumulative reward of 121.59. This shift from the optimal 2-week period in the non-rearranged data to a more extended 10-week period in the rearranged data is significant. The rearranging of features profoundly influences the model's efficiency in capturing and forecasting market trends. Compared to the reduced effectiveness in shorter durations, the enhanced performance at this longer interval underscores the impact of data sequencing on the model's predictive precision. This observation again stresses the criticality of data arrangement and preprocessing in financial time series analysis, as it can substantially alter the model's interpretation and response to market dynamics over different temporal scales. 

\subsection{Experimentation on the SMA dataset}
The analysis of the SMA dataset without data rearrangement reveals a distinct pattern in cumulative rewards over various timeframes, as shown in Figure \ref{fig:sma_rearranged}. The most significant performance is apparent in the 2-week observation window, achieving a peak cumulative reward of 184.05. This high point suggests that a 2-week observation window is particularly effective for this dataset, indicating an optimal short-term period for analysis in this context.

\begin{figure}[ht]
    \centering
    \includegraphics[width=\linewidth]{new_data_no_shuffle.pdf}
    \caption{Cumulative rewards in the SMA dataset without rearrangement}
    \label{fig:sma_rearranged}
\end{figure}

As the observation window extends, a decreasing trend in cumulative rewards is evident, particularly at 8 and 12 weeks, with rewards noted at 99.80 and 105.99, respectively. However, an unexpected increase in cumulative reward to 144.22 at the 10-week mark presents an intriguing anomaly. This inconsistency might indicate complex, possibly cyclical patterns in the SMA dataset, which the model discerns differently across various intervals. This behavior further highlights the intricate nature of these quantitative indicators and emphasizes the importance of selecting an appropriate observation window for predictive modeling.

A different outcome is observed in the analysis of the SMA dataset with feature rearrangement. The 4-week interval emerges as the most favorable, registering a peak cumulative reward of 181.84. This result contrasts the lower performance in the 2-week window, where the cumulative reward is 117.14. This discrepancy suggests that rearranging the data may significantly alter the model's ability to utilize temporal relationships in the data, affecting its effectiveness across different timeframes. The rearranged dataset's peak at a longer interval underlines the same pattern where feature arrangement enhances the model's ability to effectively capture and analyze market trends.

\begin{figure}[ht]
    \centering
    \includegraphics[width=\linewidth]{new_data_shuffled.pdf}
    \caption{Cumulative rewards in the SMA dataset with rearrangement }
    \label{fig:new_data_no_suhffle}
\end{figure}

However, an irregular trend emerges as the observation period extends beyond 4 weeks. A marked decrease in cumulative rewards is noted at 6 and 8 weeks, with figures falling to 101.04 and 90.77, respectively. Intriguingly, there is a modest reward recovery at the 10 and 12-week intervals. This pattern suggests that the model may interpret different characteristics of the rearranged SMA dataset over extended timeframes. Such fluctuations in performance underscore the added complexity due to data rearrangement and the importance of carefully choosing the observation window to maximize the model's efficacy. 

\subsection{Best Performers in the SMA Dataset}
In the next phase of our data analysis, we conducted a comparative study of optimal timeframes in the simple moving average (SMA) dataset, considering its original and rearranged forms, as shown in the plot. This revealed distinctive trends. 

In the case of the non-rearranged SMA dataset, the most effective timeframe emerges as the 2-week window, registering a peak cumulative reward of 184.05. This notable performance at the shorter interval indicates the model's ability to effectively capture the prevailing trends within the original SMA data structure. As the observation period extends, a gradual decline in cumulative rewards is observed across longer timeframes. Although there is a marginal uplift in performance at the 10-week mark, this is within the benchmark set by the 2-week observation window, which means that the pattern still highlights the dataset's responsiveness to short-term fluctuations.

The clear differentiation in performance across various timeframes suggests that the underlying dynamics of the SMA dataset are more readily discernible and exploitable in shorter intervals when the data remains in its original sequence. This insight is pivotal for financial analysts and modelers, emphasizing the need for strategic consideration of time windows in predictive modeling, especially when dealing with complex financial datasets like the SMA.
\begin{figure}[ht]
    \centering
    \includegraphics[width=\linewidth]{new_data_best.pdf}
    \caption{Best performers in the SMA dataset }
    \label{fig:sma_data_best_performers}
\end{figure}

In contrast, after rearranging the features in the SMA dataset, our analysis presents a different optimal timeframe. The 4-week window emerges as the best performer with a cumulative reward of 181.84, indicating a significant shift in the model's ability to utilize longer temporal windows. Our analysis also shows a more pronounced decline in performance for other timeframes, especially at 6 and 8 weeks. We noted that the 2-week observation size was the best performer in the non-rearranged data versus the 4-week peak in the rearranged data. Once again, the sharp contrast between the non-rearranged and rearranged data demonstrates the model's temporal processing ability.

\subsection{Best Performers overall}
Several insightful trends emerge in our final analysis of the datasets, encompassing both the SMA and Technical Indicator datasets. In its original feature arrangement, the SMA dataset exhibits strong performance in the 2-week timeframe, reaching a cumulative reward of 184.057, the highest across all datasets and timeframes. This result underscores the effectiveness of short-term observation in capturing market dynamics with this dataset. On the other hand, when the SMA features are rearranged, the 4-week window becomes the most productive, achieving a cumulative reward of 181.84. This shift suggests that market dynamics are captured more effectively over shorter temporal windows, but once features are rearranged, a slightly extended observation size proved more effective.

\begin{figure}[ht]
    \centering
    \includegraphics[width=\linewidth]{best_over_all.pdf}
    \caption{Best performers overall. }
    \label{fig:best_performers_overall}
\end{figure}

The observed trends in the Technical Indicator dataset echo those seen in the SMA dataset, particularly in the context of the original sequence. A 2-week observation window demonstrates optimal effectiveness, reaching a peak cumulative reward of 155.89. This similarity across the datasets consistently proves our decerned pattern that, without shorter observation periods, can be highly effective for predictive modeling. However, a significant shift occurs when the data sequence in the Technical Indicator dataset is rearranged. This modification leads to the 10-week timeframe becoming the most favorable, as evidenced by a cumulative reward of 121.59.

This shift indicates that the Convolutional Neural Network (CNN) becomes more adept at discerning the complex patterns between features and their temporal dynamics when the data is organized to maintain a cohesive structure for each company's features. The rearrangement enhances the model's ability to grasp longer-term trends and relationships, which may be less apparent or accessible in shorter timeframes or with non-rearranged data. This observation is crucial as it suggests that the efficacy of a CNN in financial market analysis can be significantly influenced by how the data is structured. It highlights the importance of considering the arrangement of data to optimize the predictive capabilities of models, especially in financial contexts where the relationships between various indicators and their evolution over time are crucial to understanding market movements. Thus, a flexible and context-specific approach to selecting observation periods and organizing data is paramount to maximizing the utility and accuracy of predictive models in financial analysis.

We outlines some applications of NeurFlow.
We hypothesize that, as one neuron can have multiple meanings, a DNN looks at a group of neurons rather than individually to determine the exact features of the input. Hence, we propose a metric that assesses a model's confidence in determining whether the input contains a specific visual feature. For a group $G$ with core concept neurons $\sS_G = \{s_{G,1}, \dots, s_{G, |\sS_G|}\}$, the metric denoted as $ M(v, \sS_G, \mathcal{D}) = \exp(\frac{1}{|\sS_G|}\sum_{s \in \sS_G}\log(\|\phi_s(v)/\max(\phi_s, \mathcal{D}))\|)$, where $v \in \mathcal{D}$ and $\max(\phi_s, \mathcal{D})$ is the highest value of activation of neuron $s \in \sS_G$ on dataset $\mathcal{D}$.
This returns high score when all neurons in $G$ have high activation (indicating high confidence), while resulting in almost zero if any neuron in the group has low activation (indicating low confidence). We can use this metric to determine how similar the features in the input image are to the predetermined neuron groups concept. 
The specific setup can be found in the Appendix \ref{debugging_setup}. Figures \ref{fig:img debug} and \ref{fig:debias} demonstrate the usage of the metric and the concept circuit.
We use the term \emph{NGC} to denote the concept of a neuron group. 
\begin{figure}[t]
    \centering
    \begin{minipage}{.45\textwidth}
       \vspace{-12mm}
        \includegraphics[width=0.95\columnwidth]{figures/image_debug.png}
     %   \vspace{-3mm}
        \caption{\textbf{Using NeurFlow to reveal the reason behind model's prediction.} The top concepts can be traced throughout the circuit.}\label{fig:img debug}   
        \vspace{-10pt}
    \end{minipage}
    \hfill
    \begin{minipage}{.5\textwidth}
        \vspace{-12mm}
        \includegraphics[width=0.95\columnwidth]{figures/MLLM_captioning.pdf}
    %    \vspace{-5mm}
        \caption{Demonstration for automatically labelling and explaining the relation of NGCs on class ``great white shark" using GPT4-o \citep{Gpt4o}. The captions and the names of the NGCs are highlighted in blue, while the relations are in black.} \label{fig:captioning}
        \vspace{-10pt}
    \end{minipage}
    \vspace{-5pt}
\end{figure}
% Note that, a recent work \citep{VCC}, a concept-based method that also explains the inner mechanism of DNNs, has a similar application of debugging images. They measure how close the activations of misclassified images are to the concept vectors using $l_2$ norm. However, they provide no objective proof. In contrast, one advantage of learning via neurons is that we can edit the neurons related to a concept to see whether it has significant impact on the final predictions (see section \ref{debugging}).

\subsection{Image debugging}
\label{debugging}
We aim to use the concept circuit to identify concepts contributing to false prediction, which we call \textit{image debugging}. If a concept contributes to a class when it should not, we say that the prediction (or equivalently, the model) is \textit{biased} by that concept. \citet{Debias} propose a framework for detecting biases in a vision model by generating captions for the predicted images and tracking the common keywords found in the captions. With this method, they concluded that the pretrained ResNet50 is biased by ``flower pedals" in the class ``bee". However, correlational features do not imply causation and can lead to misjudgments. We verify and enhance the causality of their claim by examining the concept circuit of class ``bee", and conducting experiments on the probabilities of the final predictions with and without neurons that related to ``flowers". Additionally, we discover that the model also suffers from ``green background" bias (resemble ``leaves"), which is not mentioned in \citet{Debias}. 

Figure \ref{fig:debias} shows the process of debugging false positive images. Three different concepts are presented in \textit{layer4.2} of ResNet50, representing ``pink pedals", ``green background", and ``bee" respectively (we choose this layer as it has a small set of NGCs, however, our following experiment is consistent for multiple layers and with different classes). We discover that most of the false positive images have high metric score for ``pedal" and ``green background". 
To further verify the impact of these biased features, we mask all neurons in the groups of the respective concepts and find that the probability of the predictions are distorted drastically (and predictions is no longer ``bee"), as opposed to masking random neurons, which yield negligible changes. 

\begin{figure}
\begin{center}
\vspace{-12mm}
\includegraphics[width=0.9\textwidth]{figures/bias.pdf} 
\end{center}
\vspace{-3mm}
\caption{(left) The metric scores of false positive images for each concept in \textit{layer4.2} of ResNet50. (right) Showing the images that have the greatest drop in the activation of the logit neuron when masking each group concept. Verifying that the neuron groups indeed reflect the concepts.} \label{fig:debias}
\vspace{-15pt}
\end{figure}
This implies the dependence on the biased concept. \textit{But how do we know that the groups reflect the respective visual features?} If these groups indeed represent the visual features, then masking them should hinder the classification probability for images that include those features. We highlight the top images that have the largest decrease in the value of the logit neuron (corresponding to class ``bee") on both validation set of the target class and augmented dataset (see Section \ref{subsec:identify_node}). As shown in Figure \ref{fig:debias}, this process indeed yields the images that contain the respective features.

To demonstrate how NeurFlow's findings differ from those of existing methods, we conduct a qualitative experiment comparing the core concept neurons identified by NeurFlow with those identified by NeuCEPT \citet{NEUCEPT}. Detailed information about this experiment is provided in Appendix \ref{sec:compare_neucept_img_debugging}. Our observations indicate that 
% NeurFlow identifies concepts more closely resembling the original images. Additionally, 
the top logit drop images identified by NeurFlow align better with the representative examples of the corresponding concepts. Moreover, masking the core concept neuron groups identified by NeurFlow resulted in more significant changes to prediction probabilities while utilizing fewer neurons compared to the groups identified by NeuCEPT.


\subsection{Automatic identification of layer-by-layer relations}
\label{labeling}
\vspace{-5pt}
While automatically discovering concepts from inner representation has been a prominent field of research \citep{CRAFT}, automatically explaining the resulting concepts is often ignored, relying on manual annotations. \citet{Invert} utilize label description in ImageNet dataset to generate caption for neurons, however, these annotations is limited and can not be used to label low level concepts. Drawing inspiration from \citet{hoang2024llm, falcon}, we go one step further and not only use MLLM to label the (group of) neurons but also explain the relations between them in consecutive layers. Thus, we show the prospect of completing the whole picture of abstracting and explaining the inner representation in a systematic manner. 

Specifically, for two consecutive layers, we ask MLLM to describe the common visual features in a NGC, then matching with those of the top NGC (with the highest weights) at the preceding layer. This can be done iteratively throughout the concept circuit, generating a comprehensive explanation without human effort. We use a popular technique \citep{Chain_of_thought} to guide GPT4-o \citep{Gpt4o} step by step in captioning and in visual feature matching. Figure \ref{fig:captioning} shows an example of applying this technique to concept circuit of class ``great white shark". We observe that MLLM can correctly identify the common visual features within exemplary images of NGCs. Furthermore, MLLM is able to match the features from lower level NGCs to those at higher level, detailing formation of new features, showing the potential of explaining in automation, capturing the gradual process of constructing the output of the model. The prompt used in this experiment is available in Appendix \ref{prompt}.


\section{Discussion}

\subsection{Interpretation of Results}
The outcomes of this study, utilizing Convolutional Neural Networks (CNNs) within a Deep Reinforcement Learning (DRL) framework for financial analytics, underscore the pivotal role of temporal precision in market predictions. The findings particularly emphasize the efficacy of short-term observation windows, with the two-week window demonstrating superior performance in capturing market dynamics. This observation resonates with the rapidity and volatility characteristic of financial markets, where new information is swiftly reflected in stock prices. Notably, the enhanced model performance achieved through feature rearrangement highlights the significant impact of feature engineering. By reorganizing features related to different stocks and technical indicators, we support the hypothesis that a more robust and generalizable representation of data can be learned, potentially increasing the model's adaptability to diverse market conditions.

\subsection{Theoretical Implications}
The success of the CNN model, employing a methodical window expansion technique, accentuates the importance of temporal dynamics in financial time-series analysis. This approach aligns with the efficient market hypothesis, positing that markets assimilate all available information into stock prices. The model's ability to adjust to the market's 'memory' is crucial for precise financial forecasting. Additionally, the effectiveness of the rearranged features approach suggests that the organization and representation of information are as critical as the information itself for the learning process. This sheds light on the interaction of features within CNN layers, stressing the role of feature engineering in refining financial models.

Our findings challenge the conventional wisdom that longer observation windows invariably lead to better predictions in financial markets. The superior performance of shorter windows, particularly in non-rearranged datasets, suggests that recent market information may carry more predictive power than extended historical data. This aligns with the concept of market efficiency, where new information is rapidly incorporated into prices.

\section{Conclusions and Future Work}

In this study, we explored the impact of different data structures and observation windows on Convolutional Neural Networks (CNNs) performance in financial market analysis. We focused on understanding how the arrangement of data and the selection of timeframes influence the model's ability to capture and predict market dynamics.

The shift in optimal performance with rearranged data suggests that a Convolutional Neural Network (CNN) becomes more proficient at identifying complex patterns between features and their temporal dynamics when the data maintains a cohesive structure for each company's features. The ability of the model to grasp longer-term trends and relationships in rearranged datasets, which are less apparent in shorter timeframes or non-rearranged data, emphasizes the importance of strategic data arrangement. This finding underlines the need for flexible and specific approaches in selecting observation periods and organizing data to enhance the utility and accuracy of CNNs in financial market analysis.

\subsection{Limitations and Challenges}

This study, while offering valuable insights into the use of Convolutional Neural Networks (CNNs) for financial market analysis, encounters several limitations and challenges that warrant attention. A primary constraint is the need for increased computational power. Exploring additional observation windows and testing larger, more complex CNN architectures necessitate substantial computational resources. This requirement becomes particularly critical when considering the intricacy and volume of financial data and the need for extensive testing to validate the robustness and accuracy of the models across various market scenarios.

Moreover, there is a significant need for research funding to support these endeavors. Enhanced funding would facilitate access to more powerful computing infrastructure and enable a broader scope of experimentation. This includes investigating a wider array of temporal windows and deploying more advanced CNN models, which could potentially uncover deeper insights and yield more precise predictive capabilities.

Future studies could explore incorporating diverse data types, like news sentiment or economic indicators, to enhance model robustness. Broadening the scope to different markets and asset types would help verify the applicability of these findings. Investigating the effects of shorter temporal windows or real-time data streams may provide insights into high-frequency trading strategies. Additionally, applying the concept of rearranged features to other forms of financial data, such as order book information or unstructured data, could pave the way for innovative advancements in financial modeling techniques.

In conclusion, our study makes several key contributions to the field of financial DRL. First, we demonstrate the critical importance of temporal window selection in CNN-based models. Second, we show that feature rearrangement can significantly alter the optimal observation period. Finally, we provide a methodological framework for systematically exploring these parameters in future research. These insights open new avenues for enhancing the accuracy and robustness of AI-driven financial analysis tools.

% \section*{Acknowledgements}
% Thanks to ...

%% The Appendices part is started with the command \appendix;
%% appendix sections are then done as normal sections
% \appendix

% \section{Appendix title 1}
%% \label{}

% \section{Appendix title 2}
%% \label{}

%% If you have bibdatabase file and want bibtex to generate the
%% bibitems, please use
%%
\section{Declaration of generative AI and AI-assisted technologies in the writing process}
% \textbf{Declaration of generative AI and AI-assisted technologies in the writing process}
During the preparation of this work the author(s) used ChatGPT  as writing assistant to draft text, improving clarity , proofreading, language refinement, and saving time. After using this tool/service, the author(s) reviewed and edited the content as needed and take(s) full responsibility for the content of the publication.


% \bibliographystyle{IEEEtran} % We choose the "plain" reference style
% \begin{thebibliography}{24}

% \begin{thebibliography}{}

\bibliography{7-biblio}

% \bibitem{parallel_drl_stock_trading}
% Wang, Xiaoyang, et al. "A parallel multi-module deep reinforcement learning algorithm for stock trading." Neurocomputing 439 (2021): 23-34. Elsevier. DOI: 10.1016/j.neucom.2021.01.128. URL: https://www.sciencedirect.com/science/article/pii/S0925231221005233.

% \bibitem{novel_drl_stock_trading}
% Zhang, Wei, et al. "A novel Deep Reinforcement Learning based automated stock trading system." Expert Systems with Applications 184 (2021): 115548. Elsevier. DOI: 10.1016/j.eswa.2021.115548. URL: https://www.sciencedirect.com/science/article/pii/S0957417423033031.

% \bibitem{market_sentiment_drl_stock_trading}
% Huang, Xin, Haoran Tan, and Qing Xu. "Market sentiment-aware deep reinforcement learning approach for stock trading." Procedia Computer Science 181 (2021): 1147-1155. Elsevier. DOI: 10.1016/j.procs.2021.01.296. URL: https://www.sciencedirect.com/science/article/pii/S2215098621000070.

% \bibitem{drl_end_to_end_stock_trading}
% Liu, Qi, et al. "Deep Reinforcement Learning Based End-to-End Stock Trading Strategy." Procedia Computer Science 184 (2021): 320-326. Elsevier. DOI: 10.1016.j.procs.2021.01.280. URL: https://www.sciencedirect.com/science/article/pii/S2215098621000197.

% \bibitem{adaptive_drl_stock_trading}
% Li, Bo, Liang Zhang, Rui Sun, and Zhaoyu Sun. "Adaptive stock trading strategies with deep reinforcement learning." ScholarX Journal 27.3 (2021): 451-467. URL: https://scholar.xjtlu.edu.cn/en/publications/adaptive-stock-trading-strategies-with-deep-reinforcement-learning.

% \bibitem{Liu2020}
% Liu, Xiao-Yang, et al. "FinRL-Meta: Market Environments and Benchmarks for Data-Driven Financial Reinforcement Learning." Deep Reinforcement Learning Workshop, 34th Conference on Neural Information Processing Systems (NeurIPS2020), 2020. Vancouver, Canada. arXiv:2011.09607v2 [q-fin.TR]. DOI: 10.48550/arXiv.2011.09607. URL: https://doi.org/10.48550/arXiv.2011.09607.

% \bibitem{Vyawahare2020}
% Vyawahare, Abides, et al. "ABIDES-Gym: Gym Environments for Multi-Agent Discrete Event Simulation and Application to Financial Markets." Proceedings of the IEEE Conference on Business Informatics (CBI), 2020.

% \bibitem{Montazeri2023}
% S. Montazeri, A. Mirzaeinia, H. Jumakhan, and A. Mirzaeinia, "CNN-DRL for Scalable Actions in Finance," presented at the 10th Annual Conf. on Computational Science \& Computational Intelligence, 2024. [Online]. Available: https://doi.org/10.48550/arXiv.2401.06179

% \bibitem{Montazeri2024}
% S. Montazeri, A. Mirzaeinia, and A. Mirzaeinia, "CNN-DRL with Shuffled Features in Finance," presented at the 10th Annual Conf. on Computational Science \& Computational Intelligence (CSCI'23), 2024. [Online]. Available: https://doi.org/10.48550/arXiv.2402.03338

% \bibitem{Zhang2019}
% Zhang, Wei, et al. "FinRL-Podracer: High Performance and Scalable Deep Reinforcement Learning for Quantitative Finance." Deep Reinforcement Learning Workshop, 34th Conference on Neural Information Processing Systems (NeurIPS2020), 2019.

% \bibitem{Tsay2010}
% Tsay, Ruey S. "Analysis of Financial Time Series." John Wiley \& Sons, 2010.

% \bibitem{Kumar2007}
% Kumar, P. R., and V. Ravi. "Bankruptcy prediction in banks and firms via statistical and intelligent techniques – A review." European Journal of Operational Research 180.1 (2007): 1-28.

% \bibitem{Atsalakis2009}
% Atsalakis, G. S., and K. P. Valavanis. "Surveying stock market forecasting techniques – Part II: Soft computing methods." Expert Systems with Applications 36.3 (2009): 5932-5941.

% \bibitem{Mnih2015}
% Mnih, Volodymyr, et al. "Human-level control through deep reinforcement learning." Nature 518.7540 (2015): 529-533.

% \bibitem{Tsantekidis2017}
% Tsantekidis, Avraam, et al. "Forecasting stock prices from the limit order book using convolutional neural networks." Proceedings of the IEEE Conference on Business Informatics (CBI), 2017.

% \bibitem{courbariaux2016binarized}
% Courbariaux, M., Hubara, I., Soudry, D., El-Yaniv, R., \& Bengio, Y. (2016). Binarized neural networks: Training deep neural networks with weights and activations constrained to +1 or −1. arXiv preprint arXiv:1602.02830.

% \bibitem{dai2021coatnet}
% Dai, Z., Liu, H., Le, Q. V., \& Tan, M. (2021). Coatnet: Marrying convolution and attention for all data sizes. arXiv preprint arXiv:2106.04803.

% \bibitem{chen2019compressing}
% Chen, K., \& Huo, Q. (2019). Compressing CNN–DBLSTM models for OCR with teacher–student learning and Tucker decomposition. Pattern Recognition, 96, 106957. Elsevier.

% \bibitem{dai2016rfcn}
% Dai, J., Li, Y., He, K., \& Sun, J. (2016). R-FCN: Object detection via region-based fully convolutional networks. IEEE Transactions on Pattern Analysis and Machine Intelligence. IEEE.

% \bibitem{chollet2017xception}
% Chollet, F. (2017). Xception: Deep learning with depthwise separable convolutions. arXiv preprint arXiv:1610.02357.

% \bibitem{LeCun2015}
% LeCun, Yann, Yoshua Bengio, and Geoffrey Hinton. "Deep learning." Nature 521.7553 (2015): 436-444.

% \bibitem{Krizhevsky2012}
% Krizhevsky, Alex, Ilya Sutskever, and Geoffrey E. Hinton. "ImageNet classification with deep convolutional neural networks." Advances in Neural Information Processing Systems 25 (2012).

% \bibitem{Goodfellow2016}
% Goodfellow, Ian, Yoshua Bengio, and Aaron Courville. "Deep Learning." MIT Press, 2016.

% \bibitem{He2016}
% He, Kaiming, et al. "Deep residual learning for image recognition." Proceedings of the IEEE Conference on Computer Vision and Pattern Recognition, 2016: 770-778.

% \bibitem{Simonyan2014}
% Simonyan, K., \& Zisserman, A. (2014). Very deep convolutional networks for large-scale image recognition. arXiv preprint arXiv:1409.1556.

% \bibitem{Szegedy2015}
% Szegedy, C., Liu, W., Jia, Y., Sermanet, P., Reed, S., Anguelov, D., Erhan, D., Vanhoucke, V., \& Rabinovich, A. (2015). Going deeper with convolutions. In Proceedings of the IEEE Conference on Computer Vision and Pattern Recognition (pp. 1-9).

% \end{thebibliography}

% \end{thebibliography}

\end{document}

\endinput
%%
%% End of file `elsarticle-template-harv.tex'.