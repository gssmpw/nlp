Here is your hypothesis section with the references added:

```latex
\section{Hypothesis}

Convolution operations are fundamental to Convolutional Neural Networks (CNNs), which are particularly effective in processing data with a grid-like topology, such as images and sequential data \cite{courbariaux2016binarized} \cite{dai2021coatnet}. The convolution operation can be understood as a mathematical process that combines two sets of information. In the context of CNNs, this involves a convolutional kernel (or filter) moving across an input signal (such as an image or time series data) to produce a feature map.

Mathematically, for continuous signals, the convolution operation is defined as:

\[
(S * K)(t) = \int_{-\infty}^{\infty} S(\tau)K(t - \tau) d\tau
\]

Here, \(S\) represents the input signal, and \(K\) represents the convolutional kernel. This integral computes the area under the product of the two functions as the kernel slides over the input signal. However, in practical applications involving digital data, the signals are discrete, and thus the convolution operation is adapted to:

\[
(S * K)[n] = \sum_{m=-M}^{M} S[m]K[n - m]
\]

In this discrete form, the convolution operation involves summing the element-wise products of the input signal and the kernel as it moves across the input. The result is a new set of values (the feature map) that highlight certain features of the input signal, such as edges in an image or patterns in sequential data \cite{chen2019compressing}.

The size of the convolutional kernel (or filter) is a critical parameter in this operation. The kernel size determines the local region from which features are extracted. A larger kernel can capture more contextual information by encompassing a wider region of the input signal, while a smaller kernel focuses on finer details. The balance between capturing local and global features is essential for the performance of CNNs \cite{dai2016rfcn}.

Additionally, the padding applied to the input signal before convolution affects the output size and the nature of the features extracted. Padding involves adding extra values (typically zeros) around the input signal, which allows the kernel to process edge regions more effectively. The output size of the convolution operation is given by:

\[
O = \frac{N - K + 2P}{S} + 1
\]

where \(N\) is the input size, \(K\) is the kernel size, \(P\) is the padding, \(S\) is the stride (the step size of the kernel), and \(O\) is the output size. Properly setting these parameters ensures that the CNN can effectively learn and extract meaningful features from the input data \cite{chollet2017xception}. Understanding these concepts is crucial for optimizing CNN architectures, especially in settings where the observation window size can significantly impact the model's performance.

The performance of Convolutional Neural Networks (CNNs) in processing sequential data is significantly influenced by the size of the observation window used in the convolutional layers. The kernel size in a convolution layer determines the local region from which features are extracted. Larger kernels can incorporate more contextual information, but excessively large kernels may dilute distinct features. The optimization of window size can be expressed through the effective window size equation:

\[
W_{\text{eff}} = W_{\text{kernel}} + (W_{\text{kernel}} - 1) \times (D - 1)
\]

where \(W_{\text{eff}}\) is the effective window size, \(W_{\text{kernel}}\) is the kernel size, and \(D\) is the dilation factor.

Furthermore, the role of padding in convolution processes influences the spatial dimensions of the output feature map, described by:

\[
O = \frac{N - K + 2P}{S} + 1
\]

where \(N\) is the input size, \(K\) is the kernel size, \(P\) is the padding, \(S\) is the stride, and \(O\) is the output size. Excessive padding can lead to overemphasis on peripheral data and potential overfitting, similar to how an over-expanded window size may cause information overload, making distinct features less discernible:

\[
\text{Information Overload} \propto \frac{W_{\text{eff}}}{\text{Distinct Features}}
\]

Therefore, a crucial balance is needed between capturing local and global features. We hypothesize that the optimal selection of a temporal window size in a CNN balances local feature detection and global contextual understanding. An optimally sized window allows the model to effectively capture relevant features without succumbing to information overload or excessive generalization, thereby enhancing accuracy and performance in sequential data processing tasks \cite{chen2019compressing}.

Given that our CNN acts as a policy for a Deep Reinforcement Learning (DRL) algorithm, the window size as a hyperparameter will be optimized through reinforcement learning. This optimal window size is found at the point where local and global feature detection are balanced:

\[
\text{Optimal Window Size} \leftrightarrow \min \left( \Delta_{\text{Local-Global}} \right)
\]

where \( \Delta_{\text{Local-Global}} \) measures the differential in information capture between local and global features. This hypothesis suggests that through careful tuning and reinforcement learning, the CNN can achieve an optimal window size that maximizes performance in sequential data tasks.
