\section{Methodology}
In this study, we have integrated Deep Reinforcement Learning (DRL), Proximal Policy Optimization (PPO), and the Markov Decision Process (MDP) framework. The integration method is adopted from FinRL \cite{Liu2022FinRL}, providing a robust and dynamic model capable of navigating the complexities of financial markets. DRL offers the foundational learning mechanism, MDP provides a structured approach to decision-making in uncertain environments, and PPO ensures efficient and stable policy optimization. Together, these methodologies create a sophisticated model capable of learning, adapting, and optimizing trading strategies in real-time financial scenarios. The upcoming sections will describe each component in detail, beginning with an overview of DRL and its significance in our framework.

\subsection{Deep Reinforcement Learning (DRL)}
Deep Reinforcement Learning (DRL) integrates the pattern recognition capabilities of Deep Learning with the decision-making framework of Reinforcement Learning. This synergy enables the development of sophisticated models that can autonomously adapt to the complex and dynamic nature of financial markets, learning to optimize strategies based on data-driven insights. By leveraging vast and varied datasets, DRL models can identify latent patterns and trends, dynamically adjusting strategies by continually learning from market data. This ability to process high-dimensional data and make real-time decisions significantly advances over traditional quantitative approaches.

DRL's ability to respond to market volatility and changes is crucial in financial markets. It addresses the high dimensionality of financial data and the need for timely decision-making. This forms the basis for integrating Proximal Policy Optimization (PPO), which enhances the stability and efficiency of our learning process.



\subsection{Markov Decision Process (MDP)}
The Markov Decision Process (MDP) provides a mathematical framework for modeling decision-making in situations where outcomes are partly random and partly under the control of a decision-maker. MDPs are fundamental to understanding reinforcement learning and are particularly relevant in financial applications where decisions must be made under uncertainty. 

In our study, MDPs model the sequential decision-making process, where each action the agent takes affects future states and rewards. We represent the trading environment as an MDP with states, actions, and rewards meticulously defined to capture the intricacies of financial markets. The state space encapsulates key financial indicators, the action space comprises various trading actions, and the reward function reflects financial gains or losses. This representation allows our DRL model to effectively learn and optimize trading strategies over time, accounting for the probabilistic nature of financial markets and the impact of each decision on future market states.  With the MDP framework providing the foundation for decision-making, we now turn to the role of feature extraction in our DRL agent, specifically through Convolutional Neural Networks (CNNs).

\subsubsection{MDP Model for Stock Trading}
The trading market is a stochastic and interactive environment in nature and can be formulated as a Markov Decision Process (MDP) with state, action, and reward.
\begin{itemize}
    \item State $s=[b, p, h, f]:$ a set that consist of balance $b$, price $p \in \mathbb{R}_+^D$, holdings of stock $h\in \mathbb{Z}_+^D$, and fundamental indicators $f$. where D is the number of stocks that we consider in the market. Fundamental indicators covers financial ratios listed in tables \ref{table:sma_data}, \ref{table:tech_ind_data}.
    \item Action $a=[sell, buy, hold]:$ a set of actions for all D stocks, consisting of sell, buy, hold which leads to a reduction, growth, or no alteration in the holdings h, correspondingly.
    \item Reward $r(s, a, s'):$ The adjustment in portfolio value upon executing action "a" in state $"s"$ and transitioning to the next state $"s'"$. The portfolio value encompasses the total value of equities in the held stocks, denoted as $p^Th$, plus the remaining balance, $"b"$.
    \item Policy $\pi(s)$: The stock trading approach in state "s" entails the probability distribution of "a" in the state "s".
    \item The action-value function $Q_\pi(s, a)$
    represents the anticipated reward obtained by taking action "a" in state "s" according to policy $\pi$.
\end{itemize}
The primary objective of this process is to optimize (maximize) the reward. Various published approaches exist for addressing this challenge, each with its own set of advantages and disadvantages \cite{sutton2018reinforcement}. We select PPO  which is commonly used and show higher performance than other approaches.

\subsubsection{Proximal Policy Optimization (PPO)}
Proximal Policy Optimization (PPO) is a cornerstone of our methodology, providing a robust approach to policy gradient optimization. PPO iteratively updates the policy in a controlled manner, minimizing the cost function while ensuring minimal deviation from the previous policy. This approach is achieved through a clipped objective function, which restricts the extent of policy updates at each iteration. PPO maintains stability during the learning process by comparing the new policy's performance to the old policy and ensuring updates occur only if they improve performance within a specified margin. 

This stability is particularly crucial in the volatile context of financial markets, where significant, risky updates could destabilize the model. PPO's multiple epochs of stochastic gradient ascent optimizes the policy, enhancing sample efficiency by reusing data. This method is valuable in financial applications where data can be scarce and costly. Including entropy terms in the objective function encourages exploration, preventing premature convergence to suboptimal policies. This makes PPO an effective choice for our DRL framework, balancing exploration and exploitation to ensure consistent and reliable trading performance. With PPO demonstrated in our methodology, let us next discuss the role of the Markov Decision Process (MDP) in modeling decision-making under uncertainty.

\subsection{CNN is as a Feature Extraction Network}
The CNN integration into FinRL is facilitated through a specialized gym environment simulating stock trading scenarios. This environment includes quantitative elements of the stock market, such as stock prices, trading volumes, and various financial ratios, which are fed into the CNN for analysis, and the CNN's role within this environment is to extract high-level features from the input data, which are then utilized by the DRL agent to make trading decisions. By transforming raw financial data into meaningful features, the CNN enables the DRL agent to learn and optimize trading strategies effectively.

Within this framework, our previous work has demonstrated that using Convolutional Neural Networks (CNNs) in Deep Reinforcement Learning (DRL) for financial applications is notably effective. The CNN model processes input states comprising stock prices and technical indicators, capturing complex patterns and relationships within the data. This enables the model to autonomously learn and adapt strategies, making informed trading decisions based on a deeper understanding of market behavior.

The CNN acts as a feature extractor within the DRL framework. It processes the raw financial data, learning to identify relevant patterns and trends. These extracted features are then fed into the DRL agent, which uses them to make trading decisions. This integration allows the model to adapt its feature extraction process based on the rewards received, creating a dynamic learning system that evolves with changing market conditions.

\subsubsection{CNN Architecture}
As mentioned before, CNN architecture used in our study is designed to handle the multidimensional nature of financial data and train on extensive datasets. Leveraging convolutional layers, batch normalization, and ReLU activation functions enhances this model's feature extraction and pattern recognition robustness. CNN's ability to capture localized features and temporal dependencies is critical in financial markets rich in temporal dynamics and complex patterns.

The feature extraction process involves CNN identifying localized features and temporal dependencies within the financial data. Its ability to capture these dynamics ensures that the DRL agent can adapt its strategies in response to changing market conditions. The effectiveness of CNN as a feature extractor is further enhanced by its capacity to handle large datasets and complex input structures. This capability allows the model to leverage vast historical and real-time market data, improving its predictive accuracy and decision-making performance.

\begin{figure*}
    \includegraphics[width=1\columnwidth]{CNN_arch_v1.5.pdf}
    \caption{Architecture of the Convolutional Neural Network}
    \label{fig:architecture_of_network}
\end{figure*}

The network comprises two primary convolutional layers: the first layer features a kernel size of 8 and a stride of 4, while the second layer has a kernel size of 4 and a stride of 2. Both layers include 2D batch normalization, enhancing the network's efficiency in learning from the data by stabilizing the learning process. The data is then flattened and fed into a fully connected neural network layer with ReLU activation, integrating the extracted features for decision-making. The parameter specifications of our CNN network architecture are listed in Table \ref{tab:net_param}. 

The choice of this specific architecture was motivated by its ability to capture both short-term price movements and longer-term trends. The kernel sizes (8 and 4) were selected to allow the model to focus on weekly and monthly patterns, respectively, while the stride values (4 and 2) help in reducing computational complexity without significant loss of information.

\begin{table}[ht]
\centering
\rowcolors{2}{gray!10}{white}
{\fontsize{9pt}{11pt}
\selectfont
\begin{tabularx}{\linewidth}{Xll}
\toprule
\hiderowcolors
\textbf{Layer (type)} & \textbf{Output Shape} & \textbf{Parameter Size} \\
\showrowcolors
\midrule
Conv2d-1       & {[}-1, 32, 12, 85{]} & 2,080     \\
BatchNorm2d-2  & {[}-1, 32, 12, 85{]} & 64        \\
ReLU-3         & {[}-1, 32, 12, 85{]} & 0         \\
MaxPool2d-4    & {[}-1, 32, 6, 42{]}  & 0         \\
Conv2d-5       & {[}-1, 64, 12, 30{]} & 32,832    \\
BatchNorm2d-6  & {[}-1, 64, 12, 30{]} & 128       \\
ReLU-7         & {[}-1, 64, 12, 30{]} & 0         \\
MaxPool2d-8    & {[}-1, 64, 6, 15{]}  & 0         \\
Conv2d-9       & {[}-1, 128, 4, 13{]} & 73,856    \\
BatchNorm2d-10 & {[}-1, 128, 4, 13{]} & 256       \\
ReLU-11        & {[}-1, 128, 4, 13{]} & 0         \\
Conv2d-12      & {[}-1, 256, 2, 11{]} & 295,168   \\
BatchNorm2d-13 & {[}-1, 256, 2, 11{]} & 512       \\
ReLU-14        & {[}-1, 256, 2, 11{]} & 0         \\
Flatten-15     & {[}-1, 5632{]}       & 0         \\
Linear-16      & {[}-1, 1024{]}       & 5,768,192 \\
ReLU-17        & {[}-1, 1024{]}       & 0         \\
Dropout-18     & {[}-1, 1024{]}       & 0         \\
Linear-19      & {[}-1, 512{]}        & 524,800   \\
ReLU-20        & {[}-1, 512{]}        & 0         \\
Dropout-21     & {[}-1, 512{]}        & 0         \\
Linear-22      & {[}-1, 128{]}        & 65,664    \\
ReLU-23        & {[}-1, 128{]}        & 0         \\ \hline
\end{tabularx}%
}
\caption{Total model params: 6,763,552 \\
Trainable params: 6,763,552 \\
Non-trainable params: 0 \\
---------------------------------------\\
Input size (MB): 0.01 \\
Forward/backward pass size (MB): 1.74 \\
Params size (MB): 25.80 \\
Estimated Total Size (MB): 27.55 \\
---------------------------------------}
\label{tab:net_param}
\end{table}

With the CNN architecture established, the next step involves integrating this model into a DRL framework tailored for financial market analysis. This integration is facilitated by developing a specialized gym environment that simulates stock trading scenarios. The environment encapsulates critical elements of the stock market, including stock prices, trading volumes, and various financial ratios, which are fed into the CNN for analysis. This environment forms the backbone of our methodology, enabling the CNN to interact with and learn from a simulated financial market dynamically.

\subsection{Iterative Window Expansion Technique}

We conducted 24 structured experiments across six temporal intervals ranging from 2 to 12 weeks, in 2-week increments. Each interval was chosen in 2-week increments, providing a range of short- to medium-term observations. We utilized two distinct dataset types for each timeframe: the Technical Indicator dataset and the Simple Moving Average (SMA) dataset. While these datasets encompass the same companies and timeframes, they include different features for each company. Each dataset was analyzed under two scenarios: one with rearranged features, grouping all columns associated with a single company, and another without rearrangement. This dual-path strategy, uniformly applied across all intervals, resulted in 24 unique experimental setups, comprehensively evaluating the CNN's performance and robustness under various temporal and data scenarios (Figure \ref{fig:journal_paper_features_old_data_not_rearranged}).

\begin{figure*}[htbp]
    \centering
    \includegraphics[width=\linewidth]{features.pdf}
    \caption{Composition of features, rearranged (left) and not rearranged (right)}
    \label{fig:journal_paper_features_old_data_not_rearranged}
\end{figure*}

\subsubsection{Initial Two-Week Window}

The study begins with a concise two-week observation window, targeting short-term market trends to establish a baseline for model performance. This initial phase is critical for understanding the model's responsiveness to recent market changes and its ability to capture short-term patterns. The two-week window helps the model make timely and accurate predictions in the fast-paced financial trading environment by focusing on the most recent and relevant data points.

\subsubsection{Bi-Weekly Expansion Strategy}

Following the initial phase, we bi-weekly expanded the observation window, incrementally integrating more historical data. This gradual enlargement enables the model to assimilate information from a widening scope of market conditions, capturing more extensive long-term trends and patterns. The bi-weekly increments strike a careful balance, incorporating fresh data while retaining the benefits of an extended historical view. This approach ensures the model remains agile, effectively responding to immediate market changes and more substantial, enduring trends.

\subsubsection{Final Twelve-Week Window}

The process culminates with a twelve-week window, providing an exhaustive perspective on market trends and behaviors over an extended duration. This elongated observation period supplies the model with a diverse and comprehensive dataset, reflecting a broad spectrum of market activities. The value of the twelve-week window lies in its ability to reveal longer-term market trends and cyclical patterns, which are pivotal for strategic decision-making in financial trading. This concluding phase is crucial for evaluating the model's capacity to generalize and maintain consistent performance across various market cycles.

\subsection{Rearranged Features Approach}

Expanding upon our previous research, this paper also investigates the impact of feature rearrangement within expanded temporal windows. The rearranged features setting entails reorganizing the columns of the input data tensor to keep related features in proximity to each other. This arrangement aims to boost the capability of the CNN in identifying relevant patterns from the data, aligning with the underlying relationships and correlations inherent in financial indicators. Presenting the CNN with inputs specifically structured to reflect the interconnected nature of financial metrics is expected to enhance the model's accuracy and generalization ability. This preprocessing strategy is particularly pertinent in financial data analysis, where the interactions between various data types (such as stock prices, transaction volumes, and technical indicators) are often more critical than the individual data points. This approach is intended to promote more efficient learning, improving the model's robustness and adaptability when deployed on a wide range of financial datasets.

\subsection{Our Datasets}

To ensure the robustness of our approach, we utilized two distinct datasets from the FinRL and FinRL Meta projects. This methodology helps confirm that the success of our methods is not merely coincidental.

\subsubsection{The SMA Dataset}

The first SMA data dataset is adapted from the FinRL Meta project. This dataset encompasses quantitative financial features, including fundamental market data such as opening, high, low, and closing prices and trading volume. Additionally, it includes a series of engineered features like MACD, Bollinger Bands, RSI, CCI, and DX over 30 days, in addition to the 30-day and 60-day closing simple moving averages (SMAs), the VIX, and a turbulence measure. This rich compilation provides an extensive perspective on market trends and volatility, crucial for the Convolutional Neural Network (CNN) model's analysis across varying timeframes.

\begin{table}[ht]
\centering
\rowcolors{2}{gray!10}{white}
{\fontsize{9pt}{11pt}\selectfont
\begin{tabularx}{\columnwidth}{Xl}
\toprule
\hiderowcolors
\textbf{Dataset Column} & \textbf{Column Description} \\
\showrowcolors
\midrule
tic & Ticker symbol \\
open & Opening price \\
high & Highest price \\
low & Lowest price \\
close & Closing price \\
volume & Trading volume \\
day & Day of the week \\
macd & MACD value \\
boll\_ub & Upper Bollinger Band \\
boll\_lb & Lower Bollinger Band \\
rsi\_30 & 30-day Relative Strength Index \\
cci\_30 & 30-day Commodity Channel Index \\
dx\_30 & 30-day Directional Movement Index \\
close\_30\_sma & 30-day closing Simple Moving Average \\
close\_60\_sma & 60-day closing Simple Moving Average \\
vix & VIX value \\
turbulence & Market turbulence \\
\bottomrule
\end{tabularx}
}
\caption{Simple Moving Average (SMA) Data}
\label{table:sma_data}
\end{table}

\subsubsection{Feature Vector: A Trading Day in the Market}
Each trading day in the stock market includes a feature vector comprising the initial monetary amount, stock prices of twenty-nine companies, their corresponding shares held, and a set of eight quantitative features for each company. These indicators include MACD, Bollinger Bands (upper and lower), RSI 30, CCI 30, DX 30, 30-day and 60-day SMAs. The total feature vector comprises 261 elements: one for the initial amount, 29 for stock prices, 29 for shares held, and 232 derived from technical indicators (eight per company). Integrating these technical indicators, which play a critical role in signaling market trends and momentum, equips the dataset as an essential tool for the CNN model. It enables the model to identify and leverage market trends effectively, facilitating precise predictions and informed decision-making in the dynamic financial trading environment.

\begin{table}[!ht]
\centering
\rowcolors{2}{gray!10}{white}
{\fontsize{9pt}{11pt}\selectfont
\begin{tabularx}{\columnwidth}{Xl}
\toprule
\hiderowcolors
\textbf{Feature Name}                            & \textbf{Size} \\
\showrowcolors
\midrule
Amount                                   & 1             \\
Price                                    & 29            \\
Share held                               & 29            \\
MACD                                     & 29            \\
Bollinger Upper Band (boll\_ub)          & 29            \\
Bollinger Lower Band (boll\_lb)          & 29            \\
RSI 30                                   & 29            \\
CCI 30                                   & 29            \\
DX 30                                    & 29            \\
Close 30 SMA                             & 29            \\
Close 60 SMA                             & 29            \\
\text{Total size of feature vector}      & 261           \\
\bottomrule
\end{tabularx}
}
\caption{Daily Feature Vector for SMA Data}
\end{table}
\FloatBarrier

\subsubsection{The Technical Indicator Dataset}

The Technical Indicator Dataset offers an in-depth view of financial performance metrics, distinguishing itself from the SMA dataset with a more extensive set of financial ratios and metrics. While it also includes fundamental trading data such as opening price, highest price, lowest price, closing prices, and trading volume, its uniqueness lies in incorporating a diverse range of financial ratios. These include Operating and Net Profit Margins, Return on Assets, Return on Equity, various liquidity ratios (Current, Quick, Cash), turnover ratios (Inventory, Accounts Receivable, Accounts Payable), Debt Ratio, Debt to Equity Ratio, and market valuation ratios like PE, PB, and Dividend Yield. This dataset is instrumental in offering a detailed assessment of instruments' financial health and market valuation, a critical aspect of the nuanced market analysis conducted by our Convolutional Neural Network (CNN) model.

\begin{table}[!ht]
\rowcolors{2}{gray!10}{white}
{\fontsize{9pt}{11pt}\selectfont
\begin{tabularx}{\columnwidth}{Xl}
    \toprule
    \hiderowcolors
    \textbf{Dataset Column} & \textbf{Description} \\
    \showrowcolors
    \midrule
    tic & Ticker symbol \\
    open & Opening price \\
    high & Highest price \\
    low & Lowest price \\
    close & Closing price \\
    volume & Trading volume \\
    OPM & Operating Profit Margin \\
    NPM & Net Profit Margin \\
    ROA & Return on Assets \\
    ROE & Return on Equity \\
    cur\_ratio & Current Ratio \\
    quick\_ratio & Quick Ratio \\
    cash\_ratio & Cash Ratio \\
    inv\_turnover & Inventory Turnover \\
    acc\_rec\_turnover & Accounts Receivable Turnover \\
    acc\_pay\_turnover & Accounts Payable Turnover \\
    debt\_ratio & Debt Ratio \\
    debt\_to\_equity & Debt to Equity Ratio \\
    PE & Price to Earnings Ratio \\
    PB & Price to Book Ratio \\
    Div\_yield & Dividend Yield \\
    \bottomrule
\end{tabularx}
}
\caption{Technical Indicator Data Columns Description}
\label{table:tech_ind_data}
\end{table}
\FloatBarrier

\subsubsection{A Different Feature Vector}
The daily feature vector within this dataset is structured to provide an exhaustive market perspective through a multidimensional data array. This table comprises several components: the initial amount, stock prices of thirty companies, the number of shares currently owned in the simulation, and fifteen distinct financial ratios for each of the thirty companies. These ratios, extracted from each company's financial statements, offer vital insights into their financial performance. The feature vector, encompassing the data for one trading day in the stock market, contains 511 elements: one for the initial amount, 30 for stock prices, 30 for shares held, and 450 derived from the financial ratios (15 per company). This elaborate dataset is essential for the CNN model, enabling the analysis and interpretation of intricate patterns and correlations within the financial markets.

\begin{table}[ht]
\rowcolors{2}{gray!10}{white}
{\fontsize{9pt}{11pt}\selectfont
\begin{tabularx}{\columnwidth}{Xc}
\toprule
\hiderowcolors
\textbf{Feature Category} & \textbf{Number of Features} \\
\showrowcolors
\midrule
Amount                          & 1    \\
Price                           & 30   \\
Share held                      & 30   \\
Financial ratios (15 * 30)      & 450  \\
Total size of feature vector    & 511  \\
\bottomrule
\end{tabularx}
}
\caption{Technical Dataset Feature Vector}
\end{table}
\FloatBarrier