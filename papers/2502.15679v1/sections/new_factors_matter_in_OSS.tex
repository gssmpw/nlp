\subsection{Factors Influencing \pb}
\label{sec:factors}

% \yy{Dan suggests to delete the whole section.}

In this section, we investigate two potential factors influencing the severity of \pb: skill difficulty and the magnitude of visual modifications. This analysis aims to provide insights for future research and inform the development of algorithms robust to \pb. We define the following two hypotheses:

\begin{enumerate}[I:]
    \item Higher skill difficulty (i.e., lower skill success rate) leads to more severe performance degradation caused by \pb for that skill.
    \item Larger magnitudes of visual modifications result in more severe performance degradation caused by \pb for the current skill.
\end{enumerate}

To test these hypotheses, we use Spearman's rank correlation coefficient ($\rho$)~\cite{spearman1961proof}, which effectively measures the strength of monotonic relationships between two variables. If hypothesis I holds, we expect a significant negative monotonic relationship (i.e., $\rho < 0$, $p < 0.05$) between skill success rate and the severity of \pb. Similarly, if hypothesis II is valid, we anticipate a significant positive monotonic relationship (i.e., $\rho > 0$, $p < 0.05$) between the magnitude of visual modifications and the severity of \pb.



\subsubsection{Test Hypothesis I (Factor: Task Difficulty)}
\begin{table}[ht]
\centering
\caption{Spearman’s rank correlation coefficients ($\rho$) and p-values ($p$) for the relationship between task difficulty and OSS severity.}
\label{tb:fac1_results}
\begin{tabular}{|c|c|c|c|}
\hline
      & BC-RESNET-RNN                                                & BC-RESNET-T                                                  & BC-VIT-T                                                     \\ \hline
Set 1 & \begin{tabular}[c]{@{}c@{}}$\rho$: -0.86\\ $p$: 0.06\end{tabular} & \begin{tabular}[c]{@{}c@{}}$\rho$: 0.02\\ $p$: 0.97\end{tabular}  & \begin{tabular}[c]{@{}c@{}}$\rho$: 0.04\\ $p$: 0.94\end{tabular}  \\ \hline
Set 2 & \begin{tabular}[c]{@{}c@{}}$\rho$: 0.89\\ $p$: 0.04\end{tabular}  & \begin{tabular}[c]{@{}c@{}}$\rho$: -0.04\\ $p$: 0.94\end{tabular} & \begin{tabular}[c]{@{}c@{}}$\rho$: 0.78\\ $p$: 0.04\end{tabular}  \\ \hline
Set 3 & \begin{tabular}[c]{@{}c@{}}$\rho$: -0.80\\ $p$: 0.20\end{tabular} & \begin{tabular}[c]{@{}c@{}}$\rho$: 0.50\\ $p$: 0.39\end{tabular}  & \begin{tabular}[c]{@{}c@{}}$\rho$: 0.55\\ $p$: 0.33\end{tabular}  \\ \hline
Set 4 & \begin{tabular}[c]{@{}c@{}}$\rho$: 0.95\\ $p$: 0.05\end{tabular}  & \begin{tabular}[c]{@{}c@{}}$\rho$: -0.21\\ $p$: 0.74\end{tabular} & \begin{tabular}[c]{@{}c@{}}$\rho$: 0.15\\ $p$: 0.80\end{tabular}  \\ \hline
Set 5 & \begin{tabular}[c]{@{}c@{}}$\rho$: 0.21\\ $p$: 0.79\end{tabular}  & \begin{tabular}[c]{@{}c@{}}$\rho$: -0.34\\ $p$: 0.57\end{tabular} & \begin{tabular}[c]{@{}c@{}}$\rho$: -0.67\\ $p$: 0.22\end{tabular} \\ \hline
Set 6 & \begin{tabular}[c]{@{}c@{}}$\rho$: 0.89\\ $p$: 0.11\end{tabular}  & \begin{tabular}[c]{@{}c@{}}$\rho$: 0.79\\ $p$: 0.11\end{tabular}  & \begin{tabular}[c]{@{}c@{}}$\rho$: 0.87\\ $p$: 0.05\end{tabular}  \\ \hline
Avg.  & \begin{tabular}[c]{@{}c@{}}$\rho$: 0.21\\ $p$: 0.21\end{tabular}  & \begin{tabular}[c]{@{}c@{}}$\rho$: 0.12\\ $p$: 0.62\end{tabular}  & \begin{tabular}[c]{@{}c@{}}$\rho$: 0.29\\ $p$: 0.40\end{tabular}  \\ \hline
\end{tabular}
\end{table}



We use the metric ``Ratio Performance Delta'' to quantify the severity of \pb and measure task difficulty using task success rate, assuming demonstrations for all skills are of similar quality. To analyze the relationship between skill difficulty and \pb, we design an experiment that isolates skill difficulty as the only variable. We select six sets of skills, where all skills within each set are performed in the same scene (e.g., kitchen-scene-1 in Libero). Identical modifications are applied to all skills within each set to simulate \pb, resulting in several pairs of skills: one unaffected by \pb and the other affected by \pb. Between pairs, the only difference is the skill's difficulty. After evaluating BCs on these pairs, we generate two lists of results for each set and algorithm: one for success rates of skills unaffected by \pb (i.e., task difficulty) and another for ``Ratio Performance Delta'' on skills affected by \pb (i.e., severity of \pb). These results are used to calculate Spearman's rank correlation coefficient ($\rho$) and the corresponding p-value. Table~\ref{tb:fac1_results} presents the findings. None of the baselines show a significant negative monotonic relationship (i.e., $\rho < 0$, $p < 0.05$), indicating no evidence to support the hypothesis that task difficulty is a key factor influencing the severity of \pb.


\subsubsection{Test Hypothesis II (Factor: Magnitude of Visual Modification)}
% \begin{table}[]
\centering
\caption{Results on the relationship between visual modification size and OSS severity.}
\label{tb:fac2_results}
\begin{tabular}{|c|c|c|c|}
\hline
                                                                 & BC-RESNET-RNN & BC-RESNET-T & BC-VIT-T \\ \hline
\begin{tabular}[c]{@{}c@{}}PD\\ (1 modification)\end{tabular} & 0.13          & 0.06        & 0.10     \\ \hline
\begin{tabular}[c]{@{}c@{}}PD\\ (2 modification)\end{tabular} & 0.25          & 0.24        & 0.13     \\ \hline
\begin{tabular}[c]{@{}c@{}}PD\\ (3 modification)\end{tabular} & 0.77          & 0.77        & 0.34     \\ \hline
\end{tabular}
\end{table}



We also use the metric ``Ratio Performance Delta'' to quantify the severity of \pb. To ensure that the magnitude of visual modification is the only variable changing, we adopt an approach similar to \bmb by increasing the number of modifications applied to the same skill. However, unlike \bmb, where modifications are generated randomly by RAMG, it is possible for skills with a larger number of modifications (e.g., affecting several small areas) to have a smaller visual modification magnitude than skills with fewer but more significant modifications (e.g., affecting a large area). To address this, we incrementally add new modifications based on existing ones, ensuring a clear correlation between the number of modifications and the size of the visual changes. After evaluating BCs on skills unaffected by \pb and skills with multiple modifications, we generate two lists for each algorithm: one representing the number of modifications (i.e., visual modification magnitude) and the other capturing the RPD (i.e., severity of \pb). Spearman's rank correlation coefficient ($\rho$) and the corresponding p-value are calculated to analyze the relationship between these two variables. For all BCs, we obtain results as $\rho = 1.0$ and $p = 0.0$, indicating a significant positive monotonic relationship between visual modification magnitude and the severity of \pb. These findings confirm that the magnitude of visual modifications is a key factor influencing the severity of \pb.