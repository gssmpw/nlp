\section{Conclusion}
% conclusion 1: \pb is a significant and common issue that could jarpardize the completion of long-horizon task. we use 3 benchmarks to prove this to show it's really an issue.
% conclusion 2: visual modification size affect \pb effect, task difficulty does NOT. This makes \pb a very severe issue especially when horizon becomes very long where later skills will accumulate many modifications.
% conclusion 3: data augmentation is not the simple solution, more specific methods are needed, such as task-relevant components attention mechanism.

This work introduces and investigates Observation Space Shift (\pb), a critical issue that hinders the completion of long-horizon robot tasks. Through three challenges in \bm, we validate the impact of \pb on several the performance of several popular baseline algorithms and present key findings:

\begin{itemize}
    \item \textbf{\pb is a common and severe problem}: Results on \bm demonstrate that \pb can substantially degrade task completion, especially in long-horizon tasks where visual modifications accumulate over time.
    \item \textbf{Visual modification magnitude matters, task difficulty may not}: Larger visual modifications lead to greater \pb severity, while task difficulty was not significantly correlated, making \pb particularly problematic as task horizons grow.
    % \item \textbf{Scalability is not sufficient}: Training on larger and more diverse datasets fails to effectively mitigate \pb and, in some cases, even degrades performance. This highlights the need for specific solutions, such as mechanisms that focus on visually task-relevant components and reduce sensitivity to irrelevant modifications. \gb{To me, this just says that your dataset is not large/diverse enough. Maybe a better way to rephrase is this is to point out the challenges of generating a sufficiently large/diverse training dataset needed to train models capable of handling OSS problem.}
    \item \textbf{Data augmentation may not be sufficient}: While increasing dataset size and diversity generally improves performance, we found that it did not fully mitigate \pb and may even degrade performance in some cases. A key challenge is that data augmentation alone cannot generate a dataset large and diverse enough to cover the vast combinatorial space of visually varying scene setups, which arise from the numerous skill compositions within long-horizon tasks. Furthermore, current baseline architectures are not explicitly designed to address \pb, highlighting the need for tailored algorithmic solutions, such as mechanisms that focus on task-relevant visual cues, and reduce sensitivity to irrelevant variations.
\end{itemize}

These findings highlight that effectively addressing \pb requires either more advanced methods for scaling the visual diversity of robotic data or innovative algorithmic designs that enhance robustness in long-horizon tasks. We hope our benchmarks and insights will inspire further research toward tackling this critical challenge.

