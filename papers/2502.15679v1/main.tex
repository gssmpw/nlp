\documentclass[lettersize,journal]{IEEEtran}
\usepackage{amsmath,amsfonts}
\usepackage{algorithmic}
\usepackage{algorithm}
\usepackage{array}
\usepackage[caption=false,font=normalsize,labelfont=sf,textfont=sf]{subfig}
\usepackage{textcomp}
\usepackage{stfloats}
\usepackage{url}
\usepackage{verbatim}
\usepackage{graphicx}
\usepackage{cite}

% pkgs added by Yue
\usepackage{xspace}
\usepackage{xcolor}
\usepackage{mwe}
\usepackage{booktabs}
\usepackage{multirow}
\usepackage{makecell} % For splitting cells
\usepackage{array}    % For centering columns vertically
\usepackage{diagbox}
% \usepackage{natbib}
\usepackage{enumerate}
\usepackage{microtype}

% todonotes
\usepackage[textsize=tiny]{todonotes}
% Set margin for todonotes
\setlength{\marginparwidth}{1.2cm}

\hyphenation{op-tical net-works semi-conduc-tor IEEE-Xplore}
% updated with editorial comments 8/9/2021

\begin{document}

\title{BOSS: Benchmark for Observation Space \\ Shift in Long-Horizon Task}

\author{
    Yue Yang\textsuperscript{1}, 
    Linfeng Zhao\textsuperscript{2}, 
    Mingyu Ding\textsuperscript{1}, 
    Gedas Bertasius\textsuperscript{1}, 
    Daniel Szafir\textsuperscript{1}%
    \thanks{This work has been submitted to the IEEE for possible publication. Copyright may be transferred without notice, after which this version may no longer be accessible.}
    \thanks{\textsuperscript{1} The University of North Carolina at Chapel Hill.}%
    \thanks{\textsuperscript{2} Northeastern University.}%
    \thanks{Contact Email: \texttt{yygx@cs.unc.edu}}%
}



% The paper headers
% \markboth{Journal of \LaTeX\ Class Files,~Vol.~14, No.~8, August~2021}%
% {Shell \MakeLowercase{\textit{et al.}}: A Sample Article Using IEEEtran.cls for IEEE Journals}

% \IEEEpubid{0000--0000/00\$00.00~\copyright~2021 IEEE}
% \IEEEpubid{...}
% Remember, if you use this you must call \IEEEpubidadjcol in the second
% column for its text to clear the IEEEpubid mark.



\definecolor{mygreen}{HTML}{00B050}
\definecolor{myred}{HTML}{FF5751}
\definecolor{myblue}{HTML}{4472c4}


\newcommand{\pb}{OSS\xspace}
\newcommand{\bm}{BOSS\xspace}
\newcommand{\bma}{BOSS-C1\xspace}
\newcommand{\bmb}{BOSS-C2\xspace}
\newcommand{\bmc}{BOSS-C3\xspace}
\newcommand{\bl}{BaselineName\xspace}
\newcommand{\taskoriginal}{\bm-44\xspace}

\newcommand{\my}[1]{\textcolor{red}{[Mingyu: #1]}}
\newcommand{\gb}[1]{{\textcolor{blue}{GB: #1}}}
\newcommand{\yy}[1]{{\textcolor{green}{(YY: #1)}}}
\newcommand{\ds}[1]{{\textcolor{yellow}{DS: #1}}}

% Todonote commands
\newcommand{\todoinline}[1]{\todo[inline, size=\small, color=orange!50]{#1}}
\newcommand{\todozlf}[2][]{\todo[color=yellow!80!black, #1]{LZ: #2}}
\newcommand{\notezlf}[1]{\todo[color=blue!50!white]{[LZ] #1}}
\newcommand{\todoilzlf}[2][]{\todo[inline, size=\small, color=yellow!80!black, #1]{LZ: #2}}
\newcommand{\todonote}[1]{\todo[color=blue!50!white]{#1}}
\newcommand{\todoilnote}[1]{\todo[inline, size=\small, color=blue!50!white]{#1}}



\maketitle



% Gedas's feedback:
% 1. switch lines in fig1, change bg color
% 2. use better examples for fig2, explain colors
% 3. modify fig5 largely
% 4. delete table 1 & 2
% 5. avoid the statement to challenge some reviews



% Discussion with Dan
% 1. Any more improvements on Fig 2?
% bottom figures, ch-2 as example, it's still successful - transparent + put some signs
% 2. Fig 4 is overwhelming, how to improve - simplify?
% dont make fig 3 and fig 4 next to each other - moving away. split it? or maybe delete the scatter plot.
% 3. Do we really hope to delete the whole section of VI if we could cut many parts by \vspace / simplify expressions? Will this make the contribution not enough?
% be attacked, cannot explain 

\begin{abstract}
% For general robot, it's desirable to let robot complete unseen complex long horizon tasks. Hirerachical IL where the robot is required to perform long horizon task companied with pre-trained skill set has proven to be a promising framework.

% However, even for the simplest long horizon task, which is a skill chain, the current skill policies cannot complete it well due to a problem identified by us as observation space modification (\pb).

% To verify the \pb problem, we evaluate on several popular and representative imitation learning algorithms on our benchmark sets \bm, containinig 3 benchmarks with increasing difficulties. Furthermore, we conduct experiments to obatain useful insights, which helps understand the problem better. To solve the challenging problem, we generate data in a scalable way and propose a simple baseline method, which provides much insights on how to solve this problem in the future. 


Robotics has long sought to develop visual-servoing robots capable of completing previously unseen long-horizon tasks. Hierarchical approaches offer a pathway for achieving this goal by executing skill combinations arranged by a task planner, with each visuomotor skill pre-trained using a specific imitation learning (IL) algorithm. However, even in simple long-horizon tasks like skill chaining, hierarchical approaches often struggle due to a problem we identify as \textbf{Observation Space Shift} (\pb), where the sequential execution of preceding skills causes shifts in the observation space, disrupting the performance of subsequent individually trained skill policies. To validate \pb and evaluate its impact on long-horizon tasks, we introduce \bm (a \underline{B}enchmark for \underline{O}bservation \underline{S}pace \underline{S}hift). \bm comprises three distinct challenges: ``Single Predicate Shift'', ``Accumulated Predicate Shift'', and ``Skill Chaining'', each designed to assess a different aspect of \pb's negative effect. We evaluated several recent popular IL algorithms on \bm, including three Behavioral Cloning methods and the Visual Language Action model OpenVLA. Even on the simplest challenge, we observed average performance drops of 67\%, 35\%, 34\%, and 54\%, respectively, when comparing skill performance with and without \pb. 
% Additionally, we explore two potential factors that may influence \pb: task difficulty and the magnitude of visual modifications. 
Additionally, we investigate a potential solution to \pb that scales up the training data for each skill with a larger and more visually diverse set of demonstrations, with our results showing it is not sufficient to resolve \pb.
%. %While this approach contributes to a large, diverse dataset and offers guidance for future research, 
The project page is: \texttt{https://boss-benchmark.github.io/}





% We evaluated state-of-the-art IL algorithms in this suite, \gb{Maybe you can emphasize that even the recent VLA-based large-scale methods struggle with this issue to make this more impactful/relevant} showing the negative impact of \pb on long-horizon task performance\gb{This is pretty vague. Would be good to cite specific numbers of how much the performance drops}. Additionally, we perform two sets of experiments to gain deeper insight into the problem.\gb{This sentence doesn't provide any useful information}. To address this challenging problem, we propose a scalable demonstration generation method and introduce a simple baseline based on the dataset generated via the proposed method\gb{too vague. Unclear at all what your method does}, offering valuable guidance for future research. \gb{Overall, the abstract is very vague and general. As a result the contribution seems limited and unclear. Would be good to rewrite to include more details + also somehow make it seem more exciting/impactful}



\end{abstract}

\begin{IEEEkeywords}
Imitation Learning, Long-Horizon Task, Robotics
\end{IEEEkeywords}

\vspace{-1.2em}

\section{Introduction}
\label{sec:introduction}
The business processes of organizations are experiencing ever-increasing complexity due to the large amount of data, high number of users, and high-tech devices involved \cite{martin2021pmopportunitieschallenges, beerepoot2023biggestbpmproblems}. This complexity may cause business processes to deviate from normal control flow due to unforeseen and disruptive anomalies \cite{adams2023proceddsriftdetection}. These control-flow anomalies manifest as unknown, skipped, and wrongly-ordered activities in the traces of event logs monitored from the execution of business processes \cite{ko2023adsystematicreview}. For the sake of clarity, let us consider an illustrative example of such anomalies. Figure \ref{FP_ANOMALIES} shows a so-called event log footprint, which captures the control flow relations of four activities of a hypothetical event log. In particular, this footprint captures the control-flow relations between activities \texttt{a}, \texttt{b}, \texttt{c} and \texttt{d}. These are the causal ($\rightarrow$) relation, concurrent ($\parallel$) relation, and other ($\#$) relations such as exclusivity or non-local dependency \cite{aalst2022pmhandbook}. In addition, on the right are six traces, of which five exhibit skipped, wrongly-ordered and unknown control-flow anomalies. For example, $\langle$\texttt{a b d}$\rangle$ has a skipped activity, which is \texttt{c}. Because of this skipped activity, the control-flow relation \texttt{b}$\,\#\,$\texttt{d} is violated, since \texttt{d} directly follows \texttt{b} in the anomalous trace.
\begin{figure}[!t]
\centering
\includegraphics[width=0.9\columnwidth]{images/FP_ANOMALIES.png}
\caption{An example event log footprint with six traces, of which five exhibit control-flow anomalies.}
\label{FP_ANOMALIES}
\end{figure}

\subsection{Control-flow anomaly detection}
Control-flow anomaly detection techniques aim to characterize the normal control flow from event logs and verify whether these deviations occur in new event logs \cite{ko2023adsystematicreview}. To develop control-flow anomaly detection techniques, \revision{process mining} has seen widespread adoption owing to process discovery and \revision{conformance checking}. On the one hand, process discovery is a set of algorithms that encode control-flow relations as a set of model elements and constraints according to a given modeling formalism \cite{aalst2022pmhandbook}; hereafter, we refer to the Petri net, a widespread modeling formalism. On the other hand, \revision{conformance checking} is an explainable set of algorithms that allows linking any deviations with the reference Petri net and providing the fitness measure, namely a measure of how much the Petri net fits the new event log \cite{aalst2022pmhandbook}. Many control-flow anomaly detection techniques based on \revision{conformance checking} (hereafter, \revision{conformance checking}-based techniques) use the fitness measure to determine whether an event log is anomalous \cite{bezerra2009pmad, bezerra2013adlogspais, myers2018icsadpm, pecchia2020applicationfailuresanalysispm}. 

The scientific literature also includes many \revision{conformance checking}-independent techniques for control-flow anomaly detection that combine specific types of trace encodings with machine/deep learning \cite{ko2023adsystematicreview, tavares2023pmtraceencoding}. Whereas these techniques are very effective, their explainability is challenging due to both the type of trace encoding employed and the machine/deep learning model used \cite{rawal2022trustworthyaiadvances,li2023explainablead}. Hence, in the following, we focus on the shortcomings of \revision{conformance checking}-based techniques to investigate whether it is possible to support the development of competitive control-flow anomaly detection techniques while maintaining the explainable nature of \revision{conformance checking}.
\begin{figure}[!t]
\centering
\includegraphics[width=\columnwidth]{images/HIGH_LEVEL_VIEW.png}
\caption{A high-level view of the proposed framework for combining \revision{process mining}-based feature extraction with dimensionality reduction for control-flow anomaly detection.}
\label{HIGH_LEVEL_VIEW}
\end{figure}

\subsection{Shortcomings of \revision{conformance checking}-based techniques}
Unfortunately, the detection effectiveness of \revision{conformance checking}-based techniques is affected by noisy data and low-quality Petri nets, which may be due to human errors in the modeling process or representational bias of process discovery algorithms \cite{bezerra2013adlogspais, pecchia2020applicationfailuresanalysispm, aalst2016pm}. Specifically, on the one hand, noisy data may introduce infrequent and deceptive control-flow relations that may result in inconsistent fitness measures, whereas, on the other hand, checking event logs against a low-quality Petri net could lead to an unreliable distribution of fitness measures. Nonetheless, such Petri nets can still be used as references to obtain insightful information for \revision{process mining}-based feature extraction, supporting the development of competitive and explainable \revision{conformance checking}-based techniques for control-flow anomaly detection despite the problems above. For example, a few works outline that token-based \revision{conformance checking} can be used for \revision{process mining}-based feature extraction to build tabular data and develop effective \revision{conformance checking}-based techniques for control-flow anomaly detection \cite{singh2022lapmsh, debenedictis2023dtadiiot}. However, to the best of our knowledge, the scientific literature lacks a structured proposal for \revision{process mining}-based feature extraction using the state-of-the-art \revision{conformance checking} variant, namely alignment-based \revision{conformance checking}.

\subsection{Contributions}
We propose a novel \revision{process mining}-based feature extraction approach with alignment-based \revision{conformance checking}. This variant aligns the deviating control flow with a reference Petri net; the resulting alignment can be inspected to extract additional statistics such as the number of times a given activity caused mismatches \cite{aalst2022pmhandbook}. We integrate this approach into a flexible and explainable framework for developing techniques for control-flow anomaly detection. The framework combines \revision{process mining}-based feature extraction and dimensionality reduction to handle high-dimensional feature sets, achieve detection effectiveness, and support explainability. Notably, in addition to our proposed \revision{process mining}-based feature extraction approach, the framework allows employing other approaches, enabling a fair comparison of multiple \revision{conformance checking}-based and \revision{conformance checking}-independent techniques for control-flow anomaly detection. Figure \ref{HIGH_LEVEL_VIEW} shows a high-level view of the framework. Business processes are monitored, and event logs obtained from the database of information systems. Subsequently, \revision{process mining}-based feature extraction is applied to these event logs and tabular data input to dimensionality reduction to identify control-flow anomalies. We apply several \revision{conformance checking}-based and \revision{conformance checking}-independent framework techniques to publicly available datasets, simulated data of a case study from railways, and real-world data of a case study from healthcare. We show that the framework techniques implementing our approach outperform the baseline \revision{conformance checking}-based techniques while maintaining the explainable nature of \revision{conformance checking}.

In summary, the contributions of this paper are as follows.
\begin{itemize}
    \item{
        A novel \revision{process mining}-based feature extraction approach to support the development of competitive and explainable \revision{conformance checking}-based techniques for control-flow anomaly detection.
    }
    \item{
        A flexible and explainable framework for developing techniques for control-flow anomaly detection using \revision{process mining}-based feature extraction and dimensionality reduction.
    }
    \item{
        Application to synthetic and real-world datasets of several \revision{conformance checking}-based and \revision{conformance checking}-independent framework techniques, evaluating their detection effectiveness and explainability.
    }
\end{itemize}

The rest of the paper is organized as follows.
\begin{itemize}
    \item Section \ref{sec:related_work} reviews the existing techniques for control-flow anomaly detection, categorizing them into \revision{conformance checking}-based and \revision{conformance checking}-independent techniques.
    \item Section \ref{sec:abccfe} provides the preliminaries of \revision{process mining} to establish the notation used throughout the paper, and delves into the details of the proposed \revision{process mining}-based feature extraction approach with alignment-based \revision{conformance checking}.
    \item Section \ref{sec:framework} describes the framework for developing \revision{conformance checking}-based and \revision{conformance checking}-independent techniques for control-flow anomaly detection that combine \revision{process mining}-based feature extraction and dimensionality reduction.
    \item Section \ref{sec:evaluation} presents the experiments conducted with multiple framework and baseline techniques using data from publicly available datasets and case studies.
    \item Section \ref{sec:conclusions} draws the conclusions and presents future work.
\end{itemize}
\section{RELATED WORK}
\label{sec:relatedwork}
In this section, we describe the previous works related to our proposal, which are divided into two parts. In Section~\ref{sec:relatedwork_exoplanet}, we present a review of approaches based on machine learning techniques for the detection of planetary transit signals. Section~\ref{sec:relatedwork_attention} provides an account of the approaches based on attention mechanisms applied in Astronomy.\par

\subsection{Exoplanet detection}
\label{sec:relatedwork_exoplanet}
Machine learning methods have achieved great performance for the automatic selection of exoplanet transit signals. One of the earliest applications of machine learning is a model named Autovetter \citep{MCcauliff}, which is a random forest (RF) model based on characteristics derived from Kepler pipeline statistics to classify exoplanet and false positive signals. Then, other studies emerged that also used supervised learning. \cite{mislis2016sidra} also used a RF, but unlike the work by \citet{MCcauliff}, they used simulated light curves and a box least square \citep[BLS;][]{kovacs2002box}-based periodogram to search for transiting exoplanets. \citet{thompson2015machine} proposed a k-nearest neighbors model for Kepler data to determine if a given signal has similarity to known transits. Unsupervised learning techniques were also applied, such as self-organizing maps (SOM), proposed \citet{armstrong2016transit}; which implements an architecture to segment similar light curves. In the same way, \citet{armstrong2018automatic} developed a combination of supervised and unsupervised learning, including RF and SOM models. In general, these approaches require a previous phase of feature engineering for each light curve. \par

%DL is a modern data-driven technology that automatically extracts characteristics, and that has been successful in classification problems from a variety of application domains. The architecture relies on several layers of NNs of simple interconnected units and uses layers to build increasingly complex and useful features by means of linear and non-linear transformation. This family of models is capable of generating increasingly high-level representations \citep{lecun2015deep}.

The application of DL for exoplanetary signal detection has evolved rapidly in recent years and has become very popular in planetary science.  \citet{pearson2018} and \citet{zucker2018shallow} developed CNN-based algorithms that learn from synthetic data to search for exoplanets. Perhaps one of the most successful applications of the DL models in transit detection was that of \citet{Shallue_2018}; who, in collaboration with Google, proposed a CNN named AstroNet that recognizes exoplanet signals in real data from Kepler. AstroNet uses the training set of labelled TCEs from the Autovetter planet candidate catalog of Q1–Q17 data release 24 (DR24) of the Kepler mission \citep{catanzarite2015autovetter}. AstroNet analyses the data in two views: a ``global view'', and ``local view'' \citep{Shallue_2018}. \par


% The global view shows the characteristics of the light curve over an orbital period, and a local view shows the moment at occurring the transit in detail

%different = space-based

Based on AstroNet, researchers have modified the original AstroNet model to rank candidates from different surveys, specifically for Kepler and TESS missions. \citet{ansdell2018scientific} developed a CNN trained on Kepler data, and included for the first time the information on the centroids, showing that the model improves performance considerably. Then, \citet{osborn2020rapid} and \citet{yu2019identifying} also included the centroids information, but in addition, \citet{osborn2020rapid} included information of the stellar and transit parameters. Finally, \citet{rao2021nigraha} proposed a pipeline that includes a new ``half-phase'' view of the transit signal. This half-phase view represents a transit view with a different time and phase. The purpose of this view is to recover any possible secondary eclipse (the object hiding behind the disk of the primary star).


%last pipeline applies a procedure after the prediction of the model to obtain new candidates, this process is carried out through a series of steps that include the evaluation with Discovery and Validation of Exoplanets (DAVE) \citet{kostov2019discovery} that was adapted for the TESS telescope.\par
%



\subsection{Attention mechanisms in astronomy}
\label{sec:relatedwork_attention}
Despite the remarkable success of attention mechanisms in sequential data, few papers have exploited their advantages in astronomy. In particular, there are no models based on attention mechanisms for detecting planets. Below we present a summary of the main applications of this modeling approach to astronomy, based on two points of view; performance and interpretability of the model.\par
%Attention mechanisms have not yet been explored in all sub-areas of astronomy. However, recent works show a successful application of the mechanism.
%performance

The application of attention mechanisms has shown improvements in the performance of some regression and classification tasks compared to previous approaches. One of the first implementations of the attention mechanism was to find gravitational lenses proposed by \citet{thuruthipilly2021finding}. They designed 21 self-attention-based encoder models, where each model was trained separately with 18,000 simulated images, demonstrating that the model based on the Transformer has a better performance and uses fewer trainable parameters compared to CNN. A novel application was proposed by \citet{lin2021galaxy} for the morphological classification of galaxies, who used an architecture derived from the Transformer, named Vision Transformer (VIT) \citep{dosovitskiy2020image}. \citet{lin2021galaxy} demonstrated competitive results compared to CNNs. Another application with successful results was proposed by \citet{zerveas2021transformer}; which first proposed a transformer-based framework for learning unsupervised representations of multivariate time series. Their methodology takes advantage of unlabeled data to train an encoder and extract dense vector representations of time series. Subsequently, they evaluate the model for regression and classification tasks, demonstrating better performance than other state-of-the-art supervised methods, even with data sets with limited samples.

%interpretation
Regarding the interpretability of the model, a recent contribution that analyses the attention maps was presented by \citet{bowles20212}, which explored the use of group-equivariant self-attention for radio astronomy classification. Compared to other approaches, this model analysed the attention maps of the predictions and showed that the mechanism extracts the brightest spots and jets of the radio source more clearly. This indicates that attention maps for prediction interpretation could help experts see patterns that the human eye often misses. \par

In the field of variable stars, \citet{allam2021paying} employed the mechanism for classifying multivariate time series in variable stars. And additionally, \citet{allam2021paying} showed that the activation weights are accommodated according to the variation in brightness of the star, achieving a more interpretable model. And finally, related to the TESS telescope, \citet{morvan2022don} proposed a model that removes the noise from the light curves through the distribution of attention weights. \citet{morvan2022don} showed that the use of the attention mechanism is excellent for removing noise and outliers in time series datasets compared with other approaches. In addition, the use of attention maps allowed them to show the representations learned from the model. \par

Recent attention mechanism approaches in astronomy demonstrate comparable results with earlier approaches, such as CNNs. At the same time, they offer interpretability of their results, which allows a post-prediction analysis. \par


\subsection{Problem Definition and Representation}
\label{sec:risk_aware_path_planning_problem_definition}

We generalize the infrastructure inspection path planning scenario as a 4D problem, which we define as $\mathbb{D}^{4} = \{ (x, y, z, v) \mid x \in [x_{min}, x_{max}], y \in [y_{min}, y_{max}], z \in [z_{min}, z_{max}], v \in [-v_{max}, v_{max}] \}$ with start and goal positions $\{(x_{a}, y_{a}, z_{a}, v_{a}), (x_{b}, y_{b}, z_{b}, v_{b})\} \in \mathbb{D}^{4}$. The goal of this MOPP problem is to find the NURBS curve $C = \{C(u): u \in [a, b]\}$ that minimizes the set of objectives $\mathcal{F}$ described below.


We choose to represent trajectories using NURBS \cite{piegl1995}, described in Section \ref{sec:nurbs}, to benefit from their inherent properties for multidimensional multi-objective optimization. These include \textit{strong convex hull} properties, \textit{local approximation} capability, and \textit{infinite differentiability} apart from knot multiplicity \cite{piegl1995}. The first two properties are helpful during optimization to constrain the curve inside a bounded area and to escape local minima by varying the curve locally. The last property ensures the smoothness of the curve according to the knot multiplicity of the curve, which is beneficial for UAVs that have strict actuator limitations, such as low acceleration. The design vector is given by:
\begin{equation}
    \label{eq:design_vector}
    \begin{aligned}
        z ={} & [w_{0} \ \ x_{1} \ \ y_{1} \ \ z_{1} \ \left \| \vec{v}_{1} \right \| \ w_{1}\\
              &\ \ \ \ \ \ \ \ \ \ \ \ \ \ \ ...\\
              &x_{n-1} \ \ y_{n-1} \ \ z_{n-1} \ \left \| \vec{v}_{n-1} \right \| \ w_{n-1} \ \ w_{n}]^{T}
    \end{aligned}
\end{equation}


We introduce the 4$^{th}$ dimension to the parametric curve $C$, representing the velocity norm at each control point, enabling velocity profile modulation along trajectories as risks evolve. The start and goal positions, along with their speeds, are fixed and excluded from the decision vector $z$. The problem is constrained by two hard limits: acceleration ($a_{max}$ in Table \ref{table:parameters_table}) to meet actuator limitations and collision avoidance to ensure feasibility.


Costs are generalized into three functions to encompass multiple scenarios: time, safety, and energy consumption, which are described in the following subsections.

%%%% TIME COST %%%%
\hfill
\subsubsection{Time cost}
\label{sec:time_cost}
The time cost is computed as:

\begin{equation}
    \label{eq:time_cost}
    \mathcal{F}_{Time} = \sum_{i=0}^{Q-1} \frac{d_{i}}{ \left \| \vec{v}_{i + 1} \right \| }
\end{equation}
where $Q$ is the number of sample points and $d_{i}$ is the distance on the path segment. The $i + 1$ velocity is used, assuming inspection drones are slow and can quickly reach low speeds. Under normal conditions, with no significant risks during an inspection mission, the algorithm minimizes this cost function, prioritizing speeding up the inspection process over safety or energy consumption.

%%%% SAFETY COST %%%%
\hfill
\subsubsection{Safety cost}
\label{sec:safety_cost}
The safety cost consists of two terms:

\begin{equation}
    \label{eq:safety_cost}
    \begin{aligned}
        \mathcal{F}_{Safety}={} & k_{a} (\frac{\sum_{i=0}^{Q} \mathcal{F}_{sdf_{i}}}{Q} + max(\mathcal{F}_{sdf})) \\ & + k_{b} (\frac{\sum_{i=0}^{Q} \mathcal{F}_{ch_{i}}}{Q} + max(\mathcal{F}_{ch}))
    \end{aligned}
\end{equation}
where $\mathcal{F}_{sdf_{i}}$ is a collision cost to ensure a safe UAV-obstacle distance and $\mathcal{F}{ch_{i}}$ denotes a non-insertion cost for obstacles forming convex hulls to keep trajectories outside critical zones. In this letter, convex hulls, modeled as oriented bounding boxes (OBB), are dynamically adjusted around cables and pylons using semantic information during power line inspections. The coefficients $k_{a}$ and $k_{b}$ in Eq. \ref{eq:safety_cost} are normalized since $\mathcal{F}{sdf_{i}}$ and $\mathcal{F}{ch_{i}}$ return normalized values. Combining mean and maximum costs ensures overall trajectory safety while maximizing the minimum obstacle distance. To support this, we use a signed distance field (SDF) \cite{jones2006} alongside our path planning algorithm to maintain obstacle information within the free space. $\mathcal{F}{sdf_{i}}$ is defined using the closest obstacle distance in the SDF:

\begin{equation}
    \label{eq:sdf_cost}
    \mathcal{F}_{sdf_{i}} = 
    \begin{cases} 
        0 & d_{obs_{i}} \geq r_{sdf_{max}} \\
        \frac{\lambda}{d_{obs_{i}}} - 1 & r_{sdf_{min}} < d_{obs_{i}} < r_{sdf_{max}} \\
        1 & d_{obs_{i}} \leq r_{sdf_{min}}
    \end{cases}
\end{equation}
where $\lambda = \frac{r_{sdf_{min}} r_{sdf_{max}}}{r_{sdf_{max}} - r_{sdf_{min}}}$ is a scaling factor and $d_{obs_{i}}$ is the distance between the UAV and the nearest obstacle. Both $r_{sdf_{min}}$ and $r_{sdf_{max}}$ are listed in Table \ref{table:parameters_table}. This cost function reflects the increasing risks as the UAV gets closer to obstacles, which rapidly increases near them. The non-insertion cost function $\mathcal{F}_{ch_{i}}$ is formulated as follows:

\begin{equation}
    \label{eq:convex_hull_cost}
    \mathcal{F}_{ch_{i}} = \sum_{i=0}^{N}
    \begin{cases}
        0 & d_{ch_{i}} \geq r_{ch_{max}} \\
        1 - \frac{d_{ch_{i}}}{r_{ch_{max}}} & 0 < d_{ch_{i}} < r_{ch_{max}} \\
        1 & d_{ch_{i}} \leq 0
    \end{cases}
\end{equation}
where $d_{ch_{i}}$ is the distance to the nearest convex hull and $r_{ch_{max}}$ is the maximum influence radius of convex hulls. Under high wind, communication, or localization risks, the algorithm prioritizes safety by selecting trajectories that minimize this cost function over time or energy efficiency.

%%%% ENERGY COST %%%%
\hfill
\subsubsection{Energy cost}
\label{sec:energy_cost}
To evaluate the energy consumption of a planned trajectory, our approach integrates physics-based principles with experimental data. \cite{guoku2022} proposed a model for quadrotor UAVs with BLDC motors, showing that energy consumption varies with the flight state. According to this model, it can be hypothesized that for a given flight speed, horizontal movements require more power than hovering, with vertical movements further affecting it. The pitch/roll power relationship with the vertical axis forms a quadric surface \cite{hilbert1999}, described by a general equation:


\begin{equation}
    \label{eq:quadric_surface}
    \begin{aligned}
        {} & ax^{2}+by^{2}+cz^{2}+dxy+eyz+fxz\\
        & \ \ \ +gx+hy+kz+L = 0 
    \end{aligned}
\end{equation}

Because of a quadrotor's symmetry and to simplify the model, coupling terms are eliminated. The term $L$ acts as a geometric translation factor and is fixed so the system doesn't return the trivial solution. With these simplifications, power data from minimally the six flight directions ($\pm Z, \pm Roll, \pm Pitch$) creates a solvable system of equations. To locate a point on this surface during optimization, we project along a unit vector $\hat{v} = (v_{x}, v_{y}, v_{z})$, parameterized by $t$: $x = t v_{x}, y = t v_{y}, z = t v_{z}$. Substituting into Eq. \ref{eq:quadric_surface} simplifies to:

\begin{equation}
    \label{eq:simplified_quadric_equation}
    At^{2} + Bt + L = 0
\end{equation}
where $A = av_{x}^{2} + bv_{y}^{2} + cv_{z}^{2}$, $B = gv_{x} + hv_{y} + kv_{z}$, and $L = 1$. The roots of this equation, solved using the quadratic formula, determine the points on the quadric surface along the unit vector, providing steady-state power consumption $P(\hat{v})$ at these coordinates. The energy consumption cost function is given by:

\begin{equation}
    \label{eq:energy_cost}
    \mathcal{F}_{Energy} = \sum_{i=1}^{Q} P(\hat{v}_{i}) \Delta t
\end{equation}

Since inspection drones are slow and quickly reach cruising speed, we assume steady-state flight energy consumption remains constant, with transient states contributing minimally. Unlike the time cost function, the energy cost function includes a directional factor influencing the optimizer. In low battery conditions, the algorithm prioritizes energy and time over safety.

% % YY: Old flow
% \begin{table}[t]
    \centering
    \caption{The performance of different pre-trained models on ImageNet and infrared semantic segmentation datasets. The \textit{Scratch} means the performance of randomly initialized models. The \textit{PT Epochs} denotes the pre-training epochs while the \textit{IN1K FT epochs} represents the fine-tuning epochs on ImageNet \citep{imagenet}. $^\dag$ denotes models reproduced using official codes. $^\star$ refers to the effective epochs used in \citet{iBOT}. The top two results are marked in \textbf{bold} and \underline{underlined} format. Supervised and CL methods, MIM methods, and UNIP models are colored in \colorbox{orange!15}{\rule[-0.2ex]{0pt}{1.5ex}orange}, \colorbox{gray!15}{\rule[-0.2ex]{0pt}{1.5ex}gray}, and \colorbox{cyan!15}{\rule[-0.2ex]{0pt}{1.5ex}cyan}, respectively.}
    \label{tab:benchmark}
    \centering
    \scriptsize
    \setlength{\tabcolsep}{1.0mm}{
    \scalebox{1.0}{
    \begin{tabular}{l c c c c  c c c c c c c c}
        \toprule
         \multirow{2}{*}{Methods} & \multirow{2}{*}{\makecell[c]{PT \\ Epochs}} & \multicolumn{2}{c}{IN1K FT} & \multicolumn{4}{c}{Fine-tuning (FT)} & \multicolumn{4}{c}{Linear Probing (LP)} \\
         \cmidrule{3-4} \cmidrule(lr){5-8} \cmidrule(lr){9-12} 
         & & Epochs & Acc & SODA & MFNet-T & SCUT-Seg & Mean & SODA & MFNet-T & SCUT-Seg & Mean \\
         \midrule
         \textcolor{gray}{ViT-Tiny/16} & & &  & & & & & & & & \\
         Scratch & - & - & - & 31.34 & 19.50 & 41.09 & 30.64 & - & - & - & - \\
         \rowcolor{gray!15} MAE$^\dag$ \citep{mae} & 800 & 200 & \underline{71.8} & 52.85 & 35.93 & 51.31 & 46.70 & 23.75 & 15.79 & 27.18 & 22.24 \\
         \rowcolor{orange!15} DeiT \citep{deit} & 300 & - & \textbf{72.2} & 63.14 & 44.60 & 61.36 & 56.37 & 42.29 & 21.78 & 31.96 & 32.01 \\
         \rowcolor{cyan!15} UNIP (MAE-L) & 100 & - & - & \underline{64.83} & \textbf{48.77} & \underline{67.22} & \underline{60.27} & \underline{44.12} & \underline{28.26} & \underline{35.09} & \underline{35.82} \\
         \rowcolor{cyan!15} UNIP (iBOT-L) & 100 & - & - & \textbf{65.54} & \underline{48.45} & \textbf{67.73} & \textbf{60.57} & \textbf{52.95} & \textbf{30.10} & \textbf{40.12} & \textbf{41.06}  \\
         \midrule
         \textcolor{gray}{ViT-Small/16} & & & & & & & & & & & \\
         Scratch & - & - & - & 41.70 & 22.49 & 46.28 & 36.82 & - & - & - & - \\
         \rowcolor{gray!15} MAE$^\dag$ \citep{mae} & 800 & 200 & 80.0 & 63.36 & 42.44 & 60.38 & 55.39 & 38.17 & 21.14 & 34.15 & 31.15 \\
         \rowcolor{gray!15} CrossMAE \citep{crossmae} & 800 & 200 & 80.5 & 63.95 & 43.99 & 63.53 & 57.16 & 39.40 & 23.87 & 34.01 & 32.43 \\
         \rowcolor{orange!15} DeiT \citep{deit} & 300 & - & 79.9 & 68.08 & 45.91 & 66.17 & 60.05 & 44.88 & 28.53 & 38.92 & 37.44 \\
         \rowcolor{orange!15} DeiT III \citep{deit3} & 800 & - & 81.4 & 69.35 & 47.73 & 67.32 & 61.47 & 54.17 & 32.01 & 43.54 & 43.24 \\
         \rowcolor{orange!15} DINO \citep{dino} & 3200$^\star$ & 200 & \underline{82.0} & 68.56 & 47.98 & 68.74 & 61.76 & 56.02 & 32.94 & 45.94 & 44.97 \\
         \rowcolor{orange!15} iBOT \citep{iBOT} & 3200$^\star$ & 200 & \textbf{82.3} & 69.33 & 47.15 & 69.80 & 62.09 & 57.10 & 33.87 & 45.82 & 45.60 \\
         \rowcolor{cyan!15} UNIP (DINO-B) & 100 & - & - & 69.35 & 49.95 & 69.70 & 63.00 & \underline{57.76} & \underline{34.15} & \underline{46.37} & \underline{46.09} \\
         \rowcolor{cyan!15} UNIP (MAE-L) & 100 & - & - & \textbf{70.99} & \underline{51.32} & \underline{70.79} & \underline{64.37} & 55.25 & 33.49 & 43.37 & 44.04 \\
         \rowcolor{cyan!15} UNIP (iBOT-L) & 100 & - & - & \underline{70.75} & \textbf{51.81} & \textbf{71.55} & \textbf{64.70} & \textbf{60.28} & \textbf{37.16} & \textbf{47.68} & \textbf{48.37} \\ 
        \midrule
        \textcolor{gray}{ViT-Base/16} & & & & & & & & & & & \\
        Scratch & - & - & - & 44.25 & 23.72 & 49.44 & 39.14 & - & - & - & - \\
        \rowcolor{gray!15} MAE \citep{mae} & 1600 & 100 & 83.6 & 68.18 & 46.78 & 67.86 & 60.94 & 43.01 & 23.42 & 37.48 & 34.64 \\
        \rowcolor{gray!15} CrossMAE \citep{crossmae} & 800 & 100 & 83.7 & 68.29 & 47.85 & 68.39 & 61.51 & 43.35 & 26.03 & 38.36 & 35.91 \\
        \rowcolor{orange!15} DeiT \citep{deit} & 300 & - & 81.8 & 69.73 & 48.59 & 69.35 & 62.56 & 57.40 & 34.82 & 46.44 & 46.22 \\
        \rowcolor{orange!15} DeiT III \citep{deit3} & 800 & 20 & \underline{83.8} & 71.09 & 49.62 & 70.19 & 63.63 & 59.01 & \underline{35.34} & 48.01 & 47.45 \\
        \rowcolor{orange!15} DINO \citep{dino} & 1600$^\star$ & 100 & 83.6 & 69.79 & 48.54 & 69.82 & 62.72 & 59.33 & 34.86 & 47.23 & 47.14 \\
        \rowcolor{orange!15} iBOT \citep{iBOT} & 1600$^\star$ & 100 & \textbf{84.0} & 71.15 & 48.98 & 71.26 & 63.80 & \underline{60.05} & 34.34 & \underline{49.12} & \underline{47.84} \\
        \rowcolor{cyan!15} UNIP (MAE-L) & 100 & - & - & \underline{71.47} & \textbf{52.55} & \underline{71.82} & \textbf{65.28} & 58.82 & 34.75 & 48.74 & 47.43 \\
        \rowcolor{cyan!15} UNIP (iBOT-L) & 100 & - & - & \textbf{71.75} & \underline{51.46} & \textbf{72.00} & \underline{65.07} & \textbf{63.14} & \textbf{39.08} & \textbf{52.53} & \textbf{51.58} \\
        \midrule
        \textcolor{gray}{ViT-Large/16} & & & & & & & & & & & \\
        Scratch & - & - & - & 44.70 & 23.68 & 49.55 & 39.31 & - & - & - & - \\
        \rowcolor{gray!15} MAE \citep{mae} & 1600 & 50 & \textbf{85.9} & 71.04 & \underline{51.17} & 70.83 & 64.35 & 52.20 & 31.21 & 43.71 & 42.37 \\
        \rowcolor{gray!15} CrossMAE \citep{crossmae} & 800 & 50 & 85.4 & 70.48 & 50.97 & 70.24 & 63.90 & 53.29 & 33.09 & 45.01 & 43.80 \\
        \rowcolor{orange!15} DeiT3 \citep{deit3} & 800 & 20 & \underline{84.9} & \underline{71.67} & 50.78 & \textbf{71.54} & \underline{64.66} & \underline{59.42} & \textbf{37.57} & \textbf{50.27} & \underline{49.09} \\
        \rowcolor{orange!15} iBOT \citep{iBOT} & 1000$^\star$ & 50 & 84.8 & \textbf{71.75} & \textbf{51.66} & \underline{71.49} & \textbf{64.97} & \textbf{61.73} & \underline{36.68} & \underline{50.12} & \textbf{49.51} \\
        \bottomrule
    \end{tabular}}}
    \vspace{-2mm}
\end{table}
% \section{Experiments}
\label{sec:experiments}
The experiments are designed to address two key research questions.
First, \textbf{RQ1} evaluates whether the average $L_2$-norm of the counterfactual perturbation vectors ($\overline{||\perturb||}$) decreases as the model overfits the data, thereby providing further empirical validation for our hypothesis.
Second, \textbf{RQ2} evaluates the ability of the proposed counterfactual regularized loss, as defined in (\ref{eq:regularized_loss2}), to mitigate overfitting when compared to existing regularization techniques.

% The experiments are designed to address three key research questions. First, \textbf{RQ1} investigates whether the mean perturbation vector norm decreases as the model overfits the data, aiming to further validate our intuition. Second, \textbf{RQ2} explores whether the mean perturbation vector norm can be effectively leveraged as a regularization term during training, offering insights into its potential role in mitigating overfitting. Finally, \textbf{RQ3} examines whether our counterfactual regularizer enables the model to achieve superior performance compared to existing regularization methods, thus highlighting its practical advantage.

\subsection{Experimental Setup}
\textbf{\textit{Datasets, Models, and Tasks.}}
The experiments are conducted on three datasets: \textit{Water Potability}~\cite{kadiwal2020waterpotability}, \textit{Phomene}~\cite{phomene}, and \textit{CIFAR-10}~\cite{krizhevsky2009learning}. For \textit{Water Potability} and \textit{Phomene}, we randomly select $80\%$ of the samples for the training set, and the remaining $20\%$ for the test set, \textit{CIFAR-10} comes already split. Furthermore, we consider the following models: Logistic Regression, Multi-Layer Perceptron (MLP) with 100 and 30 neurons on each hidden layer, and PreactResNet-18~\cite{he2016cvecvv} as a Convolutional Neural Network (CNN) architecture.
We focus on binary classification tasks and leave the extension to multiclass scenarios for future work. However, for datasets that are inherently multiclass, we transform the problem into a binary classification task by selecting two classes, aligning with our assumption.

\smallskip
\noindent\textbf{\textit{Evaluation Measures.}} To characterize the degree of overfitting, we use the test loss, as it serves as a reliable indicator of the model's generalization capability to unseen data. Additionally, we evaluate the predictive performance of each model using the test accuracy.

\smallskip
\noindent\textbf{\textit{Baselines.}} We compare CF-Reg with the following regularization techniques: L1 (``Lasso''), L2 (``Ridge''), and Dropout.

\smallskip
\noindent\textbf{\textit{Configurations.}}
For each model, we adopt specific configurations as follows.
\begin{itemize}
\item \textit{Logistic Regression:} To induce overfitting in the model, we artificially increase the dimensionality of the data beyond the number of training samples by applying a polynomial feature expansion. This approach ensures that the model has enough capacity to overfit the training data, allowing us to analyze the impact of our counterfactual regularizer. The degree of the polynomial is chosen as the smallest degree that makes the number of features greater than the number of data.
\item \textit{Neural Networks (MLP and CNN):} To take advantage of the closed-form solution for computing the optimal perturbation vector as defined in (\ref{eq:opt-delta}), we use a local linear approximation of the neural network models. Hence, given an instance $\inst_i$, we consider the (optimal) counterfactual not with respect to $\model$ but with respect to:
\begin{equation}
\label{eq:taylor}
    \model^{lin}(\inst) = \model(\inst_i) + \nabla_{\inst}\model(\inst_i)(\inst - \inst_i),
\end{equation}
where $\model^{lin}$ represents the first-order Taylor approximation of $\model$ at $\inst_i$.
Note that this step is unnecessary for Logistic Regression, as it is inherently a linear model.
\end{itemize}

\smallskip
\noindent \textbf{\textit{Implementation Details.}} We run all experiments on a machine equipped with an AMD Ryzen 9 7900 12-Core Processor and an NVIDIA GeForce RTX 4090 GPU. Our implementation is based on the PyTorch Lightning framework. We use stochastic gradient descent as the optimizer with a learning rate of $\eta = 0.001$ and no weight decay. We use a batch size of $128$. The training and test steps are conducted for $6000$ epochs on the \textit{Water Potability} and \textit{Phoneme} datasets, while for the \textit{CIFAR-10} dataset, they are performed for $200$ epochs.
Finally, the contribution $w_i^{\varepsilon}$ of each training point $\inst_i$ is uniformly set as $w_i^{\varepsilon} = 1~\forall i\in \{1,\ldots,m\}$.

The source code implementation for our experiments is available at the following GitHub repository: \url{https://anonymous.4open.science/r/COCE-80B4/README.md} 

\subsection{RQ1: Counterfactual Perturbation vs. Overfitting}
To address \textbf{RQ1}, we analyze the relationship between the test loss and the average $L_2$-norm of the counterfactual perturbation vectors ($\overline{||\perturb||}$) over training epochs.

In particular, Figure~\ref{fig:delta_loss_epochs} depicts the evolution of $\overline{||\perturb||}$ alongside the test loss for an MLP trained \textit{without} regularization on the \textit{Water Potability} dataset. 
\begin{figure}[ht]
    \centering
    \includegraphics[width=0.85\linewidth]{img/delta_loss_epochs.png}
    \caption{The average counterfactual perturbation vector $\overline{||\perturb||}$ (left $y$-axis) and the cross-entropy test loss (right $y$-axis) over training epochs ($x$-axis) for an MLP trained on the \textit{Water Potability} dataset \textit{without} regularization.}
    \label{fig:delta_loss_epochs}
\end{figure}

The plot shows a clear trend as the model starts to overfit the data (evidenced by an increase in test loss). 
Notably, $\overline{||\perturb||}$ begins to decrease, which aligns with the hypothesis that the average distance to the optimal counterfactual example gets smaller as the model's decision boundary becomes increasingly adherent to the training data.

It is worth noting that this trend is heavily influenced by the choice of the counterfactual generator model. In particular, the relationship between $\overline{||\perturb||}$ and the degree of overfitting may become even more pronounced when leveraging more accurate counterfactual generators. However, these models often come at the cost of higher computational complexity, and their exploration is left to future work.

Nonetheless, we expect that $\overline{||\perturb||}$ will eventually stabilize at a plateau, as the average $L_2$-norm of the optimal counterfactual perturbations cannot vanish to zero.

% Additionally, the choice of employing the score-based counterfactual explanation framework to generate counterfactuals was driven to promote computational efficiency.

% Future enhancements to the framework may involve adopting models capable of generating more precise counterfactuals. While such approaches may yield to performance improvements, they are likely to come at the cost of increased computational complexity.


\subsection{RQ2: Counterfactual Regularization Performance}
To answer \textbf{RQ2}, we evaluate the effectiveness of the proposed counterfactual regularization (CF-Reg) by comparing its performance against existing baselines: unregularized training loss (No-Reg), L1 regularization (L1-Reg), L2 regularization (L2-Reg), and Dropout.
Specifically, for each model and dataset combination, Table~\ref{tab:regularization_comparison} presents the mean value and standard deviation of test accuracy achieved by each method across 5 random initialization. 

The table illustrates that our regularization technique consistently delivers better results than existing methods across all evaluated scenarios, except for one case -- i.e., Logistic Regression on the \textit{Phomene} dataset. 
However, this setting exhibits an unusual pattern, as the highest model accuracy is achieved without any regularization. Even in this case, CF-Reg still surpasses other regularization baselines.

From the results above, we derive the following key insights. First, CF-Reg proves to be effective across various model types, ranging from simple linear models (Logistic Regression) to deep architectures like MLPs and CNNs, and across diverse datasets, including both tabular and image data. 
Second, CF-Reg's strong performance on the \textit{Water} dataset with Logistic Regression suggests that its benefits may be more pronounced when applied to simpler models. However, the unexpected outcome on the \textit{Phoneme} dataset calls for further investigation into this phenomenon.


\begin{table*}[h!]
    \centering
    \caption{Mean value and standard deviation of test accuracy across 5 random initializations for different model, dataset, and regularization method. The best results are highlighted in \textbf{bold}.}
    \label{tab:regularization_comparison}
    \begin{tabular}{|c|c|c|c|c|c|c|}
        \hline
        \textbf{Model} & \textbf{Dataset} & \textbf{No-Reg} & \textbf{L1-Reg} & \textbf{L2-Reg} & \textbf{Dropout} & \textbf{CF-Reg (ours)} \\ \hline
        Logistic Regression   & \textit{Water}   & $0.6595 \pm 0.0038$   & $0.6729 \pm 0.0056$   & $0.6756 \pm 0.0046$  & N/A    & $\mathbf{0.6918 \pm 0.0036}$                     \\ \hline
        MLP   & \textit{Water}   & $0.6756 \pm 0.0042$   & $0.6790 \pm 0.0058$   & $0.6790 \pm 0.0023$  & $0.6750 \pm 0.0036$    & $\mathbf{0.6802 \pm 0.0046}$                    \\ \hline
%        MLP   & \textit{Adult}   & $0.8404 \pm 0.0010$   & $\mathbf{0.8495 \pm 0.0007}$   & $0.8489 \pm 0.0014$  & $\mathbf{0.8495 \pm 0.0016}$     & $0.8449 \pm 0.0019$                    \\ \hline
        Logistic Regression   & \textit{Phomene}   & $\mathbf{0.8148 \pm 0.0020}$   & $0.8041 \pm 0.0028$   & $0.7835 \pm 0.0176$  & N/A    & $0.8098 \pm 0.0055$                     \\ \hline
        MLP   & \textit{Phomene}   & $0.8677 \pm 0.0033$   & $0.8374 \pm 0.0080$   & $0.8673 \pm 0.0045$  & $0.8672 \pm 0.0042$     & $\mathbf{0.8718 \pm 0.0040}$                    \\ \hline
        CNN   & \textit{CIFAR-10} & $0.6670 \pm 0.0233$   & $0.6229 \pm 0.0850$   & $0.7348 \pm 0.0365$   & N/A    & $\mathbf{0.7427 \pm 0.0571}$                     \\ \hline
    \end{tabular}
\end{table*}

\begin{table*}[htb!]
    \centering
    \caption{Hyperparameter configurations utilized for the generation of Table \ref{tab:regularization_comparison}. For our regularization the hyperparameters are reported as $\mathbf{\alpha/\beta}$.}
    \label{tab:performance_parameters}
    \begin{tabular}{|c|c|c|c|c|c|c|}
        \hline
        \textbf{Model} & \textbf{Dataset} & \textbf{No-Reg} & \textbf{L1-Reg} & \textbf{L2-Reg} & \textbf{Dropout} & \textbf{CF-Reg (ours)} \\ \hline
        Logistic Regression   & \textit{Water}   & N/A   & $0.0093$   & $0.6927$  & N/A    & $0.3791/1.0355$                     \\ \hline
        MLP   & \textit{Water}   & N/A   & $0.0007$   & $0.0022$  & $0.0002$    & $0.2567/1.9775$                    \\ \hline
        Logistic Regression   &
        \textit{Phomene}   & N/A   & $0.0097$   & $0.7979$  & N/A    & $0.0571/1.8516$                     \\ \hline
        MLP   & \textit{Phomene}   & N/A   & $0.0007$   & $4.24\cdot10^{-5}$  & $0.0015$    & $0.0516/2.2700$                    \\ \hline
       % MLP   & \textit{Adult}   & N/A   & $0.0018$   & $0.0018$  & $0.0601$     & $0.0764/2.2068$                    \\ \hline
        CNN   & \textit{CIFAR-10} & N/A   & $0.0050$   & $0.0864$ & N/A    & $0.3018/
        2.1502$                     \\ \hline
    \end{tabular}
\end{table*}

\begin{table*}[htb!]
    \centering
    \caption{Mean value and standard deviation of training time across 5 different runs. The reported time (in seconds) corresponds to the generation of each entry in Table \ref{tab:regularization_comparison}. Times are }
    \label{tab:times}
    \begin{tabular}{|c|c|c|c|c|c|c|}
        \hline
        \textbf{Model} & \textbf{Dataset} & \textbf{No-Reg} & \textbf{L1-Reg} & \textbf{L2-Reg} & \textbf{Dropout} & \textbf{CF-Reg (ours)} \\ \hline
        Logistic Regression   & \textit{Water}   & $222.98 \pm 1.07$   & $239.94 \pm 2.59$   & $241.60 \pm 1.88$  & N/A    & $251.50 \pm 1.93$                     \\ \hline
        MLP   & \textit{Water}   & $225.71 \pm 3.85$   & $250.13 \pm 4.44$   & $255.78 \pm 2.38$  & $237.83 \pm 3.45$    & $266.48 \pm 3.46$                    \\ \hline
        Logistic Regression   & \textit{Phomene}   & $266.39 \pm 0.82$ & $367.52 \pm 6.85$   & $361.69 \pm 4.04$  & N/A   & $310.48 \pm 0.76$                    \\ \hline
        MLP   &
        \textit{Phomene} & $335.62 \pm 1.77$   & $390.86 \pm 2.11$   & $393.96 \pm 1.95$ & $363.51 \pm 5.07$    & $403.14 \pm 1.92$                     \\ \hline
       % MLP   & \textit{Adult}   & N/A   & $0.0018$   & $0.0018$  & $0.0601$     & $0.0764/2.2068$                    \\ \hline
        CNN   & \textit{CIFAR-10} & $370.09 \pm 0.18$   & $395.71 \pm 0.55$   & $401.38 \pm 0.16$ & N/A    & $1287.8 \pm 0.26$                     \\ \hline
    \end{tabular}
\end{table*}

\subsection{Feasibility of our Method}
A crucial requirement for any regularization technique is that it should impose minimal impact on the overall training process.
In this respect, CF-Reg introduces an overhead that depends on the time required to find the optimal counterfactual example for each training instance. 
As such, the more sophisticated the counterfactual generator model probed during training the higher would be the time required. However, a more advanced counterfactual generator might provide a more effective regularization. We discuss this trade-off in more details in Section~\ref{sec:discussion}.

Table~\ref{tab:times} presents the average training time ($\pm$ standard deviation) for each model and dataset combination listed in Table~\ref{tab:regularization_comparison}.
We can observe that the higher accuracy achieved by CF-Reg using the score-based counterfactual generator comes with only minimal overhead. However, when applied to deep neural networks with many hidden layers, such as \textit{PreactResNet-18}, the forward derivative computation required for the linearization of the network introduces a more noticeable computational cost, explaining the longer training times in the table.

\subsection{Hyperparameter Sensitivity Analysis}
The proposed counterfactual regularization technique relies on two key hyperparameters: $\alpha$ and $\beta$. The former is intrinsic to the loss formulation defined in (\ref{eq:cf-train}), while the latter is closely tied to the choice of the score-based counterfactual explanation method used.

Figure~\ref{fig:test_alpha_beta} illustrates how the test accuracy of an MLP trained on the \textit{Water Potability} dataset changes for different combinations of $\alpha$ and $\beta$.

\begin{figure}[ht]
    \centering
    \includegraphics[width=0.85\linewidth]{img/test_acc_alpha_beta.png}
    \caption{The test accuracy of an MLP trained on the \textit{Water Potability} dataset, evaluated while varying the weight of our counterfactual regularizer ($\alpha$) for different values of $\beta$.}
    \label{fig:test_alpha_beta}
\end{figure}

We observe that, for a fixed $\beta$, increasing the weight of our counterfactual regularizer ($\alpha$) can slightly improve test accuracy until a sudden drop is noticed for $\alpha > 0.1$.
This behavior was expected, as the impact of our penalty, like any regularization term, can be disruptive if not properly controlled.

Moreover, this finding further demonstrates that our regularization method, CF-Reg, is inherently data-driven. Therefore, it requires specific fine-tuning based on the combination of the model and dataset at hand.

% YY: New flow as suggested
\section{The \bm Benchmark}
\label{sec:bm_challenge}

\begin{figure*}[ht]
    \centering
    \includegraphics[width=1.0\textwidth]{imgs/bms.png} 
    % \caption{This figure illustrates the three challenges of \bm where \pb occurs, with progressively increasing difficulty.\gb{Captions needs to be significantly expanded. I also feel like this figure does not look good. It has a lot of text and basic figures. Can you illustrate each challenge using concrete visual examples and fewer text? It would be more intuitive and look a lot better than the current figure. I also feel like the current figure has a lot going on in terms of arrows. You need to simplify it. For instance, I looked at it for ~30s and I still have no idea what some of these illustrations mean. Make it simpler and reduce the amount of text, arrows, and other potentially uncessary/overwhelming details}}
    % \caption{\gb{Maybe start by saying that this is an illustration of 3 BOSS challenges to immediately indicate what this figure depicts instead of going right into specific examples without any overview.} This figure uses three concrete examples, \texttt{PlaceObject(plate, cabinet)}, \texttt{PlaceObject(potato, bowl)}, and \texttt{MoveContainer(bowl, cabinet)}, to illustrate the three challenges of \bm, each examining distinct aspects of \pb. Challenge 1, ``Single Predicate Modification'', shows a case where \pb occurs due to the modification of a single predicate (i.e., circles in the figure) caused by the effect of the previous skill (e.g., \texttt{IN(potato, bowl)})\gb{Will this not be considered unfair since the training and deployment scenarios are constructed specifically so that they wouldn't overlap or at least that's how it seems?}. Challenge 2, ``Accumulated Predicates Modification'', highlights the scenario where \pb arises from multiple predicate changes (i.e., circles in the figure) due to accumulated effects from preceding skills (e.g., \texttt{On(plate, cabinet)} and \texttt{IN(potato, bowl)})\gb{I feel like this example is a bit weird. In my view, the instruction is ambiguous, i.e., you are saying move the bowl on the cabinet but actually it needs to be moved on the plate. Even if I was a person trying to execute this, I don't know exactly that this is what you would want me to do. For instance, should the bowl be placed on the plate or on some other place on the cabinet? It's also unclear what's supposed to happen if there's not enough space on the cabinet. Generally, I find this example somewhat contrived and not very natural. Maybe it would be better to tell the robot to place the bowl on the plate instead of a cabinet to avoid ambiguity.}. Challenge 3, ``Real Long-Horizon Task'', showcases how \pb impacts a real long-horizon task, where both ``Single Predicate Modification'' and ``Accumulated Predicates Modification'' occur, significantly degrading the final task performance.\gb{What do the green and red colors depict in the figure? It's not explained. What does a red circle around the plate mean? Can you explicitly describe what these are?} \gb{Also, Is this the intended font? These figure captions are pretty small and thus difficult to read without zooming in significantly}}
    
    \caption{This figure illustrates the three challenges of \bm, each examining a distinct aspect of \pb, using concrete examples: \textcolor{mygreen}{\texttt{OpenDrawer(cabinet, bottom)} (green)}, \textcolor{myred}{\texttt{PlaceObject(potato, bowl)} (red)}, and \textcolor{myblue}{\texttt{MoveContainer(bowl, cabinet)} (blue)}. Challenge 1, Single Predicate Shift (\bma), shows a case where \pb occurs due to the modification of a single predicate (i.e., the circle in the figure) caused by the effect of the previous skill (e.g., \texttt{IN(potato, bowl)}). Challenge 2, Accumulated Predicate Shift (\bmb), highlights the scenario where \pb arises from multiple predicate changes (i.e., circles in the figure) due to accumulated effects from preceding skills (e.g., \texttt{IN(potato, bowl)} and \texttt{DrawerOpen(cabinet, top)}). Challenge 3, Real Long-Horizon Task (\bmc), showcases how \pb impacts a real long-horizon task with three skills, where ``Single Predicate Shift'' and ``Accumulated Predicate Shift'' occur in the second skill and the third skill respectively, significantly degrading the final task performance.}
    \vspace{-1.6em}
    \label{fig:bm_arch}
\end{figure*}



We introduce the \bm benchmark, designed to facilitate a comprehensive empirical study of the \pb problem across modern IL methods. To address this issue, we build the environment using the Libero simulation platform~\cite{liu2024libero} (Section~\ref{subsec:bm_env}) and form the task set by leveraging selected existing tasks from Libero and scalably generating their modified versions (Section~\ref{subsec:bm_task}). \bm includes three challenges (Section~\ref{subsec:bm_1}~$\sim$~\ref{subsec:bm_3}) designed not only to validate the existence of \pb but also to demonstrate its significant negative impact on the success of long-horizon tasks. These challenges evaluate IL methods on the curated tasks within the constructed environment and feature 44, 88, and 10 tasks respectively, providing a diverse and extensive evaluation.


% \gb{Do we need a figure to highlight the main properties of the benchmark? Since this is one of our main contributions, I think it would be great to showcase that as a figure, especially if we want to emphasize particular things about the benchmark, i.e., a large number of tasks, etc?}\yy{TODO - pie chart}


\subsection{Environment}
\label{subsec:bm_env}
We build the environment of \bm on the Libero platform~\cite{liu2024libero}, which, although originally designed for lifelong robot learning tasks, provides diverse manipulation scenes featuring a Franka Emika Panda robot arm and accommodates a wide variety of tasks. Libero is highly flexible for creating and customizing tasks because they are generated using Planning Domain Definition Language (PDDL) files, which specify the planning problem including operators, predicates, and their relationships. These features make Libero an excellent foundation for developing \bm, enabling us to adapt it specifically to study the \pb problem.

\subsection{Task Design}
\label{subsec:bm_task}

Building on the Libero environment, \bm focuses on studying the impact of \pb on long-horizon tasks. To achieve this, we require a set of atomic robotic tasks, where each task involves only a single skill. These tasks are used to simulate individual skills within a skill chain that are unaffected by \pb. Additionally, we generate modified counterparts to simulate scenarios where \pb occurs. This setup enables performance comparisons between the two sets when evaluating IL methods. All three challenges are based on these 2 sets of tasks. 

To build the set of tasks unaffected by \pb, we select all the skill-level tasks from Libero-100, the most comprehensive and diverse task suite in Libero, spanning 12 manipulation scenes and covering a wide range of object interactions and motor skills. Multi-skill tasks (e.g., “open the top drawer of the cabinet and put the bowl in it”) are excluded, while single-skill tasks (e.g., “open the bottom drawer of the cabinet”) are retained. As a result, the final set comprises 44 single-skill tasks. % \gb{Why? It seems that those would be even better suited to study the OSS problem} All three challenges are constructed from this refined set.

% \yy{TODO: use some notations mentioned in problem def here.} To build the set of counterparts, we propose the \textbf{Rule-based Automatic Modification Generator (RAMG)}, which can scalably generate modified tasks. RAMG is a systematic algorithm designed to enhance diversity in pre-defined environments through structured, rule-based modifications. Operating on PDDL files, RAMG iteratively alters object positions, introduces new objects, or modifies object states within the environment. It supports three types of modifications: (1) repositioning or adding external objects to specific regions, (2) changing the states of fixtures (e.g., opening or closing cabinets), and (3) placing small objects into designated containers. These modifications follow dynamic constraints (e.g., newly added objects do not create obstacles) and maintain logical consistency (e.g., a drawer cannot be simultaneously open and closed) to ensure that they do not hinder the agent from achieving the task goal. Using deterministic yet flexible rules, RAMG provides a scalable method for generating task variations. With a single modification per task, RAMG can produce up to 1,727 modified tasks from the 44 selected tasks, and adding multiple modifications significantly increases this number. The extensive set of generated modified tasks offers flexibility and a robust foundation for designing the three challenges.

To build the set of counterparts, we propose the \textbf{Rule-based Automatic Modification Generator (RAMG)}, which can scalably generate modified tasks. RAMG is an algorithm designed to enhance the visual diversity of skills in predefined environments through structured, rule-based modifications. Operating on PDDL files, RAMG iteratively modifies predicates, $\Psi$, by altering object positions, introducing new objects, or changing object states within the environment. It supports three types of modifications: (1) repositioning existing objects or adding external objects to specific regions, (2) changing the state of fixtures (e.g., toggling a predicate like \texttt{DrawerOpen(bottom\_drawer, cabinet)}), and (3) modifying containment relations by altering predicates like \texttt{In(potato, bowl)}. These modifications adhere to dynamic constraints (e.g., newly added objects do not interfere with existing \texttt{Pre} and \texttt{Eff} conditions) and maintain logical consistency (e.g., preventing contradictions where a predicate $\psi_i$ and its negation $\neg\psi_i$ hold simultaneously). Using deterministic yet flexible rules, RAMG provides a scalable method for generating task variations. With a single modification per task, RAMG can produce up to 1,727 modified tasks from the 44 selected tasks, and adding multiple modifications significantly increases this number. The extensive set of generated modified tasks offers flexibility and a robust foundation for designing the three challenges.

% \gb{How many of these will you use? The way this is written, you are leading the reader to believe that your experiments will involve all of these 1,727 modified tasks. If the number is much smaller than that, you will need to provide a good justification why you don't do it}






\subsection{Challenge 1: Single Predicate Shift (\bma)}
\label{subsec:bm_1}

% \gb{The section structure of this seems disjoint. You go from environment to Challenge 1 without any explanation or overview. Maybe instead you could have a separate section dedicated to all the challenges and then just list each challenge as a subsection? Also, before introducing each challenge maybe you could provide a brief overview of what each challenge entails? Right now the transition between the previous section and this section seems very abrupt and not very intuitive} \yy{I added some descriptions to connect environment and the 3 challenges, will this be better now?}

% As outlined in Section~\ref{sec:problem_definition}, \pb arises when skill-irrelevant predicates, $\Psi$, are altered in ways that do not affect the feasibility of the current skill but disrupt the execution of its visuomotor policy. For example, as shown in the left box of Figure~\ref{fig:bm_arch}, \texttt{In(potato, bowl)} does not impact the feasibility of the current skill, \texttt{MoveContainer(bowl, cabinet)}, but negatively affects the skill policy's performance. In the context of skill chaining, a foundational structure in long-horizon tasks, the preceding skill is the most likely to cause \pb for the subsequent skill. Therefore, we introduce a challenge, Single Predicate Modification (\bma), specifically designed to evaluate the impact of \pb caused by single-step transitions on the performance of baseline methods (see examples in Figure~\ref{fig:bm_arch}).

As outlined in Section~\ref{sec:problem_definition}, \pb arises when skill-irrelevant predicates, $\Psi$, are altered in ways that do not affect the feasibility of the current skill but disrupt the execution of its visuomotor policy. For example, as shown in the left box of Figure~\ref{fig:bm_arch}, the predicate \texttt{In(potato, bowl)} does not impact the feasibility of the current skill, \texttt{MoveContainer(bowl, cabinet)}, since it is absent from $\texttt{Pre}$ and $\texttt{Eff}$. However, changes to this predicate can still alter the visuomotor observation space, $\mathcal{O}$, negatively affecting the skill policy's performance. In the context of skill chaining, a foundational structure in long-horizon tasks, the preceding skill is the most likely to introduce modifications in $\mathcal{O}$ that induce \pb for the subsequent skill. Therefore, we introduce a challenge, Single Predicate Shift (\bma), specifically designed to evaluate the impact of \pb caused by single-step transitions on the performance of baseline methods (see examples in Figure~\ref{fig:bm_arch}).

To investigate the impact of single-step \pb on baseline methods, we compare their performance on skills unaffected by \pb (e.g., \texttt{On(potato, table)}) versus those affected by \pb (e.g., \texttt{In(potato, bowl)}). As detailed in Section~\ref{subsec:bm_task}, for the former, we evaluate baselines on the 44 selected tasks. For the latter, we use RAMG to generate 44 corresponding randomly modified tasks, each with a single modification applied to simulate the occurrence of \pb caused by the preceding skill. 

% Libero's customization flexibility plays a crucial role here, as its task environments are generated using Planning Domain Definition Language (PDDL) files\notezlf{you may mention PDDL when introducing task planning in formulation? you can mention it defines e.g., operators, and what you need to modify here}. This allows us to easily add new objects (e.g., placing a potato in a bowl or on a tabletop) or modify object states (e.g., closing an opened drawer). These modifications enable us to simulate single-step \pb efficiently and evaluate baseline performance on the altered tasks.








\subsection{Challenge 2: Accumulated Predicate Shift (\bmb)}
\label{subsec:bm_2}
% The effect of executing a skill persists until a future skill reverses it, meaning it impacts not only the immediate next skill but also multiple subsequent skills. Over the course of a skill chain, effects from numerous skills can accumulate, resulting in a more severe \pb for a future skill. As illustrated in the middle box of Figure~\ref{fig:bm_arch}, the accumulated effects, \texttt{On(plate, cabinet)} and \texttt{In(potato, bowl)}, collectively impact the current skill, \texttt{MoveContainer(bowl, cabinet)}. Thus, it is valuable to study the differences between accumulated \pb and single-step \pb. To address this, we introduce another challenge, Accumulated Predicates Modification (\bmb), specifically designed to analyze the impact of \pb accumulation across preceding skills (check examples in Figure~\ref{fig:bm_arch}).

The effect of executing a skill persists until a future skill reverses it, meaning it impacts not only the immediate next skill but also multiple subsequent skills. Over the course of a skill chain, effects from numerous skills can accumulate, resulting in a more severe \pb for a future skill. Formally, given a sequence of operators $\{\texttt{op}_1, \texttt{op}_2, \dots, \texttt{op}_t\}$, the accumulated effects at time step $t$ are given by $\bigcup_{i=1}^{t} \texttt{Eff}_i$. As illustrated in the middle box of Figure~\ref{fig:bm_arch}, the accumulated effects, \texttt{On(plate, cabinet)} and \texttt{In(potato, bowl)}, collectively impact the current skill, \texttt{MoveContainer(bowl, cabinet)}, by modifying predicates in the observation space $\mathcal{O}$. Thus, it is valuable to study the differences between accumulated \pb and single-step \pb. To address this, we introduce another challenge, Accumulated Predicate Shift (\bmb), specifically designed to analyze the impact of \pb accumulation across preceding skills (check examples in Figure~\ref{fig:bm_arch}).


Similar to \bma, we evaluate baseline performance on the next skill under two conditions: task unaffected by \pb and task modified by previous effects. Using PDDL description files, we create two sets of modified tasks, each simulating the cumulative effects of multiple preceding skills on the current skill, with one set incorporating two modifications and the other three. However, multiple modifications can sometimes conflict. For example, a modification such as ``place an apple inside the small bowl'' could conflict with another that specifies ``place a potato inside the small bowl,'' potentially causing the Libero environment to break due to overlapping space constraints. By iteratively invoking the proposed RAMG two or three times, we easily generate the two sets of modified tasks, leveraging RAMG's ability to ensure constraint-compliant task generation.



% \gb{It would be great to provide a detailed list of modifications that your system can implement. Similarly, for completenes/reporducibility, it would be great to provide a list of all the tasks included in your dataset along with the numbers for each task. This would emphasize the benchmark contribution and also make the paper more reproducible. Right now many details about your benchmark/dataset are missing.} \yy{I could provide stats in the pie chart, but it's hard for me to list all task names considering there're thousands of tasks generated by RAMG. For listing tasks I use in these 3 challenges, I could add the names later in the appendix after the paper is published (RA-L does not include appendix but I could include one in arxiv version).}






\subsection{Challenge 3: Skill Chaining (\bmc)}
\label{subsec:bm_3}
The ultimate goal of addressing the \pb problem is to improve the success of long-horizon robot tasks. To this end, we introduce a challenge, Skill Chaining (\bmc), consisting of 10 long-horizon tasks, each comprising a chain of three skills (check examples in Figure~\ref{fig:bm_arch}). This challenge serves as a straightforward way to demonstrate the impact of the \pb problem on long-horizon task performance.

To construct this challenge, we manually select and combine skills from the 44 selected skill-level tasks, ensuring that \pb occurs in each skill while avoiding conflicts between modifications. We reset the robot to a neutral position after each skill to eliminate dynamic transition feasibility issues~\cite{chen2023sequential}, ensuring the challenge focuses solely on the impact of \pb, which specifically addresses the negative effects of changes in visual observations.



\section{The \bm Experimental Results}
\label{sec:bm_results}

\subsection{Experimental Setup}



\subsubsection{Baselines}
\label{subsec:basic_bl}

% Overview
We select four widely used imitation learning algorithms, representing diverse architectures and design approaches for baseline comparisons, to learn each skill and evaluate the impact of \pb on their performance in long-horizon tasks. Among these, three are Behavioral Cloning approaches from Libero, while one is a vision-language-action model.


% BC-RNN
% BC-Transformer
% BC-Vilt
\textbf{Behavioral Cloning (BC) in Libero}: A set of three BC algorithms~\cite{torabi2018behavioral} from Libero is adopted: BC-RESNET-RNN~\cite{mandlekar2021matters}, BC-RESNET-T~\cite{zhu2023viola}, and BC-VIT-T~\cite{kim2021vilt}. These algorithms feature diverse neural network architectures for visual and language encoding. All three use BERT embeddings~\cite{devlin2018bert} to encode the language instructions for each skill. In BC-RESNET-RNN, ResNet~\cite{he2016identity} serves as the visual backbone for encoding per-step visual observations, while an LSTM processes the sequence of encoded visual embeddings as the temporal backbone. BC-RESNET-T employs the same visual encoder, ResNet, but replaces the LSTM with a transformer decoder~\cite{vaswani2017attention} for temporal processing. BC-VIT-T uses Vision Transformer (ViT)~\cite{dosovitskiy2020image} as the visual backbone and a transformer decoder as the temporal backbone. All these BC algorithms output a multi-modal distribution over manipulation actions using a Gaussian Mixture Model (GMM) output head~\cite{bishop1994mixture}, from which an action is sampled.


% % Diffusion Policy
% \textbf{Diffusion Policy} --- Generative models have achieved remarkable success in language and vision domains~\cite{openai2024chatgpt, ramesh2022hierarchical, yang2024annotated}, making their application to robot learning a natural progression. Unlike Generative Adversarial Imitation Learning (GAIL)-like methods~\cite{ho2016generative}, which use Generative Adversarial Networks (GANs) as their backbone, diffusion policy~\cite{chi2023diffusion} achieves superior performance in robot learning due to its diffusion model backbone, offering multi-modal action distributions, handling high-dimensional outputs, and ensuring stable training. As a result, we select diffusion policy as the representative generative IL method to evaluate the impact of \pb problem.

% OpenVLA
\textbf{OpenVLA}: Applying large foundation models, such as Large Language Models (LLMs)~\cite{openai2024chatgpt} and Vision-Language Models (VLMs)~\cite{dubey2024llama}, to robotics has gained popularity, leading to vision-language-action (VLA) models. Among these, OpenVLA~\cite{kim2024openvla}, an open-sourced VLA model, outperforms other state-of-the-art methods, making it a natural choice for evaluating the \pb problem as a representative generalist policy. The language description, conditioned on each task and sourced from Libero, remains consistent, whether or not the task is affected by \pb.


\subsubsection{Metrics}
We mainly use two metrics in all the experiments: 

\textbf{Ratio Performance Delta (RPD)}: We define the metric ``Ratio Performance Delta'' as the relative change in skill success rate caused by \pb. It is calculated as the difference between a baseline's success rate on the original skill-level task and its success rate on the modified task where \pb occurs, normalized by the original success rate. A positive RPD indicates that \pb negatively impacts the current skill, while an RPD less than or equal to zero suggests no negative effect. Instances of negative RPD can occur due to the inherent randomness of IL models.

\textbf{Delta to Upper Bound Ratio (DUBR)}: Unlike \bma and \bmb, which evaluate the performance delta for individual skills, \bmc assesses the performance delta across an entire skill chain. To support this, we introduce a new metric. First, we define the ``Chain Upper Bound'' as the product of success rates for each skill in the chain when evaluated without \pb occurrence, representing the maximum achievable success rate in the absence of \pb. Then, we introduce the ``Delta to Upper Bound Ratio'', calculated as the difference between the actual success rate of completing the entire skill chain and the ``Chain Upper Bound'', divided by the ``Chain Upper Bound'', providing a normalized measure of the performance delta for the entire chain.


\subsubsection{Implementation Details}
For observation space design, we align with the baselines' default setups. OpenVLA uses only a third-person camera view as its observation space. To maintain consistency in visual observations, we define the observation space for BCs as a combination of third-person camera images, 7-DoF robot arm joint angles, and 2-DoF parallel gripper joint states, excluding only the wrist-camera view. For all the baselines, the action space is defined as a 7-dimensional relative Cartesian displacement (w.r.t. the gripper frame), and the control frequency is set to 20 Hz. For all the experiments, we use three random seeds and report only the averaged results across these runs.











\subsection{Results for \bma}
\label{subsec:results_bma}

%         % Please add the following required packages to your document preamble:
        % \usepackage{multirow}
        % \usepackage{xcolor}
        
        \begin{table*}[!htb]
        \scriptsize
        \setlength{\tabcolsep}{2pt} % Adjust the margin between columns
        \renewcommand{\arraystretch}{1.1} % Adjust the row height
        \centering
        \caption{Challenge-1 results, evaluating 4 IL methods on tasks in \taskoriginal, highlighting those with positive performance drops.}
        \label{tb:bm1_results}
        \begin{tabular}{|l|l|l|l|l|l|l|l|l|l|l|l|l|l|l|l|l|l|l|l|l|l|l|l|}
        \hline
                                   &          & 0 & 1 & 2 & 3 & 4 & 5 & 6 & 7 & 8 & 9 & 10 & 11 & 12 & 13 & 14 & 15 & 16 & 17 & 18 & 19 & 20 & 21 \\ \hline

\multirow{3}{*}{BC-RESNET-RNN} & SR - ORI & 0.98 & 0.73 & 0.78 & 0.97 & 0.00 & 0.02 & 0.63 & 0.05 & 0.15 & 0.00 & 0.07 & 0.63 & 0.00 & 0.52 & 0.40 & 1.00 & 0.98 & 0.00 & 0.27 & 0.25 & 0.63 & 0.52 \\ \cline{2-24}
                                    & SR - MOD & 1.00 & 0.20 & 0.65 & 0.32 & 0.02 & 0.00 & 0.00 & 0.02 & 0.07 & 0.00 & 0.08 & 0.23 & 0.00 & 0.60 & 0.10 & 0.37 & 0.43 & 0.00 & 0.37 & 0.27 & 0.52 & 0.32 \\ \cline{2-24}
                                    & PD    & -0.02 & \textbf{\textcolor{red}{0.53}} & \textbf{\textcolor{red}{0.13}} & \textbf{\textcolor{red}{0.65}} & -0.02 & \textbf{\textcolor{red}{0.02}} & \textbf{\textcolor{red}{0.63}} & \textbf{\textcolor{red}{0.03}} & \textbf{\textcolor{red}{0.08}} & 0.00 & -0.02 & \textbf{\textcolor{red}{0.40}} & 0.00 & -0.08 & \textbf{\textcolor{red}{0.30}} & \textbf{\textcolor{red}{0.63}} & \textbf{\textcolor{red}{0.55}} & 0.00 & -0.10 & -0.02 & \textbf{\textcolor{red}{0.12}} & \textbf{\textcolor{red}{0.20}} \\ \hline

        \multirow{3}{*}{BC-RESNET-T} & SR - ORI & 1.00 & 1.00 & 1.00 & 0.95 & 0.93 & 1.00 & 0.95 & 0.83 & 0.82 & 0.63 & 0.97 & 0.92 & 0.42 & 1.00 & 0.87 & 1.00 & 0.98 & 0.90 & 0.92 & 0.73 & 0.73 & 1.00 \\ \cline{2-24}
                                    & SR - MOD & 1.00 & 0.67 & 0.93 & 0.12 & 0.82 & 0.22 & 0.98 & 0.62 & 0.17 & 0.62 & 1.00 & 0.53 & 0.52 & 0.87 & 0.57 & 1.00 & 0.93 & 0.12 & 0.63 & 0.78 & 0.58 & 0.97 \\ \cline{2-24}
                                    & PD    & 0.00 & \textbf{\textcolor{red}{0.33}} & \textbf{\textcolor{red}{0.07}} & \textbf{\textcolor{red}{0.83}} & \textbf{\textcolor{red}{0.12}} & \textbf{\textcolor{red}{0.78}} & -0.03 & \textbf{\textcolor{red}{0.22}} & \textbf{\textcolor{red}{0.65}} & \textbf{\textcolor{red}{0.02}} & -0.03 & \textbf{\textcolor{red}{0.38}} & -0.10 & \textbf{\textcolor{red}{0.13}} & \textbf{\textcolor{red}{0.30}} & 0.00 & \textbf{\textcolor{red}{0.05}} & \textbf{\textcolor{red}{0.78}} & \textbf{\textcolor{red}{0.28}} & -0.05 & \textbf{\textcolor{red}{0.15}} & \textbf{\textcolor{red}{0.03}} \\ \hline
\multirow{3}{*}{BC-VIT-T} & SR - ORI & 1.00 & 1.00 & 0.98 & 0.95 & 0.95 & 0.98 & 0.98 & 0.87 & 0.68 & 0.55 & 0.98 & 0.90 & 0.92 & 0.90 & 0.98 & 1.00 & 1.00 & 0.98 & 0.98 & 0.70 & 0.63 & 1.00 \\ \cline{2-24}
                                    & SR - MOD & 1.00 & 0.65 & 1.00 & 0.97 & 0.82 & 0.13 & 0.97 & 0.92 & 0.30 & 0.68 & 0.73 & 0.28 & 0.88 & 0.68 & 0.77 & 1.00 & 1.00 & 0.00 & 0.73 & 0.63 & 0.67 & 1.00 \\ \cline{2-24}
                                    & PD    & 0.00 & \textbf{\textcolor{red}{0.35}} & -0.02 & -0.02 & \textbf{\textcolor{red}{0.13}} & \textbf{\textcolor{red}{0.85}} & \textbf{\textcolor{red}{0.02}} & -0.05 & \textbf{\textcolor{red}{0.38}} & -0.13 & \textbf{\textcolor{red}{0.25}} & \textbf{\textcolor{red}{0.62}} & \textbf{\textcolor{red}{0.03}} & \textbf{\textcolor{red}{0.22}} & \textbf{\textcolor{red}{0.22}} & 0.00 & 0.00 & \textbf{\textcolor{red}{0.98}} & \textbf{\textcolor{red}{0.25}} & \textbf{\textcolor{red}{0.07}} & -0.03 & 0.00 \\ \hline
\multirow{3}{*}{OpenVLA} & SR - ORI & 1.00 & 1.00 & 0.65 & 1.00 & 1.00 & 1.00 & 1.00 & 1.00 & 1.00 & 1.00 & 0.70 & 0.65 & 0.65 & 0.50 & 0.75 & 0.80 & 1.00 & 0.85 & 0.80 & 0.95 & 0.40 & 1.00 \\ \cline{2-24}
                                    & SR - MOD & 0.35 & 1.00 & 0.55 & 0.00 & 1.00 & 0.70 & 0.00 & 1.00 & 0.30 & 0.65 & 0.35 & 1.00 & 1.00 & 0.55 & 0.75 & 0.20 & 0.75 & 0.05 & 0.55 & 0.35 & 0.15 & 1.00 \\ \cline{2-24}
                                    & PD    & \textbf{\textcolor{red}{0.65}} & 0.00 & \textbf{\textcolor{red}{0.10}} & \textbf{\textcolor{red}{1.00}} & 0.00 & \textbf{\textcolor{red}{0.30}} & \textbf{\textcolor{red}{1.00}} & 0.00 & \textbf{\textcolor{red}{0.70}} & \textbf{\textcolor{red}{0.35}} & \textbf{\textcolor{red}{0.35}} & -0.35 & -0.35 & -0.05 & 0.00 & \textbf{\textcolor{red}{0.60}} & \textbf{\textcolor{red}{0.25}} & \textbf{\textcolor{red}{0.80}} & \textbf{\textcolor{red}{0.25}} & \textbf{\textcolor{red}{0.60}} & \textbf{\textcolor{red}{0.25}} & 0.00 \\ \hline
% \multirow{3}{*}{MaIL} & SR - ORI & 0.93 & 1.00 & 0.43 & 0.93 & 0.93 & 0.97 & 0.95 & 0.70 & 0.73 & 0.90 & 0.52 & 0.52 & 0.28 & 0.95 & 0.35 & 1.00 & 0.97 & 0.92 & 1.00 & 0.20 & 0.18 & 0.98 \\ \cline{2-24}
%                                     & SR - MOD & 0.95 & 0.00 & 0.58 & 0.00 & 0.00 & 0.15 & 0.00 & 0.57 & 0.28 & 0.83 & 0.92 & 0.42 & 0.13 & 0.60 & 0.00 & 0.45 & 0.83 & 0.00 & 0.92 & 0.27 & 0.37 & 0.93 \\ \cline{2-24}
%                                     & PD    & -0.02 & \textbf{\textcolor{red}{1.00}} & -0.15 & \textbf{\textcolor{red}{0.93}} & \textbf{\textcolor{red}{0.93}} & \textbf{\textcolor{red}{0.82}} & \textbf{\textcolor{red}{0.95}} & \textbf{\textcolor{red}{0.13}} & \textbf{\textcolor{red}{0.45}} & \textbf{\textcolor{red}{0.07}} & -0.40 & \textbf{\textcolor{red}{0.10}} & \textbf{\textcolor{red}{0.15}} & \textbf{\textcolor{red}{0.35}} & \textbf{\textcolor{red}{0.35}} & \textbf{\textcolor{red}{0.55}} & \textbf{\textcolor{red}{0.13}} & \textbf{\textcolor{red}{0.92}} & \textbf{\textcolor{red}{0.08}} & -0.07 & -0.18 & \textbf{\textcolor{red}{0.05}} \\ \hline

        \end{tabular}
        \end{table*}
        
        % Please add the following required packages to your document preamble:
        % \usepackage{multirow}
        % \usepackage{xcolor}




        \vspace{-30em}
        \begin{table*}[!htb]
        \scriptsize
        \setlength{\tabcolsep}{2pt} % Adjust the margin between columns
        \renewcommand{\arraystretch}{1.1} % Adjust the row height
        \centering
        \begin{tabular}{|l|l|l|l|l|l|l|l|l|l|l|l|l|l|l|l|l|l|l|l|l|l|l|l|}
        \hline
                                   &          & 22 & 23 & 24 & 25 & 26 & 27 & 28 & 29 & 30 & 31 & 32 & 33 & 34 & 35 & 36 & 37 & 38 & 39 & 40 & 41 & 42 & 43 \\ \hline

\multirow{3}{*}{BC-RESNET-RNN} & SR - ORI & 0.87 & 0.00 & 0.03 & 0.15 & 0.33 & 1.00 & 0.65 & 0.45 & 0.58 & 0.85 & 0.75 & 0.18 & 0.92 & 0.73 & 1.00 & 0.00 & 0.02 & 0.15 & 0.08 & 0.07 & 0.12 & 0.45 \\ \cline{2-24}
                                    & SR - MOD & 0.72 & 0.02 & 0.00 & 0.07 & 0.52 & 0.95 & 0.40 & 0.07 & 0.65 & 0.23 & 0.35 & 0.00 & 0.63 & 0.00 & 1.00 & 0.00 & 0.00 & 0.00 & 0.00 & 0.02 & 0.00 & 0.03 \\ \cline{2-24}
                                    & PD    & \textbf{\textcolor{red}{0.15}} & -0.02 & \textbf{\textcolor{red}{0.03}} & \textbf{\textcolor{red}{0.08}} & -0.18 & \textbf{\textcolor{red}{0.05}} & \textbf{\textcolor{red}{0.25}} & \textbf{\textcolor{red}{0.38}} & -0.07 & \textbf{\textcolor{red}{0.62}} & \textbf{\textcolor{red}{0.40}} & \textbf{\textcolor{red}{0.18}} & \textbf{\textcolor{red}{0.28}} & \textbf{\textcolor{red}{0.73}} & 0.00 & 0.00 & \textbf{\textcolor{red}{0.02}} & \textbf{\textcolor{red}{0.15}} & \textbf{\textcolor{red}{0.08}} & \textbf{\textcolor{red}{0.05}} & \textbf{\textcolor{red}{0.12}} & \textbf{\textcolor{red}{0.42}} \\ \hline

        \multirow{3}{*}{BC-RESNET-T} & SR - ORI & 1.00 & 0.27 & 0.98 & 0.47 & 0.97 & 1.00 & 0.65 & 0.57 & 1.00 & 0.98 & 0.93 & 0.98 & 0.95 & 0.85 & 1.00 & 0.47 & 0.65 & 0.60 & 0.48 & 0.90 & 0.70 & 0.53 \\ \cline{2-24}
                                    & SR - MOD & 0.97 & 0.33 & 0.95 & 0.60 & 0.90 & 1.00 & 0.38 & 0.80 & 0.90 & 0.42 & 0.75 & 0.88 & 0.98 & 0.07 & 1.00 & 0.52 & 1.00 & 0.78 & 0.00 & 0.90 & 0.30 & 0.30 \\ \cline{2-24}
                                    & PD    & \textbf{\textcolor{red}{0.03}} & -0.07 & \textbf{\textcolor{red}{0.03}} & -0.13 & \textbf{\textcolor{red}{0.07}} & 0.00 & \textbf{\textcolor{red}{0.27}} & -0.23 & \textbf{\textcolor{red}{0.10}} & \textbf{\textcolor{red}{0.57}} & \textbf{\textcolor{red}{0.18}} & \textbf{\textcolor{red}{0.10}} & -0.03 & \textbf{\textcolor{red}{0.78}} & 0.00 & -0.05 & -0.35 & -0.18 & \textbf{\textcolor{red}{0.48}} & 0.00 & \textbf{\textcolor{red}{0.40}} & \textbf{\textcolor{red}{0.23}} \\ \hline
\multirow{3}{*}{BC-VIT-T} & SR - ORI & 0.98 & 0.50 & 0.93 & 0.48 & 0.95 & 0.98 & 0.82 & 0.72 & 0.98 & 1.00 & 0.65 & 0.95 & 0.85 & 0.95 & 1.00 & 0.38 & 0.80 & 0.72 & 0.37 & 0.77 & 0.65 & 0.57 \\ \cline{2-24}
                                    & SR - MOD & 1.00 & 0.33 & 1.00 & 0.62 & 0.93 & 1.00 & 0.30 & 0.57 & 0.95 & 0.72 & 0.87 & 0.85 & 0.87 & 0.83 & 1.00 & 0.67 & 0.82 & 0.77 & 0.00 & 0.77 & 0.88 & 0.87 \\ \cline{2-24}
                                    & PD    & -0.02 & \textbf{\textcolor{red}{0.17}} & -0.07 & -0.13 & \textbf{\textcolor{red}{0.02}} & -0.02 & \textbf{\textcolor{red}{0.52}} & \textbf{\textcolor{red}{0.15}} & \textbf{\textcolor{red}{0.03}} & \textbf{\textcolor{red}{0.28}} & -0.22 & \textbf{\textcolor{red}{0.10}} & -0.02 & \textbf{\textcolor{red}{0.12}} & 0.00 & -0.28 & -0.02 & -0.05 & \textbf{\textcolor{red}{0.37}} & 0.00 & -0.23 & -0.30 \\ \hline
\multirow{3}{*}{OpenVLA} & SR - ORI & 1.00 & 0.90 & 1.00 & 0.45 & 0.95 & 1.00 & 0.65 & 0.95 & 0.75 & 0.40 & 0.60 & 0.95 & 0.85 & 0.85 & 0.90 & 1.00 & 0.35 & 1.00 & 0.00 & 1.00 & 0.70 & 0.65 \\ \cline{2-24}
                                    & SR - MOD & 1.00 & 0.55 & 0.65 & 0.35 & 0.85 & 1.00 & 0.70 & 0.65 & 0.15 & 0.05 & 0.50 & 0.60 & 0.80 & 0.95 & 0.00 & 0.65 & 0.35 & 0.30 & 0.00 & 0.30 & 1.00 & 0.00 \\ \cline{2-24}
                                    & PD    & 0.00 & \textbf{\textcolor{red}{0.35}} & \textbf{\textcolor{red}{0.35}} & \textbf{\textcolor{red}{0.10}} & \textbf{\textcolor{red}{0.10}} & 0.00 & -0.05 & \textbf{\textcolor{red}{0.30}} & \textbf{\textcolor{red}{0.60}} & \textbf{\textcolor{red}{0.35}} & \textbf{\textcolor{red}{0.10}} & \textbf{\textcolor{red}{0.35}} & \textbf{\textcolor{red}{0.05}} & -0.10 & \textbf{\textcolor{red}{0.90}} & \textbf{\textcolor{red}{0.35}} & 0.00 & \textbf{\textcolor{red}{0.70}} & 0.00 & \textbf{\textcolor{red}{0.70}} & -0.30 & \textbf{\textcolor{red}{0.65}} \\ \hline
% \multirow{3}{*}{MaIL} & SR - ORI & 0.98 & 0.82 & 0.93 & 0.27 & 0.38 & 0.68 & 0.80 & 0.90 & 0.55 & 0.85 & 0.25 & 0.97 & 0.18 & 0.95 & 1.00 & 0.57 & 0.85 & 0.62 & 0.93 & 0.53 & 0.93 & 0.10 \\ \cline{2-24}
%                                     & SR - MOD & 0.97 & 0.20 & 0.92 & 0.55 & 0.17 & 0.07 & 0.55 & 0.42 & 0.02 & 0.73 & 0.27 & 0.93 & 0.77 & 0.35 & 0.53 & 0.58 & 0.33 & 0.38 & 0.00 & 0.53 & 0.20 & 0.17 \\ \cline{2-24}
%                                     & PD    & \textbf{\textcolor{red}{0.02}} & \textbf{\textcolor{red}{0.62}} & \textbf{\textcolor{red}{0.02}} & -0.28 & \textbf{\textcolor{red}{0.22}} & \textbf{\textcolor{red}{0.62}} & \textbf{\textcolor{red}{0.25}} & \textbf{\textcolor{red}{0.48}} & \textbf{\textcolor{red}{0.53}} & \textbf{\textcolor{red}{0.12}} & -0.02 & \textbf{\textcolor{red}{0.03}} & -0.58 & \textbf{\textcolor{red}{0.60}} & \textbf{\textcolor{red}{0.47}} & -0.02 & \textbf{\textcolor{red}{0.52}} & \textbf{\textcolor{red}{0.23}} & \textbf{\textcolor{red}{0.93}} & 0.00 & \textbf{\textcolor{red}{0.73}} & -0.07 \\ \hline

        \end{tabular}
        \end{table*}
        
\begin{figure*}[ht]
    \centering
    \includegraphics[width=1.0\textwidth]{imgs/ch1_results.png} 
    \caption{This figure presents the results for \bma. In each baseline subfigure, the majority of points lie below the diagonal line, representing tasks with a positive Ratio Performance Delta, indicating that single predicate modification negatively affects tasks performance.}
    \label{fig:ch1_results}
    \vspace{-1.5em}
\end{figure*}


We evaluate baseline methods, including BC-RESNET-RNN, BC-RESNET-T, BC-VIT-T, and OpenVLA, on the 44 selected tasks and their 44 modified counterparts, calculating the Ratio Performance Delta for each baseline. The results are presented in Figure~\ref{fig:ch1_results}, where the x-axis represents the success rate on tasks unaffected by \pb, and the y-axis represents the success rate on their modified counterparts. Each point corresponds to a task pair (unaffected and modified), with darker colors indicating higher Ratio Performance Delta values. A diagonal line separates tasks: points on or above the line have non-positive Ratio Performance Delta, indicating no negative impact from \pb, while points below the line signify tasks negatively affected by \pb. As shown in Figure~\ref{fig:ch1_results}, the percentage of tasks negatively affected by \pb (points below the diagonal) is 68\%, 66\%, 50\%, and 66\% for BC-RESNET-RNN, BC-RESNET-T, BC-VIT-T, and OpenVLA, respectively. Among these tasks, the average Ratio Performance Delta is 67\%, 35\%, 34\%, and 54\% for the respective baselines, highlighting the substantial performance degradation caused by \pb. These results demonstrate that \pb frequently occurs even in single-step transitions, emphasizing its potential to jeopardize the success of long-horizon tasks.

% As defined in Section~\ref{subsec:bm_1}, we use Performance Drop (PD) as the metric, calculated as the difference between the success rate on original tasks (SR-ORI) and the success rate on modified tasks (SR-MOD).\gb{Instead of repeating explanations about the evaluation metrics, just move them to the experiments section and create a separate paragraph/subsection for them} Table~\ref{tb:bm1_results} summarizes the average results across three random seeds, showing that \pb occurs in most tasks across all baselines. The ratios of positive PD (i.e., \pb occurrence) for BC-RESNET-RNN, BC-RESNET-T, BC-VIT-T, and OpenVLA are 68\%, 66\%, 50\%, and 66\%, respectively\gb{What is positive PD? It's very counterintuitive to put two words meaning opposite things, i.e., positive drop? What does that mean? I also can't relate to what you are saying to the things in the table as it's very overwhelming}. For tasks negatively affected by \pb, the average PD ratios are 29\%, 30\%, 28\%, and 44\%\gb{What are these ratios?}, respectively, underscoring the significant performance impact. These findings demonstrate that \pb frequently occurs even in single-step transitions, providing evidence that \pb can jeopardize the successful completion of long-horizon tasks. \gb{I feel like the definitions of you are evaluation metrics and the descriptions of these results + their connection to the table is not clear. I didn't understand much in this paragraph}


\subsection{Results for \bmb}
\label{subsec:results_bmb}
\vspace{-1.2em}

% % % Please add the following required packages to your document preamble:
% % \usepackage{multirow}
% \begin{table*}[]
% \centering
% \caption{Results for BM2}
% \label{tb:bm2_results}
% \begin{tabular}{|c|c|c|c|c|c|c|}
% \hline
%                                  &                              & BC-RESNET-RNN & BC-RESNET-T & BC-VIT-T & OpenVLA & MaIL                      \\ \hline
% \multicolumn{1}{|l|}{Original}   & \multicolumn{1}{l|}{Avg. SR} & 0.43          & 0.83        & 0.84     & 0.81    & \multicolumn{1}{l|}{0.71} \\ \hline
% \multirow{2}{*}{1 Modification}  & Avg. SR                      & 0.25          & 0.67        & 0.74     & 0.54    & 0.43                      \\ \cline{2-7} 
%                                  & Avg. PD                        & 0.18          & 0.16        & 0.10     & 0.27    & 0.28                      \\ \hline
% \multirow{2}{*}{2 Modifications} & Avg. SR                      & 0.25          & 0.55        & 0.53     & 0.46    &                           \\ \cline{2-7} 
%                                  & Avg. PD                        & 0.18          & 0.28        & 0.31     & 0.35    &                           \\ \hline
% \multirow{2}{*}{3 Modifications} & Avg. SR                      & 0.23          & 0.49        & 0.56     & 0.35    &                           \\ \cline{2-7} 
%                                  & Avg. PD                        & 0.20          & 0.34        & 0.28     & 0.46    &                           \\ \hline
% \end{tabular}
% \end{table*}






% Please add the following required packages to your document preamble:
% \usepackage{multirow}
\begin{table*}[ht]
\centering
\caption{Challenge-2 results, evaluating four IL methods on \taskoriginal, with averages shown across tasks.}
\label{tb:bm2_results}
\begin{tabular}{|c|c|c|c|c|c|}
\hline
                                 &                              & BC-RESNET-RNN & BC-RESNET-T & BC-VIT-T & OpenVLA \\ \hline
\multicolumn{1}{|l|}{Original}   & \multicolumn{1}{l|}{Avg. SR} & 0.43          & 0.83        & 0.84     & 0.81    \\ \hline
\multirow{2}{*}{1 Modification}  & Avg. SR                      & 0.25          & 0.67        & 0.74     & 0.54    \\ \cline{2-6} 
                                 & Avg. PD                      & 0.18          & 0.16        & 0.10     & 0.27    \\ \hline
\multirow{2}{*}{2 Modifications} & Avg. SR                      & 0.25          & 0.55        & 0.53     & 0.46    \\ \cline{2-6} 
                                 & Avg. PD                      & 0.18          & 0.28        & 0.31     & 0.35    \\ \hline
\multirow{2}{*}{3 Modifications} & Avg. SR                      & 0.23          & 0.49        & 0.56     & 0.35    \\ \cline{2-6} 
                                 & Avg. PD                      & 0.20          & 0.34        & 0.28     & 0.46    \\ \hline
\end{tabular}
\end{table*}

% \begin{figure*}[ht]
%     \centering
%     \includegraphics[width=1.0\textwidth]{imgs/ch2_results_final.png} 
%     \caption{This figure shows the results for \bmb. In each baseline subfigure, most points fall below the diagonal line, representing tasks with a positive Ratio Performance Delta and highlighting the negative impact of accumulated predicate modifications on task performance. Two bar charts beneath the scatterplot summarize (1) the average ratio of \pb occurrence across sets with varying numbers of modifications and (2) the average positive Ratio Performance Delta for sets with different numbers of modifications.\gb{I feel like there are so many things going on in this figure that's it's difficult to understand the main takeaways quickly. For instance, it's difficult to compare 1,2,3 modification settings in the figure because those points are all over the place. I feel like it would maybe be useful to plot these separately. Fig. 3 was great: easy to follow and clear. This figure is quite overwhelming at least for me.}
%     \yy{1. Enlarge bar chart font; 2. Find a better way to visualize (?)}}
%     \label{fig:ch2_results}
% \end{figure*}


\begin{figure}[ht]
    \centering
    \includegraphics[width=0.5\textwidth]{imgs/ch2_results_final_new.png} 
    \caption{This figure shows the results for \bmb. Two bar charts summarize: (top) the average positive Ratio Performance Delta for sets with different numbers of modifications, and (bottom) the average ratio of \pb occurrence across sets with varying numbers of modifications. The upward trend of the red lines in both bar charts indicates that the accumulation of \pb progressively exacerbates its negative impact on long-horizon task completion, both in magnitude and frequency.}
    \vspace{-0.5em}
    \label{fig:ch2_results}
\end{figure}



% As detailed in Section~\ref{subsec:bm_2}, we use the proposed RAMG to generate two sets of modified tasks: one with two modifications and another with three, simulating scenarios of accumulated modifications. Similar to \bma, we evaluate baselines on the 44 unaffected tasks and their corresponding modified counterparts. Figure~\ref{fig:ch2_results} presents results for \bmb evaluations alongside \bma outcomes (i.e., single modification), providing totally three sets of results for comparison. In the scatterplot, similar to Figure~\ref{fig:ch1_results}, circles represent single-step \pb, while triangles and crosses correspond to accumulated \pb from two and three modifications, respectively. Most points fall below the diagonal line (i.e., Ratio Performance Delta is positive), reaffirming that \pb negatively impacts the current skill. Two bar charts below the scatterplot summarize: (1) the average positive Ratio Performance Delta for the three sets, and (2) the average ratio of \pb occurrence (i.e., Ratio Performance Delta is positive) across the three sets. In (1), tasks with multiple modifications display larger or comparable Ratio Performance Delta values than those with a single modification. In (2), the probability of \pb occurrence increases as the number of modifications grows.\gb{At least for me it's difficult to follow the analysis in the last two sentences. I wasn't sure what it means} Notably, across all baselines, the average Ratio Performance Delta for tasks with multiple modifications exceeds 50\%, indicating a large performance decline that is likely to result in the failure of long-horizon tasks. Since skills' effects accumulation is common in long-horizon tasks, with the number of modifications increasing along the skill chain, these results emphasize that \pb poses a significant challenge for achieving success in long-horizon tasks.


As detailed in Section~\ref{subsec:bm_2}, we use the proposed RAMG to generate two sets of modified tasks: one with two modifications and another with three, simulating scenarios of accumulated modifications. Similar to \bma, we evaluate baselines on the 44 unaffected tasks and their corresponding modified counterparts. Figure~\ref{fig:ch2_results} presents the evaluation results for \bmb alongside \bma (i.e., single modification), yielding three sets of results for comparison. Two bar charts summarize key findings: (top) the average positive Ratio Performance Delta across the three sets and (bottom) the average occurrence ratio of \pb (i.e., cases where Ratio Performance Delta is positive). Each bar chart consists of three grouped bars, where each group corresponds to sets with one, two, or three modifications, and within each group, bars represent different baselines. Additionally, a red line denotes the average performance across all baselines for each set. In both charts, this red line follows an increasing trend, indicating that as the number of modifications grows, the negative impact on task performance intensifies, both in magnitude and frequency. Notably, across all baselines, the average Ratio Performance Delta for sets with multiple modifications exceeds 50\% for all baselines (blue line in the top bar chart), signifying a substantial performance decline that is likely to cause failures in long-horizon tasks. Given that skill effect accumulation is inherent in long-horizon tasks, with modifications compounding along the skill chain, these results highlight that \pb presents a significant obstacle to achieving reliable performance.




% Table~\ref{tb:bm2_results} presents average results across three random seeds alongside \bma outcomes (single-step modifications) for straightforward comparison. Due to space constraints, we report only the average success rate (Avg. SR) and average performance drop (Avg. PD). The results reveal that tasks with multiple modifications experience greater performance degradation (i.e., higher PD) compared to single-step modifications\gb{Can you provide concrete numbers to back this up?}. Furthermore, as the number of modifications increases, the negative impact becomes more pronounced, indicating that \pb accumulation significantly exacerbates task completion difficulty\gb{Again would be nice to present some numbers here}. Since \pb accumulation can occur frequently in long-horizon tasks, with the number of modifications growing along the skill chain, these findings further validate that \pb poses a significant challenge for long-horizon task completion.


\subsection{Results for \bmc}
\label{subsec:results_bmc}

% % \begin{table*}[]
% \scriptsize
% \centering
% \caption{Results for BM3}
% \label{tb:bm3_results}
% \begin{tabular}{|c|ccc|ccc|ccc|ccc|ccc|}
% \hline
%    & \multicolumn{3}{c|}{BC-RESNET-RNN}                                                   & \multicolumn{3}{c|}{BC-RESNET-T}                                                     & \multicolumn{3}{c|}{BC-VIT-T}                                                        & \multicolumn{3}{c|}{OpenVLA}                                                           & \multicolumn{3}{c|}{MaIL}                                     \\ \hline
%    & \multicolumn{1}{c|}{Level SR}                     & \multicolumn{1}{c|}{UB}   & DUB  & \multicolumn{1}{c|}{Level SR}                     & \multicolumn{1}{c|}{UB}   & DUB  & \multicolumn{1}{c|}{Level SR}                     & \multicolumn{1}{c|}{UB}   & DUB  & \multicolumn{1}{c|}{Level SR}                      & \multicolumn{1}{c|}{UB}   & DUB   & \multicolumn{1}{c|}{Level SR} & \multicolumn{1}{c|}{UB} & DUB \\ \hline
% 1  & \multicolumn{1}{c|}{\makecell{L0: 0.85 \\ L1: 0.3 \\ L2: 0.0}}  & \multicolumn{1}{c|}{0.01} & 0.01 & \multicolumn{1}{c|}{\makecell{L0: 0.95 \\ L1: 0.1 \\ L2: 0.1}}  & \multicolumn{1}{c|}{0.95} & 0.85 & \multicolumn{1}{c|}{\makecell{L0: 1.0 \\ L1: 1.0 \\ L2: 0.2}}   & \multicolumn{1}{c|}{0.92} & 0.72 & \multicolumn{1}{c|}{\makecell{L0: 1.0 \\ L1: 0.0 \\ L2: 0.0}}    & \multicolumn{1}{c|}{0.65} & 0.65  & \multicolumn{1}{c|}{}         & \multicolumn{1}{c|}{}   &     \\ \hline
% 2  & \multicolumn{1}{c|}{\makecell{L0: 0.95 \\ L1: 0.0 \\ L2: 0.0}}  & \multicolumn{1}{c|}{0.01} & 0.01 & \multicolumn{1}{c|}{\makecell{L0: 0.9 \\ L1: 0.15 \\ L2: 0.05}} & \multicolumn{1}{c|}{0.95} & 0.90 & \multicolumn{1}{c|}{\makecell{L0: 0.95 \\ L1: 0.25 \\ L2: 0.1}} & \multicolumn{1}{c|}{0.92} & 0.82 & \multicolumn{1}{c|}{\makecell{L0: 1.0 \\ L1: 0.0 \\ L2: 0.0}}    & \multicolumn{1}{c|}{1.00} & 1.00  & \multicolumn{1}{c|}{}         & \multicolumn{1}{c|}{}   &     \\ \hline
% 3  & \multicolumn{1}{c|}{\makecell{L0: 0.55 \\ L1: 0.0 \\ L2: 0.0}}  & \multicolumn{1}{c|}{0.00} & 0.00 & \multicolumn{1}{c|}{\makecell{L0: 0.9 \\ L1: 0.0 \\ L2: 0.0}}   & \multicolumn{1}{c|}{0.77} & 0.77 & \multicolumn{1}{c|}{\makecell{L0: 1.0 \\ L1: 0.1 \\ L2: 0.1}}   & \multicolumn{1}{c|}{0.84} & 0.74 & \multicolumn{1}{c|}{\makecell{L0: 1.0 \\ L1: 0.15 \\ L2: 0.05}}  & \multicolumn{1}{c|}{0.70} & 0.65  & \multicolumn{1}{c|}{}         & \multicolumn{1}{c|}{}   &     \\ \hline
% 4  & \multicolumn{1}{c|}{\makecell{L0: 0.4 \\ L1: 0.05 \\ L2: 0.0}}  & \multicolumn{1}{c|}{0.01} & 0.01 & \multicolumn{1}{c|}{\makecell{L0: 1.0 \\ L1: 0.7 \\ L2: 0.2}}   & \multicolumn{1}{c|}{0.75} & 0.55 & \multicolumn{1}{c|}{\makecell{L0: 1.0 \\ L1: 0.75 \\ L2: 0.65}} & \multicolumn{1}{c|}{0.66} & 0.01 & \multicolumn{1}{c|}{\makecell{L0: 1.0 \\ L1: 0.75 \\ L2: 0.65}}  & \multicolumn{1}{c|}{0.70} & 0.05  & \multicolumn{1}{c|}{}         & \multicolumn{1}{c|}{}   &     \\ \hline
% 5  & \multicolumn{1}{c|}{\makecell{L0: 0.4 \\ L1: 0.0 \\ L2: 0.0}}   & \multicolumn{1}{c|}{0.07} & 0.07 & \multicolumn{1}{c|}{\makecell{L0: 0.6 \\ L1: 0.25 \\ L2: 0.0}}  & \multicolumn{1}{c|}{0.66} & 0.66 & \multicolumn{1}{c|}{\makecell{L0: 0.8 \\ L1: 0.65 \\ L2: 0.6}}  & \multicolumn{1}{c|}{0.69} & 0.09 & \multicolumn{1}{c|}{\makecell{L0: 0.8 \\ L1: 0.65 \\ L2: 0.35}}  & \multicolumn{1}{c|}{0.76} & 0.41  & \multicolumn{1}{c|}{}         & \multicolumn{1}{c|}{}   &     \\ \hline
% 6  & \multicolumn{1}{c|}{\makecell{L0: 0.45 \\ L1: 0.2 \\ L2: 0.0}}  & \multicolumn{1}{c|}{0.07} & 0.07 & \multicolumn{1}{c|}{\makecell{L0: 0.75 \\ L1: 0.5 \\ L2: 0.1}}  & \multicolumn{1}{c|}{0.66} & 0.56 & \multicolumn{1}{c|}{\makecell{L0: 0.75 \\ L1: 0.65 \\ L2: 0.3}} & \multicolumn{1}{c|}{0.69} & 0.39 & \multicolumn{1}{c|}{\makecell{L0: 0.8 \\ L1: 0.35 \\ L2: 0.1}}   & \multicolumn{1}{c|}{0.76} & 0.66  & \multicolumn{1}{c|}{}         & \multicolumn{1}{c|}{}   &     \\ \hline
% 7  & \multicolumn{1}{c|}{\makecell{L0: 0.0 \\ L1: 0.0 \\ L2: 0.0}}   & \multicolumn{1}{c|}{0.00} & 0.00 & \multicolumn{1}{c|}{\makecell{L0: 0.85 \\ L1: 0.85 \\ L2: 0.3}} & \multicolumn{1}{c|}{0.65} & 0.35 & \multicolumn{1}{c|}{\makecell{L0: 0.95 \\ L1: 0.65 \\ L2: 0.4}} & \multicolumn{1}{c|}{0.62} & 0.22 & \multicolumn{1}{c|}{\makecell{L0: 0.95 \\ L1: 0.65 \\ L2: 0.15}} & \multicolumn{1}{c|}{0.34} & 0.19  & \multicolumn{1}{c|}{}         & \multicolumn{1}{c|}{}   &     \\ \hline
% 8  & \multicolumn{1}{c|}{\makecell{L0: 0.4 \\ L1: 0.0 \\ L2: 0.0}}   & \multicolumn{1}{c|}{0.00} & 0.00 & \multicolumn{1}{c|}{\makecell{L0: 0.65 \\ L1: 0.4 \\ L2: 0.35}} & \multicolumn{1}{c|}{0.65} & 0.30 & \multicolumn{1}{c|}{\makecell{L0: 0.7 \\ L1: 0.7 \\ L2: 0.5}}   & \multicolumn{1}{c|}{0.62} & 0.12 & \multicolumn{1}{c|}{\makecell{L0: 0.5 \\ L1: 0.25 \\ L2: 0.2}}   & \multicolumn{1}{c|}{0.34} & 0.14  & \multicolumn{1}{c|}{}         & \multicolumn{1}{c|}{}   &     \\ \hline
% 9  & \multicolumn{1}{c|}{\makecell{L0: 0.8 \\ L1: 0.05 \\ L2: 0.05}} & \multicolumn{1}{c|}{0.55} & 0.50 & \multicolumn{1}{c|}{\makecell{L0: 0.9 \\ L1: 0.2 \\ L2: 0.2}}   & \multicolumn{1}{c|}{0.79} & 0.59 & \multicolumn{1}{c|}{\makecell{L0: 0.7 \\ L1: 0.5 \\ L2: 0.5}}   & \multicolumn{1}{c|}{0.62} & 0.12 & \multicolumn{1}{c|}{\makecell{L0: 0.8 \\ L1: 0.7 \\ L2: 0.7}}    & \multicolumn{1}{c|}{0.46} & -0.24 & \multicolumn{1}{c|}{}         & \multicolumn{1}{c|}{}   &     \\ \hline
% 10 & \multicolumn{1}{c|}{\makecell{L0: 1.0 \\ L1: 0.1 \\ L2: 0.1}}   & \multicolumn{1}{c|}{0.67} & 0.57 & \multicolumn{1}{c|}{\makecell{L0: 1.0 \\ L1: 0.45 \\ L2: 0.45}} & \multicolumn{1}{c|}{0.81} & 0.36 & \multicolumn{1}{c|}{\makecell{L0: 0.85 \\ L1: 0.6 \\ L2: 0.6}}  & \multicolumn{1}{c|}{0.81} & 0.21 & \multicolumn{1}{c|}{\makecell{L0: 0.9 \\ L1: 0.55 \\ L2: 0.55}}  & \multicolumn{1}{c|}{0.65} & 0.10  & \multicolumn{1}{c|}{}         & \multicolumn{1}{c|}{}   &     \\ \hline
% \end{tabular}
% \end{table*}


\begin{table*}[ht]
\scriptsize
\centering
\caption{Challenge-3 results, evaluating four IL methods on \taskoriginal, with averages shown across tasks.}
\label{tb:bm3_results}
\begin{tabular}{c|ccc|ccc|ccc|ccc}
\toprule
   & \multicolumn{3}{c|}{BC-RESNET-RNN}                                                   & \multicolumn{3}{c|}{BC-RESNET-T}                                                     & \multicolumn{3}{c|}{BC-VIT-T}                                                        & \multicolumn{3}{c}{OpenVLA}                                                           \\ \midrule
   & \multicolumn{1}{c|}{Level SR}                     & \multicolumn{1}{c|}{UB}   & DUB  & \multicolumn{1}{c|}{Level SR}                     & \multicolumn{1}{c|}{UB}   & DUB  & \multicolumn{1}{c|}{Level SR}                     & \multicolumn{1}{c|}{UB}   & DUB  & \multicolumn{1}{c|}{Level SR}                      & \multicolumn{1}{c|}{UB}   & DUB   \\ \midrule
1  & \multicolumn{1}{c|}{\makecell{L0: 0.85 \\ L1: 0.3 \\ L2: 0.0}}  & \multicolumn{1}{c|}{0.01} & 0.01 & \multicolumn{1}{c|}{\makecell{L0: 0.95 \\ L1: 0.1 \\ L2: 0.1}}  & \multicolumn{1}{c|}{0.95} & 0.85 & \multicolumn{1}{c|}{\makecell{L0: 1.0 \\ L1: 1.0 \\ L2: 0.2}}   & \multicolumn{1}{c|}{0.92} & 0.72 & \multicolumn{1}{c|}{\makecell{L0: 1.0 \\ L1: 0.0 \\ L2: 0.0}}    & \multicolumn{1}{c|}{0.65} & 0.65  \\ \midrule
2  & \multicolumn{1}{c|}{\makecell{L0: 0.95 \\ L1: 0.0 \\ L2: 0.0}}  & \multicolumn{1}{c|}{0.01} & 0.01 & \multicolumn{1}{c|}{\makecell{L0: 0.9 \\ L1: 0.15 \\ L2: 0.05}} & \multicolumn{1}{c|}{0.95} & 0.90 & \multicolumn{1}{c|}{\makecell{L0: 0.95 \\ L1: 0.25 \\ L2: 0.1}} & \multicolumn{1}{c|}{0.92} & 0.82 & \multicolumn{1}{c|}{\makecell{L0: 1.0 \\ L1: 0.0 \\ L2: 0.0}}    & \multicolumn{1}{c|}{1.00} & 1.00  \\ \midrule
3  & \multicolumn{1}{c|}{\makecell{L0: 0.55 \\ L1: 0.0 \\ L2: 0.0}}  & \multicolumn{1}{c|}{0.00} & 0.00 & \multicolumn{1}{c|}{\makecell{L0: 0.9 \\ L1: 0.0 \\ L2: 0.0}}   & \multicolumn{1}{c|}{0.77} & 0.77 & \multicolumn{1}{c|}{\makecell{L0: 1.0 \\ L1: 0.1 \\ L2: 0.1}}   & \multicolumn{1}{c|}{0.84} & 0.74 & \multicolumn{1}{c|}{\makecell{L0: 1.0 \\ L1: 0.15 \\ L2: 0.05}}  & \multicolumn{1}{c|}{0.70} & 0.65  \\ \midrule
4  & \multicolumn{1}{c|}{\makecell{L0: 0.4 \\ L1: 0.05 \\ L2: 0.0}}  & \multicolumn{1}{c|}{0.01} & 0.01 & \multicolumn{1}{c|}{\makecell{L0: 1.0 \\ L1: 0.7 \\ L2: 0.2}}   & \multicolumn{1}{c|}{0.75} & 0.55 & \multicolumn{1}{c|}{\makecell{L0: 1.0 \\ L1: 0.75 \\ L2: 0.65}} & \multicolumn{1}{c|}{0.66} & 0.01 & \multicolumn{1}{c|}{\makecell{L0: 1.0 \\ L1: 0.75 \\ L2: 0.65}}  & \multicolumn{1}{c|}{0.70} & 0.05  \\ \midrule
5  & \multicolumn{1}{c|}{\makecell{L0: 0.4 \\ L1: 0.0 \\ L2: 0.0}}   & \multicolumn{1}{c|}{0.07} & 0.07 & \multicolumn{1}{c|}{\makecell{L0: 0.6 \\ L1: 0.25 \\ L2: 0.0}}  & \multicolumn{1}{c|}{0.66} & 0.66 & \multicolumn{1}{c|}{\makecell{L0: 0.8 \\ L1: 0.65 \\ L2: 0.6}}  & \multicolumn{1}{c|}{0.69} & 0.09 & \multicolumn{1}{c|}{\makecell{L0: 0.8 \\ L1: 0.65 \\ L2: 0.35}}  & \multicolumn{1}{c|}{0.76} & 0.41  \\ \midrule
6  & \multicolumn{1}{c|}{\makecell{L0: 0.45 \\ L1: 0.2 \\ L2: 0.0}}  & \multicolumn{1}{c|}{0.07} & 0.07 & \multicolumn{1}{c|}{\makecell{L0: 0.75 \\ L1: 0.5 \\ L2: 0.1}}  & \multicolumn{1}{c|}{0.66} & 0.56 & \multicolumn{1}{c|}{\makecell{L0: 0.75 \\ L1: 0.65 \\ L2: 0.3}} & \multicolumn{1}{c|}{0.69} & 0.39 & \multicolumn{1}{c|}{\makecell{L0: 0.8 \\ L1: 0.35 \\ L2: 0.1}}   & \multicolumn{1}{c|}{0.76} & 0.66  \\ \midrule
7  & \multicolumn{1}{c|}{\makecell{L0: 0.0 \\ L1: 0.0 \\ L2: 0.0}}   & \multicolumn{1}{c|}{0.00} & 0.00 & \multicolumn{1}{c|}{\makecell{L0: 0.85 \\ L1: 0.85 \\ L2: 0.3}} & \multicolumn{1}{c|}{0.65} & 0.35 & \multicolumn{1}{c|}{\makecell{L0: 0.95 \\ L1: 0.65 \\ L2: 0.4}} & \multicolumn{1}{c|}{0.62} & 0.22 & \multicolumn{1}{c|}{\makecell{L0: 0.95 \\ L1: 0.65 \\ L2: 0.15}} & \multicolumn{1}{c|}{0.34} & 0.19  \\ \midrule
8  & \multicolumn{1}{c|}{\makecell{L0: 0.4 \\ L1: 0.0 \\ L2: 0.0}}   & \multicolumn{1}{c|}{0.00} & 0.00 & \multicolumn{1}{c|}{\makecell{L0: 0.65 \\ L1: 0.4 \\ L2: 0.35}} & \multicolumn{1}{c|}{0.65} & 0.30 & \multicolumn{1}{c|}{\makecell{L0: 0.7 \\ L1: 0.7 \\ L2: 0.5}}   & \multicolumn{1}{c|}{0.62} & 0.12 & \multicolumn{1}{c|}{\makecell{L0: 0.5 \\ L1: 0.25 \\ L2: 0.2}}   & \multicolumn{1}{c|}{0.34} & 0.14  \\ \midrule
9  & \multicolumn{1}{c|}{\makecell{L0: 0.8 \\ L1: 0.05 \\ L2: 0.05}} & \multicolumn{1}{c|}{0.55} & 0.50 & \multicolumn{1}{c|}{\makecell{L0: 0.9 \\ L1: 0.2 \\ L2: 0.2}}   & \multicolumn{1}{c|}{0.79} & 0.59 & \multicolumn{1}{c|}{\makecell{L0: 0.7 \\ L1: 0.5 \\ L2: 0.5}}   & \multicolumn{1}{c|}{0.62} & 0.12 & \multicolumn{1}{c|}{\makecell{L0: 0.8 \\ L1: 0.7 \\ L2: 0.7}}    & \multicolumn{1}{c|}{0.46} & -0.24 \\ \midrule
10 & \multicolumn{1}{c|}{\makecell{L0: 1.0 \\ L1: 0.1 \\ L2: 0.1}}   & \multicolumn{1}{c|}{0.67} & 0.57 & \multicolumn{1}{c|}{\makecell{L0: 1.0 \\ L1: 0.45 \\ L2: 0.45}} & \multicolumn{1}{c|}{0.81} & 0.36 & \multicolumn{1}{c|}{\makecell{L0: 0.85 \\ L1: 0.6 \\ L2: 0.6}}  & \multicolumn{1}{c|}{0.81} & 0.21 & \multicolumn{1}{c|}{\makecell{L0: 0.9 \\ L1: 0.55 \\ L2: 0.55}}  & \multicolumn{1}{c|}{0.65} & 0.10  \\ \midrule
\end{tabular}
\end{table*}


\begin{figure*}[ht]
    \centering
    \includegraphics[width=1.0\textwidth]{imgs/ch3_results.png} 
    \caption{This figure presents the results for \bmc, where the ``Delta to Upper Bound Ratio'' (bars) is positive and notably high in most cases, highlighting the substantial negative impact of \pb on long-horizon task completion. 
    % \my{no legend, maybe add an avg column? Show the upper bound number as a horizontal line or it's weird to have many zeros?}
    }
    \vspace{-1.75em}
    \label{fig:ch3_results}
\end{figure*}

We include \bmc to evaluate the impact of \pb on skill chaining. As described in Section~\ref{subsec:bm_3}, we test baselines on 10 manually designed tasks, each comprising a skill chain of three skills, and use the ``Delta to Upper Bound Ratio'' metric to quantify \pb's effect on performance. Figure~\ref{fig:ch3_results} presents the results, the ``Delta to Upper Bound Ratio'' is shown as bar charts, where different baselines are shown in varied colors. The ``Delta to Upper Bound Ratio'' values are positive and high in most cases. Note that some bars for BC-RESNET-RNN show 0\% values because the ``Chain Upper Bound'' for those tasks is already 0\%, indicating that BC-RESNET-RNN fails even without \pb. These results further emphasize \pb's detrimental effect on long-horizon task completion.
% \my{explain task 4/5 that different baselines perform so different}
% \subsection{Factors Influencing \pb}
\label{sec:factors}

% \yy{Dan suggests to delete the whole section.}

In this section, we investigate two potential factors influencing the severity of \pb: skill difficulty and the magnitude of visual modifications. This analysis aims to provide insights for future research and inform the development of algorithms robust to \pb. We define the following two hypotheses:

\begin{enumerate}[I:]
    \item Higher skill difficulty (i.e., lower skill success rate) leads to more severe performance degradation caused by \pb for that skill.
    \item Larger magnitudes of visual modifications result in more severe performance degradation caused by \pb for the current skill.
\end{enumerate}

To test these hypotheses, we use Spearman's rank correlation coefficient ($\rho$)~\cite{spearman1961proof}, which effectively measures the strength of monotonic relationships between two variables. If hypothesis I holds, we expect a significant negative monotonic relationship (i.e., $\rho < 0$, $p < 0.05$) between skill success rate and the severity of \pb. Similarly, if hypothesis II is valid, we anticipate a significant positive monotonic relationship (i.e., $\rho > 0$, $p < 0.05$) between the magnitude of visual modifications and the severity of \pb.



\subsubsection{Test Hypothesis I (Factor: Task Difficulty)}
\begin{table}[ht]
\centering
\caption{Spearman’s rank correlation coefficients ($\rho$) and p-values ($p$) for the relationship between task difficulty and OSS severity.}
\label{tb:fac1_results}
\begin{tabular}{|c|c|c|c|}
\hline
      & BC-RESNET-RNN                                                & BC-RESNET-T                                                  & BC-VIT-T                                                     \\ \hline
Set 1 & \begin{tabular}[c]{@{}c@{}}$\rho$: -0.86\\ $p$: 0.06\end{tabular} & \begin{tabular}[c]{@{}c@{}}$\rho$: 0.02\\ $p$: 0.97\end{tabular}  & \begin{tabular}[c]{@{}c@{}}$\rho$: 0.04\\ $p$: 0.94\end{tabular}  \\ \hline
Set 2 & \begin{tabular}[c]{@{}c@{}}$\rho$: 0.89\\ $p$: 0.04\end{tabular}  & \begin{tabular}[c]{@{}c@{}}$\rho$: -0.04\\ $p$: 0.94\end{tabular} & \begin{tabular}[c]{@{}c@{}}$\rho$: 0.78\\ $p$: 0.04\end{tabular}  \\ \hline
Set 3 & \begin{tabular}[c]{@{}c@{}}$\rho$: -0.80\\ $p$: 0.20\end{tabular} & \begin{tabular}[c]{@{}c@{}}$\rho$: 0.50\\ $p$: 0.39\end{tabular}  & \begin{tabular}[c]{@{}c@{}}$\rho$: 0.55\\ $p$: 0.33\end{tabular}  \\ \hline
Set 4 & \begin{tabular}[c]{@{}c@{}}$\rho$: 0.95\\ $p$: 0.05\end{tabular}  & \begin{tabular}[c]{@{}c@{}}$\rho$: -0.21\\ $p$: 0.74\end{tabular} & \begin{tabular}[c]{@{}c@{}}$\rho$: 0.15\\ $p$: 0.80\end{tabular}  \\ \hline
Set 5 & \begin{tabular}[c]{@{}c@{}}$\rho$: 0.21\\ $p$: 0.79\end{tabular}  & \begin{tabular}[c]{@{}c@{}}$\rho$: -0.34\\ $p$: 0.57\end{tabular} & \begin{tabular}[c]{@{}c@{}}$\rho$: -0.67\\ $p$: 0.22\end{tabular} \\ \hline
Set 6 & \begin{tabular}[c]{@{}c@{}}$\rho$: 0.89\\ $p$: 0.11\end{tabular}  & \begin{tabular}[c]{@{}c@{}}$\rho$: 0.79\\ $p$: 0.11\end{tabular}  & \begin{tabular}[c]{@{}c@{}}$\rho$: 0.87\\ $p$: 0.05\end{tabular}  \\ \hline
Avg.  & \begin{tabular}[c]{@{}c@{}}$\rho$: 0.21\\ $p$: 0.21\end{tabular}  & \begin{tabular}[c]{@{}c@{}}$\rho$: 0.12\\ $p$: 0.62\end{tabular}  & \begin{tabular}[c]{@{}c@{}}$\rho$: 0.29\\ $p$: 0.40\end{tabular}  \\ \hline
\end{tabular}
\end{table}



We use the metric ``Ratio Performance Delta'' to quantify the severity of \pb and measure task difficulty using task success rate, assuming demonstrations for all skills are of similar quality. To analyze the relationship between skill difficulty and \pb, we design an experiment that isolates skill difficulty as the only variable. We select six sets of skills, where all skills within each set are performed in the same scene (e.g., kitchen-scene-1 in Libero). Identical modifications are applied to all skills within each set to simulate \pb, resulting in several pairs of skills: one unaffected by \pb and the other affected by \pb. Between pairs, the only difference is the skill's difficulty. After evaluating BCs on these pairs, we generate two lists of results for each set and algorithm: one for success rates of skills unaffected by \pb (i.e., task difficulty) and another for ``Ratio Performance Delta'' on skills affected by \pb (i.e., severity of \pb). These results are used to calculate Spearman's rank correlation coefficient ($\rho$) and the corresponding p-value. Table~\ref{tb:fac1_results} presents the findings. None of the baselines show a significant negative monotonic relationship (i.e., $\rho < 0$, $p < 0.05$), indicating no evidence to support the hypothesis that task difficulty is a key factor influencing the severity of \pb.


\subsubsection{Test Hypothesis II (Factor: Magnitude of Visual Modification)}
% \begin{table}[]
\centering
\caption{Results on the relationship between visual modification size and OSS severity.}
\label{tb:fac2_results}
\begin{tabular}{|c|c|c|c|}
\hline
                                                                 & BC-RESNET-RNN & BC-RESNET-T & BC-VIT-T \\ \hline
\begin{tabular}[c]{@{}c@{}}PD\\ (1 modification)\end{tabular} & 0.13          & 0.06        & 0.10     \\ \hline
\begin{tabular}[c]{@{}c@{}}PD\\ (2 modification)\end{tabular} & 0.25          & 0.24        & 0.13     \\ \hline
\begin{tabular}[c]{@{}c@{}}PD\\ (3 modification)\end{tabular} & 0.77          & 0.77        & 0.34     \\ \hline
\end{tabular}
\end{table}



We also use the metric ``Ratio Performance Delta'' to quantify the severity of \pb. To ensure that the magnitude of visual modification is the only variable changing, we adopt an approach similar to \bmb by increasing the number of modifications applied to the same skill. However, unlike \bmb, where modifications are generated randomly by RAMG, it is possible for skills with a larger number of modifications (e.g., affecting several small areas) to have a smaller visual modification magnitude than skills with fewer but more significant modifications (e.g., affecting a large area). To address this, we incrementally add new modifications based on existing ones, ensuring a clear correlation between the number of modifications and the size of the visual changes. After evaluating BCs on skills unaffected by \pb and skills with multiple modifications, we generate two lists for each algorithm: one representing the number of modifications (i.e., visual modification magnitude) and the other capturing the RPD (i.e., severity of \pb). Spearman's rank correlation coefficient ($\rho$) and the corresponding p-value are calculated to analyze the relationship between these two variables. For all BCs, we obtain results as $\rho = 1.0$ and $p = 0.0$, indicating a significant positive monotonic relationship between visual modification magnitude and the severity of \pb. These findings confirm that the magnitude of visual modifications is a key factor influencing the severity of \pb.
\section{Can Data Augmentation Mitigate \pb?}
\label{sec:da}


\subsection{Data Augmentation}
\label{subsec:our_bl}
% We propose a simple baseline, \bl, which additionally provides a corresponding dataset as a valuable byproduct. As defined in Section~\ref{sec:problem_definition}, the \pb problem arises when visual changes occur due to previous skills. A straightforward solution is to collect a larger set of demonstrations, allowing algorithms to learn from a diverse observation space and become robust to novel visual modifications. To explore the effectiveness of this potential solution to \pb, we augments demonstrations and train all the aforementioned baselines based on it to provide guidance for future research on the \pb problem. 

% \gb{Subsection title is not very informative and just generally sounds weird. It also doesn't fit well with the benchmark section. Lastly, I think it hurts your paper to first introduce the benchmark and then suggest/imply that some simple approach can provide a solution to your benchmark. It defeats the purpose of your benchmark. I would formulate this as some additional baselines rather than the solution}


% \gb{You can just call this Baselines with Data Augmentation or something like that. This is essentially just data augmentation. Not sure why we need to use vague terminology such as "potential solution" to a technique that's already widely used in other fields} %, offering valuable insights for future research on addressing the \pb problem. 

% For each skill-level task, we scalably generate modified tasks with \pb problem using a \textbf{Rule-based Automatic Modification Generator (RAMG)}.\gb{To me the flow of this is not great, i.e., that you start talking about this 2 sections ago, don't conclude it, then have a separate completely unrelated subsection, and then go back to this again.} RAMG is a systematic algorithm designed to enhance diversity in pre-defined environments through structured, rule-based modifications. Operating on PDDL files, RAMG iteratively alters object positions, introduces new objects, or modifies object states within the environment. It supports three types of modifications: (1) repositioning or adding external objects to specific regions, (2) changing the states of fixtures (e.g., opening or closing cabinets), and (3) placing small objects into designated containers. These modifications follow dynamic constraints (e.g., newly added objects do not create obstacles) and maintain logical consistency (e.g., a drawer cannot be simultaneously open and closed) to ensure they do not hinder the agent from achieving the task goal. By utilizing deterministic yet flexible rules, RAMG offers a scalable approach to generating task variations, creating up to 1,727 modified tasks from the 44 tasks in \taskoriginal.


As defined in Section~\ref{sec:problem_definition}, the \pb problem occurs when visual changes caused by preceding skills disrupt the execution of the current skill. A straightforward solution is to collect demonstrations with more diverse visual content, allowing algorithms to learn from varied observation spaces and become more robust to novel visual modifications. To simulate this approach, we augment the demonstrations and train all the baselines on an expanded dataset.

Using RAMG, we generate diverse modified tasks, creating a total of 1,727 new tasks from the 44 selected tasks with a single modification applied to each. For each modified task, we replay demonstrations in the corresponding environment, filtering out failed replays to produce new demonstrations where the trajectory remains unchanged, yet with varied visual observations. This method produces a significantly larger dataset, totaling 57,000 demonstrations—nearly 30 times the size of the original Libero dataset for the 44 selected tasks (2,000 demonstrations). Notably, this figure corresponds to cases with a single modification applied; an even much larger dataset can be generated by iteratively using RAMG to apply multiple modifications. We then pre-train (BCs) or fine-tune (OpenVLA) all baselines using this expanded dataset.




% \begin{table}[]
% \centering
% \caption{This table shows results for the potential solution.\gb{Need a more detailed caption. Except for the column names, I don't know what any of these mean}}
% \label{tb:da_results}
% \begin{tabular}{|c|c|c|c|c|}
% \hline
%     & BC-RESNET-RNN & BC-RESNET-T & BC-VIT-T & OpenVLA \\ \hline
% % A   & 0.43          & 0.83        & 0.84     & 0.81    \\ \hline
% % B   & 0.30          & 0.53        & 0.64     & 0.70    \\ \hline
% Setup A   & 0.25          & 0.67        & 0.74     & 0.54    \\ \hline
% Setup B   & 0.31          & 0.54        & 0.64     & 0.54    \\ \hline
% % B-D & -0.01         & -0.01       & 0.00     & 0.16    \\ \hline
% % A-B & 0.13          & 0.30        & 0.20     & 0.11    \\ \hline
% \begin{tabular}[c]{@{}c@{}}Comparison \\ (A - B)\end{tabular} & -0.06         & 0.13        & 0.10     & 0.00    \\ \hline
% \end{tabular}
% \end{table}


\begin{table}[ht]
\centering
\caption{Comparison of baseline performance on Setup A and Setup B. Setup A: baselines trained on original demonstrations and evaluated on skills affected by \pb. Setup B: baselines trained on augmented demonstrations and evaluated on the same skills affected by \pb. The comparison reflects the difference in results between the two setups.}
\label{tb:da_results}
\begin{tabular}{c|c|c|c}
\toprule
                        & Setup A        & Setup B        & \begin{tabular}[c]{@{}c@{}}Comparison \\ (A - B)\end{tabular} \\ \midrule
BC-RESNET-RNN           & 0.25          & 0.31           & -0.06                                                      \\ 
% \hline
BC-RESNET-T             & 0.67          & 0.54           & 0.13                                                       \\ 
% \hline
BC-VIT-T                & 0.74          & 0.64           & 0.10                                                       \\ 
% \hline
OpenVLA                 & 0.54          & 0.54           & 0.00                                                       \\ \bottomrule
\end{tabular}
\vspace{-2em}
\end{table}


\subsection{Results}
\label{subsec:results_bl}


% \begin{table}[]
% \caption{Your table description here.}
% \centering
% \begin{tabular}{|c|c|c|}
% \hline
% \diagbox{Training}{Evaluation}                                 & \begin{tabular}[c]{@{}c@{}}Demonstrations \\ for Skill-X\end{tabular} & \begin{tabular}[c]{@{}c@{}}Augmented Demonstrations \\ for Skill-X\end{tabular} \\ \hline
% Skill-X                                                            & A                                                                     & B                                                                                        \\ \hline
% \begin{tabular}[c]{@{}c@{}}Skill-X \\ Affected by OSS\end{tabular} & C                                                                     & D                                                                                        \\ \hline
% \end{tabular}
% \end{table}

To investigate whether the data augmentation method mitigates the \pb problem, we compare performance between two setups to assess whether improvements exist on skills affected by \pb. Setup A: baselines are trained on original demonstrations for a skill and evaluated on the skill affected by \pb. Setup B: baselines are trained on augmented demonstrations and evaluated on the same skill affected by \pb. Table~\ref{tb:da_results} presents the performance values for both setups across all baselines, with averaged results reported for all 44 tasks. A comparison between the setups reveals that BC-RESNET-RNN shows a slight improvement under Setup B. However, other baselines, including BC-RESNET-T, BC-VIT-T, and OpenVLA, perform equivalently or worse, suggesting that data augmentation alone is limited in mitigating \pb. While larger and more diverse datasets generally improve performance, they may not fully address \pb. One possible reason is that some baselines struggle to generalize across the diverse visual distributions of demonstrations, leading to even worse performance. Another challenge is the combinatorial explosion of scene and task variations, making it impractical to cover all relevant cases through data augmentation alone. This limitation persists even with RAMG, which is constrained to simulated data generation and cannot scale to real-world data. These findings underscore the need for tailored algorithmic solutions to effectively address \pb.

% A comparison between the setups reveals that BC-RESNET-RNN shows slight improvement under Setup B. However, other baselines, including BC-RESNET-T, BC-VIT-T, and OpenVLA, perform equivalently or worse\gb{Why would the results for Setup B be worse? Can you provide an intuitive explanation of what's happening?}, highlighting the limited effectiveness of data augmentation in mitigating \pb. These findings demonstrate that simply training on larger and more diverse datasets is insufficient to address \pb effectively, underscoring the need for tailored algorithmic solutions.\gb{For me this is a very counterintuitive statement/explanation. I would actually disagree with this. We know that as long as there's sufficiently diverse/large data deep learning algorithms will learn those cases. The main issue is generalization. Therefore, for me the question after reading this statement, is whether you mean that we cannot generate all kinds of combinations of the data that might contribute to OSS? If so, maybe you should explicitly discuss this, i.e., combinatorial space explosion, and the difficulty to generate data for all possible combinations. Otherwise, if that's not an issue, I disagree with your statement, i.e., if you had sufficiently diverse+large dataset, I believe you would be able to solve this problem as evidenced by everything that's happening in deep learning in the last 5-10 years.}



\section{Conclusion}
In this work, we propose a simple yet effective approach, called SMILE, for graph few-shot learning with fewer tasks. Specifically, we introduce a novel dual-level mixup strategy, including within-task and across-task mixup, for enriching the diversity of nodes within each task and the diversity of tasks. Also, we incorporate the degree-based prior information to learn expressive node embeddings. Theoretically, we prove that SMILE effectively enhances the model's generalization performance. Empirically, we conduct extensive experiments on multiple benchmarks and the results suggest that SMILE significantly outperforms other baselines, including both in-domain and cross-domain few-shot settings.


% % Bibliography
% \bibliographystyle{IEEEtran}
% \bibliography{refs}

\bibliographystyle{IEEEtran}
\bibliography{refs}


\subsection{Lloyd-Max Algorithm}
\label{subsec:Lloyd-Max}
For a given quantization bitwidth $B$ and an operand $\bm{X}$, the Lloyd-Max algorithm finds $2^B$ quantization levels $\{\hat{x}_i\}_{i=1}^{2^B}$ such that quantizing $\bm{X}$ by rounding each scalar in $\bm{X}$ to the nearest quantization level minimizes the quantization MSE. 

The algorithm starts with an initial guess of quantization levels and then iteratively computes quantization thresholds $\{\tau_i\}_{i=1}^{2^B-1}$ and updates quantization levels $\{\hat{x}_i\}_{i=1}^{2^B}$. Specifically, at iteration $n$, thresholds are set to the midpoints of the previous iteration's levels:
\begin{align*}
    \tau_i^{(n)}=\frac{\hat{x}_i^{(n-1)}+\hat{x}_{i+1}^{(n-1)}}2 \text{ for } i=1\ldots 2^B-1
\end{align*}
Subsequently, the quantization levels are re-computed as conditional means of the data regions defined by the new thresholds:
\begin{align*}
    \hat{x}_i^{(n)}=\mathbb{E}\left[ \bm{X} \big| \bm{X}\in [\tau_{i-1}^{(n)},\tau_i^{(n)}] \right] \text{ for } i=1\ldots 2^B
\end{align*}
where to satisfy boundary conditions we have $\tau_0=-\infty$ and $\tau_{2^B}=\infty$. The algorithm iterates the above steps until convergence.

Figure \ref{fig:lm_quant} compares the quantization levels of a $7$-bit floating point (E3M3) quantizer (left) to a $7$-bit Lloyd-Max quantizer (right) when quantizing a layer of weights from the GPT3-126M model at a per-tensor granularity. As shown, the Lloyd-Max quantizer achieves substantially lower quantization MSE. Further, Table \ref{tab:FP7_vs_LM7} shows the superior perplexity achieved by Lloyd-Max quantizers for bitwidths of $7$, $6$ and $5$. The difference between the quantizers is clear at 5 bits, where per-tensor FP quantization incurs a drastic and unacceptable increase in perplexity, while Lloyd-Max quantization incurs a much smaller increase. Nevertheless, we note that even the optimal Lloyd-Max quantizer incurs a notable ($\sim 1.5$) increase in perplexity due to the coarse granularity of quantization. 

\begin{figure}[h]
  \centering
  \includegraphics[width=0.7\linewidth]{sections/figures/LM7_FP7.pdf}
  \caption{\small Quantization levels and the corresponding quantization MSE of Floating Point (left) vs Lloyd-Max (right) Quantizers for a layer of weights in the GPT3-126M model.}
  \label{fig:lm_quant}
\end{figure}

\begin{table}[h]\scriptsize
\begin{center}
\caption{\label{tab:FP7_vs_LM7} \small Comparing perplexity (lower is better) achieved by floating point quantizers and Lloyd-Max quantizers on a GPT3-126M model for the Wikitext-103 dataset.}
\begin{tabular}{c|cc|c}
\hline
 \multirow{2}{*}{\textbf{Bitwidth}} & \multicolumn{2}{|c|}{\textbf{Floating-Point Quantizer}} & \textbf{Lloyd-Max Quantizer} \\
 & Best Format & Wikitext-103 Perplexity & Wikitext-103 Perplexity \\
\hline
7 & E3M3 & 18.32 & 18.27 \\
6 & E3M2 & 19.07 & 18.51 \\
5 & E4M0 & 43.89 & 19.71 \\
\hline
\end{tabular}
\end{center}
\end{table}

\subsection{Proof of Local Optimality of LO-BCQ}
\label{subsec:lobcq_opt_proof}
For a given block $\bm{b}_j$, the quantization MSE during LO-BCQ can be empirically evaluated as $\frac{1}{L_b}\lVert \bm{b}_j- \bm{\hat{b}}_j\rVert^2_2$ where $\bm{\hat{b}}_j$ is computed from equation (\ref{eq:clustered_quantization_definition}) as $C_{f(\bm{b}_j)}(\bm{b}_j)$. Further, for a given block cluster $\mathcal{B}_i$, we compute the quantization MSE as $\frac{1}{|\mathcal{B}_{i}|}\sum_{\bm{b} \in \mathcal{B}_{i}} \frac{1}{L_b}\lVert \bm{b}- C_i^{(n)}(\bm{b})\rVert^2_2$. Therefore, at the end of iteration $n$, we evaluate the overall quantization MSE $J^{(n)}$ for a given operand $\bm{X}$ composed of $N_c$ block clusters as:
\begin{align*}
    \label{eq:mse_iter_n}
    J^{(n)} = \frac{1}{N_c} \sum_{i=1}^{N_c} \frac{1}{|\mathcal{B}_{i}^{(n)}|}\sum_{\bm{v} \in \mathcal{B}_{i}^{(n)}} \frac{1}{L_b}\lVert \bm{b}- B_i^{(n)}(\bm{b})\rVert^2_2
\end{align*}

At the end of iteration $n$, the codebooks are updated from $\mathcal{C}^{(n-1)}$ to $\mathcal{C}^{(n)}$. However, the mapping of a given vector $\bm{b}_j$ to quantizers $\mathcal{C}^{(n)}$ remains as  $f^{(n)}(\bm{b}_j)$. At the next iteration, during the vector clustering step, $f^{(n+1)}(\bm{b}_j)$ finds new mapping of $\bm{b}_j$ to updated codebooks $\mathcal{C}^{(n)}$ such that the quantization MSE over the candidate codebooks is minimized. Therefore, we obtain the following result for $\bm{b}_j$:
\begin{align*}
\frac{1}{L_b}\lVert \bm{b}_j - C_{f^{(n+1)}(\bm{b}_j)}^{(n)}(\bm{b}_j)\rVert^2_2 \le \frac{1}{L_b}\lVert \bm{b}_j - C_{f^{(n)}(\bm{b}_j)}^{(n)}(\bm{b}_j)\rVert^2_2
\end{align*}

That is, quantizing $\bm{b}_j$ at the end of the block clustering step of iteration $n+1$ results in lower quantization MSE compared to quantizing at the end of iteration $n$. Since this is true for all $\bm{b} \in \bm{X}$, we assert the following:
\begin{equation}
\begin{split}
\label{eq:mse_ineq_1}
    \tilde{J}^{(n+1)} &= \frac{1}{N_c} \sum_{i=1}^{N_c} \frac{1}{|\mathcal{B}_{i}^{(n+1)}|}\sum_{\bm{b} \in \mathcal{B}_{i}^{(n+1)}} \frac{1}{L_b}\lVert \bm{b} - C_i^{(n)}(b)\rVert^2_2 \le J^{(n)}
\end{split}
\end{equation}
where $\tilde{J}^{(n+1)}$ is the the quantization MSE after the vector clustering step at iteration $n+1$.

Next, during the codebook update step (\ref{eq:quantizers_update}) at iteration $n+1$, the per-cluster codebooks $\mathcal{C}^{(n)}$ are updated to $\mathcal{C}^{(n+1)}$ by invoking the Lloyd-Max algorithm \citep{Lloyd}. We know that for any given value distribution, the Lloyd-Max algorithm minimizes the quantization MSE. Therefore, for a given vector cluster $\mathcal{B}_i$ we obtain the following result:

\begin{equation}
    \frac{1}{|\mathcal{B}_{i}^{(n+1)}|}\sum_{\bm{b} \in \mathcal{B}_{i}^{(n+1)}} \frac{1}{L_b}\lVert \bm{b}- C_i^{(n+1)}(\bm{b})\rVert^2_2 \le \frac{1}{|\mathcal{B}_{i}^{(n+1)}|}\sum_{\bm{b} \in \mathcal{B}_{i}^{(n+1)}} \frac{1}{L_b}\lVert \bm{b}- C_i^{(n)}(\bm{b})\rVert^2_2
\end{equation}

The above equation states that quantizing the given block cluster $\mathcal{B}_i$ after updating the associated codebook from $C_i^{(n)}$ to $C_i^{(n+1)}$ results in lower quantization MSE. Since this is true for all the block clusters, we derive the following result: 
\begin{equation}
\begin{split}
\label{eq:mse_ineq_2}
     J^{(n+1)} &= \frac{1}{N_c} \sum_{i=1}^{N_c} \frac{1}{|\mathcal{B}_{i}^{(n+1)}|}\sum_{\bm{b} \in \mathcal{B}_{i}^{(n+1)}} \frac{1}{L_b}\lVert \bm{b}- C_i^{(n+1)}(\bm{b})\rVert^2_2  \le \tilde{J}^{(n+1)}   
\end{split}
\end{equation}

Following (\ref{eq:mse_ineq_1}) and (\ref{eq:mse_ineq_2}), we find that the quantization MSE is non-increasing for each iteration, that is, $J^{(1)} \ge J^{(2)} \ge J^{(3)} \ge \ldots \ge J^{(M)}$ where $M$ is the maximum number of iterations. 
%Therefore, we can say that if the algorithm converges, then it must be that it has converged to a local minimum. 
\hfill $\blacksquare$


\begin{figure}
    \begin{center}
    \includegraphics[width=0.5\textwidth]{sections//figures/mse_vs_iter.pdf}
    \end{center}
    \caption{\small NMSE vs iterations during LO-BCQ compared to other block quantization proposals}
    \label{fig:nmse_vs_iter}
\end{figure}

Figure \ref{fig:nmse_vs_iter} shows the empirical convergence of LO-BCQ across several block lengths and number of codebooks. Also, the MSE achieved by LO-BCQ is compared to baselines such as MXFP and VSQ. As shown, LO-BCQ converges to a lower MSE than the baselines. Further, we achieve better convergence for larger number of codebooks ($N_c$) and for a smaller block length ($L_b$), both of which increase the bitwidth of BCQ (see Eq \ref{eq:bitwidth_bcq}).


\subsection{Additional Accuracy Results}
%Table \ref{tab:lobcq_config} lists the various LOBCQ configurations and their corresponding bitwidths.
\begin{table}
\setlength{\tabcolsep}{4.75pt}
\begin{center}
\caption{\label{tab:lobcq_config} Various LO-BCQ configurations and their bitwidths.}
\begin{tabular}{|c||c|c|c|c||c|c||c|} 
\hline
 & \multicolumn{4}{|c||}{$L_b=8$} & \multicolumn{2}{|c||}{$L_b=4$} & $L_b=2$ \\
 \hline
 \backslashbox{$L_A$\kern-1em}{\kern-1em$N_c$} & 2 & 4 & 8 & 16 & 2 & 4 & 2 \\
 \hline
 64 & 4.25 & 4.375 & 4.5 & 4.625 & 4.375 & 4.625 & 4.625\\
 \hline
 32 & 4.375 & 4.5 & 4.625& 4.75 & 4.5 & 4.75 & 4.75 \\
 \hline
 16 & 4.625 & 4.75& 4.875 & 5 & 4.75 & 5 & 5 \\
 \hline
\end{tabular}
\end{center}
\end{table}

%\subsection{Perplexity achieved by various LO-BCQ configurations on Wikitext-103 dataset}

\begin{table} \centering
\begin{tabular}{|c||c|c|c|c||c|c||c|} 
\hline
 $L_b \rightarrow$& \multicolumn{4}{c||}{8} & \multicolumn{2}{c||}{4} & 2\\
 \hline
 \backslashbox{$L_A$\kern-1em}{\kern-1em$N_c$} & 2 & 4 & 8 & 16 & 2 & 4 & 2  \\
 %$N_c \rightarrow$ & 2 & 4 & 8 & 16 & 2 & 4 & 2 \\
 \hline
 \hline
 \multicolumn{8}{c}{GPT3-1.3B (FP32 PPL = 9.98)} \\ 
 \hline
 \hline
 64 & 10.40 & 10.23 & 10.17 & 10.15 &  10.28 & 10.18 & 10.19 \\
 \hline
 32 & 10.25 & 10.20 & 10.15 & 10.12 &  10.23 & 10.17 & 10.17 \\
 \hline
 16 & 10.22 & 10.16 & 10.10 & 10.09 &  10.21 & 10.14 & 10.16 \\
 \hline
  \hline
 \multicolumn{8}{c}{GPT3-8B (FP32 PPL = 7.38)} \\ 
 \hline
 \hline
 64 & 7.61 & 7.52 & 7.48 &  7.47 &  7.55 &  7.49 & 7.50 \\
 \hline
 32 & 7.52 & 7.50 & 7.46 &  7.45 &  7.52 &  7.48 & 7.48  \\
 \hline
 16 & 7.51 & 7.48 & 7.44 &  7.44 &  7.51 &  7.49 & 7.47  \\
 \hline
\end{tabular}
\caption{\label{tab:ppl_gpt3_abalation} Wikitext-103 perplexity across GPT3-1.3B and 8B models.}
\end{table}

\begin{table} \centering
\begin{tabular}{|c||c|c|c|c||} 
\hline
 $L_b \rightarrow$& \multicolumn{4}{c||}{8}\\
 \hline
 \backslashbox{$L_A$\kern-1em}{\kern-1em$N_c$} & 2 & 4 & 8 & 16 \\
 %$N_c \rightarrow$ & 2 & 4 & 8 & 16 & 2 & 4 & 2 \\
 \hline
 \hline
 \multicolumn{5}{|c|}{Llama2-7B (FP32 PPL = 5.06)} \\ 
 \hline
 \hline
 64 & 5.31 & 5.26 & 5.19 & 5.18  \\
 \hline
 32 & 5.23 & 5.25 & 5.18 & 5.15  \\
 \hline
 16 & 5.23 & 5.19 & 5.16 & 5.14  \\
 \hline
 \multicolumn{5}{|c|}{Nemotron4-15B (FP32 PPL = 5.87)} \\ 
 \hline
 \hline
 64  & 6.3 & 6.20 & 6.13 & 6.08  \\
 \hline
 32  & 6.24 & 6.12 & 6.07 & 6.03  \\
 \hline
 16  & 6.12 & 6.14 & 6.04 & 6.02  \\
 \hline
 \multicolumn{5}{|c|}{Nemotron4-340B (FP32 PPL = 3.48)} \\ 
 \hline
 \hline
 64 & 3.67 & 3.62 & 3.60 & 3.59 \\
 \hline
 32 & 3.63 & 3.61 & 3.59 & 3.56 \\
 \hline
 16 & 3.61 & 3.58 & 3.57 & 3.55 \\
 \hline
\end{tabular}
\caption{\label{tab:ppl_llama7B_nemo15B} Wikitext-103 perplexity compared to FP32 baseline in Llama2-7B and Nemotron4-15B, 340B models}
\end{table}

%\subsection{Perplexity achieved by various LO-BCQ configurations on MMLU dataset}


\begin{table} \centering
\begin{tabular}{|c||c|c|c|c||c|c|c|c|} 
\hline
 $L_b \rightarrow$& \multicolumn{4}{c||}{8} & \multicolumn{4}{c||}{8}\\
 \hline
 \backslashbox{$L_A$\kern-1em}{\kern-1em$N_c$} & 2 & 4 & 8 & 16 & 2 & 4 & 8 & 16  \\
 %$N_c \rightarrow$ & 2 & 4 & 8 & 16 & 2 & 4 & 2 \\
 \hline
 \hline
 \multicolumn{5}{|c|}{Llama2-7B (FP32 Accuracy = 45.8\%)} & \multicolumn{4}{|c|}{Llama2-70B (FP32 Accuracy = 69.12\%)} \\ 
 \hline
 \hline
 64 & 43.9 & 43.4 & 43.9 & 44.9 & 68.07 & 68.27 & 68.17 & 68.75 \\
 \hline
 32 & 44.5 & 43.8 & 44.9 & 44.5 & 68.37 & 68.51 & 68.35 & 68.27  \\
 \hline
 16 & 43.9 & 42.7 & 44.9 & 45 & 68.12 & 68.77 & 68.31 & 68.59  \\
 \hline
 \hline
 \multicolumn{5}{|c|}{GPT3-22B (FP32 Accuracy = 38.75\%)} & \multicolumn{4}{|c|}{Nemotron4-15B (FP32 Accuracy = 64.3\%)} \\ 
 \hline
 \hline
 64 & 36.71 & 38.85 & 38.13 & 38.92 & 63.17 & 62.36 & 63.72 & 64.09 \\
 \hline
 32 & 37.95 & 38.69 & 39.45 & 38.34 & 64.05 & 62.30 & 63.8 & 64.33  \\
 \hline
 16 & 38.88 & 38.80 & 38.31 & 38.92 & 63.22 & 63.51 & 63.93 & 64.43  \\
 \hline
\end{tabular}
\caption{\label{tab:mmlu_abalation} Accuracy on MMLU dataset across GPT3-22B, Llama2-7B, 70B and Nemotron4-15B models.}
\end{table}


%\subsection{Perplexity achieved by various LO-BCQ configurations on LM evaluation harness}

\begin{table} \centering
\begin{tabular}{|c||c|c|c|c||c|c|c|c|} 
\hline
 $L_b \rightarrow$& \multicolumn{4}{c||}{8} & \multicolumn{4}{c||}{8}\\
 \hline
 \backslashbox{$L_A$\kern-1em}{\kern-1em$N_c$} & 2 & 4 & 8 & 16 & 2 & 4 & 8 & 16  \\
 %$N_c \rightarrow$ & 2 & 4 & 8 & 16 & 2 & 4 & 2 \\
 \hline
 \hline
 \multicolumn{5}{|c|}{Race (FP32 Accuracy = 37.51\%)} & \multicolumn{4}{|c|}{Boolq (FP32 Accuracy = 64.62\%)} \\ 
 \hline
 \hline
 64 & 36.94 & 37.13 & 36.27 & 37.13 & 63.73 & 62.26 & 63.49 & 63.36 \\
 \hline
 32 & 37.03 & 36.36 & 36.08 & 37.03 & 62.54 & 63.51 & 63.49 & 63.55  \\
 \hline
 16 & 37.03 & 37.03 & 36.46 & 37.03 & 61.1 & 63.79 & 63.58 & 63.33  \\
 \hline
 \hline
 \multicolumn{5}{|c|}{Winogrande (FP32 Accuracy = 58.01\%)} & \multicolumn{4}{|c|}{Piqa (FP32 Accuracy = 74.21\%)} \\ 
 \hline
 \hline
 64 & 58.17 & 57.22 & 57.85 & 58.33 & 73.01 & 73.07 & 73.07 & 72.80 \\
 \hline
 32 & 59.12 & 58.09 & 57.85 & 58.41 & 73.01 & 73.94 & 72.74 & 73.18  \\
 \hline
 16 & 57.93 & 58.88 & 57.93 & 58.56 & 73.94 & 72.80 & 73.01 & 73.94  \\
 \hline
\end{tabular}
\caption{\label{tab:mmlu_abalation} Accuracy on LM evaluation harness tasks on GPT3-1.3B model.}
\end{table}

\begin{table} \centering
\begin{tabular}{|c||c|c|c|c||c|c|c|c|} 
\hline
 $L_b \rightarrow$& \multicolumn{4}{c||}{8} & \multicolumn{4}{c||}{8}\\
 \hline
 \backslashbox{$L_A$\kern-1em}{\kern-1em$N_c$} & 2 & 4 & 8 & 16 & 2 & 4 & 8 & 16  \\
 %$N_c \rightarrow$ & 2 & 4 & 8 & 16 & 2 & 4 & 2 \\
 \hline
 \hline
 \multicolumn{5}{|c|}{Race (FP32 Accuracy = 41.34\%)} & \multicolumn{4}{|c|}{Boolq (FP32 Accuracy = 68.32\%)} \\ 
 \hline
 \hline
 64 & 40.48 & 40.10 & 39.43 & 39.90 & 69.20 & 68.41 & 69.45 & 68.56 \\
 \hline
 32 & 39.52 & 39.52 & 40.77 & 39.62 & 68.32 & 67.43 & 68.17 & 69.30  \\
 \hline
 16 & 39.81 & 39.71 & 39.90 & 40.38 & 68.10 & 66.33 & 69.51 & 69.42  \\
 \hline
 \hline
 \multicolumn{5}{|c|}{Winogrande (FP32 Accuracy = 67.88\%)} & \multicolumn{4}{|c|}{Piqa (FP32 Accuracy = 78.78\%)} \\ 
 \hline
 \hline
 64 & 66.85 & 66.61 & 67.72 & 67.88 & 77.31 & 77.42 & 77.75 & 77.64 \\
 \hline
 32 & 67.25 & 67.72 & 67.72 & 67.00 & 77.31 & 77.04 & 77.80 & 77.37  \\
 \hline
 16 & 68.11 & 68.90 & 67.88 & 67.48 & 77.37 & 78.13 & 78.13 & 77.69  \\
 \hline
\end{tabular}
\caption{\label{tab:mmlu_abalation} Accuracy on LM evaluation harness tasks on GPT3-8B model.}
\end{table}

\begin{table} \centering
\begin{tabular}{|c||c|c|c|c||c|c|c|c|} 
\hline
 $L_b \rightarrow$& \multicolumn{4}{c||}{8} & \multicolumn{4}{c||}{8}\\
 \hline
 \backslashbox{$L_A$\kern-1em}{\kern-1em$N_c$} & 2 & 4 & 8 & 16 & 2 & 4 & 8 & 16  \\
 %$N_c \rightarrow$ & 2 & 4 & 8 & 16 & 2 & 4 & 2 \\
 \hline
 \hline
 \multicolumn{5}{|c|}{Race (FP32 Accuracy = 40.67\%)} & \multicolumn{4}{|c|}{Boolq (FP32 Accuracy = 76.54\%)} \\ 
 \hline
 \hline
 64 & 40.48 & 40.10 & 39.43 & 39.90 & 75.41 & 75.11 & 77.09 & 75.66 \\
 \hline
 32 & 39.52 & 39.52 & 40.77 & 39.62 & 76.02 & 76.02 & 75.96 & 75.35  \\
 \hline
 16 & 39.81 & 39.71 & 39.90 & 40.38 & 75.05 & 73.82 & 75.72 & 76.09  \\
 \hline
 \hline
 \multicolumn{5}{|c|}{Winogrande (FP32 Accuracy = 70.64\%)} & \multicolumn{4}{|c|}{Piqa (FP32 Accuracy = 79.16\%)} \\ 
 \hline
 \hline
 64 & 69.14 & 70.17 & 70.17 & 70.56 & 78.24 & 79.00 & 78.62 & 78.73 \\
 \hline
 32 & 70.96 & 69.69 & 71.27 & 69.30 & 78.56 & 79.49 & 79.16 & 78.89  \\
 \hline
 16 & 71.03 & 69.53 & 69.69 & 70.40 & 78.13 & 79.16 & 79.00 & 79.00  \\
 \hline
\end{tabular}
\caption{\label{tab:mmlu_abalation} Accuracy on LM evaluation harness tasks on GPT3-22B model.}
\end{table}

\begin{table} \centering
\begin{tabular}{|c||c|c|c|c||c|c|c|c|} 
\hline
 $L_b \rightarrow$& \multicolumn{4}{c||}{8} & \multicolumn{4}{c||}{8}\\
 \hline
 \backslashbox{$L_A$\kern-1em}{\kern-1em$N_c$} & 2 & 4 & 8 & 16 & 2 & 4 & 8 & 16  \\
 %$N_c \rightarrow$ & 2 & 4 & 8 & 16 & 2 & 4 & 2 \\
 \hline
 \hline
 \multicolumn{5}{|c|}{Race (FP32 Accuracy = 44.4\%)} & \multicolumn{4}{|c|}{Boolq (FP32 Accuracy = 79.29\%)} \\ 
 \hline
 \hline
 64 & 42.49 & 42.51 & 42.58 & 43.45 & 77.58 & 77.37 & 77.43 & 78.1 \\
 \hline
 32 & 43.35 & 42.49 & 43.64 & 43.73 & 77.86 & 75.32 & 77.28 & 77.86  \\
 \hline
 16 & 44.21 & 44.21 & 43.64 & 42.97 & 78.65 & 77 & 76.94 & 77.98  \\
 \hline
 \hline
 \multicolumn{5}{|c|}{Winogrande (FP32 Accuracy = 69.38\%)} & \multicolumn{4}{|c|}{Piqa (FP32 Accuracy = 78.07\%)} \\ 
 \hline
 \hline
 64 & 68.9 & 68.43 & 69.77 & 68.19 & 77.09 & 76.82 & 77.09 & 77.86 \\
 \hline
 32 & 69.38 & 68.51 & 68.82 & 68.90 & 78.07 & 76.71 & 78.07 & 77.86  \\
 \hline
 16 & 69.53 & 67.09 & 69.38 & 68.90 & 77.37 & 77.8 & 77.91 & 77.69  \\
 \hline
\end{tabular}
\caption{\label{tab:mmlu_abalation} Accuracy on LM evaluation harness tasks on Llama2-7B model.}
\end{table}

\begin{table} \centering
\begin{tabular}{|c||c|c|c|c||c|c|c|c|} 
\hline
 $L_b \rightarrow$& \multicolumn{4}{c||}{8} & \multicolumn{4}{c||}{8}\\
 \hline
 \backslashbox{$L_A$\kern-1em}{\kern-1em$N_c$} & 2 & 4 & 8 & 16 & 2 & 4 & 8 & 16  \\
 %$N_c \rightarrow$ & 2 & 4 & 8 & 16 & 2 & 4 & 2 \\
 \hline
 \hline
 \multicolumn{5}{|c|}{Race (FP32 Accuracy = 48.8\%)} & \multicolumn{4}{|c|}{Boolq (FP32 Accuracy = 85.23\%)} \\ 
 \hline
 \hline
 64 & 49.00 & 49.00 & 49.28 & 48.71 & 82.82 & 84.28 & 84.03 & 84.25 \\
 \hline
 32 & 49.57 & 48.52 & 48.33 & 49.28 & 83.85 & 84.46 & 84.31 & 84.93  \\
 \hline
 16 & 49.85 & 49.09 & 49.28 & 48.99 & 85.11 & 84.46 & 84.61 & 83.94  \\
 \hline
 \hline
 \multicolumn{5}{|c|}{Winogrande (FP32 Accuracy = 79.95\%)} & \multicolumn{4}{|c|}{Piqa (FP32 Accuracy = 81.56\%)} \\ 
 \hline
 \hline
 64 & 78.77 & 78.45 & 78.37 & 79.16 & 81.45 & 80.69 & 81.45 & 81.5 \\
 \hline
 32 & 78.45 & 79.01 & 78.69 & 80.66 & 81.56 & 80.58 & 81.18 & 81.34  \\
 \hline
 16 & 79.95 & 79.56 & 79.79 & 79.72 & 81.28 & 81.66 & 81.28 & 80.96  \\
 \hline
\end{tabular}
\caption{\label{tab:mmlu_abalation} Accuracy on LM evaluation harness tasks on Llama2-70B model.}
\end{table}

%\section{MSE Studies}
%\textcolor{red}{TODO}


\subsection{Number Formats and Quantization Method}
\label{subsec:numFormats_quantMethod}
\subsubsection{Integer Format}
An $n$-bit signed integer (INT) is typically represented with a 2s-complement format \citep{yao2022zeroquant,xiao2023smoothquant,dai2021vsq}, where the most significant bit denotes the sign.

\subsubsection{Floating Point Format}
An $n$-bit signed floating point (FP) number $x$ comprises of a 1-bit sign ($x_{\mathrm{sign}}$), $B_m$-bit mantissa ($x_{\mathrm{mant}}$) and $B_e$-bit exponent ($x_{\mathrm{exp}}$) such that $B_m+B_e=n-1$. The associated constant exponent bias ($E_{\mathrm{bias}}$) is computed as $(2^{{B_e}-1}-1)$. We denote this format as $E_{B_e}M_{B_m}$.  

\subsubsection{Quantization Scheme}
\label{subsec:quant_method}
A quantization scheme dictates how a given unquantized tensor is converted to its quantized representation. We consider FP formats for the purpose of illustration. Given an unquantized tensor $\bm{X}$ and an FP format $E_{B_e}M_{B_m}$, we first, we compute the quantization scale factor $s_X$ that maps the maximum absolute value of $\bm{X}$ to the maximum quantization level of the $E_{B_e}M_{B_m}$ format as follows:
\begin{align}
\label{eq:sf}
    s_X = \frac{\mathrm{max}(|\bm{X}|)}{\mathrm{max}(E_{B_e}M_{B_m})}
\end{align}
In the above equation, $|\cdot|$ denotes the absolute value function.

Next, we scale $\bm{X}$ by $s_X$ and quantize it to $\hat{\bm{X}}$ by rounding it to the nearest quantization level of $E_{B_e}M_{B_m}$ as:

\begin{align}
\label{eq:tensor_quant}
    \hat{\bm{X}} = \text{round-to-nearest}\left(\frac{\bm{X}}{s_X}, E_{B_e}M_{B_m}\right)
\end{align}

We perform dynamic max-scaled quantization \citep{wu2020integer}, where the scale factor $s$ for activations is dynamically computed during runtime.

\subsection{Vector Scaled Quantization}
\begin{wrapfigure}{r}{0.35\linewidth}
  \centering
  \includegraphics[width=\linewidth]{sections/figures/vsquant.jpg}
  \caption{\small Vectorwise decomposition for per-vector scaled quantization (VSQ \citep{dai2021vsq}).}
  \label{fig:vsquant}
\end{wrapfigure}
During VSQ \citep{dai2021vsq}, the operand tensors are decomposed into 1D vectors in a hardware friendly manner as shown in Figure \ref{fig:vsquant}. Since the decomposed tensors are used as operands in matrix multiplications during inference, it is beneficial to perform this decomposition along the reduction dimension of the multiplication. The vectorwise quantization is performed similar to tensorwise quantization described in Equations \ref{eq:sf} and \ref{eq:tensor_quant}, where a scale factor $s_v$ is required for each vector $\bm{v}$ that maps the maximum absolute value of that vector to the maximum quantization level. While smaller vector lengths can lead to larger accuracy gains, the associated memory and computational overheads due to the per-vector scale factors increases. To alleviate these overheads, VSQ \citep{dai2021vsq} proposed a second level quantization of the per-vector scale factors to unsigned integers, while MX \citep{rouhani2023shared} quantizes them to integer powers of 2 (denoted as $2^{INT}$).

\subsubsection{MX Format}
The MX format proposed in \citep{rouhani2023microscaling} introduces the concept of sub-block shifting. For every two scalar elements of $b$-bits each, there is a shared exponent bit. The value of this exponent bit is determined through an empirical analysis that targets minimizing quantization MSE. We note that the FP format $E_{1}M_{b}$ is strictly better than MX from an accuracy perspective since it allocates a dedicated exponent bit to each scalar as opposed to sharing it across two scalars. Therefore, we conservatively bound the accuracy of a $b+2$-bit signed MX format with that of a $E_{1}M_{b}$ format in our comparisons. For instance, we use E1M2 format as a proxy for MX4.

\begin{figure}
    \centering
    \includegraphics[width=1\linewidth]{sections//figures/BlockFormats.pdf}
    \caption{\small Comparing LO-BCQ to MX format.}
    \label{fig:block_formats}
\end{figure}

Figure \ref{fig:block_formats} compares our $4$-bit LO-BCQ block format to MX \citep{rouhani2023microscaling}. As shown, both LO-BCQ and MX decompose a given operand tensor into block arrays and each block array into blocks. Similar to MX, we find that per-block quantization ($L_b < L_A$) leads to better accuracy due to increased flexibility. While MX achieves this through per-block $1$-bit micro-scales, we associate a dedicated codebook to each block through a per-block codebook selector. Further, MX quantizes the per-block array scale-factor to E8M0 format without per-tensor scaling. In contrast during LO-BCQ, we find that per-tensor scaling combined with quantization of per-block array scale-factor to E4M3 format results in superior inference accuracy across models. 


% % Bibliography
% \begingroup
% % \footnotesize  % or \small

% \endgroup



\end{document}


