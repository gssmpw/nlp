\section{Limitations}

% The visualization effect is not perfect in XR headsets, since we simply used RGB texture for the materials,
% making the robots from different simulators look the same.
% There should be some rendering configuration in the simulator but we only use RGB texture from them.
% Secondly, IRIS's performance is restricted by the number of simulated objects.
% When the quality of simulated objects in the simulation exceeds some limitation. the performance will be significantly reduced.

% Although the IRIS have 
Based on our development and usage experience, IRIS has three main limitations.
First, the visualization in XR headsets is limited to basic RGB textures for materials, making robots from different simulators appear visually identical.
Second, IRIS currently only supports rigid objects, at least deformable object have not yet been fully explored.
Finally, IRIS has only been tested on Meta Quest 3 and HoloLens 2, and further evaluation on additional devices would be beneficial.

\section{Conclusion}

% \textcolor{red}{TODO}

% In this work, we introduced IRIS, an innovative Immersive Robot Interaction System that bridges the gap between Extended Reality (XR) technologies and robotics data collection. By addressing the challenges of reproducibility and reusability prevalent in current XR-based systems, IRIS offers a flexible and extendable framework that can be used in multiple simulators, benchmarks, real-world scenarios, and multiple-user use case.
% The user study show that IRIS outperforms previous benchmark data collection method, making it as a potential option for future data collection pipeline.
% The IRIS codebase is open-source, and it will facilitate further research and make IRIS adapted to more use case and more hardware platform.

In this work, we introduced IRIS (Immersive Robot Interaction System), an innovative framework that seamlessly integrates Extended Reality (XR) technologies with robotics data collection. IRIS addresses key challenges in reproducibility and reusability that are common in current XR-based systems. Its flexible, extendable design supports multiple simulators, benchmarks, real-world applications, and multi-user use cases.
User studies demonstrate that IRIS outperforms previous data collection methods, positioning it as a promising solution for future data collection pipelines.
As an open-source project, IRIS codebase promotes further research and adaptation across diverse use cases and hardware platforms.

