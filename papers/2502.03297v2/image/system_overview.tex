% \begin{figure*}[ht]
%     \centering
%     \includegraphics[width=\textwidth]{image/system_overview.png}
%     \caption{The graph of user study}
%     \label{fig:system_overview}
% \end{figure*}


% \begin{figure}
%     \centering
%     \includegraphics[width=\linewidth]{image/system_overview/architecture_sim.png}
%     \caption{Caption}
%     \label{fig:overview_sim}
% \end{figure}

% \begin{figure}
%     \centering
%     \includegraphics[width=\linewidth]{image/system_overview/architecture_real_wider.png}
%     % \includegraphics[width=\linewidth]{image/system_overview/architecture_real.png}
%     \caption{Caption}
%     \label{fig:overview_real}
% \end{figure}

\begin{figure*}[htbp]
    \centering
    \begin{subfigure}[b]{0.45\linewidth}
        \centering
        \includegraphics[width=\linewidth]{image/system_overview/architecture_sim.png}
        \caption{Interact with Robots in Simulation}
        \label{fig:overview_sim}
    \end{subfigure}
    \hfill
    \begin{subfigure}[b]{0.45\linewidth}
        \centering
        \includegraphics[width=\linewidth]{image/system_overview/architecture_real_wider.png}
        \caption{Interact with Robots in Real World}
        \label{fig:overview_real}
    \end{subfigure}
    \caption{
Paradigms of the system architecture in both simulation (left) and real-world (right) environments. The simulation, headsets, and sensors are connected through a Wi-Fi router.
In the left image, the simulation updates the scene using the Scene Publisher, broadcasting the information to all XR headsets on the network. A spatial anchor is used to align the virtual scenes across different headsets, allowing users to share a consistent view of the same scene.
In the right image, a sensor generates a point cloud, which is sent to the headset. This allows the operator to view the manipulated object in front of the follower robot, enhancing real-time interaction and feedback.
}
    \label{fig:system_overview}
\end{figure*}
