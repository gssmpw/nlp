\documentclass[conference]{IEEEtran}
\usepackage{times}

% numbers option provides compact numerical references in the text. 
\usepackage[numbers]{natbib}
\usepackage{multicol}
\usepackage[bookmarks=true]{hyperref}

% set monospace font for code
\usepackage{sourcecodepro}
\usepackage[T1]{fontenc}

% other packages
\usepackage{placeins}
\usepackage{pifont} % For \ding
\usepackage{xcolor} % For coloring
\definecolor{darkgreen}{RGB}{0, 125, 0} % Darker green color
\newcommand{\cmark}{\textcolor{darkgreen}{\ding{51}}} % Green check mark
\newcommand{\xmark}{\textcolor{red}{\ding{55}}}   % Red cross mark
\usepackage{graphicx}
% \usepackage{todonotes}
\newcommand{\redtodo}[1]{\todo[inline, color=white, bordercolor=white]{\textcolor{red}{#1}}}
\usepackage{booktabs}
\usepackage{tabularx} 
\usepackage{adjustbox} 
\usepackage{listings}
\usepackage{makecell}
% Define custom colors for the code
\definecolor{codegreen}{rgb}{0,0.6,0}
\definecolor{codegray}{rgb}{0.5,0.5,0.5}
\definecolor{codepurple}{rgb}{0.58,0,0.82}
\definecolor{backcolour}{rgb}{0.95,0.95,0.92}
\definecolor{codeblue}{rgb}{0.21,0.47,0.56}
\usepackage{tikz}
\usepackage{subcaption}
\usepackage{caption}
\usepackage{float} 

% Customize the style for the code
\lstdefinestyle{mystyle}{
    backgroundcolor=\color{backcolour},   % Background color
    commentstyle=\color{codegreen},      % Comments
    keywordstyle=\color{codeblue},           % Keywords
    numberstyle=\tiny\color{codegray},   % Line numbers
    stringstyle=\color{codepurple},      % Strings
    basicstyle=\scriptsize\ttfamily,                % Basic font style
    % basicstyle=\ttfamily,  
    breakatwhitespace=false,             % Automatic breaks at whitespaces
    breaklines=true,                     % Automatic line breaking
    captionpos=b,                        % Caption position
    keepspaces=true,                     % Keep spaces
    numbers=left,                        % Line numbers on the left
    numbersep=5pt,                       % Space between numbers and code
    showspaces=false,                    % Do not show spaces
    showstringspaces=false,              % Do not show string spaces
    showtabs=false,                      % Do not show tabs
    tabsize=2                            % Tab size
}

\lstset{style=mystyle}


\pdfinfo{
   /Author (Homer Simpson)
   /Title  (Robots: Our new overlords)
   /CreationDate (D:20101201120000)
   /Subject (Robots)
   /Keywords (Robots;Overlords)
}


\lstset{escapeinside={(*@}{@*)}}

\begin{document}

% paper title
\title{IRIS: An Immersive Robot Interaction System}

% You will get a Paper-ID when submitting a pdf file to the conference system
% \author{Author Names Omitted for Anonymous Review. Paper-ID [179]}
\author{\small\textbf{Xinkai Jiang}$^{*,1}$, ~~\textbf{Qihao Yuan}$^2$,~~ \textbf{Enes Ulas Dincer}$^1$, ~~\textbf{Hongyi Zhou}$^1$, ~~\textbf{Ge Li}$^1$, ~~\textbf{Xueyin Li}$^1$, \\
\textbf{Julius Haag}$^1$, ~~~\textbf{Nicolas Schreiber}$^1$, ~~~\textbf{Kailai Li}$^2$, ~~~\textbf{Gerhard Neumann}$^1$, ~~~\textbf{Rudolf Lioutikov}$^1$
\\[0.2em]
~~~~~~$^{1}$Karlsruhe Institute of Technology, Germany ~~~~~~~~$^{2}$University of Groningen, Netherlands ~~~~~~~
}

\twocolumn[{%
\renewcommand\twocolumn[1][]{#1}%
\maketitle
\begin{center}
\centering
\captionsetup{type=figure}
\vspace{-0.5cm}
% \includegraphics[width=0.95\linewidth]{image/teaser.pdf} % old layout
\includegraphics[width=0.9\linewidth]{image/COVER/BG_BLACK_ELEVATION_WEAK.png} % old layout
\captionof{figure}{
We present \textbf{IRIS},
an \textbf{I}mmersive \textbf{R}obot \textbf{I}nteraction \textbf{S}ystem designed to support various simulators, benchmarks, and real-world scenarios. The system is highly extensible, allowing seamless integration with new simulators and XR headsets.
The images shown were captured using a Meta Quest 3 headset, which leverages the Depth API \cite{metaMetaDevelopers} to enable virtual objects to blend naturally with real-world elements rather than always rendering in the foreground. To differentiate between benchmarks, we use distinct labels for each.
}
\label{fig:front_page}
\end{center}
}]

% 150words
% Replying to workplace emails that are typically long and require politeness is time-consuming and cognitively demanding.
% Replying to lengthy and polite workplace emails is often time-consuming and cognitively demanding.
% takes time to understand and reply
\red{Replying to formal emails is time-consuming and cognitively demanding, as it requires crafting polite phrasing and providing an adequate response to the sender's demands.}
% \red{Replying to formal emails, which often takes time to understand and require polite phrasing, is time-consuming and cognitively demanding.}
Although systems with Large Language Models (LLM) were designed to simplify the email replying process, users still need to provide detailed prompts to obtain the expected output.
Therefore, we proposed and evaluated an \red{LLM-powered question-and-answer (QA)-based approach} for users to reply to emails by answering a set of simple and short questions generated from the incoming email.
We developed a prototype system, \textit{ResQ}, and conducted controlled and field experiments with 12 and \red{8} participants.
Our results demonstrated that \red{the QA-based approach} improves the efficiency of replying to emails and reduces workload while maintaining email quality, compared to a conventional prompt-based approach that requires users to craft appropriate prompts to obtain email drafts.
We discuss how \red{the QA-based approach} influences the email reply process and interpersonal relationship dynamics, as well as the opportunities and challenges associated with using a QA-based approach in AI-mediated communication.

% original
% Replying to lengthy and polite workplace emails is often time-consuming and cognitively demanding.
% Although systems with Large Language Models were designed to simplify the email replying process, users still needed to provide detailed prompts to obtain the expected output.
% Therefore, we proposed and evaluated a question-and-answer-based approach for users to reply to emails by answering a set of simple and short questions generated from the incoming email.
% We developed a prototype system, \textit{ResQ}, and conducted both controlled and field experiments with 12 and 9 participants.
% Our results demonstrated that ResQ improves the efficiency of replying to emails and reduces workload while maintaining email quality compared to a conventional prompt-based approach that requires users to craft appropriate prompts to obtain email drafts.
% We discuss how ResQ influences the email reply process and interpersonal relationship dynamics, as well as the opportunities and challenges associated with using a QA-based approach in AI-mediated communication.

\IEEEpeerreviewmaketitle


\section{Introduction}

\begin{figure*}
    \centering
    \includegraphics[width=\textwidth]{figures/Introduction.pdf}
    \caption{Showing the novel problem statement applied to traffic prediction use case. Multiple unstructured observations from the past are used to reconstruct a hidden traffic state from which a full traffic state is forecast with a set of query locations. }
    \label{fig:intro}
\end{figure*}

% Was sagen denn die anderen warum Traffic Prediction gut ist? 
Forecasting the traffic in the near future is an important task for city management.
Data from the near past is used to predict future traffic states with spatio-temporal Graph Neural Networks \cite{bui22}.
Accurate prediction provides the opportunity to optimize traffic flow, reduce traffic jams and increase air quality \cite{Po19}.

% Wieso ist Sparsity in allen Dimensionen wichtig.
While traffic prediction relies on the availability of data from traffic sensors, there exists a plethora of reasons why sensors may stop working temporarily, such as simple errors, energy saving, or overloaded communication systems.
Considering small- or medium-sized cities, the coverage of sensors may be low because the sensors are too expensive or not available.
Also, the sensors are typically static and do not adapt to changes in the traffic flow (e.g. caused by a construction site), which motivates moving sensors that for example could be mounted on cars. 
However, both missing and moving sensors introduce sparsity, since measurements may not be available for all locations at all times.
This sparsity must be explicitly addressed in traffic prediction for a realistic application scenario, which is illustrated in figure \ref{fig:intro}.
From one hour of data on Sunday morning, only few observations of the traffic state are available at each timestep.
The number of observations may differ throughout the observed time and the observation itself can be distributed arbitrarily in the city. 
We assume a relatively low number of sensors to account for resource saving and sensor failure in our proposed framework SUSTeR.
The task is to predict the dense traffic state one timestep after the observations at all possible sensor locations.
We study this problem on the traffic dataset Metr-LA and PEMS-BAY to test our assumption that only a fraction of the sensor values would be enough for good predictions.
By modifying an existing traffic dataset, we are able to compare our results from very sparse observations to the bottom line with all information available.
A successful study will provide insights in how sensors in new cities can be reduced before installing them and further mobile sensors would save more resources and are able to adapt to new traffic situations.
We argue that in order to be adaptable to other cities and changes in traffic flows, prior information like the road network should be neglected and just the sparse observations considered.
This comes with the added benefit of making our solution applicable in regions where no openly available road network is maintained or pathways change frequently (e.g. flood areas, animal observations). 


The aforementioned problem is novel and more challenging than the commonly considered traffic prediction problem, since there exist very few observations in each input sample.
Current works for the traffic prediction problem do not consider any missing values. \cite{Li2021, Shao22}
A common method among state of the art approaches is the usage of Graph Neural Networks on graphs that model the sensor network \cite{bui22}.
The values of a sensor are applied to the same graph node for each timestep which prohibits any non-stationary sensors . 
With fixed sensor locations, the resulting sensor network is highly correlated with the road network.
Streets connecting two intersections with sensors should be also an interesting point for correlations in the sensor network.
However, variable observations and high temporal sparsity rates can not be modeled adequately in a static network.
We show in our experiments that the road network has only a small influence on the traffic predictions.

Besides the traffic prediction for future timesteps, some works explore the field of traffic speed imputation \cite{Cini22, Cuza22} where missing sensor values are predicted.
But the amount of missing values is assumed to be at most 80\%, which on average are still over 40 given sensors in each timestep in the Metr-LA dataset with a total of 207 sensors.
We consider up to 99.9\% missing values which are on average 2.4 observations in each timestep that are used as input.
Such high sparsity rates drastically decrease the chance that multiple values are present in one input sample from the same sensor location, which makes it challenging to recognize and learn temporal correlations for each location on its own.

High sparsity rates (>95\%) result in few sensor values, but if a reconstruction of the traffic state would be possible, we question if spatio-temporal graphs require nodes for each sensor.
In SUSTeR we utilize only a small amount of graph nodes for the encoding of information and do not relate such nodes to the sensor network.
We call this the hidden graph (see figure \ref{fig:intro}), which is still able to reconstruct the complete traffic state.
Due to the reduced number of nodes SUSTeR achieves faster runtimes, as shown in the experiments.
This hidden graph is not embedded directly in the spatial domain, which is why the assignment of observations, as well as the querying of the future traffic, is done with an encoder and a decoder, implemented as neural networks.
The decoding from the hidden graph to future values depends on a set of query locations.
Figure \ref{fig:intro} shows the query locations as given from outside and in combination with the reconstructed traffic state the future values are predicted.

To construct the hidden graph we encode observations from each timestep into from multiple graphs, one for each timestep. 
The graphs are created in a residual style and information is added to the node embeddings from the previous timesteps.
We choose this method to incorporate all timesteps equally into the hidden state because the redundant information along the past is non-existing for high sparsity rates.
From the sequence of graphs where our framework inserted the observations step by step we apply STGCN \cite{Yu18}, an algorithm for traffic prediction to find and learn the spatio-temporal correlations on our small number of graph nodes.
The first future timestep of the STGCN is our hidden graph in which the traffic state is reconstructed. 

% Recent work has an implicit embedding of the graph nodes into the spatial domain as the assignment from the sensor to graph node is fixed one by one.
% Because the graph has the same structure as the road network spatio-temporal correlations can be learned between those sensors.
% We reduce the number of nodes and use a non-linear assignment learned data-driven from the observations.

We find in the experiments that SUSTeR outperforms the plain STGCN and modern traffic prediction frameworks like D2STGNN for high sparsity rates $(\geq 99\%)$.
This is equivalent to only $0.2$ to $2.4$ observation for each timestep on average.
SUSTeR uses fewer parameters than the baselines and can train faster and with less training data.
Our main contributions can be summarized as follows:
\begin{itemize}
    \item We introduce a sparse and unstructured variant of the traffic prediction problem with sparsity in all dimensions. The sensors report only a fraction of their values and are arbitrarily distributed in the spatial domain.
    \item We propose SUSTeR, a framework around the STGCN architecture, which maps sparse observations onto a dense hidden graph to reconstruct the complete traffic state.
    Our code is available at github.\footnote{https://github.com/ywoelker/SUSTeR}
    \item We conducts experiments that show that SUSTeR outperforms the baselines in very sparse situations ($\geq 95\%$) and has a competitive performance in low sparsity rates.
    % \item SUSTeR trains a third faster than the next competitor.
\end{itemize}

\section{Related Work}

\subsection{Teleoperation-Based Data Collection on Real Robots}

Collecting data using tele-operation on real robots has been explored by many previous works. Aloha \cite{zhao2023learning} introduced a low-cost teleoperation system that collects real-world demonstrations for imitation learning. A bimanual workspace is set up, where leader robots are used to control the follower robots. Followup work \cite{aldaco2024aloha} improved the performance, ergonomics, and robustness compared to the original design. In addition, a mobile version of Aloha \cite{fu2024mobile} improved data collection outside of lab settings. GELLO \cite{wu2023gello} supports a variety robot arms through a 3D-printed low-cost leader robots with off-the-shelf motors. In order to tele-operate dexterous end effectors prior work has retrieved hand motion data through visual hand tracking \cite{Qin2023AnyTeleopAG} or customized gloves \cite{wang2024dexcap}. In contrast to IRIS, none of these approaches leverages the immersive advantages of XR.

\subsection{XR-Based Data Collection in Real World}

% A major disadvantage of tele-operation using controllers or leader robots is that, for each different kinematics of the target robot, a specialized physical control device has to be built.
% To tackle this issue, some XR-based teleoperation methods have been proposed, exploiting immersive approaches for real-world interactions with robots.
Common approaches that combine XR-based tele-operation with real-world interactions typically visualize a virtual robot to show the user how human actions are mapped to robot actuation \cite{Qin2023AnyTeleopAG}. For instance, recent work developed mobile apps to allow data collection in augmented reality without the need for XR headsets \cite{ar2-d2-pmlr-v229-duan23a, eve}. However, leveraging XR headsets, allows for more intuitive robot manipulation~\cite{arcade,armada,jiang2024comprehensive}.
% , e.g., by aligning a virtual robot arm with the user's arm.
Instead of displaying the virtual robot in a third-person view, \citet{opentelevision, openteach} directly provide the first-person camera feed of the real robot to the user.
% \citet{opentelevision} and \citet{openteach} provide the camera view from real robots directly instead of displaying a virtual robot.
Many other systems~\cite{vicarios,augmentedvisualcues,wang2024robotic,immertwin,sharedctlframework,digitaltwinmr} visualize the real-world scene in the headset and control the robot arm either with controllers~\cite{sharedctlframework} or hand tracking~\cite{wang2024robotic}.
XR-based data collection for dexterous hands has also been explored. For example, \citet{arunachalam2023holo} tracks hand motion using camera and retargets it on the real robot hand. \citet{chen2024arcap} controls robot hand and robot arm at the same time. While these approaches do use XR, the robot data collection and interaction is limited to the real world, as no simulators used in the process.

% \textcolor{red}{More work about MR/VR/AR}

% \subsection{Robot Learning by Interaction}

% \textcolor{red}{what is the combination of demonstration and correction??? Interaction is not a good word}

% \textcolor{red}{a big table of comparison to other AR robot data collection}

\subsection{XR-Based Data Collection in Simulation}

Collecting data using real robots has many limitations, such as limited scene and object diversity. For gathering demonstrations more efficiently and opening access to numerous virtual 3D assets, some works have explored fully virtual data collection.
For instance, DART \cite{dexhub-park} runs a cloud-based simulation, and users can collect demonstrations in any virtualized environment from any location. \citet{mosbach2022accelerating} collects dexterous hand manipulation data with a special glove device in physics simulations.
Although \citet{meng2023virtual} also leverages simulators, their virtual scene is a replica of the real scene, thus the flexibility of simulation is not fully exploited.

This discussion reveals that state-of-the-art, XR-based data collection in simulation has barely been explored. Hence, IRIS presents a significant contribution towards immersive robot data collection and interaction. IRIS not only supports almost all popular simulation environments, but also connects seamlessly with robots in the real world, opening up enormous possibilities in human-involved robot learning in both virtual and real-world settings.
\section{System Overview}


This section introduces the features and technical details of IRIS.
An overview of its paradigm is shown in Figure \ref{fig:system_overview}.

\section{System Overview}
\label{sec:system_overview}
The primary goal of this work is to accurately predict the MFSP in building structures subjected to various fire scenarios \revise{by utilizing the MIDR as a metric for the overall lateral stability.} To achieve this, we propose an integrated framework that combines GNNs and FEA. The system architecture is illustrated in \figref{fig:system_overview} and comprises two key components: the MIDR predictor and the MFSP predictor. \revise{There are two stages corresponding to the MIDR predictor's different modes. In stage 1, we use data with ground truth MIDR values obtained from FEA simulations to train the MIDR predictor. Then, in stage 2, with the well-trained MIDR predictor working as a differentiable agent of fire simulation, we train the MFSP predictor to determine the point that maximize the MIDR.}
This framework seamlessly integrates physics-based simulations, GNN-driven predictive modeling, and data-driven techniques, enabling efficient and accurate MFSP prediction to provide a powerful tool for proactive fire safety analysis and risk mitigation in building structures.
\begin{figure*} [h!]
    \centering
    \includegraphics[width=0.8\textwidth]{figures/system_overview.pdf}
    \caption{\revise{Proposed framework for predicting the MFSP in building structures. Trapezoids  and gray rectangles represent the NN predictor and processing, respectively. Dashed and solid lines indicate that MIDR predictor is in the respective training mode and evaluation mode with parameters fixed, acting as a differentiable agent.}}
    \label{fig:system_overview}
\end{figure*}

{\blockRevise
\subsection{Structural Stability Metrics} 
The Interstory Drift Ratio (IDR) of each node serves as a critical parameter for evaluating structural lateral stability and deformation under external forces. IDR quantifies the relative displacement between two consecutive floors (interstory displacement) as a percentage of the floor height. Mathematically, the IDR for a given node $i$ is defined as follows:
\begin{equation}
    d_i = \left.\sqrt{\left(\Delta x_i\right)^2 + \left(\Delta y_i\right)^2}\right/ H \times 100 \%,
\end{equation}
where the numerator represents the relative displacement (in the horizontal plane $xy$) of node $i$ with respect to the corresponding node on the floor below, and $H$ denotes the story height between these two floors. Excessively high drift ratios can indicate significant structural deformation, potentially leading to significant damage or collapse due to lateral instability. The Maximum IDR among all the nodes of a structure, i.e., MIDR, is chosen to be a representative example metric for assessing the overall structural stability performance during fire events. Although MIDR may not capture all possible failure modes, such as local collapses due to midspan softening, local buckling, or loss of vertical elements, we emphasize that the proposed method is not limited to MIDR. The framework can be easily adapted to other performance indicators of interest, by simply replacing the MIDR with the desired metric.

\textit{Remark:} Selecting MIDR as the primary metric for assessing the structural integrity under fire conditions is motivated by its effectiveness in quantifying global deformation patterns. In fire-induced scenarios, thermal expansion, stiffness degradation, and gravity-induced deformations contribute to structural instability, which can be captured through relative floor displacements. MIDR provides a direct and interpretable measure of structural vulnerability by identifying floors experiencing excessive lateral deformations that may lead to global instability or loss of vertical load-carrying capacity. While MIDR is commonly used for seismic and wind-induced responses, its application to fire scenarios is justified as a fire-driven thermal effect also leads to large-scale deformation patterns, particularly in multi-story steel structures. Importantly, MIDR serves as an effective proxy for overall building stability in computational frameworks where parameterizing every potential failure mode (e.g., local buckling, connection failure, progressive collapse) is infeasible. However, fire-induced failure mechanisms extend beyond interstory drift, and localized effects are not explicitly captured by MIDR. These mechanisms typically develop locally and may not always translate into immediate global structural instability. While our current framework focuses on identifying the MFSP based on a worst-case drift metric, future work could integrate alternative failure criteria to further refine the fire vulnerability predictions.

Note that ``M'' in MIDR and MFSP represents different concepts. In the case of MIDR, for a given structure and fire source point, the IDR is computed at each node, and the MIDR is defined as the maximum IDR among all nodes. In contrast, MFSP refers to the fire source location that results in the highest MIDR across all possible fire source points within the structure. In summary, an MIDR is associated with a specific structure and fire source point pair, whereas an MFSP characterizes an entire structure by identifying the most critical fire source location.

}

\subsection{MIDR \& MFSP Predictors}

The MIDR predictor is a GNN-based model designed to estimate the MIDR of a building under a given fire scenario. The inputs to this model include:
\begin{itemize}
    \item {\bf{Structural configuration}}: Building geometry, material property, and gravity loads. 
    \item {\bf{Fire location}}: The specific point where the fire is initiated within the building.
\end{itemize}
A GNN processes this input to represent the structural configuration and fire location as a graph. The MIDR predictor is trained on labeled data generated using OpenSeesRT, a robust open-source FEA framework \cite{perez2024openseesrt}. These labels represent detailed structural responses under various fire conditions. Once trained, the MIDR predictor functions as a {\em{differentiable agent}}, offering computationally efficient MIDR estimates. Its capabilities include:
\begin{itemize}
    \item {\bf{Annotating datasets}}: Assigning  MIDR values to support subsequent analyses.
    \item {\bf{Integrating with NNs}}: Reducing the computational cost typically associated with simulation-based methods.
\end{itemize}

% \subsection{MFSP Predictor}
The MFSP predictor acts as an ``argmaxer module'' for the MIDR predictor, identifing the fire location that results in the highest MIDR. This location corresponds to the point of the greatest structural vulnerability. By leveraging the structural graph as input and utilizing the MIDR predictor's outputs, the MFSP predictor efficiently pinpoints the critical fire location.

\subsection{Data Generation and Training Pipeline}
To ensure robustness and generalizability, we introduce a comprehensive data generator pipeline:
\begin{enumerate}
    \item {\bf{Structure data generator}}: This component creates synthetic datasets for diverse building configurations, including geometry, material, and gravity loads.
    \item {\bf{FEA simulations}}: With the high-fidelity FEA simulation software, OpenSeesRT, the gravity simulation is first conducted to confirm the rationality of the synthetic dataset. Further, a subset of the generated configurations undergoes fire scenario simulations using OpenSeesRT based on a rule-based thermal load generation method. These simulations produce \textbf{labeled data} detailing  the structural responses to various fire locations, forming training and testing sets for the MIDR predictor.
    \item {\bf{Unlabeled data utilization}}: The remaining configurations, without MIDR labels from the FEA simulation, are also used to train and test the MFSP predictor, leveraging the MIDR predictor as a computationally efficient, yet accurate, {\em{surrogate}} model. Although the structural configurations with unlabeled data do not undergo FEA simulations, they can be rapidly and efficiently  \emph{pseudo labeled} using this surrogate model.
\end{enumerate}



\subsection{IRIS}

% In this section,
% We separate the physical simulation and rendering into a simulation PC and MR headsets
% ScenePublisher

\subsubsection{Node Communication Protocol}
% The software architecture of IRIS utilized master node and node,
% while simulation PC is the master node and headsets as well as other devices are XR node.
% The communication between them is based on socket and ZMQ.
% Inspired by ROS, a communication protocol is created for multiple clients and transmitting data in different kinds of format.
% To make all the XR node find the master node,
% master node broadcasts UDP messages in a fixed rate to broadcast port in the network,
% then all the XR nodes will get the server ip information including ZMQ socket address and port from the master node so that XR node will start ZMQ connection to the master node.
% After that, they can communicate in request-response and publish-subscribe patterns by ZMQ.
% When the master node is offline, the XR headsets will keep listen to the discover port and wait for next master node.
% This flexible and extendable communication framework supports multiple nodes, auto-reconnection and auto-recovering from disconnection.

% IRIS operates across a simulation PC and multiple XR headsets, it requires a robust and reliable connection between them, and they need to identify each other and connect.
% In IRIS, All the simulation and XR headsets are running in the same subnet, and the XR headsets are connected by WIFI router, they should use the IP address with the same subnet.
% Therefore, IRIS expand the function of ROS from communicating in a local host to local network with different host. introduces a custom communication protocol designed to support multiple nodes across different devices and enable data transmission using request-response and publish-subscribe patterns.
% This communication protocol adopts a master-node and multi-node architecture, where the simulation PC serves as the master node and the XR headsets, along with other devices, function as XR nodes.
% Communication between these nodes is implemented using sockets and ZMQ.
% To ensure that all XR nodes can discover the master node,
% the master node broadcasts UDP messages to a fixed broadcast port on the network in a rate of five Hz.
% XR nodes receive these messages and extract the master node's details, including the ZMQ socket address and port, enabling them to establish a stable ZMQ connection with the master node.
% Once connected, communication between nodes seamlessly follows the request-response and publish-subscribe patterns supported by ZMQ.
% When the master node is offline, XR nodes continue listening the discovery port, ready to reconnect to a new master node as soon as it the master node relaunches.
% This robust communication framework supports multiple nodes, automatic reconnection, and seamless recovery from disconnections, making it highly reliable and well-suited for dynamic, multi-device XR systems.

The IRIS system operates across simulation and/or sensor-processing computers, multiple XR headsets, and other monitoring and control programs, requiring a robust and reliable network connection between them.
All devices are part of the same subnet, with all the devices connected via Wi-Fi or cable.
Inspired by the Robot Operating System (ROS \cite{quigley2009ros}), 
However, instead of using a single host IRIS leverages a local network across multiple hosts, 
while using both Request-Response and Publish-Subscribe patterns for data transmission.
This protocol follows a master-node architecture with multiple nodes, where the simulation PC serves as the master node, and the XR headsets and other devices act as XR nodes.
Communication is achieved through a combination of UDP sockets and ZeroMQ (ZMQ).

\begin{figure}[ht]
    \centering
    % \includegraphics[width=0.45\textwidth]{image/protocol.png}
    \includegraphics[width=0.45\textwidth]{image/protocol_new.png}
    \caption{The master node broadcasts UDP messages containing its details to the broadcast address (e.g., \textit{192.168.0.255}) at port 7720. Each IP address in the diagram (e.g., \textit{192.168.0.100/0}) represents a device's unique address on the network, where '0' indicates a dynamically assigned port provided by the operating system. XR nodes, upon startup, listen on the broadcast port (7720) to receive these messages, extract the master node's IP and ZMQ socket address, and build a stable connection. This architecture supports both request-response and publish-subscribe communication patterns, ensuring robust, multi-device connectivity with automatic reconnection capabilities.}
    \label{fig:protocol}
\end{figure}

To ensure node discovery, the master node broadcasts UDP messages at 5 Hz to a fixed broadcast port on the network.
When a new XR node is launched, 
it listens on the broadcast port to receive a broadcast message from the master node. 
%It extracts the master node's details, 
%including its ZMQ socket address and port, to establish a stable ZMQ connection.
Upon receiving the broadcast message, the XR node extracts the master node's details, including its ZMQ socket address and port, to establish a reliable ZMQ connection.
If the master node goes offline, XR nodes continue listening on the discovery port, allowing automatic reconnection when the master node relaunches.
This protocol (Fig. \ref{fig:protocol}) achieves \textbf{Cross-User} ability of IRIS, ensures reliable communication, automatic reconnection, and smooth recovery from disconnections,
making it ideal for dynamic multi-device XR systems.


\subsubsection{Unified Scene Specification}
% To visualize a scene with arbitrary objects in headsets from the simulation,
% the headsets need to receive the scene model from the simulation and rebuild them.
% The XR application running in the headsets is developed by C\# and Unity,
% and simulation is running by Python or C++.
% From related works, they only support specific robots and assets,
% since they need to create some identical predefined models in a model set for the XR application,
% and they don't support robots or objects which is not in the models set.
% This mechanics highly restrict the flexibility and reusability of their system.
% To fully support all the objects, robots, and assets from simulation,
% we defined a unified scene model format and directly read the scene from simulation.
% Then this information will be sent to the headset by the node communication protocol for rebuilding an identical scene in our XR application.

To visualize a scene with arbitrary objects, the XR headsets need to receive the scene model from the simulation and reconstruct it.
However, the XR application and the simulation run on different devices and use different software architectures (the XR application is developed with C\# and Unity, while simulators might be built in Python or C++).
This makes it impractical to directly transfer the scene from the simulation to the XR environment.
To address this issue, existing solutions rely on predefined models in the XR application for specific robots and assets, requiring a static set of models to be maintained within the application.
This approach restricts flexibility and reusability, preventing the support of robots or objects not included in the model set.
To address this issue, IRIS introduces a novel unified scene specification, which is generated by parsing the scene directly from the simulation.
This specification is subsequently transmitted to the headsets using the node communication protocol, enabling the XR application to accurately and dynamically recreate the scene in real time.

\begin{figure}[t]
    \centering
    \includegraphics[width=0.75\linewidth]{image/xinkai_tree.pdf}
    \caption{
The hierarchical structure of the Scene Specification begins with a root SimObject, which contains all objects in the scene. Each SimObject has a name, a list of child SimObjects, and a list of visuals. Each visual represents a geometric element attached to the object.
Within each geometric element, materials define properties such as color and texture, while meshes determine the shape. The scene's raw data includes the byte streams of meshes and textures. Since these streams are extensive, they are sent to XR headsets sequentially after the initial scene specification is transmitted.
}
    \label{fig:scene_specification}
\end{figure}

% The unified scene specification includes environment setting (e.g., lighting) and all the objects as well as their geometry elements, meshes, materials and textures attached to them.
% IRIS provides a Python Library named ScenePublisher for parsering a simulation scene to a scenespevification.
% In the scene specification, geometry element is the shape of object (e.g., Cube, sphere, capsule, cylinder, or mesh).
% The mesh contains the vertex list, faces list, normals list, and texture coordinate list.
% material is the color and eemissionColor, reflectance or texture attached to them.
% Texture is an image which repesetnt texture.
% The objects is stacked by the kinematic tree and they will be parsed into json format by the ScenePublisher.
% Since the mesh and texuture contain too much data, and they will be sent to clients at beginning because it is rather big.
% So the visual and material which have mesh and texture will hold an hash code, and wait the request from XR node to retrieve it later.

The unified scene specification (shown in Fig. \ref{fig:scene_specification}) includes all objects with their geometry, meshes, materials, and textures. 
IRIS provides a Python library called \textit{ScenePublisher} to parse a simulation scene into this scene specification.
In the specification, 
geometry defines the object's shape (e.g., cube, sphere, capsule, cylinder, or mesh).
The mesh contains vertex lists, face lists, normals, and texture coordinates. 
Materials define surface properties, including color, emission color, reflectance, and attached textures, 
and the texture is an image representing the object's surface appearance.
Objects are organized using a kinematic tree structure and are serialized into JSON by the \textit{ScenePublisher}.

Since meshes and textures contain large amounts of data, 
geometry and materials store only their hash code to reduce the transmission load. 
The XR application rebuilds the scene upon receiving the kinematic tree and then requests the meshes and textures from the simulation server in byte format.
In some cases, textures can be quite large (e.g., the textures for the RoboCasa \cite{nasiriany2024robocasa} scene exceed 700 MB). To ensure the scene loads within an acceptable time for users, we compress the textures. This compression reduces the loading time to a few seconds.
Afterwards, the \textit{ScenePublisher} continually acquires simulation states and forwards them to the XR headset. This way the positions and rotations of all the objects are updated at a fixed frequency.

The scene specification enables IRIS to support a wide range of robots and objects in simulation, facilitating both \textbf{Cross-Scene} and \textbf{Cross-Embodiment} capabilities.


\subsubsection{Extendable and Flexible Framework Support}
The unified scene specification is a general definition that does not rely on any specific simulator, providing an extensible mechanism for scene loading and updating. IRIS can be easily adapted to various simulation engines and frameworks by implementing a new simulation parser to generate the unified specification from the simulation scene and a new publisher to update the states of scene.

Currently, IRIS supports scene parsers for MuJoCo, IsaacSim, CoppeliaSim, and Genesis, with the potential to be extended to other simulation engines as desired.
IRIS has been tested in some MuJoCo-based benchmarks including \textbf{Meta World} \cite{yu2020meta}, \textbf{LIBERO} \cite{liu2024libero}, \textbf{RoboCasa} \cite{nasiriany2024robocasa}, \textbf{robosuite} \cite{zhu2020robosuite}, \textbf{Fancy Gym} \cite{fancy_gym}, and CoppeliaSim-based benchmark like 
\textbf{PyRep} \cite{james2019pyrep},
\textbf{Colosseum} \cite{pumacay2024colosseum}.
This demonstrates that IRIS can be easily adapted to various benchmarks and simulators, highlighting its \textbf{Cross-Simulator} capability.


IRIS provides a user-friendly API.
For each environment or framework,
a single line of code suffices to visualize and update the simulation in the XR headset.
Here is a short example of how to use it in the MuJoCo Simulation, where the important line is marked in bold: 
\begin{lstlisting}[language=Python]
# import scenepub
from scenepub.sim.mj_publisher import MujocoPublisher
# define the mujoco environment
model = mujoco.MjModel.from_xml_path(xml_path)
data = mujoco.MjData(model)
# define the ScenePublisher for mujoco
(*@\textbf{publisher = MujocoPublisher(model, data, host)}  @*)
# run simulation
while True:
    # run simulation step logic
    pass
\end{lstlisting}

The MujocoPublisher instance only needs to access the model and data from the MuJoCo simulation. It then creates a separate thread to run the communication protocol, automatically connecting and communicating with all available XR headsets.
IRIS provides various Simulation Publishers for different environments, all with a consistent and easy-to-use interface. This mechanism ensures a seamless experience, making the system very user-friendly.
% \begin{table}[t]
    \centering
    \begin{tabular}{cc}
        \hline
        \toprule
        \textbf{Simulation Engine} & \textbf{Benchmarks} \\
        \midrule
        Mujoco & Meta World, LIBERO, Robocasa, Fancy Gym \\
        \midrule
        IsaacSim & IsaacLab, ARNOLD \\
        \midrule
        CoppeliaSim & Colosseum, RLBench \\
        \midrule
        Genesis & N.A \\
        \bottomrule
    \end{tabular}
    \caption{Simulation engines and their associated benchmarks.}
    \label{tab:single-column}
\end{table}


\subsubsection{Real Scene Loading}
\label{sec:real_world_teleop}
% The loading and updating of real scene is similar to the simulation.
% IRIS use cameras and calibrate them for merging multiple point clouds into one point cloud.
% Then the point cloud will be cropped and down-sampled,
% each point will be sent to headsets and IRIS use particle system in Unity to visualize and update the point cloud.

% We merge and down-sample the point clouds from multiple cameras by first applying extrinsic transformations to each point cloud. 
% Then, we map the XYZ coordinates of the both pointclouds to voxel-grid indices.
% Next, we blend the colors of all points within a voxel and compute the voxel's centroid XYZ coordinates, returning both as output.
% This entire process uses Thrust \cite{bell2012thrust} and runs on the GPU to achieve low latency.

% The loading and updating of real-world scenes in IRIS follows a process similar to that used for simulation scenes, making the feature of Cross-Reality.
The loading and updating of real-world scenes in IRIS follow a process similar to that of simulation scenes, demonstrating its Cross-Reality capability.
It processes point clouds from one or multiple RGB-D cameras, which are extrinsically calibrated to a fiducial marker in the scene. A point cloud processor applies the extrinsic transformation to each point cloud before merging them. The merged point cloud is then cropped and downsampled using a voxel-grid filter.
This filter maps all 3D points to voxel-grid indices, blends the colors of points within the same voxel, and calculates the voxels centroid.
The output is a reduced point cloud that retains both color and position data.
To ensure low latency, this process runs on the GPU using Thrust \cite{bell2012thrust}.
Finally, the processed point cloud is transmitted to the headsets, where IRIS uses a particle system in Unity to visualize and dynamically update the scene in real-time.


\subsubsection{Interaction Data Collection}
\label{sec:interaction_data_collection}
IRIS provides user-friendly access to interaction data through the \textit{ScenePublisher}, which supports both Meta Quest 3 and HoloLens 2. 
Users only need to create a new device instance and assign a name to the connection.
The \textit{ScenePublisher} then waits automatically for the device to launch.
Once the device is launched, the instance receives messages from the XR headset, enabling users to access various types of interaction data through different methods.
This process makes it straightforward to retrieve data such as hand tracking, motion controller inputs, and other relevant interaction information.
The interaction data accessible through the ScenePublisher is listed in Tab. \ref{tab:feature-comparison},
and the example code can be found in the Appendix.
\ref{app:interaction_data}





\begin{table}[t]
    \centering
    \begin{tabular}{lcc}
        \hline
        \toprule
        \textbf{Feature} & \textbf{HoloLens 2} & \textbf{Meta Quest 3} \\
        \midrule
        Hand Tracking     & \cmark & \cmark \\
        Gaze Tracking     & \cmark & \cmark \\
        Motion Controller & \xmark & \cmark \\
        Head Tracking     & \cmark & \cmark \\
        % Voice             & \cmark & \cmark \\
        \bottomrule
    \end{tabular}
    \caption{
Feature comparison between HoloLens 2 and Meta Quest 3.
IRIS supports the transmission of interaction data from headsets to the simulation environment, with easy access to this data through ScenePublisher.
}
    \label{tab:feature-comparison}
\end{table}




\subsubsection{Spatial Anchor}
\label{sec:spatial_anchor}
% \textcolor{red}{rewrite this part}

% Spatial anchor serves as a reference in a 3D environment to accurately position virtual objects within physical space. 
% It allows virtual objects to maintain their position, orientation, and alignment even as users move around or leave and return to the environment.
% IRIS use QR code or motion controller (only for Meta Quest 3) as spatial anchor
% Alignment between the virtual environment and the physical world is achieved by using a trackable QR code.
% The QR code serves as the coordinate frame of the world for the virtual environment,
% and users can provide the offset.
% This method could be used for align all the virtual scene from multiple headsets and makes them look like they are sharing the same scene in the real world.
% facilitates seamless alignment of virtual and physical elements, 
% This method could also be used for synchronizing a virtual robot with its real-world counterpart in the Kinesthetic Teaching interface and multiple-collector view alignment.
% The implementation of QR tracking varies in different headsets,
% and this function is supported by all the headsets in IRIS.

A spatial anchor serves as a reference point in a 3D environment to accurately position virtual objects within physical space.
It enables virtual objects to maintain their position, orientation, and alignment, even as users move around or leave and return to the environment.
IRIS utilizes either QR codes or motion controllers (currently only implemented for the Meta Quest 3) as spatial anchors.
This approach allows for the alignment of augmented scenes across multiple headsets in the real world, making it appear as though all users are sharing the same scene.
Additionally, this method can synchronize a virtual robot with its real-world counterpart in the {Kinesthetic Teaching} (\ref{sec:kt}) interface and facilitate multiple-collector view alignment.
Fig. \ref{fig:alignment} shows the alignment between real robot and virtual robot.

\begin{figure}[h!]
    \centering
    % \includegraphics[width=0.45\textwidth]{image/objective_study.pdf}
    \includegraphics[width=\linewidth]{image/alignment/alignment.png}
    \caption{This graph shows the alignment between real robot and virtual robot by a spatial anchor, which is used for controlling virtual robot by real world Kinesthetic Teaching \ref{sec:kt}.}
    \label{fig:alignment}
\end{figure}

\subsubsection{Headset Compatibility}
% \textcolor{red}{rewrite this part}

% The basic hardware of IRIS is one PC for running the simulation and XR headsets.
% The whole architecture is extendable so that other hardware is also easy to integrate into IRIS if they use IRIS communication protocol.
% To project the scene to the MR devices,
% a local WIFI network is utilized to build communication between the simulation PC, the headset, and other devices.
% The local network has a shorter communication delay compared to the remote network or the cloud.
% Currently, we tested our system on HoloLens 2 \cite{hololens2_lolambean_2023} and Meta Quest 3.


IRIS implements an XR application using Unity, featuring essential capabilities such as node communication protocols, scene construction based on scene specifications, and synchronization with remote simulations.
The application can be directly deployed to other headset platforms using the Unity deployment pipeline, showcasing IRIS's \textbf{Cross-Platform} capability.
For input data handling (reading and sending), the application requires the use of platform-specific APIs, making it impractical to rely on a generalized framework.
This approach separates the visualization and interaction components, minimizing the effort needed to transfer IRIS codebase to new platforms.


\subsection{Intuitive Robot Control Interface}
\label{sec:intuitive_robot_interface}
% \textcolor{red}{should have some images of each interface}

In data collection tasks, robot control interfaces are used to operate the robot in both simulated and real-world environments.
Based on research in teleoperation and robot data collection \cite{jiang2024comprehensive}, Kinesthetic Teaching and Motion Controllers have been identified as the most intuitive and effective control interfaces.
Hence, we ensured that IRIS supports these two methods.
Thanks to IRIS's flexible framework, it is possible to easily customize and implement additional alternative control interfaces,
such as hand tracking, gloves, smartphones, or motion tracking systems.
This adaptability enables tailored solutions to meet specific requirements, enhancing both the usability and versatility of the system for various applications.
Fig. \ref{fig:interface} shows how these two interfaces work in IRIS. The implementation of these interfaces is outlined below.

\begin{figure}[h!]
    \centering
    % \includegraphics[width=0.45\textwidth]{image/objective_study.pdf}
    \includegraphics[width=\linewidth]{image/interface/interface.png}
    \caption{This image illustrates examples of using two interfaces to control robots in simulation: Kinesthetic Teaching (left) and Motion Controller (right), shown from a third-person perspective.}
    \label{fig:interface}
\end{figure}

\subsubsection{Kinesthetic Teaching}
\label{sec:kt}
% Kinesthetic teaching is an intuitive interface controlling robots by physically moving the real robot.
% The real robot transmit joints position and velocity to the controlled robot in the real time,
% and the controlled robot could be real robot or virtual robot.
% In both real world and simulation data collection,
% the key of Kinesthetic Teaching is the alignment of the real robot with the corresponding virtual robot in the XR headsets,
% so that usrs feel an immersive control. 
% By using spatial anchor, the robot objects in the XR and points of cloud could be perfectly match to the real robot.

Kinesthetic teaching is an intuitive interface that allows users to control robots by physically moving the real robot \cite{wrede2013user, sukkar2023guided, jiang2024comprehensive}.
The real robot transmits joint positions and velocities in real time to the controlled robot, which can be either a virtual or another physical robot.
In both real-world and simulation data collection,
the key aspect of kinesthetic teaching is ensuring alignment between the real robot and its virtual counterpart in the XR headsets.
By utilizing \textit{Spatial Anchors} (\ref{sec:spatial_anchor}), the virtual robot in the XR headsets can be perfectly aligned with the real robot, which provide users with an intuitive and immersive experience.

% In addition, our Kinesthetic Teaching interface provides force feedback for users,
% which means that users feel the resistance force from real robot when controlled robots contacts with objects,
% this feature also applied to real world data and virtual data collection.


\subsubsection{Motion Controller}

% Motion controllers are widely used in robot data collection.
% \textcolor{red}{some papers}
% They offer stable tracking and a flexible approach for various applications.
% While some XR headsets, such as the HoloLens 2, do not natively support motion controllers, this limitation can be addressed by integrating third-party motion controllers compatible with platforms like SteamVR. \textcolor{red}{citation}
% Motion controllers use inverse kinematics to control robots in Cartesian space. The movement of the robot’s end effector is dictated by the motion controller's trigger. 
% IRIS supports retrieve motion controller data from Meta Quest 3.
% For more complex scenario, users could also design their own controller based on the our IRIS.
% From our usage, motion controllers are only suitable for two-gripper end effectors, and it is challenging to use them with robotic hands to perform complex movements.

Motion controller Interfaces are commonly used in robot data collection and teleoperation \cite{pettinger2020reducing, lin2022comparison}, providing stable tracking and flexibility for various applications. 
Although some XR headsets, such as HoloLens 2, do not natively support motion controllers, this limitation can be resolved by integrating third-party controllers compatible with platforms such as SteamVR \cite{steampoweredSteamVR}.
Motion controllers utilize inverse kinematics to control robots in Cartesian space, with the movement of the robot’s end effector controlled by the controller's trigger. 
IRIS supports retrieval of motion controller data from devices like the Meta Quest 3. 
For more complex scenarios, users can design custom controllers using IRIS' flexible framework.
% Based on our experience, motion controllers are well-suited for two-gripper end effectors, 
% but are less effective for controlling robotic hands in tasks requiring complex movements.


% \subsubsection{Hand Tracking}

% Hand tracking is an intuitive method for controlling robots, especially humanoid robots or robotic arms with hand-like end effectors.
% The inside-out hand tracking (HT) interface uses the cameras of XR headsets to track hand movements and recognize gestures, 
% which is widely supported across different types of XR devices.
% This method is particularly useful for data collection tasks, as it closely mimics robotic hand movements, providing a natural and immersive control experience.
% \textcolor{red}{not sure about this part}
% IRIS provides an easy-to-use Python API (\ref{sec:interaction_data_collection}) to access hand tracking data. 
% Although this interface has only been tested with two-finger end effectors rather than humanoid robots. 
% users can easily leverage IRIS to develop custom controllers for their own robotic systems.
% However, hand tracking has some limitations.
% It requires the user's hands to stay within the headset's field of view, and tracking can be disrupted by occlusions, poor lighting, or fast hand movements, reducing its reliability for precise control.


% \textcolor{red}{should be a table of interfaces here}

\subsection{Affiliated Monitor Tools}
The extensibility of IRIS opens up numerous possibilities for creating new applications. IRIS includes a web-based monitoring tool for managing all XR headsets. This tool allows users to easily start and stop alignment processes, as well as rename devices.
Additionally, the tool supports real-time scene visualization using \textit{three.js} \cite{threejsThreejsDocs}, using the unified scene specification. An example screenshot is shown in Fig. \ref{fig:webapp}, with further technical details provided in the appendix. \textcolor{red}{}

\begin{figure}[h!]
    \centering
    \includegraphics[width=0.45\textwidth]{image/webapp.png}
    \caption{
The IRIS Dashboard, accessible via a web interface, 
allows the control and monitoring of all connected nodes. 
The scene streamed by IRIS is rendered on the right-hand side. 
The left panel displays the connected XR devices, 
providing an interface through which users can control all services made available by each device.
    }
    \label{fig:webapp}
\end{figure}




\section{System Application}

\subsection{Robot Data Collection for Benchmarks}

Through its flexible framework design, IRIS supports four simulators and various robot manipulation benchmarks.
Based on the interfaces introduced in Sec. \ref{sec:interaction_data_collection},
IRIS already provides example controllers for various benchmarks and frameworks, such as robosuite, LIBERO, RoboCasa, mink, Metaworld, Fancy Gym, and so on.
% Since all the objects in the simulation including robots and other objects are considered as one item, so naturally IRIS supports visualize any type of robot in the XR headsets from the simulation without any configuration.
Robots from all the frameworks can be controlled by Motion Controller or Kinesthetic Teaching.
Fig. \ref{fig:front_page} shows the robots controlled by IRIS including Franka Panda, Aloha 2, Barrett Wam Arm, UR5e, iiwa14, Boston Dynamics Spot, Unitree H1, and more.


% \textcolor{red}{picture with all kinds of robot control}

% \begin{figure}[h!]
    \centering
    \includegraphics[width=0.45\textwidth]{image/aloha.jpg}
    \caption{Control Aloha 2 in the Mujoco Simulation}
    \label{fig:aloha}
\end{figure}



% \subsection{Deformable Object Manipulation}

% As far as we know, no existing work has explored the manipulation of deformable objects using XR technologies.
% This is a significant gap in the field, as deformable object manipulation presents unique challenges compared to rigid object manipulation, including the need to handle continuously changing shapes, states, and dynamics in real time.
% Deformable objects, such as fabrics, ropes, or soft materials, are more complex to simulate and interact with due to their high degrees of freedom and nonlinear behaviors.

% IRIS supports deformable object manipulation by dynamically updating the mesh state in real time.
% This capability enables users to interact intuitively with deformable objects in a simulated environment,
% making it possible to train and test robotic algorithms for tasks that involve soft and flexible materials.


% To validate our framework, we conducted experiments in Isaac Sim, a state-of-the-art physics simulation environment designed for robotics applications.
% The specific task we designed for testing is folding a jacket, a representative challenge in deformable object manipulation. Jacket folding is a highly dynamic task that requires precise control and adaptability to continuously changing object shapes.
% \textcolor{red}{Qihao:}

% \subsection{Bimanual Robot Manipulation}

% % \textcolor{red}{rewrite this part}

% The collection of bimanual robot data is challenging for previous work.
% However, it is quite simple for IRIS for its compatiblity.


% Humanoid robots are inherently complex and challenging to implement and utilize effectively in the real world.
% These challenges arise from various factors, including the high cost of hardware, difficulties in maintaining precision and reliability, and the requirement for sophisticated infrastructure.
% Real-world robot training often demands extensive physical space, a controlled environment, and advanced sensors and actuators, all of which contribute to significant logistical and financial burdens.
% Furthermore, safety concerns and the potential for physical damage to robots during training add additional layers of complexity, particularly in dynamic or unstructured environments.

% Given these barriers, the use of simulation environments becomes not only a practical but also a highly effective alternative. Extended Reality (XR) technologies, in particular, have opened new doors for creating immersive and interactive virtual environments that bridge the gap between human users and robots.
% With XR, users can access and control virtual robots without the need for expensive hardware, allowing for a more flexible and cost-efficient approach to train, test and deploy their algorithm.

% In this paper, we explore using hand tracking to control humanoid robot as an intuitive and accessible interface for interacting with and testing virtual robots in simulated environments.
% Specifically, we utilize hand-tracking to evaluate robot performance across various datasets and tasks, demonstrating the feasibility and practicality of this approach in a simulated context. By leveraging the power of XR and simulation, we aim to democratize access to robotic training and lower the barriers to innovation in robotics.

\subsection{Collaborative Manipulation}

\begin{figure}[htbp]
    \centering
    \begin{subfigure}[b]{0.49\linewidth}
        \centering
        \includegraphics[width=\linewidth]{image/collaborative_collection/kt_collaborative_same_size.png}
        % \caption{Collaborative manipulation with two Franka Panda robots by Kinesthetic Teaching}
        % \label{fig:overview_sim}
    \end{subfigure}
    \hfill
    \begin{subfigure}[b]{0.49\linewidth}
        \centering
        \includegraphics[width=\linewidth]{image/collaborative_collection/aloha_two_player_same_size.png}
        % \caption{Collaborative manipulation with two Aloha 2 by Motion Controller}
        % \label{fig:overview_real}
    \end{subfigure}
    \caption{
Collaborative manipulation in the simulation via XR. The left image shows the collaborative manipulation for hand over a hammer between two Franka Panda robots by Kinesthetic Teaching, and the right image shows that collaborative manipulation for hand over a red board between two Aloha 2 Arms by Motion Controller.
}
    \label{fig:collaborative_collection}
\end{figure}

% Collaborative manipulation is a critical technology for advancing human-robot systems. It involves enabling multiple users or robots to work together seamlessly on shared tasks, leveraging their unique strengths to achieve goals that would be difficult or impossible to accomplish individually. 
% One of the major challenges in collaborative manipulation lies in ensuring smooth communication and synchronization between human participants, robots, and the virtual environment.
% To address this, our system employs a flexible communication framework that allows for easy integration of new devices and interfaces. Specifically, our system supports the addition of new XR headsets with minimal effort, enabling more users to join the collaborative environment dynamically. This flexibility makes it possible to scale up or adapt the system as needed, facilitating diverse use cases and applications.


% When a new XR application is started, the first step is to establish communication with the simulation master node. The master node serves as the central coordinator for all XR nodes and is responsible for managing the shared virtual environment. Upon connection, the master node sends the current scene data to the new XR node, ensuring that the new application has a synchronized starting point with the existing XR nodes.
% However, upon initialization, the newly added XR scene is not automatically aligned with the scenes of the existing XR nodes. To resolve this, the new XR node must locate a spatial anchor, which serves as the foundational reference point for aligning all XR nodes within the shared environment. 
% The spatial anchor ensures that virtual objects, scenes, and interactions are consistently positioned across devices, regardless of variations in the physical setup or device configurations.
% By aligning its virtual environment with the spatial anchor, the new XR node can seamlessly integrate into the shared scene, providing users with a unified and immersive experience.
% This process is critical for maintaining coherence and synchronization in multi-user XR environments, especially in applications like collaborative manipulation tasks. The spatial anchor acts as the cornerstone for all XR nodes, enabling accurate interaction and coordination among participants and ensuring that the virtual environment behaves as a single, unified system.


% By combining these technologies, our system enhances the user experience in collaborative manipulation tasks, making interactions more natural, intuitive, and efficient. This work demonstrates how XR can bridge the gap between humans and robots in shared environments, fostering better teamwork and paving the way for more advanced and scalable human-robot systems.


Collaborative manipulation, where multiple users provide demonstrations simultaneously and/or interactively, plays a crucial role in the advancement of human-robot systems \cite{tung2021learning}.
Collecting demonstration data for such tasks has been a significant challenge as it requires smooth communication and synchronization between human participants, robots, and the virtual environment.
Previous work on manipulation involving multiple humans often facilitates collaboration through multiple screens  \cite{Qin2023AnyTeleopAG}, 
which lacks the immersive experience provided by XR.
Moreover, typically only one person is in control of the demonstrations while others are limited to observing or monitoring \cite{szczurek2023multimodal}.
By leveraging the node communication protocol,
IRIS allows for easy integration of new devices and interfaces for controlling multiple robots in a shared scene.
Specifically, IRIS supports the addition of new XR headsets with minimal effort,
allowing more users to dynamically join the collaborative environment.
% This flexibility allows for scaling or adapting the system as needed, facilitating diverse use cases and applications.
% A handover task was chosen as an example, as illustrated in Fig. \textcolor{red}{image}.
% In this scenario, two operators each control two Aloha 2 robotic arms to pass a red board between them.
Fig. \ref{fig:collaborative_collection} shows collaborative manipulation for handover task.
% \textcolor{red}{more pictures!!!!!}

\subsection{Interaction with Robot Policies in Simulation}
Simulation has long been an essential tool for training robot agents, especially for reinforcement learning (RL) policies.
Its advantages -- such as safety, scalability, and cost-efficiency -- make it widely used in RL training. 
Previous works on interactive learning have utilized interfaces such as keyboards \cite{mandlekar2018roboturk}, 3D mouse \cite{luo2024precise, liu2024libero} or smartphone \cite{mandlekar2023human}. However, these methods are insufficient for tasks that require multiple viewpoints or complex motions.
% However, collecting human-robot interaction data for these agents remains challenging, as accurately modeling human behavior in simulation is inherently complex.
% IRIS offers a novel solution by bridging the gap between simulation and reality.
% By projecting the simulation into the real world, IRIS enables a real human operator and a virtual RL agent to share the same workspace.
% This immersive interaction framework overcomes the limitations of simulation-only setups. 
% It provides more flexibility in data collection with different robot models and scenes, facilitating human-robot interactive data collection for cross-embodiment and cross-scene tasks.
IRIS delivers an immersive experience, providing the appearance of "stepping into" the simulation.
Its interaction API (\ref{sec:interaction_data_collection}) allows users to effortlessly move and manipulate objects within the simulation, mirroring real-world interactions.
This capability enables IRIS to serve as a powerful tool for highly interactive tasks, such as competitive sports, where real-time responsiveness and precise control are essential.

\begin{figure}[h!]
    \centering
    \begin{subfigure}[b]{0.48\textwidth}
        \centering
        \includegraphics[width=\linewidth]{image/interact_with_policy/first_person_view.pdf}
        % \input{image/interact_with_policy/first_person_view.pdf}
        \caption{Interact with RL Agent in the first-person view (left) and simulation (right)}
        \label{fig:interact_with_policy_first_person}
    \end{subfigure}
    \vskip 1em
    \begin{subfigure}[b]{0.48\textwidth}
        \centering
        \includegraphics[width=\linewidth]{image/interact_with_policy/third_person_view.png}
        \caption{Interact with RL Agent in the third-person view}
        \label{fig:interact_with_policy_third_person}
    \end{subfigure}
    \caption{
Playing table tennis with RL agent in Fancy Gym environment, the RL agent policy is trained with \textit{Deep Black-Box Reinforcement Learning} (BBRL) \cite{otto2023deep} 
    }
    \label{fig:interact_with_policy}
\end{figure}


To demonstrate the capabilities of IRIS,
we designed an experiment in which a participant played table tennis against an episodic RL agent trained with \textit{Deep Black-Box Reinforcement Learning} (BBRL) \cite{otto2023deep} exclusively in simulation. 
The environment was adapted from Fancy Gym \cite{fancy_gym}.
Table tennis is a challenging task that requires dynamic perspective shifts and precise racket control.
Using a motion controller and XR headset, the participant learned to successfully return the ball the RL agent served within a few minutes.
This success highlights the potential of IRIS to facilitate high-quality interactive data collection, which can further enhance the training and performance of RL agents.

% table tennis

\subsection{Real World Teleoperation and Data Collection}
\label{sec:real_world_teleop}

Real robot data plays a crucial role in conducting real-world experiments. To minimize physical obstruction from humans, tele-operation using XR (Extended Reality) is commonly employed for collecting such data.
Some approaches, such as \cite{openteach}, utilize cameras to stream videos to XR headsets. However, this method lacks depth perception. Other works, like \cite{opentelevision}, employ movable platforms to track the movement of users' heads, allowing for active scene observation by adjusting the viewing perspective. While effective, this approach requires significant effort and resources to install and maintain the necessary hardware.

To overcome these limitations, we integrate a point cloud-based XR tele-operation system into our tele-operation framework, ensuring both immersion and interactivity for data collection. An ORBBEC Femto Bolt depth camera captures real-time depth and RGB data, and a QR code on the robot’s end effector enables camera-to-robot base transformation, aligning point clouds with the robot’s frame. A GPU-accelerated C++ pipeline processes (see Appendix~\ref{appendix:Point Cloud Processing Pipeline}) point clouds at 30Hz (\textasciitilde 2 million points per frame), applying voxel grid downsampling to reduce computational load and latency while maintaining real-time performance. Before downsampling, unnecessary regions are cropped, retaining only relevant portions of the workspace. The processed point clouds are transmitted to a main process, which forwards them to a Meta Quest 3 headset for real-time visualization and interaction.

\begin{figure}[h!]
    \centering
    \includegraphics[width=0.48\textwidth]{image/Real_robot/real_robot_img.png}
    \caption{Real-world teleoperation system integrating augmented reality for intuitive robot control. The setup includes two Franka Emika Panda robots: a Leader Robot (cyan) controlled by a Meta Quest 3 user and a Follower Robot (blue) mirroring its movements. An ORBBEC Femto Bolt depth camera (yellow) captures the scene for real-time point cloud visualization (green). The top-left and top-right images show the augmented view, reconstructing objects (e.g., red and yellow cups) in XR. The bottom image presents the physical setup, where the Leader Robot is directly controlled while the Follower Robot autonomously replicates actions, enabling precise, immersive remote operation.}
    \label{fig:real_robot_point_cloud_setup}
\end{figure}


Our teleoperation framework features two Franka Emika Panda \cite{Franka_Emika_Robot} robots: a leader controlled by a human in zero-torque mode and a follower mirroring it's movements, including gripper actions. This setup enables seamless remote manipulation while eliminating the need for human presence in the physical workspace. In future implementations, additional cameras can be incorporated, and point cloud fusion via Iterative Closest Point (ICP) will ensure precise alignment across multiple views. The system architecture, including point cloud processing, XR integration, and teleoperation, is illustrated in Figure~\ref{fig:real_robot_point_cloud_setup}. By integrating real-time point cloud visualization with XR-based teleoperation, our system enhances efficiency, accuracy, and scalability, enabling obstruction-free data collection for human-robot interaction studies.




% Tips: for each results section, when you write each paragraph, try to follow this:
% (1) motivation, why are you doing this?
% (2) preset result (pure data)
% (3) give interpretation


\begin{figure}[t]
	\centering
	\includegraphics[width=0.49\textwidth]{figs/Fig5_exp_platforms.pdf}
    % \includesvg[width=0.49\textwidth, inkscapelatex=false]{figs/Fig5_exp_platforms.svg} 
	\caption{Tasks for the simulation experiments. Each task is tested with various feedback types, including accurate demonstrations, noisy demonstrations, and relative corrective feedback. For the TwoArm-Lift task, partial feedback is also tested by applying feedback only to one of the robots.}
	\label{fig:tasks}
\end{figure}



% \begin{table*}
% \footnotesize
% % \setlength{\tabcolsep}{10pt}
% \caption{Experimental results in simulation under various types of feedback data. SR indicates the success rate, and CT represents the convergence timestep. A ‘$\diagdown$’ symbol denotes that the algorithm did not converge. For calculating CT, $\diagdown$ entries are replaced with the maximum allowable timestep.}
% \label{tab:data_folder_presence}
% \begin{center}
% \begin{tabular}{lcccccccccccc}
% \Xhline{0.75pt}
% Method & \multicolumn{2}{c}{CLIC-Half } & \multicolumn{2}{c}{Diffusion} & \multicolumn{2}{c}{Implicit BC} & \multicolumn{2}{c}{PVP} & \multicolumn{2}{c}{CLIC-Simplified} & \multicolumn{2}{c}{HG-DAgger/D-COACH } \\
% \hline
%  \textbf{Accurate data}& SR & CT & SR & CT & SR & CT & SR & CT &SR & CT & SR & CT \vspace{2.5pt}\\
% PushT & \textbf{0.931} & 25.6 & 0.915 & 24.1 & 0.890 & 31.8 & 0.440 & 28.5 &  0.765 & 35.6 & 0.710 & 38.6  \\
% Square & 0.930 & 43.0 & \textbf{0.953} & 48.4 & 0.732 & 54.6 &  0.000 & $\diagdown$   &0.634 & 65.9 & 0.420 & 67.0 \\
% PickCan & 0.983 & 37.8 & 0.963 & 36.1 & 0.688 & 44.2 &  0.000 & $\diagdown$    & 0.995 & 42.5 & 0.990 & 35.7 \\
% TwoArmLift & 0.970 & 34.9 & 0.990 & 23.0 & 0.000 & 2.1 &  0.000 & $\diagdown$    & 0.902 & 14.0 & 0.982 & 14.9  \\
% \hline
% \textbf{Gaussian noise} & SR & CT & SR & CT & SR & CT & SR & CT &
% SR & CT & SR & CT\vspace{2.5pt}\\ 
% PushT & 0.880 & 35.6 & \textbf{0.893} & 49.0 & 0.735 & 42.1 & 0.155 & 45.8 & 0.663 & 41.4 & 0.598 & 41.0 \\
% Square & \textbf{0.925} & 64.5 & 0.000 & 2.1 & 0.000 & 2.1 &   0.000 & $\diagdown$    & 0.238 & 71.0 & 0.060 & 77.2 \\
% PickCan & \textbf{0.973} & 37.8 & 0.467 & 68.2 & 0.070 & 70.4 &   0.000 & $\diagdown$    & 0.800 & 69.5 & 0.028 & 23.1 \\
% TwoArmLift & \textbf{0.847} & 68.0 & 0.000 & 2.1 &   0.000 & $\diagdown$    &  0.000 & $\diagdown$    & 0.433 & 39.3 & 0.008 & 63.8 \\
% \hline
% \textbf{Partial feedback} & SR & CT & SR & CT & SR & CT & SR & CT &SR & CT & SR & CT\vspace{2.5pt}\\
% TwoArmLift & \textbf{0.990} & 26.9 & 0.897 & 29.7 & 0.000 & $\diagdown$  & 0.000 & $\diagdown$ & 0.780 & 19.1 & 0.687 & 25.7 \\
% \hline
% \textbf{Relative correction}  & SR & CT & SR & CT & SR & CT & SR & CT & SR & CT & SR & CT \vspace{2.5pt}\\
% PushT & \textbf{0.853} & 40.8 & 0.060 & 72.0 & 0.400 & 58.8 &   0.110	& 50.4   & 0.733 & 43.5 & 0.520 & 49.0 \\
% Square & \textbf{0.817} & 69.0 & 0.000 & 2.1 & 0.005 & 56.3 &   0.000 & $\diagdown$    & 0.065 & 66.1 & 0.243 & 79.7 \\
% PickCan & 0.870 & 40.8 & 0.000 & 2.0 & 0.310 & 81.7 &   0.000 & $\diagdown$    & \textbf{0.890} & 67.2 & 0.693 & 62.8 \\
% TwoArmLift & \textbf{0.860} & 31.1    &  0.000 & $\diagdown$   & 0.000  & $\diagdown$  &   0.000 & $\diagdown$   & 0.613 & 18.9 & 0.115 & 64.7 \\
% \hline
% \textbf{Direction noise}  & SR & CT & SR & CT & SR & CT & SR & CT & SR & CT & SR & CT \vspace{2.5pt}\\
% PushT & 0.700 & 48.1 & 0.187 & 67.9 & 0.574 & 55.5 & NaN & NaN & NaN & NaN & NaN & NaN \\
% Square & 0.870 & 63.6 & 0.125 & 75.8 & 0.230 & 70.4 & NaN & NaN & NaN & NaN & NaN & NaN \\
% PickCan & 0.850 & 33.6 & NaN & NaN & 0.482 & 81.4 & NaN & NaN & NaN & NaN & NaN & NaN \\
% TwoArmLift & 0.907 & 20.8 & 0.885 & 46.6 & NaN & NaN & NaN & NaN & NaN & NaN & NaN & NaN \\
% \hline
% Average & \textbf{0.928} & \textbf{42.1} &  0.675 & 46.3 & 0.346  & 48.9  & 0.066 & 62.7 & & & & \\
% \Xhline{0.75pt}
% \end{tabular}
% \end{center}
% \end{table*}


% \begin{table*}
% \footnotesize
% % \setlength{\tabcolsep}{10pt}
% \caption{Experimental results in simulation under various types of feedback data. SR indicates the success rate, and CT represents the convergence timestep. A ‘$\diagdown$’ symbol denotes that the algorithm did not converge. For calculating CT, $\diagdown$ entries are replaced with the maximum allowable timestep.}
% \label{tab:data_folder_presence}
% \begin{center}
% \begin{tabular}{lcccccccc|cccc}
% \Xhline{0.75pt}
% Method & \multicolumn{2}{c}{CLIC-Half } & \multicolumn{2}{c}{Diffusion} & \multicolumn{2}{c}{Implicit BC} & \multicolumn{2}{c}{PVP} & \multicolumn{2}{c}{CLIC-Simplified} & \multicolumn{2}{c}{HG-DAgger/D-COACH } \\
%  & SR & CT & SR & CT & SR & CT & SR & CT &SR & CT & SR & CT \\
%  \hline  \textbf{Accurate data}  && & & & & & & & && & \\
% PushT & \textbf{0.931} & 25.6 & 0.915 & 24.1 & 0.890 & 31.8 & 0.440 & 28.5 &  0.765 & 35.6 & 0.710 & 38.6  \\
% Square & 0.930 & 43.0 & \textbf{0.953} & 48.4 & 0.732 & 54.6 &  0.000 & $\diagdown$   &0.634 & 65.9 & 0.420 & 67.0 \\
% PickCan & 0.983 & 37.8 & 0.963 & 36.1 & 0.688 & 44.2 &  0.000 & $\diagdown$    & 0.995 & 42.5 & 0.990 & 35.7 \\
% TwoArmLift & 0.970 & 34.9 & 0.990 & 23.0 & 0.000 & 2.1 &  0.000 & $\diagdown$    & 0.902 & 14.0 & 0.982 & 14.9  \\
% \hline
% \textbf{Gaussian noise } && & & & & & & & && &\\ 
% PushT & 0.880 & 35.6 & \textbf{0.893} & 49.0 & 0.735 & 42.1 & 0.155 & 45.8 & 0.663 & 41.4 & 0.598 & 41.0 \\
% Square & \textbf{0.925} & 64.5 & 0.000 & 2.1 & 0.000 & 2.1 &   0.000 & $\diagdown$    & 0.238 & 71.0 & 0.060 & 77.2 \\
% PickCan & \textbf{0.973} & 37.8 & 0.467 & 68.2 & 0.070 & 70.4 &   0.000 & $\diagdown$    & 0.800 & 69.5 & 0.028 & 23.1 \\
% TwoArmLift & \textbf{0.847} & 68.0 & 0.000 & 2.1 &   0.000 & $\diagdown$    &  0.000 & $\diagdown$    & 0.433 & 39.3 & 0.008 & 63.8 \\
% \hline
% \textbf{Partial feedback } && & & & & & & & && & \\
% TwoArmLift & \textbf{0.990} & 26.9 & 0.897 & 29.7 & 0.000 & $\diagdown$  & 0.000 & $\diagdown$ & 0.780 & 19.1 & 0.687 & 25.7 \\
% \hline
% \textbf{Relative correction} && & & & & & & & && &  \\
% PushT & \textbf{0.853} & 40.8 & 0.060 & 72.0 & 0.400 & 58.8 &   0.110	& 50.4   & 0.733 & 43.5 & 0.520 & 49.0 \\
% Square & \textbf{0.817} & 69.0 & 0.000 & 2.1 & 0.005 & 56.3 &   0.000 & $\diagdown$    & 0.065 & 66.1 & 0.243 & 79.7 \\
% PickCan & 0.870 & 40.8 & 0.000 & 2.0 & 0.310 & 81.7 &   0.000 & $\diagdown$    & \textbf{0.890} & 67.2 & 0.693 & 62.8 \\
% TwoArmLift & \textbf{0.860} & 31.1    &  0.000 & $\diagdown$   & 0.000  & $\diagdown$  &   0.000 & $\diagdown$   & 0.613 & 18.9 & 0.115 & 64.7 \\
% \hline
% \textbf{Direction noise }   && & & & & & & & && & \\
% PushT & 0.700 & 48.1 & 0.187 & 67.9 & 0.574 & 55.5 & NaN & NaN & NaN & NaN & NaN & NaN \\
% Square & 0.870 & 63.6 & 0.125 & 75.8 & 0.230 & 70.4 & NaN & NaN & NaN & NaN & NaN & NaN \\
% PickCan & 0.850 & 33.6 & NaN & NaN & 0.482 & 81.4 & NaN & NaN & NaN & NaN & NaN & NaN \\
% TwoArmLift & 0.907 & 20.8 & 0.885 & 46.6 & NaN & NaN & NaN & NaN & NaN & NaN & NaN & NaN \\
% \hline
% Average & \textbf{0.928} & \textbf{42.1} &  0.675 & 46.3 & 0.346  & 48.9  & 0.066 & 62.7 & & & & \\
% \Xhline{0.75pt}
% \end{tabular}
% \end{center}
% \end{table*}


\begin{table*}[t!]
\footnotesize
% \setlength{\tabcolsep}{10pt}
\caption{Experimental results in simulation under accurate feedback data. SR indicates the success rate, and CT represents the convergence timestep ($\times 10^3$). A ‘$\diagdown$’ symbol denotes that the algorithm did not converge. 
% For calculating CT, $\diagdown$ entries are replaced with the maximum allowable timestep.
}
\label{tab:sim_exp_accurate}
\begin{center}
\begin{tabular}{lcccccccccc|cccc}
\Xhline{0.75pt}
Method & \multicolumn{2}{c}{CLIC-Half (ours) } &  \multicolumn{2}{c}{CLIC-Circular (ours)} & \multicolumn{2}{c}{Diffusion Policy} & \multicolumn{2}{c}{Implicit BC} & \multicolumn{2}{c}{PVP} & \multicolumn{2}{c}{CLIC-Explicit (ours)} & \multicolumn{2}{c}{HG-DAgger} \\
 & SR & CT & SR & CT & SR & CT & SR & CT & SR & CT &SR & CT & SR & CT \\ \hline
 %  \textbf{Accurate data}  && & & & & & & & & & && & \\
Push-T & 0.931 & 25.6  & \textbf{0.955} & 28.3 & 0.915 & 24.1 & 0.890 & 31.8 & 0.440 & 28.5 &  0.765 & 35.6 & 0.710 & 38.6  \\
Square & 0.930 & 43.0 & \textbf{0.960} & 55.5 & 0.953 & 48.4 & 0.732 & 54.6 &  0.000 & $\diagdown$   &0.634 & 65.9 & 0.420 & 67.0 \\
Pick-Can & 0.983 & 37.8 & \textbf{0.990} &  38.4 & 0.980 & 36.1 & 0.688 & 44.2 &  0.000 & $\diagdown$    & 0.995 & 42.5 & 0.990 & 35.7 \\
TwoArm-Lift & 0.970 & 18.5 &  \textbf{0.990} & 12.6 & \textbf{0.990} & 23.0 & 0.000 & $\diagdown$ &  0.000 & $\diagdown$    & 0.932 & 14.7 & 0.982 & 14.9  \\
\hline
Average & {0.954} & 31.2 & \textbf{0.974} & 33.7 &  0.960 & 32.9 & 0.578  & $\diagdown$  & 0.066 & $\diagdown$ & 0.836 & 39.7 & 0.776 & 39.1 \\
\Xhline{0.75pt}
\end{tabular}
\end{center}
\end{table*}

% \vspace{-20pt}

\section{Experiments}
\label{sec:experiments}

In the experiment section, we demonstrate the effectiveness of our CLIC method through a series of simulations and real-world experiments. In Section \ref{sec:exp:simulation}, we compare CLIC with state-of-the-art methods under various types of feedback in the robot action space. Section \ref{sec:exp:ablation} presents an ablation study, analyzing the impact of key parameters and design choices. In Section \ref{sec:exp:toy_exp}, a 2D toy experiment highlights how CLIC prevents overfitting. Finally, Section \ref{sec:exp:real_rotbo} showcases the performance of CLIC in real-robot experiments.


\subsection{Simulation Experiments}
\label{sec:exp:simulation}
\textbf{Baselines} We compare CLIC with multiple baselines.
For explicit policies, we consider HG-DAgger \cite{2019_HG_DAgger} and D-COACH \cite{2019_Rodrigo_D_COACH}, which are IIL algorithms that learn from demonstration data and relative correction data, respectively. These two methods are refined for better performance, as reported in Appendix \ref{appendix:baselines}.
For implicit policies, the baselines include IBC \cite{2022_implicit_BC}, PVP \cite{2023_NIPS_PVP}, and Diffusion Policy \cite{2023_diffusionpolicy}. 
As IBC and Diffusion Policy are originally offline IL methods, to make fair comparisons, we adapt them to the IIL framework to ensure fair comparisons. 
Within this IIL framework, all methods share the same structure, differing only in their specific policy update methods.
PVP \cite{2023_NIPS_PVP} employs a loss function to assign low energy values to human actions $\bm a^h$ and high energy values to robot actions $\bm a^r$, given an observed action pair $(\bm a^r, \bm a^h)$ at state $\bm s$.
IBC, detailed in Section \ref{sec:Preliminaries}, and PVP are closely related to our method because they both involve training energy-based models. 
The Diffusion Policy is a counterpart method to IBC when learning from demonstrations. It outperforms IBC because of the improved training stability offered by the diffusion model. 
To ensure fair comparisons, we use a consistent velocity control scheme across all methods. Specifically, each method outputs a velocity command for the robot's end-effector and, if applicable, the gripper as the action at each time step.

% \textbf{Baselines} We compare CLIC with multiple baselines.
% For implicit policies, the baselines include Implicit BC (IBC) \cite{2022_implicit_BC}, PVP \cite{2023_NIPS_PVP}, and Diffusion Policy \cite{2023_diffusionpolicy}. For explicit policies, the baselines include HG-DAgger \cite{2019_HG_DAgger} and D-COACH \cite{2019_Rodrigo_D_COACH}.
% HG-DAgger and D-COACH are Interactive IL algorithms that learn from demonstration data and relative corrective data, respectively, both assuming an explicit Gaussian policy\footnote{The methods are refined for better performance, reported in Appendix \ref{appendix:HGDAgger}.}.
% When adapting offline IL baselines to the Interactive IL framework, we modify the policy update process while retaining the overall IIL structure.
% The Proxy Value Propagation (PVP) \cite{2023_NIPS_PVP} method utilizes a Proxy Value loss to assign high Q-values to human actions $\bm a^h$ and low Q-values to robot actions $\bm a^r$ that trigger human intervention.
% The IBC method, detailed in Section \ref{sec:Preliminaries}, and PVP are closely related to our method, as they both involve training energy-based models. 
% The Diffusion Policy is a counterpart method to IBC when learning from demonstrations, which outperforms IBC because of improved training stability offered by the diffusion model. 
% To ensure fair comparisons, we use a consistent velocity control scheme across all methods. Specifically, each method outputs a velocity command for the robot's end-effector as the action at each time step.



\textbf{Tasks and metrics} We compared these methods across four simulated tasks, including a Push-T task introduced in \cite{2023_diffusionpolicy} and three manipulation tasks from the robosuite benchmark \citep{2020_robosuite}, as illustrated in Fig. \ref{fig:tasks} and described in Appendix \ref{appendix:simulated_experiments_task details}.
The agent is trained by each method using an IIL framework, where the agent interacts with the environment and receives feedback from a \textbf{simulated teacher}.
This simulated teacher is employed to guarantee repeatability and fairness in training, because human teachers may not provide consistent feedback across different experimental trials. 
This is an expert policy that compares its actions with those of the learner every $n$ time steps. If the distance between these actions exceeds a threshold (set at 0.2 for all tasks), the simulated teacher provides feedback to the learner.  For all the simulation tasks, we set $n=2$.
 Each method was run for 160 episodes in every experiment, with this entire procedure repeated 3 times to calculate average final success rates and convergence time steps. 
Specifically, for each individual experiment, we calculated the final success rate by averaging the success rates of the last 8 episodes, each determined by evaluating the learned policy 10 times at the end of that episode.
We defined the convergence time step as the earliest time step when the success rate exceeded 90\% of the final success rate.

\begin{table}[t!]
\caption{Various feedback in the action space}
\centering
% \begin{tabular}{@{}ll@{}}
\begin{tabularx}{0.49\textwidth}{@{}lX@{}}
\toprule
\textbf{Type of Feedback Data}      & \textbf{Definition} \\ \midrule
Accurate absolute correction              & \( \bm a^h = \bm a^* \) \\
Gaussian noise             & \( \bm a^h = \bm a^* + \bm \omega, \bm \omega \sim \mathcal{N}(\bm0, ||\bm a^* - \bm a^r||) \) \\
Partial feedback           & \( \bm a^h \in \{[\bm a^*_{r1}, \bm a_{r2}], [\bm a_{r1}, \bm a_{r2}^*]\} \) \\ 
\hline
Accurate relative correction              & \(\bm  a^h = \bm a^r + e\bm h^*, \bm h^*= \frac{\bm a^* - \bm a^r}{||\bm a^* - \bm a^r||}  \) \\
Direction noise            & \( \bm a^h = \bm a^r + e \bm h_r \), \( \angle (\bm h_r, \bm h^*) = \beta \in [0, 90^\circ) \) \\
\bottomrule
\end{tabularx}
\label{tab:feedback-definitions}
\end{table}
% briefly show different types of feedback
\textbf{Feedback types}
In addition to accurate absolute and relative corrections, Table \ref{tab:feedback-definitions} summarizes other common types of feedback humans provide. These feedback types are also utilized in the simulation experiments. Here, the optimal action $\bm a^*$ is the original action taken by the simulated teacher. The partial feedback is utilized in the TwoArm-Lift task, where $\bm a_{r\,i}, i\in\{1, 2\}$ denotes each robot's action, and $\bm a_{ri}^*$ denotes its optimal action. 


% \begin{table*}
% \footnotesize
% % \setlength{\tabcolsep}{10pt}
% \caption{Experimental results in simulation under various types of feedback data. SR indicates the success rate, and CT represents the convergence timestep. A ‘$\diagdown$’ symbol denotes that the algorithm did not converge. For calculating CT, $\diagdown$ entries are replaced with the maximum allowable timestep.}
% \label{tab:data_folder_presence}
% \begin{center}
% \begin{tabular}{lcccccccccccc}
% \Xhline{0.75pt}
% Method & \multicolumn{2}{c}{CLIC-Half } & \multicolumn{2}{c}{Diffusion} & \multicolumn{2}{c}{Implicit BC} & \multicolumn{2}{c}{PVP} & \multicolumn{2}{c}{CLIC-Simplified} & \multicolumn{2}{c}{HG-DAgger/D-COACH } \\
% \hline
%  \textbf{Accurate data}& SR & CT & SR & CT & SR & CT & SR & CT &SR & CT & SR & CT \vspace{2.5pt}\\
% PushT & \textbf{0.931} & 25.6 & 0.915 & 24.1 & 0.890 & 31.8 & 0.440 & 28.5 &  0.765 & 35.6 & 0.710 & 38.6  \\
% Square & 0.930 & 43.0 & \textbf{0.953} & 48.4 & 0.732 & 54.6 &  0.000 & $\diagdown$   &0.634 & 65.9 & 0.420 & 67.0 \\
% PickCan & 0.983 & 37.8 & 0.963 & 36.1 & 0.688 & 44.2 &  0.000 & $\diagdown$    & 0.995 & 42.5 & 0.990 & 35.7 \\
% TwoArmLift & 0.970 & 34.9 & 0.990 & 23.0 & 0.000 & 2.1 &  0.000 & $\diagdown$    & 0.902 & 14.0 & 0.982 & 14.9  \\
% \hline
% \textbf{Gaussian noise} & SR & CT & SR & CT & SR & CT & SR & CT &
% SR & CT & SR & CT\vspace{2.5pt}\\ 
% PushT & 0.880 & 35.6 & \textbf{0.893} & 49.0 & 0.735 & 42.1 & 0.155 & 45.8 & 0.663 & 41.4 & 0.598 & 41.0 \\
% Square & \textbf{0.925} & 64.5 & 0.000 & 2.1 & 0.000 & 2.1 &   0.000 & $\diagdown$    & 0.238 & 71.0 & 0.060 & 77.2 \\
% PickCan & \textbf{0.973} & 37.8 & 0.467 & 68.2 & 0.070 & 70.4 &   0.000 & $\diagdown$    & 0.800 & 69.5 & 0.028 & 23.1 \\
% TwoArmLift & \textbf{0.847} & 68.0 & 0.000 & 2.1 &   0.000 & $\diagdown$    &  0.000 & $\diagdown$    & 0.433 & 39.3 & 0.008 & 63.8 \\
% \hline
% \textbf{Partial feedback} & SR & CT & SR & CT & SR & CT & SR & CT &SR & CT & SR & CT\vspace{2.5pt}\\
% TwoArmLift & \textbf{0.990} & 26.9 & 0.897 & 29.7 & 0.000 & $\diagdown$  & 0.000 & $\diagdown$ & 0.780 & 19.1 & 0.687 & 25.7 \\
% \hline
% \textbf{Relative correction}  & SR & CT & SR & CT & SR & CT & SR & CT & SR & CT & SR & CT \vspace{2.5pt}\\
% PushT & \textbf{0.853} & 40.8 & 0.060 & 72.0 & 0.400 & 58.8 &   0.110	& 50.4   & 0.733 & 43.5 & 0.520 & 49.0 \\
% Square & \textbf{0.817} & 69.0 & 0.000 & 2.1 & 0.005 & 56.3 &   0.000 & $\diagdown$    & 0.065 & 66.1 & 0.243 & 79.7 \\
% PickCan & 0.870 & 40.8 & 0.000 & 2.0 & 0.310 & 81.7 &   0.000 & $\diagdown$    & \textbf{0.890} & 67.2 & 0.693 & 62.8 \\
% TwoArmLift & \textbf{0.860} & 31.1    &  0.000 & $\diagdown$   & 0.000  & $\diagdown$  &   0.000 & $\diagdown$   & 0.613 & 18.9 & 0.115 & 64.7 \\
% \hline
% \textbf{Direction noise}  & SR & CT & SR & CT & SR & CT & SR & CT & SR & CT & SR & CT \vspace{2.5pt}\\
% PushT & 0.700 & 48.1 & 0.187 & 67.9 & 0.574 & 55.5 &   &   &   &   &   &   \\
% Square & 0.870 & 63.6 & 0.125 & 75.8 & 0.230 & 70.4 &   &   &   &   &   &   \\
% PickCan & 0.850 & 33.6 &   &   & 0.482 & 81.4 &   &   &   &   &   &   \\
% TwoArmLift & 0.907 & 20.8 & 0.885 & 46.6 &   &   &   &   &   &   &   &   \\
% \hline
% Average & \textbf{0.928} & \textbf{42.1} &  0.675 & 46.3 & 0.346  & 48.9  & 0.066 & 62.7 & & & & \\
% \Xhline{0.75pt}
% \end{tabular}
% \end{center}
% \end{table*}


% \begin{table*}
% \footnotesize
% % \setlength{\tabcolsep}{10pt}
% \caption{Experimental results in simulation under various types of feedback data. SR indicates the success rate, and CT represents the convergence timestep. A ‘$\diagdown$’ symbol denotes that the algorithm did not converge. For calculating CT, $\diagdown$ entries are replaced with the maximum allowable timestep.}
% \label{tab:data_folder_presence}
% \begin{center}
% \begin{tabular}{lcccccccc|cccc}
% \Xhline{0.75pt}
% Method & \multicolumn{2}{c}{CLIC-Half } & \multicolumn{2}{c}{Diffusion} & \multicolumn{2}{c}{Implicit BC} & \multicolumn{2}{c}{PVP} & \multicolumn{2}{c}{CLIC-Simplified} & \multicolumn{2}{c}{HG-DAgger/D-COACH } \\
%  & SR & CT & SR & CT & SR & CT & SR & CT &SR & CT & SR & CT \\
%  \hline  \textbf{Accurate data}  && & & & & & & & && & \\
% PushT & \textbf{0.931} & 25.6 & 0.915 & 24.1 & 0.890 & 31.8 & 0.440 & 28.5 &  0.765 & 35.6 & 0.710 & 38.6  \\
% Square & 0.930 & 43.0 & \textbf{0.953} & 48.4 & 0.732 & 54.6 &  0.000 & $\diagdown$   &0.634 & 65.9 & 0.420 & 67.0 \\
% PickCan & 0.983 & 37.8 & 0.963 & 36.1 & 0.688 & 44.2 &  0.000 & $\diagdown$    & 0.995 & 42.5 & 0.990 & 35.7 \\
% TwoArmLift & 0.970 & 34.9 & 0.990 & 23.0 & 0.000 & 2.1 &  0.000 & $\diagdown$    & 0.902 & 14.0 & 0.982 & 14.9  \\
% \hline
% \textbf{Gaussian noise } && & & & & & & & && &\\ 
% PushT & 0.880 & 35.6 & \textbf{0.893} & 49.0 & 0.735 & 42.1 & 0.155 & 45.8 & 0.663 & 41.4 & 0.598 & 41.0 \\
% Square & \textbf{0.925} & 64.5 & 0.000 & 2.1 & 0.000 & 2.1 &   0.000 & $\diagdown$    & 0.238 & 71.0 & 0.060 & 77.2 \\
% PickCan & \textbf{0.973} & 37.8 & 0.467 & 68.2 & 0.070 & 70.4 &   0.000 & $\diagdown$    & 0.800 & 69.5 & 0.028 & 23.1 \\
% TwoArmLift & \textbf{0.847} & 68.0 & 0.000 & 2.1 &   0.000 & $\diagdown$    &  0.000 & $\diagdown$    & 0.433 & 39.3 & 0.008 & 63.8 \\
% \hline
% \textbf{Partial feedback } && & & & & & & & && & \\
% TwoArmLift & \textbf{0.990} & 26.9 & 0.897 & 29.7 & 0.000 & $\diagdown$  & 0.000 & $\diagdown$ & 0.780 & 19.1 & 0.687 & 25.7 \\
% \hline
% \textbf{Relative correction} && & & & & & & & && &  \\
% PushT & \textbf{0.853} & 40.8 & 0.060 & 72.0 & 0.400 & 58.8 &   0.110	& 50.4   & 0.733 & 43.5 & 0.520 & 49.0 \\
% Square & \textbf{0.817} & 69.0 & 0.000 & 2.1 & 0.005 & 56.3 &   0.000 & $\diagdown$    & 0.065 & 66.1 & 0.243 & 79.7 \\
% PickCan & 0.870 & 40.8 & 0.000 & 2.0 & 0.310 & 81.7 &   0.000 & $\diagdown$    & \textbf{0.890} & 67.2 & 0.693 & 62.8 \\
% TwoArmLift & \textbf{0.860} & 31.1    &  0.000 & $\diagdown$   & 0.000  & $\diagdown$  &   0.000 & $\diagdown$   & 0.613 & 18.9 & 0.115 & 64.7 \\
% \hline
% \textbf{Direction noise }   && & & & & & & & && & \\
% PushT & 0.700 & 48.1 & 0.187 & 67.9 & 0.574 & 55.5 &   &   &   &   &   &   \\
% Square & 0.870 & 63.6 & 0.125 & 75.8 & 0.230 & 70.4 &   &   &   &   &   &   \\
% PickCan & 0.850 & 33.6 &   &   & 0.482 & 81.4 &   &   &   &   &   &   \\
% TwoArmLift & 0.907 & 20.8 & 0.885 & 46.6 &   &   &   &   &   &   &   &   \\
% \hline
% Average & \textbf{0.928} & \textbf{42.1} &  0.675 & 46.3 & 0.346  & 48.9  & 0.066 & 62.7 & & & & \\
% \Xhline{0.75pt}
% \end{tabular}
% \end{center}
% \end{table*}


\begin{table*}[t!]
\footnotesize
% \setlength{\tabcolsep}{10pt}
\caption{Simulation results under noisy demonstration data. SR: success rate, CT: convergence timestep ($\times 10^3$). }
\label{tab:sim_exp_noise}
\begin{center}
\begin{tabular}{lcccccccccc|cccc}
\Xhline{0.75pt}
Method & \multicolumn{2}{c}{CLIC-Half  } & \multicolumn{2}{c}{CLIC-Circular } & \multicolumn{2}{c}{Diffusion Policy} & \multicolumn{2}{c}{Implicit BC} & \multicolumn{2}{c}{PVP} & \multicolumn{2}{c}{CLIC-Explicit} & \multicolumn{2}{c}{HG-DAgger} \\
 & SR & CT & SR & CT & SR & CT & SR & CT &SR & CT & SR & CT & SR & CT \\
%  \hline  \textbf{Accurate data}  && & & & & & & & && & \\
% PushT & \textbf{0.931} & 25.6 & 0.915 & 24.1 & 0.890 & 31.8 & 0.440 & 28.5 &  0.765 & 35.6 & 0.710 & 38.6  \\
% Square & 0.930 & 43.0 & \textbf{0.953} & 48.4 & 0.732 & 54.6 &  0.000 & $\diagdown$   &0.634 & 65.9 & 0.420 & 67.0 \\
% PickCan & 0.983 & 37.8 & 0.963 & 36.1 & 0.688 & 44.2 &  0.000 & $\diagdown$    & 0.995 & 42.5 & 0.990 & 35.7 \\
% TwoArmLift & 0.970 & 34.9 & 0.990 & 23.0 & 0.000 & 2.1 &  0.000 & $\diagdown$    & 0.902 & 14.0 & 0.982 & 14.9  \\
\hline
\textbf{Gaussian noise } && & & & & & & & & & && &\\ 
Push-T & 0.880 & 35.6 &  \textbf{0.960} & 29.2 & 0.893 & 49.0 & 0.735 & 42.1 & 0.155 & 45.8 & 0.663 & 41.4 & 0.598 & 41.0 \\
Square & \textbf{0.925} & 64.5 & 0.855 & 63.9 & 0.000 &  $\diagdown$ & 0.000 &  $\diagdown$ &   0.000 & $\diagdown$    & 0.238 & 71.0 & 0.060 & 77.2 \\
Pick-Can & 0.973 & 37.8 & \textbf{1.000} & 42.8 & 0.467 & 68.2 & 0.070 & 70.4 &   0.000 & $\diagdown$    & 0.800 & 69.5 & 0.028 & 23.1 \\
TwoArm-Lift & 0.847 & 47.0 & \textbf{0.945} &  19.2& 0.000 & $\diagdown$ &   0.000 & $\diagdown$    &  0.000 & $\diagdown$    & 0.433 & 39.3 & 0.008 & 63.8 \\
\hline
% \textbf{Partial feedback } && & & & & & & & && & \\
% TwoArmLift & \textbf{0.990} & 26.9 & 0.897 & 29.7 & 0.000 & $\diagdown$  & 0.000 & $\diagdown$ & 0.780 & 19.1 & 0.687 & 25.7 \\
% \hline
% \textbf{Relative correction} && & & & & & & & && &  \\
% PushT & \textbf{0.853} & 40.8 & 0.060 & 72.0 & 0.400 & 58.8 &   0.110	& 50.4   & 0.733 & 43.5 & 0.520 & 49.0 \\
% Square & \textbf{0.817} & 69.0 & 0.000 & 2.1 & 0.005 & 56.3 &   0.000 & $\diagdown$    & 0.065 & 66.1 & 0.243 & 79.7 \\
% PickCan & 0.870 & 40.8 & 0.000 & 2.0 & 0.310 & 81.7 &   0.000 & $\diagdown$    & \textbf{0.890} & 67.2 & 0.693 & 62.8 \\
% TwoArmLift & \textbf{0.860} & 31.1    &  0.000 & $\diagdown$   & 0.000  & $\diagdown$  &   0.000 & $\diagdown$   & 0.613 & 18.9 & 0.115 & 64.7 \\
% \hline
% \hline
Average & 0.906 & 46.2 & \textbf{0.933} &  42.0 &  0.340 & $\diagdown$ & 0.268  & $\diagdown$  & 0.039 & $\diagdown$ & 0.566 & 53.3 & 0.172& 35.7 \\
\hline
\textbf{Direction noise }   && & & & & & & & & & && & \\
Push-T & 0.700 & 48.1 &  \textbf{0.950} & 27.3  & 0.187 & 67.9 & 0.574 & 55.5 & 0.000  &  $\diagdown$ & 0.638 & 44.6 & 0.473 & 43.6 \\
Square & 0.870 & 63.6 &  \textbf{0.910} & 58.9 & 0.125 & 75.8 & 0.230 & 70.4 & 0.000  &  $\diagdown$ & 0.161 & 66.9 & 0.128 & 71.6 \\
Pick-Can & \textbf{1.000} & 43.1 & \textbf{1.000} &  39.0 &   0.000  & $\diagdown$   & 0.482 & 81.4 & 0.000  & $\diagdown$  & 0.867 & 55.4 & 0.342 & 72.0 \\
TwoArm-Lift & 0.965 & 18.7 & \textbf{0.980} & 16.5  & 0.885 & 46.6 &   0.000 & $\diagdown$   &   0.000  & $\diagdown$   & 0.807 & 21.0 & 0.157 & 31.4 \\
\hline
Average & 0.884 & 43.4 &    \textbf{0.960} & 35.4 & 0.399  & $\diagdown$  & 0.429 & $\diagdown$ & 0.000 & $\diagdown$ &  0.618& 47.0 & 0.275 &  54.7 \\
\Xhline{0.75pt}
\end{tabular}
\end{center}
\end{table*}

\subsubsection{Experiments with accurate feedback}
\label{sec:exp:accurate_feedback}
% briefly introduce the tasks (or in appendix)

Table \ref{tab:sim_exp_accurate} shows the results when the teacher's feedback has no noise. 

\textbf{CLIC-Half outperforms IBC, PVP, and performs on par with Diffusion Policy} 
The results shown in Table \ref{tab:sim_exp_accurate} indicate that CLIC-Half constantly outperforms IBC and PVP in terms of success rate and convergence timesteps. 
PVP fails at the robosuite tasks because its loss function only considers the energy value of observed action pairs and cannot effectively shape the EBM. 
The optimal action assumption of IBC and Diffusion Policy is valid when the teacher's demonstration feedback is noise-free.
Under such ideal conditions, this assumption should provide more informative guidance than the assumption used by CLIC-Half.
However, IBC performance decreases as the action dimension of the task increases, with zero success rate in the TwoArm-Lift task. 
Notably, CLIC-Half achieves a $37.6 \%$ higher average success rate compared to IBC. 
Besides, CLIC-Half achieves a similar performance to that of Diffusion Policy.
These results highlight the effectiveness of training EBMs using half-space desired action spaces.

% IBC performance decreases as the action dimension of the task increases, with zero success rate in the TwoArm-Lift task. 
% CLIC-Half achieves a $37.6 \%$ higher average success rate compared to IBC. 
% Besides, CLIC-Half achieves a similar performance to that of Diffusion Policy.
% These results are remarkable because the optimal action assumption of IBC and Diffusion Policy holds true when the teacher's demonstration feedback is noise-free.
% In such ideal conditions, this assumption should provide more informative guidance than the assumption used by CLIC-Half.


 
% Therefore, this result further highlights a promising alternative approach to training policies through the estimation of the optimal actions using desired action spaces.


% motivation, why are you doing this?
% preset result (pure data)
% give interpretation
\textbf{CLIC-Circular outperforms all the baselines} 
% motivation, why are you doing this?
CLIC-Circular is the version of CLIC that mostly closely resembles IBC, as it utilizes a circular desired action space under the assumption of demonstration data. 
It reduces to IBC if all the three following conditions are met: (1) a very small radius defines the circular desired action space, (2) the temperature of the sigmoid function goes to zero, and (3) the uniform Bayes loss is used instead of policy-weighted Bayes loss.
However, CLIC-Circular outperforms IBC by a large margin. Notably,  CLIC-Circular can achieve a 99$\%$ success rate in the TwoArm-Lift task while IBC achieves zero.
CLIC-Circular also outperforms CLIC-Half as it has a more strict assumption. 
Moreover, the fact that CLIC-Circular outperforms Diffusion Policy underscores the capacity of policies represented by EBMs to surpass their diffusion-based counterparts.
Therefore, while previous studies \cite{2022_arxiv_IBC_gaps, 2023_diffusionpolicy} highlight the challenges of training EBMs, our results indicate that EBM-based policies can be trained reliably using demonstration data, by leveraging the concept of the desired action space.






\begin{table*}
\footnotesize
% \setlength{\tabcolsep}{10pt}
\caption{Simulation results under partial and relative feedback data. SR: success rate, CT: convergence timestep ($\times 10^3$). 
% A ‘$\diagdown$’ symbol denotes that the algorithm did not converge. For calculating CT, $\diagdown$ entries are replaced with the maximum allowable timestep.
}
\label{tab:sim_exp_relative_partial}
\begin{center}
\begin{tabular}{lcccccccccc|cccc}
\Xhline{0.75pt}
Method & \multicolumn{2}{c}{CLIC-Half } & \multicolumn{2}{c}{CLIC-Circular } & \multicolumn{2}{c}{Diffusion Policy} & \multicolumn{2}{c}{Implicit BC} & \multicolumn{2}{c}{PVP} & \multicolumn{2}{c}{CLIC-Explicit} & \multicolumn{2}{c}{D-COACH} \\
 & SR & CT & SR & CT & SR & CT & SR & CT &SR & CT & SR & CT & SR & CT \\\hline
\textbf{Partial feedback } && & & & & & & & && & \\
TwoArm-Lift & \textbf{0.990} & 26.9 & 0.920 &  17.8   & 0.897 & 29.7 & 0.000 & $\diagdown$  & 0.000 & $\diagdown$ & 0.863 & 18.1 & 0.687 & 25.7 \\
\hline
\textbf{Relative correction} && & &  & & & & & & & && &  \\
Push-T & \textbf{0.853} & 40.8  & 0.000 & $\diagdown$    & 0.060 & 72.0 & 0.400 & 58.8 &   0.110	& 50.4   & 0.733 & 43.5 & 0.520 & 49.0 \\
Square & \textbf{0.940} & 65.6 & 0.000 & $\diagdown$    & 0.000 & $\diagdown$ & 0.005 & 56.3 &   0.000 & $\diagdown$    & 0.065 & 66.1 & 0.243 & 79.7 \\
Pick-Can & \textbf{0.983} & 41.9 &  0.000 & $\diagdown$    & 0.000 & $\diagdown$ & 0.310 & 81.7 &   0.000 & $\diagdown$    & 0.890 & 67.2 & 0.693 & 62.8 \\
TwoArm-Lift & \textbf{0.955} & 25.3    &  0.000 & $\diagdown$    & 0.000 & $\diagdown$   & 0.000  & $\diagdown$  &   0.000 & $\diagdown$   & 0.920 & 16.8 & 0.115 & 64.7 \\
\hline
Average & \textbf{0.933} & \textbf{43.4} &  0.000 & $\diagdown$    & 0.015 & $\diagdown$ & 0.346  & $\diagdown$ & 0.066 & $\diagdown$ & 0.652 & 48.4 & 0.393 & 64.1 \\
\Xhline{0.75pt}
\end{tabular}
\end{center}
\end{table*}

\textbf{CLIC-Explicit achieves good results in uni-modal tasks, whereas PVP performs poorly}
The losses used by PVP and CLIC-Explicit share a critical similarity: they both increase the probability of human actions and decrease that of robot actions.
We denote this loss type as \textit{point-based loss} as it calculates the loss only on observed action pairs. In contrast, the losses for CLIC-Half, CLIC-Circular, and IBC are termed \textit{set-based loss}.
These losses utilize not only observed action pairs but also additional actions sampled from the EBM for loss calculation. 
Notably, PVP has zero success rate at robosuite tasks, whereas CLIC-Explicit performs well in Pick-Can and TwoArm-Lift—tasks that are both unimodal and align with the Gaussian policy assumption. 
Since PVP employs an EBM as its policy and CLIC-Explicit uses a Gaussian-parametrized policy, 
the result suggests that point-based loss is only effective for policies with simple forms, such as those based on Gaussian distributions, and fails to shape complex policies like EBMs. 
In contrast, the set-based loss provides richer information and is more effective for training EBMs. 





\subsubsection{Experiments with noisy demonstration feedback}
% motivation, why are you doing this?
% preset result (pure data)
% give interpretation

Noise is common when human teachers provide feedback to robots, either for absolute or relative corrections.
This can arise due to factors such as human fatigue or the limitations of teleoperation devices. Therefore, it is essential for algorithms to account for such noise.
To evaluate the ability of CLIC and baseline methods to learn from noisy feedback, we implemented two types of noise in simulation, as defined in Table \ref{tab:feedback-definitions}. 
For absolute corrections, we use a Gaussian distribution to model the noise, which is added to the original demonstration data. 
For relative corrections, the feedback is derived from absolute correction with a known magnitude. To introduce noise, we perturb the original direction signal by $45^\circ$ while maintaining its magnitude.
% we add direction noise to the original direction signal, resulting in a noisy direction signal with the same magnitude but has a $ 45 ^ \circ$ angle between the accurate signal. 
% To make the direction noise less challenging for the baselines, the relative correction is derived from absolute correction with known magnitude. 

% \textcolor{red}{detail more}

% \textbf{Implicit CLIC also outperforms Diffusion Policy when feedback is noisy or partial} 
\textbf{CLIC remains robust while baselines degrade under noisy feedback}
The results in Table \ref{tab:sim_exp_noise} show that, as feedback transitions from accurate to noisy, CLIC-Half and CLIC-Circular experience much smaller performance drops compared to the other baselines.
This can be observed by comparing Table \ref{tab:sim_exp_noise} with Table \ref{tab:sim_exp_accurate}.
In contrast, methods like Diffusion Policy and IBC perform significantly worse under noisy conditions.
This difference arises because these baselines depend on the strict assumption of having accurate demonstrations, making them unable to handle noisy feedback effectively. In comparison, CLIC allows adjusting the desired action space through hyperparameters, ensuring that the true optimal action remains within the desired action space even under noisy feedback.
This capability helps maintain robust performance under noisy conditions. 

\subsubsection{Experiments with relative or partial feedback}
\label{sec:exp:simulation_relative_partial}
% motivation, why are you doing this?
% Previous behavior cloning methods rely on accurate, high-quality demonstrations, limiting their use to scenarios with experts and precise teleoperation devices.  
When providing demonstrations is not possible, humans can provide feedback in more flexible ways, offering valuable information to guide improvement.
One such scenario involves partial feedback, where limitations in the control interface or a large action space make it challenging to provide complete demonstrations. In this case, we evaluated all methods on the TwoArm-Lift task, in which the teacher provides demonstration feedback to only one robot at a time.
Another scenario involves relative corrective feedback.
This feedback type is easier to provide than demonstration, because it does not require the teacher to know precisely which action should be taken; instead, it only necessitates an understanding of the general behavior the robot should exhibit.
To assess how effectively the methods handle this feedback type, we conducted experiments across four simulation tasks.


\textbf{CLIC-Half and CLIC-Explicit effectively learn from partial feedback}
The results of partial feedback experiments are shown in Table \ref{tab:sim_exp_relative_partial}.
In these experiments, human actions consist of two parts: actions on the \textit{feedback dimensions} (where human feedback is provided) and actions on the \textit{non-feedback dimensions} (which may be suboptimal).
While CLIC-Half maintains its success rate when transitioning from accurate demonstration to partial feedback, Diffusion Policy suffers from lower success rates and longer convergence times.  
This difference arises because the BC loss in Diffusion Policy attempts to imitate the entire teacher action, including non-feedback dimensions, potentially leading to suboptimal behaviors.
In contrast, CLIC-Half imitates a desired action space rather than a single action label.
It focuses on improving actions on the feedback dimensions while leaving the non-feedback dimensions unconstrained. 
As a result, CLIC-Half is robust to partial feedback as long as the optimal action lies within the desired action space.
The same reasoning explains the results of CLIC-Explicit outperforming HG-DAgger.
On the other hand, CLIC-Circular's performance drops because its circular desired action space might not include the optimal action. 
This result highlights the importance of ensuring the assumption of CLIC aligns with the data. 
% By using an intersected half-space that includes the optimal action, CLIC-Half avoids such issues, demonstrating consistent performance in partial feedback scenarios.

\textbf{CLIC-Half and CLIC-Explicit can learn from relative corrective feedback}
The results of relative corrective feedback are reported in Table  \ref{tab:sim_exp_relative_partial}. 
In these experiments, human actions improve upon robot actions but are not optimal.
CLIC-Half and CLIC-Explicit show only small performance drops compared to results in absolute corrections in Table \ref{tab:sim_exp_accurate}, because the correction magnitude information is unknown in relative correction.
In contrast, all demonstration-based baselines fail completely, achieving near-zero success rates. 
This failure occurs because the BC loss can mislead policy updates, especially when the current policy’s actions are better than the human actions in the dataset. 
Meanwhile, CLIC-Half and CLIC-Explicit construct desired action spaces to update the policy; as the policy improves, these spaces constructed from the dataset do not conflict with it and are still useful for policy improvement. 
Additionally, CLIC-Circular fails with a zero success rate, as its circular desired action space fails to include the optimal action.
Overall, the ability of CLIC-Half and CLIC-Explicit to learn from relative corrective feedback highlights their distinct advantages in such scenarios.
% As previously discussed in the analysis of partial feedback failures, the same limitations of behavior cloning methods contribute to their inability to handle relative corrective feedback effectively. CLIC’s ability to adapt highlights its distinct advantage in such scenarios.


\begin{figure}[t]
    \centering
    \includegraphics[width = 0.49\textwidth]{figs/Fig10_effects_of_alpha.pdf}
    % \includesvg[width=0.49\textwidth, inkscapelatex=false]{figs/Fig10_effects_of_alpha.svg} 
	\caption{Hyperparameter analysis of the directional certainty parameter 
$\alpha$ for CLIC-Half. The right figure visualizes how different values of 
$\alpha$ adjust the desired action space in 3D.}
 \label{fig:Fig10_effects_of_alpha}
\end{figure}


\subsection{Ablation Study}
\label{sec:exp:ablation}
In this section, we analyze the impact of various hyperparameters and loss design choices on the performance of the CLIC method.  Specifically, we focus on the directional certainty parameter $\alpha$, the temperature $T$ used in the sigmoid function of the observation model, and the different assumptions regarding the prior probability $p(\bm{a}|\bm{s})$.
During these experiments, CLIC-Half is utilized.


\subsubsection{Effects of directional certainty $\alpha$ }
The angle $\alpha$ is the main parameter utilized to control the shape of the desired action space for CLIC-Half, as shown in the right part of Fig. \ref{fig:Fig10_effects_of_alpha}.
In this experiment, we carried out experiments in the Square task. Two feedback types were considered, one with accurate feedback and another with direction noise (noise angle $\beta=45^{\circ}$). The results are reported in the left part of Fig. \ref{fig:Fig10_effects_of_alpha}.
For accurate feedback cases, the success rate decreases when $\alpha$ is larger than $120^\circ$. 
 This occurs because increasing $\alpha$ expands the desired action space to include more undesired actions, thereby providing less useful information for updating the EBM.
For direction noise, the success rate decreases for $\alpha < 2 \beta = 90^\circ$. 
This is because for any given feedback with $\alpha < 2 \beta$, the desired action space fails to include the optimal action and misguides the EBM in its update process.
These findings highlight the importance of carefully selecting $\alpha$ to balance the trade-off: maintaining an informative desired action space and ensuring that it includes the optimal action.

\begin{figure}[t]
    \centering
    \includegraphics[width = 0.49\textwidth]{figs/Fig13_exp_ablation_2.pdf}
    % \includesvg[width=0.49\textwidth, inkscapelatex=false]{figs/Fig13_exp_ablation_2.svg}
	\caption{Ablation study: (1) effects of the temperature parameter $T$. (2) Policy-weighted Bayes loss vs uniform Bayes loss.}
 \label{fig:Fig13_exp_ablation_2}
\end{figure}

\begin{figure*}[t]
	\centering
	% \includegraphics[width=\textwidth]{figs/Fig1_Illustration_generalization_across_state_2.png}
    \includegraphics[width=\textwidth]{figs/Fig1_Illustration_generalization_across_state_3.pdf}
    % \includesvg[width=\textwidth, inkscapelatex=false]{figs/Fig1_Illustration_generalization_across_state_3.svg} 
	\caption{ \textbf{Learned EBM landscapes across different trials}. The figure compares the energy landscapes learned by CLIC, PVP, and IBC after training in a 2D action space. Each row corresponds to the resulting EBMs of each trial. 
    In the middle part, we visualize the process of how CLIC-Circular reduces to IBC as $\varepsilon$ increases.
    CLIC-Circular ( with $\varepsilon=0.5$) effectively trains EBM across different trials, leading to consistent minima close to the true optimal action. In contrast, IBC overfits human actions and fails to estimate the true optimal action. Three evaluation metrics are shown in the right part of the figure.}
 \label{Fig1_Illustration_generalization_across_state}
\end{figure*}


\subsubsection{Effects of temperature $T$}
In Section \ref{section:sub:prob_desired_action_space}, the temperature $T$ is utilized to control the sharpness of the probability distribution $\text{Pr} [\bm a \in \mathcal{A} {(\bm a^r, \bm a^h)} | \bm a , \bm s] $.
We study the effects of $T$ in this experiment on the performance of CLIC, where four different values of $T$ are tested in the Square task. 
The results are presented in the left side of Fig. \ref{fig:Fig13_exp_ablation_2}.
When $T$ is very small ($\log_{10} T = -3$), the success rate drops sharply. At this extreme, the observation model becomes binary (0/1), creating a sharp boundary that is difficult for the neural network to learn. Conversely, when $T$ is too large ($T = 1$), the success rate also declines. In this case, the probabilities of actions belonging or not belonging to $\mathcal{A} {(\bm a^r, \bm a^h)}$ become nearly indistinguishable, offering limited information for policy improvement.
$T = 0.1$ proves to be a good balance between these extremes and is selected across all experiments for CLIC-Half. 


% motivation, why are you doing this?
% preset result (pure data)
% give interpretation
\subsubsection{Policy-weighted Bayes loss vs Uniform Bayes loss}
As described in Section \ref{sec:sub:sub:uniform_bayes_loss}, the uniform Bayes loss treats all actions within the desired action space equally, whereas the policy-weighted Bayes loss prioritizes actions closer to the current robot's policy.
To evaluate the effectiveness of the policy-weighted Bayes loss, we compared it against the uniform Bayes loss.  We implement the uniform variant of CLIC and evaluate it across four simulation tasks with accurate demonstration feedback. The results, shown in the right side of Fig. \ref{fig:Fig13_exp_ablation_2}, demonstrate that the uniform Bayes loss leads to significantly poorer performance compared to the policy-weighted Bayes loss.
This highlights the importance of incremental policy updates. Since the desired action space may include some undesired actions, staying close to the current policy helps avoid imitating unintended behaviors, resulting in a more stable training process.

\subsection{Toy Experiments on Noisy Feedback}
\label{sec:exp:toy_exp}




In this section, we present an example to illustrate the improved performance of CLIC over IBC. The toy task in this example consists of a single constant state with a 2D action space, where the optimal action is set to $\bm{0}$ (see Fig. \ref{Fig1_Illustration_generalization_across_state}). 
The objective is to estimate the optimal action through multiple corrective feedback.
We carried out experiments over 10 trials. In each trial, we generated a randomly sampled dataset consisting of 6 or 7 data points $(\bm s, \bm a^r, \bm a^h)$, where human actions were drawn from a Gaussian distribution centered at the optimal action. 
Each method was trained for 1,000 steps in an offline IL setting, and we visualized the trained EBMs for each method for the first two trials in Fig. \ref{Fig1_Illustration_generalization_across_state}.

To evaluate the methods, we introduced three metrics: (1) the mean square error (MSE) to optimal action: this measures MSE between each local minimum action of the EBM and the optimal action. (2) MSE to human action: this calculates the average MSE between each local minimum action of the EBM and its nearest human action. A smaller value indicates that the EBM is overfitting to the human action. (3) Variance across trials: this evaluates the variance of the EBM values over the entire action space across ten different trials.
These metrics are computed by averaging the results over the 10 trials and are reported in the right side of Fig. \ref{Fig1_Illustration_generalization_across_state}.



\textbf{CLIC learns consistent EBM landscapes across different trials}
% \cite{2017_NIPS_understanding_noise_generalization}
 The PVP-trained EBM tends to be over-optimistic about favorable actions by outputting low energy values for a large region of actions that are not present in the dataset, as shown in the left side of Fig. \ref{Fig1_Illustration_generalization_across_state}.
 This occurs because PVP calculates its loss using only observed action pairs, leaving the energy values of other actions in the action space uncontrolled. 
  On the other hand, IBC's loss function encourages human actions and discourages all other actions, even actions that are very similar to human actions. Consequently, the IBC-trained EBM overfits the data, learning minima corresponding to individual human actions. 
  This overfitting results in EBM landscapes with high variance across different trials, as shown in the bottom-right figure of Fig. \ref{Fig1_Illustration_generalization_across_state}.
 In contrast, the CLIC-trained EBM maintains consistent landscapes, with minima close to the true optimal action and low variance across different trials.
 This explains the superior performance of  CLIC over IBC and PVP, as observed in Section \ref{sec:exp:accurate_feedback}.
 


\textbf{CLIC-Circular reduces to IBC under stricter assumptions}
To illustrate how CLIC-Circular reduces to IBC, we progressively decrease the radius of the circular desired action spaces by increasing the value of $\varepsilon$ (see the middle part of Fig. \ref{Fig1_Illustration_generalization_across_state}). 
As $\varepsilon$ increases, the CLIC-trained EBM starts to split into several clusters and overfit data labels with smaller MSE to human actions, as reported in the right part of Fig. \ref{Fig1_Illustration_generalization_across_state}.
 This overfitting also leads to a larger MSE to the optimal action, indicating that the EBM becomes less effective at identifying the optimal action.
Eventually, as $\varepsilon \rightarrow 1$,  the EBM landscape closely resembles the one trained using IBC.
When the radius becomes nearly zero, imitating a circular desired action space reduces to imitating a human action label, leading to overfitting and performance drops. 
This observation highlights the key distinction between CLIC-Circular and IBC: imitating a circular desired action space rather than a single action. This distinction is crucial for training EBMs stably.
% We demonstrate the transition from CLIC-Circular to IBC by progressively adjusting its hyperparameters and analyzing their impact on the energy-based model.
% (1) For CLIC-Circular with $\varepsilon = 0.5$, by changing the policy-weighted Bayes loss to uniform Bayes loss, the learned EBM tends to imitate the whole desired action space and leads to the EBM landscape quite flat and over-optimistic.





\begin{figure*}[t]
    \centering
    \includegraphics[width = 1.0\textwidth]{figs/Fig7_InsertT_exp_results_traj_3.pdf}
    % \includesvg[width=1.0\textwidth, inkscapelatex=false]{figs/Fig7_InsertT_exp_results_traj_2.svg}
	\caption{Examples of CLIC-Half policy rollout for the Insert-T task after training. At each step, the transparent figure shows the initial state, and the orange arrow indicates the end-effector’s trajectory. The solid figure illustrates the resulting end state, which becomes the initial state for the next step.}
 \label{fig:Fig7_InsertT_exp_results_traj}
\end{figure*}

\begin{figure}[t]
    \centering
    \includegraphics[width = 0.49\textwidth]{figs/Fig7_InsertT_exp_results.pdf}
    % \includesvg[width=0.49\textwidth, inkscapelatex=false]{figs/Fig7_InsertT_exp_results.svg}
	\caption{Experiment results for the Insert-T task, categorized by difficulty levels (easy, medium, and hard). Each column shows the performance metrics for a given difficulty level, along with examples of initial states for that level.
    ``CLIC-Half (offline)'' denotes results for CLIC-Half trained offline.}
 \label{fig:Fig7_InsertT_exp_results}
\end{figure}

\subsection{Real-robot Validations}
\label{sec:exp:real_rotbo}

% \textcolor{red}{explain why we use CLIC-simplified }

Here, we use three tasks to demonstrate the practical applicability of CLIC.
The experiments include a long-horizon multi-modal Insert-T task, a dynamic ball-catching task, and a water-pouring task that necessitates precise control of the robot's end effector position and orientation. 
For the Insert-T task, we employ CLIC-Half and compare its performance against IBC and Diffusion Policy.
For the ball-catching and water-pouring tasks, we use CLIC-Explicit because it performs well in uni-modal tasks, as demonstrated in Section \ref{sec:exp:simulation}, and is more time-efficient compared to CLIC with an EBM policy (Details in Appendix \ref{appendix:time_efficiency_comparision}.).
% \footnote{Details on the time efficiency comparison between CLIC-Explicit and CLIC with an EBM policy are provided in the Appendix \ref{appendix:time_efficiency_comparision}.}
The experiments were carried out using a 7-DoF KUKA iiwa manipulator. When required, an underactuated robotic hand (1-dimensional action space) was attached to its end effector. A 6D space mouse was employed to provide feedback on the pose of the robot's end effector. Furthermore, in the ball-catching task, a keyboard provided feedback on the gripper's actuation. 
The setup of each task is detailed in Appendix \ref{appendix:real_robot_experiments_task details}, and the time durations used are reported in Appendix \ref{appendix:time_duration}. 
% The learned policy is evaluated every 5 episodes for the water-pouring task, every 10 episodes for the ball-catching task, and every 20 episodes for the Insert-T task.
% Results in Fig. \ref{fig:real_exp_figs_combined_all} show that the success rate for all tasks
% exhibits an overall improving trend as the timestep increases, demonstrating the practical applicability of CLIC. 
The experiment results are reported as follows:



\subsubsection{Insert-T—a comparison between state-of-the-art methods}
The Insert-T task requires the robot to insert a T-shaped object into a U-shaped object by pushing to adjust their positions and orientations.
Compared to the Push-T task in simulation experiments, Insert-T is more complex due to two factors: (1) it involves two objects, introducing multi-modal decisions about which object to manipulate first; and (2) it has an increased task horizon. This makes Insert-T a valuable benchmark for evaluating CLIC's performance in long-horizon, multi-modal tasks compared to state-of-the-art methods. 
To better analyze the performance of each method, we categorize the task into three difficulty levels based on the initial state of the objects and the number of contact changes required to complete the task (according to the teacher’s policy).  Examples of these categories are shown in the upper part of Fig. \ref{fig:Fig7_InsertT_exp_results}.
Tasks requiring fewer than 1 contact change are classified as ``easy”, fewer than 5 as ``medium”, and 5 or more as ``hard”. 
During each evaluation, 10 different initial states are tested for each category. To ensure a fair comparison, all methods are evaluated using the same set of initial states.
In the experiment, human-provided demonstration feedback is used to train CLIC-Half within the IIL framework. The collected data is also used to train baseline methods (Diffusion and IBC) offline. For a fair comparison, CLIC is additionally trained offline on the same dataset as the baselines.

Results for different difficulty levels are shown in Fig. \ref{fig:Fig7_InsertT_exp_results}. For easy tasks, baseline methods perform similarly to CLIC but converge more slowly. For medium and hard tasks, CLIC achieves significantly higher success rates. This is particularly evident for hard tasks, where CLIC achieves 80$\%$ success compared to 30$\%$ for Diffusion and 10$\%$ for IBC. 
The results demonstrate CLIC's ability to handle complex multi-modal tasks, thanks to the powerful encoding capabilities of EBMs and CLIC's stable EBM training. 
Furthermore, as the task difficulty increases, CLIC outperforms Diffusion and IBC by a large margin. This suggests that, for training policies, using a desired action space is more robust and efficient in real-robot tasks than relying on a single action label.
 Examples of post-training policy rollouts for CLIC in hard tasks are shown in Fig. \ref{fig:Fig7_InsertT_exp_results_traj}.

Note that CLIC is an interactive learning method, whereas Diffusion and IBC are offline methods in this experiment. Figure \ref{fig:Fig7_InsertT_exp_results} also includes results for CLIC trained with offline data, showing a similar final success rate to the online version. This indicates that CLIC can be employed to learn from offline data as well.
While CLIC is primarily based on the IIL framework, the core ideas proposed here could also benefit offline methods. We believe exploring offline training of CLIC is a promising direction for future work.



\subsubsection{Ball catching—quick coordination and partial feedback}
 The ball-catching task is challenging because of its highly dynamic nature. 
This complexity makes it difficult to provide successful demonstrations of the task to the robot, thus ruling out demonstration-based IIL methods for solving it\footnote{This limitation could be overcome with a highly reactive and precise teleoperation device. However, this also makes the solution more expensive.}.
Instead, relative corrective feedback is more intuitive and easier for this task as humans can provide direction signals occasionally to improve the robot's policy \cite{2020_DCOACH_temporal}.  
Besides, for a successful grasp, the robot must coordinate precisely the ball's motion, the end-effector's motion, along with the gripper's actuation.
This requirement makes it challenging to provide feedback on the complete action space at any given moment, and makes partial feedback suitable for this task.
With partial feedback, relative corrections can be independently provided for either the end-effector's motion or the gripper's actuation.
Therefore, this task enables testing CLIC's ability to learn effective policies from relative corrective feedback that is also partial. 
% The robot is also expected to react after an unsuccessful attempt by trying to grasp the ball again. 
% As a result, this task is particularly interesting for evaluating CLIC's performance, as it allows for the analysis of its ability to quickly coordinate multiple variables in a problem.
% Additionally, the necessary coordination between end-effector motion and gripper actuation makes it challenging to provide feedback on the complete action space at any given moment. 
% Therefore, this enables testing CLIC's ability to learn effective policies from partial feedback in real-world problems, where feedback is independently provided for either the end-effector's motion or the gripper's actuation.

% \textcolor{red}{add results analysis.}
 \begin{figure}
    \centering
    \includegraphics[width = 0.46\textwidth]{figs/Fig8_CatchBall02.pdf}
    % \includesvg[width=0.49\textwidth, inkscapelatex=false]{figs/Fig8_CatchBall02.svg}
	\caption{Experiment results for the ball-catching task.}
 \label{fig:Fig8_CatchBall}
\end{figure}

\begin{figure}
    \centering
    \includegraphics[width = 0.45\textwidth]{figs/Fig9_waterpouring.pdf}
    % \includesvg[width=0.49\textwidth, inkscapelatex=false]{figs/Fig9_waterpouring.svg}
	\caption{Experiment results for the water-pouring task.}
 \label{fig:Fig9_waterpouring}
\end{figure}

Fig. \ref{fig:Fig8_CatchBall} shows the experiment results of the ball-catching task, reporting the success rate of catching the ball within one, two, and three attempts. 
By the end of training, the robot achieves a 1.0 success rate for catching the ball within two attempts, and its first-attempt success rate continues to improve to 0.4.
One post-training policy rollout of a successful first-attempt catch is shown in Fig. \ref{fig:Fig8_CatchBall}, where the ball is caught within 1.5 seconds, an impressive result given the actuation delay of the robot hand.
This experiment demonstrates that CLIC can leverage both the relative corrective feedback and partial feedback effectively to learn challenging high-frequency tasks. 




% \begin{figure}
%     \centering
%     \includegraphics[width = 0.49\textwidth]{figs/Fig7_InsertT_exp_results_traj.png}
% 	\caption{Example of one CLIC-implicit policy rollout for the InsertT task. At each step, The transparent figure represents the initial state, the green arrow visualizes the trajectory of the end-effector, and the solid figure shows the corresponding end state, which becomes the initial state for the next step.}
%  \label{fig:Fig7_InsertT_exp_results_traj}
% \end{figure}



\subsubsection{Water pouring—learning full pose control with CLIC}
% In this experiment, the CLIC-Absolute and CLIC-Relative methods are combined to teach the robot, allowing the human teacher to select (by pressing a button) which teaching mode (absolute or relative correction) to use. 
% This approach showcases the flexibility of CLIC, which can learn from relative or absolute corrections depending on which is more appropriate at each moment. 
The water-pouring task requires the robot to control the pose of a bottle to precisely pour liquid (represented with marbles) into a small bowl. 
CLIC-Explicit is utilized in this experiment. 
The human teacher can provide either absolute or relative corrective feedback and has the flexibility to switch between these modes by pressing a specific keyboard button. 
Initially, absolute feedback was preferred as the policy was learned from scratch, and it was easier to intervene in a 6D action space. As the policy improved, relative corrections made it easier to refine the policy in specific regions of the state space.
% Furthermore, since CLIC allows partial feedback, the corrective signal could be provided in three ways: (1) position only, (2) rotation only, and (3) both position and rotation. This is useful when one part of the robot's action is correct while the other still needs improvement.
% An example
% of the learned policy and the results are shown in Fig. \ref{fig:Fig9_waterpouring}.

The experimental data is shown in Fig. \ref{fig:Fig9_waterpouring}. From Episode $1$ to Episode $16$, the teacher's feedback is provided in an absolute correction format. From Episode $16$ onward, the teacher's feedback is given as relative corrections to make small adjustments to the robot's policy. The success rate exhibits an overall improving trend, consistently increasing from 0.6 in Episode 26 to 0.9 by Episode 41. An example of the policy rollouts after training is illustrated in Fig. \ref{fig:Fig9_waterpouring}.
This experiment demonstrates the effectiveness of CLIC for learning precise control over position and orientation.














% \begin{figure*}[t]
%  \captionsetup{skip=0.2pt} % Adjust skip parameter for this figure
% 	\centering
% 	\includegraphics[width = 0.99\textwidth]{figs/real_exp_figs_combined_all.jpg}
%  % \vspace{2pt}
% 	\caption{Experiment results of the three real-world tasks, with success rate evolution in time and timesteps.
%  In the ball-catching task, success rates are recorded over different numbers of attempts.
%  }	\label{fig:real_exp_figs_combined_all}
% \end{figure*}

 





\section{Limitations}

% The visualization effect is not perfect in XR headsets, since we simply used RGB texture for the materials,
% making the robots from different simulators look the same.
% There should be some rendering configuration in the simulator but we only use RGB texture from them.
% Secondly, IRIS's performance is restricted by the number of simulated objects.
% When the quality of simulated objects in the simulation exceeds some limitation. the performance will be significantly reduced.

% Although the IRIS have 
Based on our development and usage experience, IRIS has three main limitations.
First, the visualization in XR headsets is limited to basic RGB textures for materials, making robots from different simulators appear visually identical.
Second, IRIS currently only supports rigid objects, at least deformable object have not yet been fully explored.
Finally, IRIS has only been tested on Meta Quest 3 and HoloLens 2, and further evaluation on additional devices would be beneficial.

\section{Conclusion}

% \textcolor{red}{TODO}

% In this work, we introduced IRIS, an innovative Immersive Robot Interaction System that bridges the gap between Extended Reality (XR) technologies and robotics data collection. By addressing the challenges of reproducibility and reusability prevalent in current XR-based systems, IRIS offers a flexible and extendable framework that can be used in multiple simulators, benchmarks, real-world scenarios, and multiple-user use case.
% The user study show that IRIS outperforms previous benchmark data collection method, making it as a potential option for future data collection pipeline.
% The IRIS codebase is open-source, and it will facilitate further research and make IRIS adapted to more use case and more hardware platform.

In this work, we introduced IRIS (Immersive Robot Interaction System), an innovative framework that seamlessly integrates Extended Reality (XR) technologies with robotics data collection. IRIS addresses key challenges in reproducibility and reusability that are common in current XR-based systems. Its flexible, extendable design supports multiple simulators, benchmarks, real-world applications, and multi-user use cases.
User studies demonstrate that IRIS outperforms previous data collection methods, positioning it as a promising solution for future data collection pipelines.
As an open-source project, IRIS codebase promotes further research and adaptation across diverse use cases and hardware platforms.



\bibliographystyle{plainnat}
\bibliography{main}

\clearpage

\appendix
\section{Appendix Title}
\section{Technical Details}

\subsection{Mujoco}
\label{app:mujoco}

MuJoCo \cite{todorov2012mujoco} is a fast, accurate physics engine ideal for simulating robots with complex joint structures and contact dynamics.
It supports advanced robot simulations with features like customizable actuators, collision detection, and friction modeling, enabling realistic testing of robotic control, manipulation, and learning algorithms in dynamic environments.

When starting a MuJoCo simulation, MuJoCo provides a model instance and a data instance.
The scene specification of IRIS can be generated from the model, while the simulation states can be retrieved from the data instance.
The model instance contains all the necessary assets for the scene specification, including meshes, textures, and materials. 
These assets are stored as NumPy arrays and need to be converted to byte streams with data types such as \textit{float32} or \textit{int8} for transmission.

MuJoCo uses a standard robotic coordinate system, where X is forward, Y is left, and Z is up. This differs from Unity's coordinate system, where X is right, Y is up, and Z is forward.
Therefore, object transforms, mesh vertices, faces, and normal data must be adapted for Unity using the following code:

\begin{lstlisting}[language=Python]
def mj2unity_pos(pos: List[float]) -> List[float]:
    return [-pos[1], pos[2], pos[0]]

def mj2unity_quat(quat: List[float]) -> List[float]:
    return [quat[2], -quat[3], -quat[1], quat[0]]
\end{lstlisting}

Since MuJoCo includes visual group settings that specify which object groups can be visualized, IRIS provides an API to support this feature through the \textit{MujocoPublisher} definition, as shown below:

\begin{lstlisting}[language=Python]
class MujocoPublisher(ScenePublisher):

    def __init__(
        self,
        mj_model,
        mj_data,
        host: str = "127.0.0.1",
        no_rendered_objects: Optional[List[str]] = None,
        no_tracked_objects: Optional[List[str]] = None,
        visible_geoms_groups: Optional[List[int]] = None,
    ) -> None:
\end{lstlisting}

\subsection{IsaacSim}

IsaacSim~\cite{nvidia_isaac_sim} is a robotics simulation environment based on the NVIDIA Omniverse platform~\cite{nvidia_omniverse}.
Several frameworks \cite{mittal2023orbit,gong2023arnold} are built on top of it, sharing the same underlying data structures.
IsaacSim supports ray-tracing for realistic rendering, and batched physics simulation on GPUs, significantly accelerating the training of models.

The scenes in IsaacSim are organized in a Universal Scene Description (USD) format, and data can be accessed directly with the OpenUSD API \cite{pixarUSD}.
The scene hierarchy consists of transformations (Xform), meshes, articulations (joints) among other elements. Figure~\ref{fig:isaacsim-tree} illustrates a sample scene hierarchy.

\begin{figure}[h]
    \centering
    \includegraphics[width=0.7\linewidth]{image/isaacsim/isaacsim_scene_structure.png}
    \caption{An example of USD scene hierarchy for Franka Panda robot arm in IsaacSim.}
    \label{fig:isaacsim-tree}
\end{figure}

Besides the low-level OpenUSD API, IsaacSim also provides some utility APIs for accessing scene data. In our system, we use a mix of both APIs. A (simplified) code snippet for accessing parsing the tree structure of the USD scene is given below. We use the OpenUSD API \texttt{GetChildren()} to retrieve child primitives here.

\begin{lstlisting}[language=Python]
# compute local transforms of the current prim
trans, rot, scale = self.compute_local_trans(root)

# parse material and geometry
mat_info = self.parse_prim_material(prim=root)
self.parse_prim_geometries(
    prim=root,
    prim_path=prim_path,
    sim_obj=sim_object,
    mat_info=mat_info or inherited_material,
)

# parse children of the current prim
for child in root.GetChildren():
    if obj := self.parse_prim_tree(
        root=child,
        parent_path=prim_path,
        inherited_material=mat_info or inherited_material,
    ):
        sim_object.children.append(obj)
\end{lstlisting}

Here is another snippet for computing the world transform of a primitive in the hierarchy. Instead of computing the world transform with OpenUSD API, the class in IsaacSim API \texttt{XFormPrim} is used to simplify the code.

\begin{lstlisting}[language=Python]
from pxr import Usd
from omni.isaac.core.prims import XFormPrim

def compute_world_trans(self, prim: Usd.Prim):
    prim = XFormPrim(str(prim.GetPath()))
    assert prim.is_valid()

    pos, quat = prim.get_world_pose()
    scale = prim.get_world_scale()

    return (
        pos.cpu().numpy(),
        quat_to_rot_matrix(quat.cpu().numpy()),
        scale.cpu().numpy(),
    )
\end{lstlisting}

\subsection{CoppeliaSim}
CoppeliaSim, formerly known as V-REP, is a versatile and widely used robot simulation software that supports various physics engine backbones. Scenes in CoppeliaSim are constructed using a tree structure, which includes objects such as visual shapes, dynamic shapes, and joints, as illustrated in Figure \ref{fig:coppliasim-tree}.

\begin{figure}[h]
    \centering
    \includegraphics[width=0.7\linewidth]{image/coppeliasim/CoppeliaSim_Tree.png}
    \caption{An example of a tree structure for building an ABB IRB4600 manipulator.}
    \label{fig:coppliasim-tree}
\end{figure}

In our system, we interact with CoppeliaSim using Python APIs. Depending on the version of CoppeliaSim, we utilize two groups of Python API functions:

\textbf{ZeroMQ Remote API} \footnote{https://manual.coppeliarobotics.com/en/apiFunctions.htm}. This is an official API introduced in version 4.4. It facilitates fast and straightforward communication between the simulator and a Python script, enabling users to efficiently build scenarios. For example, to retrieve mesh information for all shape objects in a CoppeliaSim scene, one can use the following call:
\begin{lstlisting}[language=Python, caption=Example ZeroMQ Remote API.]
from coppeliasim_zmqremoteapi_client import RemoteAPIClient
client = RemoteAPIClient()
sim = client.require('sim')

list_vertices, list_indices, list_normals = [], [], []

objects_id_list = sim.getObjectsInTree(sim.handle_scene, 
                                       sim.handle_all, 0)

for idx in objects_id_list:
    if sim.getObjectType(idx) == sim.sceneobject_shape:
        # We assume all the shapes are primary shapes
        vertices, indices, normals = sim.getShapeMesh(idx)
        
        list_vertices.append(vertices)
        list_indices.append(indices)
        list_list_normals.append(normals)
\end{lstlisting}


\textbf{PyRep API}\cite{james2019pyrep} \footnote{https://github.com/stepjam/PyRep}. This is a widely-used third-party Python communication interface under the CoppeliaSim version 4.1. Several notable CoppeliaSim projects and benchmarks, such as RLBench \cite{james2020rlbench}, are built on this interface. Similar to the official interface, PyRep provides various API functions and properties. For instance, to retrieve all object handles from a CoppeliaSim scene, one can use the following call:
\begin{lstlisting}[language=Python, caption=Example PyRep API.]
from pyrep.backend.sim import simGetObjectsInTree
from pyrep.backend.sim import simGetObjectType
from pyrep.backend.sim import simGetShapeMesh
from pyrep.backend.simConst import sim_handle_scene
from pyrep.backend.simConst import sim_handle_all
from pyrep.backend.simConst import sim_object_shape_type

list_vertices, list_indices, list_normals = [], [], []

objects_id_list = simGetObjectsInTree(sim_handle_scene, 
                                      sim_handle_all, 0)

for idx in objects_id_list:
    if simGetObjectType(idx) == sim_object_shape_type:
        # We assume all the shapes are primary shapes
        vertices, indices, normals = simGetShapeMesh(idx)
        
        list_vertices.append(vertices)
        list_indices.append(indices)
        list_list_normals.append(normals)

\end{lstlisting}



\subsection{Genesis}

Genesis \cite{Genesis} is a versatile physics platform designed for robotics, embodied AI, and physical AI applications. It combines a universal, re-engineered physics engine capable of simulating diverse materials and phenomena with a lightweight, ultra-fast, and user-friendly robotics simulation environment. It also features a powerful, photorealistic rendering system and a generative data engine that transforms natural language prompts into multi-modal data. By integrating various physics solvers within a unified framework, Genesis supports automated data generation through a generative agent framework, with its physics engine and simulation platform now open-source and further expansions planned.

IRIS supports Genesis by providing a Genesis parser designed to translate simulation data from the Genesis physics platform into the IRIS framework's internal representation.
It manages this by reading rigid entities, links, and geometries from a \textit{gs.Scene} object and converting them into equivalent IRIS structures such as \textit{SimScene}, \textit{SimObject}, \textit{SimVisual}, and \textit{SimMaterial}. Here's a detailed breakdown of the parsing process and its components:

The main function initializes a \textit{SimScene} and iterates through all entities in the Genesis scene.
It builds a hierarchical scene graph based on parent-child relationships between entities and links. The graph is stored in the hierarchy dictionary, which tracks each object's parent and child nodes.

Each rigid entity (\textit{RigidEntity}) from Genesis is processed to create a \textit{SimObject}. The position and rotation of the entity are retrieved and transformed is similart to Mujoco (\ref{app:mujoco}) to ensure compatibility with Unity's coordinate system.
If the Genesis scene is already built, the function pulls actual position and rotation data; otherwise, it defaults to identity transforms.

Links (RigidLink) represent individual parts of a rigid entity's structure. Each link is converted into a \textit{SimObject} and positioned within the scene based on its parent link's position and orientation.
If the link has associated visual geometries (\textit{vgeoms}), the parser processes each visual element to create \textit{SimVisual} objects, which store geometry and material data. Links without visual elements are simply added to the scene hierarchy without rendering.

Visual geometries (\textit{RigidGeom}) from Genesis are mapped to IRIS visual representations. The parser constructs a \textit{SimVisual} object, including position, rotation, and mesh data.
The mesh generation function extracts vertex and face data from the Genesis mesh and constructs a \textit{SimMesh} object for the IRIS scene, defining both the physical and visual structure of the geometry. Genesis leverages the \textit{Trimesh} \cite{trimeshTrimeshTrimesh} library to handle mesh processing, which simplifies the conversion to the IRIS mesh format.

After processing all entities, the parser organizes the parsed objects into a tree structure. Objects without parents are attached to the root of the scene, while others are linked to their respective parents based on the hierarchy dictionary.




\subsection{Real World}


The point cloud-based XR teleoperation system enables immersive teleoperation and data collection by allowing users to interact with a real-world robotic setup through an Extended Reality (XR) interface. Unlike conventional video-streaming methods, this approach provides spatial awareness and depth perception by rendering real-time 3D reconstructions of the environment in XR. The system consists of an ORBBEC Femto Bolt depth camera positioned in front of the robot to capture RGB and depth data in real time. A QR code on the robot’s end effector serves as a reference marker for pose estimation, aligning point clouds with the robot’s frame. A C++-based point cloud processing pipeline converts raw data into XYZ-RGB point clouds at 30Hz, generating approximately 2 million points per frame. To optimize computational efficiency, GPU-accelerated voxel grid downsampling is applied before transmission. The processed point clouds are sent to a main process that handles transformation, cropping, and communication with the XR system. A ZeroMQ (ZMQ) messaging protocol is used to enable efficient, low-latency communication between system components, ensuring real-time transmission of point clouds and control commands. A Meta Quest 3 headset renders the processed point clouds in real time, enabling users to visualize and interact with the scene in a fully immersive 3D environment.

\subsubsection{Camera-to-Robot Base Transformation}

\paragraph{Pose Estimation and Homogeneous Transformation}
To ensure accurate spatial alignment, the system transforms the camera’s coordinate frame into the robot’s base coordinate frame using a homogeneous transformation matrix, computed as follows:

\begin{enumerate}
    \item The QR code on the robot’s end effector provides a fixed reference point for tracking position and orientation in the camera's space.
    \item Using Forward Kinematics (FK), the end effector’s pose relative to the robot base is determined.
    \item The camera-to-robot base transformation is derived as:
    \begin{equation}
        T_{\mathrm{Robot Base}}^{\mathrm{Camera}} = T_{\mathrm{Robot Base}}^{\mathrm{End Effector}} \cdot T_{\mathrm{End Effector}}^{\mathrm{Camera}}
    \end{equation}
    where:
    \begin{itemize}
        \item $T_{\mathrm{Robot Base}}^{\mathrm{End Effector}}$ is obtained from the robot’s FK.
        \item $T_{\mathrm{End Effector}}^{\mathrm{Camera}}$ is estimated using the QR code tracking system.
    \end{itemize}
\end{enumerate}

This ensures that the captured point clouds align precisely with the robot’s coordinate frame, eliminating drift and inconsistencies in XR visualization.

\subsubsection{Point Cloud Processing Pipeline}
\label{appendix:Point Cloud Processing Pipeline}
\paragraph{Raw Data Acquisition}
The ORBBEC Femto Bolt camera captures:
\begin{itemize}
    \item Depth images (encoded as a depth map),
    \item RGB images (color information),
    \item Camera intrinsic parameters (for depth-to-3D conversion).
\end{itemize}

\paragraph{Conversion to XYZ-RGB Point Cloud}
Using the camera intrinsics, each depth pixel is converted into 3D world coordinates $(X, Y, Z)$ using:
\begin{equation}
    X = (u - c_x) \frac{Z}{f_x}, \quad Y = (v - c_y) \frac{Z}{f_y}, \quad Z = D(u,v)
\end{equation}
where:
\begin{itemize}
    \item $(u, v)$ are pixel coordinates,
    \item $(c_x, c_y)$ are the camera's principal point offsets,
    \item $(f_x, f_y)$ are focal lengths,
    \item $D(u,v)$ is the depth value at pixel $(u, v)$.
\end{itemize}
Each $(X, Y, Z)$ point is assigned an $(R, G, B)$ value from the color image, forming the XYZ-RGB point cloud.

\paragraph{Cropping and Filtering}
To reduce noise and retain only relevant portions of the workspace, the system:
\begin{itemize}
    \item Crops unnecessary regions using bounding box constraints,
    \item Applies statistical outlier removal to eliminate noise points.
\end{itemize}

\paragraph{Voxel Grid Downsampling (GPU-Accelerated)}
Since the raw point cloud consists of approximately 2 million points per frame, direct transmission is computationally expensive. To optimize efficiency, voxel grid downsampling is applied, which:
\begin{itemize}
    \item Divides the workspace into 3D voxels,
    \item Averages all points within each voxel to generate a single representative point,
    \item Reduces the point cloud size while preserving structural details.
\end{itemize}

\subsubsection{XR Visualization and Teleoperation}

\paragraph{Real-Time Communication with XR System}
The processed point cloud is transmitted to a main process, which then relays the data to the Meta Quest 3 headset using ZeroMQ. ZeroMQ ensures efficient, asynchronous, and low-latency messaging between the processing node and the XR system, allowing:  
\begin{itemize}  
    \item Real-time rendering of the scene in XR,  
    \item Full 3D immersion with accurate depth perception,  
    \item Reliable transmission of point cloud data and control commands.  
\end{itemize}  

\paragraph{Leader-Follower Teleoperation Framework}
The teleoperation system consists of:
\begin{itemize}
    \item A leader robot controlled by a human wearing the Meta Quest 3 headset,
    \item A follower robot that replicates the leader’s movements, including gripper actions.
\end{itemize}

A networked control architecture ensures:
\begin{itemize}
    \item Low-latency synchronization between the leader and follower,
    \item Seamless remote manipulation, removing the need for physical human presence.
\end{itemize}

\subsubsection{Calibration for XR-Robot Alignment}
To ensure that XR visualization aligns precisely with real-world objects, the system undergoes a calibration process consisting of:
\begin{enumerate}
    \item Robot workspace alignment: The follower robot’s workspace is adjusted to match the leader’s XR-rendered environment,
    \item Point cloud adjustment: The camera-to-robot transformation is fine-tuned,
    \item User feedback correction: Users verify object positions in XR and real-world views.
\end{enumerate}

\subsubsection{Future Extensions: Multi-Camera and ICP-Based Fusion}
Future implementations will:
\begin{itemize}
    \item Incorporate multiple cameras for wider coverage,
    \item Perform point cloud fusion using Iterative Closest Point (ICP) for improved spatial accuracy,
    \item Enhance depth perception by reconstructing high-fidelity 3D representations of the workspace.
\end{itemize}



\subsection{Affiliated Monitoring Tools}
The dashboard depicted in \ref{fig:webapp} is designed to complement the IRIS Framework by providing streamlined access to the functionality offered by XR devices. Additionally, it facilitates the rendering of the scene to a web-based canvas, enabling real-time viewing of the scene as it is being streamed.

Due to the limitations of web services, which cannot directly interface with underlying protocols such as UDP and ZMQ, this website is unable to directly connect to the IRIS server. To bridge this gap, a Python-based endpoint is deployed, which hosts the website, relays the rendering data to the web application, and forwards user commands from the interface back to the server.

Communication between the endpoint and the website is facilitated through a WebSocket connection. This setup allows the system to support multiple instances of the dashboard simultaneously, ensuring scalability. The scene rendering itself is accomplished using Three.js, a powerful JavaScript library that renders to an HTML5 canvas object through WebGL. The Python endpoint is responsible for converting the IRIS scene format into commands, which are then transmitted over the WebSocket connection. These commands are subsequently translated into high-level Three.js function calls to render the scene.

This high-performance rendering system makes it possible to display even highly complex scenes on lower-end machines. Notably, only the server hosting the web application needs to install the necessary software, enabling all connected devices to access the interface without requiring any additional installations.

To minimize latency between the initial connection and the start of rendering, primitive objects are placed in the scene immediately upon receipt by the endpoint. Larger resources, such as meshes, materials, and textures, are streamed into the scene asynchronously to prevent delays or interruptions in the rendering process.

 
\subsection{Interaction Data}
\label{app:interaction_data}

Here is an example of using \textit{ScenePub API} to retrieve the data from Meta Quest 3 and HoloLen 2.

\begin{lstlisting}[language=Python]
# import device from scenepub
from scenepub.xr_device.meta_quest3 import MetaQuest3
from scenepub.xr_device.hololens2 import HoloLens2
# do some thing about simulation
mq3_player = MetaQuest3(mq3_device_name: str)
hl2_player = HoloLens2(hl2_device_name: str)
# running simulation
while True:
    # running simulation step logic
    mc_data = mq3_player.get_mc_data()
    hand_data = mq3_player.get_hand_data()
    gaze_data = mq3_player.get_gaze_data()
    # processing input data from Meta Quest3
    hl2_hand_data = hl2_player.get_hand_data()
    hl2_gaze_data = hl2_player.get_gaze_data()
\end{lstlisting}




\end{document}


