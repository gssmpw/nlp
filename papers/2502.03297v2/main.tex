\documentclass[conference]{IEEEtran}
\usepackage{times}

% numbers option provides compact numerical references in the text. 
\usepackage[numbers]{natbib}
\usepackage{multicol}
\usepackage[bookmarks=true]{hyperref}

% set monospace font for code
\usepackage{sourcecodepro}
\usepackage[T1]{fontenc}

% other packages
\usepackage{placeins}
\usepackage{pifont} % For \ding
\usepackage{xcolor} % For coloring
\definecolor{darkgreen}{RGB}{0, 125, 0} % Darker green color
\newcommand{\cmark}{\textcolor{darkgreen}{\ding{51}}} % Green check mark
\newcommand{\xmark}{\textcolor{red}{\ding{55}}}   % Red cross mark
\usepackage{graphicx}
% \usepackage{todonotes}
\newcommand{\redtodo}[1]{\todo[inline, color=white, bordercolor=white]{\textcolor{red}{#1}}}
\usepackage{booktabs}
\usepackage{tabularx} 
\usepackage{adjustbox} 
\usepackage{listings}
\usepackage{makecell}
% Define custom colors for the code
\definecolor{codegreen}{rgb}{0,0.6,0}
\definecolor{codegray}{rgb}{0.5,0.5,0.5}
\definecolor{codepurple}{rgb}{0.58,0,0.82}
\definecolor{backcolour}{rgb}{0.95,0.95,0.92}
\definecolor{codeblue}{rgb}{0.21,0.47,0.56}
\usepackage{tikz}
\usepackage{subcaption}
\usepackage{caption}
\usepackage{float} 

% Customize the style for the code
\lstdefinestyle{mystyle}{
    backgroundcolor=\color{backcolour},   % Background color
    commentstyle=\color{codegreen},      % Comments
    keywordstyle=\color{codeblue},           % Keywords
    numberstyle=\tiny\color{codegray},   % Line numbers
    stringstyle=\color{codepurple},      % Strings
    basicstyle=\scriptsize\ttfamily,                % Basic font style
    % basicstyle=\ttfamily,  
    breakatwhitespace=false,             % Automatic breaks at whitespaces
    breaklines=true,                     % Automatic line breaking
    captionpos=b,                        % Caption position
    keepspaces=true,                     % Keep spaces
    numbers=left,                        % Line numbers on the left
    numbersep=5pt,                       % Space between numbers and code
    showspaces=false,                    % Do not show spaces
    showstringspaces=false,              % Do not show string spaces
    showtabs=false,                      % Do not show tabs
    tabsize=2                            % Tab size
}

\lstset{style=mystyle}


\pdfinfo{
   /Author (Homer Simpson)
   /Title  (Robots: Our new overlords)
   /CreationDate (D:20101201120000)
   /Subject (Robots)
   /Keywords (Robots;Overlords)
}


\lstset{escapeinside={(*@}{@*)}}

\begin{document}

% paper title
\title{IRIS: An Immersive Robot Interaction System}

% You will get a Paper-ID when submitting a pdf file to the conference system
% \author{Author Names Omitted for Anonymous Review. Paper-ID [179]}
\author{\small\textbf{Xinkai Jiang}$^{*,1}$, ~~\textbf{Qihao Yuan}$^2$,~~ \textbf{Enes Ulas Dincer}$^1$, ~~\textbf{Hongyi Zhou}$^1$, ~~\textbf{Ge Li}$^1$, ~~\textbf{Xueyin Li}$^1$, \\
\textbf{Julius Haag}$^1$, ~~~\textbf{Nicolas Schreiber}$^1$, ~~~\textbf{Kailai Li}$^2$, ~~~\textbf{Gerhard Neumann}$^1$, ~~~\textbf{Rudolf Lioutikov}$^1$
\\[0.2em]
~~~~~~$^{1}$Karlsruhe Institute of Technology, Germany ~~~~~~~~$^{2}$University of Groningen, Netherlands ~~~~~~~
}

\twocolumn[{%
\renewcommand\twocolumn[1][]{#1}%
\maketitle
\begin{center}
\centering
\captionsetup{type=figure}
\vspace{-0.5cm}
% \includegraphics[width=0.95\linewidth]{image/teaser.pdf} % old layout
\includegraphics[width=0.9\linewidth]{image/COVER/BG_BLACK_ELEVATION_WEAK.png} % old layout
\captionof{figure}{
We present \textbf{IRIS},
an \textbf{I}mmersive \textbf{R}obot \textbf{I}nteraction \textbf{S}ystem designed to support various simulators, benchmarks, and real-world scenarios. The system is highly extensible, allowing seamless integration with new simulators and XR headsets.
The images shown were captured using a Meta Quest 3 headset, which leverages the Depth API \cite{metaMetaDevelopers} to enable virtual objects to blend naturally with real-world elements rather than always rendering in the foreground. To differentiate between benchmarks, we use distinct labels for each.
}
\label{fig:front_page}
\end{center}
}]

\begin{abstract}\label{00_Abstract}
Research in the field of automated vehicles, or more generally cognitive cyber-physical systems that operate in the real world, is leading to increasingly complex systems. Among other things, artificial intelligence enables an ever-increasing degree of autonomy. In this context, the V-model, which has served for decades as a process reference model of the system development lifecycle is reaching its limits. To the contrary, innovative processes and frameworks have been developed that take into account the characteristics of emerging autonomous systems. To bridge the gap and merge the different methodologies, we present an extension of the V-model for iterative data-based development processes that harmonizes and formalizes the existing methods towards a generic framework. The iterative approach allows for seamless integration of continuous system refinement. While the data-based approach constitutes the consideration of data-based development processes and formalizes the use of synthetic and real world data. In this way, formalizing the process of development, verification, validation, and continuous integration contributes to ensuring the safety of emerging complex systems that incorporate AI. 
\end{abstract}


\begin{IEEEkeywords}
	Process Reference Model, V-Model, Continuous Integration, AI Systems, Autonomy Technology, Safety Assurance
\end{IEEEkeywords}


\IEEEpeerreviewmaketitle

\section{Introduction}

% \textcolor{red}{Still on working}

% \textcolor{red}{add label for each section}


Robot learning relies on diverse and high-quality data to learn complex behaviors \cite{aldaco2024aloha, wang2024dexcap}.
Recent studies highlight that models trained on datasets with greater complexity and variation in the domain tend to generalize more effectively across broader scenarios \cite{mann2020language, radford2021learning, gao2024efficient}.
% However, creating such diverse datasets in the real world presents significant challenges.
% Modifying physical environments and adjusting robot hardware settings require considerable time, effort, and financial resources.
% In contrast, simulation environments offer a flexible and efficient alternative.
% Simulations allow for the creation and modification of digital environments with a wide range of object shapes, weights, materials, lighting, textures, friction coefficients, and so on to incorporate domain randomization,
% which helps improve the robustness of models when deployed in real-world conditions.
% These environments can be easily adjusted and reset, enabling faster iterations and data collection.
% Additionally, simulations provide the ability to consistently reproduce scenarios, which is essential for benchmarking and model evaluation.
% Another advantage of simulations is their flexibility in sensor integration. Sensors such as cameras, LiDARs, and tactile sensors can be added or repositioned without the physical limitations present in real-world setups. Simulations also eliminate the risk of damaging expensive hardware during edge-case experiments, making them an ideal platform for testing rare or dangerous scenarios that are impractical to explore in real life.
By leveraging immersive perspectives and interactions, Extended Reality\footnote{Extended Reality is an umbrella term to refer to Augmented Reality, Mixed Reality, and Virtual Reality \cite{wikipediaExtendedReality}}
(XR)
is a promising candidate for efficient and intuitive large scale data collection \cite{jiang2024comprehensive, arcade}
% With the demand for collecting data, XR provides a promising approach for humans to teach robots by offering users an immersive experience.
in simulation \cite{jiang2024comprehensive, arcade, dexhub-park} and real-world scenarios \cite{openteach, opentelevision}.
However, reusing and reproducing current XR approaches for robot data collection for new settings and scenarios is complicated and requires significant effort.
% are difficult to reuse and reproduce system makes it hard to reuse and reproduce in another data collection pipeline.
This bottleneck arises from three main limitations of current XR data collection and interaction frameworks: \textit{asset limitation}, \textit{simulator limitation}, and \textit{device limitation}.
% \textcolor{red}{ASSIGN THESE CITATION PROPERLY:}
% \textcolor{red}{list them by time order???}
% of collecting data by using XR have three main limitations.
Current approaches suffering from \textit{asset limitation} \cite{arclfd, jiang2024comprehensive, arcade, george2025openvr, vicarios}
% Firstly, recent works \cite{jiang2024comprehensive, arcade, dexhub-park}
can only use predefined robot models and task scenes. Configuring new tasks requires significant effort, since each new object or model must be specifically integrated into the XR application.
% and it takes too much effort to configure new tasks in their systems since they cannot spawn arbitrary models in the XR application.
The vast majority of application are developed for specific simulators or real-world scenarios. This \textit{simulator limitation} \cite{mosbach2022accelerating, lipton2017baxter, dexhub-park, arcade}
% Secondly, existing systems are limited to a single simulation platform or real-world scenarios.
significantly reduces reusability and makes adaptation to new simulation platforms challenging.
Additionally, most current XR frameworks are designed for a specific version of a single XR headset, leading to a \textit{device limitation} 
\cite{lipton2017baxter, armada, openteach, meng2023virtual}.
% and there is no work working on the extendability of transferring to a new headsets as far as we know.
To the best of our knowledge, no existing work has explored the extensibility or transferability of their framework to different headsets.
These limitations hamper reproducibility and broader contributions of XR based data collection and interaction to the research community.
% as each research group typically has its own data collection pipeline.
% In addition to these main limitations, existing XR systems are not well suited for managing multiple robot systems,
% as they are often designed for single-operator use.

In addition to these main limitations, existing XR systems are often designed for single-operator use, prohibiting collaborative data collection.
At the same time, controlling multiple robots at once can be very difficult for a single operator,
making data collection in multi-robot scenarios particularly challenging \cite{orun2019effect}.
Although there are some works using collaborative data collection in the context of tele-operation \cite{tung2021learning, Qin2023AnyTeleopAG},
there is no XR-based data collection system supporting collaborative data collection.
This limitation highlights the need for more advanced XR solutions that can better support multi-robot and multi-user scenarios.
% \textcolor{red}{more papers about collaborative data collection}

To address all of these issues, we propose \textbf{IRIS},
an \textbf{I}mmersive \textbf{R}obot \textbf{I}nteraction \textbf{S}ystem.
This general system supports various simulators, benchmarks and real-world scenarios.
It is easily extensible to new simulators and XR headsets.
IRIS achieves generalization across six dimensions:
% \begin{itemize}
%     \item \textit{Cross-scene} : diverse object models;
%     \item \textit{Cross-embodiment}: diverse robot models;
%     \item \textit{Cross-simulator}: 
%     \item \textit{Cross-reality}: fd
%     \item \textit{Cross-platform}: fd
%     \item \textit{Cross-users}: fd
% \end{itemize}
\textbf{Cross-Scene}, \textbf{Cross-Embodiment}, \textbf{Cross-Simulator}, \textbf{Cross-Reality}, \textbf{Cross-Platform}, and \textbf{Cross-User}.

\textbf{Cross-Scene} and \textbf{Cross-Embodiment} allow the system to handle arbitrary objects and robots in the simulation,
eliminating restrictions about predefined models in XR applications.
IRIS achieves these generalizations by introducing a unified scene specification, representing all objects,
including robots, as data structures with meshes, materials, and textures.
The unified scene specification is transmitted to the XR application to create and visualize an identical scene.
By treating robots as standard objects, the system simplifies XR integration,
allowing researchers to work with various robots without special robot-specific configurations.
\textbf{Cross-Simulator} ensures compatibility with various simulation engines.
IRIS simplifies adaptation by parsing simulated scenes into the unified scene specification, eliminating the need for XR application modifications when switching simulators.
New simulators can be integrated by creating a parser to convert their scenes into the unified format.
This flexibility is demonstrated by IRIS’ support for Mujoco \cite{todorov2012mujoco}, IsaacSim \cite{mittal2023orbit}, CoppeliaSim \cite{coppeliaSim}, and even the recent Genesis \cite{Genesis} simulator.
\textbf{Cross-Reality} enables the system to function seamlessly in both virtual simulations and real-world applications.
IRIS enables real-world data collection through camera-based point cloud visualization.
\textbf{Cross-Platform} allows for compatibility across various XR devices.
Since XR device APIs differ significantly, making a single codebase impractical, IRIS XR application decouples its modules to maximize code reuse.
This application, developed by Unity \cite{unity3dUnityManual}, separates scene visualization and interaction, allowing developers to integrate new headsets by reusing the visualization code and only implementing input handling for hand, head, and motion controller tracking.
IRIS provides an implementation of the XR application in the Unity framework, allowing for a straightforward deployment to any device that supports Unity. 
So far, IRIS was successfully deployed to the Meta Quest 3 and HoloLens 2.
Finally, the \textbf{Cross-User} ability allows multiple users to interact within a shared scene.
IRIS achieves this ability by introducing a protocol to establish the communication between multiple XR headsets and the simulation or real-world scenarios.
Additionally, IRIS leverages spatial anchors to support the alignment of virtual scenes from all deployed XR headsets.
% To make an seamless user experience for robot learning data collection,
% IRIS also tested in three different robot control interface
% Furthermore, to demonstrate the extensibility of our approach, we have implemented a robot-world pipeline for real robot data collection, ensuring that the system can be used in both simulated and real-world environments.
The Immersive Robot Interaction System makes the following contributions\\
\textbf{(1) A unified scene specification} that is compatible with multiple robot simulators. It enables various XR headsets to visualize and interact with simulated objects and robots, providing an immersive experience while ensuring straightforward reusability and reproducibility.\\
\textbf{(2) A collaborative data collection framework} designed for XR environments. The framework facilitates enhanced robot data acquisition.\\
\textbf{(3) A user study} demonstrating that IRIS significantly improves data collection efficiency and intuitiveness compared to the LIBERO baseline.

% \begin{table*}[t]
%     \centering
%     \begin{tabular}{lccccccc}
%         \toprule
%         & \makecell{Physical\\Interaction}
%         & \makecell{XR\\Enabled}
%         & \makecell{Free\\View}
%         & \makecell{Multiple\\Robots}
%         & \makecell{Robot\\Control}
%         % Force Feedback???
%         & \makecell{Soft Object\\Supported}
%         & \makecell{Collaborative\\Data} \\
%         \midrule
%         ARC-LfD \cite{arclfd}                              & Real        & \cmark & \xmark & \xmark & Joint              & \xmark & \xmark \\
%         DART \cite{dexhub-park}                            & Sim         & \cmark & \cmark & \cmark & Cartesian          & \xmark & \xmark \\
%         \citet{jiang2024comprehensive}                     & Sim         & \cmark & \xmark & \xmark & Joint \& Cartesian & \xmark & \xmark \\
%         \citet{mosbach2022accelerating}                    & Sim         & \cmark & \cmark & \xmark & Cartesian          & \xmark & \xmark \\
%         ARCADE \cite{arcade}                               & Real        & \cmark & \cmark & \xmark & Cartesian          & \xmark & \xmark \\
%         Holo-Dex \cite{holodex}                            & Real        & \cmark & \xmark & \cmark & Cartesian          & \cmark & \xmark \\
%         ARMADA \cite{armada}                               & Real        & \cmark & \xmark & \cmark & Cartesian          & \cmark & \xmark \\
%         Open-TeleVision \cite{opentelevision}              & Real        & \cmark & \cmark & \cmark & Cartesian          & \cmark & \xmark \\
%         OPEN TEACH \cite{openteach}                        & Real        & \cmark & \xmark & \cmark & Cartesian          & \cmark & \cmark \\
%         GELLO \cite{wu2023gello}                           & Real        & \xmark & \cmark & \cmark & Joint              & \cmark & \xmark \\
%         DexCap \cite{wang2024dexcap}                       & Real        & \xmark & \cmark & \xmark & Cartesian          & \cmark & \xmark \\
%         AnyTeleop \cite{Qin2023AnyTeleopAG}                & Real        & \xmark & \xmark & \cmark & Cartesian          & \cmark & \cmark \\
%         Vicarios \cite{vicarios}                           & Real        & \cmark & \xmark & \xmark & Cartesian          & \cmark & \xmark \\     
%         Augmented Visual Cues \cite{augmentedvisualcues}   & Real        & \cmark & \cmark & \xmark & Cartesian          & \xmark & \xmark \\ 
%         \citet{wang2024robotic}                            & Real        & \cmark & \cmark & \xmark & Cartesian          & \cmark & \xmark \\
%         Bunny-VisionPro \cite{bunnyvisionpro}              & Real        & \cmark & \cmark & \cmark & Cartesian          & \cmark & \xmark \\
%         IMMERTWIN \cite{immertwin}                         & Real        & \cmark & \cmark & \cmark & Cartesian          & \xmark & \xmark \\
%         \citet{meng2023virtual}                            & Sim \& Real & \cmark & \cmark & \xmark & Cartesian          & \xmark & \xmark \\
%         Shared Control Framework \cite{sharedctlframework} & Real        & \cmark & \cmark & \cmark & Cartesian          & \xmark & \xmark \\
%         OpenVR \cite{openvr}                               & Real        & \cmark & \cmark & \xmark & Cartesian          & \xmark & \xmark \\
%         \citet{digitaltwinmr}                              & Real        & \cmark & \cmark & \xmark & Cartesian          & \cmark & \xmark \\
        
%         \midrule
%         \textbf{Ours} & Sim \& Real & \cmark & \cmark & \cmark & Joint \& Cartesian  & \cmark & \cmark \\
%         \bottomrule
%     \end{tabular}
%     \caption{This is a cross-column table with automatic line breaking.}
%     \label{tab:cross-column}
% \end{table*}

% \begin{table*}[t]
%     \centering
%     \begin{tabular}{lccccccc}
%         \toprule
%         & \makecell{Cross-Embodiment}
%         & \makecell{Cross-Scene}
%         & \makecell{Cross-Simulator}
%         & \makecell{Cross-Reality}
%         & \makecell{Cross-Platform}
%         & \makecell{Cross-User} \\
%         \midrule
%         ARC-LfD \cite{arclfd}                              & \xmark & \xmark & \xmark & \xmark & \xmark & \xmark \\
%         DART \cite{dexhub-park}                            & \cmark & \cmark & \xmark & \xmark & \xmark & \xmark \\
%         \citet{jiang2024comprehensive}                     & \xmark & \cmark & \xmark & \xmark & \xmark & \xmark \\
%         \citet{mosbach2022accelerating}                    & \xmark & \cmark & \xmark & \xmark & \xmark & \xmark \\
%         ARCADE \cite{arcade}                               & \xmark & \xmark & \xmark & \xmark & \xmark & \xmark \\
%         Holo-Dex \cite{holodex}                            & \cmark & \xmark & \xmark & \xmark & \xmark & \xmark \\
%         ARMADA \cite{armada}                               & \cmark & \xmark & \xmark & \xmark & \xmark & \xmark \\
%         Open-TeleVision \cite{opentelevision}              & \cmark & \xmark & \xmark & \xmark & \cmark & \xmark \\
%         OPEN TEACH \cite{openteach}                        & \cmark & \xmark & \xmark & \xmark & \xmark & \cmark \\
%         GELLO \cite{wu2023gello}                           & \cmark & \xmark & \xmark & \xmark & \xmark & \xmark \\
%         DexCap \cite{wang2024dexcap}                       & \xmark & \xmark & \xmark & \xmark & \xmark & \xmark \\
%         AnyTeleop \cite{Qin2023AnyTeleopAG}                & \cmark & \cmark & \cmark & \cmark & \xmark & \cmark \\
%         Vicarios \cite{vicarios}                           & \xmark & \xmark & \xmark & \xmark & \xmark & \xmark \\     
%         Augmented Visual Cues \cite{augmentedvisualcues}   & \xmark & \xmark & \xmark & \xmark & \xmark & \xmark \\ 
%         \citet{wang2024robotic}                            & \xmark & \xmark & \xmark & \xmark & \xmark & \xmark \\
%         Bunny-VisionPro \cite{bunnyvisionpro}              & \cmark & \xmark & \xmark & \xmark & \xmark & \xmark \\
%         IMMERTWIN \cite{immertwin}                         & \cmark & \xmark & \xmark & \xmark & \xmark & \xmark \\
%         \citet{meng2023virtual}                            & \xmark & \cmark & \xmark & \cmark & \xmark & \xmark \\
%         \citet{sharedctlframework}                         & \cmark & \xmark & \xmark & \xmark & \xmark & \xmark \\
%         OpenVR \cite{george2025openvr}                               & \xmark & \xmark & \xmark & \xmark & \xmark & \xmark \\
%         \citet{digitaltwinmr}                              & \xmark & \xmark & \xmark & \xmark & \xmark & \xmark \\
        
%         \midrule
%         \textbf{Ours} & \cmark & \cmark & \cmark & \cmark & \cmark & \cmark \\
%         \bottomrule
%     \end{tabular}
%     \caption{This is a cross-column table with automatic line breaking.}
% \end{table*}

% \begin{table*}[t]
%     \centering
%     \begin{tabular}{lccccccc}
%         \toprule
%         & \makecell{Cross-Scene}
%         & \makecell{Cross-Embodiment}
%         & \makecell{Cross-Simulator}
%         & \makecell{Cross-Reality}
%         & \makecell{Cross-Platform}
%         & \makecell{Cross-User}
%         & \makecell{Control Space} \\
%         \midrule
%         % Vicarios \cite{vicarios}                           & \xmark & \xmark & \xmark & \xmark & \xmark & \xmark \\     
%         % Augmented Visual Cues \cite{augmentedvisualcues}   & \xmark & \xmark & \xmark & \xmark & \xmark & \xmark \\ 
%         % OpenVR \cite{george2025openvr}                     & \xmark & \xmark & \xmark & \xmark & \xmark & \xmark \\
%         \citet{digitaltwinmr}                              & \xmark & \xmark & \xmark & \xmark & \xmark & \xmark &  \\
%         ARC-LfD \cite{arclfd}                              & \xmark & \xmark & \xmark & \xmark & \xmark & \xmark &  \\
%         \citet{sharedctlframework}                         & \cmark & \xmark & \xmark & \xmark & \xmark & \xmark &  \\
%         \citet{jiang2024comprehensive}                     & \cmark & \xmark & \xmark & \xmark & \xmark & \xmark &  \\
%         \citet{mosbach2022accelerating}                    & \cmark & \xmark & \xmark & \xmark & \xmark & \xmark & \\
%         Holo-Dex \cite{holodex}                            & \cmark & \xmark & \xmark & \xmark & \xmark & \xmark & \\
%         ARCADE \cite{arcade}                               & \cmark & \cmark & \xmark & \xmark & \xmark & \xmark & \\
%         DART \cite{dexhub-park}                            & Limited & Limited & Mujoco & Sim & Vision Pro & \xmark &  Cartesian\\
%         ARMADA \cite{armada}                               & \cmark & \cmark & \xmark & \xmark & \xmark & \xmark & \\
%         \citet{meng2023virtual}                            & \cmark & \cmark & \xmark & \cmark & \xmark & \xmark & \\
%         % GELLO \cite{wu2023gello}                           & \cmark & \xmark & \xmark & \xmark & \xmark & \xmark \\
%         % DexCap \cite{wang2024dexcap}                       & \xmark & \xmark & \xmark & \xmark & \xmark & \xmark \\
%         % AnyTeleop \cite{Qin2023AnyTeleopAG}                & \cmark & \cmark & \cmark & \cmark & \xmark & \cmark \\
%         % \citet{wang2024robotic}                            & \xmark & \xmark & \xmark & \xmark & \xmark & \xmark \\
%         Bunny-VisionPro \cite{bunnyvisionpro}              & \cmark & \cmark & \xmark & \xmark & \xmark & \xmark & \\
%         IMMERTWIN \cite{immertwin}                         & \cmark & \cmark & \xmark & \xmark & \xmark & \xmark & \\
%         Open-TeleVision \cite{opentelevision}              & \cmark & \cmark & \xmark & \xmark & \cmark & \xmark & \\
%         \citet{szczurek2023multimodal}                     & \xmark & \xmark & \xmark & Real & \xmark & \cmark & \\
%         OPEN TEACH \cite{openteach}                        & \cmark & \cmark & \xmark & \xmark & \xmark & \cmark & \\
%         \midrule
%         \textbf{Ours} & \cmark & \cmark & \cmark & \cmark & \cmark & \cmark \\
%         \bottomrule
%     \end{tabular}
%     \caption{TODO, Bruce: this table can be further optimized.}
% \end{table*}

\definecolor{goodgreen}{HTML}{228833}
\definecolor{goodred}{HTML}{EE6677}
\definecolor{goodgray}{HTML}{BBBBBB}

\begin{table*}[t]
    \centering
    \begin{adjustbox}{max width=\textwidth}
    \renewcommand{\arraystretch}{1.2}    
    \begin{tabular}{lccccccc}
        \toprule
        & \makecell{Cross-Scene}
        & \makecell{Cross-Embodiment}
        & \makecell{Cross-Simulator}
        & \makecell{Cross-Reality}
        & \makecell{Cross-Platform}
        & \makecell{Cross-User}
        & \makecell{Control Space} \\
        \midrule
        % Vicarios \cite{vicarios}                           & \xmark & \xmark & \xmark & \xmark & \xmark & \xmark \\     
        % Augmented Visual Cues \cite{augmentedvisualcues}   & \xmark & \xmark & \xmark & \xmark & \xmark & \xmark \\ 
        % OpenVR \cite{george2025openvr}                     & \xmark & \xmark & \xmark & \xmark & \xmark & \xmark \\
        \citet{digitaltwinmr}                              & \textcolor{goodred}{Limited}     & \textcolor{goodred}{Single Robot} & \textcolor{goodred}{Unity}    & \textcolor{goodred}{Real}          & \textcolor{goodred}{Meta Quest 2} & \textcolor{goodgray}{N/A} & \textcolor{goodred}{Cartesian} \\
        ARC-LfD \cite{arclfd}                              & \textcolor{goodgray}{N/A}        & \textcolor{goodred}{Single Robot} & \textcolor{goodgray}{N/A}     & \textcolor{goodred}{Real}          & \textcolor{goodred}{HoloLens}     & \textcolor{goodgray}{N/A} & \textcolor{goodred}{Cartesian} \\
        \citet{sharedctlframework}                         & \textcolor{goodred}{Limited}     & \textcolor{goodred}{Single Robot} & \textcolor{goodgray}{N/A}     & \textcolor{goodred}{Real}          & \textcolor{goodred}{HTC Vive Pro} & \textcolor{goodgray}{N/A} & \textcolor{goodred}{Cartesian} \\
        \citet{jiang2024comprehensive}                     & \textcolor{goodred}{Limited}     & \textcolor{goodred}{Single Robot} & \textcolor{goodgray}{N/A}     & \textcolor{goodred}{Real}          & \textcolor{goodred}{HoloLens 2}   & \textcolor{goodgray}{N/A} & \textcolor{goodgreen}{Joint \& Cartesian} \\
        \citet{mosbach2022accelerating}                    & \textcolor{goodgreen}{Available} & \textcolor{goodred}{Single Robot} & \textcolor{goodred}{IsaacGym} & \textcolor{goodred}{Sim}           & \textcolor{goodred}{Vive}         & \textcolor{goodgray}{N/A} & \textcolor{goodgreen}{Joint \& Cartesian} \\
        Holo-Dex \cite{holodex}                            & \textcolor{goodgray}{N/A}        & \textcolor{goodred}{Single Robot} & \textcolor{goodgray}{N/A}     & \textcolor{goodred}{Real}          & \textcolor{goodred}{Meta Quest 2} & \textcolor{goodgray}{N/A} & \textcolor{goodred}{Joint} \\
        ARCADE \cite{arcade}                               & \textcolor{goodgray}{N/A}        & \textcolor{goodred}{Single Robot} & \textcolor{goodgray}{N/A}     & \textcolor{goodred}{Real}          & \textcolor{goodred}{HoloLens 2}   & \textcolor{goodgray}{N/A} & \textcolor{goodred}{Cartesian} \\
        DART \cite{dexhub-park}                            & \textcolor{goodred}{Limited}     & \textcolor{goodred}{Limited}      & \textcolor{goodred}{Mujoco}   & \textcolor{goodred}{Sim}           & \textcolor{goodred}{Vision Pro}   & \textcolor{goodgray}{N/A} & \textcolor{goodred}{Cartesian} \\
        ARMADA \cite{armada}                               & \textcolor{goodgray}{N/A}        & \textcolor{goodred}{Limited}      & \textcolor{goodgray}{N/A}     & \textcolor{goodred}{Real}          & \textcolor{goodred}{Vision Pro}   & \textcolor{goodgray}{N/A} & \textcolor{goodred}{Cartesian} \\
        \citet{meng2023virtual}                            & \textcolor{goodred}{Limited}     & \textcolor{goodred}{Single Robot} & \textcolor{goodred}{PhysX}   & \textcolor{goodgreen}{Sim \& Real} & \textcolor{goodred}{HoloLens 2}   & \textcolor{goodgray}{N/A} & \textcolor{goodred}{Cartesian} \\
        % GELLO \cite{wu2023gello}                           & \cmark & \xmark & \xmark & \xmark & \xmark & \xmark \\
        % DexCap \cite{wang2024dexcap}                       & \xmark & \xmark & \xmark & \xmark & \xmark & \xmark \\
        % AnyTeleop \cite{Qin2023AnyTeleopAG}                & \cmark & \cmark & \cmark & \cmark & \xmark & \cmark \\
        % \citet{wang2024robotic}                            & \xmark & \xmark & \xmark & \xmark & \xmark & \xmark \\
        Bunny-VisionPro \cite{bunnyvisionpro}              & \textcolor{goodgray}{N/A}        & \textcolor{goodred}{Single Robot} & \textcolor{goodgray}{N/A}     & \textcolor{goodred}{Real}          & \textcolor{goodred}{Vision Pro}   & \textcolor{goodgray}{N/A} & \textcolor{goodred}{Cartesian} \\
        IMMERTWIN \cite{immertwin}                         & \textcolor{goodgray}{N/A}        & \textcolor{goodred}{Limited}      & \textcolor{goodgray}{N/A}     & \textcolor{goodred}{Real}          & \textcolor{goodred}{HTC Vive}     & \textcolor{goodgray}{N/A} & \textcolor{goodred}{Cartesian} \\
        Open-TeleVision \cite{opentelevision}              & \textcolor{goodgray}{N/A}        & \textcolor{goodred}{Limited}      & \textcolor{goodgray}{N/A}     & \textcolor{goodred}{Real}          & \textcolor{goodgreen}{Meta Quest, Vision Pro} & \textcolor{goodgray}{N/A} & \textcolor{goodred}{Cartesian} \\
        \citet{szczurek2023multimodal}                     & \textcolor{goodgray}{N/A}        & \textcolor{goodred}{Limited}      & \textcolor{goodgray}{N/A}     & \textcolor{goodred}{Real}          & \textcolor{goodred}{HoloLens 2}   & \textcolor{goodgreen}{Available} & \textcolor{goodred}{Joint \& Cartesian} \\
        OPEN TEACH \cite{openteach}                        & \textcolor{goodgray}{N/A}        & \textcolor{goodgreen}{Available}  & \textcolor{goodgray}{N/A}     & \textcolor{goodred}{Real}          & \textcolor{goodred}{Meta Quest 3} & \textcolor{goodred}{N/A} & \textcolor{goodgreen}{Joint \& Cartesian} \\
        \midrule
        \textbf{Ours}                                      & \textcolor{goodgreen}{Available} & \textcolor{goodgreen}{Available}  & \textcolor{goodgreen}{Mujoco, CoppeliaSim, IsaacSim} & \textcolor{goodgreen}{Sim \& Real} & \textcolor{goodgreen}{Meta Quest 3, HoloLens 2} & \textcolor{goodgreen}{Available} & \textcolor{goodgreen}{Joint \& Cartesian} \\
        \bottomrule
        \end{tabular}
    \end{adjustbox}
    \caption{Comparison of XR-based system for robots. IRIS is compared with related works in different dimensions.}
\end{table*}


\section{Related Works}
Data pruning aims to remove redundant examples, keeping the most informative subset of samples, namely the coreset.
Research in this area can be broadly categorized into two groups: \textit{score-based} and \textit{geometry-based} methods. Score-based methods define metrics representing the difficulty or importance of data points to prioritize samples with high scores. Geometry-based methods, whereas, focus more on keeping a good representation of the true data distribution.
Recent studies proposed \textit{hybrid} methods that incorporate the example difficulty score with the diversity of coreset.

\noindent\textbf{Score-based.}~EL2N~\citep{gordon2021data} calculates $L2$ norms of the error vector as an approximation of the gradient norm.
Entropy~\citep{coleman2020selectionproxyefficientdata} quantifies the information contained in the predicted probabilities at the end of training. However, the outcomes of such ``snapshot'' methods differ significantly from run to run, making it difficult to obtain a reliable score in a single run, as can be seen in \cref{fig:rank_corr}, \cref{Appendix_Experiments}.

Methods using training dynamics offer more reliability as they incorporate information throughout an entire run of training. Forgetting~\citep{toneva2018empirical} score counts the number of forgetting events, a correct prediction on a data point is flipped to a wrong prediction during training. AUM~\citep{pleiss2020identifyingmislabeleddatausing} accumulates the gap between the target probability and the second-highest prediction probability. 
Dyn-Unc~\citep{he2024large}, which strongly inspired our approach, prioritizes the uncertain samples rather than typical easy samples or hard ones during model training. The prediction uncertainty is measured by the variation of predictions in a sliding window, and the score averages the variation throughout the whole training process.
TDDS~\citep{zhang2024spanning} averages differences of Kullback-Leibler divergence loss of non-target probabilities for $T$ training epochs, where $T$ is highly dependent on the pruning ratio.
Taking the training dynamics into account proves useful for pruning because it allows one to differentiate informative but hard samples from ones with label noise \citep{he2024large}. However, despite the stability and effectiveness of these methods, they fail to provide cost-effectiveness as it requires training the model on the entire dataset.

\noindent\textbf{Geometry-based.}~Geometry-based methods focus on reducing redundancy among selected samples to provide better representation. 
SSP~\citep{sorscher2022beyond} selects the samples most distant from k-means cluster centers, while Moderate~\citep{xia2022moderate} focuses on samples with scores near the median.
However, these methods often compromise generalization performance as they underestimate the effectiveness of difficult examples.

Recently, hybrid approaches have emerged that harmonize both difficulty and diversity. CCS~\citep{zheng2022coverage} partitions difficulty scores into bins and selects an equal number of samples from each bin to ensure a balanced representation.
$\mathbb{D}^2$~\citep{maharana2023d2} employs a message-passing mechanism with a graph structure where nodes represent difficulty scores and edges encode neighboring representations, facilitating effective sample selection. BOSS~\citep{acharyabalancing} introduces a Beta function for sampling based on difficulty scores, which resembles our pruning ratio-adaptive sampling; we discuss the key difference in \cref{sec:betapruning}. 
Our DUAL pruning is a score-based approach as it considers difficulty and uncertainty by the score metric. Additionally, diversity is introduced through our proposed Beta sampling, making it a hybrid approach.
\section{System Overview}


This section introduces the features and technical details of IRIS.
An overview of its paradigm is shown in Figure \ref{fig:system_overview}.

\begin{teaserfigure}
    \centering
    \includegraphics[width=1\linewidth]{figures/hero_fig.png}
    \caption{Our system translates English text into a photorealistic ASL video with non-manual information. It starts with an English text input (first row), generates ASL tokens capturing both manual and non-manual details (second row), produces a skeletal pose sequence (third row), and finally creates the photorealistic ASL video (fourth row). \han{@all, please check the stylization.}}
    % \han{should we have a longer sentence? Ideally has 6-7 video frames. Still working on stylization...} \rotem{thinking about it again, instead of putting it here, I would have 1-2 examples here- only text->gloss->ours with expressions (or through pose), then have a no expression vs expression fig later on in the paper..}}
    \label{fig:system_overview}
\end{teaserfigure}

\subsection{IRIS}

% In this section,
% We separate the physical simulation and rendering into a simulation PC and MR headsets
% ScenePublisher

\subsubsection{Node Communication Protocol}
% The software architecture of IRIS utilized master node and node,
% while simulation PC is the master node and headsets as well as other devices are XR node.
% The communication between them is based on socket and ZMQ.
% Inspired by ROS, a communication protocol is created for multiple clients and transmitting data in different kinds of format.
% To make all the XR node find the master node,
% master node broadcasts UDP messages in a fixed rate to broadcast port in the network,
% then all the XR nodes will get the server ip information including ZMQ socket address and port from the master node so that XR node will start ZMQ connection to the master node.
% After that, they can communicate in request-response and publish-subscribe patterns by ZMQ.
% When the master node is offline, the XR headsets will keep listen to the discover port and wait for next master node.
% This flexible and extendable communication framework supports multiple nodes, auto-reconnection and auto-recovering from disconnection.

% IRIS operates across a simulation PC and multiple XR headsets, it requires a robust and reliable connection between them, and they need to identify each other and connect.
% In IRIS, All the simulation and XR headsets are running in the same subnet, and the XR headsets are connected by WIFI router, they should use the IP address with the same subnet.
% Therefore, IRIS expand the function of ROS from communicating in a local host to local network with different host. introduces a custom communication protocol designed to support multiple nodes across different devices and enable data transmission using request-response and publish-subscribe patterns.
% This communication protocol adopts a master-node and multi-node architecture, where the simulation PC serves as the master node and the XR headsets, along with other devices, function as XR nodes.
% Communication between these nodes is implemented using sockets and ZMQ.
% To ensure that all XR nodes can discover the master node,
% the master node broadcasts UDP messages to a fixed broadcast port on the network in a rate of five Hz.
% XR nodes receive these messages and extract the master node's details, including the ZMQ socket address and port, enabling them to establish a stable ZMQ connection with the master node.
% Once connected, communication between nodes seamlessly follows the request-response and publish-subscribe patterns supported by ZMQ.
% When the master node is offline, XR nodes continue listening the discovery port, ready to reconnect to a new master node as soon as it the master node relaunches.
% This robust communication framework supports multiple nodes, automatic reconnection, and seamless recovery from disconnections, making it highly reliable and well-suited for dynamic, multi-device XR systems.

The IRIS system operates across simulation and/or sensor-processing computers, multiple XR headsets, and other monitoring and control programs, requiring a robust and reliable network connection between them.
All devices are part of the same subnet, with all the devices connected via Wi-Fi or cable.
Inspired by the Robot Operating System (ROS \cite{quigley2009ros}), 
However, instead of using a single host IRIS leverages a local network across multiple hosts, 
while using both Request-Response and Publish-Subscribe patterns for data transmission.
This protocol follows a master-node architecture with multiple nodes, where the simulation PC serves as the master node, and the XR headsets and other devices act as XR nodes.
Communication is achieved through a combination of UDP sockets and ZeroMQ (ZMQ).

\subsection*{Introduction}
Welcome, and thank you for participating in our design workshop. Before we begin, I want to ensure that you've read the consent form I sent earlier. Have you had a chance to review it?

\textbf{[Response]}

\noindent Great! Just to reiterate, this session will be recorded for research purposes. Do I have your consent to proceed with the recording?

\textbf{[Response]}

\noindent Thank you. If at any point you feel uncomfortable or wish to stop the interview, please let me know.

\subsection*{General}
Let's start with some general questions about your experiences with 3D gaming and Harry Potter.

\begin{itemize}
    \item Which 3D games do you play and how long have you been playing them?
    \item (Which is your favorite?) Can you describe the 3D elements or the spatial elements in that game a little bit, please?
    \item Have you ever played in VR mode if that is available?
    \item Any AR games you have played? Like Pokémon Go?
    \item What Harry Potter books or movies have you watched?
    \item Which is your favorite?
    \item Which Hogwarts house would you put yourself in?
\end{itemize}

\subsection*{Narrative/Plot}
As mentioned in the consent form, our goal today is to explore innovative ways to improve social media. We particularly invited 3D game players with a love for Harry Potter because we believe this will help us think creatively and ``outside the box'' about what an ideal, magical social media could look like, beyond the confines of a small 2D smartphone screen.

Don't worry about being too creative---we're here to guide you through the design process, not to evaluate you. We want to explore different design directions with your help, and we'll be actively involved while also giving you the space to share your ideas.

To give our brainstorming some structure, we'll start with a specific scenario to solve during this session. I'm going to ask you some guiding questions to help you address the scenario:

\begin{quote}
\textit{Imagine you are a student at Hogwarts. You have friends who are Muggle-born, friends from magical families, and even friends at other wizarding schools around the world. You're also getting to know me, as I'm a new student you've just met at Hogwarts. Keeping in touch with everyone is tough because traditional Muggle social media like Instagram doesn't really portray your wizard self very well. You don't have a smartphone in the first place because Hogwarts people don't need electronic devices, so you can't really charge your smartphone.}\\
\textit{One day, a brilliant Muggle-born student had an idea. They decided to use magic to create a magical social media where everyone---Muggles, Hogwarts students, professors, and friends from other magical schools---could all communicate and share their lives in the most ideal way. Imagine you're using this new magical platform---or maybe it's not even a platform. Whatever it is, what do you think this would look like?}
\end{quote}

\subsection*{Character Map}
Before we dive into the designing part, let's first map out your relationships in your new life as a Hogwarts student. Use the Miro board link (Figure \ref{fig:miro}) to write down the names or nicknames of people that you would like to put in each section. For example, you might want to put ``younger sister'' or ``Jane'' under Family, some of your coworkers under Quidditch Team members, yourself under the house that you think you belong to, your older middle school friends as Muggles, etc.

Please note the two boundaries in the four rectangles for each house. You can place your closer friends in the inner box and not-so-close ones in the outer box. And you can put the same name in multiple boxes (or parallelograms) as you wish.

\begin{itemize}
    \item How do you currently communicate with or stay connected with each of these people via Muggle social media?
    \item What are your biggest pet peeves/issues with Muggle social media that you'd like to solve with magical powers?
    \item What moments do you feel bad/guilty about using social media, if at all?
    \item When does time spent on social media feel meaningless/meaningful/fulfilling, if at all?
    \item Pet peeves around privacy settings?
    \item When do you feel most connected to your friends when interacting with them on social media?
    \item What would a better social media be if you were to use adjectives to describe it? What does “better” mean to you?
\end{itemize}

\subsection*{Design}
Now, let's think about what this magical world social media might look like. Remember, there are no right or wrong answers here. We're looking for your imagination and creativity, so feel free to think outside the box and have fun with it!

\begin{itemize}
    \item What key capabilities would this thing have for communicating with [GROUP]*?
    \item If you could design this thing with any magical powers, what form would it take? (Is it a magical mushroom? A secret room? A creature?)
    \item What would you want to share about yourself on this ideal magical ``thing'' with [GROUP]?
    \item If you had [GROUP'] join this ``thing,'' how might its form adapt to include this new group?
\end{itemize}

*: Iterate for 1) Romantic Relationships and Closest Friends, 2) Closer Hogwarts Friends, 3) Muggle Friends, 4) Friends at Other Magical Schools, 5) Quidditch Team Members, 6) House Elf


\subsection*{Connecting Back}
\begin{itemize}
    \item What do you like about this magical “thing” compared to Muggle social media?
    \item What do you dislike about it?
    \item Imagine using the magical “thing” we discussed earlier. How do you think it could resolve some of the issues people might face with Muggle social media?
    \item How might this new magical “thing” better support your needs for self-presentation and identity management compared to traditional social media?
    \item What challenges might arise in using such a “thing” at Hogwarts?
\end{itemize}

Thank you so much for your time and insights. Your contributions are incredibly valuable to our research. Do you have any final thoughts or questions before we conclude?


To ensure node discovery, the master node broadcasts UDP messages at 5 Hz to a fixed broadcast port on the network.
When a new XR node is launched, 
it listens on the broadcast port to receive a broadcast message from the master node. 
%It extracts the master node's details, 
%including its ZMQ socket address and port, to establish a stable ZMQ connection.
Upon receiving the broadcast message, the XR node extracts the master node's details, including its ZMQ socket address and port, to establish a reliable ZMQ connection.
If the master node goes offline, XR nodes continue listening on the discovery port, allowing automatic reconnection when the master node relaunches.
This protocol (Fig. \ref{fig:protocol}) achieves \textbf{Cross-User} ability of IRIS, ensures reliable communication, automatic reconnection, and smooth recovery from disconnections,
making it ideal for dynamic multi-device XR systems.


\subsubsection{Unified Scene Specification}
% To visualize a scene with arbitrary objects in headsets from the simulation,
% the headsets need to receive the scene model from the simulation and rebuild them.
% The XR application running in the headsets is developed by C\# and Unity,
% and simulation is running by Python or C++.
% From related works, they only support specific robots and assets,
% since they need to create some identical predefined models in a model set for the XR application,
% and they don't support robots or objects which is not in the models set.
% This mechanics highly restrict the flexibility and reusability of their system.
% To fully support all the objects, robots, and assets from simulation,
% we defined a unified scene model format and directly read the scene from simulation.
% Then this information will be sent to the headset by the node communication protocol for rebuilding an identical scene in our XR application.

To visualize a scene with arbitrary objects, the XR headsets need to receive the scene model from the simulation and reconstruct it.
However, the XR application and the simulation run on different devices and use different software architectures (the XR application is developed with C\# and Unity, while simulators might be built in Python or C++).
This makes it impractical to directly transfer the scene from the simulation to the XR environment.
To address this issue, existing solutions rely on predefined models in the XR application for specific robots and assets, requiring a static set of models to be maintained within the application.
This approach restricts flexibility and reusability, preventing the support of robots or objects not included in the model set.
To address this issue, IRIS introduces a novel unified scene specification, which is generated by parsing the scene directly from the simulation.
This specification is subsequently transmitted to the headsets using the node communication protocol, enabling the XR application to accurately and dynamically recreate the scene in real time.

\begin{figure}[t]
    \centering
    \includegraphics[width=0.75\linewidth]{image/xinkai_tree.pdf}
    \caption{
The hierarchical structure of the Scene Specification begins with a root SimObject, which contains all objects in the scene. Each SimObject has a name, a list of child SimObjects, and a list of visuals. Each visual represents a geometric element attached to the object.
Within each geometric element, materials define properties such as color and texture, while meshes determine the shape. The scene's raw data includes the byte streams of meshes and textures. Since these streams are extensive, they are sent to XR headsets sequentially after the initial scene specification is transmitted.
}
    \label{fig:scene_specification}
\end{figure}

% The unified scene specification includes environment setting (e.g., lighting) and all the objects as well as their geometry elements, meshes, materials and textures attached to them.
% IRIS provides a Python Library named ScenePublisher for parsering a simulation scene to a scenespevification.
% In the scene specification, geometry element is the shape of object (e.g., Cube, sphere, capsule, cylinder, or mesh).
% The mesh contains the vertex list, faces list, normals list, and texture coordinate list.
% material is the color and eemissionColor, reflectance or texture attached to them.
% Texture is an image which repesetnt texture.
% The objects is stacked by the kinematic tree and they will be parsed into json format by the ScenePublisher.
% Since the mesh and texuture contain too much data, and they will be sent to clients at beginning because it is rather big.
% So the visual and material which have mesh and texture will hold an hash code, and wait the request from XR node to retrieve it later.

The unified scene specification (shown in Fig. \ref{fig:scene_specification}) includes all objects with their geometry, meshes, materials, and textures. 
IRIS provides a Python library called \textit{ScenePublisher} to parse a simulation scene into this scene specification.
In the specification, 
geometry defines the object's shape (e.g., cube, sphere, capsule, cylinder, or mesh).
The mesh contains vertex lists, face lists, normals, and texture coordinates. 
Materials define surface properties, including color, emission color, reflectance, and attached textures, 
and the texture is an image representing the object's surface appearance.
Objects are organized using a kinematic tree structure and are serialized into JSON by the \textit{ScenePublisher}.

Since meshes and textures contain large amounts of data, 
geometry and materials store only their hash code to reduce the transmission load. 
The XR application rebuilds the scene upon receiving the kinematic tree and then requests the meshes and textures from the simulation server in byte format.
In some cases, textures can be quite large (e.g., the textures for the RoboCasa \cite{nasiriany2024robocasa} scene exceed 700 MB). To ensure the scene loads within an acceptable time for users, we compress the textures. This compression reduces the loading time to a few seconds.
Afterwards, the \textit{ScenePublisher} continually acquires simulation states and forwards them to the XR headset. This way the positions and rotations of all the objects are updated at a fixed frequency.

The scene specification enables IRIS to support a wide range of robots and objects in simulation, facilitating both \textbf{Cross-Scene} and \textbf{Cross-Embodiment} capabilities.


\subsubsection{Extendable and Flexible Framework Support}
The unified scene specification is a general definition that does not rely on any specific simulator, providing an extensible mechanism for scene loading and updating. IRIS can be easily adapted to various simulation engines and frameworks by implementing a new simulation parser to generate the unified specification from the simulation scene and a new publisher to update the states of scene.

Currently, IRIS supports scene parsers for MuJoCo, IsaacSim, CoppeliaSim, and Genesis, with the potential to be extended to other simulation engines as desired.
IRIS has been tested in some MuJoCo-based benchmarks including \textbf{Meta World} \cite{yu2020meta}, \textbf{LIBERO} \cite{liu2024libero}, \textbf{RoboCasa} \cite{nasiriany2024robocasa}, \textbf{robosuite} \cite{zhu2020robosuite}, \textbf{Fancy Gym} \cite{fancy_gym}, and CoppeliaSim-based benchmark like 
\textbf{PyRep} \cite{james2019pyrep},
\textbf{Colosseum} \cite{pumacay2024colosseum}.
This demonstrates that IRIS can be easily adapted to various benchmarks and simulators, highlighting its \textbf{Cross-Simulator} capability.


IRIS provides a user-friendly API.
For each environment or framework,
a single line of code suffices to visualize and update the simulation in the XR headset.
Here is a short example of how to use it in the MuJoCo Simulation, where the important line is marked in bold: 
\begin{lstlisting}[language=Python]
# import scenepub
from scenepub.sim.mj_publisher import MujocoPublisher
# define the mujoco environment
model = mujoco.MjModel.from_xml_path(xml_path)
data = mujoco.MjData(model)
# define the ScenePublisher for mujoco
(*@\textbf{publisher = MujocoPublisher(model, data, host)}  @*)
# run simulation
while True:
    # run simulation step logic
    pass
\end{lstlisting}

The MujocoPublisher instance only needs to access the model and data from the MuJoCo simulation. It then creates a separate thread to run the communication protocol, automatically connecting and communicating with all available XR headsets.
IRIS provides various Simulation Publishers for different environments, all with a consistent and easy-to-use interface. This mechanism ensures a seamless experience, making the system very user-friendly.
% \begin{table}[t]
    \centering
    \begin{tabular}{cc}
        \hline
        \toprule
        \textbf{Simulation Engine} & \textbf{Benchmarks} \\
        \midrule
        Mujoco & Meta World, LIBERO, Robocasa, Fancy Gym \\
        \midrule
        IsaacSim & IsaacLab, ARNOLD \\
        \midrule
        CoppeliaSim & Colosseum, RLBench \\
        \midrule
        Genesis & N.A \\
        \bottomrule
    \end{tabular}
    \caption{Simulation engines and their associated benchmarks.}
    \label{tab:single-column}
\end{table}


\subsubsection{Real Scene Loading}
\label{sec:real_world_teleop}
% The loading and updating of real scene is similar to the simulation.
% IRIS use cameras and calibrate them for merging multiple point clouds into one point cloud.
% Then the point cloud will be cropped and down-sampled,
% each point will be sent to headsets and IRIS use particle system in Unity to visualize and update the point cloud.

% We merge and down-sample the point clouds from multiple cameras by first applying extrinsic transformations to each point cloud. 
% Then, we map the XYZ coordinates of the both pointclouds to voxel-grid indices.
% Next, we blend the colors of all points within a voxel and compute the voxel's centroid XYZ coordinates, returning both as output.
% This entire process uses Thrust \cite{bell2012thrust} and runs on the GPU to achieve low latency.

% The loading and updating of real-world scenes in IRIS follows a process similar to that used for simulation scenes, making the feature of Cross-Reality.
The loading and updating of real-world scenes in IRIS follow a process similar to that of simulation scenes, demonstrating its Cross-Reality capability.
It processes point clouds from one or multiple RGB-D cameras, which are extrinsically calibrated to a fiducial marker in the scene. A point cloud processor applies the extrinsic transformation to each point cloud before merging them. The merged point cloud is then cropped and downsampled using a voxel-grid filter.
This filter maps all 3D points to voxel-grid indices, blends the colors of points within the same voxel, and calculates the voxels centroid.
The output is a reduced point cloud that retains both color and position data.
To ensure low latency, this process runs on the GPU using Thrust \cite{bell2012thrust}.
Finally, the processed point cloud is transmitted to the headsets, where IRIS uses a particle system in Unity to visualize and dynamically update the scene in real-time.


\subsubsection{Interaction Data Collection}
\label{sec:interaction_data_collection}
IRIS provides user-friendly access to interaction data through the \textit{ScenePublisher}, which supports both Meta Quest 3 and HoloLens 2. 
Users only need to create a new device instance and assign a name to the connection.
The \textit{ScenePublisher} then waits automatically for the device to launch.
Once the device is launched, the instance receives messages from the XR headset, enabling users to access various types of interaction data through different methods.
This process makes it straightforward to retrieve data such as hand tracking, motion controller inputs, and other relevant interaction information.
The interaction data accessible through the ScenePublisher is listed in Tab. \ref{tab:feature-comparison},
and the example code can be found in the Appendix.
\ref{app:interaction_data}





\begin{table}[t]
    \centering
    \begin{tabular}{lcc}
        \hline
        \toprule
        \textbf{Feature} & \textbf{HoloLens 2} & \textbf{Meta Quest 3} \\
        \midrule
        Hand Tracking     & \cmark & \cmark \\
        Gaze Tracking     & \cmark & \cmark \\
        Motion Controller & \xmark & \cmark \\
        Head Tracking     & \cmark & \cmark \\
        % Voice             & \cmark & \cmark \\
        \bottomrule
    \end{tabular}
    \caption{
Feature comparison between HoloLens 2 and Meta Quest 3.
IRIS supports the transmission of interaction data from headsets to the simulation environment, with easy access to this data through ScenePublisher.
}
    \label{tab:feature-comparison}
\end{table}




\subsubsection{Spatial Anchor}
\label{sec:spatial_anchor}
% \textcolor{red}{rewrite this part}

% Spatial anchor serves as a reference in a 3D environment to accurately position virtual objects within physical space. 
% It allows virtual objects to maintain their position, orientation, and alignment even as users move around or leave and return to the environment.
% IRIS use QR code or motion controller (only for Meta Quest 3) as spatial anchor
% Alignment between the virtual environment and the physical world is achieved by using a trackable QR code.
% The QR code serves as the coordinate frame of the world for the virtual environment,
% and users can provide the offset.
% This method could be used for align all the virtual scene from multiple headsets and makes them look like they are sharing the same scene in the real world.
% facilitates seamless alignment of virtual and physical elements, 
% This method could also be used for synchronizing a virtual robot with its real-world counterpart in the Kinesthetic Teaching interface and multiple-collector view alignment.
% The implementation of QR tracking varies in different headsets,
% and this function is supported by all the headsets in IRIS.

A spatial anchor serves as a reference point in a 3D environment to accurately position virtual objects within physical space.
It enables virtual objects to maintain their position, orientation, and alignment, even as users move around or leave and return to the environment.
IRIS utilizes either QR codes or motion controllers (currently only implemented for the Meta Quest 3) as spatial anchors.
This approach allows for the alignment of augmented scenes across multiple headsets in the real world, making it appear as though all users are sharing the same scene.
Additionally, this method can synchronize a virtual robot with its real-world counterpart in the {Kinesthetic Teaching} (\ref{sec:kt}) interface and facilitate multiple-collector view alignment.
Fig. \ref{fig:alignment} shows the alignment between real robot and virtual robot.

\section{Regulating Selective Response for Improved Social Welfare with Minimal Intervention} \label{sec: regulation}
In this section, we adopt the perspective of a regulator aiming to benefit users through interventions. We show how to use the results from the previous section to ensure that the intervention will be beneficial from a welfare perspective. Additionally, we bound the revenue gap that such an intervention may create. A crucial part of our approach is that the regulator can see previous actions, but not future actions, making it closer to real-world scenarios. Specifically, for any arbitrary round $\tau$, we assume the regulator observes $x_1, \dots x_{\tau}$, but has no access to GenAI's future strategy $(x_{t})_{t = \tau + 1}^T$. 
\subsection{Sufficient Conditions for Increasing Social Welfare}
We focus on $\tau$-selective modifications that guarantee to increase welfare w.r.t. a base strategy $\bft x$. We further assume GenAI commits to a 0 response level as long as its quality is below $C$, where $C$ is the threshold from Theorem~\ref{thm: sw silence effect}. %We term this \emph{GenAI commitment}. \omer{explain why make sense}.
This commitment, formally given by $\min_{t > \tau} \{w^g_t(\bft{x}) \mid w^g_t(\bft{x}) > 0\} > C$, represents the minimum utility required from GenAI for rounds $t > \tau$. %It ensures that the potential positive effect on welfare of increased proportions after the selective response are realized, guaranteeing a net increase in social welfare.
\begin{corollary}\label{cor: welfare suficient condition}
Assume that $w^g_\tau(\bft{x}) < C$ and that GenAI commits, i.e., $\min_{t > \tau} \{w^g_t(\bft{x}) \mid w^g_t(\bft{x}) > 0\} > C$ holds for all $t > \tau$. Then, $W(\bft{x}^\tau) \geq W(\bft{x})$.
\end{corollary}
Intuitively, \Cref{cor: welfare suficient condition} ensures that the welfare improvement due to this intervention (the green shaded region in Figure~\ref{fig:welfare+}) surpasses the welfare reduction (the gray region).
\subsection{Bounding GenAI's Revenue Gap} \label{sec: rev diff}
A complementary question is to what extent \emph{forcing} a $\tau$-selective response can harm GenAI's revenue. Our goal is to establish a bound on the revenue gap between the base strategy $\bft{x}$ and the modified strategy $\bft{x}^\tau$, where the selective response occurs in round $\tau$. We stress that incomplete information about future actions makes this analysis challenging. 

By definition, $\bft{x}$ and $\bft{x}^\tau$ are identical except for round $\tau$. Consequently, they generate the same amount of data in all rounds \emph{before} $\tau$. Using a $\tau$-selective response reduces the proportion of answers in that round, which in turn increases the accumulated data available in round $\tau + 1$. Therefore, the revenue gap can be decomposed into two components: (1) The immediate effect of the proportion change in round $\tau$, $r(p_\tau(\bft{x}^\tau)) - r(p_\tau(\bft{x}))$; and (2) the downstream effects on subsequent rounds due to the change in the data generation process. Using several technical lemmas that we prove in \appnx{Appendix~\ref{appendix:regulation}}, we show that:
\begin{corollary}\label{cor: loose bound}
It holds that
\begin{align*}
&U(\bft{x}^\tau) - U(\bft{x}) \leq \\
&\gamma^{\tau - 1} \left( r(p_\tau(\bft{x}^\tau)) - r(p_\tau(\bft{x})) \right) + L_r \gamma^{\tau} \frac{p_\tau(\bft{x}^\tau) - p_\tau(\bft{x})}{1 - \gamma}.
\end{align*}
\end{corollary}
The above bound is less informative as $\gamma$ approaches~1. In \appnx{Theorem~\ref{thm: revenue bounds}}, we obtain a tighter bound by having some additional assumptions.


% The above bound becomes less informative as $\gamma$ approaches~1. Next, we obtain a tighter bound by having some additional assumptions. We consider the case where $\beta L_a \leq 1$, which is a stricter condition that the one in Assumption~\ref{assumption: data lip}. Further, let
% \[
% k = \frac{\beta}{4(1+e^{\beta w^s})^2} \min_{\mathcal{D} \in [0, T]} \frac{da(\mathcal{D})}{d\mathcal{D}}.
% \]
% \omer{Boaz to explain these parameters in color}
% Using this notation, we present the following tighter bound.
% \begin{theorem} \label{thm: revenue bounds}
% Let $\unx = \min \{ x_t \mid t > \tau, x_t > 0 \}$. If $\beta L_a \leq 1$, then
% \begin{align*}
% &U(\bft{x}^\tau) - U(\bft{x}) < \\
% &\gamma^{\tau - 1} \left( r(p_\tau(\bft{x}^\tau)) - r(p_\tau(\bft{x})) \right) + L_r \gamma^{\tau} \frac{p_\tau(\bft{x}^\tau)-p_\tau(\bft{x})}{1-\gamma \left( 1 - k \unx^2 \right)}.
% \end{align*}
% \end{theorem}
% \omer{Boaz to explain contraction}
% \boaz{Theorem~\ref{thm: revenue bounds} relies on a contraction property of the data, specifically that the data gap after round \(\tau\) decreases at an exponential rate in \((1 - k \unx^2)\).}
% To leverage the bound provided by Theorem~\ref{thm: revenue bounds}, the regulator must elicit a commitment from GenAI regarding a minimum selective response, as the theorem's applicability depends on the value of $\unx$ for $t > \tau$.


\subsubsection{Headset Compatibility}
% \textcolor{red}{rewrite this part}

% The basic hardware of IRIS is one PC for running the simulation and XR headsets.
% The whole architecture is extendable so that other hardware is also easy to integrate into IRIS if they use IRIS communication protocol.
% To project the scene to the MR devices,
% a local WIFI network is utilized to build communication between the simulation PC, the headset, and other devices.
% The local network has a shorter communication delay compared to the remote network or the cloud.
% Currently, we tested our system on HoloLens 2 \cite{hololens2_lolambean_2023} and Meta Quest 3.


IRIS implements an XR application using Unity, featuring essential capabilities such as node communication protocols, scene construction based on scene specifications, and synchronization with remote simulations.
The application can be directly deployed to other headset platforms using the Unity deployment pipeline, showcasing IRIS's \textbf{Cross-Platform} capability.
For input data handling (reading and sending), the application requires the use of platform-specific APIs, making it impractical to rely on a generalized framework.
This approach separates the visualization and interaction components, minimizing the effort needed to transfer IRIS codebase to new platforms.


\subsection{Intuitive Robot Control Interface}
\label{sec:intuitive_robot_interface}
% \textcolor{red}{should have some images of each interface}

In data collection tasks, robot control interfaces are used to operate the robot in both simulated and real-world environments.
Based on research in teleoperation and robot data collection \cite{jiang2024comprehensive}, Kinesthetic Teaching and Motion Controllers have been identified as the most intuitive and effective control interfaces.
Hence, we ensured that IRIS supports these two methods.
Thanks to IRIS's flexible framework, it is possible to easily customize and implement additional alternative control interfaces,
such as hand tracking, gloves, smartphones, or motion tracking systems.
This adaptability enables tailored solutions to meet specific requirements, enhancing both the usability and versatility of the system for various applications.
Fig. \ref{fig:interface} shows how these two interfaces work in IRIS. The implementation of these interfaces is outlined below.

\begin{figure}[h!]
    \centering
    % \includegraphics[width=0.45\textwidth]{image/objective_study.pdf}
    \includegraphics[width=\linewidth]{image/interface/interface.png}
    \caption{This image illustrates examples of using two interfaces to control robots in simulation: Kinesthetic Teaching (left) and Motion Controller (right), shown from a third-person perspective.}
    \label{fig:interface}
\end{figure}

\subsubsection{Kinesthetic Teaching}
\label{sec:kt}
% Kinesthetic teaching is an intuitive interface controlling robots by physically moving the real robot.
% The real robot transmit joints position and velocity to the controlled robot in the real time,
% and the controlled robot could be real robot or virtual robot.
% In both real world and simulation data collection,
% the key of Kinesthetic Teaching is the alignment of the real robot with the corresponding virtual robot in the XR headsets,
% so that usrs feel an immersive control. 
% By using spatial anchor, the robot objects in the XR and points of cloud could be perfectly match to the real robot.

Kinesthetic teaching is an intuitive interface that allows users to control robots by physically moving the real robot \cite{wrede2013user, sukkar2023guided, jiang2024comprehensive}.
The real robot transmits joint positions and velocities in real time to the controlled robot, which can be either a virtual or another physical robot.
In both real-world and simulation data collection,
the key aspect of kinesthetic teaching is ensuring alignment between the real robot and its virtual counterpart in the XR headsets.
By utilizing \textit{Spatial Anchors} (\ref{sec:spatial_anchor}), the virtual robot in the XR headsets can be perfectly aligned with the real robot, which provide users with an intuitive and immersive experience.

% In addition, our Kinesthetic Teaching interface provides force feedback for users,
% which means that users feel the resistance force from real robot when controlled robots contacts with objects,
% this feature also applied to real world data and virtual data collection.


\subsubsection{Motion Controller}

% Motion controllers are widely used in robot data collection.
% \textcolor{red}{some papers}
% They offer stable tracking and a flexible approach for various applications.
% While some XR headsets, such as the HoloLens 2, do not natively support motion controllers, this limitation can be addressed by integrating third-party motion controllers compatible with platforms like SteamVR. \textcolor{red}{citation}
% Motion controllers use inverse kinematics to control robots in Cartesian space. The movement of the robot’s end effector is dictated by the motion controller's trigger. 
% IRIS supports retrieve motion controller data from Meta Quest 3.
% For more complex scenario, users could also design their own controller based on the our IRIS.
% From our usage, motion controllers are only suitable for two-gripper end effectors, and it is challenging to use them with robotic hands to perform complex movements.

Motion controller Interfaces are commonly used in robot data collection and teleoperation \cite{pettinger2020reducing, lin2022comparison}, providing stable tracking and flexibility for various applications. 
Although some XR headsets, such as HoloLens 2, do not natively support motion controllers, this limitation can be resolved by integrating third-party controllers compatible with platforms such as SteamVR \cite{steampoweredSteamVR}.
Motion controllers utilize inverse kinematics to control robots in Cartesian space, with the movement of the robot’s end effector controlled by the controller's trigger. 
IRIS supports retrieval of motion controller data from devices like the Meta Quest 3. 
For more complex scenarios, users can design custom controllers using IRIS' flexible framework.
% Based on our experience, motion controllers are well-suited for two-gripper end effectors, 
% but are less effective for controlling robotic hands in tasks requiring complex movements.


% \subsubsection{Hand Tracking}

% Hand tracking is an intuitive method for controlling robots, especially humanoid robots or robotic arms with hand-like end effectors.
% The inside-out hand tracking (HT) interface uses the cameras of XR headsets to track hand movements and recognize gestures, 
% which is widely supported across different types of XR devices.
% This method is particularly useful for data collection tasks, as it closely mimics robotic hand movements, providing a natural and immersive control experience.
% \textcolor{red}{not sure about this part}
% IRIS provides an easy-to-use Python API (\ref{sec:interaction_data_collection}) to access hand tracking data. 
% Although this interface has only been tested with two-finger end effectors rather than humanoid robots. 
% users can easily leverage IRIS to develop custom controllers for their own robotic systems.
% However, hand tracking has some limitations.
% It requires the user's hands to stay within the headset's field of view, and tracking can be disrupted by occlusions, poor lighting, or fast hand movements, reducing its reliability for precise control.


% \textcolor{red}{should be a table of interfaces here}

\subsection{Affiliated Monitor Tools}
The extensibility of IRIS opens up numerous possibilities for creating new applications. IRIS includes a web-based monitoring tool for managing all XR headsets. This tool allows users to easily start and stop alignment processes, as well as rename devices.
Additionally, the tool supports real-time scene visualization using \textit{three.js} \cite{threejsThreejsDocs}, using the unified scene specification. An example screenshot is shown in Fig. \ref{fig:webapp}, with further technical details provided in the appendix. \textcolor{red}{}

\begin{figure}[h!]
    \centering
    \includegraphics[width=0.45\textwidth]{image/webapp.png}
    \caption{
The IRIS Dashboard, accessible via a web interface, 
allows the control and monitoring of all connected nodes. 
The scene streamed by IRIS is rendered on the right-hand side. 
The left panel displays the connected XR devices, 
providing an interface through which users can control all services made available by each device.
    }
    \label{fig:webapp}
\end{figure}




\section{System Application}

\subsection{Robot Data Collection for Benchmarks}

Through its flexible framework design, IRIS supports four simulators and various robot manipulation benchmarks.
Based on the interfaces introduced in Sec. \ref{sec:interaction_data_collection},
IRIS already provides example controllers for various benchmarks and frameworks, such as robosuite, LIBERO, RoboCasa, mink, Metaworld, Fancy Gym, and so on.
% Since all the objects in the simulation including robots and other objects are considered as one item, so naturally IRIS supports visualize any type of robot in the XR headsets from the simulation without any configuration.
Robots from all the frameworks can be controlled by Motion Controller or Kinesthetic Teaching.
Fig. \ref{fig:front_page} shows the robots controlled by IRIS including Franka Panda, Aloha 2, Barrett Wam Arm, UR5e, iiwa14, Boston Dynamics Spot, Unitree H1, and more.


% \textcolor{red}{picture with all kinds of robot control}

% \begin{figure}[h!]
    \centering
    \includegraphics[width=0.45\textwidth]{image/aloha.jpg}
    \caption{Control Aloha 2 in the Mujoco Simulation}
    \label{fig:aloha}
\end{figure}



% \subsection{Deformable Object Manipulation}

% As far as we know, no existing work has explored the manipulation of deformable objects using XR technologies.
% This is a significant gap in the field, as deformable object manipulation presents unique challenges compared to rigid object manipulation, including the need to handle continuously changing shapes, states, and dynamics in real time.
% Deformable objects, such as fabrics, ropes, or soft materials, are more complex to simulate and interact with due to their high degrees of freedom and nonlinear behaviors.

% IRIS supports deformable object manipulation by dynamically updating the mesh state in real time.
% This capability enables users to interact intuitively with deformable objects in a simulated environment,
% making it possible to train and test robotic algorithms for tasks that involve soft and flexible materials.


% To validate our framework, we conducted experiments in Isaac Sim, a state-of-the-art physics simulation environment designed for robotics applications.
% The specific task we designed for testing is folding a jacket, a representative challenge in deformable object manipulation. Jacket folding is a highly dynamic task that requires precise control and adaptability to continuously changing object shapes.
% \textcolor{red}{Qihao:}

% \subsection{Bimanual Robot Manipulation}

% % \textcolor{red}{rewrite this part}

% The collection of bimanual robot data is challenging for previous work.
% However, it is quite simple for IRIS for its compatiblity.


% Humanoid robots are inherently complex and challenging to implement and utilize effectively in the real world.
% These challenges arise from various factors, including the high cost of hardware, difficulties in maintaining precision and reliability, and the requirement for sophisticated infrastructure.
% Real-world robot training often demands extensive physical space, a controlled environment, and advanced sensors and actuators, all of which contribute to significant logistical and financial burdens.
% Furthermore, safety concerns and the potential for physical damage to robots during training add additional layers of complexity, particularly in dynamic or unstructured environments.

% Given these barriers, the use of simulation environments becomes not only a practical but also a highly effective alternative. Extended Reality (XR) technologies, in particular, have opened new doors for creating immersive and interactive virtual environments that bridge the gap between human users and robots.
% With XR, users can access and control virtual robots without the need for expensive hardware, allowing for a more flexible and cost-efficient approach to train, test and deploy their algorithm.

% In this paper, we explore using hand tracking to control humanoid robot as an intuitive and accessible interface for interacting with and testing virtual robots in simulated environments.
% Specifically, we utilize hand-tracking to evaluate robot performance across various datasets and tasks, demonstrating the feasibility and practicality of this approach in a simulated context. By leveraging the power of XR and simulation, we aim to democratize access to robotic training and lower the barriers to innovation in robotics.

\subsection{Collaborative Manipulation}

\begin{figure}[htbp]
    \centering
    \begin{subfigure}[b]{0.49\linewidth}
        \centering
        \includegraphics[width=\linewidth]{image/collaborative_collection/kt_collaborative_same_size.png}
        % \caption{Collaborative manipulation with two Franka Panda robots by Kinesthetic Teaching}
        % \label{fig:overview_sim}
    \end{subfigure}
    \hfill
    \begin{subfigure}[b]{0.49\linewidth}
        \centering
        \includegraphics[width=\linewidth]{image/collaborative_collection/aloha_two_player_same_size.png}
        % \caption{Collaborative manipulation with two Aloha 2 by Motion Controller}
        % \label{fig:overview_real}
    \end{subfigure}
    \caption{
Collaborative manipulation in the simulation via XR. The left image shows the collaborative manipulation for hand over a hammer between two Franka Panda robots by Kinesthetic Teaching, and the right image shows that collaborative manipulation for hand over a red board between two Aloha 2 Arms by Motion Controller.
}
    \label{fig:collaborative_collection}
\end{figure}

% Collaborative manipulation is a critical technology for advancing human-robot systems. It involves enabling multiple users or robots to work together seamlessly on shared tasks, leveraging their unique strengths to achieve goals that would be difficult or impossible to accomplish individually. 
% One of the major challenges in collaborative manipulation lies in ensuring smooth communication and synchronization between human participants, robots, and the virtual environment.
% To address this, our system employs a flexible communication framework that allows for easy integration of new devices and interfaces. Specifically, our system supports the addition of new XR headsets with minimal effort, enabling more users to join the collaborative environment dynamically. This flexibility makes it possible to scale up or adapt the system as needed, facilitating diverse use cases and applications.


% When a new XR application is started, the first step is to establish communication with the simulation master node. The master node serves as the central coordinator for all XR nodes and is responsible for managing the shared virtual environment. Upon connection, the master node sends the current scene data to the new XR node, ensuring that the new application has a synchronized starting point with the existing XR nodes.
% However, upon initialization, the newly added XR scene is not automatically aligned with the scenes of the existing XR nodes. To resolve this, the new XR node must locate a spatial anchor, which serves as the foundational reference point for aligning all XR nodes within the shared environment. 
% The spatial anchor ensures that virtual objects, scenes, and interactions are consistently positioned across devices, regardless of variations in the physical setup or device configurations.
% By aligning its virtual environment with the spatial anchor, the new XR node can seamlessly integrate into the shared scene, providing users with a unified and immersive experience.
% This process is critical for maintaining coherence and synchronization in multi-user XR environments, especially in applications like collaborative manipulation tasks. The spatial anchor acts as the cornerstone for all XR nodes, enabling accurate interaction and coordination among participants and ensuring that the virtual environment behaves as a single, unified system.


% By combining these technologies, our system enhances the user experience in collaborative manipulation tasks, making interactions more natural, intuitive, and efficient. This work demonstrates how XR can bridge the gap between humans and robots in shared environments, fostering better teamwork and paving the way for more advanced and scalable human-robot systems.


Collaborative manipulation, where multiple users provide demonstrations simultaneously and/or interactively, plays a crucial role in the advancement of human-robot systems \cite{tung2021learning}.
Collecting demonstration data for such tasks has been a significant challenge as it requires smooth communication and synchronization between human participants, robots, and the virtual environment.
Previous work on manipulation involving multiple humans often facilitates collaboration through multiple screens  \cite{Qin2023AnyTeleopAG}, 
which lacks the immersive experience provided by XR.
Moreover, typically only one person is in control of the demonstrations while others are limited to observing or monitoring \cite{szczurek2023multimodal}.
By leveraging the node communication protocol,
IRIS allows for easy integration of new devices and interfaces for controlling multiple robots in a shared scene.
Specifically, IRIS supports the addition of new XR headsets with minimal effort,
allowing more users to dynamically join the collaborative environment.
% This flexibility allows for scaling or adapting the system as needed, facilitating diverse use cases and applications.
% A handover task was chosen as an example, as illustrated in Fig. \textcolor{red}{image}.
% In this scenario, two operators each control two Aloha 2 robotic arms to pass a red board between them.
Fig. \ref{fig:collaborative_collection} shows collaborative manipulation for handover task.
% \textcolor{red}{more pictures!!!!!}

\subsection{Interaction with Robot Policies in Simulation}
Simulation has long been an essential tool for training robot agents, especially for reinforcement learning (RL) policies.
Its advantages -- such as safety, scalability, and cost-efficiency -- make it widely used in RL training. 
Previous works on interactive learning have utilized interfaces such as keyboards \cite{mandlekar2018roboturk}, 3D mouse \cite{luo2024precise, liu2024libero} or smartphone \cite{mandlekar2023human}. However, these methods are insufficient for tasks that require multiple viewpoints or complex motions.
% However, collecting human-robot interaction data for these agents remains challenging, as accurately modeling human behavior in simulation is inherently complex.
% IRIS offers a novel solution by bridging the gap between simulation and reality.
% By projecting the simulation into the real world, IRIS enables a real human operator and a virtual RL agent to share the same workspace.
% This immersive interaction framework overcomes the limitations of simulation-only setups. 
% It provides more flexibility in data collection with different robot models and scenes, facilitating human-robot interactive data collection for cross-embodiment and cross-scene tasks.
IRIS delivers an immersive experience, providing the appearance of "stepping into" the simulation.
Its interaction API (\ref{sec:interaction_data_collection}) allows users to effortlessly move and manipulate objects within the simulation, mirroring real-world interactions.
This capability enables IRIS to serve as a powerful tool for highly interactive tasks, such as competitive sports, where real-time responsiveness and precise control are essential.

\begin{figure}[h!]
    \centering
    \begin{subfigure}[b]{0.48\textwidth}
        \centering
        \includegraphics[width=\linewidth]{image/interact_with_policy/first_person_view.pdf}
        % \input{image/interact_with_policy/first_person_view.pdf}
        \caption{Interact with RL Agent in the first-person view (left) and simulation (right)}
        \label{fig:interact_with_policy_first_person}
    \end{subfigure}
    \vskip 1em
    \begin{subfigure}[b]{0.48\textwidth}
        \centering
        \includegraphics[width=\linewidth]{image/interact_with_policy/third_person_view.png}
        \caption{Interact with RL Agent in the third-person view}
        \label{fig:interact_with_policy_third_person}
    \end{subfigure}
    \caption{
Playing table tennis with RL agent in Fancy Gym environment, the RL agent policy is trained with \textit{Deep Black-Box Reinforcement Learning} (BBRL) \cite{otto2023deep} 
    }
    \label{fig:interact_with_policy}
\end{figure}


To demonstrate the capabilities of IRIS,
we designed an experiment in which a participant played table tennis against an episodic RL agent trained with \textit{Deep Black-Box Reinforcement Learning} (BBRL) \cite{otto2023deep} exclusively in simulation. 
The environment was adapted from Fancy Gym \cite{fancy_gym}.
Table tennis is a challenging task that requires dynamic perspective shifts and precise racket control.
Using a motion controller and XR headset, the participant learned to successfully return the ball the RL agent served within a few minutes.
This success highlights the potential of IRIS to facilitate high-quality interactive data collection, which can further enhance the training and performance of RL agents.

% table tennis

\subsection{Real World Teleoperation and Data Collection}
\label{sec:real_world_teleop}

Real robot data plays a crucial role in conducting real-world experiments. To minimize physical obstruction from humans, tele-operation using XR (Extended Reality) is commonly employed for collecting such data.
Some approaches, such as \cite{openteach}, utilize cameras to stream videos to XR headsets. However, this method lacks depth perception. Other works, like \cite{opentelevision}, employ movable platforms to track the movement of users' heads, allowing for active scene observation by adjusting the viewing perspective. While effective, this approach requires significant effort and resources to install and maintain the necessary hardware.

To overcome these limitations, we integrate a point cloud-based XR tele-operation system into our tele-operation framework, ensuring both immersion and interactivity for data collection. An ORBBEC Femto Bolt depth camera captures real-time depth and RGB data, and a QR code on the robot’s end effector enables camera-to-robot base transformation, aligning point clouds with the robot’s frame. A GPU-accelerated C++ pipeline processes (see Appendix~\ref{appendix:Point Cloud Processing Pipeline}) point clouds at 30Hz (\textasciitilde 2 million points per frame), applying voxel grid downsampling to reduce computational load and latency while maintaining real-time performance. Before downsampling, unnecessary regions are cropped, retaining only relevant portions of the workspace. The processed point clouds are transmitted to a main process, which forwards them to a Meta Quest 3 headset for real-time visualization and interaction.

\begin{figure}[h!]
    \centering
    \includegraphics[width=0.48\textwidth]{image/Real_robot/real_robot_img.png}
    \caption{Real-world teleoperation system integrating augmented reality for intuitive robot control. The setup includes two Franka Emika Panda robots: a Leader Robot (cyan) controlled by a Meta Quest 3 user and a Follower Robot (blue) mirroring its movements. An ORBBEC Femto Bolt depth camera (yellow) captures the scene for real-time point cloud visualization (green). The top-left and top-right images show the augmented view, reconstructing objects (e.g., red and yellow cups) in XR. The bottom image presents the physical setup, where the Leader Robot is directly controlled while the Follower Robot autonomously replicates actions, enabling precise, immersive remote operation.}
    \label{fig:real_robot_point_cloud_setup}
\end{figure}


Our teleoperation framework features two Franka Emika Panda \cite{Franka_Emika_Robot} robots: a leader controlled by a human in zero-torque mode and a follower mirroring it's movements, including gripper actions. This setup enables seamless remote manipulation while eliminating the need for human presence in the physical workspace. In future implementations, additional cameras can be incorporated, and point cloud fusion via Iterative Closest Point (ICP) will ensure precise alignment across multiple views. The system architecture, including point cloud processing, XR integration, and teleoperation, is illustrated in Figure~\ref{fig:real_robot_point_cloud_setup}. By integrating real-time point cloud visualization with XR-based teleoperation, our system enhances efficiency, accuracy, and scalability, enabling obstruction-free data collection for human-robot interaction studies.




\section{Experiment}

This section evaluates IRIS's capability to create demonstrations, focusing on the efficiency and intuitiveness of data collection.
It is followed by a performance profile analysis.

\subsection{Data Collection Evaluation in Simulation}

% IRIS is promising to collect demonstration in the simualtion,
% To verify the data collection efficiency of IRIS,
% a study was conducted to evaluate it.
% LIBERO was selected and 4 tasks is picked which represent four different movement.
% which includes
% \textit{close the microwave},
% \textit{turn off the stove},
% \textit{pick up the book in the middle and place it on the cabinet shelf},
% and
% \textit{turn on the stove and put the frying pan on it}.


% To make a fair comparison, we use the task from LIBERO and the data rather than designing other task.
% The baselines are two control interfaces from LIBERO (keyboards and 3D mouse),
% From the prior study from \cite{jiang2024comprehensive} \textcolor{red}{find another one support research} and our try,
% the hand tracking is not stable comparing to motion controller. so we choose KT and MC to conduct the user study.

% There are 8 participants in this user study, we evaluate each interface by using objective metrics ans subjective metrics.
% The objective metrics includes success rate and time consumed in each tasks,
% and the subjective metrics are conducted by questionnaires includes 
% \textit{Experience}, \textit{Usefulness}, \textit{Intuitiveness}, and \textit{Efficiency}.
% Every participant will use each interface to collect demonstration five times for each task,
% they will give a score in these four dimensions from 1 to 7.

% 
\begin{figure}[h]
    \centering
    \setlength{\tabcolsep}{2pt} % Reduce space between columns
    \renewcommand{\arraystretch}{.5} % Adjust row spacing
    \begin{tabular}{cccc}
        \includegraphics[width=0.115\textwidth]{image/user_study_task/t1.png} & 
        \includegraphics[width=0.115\textwidth]{image/user_study_task/t2.png} & 
        \includegraphics[width=0.115\textwidth]{image/user_study_task/t3.png} & 
        \includegraphics[width=0.115\textwidth]{image/user_study_task/t4.png} \\
        \scriptsize (a1) & 
        \scriptsize (b1) & 
        \scriptsize (c1) & 
        \scriptsize (d1) \\
        \includegraphics[width=0.115\textwidth]{image/user_study_task/libt1.png} & 
        \includegraphics[width=0.115\textwidth]{image/user_study_task/libt2.png} & 
        \includegraphics[width=0.115\textwidth]{image/user_study_task/libt3.png} & 
        \includegraphics[width=0.115\textwidth]{image/user_study_task/libt4.png} \\
        \scriptsize (a2) & 
        \scriptsize (b2) & 
        \scriptsize (c2) & 
        \scriptsize (d2) \\
    \end{tabular}
    \caption{
    Four tasks from LIBERO in simulation (top row: a1, b1, c1, d1) and corresponding view from Meta Quest 3 (bottom row: a2, b2, c2, d2): (a) Close the microwave, (b) Turn off the stove, (c) Pick up the book and place it on the shelf, (d) Turn on the stove and place the frying pan on it.}
    \label{fig:images_grid}
\end{figure}


To assess the data collection efficiency of IRIS, a user study was conducted by using tasks from the LIBERO benchmark \cite{liu2024libero}, which represents diverse types of movements.
Four representative tasks (see Figure~\ref{fig:images_grid}) were selected in the dimension of translation, rotation, and compound movement:
\begin{itemize}
\item[-] Task 1: \textit{close the microwave}
\item[-] Task 2: \textit{turn off the stove}
\item[-] Task 3: \textit{pick up the book in the middle and place it on the cabinet shelf}
\item[-] Task 4: \textit{turn on the stove and put the frying pan on it}.
\end{itemize}


\begin{figure}[h]
    \centering
    \setlength{\tabcolsep}{2pt} % Reduce space between columns
    \renewcommand{\arraystretch}{.5} % Adjust row spacing
    \begin{tabular}{cccc}
        \includegraphics[width=0.115\textwidth]{image/user_study_task/t1.png} & 
        \includegraphics[width=0.115\textwidth]{image/user_study_task/t2.png} & 
        \includegraphics[width=0.115\textwidth]{image/user_study_task/t3.png} & 
        \includegraphics[width=0.115\textwidth]{image/user_study_task/t4.png} \\
        \scriptsize (a1) & 
        \scriptsize (b1) & 
        \scriptsize (c1) & 
        \scriptsize (d1) \\
        \includegraphics[width=0.115\textwidth]{image/user_study_task/libt1.png} & 
        \includegraphics[width=0.115\textwidth]{image/user_study_task/libt2.png} & 
        \includegraphics[width=0.115\textwidth]{image/user_study_task/libt3.png} & 
        \includegraphics[width=0.115\textwidth]{image/user_study_task/libt4.png} \\
        \scriptsize (a2) & 
        \scriptsize (b2) & 
        \scriptsize (c2) & 
        \scriptsize (d2) \\
    \end{tabular}
    \caption{
    Four tasks from LIBERO in simulation (top row: a1, b1, c1, d1) and corresponding view from Meta Quest 3 (bottom row: a2, b2, c2, d2): (a) Close the microwave, (b) Turn off the stove, (c) Pick up the book and place it on the shelf, (d) Turn on the stove and place the frying pan on it.}
    \label{fig:images_grid}
\end{figure}



To ensure a fair comparison, the tasks and data were taken directly from LIBERO \cite{liu2024libero} without any modification.
The baselines for this study were two standard control interfaces provided by LIBERO: the Keyboard (KB) and the 3D Mouse (3M).
Based on findings from prior research \cite{jiang2024comprehensive},
hand tracking was found to be less stable than motion controllers. Therefore, we selected Kinesthetic Teaching and Motion Controller as the interfaces for the user study.

The study involved eight participants who evaluated the efficiency and intuitiveness of each interface for collecting demonstrations using both objective and subjective metrics.
The objective metrics included the success rate and the average time taken per task.
To ensure successful demonstrations, 
participants executed tasks at a very slow pace, which diluted efficiency measurements (as all interfaces appeared efficient when tasks were performed slowly), which biased participants against later interfaces \cite{jiang2024comprehensive}. 
To mitigate these biases and ensure high-quality data collection, a time limit per task was introduced,
and the time limits for four tasks are 20s, 20s, 30s, and 40s, which are quite enough for finishing the task.
The subjective metrics were assessed through a questionnaire evaluating four dimensions: \textit{Experience}, \textit{Usefulness}, \textit{Intuitiveness}, and \textit{Efficiency}.
Each participant performed each task five times using all interfaces. After finishing using one interface, they provided ratings on the subjective dimensions using a 7-point Likert scale.


\begin{table*}[!h]
\caption{Mean success rate and standard deviation per platform from the data in Fig.\ref{fig:box success rates}}
\centering
\begin{tabular}{c|c c c c}
\toprule
\textbf{Platform} & \textbf{Procedural} & \textbf{Diffusion}& \textbf{Diffusion} & \textbf{Diffusion} \\
\textbf{Heights (cm)} & \textbf{Generator} & & \textbf{VF-Offline} & \textbf{VF-Online}\\
\hline
0.10 & 0.88 $\pm$ 0.06 & 0.97 $\pm$ 0.04 & 0.98 $\pm$ 0.02 & 0.97 $\pm$ 0.02 \\
0.15 & 0.90 $\pm$ 0.07 & 0.96 $\pm$ 0.02 & 0.99 $\pm$ 0.02 & 0.99 $\pm$ 0.02 \\
0.20 & 0.91 $\pm$ 0.02 & 0.93 $\pm$ 0.05 & 0.98 $\pm$ 0.02 & 0.98 $\pm$ 0.04 \\
0.25 & 0.81 $\pm$ 0.11 & 0.93 $\pm$ 0.07 & 0.95 $\pm$ 0.08 & 0.96 $\pm$ 0.04 \\
0.30 & 0.67 $\pm$ 0.08 & 0.83 $\pm$ 0.09 & 0.86 $\pm$ 0.06 & 0.95 $\pm$ 0.03 \\
0.35 & 0.55 $\pm$ 0.13 & 0.80 $\pm$ 0.08 & 0.95 $\pm$ 0.04 & 0.93 $\pm$ 0.04 \\
0.40 & 0.62 $\pm$ 0.07 & 0.87 $\pm$ 0.09 & 0.94 $\pm$ 0.02 & 0.95 $\pm$ 0.03 \\
0.45 & 0.41 $\pm$ 0.10 & 0.79 $\pm$ 0.06 & 0.95 $\pm$ 0.05 & 0.96 $\pm$ 0.04 \\
0.50 & 0.36 $\pm$ 0.10 & 0.81 $\pm$ 0.09 & 0.97 $\pm$ 0.04 & 0.97 $\pm$ 0.04 \\
0.55 & 0.33 $\pm$ 0.05 & 0.76 $\pm$ 0.07 & 0.93 $\pm$ 0.04 & 0.98 $\pm$ 0.04 \\
0.60 & 0.18 $\pm$ 0.02 & 0.64 $\pm$ 0.05 & 0.82 $\pm$ 0.07 & 0.88 $\pm$ 0.04 \\
0.65 & 0.05 $\pm$ 0.03 & 0.21 $\pm$ 0.05 & 0.39 $\pm$ 0.06 & 0.68 $\pm$ 0.07 \\
\bottomrule
\end{tabular}
\label{tab:merged_success_rates}
\end{table*}


\begin{figure}[h!]
    \centering
    % \includegraphics[width=0.45\textwidth]{image/objective_study.pdf}
    \includegraphics[width=\linewidth]{image/objective_study_new.pdf}
    \caption{This graph shows the average task completion time (in seconds) for each interface across tasks. The KT and MC interfaces consistently perform more efficiently, while the Keyboard and 3D Mouse interfaces result in longer completion times, particularly on more complex tasks.}
    \label{fig:objective_study}
\end{figure}
For the objective metrics, Table \ref{tab:success_rate} presents the success rates for each interface across the tasks. 
A trial was considered unsuccessful if the participant failed to complete the task or exceeded the time limit. The time limits were set based on task difficulty: 20 seconds for Task 1 and Task 2, 30 seconds for Task 3, and 40 seconds for Task 4.
The data shows a success rate of over $90\%$ across all four tasks when using the KT and MC interfaces from IRIS. In contrast, the Keyboard and 3D Mouse methods from LIBERO often resulted in failures. For example, the 3D Mouse interface achieved only a $37.5\%$ success rate on task 3.
Figure \ref{fig:objective_study} shows the average time consumed for each task across four interfaces. The KT and MC interfaces consistently demonstrate lower task completion times, indicating higher efficiency. In contrast, the Keyboard and 3D Mouse interfaces show significantly higher completion times, particularly for Task 3 and Task 4, where the 3D Mouse method approaches or exceeds the task's time limit. These results align with the observed lower success rates for these interfaces, highlighting their inefficiency in time-critical tasks.


\begin{figure}[h!]
    \centering
    % \includegraphics[width=0.45\textwidth]{image/subjective_study.pdf}
    \includegraphics[width=\linewidth]{image/subjective_study_new.pdf}
    \caption{Subjective evaluation scores for usefulness, experience, intuitiveness, and efficiency across four interfaces. The KT and MC interfaces perform favorably in all categories, while the Keyboard and 3D Mouse interfaces receive lower ratings, particularly in intuitiveness and efficiency.}
    \label{fig:subjective_study}
\end{figure}

Figure \ref{fig:subjective_study} compares subjective scores for four interfaces based on usefulness, experience, intuitiveness, and efficiency.
The KT and MC interfaces consistently receive high scores across all criteria, indicating positive user perception and ease of use. 
In contrast, the Keyboard and 3D Mouse interfaces receive significantly lower ratings, particularly in intuitiveness and efficiency, reflecting the participants' difficulties in using these methods to control the robot.

The results of this user study show that IRIS received higher scores than the baseline interfaces across both objective and subjective metrics.
This indicates that the system offers a more intuitive and efficient approach for data collection.

\subsection{Performance Analysis}

The performance analysis focuses on two key aspects: network latency and headset FPS (frames per second).

Since IRIS uses asynchronous bidirectional data transfer, network latency is low, averaging around 20-30 ms. Transmission bandwidth depends on Wi-Fi capacity. Even for large scenes from RoboCasa, which contain over 200 MB of compressed assets, it takes no more than 5 seconds to transfer and generate a full scene with more than 300 objects. For real robot teleoperation, the system handles point cloud data efficiently, achieving a transmission speed of 10,000 points at 60 Hz—exceeding the camera’s frame rate.

The FPS performance depends on the hardware. On the Meta Quest 3, a scene with one robot runs at approximately 70 FPS, while on HoloLens 2, it runs around 40 FPS. As the scene size increases, FPS gradually decreases. With around 200 objects, the Meta Quest 3 headsets struggle to keep up with head movement, causing virtual objects to lag or become stuck. However, performance is additionally influenced by the complexity of the meshes and textures, as these require significant computational resources from the XR headset.

In our experiments by using Meta Quest 3, IRIS successfully handled all benchmark scenarios listed in the paper, except for some scenes from RoboCasa \cite{nasiriany2024robocasa}. These scenes have over 700 MB of assets in one single instance, which is closed to the Meta Quest 3 RAM limit. Nonetheless, IRIS was able to manage most of scenes from RoboCasa without any significant performance issues.
For real robot data collection, the optimal point cloud size is around 10,000 points, which achieves a balance between point cloud quality and FPS, maintaining a frame rate of approximately 40 FPS.


% \textcolor{red}{should be a table of performance here}



\section{Limitations}

% The visualization effect is not perfect in XR headsets, since we simply used RGB texture for the materials,
% making the robots from different simulators look the same.
% There should be some rendering configuration in the simulator but we only use RGB texture from them.
% Secondly, IRIS's performance is restricted by the number of simulated objects.
% When the quality of simulated objects in the simulation exceeds some limitation. the performance will be significantly reduced.

% Although the IRIS have 
Based on our development and usage experience, IRIS has three main limitations.
First, the visualization in XR headsets is limited to basic RGB textures for materials, making robots from different simulators appear visually identical.
Second, IRIS currently only supports rigid objects, at least deformable object have not yet been fully explored.
Finally, IRIS has only been tested on Meta Quest 3 and HoloLens 2, and further evaluation on additional devices would be beneficial.

\section{Conclusion}

% \textcolor{red}{TODO}

% In this work, we introduced IRIS, an innovative Immersive Robot Interaction System that bridges the gap between Extended Reality (XR) technologies and robotics data collection. By addressing the challenges of reproducibility and reusability prevalent in current XR-based systems, IRIS offers a flexible and extendable framework that can be used in multiple simulators, benchmarks, real-world scenarios, and multiple-user use case.
% The user study show that IRIS outperforms previous benchmark data collection method, making it as a potential option for future data collection pipeline.
% The IRIS codebase is open-source, and it will facilitate further research and make IRIS adapted to more use case and more hardware platform.

In this work, we introduced IRIS (Immersive Robot Interaction System), an innovative framework that seamlessly integrates Extended Reality (XR) technologies with robotics data collection. IRIS addresses key challenges in reproducibility and reusability that are common in current XR-based systems. Its flexible, extendable design supports multiple simulators, benchmarks, real-world applications, and multi-user use cases.
User studies demonstrate that IRIS outperforms previous data collection methods, positioning it as a promising solution for future data collection pipelines.
As an open-source project, IRIS codebase promotes further research and adaptation across diverse use cases and hardware platforms.



\bibliographystyle{plainnat}
\bibliography{main}

\clearpage

\appendix
\section{Appendix Title}
\section{Technical Details}

\subsection{Mujoco}
\label{app:mujoco}

MuJoCo \cite{todorov2012mujoco} is a fast, accurate physics engine ideal for simulating robots with complex joint structures and contact dynamics.
It supports advanced robot simulations with features like customizable actuators, collision detection, and friction modeling, enabling realistic testing of robotic control, manipulation, and learning algorithms in dynamic environments.

When starting a MuJoCo simulation, MuJoCo provides a model instance and a data instance.
The scene specification of IRIS can be generated from the model, while the simulation states can be retrieved from the data instance.
The model instance contains all the necessary assets for the scene specification, including meshes, textures, and materials. 
These assets are stored as NumPy arrays and need to be converted to byte streams with data types such as \textit{float32} or \textit{int8} for transmission.

MuJoCo uses a standard robotic coordinate system, where X is forward, Y is left, and Z is up. This differs from Unity's coordinate system, where X is right, Y is up, and Z is forward.
Therefore, object transforms, mesh vertices, faces, and normal data must be adapted for Unity using the following code:

\begin{lstlisting}[language=Python]
def mj2unity_pos(pos: List[float]) -> List[float]:
    return [-pos[1], pos[2], pos[0]]

def mj2unity_quat(quat: List[float]) -> List[float]:
    return [quat[2], -quat[3], -quat[1], quat[0]]
\end{lstlisting}

Since MuJoCo includes visual group settings that specify which object groups can be visualized, IRIS provides an API to support this feature through the \textit{MujocoPublisher} definition, as shown below:

\begin{lstlisting}[language=Python]
class MujocoPublisher(ScenePublisher):

    def __init__(
        self,
        mj_model,
        mj_data,
        host: str = "127.0.0.1",
        no_rendered_objects: Optional[List[str]] = None,
        no_tracked_objects: Optional[List[str]] = None,
        visible_geoms_groups: Optional[List[int]] = None,
    ) -> None:
\end{lstlisting}

\subsection{IsaacSim}

IsaacSim~\cite{nvidia_isaac_sim} is a robotics simulation environment based on the NVIDIA Omniverse platform~\cite{nvidia_omniverse}.
Several frameworks \cite{mittal2023orbit,gong2023arnold} are built on top of it, sharing the same underlying data structures.
IsaacSim supports ray-tracing for realistic rendering, and batched physics simulation on GPUs, significantly accelerating the training of models.

The scenes in IsaacSim are organized in a Universal Scene Description (USD) format, and data can be accessed directly with the OpenUSD API \cite{pixarUSD}.
The scene hierarchy consists of transformations (Xform), meshes, articulations (joints) among other elements. Figure~\ref{fig:isaacsim-tree} illustrates a sample scene hierarchy.

\begin{figure}[h]
    \centering
    \includegraphics[width=0.7\linewidth]{image/isaacsim/isaacsim_scene_structure.png}
    \caption{An example of USD scene hierarchy for Franka Panda robot arm in IsaacSim.}
    \label{fig:isaacsim-tree}
\end{figure}

Besides the low-level OpenUSD API, IsaacSim also provides some utility APIs for accessing scene data. In our system, we use a mix of both APIs. A (simplified) code snippet for accessing parsing the tree structure of the USD scene is given below. We use the OpenUSD API \texttt{GetChildren()} to retrieve child primitives here.

\begin{lstlisting}[language=Python]
# compute local transforms of the current prim
trans, rot, scale = self.compute_local_trans(root)

# parse material and geometry
mat_info = self.parse_prim_material(prim=root)
self.parse_prim_geometries(
    prim=root,
    prim_path=prim_path,
    sim_obj=sim_object,
    mat_info=mat_info or inherited_material,
)

# parse children of the current prim
for child in root.GetChildren():
    if obj := self.parse_prim_tree(
        root=child,
        parent_path=prim_path,
        inherited_material=mat_info or inherited_material,
    ):
        sim_object.children.append(obj)
\end{lstlisting}

Here is another snippet for computing the world transform of a primitive in the hierarchy. Instead of computing the world transform with OpenUSD API, the class in IsaacSim API \texttt{XFormPrim} is used to simplify the code.

\begin{lstlisting}[language=Python]
from pxr import Usd
from omni.isaac.core.prims import XFormPrim

def compute_world_trans(self, prim: Usd.Prim):
    prim = XFormPrim(str(prim.GetPath()))
    assert prim.is_valid()

    pos, quat = prim.get_world_pose()
    scale = prim.get_world_scale()

    return (
        pos.cpu().numpy(),
        quat_to_rot_matrix(quat.cpu().numpy()),
        scale.cpu().numpy(),
    )
\end{lstlisting}

\subsection{CoppeliaSim}
CoppeliaSim, formerly known as V-REP, is a versatile and widely used robot simulation software that supports various physics engine backbones. Scenes in CoppeliaSim are constructed using a tree structure, which includes objects such as visual shapes, dynamic shapes, and joints, as illustrated in Figure \ref{fig:coppliasim-tree}.

\begin{figure}[h]
    \centering
    \includegraphics[width=0.7\linewidth]{image/coppeliasim/CoppeliaSim_Tree.png}
    \caption{An example of a tree structure for building an ABB IRB4600 manipulator.}
    \label{fig:coppliasim-tree}
\end{figure}

In our system, we interact with CoppeliaSim using Python APIs. Depending on the version of CoppeliaSim, we utilize two groups of Python API functions:

\textbf{ZeroMQ Remote API} \footnote{https://manual.coppeliarobotics.com/en/apiFunctions.htm}. This is an official API introduced in version 4.4. It facilitates fast and straightforward communication between the simulator and a Python script, enabling users to efficiently build scenarios. For example, to retrieve mesh information for all shape objects in a CoppeliaSim scene, one can use the following call:
\begin{lstlisting}[language=Python, caption=Example ZeroMQ Remote API.]
from coppeliasim_zmqremoteapi_client import RemoteAPIClient
client = RemoteAPIClient()
sim = client.require('sim')

list_vertices, list_indices, list_normals = [], [], []

objects_id_list = sim.getObjectsInTree(sim.handle_scene, 
                                       sim.handle_all, 0)

for idx in objects_id_list:
    if sim.getObjectType(idx) == sim.sceneobject_shape:
        # We assume all the shapes are primary shapes
        vertices, indices, normals = sim.getShapeMesh(idx)
        
        list_vertices.append(vertices)
        list_indices.append(indices)
        list_list_normals.append(normals)
\end{lstlisting}


\textbf{PyRep API}\cite{james2019pyrep} \footnote{https://github.com/stepjam/PyRep}. This is a widely-used third-party Python communication interface under the CoppeliaSim version 4.1. Several notable CoppeliaSim projects and benchmarks, such as RLBench \cite{james2020rlbench}, are built on this interface. Similar to the official interface, PyRep provides various API functions and properties. For instance, to retrieve all object handles from a CoppeliaSim scene, one can use the following call:
\begin{lstlisting}[language=Python, caption=Example PyRep API.]
from pyrep.backend.sim import simGetObjectsInTree
from pyrep.backend.sim import simGetObjectType
from pyrep.backend.sim import simGetShapeMesh
from pyrep.backend.simConst import sim_handle_scene
from pyrep.backend.simConst import sim_handle_all
from pyrep.backend.simConst import sim_object_shape_type

list_vertices, list_indices, list_normals = [], [], []

objects_id_list = simGetObjectsInTree(sim_handle_scene, 
                                      sim_handle_all, 0)

for idx in objects_id_list:
    if simGetObjectType(idx) == sim_object_shape_type:
        # We assume all the shapes are primary shapes
        vertices, indices, normals = simGetShapeMesh(idx)
        
        list_vertices.append(vertices)
        list_indices.append(indices)
        list_list_normals.append(normals)

\end{lstlisting}



\subsection{Genesis}

Genesis \cite{Genesis} is a versatile physics platform designed for robotics, embodied AI, and physical AI applications. It combines a universal, re-engineered physics engine capable of simulating diverse materials and phenomena with a lightweight, ultra-fast, and user-friendly robotics simulation environment. It also features a powerful, photorealistic rendering system and a generative data engine that transforms natural language prompts into multi-modal data. By integrating various physics solvers within a unified framework, Genesis supports automated data generation through a generative agent framework, with its physics engine and simulation platform now open-source and further expansions planned.

IRIS supports Genesis by providing a Genesis parser designed to translate simulation data from the Genesis physics platform into the IRIS framework's internal representation.
It manages this by reading rigid entities, links, and geometries from a \textit{gs.Scene} object and converting them into equivalent IRIS structures such as \textit{SimScene}, \textit{SimObject}, \textit{SimVisual}, and \textit{SimMaterial}. Here's a detailed breakdown of the parsing process and its components:

The main function initializes a \textit{SimScene} and iterates through all entities in the Genesis scene.
It builds a hierarchical scene graph based on parent-child relationships between entities and links. The graph is stored in the hierarchy dictionary, which tracks each object's parent and child nodes.

Each rigid entity (\textit{RigidEntity}) from Genesis is processed to create a \textit{SimObject}. The position and rotation of the entity are retrieved and transformed is similart to Mujoco (\ref{app:mujoco}) to ensure compatibility with Unity's coordinate system.
If the Genesis scene is already built, the function pulls actual position and rotation data; otherwise, it defaults to identity transforms.

Links (RigidLink) represent individual parts of a rigid entity's structure. Each link is converted into a \textit{SimObject} and positioned within the scene based on its parent link's position and orientation.
If the link has associated visual geometries (\textit{vgeoms}), the parser processes each visual element to create \textit{SimVisual} objects, which store geometry and material data. Links without visual elements are simply added to the scene hierarchy without rendering.

Visual geometries (\textit{RigidGeom}) from Genesis are mapped to IRIS visual representations. The parser constructs a \textit{SimVisual} object, including position, rotation, and mesh data.
The mesh generation function extracts vertex and face data from the Genesis mesh and constructs a \textit{SimMesh} object for the IRIS scene, defining both the physical and visual structure of the geometry. Genesis leverages the \textit{Trimesh} \cite{trimeshTrimeshTrimesh} library to handle mesh processing, which simplifies the conversion to the IRIS mesh format.

After processing all entities, the parser organizes the parsed objects into a tree structure. Objects without parents are attached to the root of the scene, while others are linked to their respective parents based on the hierarchy dictionary.




\subsection{Real World}


The point cloud-based XR teleoperation system enables immersive teleoperation and data collection by allowing users to interact with a real-world robotic setup through an Extended Reality (XR) interface. Unlike conventional video-streaming methods, this approach provides spatial awareness and depth perception by rendering real-time 3D reconstructions of the environment in XR. The system consists of an ORBBEC Femto Bolt depth camera positioned in front of the robot to capture RGB and depth data in real time. A QR code on the robot’s end effector serves as a reference marker for pose estimation, aligning point clouds with the robot’s frame. A C++-based point cloud processing pipeline converts raw data into XYZ-RGB point clouds at 30Hz, generating approximately 2 million points per frame. To optimize computational efficiency, GPU-accelerated voxel grid downsampling is applied before transmission. The processed point clouds are sent to a main process that handles transformation, cropping, and communication with the XR system. A ZeroMQ (ZMQ) messaging protocol is used to enable efficient, low-latency communication between system components, ensuring real-time transmission of point clouds and control commands. A Meta Quest 3 headset renders the processed point clouds in real time, enabling users to visualize and interact with the scene in a fully immersive 3D environment.

\subsubsection{Camera-to-Robot Base Transformation}

\paragraph{Pose Estimation and Homogeneous Transformation}
To ensure accurate spatial alignment, the system transforms the camera’s coordinate frame into the robot’s base coordinate frame using a homogeneous transformation matrix, computed as follows:

\begin{enumerate}
    \item The QR code on the robot’s end effector provides a fixed reference point for tracking position and orientation in the camera's space.
    \item Using Forward Kinematics (FK), the end effector’s pose relative to the robot base is determined.
    \item The camera-to-robot base transformation is derived as:
    \begin{equation}
        T_{\mathrm{Robot Base}}^{\mathrm{Camera}} = T_{\mathrm{Robot Base}}^{\mathrm{End Effector}} \cdot T_{\mathrm{End Effector}}^{\mathrm{Camera}}
    \end{equation}
    where:
    \begin{itemize}
        \item $T_{\mathrm{Robot Base}}^{\mathrm{End Effector}}$ is obtained from the robot’s FK.
        \item $T_{\mathrm{End Effector}}^{\mathrm{Camera}}$ is estimated using the QR code tracking system.
    \end{itemize}
\end{enumerate}

This ensures that the captured point clouds align precisely with the robot’s coordinate frame, eliminating drift and inconsistencies in XR visualization.

\subsubsection{Point Cloud Processing Pipeline}
\label{appendix:Point Cloud Processing Pipeline}
\paragraph{Raw Data Acquisition}
The ORBBEC Femto Bolt camera captures:
\begin{itemize}
    \item Depth images (encoded as a depth map),
    \item RGB images (color information),
    \item Camera intrinsic parameters (for depth-to-3D conversion).
\end{itemize}

\paragraph{Conversion to XYZ-RGB Point Cloud}
Using the camera intrinsics, each depth pixel is converted into 3D world coordinates $(X, Y, Z)$ using:
\begin{equation}
    X = (u - c_x) \frac{Z}{f_x}, \quad Y = (v - c_y) \frac{Z}{f_y}, \quad Z = D(u,v)
\end{equation}
where:
\begin{itemize}
    \item $(u, v)$ are pixel coordinates,
    \item $(c_x, c_y)$ are the camera's principal point offsets,
    \item $(f_x, f_y)$ are focal lengths,
    \item $D(u,v)$ is the depth value at pixel $(u, v)$.
\end{itemize}
Each $(X, Y, Z)$ point is assigned an $(R, G, B)$ value from the color image, forming the XYZ-RGB point cloud.

\paragraph{Cropping and Filtering}
To reduce noise and retain only relevant portions of the workspace, the system:
\begin{itemize}
    \item Crops unnecessary regions using bounding box constraints,
    \item Applies statistical outlier removal to eliminate noise points.
\end{itemize}

\paragraph{Voxel Grid Downsampling (GPU-Accelerated)}
Since the raw point cloud consists of approximately 2 million points per frame, direct transmission is computationally expensive. To optimize efficiency, voxel grid downsampling is applied, which:
\begin{itemize}
    \item Divides the workspace into 3D voxels,
    \item Averages all points within each voxel to generate a single representative point,
    \item Reduces the point cloud size while preserving structural details.
\end{itemize}

\subsubsection{XR Visualization and Teleoperation}

\paragraph{Real-Time Communication with XR System}
The processed point cloud is transmitted to a main process, which then relays the data to the Meta Quest 3 headset using ZeroMQ. ZeroMQ ensures efficient, asynchronous, and low-latency messaging between the processing node and the XR system, allowing:  
\begin{itemize}  
    \item Real-time rendering of the scene in XR,  
    \item Full 3D immersion with accurate depth perception,  
    \item Reliable transmission of point cloud data and control commands.  
\end{itemize}  

\paragraph{Leader-Follower Teleoperation Framework}
The teleoperation system consists of:
\begin{itemize}
    \item A leader robot controlled by a human wearing the Meta Quest 3 headset,
    \item A follower robot that replicates the leader’s movements, including gripper actions.
\end{itemize}

A networked control architecture ensures:
\begin{itemize}
    \item Low-latency synchronization between the leader and follower,
    \item Seamless remote manipulation, removing the need for physical human presence.
\end{itemize}

\subsubsection{Calibration for XR-Robot Alignment}
To ensure that XR visualization aligns precisely with real-world objects, the system undergoes a calibration process consisting of:
\begin{enumerate}
    \item Robot workspace alignment: The follower robot’s workspace is adjusted to match the leader’s XR-rendered environment,
    \item Point cloud adjustment: The camera-to-robot transformation is fine-tuned,
    \item User feedback correction: Users verify object positions in XR and real-world views.
\end{enumerate}

\subsubsection{Future Extensions: Multi-Camera and ICP-Based Fusion}
Future implementations will:
\begin{itemize}
    \item Incorporate multiple cameras for wider coverage,
    \item Perform point cloud fusion using Iterative Closest Point (ICP) for improved spatial accuracy,
    \item Enhance depth perception by reconstructing high-fidelity 3D representations of the workspace.
\end{itemize}



\subsection{Affiliated Monitoring Tools}
The dashboard depicted in \ref{fig:webapp} is designed to complement the IRIS Framework by providing streamlined access to the functionality offered by XR devices. Additionally, it facilitates the rendering of the scene to a web-based canvas, enabling real-time viewing of the scene as it is being streamed.

Due to the limitations of web services, which cannot directly interface with underlying protocols such as UDP and ZMQ, this website is unable to directly connect to the IRIS server. To bridge this gap, a Python-based endpoint is deployed, which hosts the website, relays the rendering data to the web application, and forwards user commands from the interface back to the server.

Communication between the endpoint and the website is facilitated through a WebSocket connection. This setup allows the system to support multiple instances of the dashboard simultaneously, ensuring scalability. The scene rendering itself is accomplished using Three.js, a powerful JavaScript library that renders to an HTML5 canvas object through WebGL. The Python endpoint is responsible for converting the IRIS scene format into commands, which are then transmitted over the WebSocket connection. These commands are subsequently translated into high-level Three.js function calls to render the scene.

This high-performance rendering system makes it possible to display even highly complex scenes on lower-end machines. Notably, only the server hosting the web application needs to install the necessary software, enabling all connected devices to access the interface without requiring any additional installations.

To minimize latency between the initial connection and the start of rendering, primitive objects are placed in the scene immediately upon receipt by the endpoint. Larger resources, such as meshes, materials, and textures, are streamed into the scene asynchronously to prevent delays or interruptions in the rendering process.

 
\subsection{Interaction Data}
\label{app:interaction_data}

Here is an example of using \textit{ScenePub API} to retrieve the data from Meta Quest 3 and HoloLen 2.

\begin{lstlisting}[language=Python]
# import device from scenepub
from scenepub.xr_device.meta_quest3 import MetaQuest3
from scenepub.xr_device.hololens2 import HoloLens2
# do some thing about simulation
mq3_player = MetaQuest3(mq3_device_name: str)
hl2_player = HoloLens2(hl2_device_name: str)
# running simulation
while True:
    # running simulation step logic
    mc_data = mq3_player.get_mc_data()
    hand_data = mq3_player.get_hand_data()
    gaze_data = mq3_player.get_gaze_data()
    # processing input data from Meta Quest3
    hl2_hand_data = hl2_player.get_hand_data()
    hl2_gaze_data = hl2_player.get_gaze_data()
\end{lstlisting}




\end{document}


