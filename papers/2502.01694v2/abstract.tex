
\begin{abstract}
A key paradigm to improve the reasoning capabilities of large language models (LLMs) is to allocate more inference-time compute to search against a verifier or reward model. This process can then be utilized to refine the pretrained model or distill its reasoning patterns into more efficient models. In this paper, we study inference-time compute by viewing chain-of-thought (CoT) generation as a metastable Markov process: easy reasoning steps (e.g., algebraic manipulations) form densely connected clusters, while hard reasoning steps (e.g., applying a relevant theorem) create sparse, low-probability edges between clusters, leading to phase transitions at longer timescales. Under this framework, we prove that implementing a search protocol that rewards sparse edges improves CoT by decreasing the expected number of steps to reach different clusters. In contrast, we establish a limit on reasoning capability when the model is restricted to local information of the pretrained graph. We also show that the information gained by search can be utilized to obtain a better reasoning model: $(1)$~the pretrained model can be directly finetuned to favor sparse edges via policy gradient methods, and moreover $(2)$~a compressed \emph{metastable representation} of the reasoning dynamics can be distilled into a smaller, more efficient model.
\vspace{-2mm}
\end{abstract}