\begin{figure}
  \centering
  \makebox[\textwidth][c]{
    \begin{subfigure}{\shrinkfigtwo\textwidth}
      \includegraphics[trim={0.1cm 0.1cm 0.1cm 0.2cm}, clip, width=\textwidth]{pdffigs/combo.lr_curves.111M.200TPP.WD=0.1.pdf}
    \end{subfigure}
    \hspace{-1mm}
    \begin{subfigure}{\shrinkfigtwo\textwidth}
      \includegraphics[trim={0.1cm 0.1cm 0.1cm 0.2cm}, clip, width=\textwidth]{pdffigs/combo.ema_response.111M.200TPP.WD=0.1.pdf}
    \end{subfigure}
  }
  \mbox{}
  \vspace{-7mm}
  \mbox{}
  \caption{\textbf{LR schedules and their update-combination duals}:
    Each LR schedule, $\eta_t$ (left) and weight decay, $\lambda$,
    implies a weighted combination of weight \emph{updates}, with
    combination coefficients $c_{t,i}$ (right, log-scale) giving the
    contribution of $i$th update to parameters $\theta_t$ at step $t$
    (111M scale, coefficients at final step).
    %
    The more sudden the drop in LR, the less emphasis on valuable
    later updates, perhaps explaining why $\step$ underperforms
    $\cosine$ and $\cosine$ underperforms $\linear$
    decay.\label{fig:background_ema}
    %    corresponding to settings for 111M-param $\mup$ models:
    %    $\hateta$=$\maxlr$, $\rho$=$\nicefrac{1}{3}$, $\lambda$=$0.1$.
  }
\end{figure}
