\begin{wrapfigure}{r}{0.33\textwidth}
\centering
\vspace{-2mm}
\begin{tikzpicture}[scale=0.76]
    \def\growthxspikept{1.7}
    \def\growthyflatpt{0.02}
    \def\growthdescale{10}
    \begin{axis}[
        axis lines=middle,
        xlabel={TPP},
        ymin=0, ymax=0.6,
        xmin=0, xmax=10,
        width=1.6\linewidth,  % Set width relative to the wrapfigure width
        height=1.0\linewidth, % Set height to maintain aspect ratio
        x label style={at={(axis description cs:0.95,0.0)},anchor=north},
        y label style={at={(axis description cs:-0.1,.5)},rotate=90,anchor=north},
        domain=0:10,
        samples=100,
        ticks=none,
        label style={font=\bfseries}, % Make all labels bold
        % Arrow style setting
        every axis plot post/.append style={line width=1.5pt, -{Latex[length=4mm,width=4mm]}}
    ]
    % Flat line
    \addplot[blue, domain=0:9] {0.03 + 0.25 * exp(-x/3)};
    \node at (axis cs: 6.1,0.16) [right, align=left, blue, font=\bfseries] {Importance of\\high early LR};
    % Exponential growth curve starting more flat then exploding
    \addplot[red] expression[domain=0:8.5,samples=50] {x < \growthxspikept ? (\growthyflatpt) : (\growthyflatpt + pow((x-\growthxspikept)/\growthdescale, \growthxspikept))};
    \node at (axis cs: 3.35,0.45) [right, align=left, red, font=\bfseries] {Importance of\\LR decay,\\weight decay};
    \end{axis}
\end{tikzpicture}
\mbox{}
\vspace{-6mm}
\mbox{}
\caption{\textbf{HP influence vs.\ TPP}: Higher TPP means higher
  gradient noise; LR decay \& weight decay settings thus increase in
  importance with TPP.\label{fig:cartoon}}
\vspace{-4mm}
\end{wrapfigure}
