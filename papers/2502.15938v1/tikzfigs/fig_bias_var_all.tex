\begin{figure}[th]
\centering

\begin{subfigure}[t]{0.32\textwidth}
  \centering
\begin{tikzpicture}[scale=\tikzscale, every node/.style={scale=\tikztextscale}]
    % Define the dimensions of the bounding box
    \def\boxwidth{2.2} % Width for a wider appearance
    \def\rectwidth{1.5}
    \def\rectwidthTwo{0.7}
    \def\boxheight{1.5} % Less height compared to the width
    \def\rectheight{0.7}
    \def\rectheightTwo{0.7} % A different height for the second rectangle
    \def\schedshift{0.8}

    % Points for the line near the edges of rectangles
    \coordinate (LineStart) at (0.3, \rectheightTwo/2 + \schedshift); % 10pt from left edge
    \coordinate (LineEnd) at (\boxwidth - 0.3, \rectheightTwo/2 + \schedshift); % 10pt from right edge

    % Draw the rectangle in the upper-left quadrant with very light green fill
    \fill[green!20] (0, \boxheight - \rectheight) rectangle (\rectwidth, \boxheight);

    % Draw the rectangle in the bottom-right quadrant with very light red fill, different height
    \fill[red!20] (\boxwidth - \rectwidthTwo, 0) rectangle (\boxwidth, \rectheightTwo);

    % Draw the bounding box
    \draw (0, 0) rectangle (\boxwidth, \boxheight);

    % Draw a line between the modified points near the edges of the rectangles
    \draw[{Circle[length=4mm, width=4mm, color=blue]}-{Circle[length=4mm, width=4mm, color=blue]}, line width=2pt, blue] (LineStart) -- (LineEnd);

    \node[fill=white,draw=blue!80,fill opacity=.90] at ([xshift=3.5pt,yshift=-5pt]LineStart) {maxLR};
    \node[fill=white,draw=blue!80,fill opacity=.90] at ([xshift=-3pt,yshift=-5pt]LineEnd) {minLR};

    % Labels for Bias and Variance reduction
    \node[align=center,draw=green!30] (BiasLabel) at (\rectwidth/2, \boxheight + 0.2) {Bias reduction};
    \node[align=center,draw=red!30] (VarLabel) at (\boxwidth - \rectwidthTwo/2, \rectheightTwo - 0.2) {Variance reduction};
    
    % Thin lines from labels to rectangles
    \draw[thin] (BiasLabel.south) -- (\rectwidth/2 + 0.05, \boxheight - 0.05);
    \draw[thin] (VarLabel.south) -- (\boxwidth - \rectwidthTwo/2 + 0.05, \rectheightTwo -0.2 - 0.2 - 0.05);

    % Axis labels
    \node at (\boxwidth/2, -0.2) {Optimization steps};  % X-axis label
    \node[rotate=90] at (-0.2, \boxheight/2) {LR setting};  % Y-axis label

    % Arrows on axis labels
    \draw[-Stealth] (\boxwidth - 0.2, -0.2) -- (\boxwidth - 0.0, -0.2);
    \draw[-Stealth] (-0.2, \boxheight - 0.2) -- (-0.2, \boxheight - 0.0);
\end{tikzpicture}

\vspace{-0.1cm}
\caption{$\constant$, fewer steps (low TPP)\label{fig:const_low}}
\end{subfigure}%
\hfill
\begin{subfigure}[t]{0.65\textwidth}
  \centering
\begin{tikzpicture}[scale=\tikzscale, every node/.style={scale=\tikztextscale}]
    % Define the dimensions of the bounding box
    \def\boxwidth{6}
    \def\rectwidth{1.5}
    \def\rectwidthTwo{4.5}
    \def\boxheight{1.5}
    \def\rectheight{0.7}
    \def\rectheightTwo{0.7}
    \def\schedshift{0.2}

    % Points for the line near the edges of rectangles
    \coordinate (LineStart) at (0.3, \rectheightTwo/2 + \schedshift); % 10pt from left edge
    \coordinate (LineEnd) at (\boxwidth - 0.3, \rectheightTwo/2 + \schedshift); % 10pt from right edge

    % Draw the rectangle in the upper-left quadrant with very light green fill
    \fill[green!20] (0, \boxheight - \rectheight) rectangle (\rectwidth, \boxheight);

    % Draw the rectangle in the bottom-right quadrant with very light red fill, different height
    \fill[red!20] (\boxwidth - \rectwidthTwo, 0) rectangle (\boxwidth, \rectheightTwo);

    % Draw the bounding box
    \draw (0, 0) rectangle (\boxwidth, \boxheight);

    % Draw a line between the modified points near the edges of the rectangles
    \draw[{Circle[length=4mm, width=4mm, color=blue]}-{Circle[length=4mm, width=4mm, color=blue]}, line width=2pt, blue] (LineStart) -- (LineEnd);

    \node[fill=white,draw=blue!80,fill opacity=.90] at ([xshift=3.5pt,yshift=-5pt]LineStart) {maxLR};
    \node[fill=white,draw=blue!80,fill opacity=.90] at ([xshift=-3pt,yshift=-5pt]LineEnd) {minLR};

    % Labels for Bias and Variance reduction
    \node[align=center,draw=green!30] (BiasLabel) at (\rectwidth/2, \boxheight + 0.2) {Bias reduction};
    \node[align=center,draw=red!30] (VarLabel) at (\boxwidth - \rectwidthTwo/2, \rectheightTwo + 0.2) {Variance reduction};
    
    % Thin lines from labels to rectangles
    \draw[thin] (BiasLabel.south) -- (\rectwidth/2 + 0.05, \boxheight - 0.05);
    \draw[thin] (VarLabel.south) -- (\boxwidth - \rectwidthTwo/2 + 0.05, \rectheightTwo - 0.05);

    % Axis labels
    \node at (\boxwidth/2, -0.2) {Optimization steps};  % X-axis label
    \node[rotate=90] at (-0.2, \boxheight/2) {LR setting};  % Y-axis label

    % Arrows on axis labels
    \draw[-Stealth] (\boxwidth - 2.1, -0.2) -- (\boxwidth - 1.9, -0.2);
    \draw[-Stealth] (-0.2, \boxheight - 0.2) -- (-0.2, \boxheight - 0.0);
\end{tikzpicture}

\vspace{-0.1cm}
\caption{$\constant$, many steps (high TPP)\label{fig:const_high}}
\end{subfigure}

\vspace{0.2cm} % Space between the rows

\noindent

\begin{subfigure}[t]{0.32\textwidth}
  \centering
\begin{tikzpicture}[scale=\tikzscale, every node/.style={scale=\tikztextscale}]
    % Define the dimensions of the bounding box
    \def\boxwidth{2.2} % Width for a wider appearance
    \def\rectwidth{1.5}
    \def\rectwidthTwo{0.7}
    \def\boxheight{1.5} % Less height compared to the width
    \def\rectheight{0.7}
    \def\rectheightTwo{0.7} % A different height for the second rectangle
    \def\schedshift{0.17}
    \def\schedshiftTwo{-0.25}

    % Points for the line near the edges of rectangles
    \coordinate (LineStart) at (0.3,  \boxheight - \rectheight/2 + \schedshift); % 10pt from left edge
    \coordinate (LineEnd) at (\boxwidth - 0.3, \rectheightTwo/2 + \schedshiftTwo); % 10pt from right edge

    % Draw the rectangle in the upper-left quadrant with very light green fill
    \fill[green!20] (0, \boxheight - \rectheight) rectangle (\rectwidth, \boxheight);

    % Draw the rectangle in the bottom-right quadrant with very light red fill, different height
    \fill[red!20] (\boxwidth - \rectwidthTwo, 0) rectangle (\boxwidth, \rectheightTwo);

    % Draw the bounding box
    \draw (0, 0) rectangle (\boxwidth, \boxheight);

    % Draw a line between the modified points near the edges of the rectangles
    \draw[{Circle[length=4mm, width=4mm, color=blue]}-{Circle[length=4mm, width=4mm, color=blue]}, line width=2pt, blue] (LineStart) -- (LineEnd);

    \node[fill=white,draw=blue!80,fill opacity=.90] at ([xshift=3.5pt,yshift=-5pt]LineStart) {maxLR};
    \node[fill=white,draw=blue!80,fill opacity=.90] at ([xshift=-3pt,yshift=-1pt]LineEnd) {minLR};

    % Labels for Bias and Variance reduction
    \node[align=center,draw=green!30] (BiasLabel) at (\rectwidth/2, \boxheight + 0.2) {Bias reduction};
    \node[align=center,draw=red!30] (VarLabel) at (\boxwidth - \rectwidthTwo/2, \rectheightTwo + 0.2) {Variance reduction};
    
    % Thin lines from labels to rectangles
    \draw[thin] (BiasLabel.south) -- (\rectwidth/2 + 0.05, \boxheight - 0.05);
    \draw[thin] (VarLabel.south) -- (\boxwidth - \rectwidthTwo/2 + 0.05, \rectheightTwo - 0.05);

    % Axis labels
    \node at (\boxwidth/2, -0.2) {Optimization steps};  % X-axis label
    \node[rotate=90] at (-0.2, \boxheight/2) {LR setting};  % Y-axis label

    % Arrows on axis labels
    \draw[-Stealth] (\boxwidth - 0.2, -0.2) -- (\boxwidth - 0.0, -0.2);
    \draw[-Stealth] (-0.2, \boxheight - 0.2) -- (-0.2, \boxheight - 0.0);
\end{tikzpicture}

\vspace{-0.1cm}
\caption{$\dtoz$, fewer steps (low TPP)\label{fig:dtoz_low}}
\end{subfigure}%
\hfill
\begin{subfigure}[t]{0.65\textwidth}
  \centering
\begin{tikzpicture}[scale=\tikzscale, every node/.style={scale=\tikztextscale}]
    % Define the dimensions of the bounding box
    \def\boxwidth{6}
    \def\rectwidth{1.5}
    \def\rectwidthTwo{4.5}
    \def\boxheight{1.5}
    \def\rectheight{0.7}
    \def\rectheightTwo{0.7}
    \def\schedshift{-0.05}
    \def\schedshifttwo{-0.25}

    % Points for the line near the edges of rectangles
    \coordinate (LineStart) at (0.3, \boxheight - \rectheight/2 + \schedshift);
    \coordinate (LineEnd) at (\boxwidth - 0.3, \rectheightTwo/2 + \schedshifttwo);

    % Draw the rectangle in the upper-left quadrant with very light green fill
    \fill[green!20] (0, \boxheight - \rectheight) rectangle (\rectwidth, \boxheight);

    % Draw the rectangle in the bottom-right quadrant with very light red fill, different height
    \fill[red!20] (\boxwidth - \rectwidthTwo, 0) rectangle (\boxwidth, \rectheightTwo);

    % Draw the bounding box
    \draw (0, 0) rectangle (\boxwidth, \boxheight);

    % Draw a line between the modified points near the edges of the rectangles
    \draw[{Circle[length=4mm, width=4mm, color=blue]}-{Circle[length=4mm, width=4mm, color=blue]}, line width=2pt, blue] (LineStart) -- (LineEnd);

    \node[fill=white,draw=blue!80,fill opacity=.90] at ([xshift=3.5pt,yshift=-5pt]LineStart) {maxLR};
    \node[fill=white,draw=blue!80,fill opacity=.90] at ([xshift=-3pt,yshift=-5pt]LineEnd) {minLR};

    % Labels for Bias and Variance reduction
    \node[align=center,draw=green!30] (BiasLabel) at (\rectwidth/2, \boxheight + 0.2) {Bias reduction};
    \node[align=center,draw=red!30] (VarLabel) at (\boxwidth - \rectwidthTwo/2, \rectheightTwo + 0.2) {Variance reduction};
    
    % Thin lines from labels to rectangles
    \draw[thin] (BiasLabel.south) -- (\rectwidth/2 + 0.05, \boxheight - 0.05);
    \draw[thin] (VarLabel.south) -- (\boxwidth - \rectwidthTwo/2 + 0.05, \rectheightTwo - 0.05);

    % Axis labels
    \node at (\boxwidth/2, -0.2) {Optimization steps};  % X-axis label
    \node[rotate=90] at (-0.2, \boxheight/2) {LR setting};  % Y-axis label
    
    % Arrows on axis labels
    \draw[-Stealth] (\boxwidth - 2.1, -0.2) -- (\boxwidth - 1.9, -0.2);
    \draw[-Stealth] (-0.2, \boxheight - 0.2) -- (-0.2, \boxheight - 0.0);
\end{tikzpicture}

\vspace{-0.1cm}
\caption{$\dtoz$, many steps (high TPP)\label{fig:dtoz_high}}
\end{subfigure}

\caption{\textbf{Bias \& variance in LLM pre-training}: as training
  duration increases (higher TPP), the importance of variance
  reduction --- and having a lower LR --- increases.  With no decay
  ($\constant$, \ref{fig:const_low}, \ref{fig:const_high}), the
  optimal peak LR must drop significantly lower, neglecting bias
  reduction.  With $\dtoz$ (\ref{fig:dtoz_low}, \ref{fig:dtoz_high}),
  periods of bias and variance reduction can both be enjoyed without
  large shifts in peak LR.\label{fig:bias_var_all}}
\end{figure}
