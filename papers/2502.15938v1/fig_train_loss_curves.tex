\begin{wrapfigure}{r}{0.38\textwidth}
  \centering
  \vspace{-5mm}
  \begin{tikzpicture}
    % Place the image in a node
    \node[anchor=south west,inner sep=0] (image) at (0,0) {
      \scalebox{0.39}{
        \includegraphics[trim={0.35cm 0cm 0cm 0.3cm}, clip, width=\textwidth]{pdffigs/train_loss.617M.pdf}
      }
    };
    % Set up a coordinate system relative to the image
    \begin{scope}[x={(image.south east)},y={(image.north west)}]
      % FLOP savings line
      \draw[<-,thick] (0.46,0.52) -- (0.881,0.52)
      node[midway, above] {\scriptsize 60\% fewer FLOPs};
      % Vertical arrowed lines
      \draw[->,thick] (0.3,0.295) -- (0.3,0.265)
      node[midway, above] {\scriptsize 1.8\% $\downarrow$ Loss};
      \draw[->,thick] (0.3,0.18) -- (0.3,0.21)
      node[midway] {};
      % Add text above the top arrow
      % \node at (0.3, 0.3) {\scriptsize 1.8\%};
    \end{scope}
  \end{tikzpicture}
  \mbox{}
  \vspace{-7mm}
  \mbox{}
  \caption{A 610M model trained for 80~TPP with $\linear$-$\dtoz$ has
    better train (and \emph{validation}) loss than when trained for
    200~TPP with $\linear$-$\tenx$.
    %The gap
    %between $\tenx$ and $\dtoz$ increases with
    %TPP.
    \label{fig:train_loss}}
  \vspace{-9mm}
\end{wrapfigure}
