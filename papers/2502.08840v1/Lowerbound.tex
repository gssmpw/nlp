\section{Computer-Assisted Proof of Small Subgraph Preimage Uniqueness}\label{sec:algo-proof}


% We want to show the above condition holds for $\delta<\frac{2d-4}{2d-1}$. Since there are finitely many constant-size graphs, in principle we can enumerate all possible graphs and calculate the probability of them appearing. However the number of such graphs is still huge. So we use an algorithm to search over all graphs and upper bound their probability.


% \subsection{Search Algorithm that Enumerates Possible 2-Connected Components}
Recall that as discussed in Section~\ref{sec:main-idea-reconstruction}, in order to prove Theorem~\ref{thm:delta-lower}, all we need to do is check the non-existance of ambiguous graphs. Specifically, we need to prove the following.
\lowerbound*

In this section, we provide a proof for the claim, with computer assistance.
% \begin{lemma}
%     When $d=3$ and $\delta<2/5$ or when $d=4,5$ and $\delta<1/2$, any constant-size  graph $G_a$ of size at most $\frac{2\binom{d}{2}}{\frac{d-1}{d+1}-\delta} $with non-unique minimum cover satisfy
%     \[
%     \Pr(G_a\subset \pG)=o_n(1),
%     \]
% \end{lemma}
\paragraph{Depth First Search (DFS) over Hypergraphs.}
First, instead of searching over graphs $G_a$, we can search over preimages of the graphs, i.e., hypergraphs. The claim in Lemma~\ref{lem:delta-lowerbound} is equivalent to 
\begin{gather*}
    \text{For any sub-hypergraph $ \shG$ where $\cli(\proj(\shG))$ is 2-connected and  $\Pr(\shG\subset \rhG)=\Omega_n(1)$,}\\
\text{$\shG$ has unique minimum preimage.}
\end{gather*}
We will prove a sufficient condition for the claim to hold by replacing $\Pr(\shG\subset \rhG)=\Omega_n(1)$ with $\E X_{\shG}=\Omega_n(1)$.
% \begin{gather*}
%     \text{For any hypergraph $ \shG$ where $\cli(\proj(\shG))$ is 2-connected and  $\E X_{\shG}=\Omega_n(1)$,}\\
% \text{$\shG$ has unique minimum preimage}
% \end{gather*}
Therefore, we want to search over all hypergraphs $ \shG$ where $\cli(\proj(\shG))$ is 2-connected and  $\E X_{\shG}=\Omega_n(1)$. If all such graphs have unique minimum preimage, then Lemma~\ref{lem:delta-lowerbound} is proven. 


Lemma~\ref{lem:grow-contain-all-components} implies that it suffices to consider $\grow^{(t)}([d])$ for $t=1,2,\ldots$ in order to capture all $\shG$ where $\cli(\proj(\shG))$ is 2-connected. 
Based on $\grow$, we define a depth-first-search tree $T$ where nodes of the tree are hypergraphs from $\cup_{t\ge 1}\grow^{(t)}([d])$. The root of $T$ is a hypergraph with a single $d$-hyperedge $[d]$. For any node $\shG$ on the tree, its children is all hypergraphs in $\grow(\shG)$, as shown in Figure~\ref{fig:DFS}.
We therefore start with a hypergraph with a single hyperedge and perform a depth first search over the tree.



The depth of the search is also bounded by a $2/(\frac{d-1}{d+1}-\delta)=O_n(1)$, as Lemma~\ref{lem:exp-dec} tells us that each growth step decreases the expected number of appearances by a polynomial factor. 

\begin{figure}
    \centering
\begin{tikzpicture}[every node/.style={draw,circle}, 
                    level distance=2cm,
                    level 1/.style={sibling distance=6cm},
                    level 2/.style={sibling distance=3cm}]
    % Root node
    \node (G1) {$\shG_1$}
    % Level one children
    child { node (G2) {$\shG_2$}
        % Level two children of G2
        child { node {$\shG_4$} }
        child { node {$\shG_5$} }
    }
    child { node (G3) {$\shG_3$}
        % Level two children of G3
        child { node {$\shG_6$} }
        child { node {$\shG_7$} }
    };


    % Circle around G1's children and label
    \begin{scope}[on background layer]
    \node[ellipse, draw=none, fit=(G2) (G3) , inner sep=1mm,fill=violet!30] {$\grow(\shG_1)$};
    \end{scope}

\end{tikzpicture}
    \caption{Depth First Search Tree over Graphs. The children of any hypergraph $\shG$ is the set $\grow(\shG)$.}
    \label{fig:DFS}
\end{figure}

\paragraph{Pruning by Bounding the Expected Number of Appearances.}
The benefit of using the structure of $\grow$ to search instead of an arbitrary order is that we can do the following pruning.
By Lemma~\ref{lem:exp-dec}, during the depth first search, children always have a smaller expected number of appearances than the parent.
So if we reach a hypergraph $\shG$ with $o_n(1)$ expected number of appearance, we can stop the search on this branch in the depth first search tree, as any children of the graph will also have $o_n(1)$ expected number of appearances. 

% $\E X_{[d]} =\Theta(n^{1+\delta})$ so 
% \red{Explain the algorithm, the search follows the branching process in Lemma~\ref{lem:branching}}

% As in the proof of Lemma~\ref{lem:branching}, given a 2-connected component 
\paragraph{Improve the Root of the DFS Tree.}
We use the lemma below to further narrow down the search.
Given Lemma~\ref{lem:ambiguous-graph-has-connected-preimage}, instead of searching from $[d]$, we can start the search from \emph{two} hyperedges that overlap on at least 2 vertices. This turns out to dramatically decrease the depth of the search. 

\begin{lemma}\label{lem:ambiguous-graph-has-connected-preimage}
    Fix $\delta<\frac{2d-4}{2d-1}$. If $G_a$ is an ambiguous graph, then either
    \begin{itemize}
        \item $\Pr(G_a\subset \pG)=o_n(1)$ or 
        \item one of the minimum preimages of $G_a$ contains two hyperedges that share two vertices.
    \end{itemize}
\end{lemma}
\begin{proof}
Let $G_a$ be an ambiguous graph and for every minimum preimage of $G_a$, any two hyperedges in the graph share at most one vertex. 

Let $\shG_1$ and $\shG_2$ be two minimum covers of $G_a$. Let $\hE_c = \hE_{\shG_1}\cap \hE_{\shG_2}$, $\hE_1 = \hE_{\shG_1}\backslash \hE_{\shG_2}$ and $\hE_2 = \hE_{\shG_2}\backslash \hE_{\shG_1}$. Then we have $\hE_{\shG_1} $ is partitioned to $\hE_c$ and $\hE_1$, $\hE_{\shG_2} $ is partitioned to $\hE_c$ and $\hE_2$. $\hE_1\cap \hE_2=\emptyset$.
Since any two hyperedges in the graph share at most one vertex, $\proj(\hE_c)\cap \proj(\hE_1) = \emptyset$. So
\[
\proj(\hE_1) = G_a\backslash \proj(\hE_c) = \proj(\hE_2)\,.
\]
We will show that $\proj(\hE_1)$ appears in $\pG$ with $o_n(1)$ probability, then by Lemma~\ref{lem:exp-dec}, $G_a$ also appears in $\pG$ with $o_n(1)$ probability.

Suppose $|\hE_1| = k$, then so is $|\hE_2|$, as $\shG_1$ and $\shG_2 $ have the same size. The total degree of $\hE_1$ is therefore $dk$. For any node $v\in V(\hE_1)$, $v$ is in one of the hyperedges $\he$ in $\hE_2$. There are $d-1$ edges in $\proj(\hE_1)$ between $v$ and other nodes in $\he$. All $d-1$ edges are included in some hyperedges in $\hE_1$. But they cannot be in a single hyperedge in $\hE_1$, otherwise that hyperedge would be $\he$, contradicting with $\hE_1\cap \hE_2=\emptyset$. So $v$ has degree at least 2 in $\hE_1$. Therefore, 
\[
v_{\hE_1} \le \frac{dk}{\text{minimum degree}} \le dk/2\,.
\]
So by Lemma~\ref{lem:number-appearance}, 
\[
\Pr(\hE_1\subset \rhG) \le \E X_{\hE_1} = \Theta_n(n^{v_{\hE_1}}p^{e_{\hE_1}}) = O_n(n^{dk/2}p^k) = O_n(n^{-d/2+1+\delta})\,.
\]
Using $\delta<\frac{2d-4}{2d-1}$, the above is $O_n(n^{2-d/2-3/(2d-1)})$. When $d=3$, this is $O_n(n^{-1/10}) = o_n(1)$. When $d\ge 4$, this is $O_n(n^{-3/(2d-1)}) = o_n(1)$.
\end{proof}

\paragraph{The Search Algorithm}
The algorithm is given as Algorithm~\ref{alg:search}.
Lines 14 to 16 enumerates two hyperedges that overlap on at least 2 vertices and starts the DFS search from this hypergraph. The first input of the procedure is the hypergraph itself, the second input of the procedure is the expected number of the hypergraphs in $\rhG$. For two hyperedges that overlap on $k$ vertices, the expectation is $\Theta_n(n^{2d-k}p^{2})$, by Lemma~\ref{lem:number-appearance}.

In the procedure DFS, Lines 3 to 8, we check whether the graph is ambiguous. Lines 9 and 10 examine the expected number of the current graph $\shG$ in $\rhG$. If it is $o_n(1)$, this branch of the search can be pruned as any graph growing from $\shG$ would have vanishing probability of appearing. Line 11 to Line 13 continues to search from all graphs in $\grow(H)$.
% \begin{lemma}\label{lem:algo-correct}
% All hypergraph $\shG$ that $\E X_{\shG}=\Omega_n(1)$ and $\proj(\shG)$ is ambiguous will be searched by Algorithm~\ref{alg:search}.
% \end{lemma}
\paragraph{Results of Algorithm~\ref{alg:search}.}
So far we have shown that Algorithm~\ref{alg:search} will find all  hypergraphs $\shG$ such that $\E X_{\shG}=\Omega_n(1)$ and $\proj(\shG)$ is ambiguous. What remains is running the code. 

When $d=3$ and $\delta=2/5$, the algorithm only finds one ambiguous graph, $\aG$. For $d=4,5$ and $\delta=1/2$, the algorithm did not find any ambiguous graph. Thus we proved Lemma~\ref{lem:delta-lowerbound}.

Unfortunately despite our efforts to optimize the search algorithm, it still takes a rather long time when $d\ge 4$ and $\delta>1/2$. We leave it as an open problem to determine the correct threshold $\delta^*$ when $d=4,5$. We conjecture that the threshold matches the upper bound $\frac{2d-4}{2d-1}$.

\begin{algorithm}
\caption{Search Algorithm}\label{alg:search}
\begin{algorithmic}[1] 
\State Input: $d,p=n^{-d+1+\delta}$.


% \end{algorithmic}
% \end{algorithm}

% \begin{algorithm}
% \begin{algorithmic}[1]
\Procedure{DFS}{a hypergraph, $\shG$; and the expected number of $\shG$ in $\rhG$, $U$}

% \State $SizeofMinPreimage\leftarrow 0$
\For{$k=1,2,\cdots$}
\State Find all preimages of $\proj(\shG)$ with $k$ hyperedges.
\If{One such preimage is found}
\State \textbf{continue}
\EndIf
\If{Two such preimage is found}
\State \textbf{output:} An ambiguous graph $\proj(\shG)$ is found.
\EndIf
\EndFor

\If{$U=O_n(1)$}
\State \textbf{return}
\EndIf
\For{$\he\in N(\shG)$}
\For{$\shG'\in \grow(\shG,\he)$}
\State DFS($\shG'$,$ U\cdot n^{v_{\shG'}-v_{\shG}}p^{e_{\shG'}-e_{\shG}}$)
\EndFor
\EndFor
\EndProcedure
\For{$k=2,\cdots,d-1$} 
    \State Let $\shG_k$ be the graph with two hyperedges that share $k$ vertices.
    \State DFS($\shG_k$, $n^{2d-k}p^2$).
\EndFor
\end{algorithmic}
\end{algorithm}