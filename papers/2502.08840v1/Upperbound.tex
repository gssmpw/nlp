\section{Impossibility when $d\le 5$ and $\delta \ge \frac{2d-4}{2d-1}$}\label{sec:impossibility}
% MAP is optimal in the sense that it succeeds with the highest probability among all possible recovery algorithms. But it still might not achieve exact recovery if the failure probability does not vanish. 

Recall that in Lemma~\ref{lem:bad-graph-in-isolation}, we reduced the problem of proving impossibility of exact recovery to finding ambiguous graphs. 
\badgraphinisolation*
In this section we will identify an appropriate ambiguous graph and then use Lemma~\ref{lem:bad-graph-in-isolation} to prove the following theorem.
\begin{theorem}\label{thm:upper}
    When $d=3,4,5$, for any $\delta\ge \frac{2d-4}{2d-1}$, exact recovery is information theoretically impossible.
\end{theorem}
% We will use the following Lemma to characterize a sufficient condition of when does any algorithm fail on exact recovery.
% \begin{lemma}\label{lem:bad-graph-in-isolation}
%     For any constant-size graph $G_a$ with non-unique minimum preimage, and any recovery algorithm $\cA$,
%     \[
%     \Pr(\cA(\pG)\ne\rhG) \ge \frac{1}{2}\Pr(\cli(G_a)\text{ is a 2-connected component of }\cliG )
%     \]
% \end{lemma}
% \begin{proof}
% By Lemma~\ref{lem:union-min-preimage}, a minimum preimage of $\pG$ is given by the union of the minimum preimages of every 2-connected component of $\cliG$. Therefore, when $\cli(G_a)$ is a 2-connected component of $\cliG$, the minimum preimage of $\cliG$ is not unique. So no matter which hypergraph $\cA^*$ chooses, it has at least 1/2 probability of making mistake. In other words,
% \[
% \Pr(\cA(\pG)\ne\rhG|\cli(G_a)\text{ is a 2-connected component of }\cliG) \ge 1/2\,.
% \]
% The lemma follows from Bayes rule. 
% \end{proof}
\subsection{Ambiguous Graph and Its Properties}

Let us list the properties we need for an ambiguous graph $\aG$ to prove the theorem:
\begin{enumerate}
    \item The graph should be ambiguous, i.e., it should have at least two minimum preimages. We will prove this property in Lemma~\ref{lem:bad-graph-non-unique-preimage}. 
    \item The graph appears in $\pG$ with constant probability, $\Pr(\aG\subset \pG)=\Omega_n(1)$. We will prove this  property in Lemma~\ref{lem:bad-graph-contain-probability}.
    \item The graph appears as a 2-connected component in $\cliG$ with constant probability.
    % \\ $\Pr(\cli(\aG)\text{ is a 2-connected component of }\cliG)=\Omega_n(1).$ 
    We will prove this property in Corollary~\ref{cor:bad-graph-probability}.
\end{enumerate}

Recall that ambiguous graph was defined in Defn.~\ref{def:ambiguous}. We will construct a specific such graph.

\begin{definition}[Ambiguous Graph $\aG$]\label{def:aG}
   We define the graph $\aG$ as the union of the following $2d$ cliques:
\begin{itemize}
    \item the clique $u_1,v_1,v_2,\cdots, v_{d-1}$, denoted by $\he_1$,
    \item the clique $u_2,v_1,v_2,\cdots, v_{d-1}$, denoted by $\he_2$,
    \item for any $1\le i\le d-1$, the clique $u_1,v_i,w_i^{(1)},w_i^{(2)},\cdots, w_i^{(d-2)}$, denoted by $\he_i^w$,
    \item and for any $1\le i\le d-1$, the clique $u_1,v_i,z_i^{(1)},z_i^{(2)},\cdots, z_i^{(d-2)}$, denoted by $\he_i^z$.
\end{itemize}
Let $S_1 = \{\he_1^w,\cdots, \he_{d-1}^w\}$ and $S_2 = \{\he_1^z,\cdots, \he_{d-1}^z\}$. See Figure~\ref{fig:ambiguous-graph} for a drawing of the graph when $d=3$. The intuition is to create a set of size $d-1$, $v_1,v_2,\cdots, v_{d-1}$ ($\{2,3\}$ in Figure~\ref{fig:ambiguous-graph}), that can be assigned to two possible hyperedges, both yielding a minimum preimage. 
\end{definition}


Now we prove that this graph satisfies the properties stated above.

\subsection{$\aG$ Satisfies Three Desired Properties}

\paragraph{Property 1: The Graph $\aG$ is Ambiguous.}
 This is shown in the following lemma.
\begin{lemma}\label{lem:bad-graph-non-unique-preimage}
 The graph $\aG$ from Defn.~\ref{def:aG} has two minimum preimages (so it is ambiguous).
\end{lemma}
\begin{proof}
Any preimage of $\aG$ must contain hyperedges in $S_1$ and $S_2$ as each $w_i^{(j)}$ ($z_i^{(j)}$) is only included in one clique. To include edges among $v_1,v_2,\cdots, v_{d-1}$, either $\he_1$ or $\he_2$ needs to be included in the preimage. Both $S_1\cup S_2\cup \{\he_1\}$ and $S_1\cup S_2\cup \{\he_2\}$ are valid preimages, so both are minimum preimages for $\aG$.
\end{proof}

\paragraph{Property 2: The Graph $\aG$ Appears with Probability $\Omega_n(1).$}
The next lemma shows that $\aG$ appears in $\pG$ with non-negligible probability using Lemma~\ref{lem:subgraph_threshold}.
\begin{lemma}\label{lem:bad-graph-contain-probability}
Let $\aG$ be as in Defn.~\ref{def:aG}. For any $ \delta\ge \frac{2d-4}{2d-1}$, 
    \[
    \Pr(\aG\subset \pG)=\Omega_n(1)\,.
    \]
\end{lemma}

\begin{proof}
Let us focus on one possible cover of $\aG$, $S_1\cup S_2\cup \{\he_1\}$,
\[
\Pr(\aG\subset \pG)\ge \Pr(S_1\cup S_2\cup \{\he_1\}\subset \hE_\rhG)\,.
\]
Recall in Lemma~\ref{lem:subgraph_threshold}, the probability of a hypergraph $\shG$ appear as a subgraph of $\rhG$ is described by  
\[
m(\shG) = \max_{\shG'\subset \shG}\frac{e_{\shG'}}{v_{\shG'}}\,.
\]
To use Lemma~\ref{lem:subgraph_threshold}, we need to calculate $m(S_1\cup S_2\cup \{\he_1\})$. The calculation is given in the following lemma.
\begin{lemma}\label{lem:density_bad_graph}
Let $S_1,S_2$, and $h_1$ be as in the Defn.~\ref{def:aG} of graph $G_{a,d}$ above. Then
% We calculate the value of function $m$ for the preimage $S_1\cup S_2\cup \{\he_1\}$ of the ambiguous graph $\aG $ in this lemma.
    \[m(S_1\cup S_2\cup \{\he_1\}) = \frac{2d-1}{2d^2-5d+5}\,.\]
\end{lemma}
The calculation showing the lemma can be found in Appendix~\ref{sec:proof_density_bad_graph}.
Given Lemma~\ref{lem:density_bad_graph}, we have $\Pr(S_1\cup S_2\cup \{\he_1\}\subset \hE_\rhG)=\Omega_n(1)$ when 
\[
p=\Omega_n(n^{-\frac{2d^2-5d+5}{2d-1}})\,,
\]
i.e., when $\delta\ge \frac{2d-4}{2d-1}$. 
% Now by Lemma~\ref{lem:branching}, $\cli(S_1\cup S_2\cup \{\he_1\})$ has no 2-neighbor with at least constant probability. So $\aG$ is forms a 2-connected component with constant probability.
\end{proof}

\paragraph{Property 3: The Graph $\aG$ Forms a 2-connected Component with Probability $\Omega_n(1)$.}
What remains to be shown is that with probability $\Omega_n(1)$, not only does $\aG$ appears, but also $\cli(\aG)$ is a 2-connected component. In other words, we want to show that $\cli(\aG)$ has no 2-neighbors in $\cliG$, as shown in the following lemma.

\begin{lemma}\label{lem:branching}
For any $\hE_1\subset \binom{[n]}{d}$ with $|\hE_1|=O_n(1)$,
\[
    \Pr\b(\nei{\cliG}(\cli(\hE_1))\ne \emptyset | \hE_1\subset \hE_{\rhG}\b) =
    \begin{cases}
        O_n(n^{-(\frac{d-1}{d+1}-\delta)}) &\text{if }\delta<\frac{d-1}{d+1}\\
        1-\Omega_n(1) &\text{if }\delta=\frac{d-1}{d+1}
    \end{cases}
    \,.
\]
Recall that $\cli(\hE_1)=\cli(\proj(\hE_1))$.
\end{lemma}
The proof of the lemma can be found in Appendix~\ref{sec:brahcing}. The proof idea is similar to Lemma~\ref{lem:exp-dec}, where we consider all possible ways for a 2-neighbor to appear.

Since any preimage of $\aG$ has constant size, we can conclude that for any $\hE_a\subset \binom{[n]}{d} $ that is a preimage  of $\aG$, 
\[
\Pr(\nei{\cliG}(\cli(\hE_a))
= \emptyset|\hE_a\subset \hE_\rhG)=\Omega_n(1)\,.
\]
Combining this with Lemma~\ref{lem:bad-graph-contain-probability} yields the following Corollary.
\begin{corollary}\label{cor:bad-graph-probability}
    For any $\frac{2d-4}{2d-1}\le \delta\le \frac{d-1}{d+1}$, 
    \[
    \Pr\b(\cli(\aG)\text{ is a 2-connected component of }\cliG\b)=\Omega_n(1)\,.
    \]
\end{corollary}
% \begin{proof}
% Let us first prove that $\aG$ appears when $\delta\ge \frac{2d-4}{2d-1}$, then show that it form a 2-connected component using Lemma~\ref{lem:branching}. 
% \end{proof}
Now we can prove Theorem~\ref{thm:upper}.

\subsection{Proof of Theorem~\ref{thm:upper}}
We can use Lemma~\ref{lem:bad-graph-in-isolation} by setting the graph $G_a$ to be $\aG$. By Lemma~\ref{lem:bad-graph-non-unique-preimage} and Corollary~\ref{cor:bad-graph-probability}, $\aG$ is graph with two minimum preimages and appears as a 2-connected component of $\cliG$ with constant probability. So for any algorithm $\cA$,
\[
\Pr(\cA(\pG)\ne \rhG)\ge \Omega_n(1)
\]
when $\frac{2d-4}{2d-1}\le \delta\le \frac{d-1}{d+1}$. By monotonicity stated in Lemma~\ref{lem:monotone}, we prove the theorem for $d=4$ and $5$.
And for $d=3$, this shows the impossibility when $2/5\le \delta\le 1/2$. The case of $d=3$, $\delta\ge 1/2$ is already shown in Theorem~\ref{thm:large-d}.
\qed
