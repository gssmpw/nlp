
\section{MAP is Efficient for Sparse Hypergraphs}


% This allows us to output a minimum pre-image

% \begin{lemma}\label{lem:component-constant-size}
%     For any $\delta<\frac{d-1}{d+1}$, any 2-connected component of $\cliG$ has size at most $\frac{2}{\frac{d-1}{d+1}-\delta}=O_n(1)$ with high probability.
% \end{lemma}
% \begin{proof}
% For any $\he_1\in \binom{[n]}{d}$,
% let $C(\he_1)$ be the 2-connected component of $\cliG$ that contains $\he_1$ if $\he_1\in\hE_\rhG$ and $\emptyset$ otherwise. Let us control the probability that a 2-connected component grows from size $t-1$ to size $t$.
% % , we let $C(\he_1)=\emptyset$ if $\he_1\not \in \cliG$. For any $t\in\mathbb{Z}^+$,
% \[
% \begin{split}
%     &\Pr\b(|C(\he_1)|\ge t\b| |C(\he_1)|\ge t-1\b)\\
%     &=\Pr\b(\nei{\cliG}(\{\he_1,\cdots, \he_{t-1}\})\ne\emptyset \b| \exists h_1,\cdots, h_{t-1}\in\cliG \text{ s.t. }\{\he_1,\cdots, \he_{t-1}\}\text{ is 2-connected}\b)\\
%     &= \Pr\b(\nei{\cliG}(\{\he_1,\cdots, \he_{t-1}\})\ne\emptyset \b| \exists \text{a minimal cover $\hE$ of } h_1,\cdots, h_{t-1}\in\binom{[n]}{d}\text{ s.t. }   \{\he_1,\cdots, \he_{t-1}\}\text{ is 2-connected}, \hE\subset \hE_\rhG\b)
% \end{split}
% \]
% For any $t=O_n(1)$, any minimal cover of $h_1,\cdots, h_{t-1}$ also has size $O_n(1)$. So by Lemma~\ref{lem:branching}, 
% \[
% \Pr\b(|C(\he_1)|\ge t\b| |C(\he_1)|\ge t-1\b)=O_n(n^{-(\frac{d-1}{d+1}-\delta)})\,.
% \]
% So we have
% \[
% \begin{split}
% &\Pr(|C(\he_1)|\ge k)\\
% &=\Pr(\he_1\in \hE_\rhG)\prod_{t=2}^k\Pr\b(|C(\he_1)|\ge t\b| |C(\he_1)|\ge t-1\b)\\
% &=O_n(n^{-d+1+\delta}\cdot n^{-(k-1)(\frac{d-1}{d+1}-\delta)})\,.
% \end{split}
% \]
% Union bound over all $\he_1\in\binom{[n]}{d}$ and set $k=1+\frac{2}{\frac{d-1}{d+1}-\delta}$, we get 
% \[
% \Pr\B(\exists\he_1\in\binom{[n]}{d} , |C(\he_1)|\ge k\B) \le \binom{n}{d}O_n(n^{-d+1+\delta-(k-1)(\frac{d-1}{d+1}-\delta)}) = O_n(n^{\delta-1})=o_n(1)\,.
% \]
% This means any 2-connected component of $\cliG$ has size at most $k-1=\frac{2}{\frac{d-1}{d+1}-\delta}$ with probability $1-o_n(1)$.
% % Let  $\he_i$ to be a random hyperedge in $\nei{\cliG}(\{\he_0,\he_1,\cdots, \he_{i-1}$
% % We have
% % \[
% % |C(\he_0)|\ge t\Leftrightarrow \exists \he_1\in \nei{\cliG}(\he_0), \exists \he_2\in \nei{\cliG}(\{\he_0,\he_1\})\cdots, \exists \he_t\in \nei{\cliG}(\{\he_0,\he_1,\cdots, \he_{t-1}\})
% % \]
% % So 
% % \[
% % \Pr(|C(\he_0)|\ge t) = \prod_{i=1}^t \Pr(\exists \he_t\in \nei{\cliG}(\{\he_0,\he_1,\cdots, \he_{t-1}\}) | )
% % \]
% \end{proof}

\nbyp{I don't understand this lemma. It seems that $Proj(\mathcal{E}_1)$ is missing, but also what is the role of $\mathcal{E}_0$?}
\cg{rewrite this part}
\begin{lemma}\label{lem:branching}
For any $\hE_1\subset \binom{[n]}{d}$ with $|\hE_1|=O_n(1)$,
\[
    \Pr\b(\nei{\cliG}(\cli(\hE_1))\ne \emptyset | \hE_1\subset \hE_{\rhG}\b) =
    \begin{cases}
        O_n(n^{-(\frac{d-1}{d+1}-\delta)}) &\text{if }\delta<\frac{d-1}{d+1}\\
        1-\Omega_n(1) &\text{if }\delta=\frac{d-1}{d+1}
    \end{cases}
    \,.
\]


\end{lemma}


% We can show that the above bound is tight, \cg{delete}
% \begin{lemma}\label{lem:branching-tight}
%     When $\delta>\frac{d-1}{d+1}$, there exists $\hE_1\subset \binom{[n]}{d}$ such that 
%     \[
%     \Pr\b(\nei{\cliG}(\cli(\hE_1))\ne \emptyset | \hE_1\subset \hE_{\rhG}\b) = 1-o_n(1)\,.
%     \]
% \end{lemma}
\begin{proof}
% Note that the event $\nei{\cliG}(\cli(\hE_1))\ne \emptyset$ is monotonically increasing in all hyperedges, by Harris inequality, we have 
% \[
% \Pr\b(\nei{\cliG}(\cli(\hE_1))\ne \emptyset | \hE_1\subset \hE_{\rhG}, \hE_0\cap \hE_{\rhG}=\emptyset\b) \le \Pr\b(\nei{\cliG}(\cli(\hE_1))\ne \emptyset | \hE_1\subset \hE_{\rhG}\b)\,.
% \]
% In other words, we can consider the case when $\hE_0$ is empty without loss of generality. 

\end{proof}



