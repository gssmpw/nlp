\section{Deferred Proofs of Lemmas}
\label{s:deferred}



\subsection{Proof of Lemma~\ref{lem:monotone}}\label{sec:monotone}
Let $p_1 = n^{-d+1+\delta_1}$ and $p_2 = n^{-d+1+\delta_2}$. Let $\rhG_1$ and $\rhG_2$ be the random hypergraphs when hyperedge density are $p_1$ and $p_2$ respectively. Assume we have a black-box algorithm that exactly recovers $\rhG_2$. We will use it  to  recover $\rhG_1$ from $\proj(\hE_{\rhG_1})$.

The key observation is that a dense graph is the union of two sparse graphs. Specifically, let $p_3$ satisfy $p_1+(1-p_1)p_3 = p_2$, and $\rhG_3$ be a random hypergraph sampled from $\rhG(n,d,p_3)$. In the union  $\hE_{\rhG_1}\cup \hE_{\rhG_3}$, each hyperedge is included with probability $p_1+(1-p_1)p_3=p_2$. We have $\hE_{\rhG_1}\cup \hE_{\rhG_3}$ and $\hE_{\rhG_2}$ follows the same distribution.
% \[
% \cL( \hE_{\rhG_1}\cup \hE_{\rhG_3}) = \cL(\hE_{\rhG_2})\,.
% \]

Now given $\proj(\hE_{\rhG_1})$, we generate a sample of $\rhG_3$. Using the black box, we can recover $\hE_{\rhG_1}\cup \hE_{\rhG_3}$ from
\[
\proj (\hE_{\rhG_1}\cup \hE_{\rhG_3}) = \proj(\hE_{\rhG_1}) \cup \proj(\hE_{\rhG_3})\,
\]
with high probability.
By union bound over all possible hyperedges, probability that $\hE_{\rhG_1}$ and $\hE_{\rhG_3}$ has non-empty overlap is upper bounded by
\[
\Pr(\hE_{\rhG_1}\cap \hE_{\rhG_3}\ne \emptyset)\le 
\binom{n}{d}p_1p_3\le n^d p_1p_2\le n^{-d+2+\delta_1+\delta_2}=o_n(1)\,.
\]
Here we used $p_1<p_3<p_2$ in the first inequality. The last equality follows from $d\ge 4$ and $\delta_1<1$, $\delta_2\le 1$. So with high probability, $\hE_{\rhG_1}$ and $\hE_{\rhG_3}$ do not have common hyperedges, and we can recover $\hE_{\rhG_1}$ by subtracting $\hE_{\rhG_3}$ from $\hE_{\rhG_1}\cup \hE_{\rhG_3}$.
\hfill\qed



\subsection{Proof of Lemma~\ref{lem:union-min-preimage}}\label{sec:union-min-preimage}
% \begin{proof}[]
Let  $E_i = \proj(C_i)$, and $V_i$ be the node set of $C_i$. Recall that for any two sets of hyperedges, $\proj(C_1\cup C_2) = \proj(C_1)\cup \proj( C_2)$. 


\paragraph{Step 1: $\cup_i \proj(C_i)$ forms a partition of $ E_p$.} 
First, note that because $(C_i)_i$ partitions the hyperedges in $\cliG$, and then by definition of $\cliG$, 
$$
\bigcup_i \proj(C_i) = \proj(\cliG) \,.
$$
Next, if any $\{a,b\}\in C_i\cap C_j$, then there are hyperedges $h_i\in C_i$ and $h_j\in C_j$ each containing $\{a,b\}$. But then $h_i$ and $h_j$ are 2-connected, so $C_i$ and $C_j$ cannot be two separate components. 

\paragraph{Step 2: LHS $\subseteq$ RHS.} 
 Consider an arbitrary preimage $\hG\in \prim (\pE)$.
 We have $\hG\subseteq \cliG$, because $\hG$ is a clique cover of $E_p$ and $\cliG$ is the maximal clique cover. Since $\{C_i\}_i$ partitions the hyperedges in $\cliG$, 
% any $h\in \hG$ is in some $C_i$ and hence
% denoting by $\hG[V_i]$ the induced sub-hypergraph on node set $V_i= V(C_i)\subseteq[n]$, we see that 
% $h\in \hG[V_i]$
% and
% $\cliG\subseteq \cup_{i\in [m]} \cliG[V_i]$
$$H
% =
% \bigcup_{\he\in \hG}\he 
=  \bigcup_{i\in [m]}\hG\cap C_i\,.$$
% , there must exist $C_i$ such that $\he\in C_i$. 
% Denoting by $\hG[V_i]$ the induced sub-hypergraph on node set $V_i= V(C_i)\subseteq[n]$, 
We will now argue that $\hG\cap C_i\in \prim(\proj(C_i))$, i.e., $\proj(\hG\cap C_i)= \proj(C_i)$. 
It suffices to prove that $\proj(\hG\cap C_i)\supseteq \proj(C_i)$. Consider any edge $e=\{a,b\}\in \proj(C_i)\subseteq E_p$. 
Now,
$\hG$ has some hyperedge $h$ containing the endpoints of $e$, because $\proj(\hG) = E_p$. 
Secondly, 
$C_i$ has all hyperedges in $\cliG$ containing $e$,
by step 1,
and therefore contains $h$. It follows that $\proj(\hG\cap C_i)$ contains $e$.

% Then there is an $h\in C_i$ with $\{a,b\}\subset h$, and no other $\proj(C_j)$ contains $\{a,b\}$ for $j\neq i$ by step 1. But $\{a,b\}\in \proj(\hG)$, and all of the hyperedges $h_1,\dots, h_r \in \hG$ must also be in $C_i$ because they all contain $\{a,b\}$.


\paragraph{Step 3: RHS $\subseteq$ LHS.} 
 We want to show that any union on the right-hand side is in $\prim(E_p)$.
This follows immediately from step 1: if $H = \cup_{i=1}^m H_i$ for $\hG_i\in \prim(\proj(C_i))$, then $\proj(H) = \cup_i \proj(H_i) = \cup_i \proj(C_i) = E_p$. \qed

\subsection{Proof of Lemma~\ref{lem:subgraph_threshold}}\label{sec:subgraph_threshold}
Let us first state a lemma that will be used later.
\begin{lemma}\label{lem:non-zero}
    For a real-valued random variable $X$,
    \[
    \Pr(X\ne 0) \ge \frac{(\E X)^2}{(\E X)^2+\var(X)}\,.
    \]
\end{lemma}
\begin{proof}
Let $Y=\ind{X\ne 0}$. By Cauchy-Schwartz, 
\[
(\E XY)^2\le \E [X^2]\E[Y^2]\,.
\]
Since $XY=X$, the left hand side is equal to  $(\E X)^2$. Since $Y$ takes 0,1 value, $\E[Y^2]=\E[Y]=\Pr(X\ne 0)$. We have
\[
\Pr(X\ne 0)\ge \frac{(\E X)^2}{\E [X^2]} = \frac{(\E X)^2}{(\E X)^2+\var(X)}\,.
\]
\end{proof}

We prove the second and third claim using second moment method

Let $\shG_1,\shG_2,\cdots,\shG_t$ be all copies of such sub-hypergraph on the complete graph of $[n]$, we have
\[
t=\binom{n}{v_\shG}\frac{(v_\shG)!}{\aut(\shG)}=\Theta_n(n^{v_\shG})\,.
\]
Here $\aut(\shG)$ is the number of automorphisms of $\shG$.
Let $I_i$ be the indicator that $\shG_i$ is in $\rhG$. And $X_\shG = \sum_{i=1}^tI_i$ be the number of such event happening. We have
\[
\E [X_\shG] = tp^{e_\shG}(1-p) = \Theta_n(n^{v_\shG}p^{e_\shG})\,.
\]
% \red{needs editing}
And 
\[
\var(X_\shG) = \sum_{i=1}^t\sum_{j=1}^t \cov(I_iI_j) = 
\sum_{i=1}^t\sum_{j=1}^t (\Pr(I_i=I_j=1)-\Pr(I_i=1)\Pr(I_j=1))\,.
\]
We have $\Pr(I_i=1) = \Pr(I_j=1) = p^{d}(1-p)$. 
Consider pairs $(\shG_i,\shG_j)$ such that $\shG_i\cap \shG_j = \shG'$, where $\shG'\subset \shG$ is a sub-hypergraph of $\shG$ with non-empty edge set.
\begin{align*}
\var(X_\shG) &= O_n\B( \sum_{\substack{\shG' \subseteq \shG,\\ e_{\shG'} > 0}} n^{2v_\shG - v_{\shG'}} \left( p^{2e_\shG - e_{\shG'}} - p^{2e_\shG} \right) \B) \\
&= O_n\B( n^{2v_\shG} p^{2e_\shG} \sum_{\substack{\shG' \leq \shG,\\ e_{\shG'} > 0}} n^{-v_{\shG'}} p^{-e_{\shG'}} \B).
\end{align*}
Then from Lemma~\ref{lem:non-zero}, 
\[
\Pr(X_\shG\ne 0) \ge \frac{(\E X_\shG)^2}{(\E X_\shG)^2+\var(X_\shG)} = \frac{1}{1+O_n(\sum_{\substack{\shG' \leq \shG,\\ e_{\shG'} > 0}} n^{-v_{\shG'}} p^{-e_{\shG'}})}\,.
\]
We can easily check that for any $\shG'\subset \shG$, $\frac{e_\shG'}{v_\shG'}\le \frac{e_\shG}{v_\shG}$. So when $p=\Theta_n(n^{-1/m(\shG)}) = \Omega_n (n^{-v_{\shG'}/e_{\shG'}})$, we have
\[
\sum_{\substack{\shG' \subseteq \shG,\\ e_{\shG'} > 0}} n^{-v_\shG'} p^{-e_{\shG'}} = \sum_{\substack{\shG' \leq \shG,\\ e_\shG' > 0}} O_n(n^{-v_{\shG'}} n^{v_{\shG'} }) = O_n(1)\,.
\]
This means $\Pr(X_\shG\ne 0) = \Omega_n(1)$.

When $p=\omega_n(n^{-1/m(\shG)}) = \omega (n^{-v_\shG/e_\shG})$, 
we have
\[
\sum_{\substack{\shG' \leq \shG,\\ e_\shG' > 0}} n^{-v_\shG'} p^{-e_\shG'} = \sum_{\substack{\shG' \leq \shG,\\ e_\shG' > 0}} o_n(n^{-v_{\shG'}} n^{v_{\shG'} }) = o_n(1)\,.
\]
This means $\Pr(X_\shG\ne 0) = 1-o_n(1)$.

Next we prove the first claim. Let $\shG'\subset \shG$ be the sub-hypergraph that $\frac{e_{\shG'}}{v_{\shG'}} = m(\shG)$.
\[
\Pr(\shG\subset \rhG)\le \Pr(\shG'\subset \rhG)\le \E X_{\shG'} = \Theta_n(n^{v_{\shG'}}p^{e_{\shG'}})\,.
\]
When $p=o_n(n^{-1/m(\shG)})$ the above is $o_n(1)$.
\hfill\qed

\subsection{Proof of Lemma~\ref{lem:no-ambiguous}}\label{sec:proof-no-ambiguous}
Let $C_1,\cdots, C_m$ be all the 2-connected components in $\cliG=\cli(\pG)$. Let $V_i$ be the node set of $C_i$, $r_i$ be the size of the minimum preimage of $C_i$. The success probability can be written in terms of the posterior distribution.
\[
\Pr(\cA^*(\pG)=\rhG) = \E_{\pG}[p_{\rhG|\pG}(\cA^*(\pG)|\pG)] \,.
\]
We will show that this posterior probability is close to 1 with high probability.
Recall the posterior distribution is
\[
p_{\rhG|\pG}(H|\pG) =\frac{\ind{\proj(\hE_H) = \pE}p_{\rhG}(\hE_H) }{p_\pG(\pE)}\propto  \ind{\hE_H \in \prim( \pE)} \b(\frac{p}{1-p}\b)^{|\hE_H|}\,,
\]
By Lemma~\ref{lem:union-min-preimage}, $\hE_H \in \prim( \pE)$ is equivalent to $H\cap C_i\in \prim(\proj(C_i))$ for all $i$. Recall that $\hE_H=\cup_i(H\cap C_i)$. So the posterior distribution can be written as
\[
p_{\rhG|\pG}(H|\pG) \propto  \prod_{i=1}^m \B(\ind{H\cap C_i\in \prim(\proj(C_i))} \b(\frac{p}{1-p}\b)^{e(H\cap C_i)}\B)\,.
\]
Here $e(H\cap C_i)$ stands for the number of hyperedges in $H\cap C_i$. We have
\[
p_{\rhG|\pG}(\cA^*(\pG)|\pG) = \prod_{i=1}^m \frac{ \b(\frac{p}{1-p}\b)^{r_i}}{\sum_{H'\in \prim(\proj(C_i))} \b(\frac{p}{1-p}\b)^{e(H')}} =  \prod_{i=1}^m \frac{ 1}{\sum_{H'\in \prim(\proj(C_i))} \b(\frac{p}{1-p}\b)^{e(H')-r_i}}\,.
\]

By Lemma~\ref{lem:component-constant-size}, with high probability any $C_i$ has size at most $(2^d+1)/(\frac{d-1}{d+1}-\delta)$. So $|V_i|\le d(2^d+1)/(\frac{d-1}{d+1}-\delta)$. By the assumption of the lemma, any ambiguous graph $G_a$ with at most $(d2^d+1)/(\frac{d-1}{d+1}-\delta)=O_n(1)$ number of nodes has $o_n(1)$ probability of appearing in $\pG$. So by union bound, the probability of any such $G_a$ appearing in $\pG$ is $o_n(1)$.
Therefore, with probability $1-o_n(1)$, $\proj(C_i)$ is not ambiguous for any $i$. This means there is only one hypergraph in $\prim(\proj(C_i))$ with size $r_i$. So with probability $1-o_n(1)$, 
\[
\sum_{H'\in \prim(\proj(C_i))} \b(\frac{p}{1-p}\b)^{e(H')-r_i} \le 1+|\prim(\proj(C_i))|\frac{p}{1-p}=1+O_n(p)\,.
\]
The last equality is because $C_i$ is of size $O_n(1)$, so the number of possible preimages is also $O_n(1)$. Taking this back to the expression of posterior probability, we get
\[
p_{\rhG|\pG}(\cA^*(\pG)|\pG) = (1-O_n(p))^m=1-O_n(mp)\,.
\]
$m$ is the number of 2-connected component, which is bounded by the total number of hyperedges in $\rhG$. On the other hand, the total number of hyperedges in $\rhG$ follows binomial distribution $\bino(\binom{n}{d},p)$. By Chernoff bound, it is $\Theta(n^dp)$ with probability $1-o_n(1)$. So we have 
\[
p_{\rhG|\pG}(\cA^*(\pG)|\pG) = 1-o_n(1)-O_n(n^dp^2)=1-o_n(1)-O_n(n^{-d+2+2\delta})\,.
\]
Since $d\ge 3$, $\delta < \frac{d-1}{d+1}\le \frac{1}{2}$, we have $-d+2+2\delta<0$. So with high probability 
\[
p_{\rhG|\pG}(\cA^*(\pG)|\pG) = 1-o_n(1)\,,
\]
and therefore
\[
\Pr(\cA^*(\pG)=\rhG) = \E_{\pG}[p_{\rhG|\pG}(\cA^*(\pG)|\pG)] =1-o_n(1)\,.
\]
\hfill\qed


\subsection{Proof of Lemma~\ref{lem:density_bad_graph}}\label{sec:proof_density_bad_graph}
Recall the definition of $m(S_1\cup S_2\cup \{\he_1\})$ is 
\[
\max_{K\subset (S_1\cup S_2\cup \{\he_1\})} \frac{e_K}{v_K}
\]
Below we show that this is reached by the whole hypergraph, i.e., when $K=S_1\cup S_2\cup \{\he_1\}$. 
% Note that the optimal choice of $V'$ that reaches  $m(S_1\cup S_2\cup \{\he_1\})$ must be a union of the hyperedges, so we only consider such $V'$ below. 
% Let $L$ be the set of hyper edges in $S_1$ that is a subset of $V'$, $R$ be the set of hyper edges in $S_2$ that is a subset of $V'$.
Let $L$ be the set of hyperedges in $S_1$ that is a subset of $K$, $R$ be the set of hyperedges in $S_2$ that is a subset of $K$.

Case 1: $\he_1\not\in K$, $R=\emptyset$. 
\[
\frac{e_K}{v_K} = \frac{|L|}{(d-1)|L|+1}\le \frac{d-1}{(d-1)^2+1}\,.
\]
The maximum is achieved when $L=S_1$.
The case where $R\ne \emptyset$ and $L=\emptyset$ is symmetric. 

Case 2: $\he_1\not\in K$, $L,R\ne \emptyset$. Without loss of generality, assume $|L|\ge |R|$.
\[
\begin{split}
    \frac{e_K}{v_K}&= \frac{|L|+|R|}{(d-1)(|L|+|R|)+2 - \#[i:\he_i^w\in L,\he_i^z\in R]}\\
    &\le \frac{2|L|}{2(d-1)|L|+2-|L|}\\
    &\le \frac{2d-2}{(d-1)(2d-2)+2-(d-1)}\,.
\end{split}
\]
The maximum is achieved when $L=S_1$ and $R=S_2$. Easy to see that maximum in case 2 is larger than the maximum in case 1. 

Case 3: $\he_1\in K$, $R=\emptyset$.
\[
\frac{e_K}{v_K} = \frac{1+|L|}{d+|L|(d-2)}\le \frac{d}{d+(d-1)(d-2)}\,.
\]
The maximum is achieved when $L=S_1$.
The case where $R\ne \emptyset$ and $L=\emptyset$ is symmetric. 

Case 4: $\he_1\in K$, $L,R\ne \emptyset$. Without loss of generality, assume $|L|\ge |R|$.
\[
\begin{split}
    \frac{e_K}{v_K}&= \frac{1+|L|+|R|}{d+1+|L|(d-2)+|R|(d-2)}\\
    &\le \frac{2d-1}{d+1+(2d-2)(d-2)} = \frac{2d-1}{2d^2-5d+5}\,.
\end{split}
\]
The maximum is achieved when $L=S_1$ and $R=S_2$. It is easy to see that the maximum in case 4 is larger than the maximum in case 3. The maximum in case 4 has one more hyperedge than the maximum in case 2, which does not increase the number of nodes. Therefore, case 4 is the maximum overall and $m(S_1\cup S_2\cup \{\he_1\}) = \frac{2d-1}{2d^2-5d+5}$. \hfill\qed

\subsection{Proof of Lemma~\ref{lem:branching}}\label{sec:brahcing}

The high-level approach is to union bound over all possible hyperedges in $\nei{\cliG}(\cli(\hE_1))$. Let $V(\hE_1)$ be the set of nodes that are incident to one of the hyperedges in $\hE_1$. Further, let $A_k$ be the set of hyperedges that has $k$ nodes in $V(\hE_1)$, i.e., 
\[
A_k \defeq \B\{\he\in \binom{[n]}{d}\b| |\he\cap V(\hE_1)|=k\B\}\,.
\]
Here $k$ is at least 2 and at most $d$. The size of $A_k$ is at most
\[
|A_k|\le \binom{|V(\hE_1)|}{k}\binom{n}{d-k}=O_n(n^{d-k})\,.
\]
We wish to union bound the probability that any hyperedge $\he\in A_k$ being present in $\cliG.$

Let $\he\in A_k$. For $\he$ to appear in $\cliG$, every edge in $\proj(\he)\backslash \proj(\hE_1)$ should be covered in at least one hyperedge in $\hE_{\rhG}$.
Now let us look at the possible ways for this to happen.  
For any $\he\in A_k$, let $\cS^\he \defeq \{S_1^\he, S_2^\he, \cdots, S_m^\he\}$ be the set of subset of $\he$ such that $\proj(S_i)\not\subset \proj(\hE_1)$ and $\proj(S_i)\ne \emptyset$. 
If a hyperedge covers an edge in $\proj(\he)\backslash \proj(\hE_1)$, it must intersect with $h$ at one of the sets in $\cS^\he$.
Now let event $A_i^\he$ be the event that at least one hyperedge in 
\[
\hE_i^\he\defeq \B\{\he'\in \binom{[n]}{d}\b| \he'\cap \he = S_i^\he\B\}\,
\]
is in $\hE_\rhG$. Note that $\{S_i^\he\}_i$ are disjoint set of hyperedges, so $\{A_i^\he\}_i$ are independent events.
We have
\begin{equation}\label{eq:decom-inclusion}
\begin{split}
&\Pr(\he\in \cliG\b|\hE_1\in\hE_\rhG)\\ 
&=  \sum_{I\subset [m ]} \1\{(\proj(\he)\backslash \proj(\hE_1)) \subset \cup_i\proj(S_i^\he)\} \Pr\b((\cap_{i\in I}A_i^\he)\cap(\cap_{i\in [m ]\backslash [I]}(A_i^\he)^c) |\hE_1\subset \hE_\rhG\b)\\
&\le \sum_{I\subset [m ]} \1\{(\proj(\he)\backslash \proj(\hE_1)) \subset \cup_i\proj(S_i^\he)\} \Pr(\cap_{i\in I}A_i^\he|\hE_1\subset \hE_\rhG)\\
&=  \sum_{I\subset [m ]} \1\{(\proj(\he)\backslash \proj(\hE_1)) \subset \cup_i\proj(S_i^\he)\} \prod_{i\in I}\Pr(A_i^\he|\hE_1\subset \hE_\rhG)\,.
\end{split}
\end{equation}
The inequality is by inclusion of events, and the second equality is by independence of $\{A_i^\he\}_i$.
Now we show an upperbound on $\Pr(A_i^\he|\hE_1\subset \hE_\rhG)$. There are $\binom{n-|S_i^\he|}{d-|S_i^\he|}$ hyperedges in $\hE_i^\he$. And none of them are in $\hE_1$ since $\proj(S_i)\not\subset \proj(\hE_1)$. Therefore, 
\[
\Pr(A_i^\he|\hE_1\subset \hE_\rhG) = 1-(1-p)^{\binom{n-|S_i^\he|}{d-|S_i^\he|}} = O_n(pn^{d-|S_i^\he|}) = O_n(n^{-|S_i^\he|+1+\delta})\,.
\]
Note that $|S_i^\he|\ge 2$ and $\delta< 1$, this is always $o_n(1)$.
Since the number of terms in \eqref{eq:decom-inclusion} is bounded by $2^m\le 2^{2^d}$ which is $O_n(1)$,  we have
\begin{equation}\label{eq:braching-hyperedge}
\Pr(\he\in \cliG\b|\hE_1\in\hE_\rhG) = O_n\B(\max_{ \substack{I\subset[m]:\\(\proj(\he)\backslash \proj(\hE_1)) \subset \cup_i\proj(S_i^\he)}} n^{-\sum_{i\in I}(|S_i^\he|-1-\delta)}\B)\,.
\end{equation}
% \cg{improve notation}
Because $\he\in A_k$,  we know $\proj(\he)\cap \proj(\hE_1)$ is a subset of a size-$k$ clique in $\he$. So the above probability can be further relaxed to $O_n(n^{-g_k(\delta)})$. Recall
\begin{equation}\label{eq:braching-hyperedge-2}
g_k(\delta) = \min_{\substack{I\subset [m]:\\ \b(\proj(\he)\backslash \binom{U_\he}{2}\b) \subset \cup_i\proj(S_i^\he)}} \sum_{i\in I}(|S_i^\he|-1-\delta)\,.
\end{equation}
Here $U_\he$ is a size-$k$ subset of $\he$. Note that any clique in $\cS^\he$ has size at least 2, $g_k(\delta)$ is always non-negative.
Therefore, by union bound over all hyperedges in $A_k$ for any $2\le k\le d$, 
\[
\Pr\b(\nei{\cliG}(\cli(\hE_1))\ne \emptyset | \hE_1\subset \hE_{\rhG}\b) = \sum_{k=2}^d |A_k|O_n(n^{-g_k(\delta)}) = O_n(n^{\min_k \{g_k(\delta)+k-d\}})\,.
\]

Given the bound for $\min_k \{g_k(\delta)+k-d\}$ in Lemma~\ref{lem:cover-bound}, we have for any $\delta<\frac{d-1}{d+1}$,
\[
\Pr\b(\nei{\cliG}(\cli(\hE_1))\ne \emptyset | \hE_1\subset \hE_{\rhG}\b) = O_n(n^{-\frac{d-1}{d+1}+ \delta})\,.
\]

Now we prove the case when $\delta = \frac{d-1}{d+1}$. In this case, instead of using union bound, we need to be more careful and consider the correlation between different hyperedges using Harris Inequality. 

\begin{lemma}[Harris Inequality \cite{harris1960lower}]\label{lem:harris}
Let $A$ and $B$ be two events in the probability space defined by $\rhG$. If both $A$ and $B$ are increasing with respect to all possible hyperedges in $\binom{[n]}{d}$, then
\[
\Pr(A|B)\ge \Pr(A)\,.
\]
\end{lemma}


Let $A=\{\he_1,\he_2,\cdots,\he_m\}$ be the set of all hyperedges in $\nei{\cliG}(\cli(\hE_1))$, $A = \cup_{k=1}^{d|\hE_1|}A_k$. We have
\[
\begin{split}
    &\Pr\b(\nei{\cliG}(\cli(\hE_1))= \emptyset | \hE_1\subset \hE_{\rhG}\b)\\
    &=\prod_{i=1}^m \Pr\b( \he_i\not\in \hE_\rhG| \forall j<i,\he_j\not\in \hE_\rhG, \hE_1\subset \hE_{\rhG}\b)\\
    &\ge \prod_{i=1}^m \Pr\b( \he_i\not\in \hE_\rhG|  \hE_1\subset \hE_{\rhG}\b) \\
    &= \prod_{k=2}^{d}\prod_{h\in A_k}\Pr\b( \he\not\in \hE_\rhG|  \hE_1\subset \hE_{\rhG}\b)
\end{split}
\]
Here we used that $\he_i\not\in \hE_\rhG$ is decreasing event for any $i$ and applied Harris Inequality in Lemma~\ref{lem:harris}. Using the bound we get in \eqref{eq:braching-hyperedge} and \eqref{eq:braching-hyperedge-2}, we have
\[
\Pr\b(\nei{\cliG}(\cli(\hE_1))= \emptyset | \hE_1\subset \hE_{\rhG}\b) \ge \prod_{k=2}^{d} \b(1-O_n(n^{-g_k(\delta)})\b)^{|A_k|} = \prod_{k=2}^{d} \exp \B(-O_n(n^{d-k-g_k(\delta)})\B)\,.
\]
By Lemma~\ref{lem:cover-bound}, when $\delta=\frac{d-1}{d+1}$, $d-k-g_k(\delta)\ge 0$. So $\Pr\b(\nei{\cliG}(\cli(\hE_1))= \emptyset | \hE_1\subset \hE_{\rhG}\b)=\Omega_n(1)$, thus proving the second case stated in Lemma~\ref{lem:branching}. \hfill\qed

