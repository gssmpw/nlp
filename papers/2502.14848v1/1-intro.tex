\section{Introduction}
\label{sec:intro}

Large Language Models (LLMs) have demonstrated impressive capabilities in code generation~\citep{cassano2023type, li2023starcoder, roziere2023code, hui2024qwen2}, enabling complex tasks such as mathematical computation~\citep{zhou2023solving, wang2023mathcoder}, tabular reasoning~\citep{chen2022program}, and visual understanding~\citep{suris2023vipergpt, choudhury2023zero}. By generating executable code, LLMs extend their functionality beyond pre-trained parameters, empowering agent-based tasks through frameworks like AutoGen~\citep{wu2023autogen} and CodeActAgent~\citep{wang2024executable}. However, these approaches treat each program as isolated, limiting the reuse of previously generated functional modules across different tasks.

To overcome this, recent studies~\citep{wang2023voyager, cai2023large, qian2023creator, yuan2023craft, stengel2024regal} have focused on developing reusable tool libraries derived from tasks. Despite these advancements, existing methods face significant challenges: (1) \textbf{Toolset Redundancy and Inefficiency}: Many methods generate redundant tools, resulting in bloated libraries that hinder reuse. For example, Voyager~\citep{wang2023voyager} lacks a deduplication mechanism, while CREATOR~\citep{qian2023creator} and CRAFT~\citep{yuan2023craft} create one tool per task, leading to large, repetitive libraries. Regal~\citep{stengel2024regal}, though aiming for simplicity, produces libraries limited to basic arithmetic wrappers. (2) \textbf{Limited Generalizability}: Most methods are validated in narrow settings, restricting their broader applicability. For instance, Voyager~\citep{wang2023voyager} is confined to Minecraft environments, while others~\citep{cai2023large, qian2023creator, yuan2023craft, wang2024trove, stengel2024regal} focus exclusively on code generation tasks.

\begin{figure}[t]
% \vskip 0.2in
\begin{center}
\centerline{\includegraphics[width=1\linewidth]{toolnumber-all.png}}
\vskip -0.1in
\caption{Performance of \ours\ in Minecraft. \ours\ continually discovers new Minecraft items and skills during exploration, significantly outperforming other methods.}
\label{fig:toolnumber-all}
\end{center}
\vskip -0.3in
\end{figure}
In this paper, we propose \ours\ (\underline{\textit{G}}raph-based \underline{\textit{A}}daptive \underline{\textit{T}}ool \underline{\textit{E}}volution), a framework where two agents, the Task Solver and Tool Manager, dynamically interact with an Adaptive Tool Graph. The Task Solver iteratively extracts tool requirements. The Tool Manager then retrieves tools from the graph using a Graphrank Retrieval method, assembles new tools from existing ones, and refines the graph through pruning and merging. This design sets \ours\ apart from existing tool-making frameworks and addresses the three challenges we discussed as follows: (1) By assembling tools from existing ones instead of generating them from scratch, we improve tool creation efficiency. Additionally, dynamic operations such as merging and pruning ensure the tool library remains concise and manageable. (2) Through the cooperation of two agents, \ours\ dynamically extracts tool requirements based on the current environment and task, converting them into tools. This enables the system to adapt effectively to a wide range of tasks.



We evaluate \ours\ across both open-ended and closed-ended tasks. Our results demonstrate that \ours\ achieves 3.5× better item discovery and 4.3× faster tech tree mastery in Minecraft compared to the previous SOTA (state-of-the-art) method. Additionally, \ours\ outperforms baselines by 5–32\% in agent tasks and surpasses other tool-making methods by an average of 12.6\% in code generation tasks. Our analysis highlights the adaptive evolution of the tool graph across tasks. Compared to other tool-making methods, \ours\ strikes the best trade-off in terms of tool library size, complexity, and redundancy. Our contributions can be summarized as follows:
\begin{itemize}
     \setlength{\itemsep}{0.1em} % Adjust the spacing here
     \item \ours\ is the first method to construct a tool graph by leveraging the invocation relationships between tools, enabling tool evolution and efficient tool retrieval.
     \item \ours\ introduces an agent framework that effectively manages the toolset, maintaining a balanced size with complex tools while avoiding redundancy.
     \item \ours\ achieves generalizability, attaining SOTA performance across various scenarios, including open-ended and closed-ended tasks.
\end{itemize}
   
\begin{figure*}[t]
\begin{center}
\centerline{\includegraphics[width=1\linewidth]{toolgraph.pdf}}
\caption{\ours\ consists of two agents: the Task Solver and the Tool Manager, which interact with an Adaptive Tool Graph. Key processes include Tool Requirement, Generation, Pruning, Merging, and Tool Graph Updates.}
\label{fig:toolgraph}
\end{center}
\vskip -0.3in
\end{figure*}



