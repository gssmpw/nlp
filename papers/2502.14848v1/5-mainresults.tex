\section{Main Results}
\label{sec:mainresult}


\begin{figure*}[ht]
% \vskip 0.2in
\begin{center}
\centerline{\includegraphics[width=0.88\linewidth]{bucket_and_sword.png}}
% \vskip -0.2in
\caption{Zero-shot Generalization on Unseen Tasks. The figure visualizes the intermediate progress of each method on two tasks. See Figure \ref{fig:diamon and compass} for the other two tasks. ReAct and Reflexion are excluded from the plot due to their lack of meaningful progress.}
\label{fig:unseen_task}
\end{center}
\vskip -0.3in
\end{figure*}

\paragraph{\ours\ Expands Tech Tree Mastery and Exploration in Open-Ended Tasks.}
\ours\ outperforms the previous SOTA Voyager method in terms of the number of unique items and generates rarer items (Figure \ref{fig:toolnumber-all}). In Minecraft tech tree mastery, \ours\ unlocks the wooden, stone, and iron milestones 23.0×, 13.4×, and 7.5× faster than baselines, respectively (Table \ref{tab:Open-Ended Task}). Notably, \ours\ creates the Diamond Tool 4.34× faster than Voyager and navigates 2.7× longer distances, successfully exploring diverse terrains (Figure \ref{fig:trial1-map}).
\paragraph{\ours\ Enables Self-Improvement on GPT-4o and Boosts Performance on Other Models in Close-Ended Tasks.}
Table \ref{tab:Agent-Ended and signal Task} demonstrates \ours’s effectiveness across both open-source and closed-source models in close-ended tasks. \ours\ facilitates self-improvement on GPT-4o and boosts performance in other models. On average, GPT-4o shows a 5\% improvement in close-ended tasks, while other models achieve gains of 10.03\% and 9.23\% on agent and code sub-tasks, respectively. For instance, GPT-3.5-turbo-1106 improves by 32.4\% on Textcraft, and Qwen2.5-Coder-Instruct sees a 19.07\% increase on Date. These results underscore the adaptability and effectiveness of \ours\ in enhancing performance across various tasks and models.
% Our method shows significant improvements on the TextCraft and InfiAgent-DABench benchmarks, consistently outperforming baselines. As shown in Table \ref{tab:Agent-Ended and signal Task}, in TextCraft, \ours\ achieves up to 10\% improvement over the best-performing baselines, with gains ranging from 5\% to 32\% across models like \textit{Qwen2.5-7B-Instruct} and \textit{GPT-3.5-turbo-1106}. In DABench, \ours\ achieves an average improvement of over 5.6\%, surpassing Plan-Execution on complex queries and consistently outperforming Reflexion in challenging tasks. This highlights the robustness and adaptability of our approach.
\paragraph{\ours\ Achieves Significant Improvements Over Other Tool-Making Methods in Close-Ended Tasks.}
As shown in Table \ref{tab:Agent-Ended and signal Task}, \ours\ outperforms other tool-making methods by an average of 10.03\%. Some methodS, such as LATM~\citep{cai2023large} and CRAFT~\citep{yuan2023craft}, perform worse than the baseline model without additional tools, suggesting that their tool libraries may not be as effective. Contrary to the conclusions of CREATOR~\citep{qian2023creator} and CRAFTT~\citep{yuan2023craft}, which separate tool making from tool calling, our results demonstrate that directly generating code yields better performance. 

\begin{figure*}[ht]
\begin{center}
    \begin{minipage}{0.48\textwidth}
        \vspace*{0pt}
        \centering
        \includegraphics[width=\linewidth]{toolnet-main.png}
        \caption{Evolution of the tool graph. We visualize the progression of the tool graph in the Minecraft task, capturing snapshots every 40 steps. The complete evolution for other tasks is provided in the Appendix \ref{subsec:tool-graph}. For clarity, basic tools are excluded from the visualization, as they are generally connected to tools at every level.}
        \label{fig:evlove}
    \end{minipage} 
    \hfill
    \begin{minipage}{0.48\textwidth}
        \vspace*{0pt}
        \centering
        \includegraphics[width=\linewidth]{tools.png}
        \vskip -0.1in
        \caption{Layered Node Distribution of the Tool Graph. "Tool Number" represents the quantity of tools at different levels. The "cum ops" refers to the cumulative number of operations, including function calls.}
        \label{fig:tools}
    \end{minipage}
\end{center}
\vskip -0.3in
\end{figure*}



