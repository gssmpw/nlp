% This is samplepaper.tex, a sample chapter demonstrating the
% LLNCS macro package for Springer Computer Science proceedings;
% Version 2.21 of 2022/01/12
%
\documentclass[runningheads]{llncs}
%
\usepackage[T1]{fontenc}
% T1 fonts will be used to generate the final print and online PDFs,
% so please use T1 fonts in your manuscript whenever possible.
% Other font encondings may result in incorrect characters.
%
\usepackage{graphicx}
\usepackage{colortbl}
\usepackage{pseudocode} % Environment for specifying algorithms in a natural way
\usepackage{multirow}
\usepackage{enumitem}
\usepackage{url}
\usepackage{tikz}
\usepackage{balance}
\usepackage[skip=1pt]{caption}
\usepackage{booktabs}
\usepackage{cite}

\usepackage{algorithm, algpseudocode}
\algnewcommand\algorithmicforeach{\textbf{for each}}
\algdef{S}[FOR]{ForEach}[1]{\algorithmicforeach\ #1\ \algorithmicdo}
\newcommand*{\definitionautorefname}{Definition}
\newcommand*{\algorithmautorefname}{Algorithm}\usepackage{pbox}

\newcommand{\mybox}[1]
{
\vspace{1.8mm}
\noindent \hspace{-1mm} 
\setlength\fboxsep{1mm}
\fbox{\parbox{\dimexpr\linewidth-2\fboxsep-2\fboxrule}{\itshape #1}}
% \vspace{1mm}
}
\usepackage{amsmath}

\newcommand{\para}[1]{\vspace{0.1mm}\noindent\textbf{#1}.} 
\newcommand{\orion}{kumar2015learning}
\newcommand{\santoku}{kumar_demonstration_2015}
\newcommand{\hamlet}{kumarJoinNotJoin2016}
\newcommand{\f}{schleichLearningLinearRegression2016}
\newcommand{\morpheus}{chen2017towards}
\newcommand{\hamletplusplus}{shahAreKeyForeignKey2017}
\newcommand{\acdc}{khamisACDCInDatabase2018}
\newcommand{\aidaone}{dsilvaAIDAAbstractionAdvanced2018}
\newcommand{\aida}{dsilvaAIDAAbstractionAdvanced2018, dsilvaMakingRDBMSData2019}
\newcommand{\morpheusfi}{liEnablingOptimizingNonlinear2019}
\newcommand{\lmfao}{schleichLayeredAggregateEngine2019}
\newcommand{\hadad}{alotaibiHADADLightweightApproach2021}
\newcommand{\amalur}{haiAmalurNextgenerationData2022}
\newcommand{\chengone}{chengNonlinearModelsNormalized2019}
\newcommand{\chengtwo}{chengEfficientConstructionNonlinear2021}
\newcommand{\dima}{sunDimaDistributedInmemory2017}    
\newcommand{\morpheuspy}{kumar_morpheuspy_nodate}
\newcommand{\trinity}{justoPolyglotFrameworkFactorized2021}

\newcommand{\redshade}[1]{%
  \begingroup
  \setlength{\fboxsep}{0pt}%  
  \colorbox{red!15}{\reducedstrut#1\/}%
  \endgroup
}
% Used for displaying a sample figure. If possible, figure files should
% be included in EPS format.
%
% If you use the hyperref package, please uncomment the following two lines
% to display URLs in blue roman font according to Springer's eBook style:
%\usepackage{color}
%\renewcommand\UrlFont{\color{blue}\rmfamily}
%\urlstyle{rm}
%
\begin{document}
%
\title{Ilargi: a GPU Compatible Factorized ML Model Training Framework}
\author{Wenbo Sun\inst{1} \and
Rihan Hai\inst{1}}
\institute{Delft University of Technology \\
\email{w.sun-2@tudelft.nl} \\
\email{r.hai@tudelft.nl}}
%
%\titlerunning{Abbreviated paper title}
% If the paper title is too long for the running head, you can set
% an abbreviated paper title here
%
% \institute{Delft University of Technology}
% \email{w.sun-2@tudelft.nl}
% %
% \authorrunning{F. Author et al.}
% First names are abbreviated in the running head.
% If there are more than two authors, 'et al.' is used.
%
% \institute{Princeton University, Princeton NJ 08544, USA \\
% \email{\{abc,lncs\}@uni-heidelberg.de}}
%
% \author{}
\maketitle              % typeset the header of the contribution
%
\begin{abstract}
The machine learning (ML) training over disparate data sources traditionally involves materialization, which can impose substantial time and space overhead due to data movement and replication. Factorized learning, which leverages direct computation on disparate sources through linear algebra (LA) rewriting, has emerged as a viable alternative to improve computational efficiency. However, the adaptation of factorized learning to leverage the full capabilities of modern LA-friendly hardware like GPUs has been limited, often requiring manual intervention for algorithm compatibility. This paper introduces \emph{Ilargi}, a novel factorized learning framework that utilizes matrix-represented data integration (DI) metadata to facilitate automatic factorization across CPU and GPU environments without the need for costly relational joins. \emph{Ilargi} incorporates an ML-based cost estimator to intelligently selects between factorization and materialization based on data properties, algorithm complexity, hardware environments, and their interactions. This strategy ensures up to 8.9x speedups on GPUs and achieves over 20\% acceleration in batch ML training workloads, thereby enhancing the practicability of ML training across diverse data integration scenarios and hardware platforms. To our knowledge, this work is the very first effort in GPU-compatible factorized learning.


% \keywords{First keyword  \and Second keyword \and Another keyword.}
\end{abstract}

\section{Introduction}\label{sec:introduction}
% -- Outline
% ---- LLMs are popular
% ---- There're many stakeholders in the training and inference loop
% ---- Adversaries in the training loop are a problem -- malpractice, poisoning
% ---- Also, showing compliance
% ---- Need a framework to prove the integrity of the pipeline
% ---- Enter Atlas

% ---- LLMs are popular
In recent years, machine learning (ML) models, have become increasingly popular.
The pervasive use of large language models (LLMs), in particular, and multi-stakeholder
involvement in model creation and deployment exacerbate security and privacy risks.
These considerations are emphasized by the global nature and the complexity of
large-scale ML deployments with different lifecycle stages:
%gathering and sanitizing the data from different sources,
%training and inferencing across many data centers,
%compliance with local laws or corporate policies.

% ---- There're many stakeholders in the training and inference loop
%Additionally, different stages of the ML development pipeline come with their own stakeholders:
\begin{enumerate}[label=\arabic*)]
    \item Collection and sanitation of a \emph{training} dataset from several public and proprietary sources.
    %\item Solicitation and facilitation of training.
    \item Provisioning of the training environment (hardware and software).
    \item Execution of training across many data centers.
    \item Construction of a \emph{testing} dataset from several sources, and the evaluation.
    \item Deployment and use of the model for inference that is compliant with local laws or corporate policies.
    %\item Use of the model in compliance with local laws or corporate policies.
\end{enumerate}

% ---- Adversaries in the training loop are a problem -- malpractice, poisoning
Each of these stages is vulnerable to malicious or dishonest parties.
For example, data can be poisoned~\cite{biggio2012poisoning,carlini2024poisoning} during collection or training.
Service providers executing outsourced training can shorten or omit critical steps to reduce their cost.
Model providers can serve smaller models in SaaS, or even distribute malicious ones.

% ---- Also, showing compliance
On the other hand, responsible model builders and other stakeholders may be incentivised or required to provide security and trust guarantees.
They may want to prove low bias in their training data, offer easily verifiable performance claims, or guarantee end-to-end integrity of the model creation in high risk domains.

% ---- Need a framework to prove the integrity of the pipeline
To address these challenges, it is necessary to guarantee the integrity of the entire ML lifecycle --
beginning with the data, through the training, and finally, the evaluation and deployment.
Was the data modified?
Did the hardware and software environment adhere to the specification?
Did the contractor follow the specified training procedure?
Can I trust the evaluation?
How can I guarantee that I am interacting with the intended model?
These are example questions that showcase the breadth of the involved challenges that must be tackled to provide end-to-end security.

% --- Enter Atlas
In this work, we introduce \atlas, a framework for enhancing the security and transparency of the lifecycle of ML models.
\atlas establishes the baseline of fundamental components and capabilities needed for comprehensive provenance tracking
at each stage of the ML lifecycle.
Subsequently, \atlas defines the core integrity requirements for verifiable ML lifecycle transparency.
We provide a reference implementation that instantiates \atlas using hardware-based security mechanisms -- with trusted execution environment (TEE),
including attestations.% , and comprehensive metadata-based provenance tracking.
%Our implementation satisfies all \atlas requirements.

We claim the following contributions:
\begin{enumerate}[label=\arabic*.]\label{sec:introduction:contributions}
    \item We introduce \atlas, a framework designed for end-to-end ML lifecycle transparency.
    \item We instantiate \atlas using TEEs and metadata-based provenance tracking.
    \item We evaluate our \atlas prototype through two case studies:
        \begin{enumerate*}[label=\arabic*)]
            \item fine-tuning of a BERT model~\cite{lin2023metabert, lin2023metabertimpl};
            \item fine-tuning of a bge-reranker model~\cite{chen2023bge}
        \end{enumerate*}.
\end{enumerate}

%\msm{revise: Integrate this motivation into intro}
%Organizations frequently leverage pre-trained models, outsource training processes, and integrate components from multiple sources,
%making it difficult to verify the authenticity and trustworthiness of their ML systems. This complexity is further compounded
%by the potential for malicious modifications at various stages of the model lifecycle, from data preparation through deployment.
%The involvement of various third parties in ML model development and deployment
%creates critical challenges in ensuring supply chain integrity.
%
%While Software Bills of Materials (SBOMs) and AI Bills of Materials (AI BOMs) provide basic inventory tracking for model components,
%they fall short in addressing the dynamic nature of ML pipelines. These approaches typically offer point-in-time snapshots but
%fail to capture the complex transformations, fine-tuning operations, and runtime modifications that characterize modern ML workflows.
%Additionally, they lack cryptographic guarantees about the integrity of recorded information and cannot effectively track the provenance
% of model weights and training data.
%
% These approaches demonstrate the growing importance of ML supply chain security.
% However, they are typically applied in an ad-hoc fashion, highlighting the need
% for a more integrated approach that combines comprehensive lineage tracking,
% strong cryptographic properties, and practical integration capabilities with existing ML development and deployment pipelines.
%
%A comprehensive solution requires not just documentation of components, but verifiable evidence of their origins,
%transformations, and integrity throughout the entire model lifecycle. This need has driven interest in more robust
%provenance tracking mechanisms that can:
%
%\begin{itemize}
%\item Provide cryptographic proof of model lineage
%\item Track and verify all pipeline transformations
%\item Maintain tamper-evident records of training processes
%\item Ensure integrity of model artifacts across organizational boundaries
%\end{itemize}
%
%Several existing tools and frameworks
%commonly focusing on different components of the model lifecycle and provenance tracking.
%While these solutions offer valuable capabilities, they often address only specific parts of the end-to-end ML
%supply chain rather than providing comprehensive coverage.
%\msm{end-revise}
%
%\todo{add discussion of EU-CRA AI Act requirements for model documentation and FDA guidelines for AI/ML in healthcare}

%The remainder of this paper is organized as follows:
%in Section~\ref{sec:background-related} we provide an overview of the necessary background, and the related work;
%Section~\ref{sec:problem} presents the challenge of providing integrity in the ML pipeline, the threat model, and the system assumptions;
%in Section~\ref{sec:framework} we present \atlas -- our framework for providing ML integrity;
%Section~\ref{sec:implementation} covers implementation details;
%in Section~\ref{sec:eval}, we show that \atlas is effective across three dimensions: training overhead $<8\%$, the verification time increases linearly with the size of the model, and it is compatible with PyTorch and Tensorflow;
%in Section~\ref{sec:casestudies} we present the case studies;
%in Section~\ref{sec:discussion} we discuss additional considerations for \atlas,
%and Section~\ref{sec:conclusion} concludes the paper and provides directions for future work.

%--------------------------------------------------------------------------------
%	for VLDB AMALUR (cite ICDE vision) || for SIGMOD use name Ilargi
%--------------------------------------------------------------------------------



\vspace{-3mm}
\section{Ilargi: ML Over Disparate Sources}
\label{sec:system}

% \rihan{A: note we assume data is preprocessed, or rules is given; Replacing nulls as zeros, }
% To train models effectively over disparate data sources, understanding the relationships between those sources is crucial. These relationships, including schema mapping and data matching, are collectively referred to as data integration metadata.

% In this section, we introduce the Ilargi system, which utilizes matrix-represented DI metadata to facilitate model training without the need for materialization. Figure~\ref{fig:amalur} illustrates Ilargi's workflow, which accepts three inputs: source datasets, a user-defined ML algorithm (e.g., Python scripts), and metadata about the datasets, including their data integration specifics such as schema mapping and data matching. The output from Ilargi is a fully trained ML model. Ilargi has three main functions. First, given data integration metadata, Ilargi generates their matrix-based representation  (Sec. \ref{sec:matrixGen}). Such matrix-based representations enable a unified computation of data transformation and linear algebra operations (Sec. \ref{s:operations}). Second, if the scenario is centralized training, the estimator decides to factorize or materialize (Sec.~\ref{sec:cost}). Finally, the ML model is trained in the chosen strategy, i.e., materialization or factorization.
\vspace{-3mm}
As illustrated in Fig.~\ref{fig:amalur}, \emph{Ilargi} first converts the inputs, DI metadata and source tables, into sparse matrices, storing them in a compressed format. This transformation facilitates a GPU-compatible representation of schema mappings and data matching  (Sec.~\ref{sec:matrixGen}). \emph{Ilargi} then performs LA rewriting to both ML model training algorithms and data integration tasks, reconfiguring the process into a sequence of linear algebra operations (Sec.~\ref{s:operations}). The cost estimator subsequently predicts the optimal training method—either factorization or materialization—based on data characteristics, algorithmic complexity, and the hardware environment (Sec.~\ref{sec:cost}). Finally, \emph{Ilargi} trains the model with the optimal training strategy recommended by the cost estimator.

In this section, we introduce how \emph{Ilargi} utilizes matrix-represented data integration metadata to enable efficient factorized model training on both GPUs and CPUs. 


 




% \para{Implementation} The ML models supported in our system are linear regression, logistic regression, and Gaussian NMF, which can be extended. 
% \rihan{A: Which system details to add?}

\vspace{-3mm}
\subsection{DI Metadata as Matrices} 
\label{sec:matrixGen}
\vspace{-2mm}
Preparing disparate training data for ML models requires understanding the relationships between the data sources and bridging them, which is achieved primarily through data integration tasks. The relationships are normally described by following metadata: \cite{doan2012principles}: $i)$ mappings between different source schemata, i.e., schema matching and mapping \cite{rahm2001survey, fagin2009clio} and $ii)$ linkages between data instances, i.e., data matching (also known as record linkage or entity resolution)  \cite{brizan2006survey}. 
We refer to such vital information derived from data integration process as \emph{data integration metadata} (\textit{DI metadata}).

Schema mappings are traditionally formalized as first-order logic~\cite{fagin2009}, and represented and executed using query languages such as SQL~\cite{yan2001data} or XQuery~\cite{fagin2009clio}. To accelerate both data integration and machine learning operations using GPUs, it is critical to find a unified representation that is highly compatible with GPU architectures.
In Ilargi, we employ a matrix-based representation for schema mappings, which we refer to as \emph{mapping matrices}, denoted as $\mathbf{M}$. Given a source table $S_k$ and target table $T$, a mapping matrix $M_k \in \mathbf{M}$ has the size of  $c_T \times c_{k}$, and specifies how the columns of $S_k$ are mapped to the columns of $T$. 

\begin{definition}[\textit{Mapping matrix}] Mapping matrices between source tables $S_1, S_2, ..., S_n$ and target table $T$ are a set of binary matrices $\mathbf{M}= \{M_1, ..., M_n\}$. $M_k \ (k\in [1, n])$ is a matrix with the shape $c_{T} \times c_{k}$, where  

\small{
		\begin{align*}
			\begin{split} 
M_k[i,j] =  \begin{cases}1, &\text{if j-th column of $S_k$ mapped to i-th column of } T\\0, & \text{otherwise}\end{cases}
			\end{split}
		\end{align*}
	}
	\label{def:mm}
 
\end{definition}

\vspace{-2mm}
Intuitively, in $M_k[i,j]$ the vertical coordinate $i$ represents the target table column while the horizontal coordinate $j$ represents the mapped source table column.  A value of $1$ in $M_k$ specifies the existence of  column correspondences between  $S_k$ and $T$, while the value $0$ shows that the current target table attribute has no corresponding column in $S_k$. Fig.~\ref{fig:amalur} shows the mapping matrices $M_1$, $M_2$, $M_3$ for source tables $S_1$, $S_2$, $S_3$, respectively. 
% In the technical report \cite{??}, we provide more examples and algorithms that generate mapping matrix, and the compressed indicator matrix in Sec.~\ref{sssec:imGen}.
 


% It is easy to see that the binary mapping matrices are often sparse.
% %--------------------------------------------------------%
% %--------------------recover-----------------------------%
% Because each attribute in the   source table $S_k$ is mapped to only one  attribute in $T$. 
% Thus, in each row of $M_k$ at most one element is 1, while the rest are 0.  
% Moreover, if an attribute of T does not have a mapped attribute in $S_k$, the corresponding row of the mapping matrix has only values of 0. For example, $ T.o$   (column ID: 3) does not have a mapped column in $S_1$, thus, the last row of  $M_1$ has only zeros, i.e., $M_1[3] = [0,0,0]$.
% %--------------------------------------------------------%


% Could add a figure proving this point, but Morpheus also used scipy spare matrices and did not provide motivation for this either

We use the \emph{indicator matrix} \cite{chen2017towards} 
% (denoted as $I_k$) 
to preserve the row matching between each source table $S_k$ and the target table $T$. 
An indicator matrix $I_k$ of size $c_T \times c_{k}$ describes how the rows of source table $S_k$ map to the rows of target table $T$, as in Fig.~\ref{fig:amalur}. 
 
\begin{definition}[\textit{Indicator matrix} \cite{chen2017towards}] Indicator matrices between source tables $S_1, S_2, ..., S_n$ and target table $T$ are a set of row vectors $\mathbf{I}= \{I_1, ..., I_n\}$. $I_k \ (k\in [1, n])$ is a row vector  of size $r_{T}$, where 
\small{
		\begin{align*}
			\begin{split} 
					I_k[i,j] =  \begin{cases}
$1$ & \text{if $j$-th row of $S_k$ mapped to $i$-th row of $T$} \\
$0$ & \text{otherwise}
\end{cases}
			\end{split}
		\end{align*}
	\label{def:cim}
}
\end{definition}
 

% Each source table $S_k$ has an indicator matrix $I_k$ and a mapping matrix $M_k$, which describe how the rows and columns of $S_k$ map to $T$, as shown in Fig.~\ref{fig:amalur}. 

\begin{table}[t]
\caption{Notations used in the paper.}
\label{tab:notations}
\vspace{1mm}
\centering
\footnotesize
\begin{tabular}{|l|l|} \hline
\textbf{Symbol} & \textbf{Description} \\ \hline
$S_k$ & The $k$-th source table \\ \hline
$T$ & Target table \\ \hline
% $y$ & The labels \\ \hline
$I_k$ & The indicator matrix for $S_k$\\ \hline
$M_k$ & The mapping matrix for $S_k$ \\ \hline
$j_T$          & The join type of $T$ (inner/left/outer/union) \\ \hline
$r_k/r_T$      & Number of rows in $S_k/T$ \\ \hline
$c_k/c_T$      & Number of columns in $S_k/T$ \\ \hline
$m_k/m_T/m_X/m_w$      & Number of nonzero elements in $S_k/T/X/w$ \\ \hline
\end{tabular}
\vspace{-4mm}
\end{table}
% We do not exclude the possibility of further optimization of the physical design of mapping and indicator matrices, which is an interesting engineering problem for future work. 

%------------------recover-------------------------%
% While in Ilargi we assume that we need an indicator matrix for each source table, previous work did not always make this assumption \cite{chen2017towards}. In Morpheus, the authors do not include an indicator matrix for the entity table in a star schema join, which is used as the first table in an inner or left join. Instead, each row in this entity table is matched to the same row in the target table. We can take this same approach in the inner and left outer join cases. We assume the entity matrix is $S_1$ and the corresponding indicator matrix is $I_1$.
%------------------recover-------------------------%
\vspace{-5mm}
\subsection{Rewriting Rules for Factorization }
\label{s:operations}
% \para{Factorized Linear Algebra operations}
With data and metadata represented in matrices, next, we explain how to rewrite a linear algebra over a target schema to linear algebra over source schemas. 
%--------------------------------------------------------------------
% The rewriting rules, although for arithmetic operations, share similar principles of view-based query rewriting \cite{deutsch2006query,chirkova2012materialized}. 
%--------------------------------------------------------------------
Here we use the example of LA operator \emph{left matrix multiplication (LMM)}. 
% The remaining LA operations, which are element-wise scalar operations, aggregation operations and multiplication operations, can be found in our extended technical report\footnote{\url{}. Appendix \ref{s:app_operations}.}. 
% Some of these operations return a materialized matrix, while other operations the turn a normalized matrix. Operations that return a normalized matrix allow us to exploit the rewrite rules again for a subsequent linear algebra operation. 
The full set of LA rewrite rules based on mapping and indicator matrices is in our technical report \cite{tech}. 



\para{Left Matrix Multiplication} Given a matrix $X$ with the size $c_T \times c_X$, LMM of T and X is denoted as $TX$.  The  LMM result of our target table matrix $T$ and another matrix $X$ is a matrix of size $r_T \times c_X$. Our rewrite of LMM goes as follows.
\[
TX \rightarrow  I_1 S_1 M_1^T X + ... + I_n S_n M_n^T X
\]
We first compute the local result  $I_{k} S_k M_k^{T}X\ (k\in [1, n])$ for each source table, 
then assemble them for the final results.

% For instance, for the running example the actual matrix multiplication  is $(I_{1} S_1) (M_1^{T} X)$ and $(I_{2} S_2) (M_2^{T} X)$.
% \todo[]{check with the order optimal}
%
%-------omit transpose----------------------------% 
% Performing LMM between two matrices where the first matrix is transposed is the same as multiplying the untransposed first matrix with the second matrix transposed and transposing this result. Therefore, we can define the rewrite in the transposed case as follows.

% \[
% T^TX \rightarrow (X^TT)^T
% \]
%-----------------------------------------------------
% \para{Implementation-level optimization} To reduce computation overhead,  our implementation adopts the optimal matrix multiplication order algorithm  \cite{cormen2022introduction} from the version included in NumPy\footnote{\url{https://numpy.org/doc/stable/reference/generated/numpy.linalg.multi_dot.html}} for SciPy matrices, computed using matrix dimensions.
% % This method was created by Cormen \cite{cormen2022introduction}.
%-----------------------------------------------------

To showcase the effectiveness of matrix-represented DI metadata for factorized learning, we use Linear Regression as an example. In Alg.~\ref{alg:linear-regression}, we identify two key operators: matrix transpose and left matrix-matrix multiplication. By applying our rewriting strategies, these operations are pushed directly to the disparate data sources, enabling local computation and eliminating the need for data centralization, thus improving efficiency and scalability.

\begin{algorithm}[t!]
  \caption[Linear regression]{Linear regression using Gradient Descent
    ~\cite{chen2017towards}.}\label{alg:linear-regression}
  \begin{algorithmic}
    \Require $X, y , w, \gamma$
    \For{$i \in 1:n$} 
     \State $w = w - \gamma (\text{{$X^T$}}((\text{{$X w$}}) - y))$
    \EndFor
  \end{algorithmic}
\end{algorithm}
% \noindent\textbf{Why is Ilargi GPU-compatible?} Previous studies \cite{orion_learning_gen_lin_models,MorpheusFI,chen2017towards,khamis_acdc_2018} on linear algebra rewriting predominantly focused on CPU execution environments. This work leverages data integration metadata in matrix form to streamline the DI and ML pipeline with GPU-specific LA libraries, such as cuBLAS\footnote{\url{https://docs.nvidia.com/cuda/cublas/}}.
% The column and row matching between disparate sources (schema mapping and data matching) are natively sparse, as a single data source typically provides only partial information to the target table \cite{doan2012principles}. 
% To adapt the sparsity of the mapping and indicator matrices, Ilargi utilizes the Compressed Sparse Row (CSR) Format\footnote{\url{https://docs.scipy.org/doc/scipy/reference/generated/scipy.sparse.csr\_matrix.html}}. 
% Moreover, we chose CuPy \cite{cupy_learningsys2017} to translate conceptual linear operators into physical GPU implementations, significantly enhancing the performance of model training over both factorization and materialization.


% \begin{figure*}[t]
%     \centering
%     % \includegraphics[width=0.93\linewidth]
%     \includegraphics[width=0.7\linewidth]{figures/hamlet_ML.pdf}
%      \vspace{-0.5cm}
%     \caption[Model training time over six real datasets: materialization vs. factorization]{
%     Models training time over seven real datasets: materialization vs. factorization
%     % Evaluation of Ilargi ML models with the number of iterations $= 20$ on Hamlet dataset. Annotations show the speedup of factorized learning compared to learning over materialized data.
%     }
%     \vspace{-0.3cm}
%     \label{fig:eval_ML_hamlet_Ilargi}
% \end{figure*}



% \subsection{Summary}

% \para{Applicability,  assumptions, and extensibility} Fig.~\ref{fig:applicability} summarizes Ilargi's applicability compared to SOTA solution \cite{chen2017towards}. 
% Ilargi can handle both normalized and denormalized tabular data.
% % , with 
% %  mapping matrices built between the schemas of source table matrices and the target schema of the training dataset. 
% % To prepare the training data,  feature and label columns are from source datasets, which means that for each column in the target table, we can find its corresponding column from at least one source schema. 
% % For simplicity, in the running example, all rows stored in the normalized data are mapped to at least one row in the target table. 
% % We follow the common data science assumption that the normal preprocessing steps include null value replacement and transformation of categorical variables to numerical values. 
% Similar to \cite{chen2017towards}, our input source table matrices (e.g., $S_1$, $S_2$, $S_3$ in Fig.~\ref{fig:amalur}) are preprocessed datasets in formats like NumPy arrays.  For simplicity, the running example replaces null values with zero, and applies one-hot-encoding for categorical variables.  Ilargi supports input data matrices whose null values are replaced to mean/median values or user-specified values, and other methods for handling categorical variables \cite{kelleher2020fundamentals}.
% % Binary variables can be encoded into zeroes and ones, where encoding the most common category into zeroes is preferred.
% % Ilargi can handle both normalized and unnormalized tabular data, and any mapping between rows and columns. This includes overlapping columns and any combination of overlapping rows between tables, e.g., an inner join, a left outer join, a right outer join, a full outer join, and unions. 
% % In this work, we do not propose a new data integration, data preprocessing, or ML algorithm. Our focus is to understand how data integration metadata (e.g., schema mapping, row matching) affects time-wise training efficiency, and how to choose between materialization and factorization to speed up the training process, which we discuss next.
% Our approach can serve as a fundamental component of federated learning frameworks for seamlessly incorporating DI metadata during the training process.
% % Moreover, we are currently working on extending Ilargi into an end-to-end data processing and model training pipeline. 
% For future work, we will expand Ilargi's support for more transformation operations like min-max scaling to enhance Ilargi's real-world usability further.



%------------------------------------------------
%	COST ESTIMATION
%------------------------------------------------
\vspace{-4mm}
\section{Cost Estimation in Ilargi}
\label{sec:cost}
\vspace{-2mm}
Reducing ML model training time through factorization is not always guaranteed. When the target table has low redundancy compared to source tables, factorization may require more linear algebra computations, leading to longer training times than materializing the target table \cite{chen2017towards}. Existing studies \cite{chen2017towards, MorpheusFI} use empirical cost models to set sparsity-based thresholds for selecting between factorization and materialization, yet these thresholds vary with hardware.  However, in environments with both CPUs and GPUs, our experiments (Sec. \ref{sec:eva}) show that target table sparsity does not consistently correlate with factorization speedups, and identical sparsity levels can yield different speedup outcomes on different hardware platforms.

This discovery reveals the intricacy of selecting the optimal training method between factorization and materialization. Consequently, it is essential to develop a cost model that considers both algorithmic characteristics and hardware properties. To build this cost model, we begin with a complexity analysis of model training. This analysis enables a comparative evaluation of the complexities associated with materialization and factorization.
% \begin{figure}[t]
%     \centering
%     \includegraphics[width=0.9\linewidth]{figures/new_eval_figures/prelim_spasity.pdf}
%     \caption{The average speedups, expressed as $\frac{T_{materialization}}{T_{factorization}}$, vary on CPUs and GPUs depending on the sparsity of the target table. On CPUs, KMeans exhibits higher speedups than GaussianNMF, while the reverse is true on GPUs. The detailed setting are in explained Sec. \ref{sec:eva}.}
%     \label{fig:sparsity_models}
%     \vspace{-2mm}
% \end{figure}

% \input{logicalrules}



\vspace{-4mm}
\subsection{Complexity Analysis of ML Model Training Algorithms}
\label{s:complexity}
\vspace{-2mm}
We compare the computational complexity of linear algebra (LA) operations for both materialization and factorization. For materialization, the cost is based on performing LA operations directly on the materialized table. In contrast, factorization considers the operations on each source table individually.

% Below, we continue to use LMM for the explanation.

% \para{Complexity analysis in state-of-the-art} 
% In \cite{chen2017towards}, a straightforward complexity analysis is provided. For each LA operator, the decision of factorization and materialization is based on the arithmetic computation complexity, i.e., the size comparison between joinable tables and materialized table. 
% For instance, given    a matrix $T$ with the shape $r_{T} \times c_{T}$, and a matrix $X$ with the shape $c_{T} \cdot c_{X}$ ($r_{X} = c_{T}$), 
% the complexity of their left matrix multiplication $TX$ is $c_{X} \cdot c_{T} \cdot r_{T}$. Thus, in \cite{chen2017towards} the complexity of computing $TX$ for materialization is $c_{X} \cdot c_{T} \cdot r_{T}$, while for factorization is $\displaystyle \sum_{k=1}^n c_X \cdot c_k \cdot r_k$.   

% \para{Sparsity of mapping \& indicator matrices} 
% As discussed in Sec.~\ref{sec:matrixGen}, our mapping and indicator matrices are sparse. The matrix form of the source and target tables can be sparse or dense. A \emph{sparse matrix} is a matrix whose number of nonzero elements is negligible compared to the number of zeros, while a \emph{dense matrix} has more non-zero elements than zeros \cite{yuster2005fast}. 

% %-----------sparsity explained-------------------------------------------
% The null values come from the sources or from the join. 
% We inspect what sparsity looks like in Fig.~\ref{fig:example}. Here, the sparsity of table  $S_1$ Prescriptions, denoted $p_\text{Prescriptions}$, equals 6, as there are six cells with zeroes in this table. The sparsities of tables $E$, $R_1$, and $T$ are $p_E = 6$, $p_{R_1} = 2$ and $p_T = 11$.  Sparsity resulting from the data happens when data explicitly contains zero values. In our implementation, these explicit zeros are not stored in the sparse matrices. Sparsity resulting from the join occurs in the target table. These missing values are also not stored in the sparse materialized table. As missing values and explicit zeroes are not stored, the computations we perform on sparse matrices involve less operations than if we were to perform them on a dense matrix of the same size. Therefore, sparsity should be taken into account when performing cost estimation. The upcoming section details how sparsity is included in the cost estimation procedure.
% %-----------------------------------------------------------------------

% \begin{table}[t]
% \caption{LA operator computations cost comparison}
% % : materialization vs. factorization.}
% \label{tab:complexity}
% \centering
%  % \scriptsize
%   \small
% \begin{tabular}{llll}
% \toprule
% \textbf{Operation} & \textbf{Materialization}                  & \textbf{Factorization}                  \\ \midrule
% % $T \oslash x$  & \multirow{8}{*}{$\displaystyle m_T$} & \multirow{8}{*}{$\displaystyle \sum_{k=1}^n m_k$} \\
% % $T^T \oslash x$             &                                    &                                      \\
% %  $f(T)$                  &                                    &                                      \\
% % $f(T^T)$               &                                    &                                      \\
% % $\text{rowSum}(T)$       &                                    &                                      \\
% % $\text{rowSum}(T^T)$       &                                    &                                      \\
% % $\text{colSum}(T)$       &                                    &                                      \\
% % $\text{colSum}(T^T)$       &                                    &                                      \\ \hline

% $TX$               & $\displaystyle c_X \cdot m_T + r_T \cdot m_X$              & $\displaystyle \sum_{k=1}^n c_X \cdot m_k + r_k \cdot m_X$            \\
% $T^TX$               & $\displaystyle   c_X \cdot m_T + c_T \cdot m_X$              & $\displaystyle \sum_{k=1}^n    c_X \cdot m_k  + c_k \cdot m_X $          \\
% $XT$               & $\displaystyle  r_X \cdot m_T + c_T \cdot m_X $              & $\displaystyle \sum_{k=1}^n   r_X \cdot m_k + c_k \cdot m_X$      \\
% $XT^T$               & $\displaystyle r_X + m_T + r_T \cdot m_X $              & $\displaystyle \sum_{k=1}^n r_X + m_k + r_k \cdot m_X$            \\
% \bottomrule
% \end{tabular}

% \vspace{-0.4cm}
% \end{table}

\begin{table}[t]
\caption{Computations complexity comparison of common LA operators.}
% : materialization vs. factorization.}
\label{tab:complexity}
\vspace{1mm}
\centering
 % \scriptsize
  \small
\begin{tabular}{l|c|c}
\toprule
\textbf{Operation} & \textbf{Materialization}                  & \textbf{Factorization}                  \\ \midrule
$T \oslash x$  & \multirow{4}{*}{$\displaystyle m_T$} & \multirow{4}{*}{$\displaystyle \sum_{k=1}^n m_k$} \\
 $f(T)$                  &                                    &                                      \\
$\text{rowSum}(T)$       &                                    &                                      \\
$\text{colSum}(T)$       &                                    &                                      \\ \hline

$TX$               & $\displaystyle c_X \cdot m_T + r_T \cdot m_X$              & $\displaystyle \sum_{k=1}^n c_X \cdot m_k + r_k \cdot m_X$            \\
$XT$               & $\displaystyle  r_X \cdot m_T + c_T \cdot m_X $              & $\displaystyle \sum_{k=1}^n   r_X \cdot m_k + c_k \cdot m_X$      \\
\bottomrule
\end{tabular}
\vspace{-5mm}
\end{table}

% \para{Complexity analysis in Ilargi} 
Tab.~\ref{tab:complexity} summarizes the comparison. Ilargi stores matrices in sparse format, so the complexity of element-wise operations, rowSum, and colSum depends on the number of non-zero elements.

For sparse matrix multiplication, given two matrices \( T \) and \( X \), the complexity of \( TX \) is \( O(c_X \cdot m_T + r_T \cdot m_X) \), where \( m_T \) and \( m_X \) are the non-zero elements of \( T \) and \( X \) \cite{horowitz1982fundamentals}. Since our mapping and indicator matrices are sparse, we calculate matrix operation costs accordingly. With the rewriting rule (Sec.~\ref{s:operations}), the complexity of factorized multiplication is \( \sum_{k=1}^n (c_X \cdot m_k + r_k \cdot m_X) \).

The complexity of ML models depends on the specific LA operators used. We detail linear regression analysis here, with further models discussed in our technical report \cite{tech}.

% For sparse matrix operators, consider two matrices $T$ and $X$. A multiplication $TX$ over sparse format is $O(c_X \cdot m_T + r_T \cdot m_X)$, where $m_T,\ m_X$ are the numbers of nonzero elements in matrix $T$ and $X$ \cite{horowitz1982fundamentals}. 
% Since our mapping and indicator matrices are sparse, we compute the computation cost of matrix operators as in Tab.~\ref{tab:complexity}. With the rewriting rule we discussed in Sec, the computing complexity of multiplication over factorized data is the summation of complexity of multiplications over sources. As such, we have the computing complexity of factorized multiplication as $\sum_{k=1}^n c_X \cdot m_k + r_k \cdot m_X$
% As discussed in Sec.~\ref{sec:matrixGen}, our mapping and indicator matrices are sparse. Therefore, we compute the computation cost of matrix operators as in Table \ref{tab:complexity}. 
%----------------------------------------------------
% We also note the complexity of transposed operations. As described in Section \ref{s:operations}, $T^TX$ is actually defined as $(X^T T)^T$ and $XT^T$ is actually defined as $(TX^T)^T$. Here, we ignore the complexity added by the transposes, as these are not executed on data matrices.  
%----------------------------------------------------
% Here the differentiation between sparse and dense matrices is more about an intuition rather than a rigorous measurement~\cite{horowitz1982fundamentals}. 
%----------------------------------------------------
% Take the example of $TX$ for materialization, and we have $m_T \leq r_T \cdot c_T$. When the target table matrix is dense, the non-zero element number $m_T$ is close to the size of $T$, i.e., $r_T \cdot c_T$. The time for multiplying $T$ and $X$ is at most $O(r_T \cdot c_T \cdot c_X)$. 


% \begin{table}[t]
% \centering
% \scriptsize
% \begin{tabular}{lll}
% \toprule
% \textbf{Operation} & \textbf{Materialization}                  & \textbf{Factorization}                  \\ \midrule
% $T \oslash x$  & \multirow{8}{*}{$\displaystyle nnz(T)$} & \multirow{8}{*}{$\displaystyle \sum_{k=1}^K nnz(S_k)$} \\
% $T^T \oslash x$             &                                    &                                      \\
% $f(T)$               &                                    &                                      \\
% $f(T^T)$               &                                    &                                      \\
% $\text{rowSums}(T)$       &                                    &                                      \\
% $\text{rowSums}(T^T)$       &                                    &                                      \\
% $\text{colSums}(T)$       &                                    &                                      \\
% $\text{colSums}(T^T)$       &                                    &                                      \\ \hline

% $TX$               & $\displaystyle c_X \cdot nnz(T) + r_T \cdot nnz(X)$              & $\displaystyle \sum_{k=1}^K c_X \cdot nnz(S_k) + r_{S_k} \cdot nnz(X)$            \\
% $T^TX$               & $\displaystyle c_T \cdot nnz(X) + c_X \cdot nnz(T)$              & $\displaystyle \sum_{k=1}^K c_{S_k} \cdot nnz(X) + c_X \cdot nnz(S_k)$          \\
% $XT$               & $\displaystyle c_T \cdot nnz(X) + r_X \cdot nnz(T)$              & $\displaystyle \sum_{k=1}^K c_{S_k} \cdot nnz(X) + r_X \cdot nnz(S_k)$      \\
% $XT^T$               & $\displaystyle r_X + nnz(T) + r_T \cdot nnz(X) $              & $\displaystyle \sum_{k=1}^K r_X + nnz(S_k) + r_{S_k} \cdot nnz(X)$            \\
% \bottomrule
% \end{tabular}
% \caption{LA computations cost: materialization vs. factorization.}
% \label{tab:complexity}
% \end{table}


% %----------------------------------------------------------%
% \para{Computing the sparsity of $T$}\label{s:est_sparsity} Some elements in the complexity analysis can be inferred directly from the source tables, while others require estimation. Elements which can be directly inferred include $r_X$, $c_X$, $r_k$, $c_k$ and $m_k$, as these are stored in SciPy objects. The elements $r_T$ and $c_T$ cannot be inferred from the source tables directly, but can be inferred from the indicator matrices and mapping matrices, respectively. The element that requires the most complicated computations is $m_T$, which depends upon the source tables' sparsities, the indicator matrix and the mapping matrix. In order to compute $m_T$, we compute $T$ using the materialization computation $T \leftarrow  I_1 S_1 M_1^T + ... + I_K S_K M_K^T$. 
% Then, we compute the sparsity directly from the result of this computation. 
% %----------------------------------------------------------%
% \para{Limitation and future work} First, to obtain the number of non-zero elements, i.e., sparsity of $T$, we need to materialize target table $T$. 
%--------------------------------------------------------------------------%
%--------------------------------------------------------------------------%
% In preliminary experiments, we tried avoiding materialization and estimating sparsity, leading to unsatisfactory effectiveness results. Instead, we discovered that table materialization time is omittable compared to model training time, as elaborated in Sec.~\ref{sec:overhead}. Therefore, we chose to materialize the target table for a more accurate sparsity value. The goal of this work is not to avoid materialization \cite{kumar2016join}. Instead, the primary goal of this work is to ascertain whether, for a set of given source tables and learning tasks, the training of a model is expedited by conducting it over the source tables or through the use of a materialized target table.
% % Second,
% % The result of two sparse matrix multiplication is not necessarily a sparse matrix \cite{horowitz1982fundamentals}.  
%--------------------------------------------------------------------------%
%--------------------------------------------------------------------------%
% \vspace{-2mm}
% \subsection{Complexity Analysis of ML Models}\label{s:complexity_models}

 

\para{Linear regression} First, we analyze the complexity of linear regression in Algorithm \ref{alg:linear-regression} in the materialized case, which is dominated by two matrix multiplication operations, i.e., $ T^T$ $(T w)$. 
For the first matrix multiplication $T w$: 
we denote the shape of weights vector $w$ as $c_T \times 1$. Following the general assumption that $w$  is dense, we calculate $m_w$, the number of nonzero elements in $w$, with the equation $m_w = r_w \times c_w$. 
We denote the matrix multiplication result of $T w$ as $X$, which is an intermediate result
of linear regression algorithm. The size of $X$ is $r_T \times 1$, since $r_X= r_T\  and \ c_X =1$. 
Similarly, we calculate $m_X$, the number of nonzero elements in $X$, with the equation $m_X = r_X \times c_X$.  
Now, we define the complexity of linear regression in the materialized case based on $T w$ and $T^TX$.
\[
O_\text{materialized}(T) = \underbrace{c_w \cdot m_T + r_T \cdot m_w}_{Tw} + \underbrace{m_T + c_T \cdot m_X}_{T^TX}
\]

Next, we define the complexity of the factorized case. 
\[
\begin{aligned}
O_\text{factorized}(T) = & \underbrace{\sum_{k=1}^n (c_w \cdot m_k + r_k \cdot m_w)}_{Tw} + 
 \underbrace{\sum_{k=1}^n (m_k + c_k \cdot m_X)}_{T^TX}
\end{aligned}
\]


\para{Complexity ratio}
We define a variable \emph{complexity ratio} to indicate whether materialization or factorization leads to more computing. 
The complexity ratio is measured as the ratio of the materialization complexity divided by the factorization complexity.
\begin{equation}
\label{cr_value}
     \text{complexity ratio} = \frac{O_\text{materialization}(T)}{O_\text{factorization}(T)}
\end{equation}



% \subsection{Is Cost Estimation Necessary?}
% \subsection{Is Complexity Ratio All We Need?}
% \label{ssec:need_CE}
% \para{Solution so far} In Fig.~\ref{fig:ForM}, our objective is approximating the decision boundary between factorization and materialization. Two main factors -- \emph{data} (including DI metadata like schema mapping and row matching) and \emph{ML algorithm}-- are considered. The complexity ratio incorporates elements such as redundancy and operator complexity, approximating the logical-level decision boundary. Preliminary experiments, however, indicated that solely logical-level estimation falls short due to software-hardware interactions such as parallelism and memory bandwidth.







% % As illustrated in Fig.~\ref{fig:ForM}, our goal is to find an approximate decision boundary between factorization and materialization. We have considered two main factors, \emph{data} including DI metadata such as schema mapping and row matching, and \emph{ML algorithm}. As mentioned in Sec.~\ref{sec: gaps}, redundancy and sparsity play important roles. 
% % The complexity ratio incorporates relative source and target redundancies and the complexity of the operators, which approximates the decision boundary of factorization and materialization at the logical level. 
% % However, our preliminary experiments showed merely a logical-level estimation was insufficient, due to the interactions between the software implementation and hardware properties, such as parallelism and memory bandwidth.

 

% \para{The third factor} 
% One preliminary experiment reveal hardware configurations as a vital factor in the decision between factorization and materialization. Figure \ref{fig:prelim} showcases how the choice varies with changing parallelism w.r.t the numbers of CPU cores. Interestingly, factorization outperform materialization when the parallelism increases to 32, even at a complexity ratio of 3. This result is attributed to factorization's ability to decouple source table records, thus enabling parallel computational tasks and leveraging extensive hardware parallelism.


% % From experiments, we discovered that a third factor, i.e., hardware configurations, needs to be taken into consideration. Figure \ref{fig:prelim} gives a simple example of how the choice of factorization or materialization changes, with various levels of parallelism w.r.t. the number of CPU cores (parallelism). We elaborate experiment setting in Sec.~\ref{sec:eva}. 
% % When the y-axis value is less than 1.0, the actual faster strategy is factorization, while above 1.0 means materialization is faster in the experiment.
% % We observe that with 8 cores, with complexity ratio values of 3 and 6, materialization was faster. Yet, interestingly, upon increasing the core number to 32, factorization surpassed materialization, even with a complexity ratio of 3. This shift resulted in a speed boost of over 30\% compared to materialization. This can be attributed to factorization's ability to decouple records in different source tables, which in turn enables the launch of computational tasks in parallel. By leveraging the extensive computational capabilities of a hardware platform with high parallelism, factorization can indeed deliver superior performance.



% Given the intricate decision making as shown Fig. ???\ref{fig:prelim}, the goal is to design a model that can determine the most efficient method for training machine learning models over data silos. More specifically, if materialization is estimated to be faster, we train models over materialized data. Conversely, when factorization estimated to be quicker, we employ factorized training. This approach, therefore, creates a more efficient machine learning training pipeline over data silos, optimizing the overall performance.

% Next, we show how Ilargi estimates to factorize or to materialize, considering all three factors of data, model, and hardware.
 



\vspace{-2mm}
\subsection{The Hardware Factor} 
\label{sec:3rdfac}
\vspace{-2mm}
In our evaluation (detailed in Sec. \ref{fig:eval_f_m}), we found that materialized training can outperform factorized learning, despite the higher redundancy. Additionally, GPU speedups are inconsistent. These discrepancies go beyond computational complexity, highlighting the influence of hardware properties like memory bandwidth and parallelism. Traditional complexity analysis often neglects the I/O overhead from data movement, especially in multi-epoch ML training where I/O costs accumulate. To address this, we introduce a learned cost estimator that considers data characteristics, algorithm complexity, and hardware features to better determine the optimal training method.% In Sec.~\ref{sec:eva}, we extensively evaluate the performance of factorized learning on both CPUs and GPUs. One observation is that only 32\% of instances exhibited performance improvements on both types of hardware. 
% This unexpected low overlapping highlights the impact of architectural differences on the efficiency of factorization. To gain a deeper understanding of the internal dynamics of training algorithms on these two distinct platforms, we employed a performance analysis comparing the memory cost and the math cost in the profiled scenarios. 

% According to an official document from NVIDIA~\cite{nvidia-gpu-performance:online}, an effective method to predict the execution time of a GPU program is to calculate $max(T_{mem}, T_{math})$. In this formula $T_{mem}$ is the time required to transfer data to and from the GPU memory, while $T_{math}$ denotes the time needed for actual computations. This approach is in line with the inherently parallel architecture of GPUs. Should the data transfer to the GPU prove insufficiently fast, the GPU's Streaming Multiprocessors will remain idle, awaiting data. This indicates a memory-bound program. Conversely, if $T_{math} > T_{mem}$, the program is considered compute bound. This section elaborates which scenario applies to our experiments and details how this knowledge can be harnessed to estimate the runtime for machine learning training scenarios.

% \begin{figure}[t]
%   \centering
%   \begin{minipage}{0.4\textwidth}
%     \includegraphics[width=\linewidth]{figures/profiling-mem-vs-compute-materialized.pdf}
%   \end{minipage}
%   \begin{minipage}{0.4\textwidth}
%     \includegraphics[width=\linewidth]{figures/profiling-mem-vs-compute-factorized.pdf}
%   \end{minipage}
%   \caption[Memory cost vs math cost of profiled scenarios]{Memory cost ($T_{mem}$) vs compute cost ($T_{math}$) of profiled scenarios. The memory cost is computed as the total number of bytes read and written to memory divided by the measured average memory bandwidth. The math cost is the number of cycles the Streaming Multiprocessors were active divided by the measured average SM frequency.}
%   \label{fig:5-profiling-mem-vs-compute}
%   \vspace{-5mm}
% \end{figure}

% Fig.~\ref{fig:5-profiling-mem-vs-compute} shows the relationship between memory time $T_{mem}$ and computation time $T_{math}$, revealing a strong correlation ($\rho = 0.99$). The result predominantly shows that the memory cost exceeds the computational cost, as most points lie below the $y=x$ line, indicating that the operations are memory-bound. 
% From this analysis, we conclude that a threshold-based empirical cost estimator fails to account internal dynamics other than computing complexity. Therefore, we propose a learned cost estimator that leverages data characteristics, algorithmic complexity, and hardware features to more accurately determine the faster training method.



\vspace{-3mm}
\subsection{ML-based Cost Estimator}
\label{s:estimator}
\vspace{-2mm}
% Our cost estimator is designed to make a binary decision—factorization or materialization—based on data characteristics, computational complexity, and hardware features. Its goal is to select the more efficient training method. This choice is especially impactful when training multiple models, as in parameter tuning, where even slight speed improvements can translate into substantial time savings.

% While we considered analytical performance models \cite{hpc1, hpc2, hpc3} from the high-performance computing community, they lack flexibility for diverse ML models and need extensive micro-benchmark data. Given these complexities, it becomes more practical to utilize a black-box estimator \cite{blackbox1}. This type of estimator employs a statistical model to make a binary decision, effectively abstracting away from the specific details of individual hardware components and algorithmic intricacies.


% \para{Ilargi's tree boosting estimator}  Tree boosting models stand out among various statistical models, and are widely used in many cost estimators \cite{tvm, halide} due to their explainability and prediction speed. Moreover, tree boosting has the capability to identify non-linear relationships among features.
% % \para{Model Training} Our model was trained using XGBoost, a well-regarded gradient boosting framework. 
% Given these advantages, we propose a tree boosting estimator leveraging XGBoost \cite{xgboost}. The main challenge lies in identifying features that are relevant to the decision-making process between materialization and factorization.


% \para{Selection of hardware features} 
% Our estimator is generalizable and portable, focusing on macro-level features instead of micro-level. Micro-level features (e.g., cache speeds and memory latency), while valuable, can tie the model to specific hardware, limiting portability. 
% Instead, we incorporate macro-level features like number of parallelism and memory bandwidth. Shown in Sec.~\ref{ex:estimator}, combined with memory read/write costs, we can effectively estimate maximum memory access costs for both materialization and factorization. 
% In addition to the memory features, parallelism is also a crucial factor that influences the speedup of factorized learning. Therefore, the number of threads executing mathematical computations is also included our features.

Our cost estimator determines whether factorization or materialization is the more efficient training method, considering data characteristics, computational complexity, and hardware features. This choice is especially impactful when training multiple models, such as in parameter tuning, where small speed gains can lead to significant time savings.

Analytical performance models \cite{hpc2,hpc3} from high-performance computing are limited in flexibility for diverse ML models and require extensive micro-benchmark data, making a black-box estimator \cite{blackbox1} a practical alternative. This type of estimator uses a statistical model to make a binary decision, abstracting details of hardware and algorithms.

\para{Ilargi's Tree Boosting Estimator.} Tree boosting is known for its explainability and speed, making it popular in cost estimation tasks \cite{tvm,halide}. We use XGBoost for its ability to capture non-linear relationships, with a focus on identifying key features that influence the choice between materialization and factorization.

\para{Hardware Features.} To ensure portability, our estimator emphasizes macro-level hardware features, such as memory bandwidth and parallelism, over micro-level specifics like cache speed. Parallelism, particularly the number of threads used, is included due to its impact on factorized learning speedups. Combined with memory read/write costs, these features enable effective cost estimation for both methods (Sec.~\ref{ex:estimator}).




\para{Cost estimation pipeline}\label{s:design}
% The pipeline of our cost estimation approach is shown in Fig. \ref{fig:cost_estimation_amalur}. For explanatory purposes, we highlight Linear regression and its involved LA operations. 
% The optimizer takes inputs of the source tables, indicator matrix and mapping matrix. The mapping matrices and indicator matrices are generated from the input schema matching and entity resolution results, which can be generated from open-source schema matching \cite{koutras2021valentine} and  entity resolution  tools \cite{ER sol}. 
% The ML model is provided by the user, including model hyperparameters. 
The cost estimator uses three input groups: \texttt{i)} data (source matrices, mapping, and indicator matrices), \texttt{ii)} ML algorithm (operators and user-defined hyperparameters), and \texttt{iii)} hardware (e.g., parallelism, memory bandwidth). Fig. \ref{fig:cost_estimation_amalur} shows its workflow using linear regression as an example. The estimator\footnote{A full list of 33 features is detailed in our technical report \cite{tech}.} computes the complexity ratio (Sec.~\ref{s:operations}) and the theoretical memory I/O for each operator in the ML algorithm. It normalizes these features based on parallelism and memory bandwidth. The output is binary: If True, Ilargi uses factorization; otherwise, it materializes the data before training.% as input ML model. 
\begin{figure}[t]
    \centering
    \includegraphics[width=0.8\linewidth]{figures/new_eval_figures/cost-est-pipeline-simple.drawio.pdf}
    % \includegraphics[width=0.85\linewidth]{figures/new_eval_figures/cost-est-pipeline.pdf}
    \caption{Workflow of the estimator.}
    \label{fig:cost_estimation_amalur}
    \vspace{-5mm}
\end{figure}

% The estimator calculates the complexity ratio in Sec.~\ref{s:operations},
%----------------recover-------------------------------------
% \footnote{Hyperparameters such as the rank $r$ for GNMF and the number of clusters $k$ for KMeans are factored into the cost estimation. However, other hyperparameters like the learning rate $\gamma$ for linear and logistic regression do not affect the cost estimation.}, 
%----------------recover-------------------------------------
% and theoretical memory read and write quantities for each operator within the input ML algorithm to be trained. Since factorization and materialization have different complexities and memory I/O amounts, we normalize these features by dividing total amounts, further dividing these normalized features by parallelism and memory bandwidth. 
% In addition, we include the complexity ratio and tuple ratio into our estimator, comparing feature importance during evaluation (Sec.~\ref{sec:eva}). 
% We document and explain the full list of 33 features in our technical report.

% The output of the estimator is binary. Specifically, if the prediction is True, Ilargi trains the input ML model using the factorization approach. Otherwise, Ilargi first joins the source tables, subsequently training the model on the materialized target table. 

% \noindent\textbf{Why is Ilargi efficient?}
% The speedup of factorization is typically influenced by data characteristics and training algorithms. With new hardware like GPUs, the situation becomes more complex. Due to the inconsistent advantages of factorization over materialization, a practical factorized learning system requires an accurate cost estimator to determine the optimal method for overall performance improvement. Our ML-based cost estimator incorporates diverse data characteristics and memory access features specific to certain hardware, leading to effective training method selection across various hardware environments. With optimal training strategies, \emph{Ilargi} can efficiently train models over disparate data sources.








% %----------------------------------------%
% \\ \para{Limitations and applicability} The estimator provides ease of construction and captures non-linear features but requires a diverse dataset for training in advance. Its predictive accuracy can be limited for models with unseen operators due to sensitivity to operator combinations in supported ML algorithms. Additionally, its assumption of operator information availability for factorization may restrict the range of models it can support.
% %----------------------------------------%

% 
% This decision-making process enhances the overall efficiency of ML training workflows on data silos.

% Due to space restrictions, we list a few representative operators and include the full list of supported LA operators in \cite{tech}.
% \begin{myframe}{~Answer to Q3}
% \emph{1. Schema mappings can be used as pruning rules.\\
% 2. The choice of factorization and materialization depends on the datasets and their relationships, ML algorithms (LA operators), and hardware configuration.}  
% \end{myframe}
% \vspace{-2mm}



% To assess the predictive quality and generalizability of the estimator, we conducted a series of comprehensive experiments, as detailed in Section \ref{ex:estimator}. This section provides an in-depth description of the evaluation data used, along with the training configurations employed in the experiments.
% In Sec.~\ref{ssec:speedup} we discuss how empirical threshold value $\tau$ is obtained.
% The determination of $\tau$ is explained in the next section. 

% \para{Discussion} Ideally, when the complexity ratio is above 1, it indicates that a speedup should occur. However, the process of factorization brings additional  overhead \rihan{what overhead}. 
% Thus, solely relying on the complexity ratio may not lead to an accurate cost estimation. That is, a more careful design of the threshold value $\tau$telling when we shall expect speedups is needed, which we discuss in Section \ref{s:threshold}.


% , further narrowing its applicability.

\section{Evaluation}
\label{sec:eva}
% In this section we first detail the datasets as well as the hardware used for our evaluation in Sec.~\ref{s:setup}. We first evaluate the speedup of factorized individual LA operators in Sec.~\ref{ex:ops}. We find out that the speedup of LA operator factorization can be quite substantial (up to 4x) when the parallelism is high. Then, in Sec~\ref{ex:spd_training} we evaluate the speedup of factorized ML models (i.e., mixed LA operators) over synthetic data and TPC-DI benchmark, and find  that factorization is not always faster than materialization, depending on the data characteristics (e.g., redundancy) and parallelism. Essentially, this experimentally motivates the need our cost estimator (Sec.~\ref{sec:cost}) that we evaluate in Sec.~\ref{ex:estimator}. There we find that our estimator can be quite effective in predicting what strategy to use (factorize vs. materialize) with very high accuracy (90.5\% for real datasets, and 97.1\% for synthetic ones).
\vspace{-2mm}
We start this section with an overview of our experimental setup, followed by a detailed performance evaluation of machine learning workloads employing both factorization and materialization techniques on CPUs and GPUs. This evaluation specifically examines the computing performance of factorized ML model training on these platforms and explores the factors that influence the decision-making process between factorization and materialization. Building on these insights, we further evaluate the accuracy of our proposed cost estimator in predicting the optimal strategy, utilizing both real and synthetic datasets. Our results demonstrate that the cost estimator can reduce the time consumption of real-world model training workloads by more than 20\%.

\vspace{-3mm}
\subsection{Experiment Setup}
\label{s:setup}

\vspace{-1mm}
\para{Synthetic Datasets}
\label{s:setup_synthetic} 
We built a data generator\footnote{\url{https://github.com/amademicnoboday12/Ilargi/tree/main/src/data_generator}} to create synthetic datasets with diverse data characteristics. It generates source table pairs that join to form a target table based on specified parameters. Using the ranges in Tab.~\ref{tab:synth_data_char}, we produced 1800 datasets to capture the relationship between data features and training performance. Additionally, the results from these datasets serve as training data for our cost estimator.

\begin{table}[t]
\caption{Parameters used for synthetic dataset generation.}
\label{tab:synth_data_char}
\vspace{1mm}
\centering
\footnotesize
\begin{tabular}{|l|l|l|}
\hline
\textbf{Symbols}           & \textbf{Ranges}             & \textbf{Description}                                                                                                            \\ \hline
$r_T$, $r_{S_k}$   & 100k, 500k, 1M     & \begin{tabular}[c]{@{}l@{}}Number of rows in target table $T$ or source \\table $S_k$ \end{tabular}                                        \\ \hline
$c_T$, $c_{S_k}$   & {[}10, 50{]}       & \begin{tabular}[c]{@{}l@{}}Number of columns in target  table $T$ or source \\table $S_k$\end{tabular}                                    \\ \hline
% \begin{tabular}[c]{@{}l@{}}Complexity ratio \\(CR value)  \end{tabular} & {[}0.5, 9.5{]}     & \begin{tabular}[c]{@{}l@{}}Complexity ratio at model level, Equation~\ref{cr_value} in \\Sec.~\ref{s:complexity}  \end{tabular}                                 \\ \hline
$p$              & {[}0, 0.9{]}       & \begin{tabular}[c]{@{}l@{}}Sparsity of the target table\end{tabular}                             \\ \hline
$\rho_c(T)$      & {[}0.1, 1{]}       & \begin{tabular}[c]{@{}l@{}} The percentage of number of columns in source \\tables  w.r.t target table\end{tabular} \\ \hline
$j$              & \begin{tabular}[c]{@{}l@{}}Inner, left, or \\outer joins, union\end{tabular}     & Join types                                                                                                             \\ \hline
\end{tabular}
\vspace{-4mm}
\end{table}

% \vspace{-1mm}


\para{Real-World Datasets}
To validate the estimator in real scenarios, we used the Hamlet datasets \cite{2016-hamlet-sigmod}, commonly referenced in related works \cite{MorpheusFI,chen2017towards,orion_learning_gen_lin_models}. The seven datasets in Hamlet simulate ML workflows, originally designed for inner join scenarios but adapted for other join types as detailed in Tab.~\ref{tab:6-hamlet-characteristics}.

% \begin{table}[t]
%   \centering
%     \caption[Hamlet dataset characteristics]{Hamlet dataset characteristics. $r$ and $c$ indicate the number of rows and columns, and $n$ is the number of tables. Subscripts denote which table characteristics belong to. }
%   \label{tab:6-hamlet-characteristics}
%   \vspace{1mm}
%   \footnotesize
%   \begin{tabular}{|l|l|l|l|l|l|l|l|}
%   \hline
%      & Book    & Expedia & Flight  & Lastfm  & Movie   & Walmart & Yelp    \\ \midrule \midrule
%     $r_T$                                            & $253$K  & $942$K  & $66.5$K & $344$K  & $1$M    & $422$K  & $216$K  \\
%     $c_T$                                            & $81.7$K & $52.3$K & $13.7$K & $55.3$K & $13.3$K & $2.44$K & $55.6$K \\
%     $r_{S_1}$                                        & $27.9$K & $942$K  & $66.5$K & $5$K    & $6.04$K & $422$K  & $11.5$K \\
%     $r_{S_2}$                                        & $50$K   & $11.9$K & $540$   & $50$K   & $3.71$K & $2.34$K & $43.9$K \\
%     $r_{S_3}$                                        &  -       & $37$K   & $3.17$K &    -     &    -     & $45$    &   -      \\
%     $r_{S_4}$                                        &    -    &   -      & $3.17$K &    -     &   -      &   -      &      -   \\
%     $c_{S_1}$                                        & $28$K   & $27$    & $20$    & $5.02$K & $9.51$K & $1$     & $11.7$K \\
%     $c_{S_2}$                                        & $53.6$K & $12$K   & $718$   & $50.2$K & $3.84$K & $2.39$K & $43.9$K \\
%     $c_{S_3}$                                        &    -     & $40.2$K & $6.46$K &   -      &     -    & $53$    &    -     \\
%     $c_{S_4}$                                        &   -      &   -      & $6.47$K &   -      &    -     &    -     &   -      \\
%     \bottomrule
%   \end{tabular}
% \vspace{-2mm}
% \end{table}

\begin{table}[t]
  \centering
    \caption[Hamlet dataset characteristics]{Hamlet dataset characteristics. $r$ and $c$ indicate the number of rows and columns. Subscripts denote which table characteristics belong to. }
  \label{tab:6-hamlet-characteristics}
  \vspace{1mm}
  \footnotesize
  \begin{tabular}{|l|l|l|l|l|l|l|l|l|l|l|}
  \hline
     & $r_T$ & $c_T$ & $r_{S_1}$ & $r_{S_2}$ & $r_{S_3}$ & $r_{S_4}$ & $c_{S_1}$ & $c_{S_2}$ & $c_{S_3}$ & $c_{S_4}$ \\ \midrule \midrule
    Expedia & $942$K  & $52.3$K & $942$K  & $11.9$K & $37$K   & -    & $27$    & $12$K   & $40.2$K & -      \\
    Flight  & $66.5$K & $13.7$K & $66.5$K & $540$   & $3.17$K & $3.17$K & $20$    & $718$   & $6.46$K & $6.47$K \\
    Lastfm  & $344$K  & $55.3$K & $5$K    & $50$K   & -       & -    & $5.02$K & $50.2$K & -      & -       \\
    Movie   & $1$M    & $13.3$K & $6.04$K & $3.71$K & -       & -    & $9.51$K & $3.84$K & -      & -       \\
    Yelp    & $216$K  & $55.6$K & $11.5$K & $43.9$K & -       & -    & $11.7$K & $43.9$K & -      & -       \\
    \bottomrule
  \end{tabular}
\vspace{-2mm}
\end{table}

\para{TPC-DI Benchmark} 
For large-scale evaluation, we used the TPC-DI benchmark~\cite{tpcdi}, involving four datasets across three sources. The fact table, \emph{Trade}, is joined with key attributes to produce a target table with 27 features. Tab.~\ref{tab:tpc_data} outlines table cardinalities across various scale factors, highlighting data characteristics in different scenarios. 

\begin{table}[t]
\caption{Data sizes of realistic data integrations scenario based on TPC-DI benchmark. The table shows the number of rows in sources and  target table w.r.t varying scale factors.}
\label{tab:tpc_data}
\vspace{1mm}
\centering
\footnotesize
\begin{tabular}{|c|l|l|ll|l|}
\hline
              & \textbf{Source 1} & \textbf{Source 2}  & \multicolumn{2}{c|}{\textbf{Source 3}}         &    \textbf{Target}      \\ \hline
\textbf{Scale factor} & Trade   & Customer & \multicolumn{1}{l|}{Stock} & Reports &  -   \\ \hline
3             & 391.1K & 4.7K    & \multicolumn{1}{l|}{2.6K} & 98.7K  & 528.8K   \\ \hline
7             & 911.6K & 108K   & \multicolumn{1}{l|}{5.8K} & 297.7K & 1.3M \\ \hline
\end{tabular}
\vspace{-4mm}
\end{table}



% \para{Real datasets}
% \label{s:setup_real} 
% We also use the seven real-world datasets from Morpheus and its extensions \cite{kumar2016join, \hamletplusplus, chen2017towards, \morpheusfi}, as provided by Project Hamlet\footnote{Project Hamlet datasets: \url{https://adalabucsd.github.io/hamlet.html}} \cite{kumar2016join, \hamletplusplus}. Each dataset consists of two or more tables connected in a star schema with PK-FK relationships. $nnz$ indicates the number of nonzero elements. We include the characteristics of these datasets in the technical report \cite{tech}.

\para{Hardware and software} We run experiments with 16, and 32 cores of AMD EPYC 7H12 CPU, and Nvidia A40 GPU. To enable parallelized sparse matrix multiplication on both CPU and GPU, we choose MKL and CuBLAS as LA library respectively. The number of model training iterations is 200. 

\vspace{-2mm}
\subsection{Performance Evaluation with Synthetic Data}
\label{ssec:simpleDI}
\vspace{-2mm}
In this section, we conduct a performance comparison between factorized learning and learning over materialization on both CPUs and GPUs. The purpose of this comparison is to demonstrate the significant acceleration that GPUs can provide for factorized learning, as well as the performance of factorized learning across various data characteristics.

% \vspace{-1mm}
% \subsubsection{Performance on CPUs and GPUs}
% We begin by comparing the performance of factorized ML operators and training algorithms on 32-core CPU and GPU. Fig.~\ref{fig:eval_cpu_gpu} illustrates the speedups achieved by factorized linear operators and ML model training algorithms on GPU relative to their CPU counterparts. The factorized operators on GPU can achieve speedups of up to 68x. Specifically, scalar multiplication benefits most significantly from the high parallelism of GPU, achieving the greatest speedup, while matrix multiplications generally achieve more moderate speedups. This is due to the internal data dependencies during computation and the intensive memory access, which introduce considerable overheads. A notable observation is that right matrix multiplication (\(XT^T\)) is particularly impacted by these overheads due to inefficient column-wise memory access in a row-major matrix configuration.
% % We show the results first at the LA operator level, then the overall performance at the model level. 

% The observed speedups for models range from 2.5x to 8.9x, which are less significant compared to those of individual operators. Although linear operators serve as the foundational components of model training algorithms, the time consumed by the data movement of intermediate results cannot be overlooked. This issue is particularly significant in model training, which typically involves multiple epochs; thus, the overhead associated with data movement becomes more influential. Consequently, this overhead is reflected in the reduced speedups relative to the higher acceleration observed in individual linear operators.

% \mybox{\textbf{Takeaways:} 
% Factorized model training on GPUs can achieve significant speedups compared to training on CPUs. 
% }

\vspace{-3mm}
\subsubsection{Performance of factorization and materialization}
Fig.~\ref{fig:eval_f_m} shows the speedups of factorization over materialization (\(\frac{T_{materialization}}{T_{factorization}}\)) across three hardware setups. Factorized scalar operators achieve greater speedups on GPUs compared to CPUs, while matrix operators show smaller gains. On CPUs, matrix operators outperform scalar operators, but this trend reverses on GPUs. This discrepancy arises because GPU-based matrix operations are often limited by memory bandwidth, as discussed in Section~\ref{sec:3rdfac}. The high parallelism of GPUs generally benefits compute-bound tasks more than memory-bound operators.

For model training, factorization still offers speed improvements, albeit more modestly. This is due to the frequent read/write operations for intermediate results in training algorithms, which diminishes speedups, especially on GPUs where memory bandwidth constraints often dominate.

\vspace{-3mm}
\subsubsection{Speedups regarding sparsity and complexity ratio}
Fig.~\ref{fig:heatmap_cpu_gpu} illustrates how speedups vary with varying sparsity and complexity ratio on CPUs and GPUs. The speedups presented in the figure represent the average benefit when factorization proves more advantageous. From the CPU results, an increase in speedups is observed as the complexity ratio grows. Within certain intervals of complexity ratio, speedups increase as sparsity decreases. However, sparsity alone is not a reliable estimator across all complexity ratios.

The results on GPUs, in contrast, show no observable trend, suggesting that empirical threshold-based estimators may not function effectively on GPUs. This observation underscores the necessity for a learned estimator capable of integrating features of both hardware configurations and data characteristics.

% \begin{figure}[t]
%     \centering
%     \includegraphics[width=0.8\linewidth]{figures/new_eval_figures/spdup_GPU_CPU.pdf}
%      \vspace{-4mm}
%     \caption{
%     Average speedups (\(\frac{\text{Time}_{CPU}}{\text{Time}_{GPU}} \)) of LA operators and model training w.r.t varying input parameters. All operations can be generally accelerated by GPUs due to high parallelism.
%     }
%     \label{fig:eval_cpu_gpu}
%     \vspace{-5mm}
% \end{figure}


\begin{figure}[t]
    \centering
    \includegraphics[width=0.8\linewidth]{figures/new_eval_figures/spdup_f_m.pdf}
        \vspace{-1.5mm}
    \caption{
    Speedups (\(\frac{\text{Time}_{materialization}}{\text{Time}_{factorization}} \)) of LA operators and model training w.r.t varying input parameters. \emph{Here we focus on the cases that factorization performs faster than materialization.}
    }
    \label{fig:eval_f_m}
    \vspace{-4mm}
\end{figure}

\begin{figure}[t]
    \centering
    \includegraphics[width=\linewidth]{figures/new_eval_figures/heatmap_cpu_gpu.pdf}
    \caption{
        Speedups (\(\frac{\text{Time}_{materialization}}{\text{Time}_{factorization}} \)) w.r.t target table sparsity and complexity ratio on CPUs and GPUs. On CPUs, speedups increase when complexity ratio gets larger. No observable trend on GPUs. 
    }
    \label{fig:heatmap_cpu_gpu}
    \vspace{-4mm}
\end{figure}



\mybox{\textbf{Takeaways:} 
The significant variability in performance comparison between factorization and materialization, highlights \textbf{the need for a cost estimator}. 
% to identify the most efficient approach for model training when both factorization and materialization are feasible options. 
}



\vspace{-1mm}
\subsection{Effectiveness of Our Estimator}
\label{ex:estimator}
\vspace{-2mm}
This section evaluates the tree boosting cost estimator as outlined in Section \ref{s:estimator}. Our analysis focuses not only on the predictive accuracy of the estimator but also on its performance enhancement in batch model training scenarios. 

% \subsubsection{Configuration}\hfill
% \label{ex:est_config}

\para{Training and test data}
% In Sec.\ref{ssec: realDI}, we evaluated the performance of four models using both factorization and materialization techniques in realistic data integration scenarios. In this section, \rihan{delete before} 
We extract features described in Sec.~\ref{s:estimator} from the evaluation conducted on synthetic data. This process generated 7200 pairs (1800*4), each labeled to indicate whether factorization is faster than materialization (True) or not (False). The dataset, containing these features and labels, is then randomly divided into training (80\%) and test (20\%) sets.


\para{Evaluation metrics} 
We evaluate our estimator model using accuracy and F1-score, and compare its performance against other baseline estimators. Accuracy serves to gauge the effectiveness of our cost estimator in accurately recommending the faster training method—either factorization or materialization. Furthermore, we measure the end-to-end speedup of batch model training tasks when decisions between factorization and materialization are guided by our cost estimator. 

\para{Four baselines for comparison} 
% We compared the performance of our proposed estimator in Sec. \ref{s:estimator} with the following five baselines. 
% \\ 
\texttt{i)} To confirm the significance of hardware features, 
% apart from the tree boosting model in Sec. \ref{s:estimator}, 
we train another tree boosting model using the same dataset, but \emph{without hardware features}. 
\texttt{ii)} The \emph{tuple ratio and feature ratio} (TR \& FR), metrics used in Morpheus \cite{chen2017towards}, quantify the target table's redundancy relative to the joinable tables. Morpheus \cite{chen2017towards} suggests a threshold of 5 for tuple ratio and 1 for feature ratio.
 \texttt{iii)} Finally, we consider the heuristic rule proposed in \cite{MorpheusFI}.
\begin{table}[t]
\caption{Accuracy, F1-score and Overall Speedups of estimators tested on CPUs and GPUs. Our cost estimator shows superior predictive quality and effectiveness over synthetic test data.}
\label{tab:overall_accuracy}
% \vspace{1mm}
\small
\centering
\begin{tabular}{|l|l|l|l|}
\hline
\textbf{Cost estimators}                        & \textbf{Accuracy}       & \textbf{F1-score}  & \textbf{Overall Speedups} \\  \hline
\multicolumn{4}{|c|}{CPU results} \\ \midrule
Tree Boosting (ours)          & \textbf{0.971} & \textbf{0.936} & \textbf{1.24} \\ \hline
    Tree Boosting w/o hardware        &0.712 & 0.560 & 1.19\\ \hline
TR \& FR \cite{chen2017towards}            & 0.490          & 0.426 & 1.16\\ \hline
MorpheousFI             & 0.864          & 0.601        & 1.22  \\ \hline
\hline
\multicolumn{4}{|c|}{GPU results} \\ \midrule
Tree Boosting (ours)          & \textbf{0.899} & \textbf{0.821} & \textbf{1.15} \\ \hline
    Tree Boosting w/o hardware        &0.523 & 0.443 & 1.09\\ \hline
TR \& FR \cite{chen2017towards}            & 0.244          & 0.362 & 0.86\\ \hline
MorpheousFI             & 0.497          & 0.477        & 0.96  \\ \hline
\end{tabular}
\vspace{-4mm}
\end{table}

\vspace{-2mm}
\subsubsection{Results on synthetic data} 
Tab.~\ref{tab:overall_accuracy} shows that our tree boosting model outperforms others with a 97.1\% accuracy and high F1-score, confirming its strong precision-recall balance. Excluding hardware features significantly reduces both accuracy and F1-score, supporting the importance of hardware information in choosing between factorization and materialization. The two threshold-based estimators perform worse in both metrics.

Beyond quality metrics, our estimator achieves a 1.24x speedup over training exclusively with materialization, outperforming all baselines. On GPUs, however, the effectiveness of all estimators declines, with threshold-based models performing worse than default materialization. This highlights the sensitivity of empirical thresholds to hardware variations, as discussed in Sec.~\ref{sec:3rdfac}, and demonstrates the limitations of threshold-based methods in cross-platform tasks.



% \para{Robustness test}
% Fig.~\ref{fig:confusion} presents a confusion matrix from our tree boosting estimator, indicating over 20\% of test instances can benefit from factorized learning. On average, correctly predicted instances achieve a 1.42x speedup with factorization. Interestingly, even with mispredictions leading to incorrect factorization application, training still retains about 85\% of the expected performance, showcasing our estimator's robustness. A high positive correlation is observed between accurate predictions and speedups, especially in True-Negative instances.


\vspace{-5mm}
\subsubsection{Results on real-world data}
We further evaluated our estimator using the \textit{Project Hamlet} and \textit{TPC-DI} datasets to test its practical usability in realistic ML tasks, including multiple model training and hyperparameter tuning.

Tab.~\ref{tab:overall_estimator} shows that the speedups on both CPU and GPU platforms are consistent with those observed on synthetic data, with our tree boosting estimator consistently performing best on unseen data.
\begin{table}[t!]
\caption{Average speedups (\(\frac{\text{Time}_{materialization}}{\text{Time}_{factorization}} \)) of all models across all real-world data. Our tree boosting estimator can consistently accelerate batch model training workload on both CPUs and GPUs.}
\label{tab:overall_estimator}
\centering
\small
\begin{tabular}{|l|l|l|}
\hline
\textbf{Cost estimators} & \textbf{Speedups (CPU)} & \textbf{Speedups (GPU)} \\ \hline
Tree boosting (ours) & \textbf{1.21} & \textbf{1.16} \\ \hline
MorpheousFI & 1.18 & 0.88 \\ \hline
TR\&FR & 1.09 & 072 \\ \hline
\end{tabular}%
\vspace{-4mm}
\end{table}

Tab.~\ref{tab:effect-cpu} provides a breakdown of effectiveness across models and datasets on CPUs. While the estimator generally makes accurate decisions, its performance varies with different datasets. For denser datasets like \textit{flight} and \textit{lastfm}, incorrect decisions led to slower training compared to materialization. Still, when training a single model across various datasets, speedups reached up to 1.29x.

On GPUs (Tab.~\ref{tab:effect-gpu}), the results mirror the CPU findings but show slightly more mis-predictions. Speedups are generally lower on GPUs, consistent with synthetic data results. Predicting accurately for dense datasets like \textit{flight} and \textit{lastfm} remains challenging, as this high density is atypical for most star schema datasets.


\mybox{\textbf{Takeaways:}  Our tree boosting cost estimator, enhanced with hardware features, outperforms the state-of-the-art estimator, showing consistent effectiveness across diverse hardware platforms.}


% \cellcolor{white}  -> gray
\begin{table}[t!]
\caption{Detailed speedups(\(\frac{\text{Time}_{materialization}}{\text{Time}_{factorization}} \)) w.r.t different models and datasets on CPUs (32 cores). Our tree boosting estimator makes correct prediction in most cases but performs worse on \textit{flight} and \textit{lastfm}.}
\label{tab:effect-cpu}
\vspace{1mm}
\small
\centering
\begin{tabular}{p{25mm}|cccc|c}
% \hline
\tikz[diag text/.style={inner sep=0pt, font=\footnotesize},
      shorten/.style={shorten <=#1,shorten >=#1}]{%
        \node[below left, diag text] (def) {Datasets};
        \node[above right=2pt, diag text] (abc) {Models};
        \draw[shorten=4pt, very thin] (def.north west|-abc.north west) -- (def.south east-|abc.south east);}
& \cellcolor{white} G.NMF & \cellcolor{white} KMeans & \cellcolor{white} Lin.Reg & \cellcolor{white} Log.Reg & \cellcolor[HTML]{EFEFEF} Speedups/dataset \\ \hline
\rowcolor{green!5} 
\cellcolor{white} TPC-DI sf=3                                                   & 1.33                             & \cellcolor{red!10}1      & 1.12                            & 1.33                            & 1.10                                                                                    \\ \hline
\rowcolor{green!5} 
\cellcolor{white} TPC-DI sf=7                                                   & 1.23                             & \cellcolor{red!10}1      & 1.04                            & 1.20                            & 1.09                                                                                    \\ \hline
\rowcolor{green!5} 
\cellcolor{white} expedia                                                & 1.24                             & 1.56                           & \cellcolor{red!10}1       & 1.07                            & 1.17                                                                                    \\ \hline
\rowcolor{red!10}  
\cellcolor{white} flight                                                 & 0.57                             & 0.7                            & \cellcolor{green!5}1       & 0.49                            & 0.74                                                                                    \\ \hline
\rowcolor{green!5} 
\cellcolor{white} lastfm                                                 & 1                                & \cellcolor{red!10}0.9    & 1                               & 1                               & \cellcolor{red!10}0.96                                                            \\ \hline
\rowcolor{green!5} 
\cellcolor{white} movie                                                  & 1.8                              & 2.35                           & 2.32                            & 1.88                            & 2.19                                                                                    \\ \hline
\rowcolor{green!5} 
\cellcolor{white} yelp                                                   & 1.10                             & 1.24                           & 1.51                            & \cellcolor{red!10}1       & 1.28                                                                                    \\ \hline\hline
\cellcolor[HTML]{EFEFEF}\begin{tabular}[c]{@{}l@{}}Speedups/model \end{tabular} & \cellcolor{green!5}1.28     & \cellcolor{green!5}1.21   & \cellcolor{green!5}1.29    & \cellcolor{green!5}1.18    & \#                                                                                        \\ %\hline
\end{tabular}
\vspace{-3mm}
\end{table}

% Please add the following required packages to your document preamble:
% \usepackage[table,xcdraw]{xcolor}
% Beamer presentation requires \usepackage{colortbl} instead of \usepackage[table,xcdraw]{xcolor}
% red -> \cellcolor{red!10} 
\begin{table}[t]
\caption{Detailed speedups(\(\frac{\text{Time}_{materialization}}{\text{Time}_{factorization}} \)) w.r.t different models and datasets on GPU. Our tree boosting estimator makes more wrong predictions then that on CPUs but still achieves overall speedups in most cases.}
\label{tab:effect-gpu}
\vspace{1mm}
\small
\centering
\begin{tabular}{p{25mm}|cccc|c}
% \hline
\tikz[diag text/.style={inner sep=0pt, font=\footnotesize},
      shorten/.style={shorten <=#1,shorten >=#1}]{%
        \node[below left, diag text] (def) {Datasets};
        \node[above right=2pt, diag text] (abc) {Models};
        \draw[shorten=4pt, very thin] (def.north west|-abc.north west) -- (def.south east-|abc.south east);}
                                                                                      & \cellcolor{white} G.NMF & \cellcolor{white} KMeans & \cellcolor{white} Lin.Reg & \cellcolor{white} Log.Reg & \cellcolor[HTML]{EFEFEF}Speedups/dataset \\ \hline
\rowcolor{green!5}  
\cellcolor{white} TPC-DI sf=3                                                   & 1.21                             & \cellcolor{red!10} 1      & 1.09                            & 1.23                            & 1.06                                                                                    \\ \hline
\rowcolor{green!5}  
\cellcolor{white} TPC-DI sf=7                                                   & 1.17                             & \cellcolor{red!10} 1      & 1                               & 1.13                            & 1.03                                                                                    \\ \hline
\rowcolor{green!5}  
\cellcolor{white} book                                                   & 1                                & 1                              & 1                               & 1                               & 1                                                                                       \\ \hline
\rowcolor{green!5}  
\cellcolor{white} expedia                                                & 1.24                             & 1.07                           & \cellcolor{red!10} 1       & \cellcolor{red!10} 1       & 1.05                                                                                    \\ \hline
\rowcolor{red!10} 
\cellcolor{white} flight                                                 & 0.52                             & 0.67                           & \cellcolor{green!5} 1       & 0.52                            & 0.71                                                                                    \\ \hline
\rowcolor{red!10} 
\cellcolor{white} lastfm                                                 & 0.94                             & 0.89                           & \cellcolor{green!5} 1       & \cellcolor{green!5} 1       & 0.91                                                                                    \\ \hline
\rowcolor{green!5}  
\cellcolor{white} movie                                                  & 1.69                             & 1.77                           & 2.98                            & 1.78                            & 1.97                                                                                    \\ \hline

\rowcolor{green!5}  
\cellcolor{white} yelp                                                   & 1.06                             & 1.19                           & 1.33                            & 1.28                            & 1.23                                                                                    \\ \hline\hline
\cellcolor[HTML]{EFEFEF}\begin{tabular}[c]{@{}l@{}}Speedups/model\end{tabular} & \cellcolor{green!5} 1.18     & \cellcolor{green!5} 1.09   & \cellcolor{green!5} 1.21    & \cellcolor{green!5} 1.08     & \#                                                                                       \\ %\hline
\end{tabular}
\vspace{-2mm}
\end{table}





% \subsection{Feature Importance Verification}
% \label{ex:feature_importance}
% Training the boosting tree model enables extraction of node information gain for ranking feature importance, as shown Table~\ref{tab:info_gain}. We have 33 features in total. Due to space restrictions, we list top 10 features here and leave the comprehensive details documented in the technical report \cite{tech}.  
% Notably, complexity features of LMM, a common operator, exhibit high information gain. Operators like rowSum and colSum, which are difficult to parallelize, are crucial for performance determination. Besides complexity features, memory I/O features, especially tied to dense operators, notably affect performance, as computations often yield dense intermediate results, increasing I/O costs. The tuple ratio ranks lower, implying a lesser impact on cost estimation.
% \begin{table}[t]
% \caption{Top 10 information gain of features in the estimator.}
% \label{tab:info_gain}
% \small
% \begin{tabular}{|l|l|r|}
% \hline
% \textbf{Rank} & {\textbf{Top 10 features}}                  & {\textbf{Info. Gain}} \\\bottomrule 
% 1 & {rowSum mem. write}              & {30.96}                  \\ \hline
% 2 & {LMM complexity with factorization}                & {22.69}                  \\ \hline
% 3 & {Dense scalar operation complexity}                & {20.53}                                \\ \hline
% 4  & {LMM complexity with materialization}          & {8.68}                  \\ \hline
% 5 & {colSum complexity with materialization}               & {6.30}                  \\ \hline

% 6 & {Complexity ratio}            & {6.21}                   \\ \hline
% 7 & {Total mem. write with materialization} & {5.88}                   \\ \hline
% 8 & {Dense MM operation complexity}  & {5.63}                   \\ \hline
% 9 & {Dense MM mem. write}        & {5.77}                   \\ \hline

% 10 & {Feature Ratio}   & { 5.45}                   \\ \hline
% \end{tabular}
% \end{table}




% Our feature importance analysis offers valuable insights for designing a performance cost estimator for trade-offs between materialization and factorization, highlighting the importance of considering hardware features.


% \mybox{\textbf{Takeaways:} 
% \\Hardware features are crucial to build a cross-platform cost estimator for factorized learning.}
% The process of training the boosting tree model allowed us to extract the information gains of each node, which we used as indicators of feature importance. Table \ref{tab:info_gain} presents the top-10 features, ranked in descending order by their information gains.

% This occurs, for instance, when performing operations like colSum and rowSum, which produce dense vector results. The impact of these operations is often overlooked, despite their substantial I/O costs.

% An interesting observation from this analysis is the significant importance of features related to memory I/O operations, particularly for dense operators. Although the raw data used in our experiments are stored in a sparse format, the intermediate results generated during computations are often converted to a dense format. This occurs, for instance, when performing operations like colsum and rowsum. These dense, I/O-intensive operators can substantially influence the overall performance due to their high I/O costs. This effect is typically overlooked despite the competitive I/O cost associated with sparse operators.



% Furthermore, it is worth highlighting the importance of the complexity ratio. While not the most significant feature, it plays a crucial role in the estimator alongside memory I/O features. Importantly, both the materialization complexity and the factorization complexity are inherently associated with the complexity ratio. In contrast, the Tuple Ratio, which was proposed in [], does not rank high in our model, indicating its lesser impact on the overall performance estimation.

% The insights gathered from our feature importance analysis highlight key considerations for designing a performance estimator for factorized learning. They particularly underscore the value of accounting for hardware features. Given the computational demands of factorized learning and the influence of specific hardware features on its efficiency, their incorporation in the performance estimation process can provide a more comprehensive, and hence, a more accurate prediction of performance outcomes.
% \vspace{-3mm}
% \subsection{Generalizability Verification with Ablation Study} 
% To evaluate the generalizability of our cost estimator, we utilized a leave-one-out cross-validation method, specific to different ML models to be trained. 

% Fig. \ref{fig:xgb_acc_category} indicates a decrease in both accuracy and F1-score when certain models are excluded from training data. This is due to missing operator features in the training set. For example, excluding the KMeans model, which heavily uses the colSum and rowSum, creates a predictive bias due to an unbalanced feature distribution.
% Despite these constraints, the estimator remains functional with an approximate 80\% accuracy rate. This suggests that even with certain predictive biases, our estimator is resilient and generalizable in optimizing ML workloads with new ML models, offering reasonably accurate predictions across different scenarios.


% \vspace{-2mm}


\input{5related_work}
\section{Conclusion}
We have presented Digital Twin Buildings, a framework for extracting the 3D mesh of a building, for connecting the building to Google Maps Platform APIs, and for Multi-Agent Large Language Models data analytics. We demonstrate this by extracting visual description keywords and captions of the building from multi-view multi-scale images of the building. The framework can also be used to process different data modalities sourced from Google Cloud Services. This approach enables richer semantic understanding, seamless integration with geospatial data, and enhanced interaction with real-world structures, paving the way for advanced applications in urban analytics, navigation, and virtual environments.

%
%
%

%
% ---- Bibliography ----
%
% BibTeX users should specify bibliography style 'splncs04'.
% References will then be sorted and formatted in the correct style.
%
\bibliographystyle{splncs04}
\bibliography{reference}
%
% \begin{thebibliography}{8}
% \bibitem{ref_article1}
% Author, F.: Article title. Journal \textbf{2}(5), 99--110 (2016)

% \bibitem{ref_lncs1}
% Author, F., Author, S.: Title of a proceedings paper. In: Editor,
% F., Editor, S. (eds.) CONFERENCE 2016, LNCS, vol. 9999, pp. 1--13.
% Springer, Heidelberg (2016). \doi{10.10007/1234567890}

% \bibitem{ref_book1}
% Author, F., Author, S., Author, T.: Book title. 2nd edn. Publisher,
% Location (1999)

% \bibitem{ref_proc1}
% Author, A.-B.: Contribution title. In: 9th International Proceedings
% on Proceedings, pp. 1--2. Publisher, Location (2010)

% \bibitem{ref_url1}
% LNCS Homepage, \url{http://www.springer.com/lncs}, last accessed 2023/10/25
% \end{thebibliography}
\end{document}
