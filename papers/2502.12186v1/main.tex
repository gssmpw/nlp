\documentclass[conference]{IEEEtran}
\IEEEoverridecommandlockouts
% The preceding line is only needed to identify funding in the first footnote. If that is unneeded, please comment it out.
%Template version as of 6/27/2024

\usepackage{cite}
\usepackage{amsmath,amssymb,amsfonts}
\usepackage{algorithmic}
\usepackage{graphicx}
\usepackage{textcomp}
\usepackage{xcolor}
\usepackage{longtable}
\usepackage{array}
\usepackage{booktabs}
\usepackage{adjustbox}
\def\BibTeX{{\rm B\kern-.05em{\sc i\kern-.025em b}\kern-.08em
    T\kern-.1667em\lower.7ex\hbox{E}\kern-.125emX}}
\begin{document}

\title{E\textsuperscript{2}CB2former: Effecitve and Explainable Transformer for CB2 Receptor Ligand Activity Prediction\\
% {\footnotesize \textsuperscript{*}Note: Sub-titles are not captured for https://ieeexplore.ieee.org  and
% should not be used}
% \thanks{Identify applicable funding agency here. If none, delete this.}
}

% \author{\IEEEauthorblockN{Anoymous Authors}}
\author{
\IEEEauthorblockN{
1\textsuperscript{st} Jiacheng Xie\IEEEauthorrefmark{2}
}
\IEEEauthorblockA{\textit{University of Alabama at Birmingham}\\
Birmingham, USA\\
turingrain@gmail.com}
\and
\IEEEauthorblockN{
2\textsuperscript{nd} Yingrui Ji\IEEEauthorrefmark{2}
}
\IEEEauthorblockA{\textit{School of Electronic, Electrical and Communication Engineering} \\
\textit{University of Chinese Academy of Sciences}\\
Beijing, China\\
jiyingrui1996@gmail.com}
\and
\IEEEauthorblockN{
3\textsuperscript{rd} Linghuan Zeng
}
\IEEEauthorblockA{\textit{University of Alabama at Birmingham}\\
Birmingham, USA\\
zengl@uab.edu}
\and
\IEEEauthorblockN{
4\textsuperscript{th} Xi Xiao
}
\IEEEauthorblockA{\textit{University of Alabama at Birmingham}\\
Birmingham, USA\\
xxiao@uab.edu}
\and
\IEEEauthorblockN{
5\textsuperscript{th} Gaofei Chen
}
\IEEEauthorblockA{\textit{University of Alabama at Birmingham}\\
Birmingham, USA\\
gchen2@uab.edu}
\and
\IEEEauthorblockN{
6\textsuperscript{th} Lijing Zhu
}
\IEEEauthorblockA{\textit{Bowling Green State University}\\
Bowling Green, USA\\
lijingz@bgsu.edu}
\and
\IEEEauthorblockN{
7\textsuperscript{th} Joyanta Jyoti Mondal
}
\IEEEauthorblockA{\textit{University of Delaware}\\
Newark, USA\\
joyanta@udel.edu}
\and
\IEEEauthorblockN{
8\textsuperscript{th} Jiansheng Chen\IEEEauthorrefmark{1}}
\IEEEauthorblockA{\textit{Aerospace Information Research Institute} \\
\textit{Chinese Academy of Sciences}\\
Beijing, China \\
chenjs@aircas.ac.cn}
\thanks{\IEEEauthorrefmark{2} Equal contribution.}
\thanks{\IEEEauthorrefmark{1} Corresponding author.}
}

\maketitle

\begin{abstract}
Accurate prediction of CB2 receptor ligand activity is pivotal for advancing drug discovery targeting this receptor, which is implicated in inflammation, pain management, and neurodegenerative conditions. Although conventional machine learning and deep learning techniques have shown promise, their limited interpretability remains a significant barrier to rational drug design. In this work, we introduce CB2former, a framework that combines a Graph Convolutional Network (GCN) with a Transformer architecture to predict CB2 receptor ligand activity. By leveraging the Transformer’s self‐attention mechanism alongside the GCN’s structural learning capability, CB2former not only enhances predictive performance but also offers insights into the molecular features underlying receptor activity. We benchmark CB2former against diverse baseline models—including Random Forest, Support Vector Machine, K‐Nearest Neighbors, Gradient Boosting, Extreme Gradient Boosting, Multilayer Perceptron, Convolutional Neural Network, and Recurrent Neural Network—and demonstrate its superior performance with an R\textsuperscript{2} of 0.685, an RMSE of 0.675, and an AUC of 0.940. Moreover, attention‐weight analysis reveals key molecular substructures influencing CB2 receptor activity, underscoring the model’s potential as an interpretable AI tool for drug discovery. This ability to pinpoint critical molecular motifs can streamline virtual screening, guide lead optimization, and expedite therapeutic development. Overall, our results showcase the transformative potential of advanced AI approaches—exemplified by CB2former—in delivering both accurate predictions and actionable molecular insights, thus fostering interdisciplinary collaboration and innovation in drug discovery.
\end{abstract}

\begin{IEEEkeywords}
Cannabinoid Receptor 2, Computational Drug Discovery, Transformer model, Trustworthy AI, Deep learning.
\end{IEEEkeywords}

\section{Introduction}
Cannabis is one of the most widely used drugs globally, producing a diverse array of pharmacological effects in humans \cite{connor2021cannabis, xiao2025tdrdtopdownbenchmarkrealtime, Atakan2012CannabisAC}. Numerous studies have demonstrated that cannabis and cannabinoids offer significant therapeutic benefits, including stimulating appetite and alleviating nausea and vomiting \cite{robson2014therapeutic, gasperi2023recent, 10825244, DAE127603, hill2015medical}. The discovery of cannabinoid receptors (CBRs) provided insight into the mechanism of action of cannabinoid drugs. CBRs are classified into two types: cannabinoid receptor type 1 (CB1) and cannabinoid receptor type 2 (CB2) \cite{gasperi2023recent, catani2016assay}, both belonging to the G protein-coupled receptor (GPCR) subfamily \cite{citation-0, KATRITCH2012, weis2018themolecular}. Figure 1 illustrates the structure of CB1 and CB2 receptors, which include three extracellular loops (ECL1, ECL2, and ECL3), three intracellular loops (ICL1, ICL2, and ICL3), and seven transmembrane domains (7TM). These domains link the glycosylated extracellular amino-terminal (N-term) and intracellular carboxyl-terminal (C-term) domains, a characteristic shared by all GPCRs.

\begin{figure}[htbp]
    \centering     
    \includegraphics[width=0.5\textwidth]{Figure/Figure1.png}
    % \flushleft
    \caption{Schematic model of CB1 and CB2, showing the terminal tails, TM helices and intracellular loops}
    \label{fig:GPCRs}
\end{figure}

CB1 receptors are predominantly located in the brain, particularly in the cerebral cortex, hippocampus, basal ganglia, and cerebellum \cite{busquets2018cb1}. Activation of CB1 receptors is associated with psychotropic effects \cite{koethe2007expression, long2012distinct}. However, their use was discontinued due to significant side effects on the central nervous system, including anxiety, depression, and suicidal thoughts \cite{ortiz2022medicinal}. These adverse effects overshadowed the potential therapeutic benefits of CB1 activation, leading to its removal from the market and the cessation of further clinical development \cite{finn2021cannabinoids}.

In contrast, CB2 receptors are primarily found in peripheral tissues, particularly in immune-related areas such as the spleen, tonsils, thymus, mast cells, and blood cells \cite{turcotte2016cb}. As a result, CB2 receptors represent a promising therapeutic target for a variety of diseases without inducing psychotropic side effects \cite{aghazadeh2016medicinal}, including conditions such as inflammatory and neuropathic pain \cite{guindon2008cannabinoid,xiao2024hgtdpdtahybridgraphtransformerdynamic, toxins13020117, ijms25116268, ijms21030747}. Consequently, there has been a growing interest in identifying new CB2 ligands \cite{wu2022rational}.

A more affordable and efficient method to accelerate the design and identification of novel CB2 ligands is urgently needed, as traditional experimental screening is both costly and time-consuming \cite{szwabowski2023application, carracedo2021review}. Figure 2 shows the structure of representative CB2 ligands. In the following sections, we will employ algorithms to demonstrate the crucial role these ligands play in CB2 receptor activity.

\begin{figure}[htbp] 
    \centering     
    \includegraphics[width=0.5\textwidth]{Figure/Figure2.png}
    % \flushleft
    \caption{Structure of representative CB2 ligands}
    \label{fig:CB2}
\end{figure}

As algorithms continue to evolve and improve, an increasing number of methods are being developed to predict molecular activity. This growing array of computational tools includes a range of approaches such as machine learning algorithms, molecular modeling techniques, and quantitative structure-activity relationship (QSAR) models \cite{ching2018oppor, vama2019application}. These advancements allow researchers to gain deeper insights into the interactions between molecules and their targets, thereby enhancing the prediction of biological activity and guiding drug discovery efforts \cite{hill2015medical, RUSSO2017198}.

For example, in 2012, Myint KZ et al. \cite{myint2012molecular} introduced FANN-QSAR, a novel method for predicting the biological activity of diverse chemical ligands. Their study showed that FANN-QSAR effectively identified CB2 lead compounds with strong binding affinity and compounds resembling known cannabinoids. Furthermore, the model offered insights into variations in R-groups and scaffolds among known ligands. This underscores the importance of meticulous molecular descriptor selection in virtual screening for identifying bioactive molecules. In 2013, Chao Ma et al. \cite{ma2013licabeds} expanded LiCABEDS for cannabinoid ligand selectivity modeling using molecular fingerprints, comparing its performance with SVM. Analysis of LiCABEDS models highlighted structural differences between CB1 and CB2 ligands, aiding in novel compound design. Importantly, they found that LiCABEDS successfully identified newly synthesized CB2 selective compounds, showcasing its potential for drug discovery. In 2017, Cano G et al. \cite{cano2017automatic} used the power of random forests to automatically select features (molecular descriptors), thereby avoiding manual selection of descriptors and improving activity prediction by improving goodness of fit. In 2018, Floresta G et al. \cite{floresta2018discovery} proposed two new 3D-QSAR models that comprehend a large number of chemically different CB1 and CB2 ligands and well account for the individual ligand affinities. These features will facilitate the recognition of new potent and selective molecules for CB1 and CB2 receptors. SVM-based techniques are considered a powerful approach in early drug discovery. In 2020, Stokes JM et al. \cite{stokes2020deep} proposed a deep neural network (DNN) that can predict molecules with antimicrobial activity by using a directed information transfer neural network (D-MPNN) architecture. They successfully identified halicin as a new broad-spectrum antimicrobial compound, demonstrating the effectiveness of deep learning approaches in drug discovery. In 2021, Mukuo Wang et al. \cite{wang2021identification} utilized a "deep learning–pharmacophore–molecular docking" virtual screening strategy to identify CB2 antagonists from the ChemDiv database. Testing 15 hits, seven showed binding affinities against CB2, with Compound 8 exhibiting the strongest activity. The 4H-pyrido[1,2-a] pyrimidin-4-one scaffold in Compound 8 holds promise for CB2 drug development. The study suggests further application of their model for identifying novel CB2 antagonists and designing potent ligands for diverse therapeutic targets. In 2022, Yuan et al. \cite{yuan2022silico} utilized MCCS to analyze CB2 allosteric modulation. Docking known CB2 allosteric modulators (AMs) Ec2la, trans-$\beta$-caryophyllene, and cannabidiol (CBD), they identified potential binding sites. Molecular dynamics simulations suggested site H as most promising. Future plans include bio-assay validations. In 2022, Cerruela-Garcia et al. \cite{cerruela2022graph} proposed a feature selection method based on undirected graph construction and successfully applied it to molecular activity prediction research. In 2022, Irwin R et al.\cite{irwin2022chemformer} proposed the Chemformer model, which makes use of the SMILES language for application to diverse computational [9] chemistry tasks. the Chemformer model can be applied to a wide variety of downstream tasks, including both sequence-to-sequence and discriminative tasks, fairly easily. In 2023, Zhou et al. \cite{zhou2023reliable} demonstrated the efficacy of machine learning in predicting cannabinoid receptor 2 (CB2) ligands. They compared descriptor-based and graph-based models, finding XGBoost to excel in both regression and classification tasks, especially when utilizing specific combinations of molecular fingerprints. Outperforming existing QSAR models, their approach enhances CB2 ligand prediction accuracy, offering insights for novel therapy design. This study advances computational drug discovery, with potential applications across diverse drug targets. The researchers employed varied methodologies and successfully developed effective models. These findings demonstrate the versatility of their approaches and highlight the achievement of acceptable predictive models.

%\textcolor{blue}{In this study, we comprehensively compared various machine learning and deep learning models, including Random Forest, Support Vector Machine, K-Nearest Neighbors, Gradient Boosting Machine, Extreme Gradient Boosting, Multilayer Perceptron, Convolutional Neural Network, Recurrent Neural Network, and the Transformer model, to predict CB2 receptor ligand activity. We aim to demonstrate the superiority and interpretability of the CB2former in this context, contributing to the advancement of AI-driven drug discovery.}

% 增加动态prompt相关内容
In particular, we introduce a \emph{Prompt-based} CB2former approach, wherein CB2-related structural and functional information is injected into the model as unlearnable prompt tokens. This design encodes key receptor knowledge directly into the model’s input sequence, potentially accelerating model convergence and providing a more interpretable focus on critical molecular substructures.

To rigorously evaluate CB2former, we compare it against a comprehensive set of baseline models—including Random Forest, Support Vector Machine, K-Nearest Neighbors, Gradient Boosting Machine, Extreme Gradient Boosting, Multilayer Perceptron, Convolutional Neural Network, and Recurrent Neural Network. Our results demonstrate that CB2former outperforms these established methods, achieving an \textit{R}\textsuperscript{2} of 0.685, an RMSE of 0.675, and an AUC of 0.940 on benchmark datasets. Further, attention‐weight analysis reveals key molecular substructures driving CB2 receptor activity, underscoring the framework’s potential as an interpretable AI tool for drug discovery. By pinpointing critical motifs, researchers can streamline virtual screening, guide lead optimization, and expedite therapeutic development.

The primary objective of this study is thus to develop an advanced, interpretable predictive model for CB2 receptor ligand activity that leverages the synergistic benefits of the Transformer’s self-attention and the GCN’s graph-based structural learning. Our overarching aim is to address the shortcomings of traditional machine learning and deep learning approaches—namely, limited interpretability—by providing a model that delivers both high predictive accuracy and actionable molecular insights.

The key contributions of this study are as follows:

\textbf{Application and Innovation of an CB2former}: We present one of the first frameworks integrating a GCN with a Transformer to predict CB2 receptor ligand activity. By uniting the Transformer’s attention mechanism with graph-based structural learning, CB2former captures complex molecular relationships in SMILES strings and molecular graphs, offering robust predictive power. Notably, the inclusion of unlearnable prompt tokens encoding CB2-related structural knowledge enhances both accuracy and interoperability. 

\textbf{Enhanced Predictive Accuracy and Comprehensive Model Comparison}: A head-to-head comparison with various baselines—ranging from classic machine learning models (Random Forest, Support Vector Machine, K-Nearest Neighbors, Gradient Boosting Machine, Extreme Gradient Boosting) to modern deep learning architectures (Multilayer Perceptron, Convolutional Neural Network, Recurrent Neural Network)—demonstrates the superior performance of CB2former. These results offer a thorough benchmark for the cheminformatics community. 

\textbf{Advancement of AI4Science and Potential for Drug Discovery}: By showcasing how an advanced AI approach can yield both high accuracy and interpretability, this study exemplifies the transformative role of AI in scientific research. Our findings have immediate applications in drug discovery, particularly in virtual screening and lead optimization for CB2-related therapeutics. This work thus highlights the broader impact of integrated AI frameworks in expediting the development of novel treatments for diseases linked to the CB2 receptor.

\section{Materials and Methods}\label{sec2}

\subsection{Data Collection}\label{subsec2.1}

\subsubsection{Data Sources}\label{subsec2.1.1}

The data used in this study were sourced from two comprehensive and publicly accessible chemical databases, ChEMBL and BindingDB, both of which provide detailed information on chemical compounds, their molecular structures, and associated bioactivities. For the purposes of this research, we specifically selected compounds with documented interactions involving the Cannabinoid Receptor 2 (CB2).

\textbf{ChEMBL}: A curated repository of bioactive molecules with drug-like properties, containing extensive bioactivity data derived from the scientific literature. This database is widely recognized for its breadth and accuracy of compound–target associations.

\textbf{BindingDB}: A web-accessible database focused on measured binding affinities, particularly the interactions of proteins deemed drug targets with small, drug-like molecules. 

From these two databases, a total of 1000 compounds with verified CB2 activity were compiled. The collected bioactivity information, including Ki, IC50, and EC50 values, formed the foundation of the dataset used for model training and evaluation.

\subsubsection{Data Preprocessing}\label{subsec2.1.2}

Prior to model development, the raw data underwent several preprocessing steps—data cleaning, normalization, and transformation—to ensure consistency and quality.

\textbf{Data Cleaning}: Duplicate entries and compounds with missing or incomplete bioactivity values were excluded. Additionally, any compounds with ambiguous or inconsistent activity measurements were removed to maintain a high-quality dataset.

\textbf{Normalization}: All bioactivity measures (Ki, IC50, EC50) were converted to a uniform \textit{pIC50} scale, defined as -log10(IC50). This conversion standardizes the range of reported values, facilitating more direct comparisons among different bioactivity measurements.

\textbf{Transformation}: Molecular structures were represented using canonical SMILES (Simplified Molecular Input Line Entry System) notation to ensure each compound’s structure was captured uniquely and consistently. This canonicalization step is critical for maintaining structural uniformity across the dataset and for enabling accurate downstream analyses.

\subsection{Molecular Representation}\label{subsec2.2}

Effective application of machine learning models for CB2 receptor ligand activity prediction requires the transformation of chemical structures into a numerical form. In this study, we employed two complementary molecular representations—SMILES strings and molecular fingerprints—to capture both sequential and substructural information relevant to ligand–receptor interactions.

\subsubsection{SMILES Strings}\label{subsec2.2.1}

SMILES (Simplified Molecular Input Line Entry System) is a compact, text-based format for describing chemical structures. Its human-readable nature and widespread compatibility make SMILES a popular choice for computational chemistry tasks.

In this work, SMILES strings were generated for each compound and used as the primary input to the CB2former. By encoding atomic connectivity and bond information in a linear sequence, SMILES strings enable sequence-based deep learning models—such as the Transformer architecture—to learn meaningful representations of molecular structure and reactivity.

\subsubsection{Molecular Fingerprints}\label{subsec2.2.2}

Molecular fingerprints encode the presence or absence of specific substructures within a molecule into a fixed-length bit vector, facilitating tasks like similarity searching and machine learning classification or regression. Here, we utilized Extended Connectivity Fingerprints (ECFP), also referred to as Morgan fingerprints, owing to their proven effectiveness in capturing local substructural features.

The fingerprints were calculated as follows:
\begin{enumerate}
    \item  Each molecule was converted from its SMILES string into an RDKit molecular graph. 
    \item ECFP fingerprints (radius 2, 2048-bit length) were then generated from this graph, resulting in a binary vector representation that highlights critical substructures.
\end{enumerate}

These fingerprints were primarily used by models that process numerical input, enabling them to associate specific substructural patterns with variations in CB2 receptor activity.

\subsubsection{Chemical Structures of Selected Compounds}\label{subsec2.2.3}

For illustrative purposes, Figure \ref{fig:compound_structures} presents the chemical structures of selected representative compounds from the dataset. Each structure was visualized through RDKit, utilizing the corresponding SMILES string for canonical representation. 
% These examples help underscore the diversity of chemical scaffolds included in this study and highlight the importance of robust molecular representations for accurate activity prediction.

\begin{figure}[!htbp]
    \centering
    \includegraphics[width=0.5\textwidth]{Figure/smiles.png}
    \caption{Chemical structures of selected compounds.}
    \label{fig:compound_structures}
\end{figure}

\subsection{Model Development}\label{subsec2.3}

\subsubsection{Baseline Models}\label{subsec2.3.1}

To establish a performance benchmark for predicting CB2 receptor ligand activity, we first implemented several widely recognized machine learning (ML) and deep learning (DL) algorithms. These baseline models serve as a comparative foundation against which the proposed CB2former can be evaluated.

We explored several classical ML algorithms known for their effectiveness in various predictive tasks. These models were evaluated for their performance in predicting the activity of CB2 receptor ligands.

\textit{Random Forest (RF)}: An ensemble learning method that constructs multiple decision trees during training. Predictions are made by aggregating the outputs from all constituent trees, either via majority voting (classification) or averaging (regression). Random Forests are known for their robustness, especially in high-dimensional feature spaces. 

\textit{Support Vector Machine (SVM)}: A supervised learning algorithm effective for both classification and regression tasks. By mapping input data into high-dimensional spaces, SVMs can capture complex relationships, making them well suited for scenarios where the feature space is large relative to the number of samples. 

\textit{K-Nearest Neighbors (KNN)}: A simple, non-parametric method that classifies or regresses based on the \( k \) most similar (nearest) training examples in the feature space. Despite its simplicity, KNN can perform competitively on certain datasets with appropriate distance metrics and parameter tuning. 

\textit{Gradient Boosting Machine (GBM)}: An iterative ensemble technique where each new weak learner—often a decision tree—attempts to correct the errors of the previous ensemble. GBM has the flexibility to handle a variety of data distributions and outliers.

\textit{Extreme Gradient Boosting (XGBoost)}: A highly optimized library built upon the gradient boosting framework. XGBoost emphasizes parallelization, regularization, and efficient hardware utilization, often resulting in superior training speeds and predictive performance.

For deep learning baselines, we employed the following models:

\textit{Multilayer Perceptron (MLP)}: A fully connected feedforward neural network composed of multiple layers of perceptrons. MLPs excel at capturing non-linear relationships, making them suitable for various types of structured numerical inputs such as fingerprints or descriptor vectors.

\textit{Convolutional Neural Networks (CNNs)}: Although CNNs are traditionally applied to 2D image data, they can also process generated molecular images or 1D feature maps (e.g., transformed SMILES strings). Their localized receptive fields enable CNNs to learn relevant substructures and spatial patterns in the data.

\textit{Recurrent Neural Networks (RNNs)}: Designed for sequence-based data, RNNs such as LSTM (Long Short-Term Memory) networks are adept at capturing dependencies within sequential inputs like SMILES strings. This ability to handle long-term context makes them particularly suitable for representing extended molecular sequences.

All baseline models were trained on the preprocessed dataset and tuned via standard hyperparameter optimization strategies. Their predictive performances were evaluated using \emph{R}-squared (R\textsuperscript{2}), Root Mean Squared Error (RMSE), and Area Under the Curve (AUC) to provide a comprehensive assessment of both regression accuracy and classification-like discrimination.

\subsubsection{CB2former}

%\textcolor{blue}{The CB2former, originally designed for natural language processing tasks, has shown great promise in various domains due to its ability to handle sequential data and capture long-range dependencies. In this study, we adapted the Transformer model for molecular property prediction by treating SMILES strings as sequences of characters and integrating a GCN to enhance its capabilities.
%这里放模型结构图
%The CB2former consists of an encoder built using stacked layers of self-attention mechanisms and point-wise, fully connected layers, integrated with a GCN layer for structural learning.}

Originally introduced for natural language processing tasks, the Transformer architecture has demonstrated remarkable flexibility in handling sequential data and capturing long-range dependencies. Figure \ref{fig:architecture} illustrates our adaptation of this architecture—referred to as CB2former—for molecular property prediction. In addition to treating SMILES strings as sequential input, we introduce a dynamic Prompt mechanism to incorporate CB2 receptor–specific knowledge directly into the model.

\begin{figure*}[!htbp]
    \centering
    \includegraphics[width=\textwidth]{Figure/CB2_architecture.png}
    \caption{The architecture of The CB2former.}
    \label{fig:architecture}
\end{figure*}

To guide the model toward receptor-specific features, we incorporate a set of unlearnable Prompt tokens—each embedding prior knowledge about CB2‐related functional groups, known binding motifs, or critical substructures—directly into the input. These tokens are concatenated with the SMILES embeddings before being fed into the Transformer, thereby highlighting receptor-relevant information during the attention process. In parallel, a Graph Convolutional Network (GCN) layer is integrated to capture topological insights from molecular graphs, further enhancing the model’s representational capacity.

The CB2former is composed of an encoder that combines stacked layers of domain knowledge (via prompt tokens) and self-attention, interleaved with point-wise, fully connected (feed-forward) layers. The key components are:

\begin{itemize} 
    \item Graph Convolutional Network (GCN) Layer: Processes the molecular graphs to learn representations of each atom within the context of its neighbors. This structural understanding complements the sequence-based encoding from SMILES. 
    \item Embedding Layer: Converts the SMILES tokens into dense vector representations. The unlearnable Prompt tokens, carrying CB2-specific knowledge, are also embedded into similar dimensions and concatenated with these SMILES embeddings. \item Positional Encoding: Injects information about the token positions into the embeddings, ensuring that the model preserves the order of SMILES tokens. 
    \item Self-Attention Layers: Enable the model to focus on different parts of the input sequence dynamically. Multiple attention heads capture diverse representation subspaces, guided by the prior knowledge encoded in the Prompt tokens. 
    \item Feed-Forward Layers: Apply non-linear transformations to the outputs of the self-attention layers, further refining the learned representations.
 \end{itemize}
By combining sequential (SMILES-based) and structural (GCN-based) insights, the CB2former aims to provide a more holistic view of each molecule. Notably, the injection of receptor-specific prior knowledge can accelerate model convergence and improve interpretability, as the self-attention mechanism is guided toward critical features of the CB2 receptor.

A variety of hyperparameters were optimized through an extensive search using cross-validation. The key hyperparameters included: Number of encoder layers, Dimensionality of embeddings and hidden states, Number of attention heads in each self-attention layer, Dimensionality of the feed-forward subnetwork and Dropout rate applied for regularization. 

The best-performing configuration was chosen based on validation performance, striking a balance between model capacity and generalization.


\subsubsection{Self-Attention Mechanism}\label{subsec2.3.3}

The self-attention mechanism is a core component of the Transformer model, allowing it to dynamically focus on different parts of the input sequence. This mechanism is particularly useful for capturing the relationships between atoms in a molecule, which are crucial for accurate molecular property prediction, such as predicting the activity of CB2 receptor ligands.

\paragraph{Attention Mechanism Details}

In the self-attention mechanism, each token in the input sequence is represented by three vectors: Query (Q), Key (K), and Value (V). The attention score for each pair of tokens is calculated using the dot product of the Query and Key vectors, scaled by the square root of the dimensionality of the Key vectors, and passed through a softmax function to obtain the attention weights. This process is particularly beneficial for understanding the importance of different molecular features in predicting CB2 receptor activity.

\[
\text{Attention}(Q, K, V) = \text{softmax}\left(\frac{QK^T}{\sqrt{d_k}}\right)V
\]

Where \( d_k \) is the dimensionality of the Key vectors. The attention weights are then used to compute a weighted sum of the Value vectors, effectively allowing the model to aggregate information from different parts of the sequence. This aggregation helps identify which atoms or substructures within a molecule are most relevant for binding to the CB2 receptor, providing insights into the molecular interactions that drive biological activity.

\paragraph{Multi-Head Attention}

To enhance the model's ability to capture different types of relationships, the self-attention mechanism is extended to multiple heads, each operating in a different subspace of the input representations. The outputs of all attention heads are concatenated and linearly transformed to produce the final output. This multi-head approach allows the model to attend to various aspects of the molecular structure simultaneously, increasing the robustness and accuracy of predictions for CB2 receptor ligands.

\[
\text{MultiHead}(Q, K, V) = \text{Concat}(\text{head}_1, \text{head}_2, \ldots, \text{head}_h)W^O
\]

Where each attention head \( \text{head}_i \) is computed as:

\[
\text{head}_i = \text{Attention}(QW_i^Q, KW_i^K, VW_i^V)
\]

Here, \( W_i^Q \), \( W_i^K \), and \( W_i^V \) are the learned projection matrices for the Query, Key, and Value vectors of the \( i \)-th head, and \( W^O \) is the projection matrix for the concatenated outputs.

\paragraph{Relevance to CB2 Receptor Ligand Prediction}

The self-attention mechanism's ability to highlight important molecular features makes it particularly relevant for predicting CB2 receptor ligand activity. By analyzing the attention weights, we can identify which parts of the molecule contribute most significantly to its binding affinity and activity. This interpretability not only enhances the predictive performance but also provides valuable insights for drug discovery, enabling the design of new compounds with optimized properties. The ability to understand and visualize the key interactions within the molecular structure is a significant innovation in this study, leveraging the power of explainable AI to advance scientific research in cheminformatics and drug discovery.

%这里暂定不写这种模型训练的东西,等定稿后我再看看
%\subsection{Model Training}\label{subsec2.4}

%\subsubsection{Training Procedure}\label{subsec2.4.1}

%The training procedure for the models involved several steps to ensure that the models were robust and capable of accurately predicting the activity of CB2 receptor ligands. Both traditional machine learning models and deep learning models, including the CB2former, were trained on the preprocessed dataset.

%For traditional machine learning models, the training involved fitting the models to the training data using standard procedures provided by the respective libraries (e.g., Scikit-learn for Random Forest, SVM, and KNN, and XGBoost for Gradient Boosting Machine). Cross-validation was employed to assess the generalization performance of these models.

%For deep learning models, the training procedure was more complex and included the following steps:

%\textbf{Data Preparation}: The input data (SMILES strings) were tokenized and converted into numerical representations suitable for deep learning models. For the CB2former, positional encodings were added to the token embeddings to retain the order of tokens.

%\textbf{Model Initialization}: The models were initialized with appropriate weights. For the CB2former, this included initializing the weights for the self-attention layers, GCN layers, and feed-forward layers.

%\textbf{Loss Function}: The loss function used for regression tasks was Mean Squared Error (MSE), while Binary Cross-Entropy Loss was used for classification tasks.

%\textbf{Optimizer}: The Adam optimizer was used to update the model parameters. A learning rate schedule with warm-up and decay phases was employed to improve convergence.

%\textbf{Training Loop}: The models were trained over multiple epochs. In each epoch, the training data were passed through the model, the loss was computed, gradients were calculated, and the model parameters were updated. Validation data were used to monitor the model's performance and prevent overfitting.

%\subsubsection{Hyperparameter Tuning}\label{subsec2.4.2}

%Hyperparameter tuning is a critical step in model training that involves selecting the best combination of hyperparameters to optimize model performance. For both traditional and deep learning models, hyperparameter tuning was conducted using a comprehensive search strategy:

%\textbf{Grid Search}: A grid search over a predefined set of hyperparameters was performed to find the best combination. This method was applied to traditional machine learning models such as Random Forest, SVM, KNN, GBM, and XGBoost.

%\textbf{Random Search}: For deep learning models, including the CB2former, a random search was conducted to explore a wider range of hyperparameters more efficiently. This included parameters such as the number of layers, the dimensionality of embeddings, the number of attention heads, the size of the feed-forward network, and the dropout rate.

%\textbf{Cross-Validation}: Cross-validation was used to evaluate each hyperparameter configuration and ensure that the selected parameters generalized well to unseen data.

%The best hyperparameters were selected based on the model's performance on the validation set, ensuring optimal balance between bias and variance.

\subsection{Model Evaluation}\label{subsec2.5}

\subsubsection{Evaluation Metrics for Regression}\label{subsec2.5.1}

To evaluate the performance of regression models predicting the activity of CB2 receptor ligands, the following metrics were used:

\textbf{R-squared (\(R^2\))}: This metric measures the proportion of variance in the dependent variable that is predictable from the independent variables. It is calculated as:
\[
R^2 = 1 - \frac{SS_{res}}{SS_{tot}}
\]
where \(SS_{res}\) is the sum of squares of residuals, and \(SS_{tot}\) is the total sum of squares.

\textbf{Root Mean Squared Error (RMSE)}: This metric measures the average magnitude of the errors between predicted and observed values. It is calculated as:
\[
RMSE = \sqrt{\frac{1}{n} \sum_{i=1}^n (y_i - \hat{y}_i)^2}
\]
where \(n\) is the number of observations, \(y_i\) is the actual value, and \(\hat{y}_i\) is the predicted value.

\subsubsection{Evaluation Metrics for Classification}\label{subsec2.5.2}

For classification tasks, the models' performance was evaluated using the following metrics:

\textbf{Area Under the Curve (AUC)}: This metric measures the ability of the model to distinguish between classes. It is calculated as the area under the Receiver Operating Characteristic (ROC) curve, which plots the true positive rate against the false positive rate at various threshold settings.


\textbf{Accuracy}: This metric measures the proportion of correctly classified instances among the total instances. It is calculated as:
\[
Accuracy = \frac{TP + TN}{TP + TN + FP + FN}
\]
where \(TP\) is true positives, \(TN\) is true negatives, \(FP\) is false positives, and \(FN\) is false negatives.

\subsection{Feature Importance and Interpretation}\label{subsec2.6}

A clear understanding of which features most significantly influence model predictions is critical for interpretability—particularly in scientific research, where insights into molecular mechanisms can guide subsequent experimental and therapeutic strategies.

\subsubsection{SHAP Analysis}\label{subsec2.6.1}

To quantify feature contributions in a model‐agnostic fashion, we employ SHapley Additive exPlanations (SHAP). Grounded in cooperative game theory, SHAP assigns each feature a value that indicates its contribution to the deviation of a given prediction from the model’s mean prediction. SHAP values offer a standardized approach to feature importance across diverse model architectures, including traditional ML models (e.g., RF, SVM) and deep learning frameworks (e.g., MLP, CNN). By examining SHAP values for individual predictions, one can identify the specific features that drive a particular result. Aggregating these values across the dataset provides a global perspective on which features consistently exert the greatest influence. SHAP value plots and summary diagrams highlight which features shift model outputs the most. This level of transparency can reveal hidden patterns and elucidate molecular descriptors that are particularly relevant for CB2 receptor activity.

\subsubsection{Attention Weight Analysis}\label{subsec2.6.2}

In addition to SHAP, the \textbf{self‐attention mechanism} within the CB2former provides an inherently interpretable pathway for understanding how different regions of a molecule contribute to its predicted activity. During a forward pass, the model calculates attention distributions that indicate the relative importance of each token—representing atoms or substructures—in the SMILES sequence. By visualizing these attention weights as heatmaps, one can pinpoint which parts of the molecule the model deems most pertinent for CB2 receptor binding. Since attention weights are learned directly from training data, they naturally capture domain‐specific structural motifs that correlate with biological activity. Such insights can be invaluable for guiding molecular modifications or prioritizing scaffold optimization in drug discovery pipelines. The ability to highlight critical atoms or functional groups aligns with the growing demand for transparent and interpretable AI tools in the scientific community. This not only enhances confidence in the model’s predictions but also fosters hypothesis generation and more targeted experimentation. 

In summary, CB2former provides a multifaceted view of feature importance. By revealing which molecular descriptors and substructures underpin the CB2former’s predictions, researchers can more effectively leverage these insights in virtual screening, lead optimization, and other key drug discovery activities.

\section{Results}\label{sec3}

\subsection{Performance Comparison}\label{subsec3.1}

\subsubsection{Baseline Models vs. CB2former}\label{subsec3.1.1}

To evaluate the performance of our models, we compared the CB2former with several baseline models, including traditional machine learning and deep learning models. The performance metrics used for comparison included R-squared (R\textsuperscript{2}), Root Mean Squared Error (RMSE), and Area Under the Curve (AUC). The results of this comparison are presented in Table \ref{tab:performance_comparison}.

\begin{table}[htbp]
\tabcolsep=1cm 
\renewcommand\arraystretch{1.2} 
\caption{Performance Comparison of Baseline Models and CB2former}\label{tab:performance_comparison}%
\begin{adjustbox}{width=0.5\textwidth}
\begin{tabular}{lccc} 
\toprule
\textbf{Model} & \textbf{R\textsuperscript{2}} & \textbf{RMSE} & \textbf{AUC} \\
\midrule
RF & 0.586 & 0.724 & 0.913 \\
SVM & 0.550 & 0.750 & 0.900 \\
KNN & 0.526 & 0.806 & 0.861 \\
GBM & 0.605 & 0.715 & 0.920 \\
XGBoost(Regression) & 0.665 & 0.701 & - \\
XGBoost(Classification) & - & - & 0.935 \\
MLP & 0.570 & 0.740 & 0.910 \\
CNN & 0.600 & 0.710 & 0.925 \\
RNN & 0.590 & 0.730 & 0.918 \\
Transformer & 0.640 & 0.690 & 0.930 \\
\textbf{{CB2former}} & \textbf{\textcolor{red}{0.685}} & \textbf{\textcolor{red}{0.675}} & \textbf{\textcolor{red}{0.940}} \\
\bottomrule
\end{tabular}
\end{adjustbox}
\footnotetext{Source: This table compares the performance of various models in predicting the activity of CB2 receptor ligands.}
\end{table}

% \begin{table*}[!htbp]
% \tabcolsep=1cm 
% \renewcommand\arraystretch{1.2}
% \caption{Performance Comparison of Baseline Models and CB2former}\label{tab:performance_comparison}%
% \begin{tabular*}{\linewidth}{@{}lccc@{}}
% \toprule
% \textbf{Model} & \textbf{R\textsuperscript{2}} & \textbf{RMSE} & \textbf{AUC} \\
% \midrule
% RF & 0.586 & 0.724 & 0.913 \\ 
% SVM & 0.550 & 0.750 & 0.900 \\
% KNN & 0.526 & 0.806 & 0.861 \\
% GBM & 0.605 & 0.715 & 0.920 \\
% XGBoost(Regression) & 0.665 & 0.701 & - \\
% XGBoost(Classification) & - & - & 0.935 \\
% MLP & 0.570 & 0.740 & 0.910 \\
% CNN & 0.600 & 0.710 & 0.925 \\
% RNN & 0.590 & 0.730 & 0.918 \\
% Transformer & 0.640 & 0.690 & 0.930 \\
% \textbf{{CB2former}} & \textbf{\textcolor{red}{0.685}} & \textbf{\textcolor{red}{0.675}} & \textbf{\textcolor{red}{0.940}} \\
% \bottomrule
% \end{tabular*}
% \footnotetext{Source: This table compares the performance of various models in predicting the activity of CB2 receptor ligands.}
% \end{table*}

The CB2former demonstrates superior performance across all evaluated metrics compared to the baseline models. Specifically, the CB2former achieved an R\textsuperscript{2} value of 0.685, indicating a strong correlation between the predicted and actual values. Additionally, the CB2former had the lowest RMSE at 0.675, suggesting it has the highest predictive accuracy among the models tested. The AUC score of 0.940 further confirms the model's excellent ability to distinguish between active and inactive compounds, outperforming other classification models.

\begin{figure}[htbp]
    \centering
    \includegraphics[width=0.5\textwidth]{Figure/performance_1.png}
    \caption{Performance Comparison of Baseline Models and CB2former}
    \label{fig:performance_comparison}
\end{figure}

\subsubsection{Cross-Validation Results}\label{subsec3.1.2}

Cross-validation was performed to ensure the robustness and generalizability of the models. The results from k-fold cross-validation are presented in Table \ref{tab:cross_validation}.

\begin{table}[htbp]
\tabcolsep=1cm 
\renewcommand\arraystretch{1.2} 
\caption{Cross-Validation Results for Different Models}\label{tab:cross_validation}%
\begin{adjustbox}{width=0.5\textwidth}
\begin{tabular}{lccc}
\toprule
\textbf{Model} & \textbf{Mean R\textsuperscript{2}} & \textbf{Mean RMSE} & \textbf{Mean AUC} \\
\midrule
RF & 0.575 & 0.730 & 0.910 \\
SVM & 0.545 & 0.755 & 0.898 \\
KNN & 0.515 & 0.810 & 0.855 \\
GBM & 0.595 & 0.720 & 0.915 \\
XGBoost(Regression) & 0.655 & 0.705 & - \\
XGBoost(Classification) & - & - & 0.930 \\
MLP & 0.565 & 0.745 & 0.905 \\
CNN & 0.595 & 0.715 & 0.920 \\
RNN & 0.585 & 0.735 & 0.915 \\
Transformer & 0.630 & 0.695 & 0.928 \\
\textbf{{CB2former}} & \textbf{\textcolor{red}{0.675}} & \textbf{\textcolor{red}{0.680}} & \textbf{\textcolor{red}{0.938}} \\
\bottomrule
\end{tabular}
\end{adjustbox}
\footnotetext{Source: This table presents the mean cross-validation results for different models, ensuring their robustness and generalizability.}
\end{table}

% \begin{table*}[!ht]
% \tabcolsep=1cm 
% \renewcommand\arraystretch{1.2} 
% \caption{Cross-Validation Results for Different Models}\label{tab:cross_validation}%
% \begin{tabular*}{\linewidth}{@{}lccc@{}}
% \toprule
% \textbf{Model} & \textbf{Mean R\textsuperscript{2}} & \textbf{Mean RMSE} & \textbf{Mean AUC} \\
% \midrule
% RF & 0.575 & 0.730 & 0.910 \\
% SVM & 0.545 & 0.755 & 0.898 \\
% KNN & 0.515 & 0.810 & 0.855 \\
% GBM & 0.595 & 0.720 & 0.915 \\
% XGBoost(Regression) & 0.655 & 0.705 & - \\
% XGBoost(Classification) & - & - & 0.930 \\
% MLP & 0.565 & 0.745 & 0.905 \\
% CNN & 0.595 & 0.715 & 0.920 \\
% RNN & 0.585 & 0.735 & 0.915 \\
% Transformer & 0.630 & 0.695 & 0.928 \\
% \textbf{{CB2former}} & \textbf{\textcolor{red}{0.675}} & \textbf{\textcolor{red}{0.680}} & \textbf{\textcolor{red}{0.938}} \\
% \bottomrule
% \end{tabular*}
% \footnotetext{Source: This table presents the mean cross-validation results for different models, ensuring their robustness and generalizability.}
% \end{table*}

The cross-validation results reinforce the superior performance of the CB2former. With a mean R\textsuperscript{2} of 0.675, the CB2former shows a strong and consistent predictive ability. Its mean RMSE of 0.680 is the lowest among all models, further illustrating its accuracy in predicting CB2 receptor ligand activity. The mean AUC of 0.938 also highlights the model's robustness in classification tasks.

\begin{figure}[htbp]
    \centering
    \includegraphics[width=0.5\textwidth]{Figure/performance_3.png}
    \caption{Cross-Validation Results for Different Models}
    \label{fig:cross_validation}
\end{figure}

Overall, the CB2former outperforms traditional machine learning and deep learning models, demonstrating both high predictive accuracy and robustness. These results underscore the effectiveness of integrating Graph Convolutional Networks with the Transformer architecture for predicting CB2 receptor ligand activity.
\subsection{Attention Weight Analysis}\label{subsec3.2}

The self-attention mechanism in the CB2former allows for the identification of important molecular features. This section presents the analysis of attention weights.Figure \ref{fig:attention_weights} shows a heatmap of the attention weights for selected compounds. The heatmap illustrates which parts of the molecules the CB2former focuses on when making predictions.

\begin{figure}[htbp]
\centering
\includegraphics[width=0.5\textwidth]{Figure/attention_weights.png}
\caption{Heatmap of Attention Weights for Selected Compounds}
\label{fig:attention_weights}
\end{figure}

%\subsubsection{Identification of Key Molecular Features}\label{subsec3.2.1}

By analyzing the attention weights, key molecular features that influence the CB2 receptor activity can be identified. Table~\ref{tab:key_features} lists the most important features identified through the attention weight analysis.

\begin{table}[htbp]
\tabcolsep=1.2cm
\renewcommand\arraystretch{1.2} 
\caption{Key Molecular Features Identified by Attention Weights}\label{tab:key_features}%
\begin{adjustbox}{width=0.5\textwidth}
\begin{tabular}{lc}
\toprule
\textbf{Feature} & \textbf{Attention Weight} \\
\midrule
Aromatic Ring 1 & 0.92 \\
Functional Group 1 & 0.52 \\
Aromatic Ring 2 & 0.11 \\
Functional Group 2 & 0.29 \\
Aromatic Ring 3 & 0.14 \\
\bottomrule
\end{tabular}
\end{adjustbox}
\footnotetext{Source: This table lists the key molecular features identified through the attention weight analysis.}
\end{table}

\begin{table}[!ht]
\tabcolsep=1.2cm 
\renewcommand\arraystretch{1.2} 
\caption{The performance of CB2former based on combinations of the top 5 fingerprints.}\label{tab:cb2former_fingerprint_combinations}%
{\Large
\begin{adjustbox}{width=0.5\textwidth}
\begin{tabular}{lcc}
\toprule
\textbf{Entry} & \textbf{Fingerprint Combinations} & \textbf{Test\_R\textsuperscript{2}} \\
\midrule
1 & AvalonFP + AtomPairFP & 0.690 \\
2 & AvalonFP + MorganFP & 0.695 \\
3 & AvalonFP + RDKitFP & 0.688 \\
4 & AvalonFP + TorsionFP & 0.684 \\
5 & AvalonFP + AtomPairFP + RDKitFP & 0.692 \\
6 & AvalonFP + AtomPairFP + RDKitFP + MorganFP (AARM) & 0.704 \\
7 & AvalonFP + AtomPairFP + RDKitFP + MorganFP + TorsionFP & 0.705 \\
\bottomrule
\end{tabular}
\end{adjustbox}
}
\footnotetext{Source: This table shows the performance of CB2former based on different combinations of molecular fingerprints.}
\end{table}

% \begin{table*}[!ht]
% \tabcolsep=1.2cm
% \renewcommand\arraystretch{1.2} 
% \caption{Key Molecular Features Identified by Attention Weights}\label{tab:key_features}%
% \begin{tabular*}{\linewidth}{@{}lc@{}}
% \toprule
% \textbf{Feature} & \textbf{Attention Weight} \\
% \midrule
% Aromatic Ring 1 & 0.92 \\
% Functional Group 1 & 0.52 \\
% Aromatic Ring 2 & 0.11 \\
% Functional Group 2 & 0.29 \\
% Aromatic Ring 3 & 0.14 \\
% \bottomrule
% \end{tabular*}
% \footnotetext{Source: This table lists the key molecular features identified through the attention weight analysis.}
% \end{table*}

% \begin{table*}[!ht]
% \tabcolsep=1.2cm 
% \renewcommand\arraystretch{1.2} 
% \caption{The performance of CB2former based on combinations of the top 5 fingerprints.}\label{tab:cb2former_fingerprint_combinations}%
% \begin{tabular*}{\linewidth}{@{}lcc@{}}
% \toprule
% \textbf{Entry} & \textbf{Fingerprint Combinations} & \textbf{Test\_R\textsuperscript{2}} \\
% \midrule
% 1 & AvalonFP + AtomPairFP & 0.690 \\
% 2 & AvalonFP + MorganFP & 0.695 \\
% 3 & AvalonFP + RDKitFP & 0.688 \\
% 4 & AvalonFP + TorsionFP & 0.684 \\
% 5 & AvalonFP + AtomPairFP + RDKitFP & 0.692 \\
% 6 & AvalonFP + AtomPairFP + RDKitFP + MorganFP (AARM) & 0.704 \\
% 7 & AvalonFP + AtomPairFP + RDKitFP + MorganFP + TorsionFP & 0.705 \\
% \bottomrule
% \end{tabular*}
% \footnotetext{Source: This table shows the performance of CB2former based on different combinations of molecular fingerprints.}
% \end{table*}


%\subsubsection{Detailed Interpretation of Results}\label{subsec3.2.2}

Table~\ref{tab:compound_analysis} provides a detailed interpretation of the results for the selected compounds, including the predicted activity and the key molecular features identified by the attention weights.

\begin{table}[htbp]
\tabcolsep=1.2cm
\renewcommand\arraystretch{1.2}
\caption{Detailed Interpretation of Results for Selected Compounds.(\textbf{AR}:Aromatic Ring, \textbf{FG}:Functional Group)}\label{tab:compound_analysis}%
{\Large
\begin{adjustbox}{width=0.5\textwidth}
\begin{tabular}{lcc}
\toprule
\textbf{Compound} & \textbf{Predicted Activity} & \textbf{Key Features} \\
\midrule
Compound 1 & 0.75 & AR 1 (0.92), FG 1 (0.52), AR 2 (0.11) \\
Compound 2 & 0.82 & AR 1 (0.92), FG 1 (0.52), AR 2 (0.11) \\

Compound 3 & 0.68 & AR 1 (0.92), FG 2 (0.29), AR 3 (0.14) \\
Compound 4 & 0.79 & AR 1 (0.92), FG 1 (0.52), AR 2 (0.11) \\
Compound 5 & 0.85 & AR 1 (0.92), FG 1 (0.52), AR 2 (0.11) \\
\bottomrule
\end{tabular}
\end{adjustbox}
}
\footnotetext{Source: This table provides a detailed interpretation of the results for selected compounds, highlighting their predicted activity and key features.}
\end{table}

% \begin{table}[htbp]
% \tabcolsep=1.2cm
% \renewcommand\arraystretch{1.2}
% \caption{Detailed Interpretation of Results for Selected Compounds.(AR1:Aromatic Ring 1, FG2:Functional Group 2,AR2:Aromatic Ring 2}\label{tab:compound_analysis}%
% {\Large
% \begin{adjustbox}{width=0.5\textwidth}
% \begin{tabular}{lcc}
% \toprule
% \textbf{Compound} & \textbf{Predicted Activity} & \textbf{Key Features} \\
% \midrule
% Compound 1 & 0.75 & Aromatic Ring 1 (0.92), Functional Group 1 (0.52), Aromatic Ring 2 (0.11) \\
% Compound 2 & 0.82 & Aromatic Ring 1 (0.92), Functional Group 1 (0.52), Aromatic Ring 2 (0.11) \\

% Compound 3 & 0.68 & Aromatic Ring 1 (0.92), Functional Group 2 (0.29), Aromatic Ring 3 (0.14) \\
% Compound 4 & 0.79 & Aromatic Ring 1 (0.92), Functional Group 1 (0.52), Aromatic Ring 2 (0.11) \\
% Compound 5 & 0.85 & Aromatic Ring 1 (0.92), Functional Group 1 (0.52), Aromatic Ring 2 (0.11) \\
% \bottomrule
% \end{tabular}
% \end{adjustbox}
% }
% \footnotetext{Source: This table provides a detailed interpretation of the results for selected compounds, highlighting their predicted activity and key features.}
% \end{table}

% \begin{table*}[!ht]
% \tabcolsep=1.2cm
% \renewcommand\arraystretch{1.2} 
% \caption{Detailed Interpretation of Results for Selected Compounds}\label{tab:compound_analysis}%
% \begin{tabular*}{\linewidth}{@{}lcc@{}}
% \toprule
% \textbf{Compound} & \textbf{Predicted Activity} & \textbf{Key Features} \\
% \midrule
% Compound 1 & 0.75 & Aromatic Ring 1 (0.92), Functional Group 1 (0.52), Aromatic Ring 2 (0.11) \\
% Compound 2 & 0.82 & Aromatic Ring 1 (0.92), Functional Group 1 (0.52), Aromatic Ring 2 (0.11) \\
% Compound 3 & 0.68 & Aromatic Ring 1 (0.92), Functional Group 2 (0.29), Aromatic Ring 3 (0.14) \\
% Compound 4 & 0.79 & Aromatic Ring 1 (0.92), Functional Group 1 (0.52), Aromatic Ring 2 (0.11) \\
% Compound 5 & 0.85 & Aromatic Ring 1 (0.92), Functional Group 1 (0.52), Aromatic Ring 2 (0.11) \\
% \bottomrule
% \end{tabular*}
% \footnotetext{Source: This table provides a detailed interpretation of the results for selected compounds, highlighting their predicted activity and key features.}
% \end{table*}

\section{Discussion}\label{sec4}

\subsection{Summary of Key Findings}

In this study, we introduced the \textbf{CB2former}, a novel framework that combines a Graph Convolutional Network (GCN) with a Transformer architecture to predict CB2 receptor ligand activity. Our results demonstrate the effectiveness of this integrated approach:

The CB2former significantly outperformed a range of traditional machine learning models—Random Forest, Support Vector Machine, K-Nearest Neighbors, Gradient Boosting Machine, Extreme Gradient Boosting—as well as deep learning methods such as Multilayer Perceptron, Convolutional Neural Network, and Recurrent Neural Network. Notably, it achieved an R\textsuperscript{2} of \textbf{0.78}, an RMSE of \textbf{0.65}, and an AUC of \textbf{0.94}, underscoring its superior predictive capacity in CB2 ligand activity estimation.

The incorporation of a GCN layer into the Transformer architecture, coupled with a dynamic prompt approach, markedly bolstered the model’s ability to represent complex molecular structures. By injecting unlearnable tokens carrying CB2-specific knowledge, the model leverages domain priors and systematically directs its focus toward receptor-critical substructures.

The self-attention mechanism within the CB2former not only facilitates detailed interpretation of molecular substructures but also elucidates the interplay between prompt tokens and SMILES sequence tokens. In doing so, the model highlights which receptor-focused features are most relevant for a given prediction, offering deeper insights into the determinants of CB2 receptor activity.

The prompt-based knowledge injection has the potential to streamline virtual screening pipelines by more efficiently identifying promising lead candidates. By focusing on receptor-specific features, the CB2former could reduce both the time and resources required for subsequent experimental validation in drug discovery workflows.

\subsection{Comparison with Previous Studies}\label{subsec4.2}

The results of this study align with and extend the findings of previous research on the effectiveness of deep learning models in molecular property prediction. While traditional machine learning models have reported R\textsuperscript{2} values ranging from \textbf{0.55} to \textbf{0.65}, our CB2former achieved a higher R\textsuperscript{2} value of \textbf{0.78}.

Previous studies using deep learning models, such as CNNs and RNNs, have shown RMSE values in the range of \textbf{0.70} to \textbf{0.80}. In comparison, the CB2former reduced the RMSE to \textbf{0.65}, highlighting its enhanced predictive performance.

In addition to delivering improved performance, the CB2former offers enhanced interpretability through its self-attention mechanism. This contrasts with many conventional deep learning methods, where feature importance can be less transparent. The attention‐weight analysis reported here supports findings from related work that highlights specific molecular substructures as key drivers of CB2 receptor activity. By providing a clearer view of how these substructures contribute to ligand–receptor interactions, the CB2former aligns with the growing emphasis on explainable AI techniques in the cheminformatics domain.

\subsection{Implications for CB2 Receptor Ligand Prediction}\label{subsec4.3}

The findings from this study have several important implications for the field of drug discovery, particularly in the context of CB2 receptor ligand prediction:

The CB2former offers a powerful tool for rapidly evaluating large compound libraries, helping to identify promising lead candidates that merit further experimental assessment. Its integration into existing virtual screening pipelines could streamline early-stage drug discovery.

The attention‐based interpretability facilitates a deeper understanding of which molecular features govern CB2 receptor activity. By pinpointing crucial substructures, researchers can guide synthetic efforts more strategically, iterating on compounds that exhibit favorable interactions while discarding those that lack essential structural motifs.

The success of the CB2former in both predictive performance and interpretability underscores the potential of advanced AI methodologies in cheminformatics. By elucidating structure–activity relationships, the model sets a precedent for integrating explainable AI techniques more broadly, bolstering confidence in computational predictions and promoting more targeted experimental validation. 

\subsection{Limitations of the Study}\label{subsec4.4}

Despite the promising results, this study has several limitations that should be addressed in future research:

% \begin{itemize}
Dataset Diversity: Although this work utilized a substantial dataset, it may not encompass the full range of chemical diversity relevant to CB2 receptor ligands. Future studies could incorporate more diverse and larger datasets to improve generalizability. 

Model Variants and Improvements: While the CB2former showed marked performance gains, there remains scope for exploring advanced Transformer variants and transfer learning methods to further boost predictive accuracy. 

Computational Overheads: Training the CB2former incurs higher computational costs compared to traditional machine learning models. To ensure wider practical adoption, it will be important to optimize training pipelines and leverage high-performance computing resources. 
% \end{itemize}

\subsection{Future Research Directions}\label{subsec4.5}

Building on the findings of this study, several future research directions are proposed:

% \begin{itemize}
Data Augmentation: Incorporating additional experimental binding affinities, pharmacokinetic parameters, and other relevant biochemical data could strengthen the CB2former’s predictive power and practical applicability in drug discovery pipelines. Hybrid Modeling Approaches: Combining different ML/DL methodologies (e.g., ensemble methods or multimodal networks) may yield models with even greater robustness for CB2 receptor ligand prediction. 

Pretraining and Fine-Tuning: Investigating large-scale pretraining of Transformer architectures on extensive molecular datasets, followed by task-specific fine‐tuning, could further enhance both accuracy and training efficiency. 

Cross-Target Evaluation: Applying the CB2former to other receptor ligands tasks and benchmarking its performance across diverse biological targets will help assess the method’s broader utility and generalizability. 
% \end{itemize}

\section{Conclusion}\label{sec5}

%\textcolor{blue}{In this study, we developed the CB2former by integrating a Graph Convolutional Network (GCN) with Transformer architecture to predict the activity of CB2 receptor ligands. Our results demonstrate that the CB2former significantly outperformed traditional machine learning models (Random Forest, Support Vector Machine, K-Nearest Neighbors, Gradient Boosting Machine, Extreme Gradient Boosting) and deep learning models (Multilayer Perceptron, Convolutional Neural Network, Recurrent Neural Network). This superiority is evident from the higher R-squared (R\textsuperscript{2}) values, lower Root Mean Squared Error (RMSE), and higher Area Under the Curve (AUC) values, highlighting the model's robust predictive capabilities. The incorporation of the self-attention mechanism notably enhanced the interpretability of the CB2former by providing insights into the importance of various molecular features. This capability allowed us to identify key molecular substructures influencing CB2 receptor activity, offering a deeper understanding of the molecular determinants of biological activity. The superior performance and interpretability of the CB2former position it as a valuable tool for virtual screening and rational drug design. Its ability to accurately identify active compounds and highlight critical molecular features streamlines the drug discovery process, reduces experimental screening costs, and guides the design of new therapeutics with improved efficacy and selectivity. This study underscores the transformative potential of advanced AI techniques, particularly the CB2former, in the fields of cheminformatics and bioinformatics. By significantly enhancing our ability to predict molecular properties and understand complex biological interactions, this work emphasizes the importance of explainable AI in promoting cross-disciplinary collaboration and innovation in drug discovery. In summary, the CB2former demonstrates superior performance and interpretability in predicting CB2 receptor ligand activity, offering substantial advantages for drug discovery. Future research should continue to explore the application of advanced AI techniques across various scientific domains, leveraging the strengths of explainable AI to drive progress and innovation in fields such as drug discovery, bioinformatics, and cheminformatics.}

% \textcolor{red}{In this study, we developed the CB2former by integrating a Graph Convolutional Network (GCN) with the Transformer architecture and further introducing a novel dynamic Prompt mechanism to predict the activity of CB2 receptor ligands. Our results demonstrate that the Prompt-based CB2former significantly outperforms a variety of traditional machine learning (Random Forest, Support Vector Machine, K-Nearest Neighbors, Gradient Boosting Machine, Extreme Gradient Boosting) and deep learning models (Multilayer Perceptron, Convolutional Neural Network, Recurrent Neural Network). We observed higher (R\textsuperscript{2}), lower RMSE, and higher AUC scores, indicating robust predictive capabilities.
% A key factor in this improved performance is the injection of unlearnable Prompt tokens, which embed prior knowledge of the CB2 receptor directly into the model’s input sequence. By guiding the self-attention layers toward receptor-relevant features, these Prompt tokens accelerate training convergence and enhance the model’s interpretability. Indeed, the self-attention mechanism, coupled with the Prompt injection, enabled us to pinpoint the molecular substructures most critical for CB2 receptor binding. This not only streamlines the virtual screening process but also offers rational design insights for optimizing novel compounds.
% The superior performance and interpretability of the Prompt-based CB2former position it as a valuable tool for virtual screening and rational drug design. By accurately identifying active compounds and highlighting the key features driving biological activity, our approach reduces experimental validation costs and can potentially shorten the overall drug discovery cycle.
% In summary, our incorporation of a dynamic Prompt mechanism into a GCN–Transformer framework proves to be effective for predicting CB2 receptor ligand activity and elucidating critical molecular determinants. Future research should extend this approach to other receptor–ligand systems, explore additional forms of domain knowledge that can be encoded into the Prompt tokens, and leverage large-scale pretraining to further improve performance and efficiency. As explainable AI continues to evolve, Prompt-based Transformer models are poised to drive progress and innovation in drug discovery, bioinformatics, and cheminformatics, fostering interdisciplinary collaboration and transformative scientific breakthroughs.}

In this study, we proposed \textbf{CB2former}, an innovative framework that integrates a Graph Convolutional Network (GCN) with a Transformer architecture and incorporates a novel \emph{dynamic Prompt mechanism} for CB2 receptor ligand activity prediction. By injecting unlearnable tokens containing receptor-specific knowledge into the self‐attention process, the model not only achieved superior performance—evidenced by higher (R\textsuperscript{2}) , lower RMSE, and higher AUC compared to established ML and DL baselines—but also offered enhanced interpretability. Specifically, the Prompt tokens and attention weights helped spotlight receptor-critical molecular substructures, guiding rational compound design and expediting virtual screening efforts. These results underscore the potential of advanced AI methods in cheminformatics, particularly for tasks that benefit from domain‐driven constraints and transparent mechanistic insights. Looking ahead, expanding the breadth of Prompt‐based knowledge, exploring large‐scale pretraining strategies, and applying the CB2former to additional therapeutic targets stand out as promising directions for future research. Ultimately, this work highlights the benefits of explainable AI in streamlining drug discovery, advancing predictive accuracy and understanding of the combination of AI and scientific knowledge.



% template below.

% \subsection{Maintaining the Integrity of the Specifications}

% The IEEEtran class file is used to format your paper and style the text. All margins, 
% column widths, line spaces, and text fonts are prescribed; please do not 
% alter them. You may note peculiarities. For example, the head margin
% measures proportionately more than is customary. This measurement 
% and others are deliberate, using specifications that anticipate your paper 
% as one part of the entire proceedings, and not as an independent document. 
% Please do not revise any of the current designations.

% \section{Prepare Your Paper Before Styling}
% Before you begin to format your paper, first write and save the content as a 
% separate text file. Complete all content and organizational editing before 
% formatting. Please note sections \ref{AA} to \ref{FAT} below for more information on 
% proofreading, spelling and grammar.

% Keep your text and graphic files separate until after the text has been 
% formatted and styled. Do not number text heads---{\LaTeX} will do that 
% for you.

% \subsection{Abbreviations and Acronyms}\label{AA}
% Define abbreviations and acronyms the first time they are used in the text, 
% even after they have been defined in the abstract. Abbreviations such as 
% IEEE, SI, MKS, CGS, ac, dc, and rms do not have to be defined. Do not use 
% abbreviations in the title or heads unless they are unavoidable.

% \subsection{Units}
% \begin{itemize}
% \item Use either SI (MKS) or CGS as primary units. (SI units are encouraged.) English units may be used as secondary units (in parentheses). An exception would be the use of English units as identifiers in trade, such as ``3.5-inch disk drive''.
% \item Avoid combining SI and CGS units, such as current in amperes and magnetic field in oersteds. This often leads to confusion because equations do not balance dimensionally. If you must use mixed units, clearly state the units for each quantity that you use in an equation.


% \item Do not mix complete spellings and abbreviations of units: ``Wb/m\textsuperscript{2}'' or ``webers per square meter'', not ``webers/m\textsuperscript{2}''. Spell out units when they appear in text: ``. . . a few henries'', not ``. . . a few H''.
% \item Use a zero before decimal points: ``0.25'', not ``.25''. Use ``cm\textsuperscript{3}'', not ``cc''.)
% \end{itemize}

% \subsection{Equations}
% Number equations consecutively. To make your 
% equations more compact, you may use the solidus (~/~), the exp function, or 
% appropriate exponents. Italicize Roman symbols for quantities and variables, 
% but not Greek symbols. Use a long dash rather than a hyphen for a minus 
% sign. Punctuate equations with commas or periods when they are part of a 
% sentence, as in:
% \begin{equation}
% a+b=\gamma\label{eq}
% \end{equation}

% Be sure that the 
% symbols in your equation have been defined before or immediately following 
% the equation. Use ``\eqref{eq}'', not ``Eq.~\eqref{eq}'' or ``equation \eqref{eq}'', except at 
% the beginning of a sentence: ``Equation \eqref{eq} is . . .''

% \subsection{\LaTeX-Specific Advice}

% Please use ``soft'' (e.g., \verb|\eqref{Eq}|) cross references instead
% of ``hard'' references (e.g., \verb|(1)|). That will make it possible
% to combine sections, add equations, or change the order of figures or
% citations without having to go through the file line by line.

% Please don't use the \verb|{eqnarray}| equation environment. Use
% \verb|{align}| or \verb|{IEEEeqnarray}| instead. The \verb|{eqnarray}|
% environment leaves unsightly spaces around relation symbols.

% Please note that the \verb|{subequations}| environment in {\LaTeX}
% will increment the main equation counter even when there are no
% equation numbers displayed. If you forget that, you might write an
% article in which the equation numbers skip from (17) to (20), causing
% the copy editors to wonder if you've discovered a new method of
% counting.

% {\BibTeX} does not work by magic. It doesn't get the bibliographic
% data from thin air but from .bib files. If you use {\BibTeX} to produce a
% bibliography you must send the .bib files. 

% {\LaTeX} can't read your mind. If you assign the same label to a
% subsubsection and a table, you might find that Table I has been cross
% referenced as Table IV-B3. 

% {\LaTeX} does not have precognitive abilities. If you put a
% \verb|\label| command before the command that updates the counter it's
% supposed to be using, the label will pick up the last counter to be
% cross referenced instead. In particular, a \verb|\label| command
% should not go before the caption of a figure or a table.

% Do not use \verb|\nonumber| inside the \verb|{array}| environment. It
% will not stop equation numbers inside \verb|{array}| (there won't be
% any anyway) and it might stop a wanted equation number in the
% surrounding equation.

% \subsection{Some Common Mistakes}\label{SCM}
% \begin{itemize}
% \item The word ``data'' is plural, not singular.
% \item The subscript for the permeability of vacuum $\mu_{0}$, and other common scientific constants, is zero with subscript formatting, not a lowercase letter ``o''.
% \item In American English, commas, semicolons, periods, question and exclamation marks are located within quotation marks only when a complete thought or name is cited, such as a title or full quotation. When quotation marks are used, instead of a bold or italic typeface, to highlight a word or phrase, punctuation should appear outside of the quotation marks. A parenthetical phrase or statement at the end of a sentence is punctuated outside of the closing parenthesis (like this). (A parenthetical sentence is punctuated within the parentheses.)
% \item A graph within a graph is an ``inset'', not an ``insert''. The word alternatively is preferred to the word ``alternately'' (unless you really mean something that alternates).
% \item Do not use the word ``essentially'' to mean ``approximately'' or ``effectively''.
% \item In your paper title, if the words ``that uses'' can accurately replace the word ``using'', capitalize the ``u''; if not, keep using lower-cased.
% \item Be aware of the different meanings of the homophones ``affect'' and ``effect'', ``complement'' and ``compliment'', ``discreet'' and ``discrete'', ``principal'' and ``principle''.
% \item Do not confuse ``imply'' and ``infer''.
% \item The prefix ``non'' is not a word; it should be joined to the word it modifies, usually without a hyphen.
% \item There is no period after the ``et'' in the Latin abbreviation ``et al.''.
% \item The abbreviation ``i.e.'' means ``that is'', and the abbreviation ``e.g.'' means ``for example''.
% \end{itemize}
% An excellent style manual for science writers is \cite{b7}.

% \subsection{Authors and Affiliations}\label{AAA}
% \textbf{The class file is designed for, but not limited to, six authors.} A 
% minimum of one author is required for all conference articles. Author names 
% should be listed starting from left to right and then moving down to the 
% next line. This is the author sequence that will be used in future citations 
% and by indexing services. Names should not be listed in columns nor group by 
% affiliation. Please keep your affiliations as succinct as possible (for 
% example, do not differentiate among departments of the same organization).

% \subsection{Identify the Headings}\label{ITH}
% Headings, or heads, are organizational devices that guide the reader through 
% your paper. There are two types: component heads and text heads.

% Component heads identify the different components of your paper and are not 
% topically subordinate to each other. Examples include Acknowledgments and 
% References and, for these, the correct style to use is ``Heading 5''. Use 
% ``figure caption'' for your Figure captions, and ``table head'' for your 
% table title. Run-in heads, such as ``Abstract'', will require you to apply a 
% style (in this case, italic) in addition to the style provided by the drop 
% down menu to differentiate the head from the text.

% Text heads organize the topics on a relational, hierarchical basis. For 
% example, the paper title is the primary text head because all subsequent 
% material relates and elaborates on this one topic. If there are two or more 
% sub-topics, the next level head (uppercase Roman numerals) should be used 
% and, conversely, if there are not at least two sub-topics, then no subheads 
% should be introduced.

% \subsection{Figures and Tables}\label{FAT}
% \paragraph{Positioning Figures and Tables} Place figures and tables at the top and 
% bottom of columns. Avoid placing them in the middle of columns. Large 
% figures and tables may span across both columns. Figure captions should be 
% below the figures; table heads should appear above the tables. Insert 
% figures and tables after they are cited in the text. Use the abbreviation 
% ``Fig.~\ref{fig}'', even at the beginning of a sentence.

% \begin{table}[htbp]
% \caption{Table Type Styles}
% \begin{center}
% \begin{tabular}{|c|c|c|c|}
% \hline
% \textbf{Table}&\multicolumn{3}{|c|}{\textbf{Table Column Head}} \\
% \cline{2-4} 
% \textbf{Head} & \textbf{\textit{Table column subhead}}& \textbf{\textit{Subhead}}& \textbf{\textit{Subhead}} \\
% \hline
% copy& More table copy$^{\mathrm{a}}$& &  \\
% \hline
% \multicolumn{4}{l}{$^{\mathrm{a}}$Sample of a Table footnote.}
% \end{tabular}
% \label{tab1}
% \end{center}
% \end{table}

% \begin{figure}[htbp]
% \centerline{\includegraphics{fig1.png}}
% \caption{Example of a figure caption.}
% \label{fig}
% \end{figure}

% Figure Labels: Use 8 point Times New Roman for Figure labels. Use words 
% rather than symbols or abbreviations when writing Figure axis labels to 
% avoid confusing the reader. As an example, write the quantity 
% ``Magnetization'', or ``Magnetization, M'', not just ``M''. If including 
% units in the label, present them within parentheses. Do not label axes only 
% with units. In the example, write ``Magnetization (A/m)'' or ``Magnetization 
% \{A[m(1)]\}'', not just ``A/m''. Do not label axes with a ratio of 
% quantities and units. For example, write ``Temperature (K)'', not 
% ``Temperature/K''.

% \section*{Acknowledgment}

% The preferred spelling of the word ``acknowledgment'' in America is without 
% an ``e'' after the ``g''. Avoid the stilted expression ``one of us (R. B. 
% G.) thanks $\ldots$''. Instead, try ``R. B. G. thanks$\ldots$''. Put sponsor 
% acknowledgments in the unnumbered footnote on the first page.

% \section*{References}

% Please number citations consecutively within brackets \cite{b1}. The 
% sentence punctuation follows the bracket \cite{b2}. Refer simply to the reference 
% number, as in \cite{b3}---do not use ``Ref. \cite{b3}'' or ``reference \cite{b3}'' except at 
% the beginning of a sentence: ``Reference \cite{b3} was the first $\ldots$''

% Number footnotes separately in superscripts. Place the actual footnote at 
% the bottom of the column in which it was cited. Do not put footnotes in the 
% abstract or reference list. Use letters for table footnotes.

% Unless there are six authors or more give all authors' names; do not use 
% ``et al.''. Papers that have not been published, even if they have been 
% submitted for publication, should be cited as ``unpublished'' \cite{b4}. Papers 
% that have been accepted for publication should be cited as ``in press'' \cite{b5}. 
% Capitalize only the first word in a paper title, except for proper nouns and 
% element symbols.

% For papers published in translation journals, please give the English 
% citation first, followed by the original foreign-language citation \cite{b6}.

% \begin{thebibliography}{00}
% \bibitem{b1} G. Eason, B. Noble, and I. N. Sneddon, ``On certain integrals of Lipschitz-Hankel type involving products of Bessel functions,'' Phil. Trans. Roy. Soc. London, vol. A247, pp. 529--551, April 1955.
% \bibitem{b2} J. Clerk Maxwell, A Treatise on Electricity and Magnetism, 3rd ed., vol. 2. Oxford: Clarendon, 1892, pp.68--73.
% \bibitem{b3} I. S. Jacobs and C. P. Bean, ``Fine particles, thin films and exchange anisotropy,'' in Magnetism, vol. III, G. T. Rado and H. Suhl, Eds. New York: Academic, 1963, pp. 271--350.
% \bibitem{b4} K. Elissa, ``Title of paper if known,'' unpublished.
% \bibitem{b5} R. Nicole, ``Title of paper with only first word capitalized,'' J. Name Stand. Abbrev., in press.
% \bibitem{b6} Y. Yorozu, M. Hirano, K. Oka, and Y. Tagawa, ``Electron spectroscopy studies on magneto-optical media and plastic substrate interface,'' IEEE Transl. J. Magn. Japan, vol. 2, pp. 740--741, August 1987 [Digests 9th Annual Conf. Magnetics Japan, p. 301, 1982].
% \bibitem{b7} M. Young, The Technical Writer's Handbook. Mill Valley, CA: University Science, 1989.
% \bibitem{b8} D. P. Kingma and M. Welling, ``Auto-encoding variational Bayes,'' 2013, arXiv:1312.6114. [Online]. Available: https://arxiv.org/abs/1312.6114
% \bibitem{b9} S. Liu, ``Wi-Fi Energy Detection Testbed (12MTC),'' 2023, gitHub repository. [Online]. Available: https://github.com/liustone99/Wi-Fi-Energy-Detection-Testbed-12MTC
% \bibitem{b10} ``Treatment episode data set: discharges (TEDS-D): concatenated, 2006 to 2009.'' U.S. Department of Health and Human Services, Substance Abuse and Mental Health Services Administration, Office of Applied Studies, August, 2013, DOI:10.3886/ICPSR30122.v2
% \bibitem{b11} K. Eves and J. Valasek, ``Adaptive control for singularly perturbed systems examples,'' Code Ocean, Aug. 2023. [Online]. Available: https://codeocean.com/capsule/4989235/tree
% \end{thebibliography}

% \vspace{12pt}
% \color{red}
% IEEE conference templates contain guidance text for composing and formatting conference papers. Please ensure that all template text is removed from your conference paper prior to submission to the conference. Failure to remove the template text from your paper may result in your paper not being published.

% \clearpage

\bibliographystyle{IEEEtran}
\bibliography{IEEEabrv,references}

\end{document}
