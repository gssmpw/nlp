\section{Conclusions}
We proposed CellNuc-DETR, a detection-based approach for cell analysis in histopathological WSIs, shifting from the traditional segmentation methods used in digital pathology. Our method focuses on directly detecting and classifying cell nuclei, avoiding the computational burden of segmentation. Experiments on the PanNuke dataset demonstrate that CellNuc-DETR achieves state-of-the-art performance in both cell nuclei detection and classification. Cross-dataset evaluations on CoNSeP and MoNuSeg further confirm its robustness and generalization ability across diverse datasets and conditions.

In addition to its superior accuracy, CellNuc-DETR significantly enhances inference efficiency on large WSIs, being twice as fast as the fastest segmentation-based method, HoVer-NeXt, while achieving better accuracy. Compared to CellViT, CellNuc-DETR provides higher classification accuracy and operates approximately ten times more efficiently during WSI inference. These advantages make CellNuc-DETR highly suitable for clinical and high-throughput research settings where both speed and accuracy are essential.

Overall, CellNuc-DETR offers a new standard for cell analysis in digital pathology by combining high performance with computational efficiency. Future work could explore further optimizations for clinical applications and extend the method to additional histopathological tasks.