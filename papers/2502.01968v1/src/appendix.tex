
{\LARGE \bf Appendix}


\section*{Organization of the Appendix}
The Appendix is organized as follows. 
\squishlist
    \item Section~\ref{sec:apdx_limitations} discusses the potential limitations of our work.
    \item Section~\ref{app:proof_noisy_bound} provides a full proof for Theorems shown in Section \ref{sec:thm_analyses}.
    \item Section~\ref{sec:apx_experiment_details} illustrates the experimental details including data pool, evaluation benchmarks, and training details.
    \item Section~\ref{sec:apx_more_exp_results} demonstrates detailed omitted performance results. 
    \item Section~\ref{sec:apx_examples_with_tokens} provides several samples with tokens selected by the self-evolving cleaning pipeline.
\squishend


\section{Limitations}\label{sec:apdx_limitations}

While the proposed token cleaning pipelines demonstrate competitive performance compared to other baselines, we acknowledge that there are still potential limitations:
\begin{itemize}[left=-2pt]
    \item \textbf{Base Model Scale}. Our experiments are primarily conducted on a base model with a 3B-8B scale. It remains uncertain how well the pipeline would perform on larger-scale base models.
    \item \textbf{Data Pool Scale}. Due to cost considerations, our data pool is limited to 50k samples.  The performance of the proposed pipeline on a larger-scale data pool remains uncertain.
\end{itemize}


\section{Proof for Theorem~\ref{thm:noisy_bound}}\label{app:proof_noisy_bound}

We first reproduce the definitions as follows.


Denote by $\widetilde D:=\{(x_{i,j}, \bm x_{i,:j}, \tilde y_{i,j}), \forall i,j\}$ the full-token dataset.
By minimizing the noisy loss
$$
    \widehat {\mathcal{L}}_{\widetilde D}(\theta) =   \frac{1}{\sum_{(i,j)\in S} \tilde y_{i,j}} \sum_{(i,j)\in S} \tilde y_{i,j} \BR( \theta (\bm x_{i,:j}), x_{i,j}),
$$
where $S:=\{(i,j) | i\in [N], j\in [L_i]\}$, $[N]:=\{1, 2,\cdots, N\}$.
we can get model $\hat \theta_{\widetilde D}:=\argmin_{\theta}~  \widehat {\mathcal{L}}_{\widetilde D}(\theta)$. When train with full tokens, we have $\tilde y_{i,j}=1, \forall i,j$. The corresponding expected loss can be denoted by 
$$
{\mathcal{L}}_{\widetilde {\mathcal D}}(\theta) = \E\left[ \widetilde Y \cdot \BR{(\theta(\bm X_{\text{prev}}), X_{\text{next}})}  \right] = \E\left[  \BR{(\theta(\bm X_{\text{prev}}), X_{\text{next}})}  \right],
$$
where $\widetilde {\mathcal D}$ is the distribution of $\widetilde D$. Denote by $\widetilde Y$ the random variable for noisy token label $\tilde y_{i,j}$, and $Y$ the random variable for the ground-truth token label $y_{i,j}$. Accordingly, with ground-truth token labels, the expected loss is
$$
{\mathcal{L}}_{{\mathcal D}}(\theta) = \frac{1}{\E[Y]} \E\left[ Y \cdot \BR{(\theta(\bm X_{\text{prev}}), X_{\text{next}})}  \right].
$$







\textbf{Decomposition}~
With the above definitions, the generalization error of model $\hat \theta_{\widetilde D}$ on the clean distribution could be decomposed as $${\mathcal L}_{{\mathcal D}}(\hat \theta_{\widetilde D})=  
\underbrace{({\mathcal L}_{{\mathcal D}}(\hat \theta_{\widetilde D}) - {\mathcal L}_{\widetilde{\mathcal D}}(\hat \theta_{\widetilde D})) }_{\text{Term-1}}
+\underbrace{{\mathcal L}_{\widetilde{\mathcal D}}(\hat \theta_{\widetilde D}) }_{\text{Term-2}},$$
where {\bf Term-1} 
transforms the evaluation of $\hat \theta_{\widetilde D}$ from clean distribution $\mathcal D$ to the noisy distribution $\widetilde{\mathcal D}$. {\bf Term-2} 
is the generalization error but the model is trained and evaluated on noisy distribution $\widetilde{\mathcal D}$.
Both terms are analyzed as follows.

\subsection{Term-1 Upper Bound}

For a certain model $\theta$, there always exist a random variable $\widetilde X_{\text{next}}$ such that when $Y=1$, $X_{\text{next}} = \widetilde X_{\text{next}}$  and when $Y=0$,
$$
 \E_{X|Y=1}\left[ \BR{(\theta(\bm X_{\text{prev}}), X_{\text{next}})}  \right] 
= \E_{X|Y=0}\left[ \BR{(\theta(\bm X_{\text{prev}}), \widetilde X_{\text{next}})}  \right].
$$
Then we have
\begin{align*}
    & \E\left[ \widetilde Y \cdot \BR{(\theta(\bm X_{\text{prev}}), \widetilde X_{\text{next}})}  \right] \\
  = & \PP(Y=1) \cdot \E_{X|Y=1}\left[ \BR{(\theta(\bm X_{\text{prev}}), \widetilde X_{\text{next}})}  \right] + \PP(Y=0) \cdot \E_{X|Y=0}\left[ \BR{(\theta(\bm X_{\text{prev}}), \widetilde X_{\text{next}})}  \right] \\
  = & \PP(Y=1) \cdot \E_{X|Y=1}\left[ \BR{(\theta(\bm X_{\text{prev}}),  X_{\text{next}})}  \right] + \PP(Y=0) \cdot \E_{X|Y=1}\left[ \BR{(\theta(\bm X_{\text{prev}}),  X_{\text{next}})}  \right] \\
  = & \E_{X|Y=1}\left[ \BR{(\theta(\bm X_{\text{prev}}),  X_{\text{next}})}  \right] \\
  = & \frac{1}{\E[Y]} \cdot \E\left[ Y \cdot \BR{(\theta(\bm X_{\text{prev}}),  X_{\text{next}})}  \right]
\end{align*}

\begin{align*}
     &{\mathcal L}_{\mathcal D}(\theta) - {\mathcal L}_{\widetilde{\mathcal D}}(\theta) \\
    =& \int_{X} \left( \frac{Y}{\E[Y]} \cdot \PP({\theta(\bm X_{\text{prev}})\ne  X_{\text{next}} }| X) - \widetilde Y \cdot \PP({\theta(\bm X_{\text{prev}}) \ne X_{\text{next}}} | X) \right) \PP(X)   ~dX \\
    =& \int_{X} \left( \widetilde Y \cdot \PP({\theta(\bm X_{\text{prev}})\ne \widetilde X_{\text{next}} }| X) - \widetilde Y \cdot \PP({\theta(\bm X_{\text{prev}}) \ne X_{\text{next}}} | X) \right) \PP(X)   ~dX \\
    =& \int_{X} \left(  \PP({\theta(\bm X_{\text{prev}})\ne \widetilde X_{\text{next}} }| X) -  \PP({\theta(\bm X_{\text{prev}}) \ne X_{\text{next}}} | X) \right) \PP(X)   ~dX \\
    =& \int_{X} \left(  \PP({\theta(\bm X_{\text{prev}}) =  X_{\text{next}} }| X) -  \PP({\theta(\bm X_{\text{prev}}) = \widetilde X_{\text{next}}} | X) \right) \PP(X)   ~dX \\
    \le & \frac{1}{2} \int_{X} \Bigg(\left| \PP({\color{red}\theta(\bm X_{\text{prev}})=\widetilde X_{\text{next}} }| X) - \PP({\color{red}\theta(\bm X_{\text{prev}}) = X_{\text{next}}} | X) \right| \\
    & + \left| \PP({\color{red}\theta(\bm X_{\text{prev}}) \ne \widetilde X_{\text{next}} }| X) - \PP({\color{red}\theta(\bm X_{\text{prev}}) \ne  X_{\text{next}}} | X) \right| \Bigg) \PP(X)  ~dX  \\
    \overset{(a)}{=} &  \int_{X} \text{TD}(\widetilde X_{\text{next}}(\bm X_{\text{prev}}; \theta) || X_{\text{next}}(\bm X_{\text{prev}};\theta))  \PP(X)    ~dX \\
    \overset{(b)}{\le} &   \frac{1}{2}\int_{X}\sum_{k\in[K]}\left | \PP(\widetilde X_{\text{next}} = k | X) - \PP(X_{\text{next}}=k| X)  \right | \PP(X) ~dX\\
    = & \PP(Y=1) \cdot \frac{1}{2}\int_{X}\sum_{k\in[K]}\left | \PP(\widetilde X_{\text{next}} = k | X, Y=1) - \PP(X_{\text{next}}=k| X, Y=1)  \right | \PP(X|Y=1)  ~dX \\
    & + \PP(Y=0) \cdot \frac{1}{2}\int_{X}\sum_{k\in[K]}\left | \PP(\widetilde X_{\text{next}} = k | X, Y=0) - \PP(X_{\text{next}}=k| X, Y=0)  \right | \PP(X|Y=0)  ~dX\\
    \le & \PP(Y=0) \\
    = & \PP(Y\ne \widetilde Y) \\
    = & \eta(\widetilde D).
\end{align*}
where in \textbf{equality (a)}, given model $\theta$ and previous tokens, we can treat $\widetilde X_{\text{next}}(\bm X_{\text{prev}}; \theta)$ as a Bernoulli random variable such that
\[
\PP(\widetilde{X}_{\text{next}}(\bm X_{\text{prev}};\theta) = +) = \PP({\theta(\bm X_{\text{prev}})=\widetilde X_{\text{next}} }| X) \text{~~and~~} \PP(\widetilde{X}_{\text{next}}(\bm X_{\text{prev}}; \theta) = -) = \PP({\theta(\bm X_{\text{prev}})\ne \widetilde X_{\text{next}} }| X). 
\]
Then according to the definition of total variation of two distributions, i.e.,
\[
\text{TD}(P||Q) := \frac{1}{2} \int_u |\frac{dP}{du} - \frac{dQ}{du}|du,
\]
we can summarize the integrand as the total variation between $\widetilde{X}_{\text{next}}(\bm X_{\text{prev}};\theta)$ and ${X}_{\text{next}}(\bm X_{\text{prev}};\theta)$.


\textbf{Inequality (b)} holds due to the data processing inequality since the probabilities $[\PP(\widetilde X_{\text{next}}=\theta(\bm X_{\text{prev}})), \PP(\widetilde X_{\text{next}}\ne \theta(\bm X_{\text{prev}}))]$ are generated by $[\PP(\widetilde X_{\text{next}}=i), \forall i\in[K]]$, and the probabilities $[\PP(X_{\text{next}}=\theta(\bm X_{\text{prev}})), \PP(X_{\text{next}}\ne \theta(\bm X_{\text{prev}}))]$ are generated by $[\PP(X_{\text{next}}=k), \forall k\in[K]]$.


The results also hold for an arbitrary $\widetilde Y$. To prove it, we need to split the loss into two part: 1) $\widetilde Y=1$ and 2) $\widetilde Y=0$.
The first part follows the previous proof, yields an upper bound of $\PP(\widetilde Y=1, Y=0)$. The second part leads to an upper bound of $\PP(\widetilde Y=0, Y=1)$. By summing up two bounds, we still have the same result, i.e., an upper bound of $\PP(Y\ne \widetilde Y)$.

\subsection{Term-2 Upper Bound}\label{appendix:Term2_ub}

Recall that 
$$
\widehat {\mathcal{L}}_{\widetilde D}(\theta) =  \frac{1}{\sum_{(i,j)\in S} \tilde y_{i,j}} \sum_{(i,j)\in S} \tilde y_{i,j} \ell(x_{i,j} | \bm x_{i,:j}; \theta).
$$
Let $\hat \theta_{\widetilde D}$ denote the model trained by minimizing 0-1 loss with noisy dataset ${\widetilde D}$, i.e., $$\hat \theta_{\widetilde D}:=\argmin_{\theta}~ \hat {\mathcal L}_{\widetilde D}(\theta).$$

Recall that the expected error of model $\theta$ on distribution $\mathcal {\widetilde D}$ is 
$$
{\mathcal{L}}_{{\mathcal {\widetilde D}}}(\theta) =  \E\left[ \widetilde Y \cdot \BR{(\theta(\bm X_{\text{prev}}), X_{\text{next}})}  \right].
$$
The optimal classifier is denoted by $$\theta_{\mathcal {\widetilde D}}:=\argmin_\theta {\mathcal L}_{\mathcal {\widetilde D}}(\theta).$$ 
Denote by $M:=\sum_{i=1}^N L_i$ the number of toekns. With probability at least $1-\delta$, we have:
\begin{equation*}
    \begin{split}
      &  {\mathcal L}_{\mathcal {\widetilde D}}(\hat \theta_{\widetilde D}) - {\mathcal L}_{\mathcal {\widetilde D}}(\theta_{\mathcal {\widetilde D}}) \\
    = & \hat {\mathcal L}_{\widetilde D}(\hat \theta_{\widetilde D}) - \hat {\mathcal L}_{\widetilde D}(\theta_{\mathcal {\widetilde D}})
    + \left( {\mathcal L}_{\mathcal {\widetilde D}}(\hat \theta_{\widetilde D}) 
    - \hat {\mathcal L}_{\widetilde D}(\hat \theta_{\widetilde D})\right)   + \left(\hat {\mathcal L}_{\widetilde D}(\theta_{\mathcal {\widetilde D}})
    - {\mathcal L}_{\mathcal {\widetilde D}}(\theta_{\mathcal {\widetilde D}}) \right)\\ 
  \overset{(a)}{\le} & 0 +  { |\hat {\mathcal L}_{\widetilde D}(\hat \theta_{\widetilde D}) - {\mathcal L}_{\mathcal {\widetilde D}}(\hat \theta_{\widetilde D})| + |\hat {\mathcal L}_{\widetilde D}(\theta_{\mathcal {\widetilde D}}) - {\mathcal L}_{\mathcal {\widetilde D}}(\theta_{\mathcal {\widetilde D}})|}  \\
  \overset{(b)}{\le} &  \sqrt{\frac{2\log({4}/\delta)}{M}},
    \end{split}
\end{equation*}
where inequality (a) holds since 1) $\hat {\mathcal L}_{\widetilde D}(\hat \theta_{\widetilde D})$ achieves the minimum empirical risk according to its definition, thus $\hat {\mathcal L}_{\widetilde D}(\hat \theta_{\widetilde D}) - \hat {\mathcal L}_{\widetilde D}(\theta_{\mathcal {\widetilde D}})\le 0$; {2) each of the following two terms will be no greater than the corresponding gap $|\hat {\mathcal L}_{\widetilde D}(\theta) - {\mathcal L}_{\mathcal {\widetilde D}}(\theta)|$. Specifically, inequality (b) holds due to the Hoeffding’s inequality, i.e., given any classifier $\hat \theta_{\widetilde D}$, and $\theta_{\mathcal {\widetilde D}}$, with probability at least $1-\delta/2$, we have the following bounds independently:
\[
 |\hat {\mathcal L}_{\widetilde D}(\hat \theta_{\widetilde D}) - {\mathcal L}_{\mathcal {\widetilde D}}(\hat \theta_{\widetilde D})| \le \sqrt{\frac{\log(4/\delta)}{2M}}, \quad  |\hat {\mathcal L}_{\widetilde D}(\theta_{\mathcal {\widetilde D}}) - {\mathcal L}_{\mathcal {\widetilde D}}(\theta_{\mathcal {\widetilde D}})| \le \sqrt{\frac{\log(4/\delta)}{2M}}.
\]
By the union bound, we have inequality (b) with probability at least $1-\delta$.

}

Supposing a unique next token for given previous tokens in distribution $\mathcal{\widetilde D}$, we have $ {\mathcal L}_{\mathcal {\widetilde D}}(\theta_{\mathcal {\widetilde D}}) = 0$.



\section{Experimental Details}\label{sec:apx_experiment_details}

\subsection{Data Pool}\label{sec:apx_data_pool}
The data pool used in this work is primarily composed of five widely used SFT datasets, derived either from human annotations or generated by advanced LLMs. Further details on these datasets can be found in Table~\ref{tab:data-pool}. Notably, these datasets differ in format, quality, prompt length, and target tasks, highlighting the diversity of our foundational data pool.
For consistency, we adopt the ``Tulu" template format proposed by \citet{wang2023far} to standardize these datasets. The ``Tulu" template includes two primary tags,  \texttt{<|User|>} and \texttt{<|Assistant|>}, which denote the roles of the user and the assistant, respectively.

\begin{table}[h]
\centering
\caption{Statistical summary of our 50k data pool. The average prompt length and the average response length are measured based on \texttt{LLaMA-3.2-3B}.}
\vspace{0.1in}
\resizebox{1\linewidth}{!}{
\begin{tabular}{llcccccc}
\toprule
\textbf{Datasets} & \textbf{Data source} & \textbf{Data quality} & \# \textbf{Data size} & \textbf{Prompt\_Len} & \textbf{Completion\_Len} & \textbf{Overall\_Len}\\
\midrule
\textsc{Open-Assistant 1} & Human-generated & Both & 2418 & 45.4 & 364.1 & 409.5 \\
\textsc{Stanford Alpaca} & Generated w/ Davinci-003 & Normal & 4598 & 22.5 & 84.8 & 107.2 \\
\textsc{WizardLM} & ChatGPT-generated & High & 34772 & 126.1 & 419.23 & 545.4\\
\textsc{Dolly} & Human-generated & Normal & 1567 & 119.1 & 133.7 & 252.9 \\
\textsc{Flan V2} & Human-generated & Normal & 6645 & 475.1 & 38.6 & 513.7 \\
\bottomrule
\end{tabular}
}
\label{tab:data-pool}
\end{table}




\subsection{Evaluation Benchmarks}\label{sec:appendix_evalution_details}

For MMLU, TruthfulQA, LogiQA, ARC-C, BoolQ, and HellaSwag datasets, we directly use accuracy as the performance metric.
In particular, for the TruthfulQA dataset (a multiple-choice question benchmark), we utilize the MC2 metric, which evaluates only the answer assigned the highest probability by the model (i.e., the model's most confident choice). 
For the TydiQA dataset, we consider using the 1-shot F1 score. One can double-check these performance metrics in the \textsc{lm-evaluation-harness} repository. By default, we use all benchmark samples to conduct evaluation.


\subsection{Training Details}
Following the experimental setup \citep{wang2023far}, we apply a Lora technique with a rank-size of 64 and a scaling factor of 16. The overall batch size is 48, with the learning rate at 1e-4 as well as 1 training epoch. By default, the maximum input length is 2048. All experiments are conducted on eight NVIDIA L40S GPUs. Fine-tuning 7B or 8B models on a dataset of 50k samples typically takes approximately 3 hours.

\section{Additional Experimental Results}\label{sec:apx_more_exp_results}




\subsection{Empirical Results Still Follows the Observation of the Matthew Effort}

Here, we also provide the empirical results of the self-evolving cleaning strategy on the other two base models. As shown in Table~\ref{tab:iteration_results_7b_and_8b}, the performance results over iterations also align with the theoretical observation, as discussed in Section~\ref{sec:matthew_effect}.


\begin{table*}[ht]
\centering
\caption{Performance results of self-evolving cleaning pipeline over iterations (checkpoints) on seven benchmarks. Base models: \texttt{LLaMA-3.1-8B} and \texttt{Mistral-7B-v0.3}. These performance results still align with three observations arising from the Matthew effect.}
\vspace{1ex}
\begin{tabular}{l|ccccccc}
\toprule
\textbf{Model} 
& \textbf{TruthfulQA} 
& \textbf{TydiQA} 
& \textbf{LoqiQA} 
& \textbf{MMLU} 
& \textbf{HellaSwag} 
& \textbf{ARC-C} 
& \textbf{BoolQ} \\
\midrule
\multicolumn{8}{c}{\cellcolor{blue!10} \textbf{Base model: LLaMA-3.1-8B}} \\
\midrule
\textsc{Reference-1} 
 & 53.45 & 58.62 & 28.68 & 65.65 & 61.58 & 55.56 & 82.98 \\
\textsc{Reference-2} 
& 56.77 & 63.23 & 27.91 & 65.40 & 62.29 & 56.16 & 82.64  \\
\textsc{Reference-3} 
& 59.01 & 65.27 & 26.05 & 65.18 & 62.56 & 55.81 & 82.74  \\
\textsc{Reference-4} 
& 59.58 & 63.58 & 26.05 & 65.07 & 62.67 & 54.87 & 82.49  \\
\midrule
\multicolumn{8}{c}{\cellcolor{blue!10} \textbf{Base model: Mistral-7B-v0.3}} \\

\midrule
\textsc{Reference-1} 
& 44.36 & 55.79 & 26.20 & 62.31 & 61.32 & 51.34 & 84.03 \\
\textsc{Reference-2} 
& 44.78 & 56.04 & 27.13 & 62.41 & 61.37 & 51.25 & 83.82  \\
\textsc{Reference-3} 
& 44.93 & 55.69 & 27.44 & 62.32 & 61.30 & 50.82 & 80.36 \\
\textsc{Reference-4}  
& 45.41 & 56.17 & 27.44 & 62.30 & 61.40 & 50.65 & 81.28  \\
\bottomrule
\end{tabular}
\label{tab:iteration_results_7b_and_8b}
\end{table*}


\subsection{Impact of Selected Token Proportion}

This section provides more experimental results to explicitly demonstrate the impact of the selected token proportion across iterations. Here, Table~\ref{tab:impact_of_data_prop_with_iteration} and Table~\ref{tab:impact_of_data_prop_full} show the full performance results of the self-evolving cleaning pipeline.



\begin{table*}[ht]
    \centering
    \caption{Performance comparison of self-evolving cleaning pipeline over different iterations on various benchmarks under different selected token proportions. } 
    \vspace{10pt}
    \resizebox{0.8\linewidth}{!}{
    \begin{tabular}{l|ccccccc}
    \toprule
    \textbf{Model} & \textbf{TruthfulQA} & \textbf{TydiQA} & \textbf{LoqiQA} & \textbf{MMLU} & \textbf{HellaSwag} & \textbf{ARC-C} & \textbf{BoolQ} \\
    \toprule
     \multicolumn{8}{c}{\cellcolor{blue!10} \textbf{Base Model: LLaMA-3.2-3B}} \\
    \midrule
\textsc{Reference}-1 (0.3) & 45.68 & 49.36 & 27.60 & 57.23 & 56.06 & 45.56 & 77.08 \\
\textsc{Reference}-2 (0.3)& 46.84 & 50.10 & 28.22 & 56.53 & 56.01 & 47.11 & 77.18 \\
\textsc{Reference}-3 (0.3)& 49.15 & 50.69 & 27.44 & 56.11 & 55.83 & 46.51 & 77.15 \\
\textsc{Reference}-4 (0.3)& 51.57 & 52.96 & 26.51 & 55.42 & 55.55 & 46.51 & 77.24 \\
\midrule
\textsc{Reference}-1 (0.4)  & 45.56 & 46.88 & 26.98 & 57.32 & 56.29 & 45.99 & 77.15 \\
\textsc{Reference}-2 (0.4) & 46.95 & 51.73 & 29.30 & 56.82 & 56.34 & 46.94 & 77.18 \\
\textsc{Reference}-3 (0.4) & 48.72 & 54.56 & 28.84 & 56.52 & 56.05 & 46.68 & 77.52  \\
\textsc{Reference}-4 (0.4) & 50.69 & 55.20 & 27.91 & 56.27 & 55.77 & 46.86 & 77.36 \\
\midrule
\textsc{Reference}-1 (0.5) & 45.47 & 47.49 & 27.13 & 57.18 & 56.22 & 45.82 & 76.99 \\
\textsc{Reference}-2 (0.5)& 46.93 & 53.02 & 27.44 & 57.04 & 56.45 & 46.68 & 77.30  \\
\textsc{Reference}-3 (0.5) & 48.73 & 54.65 & 28.37 & 56.55 & 56.10 & 46.94 & 77.33 \\
\textsc{Reference}-4 (0.5) & 50.42 & 55.19 & 28.37 & 56.25 & 55.83 & 46.77 & 77.12\\
\midrule
\textsc{Reference}-1 (0.6)& 45.46 & 50.05 & 27.44 & 57.31 & 56.10 & 45.91 & 76.87 \\
\textsc{Reference}-2 (0.6)& 46.67 & 53.18 & 27.44 & 56.89 & 56.25 & 46.51 & 77.15 \\
\textsc{Reference}-3 (0.6)& 48.91 & 54.36 & 28.22 & 56.43 & 56.13 & 46.43 & 77.36 \\
\textsc{Reference}-4 (0.6)& 51.07 & 56.38 & 28.22 & 56.18 & 55.81 & 45.99 & 77.33 \\
\midrule
\textsc{Reference}-1 (0.7)& 45.34 & 50.45 & 27.60 & 57.42 & 56.09 & 45.65 & 76.90 \\
\textsc{Reference}-2 (0.7)& 46.82 & 52.02 & 26.98 & 56.86 & 56.34 & 46.43 & 77.30 \\
\textsc{Reference}-3 (0.7)& 48.48 & 55.27 & 27.91 & 56.38 & 56.18 & 46.77 & 77.21 \\
\textsc{Reference}-4 (0.7)& 50.66 & 55.47 & 27.91 & 56.10 & 55.93 & 46.43 & 77.36 \\
\midrule
\textsc{Reference}-1 (0.8)& 45.09 & 56.21 & 26.67 & 57.50 & 45.65 & 32.73 & 76.87 \\
\textsc{Reference}-2 (0.8)& 46.25 & 56.39 & 26.67 & 57.21 & 46.08 & 32.93 & 77.08 \\
\textsc{Reference}-3 (0.8)& 47.30 & 56.34 & 26.67 & 57.05 & 45.99 & 33.33 & 77.33 \\
\textsc{Reference}-4 (0.8)& 48.40 & 56.23 & 26.82 & 56.89 & 45.91 & 33.33 & 77.18 \\
\midrule
\textsc{Reference}-1 (0.9) & 44.75 & 46.88 & 26.20 & 57.52 & 56.10 & 45.39 & 76.68  \\
\textsc{Reference}-2 (0.9) & 45.06 & 50.47 & 26.20 & 57.27 & 56.42 & 46.08 & 76.96  \\
\textsc{Reference}-3 (0.9)& 45.23 & 50.36 & 25.43 & 57.37 & 56.45 & 46.08 & 77.02  \\
\textsc{Reference}-4 (0.9) & 46.36 & 51.68 & 26.98 & 57.25 & 56.34 & 45.65 & 76.90 \\
\bottomrule
    \end{tabular}
    }
    \label{tab:impact_of_data_prop_with_iteration}
\end{table*}




\begin{table*}[ht]
    \centering
    \caption{ Performance comparison of self-evolving cleaning pipeline on various benchmarks under different selected token proportions. The following discrete value (e.g., 0.3) denotes the selected token proportion.} 
    \vspace{1ex}
    \resizebox{0.9\linewidth}{!}{
    \begin{tabular}{l|ccccccc|c}
    \toprule
    \textbf{Model} & \textbf{TruthfulQA} & \textbf{TydiQA} & \textbf{LoqiQA} & \textbf{MMLU} & \textbf{HellaSwag} & \textbf{ARC-C} & \textbf{BoolQ} & \textbf{AVG} \\
    \toprule
     \multicolumn{9}{c}{\cellcolor{blue!10} \textbf{Base Model: LLaMA-3.2-3B}} \\
    \midrule
\textsc{Self-evolving cleaning} (0.3) & 51.57 & 52.96 & 26.51 & 55.42 & 55.55 & 46.51 & 77.24 & 52.25\\
\textsc{Self-evolving cleaning} (0.4) & 50.69 & 55.20 & 27.91 & 56.27 & 55.77 & 46.86 & 77.36 & 52.87\\
\textsc{Self-evolving cleaning} (0.5) & 50.42 & 55.19 & 28.37 & 56.25 & 55.83 & 46.77 & 77.12 & 52.85\\
\textsc{Self-evolving cleaning} (0.6) & 51.07 & 56.38 & 28.22 & 56.18 & 55.81 & 45.99 & 77.33 & 53.0\\
\textsc{Self-evolving cleaning} (0.7) & 50.66 & 55.47 & 27.91 & 56.10 & 55.93 & 46.43 & 77.36 & 52.83\\
\textsc{Self-evolving cleaning} (0.8) & 48.40 & 56.23 & 26.82 & 56.89 & 45.91 & 33.33 & 77.18 & 49.25\\
\textsc{Self-evolving cleaning} (0.9) & 46.36 & 51.68 & 26.98 & 57.25 & 56.34 & 45.65 & 76.90 & 51.59  \\
\textsc{Self-evolving cleaning} (1.0)& 43.32 & 49.60 & 24.34 & 56.87 & 55.57 & 44.44 & 74.98 & 49.87 \\
\bottomrule
    \end{tabular}
    }
    \label{tab:impact_of_data_prop_full}
\end{table*}




\subsection{Impact of Reference Model}

Here, we provide the full performance results under different reference models, as shown in Table~\ref{tab:impact_of_reference_model_full}. Here, we select \texttt{LLaMA-3.1-8B-Instruct}\footnote{\url{https://huggingface.co/meta-llama/Llama-3.1-8B-Instruct}.} as our reference model.

\begin{table}[ht]
    \centering
    \caption{Performance results under different reference models. Base model: LLaMA-3.1-3B. Additional reference model: LLaMA-3.1-8B-Instruct.}
    \vspace{1ex}
    \resizebox{0.9\linewidth}{!}{
    \begin{tabular}{l|ccccccc|c}
    \toprule
    \textbf{Model} & \textbf{TruthfulQA} & \textbf{TydiQA} & \textbf{LoqiQA} & \textbf{MMLU} & \textbf{HellaSwag} & \textbf{ARC-C} & \textbf{BoolQ} & \textbf{AVG} \\
    \toprule
    \textsc{Base Model}& 39.39 & 21.10 & 22.17 & 56.29 & 55.24 & 42.20 & 72.95 & 44.19 \\
    \midrule 
        \textsc{LLaMA-8B-Inst} & 54.12  & 21.63 & 31.47 & 68.18 &  59.0 & 51.77 & 84.03 & 52.89 \\
      \textsc{Rho (LLaMA)}   & 44.06  & 39.37 &  25.58 & 57.04&  55.35 &  46.08  & 75.94     & 49.1 \\ 
      \textsc{Fixed-Model cleaning (LLaMA)}   & 44.27 & 39.17 & 24.96 & 56.96 & 55.22 & 46.08 & 76.37 & 49.0\\
      \midrule
      \textsc{Warmup Reference Model} & 43.35 & 41.20 & 24.96 & 56.93 & 55.64 & 44.62 & 74.80 & 48.79 \\
      \textsc{Rho (Warmup) } & 45.57 & 53.60 & 26.05 & 57.10 & 55.16 & 45.39 & 77.36 & 51.46 \\ 
    \textsc{Fixed-Model cleaning (Warmup)}   & 48.96 & 52.60 & 25.89 & 57.09 & 56.43 & 45.39 & 77.52 & 51.98 \\
    \bottomrule
    \end{tabular}}
    \label{tab:impact_of_reference_model_full}
\end{table}



\section{Illustrative Examples with Selected Tokens}\label{sec:apx_examples_with_tokens}
 Intuitively, those common tokens are less likely to be selected by our proposed strategies.
Here, we provide several samples with tokens selected by the proposed self-evolving cleaning strategy. The selected tokens are highlighted in \textbf{{\color{blue}blue}}.

\begin{tcolorbox}[colframe=gray!70!black,colback=white, title=Sample 1]
\textbf{User:}
What is the meaning of "real life"? How does one distinguish the real world from the imaginary? Can virtual reality experiences be considered real life? How do our senses interpret reality? \vspace{0.1in}
\\
\textbf{Assistant:}
\textcolor{blue}{The}\textcolor{blue}{~meaning}\textcolor{blue}{~of}\textcolor{blue}{~"}\textcolor{blue}{real}\textcolor{blue}{~life}\textcolor{blue}{"}\textcolor{blue}{~can} be\textcolor{blue}{~subjective}\textcolor{blue}{~and}\textcolor{blue}{~varies}\textcolor{blue}{~from}\textcolor{blue}{~person}\textcolor{blue}{~to} person\textcolor{blue}{.}\textcolor{blue}{~Generally}\textcolor{blue}{,}\textcolor{blue}{~it} refers to experiences and interactions\textcolor{blue}{~that} occur\textcolor{blue}{~in} the physical\textcolor{blue}{~world}\textcolor{blue}{,}\textcolor{blue}{~as}\textcolor{blue}{~opposed} to\textcolor{blue}{~in}\textcolor{blue}{~one}\textcolor{blue}{'s} imagination or virtual\textcolor{blue}{~reality}\textcolor{blue}{.}\textcolor{blue}{~
}\textcolor{blue}{D}\textcolor{blue}{istingu}ishing\textcolor{blue}{~the}\textcolor{blue}{~real}\textcolor{blue}{~world}\textcolor{blue}{~from}\textcolor{blue}{~the}\textcolor{blue}{~imaginary}\textcolor{blue}{~can}\textcolor{blue}{~be}\textcolor{blue}{~difficult}\textcolor{blue}{,} as imagination\textcolor{blue}{~and} perceptions can be\textcolor{blue}{~influenced}\textcolor{blue}{~by} personal beliefs\textcolor{blue}{,} emotions\textcolor{blue}{,}\textcolor{blue}{~and} past experiences\textcolor{blue}{.}\textcolor{blue}{~However}\textcolor{blue}{,} in\textcolor{blue}{~general}\textcolor{blue}{,}\textcolor{blue}{~the}\textcolor{blue}{~real}\textcolor{blue}{~world}\textcolor{blue}{~is} considered\textcolor{blue}{~to} be the\textcolor{blue}{~physical} environment we inhabit and\textcolor{blue}{~experience} through our senses, whereas\textcolor{blue}{~the}\textcolor{blue}{~imaginary}\textcolor{blue}{~world}\textcolor{blue}{~is}\textcolor{blue}{~one}\textcolor{blue}{~that}\textcolor{blue}{~exists} solely\textcolor{blue}{~in} our\textcolor{blue}{~minds} and\textcolor{blue}{~is}\textcolor{blue}{~not}\textcolor{blue}{~physically}\textcolor{blue}{~present}\textcolor{blue}{.
}\textcolor{blue}{Virtual}\textcolor{blue}{~reality} experiences can\textcolor{blue}{~provide}\textcolor{blue}{~a} sense of\textcolor{blue}{~immersion}\textcolor{blue}{~that}\textcolor{blue}{~can}\textcolor{blue}{~feel} very\textcolor{blue}{~real}\textcolor{blue}{,}\textcolor{blue}{~but}\textcolor{blue}{~they}\textcolor{blue}{~are}\textcolor{blue}{~still}\textcolor{blue}{~created} and\textcolor{blue}{~controlled} by\textcolor{blue}{~technology},\textcolor{blue}{~and}\textcolor{blue}{~not}\textcolor{blue}{~experienced} in\textcolor{blue}{~the}\textcolor{blue}{~physical}\textcolor{blue}{~world}\textcolor{blue}{.}\textcolor{blue}{~As}\textcolor{blue}{~such},\textcolor{blue}{~they}\textcolor{blue}{~are}\textcolor{blue}{~not} typically\textcolor{blue}{~considered}\textcolor{blue}{~"}\textcolor{blue}{real}\textcolor{blue}{~life}\textcolor{blue}{."}\textcolor{blue}{~
}\textcolor{blue}{Our}\textcolor{blue}{~senses} play\textcolor{blue}{~a}\textcolor{blue}{~critical} role in interpreting\textcolor{blue}{~and}\textcolor{blue}{~experiencing}\textcolor{blue}{~reality}\textcolor{blue}{.}\textcolor{blue}{~Our} brains\textcolor{blue}{~receive} information from our\textcolor{blue}{~eyes}\textcolor{blue}{,} ears,\textcolor{blue}{~nose},\textcolor{blue}{~skin}\textcolor{blue}{,} and other sensory\textcolor{blue}{~receptors},\textcolor{blue}{~and}\textcolor{blue}{~then}\textcolor{blue}{~process} and interpret\textcolor{blue}{~that}\textcolor{blue}{~information} to create our\textcolor{blue}{~experience}\textcolor{blue}{~of}\textcolor{blue}{~the}\textcolor{blue}{~world}. However\textcolor{blue}{,}\textcolor{blue}{~our} senses are not perfect and\textcolor{blue}{~can}\textcolor{blue}{~be}\textcolor{blue}{~influenced}\textcolor{blue}{~by} factors such as fatigue\textcolor{blue}{,} illness, or external\textcolor{blue}{~stimuli}, which\textcolor{blue}{~can}\textcolor{blue}{~alter}\textcolor{blue}{~our}\textcolor{blue}{~perceptions}\textcolor{blue}{~of} reality\textcolor{blue}{.}\textcolor{blue}{}
\end{tcolorbox}



\begin{tcolorbox}[colframe=gray!70!black,colback=white, title=Sample 2]
\textbf{User:}
In a restaurant kitchen, a chef uses a food processor to prepare a sauce. The recipe calls for 2 cups of chopped tomatoes, 1 cup of diced onions, and 1/2 cup of minced garlic. The food processor has a maximum capacity of 4 cups. If the chef wants to make enough sauce for 6 servings, how many times must the ingredients be processed in the food processor? 
Hint: Assume that the food processor can process the entire recipe at once. \vspace{0.1in}
\\
\textbf{Assistant:}
\textcolor{blue}{The}\textcolor{blue}{~recipe}\textcolor{blue}{~requires}\textcolor{blue}{~a}\textcolor{blue}{~total}\textcolor{blue}{~of}\textcolor{blue}{~}2 +\textcolor{blue}{~}\textcolor{blue}{1}\textcolor{blue}{~+}\textcolor{blue}{~}\textcolor{blue}{0}\textcolor{blue}{.}\textcolor{blue}{5} =\textcolor{blue}{~}\textcolor{blue}{3}\textcolor{blue}{.}\textcolor{blue}{5}\textcolor{blue}{~cups}\textcolor{blue}{~of}\textcolor{blue}{~ingredients}\textcolor{blue}{.}\textcolor{blue}{~To}\textcolor{blue}{~make}\textcolor{blue}{~enough}\textcolor{blue}{~sauce}\textcolor{blue}{~for}\textcolor{blue}{~}\textcolor{blue}{6}\textcolor{blue}{~servings},\textcolor{blue}{~the}\textcolor{blue}{~chef}\textcolor{blue}{~needs} to\textcolor{blue}{~prepare} 6 *\textcolor{blue}{~}3.\textcolor{blue}{5} =\textcolor{blue}{~}\textcolor{blue}{21}\textcolor{blue}{~cups}\textcolor{blue}{~of} ingredients in\textcolor{blue}{~total}\textcolor{blue}{.}\textcolor{blue}{~
}\textcolor{blue}{Since}\textcolor{blue}{~the}\textcolor{blue}{~food}\textcolor{blue}{~processor} can\textcolor{blue}{~process}\textcolor{blue}{~up} to\textcolor{blue}{~}\textcolor{blue}{4}\textcolor{blue}{~cups} of\textcolor{blue}{~ingredients}\textcolor{blue}{~at} a\textcolor{blue}{~time},\textcolor{blue}{~the}\textcolor{blue}{~chef} needs\textcolor{blue}{~to}\textcolor{blue}{~process}\textcolor{blue}{~the}\textcolor{blue}{~ingredients} 21\textcolor{blue}{~/}\textcolor{blue}{~}\textcolor{blue}{4}\textcolor{blue}{~=}\textcolor{blue}{~}\textcolor{blue}{5}\textcolor{blue}{.}\textcolor{blue}{25}\textcolor{blue}{~times}\textcolor{blue}{.} 
\textcolor{blue}{However}\textcolor{blue}{,} since\textcolor{blue}{~the} chef\textcolor{blue}{~cannot} process\textcolor{blue}{~ingredients} a fraction\textcolor{blue}{~of}\textcolor{blue}{~a}\textcolor{blue}{~time}, they\textcolor{blue}{~will} need\textcolor{blue}{~to} process the ingredients 6\textcolor{blue}{~times}\textcolor{blue}{~to} make\textcolor{blue}{~enough} sauce for\textcolor{blue}{~}\textcolor{blue}{6}\textcolor{blue}{~servings}\textcolor{blue}{.}\textcolor{blue}{}
\end{tcolorbox}




\begin{tcolorbox}[colframe=gray!70!black,colback=white, title=Sample 3]
\textbf{User:}
Write a response that appropriately completes the request.\\
\#\#\# Instruction:
Does Palmeiras have a Fifa club world cup?\\
\#\#\# Response:  \vspace{0.1in}
\\
\textbf{Assistant:}
\textcolor{blue}{Despite} many claims that\textcolor{blue}{~Pal}\textcolor{blue}{me}\textcolor{blue}{iras}\textcolor{blue}{~won}\textcolor{blue}{~a}\textcolor{blue}{~club} world\textcolor{blue}{~cup}\textcolor{blue}{~in}\textcolor{blue}{~}\textcolor{blue}{195}\textcolor{blue}{1},\textcolor{blue}{~the} official\textcolor{blue}{~list}\textcolor{blue}{~of}\textcolor{blue}{~the} Fif\textcolor{blue}{a} World\textcolor{blue}{~Club} winners\textcolor{blue}{~starts} in the year\textcolor{blue}{~}2000\textcolor{blue}{~with} Corinthians\textcolor{blue}{~being}\textcolor{blue}{~the} first champion of\textcolor{blue}{~the}\textcolor{blue}{~tournament}\textcolor{blue}{~and}\textcolor{blue}{~Pal}me\textcolor{blue}{iras}\textcolor{blue}{~doesn}\textcolor{blue}{'t}\textcolor{blue}{~have}\textcolor{blue}{~won}\textcolor{blue}{~any}\textcolor{blue}{~Fif}\textcolor{blue}{a} World\textcolor{blue}{~Cup}\textcolor{blue}{~tournament} since\textcolor{blue}{.}\textcolor{blue}{}
\end{tcolorbox}



\begin{tcolorbox}[colframe=gray!70!black,colback=white, title=Sample 4]
\textbf{User:}
Seven Wonders of the Underwater World. Answer:  \vspace{0.1in}
\\
\textbf{Assistant:}
\textcolor{blue}{Here}'s\textcolor{blue}{~a}\textcolor{blue}{~list}\textcolor{blue}{~of} the\textcolor{blue}{~Seven}\textcolor{blue}{~W}onders\textcolor{blue}{~of}\textcolor{blue}{~the}\textcolor{blue}{~Under}\textcolor{blue}{water}\textcolor{blue}{~World}\textcolor{blue}{:
}- Pal\textcolor{blue}{au}\textcolor{blue}{
}\textcolor{blue}{-}\textcolor{blue}{~Belize}\textcolor{blue}{~Barrier}\textcolor{blue}{~Reef},\textcolor{blue}{~Belize}\textcolor{blue}{
}\textcolor{blue}{-} Great\textcolor{blue}{~Barrier}\textcolor{blue}{~Reef}\textcolor{blue}{,}\textcolor{blue}{~Australia}\textcolor{blue}{
}\textcolor{blue}{-}\textcolor{blue}{~Deep}\textcolor{blue}{-se}a\textcolor{blue}{~hydro}\textcolor{blue}{thermal}\textcolor{blue}{~vents} (\textcolor{blue}{world}wide\textcolor{blue}{)
}\textcolor{blue}{-} Galápagos\textcolor{blue}{~Islands}\textcolor{blue}{,}\textcolor{blue}{~Ecuador}\textcolor{blue}{
}\textcolor{blue}{-} Lake Ba\textcolor{blue}{ikal}, Russia\textcolor{blue}{
}\textcolor{blue}{-}\textcolor{blue}{~Northern}\textcolor{blue}{~Red}\textcolor{blue}{~Sea}, bordered\textcolor{blue}{~by} Saudi Arabia and\textcolor{blue}{~Yemen} on\textcolor{blue}{~the}\textcolor{blue}{~eastern} shore, and Egypt\textcolor{blue}{,}\textcolor{blue}{~Sudan}, Eritrea, and Djibouti on the western shore\textcolor{blue}{}
\end{tcolorbox}


\begin{tcolorbox}[colframe=gray!70!black,colback=white, title=Sample 5]
\textbf{User:}
What's the furthest back we can prove human existence? Is there good reason to believe it goes further then this point? \vspace{0.1in}
\\
\textbf{Assistant:}
\textcolor{blue}{Human} is an\textcolor{blue}{~ever}\textcolor{blue}{-ev}\textcolor{blue}{olving}\textcolor{blue}{~species}. If\textcolor{blue}{~we}\textcolor{blue}{~starts}\textcolor{blue}{~from} \textcolor{blue}{H}\textcolor{blue}{omo}\textcolor{blue}{~sapi}\textcolor{blue}{ens}\textcolor{blue}{~the}\textcolor{blue}{~species},\textcolor{blue}{~there} are\textcolor{blue}{~traces}\textcolor{blue}{~of}\textcolor{blue}{~human} in\textcolor{blue}{~Africa} since\textcolor{blue}{~}\textcolor{blue}{300}\textcolor{blue}{,}000\textcolor{blue}{~years}\textcolor{blue}{~ago}\textcolor{blue}{.} For H\textcolor{blue}{omo}genus\textcolor{blue}{,}\textcolor{blue}{~earliest} fossil\textcolor{blue}{~record}\textcolor{blue}{~discovered} so\textcolor{blue}{~far} is\textcolor{blue}{~dated}\textcolor{blue}{~around}\textcolor{blue}{~}\textcolor{blue}{2}\textcolor{blue}{.}\textcolor{blue}{8}\textcolor{blue}{~million}\textcolor{blue}{~years}\textcolor{blue}{~ago}\textcolor{blue}{.
}\textcolor{blue}{The} above\textcolor{blue}{~dates} are\textcolor{blue}{~based} on\textcolor{blue}{~fossil} records\textcolor{blue}{.}\textcolor{blue}{~It}\textcolor{blue}{~is}\textcolor{blue}{~completely}\textcolor{blue}{~possible} for new fossil\textcolor{blue}{~emerges}\textcolor{blue}{~with}\textcolor{blue}{~an}\textcolor{blue}{~earlier} date\textcolor{blue}{.} It\textcolor{blue}{~is} also possible\textcolor{blue}{~that} carbon\textcolor{blue}{~dating}\textcolor{blue}{,}\textcolor{blue}{~technique}\textcolor{blue}{~used}\textcolor{blue}{~to}\textcolor{blue}{~determine} fossil\textcolor{blue}{~age},\textcolor{blue}{~may}\textcolor{blue}{~not} be\textcolor{blue}{~accurate}\textcolor{blue}{.} For example\textcolor{blue}{~if}\textcolor{blue}{~there} are\textcolor{blue}{~more}\textcolor{blue}{~carbon}-\textcolor{blue}{14}\textcolor{blue}{~than} we thought\textcolor{blue}{~there}\textcolor{blue}{~would} be in\textcolor{blue}{~the} past\textcolor{blue}{,}\textcolor{blue}{~we} would underestimate\textcolor{blue}{~the}\textcolor{blue}{~age}\textcolor{blue}{~of} existing fossils,\textcolor{blue}{~pushing}\textcolor{blue}{~human} history\textcolor{blue}{~further} back\textcolor{blue}{.}\textcolor{blue}{}\textcolor{blue}{
}\vspace{0.1in}
\\
\textbf{User:}
How reliable is carbon dating exactly?\vspace{0.1in}
\\
\textbf{Assistant:}
There is\textcolor{blue}{~no}\textcolor{blue}{~exact} answer\textcolor{blue}{~to}\textcolor{blue}{~how}\textcolor{blue}{~reliable}\textcolor{blue}{~carbon}\textcolor{blue}{~dating}\textcolor{blue}{~is}.\textcolor{blue}{~It}\textcolor{blue}{~depends}\textcolor{blue}{~on} many\textcolor{blue}{~factors}, such as\textcolor{blue}{~the}\textcolor{blue}{~quality} of\textcolor{blue}{~the}\textcolor{blue}{~sample}\textcolor{blue}{,}\textcolor{blue}{~the}\textcolor{blue}{~calibration} of the\textcolor{blue}{~equipment}\textcolor{blue}{,}\textcolor{blue}{~and}\textcolor{blue}{~the} assumptions made\textcolor{blue}{~about} the past\textcolor{blue}{.}\textcolor{blue}{~Generally} speaking\textcolor{blue}{,}\textcolor{blue}{~carbon}\textcolor{blue}{~dating} can\textcolor{blue}{~be}\textcolor{blue}{~accurate} up to\textcolor{blue}{~about}\textcolor{blue}{~}50\textcolor{blue}{,}000 years\textcolor{blue}{~ago}\textcolor{blue}{,} but\textcolor{blue}{~beyond}\textcolor{blue}{~that} it\textcolor{blue}{~becomes} less\textcolor{blue}{~reliable}\textcolor{blue}{.} The\textcolor{blue}{~margin}\textcolor{blue}{~of}\textcolor{blue}{~error}\textcolor{blue}{~can}\textcolor{blue}{~range}\textcolor{blue}{~from} a\textcolor{blue}{~few}\textcolor{blue}{~years}\textcolor{blue}{~to}\textcolor{blue}{~several} thousand years\textcolor{blue}{,}\textcolor{blue}{~depending} on the situation\textcolor{blue}{.}\textcolor{blue}{}
\end{tcolorbox}



\begin{tcolorbox}[colframe=gray!70!black,colback=white, title=Sample 6]
\textbf{User:}
Develop a comprehensive metric that takes into consideration various factors such as user engagement, user feedback, user retention, user conversion rate, and user satisfaction to evaluate the effectiveness of a given phone application designed to connect people with mental health professionals. The metric should also take into account the different types of mental health conditions that the application caters to, such as depression, anxiety, bipolar disorder, and schizophrenia, and should provide separate scores for each condition. Additionally, the metric should consider the geographical location of the users and the availability of mental health professionals in their area. The evaluation should be done over a period of at least six months to account for any seasonal variations. Finally, the metric should be presented in a format that is easily understandable by both technical and non-technical stakeholders, such as a dashboard with charts and graphs, and should also include recommendations for improving the application based on the evaluation results. \vspace{0.1in} \\
\textbf{Assistant:}
\textcolor{blue}{The} following is\textcolor{blue}{~a}\textcolor{blue}{~comprehensive}\textcolor{blue}{~metric}\textcolor{blue}{~that}\textcolor{blue}{~takes}\textcolor{blue}{~into}\textcolor{blue}{~consideration} various\textcolor{blue}{~factors}\textcolor{blue}{~to}\textcolor{blue}{~evaluate}\textcolor{blue}{~the}\textcolor{blue}{~effectiveness}\textcolor{blue}{~of}\textcolor{blue}{~a} given\textcolor{blue}{~phone}\textcolor{blue}{~application}\textcolor{blue}{~designed}\textcolor{blue}{~to}\textcolor{blue}{~connect}\textcolor{blue}{~people}\textcolor{blue}{~with}\textcolor{blue}{~mental}\textcolor{blue}{~health}\textcolor{blue}{~professionals}:
\textcolor{blue}{1}\textcolor{blue}{.}\textcolor{blue}{~User} Engagement\textcolor{blue}{:}\textcolor{blue}{~This}\textcolor{blue}{~factor}\textcolor{blue}{~measures}\textcolor{blue}{~the} level\textcolor{blue}{~of} user interaction\textcolor{blue}{~with} the\textcolor{blue}{~application}\textcolor{blue}{.}\textcolor{blue}{~The} following\textcolor{blue}{~metrics} can\textcolor{blue}{~be}\textcolor{blue}{~used}\textcolor{blue}{~to} evaluate\textcolor{blue}{~user} engagement\textcolor{blue}{:
}\textcolor{blue}{-}\textcolor{blue}{~Number} of downloads:\textcolor{blue}{~The} total number of\textcolor{blue}{~downloads}\textcolor{blue}{~of}\textcolor{blue}{~the}\textcolor{blue}{~application}\textcolor{blue}{~over} a\textcolor{blue}{~period}\textcolor{blue}{~of}\textcolor{blue}{~six}\textcolor{blue}{~months}\textcolor{blue}{.
}\textcolor{blue}{-} Active Users\textcolor{blue}{:}\textcolor{blue}{~The} number of\textcolor{blue}{~users} who actively\textcolor{blue}{~use} the\textcolor{blue}{~application} on\textcolor{blue}{~a} weekly\textcolor{blue}{~or}\textcolor{blue}{~monthly} basis\textcolor{blue}{.
}\textcolor{blue}{-} Session Duration\textcolor{blue}{:}\textcolor{blue}{~The} average time spent by\textcolor{blue}{~users}\textcolor{blue}{~on}\textcolor{blue}{~the}\textcolor{blue}{~application} during each\textcolor{blue}{~session}\textcolor{blue}{.
}\textcolor{blue}{-} Frequency\textcolor{blue}{~of} Use\textcolor{blue}{:}\textcolor{blue}{~The} number of times\textcolor{blue}{~users} access\textcolor{blue}{~the}\textcolor{blue}{~application} in\textcolor{blue}{~a}\textcolor{blue}{~week}\textcolor{blue}{~or}\textcolor{blue}{~month}\textcolor{blue}{.
}2.\textcolor{blue}{~User}\textcolor{blue}{~Feedback}\textcolor{blue}{:}\textcolor{blue}{~This}\textcolor{blue}{~factor}\textcolor{blue}{~measures} the\textcolor{blue}{~level} of satisfaction of\textcolor{blue}{~users}\textcolor{blue}{~with}\textcolor{blue}{~the}\textcolor{blue}{~application}\textcolor{blue}{.}\textcolor{blue}{~The} following\textcolor{blue}{~metrics} can be used to\textcolor{blue}{~evaluate}\textcolor{blue}{~user}\textcolor{blue}{~feedback}\textcolor{blue}{:
}\textcolor{blue}{-} App\textcolor{blue}{~Rating}\textcolor{blue}{:}\textcolor{blue}{~The} average rating\textcolor{blue}{~of}\textcolor{blue}{~the}\textcolor{blue}{~application}\textcolor{blue}{~on}\textcolor{blue}{~app}\textcolor{blue}{~stores} such as Google Play Store\textcolor{blue}{~and}\textcolor{blue}{~Apple} App\textcolor{blue}{~Store}\textcolor{blue}{.
}-\textcolor{blue}{~User} Reviews\textcolor{blue}{:}\textcolor{blue}{~The}\textcolor{blue}{~number}\textcolor{blue}{~and} content of\textcolor{blue}{~user}\textcolor{blue}{~reviews}\textcolor{blue}{~on}\textcolor{blue}{~app}\textcolor{blue}{~stores} and other forums.
\textcolor{blue}{-} Net\textcolor{blue}{~Prom}oter\textcolor{blue}{~Score}\textcolor{blue}{~(}NPS\textcolor{blue}{):} A\textcolor{blue}{~metric}\textcolor{blue}{~that}\textcolor{blue}{~measures} the likelihood\textcolor{blue}{~of}\textcolor{blue}{~users}\textcolor{blue}{~to}\textcolor{blue}{~recommend}\textcolor{blue}{~the}\textcolor{blue}{~application}\textcolor{blue}{~to}\textcolor{blue}{~others}\textcolor{blue}{.
}3.\textcolor{blue}{~User}\textcolor{blue}{~Ret}ention\textcolor{blue}{:}\textcolor{blue}{~This}\textcolor{blue}{~factor}\textcolor{blue}{~measures}\textcolor{blue}{~the} ability\textcolor{blue}{~of}\textcolor{blue}{~the}\textcolor{blue}{~application} to\textcolor{blue}{~retain}\textcolor{blue}{~users} over\textcolor{blue}{~a}\textcolor{blue}{~period}\textcolor{blue}{~of} time\textcolor{blue}{.}\textcolor{blue}{~The} following metrics can be used to\textcolor{blue}{~evaluate} user retention\textcolor{blue}{:
}\textcolor{blue}{-} Churn Rate\textcolor{blue}{:}\textcolor{blue}{~The} percentage of\textcolor{blue}{~users} who stop using the\textcolor{blue}{~application}\textcolor{blue}{~over}\textcolor{blue}{~a}\textcolor{blue}{~period}\textcolor{blue}{~of} time\textcolor{blue}{.
}\textcolor{blue}{-}\textcolor{blue}{~Ret}\textcolor{blue}{ention}\textcolor{blue}{~Rate}\textcolor{blue}{:}\textcolor{blue}{~The}\textcolor{blue}{~percentage} of users who continue\textcolor{blue}{~to} use the\textcolor{blue}{~application}\textcolor{blue}{~over}\textcolor{blue}{~a}\textcolor{blue}{~period} of time\textcolor{blue}{.
}4.\textcolor{blue}{~User}\textcolor{blue}{~Conversion}\textcolor{blue}{~Rate}\textcolor{blue}{:}\textcolor{blue}{~This}\textcolor{blue}{~factor} measures the ability of\textcolor{blue}{~the} application to\textcolor{blue}{~convert}\textcolor{blue}{~users} into paying customers\textcolor{blue}{.}\textcolor{blue}{~The}\textcolor{blue}{~following}\textcolor{blue}{~metrics} can be\textcolor{blue}{~used} to\textcolor{blue}{~evaluate}\textcolor{blue}{~user}\textcolor{blue}{~conversion}\textcolor{blue}{~rate}\textcolor{blue}{:
}\textcolor{blue}{-}\textcolor{blue}{~Conversion}\textcolor{blue}{~Rate}\textcolor{blue}{:}\textcolor{blue}{~The}\textcolor{blue}{~percentage} of\textcolor{blue}{~users} who upgrade\textcolor{blue}{~to}\textcolor{blue}{~a}\textcolor{blue}{~paid}\textcolor{blue}{~subscription} or purchase additional\textcolor{blue}{~services}\textcolor{blue}{.
}\textcolor{blue}{-}\textcolor{blue}{~Average} Revenue Per\textcolor{blue}{~User}\textcolor{blue}{~(}AR\textcolor{blue}{PU}\textcolor{blue}{):}\textcolor{blue}{~The}\textcolor{blue}{~average} amount\textcolor{blue}{~of}\textcolor{blue}{~revenue}\textcolor{blue}{~generated}\textcolor{blue}{~per}\textcolor{blue}{~user}.
\textcolor{blue}{5}. User\textcolor{blue}{~Satisfaction}:\textcolor{blue}{~This} factor measures the\textcolor{blue}{~level}\textcolor{blue}{~of}\textcolor{blue}{~satisfaction} of\textcolor{blue}{~users} with the\textcolor{blue}{~mental}\textcolor{blue}{~health}\textcolor{blue}{~professionals}\textcolor{blue}{~they} connect\textcolor{blue}{~with} through\textcolor{blue}{~the}\textcolor{blue}{~application}\textcolor{blue}{.}\textcolor{blue}{~The}\textcolor{blue}{~following}\textcolor{blue}{~metrics}\textcolor{blue}{~can}\textcolor{blue}{~be}\textcolor{blue}{~used}\textcolor{blue}{~to}\textcolor{blue}{~evaluate}\textcolor{blue}{~user}\textcolor{blue}{~satisfaction}\textcolor{blue}{:
}\textcolor{blue}{-}\textcolor{blue}{~Patient}\textcolor{blue}{~Satisfaction}\textcolor{blue}{~Score}\textcolor{blue}{~(}\textcolor{blue}{P}\textcolor{blue}{SS}\textcolor{blue}{):}\textcolor{blue}{~A}\textcolor{blue}{~metric}\textcolor{blue}{~that}\textcolor{blue}{~measures}\textcolor{blue}{~the} satisfaction\textcolor{blue}{~of} patients\textcolor{blue}{~with} the\textcolor{blue}{~mental}\textcolor{blue}{~health}\textcolor{blue}{~professionals}\textcolor{blue}{~they}\textcolor{blue}{~connect}\textcolor{blue}{~with} through\textcolor{blue}{~the}\textcolor{blue}{~application}\textcolor{blue}{.
}\textcolor{blue}{-}\textcolor{blue}{~Mental}\textcolor{blue}{~Health} Professional\textcolor{blue}{~Satisfaction}\textcolor{blue}{~Score}\textcolor{blue}{~(}MHP\textcolor{blue}{SS}\textcolor{blue}{):}\textcolor{blue}{~A}\textcolor{blue}{~metric} that\textcolor{blue}{~measures}\textcolor{blue}{~the} satisfaction\textcolor{blue}{~of}\textcolor{blue}{~mental}\textcolor{blue}{~health}\textcolor{blue}{~professionals}\textcolor{blue}{~with}\textcolor{blue}{~the}\textcolor{blue}{~application} and\textcolor{blue}{~the}\textcolor{blue}{~quality}\textcolor{blue}{~of}\textcolor{blue}{~patients}\textcolor{blue}{~they}\textcolor{blue}{~connect}\textcolor{blue}{~with}\textcolor{blue}{.
}6.\textcolor{blue}{~Mental}\textcolor{blue}{~Health} Condition:\textcolor{blue}{~This}\textcolor{blue}{~factor} takes into account\textcolor{blue}{~the}\textcolor{blue}{~different}\textcolor{blue}{~types}\textcolor{blue}{~of}\textcolor{blue}{~mental}\textcolor{blue}{~health}\textcolor{blue}{~conditions}\textcolor{blue}{~that}\textcolor{blue}{~the}\textcolor{blue}{~application}\textcolor{blue}{~cat}ers\textcolor{blue}{~to}, such\textcolor{blue}{~as}\textcolor{blue}{~depression}\textcolor{blue}{,} anxiety\textcolor{blue}{,} bipolar\textcolor{blue}{~disorder}\textcolor{blue}{,}\textcolor{blue}{~and}\textcolor{blue}{~schizophrenia}\textcolor{blue}{.} The following\textcolor{blue}{~metrics} can be\textcolor{blue}{~used} to evaluate\textcolor{blue}{~the}\textcolor{blue}{~effectiveness}\textcolor{blue}{~of}\textcolor{blue}{~the}\textcolor{blue}{~application}\textcolor{blue}{~for}\textcolor{blue}{~each}\textcolor{blue}{~condition}\textcolor{blue}{:
}\textcolor{blue}{-} Condition\textcolor{blue}{-specific}\textcolor{blue}{~Engagement}\textcolor{blue}{:}\textcolor{blue}{~The}\textcolor{blue}{~level} of\textcolor{blue}{~engagement} of\textcolor{blue}{~users}\textcolor{blue}{~with}\textcolor{blue}{~the}\textcolor{blue}{~application}\textcolor{blue}{~for}\textcolor{blue}{~each}\textcolor{blue}{~mental}\textcolor{blue}{~health}\textcolor{blue}{~condition}\textcolor{blue}{.
}\textcolor{blue}{-}\textcolor{blue}{~Condition}\textcolor{blue}{-specific}\textcolor{blue}{~Feedback}\textcolor{blue}{:} The\textcolor{blue}{~level} of\textcolor{blue}{~satisfaction}\textcolor{blue}{~of}\textcolor{blue}{~users} with the application\textcolor{blue}{~for} each\textcolor{blue}{~mental}\textcolor{blue}{~health} condition\textcolor{blue}{.
}\textcolor{blue}{-}\textcolor{blue}{~Condition}\textcolor{blue}{-specific}\textcolor{blue}{~Ret}ention\textcolor{blue}{:} The ability of\textcolor{blue}{~the}\textcolor{blue}{~application}\textcolor{blue}{~to}\textcolor{blue}{~retain}\textcolor{blue}{~users}\textcolor{blue}{~with}\textcolor{blue}{~each}\textcolor{blue}{~mental} health condition over\textcolor{blue}{~a}\textcolor{blue}{~period}\textcolor{blue}{~of} time\textcolor{blue}{.
}\textcolor{blue}{-}\textcolor{blue}{~Condition}\textcolor{blue}{-specific}\textcolor{blue}{~Conversion}\textcolor{blue}{~Rate}\textcolor{blue}{:}\textcolor{blue}{~The}\textcolor{blue}{~ability} of\textcolor{blue}{~the}\textcolor{blue}{~application}\textcolor{blue}{~to}\textcolor{blue}{~convert}\textcolor{blue}{~users}\textcolor{blue}{~with}\textcolor{blue}{~each}\textcolor{blue}{~mental}\textcolor{blue}{~health}\textcolor{blue}{~condition} into\textcolor{blue}{~paying}\textcolor{blue}{~customers}\textcolor{blue}{.
}\textcolor{blue}{-}\textcolor{blue}{~Condition}\textcolor{blue}{-specific}\textcolor{blue}{~Satisfaction}\textcolor{blue}{:}\textcolor{blue}{~The}\textcolor{blue}{~level} of\textcolor{blue}{~satisfaction}\textcolor{blue}{~of} users with the\textcolor{blue}{~mental}\textcolor{blue}{~health}\textcolor{blue}{~professionals}\textcolor{blue}{~they}\textcolor{blue}{~connect}\textcolor{blue}{~with} through\textcolor{blue}{~the}\textcolor{blue}{~application}\textcolor{blue}{~for} each\textcolor{blue}{~mental}\textcolor{blue}{~health}\textcolor{blue}{~condition}\textcolor{blue}{.
}\textcolor{blue}{7}.\textcolor{blue}{~Ge}\textcolor{blue}{ographical}\textcolor{blue}{~Location}\textcolor{blue}{:}\textcolor{blue}{~This} factor takes into\textcolor{blue}{~account}\textcolor{blue}{~the} geographical\textcolor{blue}{~location}\textcolor{blue}{~of} the\textcolor{blue}{~users}\textcolor{blue}{~and}\textcolor{blue}{~the}\textcolor{blue}{~availability}\textcolor{blue}{~of}\textcolor{blue}{~mental}\textcolor{blue}{~health}\textcolor{blue}{~professionals}\textcolor{blue}{~in}\textcolor{blue}{~their}\textcolor{blue}{~area}\textcolor{blue}{.}\textcolor{blue}{~The}\textcolor{blue}{~following}\textcolor{blue}{~metrics}\textcolor{blue}{~can}\textcolor{blue}{~be}\textcolor{blue}{~used}\textcolor{blue}{~to}\textcolor{blue}{~evaluate} the effectiveness\textcolor{blue}{~of}\textcolor{blue}{~the}\textcolor{blue}{~application} for each\textcolor{blue}{~geographical}\textcolor{blue}{~location}\textcolor{blue}{:
}\textcolor{blue}{-} Location\textcolor{blue}{-specific} Engagement\textcolor{blue}{:} The\textcolor{blue}{~level}\textcolor{blue}{~of} engagement\textcolor{blue}{~of} users with\textcolor{blue}{~the}\textcolor{blue}{~application} for\textcolor{blue}{~each}\textcolor{blue}{~geographical}\textcolor{blue}{~location}\textcolor{blue}{.
}\textcolor{blue}{-}\textcolor{blue}{~Location}\textcolor{blue}{-specific} Feedback\textcolor{blue}{:}\textcolor{blue}{~The}\textcolor{blue}{~level}\textcolor{blue}{~of}\textcolor{blue}{~satisfaction}\textcolor{blue}{~of}\textcolor{blue}{~users}\textcolor{blue}{~with}\textcolor{blue}{~the}\textcolor{blue}{~application}\textcolor{blue}{~for}\textcolor{blue}{~each}\textcolor{blue}{~geographical}\textcolor{blue}{~location}\textcolor{blue}{.
}\textcolor{blue}{-}\textcolor{blue}{~Location}\textcolor{blue}{-specific} Retention:\textcolor{blue}{~The}\textcolor{blue}{~ability}\textcolor{blue}{~of}\textcolor{blue}{~the}\textcolor{blue}{~application}\textcolor{blue}{~to}\textcolor{blue}{~retain}\textcolor{blue}{~users} in\textcolor{blue}{~each}\textcolor{blue}{~geographical}\textcolor{blue}{~location}\textcolor{blue}{~over}\textcolor{blue}{~a}\textcolor{blue}{~period}\textcolor{blue}{~of}\textcolor{blue}{~time}\textcolor{blue}{.
}-\textcolor{blue}{~Location}\textcolor{blue}{-specific}\textcolor{blue}{~Conversion}\textcolor{blue}{~Rate}\textcolor{blue}{:}\textcolor{blue}{~The}\textcolor{blue}{~ability}\textcolor{blue}{~of}\textcolor{blue}{~the}\textcolor{blue}{~application}\textcolor{blue}{~to}\textcolor{blue}{~convert}\textcolor{blue}{~users}\textcolor{blue}{~in}\textcolor{blue}{~each}\textcolor{blue}{~geographical}\textcolor{blue}{~location}\textcolor{blue}{~into}\textcolor{blue}{~paying}\textcolor{blue}{~customers}\textcolor{blue}{.
}-\textcolor{blue}{~Location}\textcolor{blue}{-specific}\textcolor{blue}{~Satisfaction}\textcolor{blue}{:}\textcolor{blue}{~The}\textcolor{blue}{~level}\textcolor{blue}{~of}\textcolor{blue}{~satisfaction}\textcolor{blue}{~of} users\textcolor{blue}{~with} the\textcolor{blue}{~mental}\textcolor{blue}{~health}\textcolor{blue}{~professionals} they\textcolor{blue}{~connect}\textcolor{blue}{~with}\textcolor{blue}{~through}\textcolor{blue}{~the}\textcolor{blue}{~application}\textcolor{blue}{~in}\textcolor{blue}{~each}\textcolor{blue}{~geographical}\textcolor{blue}{~location}\textcolor{blue}{.
}\textcolor{blue}{The}\textcolor{blue}{~evaluation} should\textcolor{blue}{~be}\textcolor{blue}{~done}\textcolor{blue}{~over}\textcolor{blue}{~a}\textcolor{blue}{~period} of\textcolor{blue}{~at} least\textcolor{blue}{~six} months to\textcolor{blue}{~account}\textcolor{blue}{~for} any\textcolor{blue}{~seasonal}\textcolor{blue}{~variations}\textcolor{blue}{.}\textcolor{blue}{~The}\textcolor{blue}{~metric}\textcolor{blue}{~can}\textcolor{blue}{~be}\textcolor{blue}{~presented}\textcolor{blue}{~in}\textcolor{blue}{~a} format\textcolor{blue}{~that}\textcolor{blue}{~is}\textcolor{blue}{~easily}\textcolor{blue}{~understandable}\textcolor{blue}{~by}\textcolor{blue}{~both}\textcolor{blue}{~technical}\textcolor{blue}{~and}\textcolor{blue}{~non}\textcolor{blue}{-}\textcolor{blue}{technical}\textcolor{blue}{~stakeholders},\textcolor{blue}{~such}\textcolor{blue}{~as}\textcolor{blue}{~a}\textcolor{blue}{~dashboard}\textcolor{blue}{~with}\textcolor{blue}{~charts}\textcolor{blue}{~and} graphs\textcolor{blue}{.}\textcolor{blue}{~The}\textcolor{blue}{~dashboard}\textcolor{blue}{~should}\textcolor{blue}{~include} separate\textcolor{blue}{~scores}\textcolor{blue}{~for}\textcolor{blue}{~each}\textcolor{blue}{~mental}\textcolor{blue}{~health}\textcolor{blue}{~condition}\textcolor{blue}{~and} each\textcolor{blue}{~geographical}\textcolor{blue}{~location}\textcolor{blue}{.} The dashboard\textcolor{blue}{~should}\textcolor{blue}{~also}\textcolor{blue}{~include}\textcolor{blue}{~recommendations}\textcolor{blue}{~for}\textcolor{blue}{~improving}\textcolor{blue}{~the}\textcolor{blue}{~application}\textcolor{blue}{~based} on\textcolor{blue}{~the}\textcolor{blue}{~evaluation}\textcolor{blue}{~results}.\textcolor{blue}{}
\end{tcolorbox}



    




    
    

















