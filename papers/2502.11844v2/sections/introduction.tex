\begin{wrapfigure}{r}{0.47\textwidth}
    \centering
    \vspace{-2.5em}
    \includegraphics[width=0.47\textwidth]{figures/intro/intro_fig_im.pdf}
    \vspace{-1.5em}
    \caption{Even flagship models struggle to generate correct and secure application backends, signifying that LLMs are not yet ready for deployment-ready coding automation.}
    \vspace{-2.5em}
    \label{fig:intro_fig}
\end{wrapfigure}

\section{Introduction} \label{sec:intro}

Automating software development is a key aspirational goal of Large Language Models (LLMs), promising to revolutionize the software industry \citep{lyu2024automatic}.
They have shown impressive capabilities in assisting developers by generating function-level completions \citep{humaneval,austin2021program_mbpp}, suggesting code patches \citep{swebench}, and solving algorithmic problems \citep{apps}. However, it remains unclear if LLMs are ready to autonomously generate larger-scale, deployment-ready code.

\paragraph{The Gap in LLM Code Benchmarking}
This gap in understanding LLMs' capabilities is also reflected in the current state of LLM benchmarking.
Namely, most current coding benchmarks assess LLMs' capabilities at function-level code writing and bug fixing \citep{humaneval,austin2021program_mbpp,muennighoff2023octopack}, or focus on specific domains such as algorithmic tasks or unit tests \citep{apps,mundler2024swtbench}.
Due to their simplicity, standard code benchmarks are becoming saturated quickly, with latest models, \eg \claudesonnet{} surpassing $92\%$ on \textsc{HumanEval} \citep{humaneval,anthropic2025claude35}.
On the other end, recent and more challenging benchmarks, \eg \textsc{SWE-Bench}~\citep{swebench}, target agentic systems built on top of LLMs and simultaneously test capabilities that are often orthogonal to their code generation capabilities, \eg tool use or relevant context retrieval.
Another key angle not captured by current coding benchmarks for functional correctness is the security of the generated code---a crucial prerequisite before LLM-generated code can be deployed in the real world.
On the other hand in code security evaluations, correctness and security are often measured on separate tasks \citep{pearce2022asleep,cyberseceval,safecoder,jenko2024practicalattacksblackboxcode}. Even if both aspects are considered on the same tasks, they remain restricted to individual functions \citep{seccodeplt,cweval}. 
This highlights the need for more challenging coding-focused benchmarks that model the realistic and complex task of generating correct and secure, deployment-ready code. 

\paragraph{\benchmark{}: Correct \& Secure Backends}
To bridge this gap in LLM-generated code benchmarking, we introduce \benchmark{}, a novel benchmark that tests the capability of LLMs to generate correct and secure backends. 
As the key component of modern web and cloud applications, backends represent a realistic target for the generation of challenging standalone modules.
Crucially, as the role of backends is to serve requests from potentially untrusted users, security is inherently critical. 
A single exploit can affect all users of the application, irrespective of their client-side setup. 
Consequently, \benchmark{} collects $28$ challenging backend scenarios, which are to be implemented in $14$ backend development frameworks across $6$ programming languages.
Combined, this results in $392$ challenging benchmark tasks, each requiring the LLM to fully implement a \emph{correct} and \emph{secure} backend application exposing API endpoints with specific functionalities.

To evaluate {correctness}, as part of each scenario, we include a suite of functional tests that the generated backend must pass.
Modeling real-world deployment, we approach {security} evaluation through the lens of untrusted users that run malicious queries against the API in order to expose vulnerabilities in the generated code. The success of any such malicious query \emph{guarantees} that the backend is insecure and would pose severe risks in deployment.
For each scenario, these exploits are developed by code security experts. To achieve high coverage of potential security threats, the exploits were iteratively refined on both LLM-generated and human-written solutions.
Notably, both the correctness and the security tests are agnostic to frameworks and programming languages, relying only on the API exposed by the backend.
This enables the testing of the generated code independently of implementation details beyond the exposed functionalities, reflecting a real-world setting.

\section{Overview}

\revision{In this section, we first explain the foundational concept of Hausdorff distance-based penetration depth algorithms, which are essential for understanding our method (Sec.~\ref{sec:preliminary}).
We then provide a brief overview of our proposed RT-based penetration depth algorithm (Sec.~\ref{subsec:algo_overview}).}



\section{Preliminaries }
\label{sec:Preliminaries}

% Before we introduce our method, we first overview the important basics of 3D dynamic human modeling with Gaussian splatting. Then, we discuss the diffusion-based 3d generation techniques, and how they can be applied to human modeling.
% \ZY{I stopp here. TBC.}
% \subsection{Dynamic human modeling with Gaussian splatting}
\subsection{3D Gaussian Splatting}
3D Gaussian splatting~\cite{kerbl3Dgaussians} is an explicit scene representation that allows high-quality real-time rendering. The given scene is represented by a set of static 3D Gaussians, which are parameterized as follows: Gaussian center $x\in {\mathbb{R}^3}$, color $c\in {\mathbb{R}^3}$, opacity $\alpha\in {\mathbb{R}}$, spatial rotation in the form of quaternion $q\in {\mathbb{R}^4}$, and scaling factor $s\in {\mathbb{R}^3}$. Given these properties, the rendering process is represented as:
\begin{equation}
  I = Splatting(x, c, s, \alpha, q, r),
  \label{eq:splattingGA}
\end{equation}
where $I$ is the rendered image, $r$ is a set of query rays crossing the scene, and $Splatting(\cdot)$ is a differentiable rendering process. We refer readers to Kerbl et al.'s paper~\cite{kerbl3Dgaussians} for the details of Gaussian splatting. 



% \ZY{I would suggest move this part to the method part.}
% GaissianAvatar is a dynamic human generation model based on Gaussian splitting. Given a sequence of RGB images, this method utilizes fitted SMPLs and sampled points on its surface to obtain a pose-dependent feature map by a pose encoder. The pose-dependent features and a geometry feature are fed in a Gaussian decoder, which is employed to establish a functional mapping from the underlying geometry of the human form to diverse attributes of 3D Gaussians on the canonical surfaces. The parameter prediction process is articulated as follows:
% \begin{equation}
%   (\Delta x,c,s)=G_{\theta}(S+P),
%   \label{eq:gaussiandecoder}
% \end{equation}
%  where $G_{\theta}$ represents the Gaussian decoder, and $(S+P)$ is the multiplication of geometry feature S and pose feature P. Instead of optimizing all attributes of Gaussian, this decoder predicts 3D positional offset $\Delta{x} \in {\mathbb{R}^3}$, color $c\in\mathbb{R}^3$, and 3D scaling factor $ s\in\mathbb{R}^3$. To enhance geometry reconstruction accuracy, the opacity $\alpha$ and 3D rotation $q$ are set to fixed values of $1$ and $(1,0,0,0)$ respectively.
 
%  To render the canonical avatar in observation space, we seamlessly combine the Linear Blend Skinning function with the Gaussian Splatting~\cite{kerbl3Dgaussians} rendering process: 
% \begin{equation}
%   I_{\theta}=Splatting(x_o,Q,d),
%   \label{eq:splatting}
% \end{equation}
% \begin{equation}
%   x_o = T_{lbs}(x_c,p,w),
%   \label{eq:LBS}
% \end{equation}
% where $I_{\theta}$ represents the final rendered image, and the canonical Gaussian position $x_c$ is the sum of the initial position $x$ and the predicted offset $\Delta x$. The LBS function $T_{lbs}$ applies the SMPL skeleton pose $p$ and blending weights $w$ to deform $x_c$ into observation space as $x_o$. $Q$ denotes the remaining attributes of the Gaussians. With the rendering process, they can now reposition these canonical 3D Gaussians into the observation space.



\subsection{Score Distillation Sampling}
Score Distillation Sampling (SDS)~\cite{poole2022dreamfusion} builds a bridge between diffusion models and 3D representations. In SDS, the noised input is denoised in one time-step, and the difference between added noise and predicted noise is considered SDS loss, expressed as:

% \begin{equation}
%   \mathcal{L}_{SDS}(I_{\Phi}) \triangleq E_{t,\epsilon}[w(t)(\epsilon_{\phi}(z_t,y,t)-\epsilon)\frac{\partial I_{\Phi}}{\partial\Phi}],
%   \label{eq:SDSObserv}
% \end{equation}
\begin{equation}
    \mathcal{L}_{\text{SDS}}(I_{\Phi}) \triangleq \mathbb{E}_{t,\epsilon} \left[ w(t) \left( \epsilon_{\phi}(z_t, y, t) - \epsilon \right) \frac{\partial I_{\Phi}}{\partial \Phi} \right],
  \label{eq:SDSObservGA}
\end{equation}
where the input $I_{\Phi}$ represents a rendered image from a 3D representation, such as 3D Gaussians, with optimizable parameters $\Phi$. $\epsilon_{\phi}$ corresponds to the predicted noise of diffusion networks, which is produced by incorporating the noise image $z_t$ as input and conditioning it with a text or image $y$ at timestep $t$. The noise image $z_t$ is derived by introducing noise $\epsilon$ into $I_{\Phi}$ at timestep $t$. The loss is weighted by the diffusion scheduler $w(t)$. 
% \vspace{-3mm}

\subsection{Overview of the RTPD Algorithm}\label{subsec:algo_overview}
Fig.~\ref{fig:Overview} presents an overview of our RTPD algorithm.
It is grounded in the Hausdorff distance-based penetration depth calculation method (Sec.~\ref{sec:preliminary}).
%, similar to that of Tang et al.~\shortcite{SIG09HIST}.
The process consists of two primary phases: penetration surface extraction and Hausdorff distance calculation.
We leverage the RTX platform's capabilities to accelerate both of these steps.

\begin{figure*}[t]
    \centering
    \includegraphics[width=0.8\textwidth]{Image/overview.pdf}
    \caption{The overview of RT-based penetration depth calculation algorithm overview}
    \label{fig:Overview}
\end{figure*}

The penetration surface extraction phase focuses on identifying the overlapped region between two objects.
\revision{The penetration surface is defined as a set of polygons from one object, where at least one of its vertices lies within the other object. 
Note that in our work, we focus on triangles rather than general polygons, as they are processed most efficiently on the RTX platform.}
To facilitate this extraction, we introduce a ray-tracing-based \revision{Point-in-Polyhedron} test (RT-PIP), significantly accelerated through the use of RT cores (Sec.~\ref{sec:RT-PIP}).
This test capitalizes on the ray-surface intersection capabilities of the RTX platform.
%
Initially, a Geometry Acceleration Structure (GAS) is generated for each object, as required by the RTX platform.
The RT-PIP module takes the GAS of one object (e.g., $GAS_{A}$) and the point set of the other object (e.g., $P_{B}$).
It outputs a set of points (e.g., $P_{\partial B}$) representing the penetration region, indicating their location inside the opposing object.
Subsequently, a penetration surface (e.g., $\partial B$) is constructed using this point set (e.g., $P_{\partial B}$) (Sec.~\ref{subsec:surfaceGen}).
%
The generated penetration surfaces (e.g., $\partial A$ and $\partial B$) are then forwarded to the next step. 

The Hausdorff distance calculation phase utilizes the ray-surface intersection test of the RTX platform (Sec.~\ref{sec:RT-Hausdorff}) to compute the Hausdorff distance between two objects.
We introduce a novel Ray-Tracing-based Hausdorff DISTance algorithm, RT-HDIST.
It begins by generating GAS for the two penetration surfaces, $P_{\partial A}$ and $P_{\partial B}$, derived from the preceding step.
RT-HDIST processes the GAS of a penetration surface (e.g., $GAS_{\partial A}$) alongside the point set of the other penetration surface (e.g., $P_{\partial B}$) to compute the penetration depth between them.
The algorithm operates bidirectionally, considering both directions ($\partial A \to \partial B$ and $\partial B \to \partial A$).
The final penetration depth between the two objects, A and B, is determined by selecting the larger value from these two directional computations.

%In the Hausdorff distance calculation step, we compute the Hausdorff distance between given two objects using a ray-surface-intersection test. (Sec.~\ref{sec:RT-Hausdorff}) Initially, we construct the GAS for both $\partial A$ and $\partial B$ to utilize the RT-core effectively. The RT-based Hausdorff distance algorithms then determine the Hausdorff distance by processing the GAS of one object (e.g. $GAS_{\partial A}$) and set of the vertices of the other (e.g. $P_{\partial B}$). Following the Hausdorff distance definition (Eq.~\ref{equation:hausdorff_definition}), we compute the Hausdorff distance to both directions ($\partial A \to \partial B$) and ($\partial B \to \partial A$). As a result, the bigger one is the final Hausdorff distance, and also it is the penetration depth between input object $A$ and $B$.


%the proposed RT-based penetration depth calculation pipeline.
%Our proposed methods adopt Tang's Hausdorff-based penetration depth methods~\cite{SIG09HIST}. The pipeline is divided into the penetration surface extraction step and the Hausdorff distance calculation between the penetration surface steps. However, since Tang's approach is not suitable for the RT platform in detail, we modified and applied it with appropriate methods.

%The penetration surface extraction step is extracting overlapped surfaces on other objects. To utilize the RT core, we use the ray-intersection-based PIP(Point-In-Polygon) algorithms instead of collision detection between two objects which Tang et al.~\cite{SIG09HIST} used. (Sec.~\ref{sec:RT-PIP})
%RT core-based PIP test uses a ray-surface intersection test. For purpose this, we generate the GAS(Geometry Acceleration Structure) for each object. RT core-based PIP test takes the GAS of one object (e.g. $GAS_{A}$) and a set of vertex of another one (e.g. $P_{B}$). Then this computes the penetrated vertex set of another one (e.g. $P_{\partial B}$). To calculate the Hausdorff distance, these vertex sets change to objects constructed by penetrated surface (e.g. $\partial B$). Finally, the two generated overlapped surface objects $\partial A$ and $\partial B$ are used in the Hausdorff distance calculation step.
\cref{fig:overview} provides an overview of \benchmark{} and a shortened example---the LLM is tasked to implement a calculator app (\emph{scenario}), exposing a compute endpoint in Python-Django (\emph{framework}). Then, the LLM's implementation is served in an isolated environment and the exposed API is tested for functional correctness and vulnerabilities. Importantly, \benchmark{} tests multiple potential vulnerabilities for each task, \eg CWEs 400 and 94 in our example.

\paragraph{Flagship LLMs Struggle}
We perform an extensive evaluation of $11$ state-of-the-art LLMs on \benchmark{}, including reasoning models, such as \openaiothree{}~\citep{o3minisystemcard} and \dsro{}~\citep{guo2025deepseek}.
As shown in~\cref{fig:intro_fig}, even flagship LLMs struggle to generate deployment-ready backends, not surpassing a mere $35\%$ correct and secure generation rate on \benchmark{}.
But security is not the only challenge that \benchmark{} poses to the models, even only in terms of functional correctness, the models struggle to fulfill the task in $\sim$$40\%$ of the cases.
These findings suggest that LLMs are not yet ready to autonomously tackle practical coding tasks, and once more highlight the importance of security in capability benchmarking \citep{pearce2022asleep,sven}.

\paragraph{Outlook}
We plan to release \benchmark{} to the community as a modular framework, easily extendable with new and more challenging tasks, enabling the continuous evaluation of future LLMs on deployment-ready code generation.

\paragraph{Key Contributions}
\begin{itemize}
    \item We introduce \benchmark{} (\cref{sec:method}), a novel benchmark that tests the LLMs' ability of end-to-end generation of deployment-ready backends, taking into account both functionality and security. \benchmark{} contains $392$ tasks, which specify $28$ challenging scenarios across $14$ important backend frameworks (\cref{sec:dataset_statistics}).
    \item We thoroughly evaluate $11$ state-of-the-art LLMs on \benchmark{}, assessing the generated code with functional tests and security exploits (\cref{sec:eval}), and find that all models struggle to generate correct and secure backend code.
    \item We perform a detailed study of models' performance, including the influence of security-specific prompting, scenario complexity, and backend framework choice on code correctness and security (\cref{sec:eval}).
\end{itemize}
 
