\section{\benchmark{} Statistics}
\label{sec:dataset_statistics}


\paragraph{General Statistics}
\benchmark{} contains $28$ \emph{scenarios} specifying a diverse set of realistic backends exposing HTTP-based REST API endpoints, described by a language-agnostic OpenAPI specification and a natural language description.
Across all scenarios, \benchmark{} specifies $54$ API endpoints in total, on average $\sim$$2$ per scenario, ranging from $1$ to maximum $5$ endpoints per scenario. Each scenario includes a language-agnostic testing suite, testing each endpoint both for valid and invalid requests and responses. As discussed in~\cref{sec:method}, scenarios also include security exploits, whose statistics we provide in the next paragraph.
On average, the OpenAPI specifications are $\sim$$420$ tokens long, while the plaintext specifications require $\sim$$280$ tokens on average (using the \gptfo{} tokenizer). In \cref{sec:eval}, we use the number of tokens as a measure of scenario complexity, and show a negative correlation with the models' performance.
\benchmark{} supports $14$ frameworks across $6$ programming languages.
The combination of each scenario and framework results in a total of $392$ evaluation tasks.
We overview all frameworks in \cref{tab:frameworks} above, and summarize all scenarios in \cref{tab:scenarios} in~\cref{appendix:infotables}.

\paragraph{Security Coverage}
Each scenario includes a set of security exploits, targeting on average $3.3$ CWEs per scenario, with a maximum of $5$ exposed CWEs for one scenario.
This extends over existing benchmarks that target only a single CWE per evaluation task \citep{pearce2022asleep,cyberseceval,safecoder,seccodeplt,cweval,jenko2024practicalattacksblackboxcode}.
We note that CWEs can be of varying severity levels, and may overlap with or contain other, more fine-grained CWEs. Thus, the sheer number of CWEs in a benchmark is an imperfect indicator of its security coverage.

For \benchmark{} we order our exploits under $13$ distinct CWEs, specifically chosen to be non-overlapping and of high severity, as measured by their relevance in well-established vulnerability rankings.
Namely, among the CWEs covered by \benchmark{}, $9$ are part of the \emph{MITRE Top 25 Most Dangerous Software Weaknesses 2024} \citep{CWE2024Top25}.
Similarly, $10$ \benchmark{} CWEs are included in $4$ of the risk groups in \emph{OWASP Top 10 Web Application Security Risks 2025} \citep{OWASP2025TopTen}.
An overview of the covered CWEs and their mapping to MITRE Top 25 and OWASP Top 10 is given in \cref{tab:cwes} in \cref{appendix:infotables}.
