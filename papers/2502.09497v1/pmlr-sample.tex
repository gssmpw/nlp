\documentclass[pmlr]{jmlr}% new name PMLR (Proceedings of Machine Learning Research)

 % The following packages will be automatically loaded:
 % amsmath, amssymb, natbib, graphicx, url, algorithm2e

 %\usepackage{rotating}% for sideways figures and tables
\usepackage{longtable}% for long tables

 % The booktabs package is used by this sample document
 % (it provides \toprule, \midrule and \bottomrule).
 % Remove the next line if you don't require it.
\usepackage{booktabs}
 % The siunitx package is used by this sample document
 % to align numbers in a column by their decimal point.
 % Remove the next line if you don't require it.
\usepackage[load-configurations=version-1]{siunitx} % newer version
 %\usepackage{siunitx}

 % The following command is just for this sample document:
\newcommand{\cs}[1]{\texttt{\char`\\#1}}

 % Define an unnumbered theorem just for this sample document:
\theorembodyfont{\upshape}
\theoremheaderfont{\scshape}
\theorempostheader{:}
\theoremsep{\newline}
\newtheorem*{note}{Note}
\newcommand{\joey}[1]{\textcolor{red}{\textbf{Joey:} #1}}
\usepackage{booktabs}
\usepackage{makecell}
\usepackage{multirow}

 % change the arguments, as appropriate, in the following:
\jmlrvolume{}
% \jmlryear{2025}
% \jmlrheading{1}
\jmlrpages{}
\jmlrproceedings{JMLR}{}
% \jmlrproceedings{JMLR}{2025 Conference on Artificial Intelligence}%
\jmlrworkshop{}
% \jmlrworkshop{Innovation and Responsibility in AI-Supported Education}

% \title[Short Title]{Full Title of Article\titlebreak This Title Has A Line Break\titletag{\thanks{sample footnote}}}

\title[Linguistic Features for LLM-based AES]{Improve LLM-based Automatic Essay Scoring with Linguistic Features}

 % Use \Name{Author Name} to specify the name.

 % Spaces are used to separate forenames from the surname so that
 % the surnames can be picked up for the page header and copyright footer.
 
 % If the surname contains spaces, enclose the surname
 % in braces, e.g. \Name{John {Smith Jones}} similarly
 % if the name has a "von" part, e.g \Name{Jane {de Winter}}.
 % If the first letter in the forenames is a diacritic
 % enclose the diacritic in braces, e.g. \Name{{\'E}louise Smith}

 % *** Make sure there's no spurious space before \nametag ***

 % Two authors with the same address
  % \author{\Name{Author Name1\nametag{\thanks{with a note}}} \Email{abc@sample.com}\and
  %  \Name{Author Name2} \Email{xyz@sample.com}\\
  %  \addr Address}

 % Three or more authors with the same address:
 \author{
    \Name{Zhaoyi Joey Hou} \Email{joey.hou@pitt.edu}\\
    \Name{Alejandro Ciuba} \Email{alejandrociuba@pitt.edu}\\
    \Name{Xiang Lorraine Li} \Email{xianglli@pitt.edu}\\
    % \addr 130 N Bellefield Ave, Pittsburgh, PA
}


 % Authors with different addresses:
 % \author{\Name{Author Name1} \Email{abc@sample.com}\\
 % \addr Address 1
 % \AND
 % \Name{Author Name2} \Email{xyz@sample.com}\\
 % \addr Address 2
 %}

% \editor{Editor's name}
 % \editors{List of editors' names}

\begin{document}

\maketitle

\begin{abstract}
Large language model (LLM)-based agents have shown promise in tackling complex tasks by interacting dynamically with the environment. 
Existing work primarily focuses on behavior cloning from expert demonstrations and preference learning through exploratory trajectory sampling. However, these methods often struggle in long-horizon tasks, where suboptimal actions accumulate step by step, causing agents to deviate from correct task trajectories.
To address this, we highlight the importance of \textit{timely calibration} and the need to automatically construct calibration trajectories for training agents. We propose \textbf{S}tep-Level \textbf{T}raj\textbf{e}ctory \textbf{Ca}libration (\textbf{\model}), a novel framework for LLM agent learning. 
Specifically, \model identifies suboptimal actions through a step-level reward comparison during exploration. It constructs calibrated trajectories using LLM-driven reflection, enabling agents to learn from improved decision-making processes. These calibrated trajectories, together with successful trajectory data, are utilized for reinforced training.
Extensive experiments demonstrate that \model significantly outperforms existing methods. Further analysis highlights that step-level calibration enables agents to complete tasks with greater robustness. 
Our code and data are available at \url{https://github.com/WangHanLinHenry/STeCa}.
\end{abstract}
\begin{keywords}
Automatic Essay Evaluation, Large Language Model, Zero shot Learning, LLM as Evaluator, Linguistic Features
\end{keywords}

% \section{Introduction}

Deep Reinforcement Learning (DRL) has emerged as a transformative paradigm for solving complex sequential decision-making problems. By enabling autonomous agents to interact with an environment, receive feedback in the form of rewards, and iteratively refine their policies, DRL has demonstrated remarkable success across a diverse range of domains including games (\eg Atari~\citep{mnih2013playing,kaiser2020model}, Go~\citep{silver2018general,silver2017mastering}, and StarCraft II~\citep{vinyals2019grandmaster,vinyals2017starcraft}), robotics~\citep{kalashnikov2018scalable}, communication networks~\citep{feriani2021single}, and finance~\citep{liu2024dynamic}. These successes underscore DRL's capability to surpass traditional rule-based systems, particularly in high-dimensional and dynamically evolving environments.

Despite these advances, a fundamental challenge remains: DRL agents typically rely on deep neural networks, which operate as black-box models, obscuring the rationale behind their decision-making processes. This opacity poses significant barriers to adoption in safety-critical and high-stakes applications, where interpretability is crucial for trust, compliance, and debugging. The lack of transparency in DRL can lead to unreliable decision-making, rendering it unsuitable for domains where explainability is a prerequisite, such as healthcare, autonomous driving, and financial risk assessment.

To address these concerns, the field of Explainable Deep Reinforcement Learning (XRL) has emerged, aiming to develop techniques that enhance the interpretability of DRL policies. XRL seeks to provide insights into an agent’s decision-making process, enabling researchers, practitioners, and end-users to understand, validate, and refine learned policies. By facilitating greater transparency, XRL contributes to the development of safer, more robust, and ethically aligned AI systems.

Furthermore, the increasing integration of Reinforcement Learning (RL) with Large Language Models (LLMs) has placed RL at the forefront of natural language processing (NLP) advancements. Methods such as Reinforcement Learning from Human Feedback (RLHF)~\citep{bai2022training,ouyang2022training} have become essential for aligning LLM outputs with human preferences and ethical guidelines. By treating language generation as a sequential decision-making process, RL-based fine-tuning enables LLMs to optimize for attributes such as factual accuracy, coherence, and user satisfaction, surpassing conventional supervised learning techniques. However, the application of RL in LLM alignment further amplifies the explainability challenge, as the complex interactions between RL updates and neural representations remain poorly understood.

This survey provides a systematic review of explainability methods in DRL, with a particular focus on their integration with LLMs and human-in-the-loop systems. We first introduce fundamental RL concepts and highlight key advances in DRL. We then categorize and analyze existing explanation techniques, encompassing feature-level, state-level, dataset-level, and model-level approaches. Additionally, we discuss methods for evaluating XRL techniques, considering both qualitative and quantitative assessment criteria. Finally, we explore real-world applications of XRL, including policy refinement, adversarial attack mitigation, and emerging challenges in ensuring interpretability in modern AI systems. Through this survey, we aim to provide a comprehensive perspective on the current state of XRL and outline future research directions to advance the development of interpretable and trustworthy DRL models.
%!TEX root = gcn.tex
\section{Introduction}
Graphs, representing structural data and topology, are widely used across various domains, such as social networks and merchandising transactions.
Graph convolutional networks (GCN)~\cite{iclr/KipfW17} have significantly enhanced model training on these interconnected nodes.
However, these graphs often contain sensitive information that should not be leaked to untrusted parties.
For example, companies may analyze sensitive demographic and behavioral data about users for applications ranging from targeted advertising to personalized medicine.
Given the data-centric nature and analytical power of GCN training, addressing these privacy concerns is imperative.

Secure multi-party computation (MPC)~\cite{crypto/ChaumDG87,crypto/ChenC06,eurocrypt/CiampiRSW22} is a critical tool for privacy-preserving machine learning, enabling mutually distrustful parties to collaboratively train models with privacy protection over inputs and (intermediate) computations.
While research advances (\eg,~\cite{ccs/RatheeRKCGRS20,uss/NgC21,sp21/TanKTW,uss/WatsonWP22,icml/Keller022,ccs/ABY318,folkerts2023redsec}) support secure training on convolutional neural networks (CNNs) efficiently, private GCN training with MPC over graphs remains challenging.

Graph convolutional layers in GCNs involve multiplications with a (normalized) adjacency matrix containing $\numedge$ non-zero values in a $\numnode \times \numnode$ matrix for a graph with $\numnode$ nodes and $\numedge$ edges.
The graphs are typically sparse but large.
One could use the standard Beaver-triple-based protocol to securely perform these sparse matrix multiplications by treating graph convolution as ordinary dense matrix multiplication.
However, this approach incurs $O(\numnode^2)$ communication and memory costs due to computations on irrelevant nodes.
%
Integrating existing cryptographic advances, the initial effort of SecGNN~\cite{tsc/WangZJ23,nips/RanXLWQW23} requires heavy communication or computational overhead.
Recently, CoGNN~\cite{ccs/ZouLSLXX24} optimizes the overhead in terms of  horizontal data partitioning, proposing a semi-honest secure framework.
Research for secure GCN over vertical data  remains nascent.

Current MPC studies, for GCN or not, have primarily targeted settings where participants own different data samples, \ie, horizontally partitioned data~\cite{ccs/ZouLSLXX24}.
MPC specialized for scenarios where parties hold different types of features~\cite{tkde/LiuKZPHYOZY24,icml/CastigliaZ0KBP23,nips/Wang0ZLWL23} is rare.
This paper studies $2$-party secure GCN training for these vertical partition cases, where one party holds private graph topology (\eg, edges) while the other owns private node features.
For instance, LinkedIn holds private social relationships between users, while banks own users' private bank statements.
Such real-world graph structures underpin the relevance of our focus.
To our knowledge, no prior work tackles secure GCN training in this context, which is crucial for cross-silo collaboration.


To realize secure GCN over vertically split data, we tailor MPC protocols for sparse graph convolution, which fundamentally involves sparse (adjacency) matrix multiplication.
Recent studies have begun exploring MPC protocols for sparse matrix multiplication (SMM).
ROOM~\cite{ccs/SchoppmannG0P19}, a seminal work on SMM, requires foreknowledge of sparsity types: whether the input matrices are row-sparse or column-sparse.
Unfortunately, GCN typically trains on graphs with arbitrary sparsity, where nodes have varying degrees and no specific sparsity constraints.
Moreover, the adjacency matrix in GCN often contains a self-loop operation represented by adding the identity matrix, which is neither row- nor column-sparse.
Araki~\etal~\cite{ccs/Araki0OPRT21} avoid this limitation in their scalable, secure graph analysis work, yet it does not cover vertical partition.

% and related primitives
To bridge this gap, we propose a secure sparse matrix multiplication protocol, \osmm, achieving \emph{accurate, efficient, and secure GCN training over vertical data} for the first time.

\subsection{New Techniques for Sparse Matrices}
The cost of evaluating a GCN layer is dominated by SMM in the form of $\adjmat\feamat$, where $\adjmat$ is a sparse adjacency matrix of a (directed) graph $\graph$ and $\feamat$ is a dense matrix of node features.
For unrelated nodes, which often constitute a substantial portion, the element-wise products $0\cdot x$ are always zero.
Our efficient MPC design 
avoids unnecessary secure computation over unrelated nodes by focusing on computing non-zero results while concealing the sparse topology.
We achieve this~by:
1) decomposing the sparse matrix $\adjmat$ into a product of matrices (\S\ref{sec::sgc}), including permutation and binary diagonal matrices, that can \emph{faithfully} represent the original graph topology;
2) devising specialized protocols (\S\ref{sec::smm_protocol}) for efficiently multiplying the structured matrices while hiding sparsity topology.


 
\subsubsection{Sparse Matrix Decomposition}
We decompose adjacency matrix $\adjmat$ of $\graph$ into two bipartite graphs: one represented by sparse matrix $\adjout$, linking the out-degree nodes to edges, the other 
by sparse matrix $\adjin$,
linking edges to in-degree nodes.

%\ie, we decompose $\adjmat$ into $\adjout \adjin$, where $\adjout$ and $\adjin$ are sparse matrices representing these connections.
%linking out-degree nodes to edges and edges to in-degree nodes of $\graph$, respectively.

We then permute the columns of $\adjout$ and the rows of $\adjin$ so that the permuted matrices $\adjout'$ and $\adjin'$ have non-zero positions with \emph{monotonically non-decreasing} row and column indices.
A permutation $\sigma$ is used to preserve the edge topology, leading to an initial decomposition of $\adjmat = \adjout'\sigma \adjin'$.
This is further refined into a sequence of \emph{linear transformations}, 
which can be efficiently computed by our MPC protocols for 
\emph{oblivious permutation}
%($\Pi_{\ssp}$) 
and \emph{oblivious selection-multiplication}.
% ($\Pi_\SM$)
\iffalse
Our approach leverages bipartite graph representation and the monotonicity of non-zero positions to decompose a general sparse matrix into linear transformations, enhancing the efficiency of our MPC protocols.
\fi
Our decomposition approach is not limited to GCNs but also general~SMM 
by 
%simply 
treating them 
as adjacency matrices.
%of a graph.
%Since any sparse matrix can be viewed 

%allowing the same technique to be applied.

 
\subsubsection{New Protocols for Linear Transformations}
\emph{Oblivious permutation} (OP) is a two-party protocol taking a private permutation $\sigma$ and a private vector $\xvec$ from the two parties, respectively, and generating a secret share $\l\sigma \xvec\r$ between them.
Our OP protocol employs correlated randomnesses generated in an input-independent offline phase to mask $\sigma$ and $\xvec$ for secure computations on intermediate results, requiring only $1$ round in the online phase (\cf, $\ge 2$ in previous works~\cite{ccs/AsharovHIKNPTT22, ccs/Araki0OPRT21}).

Another crucial two-party protocol in our work is \emph{oblivious selection-multiplication} (OSM).
It takes a private bit~$s$ from a party and secret share $\l x\r$ of an arithmetic number~$x$ owned by the two parties as input and generates secret share $\l sx\r$.
%between them.
%Like our OP protocol, o
Our $1$-round OSM protocol also uses pre-computed randomnesses to mask $s$ and $x$.
%for secure computations.
Compared to the Beaver-triple-based~\cite{crypto/Beaver91a} and oblivious-transfer (OT)-based approaches~\cite{pkc/Tzeng02}, our protocol saves ${\sim}50\%$ of online communication while having the same offline communication and round complexities.

By decomposing the sparse matrix into linear transformations and applying our specialized protocols, our \osmm protocol
%($\prosmm$) 
reduces the complexity of evaluating $\numnode \times \numnode$ sparse matrices with $\numedge$ non-zero values from $O(\numnode^2)$ to $O(\numedge)$.

%(\S\ref{sec::secgcn})
\subsection{\cgnn: Secure GCN made Efficient}
Supported by our new sparsity techniques, we build \cgnn, 
a two-party computation (2PC) framework for GCN inference and training over vertical
%ly split
data.
Our contributions include:

1) We are the first to explore sparsity over vertically split, secret-shared data in MPC, enabling decompositions of sparse matrices with arbitrary sparsity and isolating computations that can be performed in plaintext without sacrificing privacy.

2) We propose two efficient $2$PC primitives for OP and OSM, both optimally single-round.
Combined with our sparse matrix decomposition approach, our \osmm protocol ($\prosmm$) achieves constant-round communication costs of $O(\numedge)$, reducing memory requirements and avoiding out-of-memory errors for large matrices.
In practice, it saves $99\%+$ communication
%(Table~\ref{table:comm_smm}) 
and reduces ${\sim}72\%$ memory usage over large $(5000\times5000)$ matrices compared with using Beaver triples.
%(Table~\ref{table:mem_smm_sparse}) ${\sim}16\%$-

3) We build an end-to-end secure GCN framework for inference and training over vertically split data, maintaining accuracy on par with plaintext computations.
We will open-source our evaluation code for research and deployment.

To evaluate the performance of $\cgnn$, we conducted extensive experiments over three standard graph datasets (Cora~\cite{aim/SenNBGGE08}, Citeseer~\cite{dl/GilesBL98}, and Pubmed~\cite{ijcnlp/DernoncourtL17}),
reporting communication, memory usage, accuracy, and running time under varying network conditions, along with an ablation study with or without \osmm.
Below, we highlight our key achievements.

\textit{Communication (\S\ref{sec::comm_compare_gcn}).}
$\cgnn$ saves communication by $50$-$80\%$.
(\cf,~CoGNN~\cite{ccs/KotiKPG24}, OblivGNN~\cite{uss/XuL0AYY24}).

\textit{Memory usage (\S\ref{sec::smmmemory}).}
\cgnn alleviates out-of-memory problems of using %the standard 
Beaver-triples~\cite{crypto/Beaver91a} for large datasets.

\textit{Accuracy (\S\ref{sec::acc_compare_gcn}).}
$\cgnn$ achieves inference and training accuracy comparable to plaintext counterparts.
%training accuracy $\{76\%$, $65.1\%$, $75.2\%\}$ comparable to $\{75.7\%$, $65.4\%$, $74.5\%\}$ in plaintext.

{\textit{Computational efficiency (\S\ref{sec::time_net}).}} 
%If the network is worse in bandwidth and better in latency, $\cgnn$ shows more benefits.
$\cgnn$ is faster by $6$-$45\%$ in inference and $28$-$95\%$ in training across various networks and excels in narrow-bandwidth and low-latency~ones.

{\textit{Impact of \osmm (\S\ref{sec:ablation}).}}
Our \osmm protocol shows a $10$-$42\times$ speed-up for $5000\times 5000$ matrices and saves $10$-2$1\%$ memory for ``small'' datasets and up to $90\%$+ for larger ones.


\section{Related Works}
\label{related-works}
\subsection{Automatic Essay Scoring}
\label{automatic-essay-scoring}
% Recent studies in AES can be roughly divided into the following two perspectives.

\paragraph{Feature Engineering} approaches leverage various features to predict essay scores, including linguistic features, e.g., readability metrics and word length ~\cite{ridley2020promptagnosticessayscorer, uto-etal-2020-neural, jin-etal-2018-tdnn, FoltLahaLand19993j, chen-he-2013-automated}, and content features, e.g., content quality and organization ~\cite{mathias-bhattacharyya-2018-asap, crossley2023english}. Models that utilize these features range from simple logistic regression models ~\cite{chen-he-2013-automated} to deep neural networks ~\cite{uto-etal-2020-neural}. These approaches assess the quality of essays in an interpretable manner with well-defined features.

\paragraph{Language-model-based} approaches emerge with the rising popularity of Transformer architecture, including BERT-based methods that require supervised fine-tuning \cite{wang-etal-2022-use, mutlitaskAESforEssayGrading, hierarchicalbert} and LLM-based methods that focus on prompt-engineering \cite{mansour-etal-2024-large, stahl-etal-2024-exploring}. In particular, \cite{stahl-etal-2024-exploring} explores zero-shot prompting with persona prompts and analysis instructions. Building on this, our work aims to utilize linguistic features in LLM prompting.

\subsection{LLM as Evaluator}
\label{llm-as-evaluator}
Given the increasing capability of LLMs and their scalable nature, researchers in various domains have explored how to use them for the automatic evaluation of text content \cite{zubiaga-etal-2024-llm, alhafni-etal-2024-personalized, gao2024llmbasednlgevaluationcurrent, fu-etal-2024-gptscore}. Although some research has shown proper prompt tuning, such as explanation-guided generation, clear rubric guidance, and chain-of-thought (COT) could improve the alignment between human and LLMs \cite{chiang-lee-2023-closer, liu-etal-2023-g, hashemi-etal-2024-llm}, the LLM-based evaluators still perform underwhelming in more complex tasks, such as reviewing papers \cite{zhou-etal-2024-llm} and scoring students essay \cite{mansour-etal-2024-large, stahl-etal-2024-exploring}. In this work, we specifically focus on improving LLMs as student essay graders by incorporating the linguistic features of essays. Additionally, we examine the transferability of the prompts, i.e., how a prompt that is tuned in the in-distribution data would perform out-of-distribution in the same task.

\section{The Flow-Based Combinatorial Auction Menu Network}

As discussed in the introduction, the major challenge in learning menus for CAs is 
to provide an expressive representation of distributions over bundles
to associate with each menu element while retaining efficiency, so that
the exponential number of possible bundles does not become a bottleneck.
%dcp cut Given that the number of bundles grows exponentially in the number of items, this requirement greatly increases the complexity of the menu representation. 
Moreover, training these representations adds another layer of difficulty: the menu must be not only concise but also easily differentiable to support training. 

\subsection{Menu representation}

Our key idea, following from score-based diffusion models and continuous normalizing flow, is to construct a concise and differentiable representation of a bundle distribution by modeling it through the solution of an ordinary differential equation (ODE). Specifically, the $k$th menu element generates its bundle distribution by the ODE,
%
\begin{equation}
    d\vs^{(k)}_t = \varphi^{(k)}(t,\vs^{(k)}_t) dt,\label{equ:de}
\end{equation}
%
for a suitable choice of vector field $\varphi^{(k)}$.
%
Here, we refer to $\vs^{(k)}_t\in \mathbb{R}^m$ as the \emph{bundle variable at time $t$}, where $m$ is the number of items. At time $T$, we require that a bundle variable $\vs^{(k)}_T$ represents a meaningful bundle, so that all entries are 0s or 1s,
and we adopt $\alpha^{(k)}_T(\vs^{(k)}_T)$ to denote the corresponding allocation probability. 
For simplicity, we omit the superscript $(k)$ when this is clear from the context.

By the  Liouville equation~\cite{liouville1838note}, the probability density at $T$ derived from Eq.~\ref{equ:de} satisfies:
\begin{align}
    \log \alpha_T(\vs_T) = \log \alpha_0(\vs_0) - \int_0^T \nabla\cdot \varphi(t, \vs_t) dt,\label{equ:liouville}
\end{align}
where $\alpha_0(\vs_0)$ denotes the initial distribution at time $0$, on initial bundle variables $\vs_0$, and $\nabla\cdot \varphi(t, \vs_t)$ is the divergence of $\varphi$.   Eq.~\ref{equ:liouville} is applicable to any $\vs_0$, and a bundle variable $\vs_t$ is generated from $\vs_0$ by $\vs_t(\vs_0)=\vs_0+\int_0^t \varphi(\tau,\vs_\tau) d\tau$. For clarity, we omit the explicit dependence on $\vs_0$ and simply write $\vs_t$.
%\dcp{as per my comment in the intro, it may be worth being a bit more pedantic here, and explaining that this can be applied to any $s_0$, and that $s_T$ is the rv at time $T$ that corresponds to this particular $s_0$.} 

\textbf{Training scheme}. Both the vector field $\varphi$ and the initial distribution $\alpha_0$ can influence the final distribution $\alpha_T$. 
Our method proceeds in two stages, involving the training of  each of
these two components in turn: 

$\quad$ \emph{(1) Flow Initialization.} We  fix  the initial distribution $\alpha_0$ and train the vector field $\varphi(t, \cdot)$ so that the final distribution, $\alpha_T$,
provides a reasonable coverage over bundles. 

$\quad$ \emph{(2) Menu Optimization.} We  fix the vector field from Stage 1, and backpropagate the revenue-maximizing loss through the flow to update the initial distribution $\alpha_0^{(k)}$ 
for each menu element $k$.

% by learning the weights $w_d^{(k)}$ and means $\bm\mu_d^{(k)}$ associated with the menu element. 

$\varphi$ and $\alpha_0(\vs_0)$ play a crucial role in maintaining a concise and easily differentiable representation and ensuring efficient training. We next propose specific functional forms for these two components that meet these criteria.

\textbf{Vector field}. We adopt the following functional form for the vector field,
%
\begin{align}
    \varphi(t, \vs_t;\xi,\theta) = \eta(t;\xi) Q(\vs_0;\theta) \vs_t,\label{equ:varphi}
\end{align}
where $Q: \mathbb{R}^{m}\rightarrow\mathbb{R}^{m\times m}$, written as a function of $\vs_0$, and the scalar factor $\eta:\mathbb{R}\rightarrow \mathbb{R}$, written as a function of  the ODE time $t\in[0,T]$, are neural networks with learnable parameters $\theta$ and $\xi$, respectively. 
%
%
We  omit dependence on $\theta$ and $\xi$ when the context is clear. This formulation's advantage becomes apparent when we consider its divergence:
%
\begin{align}
    \nabla\cdot \varphi(t,\vs_t) &= \sum_{i=1}^m \frac{\partial \varphi_i}{\partial s_{t,i}}
    = \sum_{i=1}^m \frac{\partial}{\partial s_{t,i}} \eta(t) Q_i(\vs_0) \vs_t = \sum_{i=1}^m  \eta(t) Q_{ii}(\vs_0) \\
    & = \eta(t)\Tr[Q(\vs_0)].
\end{align}

Here, $\varphi_i$ and $s_{t,i}$ are the $i$th element of $\varphi$ and $\vs_t$, respectively, $Q_i$ is the $i$th row of $Q$, and $Q_{ii}$ is the $i$th diagonal element of $Q$. Thus, the probability density at $T$ becomes
\begin{align}
    \log \alpha_T(\vs_T) = \log \alpha_0(\vs_0) - \Tr[Q(\vs_0)]\int_0^T \eta(t) dt.\label{equ:likelihood}
\end{align}

The integral in Eq.~\ref{equ:likelihood} 
is tractable as it only involves a scalar function, instead of bundle variables.
We can efficiently estimate this integral by time discretization. 

\textbf{Initial distribution}. In Stage 1, we use a mixture-of-Gaussian distribution for
the initial distribution $\alpha_0(\vs_0)$ on bundle variables $\vs_0$, with
%
\begin{align}
    \vs_0 \sim\sum_{d=1}^D w_d \mathcal{N}(\bm\mu_d, \sigma_d^2\mI_m),\label{equ:init_dist}
\end{align}
where, for $D$ components,
$\bm\mu_d\in\mathbb{R}^m$, $\sigma_d\in\mathbb{R}_{>0}$, $\mI_m$ is the $m\times m$ identity matrix, and $w_d\geq 0$ are weights satisfying $\sum_{d=1}^D w_d=1$. In Stage 2, as discussed later, we ensure DSIC by adopting a mixture-of-Dirac distribution, which is practically implemented by setting a very small variance $\sigma_d$ in a mixture-of-Gaussian distribution.

% (1,1,\cdots,1)^\Tau We expect the flow to transport a sample $z_0$ to a bundle $S\in\{0,1\}^m$.

\subsection{Stage 1: Flow initialization}

The aim of the first stage is to guarantee that the flow can transport any initial bundle variable $\vs_0$ to a feasible bundle $S\in 2^M$.  We use $\vs=(\mathbb{I}{\{i\in S\}})$ to denote the vectorization of set $S$, i.e., the $i$-th component of $\vs$ is 1 if item $i$ is in $S$ and 0 otherwise.

In practice, numerical issues make it challenging to exactly obtain an feasible bundle $\vs$; i.e., a bundle variable
with only 0s and 1s. To account for this, we allow a small region around $\vs$ to be approximated as $\vs$ by modeling the bundle as a Gaussian variable,
%
\begin{align}
    S_{\sigma_z} = \mathcal{N}(\vs, \sigma_z^2\mI_m).
\end{align}
% the random variable representing the bundle variables at $t=0$.  where $Z_0$ is 

We train the vector field networks using rectified flow (\citet{liu2022flow}, Eq.~\ref{equ:rf}). For this  stage, we fix the initial distribution $\alpha_0(\vs_0)$ to a mixture-of-Gaussian model $\alpha_0(\vs_0)=\sum_{d=1}^D w_d \mathcal{N}(\bm\mu_d, \sigma_d^2\mI_m)$ with $D$ components.
We  define
$\alpha_T(\vs_T)$ as a uniform mixture-of-Gaussian model, with  components centered around each feasible bundle, and
$\alpha_T(\vs_T)=\frac{1}{2^m}\sum_{S\in 2^M} \mathcal{N}(\vs, \sigma_z^2\mI_m)=\frac{1}{2^m}\sum_{S\in 2^M}S_{\sigma_z}$.
This target distribution only applies in Stage 1, where it serves to encourage a balanced coverage of the final distribution over feasible bundles. In Stage 2, we have an optimization problem, and there is no longer a fixed target distribution.
%

We  follow the idea of  rectified flow, and define the {\em flow training loss} as
%
\begin{align}
    \mathcal{L}_{\textsc{Flow}}(\theta,\xi) =& \mathbb{E}_{(\vs_0,\vs_T)\sim (\alpha_0,\alpha_T), t\sim [0,T]} \left[\|(\vs_T-\vs_0)-\varphi(t, \vs_t; \theta,\xi)\|^2\right], \label{equ:flow_loss}\ \ \mbox{where}\\
    & \vs_t = t\cdot \vs_T + (1-t)\cdot \vs_0,\\
    & \varphi(t, \vs_t; \theta,\xi)=\eta(t;\xi)Q(\vs_0;\theta)\vs_t.
\end{align}

% \dcp{i think you want $\vs_T$ not $\vs_1$ in eqn 16}
This loss is used to update the neural networks $Q$ and $\eta$ to encourage the vector field at interpolated points $\vs_t$ to point from $\vs_0$ to $\vs_T$.
%\dcp{I think you can say a bit more here---you've defined $Z_1$ to be a uniform distr over all bundles, so this is trying to get coverage of bundles.}
%
The expectation in the flow training loss is taken over $(\alpha_0,\alpha_T)$, 
but directly sampling from $\alpha_T$ is intractable as it involves $2^m$ bundles.

Crucially, using a flow-based representation provides a workaround. We first draw $\vs_0\sim \alpha_0$, which is straightforward given that $\alpha_0$ comprises a manageable number of components ($D$). We then round $\vs_0$ to
the nearest feasible bundle, $\vs=\mathbb{I}(\vs_0\ge 0.5)\in\{0,1\}^m$,
and sample $\vs_T\sim \mathcal{N}(\vs, \sigma_z^2\mI_m)$. This approach underscores an advantage of deep learning. Although we cannot enumerate all possible bundles, the generalization ability of neural networks allows for learning the mapping from $\alpha_0$ to $\alpha_T$ given enough training samples.


\subsection{Stage 2: Menu optimization}\label{sec:method:opt}

In the second stage, we train the menu to seek to maximize the expected revenue for the auctioneer. For each menu element $k$,  the 
trainable parameters comprise the price $\beta^{(k)}$, as well as the parameters $w_d^{(k)}$ and $\bm\mu_d^{(k)}$ that define the initial distribution $\alpha^{(k)}_0$ on the bundle variable. 
The vector field $\varphi$ is  fixed in this stage and shared
among all menu elements.

Given a bidder with a value function $v$, the payment to the auctioneer is the price associated with the menu element that provides the highest utility to the bidder. 
Thus, computing the utility of each menu element is central to evaluating the revenue objective.
We always maintain a null menu element (zero allocation, zero price), which ensures 
  individual rationality (IR), so that the bidder has  non-negative expected
utility. 

Computing the expected 
utility corresponding to a menu element with bundle distribution $\alpha^{(k)}$ 
is intractable when done with a direct calculation,
because
%
\begin{equation}
    u^{(k)}(v) = \sum_{S\in 2^M} \alpha^{(k)}(S)v(S)
\end{equation}
%
requires enumerating $2^m$ bundles for a general valuation function.
However, with our flow-based representation, we can get the bundle allocation probabilities by applying the flow to the initial distribution. Specifically, we have
\begin{align}
    u^{(k)}(v) = \mathbb{E}_{\vs_0\sim \alpha^{(k)}_0, \vs=\mathbb{I}(\phi(T,\vs_0)\ge 0.5)} \left[v(\vs) \alpha^{(k)}_0(\vs_0)\exp\left(-\Tr[Q(\vs_0)]\int_0^T \eta(t) dt]\right)\right],\label{equ:u}
\end{align}
by applying the exponential operation to both sides of Eq.~\ref{equ:likelihood}. Here, $\phi(T,\vs_0)$ is the solution of the ODE solved by forward Euler,
%
\begin{align}
    \phi(T,\vs_0) = \vs_0 + Q(\vs_0)\int_0^T \eta(t)\vs_t dt,\label{equ:phi_T_s_0}
\end{align}
%
and $\vs=\mathbb{I}(\phi(T,\vs_0)\ge 0.5)$ is the rounded final bundle. Due to its simple form, a modern ODE solver can efficiently solve the ODE (Eq.~\ref{equ:phi_T_s_0}) in just a few steps. Therefore, the calculation of $u^{(k)}(v)$ becomes tractable when we make the initial distribution simple.

% Here we see another advantage of our choice for the functional form of the vector field $\phi$ (Eq.~\ref{equ:varphi}): $u^{(k)}(v)$ can be determined by the initial distribution, without calculating bundle variables at $t>0$.  Moreover, 
% \tw{TODO: double check the reference.}  \dcp{XX STILL TO DO! XX}
% designed for diffusion models 
%

To ensure DSIC, we need to accurately calculate the expectation in Eq.~\ref{equ:u} to get the exact
utility to the bidder. We accomplish this by employing a mixture-of-Dirac distribution as the initial distribution, which has finite support. To implement this in practice, we set, for Stage 2 only, a very small variance to the Gaussian components in Eq.~\ref{equ:init_dist}, with $\sigma_d=1e\shortn 20$ for every component $d$. In this way, the utility can be obtained by enumerating over the finite support of the initial distribution:
% In this way, each Gaussian in the mixture is effectively a Dirac delta, 
\begin{align}
    u^{(k)}(v) = \sum_{d=1}^D \left[v(\vs(\bm\mu^{(k)}_d)) \alpha^{(k)}_0(\bm\mu^{(k)}_d)\exp\left(-\Tr[Q(\bm\mu^{(k)}_d)]\int_0^T \eta(t) dt]\right)\right],\label{equ:u_finite}
\end{align}
where $\vs(\bm\mu^{(k)}_d)=\mathbb{I}(\phi(T,\bm\mu^{(k)}_d)\ge 0.5)$. 
That is, the support of $\alpha^{(k)}_0$ 
consists, in effect, of the set of means, one for each component. %\dcp{this is where I think we could just say this is a mixture-of-Dirac and avoid this small variance mix-of-Gaussian discussion}. \tw{In Sec. 3.1, before we introduce Stage 1 and 2, we said the initial distribution is mixture-of-Gaussian. I suggest a minimal modification (by setting different variance in the two stages) here, but am willing to rework both Sec. 3.1 and 3.3.} 
It is worth noting that $D$ in Eq.~\ref{equ:u_finite} does not need to be the same $D$ as in Stage 1, and it could even vary across menu elements.
% \dcp{also, is it worth commenting that this $D$ does not need to be the same $D$ as in Stage 1? I suppose in principle it could even vary across elements...}
%\dcp{a bit confusing, at least to me --- I don't see where you sample different weighted combinations of the means?} \tw{$\alpha_0(\bm\mu^{(k)}_d)$ in Eq.~\label{equ:u_finite}?}

In this Stage 2, we fix the vector field $\varphi$ ($Q$ and $\eta$ networks) in Eq.~\ref{equ:u_finite} and update trainable parameters associated with the price and 
initial distribution $\alpha_0^{(k)}$ for each menu element $k$ 
during menu optimization, \ie, $\beta^{(k)}$, $\{w^{(k)}_d\}_{d=1}^{D}$, and $\{\bm\mu^{(k)}_d\}_{d=1}^{D}$. Therefore, given a set of bidder valuations $\mathcal{V}$,
the {\em revenue-maximization loss} is defined as
%
\begin{align}
    \mathcal{L}_{\textsc{Rev}}\left(\{\beta^{(k)}\}_{k=1}^K, \bigl\{ w_d^{(k)} \bigr\}_{\substack{d\in[D] \\ k\in[K]}},\bigl\{\bm\mu_d^{(k)} \bigr\}_{\substack{d\in[D] \\ k\in[K]}}\right) = -\frac{1}{|\mathcal{V}|}\sum_{v\in \mathcal{V}}\left[\sum_{k\in[K]}z^{(k)}(v)\beta^{(k)} \right],
\end{align}
%
where $z^{(k)}(v)$ is obtained by applying the differentiable SoftMax function to the utility of the bidder being allocated the $k$-th menu choice, i.e.,
%
\begin{align}
    z^{(k)}(v) = \mathsf{SoftMax}_k\left(\lambda_{\textsc{SoftMax}}\cdot u^{(1)}(v),\ldots,\lambda_{\textsc{SoftMax}}\cdot u^{(K)}(v)\right),\label{equ:softmax_in_loss}
\end{align}
%
where $\lambda_{\textsc{SoftMax}}$ is a scaling factor, and $u^{(k)}(v)$ is calculated by Eq.~\ref{equ:u_finite}.
%\dcp{do you want $\mathsf{SoftMax}_K$ not $\mathsf{SoftMax}_k$?} \tw{I used the notation of the JACM paper. $k$ acts like an index here?}
When optimizing $\mathcal{L}_{\textsc{Rev}}$, the gradients with respect to $\beta^{(k)}$ are straightforward to compute. Moreover, although $Q$ remains fixed, gradients can still backpropagate through this network to update its input,
which is $\bm\mu^{(k)}_d$. Gradients also flow through $z^{(k)}$ back into $\alpha_0$, enabling updates to the mixture weights $w^{(k)}_d$. All these gradients are automatically handled by standard deep learning frameworks.



% The probability of
% \begin{align}
%     \bm\alpha^{(k)}_{ij} = \sum_{z_1=f(z_0)=S_j} D_0(z_0)
% \end{align}

\subsection{Discussion}\label{sec:multi-bidder}

%\dcp{mention somewhere, perhaps here, that softmax becomes hardmax at test time to give DSIC}

\textbf{DSIC}. The seminal work by \citet{hammond1979straightforward} establishes necessary and sufficient conditions for a strategyproof menu-based auction: (1) Self-bid independent: the menu is independent of the bidder's bid; (2) Agent-optimizing: the bidder is assigned the menu element that maximizes their utility. As we analyze here, our method satisfies these two properties.

In \name, all element prices, as well as bundle allocations, which depend on initial distributions and the vector field, are trained on values sampled from the distribution $F$, without using any information about the bidder's specific valuation. Therefore, menus learned by \name~are self-bid independent. 
%
As discussed in Sec.~\ref{sec:method:opt}, we require the initial distribution for each menu
element to have finite support, which means that the bundle distribution for each menu element 
can be reconstructed without any approximation error.
This guarantees exact utility calculation for every menu element. Moreover, unlike the SoftMax in Eq.~\ref{equ:softmax_in_loss}, we use hard argmax at test time, thereby selecting the menu element with the highest utility to the bidder. In this way, \name\ is strictly agent-optimizing.

\textbf{Expressiveness}. In Stage 1, we initialize the vector field $\varphi$. After this stage, given appropriate initial distributions, the final distribution can in principle cover all $2^m$ bundles and is trained to seek to achieve this.
In Stage 2, since the initial distribution for a menu element has finite support of size $D$, the bundle distribution for a menu element is also limited to finite support of size $D$. 
What is crucial, though, is that we can learn which (up to) $D$ bundles are represented in the distribution
that corresponds to a menu element. In practice, we find that a bounded  $D$
that is much smaller than $2^m$ still gives very high expected revenue.


%\dcp{mention somewhere that whereas the realized distributions per menu element are limited to $D$ bundles, with $D$ bounded or at least small compared to $2^m$, the final distribution achieved
%from the vector field in Stage 1 can, in principle, have support on all $2^m$ bundles.}

\textbf{Extension to multi-bidder settings}. By providing an expressive and concise %\dcp{this also said `concise and flexible` but I think concise is the same as efficient and flexible the same as expressive}
representation of single-bidder menus for the CA setting, our method opens up the possibilities of developing a general DSIC multi-bidder CA mechanism. A principled approach is to 
adapt the idea of GemNet~\cite{wang2024gemnet}. 
%
First, we can learn a separate \name~menu for each bidder. The modification in the network architecture is that these menus should now also depend on other bidders' bids $\vb_{\shortn i}$. To achieve this, we can condition the vector field, specifically the $Q$ and $\sigma$ networks, on $\vb_{\shortn i}$ by concatenating them to the inputs. For the price of each menu element, we can model them as the output of a neural network whose input is $\vb_{\shortn i}$. During training, we can also introduce a compatibility loss in the same way as that used in GemNet. This loss penalizes any over-allocation of items in the selected agent-optimizing elements from individual menus.

The major challenge in adapting GemNet to the CA setting  
arises during the post-training stage of GemNet, which adjusts prices of menu elements so that there is provably never any over-allocation of items. For this, GemNet constructs a grid over the space of bidder values. On each grid point, GemNet formulates a mixed-integer linear program (MILP) to adjust prices to ensure that, the utility of the best  element that is compatible with the choices of others in the sense of not over-allocating items is larger than that of all other elements by a safety margin. These safety margins prevent an incompatible menu element from being selected in the regions between grid points. Although the concise \name~menu representation, in principle, enables this MILP to 
be directly adapted to the combinatorial setting and used to adjust \name~menus to obtain a DSIC CA, the main issue is that the space of bidder values exhibits exponential dimensionality in the CA setting, resulting in an excessively large grid. Reducing this complexity represents the crucial 
remaining step in future work to enable 
a general, DSIC, and multi-bidder CA mechanism.

\section{Experiment}
We evaluate our proposed method with strong baselines and further analyze contributions of different components, and the impact of key parameters.

\subsection{Experiment Setup}
\textbf{Dataset.}
We evaluate all the methods on Inter-X dataset, which consists about 9K training samples and 1,708 test samples. Each sample is an action-reaction sequence and three corresponding textual description.
As supplementation, we mix our pre-training data with single person motion-text dataset HumanML3D~\citep{humanml3d}, which consists more than 23K annotated motion sequences.
We uniformly sample frames for both datasets to 30 FPS. 

\textbf{Evaluation Metrics.}
Following single-person motion generation~\citep{t2mgpt}, we adopt the these metrics to quantitatively evaluate the generated motion: R-Precision measures the ranking of Euclidean distances between motion and text features. Accuracy (Acc.) assesses how likely a generated motion could be successfully recognized as its interaction label, like ``high-five''. Frechet Inception Distance~\citep{fid} (FID) evaluates the similarity in feature space between predicted and ground-truth motion. Multimodal Distance (MMDist.) calculates the average Euclidean distance between generated motion and the corresponding text description. Diversity (Div.) measures the feature diversity within generated motions. All the metrics reported are calculated with batch size set to 32, and accumulated across the test dataset, and we evaluate each method for 20 times with different seeds to calculate the final results at 95\% confidence interval.

\textbf{Evaluation Model.} \label{sec:eval}
Every metric mentioned above requires an encoder $\mathcal{M}$ to extract motion feature.
For single person text-to-motion generation tasks, a motion-text matching model are commonly trained as human motion feature extractor.
A simple way to transfer this method to interaction domain is to directly train an interaction-to-text matching model $\mathcal{M}(\mathbf{a}, \hat{\mathbf{b}}, text)$, where action sequence $\mathbf{a}$ and predicted reaction sequence $\hat{\mathbf{b}}$ together is regarded as a generated interaction sequence, or a reaction-to-text match model $\mathcal{M}(\hat{\mathbf{b}}, text)$.
However, the former one may focus too much on the ground-truth action input, leading insufficient discriminative power of $\hat{\mathbf{b}}$'s quality, while the latter one lacks semantics provided by action, thus leading to subpar matching capability.

To address the issue, we simply uniformly mask off a large portion of $\mathbf{a}$, obtaining down-sampled action motion sequence $\mathbf{a}'$ (downsampled to 1 FPS in our setting), which serves as a semantic hint for the matching process while not introducing too much emphasis on input action sequence.
The final evaluation model consists of an masked interaction encoder and a text encoder.
We use contrastive loss following CLIP~\citep{clip}, which encourages paired motion and text features to be close geometrically.
In addition, we add a classification head after the predicted motion features, to simultaneously predict interaction labels, such as ``high-five''.

\textbf{Baselines.} To evaluate the performance of our method \ModelAbbr~on online and unconstrained setting, we compare \ModelAbbr~with the following baselines:
1) \textbf{InterFormer}~\citep{interformer} is a transformer based action-to-reaction generation model that leverages human skeleton as prior knowledge for efficient attention process.
2) \textbf{MotionGPT}~\citep{motiongpt} is a motion-language model that leverages an LLM for motion and text generation. We extend the motion tokenizer of MotionGPT to encode multi-person motion, while keeping other settings unchanged.
3) \textbf{InterGen}~\citep{intergen} proposes a mutual attention mechanism within diffusion process for human interaction generation, we reproduce and adapt IngerGen to action-to-reaction generation.
4) \textbf{ReGenNet}~\citep{regennet} is latest state-of-the-art model on action-to-reaction generation. It adopts a transformer decoder based diffusion model, which directly predicts human reaction given action input in unconstrained and online manner as ours.


\textbf{Implementation Details.}
For the LLM, we adopt Flan-T5-base~\citep{flan,t5} as our base model, with extended vocabulary. We warm up the learning rate for 1,000 steps, peaking at 1e-4 for the pre-training phase, and use the same learning rate for fine-tuning.
Both the pre-training and fine-tuning phases are trained on a single machine with 8 Tesla V100 GPUs. The training batch size is set to 32 for the LLM and we monitor the validation loss and reaction generation metrics for early-stopping, resulting about 100K pre-training steps and 40K fine-tuning steps.
We set the re-thinking interval $N_r$ to 4 tokens and divide each space signal into $N_b=10$ bins.

\begin{table}[t]
\centering
\tiny
\caption{Comparison to state-of-the-art baselines and ablation studies of our method on Inter-X dataset. $\uparrow$ or $\downarrow$ denotes a higher or lower value is better, and $\rightarrow$ means that the value closer to real is better. We use $\pm$ to represent 95\% confidence interval and highlight the best results in \textbf{bold}. For ablation methods (in grey), PT, M, P, S, and SP are abbreviations for pre-training, motion, pose, space, and single-person data, respectively.}
\label{tab:main}
\begin{tabular}{l|ccccccc}
\toprule
\multirow{2}{*}{Methods} &  \multicolumn{3}{c}{R-Precision$\uparrow$} & \multirow{2}{*}{Acc.$\uparrow$}& \multirow{2}{*}{FID$\downarrow$}         & \multirow{2}{*}{MMDist$\downarrow$} & \multirow{2}{*}{Div.$\rightarrow$} \\
               & Top-1       & Top-2       & Top-3     &     &         &                               &                       \\ \midrule
Real                    & $0.511^{\pm.003}$ & $0.682^{\pm.002}$ & $0.776^{\pm.002}$ & $0.463^{\pm.000}$  & $0.000^{\pm.000}$         & $5.348^{\pm.002}$         & $2.498^{\pm.005}$           \\ \midrule
InterFormer             & $0.172^{\pm.012}$ & $0.292^{\pm.013}$ & $0.343^{\pm.012}$ & $0.171^{\pm.009}$ & $10.468^{\pm.021}$        &  $7.831^{\pm.018}$         & $3.505^{\pm.023}$           \\
MotionGPT &            $0.238^{\pm.003}$        &     $0.354^{\pm.004}$        &     $0.441^{\pm.003}$   &    $0.186^{\pm.002}$            &     $5.823^{\pm.048}$               &      $6.211^{\pm.005}$           &      $2.615^{\pm.007}$ \\
InterGen                & $0.326^{\pm.036}$ & $0.423^{\pm.063}$ & $0.525^{\pm.053}$ & $0.254^{\pm.019}$  & $5.506^{\pm.257}$         & $6.182^{\pm.038}$         & $2.284^{\pm.009}$           \\
ReGenNet                & $0.384^{\pm.005}$ & $0.483^{\pm.002}$ & $0.572^{\pm.003}$ & $0.297^{\pm.004}$  & $3.988^{\pm.048}$         & $5.867^{\pm.009}$         & $\mathbf{2.502^{\pm.001}}$           \\ \midrule
% \rowcolor[HTML]{EFEFEF}
\ModelAbbr~(Ours)       & $\mathbf{0.423^{\pm.005}}$ & $\mathbf{0.599^{\pm.003}}$ & $\mathbf{0.693^{\pm.003}}$ & $\mathbf{0.318^{\pm.003}}$  & $\mathbf{1.942^{\pm.017}}$         & $\mathbf{5.643^{\pm.003}}$         & $2.629^{\pm.006}$           \\
\rowcolor[HTML]{EFEFEF}
w/o Think           & $0.367^{\pm.003}$ & $0.491^{\pm.027}$ & $0.584^{\pm.008}$ & $0.230^{\pm.036}$ & $3.828^{\pm.016}$         & $6.186^{\pm.055}$         & $2.609^{\pm.006}$           \\
\rowcolor[HTML]{EFEFEF}
w/o All PT.         & $0.398^{\pm.007}$ & $0.531^{\pm.002}$ & $0.628^{\pm.003}$ & $0.288^{\pm.002}$ & $3.467^{\pm.113}$         & $5.822^{\pm.003}$         & $2.909^{\pm.053}$           \\
\rowcolor[HTML]{EFEFEF}
w/o M-M PT. & $0.408^{\pm.005}$ & $0.563^{\pm.004}$ & $0.646^{\pm.005}$ & $0.293^{\pm.002}$ & $2.874^{\pm.020}$         & $5.736^{\pm.003}$         & $2.553^{\pm.006}$           \\
\rowcolor[HTML]{EFEFEF}
w/o P-S PT. & $0.417^{\pm.004}$ & $0.582^{\pm.004}$ & $0.664^{\pm.004}$ & $0.308^{\pm.003}$ & $2.685^{\pm.024}$         & $5.699^{\pm.004}$         & $2.859^{\pm.007}$           \\
\rowcolor[HTML]{EFEFEF}
w/o M-T PT. & $0.406^{\pm.003}$ & $0.557^{\pm.004}$ & $0.637^{\pm.004}$ & $0.304^{\pm.003}$ & $2.580^{\pm.021}$         & $5.822 ^{\pm.003}$         & $2.889^{\pm.005}$           \\
\rowcolor[HTML]{EFEFEF}
w/o SP Data     & $0.414^{\pm.004}$ & $0.592^{\pm.005}$ & $0.685^{\pm.003}$ & $0.315^{\pm.004}$ & $2.007^{\pm.015}$         & $5.667^{\pm.003}$         & $2.611^{\pm.005}$           \\
\bottomrule
\end{tabular}
\end{table}



\begin{figure}
    \centering
    \includegraphics[width=\linewidth]{figs/tsne.pdf}
    \caption{Visualization of a person's motion sequences in Inter-X dataset and HumanML3D dataset.}
    \label{fig:tsne}
\end{figure}

\subsection{Comparison to Baselines}\label{sec:sota}
As shown in the upper side of Table~\ref{tab:main}, our method \ModelAbbr~significantly outperforms baseline methods in terms of ranking, accuracy, FID and multimodal distance, showing superior human reaction generation quality.
Compared to MotionGPT, which adopts a similar motion-language architecture, \ModelAbbr~expresses stronger performance, which we attribute to our unified representation of motion via space and pose tokenizers, enabling effective individual pose and inter-person spatial relationship representation.
\ModelAbbr~also surpasses the diffusion-based methods, InterGen and ReGenNet, with our think-then-react architecture, improving generated motions by describing observed action and reasoning what reaction is expected on semantic level. In addition, ReGenNet and MotionGPT get closer diversity to the real than our model. We mainly attribute to that, \ModelAbbr~may conduct multiple re-thinking processes during inference, and the inferred semantics may bring a higher diversity.


\subsection{Ablation Study of Key Components}
To evaluate the effectiveness of our proposed key designs, we conduct detailed ablation studies by removing each of them to observe how much drop compared to the full version of our \ModelAbbr~method. The larger drop indicates more contribution. The results are shown in gray lines of Table~\ref{tab:main}. According to the drops in FID, all designs, including thinking, pre-training tasks and using single person data in pre-training, have positive contributions to the final performance, and thinking contributes the most. Some detailed findings and analyses are as follows.

First, we skip \textbf{thinking} stage during inference, and find the performance drops significantly in FID from 1.9 to 3.8. This supports the necessity of our proposed thinking process before reacting. We also notice decreasing diversity of generated samples, as the model relies solely on input action, and cannot explicitly capture and infer action's intent, thus leading to more rigid motion in some cases.

Second, to evaluate the effectiveness of \textbf{pre-training}, we omit the pre-training stage, and directly train our model \ModelAbbr~for thinking and reacting tasks. As shown in Table~\ref{tab:main}, our model's performance deteriorates without a fine-grained pre-training phase from 1.9 to 3.4 in FID. This indicates that pre-training can effectively adapt a language model (Flan-T5-base) into a motion and language model. We further removing three kinds of pre-training tasks: motion-motion (M-M PT.), pose-space (P-S PT.), and motion-text (M-T PT.). The results show that the without any task, the performance obviously gets worse, from 1.9 to 2.5 - 2.8 in FID, indicating their positive contribution to the final performance and complementary values to each other.

Third, to see how much \textbf{single-person data} helps reaction generation, we remove single person motion-text data, i.e., the data from HumanML3D dataset, from our training set. The result (w/o SP Data) shows that the model performs worse without training on HumanML3D, which proves that our unified motion encoder and motion-language architecture can leverage both single- and multi-person data, alleviating the insufficiency of training data. However, the benefit from single-person data is not as large as we expect. 

% What's more, we evaluate the necessity of \textbf{decoupled space-motion tokenizer}, and the results are shown in Table~\ref{tab:vqvae}. We design a plain motion VQ-VAE with unnormalized action and reaction as input, maintaining absolute space and pose features. With the trained motion VQ-VAE, we encode action/reaction into tokens, which are then fed into TTR for reaction prediction task. First, without normalized motion as input, the reconstruction FID significantly rises from 0.262 to 0.983, showing deteriorated reconstruction performance due to insufficient utilization of codebook. Second, in the reaction generation phase, TTR's performance drops dramatically, as the badly constructed codebook leads to inaccurate action understanding and reaction prediction, highlighting the necessity of decoupling token representation of space and pose features in multi-person scenario.

\begin{figure}
    \centering
    \includegraphics[width=\linewidth]{figs/case_study.pdf}
    \caption{Visualized cases of our predicted reactions (in green) to input action (in blue) and corresponding thinking results. We also provide a failure case in figure (d), where TTR misunderstands the input action as ``wrestling'', which should be ``embracing''.}
    \label{fig:case_study}
\end{figure}


\subsection{Analysis on Overlapping between Single- and Multi-Person Motions}
To investigate the reason of small contribution from single-person data, we further visualize motion sequences of single-person motion (HumanML3D), two-person action (Inter-X Action) and reaction (Inter-X Reaction) in the same space, as presented in Figure~\ref{fig:tsne}. Specifically, we use t-SNE tool~\cite{tsne} to project motion token sequence features into two-dimension. As shown in Figure~\ref{fig:tsne}, the single- and two-person motion sequences have little overlap. When doing case studies, we find that most two-person motion are unique, e.g., massage and being pulled, and will never be used in single-person motion. Similarly, most single-person motions are unique too, e.g., T-pose, and seldom appear in multi-person interaction. There are only a few overlapped motions, e.g., standing still. In addition, when comparing action and reaction sequences in multi-person interaction, we have some interesting findings. When reactions are close to actions, the motion usually belongs to symmetrical interactions, e.g., pulling or being pulled; whereas, when actions are far from reactions, the motion usually belongs to asymmetrical interaction, e.g., massage.


\subsection{Impact of Down-Sampling Parameter in Matching Model for Evaluation}

As described in Section~\ref{sec:eval}, we propose downsampling action motion sequence to avoid matching models for evaluation pay too much attention to input action rather than output reaction. We conduct an experiment to change the downsampling parameter frame rate and calculate the difference between taking ground-truth action and random action as the input of $\mathcal{M}$, in terms of summed ranking scores (Top-1, Top-2, Top-3 and Acc.). As presented in Figure~\ref{fig:discriminative}, 
difference is lowest when FPS equals to 0, which meaning we only match generated reaction motion with text. It goes up to the peak when FPS equals 1 and quickly goes down to low values, even close to the lowest when FPS is about 15. This indicates that it is necessary to concatenate input action with generated reaction to compose a meaningful interaction in evaluation, otherwise the motion-text matching model cannot well recognize the interaction. However, only 1 FPS is enough. With larger FPS, the matching models will be disturbed by input action rather than the generated reaction. Thus, we choose 1 FPS, corresponding to the largest difference, as our final setting.

\subsection{Impact of Re-thinking Interval}
% Our aim is to generate real-time reaction online, and thus time interval is an important parameter to generation quality. 
We change the re-thinking interval $N_r$ from about 1 to 100 timesteps (about 0.1 to 10 seconds) and observe how it impacts generative quality measure FID. As shown in Figure~\ref{fig:latency}, FID falls down first until $N_r=4$ (about 0.5 second) and then continues rising up. This indicate that the best time interval is about 0.5 second. When the time interval is too short, our \ModelAbbr~model cannot get enough information to re-think what the input action means and will bring some randomness into predicting appropriate reaction. When the time interval gets too long, our \ModelAbbr~model give slow responses to the input action sequences and generates coarse-grained reaction.

We also evaluate the average inference time per step (AITS) with respect to the re-thinking interval. As shown in Figure~\ref{fig:latency}, the inference time significantly decreases as the re-thinking interval increases, eventually converging to approximately 10 milliseconds per step (100 FPS). In our setup, we opt to re-think every four steps, resulting in an inference time of less than 50 milliseconds, which meets the requirements for a real-time system.

\subsection{User Study}
To further evaluate our model qualitatively, we conduct a user study on TTR vs. the latest SOTA method ReGenNet, and the results are shown in Figure~\ref{fig:user_study}. We randomly sample 100 action sequences from Inter-X dataset, which are fed into TTR and ReGenNet to predict reactions, and ask four real human to choose the better ones. It can be seen that TTR surpasses ReGenNet on all the duration range, and the winning rate rises significantly when motion duration is longer. We mainly contribute this to our explicit thinking and re-thinking procedure, which ensures semantics matching and alleviates accumulated errors. 

\begin{figure}[t]
    \centering
    \begin{minipage}[b]{0.3\textwidth}
        \centering
        \includegraphics[width=\textwidth]{figs/discriminative_power.pdf}
        \caption{Impact of input action FPS to summed ranking score differences.}
        \label{fig:discriminative}
    \end{minipage}
    \hfill
    \begin{minipage}[b]{0.35\textwidth}
        \centering
        \includegraphics[width=\textwidth]{figs/latency.pdf}
        \caption{Impact of re-thinking interval to FID and average inference time per step (AITS).}
        \label{fig:latency}
    \end{minipage}
    \hfill
    \begin{minipage}[b]{0.3\textwidth}
        \centering
        \includegraphics[width=\linewidth]{figs/user_study.pdf}
        \caption{User preference between TTR and ReGenNet on different motion duration.}
        \label{fig:user_study}
    \end{minipage}
\end{figure}


\section{Discussion}
\label{sec:discussion}

\rv{
While the results show that \name is able to simulate human-like eye movements when performing analytical tasks, there is a need to expand on our discussion of the model's practical implications, the generalizability of the modeling approach, and the limitations and potential for supporting sophisticated chart-based question answering.
}

\subsection{\rv{Applications}}

\rv{
\paragraph{Visualization design evaluation}
\name can assist in evaluation of chart design.
With well-controlled experiment conditions, eye tracking data afford valuable insight into chart designs, especially relative to alternative designs.
For example, \citet{goldberg2011eye} showcased eye tracking's value in comparing line and radial graphs for reading of values, by allowing researchers to understand the viewing order of AOIs and the task completion time.
\name holds potential to replace human input to evaluation based on eye tracking.
With the simulated scanpaths from \name, chart designers can obtain quick and cost-effective feedback that yields the benefits from eye tracking without requiring an expensive empirical study.
}
\rv{
\paragraph{Visualization design optimization}
Beyond evaluation, another potential usage application of \name is to help optimize visualization design~\cite{shin2023perceptual}. 
Like other fields of design, visualization design requires user feedback for continual iteration. When visualization designers create charts for specific tasks, they may wonder if the design is suitable for delivery.
With the predicted scanpaths from the model, they can easily access quick and affordable feedback before deeming a candidate design ready for expensive evaluation in a user study.
Predictive models could offer feedback to designers or even provide optimization goals in automated visualization design frameworks.
The ultimate goal is grounding for recommendations for visualizations that support specific tasks~\cite{albers2014task} and even automation of visualization design in real time.
Today's human-in-the-loop design optimization paradigm~\cite{kadner2021adaptifont} could shift to a user-agent-in-the-loop approach, wherein a computational agent that simulates human feedback enables scalable and efficient design evaluation.
}

\rv{
\paragraph{Explainable AI in chart question answering}
Systems for answering questions via charts~\cite{masry2022chartqa} are typically viewed as black boxes that generate answers directly from a given chart and natural-language question. 
In contrast, \name introduces a glass-box approach that answers questions through a step-by-step reasoning process. This method enhances the alignment between human and machine attention~\cite{sood2023multimodal}.
We anticipate that this approach could lead to significant improvements in chart question answering~\cite{masry2022chartqa} and greater compatibility with explainable AI systems.
}

\subsection{\rv{Extending the Model beyond Bar Charts}}

\rv{
Our modeling approach can be extended to many visualization types besides bar charts.
We analyzed the visualization taxonomy outlined in prior work~\cite{borkin2013makes, borkin2015beyond}, including area, circle, diagram, distribution, grid, line, map, point, table, text, tree, and network, then categorize these visualization techniques into two groups: those that are feasible to extend with minor changes and those that are out of reach, requiring additional features.
}

\begin{figure}[!t]
    \centering
    \subfigure[\rv{An \textit{RV} task with a line chart: ``What was the revenue from newspaper advertising in 1980?''}]{\label{fig:a}\includegraphics[width=0.48\textwidth]{Images/line-case.png}}
    \hspace{0.02\columnwidth}
    \subfigure[\rv{An \textit{F} task with a scatterplot: ``In which countries do people anticipate spending about \$700 for personal Christmas gifts?''}]{\label{fig:b}\includegraphics[width=0.48\textwidth]{Images/point-case.png}}
    \caption{\rv{Two cases that illustrate the generalizability of the modeling approach, showing the extension of \name to a line chart and a scatterplot. The model's predictions are spatially similar to human ground-truth scanpaths.}}
    \label{fig:case}
    % \vspace{-5mm}
\end{figure}

\rv{
Our modeling approach can be applied to most statistical charts either directly or upon rectification of minor issues. For instance, extending the model to interpret \textit{line charts} and \textit{area charts} is feasible when the axis labels are clearly defined. The trend patterns of lines and areas can be perceived by the peripheral vision as visual guidance.
For \textit{point charts}, such as scatterplots, the model performs well in conditions of sparse data points. However, individual points may be obscured in dense scatterplots, making it difficult to label data when points are cluttered or overlapping. 
\textit{Distribution charts}, such as histograms, and \textit{circle charts}, such as pie charts, are similar to bar charts in that they use the area of marks to represent values. Retrieving exact values from these two presentation types can be imprecise on account of the ranges of the bins and inaccuracies in estimating angles or arc lengths.
Reading \textit{grid charts} (e.g., heatmaps) too is feasible; however, identifying the values necessitates understanding color intensity, a factor that can sometimes lead to ambiguity.
Modeling scanpaths on \textit{tables} or \textit{text} for retrieval tasks is tractable under the current modeling approach, but a lack of visual pattern recognition may render the results poor.
To further examine the generalizability of this category, we considered two additional cases, using a line chart and a scatterplot. We manually labeled the charts, trained the model, and made predictions. As Figure~\ref{fig:case} attests, the trained model performs well for these two chart types when compared to human ground-truth scanpaths.
}

\rv{
Other, sophisticated visualization types are out of reach because they require additional features, particularly prior knowledge and advanced reasoning abilities. For instance, reading \textit{maps} involves associating spatial regions with colors, sizes, or symbols to retrieve related values. Also, when interpreting maps, people rely heavily on preexisting geographical knowledge as a basis for efficient visual searches. Complex designs with intricate structures, such as \textit{diagrams}, \textit{trees}, and \textit{network graphs}, typically require advanced reasoning based on connections. All these skill requirements point to a need for further study in this area.
}

\subsection{\rv{Paths toward Sophisticated Tasks in Chart Question Answering}}

\rv{
Although the model focuses primarily on gaze prediction, it is worth exploring potential improvements for enriching its sophisticated question answering capabilities. We also discuss its limitations.
}

\rv{
Our current model does not achieve the same level of accuracy as the state-of-the-art models represented by the ChartQA benchmark~\cite{masry2022chartqa}. 
Unlike other models that can access the full chart image, \name is limited by its foveal vision and restricted spatial reasoning abilities. For instance, if a bar's height falls between two labeled values, such as 10 and 15, the model might choose either 10 or 15 as its answer when interpreting the axis, failing to provide a more precise value. 
This limitation stems from the constrained spatial perception capabilities of LLMs, which are central to cognitive control.
One possible solution is integrating multi-modal LLMs~\cite{cuarbune2024chart}, for which recent research has demonstrated an accuracy rate of 81.3\%.
}

\rv{
The sense-making process for complex visualizations may be inherently challenging. Even humans often struggle with understanding how the data are encoded, recognizing a given chart's purpose, tackling readability issues, performing numerical calculations, identifying relationships among data points, and navigating the spatial arrangement of graphical elements~\cite{rezaie2024struggles}.
Our model is designed to be straightforward and objective, focusing on analysis tasks related to statistical charts, but it does not fully capture the complexities of visualizations.
A possible enhancement in this respect would be to integrate the model with human sense-making practices~\cite{rezaie2024struggles} or to incorporate a framework of human understanding~\cite{albers2014task}. Such integration could facilitate better simulation of a human-like problem-solving process.
}


\section{Conclusion}
In this study, we introduce \ours, a novel framework designed to achieve lossless acceleration in generating ultra-long sequences with \acp{llm}. By analyzing and addressing three challenges, \ours significantly enhances the efficiency of the generation process. Our experimental results demonstrate that \ours achieves over $3\times$ acceleration across various model scales and architectures. Furthermore, \ours effectively mitigates issues related to repetitive content, ensuring the quality and coherence of the generated sequences. These advancements position \ours as a scalable and effective solution for ultra-long sequence generation tasks.


\section{Limitations}
\label{limitations}

During our experiments, we noted several limitations that future work could expand upon and resolve. Firstly, there is only one open-source and one close-source one. Future work could look at including more LLMs for comparison. Secondly, the prediction target, holistic score, is still hard to interpret, whereas some subsets (7 and 8) of ASAP and the entire set of ELLIPSE do have fine-grained essay score annotations (see examples in Appendix \ref{ellipse-rubric}). Incorporating them into the overall score prediction process would make the overall score more transparent. Thirdly, the persona section of the prompt template mentions ``grade 7 to 10,'' which is the age range for students in ASAP; however, the students in the ELLIPSE dataset are from grades 8 to 12, which might lead to performance differences among those two datasets. Lastly, our datasets have a clear western bias, especially ELLIPSE, which focuses on ESL students in the United States. We believe the community would benefit from more diverse and inclusive datasets.


% \acks{Acknowledgements go here.}

\bibliography{citations/alejandro-citations, citations/joey-citations}
% \bibliography{citations/joey-citations}

\appendix

\pagebreak
\section{ELLIPSE Rubric}
\label{ellipse-rubric}
% \begin{table}[h!]
%     \centering
%     \small
\begin{center}
    \small
    \begin{longtable}{|p{0.12\linewidth}|p{0.15\linewidth}|p{0.15\linewidth}|p{0.15\linewidth}|p{0.15\linewidth}|p{0.15\linewidth}|}
        \hline
        Score \ \ \ \ Category & 5                                                                                                                                                                                                                                                & 4                                                                                                                                                                                                                       & 3                                                                                                                                                                                                                                                                         & 2                                                                                                                                                                                                               & 1                                                                                                                                                                                            \\ \hline
        Overall                       & Native-like facility in the use of language with syntactic variety, Appropriate word choice and phrases; well-controlled text organization; precise use of grammar and conventions; rare language inaccuracies that do not impede communication. & Facility in the use of language with syntactic variety and range of words and phrases; controlled organization; accuracy in grammar and conventions; occasional language inaccuracies that rarely impede communication. & Facility limited to the use of common structures and generic vocabulary; organization generally controlled although connection sometimes absent or unsuccessful; errors in grammar and syntax and usage. Communication is impeded by language inaccuracies in some cases. & Inconsistent facility in sentence formation, word choice, and mechanics; organization partially developed but may be missing or unsuccessful. Communication impeded in many instances by language inaccuracies. & A limited range of familiar words or phrases loosely strung together; frequent errors in grammar (including syntax) and usage. Communication impeded in most cases by language inaccuracies. \\ \hline
        Cohesion                      & Text organization consistently well controlled using a variety of effective linguistic features such as reference and transitional words and phrases to connect ideas across sentences and paragraphs; appropriate overlap of ideas.             & Organization generally well controlled; a range of cohesive devices used appropriately such as reference and transitional words and phrases to connect ideas; generally appropriate overlap of ideas                    & Organization generally controlled; cohesive devices used but limited in type; Some repetitive, mechanical, or faulty use of cohesion use within and/or between sentences and paragraphs.                                                                                  & Organization only partially developed with a lack of logical sequencing of ideas; some basic cohesive devices used but with inaccuracy or repetition.                                                           & No clear control of organization; cohesive devices not present or unsuccessfully used; presentation of ideas unclear.                                                                        \\ \hline
        Syntax                        & Flexible and effective use of a full range of syntactic structures including simple, compound, and complex sentences; There may be rare minor and negligible errors in sentence formation.                                                       & Appropriate use of a variety of syntactic structures, such as simple, compound, and complex sentences; occasional errors or inappropriateness in sentence formation.                                                    & Simple, compound, and complex syntactic structures present although the range may be limited; some apparent errors in sentence formation, especially in more complex sentences.                                                                                           & Some sentence variation used; many sentence structure problems.                                                                                                                                                 & Pervasive and basic errors in sentence structure and word order that cause confusion; basic sentences errors common.                                                                         \\ \hline
        Vocabulary                    & Wide range of vocabulary flexibly and effectively used to convey precise meanings; skillful use of topic-related terms and less common words; rare negligible inaccuracies in word use.                                                          & Sufficient range of vocabulary to allow flexibility and precision; appropriate use of topic-related terms and less common lexical items                                                                                 & Minimally adequate range of vocabulary for the topic; no precise use of subtle word meanings; topic related terms only used occasionally; attempts to use less common vocabulary but with some inaccuracy                                                                 & Narrow range of vocabulary to convey basic and elementary meaning; topic related terms used inappropri ately; errors in word formation and word choice that may distort meanings                                & Limited vocabulary often inappropriately used; limited control of word choice and word forms; little attempt to use topic-related terms                                                      \\ \hline
        Phraseology                   & Flexible and effective use of a variety of phrases, such as idioms, collocations, and lexical bundles, to convey precise and subtle meanings; rare minor inaccuracies that are negligible.                                                       & Appropriate use of a variety of phrases, such as idioms, collocations, and lexical bundles; occasional inaccuracies and colloquialisms.                                                                                 & Evident use of phrases such as idioms, collocations, and lexical bundles but without much variety; some noticeable repetitions and misuses.                                                                                                                               & Narrow range of phrases, such as collocations and lexical bundles, used to convey basic and elementary meaning; many repetitions and /or misuses of phrases.                                                    & Memorized chunks of language, or simple phrasal patterns predominate; many repetitions and misuses of phrases.                                                                               \\ \hline
        Grammar                       & Command of grammar and usage with few or no errors.                                                                                                                                                                                              & Minimal errors in grammar and usage.                                                                                                                                                                                    & Some errors in grammar and usage.                                                                                                                                                                                                                                         & Many errors in grammar and usage.                                                                                                                                                                               & Errors in grammar and usage throughout.                                                                                                                                                      \\ \hline
        Conventions                   & Consistent use of appropriate conventions to convey meaning; spelling, capitalization, and punctuation errors nonexistent or negligible.                                                                                                         & Generally consistent use of appropriate conventions to convey meaning; spelling, capitalizatio n, and punctuation errors few and not distracting.                                                                       & Developing use of conventions to convey meaning; errors in spelling, capitalization, and punctuation that are sometimes distracting.                                                                                                                                      & Variable use of conventions; spelling, capitalization, and punctuation errors frequent and distracting.                                                                                                         & Minimal use of conventions; spelling, capitalizatio n, and punctuation errors throughout.                                                                                                    \\ \hline
    \caption{The ELLIPSE rubric, gotten directly from the original paper.}
    \label{tab:ellipse-rubric}
    \end{longtable}
\end{center}
% \end{table}



\section{Supervised Baseline Details}
\label{supervised-baseline-details}
Our supervised baseline~\cite{wang-etal-2022-use}\footnote{\url{https://github.com/lingochamp/Multi-Scale-BERT-AES}}is a BERT-based architecture comprised of two sub-components---each pretrained BERT models (\cite{devlin})---which analyze three main feature classes: document-, token- and segment-scale features. The first sub-component receives the document- and token-scale features. It is fine-tuned to learn the document-scale feature representation through the \texttt{[CLS]} (start) token\footnote{There can be multiple text segments per essay as their input length is set to 510.} and the token-scale features through the BERT word embeddings. Its output goes through a final max pooling layer to represent the sub-component's score. The segment-scale features are received by the second sub-component, which takes in an essay as a series of segments each of size $k$ (except the last segment, which is smaller). A list of these segment series of varying sizes $k_{i}$ are input into the model sequentially, and a final LSTM and attention and dense pooling layer is used to output the sub-component's score. Lastly, the output from the two sub-components are added together to produce the final holistic score. The model's loss function is additive between mean squared error ($MSE$), cosine similarity ($CS$) and margin ranking loss ($MLR$): $\mathcal{L}_{\mathrm{Total}}(\mathbf{x}, \mathbf{y}) = \alpha \mathcal{L}_{MSE}(\mathbf{x}, \mathbf{y}) + \beta \mathcal{L}_{CS}(\mathbf{x}, \mathbf{y}) + \gamma \mathcal{L}_{MLR}(\mathbf{x}, \mathbf{y})$.

We base our fine-tuning on the authors' available code\footnote{Available upon acceptance.}. The authors only released their fine-tuned model for ASAP prompt 8, so---to obtain models for all prompts---we fine-tuned \texttt{bert-base-uncased} as they did in the original paper. We used our splits of the ASAP dataset for fine-tuning, validation and testing (see \ref{asap-and-asap++}). We fine-tune for 80 epochs, our hyperparameters for $\alpha$, $\beta$ and $\gamma$ were all set to 0.5 and with cosine similarity \texttt{dim=1} and margin ranking loss \texttt{margin=0}. Everything is implemented in PyTorch (\cite{paszke2019}) and HuggingFace (\cite{wolf2019}) using \texttt{google-bert/bert-base-uncased}. We run the test set on the prompt's model with the best loss.


\section{Zero-shot Essay Scoring Prompts}
\label{app-scoring-prompts}
% \subsection{Zero-shot Essay Scoring}
Here are some examples of zero-shot essay scoring prompts. Note that the exact phrasing and wording are not exactly the same as \cite{stahl-etal-2024-exploring} paper. That is because we have failed to reproduce the exact same results in their paper, motivating us to conduct a limited prompt tuning in the \texttt{dev} set of ASAP. To reduce complexity, the tuning is done only in phrasing and formatting, without changing the overall structure of the prompt compared to the original design.

\subsubsection{No Linguistic Feature}
\textit{
You are part of an educational research team analyzing the writing skills of students in grades 7 to 10. You have been given a student's essay and the prompt they responded to.  \\
\#\#\# Essay Prompt: More and more people use computers, but not everyone agrees that this benefits society. Those who support advances in technology believe that computers have a positive effect on people. They teach hand-eye coordination, give people the ability to learn about faraway places and people, and even allow people to talk online with other people. Others have different ideas. Some experts are concerned that people are spending too much time on their computers and less time exercising, enjoying nature, and interacting with family and friends. Write a letter to your local newspaper in which you state your opinion on the effects computers have on people. Persuade the readers to agree with you.  \\
}
\textit{
\#\#\# Analysis Task: Grade the given essay with the following requirements:  \\
- Use those score ranges: Overall: from 1 to 6.  \\
- Provide an explanation for your score as well.  \\
}
\textit{
\#\#\# Analyzed Student Essay: Dear, @CAPS1 @CAPS2 @CAPS3 More and more people use computers, but not everyone agrees that this benefits society. Those who support advances in technology believe that computers have a positive effect on people. Others have different ideas. A great amount in the world today are using computers, some for work and spme for the fun of it. Computers is one of mans greatest accomplishments. Computers are helpful in so many ways, @CAPS4, news, and live streams. Don't get me wrong way to much people spend time on the computer and they should be out interacting with others but who are we to tell them what to do. When I grow up I want to be a author or a journalist and I know for a fact that both of those jobs involve lots of time on time on the computer, one @MONTH1 spend more time then the other but you know exactly what @CAPS5 getting at. So what if some expert think people are spending to much time on the computer and not exercising, enjoying natures and interacting with family and friends. For all the expert knows that its how must people make a living and we don't know why people choose to use the computer for a great amount of time and to be honest it's non of my concern and it shouldn't be the so called experts concern. People interact a thousand times a day on the computers. Computers keep lots of kids of the streets instead of being out and causing trouble. Computers helps the @ORGANIZATION1 locate most wanted criminals. As you can see computers are more useful to society then you think, computers benefit society. \\
}
\textit{
\#\#\# Analysis: Conclude your analysis with a grade and comments in the following format:  \\
\#\#\# Explanation:  \\
\#\#\# Score: \\
- Overall:
}


\subsubsection{Top-10 Features}
\textit{
You are part of an educational research team analyzing the writing skills of students in grades 7 to 10. You have been given a student's essay and the prompt they responded to.  \\
\#\#\# Essay Prompt: More and more people use computers, but not everyone agrees that this benefits society. Those who support advances in technology believe that computers have a positive effect on people. They teach hand-eye coordination, give people the ability to learn about faraway places and people, and even allow people to talk online with other people. Others have different ideas. Some experts are concerned that people are spending too much time on their computers and less time exercising, enjoying nature, and interacting with family and friends. Write a letter to your local newspaper in which you state your opinion on the effects computers have on people. Persuade the readers to agree with you.  \\
}
\textit{
\#\#\# Analysis Task: Grade the given essay with the following requirements:  \\
- Use those score ranges: Overall: from 1 to 6.  \\
- Provide an explanation for your score as well.  \\
}
\textit{
\#\#\# Analyzed Student Essay: Dear, @CAPS1 @CAPS2 @CAPS3 More and more people use computers, but not everyone agrees that this benefits society. Those who support advances in technology believe that computers have a positive effect on people. Others have different ideas. A great amount in the world today are using computers, some for work and spme for the fun of it. Computers is one of mans greatest accomplishments. Computers are helpful in so many ways, @CAPS4, news, and live streams. Don't get me wrong way to much people spend time on the computer and they should be out interacting with others but who are we to tell them what to do. When I grow up I want to be a author or a journalist and I know for a fact that both of those jobs involve lots of time on time on the computer, one @MONTH1 spend more time then the other but you know exactly what @CAPS5 getting at. So what if some expert think people are spending to much time on the computer and not exercising, enjoying natures and interacting with family and friends. For all the expert knows that its how must people make a living and we don't know why people choose to use the computer for a great amount of time and to be honest it's non of my concern and it shouldn't be the so called experts concern. People interact a thousand times a day on the computers. Computers keep lots of kids of the streets instead of being out and causing trouble. Computers helps the @ORGANIZATION1 locate most wanted criminals. As you can see computers are more useful to society then you think, computers benefit society. \\
}
\textit{
\#\#\# Additional Information: Studies show that the following features are highly, positively correlated with the grade of the essay (i.e., higher features typically means higher end score)  \\
- total number of unique words in the essay: 113  \\
- total number of words in the essay.: 279  \\
- total number of sentences present: 14  \\
- total number of characters: 279  \\
- total number of lemma: 133  \\
- total number of nouns: 50  \\
- total number of stopwords: 71  \\
- total number of words that are not in the Dale-Chall word list of 3000 words recognized by 80\% of fifth graders: 80  \\
- total number of characters: 1229  \\
}
\textit{
\#\#\# Analysis: Conclude your analysis with a grade and comments in the following format:  \\
\#\#\# Explanation:  \\
\#\#\# Score: \\
- Overall:
}

\section{Linguistic Features}
\label{app:linguistic-features}
\textbf{Unique words} refers to the number of single-instance words in the essay. For \textbf{essay character length}, we only count the number of non-space, non-punctuation characters. Words are not normalized before these metrics as in the original paper. \textbf{Total word count} and \textbf{total sentence count} per essay are gotten via \texttt{nltk} (\cite{loper2002}) tokenizers. We additionally utilize the \texttt{en\_core\_web\_sm} in \texttt{spaCy} (\cite{honnibal2020}) to get \textbf{separate counts for lemma, noun, and stop-words.} Finally, we get \textbf{the Dale-Chall} (\cite{dale1948}) \textbf{word count}, \textbf{total character count} and \textbf{long word count} with the \texttt{readability}\footnote{\url{https://pypi.org/project/readability/}} Python package.

Our implementation is based on the code \footnote{\url{https://github.com/robert1ridley/cross-prompt-trait-scoring/blob/main/features.py}} from the original paper \cite{ridleyAutomatedCrosspromptScoring2021}. Our implementation will be made available upon acceptance.
\section{Parsing Module}
\label{app-parsing}
\subsection{Configurations}
\begin{itemize}
    \item Model: Mistral-7B (the same configuration as the scoring model)
    \item Overall parsing error is less than 7\%.
\end{itemize}

\subsection{Few-shot Output Parsing}
\textit{
    You are an AI agent that specialized in converting text input into JSON format.\\
    Instruction: \\
    - Input: text with one or more score and some other relevant information (e.g., explanation, feedbacks, etc.)\\
    - Output: JSON text with `Score' as a mandatory key and other information organized by their field names\\
    - Make sure ONLY return the VALID JSON data, without any additional text or characters.\\
    Here are some examples
}\\

\textit{
Example Input:\\
\#\#\# Explanation: The student's essay demonstrates a limited understanding of the topic and a lack of cohesion. The essay jumps from one idea to another without a clear connection between them. The writing is also filled with numerous grammatical errors, misspellings, and inconsistent capitalization. \\
\#\#\# Score:\\
- Overall: 1 The essay demonstrates a very limited understanding of the topic and contains numerous errors in grammar, spelling, and capitalization. The writing lacks cohesion and a clear thesis statement, and the arguments are not well-supported with evidence or examples. \\
Example Output:\\
\{
    ``Score'': \{
        ``Overall'': 1
    \},
    ``Explanation'': ``The student's essay demonstrates a limited understanding of the topic and a lack of cohesion. The essay jumps from one idea to another without a clear connection between them. The writing is also filled with numerous grammatical errors, misspellings, and inconsistent capitalization.''
\}
}\\

\textit{
Example Input:\\
\#\#\# Explanation: The student's essay demonstrates a basic understanding of the topic and presents a clear position, but the writing is disorganized and contains numerous errors in language conventions. The essay jumps between discussing censorship in libraries and specific examples of offensive music, making it difficult to follow the main argument. \\
\#\#\# Score: \\
- Writing Applications: 2 The essay presents a viewpoint on the issue of censorship, but the argument is not well-developed or clearly stated. The student uses personal experiences and examples. 
- Language Conventions: 1 The essay contains numerous errors in language conventions, including incorrect capitalization, punctuation, and sentence structure. \\
Example Output:\\
\{
    ``Score'': \{
        ``Writing Applications'': 2,
        ``Language Conventions'': 1
    \}
    ``Explanation'': ``The student's essay demonstrates a basic understanding of the topic and presents a clear position, but the writing is disorganized and contains numerous errors in language conventions. The essay jumps between discussing censorship in libraries and specific examples of offensive music, making it difficult to follow the main argument.''
\} 
}\\

\textit{
Example Input:\\
\#\#\# Explanation: The student's essay demonstrates a moderate level of awareness of the audience, as they address the readers directly and use a conversational tone. \\
\#\#\# Feedbacks: the essay could have been more effective if the student had used more formal language and addressed specific concerns of the local community regarding the overuse of computers. \\
\#\#\# Score: \\
- Overall: 3 The student's essay shows some awareness of the audience, but there is room for improvement in terms of language and organization. The essay could benefit from more specific examples and a clearer, more focused argument. \\
Example Output:
\{
    ``Score'': \{
        ``Overall'': 3
    \},
    ``Explanation'': ``The student's essay demonstrates a moderate level of awareness of the audience, as they address the readers directly and use a conversational tone.'',
    ``Feedbacks'': "the essay could have been more effective if the student had used more formal language and addressed specific concerns of the local community regarding the overuse of computers.''
\} 
}\\
\textit{
Now work on the following input:\\
Input:\\
\{LLM OUTPUT\} \\
Output:
% \end{quote}
}

\end{document}
