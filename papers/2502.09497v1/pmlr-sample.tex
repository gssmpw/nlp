\documentclass[pmlr]{jmlr}% new name PMLR (Proceedings of Machine Learning Research)

 % The following packages will be automatically loaded:
 % amsmath, amssymb, natbib, graphicx, url, algorithm2e

 %\usepackage{rotating}% for sideways figures and tables
\usepackage{longtable}% for long tables

 % The booktabs package is used by this sample document
 % (it provides \toprule, \midrule and \bottomrule).
 % Remove the next line if you don't require it.
\usepackage{booktabs}
 % The siunitx package is used by this sample document
 % to align numbers in a column by their decimal point.
 % Remove the next line if you don't require it.
\usepackage[load-configurations=version-1]{siunitx} % newer version
 %\usepackage{siunitx}

 % The following command is just for this sample document:
\newcommand{\cs}[1]{\texttt{\char`\\#1}}

 % Define an unnumbered theorem just for this sample document:
\theorembodyfont{\upshape}
\theoremheaderfont{\scshape}
\theorempostheader{:}
\theoremsep{\newline}
\newtheorem*{note}{Note}
\newcommand{\joey}[1]{\textcolor{red}{\textbf{Joey:} #1}}
\usepackage{booktabs}
\usepackage{makecell}
\usepackage{multirow}

 % change the arguments, as appropriate, in the following:
\jmlrvolume{}
% \jmlryear{2025}
% \jmlrheading{1}
\jmlrpages{}
\jmlrproceedings{JMLR}{}
% \jmlrproceedings{JMLR}{2025 Conference on Artificial Intelligence}%
\jmlrworkshop{}
% \jmlrworkshop{Innovation and Responsibility in AI-Supported Education}

% \title[Short Title]{Full Title of Article\titlebreak This Title Has A Line Break\titletag{\thanks{sample footnote}}}

\title[Linguistic Features for LLM-based AES]{Improve LLM-based Automatic Essay Scoring with Linguistic Features}

 % Use \Name{Author Name} to specify the name.

 % Spaces are used to separate forenames from the surname so that
 % the surnames can be picked up for the page header and copyright footer.
 
 % If the surname contains spaces, enclose the surname
 % in braces, e.g. \Name{John {Smith Jones}} similarly
 % if the name has a "von" part, e.g \Name{Jane {de Winter}}.
 % If the first letter in the forenames is a diacritic
 % enclose the diacritic in braces, e.g. \Name{{\'E}louise Smith}

 % *** Make sure there's no spurious space before \nametag ***

 % Two authors with the same address
  % \author{\Name{Author Name1\nametag{\thanks{with a note}}} \Email{abc@sample.com}\and
  %  \Name{Author Name2} \Email{xyz@sample.com}\\
  %  \addr Address}

 % Three or more authors with the same address:
 \author{
    \Name{Zhaoyi Joey Hou} \Email{joey.hou@pitt.edu}\\
    \Name{Alejandro Ciuba} \Email{alejandrociuba@pitt.edu}\\
    \Name{Xiang Lorraine Li} \Email{xianglli@pitt.edu}\\
    % \addr 130 N Bellefield Ave, Pittsburgh, PA
}


 % Authors with different addresses:
 % \author{\Name{Author Name1} \Email{abc@sample.com}\\
 % \addr Address 1
 % \AND
 % \Name{Author Name2} \Email{xyz@sample.com}\\
 % \addr Address 2
 %}

% \editor{Editor's name}
 % \editors{List of editors' names}

\begin{document}

\maketitle

\begin{abstract}
\begin{abstract}
Out-of-distribution (OOD) detection and OOD generalization are widely studied in Deep Neural Networks (DNNs), yet their relationship remains poorly understood. We empirically show that the degree of Neural Collapse (NC) in a network layer is inversely related with these objectives: stronger NC improves OOD detection but degrades generalization, while weaker NC enhances generalization at the cost of detection. This trade-off suggests that a single feature space cannot simultaneously achieve both tasks. To address this, we develop a theoretical framework linking NC to OOD detection and generalization. We show that entropy regularization mitigates NC to improve generalization, while a fixed Simplex Equiangular Tight Frame (ETF) projector enforces NC for better detection. Based on these insights, we propose a method to control NC at different DNN layers. In experiments, our method excels at both tasks across OOD datasets and DNN architectures. 

\begin{comment}   

Out-of-distribution (OOD) detection and OOD generalization are critical for deploying machine learning models in real-world scenarios. While substantial progress has been made in addressing these problems independently, few works have attempted to tackle them jointly. However, existing methods often rely on auxiliary OOD training data and primarily focus on covariate-shifted OOD data that share labels with in-distribution (ID) data. In contrast, we tackle the more realistic and challenging task of jointly detecting and generalizing to semantic OOD data with disjoint labels from the ID data, without auxiliary OOD training data.
Achieving both objectives simultaneously is inherently difficult due to a fundamental conflict — OOD generalization requires enhanced transferability, while OOD detection necessitates the inhibition of transfer.
To address this, we leverage insights from neural collapse (NC) — a phenomenon in deep networks where top-layer representations suppress feature variability and adopt a Simplex Equiangular Tight Frame (ETF) structure, impairing transferability. By controlling NC, we unify OOD detection and generalization: preventing NC enhances OOD transfer while inducing NC improves OOD detection.
Our proposed method excels at both tasks across various OOD datasets and architectures. 

\end{comment}


\end{abstract}
\end{abstract}
\begin{keywords}
Automatic Essay Evaluation, Large Language Model, Zero shot Learning, LLM as Evaluator, Linguistic Features
\end{keywords}

% \section{Introduction}

Chain-of-Thought (CoT) prompting~\cite{Nye:2021, cot, Kojima:2022cotzero} has emerged as a cornerstone strategy for enhancing Large Language Models (LLMs) in complex reasoning tasks. By eliciting step-by-step inference, CoT enables LLMs to decompose intricate problems into manageable subtasks, thereby improving their problem-solving performance~\cite{Yao:2023tot, Wang:2023self-consistency, Zhou:2023least, Shinn:2023Reflexion}. Recent advancements, such as OpenAI's o1~\cite{o1} and DeepSeek-R1~\cite{deepseekr1}, further demonstrate that scaling up CoT lengths from hundreds to thousands of reasoning steps could continuously improve LLM reasoning. These breakthroughs have underscored CoT’s potential to advance LLM capabilities, expanding the boundaries of AI-driven problem-solving.

\begin{figure}[t]
\centering
    \includegraphics[width=0.95\columnwidth]{fig/intro.pdf}
    \caption{In contrast to vanilla CoT that generates all reasoning tokens sequentially, \method enables LLMs to \textit{skip} tokens with less semantic importance (\textit{e.g.,} \includegraphics[width=7pt]{fig/token.pdf}~) and learn shortcuts between critical reasoning tokens, facilitating controllable CoT compression.}
    \label{fig:intro}
\end{figure}

Despite its effectiveness, the increased length of CoT sequences introduces substantial computational overhead. Due to the autoregressive nature of LLM decoding, longer CoT outputs lead to proportional increases in both inference latency and memory footprints of key-value cache. Additionally, the quadratic computational cost of attention layers further exacerbates this burden. These issues become particularly pronounced when CoT sequences extend into thousands of reasoning steps, resulting in significant computational costs and prolonged response times. While prior research has explored methods for selectively skipping reasoning steps~\cite{Ding:2024cotshortcut, liu2024skipstep}, recent findings~\cite{jin:2024cotlength, Merrill:2024cotlength} suggest that such reductions may conflict with test-time scaling~\cite{o1-blog, snell2025scaling}, ultimately impairing LLM reasoning performance. Therefore, striking an optimal balance between CoT efficiency and reasoning accuracy remains a critical open challenge.

In this work, we delve into CoT efficiency and seek the answer to an important question: \textit{``Does every token in the CoT output contribute equally to deriving the answer?''} We empirically analyze the semantic importance of tokens within CoT outputs and reveal that their contributions to the reasoning performance vary, as depicted in Figure 2. Building on this insight, we introduce \method, a simple yet effective approach that enables LLMs to \textit{skip} less important tokens within CoT sequences and learn shortcuts between critical reasoning tokens, thereby allowing for controllable CoT compression with adjustable ratios. Specifically, as shown in Figure~\ref{fig:intro}, \method constructs compressed CoT training data with various compression ratios, by pruning unimportance tokens from original LLM CoT trajectories. Then, it conducts a general supervised fine-tuning process on target LLMs with this training data, facilitating LLMs to automatically trim redundant tokens during reasoning.

We conduct extensive experiments across various models, including LLaMA-3.1-8B-Instruct and the Qwen2.5-Instruct series, using two widely recognized math reasoning benchmarks: GSM8K and MATH-500. The results validate the effectiveness of \method in compressing CoT outputs while maintaining robust reasoning performance. Notably, Qwen2.5-14B-Instruct exhibits almost \textbf{NO} performance drop (less than $0.4\%$) with a $\bm{40\%}$ reduction in token usage on GSM8K. On the challenging MATH-500 dataset, LLaMA-3.1-8B-Instruct effectively reduces CoT token usage by $\bm{30}\%$ with a performance decline of less than $4\%$, resulting in a $\bm{1.4}\times$ inference speedup. Further analysis underscores the coherence of \method in specified compression ratios and its potential scalability with stronger compression techniques.

\method is distinguished by its low training cost. For Qwen2.5-14B-Instruct, \method fine-tunes only 0.2\% of the model's parameters using LoRA. The size of the compressed CoT training data is no larger than that of the original training set, with 7,473 examples in GSM8K and 7,500 in MATH. The training is completed in approximately 2 hours for the 7B model and 2.5 hours for the 14B model on two 3090 GPUs. These characteristics make \method an efficient and reproducible approach, suitable for use in efficient and cost-effective LLM deployment.

To sum up, our key contributions are:
\begin{enumerate}
    \item To the best of our knowledge, this work is the \textit{first} to investigate the potential of enhancing CoT efficiency through \textit{token skipping}, inspired by the varying semantic importance of tokens in CoT trajectories of LLMs.
    \item We introduce \method, a simple yet effective approach that enables LLMs to skip redundant tokens within CoTs and learn shortcuts between critical tokens, facilitating CoT compression with adjustable ratios.
    \item Our experiments validate the effectiveness of \method. When applied to Qwen2.5-14B-Instruct, \method reduces reasoning tokens by $40\%$ (from 313 to 181) on GSM8K, with less than a $0.4\%$ performance drop.
\end{enumerate}

\section{Introduction}
\label{sec:intro}

Foundational models (FMs)~\cite{zhang2024data, zhou2023comprehensive} have shown remarkable progress in the healthcare domain, enabling professional-like assessment of disease diagnosis, treatment decision-making, and monitoring~\cite{zhang2023text, wang2022medclip, lu2023mi-zero}. 
Examples include LLaVA-Med~\cite{li2023llava}, Med-PaLM Multimodal~\cite{tu2024towards}, and Med-Flamingo~\cite{moor2023med}, have demonstrated their capacity on question answering, medical image analysis, and report generation.
These studies follow a predominant top-down model development strategy that requires upstream developers to collect data and train models for downstream tasks. 
Consequently, the developed model capabilities are heavily dependent on the training data, limiting their generalization performance in diverse clinical scenarios. 
For instance, Med-Gemini~\cite{yang2024advancing} reveals promising general capabilities in report generation while it lags behind state-of-the-art (SoTA) models on classification tasks, especially for out-of-domain applications. 
This indicates that while the generalizability of the foundation model is promising, more solutions are expected to meet the various specialized clinical needs.

To address these challenges, multi-center data centralization becomes essential to enhance model capacity and robustness across varied clinical scenarios~\cite{rajpurkar2022ai}. 
Centralizing distributed data can significantly improve model training and inference performance.
However, the process of medical data storage, transfer, and aggregation among centers requires extra efforts to ensure data security and system interoperability~\cite{bradford2020international}.
Moreover, a growing concern for patient privacy makes large-scale multi-center data sharing particularly challenging. 
While efforts like federated learning~\cite{wen2023survey, li2020review} can achieve good model performance on local data, the need for synchronized system coordination presents significant challenges, as clients are unable to update asynchronously. This limitation greatly restricts the practical capability of such approaches.
As a result, without a flexible collaboration, medical community still struggles to fully utilize the isolated data and local computation resources for comprehensive medical AI model development. 
To address this dilemma, open-source platforms encourage public data sharing and knowledge integration~\cite{markiewicz2021openneuro, zenodo}.
However, these platforms focus solely on raw data sharing while seldom providing collaborative model training or cooperation between different institutions.
Recently, collaborative learning has emerged as a viable approach for enhancing multi-model robustness~\cite{boulemtafes2020review}. 
For instance, software-like model development~\cite{raffel2023building} mimics software engineering practices by introducing structured workflows, enabling merging, version control, and continuous model integration.
Under this design, model ability can be strengthened with incremental knowledge updates similar to the version updating in software development. 

Although collaborative learning provides a multi-model collaboration, two key challenges remain in the leakage of raw data during collaboration~\cite{huang2023lorahub} and the synchronization of multiple collaborators~\cite{mcmahan2017communication} in the medical AI community. It is still challenging to integrate decentralized, privacy-sensitive data across institutions, leading to under-utilized insights and fragmented knowledge sharing~\cite{kaissis2020secure, rajpurkar2022ai, abdullah2021ethics}.
 To address these challenges, inspired by the collaborative software development, we propose \textbf{Med}ical \textbf{Fo}undation Models Me\textbf{rg}ing (\textbf{MedForge}), a cooperative workflow enabling continuously community-driven foundation model (FM) development.
MedForge enables a lightweight manner for individual centers to share their knowledge among multiple centers, minimizing the burden of data transmission and integration while enhancing model robustness.
Meanwhile, MedForge facilitates asynchronous and flexible collaboration, allowing individual centers to continuously update and improve medical FMs without the need for real-time synchronization.
Similar to open-source software development, MedForge incrementally updates medical knowledge and follows a sustainable model development scheme. 
This key design emphasizes a bottom-up construction of a multi-task medical FM, allowing downstream users to collaboratively build, refine, and update the upstream model according to their local resources. Our major contributions of MedForge are as below: 
\begin{enumerate}
    \item[$\bullet$] We introduce a collaborative workflow to promote the merging scheme of open-source software development. Our proposed MedForge allows distributed clinical centers to asynchronously contribute to comprehensive medical model construction while reducing transmitting costs among centers and avoiding the leakage of raw data, thus enhancing the utilization of private resources in the healthcare system. 
    \item[$\bullet$] We propose two effective knowledge-merging strategies for the asynchronous branch contribution. The MedForge-Fusion strategy updates the plugin module parameters of the main model during the merging phase, whereas the MedForge-Mixture strategy integrates the output of the plugin module by memorizing each contributor's coefficient. These strategies make MedForge more flexible and versatile. MedForge-Fusion is friendly to implement, while the MedForge-Mixture offers better performance and robustness.
    \item[$\bullet$]  We comprehensively evaluate model merging strategies to accumulate medical knowledge among multiple branch plugin modules. MedForge yields superior performance on medical classification tasks compared to other collaborative baselines across multiple datasets. We demonstrate the robustness of MedForge by shuffling the task order and evaluating various configurations of plugin modules and dataset distillation methods.
\end{enumerate}




\section{Related Works}
\label{related-works}
\subsection{Automatic Essay Scoring}
\label{automatic-essay-scoring}
% Recent studies in AES can be roughly divided into the following two perspectives.

\paragraph{Feature Engineering} approaches leverage various features to predict essay scores, including linguistic features, e.g., readability metrics and word length ~\cite{ridley2020promptagnosticessayscorer, uto-etal-2020-neural, jin-etal-2018-tdnn, FoltLahaLand19993j, chen-he-2013-automated}, and content features, e.g., content quality and organization ~\cite{mathias-bhattacharyya-2018-asap, crossley2023english}. Models that utilize these features range from simple logistic regression models ~\cite{chen-he-2013-automated} to deep neural networks ~\cite{uto-etal-2020-neural}. These approaches assess the quality of essays in an interpretable manner with well-defined features.

\paragraph{Language-model-based} approaches emerge with the rising popularity of Transformer architecture, including BERT-based methods that require supervised fine-tuning \cite{wang-etal-2022-use, mutlitaskAESforEssayGrading, hierarchicalbert} and LLM-based methods that focus on prompt-engineering \cite{mansour-etal-2024-large, stahl-etal-2024-exploring}. In particular, \cite{stahl-etal-2024-exploring} explores zero-shot prompting with persona prompts and analysis instructions. Building on this, our work aims to utilize linguistic features in LLM prompting.

\subsection{LLM as Evaluator}
\label{llm-as-evaluator}
Given the increasing capability of LLMs and their scalable nature, researchers in various domains have explored how to use them for the automatic evaluation of text content \cite{zubiaga-etal-2024-llm, alhafni-etal-2024-personalized, gao2024llmbasednlgevaluationcurrent, fu-etal-2024-gptscore}. Although some research has shown proper prompt tuning, such as explanation-guided generation, clear rubric guidance, and chain-of-thought (COT) could improve the alignment between human and LLMs \cite{chiang-lee-2023-closer, liu-etal-2023-g, hashemi-etal-2024-llm}, the LLM-based evaluators still perform underwhelming in more complex tasks, such as reviewing papers \cite{zhou-etal-2024-llm} and scoring students essay \cite{mansour-etal-2024-large, stahl-etal-2024-exploring}. In this work, we specifically focus on improving LLMs as student essay graders by incorporating the linguistic features of essays. Additionally, we examine the transferability of the prompts, i.e., how a prompt that is tuned in the in-distribution data would perform out-of-distribution in the same task.

\section{3DMolFormer}
\begin{figure}[t]
    \centering
    \includegraphics[width=\textwidth]{imgs/Main.pdf}
    \vspace{-0.4cm}
    \caption{Overview of 3DMolFormer. The left shows the dual-channel model architecture, the top right illustrates the input and output of the two SBDD tasks in a parallel sequence, and the bottom right outlines the pre-training and fine-tuning process.}
    \label{Overview}
    \vspace{-0.2cm}
\end{figure}

\subsection{Format of Pocket and Ligand Sequences with 3D Coordinates}

To leverage a causal language model for handling 3D protein pockets and small molecules while explicitly separating discrete structural information from continuous spatial coordinates, we design a parallel sequence format. This format consists of a discrete token sequence $s_\mathrm{tok}$ and a continuous numerical sequence $s_\mathrm{num}$, both of which share the same length and align element-wise. The token sequence consists of tokens in a predefined vocabulary, while the numerical sequence contains floating-point values.

As shown in Figure~\ref{PocketSeq}, the sequence for a protein pocket $s^\mathrm{poc}$ consists of two parts: the first $s^\mathrm{poc\_atoms}$ represents an atomic list, and the second $s^\mathrm{poc\_coord}$ contains 3D coordinate information. 
The atomic list is encoded in the token sequence, which includes all atoms in the protein pocket except for hydrogen atoms. Aside from alpha carbon atoms, denoted as 'CA', other atoms are represented by their element type, such as 'C', 'O', 'N', and 'S'. The sequence of atoms follows the order of the pdb file, where each amino acid begins with ['N', 'CA', 'C', 'O'] followed by the side-chain atoms.
The normalized 3D coordinates for each atom in the atomic list are included in the numerical sequence in the same order, with each dimension ('x', 'y', 'z') occupying a separate position. The length of the 3D coordinate sequence is always three times the length of the atomic list.
Moreover, in the token sequence, the start and end of the atomic list are marked by 'PS' and 'PE', while the 3D coordinates are delineated by 'PCS' and 'PCE' at the start and end, respectively.
In the numerical sequence, numbers that do not correspond to 3D coordinates are padded with 1.0.

As illustrated in Figure~\ref{LigandSeq}, the sequence for a small molecule $s^\mathrm{lig}$ is similar to that of the protein pocket, comprising both a SMILES string section $s^\mathrm{lig\_smiles}$ and a 3D coordinate section $s^\mathrm{lig\_coord}$.
After atom-level tokenization~\citep{SMILEStokenization}, the SMILES string of the small molecule is encoded in the token sequence, excluding hydrogen atoms. It is important to note that some tokens may not correspond to atoms, and thus, no 3D coordinates will be associated with them.
The normalized 3D coordinates for each atom in the tokenized SMILES string are included in the numerical sequence, with each coordinate dimension ('x', 'y', 'z') occupying a separate position. The length of the 3D coordinate sequence is always three times the number of atoms in the small molecule.
In the token sequence, the start and end of the SMILES tokens are marked by 'LS' and 'LE', while the 3D coordinates of the corresponding atoms are marked by 'LCS' and 'LCE' at the start and end, respectively.
In the numerical sequence, numbers not corresponding to 3D coordinates are similarly padded with 1.0.


When the sequence of a protein pocket is concatenated with that of a small molecule ligand, it forms a pocket-ligand complex sequence along with their 3D coordinates $s^\mathrm{poc-lig}$. This sequence format offers three advantages:
\begin{itemize}[leftmargin=*]
    \item It fully encapsulates the structural and 3D coordinate information of both the protein pocket and the small molecule ligand.
    \item Discrete structural information and continuous numerical data are separated into two parallel sequences, enabling independent processing of each data type.
    \item The sequence of the pocket-ligand complex maintains causal logic. As depicted in the upper right of Figure~\ref{Overview}, this sequence structure allows autoregressive prediction, which can effectively represent both pocket-ligand docking and pocket-aware drug design tasks.
\end{itemize}

Specifically, we normalize the coordinates of all pocket-ligand complexes by translating their center of mass to the origin $(0,0,0)$. Additionally, to ensure numerical stability during training~\citep{DeepLearning}, we scale the coordinate values by a factor $q>1$ to reduce the range of their distribution:
\begin{equation}
\label{coordnorm}
(x_i',y_i',z_i')=\Big(\frac{x_i-x_c}{q},\frac{y_i-y_c}{q},\frac{z_i-z_c}{q}\Big),
\end{equation}
where $(x_i,y_i,z_i)$ is the original coordinate of the $i$-th atom, $(x_c,y_c,z_c)$ is the coordinate of the center of mass, and $(x_i',y_i',z_i')$ refers to the normalized values used in the numerical sequence.

\subsection{Model Architecture}
To process the aforementioned parallel sequences, we require an autoregressive language model that can simultaneously take a discrete token sequence and a continuous floating-point sequence as input, while predicting both the next token and the next numerical value. Inspired by xVal~\citep{xVal}, we propose a dual-channel transformer architecture for 3DMolFormer, as illustrated in the left part of Figure~\ref{Overview}. The module handling the token sequence is based on the GPT-2 model~\citep{GPT-2}, featuring identical token embeddings, positional embeddings, multiple transformer layers, and a prediction head for logits. On top of this, we introduce a parallel numerical channel at both the input and output stages.

At the input stage, we multiply the embedding of each token in the token sequence with the corresponding value in the numerical sequence, using this product as the input to the positional embedding. This is why numerical values that lack meaningful information are padded with 1.0. At the output stage, in parallel with the token prediction head, we add a number head to predict the next floating-point value.

During inference with 3DMolFormer, the outputs are handled in two modes:
\begin{itemize}[leftmargin=*]
    \item Token Mode: In the drug design task, when predicting ligand SMILES tokens, the corresponding numerical output holds no meaningful value and is therefore padded with 1.0.
    \item Numerical Mode: In docking and drug design tasks, once the ligand SMILES is determined, the length of the 3D coordinate sequence and its tokens are also fixed. Therefore, the token output no longer holds meaningful information and is filled with the expected tokens (from ['x', 'y', 'z', 'LCS', 'LCE']). When the position corresponds to ['x', 'y', 'z'], the predicted floating-point values are appended to the input numerical sequence. For tokens corresponding to ['LCS', 'LCE'], the numerical values are also set to 1.0.
\end{itemize}


\subsection{Self-supervised Pre-training}
To enable the 3DMolFormer model to learn the general patterns of pocket-ligand complex sequences, we conduct large-scale pre-training on 3D data, which includes three datasets: approximately 3.2M protein pockets, about 209M small molecule conformations, and around 167K pocket-ligand complexes. The first two datasets were collected by Uni-Mol~\citep{Uni-Mol} for large-scale pre-training on 3D protein pockets and small molecules, while the last dataset was generated by CrossDocked2020~\citep{CrossDocked}.

In order for the dual-channel autoregressive model to capture both the token sequence format and the 3D coordinate patterns of pocket-ligand complexes, we adopt a composite loss function for the prediction of the next token and the corresponding numerical value. This loss function incorporates the cross-entropy (CE) loss for the whole token sequence and the mean squared error (MSE) loss for the numerical sequence corresponding to the 3D coordinates:
\begin{equation}
\label{pretrainloss}
L(\hat{s}, s)=\mathrm{CE}(\hat{s}_\mathrm{tok}, s_\mathrm{tok})+\alpha\cdot \mathrm{MSE}(\hat{s}_\mathrm{num}^\mathrm{coord}, s_\mathrm{num}^\mathrm{coord}),
\end{equation}
where $\hat{s}$ represents the sequence predicted by 3DMolFormer, $s$ refers to the training data, and $\alpha$ is a coefficient that controls the balance between the CE loss and the MSE loss. This composite loss is applied to all of the three types of pre-training data.

Additionally, we employ large-batch training \citep{large-batch-training} through gradient accumulation, which we found to be crucial for the pre-training stability of 3DMolFormer. For further details on pre-training and hyper-parameter settings, please refer to Section~\ref{experiments} and Appendix~\ref{app2}.




\subsection{Fine-tuning}
After the large-scale pre-training, we further fine-tune the 3DMolFormer model on two downstream drug discovery tasks: supervised fine-tuning for pocket-ligand docking, and reinforcement learning (RL) fine-tuning for pocket-aware drug design.

\subsubsection{Supervised Fine-tuning for Protein-ligand Binding Pose Prediction}
In the protein-ligand binding pose prediction (docking) task, as illustrated in Figure~\ref{Overview}, each sample consists of a pocket-ligand complex. The input sequence contains the atoms of the protein pocket and their 3D coordinates, along with the SMILES sequence of the ligand. The output is the 3D coordinates of each atom in the ligand.

The pre-training data for 3DMolFormer already includes about 167K pocket-ligand complexes from CrossDocked2020~\citep{CrossDocked}; however, these complexes are generated using the docking software Smina~\cite{Smina}, which means that the docking performance of models trained with this data would not exceed that of Smina. To improve the upper limit of our model's docking performance, we fine-tune it on the experimentally determined PDBBind dataset~\cite{PDBbind}, which contains approximately 17K ground-truth pocket-ligand complexes. Additionally, we employ a task-specific loss function that computes the mean squared error (MSE) loss only for the 3D coordinates of the ligand in the context of next numerical value prediction, since the inference process of docking operates entirely in numerical mode:
\begin{equation}
L_{\mathrm{docking}}(\hat{s}^\mathrm{lig\_coord},s^\mathrm{lig\_coord})=\mathrm{MSE}(\hat{s}_\mathrm{num}^\mathrm{lig\_coord}, s_\mathrm{num}^\mathrm{lig\_coord}).
\end{equation}

To mitigate overfitting during supervised fine-tuning, SMILES randomization~\citep{SMILESRandomization} and random rotation of the 3D coordinates of complexes are used as data augmentation strategies. For further details on docking fine-tuning, please refer to Section~\ref{exp-docking} and Appendix~\ref{app3}.

\subsubsection{RL Fine-tuning for Pocket-aware 3D Drug Design}
In the pocket-aware drug design task, as illustrated in Figure~\ref{Overview}, each sample is also a pocket-ligand pair. The input sequence includes the atoms of the protein pocket and their 3D coordinates, while the output consists of the ligand SMILES sequence and the 3D coordinates of its atoms.

Inspired by 1D RL-based molecular generation methods \citep{Reinvent}, an RL agent with the 3DMolFormer architecture is initialized with the pre-trained weights, and a molecular property scoring function for each protein pocket is designed as the RL reward. Then, the agent is iteratively optimized to maximize the expected reward of its outputs. Specifically, at each RL step, the agent samples a batch of 3D ligands, and the regularized maximum likelihood estimation (MLE) loss \citep{SMILES_RL} of each ligand is computed and used to update the agent:
\begin{equation}
\label{RLloss}
L_{\mathrm{design}}(\hat{s}^\mathrm{lig})=\big(\log\pi_\text{pre-trained}(\hat{s}_\mathrm{tok}^\mathrm{lig\_smiles})+\sigma\cdot R(m)-\log\pi_\text{agent}(\hat{s}_\mathrm{tok}^\mathrm{lig\_smiles})\big)^2,
\end{equation}
where $\hat{s}^\mathrm{lig}$ ($\hat{s}^\mathrm{lig\_smiles}$ and $\hat{s}^\mathrm{lig\_coord}$) is a sample generated by the RL agent, $m$ is the 3D molecule represented by $\hat{s}^\mathrm{lig}$, and $R(\cdot)$ is reward function evaluating the property of the molecule. $\pi_\text{pre-trained}(s)$ is the likelihood of the pre-trained 3DMolFormer model for generating the sequence $s$, $\pi_\text{agent}(s)$ is the corresponding likelihood of the agent model, and $\sigma$ is a coefficient hyper-parameter to control the importance of the reward. This loss function encourages the agent to generate molecules with higher expected rewards while retaining a low deviation from the pre-trained weights.

It is important to note that to leverage the duality of the two SBDD tasks, the sampling of ligand SMILES utilizes the weights of the RL agent's model, which are continuously updated during fine-tuning. In contrast, the generation of atomic 3D coordinates uses the weights from the model fine-tuned for docking, which remains unchanged during this process. For additional details on RL fine-tuning and hyper-parameter settings, please refer to Section~\ref{exp-drug-design} and Appendix~\ref{app4}.

\section{Experiment}
We evaluate our proposed method with strong baselines and further analyze contributions of different components, and the impact of key parameters.

\subsection{Experiment Setup}
\textbf{Dataset.}
We evaluate all the methods on Inter-X dataset, which consists about 9K training samples and 1,708 test samples. Each sample is an action-reaction sequence and three corresponding textual description.
As supplementation, we mix our pre-training data with single person motion-text dataset HumanML3D~\citep{humanml3d}, which consists more than 23K annotated motion sequences.
We uniformly sample frames for both datasets to 30 FPS. 

\textbf{Evaluation Metrics.}
Following single-person motion generation~\citep{t2mgpt}, we adopt the these metrics to quantitatively evaluate the generated motion: R-Precision measures the ranking of Euclidean distances between motion and text features. Accuracy (Acc.) assesses how likely a generated motion could be successfully recognized as its interaction label, like ``high-five''. Frechet Inception Distance~\citep{fid} (FID) evaluates the similarity in feature space between predicted and ground-truth motion. Multimodal Distance (MMDist.) calculates the average Euclidean distance between generated motion and the corresponding text description. Diversity (Div.) measures the feature diversity within generated motions. All the metrics reported are calculated with batch size set to 32, and accumulated across the test dataset, and we evaluate each method for 20 times with different seeds to calculate the final results at 95\% confidence interval.

\textbf{Evaluation Model.} \label{sec:eval}
Every metric mentioned above requires an encoder $\mathcal{M}$ to extract motion feature.
For single person text-to-motion generation tasks, a motion-text matching model are commonly trained as human motion feature extractor.
A simple way to transfer this method to interaction domain is to directly train an interaction-to-text matching model $\mathcal{M}(\mathbf{a}, \hat{\mathbf{b}}, text)$, where action sequence $\mathbf{a}$ and predicted reaction sequence $\hat{\mathbf{b}}$ together is regarded as a generated interaction sequence, or a reaction-to-text match model $\mathcal{M}(\hat{\mathbf{b}}, text)$.
However, the former one may focus too much on the ground-truth action input, leading insufficient discriminative power of $\hat{\mathbf{b}}$'s quality, while the latter one lacks semantics provided by action, thus leading to subpar matching capability.

To address the issue, we simply uniformly mask off a large portion of $\mathbf{a}$, obtaining down-sampled action motion sequence $\mathbf{a}'$ (downsampled to 1 FPS in our setting), which serves as a semantic hint for the matching process while not introducing too much emphasis on input action sequence.
The final evaluation model consists of an masked interaction encoder and a text encoder.
We use contrastive loss following CLIP~\citep{clip}, which encourages paired motion and text features to be close geometrically.
In addition, we add a classification head after the predicted motion features, to simultaneously predict interaction labels, such as ``high-five''.

\textbf{Baselines.} To evaluate the performance of our method \ModelAbbr~on online and unconstrained setting, we compare \ModelAbbr~with the following baselines:
1) \textbf{InterFormer}~\citep{interformer} is a transformer based action-to-reaction generation model that leverages human skeleton as prior knowledge for efficient attention process.
2) \textbf{MotionGPT}~\citep{motiongpt} is a motion-language model that leverages an LLM for motion and text generation. We extend the motion tokenizer of MotionGPT to encode multi-person motion, while keeping other settings unchanged.
3) \textbf{InterGen}~\citep{intergen} proposes a mutual attention mechanism within diffusion process for human interaction generation, we reproduce and adapt IngerGen to action-to-reaction generation.
4) \textbf{ReGenNet}~\citep{regennet} is latest state-of-the-art model on action-to-reaction generation. It adopts a transformer decoder based diffusion model, which directly predicts human reaction given action input in unconstrained and online manner as ours.


\textbf{Implementation Details.}
For the LLM, we adopt Flan-T5-base~\citep{flan,t5} as our base model, with extended vocabulary. We warm up the learning rate for 1,000 steps, peaking at 1e-4 for the pre-training phase, and use the same learning rate for fine-tuning.
Both the pre-training and fine-tuning phases are trained on a single machine with 8 Tesla V100 GPUs. The training batch size is set to 32 for the LLM and we monitor the validation loss and reaction generation metrics for early-stopping, resulting about 100K pre-training steps and 40K fine-tuning steps.
We set the re-thinking interval $N_r$ to 4 tokens and divide each space signal into $N_b=10$ bins.

\begin{table}[t]
\centering
\tiny
\caption{Comparison to state-of-the-art baselines and ablation studies of our method on Inter-X dataset. $\uparrow$ or $\downarrow$ denotes a higher or lower value is better, and $\rightarrow$ means that the value closer to real is better. We use $\pm$ to represent 95\% confidence interval and highlight the best results in \textbf{bold}. For ablation methods (in grey), PT, M, P, S, and SP are abbreviations for pre-training, motion, pose, space, and single-person data, respectively.}
\label{tab:main}
\begin{tabular}{l|ccccccc}
\toprule
\multirow{2}{*}{Methods} &  \multicolumn{3}{c}{R-Precision$\uparrow$} & \multirow{2}{*}{Acc.$\uparrow$}& \multirow{2}{*}{FID$\downarrow$}         & \multirow{2}{*}{MMDist$\downarrow$} & \multirow{2}{*}{Div.$\rightarrow$} \\
               & Top-1       & Top-2       & Top-3     &     &         &                               &                       \\ \midrule
Real                    & $0.511^{\pm.003}$ & $0.682^{\pm.002}$ & $0.776^{\pm.002}$ & $0.463^{\pm.000}$  & $0.000^{\pm.000}$         & $5.348^{\pm.002}$         & $2.498^{\pm.005}$           \\ \midrule
InterFormer             & $0.172^{\pm.012}$ & $0.292^{\pm.013}$ & $0.343^{\pm.012}$ & $0.171^{\pm.009}$ & $10.468^{\pm.021}$        &  $7.831^{\pm.018}$         & $3.505^{\pm.023}$           \\
MotionGPT &            $0.238^{\pm.003}$        &     $0.354^{\pm.004}$        &     $0.441^{\pm.003}$   &    $0.186^{\pm.002}$            &     $5.823^{\pm.048}$               &      $6.211^{\pm.005}$           &      $2.615^{\pm.007}$ \\
InterGen                & $0.326^{\pm.036}$ & $0.423^{\pm.063}$ & $0.525^{\pm.053}$ & $0.254^{\pm.019}$  & $5.506^{\pm.257}$         & $6.182^{\pm.038}$         & $2.284^{\pm.009}$           \\
ReGenNet                & $0.384^{\pm.005}$ & $0.483^{\pm.002}$ & $0.572^{\pm.003}$ & $0.297^{\pm.004}$  & $3.988^{\pm.048}$         & $5.867^{\pm.009}$         & $\mathbf{2.502^{\pm.001}}$           \\ \midrule
% \rowcolor[HTML]{EFEFEF}
\ModelAbbr~(Ours)       & $\mathbf{0.423^{\pm.005}}$ & $\mathbf{0.599^{\pm.003}}$ & $\mathbf{0.693^{\pm.003}}$ & $\mathbf{0.318^{\pm.003}}$  & $\mathbf{1.942^{\pm.017}}$         & $\mathbf{5.643^{\pm.003}}$         & $2.629^{\pm.006}$           \\
\rowcolor[HTML]{EFEFEF}
w/o Think           & $0.367^{\pm.003}$ & $0.491^{\pm.027}$ & $0.584^{\pm.008}$ & $0.230^{\pm.036}$ & $3.828^{\pm.016}$         & $6.186^{\pm.055}$         & $2.609^{\pm.006}$           \\
\rowcolor[HTML]{EFEFEF}
w/o All PT.         & $0.398^{\pm.007}$ & $0.531^{\pm.002}$ & $0.628^{\pm.003}$ & $0.288^{\pm.002}$ & $3.467^{\pm.113}$         & $5.822^{\pm.003}$         & $2.909^{\pm.053}$           \\
\rowcolor[HTML]{EFEFEF}
w/o M-M PT. & $0.408^{\pm.005}$ & $0.563^{\pm.004}$ & $0.646^{\pm.005}$ & $0.293^{\pm.002}$ & $2.874^{\pm.020}$         & $5.736^{\pm.003}$         & $2.553^{\pm.006}$           \\
\rowcolor[HTML]{EFEFEF}
w/o P-S PT. & $0.417^{\pm.004}$ & $0.582^{\pm.004}$ & $0.664^{\pm.004}$ & $0.308^{\pm.003}$ & $2.685^{\pm.024}$         & $5.699^{\pm.004}$         & $2.859^{\pm.007}$           \\
\rowcolor[HTML]{EFEFEF}
w/o M-T PT. & $0.406^{\pm.003}$ & $0.557^{\pm.004}$ & $0.637^{\pm.004}$ & $0.304^{\pm.003}$ & $2.580^{\pm.021}$         & $5.822 ^{\pm.003}$         & $2.889^{\pm.005}$           \\
\rowcolor[HTML]{EFEFEF}
w/o SP Data     & $0.414^{\pm.004}$ & $0.592^{\pm.005}$ & $0.685^{\pm.003}$ & $0.315^{\pm.004}$ & $2.007^{\pm.015}$         & $5.667^{\pm.003}$         & $2.611^{\pm.005}$           \\
\bottomrule
\end{tabular}
\end{table}



\begin{figure}
    \centering
    \includegraphics[width=\linewidth]{figs/tsne.pdf}
    \caption{Visualization of a person's motion sequences in Inter-X dataset and HumanML3D dataset.}
    \label{fig:tsne}
\end{figure}

\subsection{Comparison to Baselines}\label{sec:sota}
As shown in the upper side of Table~\ref{tab:main}, our method \ModelAbbr~significantly outperforms baseline methods in terms of ranking, accuracy, FID and multimodal distance, showing superior human reaction generation quality.
Compared to MotionGPT, which adopts a similar motion-language architecture, \ModelAbbr~expresses stronger performance, which we attribute to our unified representation of motion via space and pose tokenizers, enabling effective individual pose and inter-person spatial relationship representation.
\ModelAbbr~also surpasses the diffusion-based methods, InterGen and ReGenNet, with our think-then-react architecture, improving generated motions by describing observed action and reasoning what reaction is expected on semantic level. In addition, ReGenNet and MotionGPT get closer diversity to the real than our model. We mainly attribute to that, \ModelAbbr~may conduct multiple re-thinking processes during inference, and the inferred semantics may bring a higher diversity.


\subsection{Ablation Study of Key Components}
To evaluate the effectiveness of our proposed key designs, we conduct detailed ablation studies by removing each of them to observe how much drop compared to the full version of our \ModelAbbr~method. The larger drop indicates more contribution. The results are shown in gray lines of Table~\ref{tab:main}. According to the drops in FID, all designs, including thinking, pre-training tasks and using single person data in pre-training, have positive contributions to the final performance, and thinking contributes the most. Some detailed findings and analyses are as follows.

First, we skip \textbf{thinking} stage during inference, and find the performance drops significantly in FID from 1.9 to 3.8. This supports the necessity of our proposed thinking process before reacting. We also notice decreasing diversity of generated samples, as the model relies solely on input action, and cannot explicitly capture and infer action's intent, thus leading to more rigid motion in some cases.

Second, to evaluate the effectiveness of \textbf{pre-training}, we omit the pre-training stage, and directly train our model \ModelAbbr~for thinking and reacting tasks. As shown in Table~\ref{tab:main}, our model's performance deteriorates without a fine-grained pre-training phase from 1.9 to 3.4 in FID. This indicates that pre-training can effectively adapt a language model (Flan-T5-base) into a motion and language model. We further removing three kinds of pre-training tasks: motion-motion (M-M PT.), pose-space (P-S PT.), and motion-text (M-T PT.). The results show that the without any task, the performance obviously gets worse, from 1.9 to 2.5 - 2.8 in FID, indicating their positive contribution to the final performance and complementary values to each other.

Third, to see how much \textbf{single-person data} helps reaction generation, we remove single person motion-text data, i.e., the data from HumanML3D dataset, from our training set. The result (w/o SP Data) shows that the model performs worse without training on HumanML3D, which proves that our unified motion encoder and motion-language architecture can leverage both single- and multi-person data, alleviating the insufficiency of training data. However, the benefit from single-person data is not as large as we expect. 

% What's more, we evaluate the necessity of \textbf{decoupled space-motion tokenizer}, and the results are shown in Table~\ref{tab:vqvae}. We design a plain motion VQ-VAE with unnormalized action and reaction as input, maintaining absolute space and pose features. With the trained motion VQ-VAE, we encode action/reaction into tokens, which are then fed into TTR for reaction prediction task. First, without normalized motion as input, the reconstruction FID significantly rises from 0.262 to 0.983, showing deteriorated reconstruction performance due to insufficient utilization of codebook. Second, in the reaction generation phase, TTR's performance drops dramatically, as the badly constructed codebook leads to inaccurate action understanding and reaction prediction, highlighting the necessity of decoupling token representation of space and pose features in multi-person scenario.

\begin{figure}
    \centering
    \includegraphics[width=\linewidth]{figs/case_study.pdf}
    \caption{Visualized cases of our predicted reactions (in green) to input action (in blue) and corresponding thinking results. We also provide a failure case in figure (d), where TTR misunderstands the input action as ``wrestling'', which should be ``embracing''.}
    \label{fig:case_study}
\end{figure}


\subsection{Analysis on Overlapping between Single- and Multi-Person Motions}
To investigate the reason of small contribution from single-person data, we further visualize motion sequences of single-person motion (HumanML3D), two-person action (Inter-X Action) and reaction (Inter-X Reaction) in the same space, as presented in Figure~\ref{fig:tsne}. Specifically, we use t-SNE tool~\cite{tsne} to project motion token sequence features into two-dimension. As shown in Figure~\ref{fig:tsne}, the single- and two-person motion sequences have little overlap. When doing case studies, we find that most two-person motion are unique, e.g., massage and being pulled, and will never be used in single-person motion. Similarly, most single-person motions are unique too, e.g., T-pose, and seldom appear in multi-person interaction. There are only a few overlapped motions, e.g., standing still. In addition, when comparing action and reaction sequences in multi-person interaction, we have some interesting findings. When reactions are close to actions, the motion usually belongs to symmetrical interactions, e.g., pulling or being pulled; whereas, when actions are far from reactions, the motion usually belongs to asymmetrical interaction, e.g., massage.


\subsection{Impact of Down-Sampling Parameter in Matching Model for Evaluation}

As described in Section~\ref{sec:eval}, we propose downsampling action motion sequence to avoid matching models for evaluation pay too much attention to input action rather than output reaction. We conduct an experiment to change the downsampling parameter frame rate and calculate the difference between taking ground-truth action and random action as the input of $\mathcal{M}$, in terms of summed ranking scores (Top-1, Top-2, Top-3 and Acc.). As presented in Figure~\ref{fig:discriminative}, 
difference is lowest when FPS equals to 0, which meaning we only match generated reaction motion with text. It goes up to the peak when FPS equals 1 and quickly goes down to low values, even close to the lowest when FPS is about 15. This indicates that it is necessary to concatenate input action with generated reaction to compose a meaningful interaction in evaluation, otherwise the motion-text matching model cannot well recognize the interaction. However, only 1 FPS is enough. With larger FPS, the matching models will be disturbed by input action rather than the generated reaction. Thus, we choose 1 FPS, corresponding to the largest difference, as our final setting.

\subsection{Impact of Re-thinking Interval}
% Our aim is to generate real-time reaction online, and thus time interval is an important parameter to generation quality. 
We change the re-thinking interval $N_r$ from about 1 to 100 timesteps (about 0.1 to 10 seconds) and observe how it impacts generative quality measure FID. As shown in Figure~\ref{fig:latency}, FID falls down first until $N_r=4$ (about 0.5 second) and then continues rising up. This indicate that the best time interval is about 0.5 second. When the time interval is too short, our \ModelAbbr~model cannot get enough information to re-think what the input action means and will bring some randomness into predicting appropriate reaction. When the time interval gets too long, our \ModelAbbr~model give slow responses to the input action sequences and generates coarse-grained reaction.

We also evaluate the average inference time per step (AITS) with respect to the re-thinking interval. As shown in Figure~\ref{fig:latency}, the inference time significantly decreases as the re-thinking interval increases, eventually converging to approximately 10 milliseconds per step (100 FPS). In our setup, we opt to re-think every four steps, resulting in an inference time of less than 50 milliseconds, which meets the requirements for a real-time system.

\subsection{User Study}
To further evaluate our model qualitatively, we conduct a user study on TTR vs. the latest SOTA method ReGenNet, and the results are shown in Figure~\ref{fig:user_study}. We randomly sample 100 action sequences from Inter-X dataset, which are fed into TTR and ReGenNet to predict reactions, and ask four real human to choose the better ones. It can be seen that TTR surpasses ReGenNet on all the duration range, and the winning rate rises significantly when motion duration is longer. We mainly contribute this to our explicit thinking and re-thinking procedure, which ensures semantics matching and alleviates accumulated errors. 

\begin{figure}[t]
    \centering
    \begin{minipage}[b]{0.3\textwidth}
        \centering
        \includegraphics[width=\textwidth]{figs/discriminative_power.pdf}
        \caption{Impact of input action FPS to summed ranking score differences.}
        \label{fig:discriminative}
    \end{minipage}
    \hfill
    \begin{minipage}[b]{0.35\textwidth}
        \centering
        \includegraphics[width=\textwidth]{figs/latency.pdf}
        \caption{Impact of re-thinking interval to FID and average inference time per step (AITS).}
        \label{fig:latency}
    \end{minipage}
    \hfill
    \begin{minipage}[b]{0.3\textwidth}
        \centering
        \includegraphics[width=\linewidth]{figs/user_study.pdf}
        \caption{User preference between TTR and ReGenNet on different motion duration.}
        \label{fig:user_study}
    \end{minipage}
\end{figure}


\section{Discussion}

\paragraph{Quadratic Programming vs. Logistic Regression.}  
Our formulation estimates the attribute weights $\mathbf{p}$ by transforming the Bradley-Terry loss into a quadratic program. An alternative approach based on logistic regression—which assigns absolute labels of 1 and 0 to win/lose responses—can also be used, as demonstrated by \citep{go2023compositional}. 
We compared these two formulations using Drift attributes in Table~\ref{fig:discussion}. The logistic regression approach proves highly unstable and shows lower performance when training examples are limited. We interpret this instability as follows: preference judgments are inherently relative—what constitutes a winning response in one context might be considered a losing response when compared to an even better alternative. Thus, imposing absolute labels through regression can lead to overfitting, particularly when data are scarce. Our results suggest that approaching preference problems from a relative perspective is crucial for effective preference modeling.
\begin{figure}[ht]
\centering
\includegraphics[trim=7 8 2 2, clip, width=0.65\columnwidth]{figs/discussion.pdf}
\caption{Few-shot preference modeling results for \texttt{user1008} in the PRISM with quadratic programming (QP) and logistic regression (LQ).}
\label{fig:discussion}
\vspace{-5mm}
\end{figure}


\paragraph{Compatible with samplers.}
\label{sec:practical-2}
Autoregressive sampling in LLMs has various decoding strategies at the token-level distribution. Drift steers distributions at the logit level—applying its computations before the softmax—making it compatible with a wide range of sampling methods tailored to different objectives~\citep{vijayakumar2016diverse, fan2018hierarchical, holtzman2019curious}. our analysis indicates that the backbone LLM exhibits an average next-token entropy of about 0.27 bits, which increases to approximately 0.63 bits after applying Drift. While this boost in entropy can substantially enhance generation diversity, it may also increase the likelihood of selecting unreliable tokens. Therefore, we recommend combining Drift with top-p or top-k sampling strategies to control an optimal balance between diversity and reliability.

\paragraph{Practical Implications.}
While traditional RLHF methods may eventually surpass Drift when user data becomes abundant, Drift offers several advantages in practical settings. 
First, conventional reward models struggle with \textit{continual learning}; retraining on an ever-expanding user dataset is impractical. In contrast, Drift can be updated instantly by simply appending new instances to the $W-L$—no retraining required. 
Second, personal preferences often \textit{change more rapidly than general preferences}. Drift’s interpretability allows real-time tracking of preference shifts, enabling dynamic adjustments for improved personalization. 
Third, when collecting additional user annotations, the variance observed in each attribute can inform an \textit{active learning} strategy~\citep{miller2020active} for efficient data collection. These benefits make Drift an attractive complement to existing RLHF pipelines in personalized applications.


\section{Conclusion}

This work analysed the results of evolutionary wrapper approaches using decision tree based models as proxies and compared them with common \gls{FE} techniques on a \gls{HL} detection problem. Three experiments were conducted using the proposed framework, each employing different proxy models.

When comparing the three experiments, an interesting behaviour of the framework was discovered, when changing the proxy model. The \gls{DT} experiment drastically reduced the number of features, while the other models did not. To further reduce the number of features, one could bias the grammar or apply some penalty in the fitness function for the individuals that use a large number of features. This might not change the behaviour when using different models other than a \gls{DT}, but it forcefully reduces the number of features.  

The results confirm that FEDORA can reduce the dimensionality of the data while statistically maintaining baseline performance, in every experiment. The framework consistently outperforms the remaining \gls{FE} methods, with statistical significance and large effect sizes, proving itself as a viable alternative.

The best result obtained is 76.2\% balanced accuracy using an individual from the \gls{RF} experiment, and a \gls{XGB} algorithm as the testing model, using 57 total features (45 Original, 6 Engineered and 6 Complex) out of the 60 original ones. When using the least amount of features, the best result is 72,8\% balanced accuracy using an individual from the \gls{DT} experiment and a \gls{RF} algorithm as the testing model, using a single complex feature.

In future work, exploring the above-mentioned behaviours might be relevant to better understanding them, namely when biasing the grammar or penalizing the use of many features in the fitness function. Concerning the explainability of the FEDORA transformations, researching meaningful grammar operators might prove useful in addressing problem-specific needs. In this case, having logical operators for the boolean features, which have values of "yes" or "no", and the choice of a simple decision algorithm as the proxy, may increase explainability. Additionally, the previous study has identified several areas for future research, yet to be addressed. For instance, comparing the framework with other common and more complex methods and completing the full \gls{ML} pipeline through the use of a method that addresses the \gls{CASH}, such as \cite{assunccao2020evolution}, and comparing it to other full pipeline frameworks, could be beneficial for contextualizing and evaluating the framework within the \gls{AutoML} and \gls{EC} domains. The framework still needs to be analysed with different datasets to properly assess its generalization capabilities.

\section{Limitations}
\label{limitations}

During our experiments, we noted several limitations that future work could expand upon and resolve. Firstly, there is only one open-source and one close-source one. Future work could look at including more LLMs for comparison. Secondly, the prediction target, holistic score, is still hard to interpret, whereas some subsets (7 and 8) of ASAP and the entire set of ELLIPSE do have fine-grained essay score annotations (see examples in Appendix \ref{ellipse-rubric}). Incorporating them into the overall score prediction process would make the overall score more transparent. Thirdly, the persona section of the prompt template mentions ``grade 7 to 10,'' which is the age range for students in ASAP; however, the students in the ELLIPSE dataset are from grades 8 to 12, which might lead to performance differences among those two datasets. Lastly, our datasets have a clear western bias, especially ELLIPSE, which focuses on ESL students in the United States. We believe the community would benefit from more diverse and inclusive datasets.


\input{sections/8-Acknowledgements}

\bibliography{citations/alejandro-citations, citations/joey-citations}
% \bibliography{citations/joey-citations}

\appendix

\pagebreak
\section{ELLIPSE Rubric}
\label{ellipse-rubric}
% \begin{table}[h!]
%     \centering
%     \small
\begin{center}
    \small
    \begin{longtable}{|p{0.12\linewidth}|p{0.15\linewidth}|p{0.15\linewidth}|p{0.15\linewidth}|p{0.15\linewidth}|p{0.15\linewidth}|}
        \hline
        Score \ \ \ \ Category & 5                                                                                                                                                                                                                                                & 4                                                                                                                                                                                                                       & 3                                                                                                                                                                                                                                                                         & 2                                                                                                                                                                                                               & 1                                                                                                                                                                                            \\ \hline
        Overall                       & Native-like facility in the use of language with syntactic variety, Appropriate word choice and phrases; well-controlled text organization; precise use of grammar and conventions; rare language inaccuracies that do not impede communication. & Facility in the use of language with syntactic variety and range of words and phrases; controlled organization; accuracy in grammar and conventions; occasional language inaccuracies that rarely impede communication. & Facility limited to the use of common structures and generic vocabulary; organization generally controlled although connection sometimes absent or unsuccessful; errors in grammar and syntax and usage. Communication is impeded by language inaccuracies in some cases. & Inconsistent facility in sentence formation, word choice, and mechanics; organization partially developed but may be missing or unsuccessful. Communication impeded in many instances by language inaccuracies. & A limited range of familiar words or phrases loosely strung together; frequent errors in grammar (including syntax) and usage. Communication impeded in most cases by language inaccuracies. \\ \hline
        Cohesion                      & Text organization consistently well controlled using a variety of effective linguistic features such as reference and transitional words and phrases to connect ideas across sentences and paragraphs; appropriate overlap of ideas.             & Organization generally well controlled; a range of cohesive devices used appropriately such as reference and transitional words and phrases to connect ideas; generally appropriate overlap of ideas                    & Organization generally controlled; cohesive devices used but limited in type; Some repetitive, mechanical, or faulty use of cohesion use within and/or between sentences and paragraphs.                                                                                  & Organization only partially developed with a lack of logical sequencing of ideas; some basic cohesive devices used but with inaccuracy or repetition.                                                           & No clear control of organization; cohesive devices not present or unsuccessfully used; presentation of ideas unclear.                                                                        \\ \hline
        Syntax                        & Flexible and effective use of a full range of syntactic structures including simple, compound, and complex sentences; There may be rare minor and negligible errors in sentence formation.                                                       & Appropriate use of a variety of syntactic structures, such as simple, compound, and complex sentences; occasional errors or inappropriateness in sentence formation.                                                    & Simple, compound, and complex syntactic structures present although the range may be limited; some apparent errors in sentence formation, especially in more complex sentences.                                                                                           & Some sentence variation used; many sentence structure problems.                                                                                                                                                 & Pervasive and basic errors in sentence structure and word order that cause confusion; basic sentences errors common.                                                                         \\ \hline
        Vocabulary                    & Wide range of vocabulary flexibly and effectively used to convey precise meanings; skillful use of topic-related terms and less common words; rare negligible inaccuracies in word use.                                                          & Sufficient range of vocabulary to allow flexibility and precision; appropriate use of topic-related terms and less common lexical items                                                                                 & Minimally adequate range of vocabulary for the topic; no precise use of subtle word meanings; topic related terms only used occasionally; attempts to use less common vocabulary but with some inaccuracy                                                                 & Narrow range of vocabulary to convey basic and elementary meaning; topic related terms used inappropri ately; errors in word formation and word choice that may distort meanings                                & Limited vocabulary often inappropriately used; limited control of word choice and word forms; little attempt to use topic-related terms                                                      \\ \hline
        Phraseology                   & Flexible and effective use of a variety of phrases, such as idioms, collocations, and lexical bundles, to convey precise and subtle meanings; rare minor inaccuracies that are negligible.                                                       & Appropriate use of a variety of phrases, such as idioms, collocations, and lexical bundles; occasional inaccuracies and colloquialisms.                                                                                 & Evident use of phrases such as idioms, collocations, and lexical bundles but without much variety; some noticeable repetitions and misuses.                                                                                                                               & Narrow range of phrases, such as collocations and lexical bundles, used to convey basic and elementary meaning; many repetitions and /or misuses of phrases.                                                    & Memorized chunks of language, or simple phrasal patterns predominate; many repetitions and misuses of phrases.                                                                               \\ \hline
        Grammar                       & Command of grammar and usage with few or no errors.                                                                                                                                                                                              & Minimal errors in grammar and usage.                                                                                                                                                                                    & Some errors in grammar and usage.                                                                                                                                                                                                                                         & Many errors in grammar and usage.                                                                                                                                                                               & Errors in grammar and usage throughout.                                                                                                                                                      \\ \hline
        Conventions                   & Consistent use of appropriate conventions to convey meaning; spelling, capitalization, and punctuation errors nonexistent or negligible.                                                                                                         & Generally consistent use of appropriate conventions to convey meaning; spelling, capitalizatio n, and punctuation errors few and not distracting.                                                                       & Developing use of conventions to convey meaning; errors in spelling, capitalization, and punctuation that are sometimes distracting.                                                                                                                                      & Variable use of conventions; spelling, capitalization, and punctuation errors frequent and distracting.                                                                                                         & Minimal use of conventions; spelling, capitalizatio n, and punctuation errors throughout.                                                                                                    \\ \hline
    \caption{The ELLIPSE rubric, gotten directly from the original paper.}
    \label{tab:ellipse-rubric}
    \end{longtable}
\end{center}
% \end{table}



\section{Supervised Baseline Details}
\label{supervised-baseline-details}
Our supervised baseline~\cite{wang-etal-2022-use}\footnote{\url{https://github.com/lingochamp/Multi-Scale-BERT-AES}}is a BERT-based architecture comprised of two sub-components---each pretrained BERT models (\cite{devlin})---which analyze three main feature classes: document-, token- and segment-scale features. The first sub-component receives the document- and token-scale features. It is fine-tuned to learn the document-scale feature representation through the \texttt{[CLS]} (start) token\footnote{There can be multiple text segments per essay as their input length is set to 510.} and the token-scale features through the BERT word embeddings. Its output goes through a final max pooling layer to represent the sub-component's score. The segment-scale features are received by the second sub-component, which takes in an essay as a series of segments each of size $k$ (except the last segment, which is smaller). A list of these segment series of varying sizes $k_{i}$ are input into the model sequentially, and a final LSTM and attention and dense pooling layer is used to output the sub-component's score. Lastly, the output from the two sub-components are added together to produce the final holistic score. The model's loss function is additive between mean squared error ($MSE$), cosine similarity ($CS$) and margin ranking loss ($MLR$): $\mathcal{L}_{\mathrm{Total}}(\mathbf{x}, \mathbf{y}) = \alpha \mathcal{L}_{MSE}(\mathbf{x}, \mathbf{y}) + \beta \mathcal{L}_{CS}(\mathbf{x}, \mathbf{y}) + \gamma \mathcal{L}_{MLR}(\mathbf{x}, \mathbf{y})$.

We base our fine-tuning on the authors' available code\footnote{Available upon acceptance.}. The authors only released their fine-tuned model for ASAP prompt 8, so---to obtain models for all prompts---we fine-tuned \texttt{bert-base-uncased} as they did in the original paper. We used our splits of the ASAP dataset for fine-tuning, validation and testing (see \ref{asap-and-asap++}). We fine-tune for 80 epochs, our hyperparameters for $\alpha$, $\beta$ and $\gamma$ were all set to 0.5 and with cosine similarity \texttt{dim=1} and margin ranking loss \texttt{margin=0}. Everything is implemented in PyTorch (\cite{paszke2019}) and HuggingFace (\cite{wolf2019}) using \texttt{google-bert/bert-base-uncased}. We run the test set on the prompt's model with the best loss.


\section{Zero-shot Essay Scoring Prompts}
\label{app-scoring-prompts}
% \subsection{Zero-shot Essay Scoring}
Here are some examples of zero-shot essay scoring prompts. Note that the exact phrasing and wording are not exactly the same as \cite{stahl-etal-2024-exploring} paper. That is because we have failed to reproduce the exact same results in their paper, motivating us to conduct a limited prompt tuning in the \texttt{dev} set of ASAP. To reduce complexity, the tuning is done only in phrasing and formatting, without changing the overall structure of the prompt compared to the original design.

\subsubsection{No Linguistic Feature}
\textit{
You are part of an educational research team analyzing the writing skills of students in grades 7 to 10. You have been given a student's essay and the prompt they responded to.  \\
\#\#\# Essay Prompt: More and more people use computers, but not everyone agrees that this benefits society. Those who support advances in technology believe that computers have a positive effect on people. They teach hand-eye coordination, give people the ability to learn about faraway places and people, and even allow people to talk online with other people. Others have different ideas. Some experts are concerned that people are spending too much time on their computers and less time exercising, enjoying nature, and interacting with family and friends. Write a letter to your local newspaper in which you state your opinion on the effects computers have on people. Persuade the readers to agree with you.  \\
}
\textit{
\#\#\# Analysis Task: Grade the given essay with the following requirements:  \\
- Use those score ranges: Overall: from 1 to 6.  \\
- Provide an explanation for your score as well.  \\
}
\textit{
\#\#\# Analyzed Student Essay: Dear, @CAPS1 @CAPS2 @CAPS3 More and more people use computers, but not everyone agrees that this benefits society. Those who support advances in technology believe that computers have a positive effect on people. Others have different ideas. A great amount in the world today are using computers, some for work and spme for the fun of it. Computers is one of mans greatest accomplishments. Computers are helpful in so many ways, @CAPS4, news, and live streams. Don't get me wrong way to much people spend time on the computer and they should be out interacting with others but who are we to tell them what to do. When I grow up I want to be a author or a journalist and I know for a fact that both of those jobs involve lots of time on time on the computer, one @MONTH1 spend more time then the other but you know exactly what @CAPS5 getting at. So what if some expert think people are spending to much time on the computer and not exercising, enjoying natures and interacting with family and friends. For all the expert knows that its how must people make a living and we don't know why people choose to use the computer for a great amount of time and to be honest it's non of my concern and it shouldn't be the so called experts concern. People interact a thousand times a day on the computers. Computers keep lots of kids of the streets instead of being out and causing trouble. Computers helps the @ORGANIZATION1 locate most wanted criminals. As you can see computers are more useful to society then you think, computers benefit society. \\
}
\textit{
\#\#\# Analysis: Conclude your analysis with a grade and comments in the following format:  \\
\#\#\# Explanation:  \\
\#\#\# Score: \\
- Overall:
}


\subsubsection{Top-10 Features}
\textit{
You are part of an educational research team analyzing the writing skills of students in grades 7 to 10. You have been given a student's essay and the prompt they responded to.  \\
\#\#\# Essay Prompt: More and more people use computers, but not everyone agrees that this benefits society. Those who support advances in technology believe that computers have a positive effect on people. They teach hand-eye coordination, give people the ability to learn about faraway places and people, and even allow people to talk online with other people. Others have different ideas. Some experts are concerned that people are spending too much time on their computers and less time exercising, enjoying nature, and interacting with family and friends. Write a letter to your local newspaper in which you state your opinion on the effects computers have on people. Persuade the readers to agree with you.  \\
}
\textit{
\#\#\# Analysis Task: Grade the given essay with the following requirements:  \\
- Use those score ranges: Overall: from 1 to 6.  \\
- Provide an explanation for your score as well.  \\
}
\textit{
\#\#\# Analyzed Student Essay: Dear, @CAPS1 @CAPS2 @CAPS3 More and more people use computers, but not everyone agrees that this benefits society. Those who support advances in technology believe that computers have a positive effect on people. Others have different ideas. A great amount in the world today are using computers, some for work and spme for the fun of it. Computers is one of mans greatest accomplishments. Computers are helpful in so many ways, @CAPS4, news, and live streams. Don't get me wrong way to much people spend time on the computer and they should be out interacting with others but who are we to tell them what to do. When I grow up I want to be a author or a journalist and I know for a fact that both of those jobs involve lots of time on time on the computer, one @MONTH1 spend more time then the other but you know exactly what @CAPS5 getting at. So what if some expert think people are spending to much time on the computer and not exercising, enjoying natures and interacting with family and friends. For all the expert knows that its how must people make a living and we don't know why people choose to use the computer for a great amount of time and to be honest it's non of my concern and it shouldn't be the so called experts concern. People interact a thousand times a day on the computers. Computers keep lots of kids of the streets instead of being out and causing trouble. Computers helps the @ORGANIZATION1 locate most wanted criminals. As you can see computers are more useful to society then you think, computers benefit society. \\
}
\textit{
\#\#\# Additional Information: Studies show that the following features are highly, positively correlated with the grade of the essay (i.e., higher features typically means higher end score)  \\
- total number of unique words in the essay: 113  \\
- total number of words in the essay.: 279  \\
- total number of sentences present: 14  \\
- total number of characters: 279  \\
- total number of lemma: 133  \\
- total number of nouns: 50  \\
- total number of stopwords: 71  \\
- total number of words that are not in the Dale-Chall word list of 3000 words recognized by 80\% of fifth graders: 80  \\
- total number of characters: 1229  \\
}
\textit{
\#\#\# Analysis: Conclude your analysis with a grade and comments in the following format:  \\
\#\#\# Explanation:  \\
\#\#\# Score: \\
- Overall:
}

\section{Linguistic Features}
\label{app:linguistic-features}
\textbf{Unique words} refers to the number of single-instance words in the essay. For \textbf{essay character length}, we only count the number of non-space, non-punctuation characters. Words are not normalized before these metrics as in the original paper. \textbf{Total word count} and \textbf{total sentence count} per essay are gotten via \texttt{nltk} (\cite{loper2002}) tokenizers. We additionally utilize the \texttt{en\_core\_web\_sm} in \texttt{spaCy} (\cite{honnibal2020}) to get \textbf{separate counts for lemma, noun, and stop-words.} Finally, we get \textbf{the Dale-Chall} (\cite{dale1948}) \textbf{word count}, \textbf{total character count} and \textbf{long word count} with the \texttt{readability}\footnote{\url{https://pypi.org/project/readability/}} Python package.

Our implementation is based on the code \footnote{\url{https://github.com/robert1ridley/cross-prompt-trait-scoring/blob/main/features.py}} from the original paper \cite{ridleyAutomatedCrosspromptScoring2021}. Our implementation will be made available upon acceptance.
\section{Parsing Module}
\label{app-parsing}
\subsection{Configurations}
\begin{itemize}
    \item Model: Mistral-7B (the same configuration as the scoring model)
    \item Overall parsing error is less than 7\%.
\end{itemize}

\subsection{Few-shot Output Parsing}
\textit{
    You are an AI agent that specialized in converting text input into JSON format.\\
    Instruction: \\
    - Input: text with one or more score and some other relevant information (e.g., explanation, feedbacks, etc.)\\
    - Output: JSON text with `Score' as a mandatory key and other information organized by their field names\\
    - Make sure ONLY return the VALID JSON data, without any additional text or characters.\\
    Here are some examples
}\\

\textit{
Example Input:\\
\#\#\# Explanation: The student's essay demonstrates a limited understanding of the topic and a lack of cohesion. The essay jumps from one idea to another without a clear connection between them. The writing is also filled with numerous grammatical errors, misspellings, and inconsistent capitalization. \\
\#\#\# Score:\\
- Overall: 1 The essay demonstrates a very limited understanding of the topic and contains numerous errors in grammar, spelling, and capitalization. The writing lacks cohesion and a clear thesis statement, and the arguments are not well-supported with evidence or examples. \\
Example Output:\\
\{
    ``Score'': \{
        ``Overall'': 1
    \},
    ``Explanation'': ``The student's essay demonstrates a limited understanding of the topic and a lack of cohesion. The essay jumps from one idea to another without a clear connection between them. The writing is also filled with numerous grammatical errors, misspellings, and inconsistent capitalization.''
\}
}\\

\textit{
Example Input:\\
\#\#\# Explanation: The student's essay demonstrates a basic understanding of the topic and presents a clear position, but the writing is disorganized and contains numerous errors in language conventions. The essay jumps between discussing censorship in libraries and specific examples of offensive music, making it difficult to follow the main argument. \\
\#\#\# Score: \\
- Writing Applications: 2 The essay presents a viewpoint on the issue of censorship, but the argument is not well-developed or clearly stated. The student uses personal experiences and examples. 
- Language Conventions: 1 The essay contains numerous errors in language conventions, including incorrect capitalization, punctuation, and sentence structure. \\
Example Output:\\
\{
    ``Score'': \{
        ``Writing Applications'': 2,
        ``Language Conventions'': 1
    \}
    ``Explanation'': ``The student's essay demonstrates a basic understanding of the topic and presents a clear position, but the writing is disorganized and contains numerous errors in language conventions. The essay jumps between discussing censorship in libraries and specific examples of offensive music, making it difficult to follow the main argument.''
\} 
}\\

\textit{
Example Input:\\
\#\#\# Explanation: The student's essay demonstrates a moderate level of awareness of the audience, as they address the readers directly and use a conversational tone. \\
\#\#\# Feedbacks: the essay could have been more effective if the student had used more formal language and addressed specific concerns of the local community regarding the overuse of computers. \\
\#\#\# Score: \\
- Overall: 3 The student's essay shows some awareness of the audience, but there is room for improvement in terms of language and organization. The essay could benefit from more specific examples and a clearer, more focused argument. \\
Example Output:
\{
    ``Score'': \{
        ``Overall'': 3
    \},
    ``Explanation'': ``The student's essay demonstrates a moderate level of awareness of the audience, as they address the readers directly and use a conversational tone.'',
    ``Feedbacks'': "the essay could have been more effective if the student had used more formal language and addressed specific concerns of the local community regarding the overuse of computers.''
\} 
}\\
\textit{
Now work on the following input:\\
Input:\\
\{LLM OUTPUT\} \\
Output:
% \end{quote}
}

\end{document}
