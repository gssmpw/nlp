\section{Parsing Module}
\label{app-parsing}
\subsection{Configurations}
\begin{itemize}
    \item Model: Mistral-7B (the same configuration as the scoring model)
    \item Overall parsing error is less than 7\%.
\end{itemize}

\subsection{Few-shot Output Parsing}
\textit{
    You are an AI agent that specialized in converting text input into JSON format.\\
    Instruction: \\
    - Input: text with one or more score and some other relevant information (e.g., explanation, feedbacks, etc.)\\
    - Output: JSON text with `Score' as a mandatory key and other information organized by their field names\\
    - Make sure ONLY return the VALID JSON data, without any additional text or characters.\\
    Here are some examples
}\\

\textit{
Example Input:\\
\#\#\# Explanation: The student's essay demonstrates a limited understanding of the topic and a lack of cohesion. The essay jumps from one idea to another without a clear connection between them. The writing is also filled with numerous grammatical errors, misspellings, and inconsistent capitalization. \\
\#\#\# Score:\\
- Overall: 1 The essay demonstrates a very limited understanding of the topic and contains numerous errors in grammar, spelling, and capitalization. The writing lacks cohesion and a clear thesis statement, and the arguments are not well-supported with evidence or examples. \\
Example Output:\\
\{
    ``Score'': \{
        ``Overall'': 1
    \},
    ``Explanation'': ``The student's essay demonstrates a limited understanding of the topic and a lack of cohesion. The essay jumps from one idea to another without a clear connection between them. The writing is also filled with numerous grammatical errors, misspellings, and inconsistent capitalization.''
\}
}\\

\textit{
Example Input:\\
\#\#\# Explanation: The student's essay demonstrates a basic understanding of the topic and presents a clear position, but the writing is disorganized and contains numerous errors in language conventions. The essay jumps between discussing censorship in libraries and specific examples of offensive music, making it difficult to follow the main argument. \\
\#\#\# Score: \\
- Writing Applications: 2 The essay presents a viewpoint on the issue of censorship, but the argument is not well-developed or clearly stated. The student uses personal experiences and examples. 
- Language Conventions: 1 The essay contains numerous errors in language conventions, including incorrect capitalization, punctuation, and sentence structure. \\
Example Output:\\
\{
    ``Score'': \{
        ``Writing Applications'': 2,
        ``Language Conventions'': 1
    \}
    ``Explanation'': ``The student's essay demonstrates a basic understanding of the topic and presents a clear position, but the writing is disorganized and contains numerous errors in language conventions. The essay jumps between discussing censorship in libraries and specific examples of offensive music, making it difficult to follow the main argument.''
\} 
}\\

\textit{
Example Input:\\
\#\#\# Explanation: The student's essay demonstrates a moderate level of awareness of the audience, as they address the readers directly and use a conversational tone. \\
\#\#\# Feedbacks: the essay could have been more effective if the student had used more formal language and addressed specific concerns of the local community regarding the overuse of computers. \\
\#\#\# Score: \\
- Overall: 3 The student's essay shows some awareness of the audience, but there is room for improvement in terms of language and organization. The essay could benefit from more specific examples and a clearer, more focused argument. \\
Example Output:
\{
    ``Score'': \{
        ``Overall'': 3
    \},
    ``Explanation'': ``The student's essay demonstrates a moderate level of awareness of the audience, as they address the readers directly and use a conversational tone.'',
    ``Feedbacks'': "the essay could have been more effective if the student had used more formal language and addressed specific concerns of the local community regarding the overuse of computers.''
\} 
}\\
\textit{
Now work on the following input:\\
Input:\\
\{LLM OUTPUT\} \\
Output:
% \end{quote}
}