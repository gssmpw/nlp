\section{Linguistic Features}
\label{app:linguistic-features}
\textbf{Unique words} refers to the number of single-instance words in the essay. For \textbf{essay character length}, we only count the number of non-space, non-punctuation characters. Words are not normalized before these metrics as in the original paper. \textbf{Total word count} and \textbf{total sentence count} per essay are gotten via \texttt{nltk} (\cite{loper2002}) tokenizers. We additionally utilize the \texttt{en\_core\_web\_sm} in \texttt{spaCy} (\cite{honnibal2020}) to get \textbf{separate counts for lemma, noun, and stop-words.} Finally, we get \textbf{the Dale-Chall} (\cite{dale1948}) \textbf{word count}, \textbf{total character count} and \textbf{long word count} with the \texttt{readability}\footnote{\url{https://pypi.org/project/readability/}} Python package.

Our implementation is based on the code \footnote{\url{https://github.com/robert1ridley/cross-prompt-trait-scoring/blob/main/features.py}} from the original paper \cite{ridleyAutomatedCrosspromptScoring2021}. Our implementation will be made available upon acceptance.