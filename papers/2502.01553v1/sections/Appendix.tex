\section{Appendix}

\subsection{Sample Images of VTubers \& Bullet Chats}

\begin{figure}[h!]
    \centering
    \includegraphics[width=\linewidth]{figs/Nijisanji_2022.png}
    \caption{VTubers (as in 2D avatars) affiliated with Nijisanji, a Japanese VTuber agency. The image is from fandom wiki. \url{https://virtualyoutuber.fandom.com/wiki/NIJISANJI}}
    \label{fig:nijisanji}
\end{figure}

\begin{figure}[h!]
    \centering
    \includegraphics[width=\linewidth]{figs/nanako.png}
    \caption{Screenshot of a livestreaming session on Bilibili of \zh{菜菜子Nanako} (\url{https://space.bilibili.com/595407557}), a Chinese VTuber. The purple indicates the VTuber 2D avatar. The avatar is automatically animated, in-line with the human operator's movements. The orange indicates (some of) the bullet chats sent by the viewers. The red indicates the listing view of the bullet chats. The blue indicates the superchat.}
    \label{fig:nanako}
\end{figure}

\newpage

\subsection{Data Description}
\label{subsec:appendix_data_description}

\begin{table}[h!]
\centering \small
\begin{tabular}{|l|l|L{14em}|}
\hline
\textbf{Field}           & \textbf{Type}    & \textbf{Description}                                                             \\ \hline
\texttt{uId}             & Integer          & Unique identifier for the streamer.                                                   \\ \hline
\texttt{uName}           & String           & Name of the streamer.                                \\ \hline
\texttt{liveId}          & String           & Unique identifier for the live session.                                           \\ \hline
\texttt{parentArea}      & String           & The parent category or area of the live session.                                  \\ \hline
\texttt{area}            & String           & The specific area or category of the live session.                                \\ \hline
\texttt{coverUrl}        & String           & URL of the cover image for the live session.                                      \\ \hline
\texttt{startDate}       & Integer (Epoch)  & Start time of the live session, represented as a Unix timestamp.                  \\ \hline
\texttt{stopDate}        & Integer (Epoch)  & Stop time of the live session, represented as a Unix timestamp.                   \\ \hline
\texttt{title}           & String           & Title of the live session.                                                        \\ \hline
\end{tabular}
\caption{Description of data fields for live session}
\label{table:data_description}
\end{table}


\begin{table}[h!]
\centering \small
\begin{tabular}{|l|l|L{14em}|}
\hline
\textbf{Field}           & \textbf{Type}             & \textbf{Description}                                                             \\ \hline
\texttt{uId}             & Integer                   & Unique identifier for the user who sent the interaction.                             \\ \hline
\texttt{uName}           & String                    & Name of the user who sent the interaction.                                           \\ \hline
\texttt{type}            & Integer                   & Type of the interaction.                                                             \\ \hline
\texttt{sendDate}        & Integer (Epoch)           & Time when the interaction was sent, represented as a Unix timestamp.                 \\ \hline
\texttt{message}         & String                    & Content of the interaction message.                                                  \\ \hline
\texttt{price}           & Integer                   & Price associated with the interaction, if any.                                       \\ \hline
\texttt{count}           & Integer                   & Count associated with the interaction, if any.                                       \\ \hline
\end{tabular}
\caption{Description of data fields for viewer interaction}
\label{table:danmakus_description}
\end{table}


\subsection{Ethical Considerations}
\label{subsec:appendix_ethics}

All data used in this paper is publicly available, and accessible to all Bilibili users. We collect bullet chats and other publicly displayed messages within live streaming sessions on the Bilibili platform. This information is intended to be openly visible for any viewer. We record the sender's anonymous ID, the content of the message, and the corresponding timestamp.
% 
It is important to note that we do not, nor are we able to, associate the user IDs on Bilibili with real personal identities. We do not analyze individual users but rather aggregate the data to gain a high-level understanding.
% 
We use \texttt{danmakus.com} to retrieve live streaming session data. The website states that the data can be used for research purposes, provided the source is acknowledged. We have IRB approval.

\newpage

\subsection{Activity Metrics}
\label{subsec:appendix_activity_metrics}

\pb{Viewing Activity Regarding the Membership-ed VTuber.}
We use the following metrics:

\begin{itemize}[leftmargin=*]
    \item \textbf{Rate of the VTuber's livestream sessions watched:} Number of the membership-ed VTuber's livestream sessions watched by the user, divided by the total number of the VTuber's livestream sessions.
    \item \textbf{Rate of the VTubers's livestream sessions watched on-time:} Number of the membership-ed  VTuber's livestream sessions watched on-time by the user, divided by the number of the VTuber's livestream sessions watched by the user. 
    If the user enters the livestream session within 10 minutes of its start, we classify this as ``on-time''.
    \item \textbf{Proportion of time of the VTuber's livestream sessions watched:} The total time that the user spends watching the membership-ed VTuber's livestream sessions divided by the total time of the VTuber's livestream sessions.
    We estimate the watch time for each session by subtracting the earliest timestamp of the user's activity in that session from the latest timestamp of their activity.
\end{itemize}

\pb{Platform-wide Viewing Activity.}
We analyze the platform-wide viewing behavior of users by using the following three metrics:

\begin{itemize}[leftmargin=*]
    \item \textbf{Active days:} Number of active days of the user.
    \item \textbf{livestream sessions watched:} Number of livestream sessions watched by the user.
    \item \textbf{Streamers watched:} Number of streamers watched by the user.
\end{itemize}
% 

Figure \ref{fig:rq1_appendix} presents the CDFs of the specified metrics over a 45-day period for both members (pre \& post) and non-members. For non-members, we combine the two 45-day periods as they are almost identical. This practice also applies to all other subsequent figures in the paper unless otherwise specified.
Contrary to our initial hypothesis, we see there is no significant difference in viewing activity between members and non-members.


\begin{figure}[h!]
    \centering
    \includegraphics[width=\linewidth]{figs/rq1_appendix.pdf}
    \caption{The CDF of the metrics for platform-wide viewing activity of the members and non-members.}
    \label{fig:rq1_appendix}
\end{figure}


\pb{Bullet Chat Activity.}
We use the following measures:


\begin{itemize}[leftmargin=*]
    \item \textbf{Number of chats:} Total number of chats sent by the user.
    \item \textbf{Number of chats per the VTuber's livestream session:} The total number of chats sent by the user to the membership-ed VTuber's livestream sessions, divided by the number of VTuber's livestream sessions watched.
    
\end{itemize}

\pb{Gift \& Superchat Activity.}
We use the following measures:

\begin{itemize}[leftmargin=*]
    \item \textbf{Number of gifts and superchats:} Total number of gifts and superchats sent by the user
    \item \textbf{Value of gifts and superchats:} Total monetary value of gifts and superchats sent by the user.
    \item \textbf{Number of gifts and SCs to the VTuber:}
    Total number of gifts and SCs sent to the membership-ed VTuber by the user. 
    \item \textbf{Value of gifts and SCs to the VTuber:}
    Total monetary value of gifts and SCs sent to the membership-ed VTuber by the user.
\end{itemize}


\subsection{Chi-square Test of Independence}
\label{subsec:appendix_chi_square}

We employ the Chi-square test of independence to investigate the relationship between the decline in six engagement metrics and membership renewal.
% 
The analysis utilizes the same metrics as detailed in Section \ref{subsec:rq1_temporal}. Specifically, we calculate the metrics for each 15-day interval, starting from 45, 30, and 15 days before the user became a member (T-45, T-30, and T-15), and then for the periods of 0, 15, and 30 days after the user became a member (T+0, T+15, and T+30).

To analyze the decrease, we subtract the metrics at T+30 from those at T+0. We then categorize the results into three groups: an increase is denoted by 1, a decrease by -1, and no change by 0. Finally, we calculate the Point-Biserial Correlation Coefficient to examine the correlation between the decrease in the six metrics and membership renewal, respectively. The results are presented in Table \ref{table:metrics_correlation}.

\begin{table}[h]
\centering
\small
\begin{tabular}{rrr}
\hline\hline
{}                 & \textbf{Chi-square Statistic}               & \textbf{p-value}                  \\ 
\textbf{Metric}             & {}                      & {}                       \\  
\hline
Live Watch Rate & 1288.17 & $< 10^{-280}$ \\
On Time Rate & 3014.12 & $< 10^{-280}$ \\
Live Watch Time & 3916.37 & $< 10^{-280}$ \\
Chat per Live & 6892.15 & $< 10^{-280}$ \\
Gift \& SC to the VTuber & 8090.92 & $< 10^{-280}$ \\
Gift \& SC Value to the VTuber & 3627.77 & $< 10^{-280}$ \\
\hline\hline
\end{tabular}
\caption{Chi-square test of independence of the decrease of engagement metrics and membership renewal.}
\label{table:metrics_correlation}
\end{table}


\subsection{Topic Analysis of Members' Chat Messages}
\label{subsec:appendix_similarity_topic_analysis}

\pb{Methodology.}
We employ BerTopic with text embeddings (embeddings delineated in Section~\ref{subsec:nlp_methods}). For each VTuber, we construct models of the topics present within members' chat messages over distinct 15-day intervals. These intervals commence at 45, 30, and 15 days preceding the date when a user joins as a member (denoted as T-45, T-30, and T-15, respectively), and extend to include the periods of 0, 15, and 30 days subsequent to membership commencement (denoted as T+0, T+15, and T+30, respectively). We restrict the topic count to a maximum of twenty.

\pb{Findings.}
We discern an uptrend in the prevalence of five specific categories of topics across the most of the top 100 VTubers.

\begin{itemize}[leftmargin=*]
    \item \textbf{Endearments for the VTuber (100\% of the investigated VTubers)}: Fans invariably use a term of endearment when referring to the VTuber. 
    \item \textbf{Self-referential Terms (100\% of the investigated VTubers)}: Similarly, there exists a specific term that the VTuber uses to address their fans, which, in turn, becomes the term by which the fans identify themselves. 
    \item \textbf{Prescribed Viewer Reactions (98\% of the investigated VTubers)}: There are established conventions dictating how viewers should react to certain events about the VTuber. For example, these conventions may apply when the VTuber wins or loses a game, or engages in humorous or silly behavior, etc. 
    \item \textbf{Slang-Associated Entities (95\% of the investigated VTubers)}: This category encompasses entities closely tied to the VTuber, which can include another streamer, real-life or virtual friends, family members, pets, or places associated with the VTuber, etc. These entities often have various slang terms linked to them. 
    \item \textbf{Abbreviations Using Pinyin Initials (88\% of the investigated VTubers)}: Often, abbreviations are used that consist solely of the initial letters in Pinyin. Regrettably, in the majority of cases, we are unable to decipher these abbreviations. However, we hypothesize that they typically pertain to one of the aforementioned four categories. 
\end{itemize}


\subsection{Topic Analysis of Toxic Chat Messages}
\label{subsec:appendix_toxic_topic_analysis}

\pb{Methodology.}
We employ BerTopic with text embeddings (embeddings delineated in \S\ref{subsec:nlp_methods}). 
For the three categories, sexual, harassment, and violence, we construct models of the topics present within members' chat messages in the category. We then use the embeddings strategy\footnote{\url{https://maartengr.github.io/BERTopic/getting\_started/outlier\_reduction/outlier\_reduction.html\#embeddings}} to reduce the outliers. 

We list the top 10 topics for sexual, harassment, and violence chat messages in Tables \ref{tab:toxic_sexual}, \ref{tab:toxic_harassment}, and \ref{tab:toxic_violence}.
\textcolor{blue}{\textbf{Caution: For transparency, we present each topic as it is. Therefore, the tables may contain words that some may find disturbing or offensive.}}
In each category, the top 10 topics encompass the 94.6\%, 96.9\%, and 85.3\% of the chat messages, respectively.

\begin{figure}[h!]
    \centering
    \includegraphics[width=\linewidth]{figs/rq3_2_v2.pdf}
    \caption{Permutation feature importance for Histogram-based Gradient Boosting and Random Forest, with importance values scaled to a range of 0-1 using min-max scaling.}
    \label{fig:rq3_importance}
\end{figure}

\newpage

\subsection{Features}
\label{subsec:appendix_features}
\begin{table}[h!]
    \centering
    \small
    \begin{tabular}{rrrr}
    \hline\hline
         {} & \textbf{Ref.} & \textbf{Measured} & \textbf{Diff.} \\
         \textbf{Feature} & {} & {} & {} \\
    \hline
        Chat Sent & \S\ref{subsec:rq1_compare} & T-30 \& T-15$^1$ & None \\
        Gift \& SC & \S\ref{subsec:rq1_compare} & T-30 \& T-15 & None \\
        Gift \& SC (in CNY) & \S\ref{subsec:rq1_compare} & T-30 \& T-15 & None \\
        Live Watch Rate & \S\ref{subsec:rq1_compare} & T-30 \& T-15 & T-45 \vs T-15$^2$ \\
        Live Watch On-time Rate & \S\ref{subsec:rq1_compare} & T-30 \& T-15 & T-45 \vs T-15 \\
        Live Watch Time & \S\ref{subsec:rq1_compare} & T-30 \& T-15 & T-45 \vs T-15 \\
        Chat Per Live & \S\ref{subsec:rq1_compare} & T-30 \& T-15 & T-45 \vs T-15 \\
        Gift \& SC to the Vtb. & \S\ref{subsec:rq1_compare} & T-30 \& T-15 & T-45 \vs T-15 \\
        Gift \& SC (in CNY) to the Vtb. & \S\ref{subsec:rq1_compare} & T-30 \& T-15 & T-45 \vs T-15 \\
        ChatSim (Average)$^3$ & \S\ref{subsec:rq2_similarity} & T-30 \& T-15 & T-45 \vs T-15 \\
        Sexual (Binary)$^4$ & \S\ref{subsec:rq2_toxicity} & T-30 \& T-15 & None \\
        Harassment (Binary) & \S\ref{subsec:rq2_toxicity} & T-30 \& T-15 & None \\
        Violence (Binary) & \S\ref{subsec:rq2_toxicity} & T-30 \& T-15 & None \\
    \hline\hline
    \end{tabular}
    \\
    \begin{flushleft}
        1. Result measured \{from T-30 to T-15\} and result measured \{from T-15 to T-0\}. The current livestreaming session is not included. \\
        2. Calculated by subtracting the result measured \{from T-15 to T-0\} from the result measured \{from T-45 to T-30\}. \\
        3. Calculated by taking the average ChatSim for all live streaming sessions of the VTuber throughout the measurement period. \\
        4. A boolean value that is True if the user send such chats to the VTuber during the measurement period.
    \end{flushleft}
    \caption{Features used for machine learning models.}
    \label{tab:rq3_feature}
\end{table}



\subsection{Model Hyperparameters}
\label{subsec:appendix_parameters}

\begin{table}[h!] 
    \centering 
    \small
    \begin{tabular}{rR{23em}} 
    \hline\hline
    \textbf{Algorithm} & \textbf{Parameters} \\ 
    \hline 
        LiR & - \\ 
        LoR & penalty=L2, C=1.0 \\ 
        RF & n\_estimators=200, max\_depth=16 \\ 
        HGB & learning\_rate=0.1, max\_iter=100, max\_depth=None \\ 
        KNN & n\_neighbors=50, leaf\_size=30, p=2, metric=minkowski \\ 
    \hline\hline
    \end{tabular} 
    \caption{Hyperparameters Used for Each Model.} 
    \label{tab:classifier_parameters} 
\end{table}


\subsection{Feature Importance}

Figure \ref{fig:rq3_importance} presents the feature importance of the two top performing FanRanker models: Random Forest and Histogram Based Gradient Boosting.

\vspace{10em}


\begin{table*}[]
    \centering
    \begin{tabular}{crrR{35em}}
        \hline\hline
            & \% & Keyword & Explanation \\
            Rank &  &  &  \\
        \hline
            1 & 25.4\% & \zh{亲} (Kiss) & - \\
            2 & 24.3\% & \zh{舔} (Lick) & In a sexual context \\
            3 & 10.8\% & \zh{屁} (Fart, Butt) & - \\
            4 & 8.2\% & \zh{舔、摸} (Lick, Touch) & In a sexual context \\
            5 & 7.8\% & \zh{汁} (Juice) & In a sexual context \\
            6 & 7.3\% & \zh{臀、屁股} (Hip, Butt) & - \\
            7 & 4.8\% & \zh{熊、奈子} (Breast) & Slangs \\
            8 & 2.6\% & \zh{仙女棒} (Vibrator) & Slangs \\
            9 & 2.0\% & \zh{裸} (Naked) & - \\
            10 & 1.4\% & \zh{妈} (Mommy) & - \\
        \hline\hline
    \end{tabular}
    \caption{Topics for toxic chat messages in the sexual category.}
    \label{tab:toxic_sexual}
\end{table*}

\begin{table*}[]
    \centering
    \begin{tabular}{crrR{35em}}
        \hline\hline
            & \% & Keyword & Explanation \\
            Rank &  &  &  \\
        \hline
            1 & 40.8\% & \zh{可恶、你小子} (Bastard) & More likely to be playful or teasing instead of Harassment \\
            2 & 21.6\% & \zh{笨} (Stupid) & Traditionally considered insulting, but it can also be playful or teasing with less hostility, depending on the context \\
            3 & 16.3\% & \zh{傻} (Stupid) & Insulting, but its hostility has diminished over time, unless used in combination with more aggressive words \\
            4 & 5.1\% & \zh{臭} (Stinky) & More likely to be playful or teasing instead of Harassment  \\
            5 & 4.5\% & \zh{坏女人} (Bad Woman) & Traditionally considered insulting, but it can also be playful or teasing with less hostility, depending on the context \\
            6 & 4.0\% & \zh{渣、卑鄙} (Despicable) & Traditionally considered insulting, but it can also be playful or teasing with less hostility, depending on the context \\
            7 & 1.9\% & Animal Related & Many VTubers' avatars have the design elements from animals \\
            8 & 1.1\% & \zh{秃} (Bald) & - \\
            9 & 1.0\% & \zh{讨厌你} (Hate you) & - \\
            10 & 0.6\% & \zh{小丑} (Clown) & The insult emphasizes a lack of respect for the person's abilities or behavior, reducing them to a figure of mockery or ridicule \\
        \hline\hline
    \end{tabular}
    \caption{Topics for toxic chat messages in the harassment category.}
    \label{tab:toxic_harassment}
\end{table*}

\begin{table*}[]
    \centering
    \begin{tabular}{crrR{35em}}
        \hline\hline
            & \% & Keyword & Explanation \\
            Rank &  &  &  \\
        \hline
            1 & 23.3\% & \zh{踩} (Step-on, Trample) & In a sexual context \\
            2 & 18.3\% & \zh{吃我一拳} (Take my punch) & A meme \cite{take-my-punch}, usually with no actual violent meaning \\
            3 & 15.5\% & \zh{杀 ...} (Kill something) & Kill something, typically in a gaming context \\
            4 & 8.1\% & Knife Related & Typically in a gaming context \\
            5 & 6.3\% & \zh{杀} (Kill) & Kill, while without a direct object, typically to express emotion or give instructions in a gaming context \\
            6 & 4.7\% & \zh{吞} (Swallow) & Typically used with slangs with no violent meaning \\
            7 & 2.6\% & \zh{猎杀} (Hunt) & Typically in a gaming context \\
            8 & 2.0\% & Gun Related & Typically in a gaming context \\
            9 & 2.0\% & \zh{毁灭} (Destroy) & Typically used to express emotion in a exaggerated way \\
            10 & 1.5\% & \zh{炸} (Explode) & Typically used to express emotion in a exaggerated way  \\
        \hline\hline
    \end{tabular}
    \caption{Topics for toxic chat messages in the violence category.}
    \label{tab:toxic_violence}
\end{table*}



\subsection{\toolname~Additional Experiments}

\pb{Baseline Comparison.}
We have experimented with recommendations based on historical gift/superchat activity and chat frequency as a baseline. 
Specifically, we ranked users by the number of gifts or chat messages sent to the membership-ed VTuber from T-30 to T-1. 
We then compared the results of this baseline approach to our FanRanker.
We find that the performance of these baseline methods is significantly worse than our Random Forest or Histogram-Based Gradient Boosting models. Ranking based on gifts proves very ineffective, as the majority of viewers have never sent a gift (as illustrated in Figure \ref{fig:rq1_2}d), resulting in most viewers having the same rank. Ranking based on the number of chat messages does yield a reasonable result, but it is worse than our approach, as shown in the Table \ref{tab:baseline_tab}.
This shows the same metric as Figure \ref{fig:rq3_result}, which is the ranking performance metrics outlined in \S\ref{subsec:rq3_eval}. That is, the ranking position and percentile of the user who actually purchased the membership during the livestreaming session.

\begin{table}[h]
    \centering
    
    \begin{tabular}{r|rrr|rrr}
        \hline\hline
        {} & \multicolumn{3}{c}{Rank} & \multicolumn{3}{c}{Percentile} \\
        {} & Chat  & HGB   & RF    & Chat & HGB & RF \\
        \hline
        Mean      & 52.540 & 32.241 & 30.002 & 0.124  & 0.070  & 0.066  \\
        25\%      & 1.000  & 0.000  & 0.000  & 0.031  & 0.000  & 0.000  \\
        50\%      & 9.000  & 0.000  & 0.000  & 0.096  & 0.000  & 0.000  \\
        75\%      & 52.000 & 19.000 & 18.000 & 0.182  & 0.091  & 0.086  \\
        \hline\hline
    \end{tabular}
    \caption{The performance of \toolname~as compared to baseline method using chat frequency.}
    \label{tab:baseline_tab}
\end{table}

\pb{Ablation Study of ChatSim.}
We conduct an ablation study for the HGB and RF models, excluding all ChatSim-related features. The results are presented in the Table \ref{tab:ablation_1} and \ref{tab:ablation_2}.
The tables show the same metric as Figure \ref{fig:rq3_result}, which is the ranking performance metrics outlined in \S\ref{subsec:rq3_eval}. That is, the ranking position and percentile of the user who actually purchased the membership during the livestreaming session.
Overall, we see that without ChatSim, the performance of the models drops clearly.


\begin{table}[h]
    \centering
    \small
    \begin{tabular}{r|rr|rr}
        \hline\hline
        {} & HGB & HGB (w/o ChatSim) & RF & RF (w/o ChatSim) \\ \midrule
        Mean       & 32.241 & 35.266 & 30.002 & 36.454 \\
        25\%       & 0.000  & 0.000  & 0.000  & 0.000  \\
        50\%       & 0.000  & 3.000  & 0.000  & 4.000  \\
        75\%       & 19.000 & 29.000 & 18.000 & 31.000 \\ 
        \hline\hline
    \end{tabular}
    \caption{Rank position meetric of the performance of \toolname with and without ChatSim features.}
    \label{tab:ablation_1}
\end{table}


\begin{table}[h]
    \centering
    \small
    \begin{tabular}{r|rr|rr}
        \hline\hline
        {} & HGB & HGB (w/o ChatSim) & RF & RF (w/o ChatSim) \\ 
        \hline
        Mean       & 0.070 & 0.089 & 0.066 & 0.093 \\
        25\%       & 0.000 & 0.000 & 0.000 & 0.000 \\
        50\%       & 0.000 & 0.038 & 0.000 & 0.044 \\
        75\%       & 0.091 & 0.128 & 0.086 & 0.137 \\
        \hline\hline
    \end{tabular}
    \caption{Percentile of rank position Metric of the performance of \toolname with and without ChatSim features.}
    \label{tab:ablation_2}
\end{table}



% \subsection{Additional Related Work}

