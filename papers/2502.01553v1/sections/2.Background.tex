\vspace{-2ex}
\section{Background}
\label{sec:back}

% \subsection{A Primer on VTubers}
% \label{sec:back:vtubers}

% \vspace{-0.5ex}
\pb{The VTuber Concept.}
VTubers originated in Japan and have rapidly gained popularity since their debut in 2016. Initially, VTubers focused on uploading videos to YouTube. However, with the rise of online live streaming, live streaming became their primary activity \cite{10.1145/3604479.3604523}.
% 
A VTuber is an animated virtual avatar that performs in live video streams or recorded videos. These avatars are often voiced by actors known as Nakanohitos in Japanese. Typically, VTubers use half-body 2D avatars created with tools like Live2D, which capture the actor’s facial movements to animate the avatar’s expressions(see Figure \ref{fig:nijisanji} in the Appendix for examples). Additional body movements can be triggered within these programs using commands from desktop computers. VTubers with access to full-body motion capture systems can perform using 3D avatars, allowing for a wider range of motion. Like real-person streamers, VTubers often interact with their audience by reading and responding to chat messages during streams.



\pb{VTubers in China.}
VTubers in China have experienced rapid growth since 2021, following the exit of Japanese agency-based VTubers from the Chinese market for various reasons \cite{holowiki}.
Bilibili is the primary platform for VTuber livestreams in China. According to \texttt{VTBs.moe} (a website that indexes and tracks information about Vtubers in China), there are over 6,000 active Vtubers listed, with the most popular one boasting 4.5 million followers. Additionally, 35 indexed Vtubers have more than 1 million followers each. 

\pb{Livestreaming on Bilibili.}
As one of the leading video streaming platforms in China, Bilibili hosts a diverse range of streamers besides VTubers, encompassing sports, esports, gaming, arts, and more.
In addition to the standard features found on other livestreaming platforms, Bilibili provides two advanced features, membership and bullet chat.


\pb{Bilibili Membership System.}
Bilibili offers a distinctive membership system, officially called \emph{guards}. Users have the option to pay a monthly fee of \textasciitilde\$24 USD to become a member of a particular streamer, with more premium tiers available at \textasciitilde\$280 USD and \textasciitilde\$2800 USD. This system resembles the paid subscription model on Twitch but is more expensive and comes with additional privileges. These privileges include having the member's name displayed beside the live streaming screen, showing notifications upon entering the room, and enjoying more prominent visibility in chat interactions. Furthermore, streamers often invite their members to join private chat groups, assist in moderating live streaming sessions, and sometimes organize meetings offline. This creates a very different dynamic, compared to other platforms.

VTubers, or rather their fans, are the primary users of this feature on Bilibili. According to official rankings \cite{bilibili-membership-ranking}, at the time of writing, VTubers occupy all of the top 10 spots among streamers with the most members, and 39 out of the top 50 streamers are VTubers.
The VTubers on Bilibili, in total, have succeeded in attracting 152,000 paying members, generating a monthly revenue of \$3.5 million USD \cite{vtb-moe}.  
Consequently, this membership system offers a valuable framework for exploring the core audience of VTubers, the community among VTubers and their fans, and the monetization tactics of VTubers.

\pb{Bilibili Bullet Chat.}
Bullet chat (also known as danmaku in Japanese and danmu in Chinese) \cite{huang2023} is a comment system originally introduced by the Japanese website Niconico. It allows viewers to post their comments on the screen during livestreaming, where they appear as floating, moving text, as depicted in Figure \ref{fig:nanako} in Appendix. 
In comparison to traditional comment systems like those on YouTube or Twitch live streams, bullet chat offers a more engaging real-time interaction experience for users \cite{huang2023}. Due to the timely and straightforward nature of bullet chats, usually, a large number of comments are posted by viewers during livestreaming sessions.

\pb{Bilibili Gift \& Superchat.}
Similar to other mainstream platforms, Bilibili also offers common monetization avenues. Viewers have the option to purchase virtual gifts and donate them to streamers. Additionally, there's a superchat (SC) system in place, allowing users to pay for sending a special message that gets pinned at the top of the chat column for a certain period. These features provide streamers with alternative methods of generating revenue.