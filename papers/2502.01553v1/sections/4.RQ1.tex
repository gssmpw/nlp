

% \vspace{-2ex}
\section{Characterizing User Activity}
\label{sec:RQ1}

In this section we explore the activity patterns of members (\textbf{RQ1}).
The online activity of a user is our first step in understanding their preferences, interests, and behavioral patterns \cite{Gyarmati2010, Zhu2023}. We posit that this insight can help VTubers understand how to identify potential future members effectively.

\begin{figure}[]
    \centering
    \includegraphics[width=\linewidth]{figs/rq1_new_1.pdf}
    \vspace{-5ex}
    \caption{The CDF of the viewing activity metrics to the membership-ed VTuber for members and non-members.}
    \label{fig:rq1_1}
    % \vspace{-4.5ex}
\end{figure}

\begin{figure*}
    \centering
    \includegraphics[width=0.66\textwidth]{figs/rq1_new_2.pdf}
    \includegraphics[width=0.32\textwidth]{figs/rq1_new_3.pdf}
    \vspace{-2ex}
    \caption{The CDF of the metrics for chat and gift \& superchat activity for members and non-members.}
    % \vspace{-3ex}
    \label{fig:rq1_2}
\end{figure*}

% \vspace{-1.5ex}
\subsection{Comparing Members and Non-members}
\label{subsec:rq1_compare}

We begin by comparing the overall activity of members \vs non-members. Intuitively, we expect that members will exhibit higher levels of activity compared to non-members, as members are likely to be heavy users of the platform and more engaged with the membership-ed VTuber's live streaming.

\pb{Methodology.}
We use several activity metrics (see Appendix \ref{subsec:appendix_activity_metrics} for details) that have been widely used in previous studies of streaming platforms \cite{10.1145/3311350.3347149}.
For each member and non-member, we compute each metric over a 90-day period. 
Specifically, for each user and the corresponding livestream session they join as a member, we consider a timeframe that spans 45 days \emph{prior} to the livestream session (T-45) and 45 days \emph{following} the session (T+45). This allows us to capture their behaviors before and after they become a member.

\pb{Viewing Activity.}
We first analyze the viewing activity of members regarding the livestream sessions of the membership-ed VTuber.
We evaluate this using three metrics:
\one the proportion of livestream sessions watched, \two the proportion of on-time attendance for livestream sessions, 
and
\three the proportion of the watch time relative to the total streaming time of the VTuber.  

Figure \ref{fig:rq1_1} presents the CDFs of the specified metrics over a 45-day period for members (pre \& post purchase) and non-members. Note, for non-members, we combine the two 45-day periods as they are almost identical (this applies to all other subsequent figures in the paper unless otherwise specified).
We see that members pre-purchase significantly outperform non-members in all three metrics.
Specifically, members pre-purchase exhibit a higher live watch rate (mean 26\% \vs 20\%), a greater on-time rate (mean 18\% \vs 9\%), and an extended duration of watch time (mean 17\% \vs 7\%). This indicates that before they become a member, members begin to demonstrate a more dedicated viewing behavior, which provides insights into how to identify potential members.











\pb{Bullet Chat Activity.}
We next investigate the bullet chat (see \S\ref{sec:back}) activity of users --- a more nuanced indicator of user engagement. A significant difference is observed both in the total number of chats sent across the platform, and in the average number of chats sent per livestream session of the membership-ed VTuber.
% 
Figure \ref{fig:rq1_2} (a-b) displays the CDF for these two metrics. We see that members surpass non-members regarding the total number of chats sent, with averages of 876 for members pre-purchase \vs 161 non-members. A similar pattern is observed for the number of chats per the VTuber's livestream session, where members significantly outperform non-members, with an average of 12.9 for members pre-purchase \vs 3.2 non-members.

The findings indicate that members pre-purchase are indeed more engaged in terms of interaction with streamers. In contrast to viewing activity, this pattern is not limited to the membership-ed VTubers but also applies across the entire livestreaming platform. We argue that this serves as a valuable indicator for VTubers to select which viewers may be future members. We later exploit this for identifying potential members (\S\ref{sec:RQ3}).

\pb{Gift \& Superchat Activity.}
We finally delve into the gifts and superchats from viewers. 
Intuitively, considering users who have previously made purchases as potential members is a reasonable approach. Here, we evaluate this conjecture.
Figure \ref{fig:rq1_2} (c-f) illustrates the CDF for both the quantity and monetary value of gifts and superchats sent by the user across the entire platform, and directed specifically towards the membership-ed VTuber. 

It is evident that members consistently surpass non-members in all four metrics.
In terms of platform-wide activity, the average for members pre-purchase are 48 (quantity) and 9190 (monetary value), significantly higher than 2.7 (quantity) and 2.6 (monetary value) for non-members. Furthermore, 12\% of members pre-purchase  send at least one gift or superchat, compared to only 4.5\% of non-members. 

Moreover, while the pattern of gift and superchat activity towards the membership-ed VTuber shows a similar trend to the platform-wide activity, a notable distinction exists. Specifically, 5\% of members pre-purchase have previously sent a gift or superchat to the membership-ed VTuber, while 12\% members pre-purchase have sent to any streamer. Curiously, this indicates that 7\% of members pre-purchase have sent a gift but not specifically towards the membership-ed VTuber. 
This suggests that members pre-purchase are more engaged in terms of gifts and superchats, which is intuitive since it reflects a user's willingness to contribute (financially), even if the gifting may not be directed to the membership-ed VTuber.


Overall, these results confirm that considering users who have previously made purchases as potential members is a reasonable strategy to identify potential members. However, this strategy, while effective, is not sufficient on its own, given that there are still 88\% of members who have not engaged in paid gifting or superchats before buying a membership. This highlights the complexity of identifying potential members, and underscores the need for a more nuanced understanding of user behavior and motivations.






























\begin{figure*}
    \centering
    \includegraphics[width=\textwidth]{figs/rq1_3_v2.pdf}
    % \includegraphics[width=0.192\textwidth]{figs/rq1_3b.pdf}
    \vspace{-4.5ex}
    \caption{The time-series plots of the activity metrics for members with the membership-ed VTuber.}
    % \vspace{-4ex}
    \label{fig:rq1_3}
\end{figure*}

\vspace{-1.5ex}
\subsection{Temporal Analysis}
\label{subsec:rq1_temporal}

In \S\ref{subsec:rq1_compare}, we find that engagement with the membership-ed VTuber significantly differs between members and non-members. 
We also observe that, among all metrics investigated (in Figure \ref{fig:rq1_1} and \ref{fig:rq1_2}),  members post-purchase exhibits higher values than members pre-purchase.
Consequently, we are intrigued by how exactly the engagement evolves over time. 
We anticipate an upward trend as it approaches the time when they become a member. 
Thus, we next conduct a daily temporal analysis. 


\pb{Methodology.}
As the first step, we confirmed that the increase of the platform-wide activity is due to the heightened activity towards the membership-ed VTuber.
Thus, we only consider the metrics in \S\ref{subsec:rq1_compare} regarding the membership-ed VTuber.
However, instead of analyzing the 90-day period as a whole, we analyze it on a daily basis. For each individual day within the 90-day span, we aggregate the data for all members. For the first four metrics ---- Live Watch Rate, Live Watch On-Time Rate, Live Watch Time, and Number of Chats Sent --- we calculate the mean average. Meanwhile, for the metrics concerning the gifts and superchats sent, we aggregate them as the proportion of members who send at least one.


\pb{Results.}
Figure \ref{fig:rq1_3} displays the time-series plots for the aggregated six metrics of member activity over the 90-day period. A consistent pattern emerges across all metrics: starting at its lowest point at T-45, the trend exhibits a steady increase, culminating in a peak at T+0. However, despite maintaining a relatively high level, a noticeable decline follows.
Specifically, for the Live Watch Rate, Live Watch Time, and Number of Gifts Sent, the figures at T+45 drop to approximately the same levels as those observed at T-15.


\pb{Explaining the Decline.}
The above observation confirms that member engagement intensifies during the lead-up to membership, but surprisingly declines soon afterwards. This raises intriguing questions about the factors that contribute to the observed decrease in engagement post-purchase.
An intuitive explanation for the decline in engagement after membership could be a diminishing interest in the membership-ed VTuber.
Thus, we investigate a key indicator for this: whether members renew their membership in the subsequent month. 
We note that due to the automatic subscription renewal mechanism, we might not capture this information if the VTuber does not stream when the renewal occurs. Thus, our analysis is confined to the instances where the VTuber is streaming in the following month when the renewal is expected to occur. This covers 67\% of the membership subscriptions.

We find only 8.9\% of members renew their membership in the following month. To further evidence this, we perform the $\chi^2$ test of independence \cite{pearson1900} to examine the relationship between the decline in six engagement metrics and the membership renewal (See Appendix \ref{subsec:appendix_chi_square} for details). We observe a clear correlation across all six metrics with significant p-values. This confirms that declining engagement is indeed related to lower renewal rates.
Contrary to initial expectations, the findings suggest that purchasing a membership appears to be an \emph{one-time} behavior. Viewers initially become increasingly engaged with the VTuber, leading them to subscribe. However, once they become members, roughly the first 2 weeks represent the peak of their activity. After this period, their interest seems to wane, as indicated by a decrease in engagement, with the majority choosing not to renew their membership.

This differs from previous research on Twitch subscriptions \cite{10.1145/3311350.3347160}, which emphasizes the distinction between (continuous) subscriptions and one-time donations. We suspect that the high cost of membership (\textasciitilde\$24 per month \vs \$5.99 on Twitch) could be a contributing factor. Overall, the results underscore the need to reconsider membership strategies for VTubers, in stark contrast to prior membership systems. For instance, membership could be optimized by introducing shorter-term options, such as two-week memberships at lower prices. Additionally, this also highlights the necessity for VTubers to implement more effective retention strategies to maintain member interest after the initial period.

 




