% \vspace{-2ex}
\section{Introduction}

Live streaming has gained significant popularity globally, encompassing a wide range of themes, including personal experiences \cite{lottridge2017thirdwave, lu2019vicariously, tang2016meerkat}, artistic creation \cite{fraser2019sharing, lu2019responsibility}, educational content \cite{lu2018watch, faas2018watch, lu2018streamwiki}, and gaming sessions \cite{hamilton2014streaming, 7382994}. 
As part of this trend, we have recently witnessed the emergence of a novel type of streamer, referred to as \emph{VTubers} (Virtual YouTubers) \cite{lu2021kawaii}. 
A VTuber is an animated 2D or 3D virtual avatar that performs in live video streams.
These streams are produced using tools like Live2D \cite{Live2D} that capture the actor's movements and control the avatar accordingly.
The avatar is therefore voiced and controlled by a specific person.
VTuber content covers a wide range of topics, including anything that does not require a real human to present.


Although seemingly fringe, VTubers have been gaining significant popularity, with dedicated fan bases and corporate sponsorship deals.
There are tens of thousands of highly active and influential VTubers --- the most popular ones have millions of followers \cite{vtb-fan-ranking}.
Thanks to the various monetization mechanisms offered by live streaming platforms, VTubers earn substantial incomes. 
Among the top 50 highest-earning YouTubers based on all-time superchats (viewer donations), 31 are VTubers, each earning between \$1.1 million and \$3.2 million USD \cite{vtb-superchat-ranking}. 

In contrast to prior studies showing that a sense of realism is a crucial factor in audience engagement \cite{tang2017crowdcasting, haimson2017live}, VTubers intentionally conceal their real identities and present themselves to viewers through virtual avatars.
This approach completely reconstructs the self-representation \cite{10.1145/3637357, doi:10.1080/10510974.2024.2337955} of the streamer, and also the relationship between the streamer and their fanbase \cite{Xu_Niu_2023, lu2021kawaii, Turner1676326, 10058945}. This is driven by an intersection of Japanese otaku culture and idol culture \cite{lu2021kawaii, 10.1145/3604479.3604523}, with the core viewer communities of VTubers exhibiting similar traits. As a result, these online viewer communities play a much more significant role compared to those of other streamers. The core viewers of a VTuber usually create and share their own subculture, and are active and passionate. Core viewers engage not only in live streaming but also contribute in various other ways to expand the VTuber's influence. For instance, there are over 420,000 dedicated fan artworks on Pixiv \cite{pixiv-hololive,wei2024understanding} for VTubers affiliated with Hololive (a ``celebrity'' VTuber agency). 

Consequently, building such a core viewer community is essential to a VTuber's success, both in terms of fame and monetization. Yet, we still possess only a limited knowledge of the characteristics and behaviors of VTuber's viewers. We argue that gaining a deeper understanding is crucial for VTubers to enhance engagement with their audience, foster a stronger fan community, and thus attract more viewers (thereby generating greater monetary income). This insight can be especially beneficial for self-operated and startup VTubers, who do not belong to a management agency, as it will help them better understand their audience and identify potential core viewers.

To bridge this gap, we conduct the first large-scale measurement study of VTuber core viewers, focusing on Bilibili as an exemplar platform. 
Bilibili is a major website for user-generated video and live streaming. It currently hosts the largest number of VTubers in China, with almost all Chinese VTubers streaming on the platform. 
% 
Importantly, on Bilibili, core viewers can be easily identified through a membership system, whereby users can pay a monthly fee to gain membership for a streamer. That is, ``members'' are the core viewers.
This membership is similar to a paid subscription on Twitch but more expensive and with additional privileges \eg participation in a private chat group. 
Consequently, purchasing a membership signifies a strong commitment to the streamer, making members of a VTuber and their community an ideal target for our study.

To investigate the members of VTubers on Bilibili, we compile a dataset encompassing all live streaming records (2.7 million live records with 3.6 billion chat messages in total) of 4.9k VTubers, along with 21k other streamers on Bilibili, from June 2022 to August 2023. This dataset includes almost all streamers with a moderate fanbase on Bilibili, representing the largest dataset of VTubers available, to the best of our knowledge.
Exploiting this dataset, we focus on the following research questions:


\begin{itemize}[leftmargin=*]
    \item \textbf{RQ1:} How do the behaviors of members differ from those of non-members? How do these patterns change before and after they become a member? Understanding these behavioral variations can assist in identifying potential future members and improving strategies for member promotion and retention.

    \item \textbf{RQ2:} How do the chat messages sent by members differ from non-members? How do these chat patterns change before and after they become members?
    Besides identifying potential members, analyzing these communication patterns can offer insights into the unique culture of the community among members, guiding efforts in better community building.
    
    \item \textbf{RQ3:} Leveraging the findings from RQ1 and RQ2, can we develop a tool that assists VTubers in identifying viewers likely to become members in the future? 
    While numerous studies have explored recommending streams to viewers, the inverse --- identifying valuable viewers for streamers --- remains unexplored. By flipping the problem, this tool addresses a crucial gap, offering VTubers a strategic advantage in cultivating a loyal and engaged community.
\end{itemize}




\noindent
By answering these RQs, our findings include:

\begin{enumerate}[leftmargin=*]
    

    \item Members' activity escalates in the period leading up to their decision to purchase a membership, but interestingly decreases after purchase. Only 8.9\% of members renew their membership in the following month, and we find that the likelihood to renew is correlated with the decrease in activity level ($\chi^2$ test p-values all less than $10^{-280}$). This underscores the necessity for VTubers to implement more effective retention strategies to maintain member interest beyond the initial period. (\S\ref{subsec:rq1_temporal}) 
    
    \item Members' chat content exhibits higher alignment with the session’s chat environment compared to non-members, indicating a stronger level of engagement and familiarity. This alignment increases before the viewer become a member. We find this is because members become adept at using the unique conventions and interactive styles of the specific VTuber, highlighting the crucial role of unique community culture in fostering community growth. (\S\ref{subsec:rq2_similarity})
    
    \item Members are \emph{more} likely to send toxic chat messages than non-members, with 12\% having sent harassment and 6\% having sent sexual content, while only 2\% non-members have sent such messages. Our qualitative analysis finds that these messages are not necessarily malicious but rather a problematic form of cultural ``overstepping'' interactions. This emphasizes the need for better and more flexible moderation. (\S\ref{subsec:rq2_toxicity})

    \item Leveraging the above findings, we show that it is possible to help VTubers identify a small group of viewers (from their thousands of viewers) who are likely to be open to becoming a member. By focusing more attention on this smaller group, VTubers can increase the likelihood of converting potential future members into actual members, thereby enhancing fan community building and monetization. (\S\ref{sec:RQ3})
\end{enumerate}

