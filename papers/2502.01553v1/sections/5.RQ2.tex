\vspace{-2ex}
\section{Analysis of Linguistic Behaviors}
\label{sec:RQ2}

Given that members send more chats than non-members, with a noticeable increase leading up to their membership (\S\ref{sec:RQ1}), we next explore the messages sent by members (\textbf{RQ2}).
We are particularly interested in two aspects. 
First, we wish to explore the presence of distinct community subcultures and community norms, in terms of how members interact with VTubers. 
Second, we wish to measure the potential harms that may be associated with this subculture. For example, concerns have been raised about the sexualization of VTuber avatars \cite{10.1145/3637357}, and there are VTubers whose fan communities are notorious for engaging in online abuse and harassment \cite{seren, nyaru, jiaran}. Thus, we next explore these matters.












% \vspace{-1.5ex}
\subsection{Similarity to the Chat Culture}
\label{subsec:rq2_similarity}

We begin by examining the similarity between an individual viewer's chat messages and the overall chat environment (\ie culture) during a livestreaming session. Intuitively, a high level of similarity between a user and the remaining chat messages would suggest a robust engagement with both the livestreaming and the VTuber.

\pb{Calculating the Similarity.}
To quantify this similarity, we introduce a metric termed \textbf{ChatSim} to capture the similarity between a viewer's chat messages and the overall chat environment in a livestreaming session.
To compute ChatSim, we first generate embeddings for all chat messages using the methods outlined in \S\ref{subsec:nlp_methods}. Subsequently, for each livestreaming session, we compute the average of all chat embeddings within that session to represent the overall chat environment. Then, for each viewer in the session, we calculate the average of the embeddings for that viewer's chat messages. We then compute the cosine similarity between this average and the overall chat environment. This yields the ChatSim metric for the viewer in this livestreaming session. 


\pb{Comparing Members vs.\ Non-Members.}
We begin by comparing the ChatSim scores of members and non-members in livestreaming sessions. 
The analysis is conducted on a 90-day period that includes 45 days before the livestreaming session (T-45) and 45 days after the session (T+45), on users in our designated member and non-member lists, as described in Section \ref{subsec:construct_dataset}. Figure \ref{fig:rq2_1}a plots the CDFs.

It is evident that members exhibit higher ChatSim scores than non-members in the livestreaming sessions of the VTubers they support. The average (mean) scores are 0.80 for members pre-purchase, 0.83 for members post-purchase.
This can be compared to just 0.74 for non-members. 
% 
This suggests that members have higher alignment with the chat environment. We posit that this pattern could be helpful in identifying potential members.

Intuitively, this might be explained by the fact that the topic of chat discussions is closely related to the content the VTuber is streaming. However, we note that this is unlikely to result in a clear distinction between members and non-members. We therefore hypothesize that this is due to the community subculture among members and the specific norms how they interact with the VTuber, which we explore in the following paragraphs.


\pb{Temporal Chat Analysis.}
In the previous paragraph, we noticed that ChatSim seems to rise after the purchase of the membership (Figure \ref{fig:rq2_1}a).
Thus, we next inspect how exactly ChatSim scores change before and after a user becomes a member for a VTuber. 
Figure \ref{fig:rq2_1}b displays the ChatSim scores of members during the livestreaming sessions of the membership-ed VTuber, grouped by the number of days relative to the day they become members. We illustrate the average (mean) values and the first to third quartile range (Q1 to Q3) with shading.

We observe an increasing trend before T+0, with the mean rising from 0.78 at T-45 to 0.82 at T-1, and reaching a peak of 0.86 at T+0. This trend is consistent with our findings on audience activity in  \S\ref{subsec:rq1_temporal}. However, after T+0, there is a decline, with the mean decreasing to 0.83 at T+45. This slightly differs from the observations in  \S\ref{subsec:rq1_temporal}: although a decrease is noted, it is minor and the mean remains higher than it was before T+0.
We posit that this increasing pattern could be helpful in identifying potential members. 
However, the result (contrary to \S\ref{subsec:rq1_temporal}) suggests that the change of the ChatSim score is less relevant to the activity level of the member. 


\pb{Chat Topic Analysis.}
To gain a deeper understanding of the factors contributing to the differences (as compared to non-members) and the increases (over time) in ChatSim scores among members, we conduct a content analysis of chat messages. 
We use BerTopic \cite{grootendorst2022bertopic} to extract the key topics from the chat messages.
However, we find that the topics for each VTubers differ significantly. 
Thus, we build the topic model respectively for the 100 most popular  VTubers's (ranked by the number of chats), and manually validate each one. 
The specifics of this analysis are detailed in Appendix \ref{subsec:appendix_similarity_topic_analysis}.
Overall, our analysis shows that the most notably increased topics are associated with \one endearments for the VTuber and the fans (increased for 100\% of the selected VTubers), 
\two prescribed viewer reactions (98\%),
and 
\three slang/catchphrases/memes (95\%). 
We emphasize that these are unique to each specific VTuber.  

This indicates that members become familiar with the unique subculture and interactive styles of the VTuber. 
This aligns with findings from prior human-factors research on VTubers \cite{lu2021kawaii} and other streamers \cite{10.1145/3025453.3025854} which shows the importance of a dedicated viewer community with its own  culture, and how this community and culture are established.
This observation also helps explain why there is no significant decrease in ChatSim scores afterwards: although members may become less active over time, they still know the subulture and are able to interact as a ``proficient'' participant. 



% \vspace{-1.5ex}
\subsection{Toxicity}
\label{subsec:rq2_toxicity}

\begin{figure}[]
    \centering
    \includegraphics[width=0.48\linewidth]{figs/rq2_1_v2.pdf}
    \hfill
    \includegraphics[width=0.48\linewidth]{figs/rq2_2_v2.pdf}
    \vspace{-1.5ex}
    \caption{(a) The CDF of the ChatSim score for members and non-members in the livestream sessions of the membership-ed VTuber; (b) Daily average ChatSim score for members in the livestream sessions of the membership-ed VTuber.}
    \vspace{-4ex}
    \label{fig:rq2_1}
\end{figure}

\begin{figure*}
    \centering
    \includegraphics[width=0.22\textwidth]{figs/rq2_3_v2.pdf}
    \includegraphics[width=0.54\textwidth]{figs/rq2_4_v2.pdf}
    \includegraphics[width=0.22\textwidth]{figs/rq2_5.pdf}
    \vspace{-2.5ex}
    \caption{(a) The proportion of toxic messages sent by members and non-members to the membership-ed VTuber; (b) The CDF of the proportion of toxic messages sent by members and non-members to the membership-ed VTuber; (c) The proportion of members who have sent a toxic chat messages to the membership-ed VTuber on the day.}
    \vspace{-2.5ex}
    \label{fig:rq2_2}
\end{figure*}

The presence of toxicity in chats, including sexual content, harassment, and violence, remains a contentious issue. 
On the one hand, VTubers may prefer to avoid such toxic chats as they could potentially harm the VTuber, affect the quality of the livestream, and negatively impact the audience experience.
Yet, on the other hand, it can also be a consequence of high audience engagement in livestreaming. For instance, a VTuber who focuses on content that borders on the erotic might not view chats containing sexual content as toxic and might even encourage such interactions \cite{10.1145/3544548.3580730, 10.1145/3543507.3583210}. Similarly, livestreams dedicated to highly competitive esports often naturally attract comments that are violent in nature \cite{Jiang_Shen_Wen_Sha_Chu_Liu_Backes_Zhang_2024}.
% 
Thus, given the complex impact that toxic interactions can have on both a VTuber and their audiences, it is crucial to examine the role and extent of toxicity in chat messages to gain a better understanding. 


\pb{Methodology.}
To analyze the toxicity, we choose the 90-day period that includes 45 days before the livestreaming session (T-45) and 45 days after the session (T+45). 
Using the method described in \S\ref{subsec:nlp_methods}, we label all chat messages  in livestreaming sessions (within the 90-day period) related to the viewers included in our member and non-member lists (\S\ref{subsec:construct_dataset}). 
Since we find minimal cases of other categories, we concentrate on three main toxicity categories: sexual, harassment (insulting or degrading someone), and violence among the categories of toxicity.


\pb{Comparing Members vs.\ Non-Members.}
We first analyze the proportion of toxic messages. The results are presented in Figure \ref{fig:rq2_2}a. 
Interestingly, the proportion of toxic messages from members pre-purchase is slightly \emph{lower} than that of members post-purchase for sexual (0.68\textperthousand \vs 0.71\textperthousand), harassment (2.5\textperthousand \vs 2.7\textperthousand), and violence categories (2.9\textperthousand \vs 3.2\textperthousand). 
While for non-members, the proportion for sexual (0.43\textperthousand), harassment (1.7\textperthousand), and violence (2.9\textperthousand) are all lower than members.
This suggests that members are more inclined to send toxic chat messages, and that the frequency of toxic messages tends to increase after paying for membership.

Further, we compare the data at an individual viewer-level. Figure \ref{fig:rq2_2}b shows the CDF of the proportion of toxic messages sent to the membership-ed VTuber out of all messages sent to the VTuber by each individual viewer. We see noticeable disparity across all three categories, with members significantly outpacing non-members, and members post-purchase significantly outpacing members-pre. Specifically, 6.4\% members post-purchase sent sexual messages, 12.1\% members-post sent harassment messages, and 14.5\% members-post sent messages containing violence.
Overall, the results confirm that members are more inclined to send toxic chat messages.


\pb{Temporal Analysis.}
To further explore this, we next conduct additional temporal analysis on the proportion of members who have sent toxic chat messages to the membership-ed VTuber. Our previous analysis indicated a significant increase in this proportion after the membership purchase (as illustrated in Figure \ref{fig:rq2_2}b).
% 
In Figure \ref{fig:rq2_2}c, the results are presented on a daily basis. It is evident that all three categories exhibit a similar pattern to the ChatSim score shown in Figure \ref{fig:rq2_1}b: a rise leading up to T+0, a peak at T+0, followed by a slight decrease. This alignment may be attributable to a similar reason as the ChatSim score trend: toxic chat messages could potentially be a form of cultural engagement by some members, where as they get more acquainted, they start to act differently.


\pb{Content Analysis of Toxic Chats.}
We therefore next inspect the precise content of the toxicity in chat messages, to clarify \one how exactly this reflects viewer engagement, and \two whether such messages are culturally acceptable or require moderation.
For the three categories (sexual, harassment, and violence) we respectively construct topic models within members' chat messages in the category, using BerTopic \cite{grootendorst2022bertopic}. 
We present the details in Appendix \ref{subsec:appendix_toxic_topic_analysis}.

We find that sexual chats clearly pose a potential issue. The top 10 topics (covering 94.6\% of the sexual chats) predominantly revolve around sexual behaviors towards the VTuber (\eg ``kiss'' for 25.4\%, ``lick'' for 24.3\%) or direct references to sexual body parts (\eg ``Hip'' for 7,3\%, ``Chest'' for 4.8\%). 
It is unclear whether the VTuber is anticipating or merely tolerating it. Even if they are expecting such content, it may not be appropriate for a general purpose livestreaming platform and could offend other viewers. 

Similar issues arise with harassment chat messages. The top 10 topics (covering 96.9\% of harassment messages) all contain insulting language that is, by definition, harassment. However, many of these messages are characterized with a somewhat light-hearted tone (\eg ``Stinky'' for 5.1\%, ``Despicable'' for 4\%), blurring the line between problematic harassment and jests. Therefore, rather than a clear-cut need for moderation, as is the case with sexual chats, it may be left to the discretion of the VTuber to determine whether they can handle such comments to make the (potential) members pleased.


Violent chat messages appears to have two types. The first type is actually flirting that is similarly problematic, akin to sexual chat messages, \eg ``Step-on/Trample'' (23.3\%).
The second type appears broadly acceptable, which covers 8 out of the top 10 topics. These messages do contain violent language, such as ``kill'' (21.8\%). However, these actions are typically directed towards a virtual entity (\eg in video games) or are actually memes. 

The content analysis reveals that, as viewers become more familiar with the VTuber, this increased familiarity leads to more toxic behavior, including expressing their strong admiration for the VTuber through sexualized messages and making aggressive jokes that border on harassment. 
While such behavior may indicate high engagement, it also raises concerns about VTubers' online safety, particularly with the normalization of sexualized and aggressive interactions. Unlike previous research on harassment toward content creators \cite{10.1145/3613904.3641949, 10.1145/3491102.3501879}, this harassment notably originates from within the VTuber's own fanbase and paying supporters. This underscores the need for new mechanisms to address this complex issue, where platforms must develop and implement stronger and flexible moderation and support systems to protect VTubers from escalating harassment and ensure a safer online environment.





