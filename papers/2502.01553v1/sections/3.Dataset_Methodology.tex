\vspace{-2ex}
\section{Data Collection Methodology}
\label{sec:dataset}



\pb{Target Streamers.}
First, we compile a list of target streamers. 
To accomplish this, we rely on \texttt{VTBs.moe}, an indexing website for VTubers on Bilibili; and \texttt{danmakus.com}, a website dedicated to indexing and archiving popular livestreaming sessions on Bilibili.
We include all streamers indexed by these two websites, and obtain a list of 4.9k VTubers and 21k other streamers, covering almost all streamers with a moderately large fanbase on Bilibili. 


\pb{Data Collection of Livestream Sessions.}
For each target streamer, we collect data for all livestreaming sessions from \texttt{2022-06-01} to \texttt{2023-09-01}. The data encompasses the metadata of the livestreaming sessions, such as the time, duration, title, area, \etc of each session.  The data also encompasses all interactions from viewers during the livestreaming sessions. These interactions include entering the livestream, chatting, gifting, and paying for membership. We provide a full data schema description in Appendix \ref{subsec:appendix_data_description}.
In total, we obtain a dataset of 2.7 million live sessions, with 10.7 billion interaction records, covering 3.6 billion chat messages. We also note that part of our collected data comes from the \texttt{danmakus.com} website.
See Appendix \ref{subsec:appendix_ethics} for ethical considerations.

%%
\pb{Identifying Members.}
\label{subsec:construct_dataset}
Recall, our study focuses on members (\ie viewers who have paid for a membership). 
Thus, it is necessary to compile a list of members from our collected dataset.
We therefore compile two separate lists of \one~members, and \two~non-members (for comparison). 
We search our collected dataset for any membership purchase record from the period \texttt{2022-07-16} to \texttt{2023-07-16}. This start and end date ensures we have data for a sufficiently long period (45 days) before and after the membership.
For each paid membership, we add it to the member list if it is the \emph{first time} the user has paid for a membership for this VTuber. If it is not the first time, it will not be included in either the member list or the non-member list, as it would be impossible for us to study their before-membership period. 
Then, we add to the non-member list all other viewers in the same livestreaming session who have \emph{never} paid for a membership during our measurement period. 
Note, a viewer can appear multiple times if they participate in different livestreaming sessions. In total, we obtain 608k membership records from 337k unique viewers. 



Throughout the remainder of this paper, we refer to ``member(s)'', even if our analysis covers the time period before they had paid for their membership. We use ``members pre-purchase'' and ``members post-purchase'' to refer to members before and after the purchase of membership. We use \textbf{\emph{the ``membership-ed'' VTuber}} to refer to the VTuber that the membership pertains to.





\label{subsec:nlp_methods}
\pb{Chat Messages Preprocessing.}
We later analyze chat messages. To facilitate this, we convert all chat messages into their text embeddings, to provide greater semantic insight into the content. 
For this, we use the \texttt{text2vec-base-chinese} embedding model, which has been reported to have high overall performance \cite{Text2vec, embeddingTest}.

We also inspect the toxicity in chat messages. To quantify this, we employ the OpenAI moderation API to label all messages. This tool allows us to obtain toxicity scores (0--1) across various categories, such as sexual content, harassment, hate speech, and violence, \etc We apply a threshold of 0.5 to determine whether a chat message is considered toxic in a given category.




