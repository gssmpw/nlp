\section*{Methods}
%We propose DCFormer, a novel token mixer to efficiently process 3D CT images in the context of a CLIP-based framework. DCFormer addresses the computational challenges inherit in volumetric 3D CT scans by decomposing the computationally expensive 3D convolution into several 1D operations across all spatial dimensions. DCFormer utilizes MetaFormer\cite{yu2022metaformer, yu2023metaformer} as the backbone block design and uses decomposed convolutions as the token mixer for feature extraction. It also enjoys a hierarchical structure which ultimately enhances feature extraction capacity for 3D CT images. Finally, the proposed DCFormer-based image encoder is integrated into a CLIP framework to achieve text-image alignment. As we will show later, such a decomposition strategy significantly reduces parameter count and computational costs while preserving high zero-shot and fine-tuning performance. 

%\subsection*{DCFormer}


Processing 3D CT images presents significant computational challenges due to their high resolution and volumetric complexity. 3D convolution is computationally expensive due to its cubic scaling with kernel size, and the self-attention mechanism further amplifies the burden by scaling quadratically with input size. To address these challenges, we propose DCFormer, a hybrid architecture that decomposes 3D convolution into multiple 1D components, significantly reducing computational overhead. DCFormer adopts MetaFormer \cite{yu2022metaformer, yu2023metaformer} as its backbone and employs decomposed convolutions as the token mixer for feature extraction. Its hierarchical structure further enhances feature representation in 3D CT images. Finally, the DCFormer-based image encoder is integrated into a CLIP framework for image-text alignment. As demonstrated in our results, this decomposition strategy substantially reduces parameter count and computational cost while maintaining high performance in both zero-shot and fine-tuning settings.

% [old version] Processing 3D CT images brings significant computational challenges due to their high input resolution and the complexity of volumetric data. Especially in the CLIP framework using 3D vision encoder, the computational burden is amplified since a text encoder must be jointly trained with the vision encoder. Although 3D convolution is very effective in capturing spatial features across three dimensions, it is computationally expensive, especially when applied to high-resolution volumetric images. Similarly, the self-attention mechanism also suffers from quadratic scaling with respect to the input size which further amplifies the computational burden when applied to 3D volumetric data. To address these challenges, we propose DCFormer, a hybrid architectural family that decomposes 3D convolution into several 1D components which ultimately reduces computational overhead. DCFormer utilizes MetaFormer\cite{yu2022metaformer, yu2023metaformer} as the backbone block design and uses decomposed convolutions as the token mixer for feature extraction. It also enjoys a hierarchical structure which ultimately enhances feature extraction capacity for 3D CT images. Finally, the proposed DCFormer-based image encoder is integrated into a CLIP framework to achieve text-image alignment. As we show in the results section, such a decomposition strategy significantly reduces parameter count and computational costs while preserving high zero-shot and fine-tuning performance. 


\subsubsection*{Formulation of DCFormer block} %% may need some updates attention part

The core concept, \textit{MetaFormer} \cite{yu2022metaformer, yu2023metaformer}, is a general architecture abstracted from transformers \cite{vaswani2017attention}. In the Metaformer architecture, the token mixer (e.g. self-attention, depthwise convolution) is not specified, while normalization, a channel mixer (e.g. MLP), and residual connections \cite{he2016deep} are retained. The input \( I \) is first passed into a patch embedding block, such as a convolution:
\begin{align}
X = \text{PatchEmbed}(I)=Flatten(\text{Conv}^{(k, s, p)}_{C_0 \rightarrow C} (I)) \label{eq1}
\end{align}
where \( X \in \mathbb{R}^{N \times C} \) denotes the embedded tokens with sequence length \( N \) and embedding dimension \( C \). Here, $C_0, k, s$ and $p$ represent input image channels, kernel size, stride and padding, respectively. These embedded tokens are then fed into the Metaformer architecture: 
\begin{align}
X' &= X + \text{TokenMixer}(\text{Norm}_1(X)), \label{eq2} \\
X'' &= X' + \sigma(\text{Norm}_2(X')W_1)W_2. \label{eq3}
\end{align}
Here, $\text{Norm}_1$ and $\text{Norm}_2$ are typically batch normalization \cite{ioffe2015batch} or layer normalization \cite{ba2016layer}. The TokenMixer serves as the core module for spatial information interaction, $W_1$ and $W_2$ are learnable weights in a two-layer channel MLP, and $\sigma$ is a non-linear activation \cite{nair2010rectified, hendrycks2016gaussian}. 


% [Old version] The core concept \textit{MetaFormer} \cite{yu2022metaformer, yu2023metaformer} is a general architecture abstracted from transformers \cite{vaswani2017attention}. In a general Metaformer architecture, the token mixer (e.g. self-attention, depthwise convolution) is typically not specified while normalization, channel-mixer (e.g. MLP) and residual connections \cite{he2016deep} are retained. The input \( I \) is first passed into a patch embedding block such as convolution:
% \begin{align}
% X = \text{PatchEmbed}(I)=Flatten(\text{Conv}^{(k, s, p)}_{C_0 \rightarrow C} (X)) \label{eq1}
% \end{align}
% where \( X \in \mathbb{R}^{N \times C} \) denotes the embedded tokens with sequence length \( N \) and embedding dimension \( C \). $C_0, k, s$ and $p$ represent input image channels, kernel size, stride and padding, respectively. Then these embedded tokens are fed into the Metaformer architecture: 
% \begin{align}
% X' &= X + \text{TokenMixer}(\text{Norm}_1(X)), \label{eq2} \\
% X'' &= X' + \sigma(\text{Norm}_2(X')W_1)W_2. \label{eq3}
% \end{align}
% Here, $\text{Norm}_1$ and $\text{Norm}_2$ are typically either batch \cite{ioffe2015batch} or layer \cite{ba2016layer} normalization. TokenMixer is the main block for spatial information interaction, $W_1$ and $W_2$ are learnable weights in a two-layer channel MLP and $\sigma$ is a non-linear activation \cite{nair2010rectified, hendrycks2016gaussian}. 


To further enhance the MetaFormer architecture, we introduce Decomposed Convolution as the token mixer within the DCFormer block. This design leverages the computational efficiency of decomposed 1D convolutional operations along each spatial axis (height, width, and depth). By splitting the 3D convolution into three parallel 1D convolutions, Decomposed Convolution captures spatial features while significantly reducing the number of parameters and computational cost. Thus, the DCFormer block integrates decomposed convolutions as a lightweight yet powerful token mixer. Let \( X \in \mathbb{R}^{B \times C \times H \times W \times D} \) denote the input feature map, where \( B \) is the batch size, \( C \) is the number of channels, and \( H \), \( W \), and \( D \) represent the spatial dimensions (height, width, and depth), respectively. The decomposed convolution consists of three 1D depthwise convolutions, processing the input tensor along each spatial axis:
\begin{align}
X^*_{\text{h}} &= \text{DWConv}^{k_h \times 1 \times 1}_{C \rightarrow C} (X), \\
X^*_{\text{w}} &= \text{DWConv}^{1 \times k_w \times 1}_{C \rightarrow C} (X), \\
X^*_{\text{d}} &= \text{DWConv}^{1 \times 1 \times k_d}_{C \rightarrow C} (X) 
\end{align}
where ($k_h, k_w, k_d$) represents the kernel sizes in height, width and depth dimensions, respectively. In our implementation, we set $k_h = k_w = k_d = k \in \{13, 11, 9, 7\}$ to leverage large kernels while maintaining computational efficiency through decomposition.
After applying decomposed convolutions along each spatial axis, we normalize the resulting features separately and then combine them using elementwise summation to form the DCFormer block: 
\begin{align}
X' &= X + \text{Norm}_h(X^*_{\text{h}}) + \text{Norm}_w(X^*_{\text{w}}) + \text{Norm}_d(X^*_{\text{d}})
\end{align}

An illustration of the DCFormer and its Pytorch-like implementation are shown in Figure \ref{fig:deconvnext} and Algorithm \ref{algo:decompconv3d}, respectively.

% [Old version] To further enhance the MetaFormer architecture, we introduce Decomposed Convolution as the token mixer within the DCFormer block. This design leverages the computational efficiency of decomposed 1D convolutional operations along each spatial axis (height, width, and depth) instead of traditional 3D convolutions. By splitting the 3D convolution into three parallel 1D convolutions, Decomposed Convolution can capture spatial features while significantly reducing the number of parameters and computational costs. Thus, the DCFormer block integrates decomposed convolutions as a lightweight yet powerful token mixer. Let the input tensor \( X \in \mathbb{R}^{N \times C \times H \times W \times D} \) represent the input feature map with batch size \( N \), channels \( C \), height \( H \), width \( W \), and depth \( D \). The decomposed convolution consists of three 1D convolutions, processing the input in each spatial axis:
% \begin{align}
% X^*_{\text{h}} &= \text{DWConv}^{k_h \times 1 \times 1}_{g \rightarrow g} (X), \\
% X^*_{\text{w}} &= \text{DWConv}^{1 \times k_w \times 1}_{g \rightarrow g} (X), \\
% X^*_{\text{d}} &= \text{DWConv}^{1 \times 1 \times k_d}_{g \rightarrow g} (X) 
% \end{align}
% where ($k_h, k_w, k_d$) represent the kernel sizes in height, width and depth dimensions, respectively. Note that in our implementation, $k_h = k_w = k_d = k \in \{13, 11, 9, 7\}$, aiming to leverage large kernels with our decomposition strategy.
% After applying decomposed convolutions along each spatial axis, we normalize each of the resulting components separately and then combine them using elementwise summation to form the DCFormer block: 
% \begin{align}
% X' &= X + \text{Norm}_h(X^*_{\text{h}}) + \text{Norm}_w(X^*_{\text{w}}) + \text{Norm}_d(X^*_{\text{d}})
% \end{align}
% The illustration of the DCFormer and its Pytorch-like code is shown in Figure \ref{fig:deconvnext} and Algorithm \ref{algo:decompconv3d}, respectively.

\begin{figure*}[t]
\centering
\includegraphics[width=12cm]{figures/deconvnext.png}\\
\caption{Block illustration of MetaNeXt, ConvNeXt and \dc.}
\label{fig:deconvnext} 
\end{figure*}

\section{The general case: Proof of \texorpdfstring{\Cref{thm:main-decomp}}{Theorem 1.6}}\label{sec:algo}

First, we show that data structure of \Cref{l:max_min_query} can be used to compute distances witnessed by shortest paths that pass through a constant-size separator.

\begin{lemma}\label{l:single_adhesion}
Fix a constant $k \in \mathbb{N}$. There exists an algorithm which as the input receives an edge-weighted graph $G$ on $n$ vertices and $m$ edges together with a partition of its vertices into three sets $A, B, C$ such that $|B| \leq k$ and there are no edges between $A$ and $C$, and as the output computes $\max_{c \in C} \dist(a, c)$ for every $a \in A$. The running time is $\Oh(m \log n + n \log^{k - 1} n)$.
\end{lemma}

\begin{proof}
Let $B = \{b_1, \ldots, b_k\}$. For any $a \in A, c \in C$, we have $\dist(a, c) = \min_{i \in [k]} \dist(a, b_i) + \dist(c, b_i)$. First, we run Dijkstra's algorithm from every vertex in $B$ to find $\dist(v, b_i)$ for every $v \in V(G)$ and $i \in [k]$. Next, we use \Cref{l:max_min_query} to construct a data structure $\mathbb{D}$ for the point set $\{(\dist(c, b_1), \dots, \dist(c, b_k))\colon c\in C\}\subseteq \mathbb{R}^k$. Now, the value $\max_{c \in C} \dist(a, c)$ for any given $a$ is equal to the answer of $\mathbb{D}$ to the query with argument $(\dist(a, b_1), \dots, \dist(a, b_k))$.
\end{proof}

After computing the distances over a constant-size separator, we will use the following observation to simplify one of the sides of the separation.

\begin{lemma}\label{l:inserting_paths}
Let $G$ be a edge-weighted connected graph and let $A, B, C$ be a partition of its vertices such that there are no edges between $A$ and $C$. For every pair of vertices $u, v \in B$, let $P_{u, v}$ be any shortest path from $u$ to $v$ with all internal vertices in $C$ (assuming such a path exists).

Let $G'$ denote a graph obtained from $G[A \cup B]$ by adding an edge from $u$ to $v$ of weight equal to the length of $P_{u, v}$, for all $u, v \in B$ for which $P_{u, v}$ exists. Then,  $$\dist_G(s, t) = \dist_{G'}(s, t)\qquad\textrm{for all }s,t\in A\cup B.$$
\end{lemma}
\begin{proof}
Let $G''$ be the graph obtained by adding new edges of $G'$ to $G$.
Fix any $s, t \in A \cup B$ and let $P$ denote the shortest path from $s$ to $t$ in $G''$ which minimizes the number of vertices from $C$ visited. Naturally, the weight of $P$ is equal $\dist_G(s, t)$. Assume that such path visits at least one vertex of $C$. Then, the path $P$ is of the form $s \xrightarrow{P_1} x \xrightarrow{P_2} y \xrightarrow{P_3} t$, where $x, y \in B$ and all the internal vertices of $P_2$ are in $C$. By the construction of $G'$, $P_2$ can be replaced with a direct edge from $x$ to $y$ of the same weight. We obtain a same weight path with a smaller number of vertices of $C$ visited, which is a contradiction. Therefore, $P$ is entirely contained in $A \cup B$, hence it exists in $G'$. This shows that $\dist_G(s, t) = \dist_{G'}(s, t)$.
\end{proof}


The next lemma encapsulates the main algorithmic content of the proof of \Cref{thm:main-decomp}. The algorithm will split the tree decomposition provided on input into smaller parts for which the eccentricities are easier to calculate. We use the following lemma to handle a single such part.
\begin{lemma}\label{l:star}
Fix constants $k, g \in \mathbb{N}, 0 < \delta < \frac{1}{54}$. Assume we are given $n \in \mathbb{N}$, an edge-weighted graph $G$ on at most $n$ vertices with a weight function $w \colon E(G) \to \mathbb{N}$, a vertex subset $A$ and a collection of non-empty vertex subsets $V_0, V_1, \dots, V_\ell$ satisfying the following conditions:
\begin{itemize}[nosep]
	\item The sum of weights of all the edges in $G$ is bounded by $\Oh(n)$.
	\item $V(G) \setminus A = V_0 \cup V_1 \cup \dots \cup V_\ell$.
	\item $|A| \leq k$.
	\item For every $i \in [\ell]$, $G[V_i \setminus V_0]$ is connected, $N_G(V_i \setminus V_0) = V_i \cap V_0$, $|V_i| = \Oh(n^\delta)$, and $|V_0 \cap V_i| \leq 4$.
	\item For all $i, j \in [\ell], i \neq j$, $V_i \setminus V_0$ and $V_j \setminus V_0$ are disjoint and non-adjacent in $G$.
	\item Every edge $uv \in E(G)$ with $u, v \not\in A$ is contained in $G[V_i]$ for some $i\in \{0,1,\ldots,\ell\}$.
	\item The graph obtained by taking $G[V_0]$ and adding a clique on $V_0 \cap V_i$ for every $i \in [\ell]$ has Euler genus bounded by $g$.
\end{itemize}
Then, we can compute the eccentricity of every vertex of $G$ in time $\Oh \left( n^{1 + \frac{150 + 54 \delta}{151}} \log^k n \right)$.
\end{lemma}

\begin{proof}
Fix $\delta' = \frac{1 + 97 \delta}{151}$; we have $\delta' - \delta = \frac{1 - 54\delta}{151} > 0$.
Let $E_i$ denote the set of edges with one endpoint in $V_i$ and the other endpoint in $V_i \setminus V_0$. For $i \in [\ell]$, we shall say that $V_i$ is {\em{heavy}} if the sum of weights of $E_i$ is larger than $n^{\delta'}$. Since the sets $E_i$ are pairwise disjoint and the total sum of weights of all the edges is bounded by $\Oh(n)$, the number of heavy subsets is bounded by $\Oh(n^{1 - \delta'})$. Without loss of generality, we may assume that $V_{\ell' + 1}, \dots, V_\ell$ are heavy and $V_1, \dots, V_{\ell'}$ are not, for some $\ell'\in \{0,\ldots,\ell\}$.


For any source vertex $s$, we can calculate distances from $s$ to every vertex of $G$  using breadth first search in time $\Oh(\sum_{e \in E(G)} w(e)) = \Oh(n)$.
In particular, for every $\ell' < i \leq \ell$, we can compute the distances from every vertex of $V_i$ to every vertex of $G$ in total time $\Oh(n^{2 - \delta' + \delta})$, because $$|V_{\ell'+1}\cup \ldots\cup V_{\ell}|\leq n^{1-\delta'}\cdot \Oh(n^\delta)=\Oh(n^{1-\delta'+
\delta}).$$
Additionally, we calculate distances $\dist_G(a, v)$ for every $a \in A, v \in V(G)$ in time $O(n)$.

For every $i \in [\ell]$ and $u,v \in V_0 \cap V_i$, there exists a shortest path $P_{i,u,v}$ from $u$ to $v$ with all internal vertices belonging to $V_i - V_0$ due to the assumption that $G[V_i - V_0]$ is connected and $N_G(V_i - V_0) = V_i \cap V_0$. Therefore, the distance from $u$ to $v$ is bounded by the sum of weights of edges in $E_i$. In particular, for $i \in [\ell']$, $\dist_G(u, v) \leq n^{\delta'}$.

We define $\widetilde{G}$ to be the graph obtained by taking $G[A \cup V_0 \cup \dots \cup V_{\ell'}]$ and applying the following operation for every $i \in \{\ell' + 1, \dots, \ell\}$:
for each pair of vertices $u, v \in A \cup (V_0 \cap V_i)$, add an edge in $\widetilde{G}$ between $u$ and $v$ with weight equal to the total weight of $P_{i,u,v}$. For a fixed $i, u$, we can find $P_{i, u, v}$ for all $v$ using breadth first search in time $\Oh(n)$. Taking a sum over all $i, u$, we get that $\tilde{G}$ can be computed in total time $\Oh(n^{2 - \delta'})$.


\begin{claim}\label{cl:wG}
The sum of the edge weights in $\widetilde{G}$ is $\Oh(n)$. Moreover, for all $u, v \in V(\widetilde{G})$, we have $\dist_{\widetilde{G}}(u, v) = \dist_{G}(u, v)$.
\end{claim}

\begin{proof}
Consider $i \in \{\ell' + 1, \dots, \ell\}$ and any $u, v \in A \cup (V_0 \cap V_i)$ for which we added an edge. Its weight is bounded by the sum of weights of edges in $E_i$. Therefore, the total weight of all edges added is at most
$$
\sum_{i \in \{\ell' + 1, \dots, \ell\}} \left( |A \cup (V_0 \cap V_i)|^2 \sum_{e \in E_i} w(e) \right) \leq (4 + k)^2 \sum_{e \in E(G)} w(e) = \Oh(n).
$$
This proves the first part of the claim.

For the second part of the claim, consider any $i \in \{\ell' + 1, \dots, \ell \}$ and observe that by our assumptions, $A \cup (V_0 \cap V_i)$ separates $(V_0 \cup \dots \cup V_{\ell'} \cup V_{i + 1} \cup \dots \cup V_\ell) \setminus V_i$ from $V_i \setminus V_0$. Hence it suffices to repeatedly apply \Cref{l:inserting_paths}.
\end{proof}

For every $u \in V(\widetilde{G})$, we have $\ecc_G(u) = \max(\ecc_{\widetilde{G}}(v), \max_{v \in V(G) \setminus V(\widetilde{G})} \dist_G(u, v))$. Note, that we already know all the distances $\dist_G(u, v)$ for $v \in V(G) \setminus V(\widetilde{G})$. Similarly, we can already compute $\ecc_G(u)$ for every $u \in V(G) \setminus V(\widetilde{G})$. Therefore, it remains to compute $\ecc_{\widetilde{G}}(v)$ for each $v \in V(\widetilde{G})$. Our goal is to show that this can be done efficiently using \Cref{l:main_ecc}.

Now, let $G'$ be the graph obtained from $\tilde{G}$ by replacing every edge $e$ non-indicent to $A$ with $w(e)\geq 2$ with a path of length $w(e)$ consisting of unit-weight edges. This operation again preserves the distances. Since the sum of edge weights in $\tilde{G}$ is of $\Oh(n)$, the total number of vertices in $G'$ is of $\Oh(n)$. For $0 \leq i \leq \ell'$, we write $V'_i$ to denote the set $V_i$ together with all the vertices added as a part of a path between two endpoints in $V_i$.
As $V_i$ is not heavy for each $i\in [\ell']$, we have
$$
|V'_i \setminus V'_0| \leq |V_i| + \sum_{e \in E_i} w(e) = \Oh(n^{\delta'})\qquad \textrm{for all }i\in [\ell'].
$$

Let $G_0$ denote the graph $G'[V'_0]$ and let $G_0^*$ denote the graph $G'- A$ with $V'_i - V'_0$ contracted to a single vertex $v_i^*$, for each $i \in [\ell']$; note that, all edges of $G_0$ and $G_0^*$ have unit weight.

\begin{claim}
	The graph $G_0^*$ is does not contain $K_{t}$ as a minor, where $t = \Oh(\sqrt{g})$.
\end{claim}

\begin{proof}
Let $\bar{G}_0$ denote the graph obtained by taking $G_0$ and adding a clique on $V_0 \cap V_i$ for every $i \in [\ell']$.
By lemma assumptions and the fact that subdividing edges does not increase the Euler genus, $\bar{G}_0$ has Euler genus at most $g$. In particular, $\bar{G}_0$ is $K_{t'}$-minor-free for some $t' = \Oh(\sqrt{g})$, because the Euler genus of $K_{t'}$ is $\Omega({t'}^2)$.

Similarly, let $\bar{G}_0^*$ be the graph obtained by taking $G_0^*$ and adding a clique on each $V_0 \cap V_i$.
Note, that $\bar{G}_0^* - \{v_1^*, \dots, v_{\ell'}^*\}$ is precisely $\bar{G}_0$. Let $t = \max(t', 6)$.
Recall that a minor model of a clique $K_t$ consists of $t$ pairwise vertex-disjoint connected subgraphs, called
branch sets, such that there is at least one edge between each pair of the branch sets.
Consider a minor model $\varphi$ of $K_{t}$ in $\bar{G}^*_0$.
Note that $\varphi$ cannot contain any singleton branch set of the form $\{v^*_i\}$, for the degree of $v^*_i$ in $\bar{G}^*_0$ is at most $4 < t - 1$. Furthermore, since $N_{\bar{G}^*_0}(v^*_i) = V_0 \cap V_i$, any branch set containing $v^*_i$ and at least one other vertex contains some $u \in V_0 \cap V_i$, and $N_{\bar{G}^*_0}(v^*_i)\subseteq N_{\bar{G}^*_0}(u)$, hence removing $v^*_i$ from this branch set preserves the model. Therefore, we can assume without loss of generality that all branch sets of $\varphi$ are disjoint from $\{v^*_1, \dots, v^*_{\ell'}\}$, hence $\varphi$ is a minor model of $K_{t}$ in $\bar{G}_0$. This is a contradiction, as $t \geq t'$ and $\bar{G}_0$ is $K_{t'}$-minor-free. Therefore, $\bar{G}_0^*$ is $K_t$-minor-free, hence $G_0^*$ also.
\end{proof}

Let $\rho' = \frac{2 - 108 \delta}{151} > 0$. The graph $G^*_0$ is a unit-weight graph and is $K_{t}$-minor-free.
Hence, by applying \Cref{t:r_division} to $G^*_0$ (with $\varepsilon = \rho'/2$)
we obtain an $n^{\rho'}$-division $\mathcal{R}_0$ in time $\Oh(n^{1 + \rho'})$.
We extend it to $G' - A$ by mapping every contracted vertex $v^*_i$ to $N_{G' - A}[V'_i - V'_0] = (V'_i - V'_0) \cup (V_0 \cap V_i)$. Formally, we put $V''_i \coloneqq N_{G' - A}[V'_i - V'_0]$ and 
$$
\mathcal{R} \coloneqq \left\{ (R_0 \cap V'_0) \cup \bigcup_{i \colon v^*_i \in R_0} V''_i \colon R_0 \in \mathcal{R}_0 \right\}.
$$

Now, we argue that $\mathcal{R}$ is a reasonable division of $G' - A$. Clearly, all sets $R \in \mathcal{R}$ are connected in $G' - A$. Pick any $R \in \mathcal{R}$ and let $R_0$ be its corresponding set in $\mathcal{R}_0$.
Every vertex $v^*_i$ is mapped to a set of size $\Oh(n^{\delta'})$, therefore
$$|R| \leq |R_0| \cdot \Oh(n^{\delta'}) = \Oh(n^{\rho' + \delta'}).$$

By our construction, for every $i \in [\ell']$, $R$ is either disjoint from $V'_i - V'_0$ or contains whole $N_{G' - A}[V'_i - V'_0]$. This means that no vertex belonging to any $V'_i - V'_0$ can be in $\partial R$, hence $\partial R \subseteq V'_0$.

Pick any $u \in \partial R \cap R_0$. Assume that $u \not\in \partial R_0$. Then every vertex of $N_{G_0^*}(u)$ must be in $R_0$, hence $N_{G - A'}(u) \subseteq R$, which is a contradiction. This means that $\partial R \cap R_0 \subseteq \partial R_0$.

Pick any $u \in \partial R - R_0$. Then, $u \in V_0 \cap V_i$ for some $i \in [\ell']$ such that $v_i^* \in R_0$. Moreover, $v_i^* \in \partial R_0$ and is adjacent to $u$ in $G_0^*$. The number of such $u$ is bounded by $4 |\partial R_0 \cap \{ v_1^*, \dots, v_{\ell'}^* \}|$.

Putting two cases together, we obtain:
$$
\sum_{R \in \mathcal{R}} |\partial R| = \sum_{R \in \mathcal{R}} \left(|\partial R \cap R_0| + |\partial R - R_0|\right) \leq \sum_{R_0 \in \mathcal{R}_0} \left(|\partial R_0| + 4 |\partial R_0 \cap \{ v_1^*, \dots, v_{\ell'}^* \}|\right) = \Oh(n^{1 - \frac{1}{2}\rho'}).
$$

It remains to show the following claim.

\begin{claim}
Pick any $R \in \mathcal{R}, s_R \in R$. The number of different distance profiles on $R$ relative to $s_R$ in $G' - A$ is of $\Oh(n^{48\rho' + 54\delta'})$.
\end{claim}
\begin{proof}
We look at every vertex $v \in V(G') \setminus A$ and consider three cases: $v \in R$, $v \in V'_0$, and $v \in V'_i \setminus (V'_0 \cup R)$ for some $i \in [\ell']$. By our construction, $R \cap V'_0$ is non-empty, hence w.l.o.g. we can assume that $s_R \in V'_0$ as whether two vertices have the same profile on $R$ is independent of the choice of the pivot vertex.

In the first case, there are at most $|R| = \Oh(n^{\rho' + \delta'})$ such vertices, hence they realise at most that many profiles.

In the second case, we want to observe that profile of any vertex $v \in V'_0$ on $R$ depends only on its profile on $R \cap V'_0$ (relative to $s_R$). Pick any $t \in R - V'_0$. Then $t \in V'_i - V'_0$ for some $i \in [\ell']$, $V_i \cap V_0 \subseteq R \cap V'_0$, and every path from $v$ to $t$ intersects $V_i \cap V_0$. In particular, distances from $v$ to vertices of $V_i \cap V_0$ determine its distance to $t$, which proves the observation.

Let $\tilde{G}_0$ denote the graph obtained by taking $G'[V'_0]$ and for every $i \in [\ell'], u, v \in V_0 \cap V_i$ adding a disjoint path from $u$ to $v$ of length $\dist(u, v)$. Let $P_i$ denote the vertex set of paths added between $V_0 \cap V_i$. For every $t \in V'_0$ we have $\dist_{G' - A}(v, t) = \dist_{\tilde{G}_0}(v, t)$, so it suffices to bound the number of profiles on $R \cap V'_0$ in $\tilde{G}_0$. By our assumptions, $\tilde{G}_0$ has Euler genus bounded by $g$ and all $P_i$ are of size $\Oh(n^{\delta'})$.

Let $R_0$ be the set of $\mathcal{R}_0$ corresponding to $R$. Let $\tilde{R}_0$ denote the set $(R \cap V'_0) \cup \bigcup_{i : v^*_i \in R_0} P_i$. Such set is connected in $\tilde{G}_0$. Moreover, similarly to $R$, its size is $\Oh(n^{\rho' + \delta'})$. Applying \Cref{thm:distprofiles}, we get that the number of distance profiles on $\tilde{R}_0$ in $\tilde{G}_0$ is $\Oh(n^{12(\rho' + \delta')})$, which also bounds the number of profiles on $R$ in $G' - A$ realised by $V'_0$.

For the third case, assume $v \in V'_i \setminus (V'_0 \cup R)$ for some $i\in [\ell']$. Every path from $v$ to any vertex of $R$ in $G' - A$ intersects $V_i \cap V_0$. Let $v_1, \dots v_p$ be the vertices of $V_i \cap V_0$, where $p \leq 4$. The profile of $v$ on $R$ is then determined by the following:
\begin{itemize}[nosep]
 \item[(a)] the profile of each $v_j$ on $R$,
 \item[(b)] $\dist_{G' - A}(v, v_j) - \dist_{G' - A}(v, v_1)$ for each $2 \leq j \leq p$, and
 \item[(c)] $\dist_{G' - A}(s_R, v_j) - \dist_{G' - A}(s_R, v_1)$ for each $2 \leq j \leq p$ where $s_R$ is some pivot vertex of $R$.
\end{itemize}
By the previous case, the number of distance profiles of each $v_j$ is $\Oh(n^{12(\rho' + \delta')})$. The distances between $v$ and $v_j$ are bounded by $|V'_i|$, hence each quantity described in (b) can take $\Oh(n^{\delta'})$ different possible values. Similarly, since $v_1$ and $v_j$ are connected via $V'_i$, $|\dist_{G' - A}(s_R, v_j) - \dist_{G' - A}(s_R, v_1)| \leq \Oh(n^{\delta'})$. The number of different possible profiles of such $v$ is therefore bounded by $\Oh(n^{48(\rho' + \delta') + 6\delta'}) = \Oh(n^{48\rho' + 54\delta'})$. This finishes the proof of the claim.
\end{proof}

Now we can apply \Cref{l:main_ecc} to graph $G'$ with apex set $A$, $X = V(\widetilde{G})$, and the following constants: $$\rho = \rho' + \delta',\qquad \gamma = 1 - \frac{1}{2}\rho',\quad \textrm{and}\quad \alpha = 48\rho' + 54 \delta'.$$ This allows us to calculate all $V(\widetilde{G})$-eccentricities in $G'$ in time
$$
\Oh \left( \left(
	n^{ 2 - \frac{1}{2} \rho' } +
	n^{ 1 + 48\rho' + 54 \delta' }
\right) \log^k n \right) =
\Oh \left( n^{1 + \frac{150 + 54 \delta}{151}} \log^k n \right).
$$
Since for each $v\in V(\widetilde{G})$ we have $\ecc_{\widetilde{G}}(v) = \max_{u \in V(\widetilde{G})} \dist_{\widetilde{G}}(v, u) = \max_{u \in V(\widetilde{G})} \dist_{G'}(v, u)$, this means that we have successfully computed all the eccentricities in $\widetilde{G}$; and as we argued, this is enough to compute all the eccentricities in $G$ as well.

Finally, the total running time of the algorithm is
$$
\Oh \left( n^{1 + \frac{150 + 54 \delta}{151}} \log^k n + n^{2 - \delta' + \delta} \right) =
\Oh \left( n^{1 + \frac{150 + 54 \delta}{151}} \log^k n \right).
$$\qedhere
\end{proof}


\begin{lemma}\label{l:star2}
Fix constants $k, g \in \mathbb{N}, 0 < \delta < \frac{1}{54}$. Assume we are given $n \in \mathbb{N}$, an edge-weighted graph $G$ on at most $n$ vertices with a weight function $w \colon E(G) \to \mathbb{N}$, a vertex subset $A$ and a collection of non-empty vertex subsets $V_0, V_1, \dots, V_\ell$ satisfying the same conditions as in \Cref{l:star} with the following differences:
\begin{itemize}
	\item we don't require $G[V_i - V_0]$ to be connected and $V_i - V_0$ to be adjacent to whole $V_i \cap V_0$;
	\item instead of $|V_0 \cap V_i| \leq 4$, we require $|V_0 \cap V_i| \leq k$.
\end{itemize}
Then, we can compute the eccentricity of every vertex of $G$ in time $\Oh \left( n^{1 + \frac{150 + 54 \delta}{151}} \log^{k + 5g} n \right)$.
\end{lemma}

\begin{proof}
We will reduce our input to one which will satisfy the conditions of \Cref{l:star}. We start by addressing the adhesions $V_0 \cap V_i$ containing too many vertices.

Let $G_0$ denote the graph $G[V_0]$ with cliques placed at $V_0 \cap V_i$ for every $i \in [\ell]$.
For every $i \in [\ell]$ we repeat the following procedure: while $|V_0 \cap V_i| > 4$,
remove arbitrary $5$ vertices from $V_0 \cap V_i$. Since $|V_0 \cap V_i| \leq k$ for each $i\in [\ell]$,
this procedure can be implemented in total time $\Oh(n)$. As a result, at the end we have $|V_0 \cap V_i| \leq 4$ for all $i \in [\ell]$. Let $M$ be the set of all the removed vertices. By our assumptions, $G_0$ has Euler genus bounded by $g$, hence it cannot contain $g + 1$ pairwise disjoint copies of $K_5$
(as the Euler genus of a graph is the sum of the Euler genera of its 2-connected components~\cite{StahlB77} and $K_5$ is not planar). Each removed quintiple of vertices induces a $K_5$ in $G_0$, hence we have $|M| \leq 5g$. We set $A' = A \cup M$ and may thus assume that $V_i$ is disjoint from $A'$ for all $0 \leq i \leq \ell$.

Now, fix $i \in [\ell]$. Let $C^i_1, \dots, C^i_{r_i}$ denote the connected components of $V_i - V_0$ in $G - A'$. We define $W^i_j := N_{G - A'}[C^i_j]$ for every $j \in [r_i]$. Clearly, all $W^i_j$ induce a connected subgraph of $G$ and satisfy $N_{G - A'}(W^i_j - V_0) = W^i_j \cap V_0$. We put $V'_0 := V_0$ and enumerate
$$
\{V'_1, V'_2, \dots V'_{\ell'}\} := \{ W^i_j \colon i \in [\ell], j \in [r_i] \}.
$$
It is easy to verify that the sets $A'$ and $V'_0, V'_1, \dots, V'_{\ell'}$ satisfy the conditions of \Cref{l:star}. We apply said lemma to calculate the eccentricity of every vertex of $G$ in the desired time.
\end{proof}



The next statement is a reformulation of \Cref{thm:main-decomp}.

\begin{theorem}
Fix constants $k, g \in \mathbb{N}$. Assume we are given a graph $G$ on $n$ vertices together with its tree decomposition $(T, \beta)$ and a set of private apices $A_t \subseteq \beta(t)$ for each node $t\in V(T)$ such that the following conditions hold:
\begin{itemize}[nosep]
 \item For every node $t \in V(T)$, we have $|A_t| \leq k$.
 \item For every edge $st \in E(T)$,  we have $|\beta(v) \cap \beta(u)|\leq k$.
 \item For every node $t \in V(T)$, graph obtained by taking $G[\beta(t)] - A_t$ and turning  $(\beta(t) \cap \beta(s))\setminus A_t$ into a clique for every edge $st \in E(T)$ has Euler genus bounded by $g$.
\end{itemize}
Then, we can compute the eccentricity of every vertex of $G$ in time $\Oh \left( n^{1 + \frac{355}{356}} \log^{k + 5g} n \right)$.
\end{theorem}

\begin{proof}
We may assume that $|V(T)|\leq n$, for every tree decomposition with no two bags comparable by inclusion has this property; and adjacent comparable bags can be merged by contracting the edge between them.

For a node $t\in V(T)$, by the {\em{weight}} of $t$ we mean the size of the corresponding bag, that is, $|\beta(t)|$. For any subset of nodes $S \subseteq V(T)$, we define $\beta(S) \coloneqq \bigcup_{t \in S} \beta(t)$ By the {\em{weight}} of $S$, we mean the total weight of the elements of $S$, that is, $\sum_{t\in S} |\beta(t)|$. 

\begin{claim}\label{cl:weight-T}
The weight of $V(T)$ is of $\Oh(n)$.
\end{claim}

\begin{proof}
The sets $\beta'(t) := \beta(t) - \bigcup_{s \in N_T(t)} \beta(s)$ are pairwise disjoint. We have
$$
\sum_{t \in V(T)} |\beta(t)| =
\sum_{t \in V(T)} |\beta'(t)| + 2 \cdot \sum_{st \in E(T)} |\beta(s) \cap \beta(t)| \leq
|V(T)| + 2k|E(T)| = \Oh(n).
$$
\end{proof}

Since every bag induces a graph of bounded Euler genus, the number of edges contained in a bag is linear in its size. In particular, this implies that the total number of edges of $G$ is also bounded by $\Oh(n)$.

We set $$\delta \coloneqq \frac{1}{356}\qquad\textrm{and}\qquad \Delta \coloneqq \frac{355}{356}.$$ Root the tree $T$ in an arbitrarily chosen node; this naturally imposes an ancestor-descendant relation in $T$ (for convenience, every node is considered its own ancestor and descendant).

We start by partitioning $T$ into connected subtrees using the following procedure.
We proceed bottom-up over $T$, processing nodes in any order so that a node is processed after all its strict descendants have been processed. Along the way, we mark some nodes and split the edges of $T$ into heavy and light. Let $t \in V(T)$ be the currently processed non-root node of $T$ and let $e \in E(T)$ be the edge connecting $t$ with its parent. If the total weight of all the unmarked nodes that are descendants of $t$ is at least $n^\delta$ (recall that this includes $t$ itself as well), then we declare $e$ heavy and mark all the descendants of $t$ that were unmarked so far. Otherwise, the edge $e$ is declared light and the procedure proceeds to further nodes of $T$.

Observe that
removing all heavy edges splits $T$ into connected subtrees, say $T'_1, \cdots T'_m$. All of the subtrees, except for possibly the subtree containing the root node, are of weight at least $n^\delta$. In particular, the number of subtrees $m$, and therefore the number of heavy edges, is  bounded by $\Oh(n^{1 - \delta})$. Moreover, in every subtree $T'_i$, removing the node closest to the root splits $T'_i$ into smaller components, each of weight less than $n^\delta$.

Fix a heavy edge $e$ and let $T^e_1$ and $T^e_2$ be the two subtrees into which $T$ splits after removing~$e$. Let $X^e_i = \beta(T^e_i)$ for $i \in \{1, 2\}$. Put $A_e = X^e_1 \setminus X^e_2$, $C_e = X^e_2 \setminus X^e_1$, and $B_e = X^e_1 \cap X^e_2$. By the properties of tree decompositions, such choice of $A_e, B_e, C_e$ satisfies the conditions of \Cref{l:single_adhesion}, hence in time $\Oh(n \log^{k - 1} n)$ we can compute $\max_{v \in X^e_2} \dist_G(u,v)$ for every $u \in X^e_1$, and $\max_{u \in X^e_1} \dist_G(u,v)$ for every $v \in X^e_2$. Computing this for every heavy edge $e$ takes total time $\Oh(n^{2 - \delta} \log^{k - 1} n)$.

Fix any subtree $T'=T'_j$. Let $e_1 = t^{e_1}_1t^{e_1}_2, e_2 = t^{e_2}_1 t^{e_2}_2, \dots, e_\ell = t^{e_\ell}_1 t^{e_\ell}_2$ denote the heavy edges incident to $T'$, where $t^{e_i}_1 \in V(T')$ and $V(T') \subseteq V(T_1^{e_i})$ for every $i \in [\ell]$.
For a vertex $v \in \beta(T')$, let
$$d_0(v) = \max_{u \in \beta(T')} \dist_G(v, u)\qquad\textrm{and}\qquad d_i(v) = \max_{u \in X_2^{e_i}}\dist_G(v,u),\quad\textrm{for } i \in [\ell].$$ We have $\ecc(v) = \max \{ d_i(v)\colon i\in \{0,1,\ldots,\ell\}\}$.The values of $d_i(v)$ are already calculated for all $i\in [\ell]$, hence it remains to compute $d_0(v)$.

For every $i \in [\ell]$ and every pair of vertices $u, v \in \beta(t^{e_i}_1) \cap \beta(t^{e_i}_2)$ we find a shortest path between $u$ and $v$ with all internal vertices inside $X^{e_i}_2$ (or determine that it doesn't exist). For a fixed $u, v$ this can be done in time $\Oh(n)$. Since in total we perform this step at most $2k^2$ times per heavy edge, it takes $\Oh(n^{2 - \delta})$ time in total. Let $P_{i, u, v}$ denote such path, assuming it exists.

Let $G'$ denote the graph obtained from $G[\beta(T')]$ by taking every $i, u, v$ for which $P_{i, u, v}$ exists and adding an edge between $u$ and $v$ of weight equal to the total weight of $P_{i, u, v}$.
The weight of every edge inserted in $\beta(t^{e_i}_1) \cap \beta(t^{e_i}_2)$ is bounded by $|X^{e_i}_2|+1$. The total weight of all edges inserted is therefore at most
$$
\sum_{i \in [\ell]} |\beta(t^{e_i}_1) \cap \beta(t^{e_i}_2)|^2 \cdot (|X^{e_i}_2|+1) \leq
k^2 \sum_{i \in [\ell]} (|X^{e_i}_2|+1) = \Oh(n),
$$
where the last equality follows from the fact that all the trees $T^{e_i}_2$ are pairwise disjoint.
By \Cref{l:inserting_paths}, we have $\dist_{G'}(u, v) = \dist_G(u, v)$ for each $u, v \in \beta(T')$. Hence, computing $d_0(v)$ for every $v \in \beta(T')$ is equivalent to computing the eccentricity of every vertex in $G'$.

If the size of $\beta(T')$ is smaller than $n^\Delta$, we compute the eccentricities naively in time $\Oh(|\beta(T')|^2)$, 
noting that $G'$ has $\Oh(|\beta(T')|)$ edges (thanks to Claim~\ref{cl:weight-T} and bounded genus assumption 
of the last bullet of the theorem statement). Otherwise, we argue that we can use the algorithm in \Cref{l:star} as follows.

Let $t$ be the node of $T'$ closest to the root. Let $s_1, \dots, s_p$ be the children of $t$ in $T$ and let $T''_i$ denote the connected component of $T' - \{t\}$ containing $s_i$. Set $V_0 = \beta(t)$ and $V_i = \beta(T''_i)$ for $i \in [p]$.

It is now easy to verify that $G'$ and sets $A, \{V_i\colon 0\leq i\leq p\}$ selected this way satisfy the assumptions of \Cref{l:star2}. This allows us to use it to compute the eccentricities in $G'$ in time
$$
\Oh \left( n^{1 + \frac{150 + 54\delta}{151}} \log^{k + 5g} n \right) =
\Oh \left( n^{1 + \frac{354}{356}} \log^{k + 5g} n \right).
$$
As we argued, from these eccentricities, we may easily compute all the eccentricities in $G$.

Now, let us analyse the total running time of the whole algorithm. We invoke \Cref{l:star} $\Oh(n^{1 - \Delta})$ times, since we apply it only to subtrees $T'_i$ of size at least $n^\Delta$. The total running time of those applications is hence
$$
\Oh \left( n^{2 + \frac{354}{356} - \Delta} \log^{k + 5g} n \right) =
\Oh \left( n^{1 + \frac{355}{356}} \log^{k + 5g} n \right).
$$
We compute the eccentricities naively for subtrees smaller than $n^\Delta$, hence the total running time of this computation is
$$
\sum_{i \in [m] \colon |\beta(T'_i)| \leq n^\Delta} |\beta(T'_i)|^2 \leq
n^\Delta \cdot \sum_{i \in m} |\beta(T'_i)| = \Oh(n^{1 + \Delta})=\Oh\left(n^{1+\frac{355}{356}}\right).
$$
The rest of computation can be done in $\Oh(n^{2 - \delta} \log^k n)$. Therefore, the whole algorithm runs in time $\Oh \left( n^{1 + \frac{355}{356}} \log^{k + 5g} n \right)$.
\end{proof}


\subsubsection*{Complexity analysis}
Given an input tensor \( X \in \mathbb{R}^{1 \times C \times H \times W \times D} \) with batch size \( B=1 \) a 3D depthwise convolution \cite{chollet2017xception} with a kernel size ($k_h, k_w, k_d$) where $k_h = k_w = k_d = k$, the total parameter count and FLOPs can be calculated as $Ck^3$ and $2CHWDk^3$, respectively (bias is omitted for simplicity). Note that both parameters and FLOPS increase cubically as $k$ increases. This introduces extensive computational burden, especially for large kernel sizes and when these convolutions are used repeatedly in an end-to-end network, where the effect of the cubic scaling compounds over multiple layers. 

By decomposing the 3D depthwise convolution into three 1D depthwise convolutions, the parameter count and FLOPS are significantly reduced to $3Ck$ and $6CHWDk$, respectively, as each 1D convolution operates independently along a single spatial dimension rather than all three simultaneously. A comparison of FLOPs and parameter counts for 3D depthwise and decomposed convolutions is shown in Figure \ref{fig:compdecomp}. For the decomposed convolution, the computational cost scales linearly with respect to kernel size $k$, as opposed to the cubic scaling in the 3D depthwise convolution. 

When the kernel size is larger than 3, the complexity increase becomes much more significant. Large kernel sizes such as 7, 11 \cite{liu2022convnet, yu2024inceptionnext} and even up to 51 \cite{ding2022scaling, liu2022more} have been shown to improve model performance. In 3D imaging, our decomposition approach becomes crucial, particularly in deep networks where multiple convolution layers are applied. 

% [Old version] Given an input tensor \( X \in \mathbb{R}^{1 \times C \times H \times W \times D} \) with batch size \( N=1 \) and \( C \), \( H \), \( W \), \( D \) representing channel, height, width and depth and a 3D depthwise convolution \cite{chollet2017xception} with a kernel size ($k_h, k_w, k_d$) where $k_h = k_w = k_d = k$, the total parameter count and FLOPs can be calculated as $Ck^3$ and $2CHWDk^3$, respectively (bias is omitted for simplicity). Note that both parameters and FLOPS increase cubically as $k$ increases. This introduces extensive computational burden, especially for large kernel sizes and when these convolutions are used repeatedly in an end-to-end network, where the effect of the cubic scaling compounds over multiple layers. By decomposing the 3D depthwise convolution into three 1D depthwise convolutions, the parameter count and FLOPS become $3Ck$ and $6CHWDk$, respectively. Such a decomposition strategy significantly reduces both the parameter count and FLOPs, as each 1D convolution is applied in a parallel manner along one spatial dimension rather than across all three dimensions at once. Comparisons of both FLOPS and parameters for 3D depthwise and decomposed convolutions are shared in Figure \ref{fig:convflop_merge}b. For the decomposed convolution, the computational cost scales linearly with respect to $k$, as opposed to the cubic scaling in the 3D depthwise convolution. Especially when the kernel size is larger than 3, the complexity increase becomes much more significant. Given that, large kernel sizes such as 7, 11 \cite{liu2022convnet, yu2024inceptionnext} and up to 51 \cite{ding2022scaling, liu2022more} are shown to improve the performance, the decomposition approach becomes  much more critical, especially when used in deep networks where multiple layers of convolution are applied. 
%\begin{figure*}[t]
\centering
\includegraphics[width=10cm]{figures/flop.png}\\
\caption{FLOP and parameter comparison between depthwise convolution and decomposed depthwise convolution.}
\label{fig:flop} 
\end{figure*}

\begin{figure*}[t]
\begin{center}
\includegraphics[width=.85\linewidth]{fig_overview_v3.pdf}
\end{center}
\caption{
FastAtlas Overview: In each frame, we compute charts spanning fully or partially visible triangles (a), determine texture space bounding boxes for the visible portions of the view-space projections of each chart, and tightly pack these boxes into atlases (b, here $2K \times 2K$). We simultaneously bijectively parameterize and shade the charts into their atlas boxes, obtaining high quality texture space shading (c), and use this shading to render the shaded frames (d).}
\label{fig:overview}
\label{fig:alg_overview}
\end{figure*}

\section{Overview}
\label{sec:overview}
Our work has two core contributions: a real-time, GPU-based algorithm for tight packing of general parameterized charts into compact atlases; and a real-time TSS method that
utilizes this packing.  

\paragraph*{FastAtlas Packing.}
FastAtlas runs entirely on the GPU as a series of compute shaders. It takes the bounding boxes of parameterized charts as input, and packs them into an atlas (Fig~\ref{fig:overview}b, Sec.~\ref{sec:pack}). As such, the only input it requires are the dimensions of the bounding boxes.
Its outputs are deterministic; identical input charts are packed into identical atlases. This is critical for TSS and similar applications, as it ensures that consecutive frames taken from the same camera view have the same shading. Even minute shading differences across such frames can cause sampling jitter, leading to undesirable flicker \cite{baker2012rock}. 
While prior methods such as \cite{mueller2018shading,hladky2019tessellated,hladky2021snakebinning,Neff2022MSA} cap the dimensions of the charts that can be packed as-is for a given atlas size, and scale down all charts that exceed these dimensions, we scale all charts by the same factor, and do so only when strictly necessary to achieve packing success (Figs~\ref{fig:atlas},~\ref{fig:sas_issues}). 

\paragraph*{TSS using FastAtlas.}
Our end-to-end TSS atlas generation method combines the packing method above with a novel approach for computing seamless per-frame charts. 
We define our charts as the connected components of the visible surfaces in each frame (Fig.~\ref{fig:overview}a), and efficiently compute them using a parallel union-find algorithm (Sec.~\ref{sec:visible}). Since the boundaries of these charts coincide with the contours of the rendered surface, they are {\em invisible} to the viewer. This approach 
eliminates the artifacts caused by shading discontinuities along visible seams (Fig.~\ref{fig:seams}). 

\begin{parWithWrapFigure}
\begin{wrapfigure}{l}{.27\columnwidth}%
\includegraphics[width=\linewidth]{fig_inset_view_plane.pdf}%
\end{wrapfigure}
We bijectively parametrize the {\em visible portions} of our charts by projecting them to view space (inset). This maps a constant number of texels to each pixel in the final rendered output, evenly distributing residual undersampling error across all image pixels. While conceptually straightforward, efficiently parameterizing charts containing partially visible triangles using viewspace projection is non-trivial, as the visible portions may no longer be triangular (e.g. green triangle in the inset); applying naive projection to triangles with vertices behind the camera may produce ill-posed results. Clipping triangles before projection is both computationally expensive and significantly complicates downstream operations. We avoid explicit clipping by observing that all that is required for atlas packing is the dimensions of, potentially conservative, bounding boxes of these projected visible portions. We compute such bounding boxes without explicit chart clipping by adapting a conservative screen coverage estimator \shortcite{Blinn:CalculatingScreenCoverage} (Sec.~\ref{sec:box}). We then pack the computed boxes using FastAtlas. 
\end{parWithWrapFigure}

Finally, we shade the visible portion of each chart into its corresponding atlas bounding box (Fig~\ref{fig:overview}c). 
The resulting texture is then used during rasterization as a standard texture map (Fig. ~\ref{fig:overview}d). 
Our framework is compatible with all existing approaches for texture space shading, including forward shading, raytraced illumination, or deferred shading in texture space \cite{baker:2016}. In the examples shown, we use the standard forward shading based rendering pipeline included in the G3D Innovation Engine \cite{G3D17}, a commercial grade renderer.


\subsubsection*{Building a vision encoder with \dc}
Based on the above findings, we formulate DCFormer as the token mixer and develop a family of architectures of varying sizes, using DCFormer as the main building block. Figure \ref{fig:overview} illustrates the general architecture of our proposed DCFormer. Following previous work \cite{he2016deep, liu2021swin, liu2022convnet, yu2022metaformer, wang2023internimage, yu2024inceptionnext}, DCFormer adopts a hierarchical structure with four stages and an initial stem stage following \cite{wang2023internimage, dai2021coatnet}. 

In the stem stage, we use a decomposed convolution with a kernel size of 7 and a stride of 4. The downsampled images are then processed with three additional decomposed convolutions, each with a kernel size of 3 and a stride of 1. The outputs of the stem stage are passed into the hierarchical structure, where each stage produces tokens of sizes \(\frac{H}{8} \times \frac{W}{8} \times \frac{D}{8}\), \(\frac{H}{16} \times \frac{W}{16} \times \frac{D}{16}\), \(\frac{H}{32} \times \frac{W}{32} \times \frac{D}{32}\), and \(\frac{H}{64} \times \frac{W}{64} \times \frac{D}{64}\), respectively, where $H$, $W$, $D$ represent height, width and depth of the input image volume. 

We propose four model variants: nano, naive, tiny, small. The number of layers in each stage are [1, 1, 1, 1] for nano, [2, 2, 2, 2] for naive, [2, 3, 3, 2] for tiny and [2, 3, 6, 2] for small models. Additionally, each stage begins with patch embedding where a 3D max pooling operation with a kernel size of 3 and a stride of 2 is used to reduce the image dimensions \cite{dai2021coatnet}. Following \cite{dosovitskiy2020image, liu2022convnet, yu2024inceptionnext}, we set the MLP ratio to 4, while the kernel size for the decomposed convolution is set to 7, as in \cite{liu2021swin, liu2022convnet}. Further configuration details are provided in Table \ref{config}. 
\usepackage[utf8]{inputenc} % allow utf-8 input
\usepackage[T1]{fontenc}    % use 8-bit T1 fonts
\usepackage{microtype,inconsolata}
\usepackage{times,latexsym}
\usepackage{graphicx} \graphicspath{{figures/}}
\usepackage{amsmath,amssymb,mathabx,mathtools,amsthm,nicefrac}
\usepackage[linesnumbered,ruled,vlined]{algorithm2e}
\usepackage{acronym}
\usepackage{enumitem}
\usepackage[pagebackref,breaklinks,colorlinks]{hyperref}
\usepackage{balance}
\usepackage{xspace}
\usepackage{setspace}
\usepackage[skip=3pt,font=small]{subcaption}
\usepackage[skip=3pt,font=small]{caption}
\usepackage[capitalise,noabbrev,nameinlink]{cleveref}
\usepackage{booktabs,tabularx,colortbl,multirow,multicol,array,makecell,tabularray}
\usepackage{overpic,wrapfig}
\usepackage{dblfloatfix}
\usepackage[misc]{ifsym}
\usepackage{pifont}
\usepackage{fancyvrb}

% Add a period to the end of an abbreviation unless there's one
% already, then \xspace.
\makeatletter
\DeclareRobustCommand\onedot{\futurelet\@let@token\@onedot}
\def\@onedot{\ifx\@let@token.\else.\null\fi\xspace}

\def\eg{\emph{e.g}\onedot} \def\Eg{\emph{E.g}\onedot}
\def\ie{\emph{i.e}\onedot} \def\Ie{\emph{I.e}\onedot}
\def\cf{\emph{c.f}\onedot} \def\Cf{\emph{C.f}\onedot}
\def\etc{\emph{etc}\onedot} \def\vs{\emph{vs}\onedot}
\def\wrt{w.r.t\onedot} \def\dof{d.o.f\onedot}
\def\etal{\emph{et al}\onedot}

\makeatother

\acrodef{sota}[SOTA]{State-of-the-Art}
\acrodef{method}[\textsc{PRA}]{Preference-based Robot Assistant}
\acrodef{pbp}[\textsc{PbP}]{Preference-based Planning}
\acrodef{vln}[VLN]{Vision-and-Language Navigation}
\acrodef{llm}[LLM]{Large Language Model}
\acrodef{EILEV}[EILEV]{Efficient In-context Learning on Egocentric Videos}
\acrodef{vlm}[VLM]{Vision-Language Model}
\acrodef{vivit}[ViViT]{Video Vision Transformer}
\acrodef{llava}[LLaVA]{Large Language and Vision Assistant}
\acrodef{ai}[AI]{Artificial Intelligence}
\acrodef{ik}[IK]{Inverse Kinematics}
\acrodef{ompl}[OMPL]{Open Motion Planning Library}
\acrodef{sem}[SEM]{Structural Equation Model}

% Spacing
% \medmuskip=2mu   % reduce spacing around binary operators
% \thickmuskip=3mu % reduce spacing around relational operators
\setlength{\abovedisplayskip}{3pt}
\setlength{\belowdisplayskip}{3pt}
\setlength{\abovecaptionskip}{3pt}
\setlength{\belowcaptionskip}{3pt}
% \setlength\floatsep{1\baselineskip plus 3pt minus 2pt}
% \setlength\textfloatsep{1\baselineskip plus 3pt minus 2pt}
% \setlength\dbltextfloatsep{1\baselineskip plus 3pt minus 2pt}
% \setlength\intextsep{1\baselineskip plus 3pt minus 2pt}

\newcolumntype{x}{>{\columncolor{LightCyan1}}c}
\newcolumntype{y}{>{\columncolor{MistyRose}}c}

% [Old version] Based on the above findings, we formulate DCFormer as the token mixer and build a family of architectures for different sizes using DCFormer as the main building block. Figure. \ref{fig:overview} demonstrates the general architecture of our proposed DCFormer. Following previous work \cite{he2016deep, liu2021swin, liu2022convnet, yu2022metaformer, wang2023internimage, yu2024inceptionnext}, DCFormer adapts a hierarchical structure with 4 stages and a stem stage following \cite{wang2023internimage, dai2021coatnet}. As for the initial stem stage, we use a decomposed convolution with a kernel size of 7 and a stride of 4. Then the downsampled images are processed further with a sequence of another three decomposed convolutions with a kernel size of 3 and a stride of 1. Finally, the outputs of the stem stage are passed into the hierarchical structure, each stage having \(\frac{H}{8} \times \frac{W}{8} \times \frac{D}{8}\), \(\frac{H}{16} \times \frac{W}{16} \times \frac{D}{16}\), \(\frac{H}{32} \times \frac{W}{32} \times \frac{D}{32}\), and \(\frac{H}{64} \times \frac{W}{64} \times \frac{D}{64}\) tokens respectively, where $H$, $W$, $D$ represent height, width and depth of the input image. We propose 4 variants including naive, tiny, small and base models. The layer numbers for each stage are [1, 1, 1, 1] for nano, [2, 2, 2, 2] for naive, [2, 3, 3, 2] for tiny and [2, 3, 6, 2] for small models. Moreover, each stage begins with patch embedding where a 3D max-pooling with a kernel size of 3 and a stride of 2 is utilized to reduce the image dimension \cite{dai2021coatnet}. Following \cite{dosovitskiy2020image, liu2022convnet, yu2024inceptionnext}, the MLP ratio is set to 4. Following \cite{liu2021swin, liu2022convnet}, the kernel size for the decomposed convolution is set to 7. More details for the configuration details are shared in Table. \ref{config}. 


\subsubsection*{Developing a CLIP framework using \dc}
The CLIP framework aligns visual and contextual embeddings in a shared latent space through contrastive learning (Figure \ref{fig:clip}(a)). In this paper, we integrate our proposed DCFormer as the vision encoder in CLIP, serving as the primary module for efficient processing of 3D medical images. For the text encoder, we incorporate CXR-BERT \cite{boecking2022making}, following the implementation in Hamamci et al.\cite{hamamci2024foundation}. To compute similarities between images and radiology reports, we project visual and textual features into a shared 512-dimensional embedding space. Specifically, we apply global average pooling, followed by a linear projection to the output of the image encoder, while the text encoder output is transformed using a separate linear layer. Finally, visual and textual embeddings are aligned using a contrastive loss function \cite{radford2021learning}:
\begin{align}
    \mathcal{L}_{\text{CLIP}} = - \frac{1}{B} \sum_{i=1}^B \left[ \log \frac{\exp\left(\text{sim}(z_i^t, z_i^v)/\tau\right)}{\sum_{j=1}^B \exp\left(\text{sim}(z_i^t, z_j^v)/\tau\right)} + \log \frac{\exp\left(\text{sim}(z_i^v, z_i^t)/\tau\right)}{\sum_{j=1}^B \exp\left(\text{sim}(z_i^v, z_j^t)/\tau\right)} \right]
\end{align}
where $\text{sim}(z_i^t, z_j^v)$ represents the cosine similarity between the text embedding $z_i^t$ for the $i-$th radiology report and the visual embedding $z_j^v$ for the $j-th$ 3D image, $\tau$ is the temperature parameter (set to 1), and $B$ is the number of text-image pairs in a training batch. 

Unlike CT-CLIP \cite{hamamci2024foundation}, which lacks explicit multi-scale feature extraction, our DCFormer-based CLIP framework introduces a hierarchical structure that captures multi-scale features across different spatial resolutions. This design enhances representation learning and improves the alignment between image and text embeddings by preserving both global context and fine-grained details—crucial for accurate 3D CT image interpretation. Additionally, integrating DCFormer helps address the computational challenges associated with 3D CT imaging. By utilizing decomposed convolutions, our framework effectively reduces both the parameter count and computational overhead while maintaining robust feature extraction capabilities. As a result, the DCFormer-based CLIP framework is well-suited for large-scale 3D medical imaging applications.

\begin{figure*}[t]
\centering
\includegraphics[width=15.5cm]{figures/clip_zeroshot.png}\\
\caption{CLIP a) training and b) zero-shot inference framework}
\label{fig:clip} 
\end{figure*}


% [old version] The CLIP framework operates by aligning the embeddings of text and image modalities in a shared latent space, enabling cross-modal tasks such as zero-shot classification and retrieval (Figure. \ref{fig:clip}). In this article, we integrate our proposed DCFormer as the vision encoder into the CLIP framework which acts as the primary module for processing 3D medical images efficiently. As for the text encoder, we integrate CXR-Bert\cite{boecking2022making} by following the implementation of Hamamci et al. \cite{hamamci2024foundation}. To compute the similarities between embedded images \( E_{\text{img}} \in \mathbb{R}^{C_{\text{embed}}} \) and radiology reports \( E_{\text{txt}} \in \mathbb{R}^{C_{\text{embed}}} \), we apply global average pooling and a linear layer with 512 dimensions. The CXR-Bert text encoder tokens which are in 768 dimensions, are linearly transformed into a 512 layer. Finally these two embedding spaces are aligned using a contrastive loss function \cite{radford2021learning}:
% \begin{align}
%     \mathcal{L}_{\text{CLIP}} = - \frac{1}{N} \sum_{i=1}^N \left[ \log \frac{\exp\left(\text{sim}(z_i^t, z_i^v)/\tau\right)}{\sum_{j=1}^N \exp\left(\text{sim}(z_i^t, z_j^v)/\tau\right)} + \log \frac{\exp\left(\text{sim}(z_i^v, z_i^t)/\tau\right)}{\sum_{j=1}^N \exp\left(\text{sim}(z_i^v, z_j^t)/\tau\right)} \right]
% \end{align}
% where $\text{sim}(z_i^t, z_j^v)$ represents the cosine similarity between the text embedding $z_i^t$ and the visual embedding $z_j^v$, $\tau$ is the temperature parameter which is set to 1, and $N$ is the number of text-image pairs. It should be noted that to further enhance the effectiveness of the CLIP framework for 3D CT images, we leverage DCFormer’s decomposed convolutions to handle the large input resolution and volumetric nature of CT images while maintaining computational efficiency. Different from CT-CLIP \cite{hamamci2024foundation}, DCFormer-based CLIP framework is composed of a hierarchical structure to effectively capture multi-scale features across different spatial resolutions, enabling improved representation learning and better alignment between image and text embeddings. Such a structure enables the model to focus on both global context and fine-grained details, which is critical for accurately interpreting 3D CT images. Unlike CT-CLIP \cite{hamamci2024foundation}, which does not explicitly incorporate multi-scale feature extraction, our proposed DCFormer-based CLIP framework processes input images through a hierarchical structure which ensures that features from different scales are represented in the learned embedding space. Furthermore, integrating the DCFormer architecture helps tackle the significant computational challenges associated with 3D CT images as it utilizes decomposed convolutions to extract features which effectively reduces the number of parameters and computational overhead.

\subsubsection*{Zero-shot multi-abnormality detection with DCFormer}
For zero-shot multi-abnormality detection, we follow the same approach as CheXzero\cite{tiu2022expert} and CT-CLIP\cite{hamamci2024foundation}. Specifically, two similarity scores are computed using the cosine similarity between the embedding of the CT image of interest and the embeddings of a positive ('\{\textit{Pathology}\} is present.') prompt and a negative ('\{\textit{Pathology}\} is not present.') prompt. 
A softmax function is then applied to these similarity scores to estimate the likelihood of each abnormality being present in the given CT image (see Figure. \ref{fig:clip} (b) for illustration). It should be noted that the zero-shot performance can potentially be improved by experimenting with different prompt variations. For instance, possible prompt templates including: 'Signs of \{\textit{Pathology}\} are/are not detected in the image.', 'There is/is not \{\textit{Pathology}\}.', '\{\textit{Pathology}\} is/is not observed.' 'Indication/No indication of \{\textit{Pathology}\}.'.  Although prompt engineering plays a critical role in improving image-text alignment, we adopt the prompt structure proposed by Hamamci et al. \cite{hamamci2024foundation} ('\{ \textit{Pathology}\} is/is not present.') to maintain consistency with prior work and ensure a fair comparison.

% [old version] For zero-shot multi-abnormality detection, we follow the same practice as CheXzero\cite{tiu2022expert} and CT-CLIP\cite{hamamci2024foundation}. That is, the output scores are extracted using the CT image of interest as well as both positive: '\{\textit{Pathology}\} is present.' and negative: '\{\textit{Pathology}\} is not present.' prompts. Then, a softmax function is applied to these scores to achieve the likelihood of each abnormality being present in the given CT image (see Figure. \ref{fig:clip}b for the illustration). It should be noted that it is possible to enhance the zero-shot performance by experimenting with different prompts associated with medical imaging domain. For instance, 'Signs of \{\textit{Pathology}\} are/are not detected in the image.', 'There is/is not \{\textit{Pathology}\}.', '\{\textit{Pathology}\} is/is not observed.' 'Indication/No indication of \{\textit{Pathology}\}.' etc., can be some prompt templates. Although such prompt engineering plays a critical role in determining how effectively the model aligns text and image embeddings,  we strictly utilize the prompt structure proposed by Hamamci et al. \cite{hamamci2024foundation} ('\{ \textit{Pathology}\} is/is not present.') to maintain consistency with prior work and enable a fair comparison.


%\subsubsection*{Fine-tuning DCFormer for multi-abnormality detection}
%To further enhance the multi-abnormality detection performance of DCFormer, we fine-tune the DCFormer-based image encoder using a supervised learning approach (see Figure. \ref{fig:clip}(c)). Specifically, fine-tuning is performed in two  modes: (1) Frozen mode, where the pre-trained encoder weights remain unchanged, and only a task-specific classification layer is trained; and (2) Trainable mode, where the image encoder weights are updated alongside the task-specific classification layer, allowing the image encoder to adapt to the multi-abnormality detection task. The supervised model is trained to minimize the Binary Cross-Entropy (BCE) loss for each abnormality.

% [Old Version] To further improve the multi-abnormality detection performance of DCFormer, we fine-tune the DCFormer-based image encoder using a supervised learning approach (see Figure. \ref{fig:clip}c). Specifically, we perform the fine-tuning process in two different modes: (1) with the image encoder weights frozen and (2) with the image encoder weights trainable. In the frozen mode, the pre-trained weights of the encoder remain unchanged and only a task-specific additional classification layer is trained to perform the classification task. In the trainable mode, the image encoder weights are updated alongside the task-specific classification layer, enabling the model to adapt more closely to the multi-abnormality detection task. Then, the supervised model is trained to minimize the Binary Cross-Entropy (BCE) loss for each abnormality.

\subsubsection*{State-of-the-art image encoders}
The pioneering work ViT \cite{dosovitskiy2020image} introduced the concept of processing images as sequences of patches, leveraging the self-attention mechanism for global feature extraction. However, its quadratic complexity imposes a significant computational burden, particularly for 3D medical images. TransUNet \cite{chen2021transunet} was the first model to incorporate ViTs into medical image segmentation, employing a hybrid architecture that combines CNNs with transformers. ConvNeXt \cite{liu2022convnet} demonstrated that 7×7 depthwise convolutions can serve as an effective token mixer, improving performance over self-attention while maintaining computational efficiency. PoolFormer \cite{yu2022metaformer} introduced a more efficient feature extraction mechanism than ConvNeXt, replacing self-attention with a simple average pooling operation to minimize computational overhead while preserving strong performance.

% [old version] The pioneering work ViT introduced the concept of processing images as sequences of patches which then leverages the self-attention mechanism for global feature extraction. However, especially for 3D medical images, its quadratic complexity brings extensive computational burden. Similarly, TransUNet is the first model proposed to use ViTs in medical imaging tasks. It enjoys a hybrid model that combines CNNs with transformers. ConvNeXt shows that 7x7 depthwise convolutions as the token mixer can improve the performance over self-attention while retaining the computational efficiency. PoolFormer introduces efficient implementation for feature extraction compared to ConvNeXt, where it employs simple average pooling operation to minimize computational overhead while aiming to maintain good performance. 


\subsubsection*{Datasets}

For model training and evaluation, we used the open-source CT-RATE dataset \cite{hamamci2024foundation}, which consists of 50,188 reconstructed CT volumes from 25,692 distinct CT experiments involving 21,304 patients. Each CT volume is paired with a radiology report. The dataset also includes 18 distinct abnormalities extracted from medical reports of each CT scan (findings and impressions). Table \ref{ctrate} provides an in-depth overview of CT-RATE, detailing the distribution of abnormalities across training and validation subsets. Each abnormality is associated with the number of samples in both sets, along with their respective ratios within the dataset. These ratios represent the proportion of samples for each abnormality relative to the total dataset. Notably, the ratios for the training and validation sets remain nearly identical across all abnormalities, ensuring that the validation set accurately reflects the training distribution, making it reliable for evaluating model performance.

% [old version] As for our first experimental dataset, we use CT-RATE \cite{hamamci2024foundation} which consists of 50,188 reconstructed CT volumes from 25,692 distinct CT experiments from 21,304 different patients. For zero-shot and fine-tuning purposes, the dataset consists of 18 distinct abnormalities extracted from medical reports of each CT scan (i.e, findings and impressions) Table \ref{ctrate} provides an in-depth view of the CT-RATE dataset, focusing on the distribution of various abnormalities across training and validation datasets. Each abnormality is associated with the number of samples in both sets, as well as their respective ratios within the dataset. These ratios show the proportion of samples for a given abnormality relative to the total samples in each subset. Note that the ratios for the training and validation sets are nearly identical across all abnormalities, which ensures that the validation set adequately reflects the distribution of the training set, making it reliable for evaluating model performance.


%\subsubsection*{RAD-ChestCT}

%\subsubsection*{One more dataset}

\subsubsection*{Implementation details}

For model training, we follow the same data splitting and pre-processing strategy as in \cite{hamamci2024foundation}. Specifically, we use 20,000 patients for training and 1,304 for validation. For pre-processing, we first resize each CT volume to a spacing of 0.75 mm on the x-axis and y-axis and 1.5 mm on the z-axis. Then, the volumes are center-cropped or padded to a fixed size of 512 x 512 x 256. Finally, we clip the Hounsfield Unit (HU) values of each CT image volume to [-1000,1000] and normalize them to [-1,1].  We train models using the AdamW optimizer \cite{kingma2014adam} with an initial learning rate $1^{-5}$. We dot not apply learning rate scheduling or warmup, as we have not observed any significant improvements. All models are trained for 15 epochs.

% [old version] As for the training procedure, we follow the same splitting and pre-processing strategy as Hamamci et al.\cite{hamamci2024foundation}. More specifically, we use 20,000 patients for training and 1304 for validation. For pre-processing operations, we first resize each volume to achieve a uniform spacing of 0.75 mm on the x-axis and y-axis and 1.5 mm on the z-axis. Then, the volumes are center-cropped or padded to achieve a constant resolution of 512 x 512 x 256. Finally, we convert each CT volume to Hounsfield Unit (HU) values using intercept and slope. HU values are clipped to a range of -1000 to 1000 and normalized to have values between -1 and 1 during training. For all experiments, we use the input size 512 x 512 x 256. We use AdamW \cite{kingma2014adam} with an initial learning rate $1^{-5}$ as the optimizer. We did not apply any learning rate scheduling or warmup as we have not seen any significant improvements. All models are trained for 15 epochs.
\begin{table*}[!th]
\scriptsize
\renewcommand\arraystretch{1.4}
\begin{center}
    
\begin{tabular}{|l|c|c|c|c|}
\hline
Abnormality                        & \multicolumn{1}{l|}{Number of samples (train)} & \multicolumn{1}{l|}{Number of samples (validation)} & \multicolumn{1}{l|}{Ratio in the dataset (train)} & \multicolumn{1}{l|}{Ratio in the dataset (validation)} \\ \hline
Medical Material                   & 5818                                           & 313                                                 & 0.123                              & 0.103                                   \\ \hline
Arterial wall calcification        & 13377                                          & 867                                                 & 0.284                              & 0.285                                   \\ \hline
Cardiomegaly                       & 5308                                           & 325                                                 & 0.113                              & 0.107                                   \\ \hline
Pericardial effusion               & 3412                                           & 226                                                 & 0.072                              & 0.074                                   \\ \hline
Coronary artery wall calcification & 12025                                          & 765                                                 & 0.255                              & 0.252                                   \\ \hline
Hiatal hernia                      & 6751                                           & 417                                                 & 0.143                              & 0.137                                   \\ \hline
Lymphadenopathy                    & 12221                                          & 789                                                 & 0.259                              & 0.260                                   \\ \hline
Emphysema                          & 9122                                           & 600                                                 & 0.193                              & 0.197                                   \\ \hline
Atelectasis                        & 12263                                          & 713                                                 & 0.260                              & 0.235                                   \\ \hline
Lung nodule                        & 21382                                          & 1361                                                & 0.453                              & 0.448                                   \\ \hline
Lung opacity                       & 17420                                          & 1184                                                & 0.369                              & 0.390                                   \\ \hline
Pulmonary fibrotic sequela         & 12589                                          & 831                                                 & 0.267                              & 0.273                                   \\ \hline
Pleural effusion                   & 5705                                           & 376                                                 & 0.121                              & 0.124                                   \\ \hline
Mosaic attenuation pattern         & 3638                                           & 253                                                 & 0.077                              & 0.083                                   \\ \hline
Peribronchial thickening           & 4973                                           & 355                                                 & 0.105                              & 0.117                                   \\ \hline
Consolidation                      & 8319                                           & 581                                                 & 0.176                              & 0.191                                   \\ \hline
Bronchiectasis                     & 4732                                           & 330                                                 & 0.100                              & 0.109                                   \\ \hline
Interlobular septal thickening     & 3745                                           & 249                                                 & 0.079                              & 0.082                                   \\ \hline
\end{tabular}
\label{table1}
\end{center}
\caption{\label{ctrate} Detailed overview of the CT-RATE dataset for each abnormality and their distributions in the training and validation sets. \cite{hamamci2024foundation}}
\end{table*}
