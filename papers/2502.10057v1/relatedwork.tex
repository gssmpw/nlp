\section{Related Work}
Recent research in membrane-based actuation reveals similar approaches for estimating membrane stretch across studies. Most commonly, this involves calculating principal stretches by integrating over infinitesimally small membrane segments, offering a detailed yet computationally intensive method~\cite{Zhou2018AnMembrane, Wang2017AnomalousActuation, Shi2022ModellingFingertip, Shi2023ModellingLoad}. However, this technique is primarily applicable to small expansion profiles, as it tends to produce errors in scenarios involving high strain. On the other hand, Herzig et al. \cite{Herzig2021ModelMembrane} proposed a geometric model for highly extensible membranes where the deformed membrane is approximated as a spherical segment under specific conditions.  The approach takes many simplifications describing the membrane's deformation configurations regardless of whether the external force is applied or not. While various methods are employed to estimate stretch and strain energy, the core concept involves balancing the total potential energy of the inflated membrane based on the principle of minimum potential energy. When a compressible fluid is used as the driving medium, additional formulations are required to account for compressibility, making it more complex to relate input volume with actuator shape, thus complicating state estimation~\cite{Herzig2021ModelMembrane,Shi2023ModellingLoad}. In contrast, using liquid as a driving medium eliminates the need for these equations, simplifying the modeling process.

As shown in Fig.~\ref{fig:four_images}, the state variables are selected to be $(h_1,\;h_2,\; F)$, where $h_1$ represents the virtual unindented height of the membrane from the baseline assuming that the membrane is in the non-contact phase and $h_2$ is the actual depth to which the membrane is pushed down from the unindented height $h_1$ due to the applied external force $F$ when the membrane is in the contact phase.

As illustrated in Fig.~\ref{fig: flow_chart}, the generic way to compute state variables of liquid-driven ballooning membrane actuators (BMAs) is as follows:
The model starts with a volume input representing the fluid inside the membrane. From the volume, the unindented height $h_1$ is calculated. At this stage, the membrane takes on a specific shape, which needs to be calculated based on geometric assumptions. With the geometry determined, the membrane’s deformation can be calculated. The principal stretches $\lambda$ refers to the change in lengths along the membrane's principal directions, which are required for further energy calculations. The strain energy function $W$ is computed using the material's properties based on the principal stretches. The total potential energy 
$E_p$ of the BMA represents the work done to deform the membrane elastically and combines three components: the strain energy stored in the membrane due to deformation, the work done by the relative pressure $p$ to the environment over the volume of fluid, and the work from an external force 
$F$ acting over the indentation height difference $h_1-h_2$.
\begin{equation}
    E_p=\int_{V_m} W \,dV-\int_{V_f} p \,dV + F(h_1-h_2)
    \label{Eq:E_p_prior_art}
\end{equation}
where $V_m$ is the membrane's volume and $V_f$ is the fluid volume within the membrane.
\begin{figure}[thpb]
  \centering
  \includegraphics[width=1\linewidth]{Figures/flow_chart.PDF}
  \caption{Flow diagram: intrinsic quasi-static modeling, the state variables are indicated in red color.}
  \label{fig: flow_chart}
\end{figure}