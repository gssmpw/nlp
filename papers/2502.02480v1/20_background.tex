\section{Background}\label{sec:background}

\subsection{Stability of dynamical systems}\label{sec:stability}

We consider autonomous dynamical systems as first-order ODEs
\begin{equation}\label{eq:autonomous_system}
    \dot\bsx=\bsf(\bsx)
\end{equation}
with $\bsx\in\bbR^n$ denoting the systems state and function $\bsf:\bbR^n\mapsto\bbR^n$. Suppose $\bsf$ is sufficiently smooth to ensure a unique solution of \cref{eq:autonomous_system} for any initial condition $\bsx(t_0)$. A solution $\hat\bsx(t)$ is labeled \emph{stable} in the sense of Lyapunov if, for any small perturbation in the initial condition, the perturbed solution $\Bar\bsx(t)$ remains close for all times, that is, if for any $\epsilon>0$, there exists a $\delta>0$ such that $\norm{\hat\bsx(t_0) - \Bar{\bsx}(t_0)} < \delta$ implies  $\norm{\hat\bsx(t) - \Bar\bsx(t)} < \epsilon$ for all $t>t_0$ \cite{malkin1959, verhulst1990}.
A stronger version of stability, namely \emph{asymptotic stability}, is obtained if additionally the perturbed solution $\bar\bsx(t)$ converges to $\hat\bsx(t)$ for $t\rightarrow\infty$ for initial conditions $\norm{\hat\bsx(t_0) - \Bar{\bsx}(t_0)} < \Delta$ with some\, ${\Delta>0}$~\cite{verhulst1990}.

Lyapunov stability is a \emph{local} property, in the sense that for sufficiently large perturbations, the perturbed solution may stray arbitrarily far even from asymptotically stable solutions \cite{seydel2010}. 
Quantifying the permissible extent of perturbations on the initial condition such that the perturbed solution still converges to the unperturbed solution leads to the concept of a \emph{basin of attraction} \cite{layek2015}.
Asymptotically stable equilibria, for which the basin of attraction extends to the entire phase space, are called \emph{globally asymptotically stable} \cite{layek2015}. A system with a globally stable equilibrium cannot have any other equilibria, as their existence would imply that there are solutions (namely the other equilibria) that do not converge to the globally stable equilibrium.
For linear \gls{ODE} systems, stability is always global, i.e., all solutions share the same stability property. In particular, an asymptotically stable equilibrium of a linear system is always globally asymptotically stable.

In the following, we consider the case of an equilibrium solution $\bsx(t)=\bszero$, i.e., $\bsf(\bszero)=\bszero$. Placing the equilibrium at the origin is done for ease of notation and is without loss of generality, as arbitrary equilibria can be translated to the origin with a time-independent transformation. 
The stability of the equilibrium can be determined through the use of Lyapunov's stability theory by finding a suitable scalar function $V$ called a Lyapunov function \cite{verhulst1990, khalil2002}. Let $V: D\mapsto\bbR$ be continuously differentiable and positive definite, that is, $V(\bszero)=0$ and $ V(\bsx) > 0$ in $D\setminus\{\bszero\}$, where $D\subseteq\bbR^n$ is a neighborhood of the origin. If the value of $V$ is nonincreasing along trajectories of the system in $D$, then the equilibrium is stable in the sense of Lyapunov. Mathematically, this is the case if $\dot V(\bsx)$ is negative semi-definite:
\begin{equation}\label{eq:decrease_condition}
    \dot V(\bsx)=\partdiff{V}{\bsx}^{\intercal}\dot\bsx = \partdiff{V}{\bsx}^{\intercal}\bsf(\bsx) \leq0\quad\forall\bsx\in D\setminus\{\bszero\}.
\end{equation}
In the case of a negative definiteness ($\dot V(\bsx)<0$ in $D\setminus\{\bszero\}$), \emph{asymptotic} stability can be concluded.
To obtain \emph{global} asymptotic stability, these conditions must be satisfied \emph{globally}, that is, for all ${\bsx\in D=\bbR^n}$, \emph{and} the Lyapunov function must be radially unbounded, i.e., $V(\bsx)\rightarrow\infty$ for $\norm{\bsx}\rightarrow\infty$ \cite{khalil2002}.
The latter condition avoids the existence of solutions that do not converge to the equilibrium by diverging to infinity without violating the decrease condition in \cref{eq:decrease_condition}.

\subsection{Port-Hamiltonian systems}\label{sec:phs}

\Glspl{PHS} theory provides a framework for modeling physical systems that generalizes the underlying mathematical structure of Hamiltonian dynamics. It incorporates the modeling of energy-dissipating elements, which are typically absent in classical Hamiltonian systems \cite{vanderschaft2014}. Additionally, \glspl{PHS} theory has roots in control theory, emphasizing the interactions of dynamic systems with their environment.
In this work, we consider \glspl{PHS} of the form
\begin{equation}\label{eq:isphs_evolution}
    \dot{\bsx} = \left[\bsJ(\bsx) - \bsR(\bsx)\right]\partdiff{\hamiltonian}{\bsx}(\bsx) + \bsG(\bsx)\bsu(t).
\end{equation}
Here, $\bsx(t)\in\bbR^n$ describes the system's state, ${\bsu(t)\in\bbR^m}$ its input, and $\hamiltonian:\bbR^n\rightarrow\bbR$ denotes the Hamiltonian, which typically represents the total energy of the system \cite{vanderschaft2014, beattie2018}.
The skew-symmetric \emph{structure matrix} ${\bsJ:\bbR^n\mapsto\bbR^{n\times n}}$ describes the conservative energy flux within the system. 
The symmetric positive semi-definite matrix ${\bsR:\bbR^n\mapsto\bbR^{n\times n}}$ is called \emph{dissipation matrix} and describes the energy losses. Finally, the \emph{input matrix} ${\bsG:\bbR^n\mapsto\bbR^{n\times m}}$ describes how energy enters and exits the system via the inputs.
The matrices are required to satisfy
\begin{align}\label{eq:isophs_psd_condition}
    \bsJ(\bsx)=-\bsJ^{\intercal}(\bsx), && \bsR(\bsx)=\bsR^{\intercal}(\bsx)\succeq0.
\end{align}
As direct consequence, \glspl{PHS} extend the energy conservation of Hamiltonian systems to the energy dissipation inequality:
\begin{equation}\label{eq:isphs_energy_balance}\begin{split}
    \dot\hamiltonian &=     \underbrace{\partdiff{\hamiltonian}{\bsx}^{\intercal}\mkern-8mu\bsJ\partdiff{\hamiltonian}{\bsx}}_{\substack{\vphantom{\geq}=0}} - \underbrace{\partdiff{\hamiltonian}{\bsx}^{\intercal}\mkern-8mu\bsR\partdiff{\hamiltonian}{\bsx}}_{\substack{\geq0}} + \underbrace{\partdiff{\hamiltonian}{\bsx}^{\intercal}\mkern-8mu\bsG\bsu(t)}_{\substack{s(\bsx, \bsu)}}
        \leq s(\bsx,\bsu).
\end{split}
\end{equation}
In the language of system theory, the system is dissipative\footnote{The technical definition of dissipative systems requires $\hamiltonian\geq0$, which will be ensured later. If $\hamiltonian\in\bbR$, then the system \cref{eq:isphs_evolution} is called \emph{cyclo-dissipative}.} with respect to the supply rate $s(\bsx,\bsu)$, which is often stated in the integral form
\begin{equation}
    \hamiltonian(\bsx(t_1)) - \hamiltonian(\bsx(t_0)) \leq \int_{t_0}^{t_1} s(\bsx,\bsu) \dif t.
\end{equation}
Since the supply rate $s(\bsx,\bsu)$ is linear in the input $\bsu$, it vanishes for zero-input $s(\bsx,\bszero)=0$. The inequality thus ensures that the system cannot gain energy in the absence of an external excitation ($\bsu=\bszero$). Besides this physical motivation, the dissipativity property of \glspl{PHS} plays a crucial role in stability analysis. 
For the unforced system with ${s=0}$, the Hamiltonian inherently fulfills the decrease condition $\dot\hamiltonian\leq0$.
Thus, by using $\hamiltonian$ as a Lyapunov function, it can be shown that any strict minimum in the Hamiltonian implies a stable equilibrium of the unforced dynamics \cite{vanderschaft2014}.  
Furthermore, the equilibrium is asymptotically stable if the system continuously dissipates energy ($\bsR\succ0$) for states near the minimum.

