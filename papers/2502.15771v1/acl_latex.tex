% This must be in the first 5 lines to tell arXiv to use pdfLaTeX, which is strongly recommended.
\pdfoutput=1
% In particular, the hyperref package requires pdfLaTeX in order to break URLs across lines.

\documentclass[11pt]{article}

% Change "review" to "final" to generate the final (sometimes called camera-ready) version.
% Change to "preprint" to generate a non-anonymous version with page numbers.
\usepackage[preprint]{acl}

% Standard package includes
\usepackage{times}
\usepackage{latexsym}

% For proper rendering and hyphenation of words containing Latin characters (including in bib files)
\usepackage[T1]{fontenc}
% For Vietnamese characters
% \usepackage[T5]{fontenc}
% See https://www.latex-project.org/help/documentation/encguide.pdf for other character sets

% This assumes your files are encoded as UTF8
\usepackage[utf8]{inputenc}

% This is not strictly necessary, and may be commented out,
% but it will improve the layout of the manuscript,
% and will typically save some space.
\usepackage{microtype}

% This is also not strictly necessary, and may be commented out.
% However, it will improve the aesthetics of text in
% the typewriter font.
\usepackage{inconsolata}

%Including images in your LaTeX document requires adding
%additional package(s)
\usepackage{graphicx}
\usepackage{url}
\usepackage{enumitem}
\usepackage{booktabs}
\usepackage{makecell}
\usepackage{multirow}
\usepackage{algorithm}
\usepackage{algpseudocode}
\usepackage{amssymb}
\usepackage{amsmath}
\usepackage{pifont}
\usepackage[most]{tcolorbox}
\usepackage{soul}
\usepackage{xcolor}
\usepackage{ulem}
% \usepackage{minted}

\usepackage{pgfplotstable}
\usepackage{subfigure}
\usepackage{pgfplots}
\usepackage{tikz}
\usetikzlibrary{arrows.meta}
\usetikzlibrary{decorations}
\usetikzlibrary{pgfplots.groupplots}
\usepgfplotslibrary{fillbetween}
\usetikzlibrary{shapes.multipart}
\usetikzlibrary{shapes.geometric}

\definecolor{ugreen}{cmyk}{1,0,1,0.498}
\definecolor{lyyblue}{cmyk}{0.8278,0.3333,0,0.2941}
\definecolor{lyygreen}{cmyk}{0.6813,0,0.725,0.3725}
\definecolor{lyyred}{cmyk}{0,0.8855,0.8767,0.1098}
\definecolor{dblue}{cmyk}{1,0.5487,0,0.5569}
\definecolor{royalblue}{HTML}{4169e1}

\definecolor{myred}{HTML}{E33222}

\newcommand{\cmark}{\ding{52}}%
\newcommand{\xmark}{\ding{56}}%

\definecolor{gr}{RGB}{0, 146, 0}
\newcommand{\gr}[1]{\textcolor{gr}{#1}}
\newcommand{\red}[1]{\textcolor{myred}{#1}}
\newcommand{\purp}[1]{\textcolor{purple}{#1}}
\newcommand{\org}[1]{\textcolor{orange}{#1}}
\newcommand{\blue}[1]{\textcolor{blue}{#1}}
\newcommand{\teal}[1]{\textcolor{teal}{#1}}
\newcommand{\gray}[1]{\textcolor{lightgray}{#1}}

\algnewcommand{\LineComment}[1]{\State \(\triangleright\) #1}

\newcommand{\greencheck}{\textcolor{gr}{\ding{51}}}
\newcommand{\redcross}{\textcolor{myred}{\ding{55}}}

\newcommand{\hlc}[2][yellow]{{%
    \colorlet{foo}{#1}%
    \sethlcolor{foo}\hl{#2}}%
}

\newcommand{\method}{\textsc{OpTune}}

% If the title and author information does not fit in the area allocated, uncomment the following
%
%\setlength\titlebox{<dim>}
%
% and set <dim> to something 5cm or larger.

\title{Learning to Reason from Feedback at Test-Time}

% Author information can be set in various styles:
% For several authors from the same institution:
% \author{Author 1 \and ... \and Author n \\
%         Address line \\ ... \\ Address line}
% if the names do not fit well on one line use
%         Author 1 \\ {\bf Author 2} \\ ... \\ {\bf Author n} \\
% For authors from different institutions:
% \author{Author 1 \\ Address line \\  ... \\ Address line
%         \And  ... \And
%         Author n \\ Address line \\ ... \\ Address line}
% To start a separate ``row'' of authors use \AND, as in
% \author{Author 1 \\ Address line \\  ... \\ Address line
%         \AND
%         Author 2 \\ Address line \\ ... \\ Address line \And
%         Author 3 \\ Address line \\ ... \\ Address line}

% \author{First Author \\
%   Affiliation / Address line 1 \\
%   Affiliation / Address line 2 \\
%   Affiliation / Address line 3 \\
%   \texttt{email@domain} \\\And
%   Second Author \\
%   Affiliation / Address line 1 \\
%   Affiliation / Address line 2 \\
%   Affiliation / Address line 3 \\
%   \texttt{email@domain} \\}

%\author{
%  \textbf{First Author\textsuperscript{1}},
%  \textbf{Second Author\textsuperscript{1,2}},
%  \textbf{Third T. Author\textsuperscript{1}},
%  \textbf{Fourth Author\textsuperscript{1}},
%\\
%  \textbf{Fifth Author\textsuperscript{1,2}},
%  \textbf{Sixth Author\textsuperscript{1}},
%  \textbf{Seventh Author\textsuperscript{1}},
%  \textbf{Eighth Author \textsuperscript{1,2,3,4}},
%\\
%  \textbf{Ninth Author\textsuperscript{1}},
%  \textbf{Tenth Author\textsuperscript{1}},
%  \textbf{Eleventh E. Author\textsuperscript{1,2,3,4,5}},
%  \textbf{Twelfth Author\textsuperscript{1}},
%\\
%  \textbf{Thirteenth Author\textsuperscript{3}},
%  \textbf{Fourteenth F. Author\textsuperscript{2,4}},
%  \textbf{Fifteenth Author\textsuperscript{1}},
%  \textbf{Sixteenth Author\textsuperscript{1}},
%\\
%  \textbf{Seventeenth S. Author\textsuperscript{4,5}},
%  \textbf{Eighteenth Author\textsuperscript{3,4}},
%  \textbf{Nineteenth N. Author\textsuperscript{2,5}},
%  \textbf{Twentieth Author\textsuperscript{1}}
%\\
%\\
%  \textsuperscript{1}Affiliation 1,
%  \textsuperscript{2}Affiliation 2,
%  \textsuperscript{3}Affiliation 3,
%  \textsuperscript{4}Affiliation 4,
%  \textsuperscript{5}Affiliation 5
%\\
%  \small{
%    \textbf{Correspondence:} \href{mailto:email@domain}{email@domain}
%  }
%}

\author{Yanyang Li, Michael R. Lyu, Liwei Wang\thanks{Corresponding author.}\\
The Chinese University of Hong Kong \\
\texttt{\{yyli21,lyu,lwwang\}@cse.cuhk.edu.hk}
}

\begin{document}
\maketitle
\begin{abstract}
Solving complex tasks in a single attempt is challenging for large language models (LLMs). Iterative interaction with the environment and feedback is often required to achieve success, making effective feedback utilization a critical topic. Existing approaches either struggle with length generalization or rely on naive retries without leveraging prior information. In this paper, we introduce FTTT, a novel paradigm that formulates feedback utilization as an optimization problem at test time. Additionally, we propose a learnable test-time optimizer, \method{}, to effectively exploit feedback. Experiments on two LLMs across four reasoning datasets demonstrate that FTTT and \method{} achieve superior scalability and performance\footnote{\url{https://github.com/LaVi-Lab/FTTT}}.
\end{abstract}

\section{Introduction}

Leveraging external feedback from interactions with the environment during test time has emerged as a promising approach for large language models (LLMs). This includes applications such as LLM-based agents~\cite{DBLP:conf/iclr/YaoZYDSN023,DBLP:conf/nips/ShinnCGNY23} and, more recently, test-time scaling~\cite{wu2024inferencescalinglawsempirical,DBLP:journals/corr/abs-2408-03314,liu20251bllmsurpass405b}. Such methods further enhance the potential of LLMs to solve challenging tasks, e.g., Olympiad-level math problems~\cite{guan2025rstarmathsmallllmsmaster} and competitive programming~\cite{openai2025competitiveprogramminglargereasoning}.

Significant progress in this area typically falls into two categories~\cite{DBLP:journals/corr/abs-2408-03314}, as illustrated in Figure~\ref{fig:compare}: sequential revision and parallel sampling. Sequential revision methods~\cite{DBLP:conf/nips/ShinnCGNY23,DBLP:conf/nips/MadaanTGHGW0DPY23} incorporate previous attempts into the LLM's context, while parallel sampling methods~\cite{DBLP:journals/corr/abs-2407-21787,DBLP:conf/nips/XieKZZKHX23} generate new attempts independently of prior failures.
However, both approaches have notable limitations. Sequential revision is computationally expensive due to long context lengths and faces challenges~\cite{muennighoff2025s1simpletesttimescaling}, such as position bias~\cite{DBLP:journals/tacl/LiuLHPBPL24} and attention noise~\cite{ye2024differentialtransformer}. In contrast, parallel sampling, while efficient, fails to learn from previous errors~\cite{DBLP:journals/corr/abs-2407-21787}.
Unlike these paradigms, human reasoning follows a different pattern: humans store recent experiences in ``fast weights''~\cite{DBLP:conf/nips/BaHMLI16}, enabling them to neither revisit past errors explicitly nor start each attempt without any prior knowledge. Recent research suggests that the weights of neural networks could serve as a natural memory mechanism during test time~\cite{DBLP:journals/corr/abs-2401-11504}.

% poor length generalization \cite{muennighoff2025s1simpletesttimescaling}

% in-context learning is similar to gradient descent~\cite{DBLP:conf/icml/OswaldNRSMZV23}

% store not only to parameters~\cite{DBLP:journals/corr/abs-2401-11504} but also activations~\cite{DBLP:conf/iclr/DathathriMLHFMY20,DBLP:journals/corr/abs-2404-03592}

Building on these observations, we propose a novel paradigm that leverages Test-Time Training (TTT)~\cite{DBLP:conf/icml/SunWLMEH20,DBLP:journals/corr/abs-2310-13807} to store past experiences in model weights rather than in the context. This approach bridges the gap between sequential revision and parallel sampling by indirectly incorporating knowledge into the LLM without disrupting in-context reasoning. Specifically, we introduce \textbf{F}eedback-based \textbf{T}est-\textbf{T}ime \textbf{T}raining (FTTT), which employs a carefully designed TTT task enriched with feedback through self-reflection. We demonstrate that FTTT improves test-time computation scalability on two mathematical reasoning and two code generation datasets, using \texttt{Llama-3.1-8B-Instruct}~\cite{DBLP:journals/corr/abs-2407-21783} and \texttt{Mistral-7B-Instruct-v0.3}~\cite{DBLP:journals/corr/abs-2310-06825}.

\begin{figure*}[t!]
    \newdimen\base
    \base=0.7cm

    \tikzstyle{textnode} = [rectangle,font=\scriptsize,draw=black,inner sep=0pt,outer sep=0pt,minimum width=2\base,minimum height=\base,rounded corners=2pt]

    \hspace*{\fill}
    \subfigure[Sequential Revision]
    {
        \centering
        \begin{tikzpicture}
        
        \node[textnode, fill=lyyblue!40!white] (question) at (0,0) {Question};
        \node[textnode, anchor=north, fill=red!30!white] (answer1) at ([yshift=-0.5\base]question.south) {Attempt \#1};
        \node[textnode, anchor=north, fill=red!30!white] (answer2) at ([yshift=-0.5\base]answer1.south) {Attempt \#2};
        \node[textnode, anchor=north, fill=red!30!white] (answer3) at ([yshift=-0.5\base]answer2.south) {Attempt \#3};
        \node[textnode, anchor=north, fill=ugreen!30!white] (answer4) at ([yshift=-0.5\base]answer3.south) {Attempt \#4};
        
        \draw[-latex] (question.east) to[out=0, in=0, looseness=3] ([xshift=\base, yshift=-0.25\base]question.south) to ([xshift=-\base, yshift=0.25\base]answer1.north) to[out=180, in=180, looseness=3] (answer1.west);
        \draw[-latex] (answer1.east) to[out=0, in=0, looseness=3] ([xshift=\base, yshift=-0.25\base]answer1.south) to ([xshift=-\base, yshift=0.25\base]answer2.north) to[out=180, in=180, looseness=3] (answer2.west);
        \draw[-latex] (answer2.east) to[out=0, in=0, looseness=3] ([xshift=\base, yshift=-0.25\base]answer2.south) to ([xshift=-\base, yshift=0.25\base]answer3.north) to[out=180, in=180, looseness=3] (answer3.west);
        \draw[-latex] (answer3.east) to[out=0, in=0, looseness=3] ([xshift=\base, yshift=-0.25\base]answer3.south) to ([xshift=-\base, yshift=0.25\base]answer4.north) to[out=180, in=180, looseness=3] (answer4.west);
        
        \end{tikzpicture}
        \label{fig:revision}
    }
    \hfill
    \subfigure[Parallel Sampling]
    {
        \centering
        \begin{tikzpicture}
        
        \node[textnode, fill=lyyblue!40!white] (question) at (0,0) {Question};
        \node[textnode, anchor=west, fill=red!30!white] (answer1) at ([xshift=2\base, yshift=3\base]question.east) {Attempt \#1};
        \node[textnode, anchor=west, fill=ugreen!30!white] (answer2) at ([xshift=2\base, yshift=\base]question.east) {Attempt \#2};
        \node[textnode, anchor=west, fill=red!30!white] (answer3) at ([xshift=2\base, yshift=-\base]question.east) {Attempt \#3};
        \node[textnode, anchor=west, fill=red!30!white] (answer4) at ([xshift=2\base, yshift=-3\base]question.east) {Attempt \#4};

        \draw[-latex] (question.east) to[out=0, in=180] (answer1.west);
        \draw[-latex] (question.east) to[out=0, in=180] (answer2.west);
        \draw[-latex] (question.east) to[out=0, in=180] (answer3.west);
        \draw[-latex] (question.east) to[out=0, in=180] (answer4.west);
        
        \end{tikzpicture}
      \label{fig:search}
    }
    \hfill
    \subfigure[Feedback-based Test-Time Training]
    {
        \centering
        \begin{tikzpicture}
        
        \node[textnode, fill=lyyblue!40!white] (question) at (0,0) {Question};
        \node[textnode, anchor=west, fill=red!30!white] (answer1) at ([xshift=2\base, yshift=3\base]question.east) {Attempt \#1};
        \node[textnode, anchor=west, fill=red!30!white] (answer2) at ([xshift=2\base, yshift=\base]question.east) {Attempt \#2};
        \node[textnode, anchor=west, fill=red!30!white] (answer3) at ([xshift=2\base, yshift=-\base]question.east) {Attempt \#3};
        \node[textnode, anchor=west, fill=ugreen!30!white] (answer4) at ([xshift=2\base, yshift=-3\base]question.east) {Attempt \#4};

        \draw[-latex] (question.east) to[out=0, in=180] (answer1.west);
        \draw[-latex] (question.east) to[out=0, in=180] (answer2.west);
        \draw[-latex] (question.east) to[out=0, in=180] (answer3.west);
        \draw[-latex] (question.east) to[out=0, in=180] (answer4.west);
        
        \draw[-latex, densely dashed] (answer1.east) to[out=0, in=0, looseness=2] ([xshift=\base, yshift=-0.5\base]answer1.south) to ([xshift=-\base, yshift=0.5\base]answer2.north) to[out=180, in=180, looseness=2] (answer2.west);
        \draw[-latex, densely dashed] (answer2.east) to[out=0, in=0, looseness=2] ([xshift=\base, yshift=-0.5\base]answer2.south) to ([xshift=-\base, yshift=0.5\base]answer3.north) to[out=180, in=180, looseness=2] (answer3.west);
        \draw[-latex, densely dashed] (answer3.east) to[out=0, in=0, looseness=2] ([xshift=\base, yshift=-0.5\base]answer3.south) to ([xshift=-\base, yshift=0.5\base]answer4.north) to[out=180, in=180, looseness=2] (answer4.west);
        
        \end{tikzpicture}
      \label{fig:ttt}
    }
    \hspace*{\fill}
    \caption{Comparison between sequential revision, parallel sampling, and feedback-based test-time training. {\protect\tikz \protect\draw[color=black, fill=red!30!white] plot[mark=square*, mark options={scale=1.4}] (0,0);} is the failed attempt and {\protect\tikz \protect\draw[color=black, fill=ugreen!30!white] plot[mark=square*, mark options={scale=1.4}] (0,0);} is the successful attempt. {\protect\tikz {\protect\draw[-latex, thick, color=black] (0,0.5) -- (0.5,0.5);\protect\draw[opacity=0] (0,0.4) -- (0.5,0.4);}} indicates the LLM generation with the input on the left of the arrow and the output on the right. {\protect\tikz {\protect\draw[-latex, thick, densely dashed, color=black] (0,0.5) -- (0.5,0.5);\protect\draw[opacity=0] (0,0.4) -- (0.5,0.4);}} denotes the LLM training, where the left of the arrow is the training data.}
    \label{fig:compare}
\end{figure*}

Inspired by advancements in learning to optimize~\cite{DBLP:journals/jmlr/ChenCCH0WY22}, we explore training learnable test-time optimizers to yield Pareto-optimal cost-performance tradeoffs. Our proposed learnable optimizer, \method{}, is a lightweight neural network that predicts weight updates from the gradients of the previous attempt. Unlike traditional parameter-efficient fine-tuning (PEFT) methods, \method{} works on the gradient rather than the activation space. Experiments on three reasoning datasets and two different LLMs demonstrate the effectiveness of \method{}, outperforming five widely used PEFT baselines.

% compatible with advanced search by deeply modifying the static policy

\section{Feedback-based Test-Time Training}
\label{sec:ttt}

\subsection{The Test-Time Training Task}

The problem of exploiting test-time feedback is as~\cite{DBLP:conf/nips/ShinnCGNY23}: given a question $Q$, a model $M$ attempts to solve $Q$ within a budget of $N$ attempts. A verifier $V$ evaluates each attempt, such as the $n$-th attempt $A_n$, and provides feedback $V(A_n)$. This work focuses on binary verifiers, which determine whether $A_n$ is correct. These verifiers are well-established, rule-based systems that are both cost-effective and efficient to evaluate.

When the model generates attempts sequentially, our goal is to enable $M$ to learn from previous attempts to improve subsequent ones. To achieve this, we frame learning from previous attempts as a training problem: at each step $n$, we optimize $M$ using $Q$, $A_n$, and $V(A_n)$, aiming for $M$ to generate a better $A_{n+1}$. This way internalizes the past attempts into weights for efficient inference of $A_{n+1}$. As a result, the sequence of attempts can be viewed as an $N$-step optimization process.

A key challenge is designing an effective supervised task using $Q$, $A_n$, and $V(A_n)$ to improve the model's ability to solve $Q$. We build on the intuition that \textit{a model capable of judging the correctness of a solution should also be able to solve the question itself}. Concretely, given $Q$ and $A_n$, we train $M$ to predict verbal feedback $F$ that aligns with $V(A_n)$. This leads to our FTTT loss:
\begin{equation}
\scalebox{0.8}{\ensuremath{
    \mathcal{L}_\mathrm{FTTT}\left(Q, A_n\right)=-\frac{1}{l_0}\log M_{n-1}\left(F\mid Q, A_n\right)
}}
    \label{eqn:ttt-loss}
\end{equation}
where $l_0$ is the length of $F$ and $M_0$ denotes the raw LLM. In this work, $F$ is set to ``Your answer is incorrect.'' when $V\left( A_n\right)$ implies an incorrect $A_n$.

\subsection{Self-Reflected Feedback}

Since we are working with a binary verifier, the learning signal is limited at each interaction. Previous research suggests that LLMs can self-correct errors when provided with external signals~\cite{DBLP:conf/iclr/0009CMZYSZ24}. Inspired by this, we aim to enhance the learning signal by leveraging the model to generate silver-standard training labels.

We first sample a reflection $R_n$ from the model given $Q$, $A_n$, $F$ and the instruction $P$:
\begin{equation}
% \scalebox{0.8}{\ensuremath{
    R_n\sim M_0\left(R\mid Q,A_n,F,P\right)
% }}
\end{equation}
In practice, we use $M_0$ to generate $R_n$ to mitigate the risk of degraded self-reflection ability after training.
% Below is an example for \texttt{Llama-3.1-8B-Instruct}~\cite{DBLP:journals/corr/abs-2407-21783}:
% \patchcmd{\quote}{\rightmargin}{\leftmargin 15pt \rightmargin}{}{}
% \begin{quote}
% \small %\it
% \begin{tcolorbox}[breakable, colback=white, colbacktitle=blue!5!white, colframe=black, boxrule=1pt, title={\textcolor{black}{\textbf{Reflection Generation Example}}}]
% \textbf{User:} Solve the following math problem $\ldots$\\
% \textbf{Assistant:} $\ldots$ the final answer is: $\ldots$\\
% \textbf{User:} Your answer is incorrect. \textcolor{gray}{Please carefully check the solution and summarize all mistakes in short. Do NOT provide the corrected solution. Do NOT say ``my solution''.}\\
% \textbf{Assistant:} \textcolor{royalblue}{Here is the summary of the mistakes in the previous solution $\ldots$}
% \end{tcolorbox}
% \end{quote}
% \patchcmd{\quote}{\rightmargin}{\leftmargin 26pt \rightmargin}{}{}
% The sentence in \textcolor{gray}{gray} is $P$ and the one in \textcolor{royalblue}{blue} is the generated reflection $R_n$.
The auxiliary loss is then defined as:
\begin{equation}
\scalebox{0.8}{\ensuremath{
    \mathcal{L}_\mathrm{aux}\left(Q, A_n, R_n\right)=-\frac{1}{l_n} \log M_{n-1}\left(R_n\mid Q, A_n, F\right)
}}
    \label{eqn:reflect-loss}
\end{equation}
where $l_n$ is the length of $R_n$. Eq.~\ref{eqn:reflect-loss} can be interpreted as a sequence-level distillation loss~\cite{kim-rush-2016-sequence}, where knowledge from the raw model $M_0$ is distilled into the trained model $M_{n-1}$ to prevent overfitting.
Finally, the overall loss is as:
\begin{equation}
% \scalebox{0.8}{\ensuremath{
    \mathcal{L}_\mathrm{final}=\mathcal{L}_\mathrm{FTTT} + \mathcal{L}_\mathrm{aux}
% }}
\end{equation}
Below is a training example with self-reflection, where underlined sentences are the training target:

\patchcmd{\quote}{\rightmargin}{\leftmargin 15pt \rightmargin}{}{}
\begin{quote}
\small %\it
\begin{tcolorbox}[breakable, colback=white, colbacktitle=blue!5!white, colframe=black, boxrule=1pt, title={\textcolor{black}{\textbf{Training Example with Self-Reflection}}}]
\textbf{User:} Solve the following math problem $\ldots$\\
\textbf{Assistant:} $\ldots$ the final answer is: $\ldots$\\
\textbf{User:} \ul{Your answer is incorrect.} \uwave{Here is the summary of the mistakes in the previous solution $\ldots$}
\end{tcolorbox}
\end{quote}
\patchcmd{\quote}{\rightmargin}{\leftmargin 26pt \rightmargin}{}{}
The \ul{underlined} sentence corresponds to $F$ in Eq.~\ref{eqn:ttt-loss} and the \uwave{wave-underlined} sentence represents $R_n$ for Eq.~\ref{eqn:reflect-loss}.
Algorithm~\ref{alg:ttt} summarizes our FTTT.


\section{\thename}
\subsection{End-to-End Driving Policy}
The overall framework of \thename{} is depicted in Fig.~\ref{fig:framework}. 
\thename{} takes multi-view image sequences as input, transforms the sensor data into scene token embeddings, outputs the probabilistic distribution of actions, and samples an action to control the vehicle. 

\boldparagraph{BEV Encoder.} 
We first employ a BEV encoder~\cite{li2022bevformer} to transform multi-view image features from the perspective view to the Bird's Eye View (BEV), obtaining a feature map in the BEV space. This feature map is then used to learn instance-level map features and agent features.

\boldparagraph{Map Head.} 
Then we utilize a group of map tokens~\cite{maptrv2, liao2022maptr, lanegap} to learn the vectorized map elements of the driving scene from the BEV feature map, including lane centerlines, lane dividers, road boundaries, arrows, traffic signals, \etc.

\boldparagraph{Agent Head.} 
Besides, a group of agent tokens~\cite{jiang2022pip} is adopted to predict the motion information of other traffic participants, including location, orientation, size, speed, and multi-mode future trajectories.

\boldparagraph{Image Encoder.} 
Apart from the above instance-level map and agent tokens, we also use an individual image encoder~\cite{vit,he2016resnet} to transform the original images into image tokens. These image tokens provide dense and rich scene information for planning, complementary to the instance-level tokens.

\begin{figure}[t]
\centering
\includegraphics[width=0.98\linewidth]{fig/post-training-2.pdf} 
\caption{\textbf{Post-training.}  $N$  workers parallelly run. The generated rollout data $(s_t,a_t, r_{t+1},s_{t+1},...)$ are recorded in a rollout buffer. Rollout data and human driving demonstrations are used in RL- and IL-training steps to fine-tune the AD policy synergistically.
}
\label{fig:post-training}
\end{figure}

\boldparagraph{Action Space.} 
To accelerate the convergence of RL training, we design a decoupled discrete action representation. 
We divide the action into two independent components: lateral action and longitudinal action. 
The action space is constructed over a short $0.5$-second time horizon, during which the vehicle's motion is approximated by assuming constant linear and angular velocities. 
Under this assumption, the lateral action $a^x$ and longitudinal action $a^y$ can be directly computed based on the current linear and angular velocities.
By combining decoupling with a limited temporal scope and simplified motion model, our approach effectively reduces the dimensionality of the action space, accelerating training convergence.


\boldparagraph{Planning Head.} 
We use $E_\text{scene}$ to denote the scene representation, which consists of map tokens, agent tokens, and image tokens. We initialize a planning embedding denoted as $E_\text{plan}$. A cascaded Transformer decoder $\phi$ takes the planning embedding $E_\text{plan}$ as the query and the scene representation $E_\text{scene}$ as both key and value.

The output of the decoder $\phi$ is then combined with navigation information $E_\text{navi}$ and ego state $E_\text{state}$ to output the probabilistic distributions of the lateral action $a^x$ and the longitudinal action $a^y$:
\begin{equation}
\begin{aligned}
     \pi(a^x\mid s) = & \text{softmax}(\text{MLP}(\phi(E_\text{plan}, E_\text{scene}) \\
    & + E_\text{navi} + E_\text{state})), \\
     \pi(a^y\mid s) = & \text{softmax}(\text{MLP}(\phi(E_\text{plan}, E_\text{scene}) \\
     & + E_\text{navi} + E_\text{state})),
\label{eq:action distribution}
\end{aligned}
\end{equation}
where $E_\text{plan}$, $E_\text{navi}$, $E_\text{state}$, and the output of $\text{MLP}$ are all of the same dimension ($1 \times D$).

The planning head also outputs the value functions $V_x(s)$ and $V_y(s)$, which estimate the expected cumulative rewards for the lateral and longitudinal actions, respectively: 
\begin{equation}
\begin{aligned}
    & V_x(s) = \text{MLP}(\phi(E_\text{plan}, E_\text{scene}) + E_\text{navi} + E_\text{state}), \\
    & V_y(s) = \text{MLP}(\phi(E_\text{plan}, E_\text{scene}) + E_\text{navi} + E_\text{state}).
\end{aligned}
\end{equation}
The value functions are used in RL training (Sec.~\ref{sec:optimization}).

\subsection{Training Paradigm}
We adopt a three-stage training paradigm: perception pre-training, planning pre-training, and reinforced post-training, as shown in Fig.~\ref{fig:framework}.

\boldparagraph{Perception Pre-Training.} 
Information in the image is sparse and low-level. In the first stage,  
the map head and the agent head explicitly output map elements and agent motion information, which are supervised with ground-truth labels. Consequently,  
map tokens and agent tokens implicitly encode the corresponding high-level information.  
In this stage, we only update the parameters of the BEV encoder, the map head, and the agent head.



\boldparagraph{Planning Pre-Training.} 
In the second stage, to prevent the unstable cold start of RL training, IL is first performed to initialize the probabilistic distribution of actions based on large-scale real-world driving demonstrations from expert drivers. In this stage, we only update the parameters of the image encoder and the planning head, while the parameters of the BEV encoder, map head, and agent head are frozen. The optimization objectives of perception tasks and planning tasks may conflict with each other. However, with the training stage and parameters decoupled, such conflicts are mostly avoided.

\boldparagraph{Reinforced Post-Training.} 
In the reinforced post-training, RL and IL synergistically fine-tune the distribution. RL aims to guide the policy to be sensitive to critical risky events and adaptive to out-of-distribution situations. IL serves as the regularization term to keep the policy's behavior similar to that of humans.

We select a large amount of risky dense-traffic clips from collected driving demonstrations. For each clip, we train an independent 3DGS model that reconstructs the clip and serves as a digital driving environment.  
As shown in Fig.~\ref{fig:post-training}, we set $N$ parallel workers.  
Each worker randomly samples a 3DGS environment and begins rollout, i.e., the AD policy controls the ego vehicle to move and iteratively interacts with the 3DGS environment. After the rollout process of this 3DGS environment ends, the generated rollout data $(s_t,a_t, r_{t+1},s_{t+1},...)$ are recorded in a rollout buffer, and the worker will sample a new 3DGS environment for another round of rollout.

As for policy optimization, we iteratively perform RL-training steps and IL-training steps. For RL-training steps, we sample data from the rollout buffer and follow the Proximal Policy Optimization (PPO) framework~\cite{PPO} to update the AD policy. For IL-training steps, we use real-world driving demonstrations to update the policy. After a fixed number of training steps, the updated AD policy is sent to every worker to replace the old one, to avoid a distribution shift between data collection and optimization.
We only update the parameters of the image encoder and the planning head. The parameters of the BEV encoder, the map head, and the agent head are frozen.  
The detailed RL design is presented below.

\subsection{Interaction Mechanism between AD Policy and 3DGS Environment}
In the 3DGS environment, the ego vehicle acts according to the AD policy. Other traffic participants act according to real-world data in a log-replay manner.  
A simplified kinematic bicycle model is employed to iteratively update the ego vehicle's pose at every $\Delta t$ seconds as follows:  
\begin{equation}
\begin{aligned}
x_{t+1}^{w} & = x_{t}^w + v_t \cos \left(\psi_{t}^w\right) \Delta t, \\
y_{t+1}^{w} & = y_{t}^w + v_t \sin \left(\psi_{t}^w\right) \Delta t, \\
\psi_{t+1}^{w} & = \psi_{t}^w + \frac{v_t}{L} \tan \left(\delta_t\right) \Delta t,
\label{equation:kinematic_model}
\end{aligned}
\end{equation}  
where $x_t^{w}$ and $y_t^{w}$ denote the position of the ego vehicle relative to the world coordinate; $\psi_t^w$ is the heading angle that defines the vehicle's orientation with respect to the world $x$-coordinate; $v_t$ is the linear velocity of the ego vehicle; $\delta_t$ is the steering angle of the front wheels; and $L$ is the wheelbase, i.e., the distance between the front and rear axles.

During the rollout process, the AD policy outputs actions $(a_t^x, a_t^y)$ for a $0.5$-second time horizon at time step $t$. We derive the linear velocity $v_t$ and steering angle $\delta_t$ based on $(a_t^x, a_t^y)$.  
Based on the kinematic model in Eq.~\ref{equation:kinematic_model},  
the pose of the ego vehicle in the world coordinate system is updated from ${p}_t = (x_{t}^w, y_{t}^w, \psi_{t}^w)$ to ${p}_{t+1} = (x_{t+1}^{w}, y_{t+1}^{w}, \psi_{t+1}^{w})$.  

Based on the updated ${p}_{t+1}$, the 3DGS environment computes the new ego vehicle's state $s_{t+1}$. The updated pose ${p}_{t+1}$ and state $s_{t+1}$ serve as the input for the next iteration of the inference process.

The 3DGS environment also generates rewards $\mathcal{R}$ (Sec.~\ref{sec:reward}) according to multi-source information (including trajectories of other agents, map information, the expert trajectory of the ego vehicle, and the parameters of Gaussians), which are used to optimize the AD policy (Sec.~\ref{sec:optimization}).

\begin{figure}[t]
\centering
\includegraphics[width=1.0\linewidth]{fig/reward.pdf} 
\caption{\textbf{Example diagram of four types of reward sources.}  (1): Collision with a dynamic obstacle ahead triggers a reward $r_{\text{dc}}$. (2): Hitting a static roadside obstacle incurs a reward $r_{\text{sc}}$. (3): Moving onto the curb exceeds the positional deviation threshold $d_{\text{max}}$, triggering a reward $r_{\text{pd}}$. (4): Drifting toward the adjacent lane exceeds the heading deviation threshold $\psi_{\text{max}}$, triggering a reward $r_{\text{hd}}$.
}
\label{fig: reward source}
\end{figure}
\subsection{Reward Modeling}
\label{sec:reward}
The reward is the source of the training signal, which determines the optimization direction of RL. The reward function is designed to guide the ego vehicle's behavior by penalizing unsafe actions and encouraging alignment with the expert trajectory. It is composed of four reward components: (1) collision with dynamic obstacles, (2) collision with static obstacles, (3) positional deviation from the expert trajectory, and (4) heading deviation from the expert trajectory:
\begin{equation}
\begin{aligned}
\mathcal{R} = \{r_{\text{dc}}, r_{\text{sc}}, r_{\text{pd}}, r_{\text{hd}}  \}. 
\end{aligned}
\end{equation}

As illustrated in Fig.~\ref{fig: reward source}, these reward components are triggered under specific conditions.  
In the 3DGS environment, dynamic collision is detected if the ego vehicle's bounding box overlaps with the annotated bounding boxes of dynamic obstacles, triggering a negative reward $r_{\text{dc}}$. Similarly, static collision is identified when the ego vehicle's bounding box overlaps with the Gaussians of static obstacles, resulting in a negative reward $r_{\text{sc}}$.  
Positional deviation is measured as the Euclidean distance between the ego vehicle's current position and the closest point on the expert trajectory. A deviation beyond a predefined threshold $d_{\text{max}}$ incurs a negative reward $r_{\text{pd}}$.  
Heading deviation is calculated as the angular difference between the ego vehicle's current heading angle $ \psi_t $ and the expert trajectory's matched heading angle $\psi_{\text{expert}}$. A deviation beyond a threshold $ \psi_{\text{max}}$ results in a negative reward $r_{\text{hd}}$.

Any of these events, including dynamic collision, static collision, excessive positional deviation, or excessive heading deviation, triggers immediate episode termination. Because after such events occur, the 3DGS environment typically generates noisy sensor data, which is detrimental to RL training.

\subsection{Policy Optimization}
\label{sec:optimization}
In the closed-loop environment, the error in each single step accumulates over time. The aforementioned rewards are not only caused by the current action but also by the actions of the preceding steps.  
The rewards are propagated forward with Generalized Advantage Estimation (GAE)~\cite{gae} to optimize the action distribution of the preceding steps.

Specifically, for each time step $t$, we store the current state $s_t$, action $a_t$, reward $r_t$, and the estimate of the value $V(s_t)$.  
Based on the decoupled action space, and considering that different rewards have different correlations to lateral and longitudinal actions, the reward $r_t$ is divided into lateral reward $r_t^x$ and longitudinal reward $r_t^y$:
\begin{equation}
\begin{aligned}
r_t^x &= r_t^{\text{sc}} + r_t^{\text{pd}} + r_t^{\text{hd}}, \\
r_t^y &= r_t^{\text{dc}}.
\label{eq:reward-decouple}
\end{aligned}
\end{equation}
Similarly, the value function $V(s_t)$ is decoupled into two components: $V_x(s_t)$ for the lateral dimension and $V_y(s_t)$ for the longitudinal dimension. These value functions estimate the expected cumulative rewards for the lateral and longitudinal actions, respectively. The advantage estimates $\hat{A}_t^x$ and $\hat{A}_t^y$ are then computed as follows:
\begin{equation}
\begin{aligned}
\delta_t^x &= r_t^x + \gamma V_x(s_{t+1}) - V_x(s_t), \\
\delta_t^y &= r_t^y + \gamma V_y(s_{t+1}) - V_y(s_t), \\
\hat{A}_t^x &= \sum_{l=0}^{\infty}(\gamma \lambda)^l \delta_{t+l}^x, \\
\hat{A}_t^y &= \sum_{l=0}^{\infty}(\gamma \lambda)^l \delta_{t+l}^y,
\label{eq:advantage}
\end{aligned}
\end{equation}
where $\delta_t^x$ and $\delta_t^y$ are the temporal difference errors for the lateral and longitudinal dimensions, $\gamma$ is the discount factor, and $\lambda$ is the GAE parameter that controls the trade-off between bias and variance.

To further clarify the relationship between the advantage estimates and the reward components, we decompose $\hat{A}_t^x$ and $\hat{A}_t^y$ based on the reward decomposition in Eq.~\ref{eq:reward-decouple} and the advantage estimation in Eq.~\ref{eq:advantage}. Specifically, we derive the following decomposition:
\begin{equation}
\begin{aligned}
\hat{A}_t^x &= \hat{A}_t^{\text{sc}} + \hat{A}_t^{\text{pd}} + \hat{A}_t^{\text{hd}}, \\
\hat{A}_t^y &= \hat{A}_t^{\text{dc}},
\end{aligned}
\end{equation}
where $\hat{A}_t^{\text{sc}}$ is the advantage estimate for avoiding static collisions, $\hat{A}_t^{\text{pd}}$ is the advantage estimate for minimizing positional deviations, $\hat{A}_t^{\text{hd}}$ is the advantage estimate for minimizing heading deviations, and $\hat{A}_t^{\text{dc}}$ is the advantage estimate for avoiding dynamic collisions.

These advantage estimates are used to guide the update of the AD policy $\pi_{\theta}$, following the PPO framework~\cite{PPO}. By leveraging the decomposed advantage estimates $\hat{A}_t^x$ and $\hat{A}_t^y$, we can independently optimize the lateral and longitudinal dimensions of the policy. This is achieved by defining separate objective functions $\mathcal{L}_x^{\text{CLIP}}(\theta)$ and $\mathcal{L}_y^{\text{CLIP}}(\theta)$ for each dimension,  as follows:
\begin{equation}
\begin{aligned}
\mathcal{L}_x^{\text{PPO}}(\theta) &= \mathbb{E}_t \left[ \min \left( \rho_t^x \hat{A}_t^x, \ \text{clip}(\rho_t^x, 1-\epsilon_x, 1+\epsilon_x) \hat{A}_t^x \right) \right], \\
\mathcal{L}_y^{\text{PPO}}(\theta) &= \mathbb{E}_t \left[ \min \left( \rho_t^y \hat{A}_t^y, \ \text{clip}(\rho_t^y, 1-\epsilon_y, 1+\epsilon_y) \hat{A}_t^y \right) \right], \\
\mathcal{L}^{\text{PPO}}(\theta) &= \mathcal{L}_x^{\text{PPO}}(\theta) + \mathcal{L}_y^{\text{PPO}}(\theta),
\end{aligned}
\end{equation}
where $\rho_t^x = \frac{\pi_{\theta}(a_t^x \mid s_t)}{\pi_{\theta_{\text{old}}}(a_t^x \mid s_t)}$ is the importance sampling ratio for the lateral dimension, $\rho_t^y = \frac{\pi_{\theta}(a_t^y \mid s_t)}{\pi_{\theta_{\text{old}}}(a_t^y \mid s_t)}$ is the importance sampling ratio for the longitudinal dimension, $\epsilon_x$ and $\epsilon_y$ are small constants that control the clipping range for the lateral and longitudinal dimensions, ensuring stable policy updates.

The clipped objective function $\mathcal{L}^{\text{PPO}}(\theta)$ prevents excessively large updates to the policy parameters $\theta$, thereby maintaining training stability.

\begin{table*}[ht]
    \centering
{
\begin{tabular}{lccccccccc}
    \toprule
    RL:IL & CR$\downarrow$ & DCR$\downarrow$ & SCR$\downarrow$ & DR$\downarrow$ & PDR$\downarrow$ & HDR$\downarrow$ &ADD$\downarrow$ & Long. Jerk$\downarrow$ & Lat. Jerk$\downarrow$ \\
    \midrule
     0:1  & 0.229 & 0.211 & 0.018 & 0.066 & 0.039 & 0.027  & 0.238 & 3.928 & 0.103\\
     1:0  & 0.143 & 0.128 & 0.015 &0.080 &0.065 &0.015 &0.345 &4.204 &0.085\\
     2:1 & 0.137 & 0.125 & 0.012 & 0.059 & 0.050 & 0.009  & 0.274 & 4.538 & 0.092\\
     4:1 & 0.089 & 0.080 & 0.009 & 0.063 & 0.042 & 0.021  & 0.257 & 4.495 & 0.082 \\
     8:1 & 0.125 & 0.116 & 0.009 & 0.084 & 0.045 & 0.039  & 0.323 & 5.285 & 0.115\\
    \bottomrule
\end{tabular}
}
    \caption{\textbf{Ablation on RL-to-IL step mixing ratios in the reinforced post-training stage.}}
    \label{tab:ratio}
\end{table*}

\subsection{Auxiliary Objective}
RL usually faces the challenge of sparse rewards, which makes the convergence process unstable and slow. To speed up convergence, we introduce auxiliary objectives that provide dense guidance to the entire action distribution.

The auxiliary objectives are designed to penalize undesirable behaviors by incorporating specific reward sources, including dynamic collisions, static collisions, positional deviations, and heading deviations. These objectives are computed based on the actions \( a_t^{x, \text{old}} \) and \( a_t^{y, \text{old}} \) selected by the old AD policy \( \pi_{\theta_{\text{old}}} \) at time step \( t \). To facilitate the evaluation of these actions, we separate the probability distribution of the action into four parts:
\begin{equation}
\begin{aligned}
\Delta \pi_y^{\text{dec}} &= \sum_{a_t^y < a_t^{y, \text{old}}} \pi_\theta(a_t^y \mid s_t), \\
\Delta \pi_y^{\text{acc}} &= \sum_{a_t^y > a_t^{y, \text{old}}} \pi_\theta(a_t^y \mid s_t), \\
\Delta \pi_x^{\text{left}} &= \sum_{a_t^x < a_t^{x, \text{old}}} \pi_\theta(a_t^x \mid s_t), \\
\Delta \pi_x^{\text{right}} &= \sum_{a_t^x > a_t^{x, \text{old}}} \pi_\theta(a_t^x \mid s_t).
\end{aligned}
\end{equation}
Here, \( \Delta \pi_y^{\text{dec}} \) represents the total probability of deceleration actions, \( \Delta \pi_y^{\text{acc}} \) represents the total probability of acceleration actions, \( \Delta \pi_x^{\text{left}} \) represents the total probability of leftward steering actions, and \( \Delta \pi_x^{\text{right}} \) represents the total probability of rightward steering actions.

\boldparagraph{Dynamic Collision Auxiliary Objective.}  
The dynamic collision auxiliary objective adjusts the longitudinal control action \(a_t^y\) based on the location of potential collisions relative to the ego vehicle. If a collision is detected ahead, the policy prioritizes deceleration actions (\(a_t^y < a_t^{y, \text{old}}\)); if a collision is detected behind, it encourages acceleration actions (\(a_t^y > a_t^{y, \text{old}}\)). To formalize this behavior, we define a directional factor \(f_\text{dc}\):
\begin{equation}
\begin{aligned}
f_\text{dc} = \begin{cases} 
1 & \text{if the collision is ahead}, \\
-1 & \text{if the collision is behind}.
\end{cases} 
\end{aligned}
\end{equation}

The auxiliary objective for dynamic collision avoidance is defined as:
\begin{equation}
\begin{aligned}
\mathcal{L}_\text{dc}(\theta_y) = \mathbb{E}_t \left[ 
    \hat{A}_t^\text{dc} \cdot f_\text{dc} \cdot (\Delta \pi_y^{\text{dec}} - \Delta \pi_y^{\text{acc}})
\right],
\end{aligned}
\end{equation}
where \(\hat{A}_t^\text{dc}\) is the advantage estimate for dynamic collision avoidance.

\boldparagraph{Static Collision Auxiliary Objective.}  
The static collision auxiliary objective adjusts the steering control action $a_t^x$ based on the proximity to static obstacles. If the static obstacle is detected on the left side, the policy promotes rightward steering actions ($a_t^x > a_t^{x,\text{old}}$); if the static obstacle is detected on the right side, it promotes leftward steering actions ($a_t^x < a_t^{x,\text{old}}$). To formalize this behavior, we define a directional factor $f_\text{sc}$:  
\begin{equation}
\begin{aligned}
f_\text{sc} = \begin{cases} 
1 & \text{if static obstacle is on the left}, \\
-1 & \text{if static obstacle is on the right}.
\end{cases} 
\end{aligned}
\end{equation}

The auxiliary objective for static collision avoidance is defined as:  
\begin{equation}
\begin{aligned}
\mathcal{L}_\text{sc}(\theta_x) = \mathbb{E}_t \left[ 
    \hat{A}_t^\text{sc} \cdot f_\text{sc} \cdot (\Delta \pi_x^{\text{right}} - \Delta \pi_x^{\text{left}})
\right],
\end{aligned}
\end{equation}  
where $\hat{A}_t^\text{sc}$ is the advantage estimate for static collision avoidance.  

\boldparagraph{Positional Deviation Auxiliary Objective.}  
The positional deviation auxiliary objective adjusts the steering control action $a_t^x$ based on the ego vehicle's lateral deviation from the expert trajectory. If the ego vehicle deviates leftward, the policy promotes rightward corrections ($a_t^x > a_t^{x,\text{old}}$); if it deviates rightward, it promotes leftward corrections ($a_t^x < a_t^{x,\text{old}}$). We formalize this with a directional factor $f_\text{pd}$:  
\begin{equation}
\begin{aligned}
f_\text{pd} = \begin{cases} 
1 & \text{if ego vehicle deviates leftward}, \\
-1 & \text{if ego vehicle deviates rightward}.
\end{cases} 
\end{aligned}
\end{equation}

The auxiliary objective for positional deviation correction is:
\begin{equation}
\begin{aligned}
\mathcal{L}_\text{pd}(\theta_x) = \mathbb{E}_t \left[ 
    \hat{A}_t^\text{pd} \cdot f_\text{pd} \cdot (\Delta \pi_x^{\text{right}} - \Delta \pi_x^{\text{left}})
\right],
\end{aligned}
\end{equation}  
where $\hat{A}_t^\text{pd}$ estimates the advantage of trajectory alignment.

\boldparagraph{Heading Deviation Auxiliary Objective.}  
The heading deviation auxiliary objective adjusts the steering control action $a_t^x$ based on the angular difference between the ego vehicle’s current heading and the expert’s reference heading. If the ego vehicle deviates counterclockwise, the policy promotes clockwise corrections ($a_t^x > a_t^{x,\text{old}}$); if it deviates clockwise, it promotes counterclockwise corrections ($a_t^x < a_t^{x,\text{old}}$). To formalize this behavior, we define a directional factor $f_\text{hd}$:  
\begin{equation}
\begin{aligned}
f_\text{hd} = \begin{cases} 
1 & \text{if ego vehicle deviates clockwise}, \\
-1 & \text{if ego vehicle deviates counterclockwise}.
\end{cases} 
\end{aligned}
\end{equation}

The auxiliary objective for heading deviation correction is then defined as:  
\begin{equation}
\begin{aligned}
\mathcal{L}_\text{hd}(\theta_x) = \mathbb{E}_t \left[ 
    \hat{A}_t^\text{hd} \cdot f_\text{hd} \cdot (\Delta \pi_x^{\text{right}} - \Delta \pi_x^{\text{left}})
\right],
\end{aligned}
\end{equation}  
where $\hat{A}_t^\text{hd}$ is the advantage estimate for heading alignment.  

\begin{table*}[ht]
\begin{center}
\centering
\resizebox{0.98\textwidth}{!}{
\begin{tabular}{cccccccccccccc}
\toprule
\multirow{2}{*}{ID} & Dynamic & Static & Position & Heading & \multirow{2}{*}{CR$\downarrow$} &\multirow{2}{*}{DCR$\downarrow$} &\multirow{2}{*}{SCR$\downarrow$} &\multirow{2}{*}{DR$\downarrow$} &\multirow{2}{*}{PDR$\downarrow$} &\multirow{2}{*}{HDR$\downarrow$} &\multirow{2}{*}{ADD$\downarrow$} &\multirow{2}{*}{Long. Jerk$\downarrow$} &\multirow{2}{*}{Lat. Jerk$\downarrow$}\\
& Collision & Collision & Deviation & Deviation & & & & & & & & & \\
\midrule
1 & \cmark  &  &  &  & 0.172 & 0.154 & 0.018 & 0.092 & 0.033 & 0.059  & 0.259 & 4.211 & 0.095 \\
2 &  & \cmark & \cmark & \cmark & 0.238 & 0.217 & 0.021 & 0.090 & 0.045 & 0.045  & 0.241 & 3.937 & 0.098 \\
3 & \cmark &  & \cmark & \cmark & 0.146 & 0.128 & 0.018 & 0.060 & 0.030 & 0.030  & 0.263 & 3.729 & 0.083\\
4 & \cmark & \cmark &  & \cmark & 0.151 & 0.142 & 0.009 & 0.069 & 0.042 & 0.027 & 0.303 & 3.938 & 0.079\\
5 & \cmark & \cmark & \cmark &  & 0.166 & 0.157 & 0.009 & 0.048 & 0.036 & 0.012 & 0.243 & 3.334 & 0.067\\
6 & \cmark & \cmark & \cmark & \cmark & 0.089 & 0.080 & 0.009 & 0.063 & 0.042 & 0.021 & 0.257 & 4.495 & 0.082 \\
\bottomrule
\end{tabular}
}
\end{center}
\vspace{-2mm}
\caption{\textbf{Ablation on reward sources.} The table shows the impact of different reward components on performance.}
\label{tab:reward_ablation}
\end{table*}

\begin{table*}[ht]
\begin{center}
\centering
\resizebox{0.98\textwidth}{!}{
\begin{tabular}{ccccccccccccccc}
\toprule
\multirow{2}{*}{ID} & \multirow{2}{*}{PPO Obj.}  & Dynamic Col. & Static Col. & Position Dev. & Heading Dev. & \multirow{2}{*}{CR$\downarrow$} & \multirow{2}{*}{DCR$\downarrow$}  & \multirow{2}{*}{SCR$\downarrow$} & \multirow{2}{*}{DR$\downarrow$} & \multirow{2}{*}{PDR$\downarrow$} & \multirow{2}{*}{HDR$\downarrow$} & \multirow{2}{*}{ADD$\downarrow$} & \multirow{2}{*}{Long. Jerk$\downarrow$} & \multirow{2}{*}{Lat. Jerk$\downarrow$} \\
& & Auxiliary Obj. & Auxiliary Obj. & Auxiliary Obj. & Auxiliary Obj. & & & & & & & & & \\
\midrule
1 &\cmark&  &  &  &  & 0.249 & 0.223 & 0.026 & 0.077 & 0.047 & 0.030  & 0.266 & 4.209 & 0.104 \\
2 &\cmark& \cmark &  &  &  & 0.178 & 0.163 & 0.015 & 0.151 & 0.101 & 0.050 & 0.301 & 3.906 & 0.085 \\
3 &\cmark&  & \cmark & \cmark & \cmark & 0.137 & 0.125 & 0.012 & 0.157 & 0.145 & 0.012 & 0.296 & 3.419 & 0.071 \\
4 &\cmark& \cmark &  & \cmark & \cmark & 0.169 & 0.151 & 0.018 & 0.075 & 0.042 & 0.033 & 0.254 & 4.450 & 0.098 \\
5 &\cmark& \cmark & \cmark &  & \cmark & 0.149 & 0.134 & 0.015 & 0.063 & 0.057 & 0.006 & 0.324 & 3.980 & 0.086 \\
6 &\cmark& \cmark & \cmark & \cmark & & 0.128 & 0.119  & 0.009 & 0.066 & 0.030 & 0.036  & 0.254 & 4.102 & 0.092 \\
7 &&\cmark  &\cmark  &\cmark  &\cmark  & 0.187 &0.175  &0.012 &0.077 &0.056  &0.021  &0.309  &5.014  &0.112  \\
8 &\cmark& \cmark & \cmark & \cmark & \cmark & 0.089 & 0.080 & 0.009 & 0.063 & 0.042 & 0.021  & 0.257 & 4.495 & 0.082 \\
\bottomrule
\end{tabular}
}
\end{center}
\vspace{-2mm}
\caption{\textbf{Ablation on auxiliary objectives.} The table shows the impact of different auxiliary objectives on performance.}
\label{tab:auxiliary_ablation}
\end{table*}

\boldparagraph{Overall Auxiliary Objectives.}  
The overall auxiliary objectives are a weighted sum of the individual objectives:
\begin{equation}
\begin{aligned}
\mathcal{L}_\text{aux}(\theta) = &\lambda_1 \mathcal{L}_\text{dc}(\theta_y) + \lambda_2 \mathcal{L}_\text{sc}(\theta_x)  + \\ 
&\lambda_3 \mathcal{L}_\text{pd}(\theta_x) +\lambda_4 \mathcal{L}_\text{hd}(\theta_x),
\end{aligned}
\end{equation}
where $\lambda_1$, $\lambda_2$, $\lambda_3$, and $\lambda_4$ are weighting coefficients that balance the contributions of each auxiliary objective.

\boldparagraph{Optimization Objective.}  
The final optimization objective combines the clipped PPO objective with the auxiliary objective:
\begin{equation}
\mathcal{L}(\theta) = \mathcal{L}^{\text{PPO}}(\theta) + \mathcal{L}_\text{aux}(\theta).
\end{equation}


\paragraph{Discussion.}

As shown in Table~\ref{tab:compare}, FTTT combines the strengths of sequential revision and parallel sampling. Like sequential revision, it leverages memory (i.e., weights) to store past failed attempts, while avoiding the length generalization issues as in parallel sampling. Figure~\ref{fig:compare} highlights the advantages of FTTT from a probabilistic graphical model perspective, where both sequential revision and parallel sampling are special cases of FTTT with additional independence assumptions. In particular, sequential revision assumes that attempts form a Markov process, whereas parallel sampling treats each attempt as independent.

In terms of efficiency, FTTT is highly optimized, requiring one additional backward propagation computed in parallel for all tokens in one attempt, whose cost is negligible. The dominant overhead of FTTT is generating $R_n$. However, FTTT is still much faster than sequential revision (see Section~\ref{sec:result1}), as $R_n$ is typically short.

Moreover, FTTT closely resembles vanilla TTT~\cite{DBLP:conf/icml/SunWLMEH20}, but with additional inputs beyond $Q$, such as $A_n$, $V\left(A_n\right)$, and $R_n$, as defined in the problem. Consequently, it inherits the same convergence guarantees as TTT.
% Some advent algorithms ~\cite{DBLP:journals/corr/abs-2403-19094,DBLP:conf/nips/YaoYZS00N23,DBLP:journals/corr/abs-2405-03553} may also bridge the gap between sequential revision and parallel sampling to some extent. But they merely manipulate the output logits in the shallow layer and are unable to revise the ranking of unexplored candidates.

\begin{table}
\centering
% \footnotesize
\resizebox{\linewidth}{!}{
\setlength{\tabcolsep}{5pt}
\begin{tabular}[t]{l|ccc}
\toprule
 \makecell[c]{\textbf{Method}} & \makecell[c]{\textbf{Self}\\\textbf{Reflection}} & \makecell[c]{\textbf{Memory}} & \makecell[c]{\textbf{Length}\\\textbf{Generalization}} \\
\midrule
Revision~\cite{DBLP:journals/corr/abs-2408-03314} & \redcross & \greencheck & \redcross \\
Self-Refine~\cite{DBLP:conf/nips/MadaanTGHGW0DPY23} & \greencheck & \greencheck & \redcross \\
Best-of-N~\cite{DBLP:journals/corr/abs-2407-21787} & \redcross & \redcross & \greencheck \\
Beam Search~\cite{ow1988filtered} & \redcross & \redcross & \greencheck \\
Guided Beam Search~\cite{DBLP:conf/nips/XieKZZKHX23} & \greencheck & \redcross & \greencheck \\
\midrule
\textbf{FTTT (ours)} & \greencheck & \greencheck & \greencheck \\
\bottomrule
\end{tabular}
}
% \vspace{-5pt}
\caption{Comparing the advantages and drawbacks of FTTT and related works.}
\label{tab:compare}
% \vspace{-0.5cm}
\end{table}

\section{A Learnable Test-Time Optimizer}

\subsection{The Learning to Optimize Problem}

Although FTTT achieves success (see Section~\ref{sec:result1}), it simply accumulates the gradients of the feedback received so far to update the weights. This raises the question: can we design a better test-time optimizer that more effectively exploits feedback?

Motivated by learning to optimize~\cite{DBLP:journals/jmlr/ChenCCH0WY22}, we adopt a neural network as the test-time optimizer. Concretely, this learnable test-time optimizer is formulated as $f_\theta\left(Q, \left\{A_i, V\left(A_i\right)\right\}^n_{i=1}\right)$, which predicts updates for all LLM weights based on the previous $n$ attempts, and $\theta$ is the optimizer parameter. However, this direct formulation leads to prohibitively large networks due to high-dimensional input and output spaces. For a maximum number of $m$ tokens per attempt and an $l$-layer LLM, the input space grows to $n \times m \times l$, even when updating only a scalar (we exclude the token count of $Q$, as it is significantly smaller than $m$). Since updates for all weight matrices across all layers are predicted jointly, the dimensionality of the input and output spaces becomes unmanageable. We therefore simplify $f_\theta$ by introducing the following assumptions:
\begin{enumerate}[noitemsep, nolistsep, label={(\bfseries A\arabic*):},left=\parindent]
    \item \textbf{Markov Property}: The latest attempt captures all relevant information from previous attempts.
    \item \textbf{Independent Update}: The optimizer predicts updates for each parameter independently, similar to conventional optimizers.
\end{enumerate}

\textbf{A1} eliminates the dependency on $n$ and \textbf{A2} enables updates to be predicted independently for each weight, significantly reducing the size of the output space. The learnable test-time optimizer now becomes $\tilde{\nabla}_{W_i}=f_{\theta_{W_i}}\left(Q, A_n, V\left(A_n\right)\right)$, where it predicts the update for the weight $W_i$ in the $i$-th layer based on $Q$ and the latest attempt $A_n$.

To train all $f$, we define the following loss:
\begin{equation}
\scalebox{0.8}{\ensuremath{
    \mathcal{L}_\mathrm{meta}=-\log M\left(\hat{A} \mid Q, \left\{W+\tilde{\nabla}_{W}\mid\forall \ W \in \mathcal{W}\right\}\right)
}}
    \label{eqn:meta}
\end{equation}
where $\hat{A}$ is the ground-truth for $Q$ and $\mathcal{W}$ is the set of LLM weights. Eq.~\ref{eqn:meta} encourages $f$ to predict updates that increase the likelihood of generating the correct answer after applying the updates.

\subsection{A Parameter-Efficient Architecture in The Gradient Space}

Given the limited learning signal at test time, we design the learnable optimizer to be parameter-efficient to alleviate overfitting. However, the input and output spaces of $f_{\theta_{W_i}}\left(Q, A_n, V\left(A_n\right)\right)$ are large due to their lengths, making even a simple linear projection parameter-intensive. Additionally, $V\left(A_n\right)$ may be heterogeneous to $Q$ and $A_n$, e.g., a scalar, posing challenges for modeling.

Inspired by the success of FTTT in Section~\ref{sec:ttt} and recent works~\cite{DBLP:conf/iclr/MitchellLBFM22,DBLP:journals/corr/abs-2401-11504}, we propose a parameter-efficient architecture in the gradient space as the learnable optimizer.

\paragraph{Gradient-based Input Compression.}

Instead of directly inputting $Q$, $A_n$, and $V\left(A_n\right)$, we first project them into the gradient space, since recent work suggests that long context can be effectively compressed by gradients~\cite{DBLP:journals/corr/abs-2401-11504}. This way reduces the token count $m$ in $A_n$ to a constant and unifies the spaces of $Q$, $A_n$, and $V\left(A_n\right)$ to ease the modeling. To compress $Q$ and $A_n$, we use the next token prediction loss, while for $V\left(A_n\right)$, we include $\mathcal{L}_\mathrm{FTTT}$ in Eq.~\ref{eqn:ttt-loss}. The final loss for compressing the optimizer input is:
\begin{equation}
% \scalebox{0.8}{\ensuremath{
    \mathcal{L}_\mathrm{compress}=-\frac{1}{m}\log M\left(A_n\mid Q\right) + \mathcal{L}_\mathrm{FTTT}
% }}
\end{equation}
The input of $f_{\theta_{W_i}}$ to predict the update of $W_i$ now is the gradient $\nabla_{W_i}$ of $\mathcal{L}_\mathrm{compress}$ w.r.t. $W_i$. Consequently, $f_{\theta_{W_i}}$ receives a fixed-size tensor as input rather than a variable-length sequence.

\paragraph{Gradient Decomposition.}

Although $f_{\theta_{W_i}}$ operates on a smaller space after compression, the dimensionality of the gradient space remains large for direct processing. We utilize the observation that $\nabla_{W_i}\in \mathbb{R}^{d\times d}$ (assuming $W_i\in \mathbb{R}^{d\times d}$) can be decomposed into two vectors to further reduce the dimensionalities: the input to a linear projection with weight $W_i$, $u_i\in \mathbb{R}^{d \times 1}$, and the gradient of $\mathcal{L}_\mathrm{compress}$ w.r.t. the output of the projection, $\delta_{i+1}\in \mathbb{R}^{d \times 1}$~\cite{DBLP:conf/iclr/MitchellLBFM22}. In this framework, $f_{\theta_{W_i}}$ takes the decomposed $u_i$ and $\delta_{i+1}$ as its input and predicts $\tilde{u}_i$ and $\tilde{\delta}_{i+1}$. The update is then reconstructed as $\tilde{\nabla}_{W_i}=\tilde{\delta}_{i+1}\tilde{u}_i^T$. This approach reduces the dimension from $d^2$ to $2d$.

\begin{figure}[t!]
    \newdimen\base
    \base=0.5cm
    \newdimen\xsep
    \xsep=0.3cm

    \tikzstyle{common} = [font=\scriptsize]
    \tikzstyle{split2} = [rectangle split, rectangle split parts=2, draw=black, inner sep=0pt, outer sep=0pt, minimum width=\base, minimum height=\base, inner ysep=0.5\base, rounded corners=2pt]
    
    \centering
    \begin{tikzpicture}
    
    \node[common, inner sep=0pt] (input) at (0,0) {$\nabla_{W_i}$};
    \node[common, split2, anchor=west, fill=lyyblue!40!white] (decomposein) at ([xshift=\xsep]input.east) {\rotatebox{-90}{$u_i$} \nodepart{two} \rotatebox{-90}{$\delta_{i+1}$}};
    \node[common, circle, draw=black, inner sep=1pt, anchor=west, fill=orange!30!white] (norm) at ([xshift=\xsep]decomposein.east) {norm};
    \node[common, split2, anchor=west, fill=lyyblue!40!white] (normout) at ([xshift=\xsep]norm.east) {\rotatebox{-90}{$\bar{u}_i$} \nodepart{two} \rotatebox{-90}{$\bar{\delta}_{i+1}$}};
    \node[common, trapezium, trapezium angle=30, rotate=-90, draw=black, anchor=south, fill=ugreen!30!white, minimum height=\base, minimum width=3\base, inner sep=0pt, outer sep=0pt, trapezium stretches, text width=2.5\base, align=center] (down) at ([xshift=\xsep]normout.east) {Down};
    \node[common, rectangle, draw=black, anchor=west, fill=red!30!white, minimum height=2.5\base] (hidden) at ([xshift=\xsep]down.north) {\rotatebox{-90}{Dropout}};
    \node[common, trapezium, trapezium angle=30, rotate=90, draw=black, anchor=north, fill=ugreen!30!white, minimum height=\base, minimum width=3\base, inner sep=0pt, outer sep=0pt, trapezium stretches, text width=2.5\base, align=center] (up) at ([xshift=\xsep]hidden.east) {Up};
    \node[common, anchor=west, inner sep=0pt] (residual) at ([xshift=0.7\xsep]up.south) {$\bigoplus$};
    \node[common, split2, anchor=west, fill=lyyblue!40!white] (decomposeout) at ([xshift=0.7\xsep]residual.east) {\rotatebox{-90}{$\tilde{u}_i$} \nodepart{two} \rotatebox{-90}{$\tilde{\delta}_{i+1}$}};
    \node[common, inner sep=0pt, anchor=west] (out) at ([xshift=\xsep]decomposeout.east) {$\tilde{\nabla}_{W_i}$};

    \draw[-latex] (input) to (decomposein);
    \draw[-latex] (decomposein) to (norm);
    \draw[-latex] (norm) to (normout);
    \draw[-latex] (normout) to (down);
    \draw[-latex] (down) to (hidden);
    \draw[-latex] (hidden) to (up);
    \draw[-latex] (up) to (residual);
    \draw[-latex] (residual) to (decomposeout);
    \draw[-latex] (decomposeout) to (out);

    \draw[-latex] (normout.north) |- ($(normout.north) + (0,0.5\base)$) -| (residual.north);
    
    \end{tikzpicture}
    \caption{The model architecture of \method{}.}
    \label{fig:arch}
\end{figure}

\paragraph{Model Architecture.}

The architecture of $f_{\theta_{W_i}}\left(u_i, \delta_{i+1}\right)$, named \method{}, is shown in Figure~\ref{fig:arch} and defined as follows:
\begin{align}
    \left[\bar{u}_i, \bar{\delta}_{i+1}\right]&=\mathrm{Norm}\left(\left[u_i, \delta_{i+1}\right]\right)\\
    h_i&=\theta_2\mathrm{Dropout}\left(\theta_1\left[\bar{u}_i, \bar{\delta}_{i+1}\right]\right)\\
    \left[\tilde{u}_i, \tilde{\delta}_{i+1}\right]&=h_i + \left[\bar{u}_i, \bar{\delta}_{i+1}\right]
\end{align}
where $\theta_1 \in \mathbb{R}^{r \times 2d}$ and $\theta_2 \in \mathbb{R}^{2d \times r}$ are the optimizer parameters with $r \ll d$. $[\cdot]$ denotes the vector concatenation. $\mathrm{Norm}$ normalizes $u_i$ and $\delta_{i+1}$ to have zero mean and unit variance separately. $\mathrm{Dropout}$ is the dropout regularization~\cite{DBLP:journals/jmlr/SrivastavaHKSS14}. In practice, $\theta_1$ and $\theta_2$ are shared across all weights with the same shape. \method{} is similar to the Bottleneck Adapter~\cite{DBLP:conf/icml/HoulsbyGJMLGAG19}, with the key difference that its input is gradients and its output is the weight update. As such, \method{} can also be regarded as a specialized PEFT technique tailored to reasoning.

% \textcolor{red}{a two-column workflow diagram}

\begin{table*}
\centering
% \footnotesize
\resizebox{\linewidth}{!}{
% \setlength{\tabcolsep}{5pt}
\begin{tabular}[t]{l|llll|c}
\toprule
 \makecell[c]{\textbf{Method}} & \makecell[c]{\textbf{MATH}} & \makecell[c]{\textbf{GSM8K}} & \makecell[c]{\textbf{MBPP}} & \makecell[c]{\textbf{HumanEval}} & \makecell[c]{\textbf{Avg.}} \\
\midrule
\multicolumn{6}{c}{\texttt{Llama-3.1-8B-Instruct}} \\
\midrule
Revision~\cite{DBLP:journals/corr/abs-2408-03314} & 0.2960 & 0.4570 & 0.2991 & 0.3889 & 0.3603 \\
Beam Search~\cite{ow1988filtered} & 0.2560 & 0.4842 & 0.1869 & 0.2407 & 0.2920 \\
% Beam Search Sampling~\cite{DBLP:journals/corr/abs-1910-03771} & 0.2520$_{0.0000}$ & 0.4842$_{0.0000}$ & 0.1822$_{0.0000}$ & 0.2407$_{0.0000}$ & 0.2898 \\
Self-Consistency~\cite{DBLP:conf/iclr/0002WSLCNCZ23} & 0.3000$_{0.0057}$ & 0.4525$_{0.0111}$ & 0.1371$_{0.0180}$ & 0.1235$_{0.0531}$ & 0.2533 \\
Self-Refine~\cite{DBLP:conf/nips/MadaanTGHGW0DPY23} & 0.4693$_{0.0207}$ & 0.7828$_{0.0064}$ & 0.2305$_{0.0088}$ & 0.2963$_{0.0800}$ & 0.4447 \\
Best-of-N~\cite{DBLP:journals/corr/abs-2407-21787} & 0.6427$_{0.0154}$ & \underline{0.8069}$_{0.0043}$ & \underline{0.5452}$_{0.0154}$ & 0.6728$_{0.0087}$ & 0.6669 \\
\midrule
FTTT & \underline{0.6707}$_{0.0222}$ & \textbf{0.8100}$_{0.0037}$ & \textbf{0.5607}$_{0.0212}$ & \textbf{0.6852}$_{0.0302}$ & 0.6817 \\
+ w/o Self-Reflected Feedback & \textbf{0.6720}$_{0.0113}$ & 0.8054$_{0.0133}$ & 0.5405$_{0.0117}$ & \underline{0.6790}$_{0.0087}$ & 0.6742 \\
\toprule
\multicolumn{6}{c}{\texttt{Mistral-7B-Instruct-v0.3}} \\
\midrule
Revision~\cite{DBLP:journals/corr/abs-2408-03314} & 0.0497 & 0.1686 & 0.1351 & 0.1000 & 0.1134 \\
Beam Search~\cite{ow1988filtered} & 0.1783 & 0.4537 & 0.1318 & 0.1600 & 0.2310 \\
% Beam Search Sampling~\cite{DBLP:journals/corr/abs-1910-03771} & 0.1716$_{0.0032}$ & 0.4533$_{0.0016}$ & 0.1318$_{0.0000}$ & 0.1500$_{0.0000}$ & 0.2267 \\
Self-Consistency~\cite{DBLP:conf/iclr/0002WSLCNCZ23} & 0.1618$_{0.0011}$ & 0.3549$_{0.0052}$ & 0.1243$_{0.0135}$ & 0.1033$_{0.0236}$ & 0.1861 \\
Self-Refine~\cite{DBLP:conf/nips/MadaanTGHGW0DPY23} & 0.1287$_{0.0139}$ & 0.3312$_{0.0078}$ & 0.2347$_{0.0091}$ & 0.3533$_{0.0309}$ & 0.2620 \\
Best-of-N~\cite{DBLP:journals/corr/abs-2407-21787} & 0.4688$_{0.0138}$ & 0.7807$_{0.0055}$ & \textbf{0.4962}$_{0.0171}$ & 0.6500$_{0.0141}$ & 0.5989 \\
\midrule
FTTT & \underline{0.4733}$_{0.0087}$ & \underline{0.7820}$_{0.0045}$ & \textbf{0.4962}$_{0.0015}$ & \underline{0.6633}$_{0.0125}$ & 0.6037 \\
+ w/o Self-Reflected Feedback & \textbf{0.4876}$_{0.0133}$ & \textbf{0.7858}$_{0.0021}$ & \underline{0.4941}$_{0.0040}$ & \textbf{0.6833}$_{0.0205}$ & 0.6127 \\
\bottomrule
\end{tabular}
}
% \vspace{-5pt}
\caption{Experimental results on four datasets with a budget of 32. For stochastic algorithms, we report the mean of three runs with different random seeds and standard deviation in the subscript. \textbf{Bold} entries are the best results, and \underline{underlined} entries are the second-best results.}
\label{tab:pilot}
% \vspace{-0.5cm}
\end{table*}

\section{Experiments}
\label{sec:exp}

\subsection{Setup}

\paragraph{Datasets.}

We evaluate both baselines and our method on math and coding reasoning tasks:
(a) Mathematical reasoning: MATH~\cite{DBLP:conf/nips/HendrycksBKABTS21} and GSM8K~\cite{DBLP:journals/corr/abs-2110-14168}, using the test split from~\citet{DBLP:conf/iclr/LightmanKBEBLLS24} for MATH.
(b) Code generation: MBPP~\cite{DBLP:journals/corr/abs-2108-07732} and HumanEval~\cite{DBLP:journals/corr/abs-2107-03374}.
For all datasets, we report results on subsets where models fail with greedy decoding. We use Exact Match as the evaluation metric as well as the verifier for math tasks and Pass@1 for code.

\paragraph{Models.}

We conduct experiments with \texttt{Llama-3.1-8B-Instruct}~\cite{DBLP:journals/corr/abs-2407-21783} and \texttt{Mistral-7B-Instruct-v0.3}~\cite{DBLP:journals/corr/abs-2310-06825}.
We evaluate both models with zero-shot prompting and follow the official instructions when evaluating the model on each dataset\footnote{\url{https://huggingface.co/datasets/meta-llama/Llama-3.1-8B-Instruct-evals}}.

\paragraph{Baselines.}

We compare FTTT with the following test-time scaling methods:
\begin{itemize}[noitemsep, nolistsep]
    \item \textbf{Beam Search}~\cite{ow1988filtered} is a popular search algorithm that approximates the most confident model prediction.
    % \item \textbf{Self-Evaluation Guided Beam Search}~\cite{DBLP:conf/nips/XieKZZKHX23}
    % \item \textbf{Stochastic Beam Search}~\cite{DBLP:conf/icml/KoolHW19}
    % \item \textbf{Beam Search Sampling}~\cite{DBLP:journals/corr/abs-1910-03771} is a randomized version of beam search. Instead of keeping the top-$B$ candidates deterministically, it samples them stochastically to avoid similar predictions, where $B$ is the beam size. The best prediction is finally selected based on feedback.
    \item \textbf{Self-Consistency}~\cite{DBLP:conf/iclr/0002WSLCNCZ23} samples multiple predictions and selects the most frequent answer.
    \item \textbf{Best-of-N}~\cite{DBLP:journals/corr/abs-2407-21787} samples $N$ predictions independently and picks the best one based on external feedback.
    \item \textbf{Revision}~\cite{DBLP:journals/corr/abs-2408-03314} iteratively refines answers by conditioning the model on previous attempts.
    \item \textbf{Self-Refine}~\cite{DBLP:conf/nips/MadaanTGHGW0DPY23} alternates between self-critique and refinement. We select the best solution based on feedback.
\end{itemize}

\pgfplotstableread[col sep=space,string type]{
id num mean1 std1 mean2 std2 mean3 std3 mean4 std4 mean5 std5
0 2 0.14 0.0 0.164 0.011775681155103792 0.04 0.009797958971132711 0.156 0.014966629547095756 0.25866666666666666 0.015084944665313026
1 4 0.256 0.0 0.19200000000000003 0.024657656011875903 0.11466666666666665 0.006798692684790379 0.29066666666666663 0.009428090415820642 0.3746666666666667 0.004988876515698593
2 8 0.28 0.0 0.23066666666666666 0.02174600857373346 0.2293333333333333 0.013597385369580758 0.428 0.02262741699796954 0.48 0.008640987597877129
3 16 0.292 0.0 0.2733333333333334 0.004988876515698593 0.35200000000000004 0.021416504538945343 0.532 0.019866219234335136 0.6026666666666666 0.008219218670625309
4 32 0.296 0.0 0.3 0.005656854249492386 0.4653333333333333 0.020997354330698156 0.6386666666666666 0.015434449203720314 0.6707 0.0222
}{\llamamath}

\pgfplotstableread[col sep=space,string type]{
id num mean1 std1 mean2 std2 mean3 std3 mean4 std4 mean5 std5
0 2 0.33031674208144796 0.0 0.334841628959276 0.01954974569655463 0.07692307692307693 0.006399156390828481 0.3197586726998492 0.023752663270020555 0.4962292609351433 0.005643525470247277
1 4 0.4434389140271493 0.0 0.38009049773755654 0.016930576410741804 0.2730015082956259 0.026211383404197246 0.5113122171945701 0.012798312781656976 0.6003016591251885 0.007690828828948417
2 8 0.45701357466063347 0.0 0.43288084464555054 0.01297484957321663 0.4841628959276018 0.037495634674678896 0.645550527903469 0.007690828828948402 0.6787330316742081 0.016930576410741797
3 16 0.45701357466063347 0.0 0.4419306184012066 0.007690828828948391 0.669683257918552 0.0230724864868452 0.7345399698340875 0.018958982036163686 0.7601809954751131 0.009774872848277315
4 32 0.45701357466063347 0.0 0.45248868778280543 0.011083663994494026 0.779788838612368 0.005643525470247252 0.8054298642533938 0.006399156390828514 0.8099547511312218 0.0036945546648313263
}{\llamagsm}

\pgfplotstableread[col sep=space,string type]{
id num mean1 std1 mean2 std2 mean3 std3 mean4 std4 mean5 std5
0 2 0.1308411214953271 0.0 0.14174454828660435 0.01803089860247698 0.040498442367601244 0.005828126770675922 0.14174454828660435 0.01803089860247698 0.21495327102803738 0.0066084745905284825
1 4 0.24299065420560748 0.0 0.14174454828660435 0.01803089860247698 0.09190031152647975 0.009601890970356661 0.2554517133956386 0.009601890970356651 0.30062305295950154 0.014444888622267452
2 8 0.2850467289719626 0.0 0.14174454828660435 0.01803089860247698 0.1308411214953271 0.01375663686343901 0.3707165109034268 0.014444888622267452 0.3862928348909658 0.009601890970356661
3 16 0.29906542056074764 0.0 0.14330218068535824 0.01720461217630414 0.19003115264797507 0.007942397996250445 0.4657320872274144 0.01957913565416906 0.48286604361370716 0.0223562306766469
4 32 0.29906542056074764 0.0 0.13707165109034267 0.01803089860247698 0.22897196261682243 0.010094611679763024 0.5420560747663551 0.01375663686343899 0.5607476635514018 0.02124327367131751
}{\llamambpp}

\pgfplotstableread[col sep=space,string type]{
id num mean1 std1 mean2 std2 mean3 std3 mean4 std4 mean5 std5
0 2 0.16666666666666666 0.0 0.14814814814814814 0.0604812282168686 0.043209876543209874 0.017459426695964134 0.14814814814814814 0.0604812282168686 0.24691358024691357 0.03147542909625176
1 4 0.35185185185185186 0.0 0.14814814814814814 0.0604812282168686 0.1111111111111111 0.01512030705421715 0.3518518518518518 0.02618914004394619 0.3703703703703704 0.01512030705421715
2 8 0.3888888888888889 0.0 0.14814814814814814 0.0604812282168686 0.1728395061728395 0.031475429096251756 0.4691358024691357 0.017459426695964137 0.49382716049382713 0.04619330107128325
3 16 0.3888888888888889 0.0 0.14814814814814814 0.0604812282168686 0.24691358024691357 0.057244558614171014 0.5802469135802469 0.008729713347982055 0.5987654320987654 0.03805193828993194
4 32 0.3888888888888889 0.0 0.12345679012345678 0.05310077325334955 0.2962962962962963 0.08000914442478839 0.6666666666666666 0.01512030705421717 0.6851851851851851 0.0302406141084343
}{\llamahumaneval}

\pgfplotstableread[col sep=space,string type]{
id num mean1 std1 mean2 std2 mean3 std3 mean4 std4 mean5 std5
0 2 0.0 0.0 0.06922498118886382 0.011109723897843043 0.05417607223476298 0.013915155762909656 0.06696764484574869 0.009276770508606438 0.11361926260346127 0.004256474228361455
1 4 0.11254019292604502 0.0 0.08653122648607976 0.011109723897843041 0.07072987208427389 0.011109723897843043 0.15951843491346876 0.013833541242174742 0.18434913468773514 0.005924761379993844
2 8 0.13504823151125403 0.0 0.109104589917231 0.0069778920206890185 0.08728367193378479 0.007673468041524134 0.25959367945823925 0.012901751843101775 0.27915726109857036 0.015675445188863543
3 16 0.13504823151125403 0.0 0.1361926260346125 0.009091832937241967 0.10609480812641083 0.008446179202649982 0.3829947328818661 0.008512948456722924 0.380737396538751 0.014783207452512045
4 32 0.13504823151125403 0.0 0.1617757712565839 0.001064118557090362 0.12641083521444693 0.012086063509562834 0.46501128668171554 0.01026198773287123 0.4740406320541761 0.008033918925531462
}{\mistralmath}

\pgfplotstableread[col sep=space,string type]{
id num mean1 std1 mean2 std2 mean3 std3 mean4 std4 mean5 std5
0 2 0.0 0.0 0.18166455428812844 0.006890370270511227 0.13181242078580482 0.0017924126265818718 0.16603295310519647 0.004742278056747701 0.3565694972539079 0.01285604060388926
1 4 0.15716096324461343 0.0 0.2336290663286861 0.0043084237546200084 0.18250950570342206 0.006209099474735556 0.3967046894803549 0.005377237879745629 0.4879594423320659 0.011753635608993267
2 8 0.1634980988593156 0.0 0.2995352767215885 0.004308423754620012 0.22475707646810306 0.009041374972130423 0.5792141951837769 0.010192089633979539 0.6117448246725813 0.012856040603889275
3 16 0.16603295310519645 0.0 0.33586818757921416 0.0027379555126353537 0.2691170257710182 0.0041822961286910295 0.694972539079003 0.0041822961286910295 0.7165188001689903 0.0033265770485897185
4 32 0.16856780735107732 0.0 0.3548795944233207 0.005174249562279624 0.3269961977186312 0.005174249562279624 0.7777777777777778 0.0047797670042182905 0.7828474862695396 0.0048905098871103924
}{\mistralgsm}

\pgfplotstableread[col sep=space,string type]{
id num mean1 std1 mean2 std2 mean3 std3 mean4 std4 mean5 std5
0 2 0.0 0.0 0.11361200428724545 0.012407113507813749 0.09860664523043944 0.004010350896863818 0.11361200428724545 0.012407113507813749 0.1747052518756699 0.009939569662910722
1 4 0.11254019292604502 0.0 0.11146838156484458 0.010930374091302864 0.12754555198285103 0.0015157701633152123 0.22186495176848875 0.005250781870917845 0.24544480171489816 0.028919051582490973
2 8 0.13504823151125403 0.0 0.11789924973204717 0.015383387025088234 0.1639871382636656 0.004547310489945637 0.31939978563772775 0.012950745952405758 0.34941050375133975 0.020051754484319093
3 16 0.13504823151125403 0.0 0.12433011789924973 0.01183854342678163 0.195069667738478 0.007578850816576075 0.4030010718113612 0.017870666667238022 0.4308681672025723 0.01837773654821248
4 32 0.13504823151125403 0.0 0.12433011789924973 0.01347245990351183 0.23365487674169347 0.009939569662910722 0.4919614147909968 0.018377736548212457 0.4962486602357985 0.0015157701633152255
}{\mistralmbpp}

\pgfplotstableread[col sep=space,string type]{
id num mean1 std1 mean2 std2 mean3 std3 mean4 std4 mean5 std5
0 2 0.0 0.0 0.13 0.035590260840104374 0.13333333333333333 0.020548046676563257 0.13 0.035590260840104374 0.18666666666666668 0.01699673171197594
1 4 0.1 0.0 0.12 0.03741657386773942 0.19666666666666666 0.0262466929133727 0.2933333333333334 0.04642796092394707 0.31666666666666665 0.009428090415820642
2 8 0.1 0.0 0.12 0.03741657386773942 0.24333333333333332 0.02867441755680877 0.41 0.029439202887759492 0.43333333333333335 0.012472191289246483
3 16 0.1 0.0 0.10333333333333333 0.02357022603955158 0.30666666666666664 0.016996731711975962 0.5366666666666667 0.026246692913372727 0.5533333333333333 0.009428090415820642
4 32 0.1 0.0 0.10333333333333333 0.02357022603955158 0.3466666666666667 0.0262466929133727 0.64 0.016329931618554536 0.6633333333333334 0.012472191289246483
}{\mistralhumaneval}

\begin{figure*}[t!]
    \centering
    \makeatletter
    % temporally disable hyperref
    \let\ref\@refstar
    \ref{grouplegend}
    \makeatother
    % \vspace{-0.2in}
    \begin{tikzpicture}
        \begin{groupplot}[
            group style={group size=4 by 2, horizontal sep=45pt, vertical sep=15pt},
            width=1.0\textwidth,
            height=0.3\textwidth,
            legend cell align={left},
            legend pos=north west,
            enlargelimits=0.1,
            legend style={
                font=\small,
                draw=none,
                column sep=5pt,
                legend columns=5,
            },
        ]
        \nextgroupplot[
            width=0.25\textwidth,height=0.3\textwidth,
            yticklabel style={/pgf/number format/fixed,/pgf/number format/precision=1},
            xticklabels from table={\llamamath}{num},
            ylabel={MATH},
            ylabel near ticks,
            % xlabel={Budget},
            xlabel near ticks,
            xmajorgrids=true,
            ymajorgrids=true,
            legend pos=south west,
            grid style=dashed,
            xtick={0,...,4},
            every tick label/.append style={font=\small},
            label style={font=\small},
            ylabel style={yshift=0pt},
            legend to name=grouplegend,
            xmin=0,xmax=4,
            % ymax=60,ymin=0,
        ]
            \foreach \c/\marker/\name [count=\n] in {purple/{otimes}/{Revision}, orange/{star}/{Self-Consistency}, lyygreen/{diamond*}/{Self-Refine}, lyyred/{*}/{Best-of-N}, lyyblue/{square*}/{FTTT}}{
                \addplot [name path=lower, fill=none, draw=none, forget plot] table [
                    x=id, y expr=\thisrow{mean\n} - \thisrow{std\n}] {\llamamath};
                \addplot [name path=upper, fill=none, draw=none, forget plot] table [
                    x=id, y expr=\thisrow{mean\n} + \thisrow{std\n}] {\llamamath};
                \expandafter\addplot\expandafter [\c!60, semitransparent, forget plot] fill between[of=lower and upper];
                \edef\tempoptions{\c,thick,mark=\marker,mark options={scale=0.7}}
                \expandafter\addplot\expandafter [\tempoptions] table [
                    x=id, y=mean\n,
                ] {\llamamath};
                \expandafter\addlegendentry\expandafter{\name}
                % \pgfplotstablegetelem{\n}{name}\of{\modeldata}
                % \addlegendentry{\pgfplotsretval}
            }
            % https://tex.stackexchange.com/questions/705041/newcommand-and-macro-expansion-in-pgfplots
        \nextgroupplot[
            width=0.25\textwidth,height=0.3\textwidth,
            yticklabel style={/pgf/number format/fixed,/pgf/number format/precision=1},
            xticklabels from table={\llamagsm}{num},
            ylabel={GSM8K},
            ylabel near ticks,
            % xlabel={Budget},
            xlabel near ticks,
            xmajorgrids=true,
            ymajorgrids=true,
            grid style=dashed,
            xtick={0,...,4},
            every tick label/.append style={font=\small},
            label style={font=\small},
            ylabel style={yshift=0pt},
            xmin=0,xmax=4,
            % ymax=60,ymin=0,
        ]
            \foreach \c/\marker/\name [count=\n] in {purple/{otimes}/{Revision}, orange/{star}/{Stochastic Beam Search}, lyygreen/{diamond*}/{Self-Refine}, lyyred/{*}/{Best-of-N}, lyyblue/{square*}/{TTT}}{
                \addplot [name path=lower, fill=none, draw=none, forget plot] table [
                    x=id, y expr=\thisrow{mean\n} - \thisrow{std\n}] {\llamagsm};
                \addplot [name path=upper, fill=none, draw=none, forget plot] table [
                    x=id, y expr=\thisrow{mean\n} + \thisrow{std\n}] {\llamagsm};
                \expandafter\addplot\expandafter [\c!60, semitransparent, forget plot] fill between[of=lower and upper];
                \edef\tempoptions{\c,thick,mark=\marker,mark options={scale=0.7}}
                \expandafter\addplot\expandafter [\tempoptions] table [
                    x=id, y=mean\n,
                ] {\llamagsm};
            }
        \nextgroupplot[
            width=0.25\textwidth,height=0.3\textwidth,
            yticklabel style={/pgf/number format/fixed,/pgf/number format/precision=1},
            xticklabels from table={\llamambpp}{num},
            ylabel={MBPP},
            ylabel near ticks,
            % xlabel={Budget},
            xlabel near ticks,
            xmajorgrids=true,
            ymajorgrids=true,
            grid style=dashed,
            xtick={0,...,4},
            every tick label/.append style={font=\small},
            label style={font=\small},
            ylabel style={yshift=0pt},
            xmin=0,xmax=4,
            % ymax=60,ymin=0,
        ]
            \foreach \c/\marker/\name [count=\n] in {purple/{otimes}/{Revision}, orange/{star}/{Stochastic Beam Search}, lyygreen/{diamond*}/{Self-Refine}, lyyred/{*}/{Best-of-N}, lyyblue/{square*}/{TTT}}{
                \addplot [name path=lower, fill=none, draw=none, forget plot] table [
                    x=id, y expr=\thisrow{mean\n} - \thisrow{std\n}] {\llamambpp};
                \addplot [name path=upper, fill=none, draw=none, forget plot] table [
                    x=id, y expr=\thisrow{mean\n} + \thisrow{std\n}] {\llamambpp};
                \expandafter\addplot\expandafter [\c!60, semitransparent, forget plot] fill between[of=lower and upper];
                \edef\tempoptions{\c,thick,mark=\marker,mark options={scale=0.7}}
                \expandafter\addplot\expandafter [\tempoptions] table [
                    x=id, y=mean\n,
                ] {\llamambpp};
            }
        \nextgroupplot[
            width=0.25\textwidth,height=0.3\textwidth,
            yticklabel style={/pgf/number format/fixed,/pgf/number format/precision=1},
            xticklabels from table={\llamahumaneval}{num},
            ylabel={HumanEval},
            ylabel near ticks,
            % xlabel={Budget},
            xlabel near ticks,
            xmajorgrids=true,
            ymajorgrids=true,
            grid style=dashed,
            xtick={0,...,4},
            every tick label/.append style={font=\small},
            label style={font=\small},
            ylabel style={yshift=0pt},
            xmin=0,xmax=4,
            % ymax=60,ymin=0,
        ]
            \foreach \c/\marker/\name [count=\n] in {purple/{otimes}/{Revision}, orange/{star}/{Stochastic Beam Search}, lyygreen/{diamond*}/{Self-Refine}, lyyred/{*}/{Best-of-N}, lyyblue/{square*}/{TTT}}{
                \addplot [name path=lower, fill=none, draw=none, forget plot] table [
                    x=id, y expr=\thisrow{mean\n} - \thisrow{std\n}] {\llamahumaneval};
                \addplot [name path=upper, fill=none, draw=none, forget plot] table [
                    x=id, y expr=\thisrow{mean\n} + \thisrow{std\n}] {\llamahumaneval};
                \expandafter\addplot\expandafter [\c!60, semitransparent, forget plot] fill between[of=lower and upper];
                \edef\tempoptions{\c,thick,mark=\marker,mark options={scale=0.7}}
                \expandafter\addplot\expandafter [\tempoptions] table [
                    x=id, y=mean\n,
                ] {\llamahumaneval};
            }
            
        \nextgroupplot[
            width=0.25\textwidth,height=0.3\textwidth,
            yticklabel style={/pgf/number format/fixed,/pgf/number format/precision=1},
            xticklabels from table={\mistralmath}{num},
            ylabel={MATH},
            ylabel near ticks,
            xlabel={Budget},
            xlabel near ticks,
            xmajorgrids=true,
            ymajorgrids=true,
            legend pos=south west,
            grid style=dashed,
            xtick={0,...,4},
            every tick label/.append style={font=\small},
            label style={font=\small},
            ylabel style={yshift=0pt},
            xmin=0,xmax=4,
            % ymax=60,ymin=0,
        ]
            \foreach \c/\marker/\name [count=\n] in {purple/{otimes}/{Revision}, orange/{star}/{Self-Consistency}, lyygreen/{diamond*}/{Self-Refine}, lyyred/{*}/{Best-of-N}, lyyblue/{square*}/{FTTT}}{
                \addplot [name path=lower, fill=none, draw=none, forget plot] table [
                    x=id, y expr=\thisrow{mean\n} - \thisrow{std\n}] {\mistralmath};
                \addplot [name path=upper, fill=none, draw=none, forget plot] table [
                    x=id, y expr=\thisrow{mean\n} + \thisrow{std\n}] {\mistralmath};
                \expandafter\addplot\expandafter [\c!60, semitransparent, forget plot] fill between[of=lower and upper];
                \edef\tempoptions{\c,thick,mark=\marker,mark options={scale=0.7}}
                \expandafter\addplot\expandafter [\tempoptions] table [
                    x=id, y=mean\n,
                ] {\mistralmath};
            }
        \nextgroupplot[
            width=0.25\textwidth,height=0.3\textwidth,
            yticklabel style={/pgf/number format/fixed,/pgf/number format/precision=1},
            xticklabels from table={\mistralgsm}{num},
            ylabel={GSM8K},
            ylabel near ticks,
            xlabel={Budget},
            xlabel near ticks,
            xmajorgrids=true,
            ymajorgrids=true,
            grid style=dashed,
            xtick={0,...,4},
            every tick label/.append style={font=\small},
            label style={font=\small},
            ylabel style={yshift=0pt},
            xmin=0,xmax=4,
            % ymax=60,ymin=0,
        ]
            \foreach \c/\marker/\name [count=\n] in {purple/{otimes}/{Revision}, orange/{star}/{Stochastic Beam Search}, lyygreen/{diamond*}/{Self-Refine}, lyyred/{*}/{Best-of-N}, lyyblue/{square*}/{FTTT}}{
                \addplot [name path=lower, fill=none, draw=none, forget plot] table [
                    x=id, y expr=\thisrow{mean\n} - \thisrow{std\n}] {\mistralgsm};
                \addplot [name path=upper, fill=none, draw=none, forget plot] table [
                    x=id, y expr=\thisrow{mean\n} + \thisrow{std\n}] {\mistralgsm};
                \expandafter\addplot\expandafter [\c!60, semitransparent, forget plot] fill between[of=lower and upper];
                \edef\tempoptions{\c,thick,mark=\marker,mark options={scale=0.7}}
                \expandafter\addplot\expandafter [\tempoptions] table [
                    x=id, y=mean\n,
                ] {\mistralgsm};
            }
        \nextgroupplot[
            width=0.25\textwidth,height=0.3\textwidth,
            yticklabel style={/pgf/number format/fixed,/pgf/number format/precision=1},
            xticklabels from table={\mistralmbpp}{num},
            ylabel={MBPP},
            ylabel near ticks,
            xlabel={Budget},
            xlabel near ticks,
            xmajorgrids=true,
            ymajorgrids=true,
            grid style=dashed,
            xtick={0,...,4},
            every tick label/.append style={font=\small},
            label style={font=\small},
            ylabel style={yshift=0pt},
            xmin=0,xmax=4,
            % ymax=60,ymin=0,
        ]
            \foreach \c/\marker/\name [count=\n] in {purple/{otimes}/{Revision}, orange/{star}/{Stochastic Beam Search}, lyygreen/{diamond*}/{Self-Refine}, lyyred/{*}/{Best-of-N}, lyyblue/{square*}/{FTTT}}{
                \addplot [name path=lower, fill=none, draw=none, forget plot] table [
                    x=id, y expr=\thisrow{mean\n} - \thisrow{std\n}] {\mistralmbpp};
                \addplot [name path=upper, fill=none, draw=none, forget plot] table [
                    x=id, y expr=\thisrow{mean\n} + \thisrow{std\n}] {\mistralmbpp};
                \expandafter\addplot\expandafter [\c!60, semitransparent, forget plot] fill between[of=lower and upper];
                \edef\tempoptions{\c,thick,mark=\marker,mark options={scale=0.7}}
                \expandafter\addplot\expandafter [\tempoptions] table [
                    x=id, y=mean\n,
                ] {\mistralmbpp};
            }
        \nextgroupplot[
            width=0.25\textwidth,height=0.3\textwidth,
            yticklabel style={/pgf/number format/fixed,/pgf/number format/precision=1},
            xticklabels from table={\mistralhumaneval}{num},
            ylabel={HumanEval},
            ylabel near ticks,
            xlabel={Budget},
            xlabel near ticks,
            xmajorgrids=true,
            ymajorgrids=true,
            grid style=dashed,
            xtick={0,...,4},
            every tick label/.append style={font=\small},
            label style={font=\small},
            ylabel style={yshift=0pt},
            xmin=0,xmax=4,
            % ymax=60,ymin=0,
        ]
            \foreach \c/\marker/\name [count=\n] in {purple/{otimes}/{Revision}, orange/{star}/{Stochastic Beam Search}, lyygreen/{diamond*}/{Self-Refine}, lyyred/{*}/{Best-of-N}, lyyblue/{square*}/{FTTT}}{
                \addplot [name path=lower, fill=none, draw=none, forget plot] table [
                    x=id, y expr=\thisrow{mean\n} - \thisrow{std\n}] {\mistralhumaneval};
                \addplot [name path=upper, fill=none, draw=none, forget plot] table [
                    x=id, y expr=\thisrow{mean\n} + \thisrow{std\n}] {\mistralhumaneval};
                \expandafter\addplot\expandafter [\c!60, semitransparent, forget plot] fill between[of=lower and upper];
                \edef\tempoptions{\c,thick,mark=\marker,mark options={scale=0.7}}
                \expandafter\addplot\expandafter [\tempoptions] table [
                    x=id, y=mean\n,
                ] {\mistralhumaneval};
            }
        \end{groupplot}
    \end{tikzpicture}
    \caption{The scaling trends of different methods under varying budgets. The colored area around the line denotes the standard deviation. The first row is the results of \texttt{Llama-3.1-8B-Instruct} and the second row is \texttt{Mistral-7B-Instruct-v0.3}.}
    \label{fig:scale}
    % \vspace{-0.4cm}
\end{figure*}

For sampling-based methods, we use nucleus sampling~\cite{DBLP:conf/iclr/HoltzmanBDFC20} with a temperature of 0.6 and $p=0.95$, following~\citet{DBLP:journals/corr/abs-2407-21787}. All methods are allocated a budget of 32. For FTTT, we fine-tune the model with LoRA~\cite{DBLP:conf/iclr/HuSWALWWC22}, using a rank of 4 and a dropout ratio of 0.05. We use the Adam optimizer~\cite{DBLP:journals/corr/KingmaB14} with a learning rate of 1e-5, except for \texttt{Mistral-7B-Instruct-v0.3} in coding tasks, where we use 2e-5.

As \method{} is a specialized PEFT method, we compare it with the following PEFT approaches: \textbf{Adapter}~\cite{DBLP:conf/icml/HoulsbyGJMLGAG19}, \textbf{(IA)$^3$}~\cite{DBLP:conf/nips/LiuTMMHBR22}, \textbf{LoRA}~\cite{DBLP:conf/iclr/HuSWALWWC22} and \textbf{LN-Tuning}~\cite{DBLP:conf/iclr/ZhaoT0MX24}. We also include \textbf{full fine-tuning} that updates all LLM weights. For \method{}, $r=16$ and the dropout ratio is 0.1. We only apply \method{} to the query and value projections in the last two layers of the LLM. Detailed configurations are in Appendix~\ref{app:setting}.

% BitFit~\cite{ben-zaken-etal-2022-bitfit}

% ReFT~\cite{DBLP:journals/corr/abs-2404-03592}

% \textbf{Prefix-Tuning}~\cite{li-liang-2021-prefix}

\subsection{Training-Free Results}
\label{sec:result1}

Table~\ref{tab:pilot} compares FTTT with various baselines across four reasoning datasets. FTTT, both with and without self-reflected feedback, outperforms conventional test-time scaling methods on average. This success is partially explained by the findings of \citet{ye2024physicslanguagemodels22}, which show that training with error-correction data enhances reasoning capabilities and models do not retry during inference. FTTT is also efficient. For instance, the inference time of \texttt{Llama-3.1-8B-Instruct} on GSM8K with a budget of 32 is 3 GPU hours for the best parallel sampling method (Best-of-N) and 20 GPU hours for the best sequential revision method (Self-Refine). In contrast, FTTT achieves inference times of approximately 3 GPU hours without self-reflected feedback and 4 GPU hours with self-reflected feedback.

% the inference time of \texttt{Llama-3.1-8B-Instruct} with FTTT on GSM8K is 46 minutes on 4 NVIDIA H800, while for Best-of-N and Self-Refine, their inference time is 47 minutes and 5 hours

Notably, self-reflected feedback does not always improve results. Its effectiveness appears to depend on the LLM's self-reflection ability. To test this, we computed the Spearman rank correlation between FTTT and Self-Refine, a self-reflection-based algorithm. The Spearman coefficient ($r = 0.8656$, $p \leq 0.05$) indicates a strong positive correlation, supporting our hypothesis. We also observe that Self-Consistency performs poorly on code tasks because sampled code snippets rarely match exactly, making majority voting akin to random selection.

Figure~\ref{fig:scale} illustrates performance for FTTT and baselines under varying budgets. FTTT consistently outperforms baselines, with greater gains under constrained budgets. In contrast, Revision and Self-Consistency do not scale well. Revision struggles with long-context reasoning due to length generalization issues~\cite{li-etal-2024-making}, while Self-Consistency fails to leverage feedback, often discarding correct answers during majority voting due to long-tailed distributions of correct answers~\cite{DBLP:journals/corr/abs-2407-21787}.

% a line plot that shows the number of additional problems solved given more budgets

% the portion of problems solved in different difficulty levels in MATH

\subsection{Fine-Tuning Results}

\newcommand\hmmax{0}
\newcommand\bmmax{0}


\usepackage{eqndefns-left} 
\RequirePackage{tgtermes}
\RequirePackage{newtxtext}
\RequirePackage{newtxmath}
\RequirePackage{bm}
\RequirePackage{endnotes}

\OneAndAHalfSpacedXII
\usepackage{algorithm}
\usepackage{algpseudocode}
\usepackage{tikz}

\usepackage{natbib}
 \bibpunct[, ]{(}{)}{,}{a}{}{,}%
 \def\bibfont{\small}%
 \def\bibsep{\smallskipamount}%
 \def\bibhang{24pt}%
 \def\newblock{\ }%
 \def\BIBand{and}%


\EquationsNumberedThrough   
\TheoremsNumberedThrough    
\ECRepeatTheorems 
\MANUSCRIPTNO{MNSC-0001-2024.00}


\usepackage[utf8]{inputenc} % allow utf-8 input
\usepackage[T1]{fontenc}    % use 8-bit T1 fonts
\usepackage{hyperref}       % hyperlinks
\usepackage{url}            % simple URL typesetting
\usepackage{booktabs}       % professional-quality tables
\usepackage{amsfonts}       % blackboard math symbols
\usepackage{nicefrac}       % compact symbols for 1/2, etc.
\usepackage{microtype}      % microtypography
\usepackage{xcolor}         % colors
\usepackage{amsmath}
\usepackage{thmtools}
\usepackage{thm-restate}
\usepackage{graphicx}
\usepackage{subcaption}
\usepackage{algorithm}
\usepackage{bbm}
\usepackage{times}
\usepackage{booktabs}
\usepackage{fancyhdr,graphicx,amsmath,amssymb}
\declaretheorem[name=Theorem,numberwithin=section]{thm}


\begin{table*}[t]
\caption{Result comparison  under  fine-tuning setting. See Reward Margins in~\Cref{tab:rm}. \vspace{-0.5em}}
\label{tab:finetune}
\vskip 0.1in
\setlength{\tabcolsep}{14pt} 
\centering
\scalebox{0.8}{
\begin{tabular}{l|cccc|cccc}
\toprule
\multicolumn{1}{l|}{\multirow{2}{*}{\textbf{Method} }}  & \multicolumn{4}{|c|}{\textbf{HH-RLHF}} & \multicolumn{4}{|c}{\textbf{PKU-SafeRLHF}} \\
\cline{2-9}
 & \textbf{BLEU}$\uparrow$ & \textbf{Reward}$\uparrow$ & $\textbf{RM}_{\text{DPO}}$$\uparrow$ & $\textbf{RM}_{\text{R-DPO}}$$\uparrow$ 
& \textbf{BLEU}$\uparrow$ & \textbf{Reward}$\uparrow$ & $\textbf{RM}_{\text{DPO}}$$\uparrow$ & $\textbf{RM}_{\text{R-DPO}}$$\uparrow$ \\
\midrule
\textbf{SFT Model} & 0.220 & 0.425 & - & - & 0.294 & 0.406 & - & - \\
\midrule
\textbf{DPO-PL} & 0.230 & 0.430 & -0.795 & -1.448 & 0.305 & 0.412 & -6.852 & -5.961\\
\textbf{DPO-BT} & 0.230 & \textbf{0.431} & 0.349 & -0.455 & \textbf{0.306} & \textbf{0.417} & -5.441 & -6.167 \\
\rowcolor{mygray}
\textbf{DPO-HPS} & \textbf{{0.232}} & 0.430 & \textbf{{2.723}} & \textbf{{2.040}}& \textbf{0.306} & 0.407 & \textbf{-5.359} & \textbf{-5.851} \\
\midrule
\textbf{EXO-PL} & \textbf{{0.232}}  & \textbf{{0.432}}  & -0.724  & -1.406  & 0.303  & 0.409  & \textbf{-5.455 }& -6.128  \\
\textbf{EXO-BT} & 0.231 & 0.430 & 0.816 & 0.215 & \textbf{{0.324}} & 0.421 & -5.553 & -6.164 \\
\rowcolor{mygray}
\textbf{EXO-HPS} & \textbf{{0.232}} & \textbf{{0.432}} & \textbf{1.079} & \textbf{0.410} & 0.314 & \textbf{0.425} & -5.495 & \textbf{-6.031} \\
\midrule
\textbf{IPO-PL} & 0.223 & \textbf{0.429} & -5.199 & -5.264 & 0.309 & 0.401 & -66.337 & -66.946 \\
\textbf{IPO-BT} & \textbf{0.232} & 0.428 & -0.382 & -1.254 & \textbf{0.310} & 0.405 & -23.070 & -23.678 \\
\rowcolor{mygray}
\textbf{IPO-HPS} & 0.231 & 0.424 & \textbf{-0.321} & \textbf{-0.926} & 0.308 & \textbf{0.406} & \textbf{-21.607}& \textbf{-22.215} \\
\midrule
\textbf{SPPO-PL} & 0.225 & 0.430 & -7.630 &-7.674 & 0.297 & 0.413 & -67.474 & -68.082 \\
\textbf{SPPO-BT} & \textbf{0.231} & \textbf{{0.432}} & -0.978 & -1.411 & 0.297 & 0.433 & -13.442 & -14.050 \\
\rowcolor{mygray}
\textbf{SPPO-HPS} & \textbf{0.231} & \textbf{{0.432}} & \textbf{-0.969} & \textbf{-1.302} & \textbf{0.298} & \textbf{{0.435}} & \textbf{-5.273} & \textbf{-5.881} \\
\midrule
\textbf{NCA-PL} & 0.221 & 0.431 & -5.760 &	-5.819 & 0.300 & \textbf{0.411} & -70.910 & -71.518 \\
\textbf{NCA-BT} & 0.229 & \textbf{{0.432}} & -1.702 & -3.121 & \textbf{0.305} & 0.410 & -5.135 & -5.644 \\
\rowcolor{mygray}
\textbf{NCA-HPS} & \textbf{0.231} & \textbf{{0.432}} & \textbf{-0.822} & \textbf{-1.268} & 0.304 & \textbf{0.411} & \textbf{{-5.109}} & \textbf{{-5.318}} \\
\bottomrule
\end{tabular}}
\vspace{-1.0em}
\end{table*}


We present the results of PEFT baselines and \method{} with a budget of 32 in Table~\ref{tab:meta}. Best-of-N is applied to PEFT baselines to exploit test-time feedback. HumanEval is excluded as it lacks a training set. Table~\ref{tab:meta} highlights the effectiveness of \method{}, outperforming all PEFT baselines by at least 2.58\% on average. \method{} is also parameter-efficient, with 439K trainable parameters that are comparable to the most lightweight PEFT method (LN-Tuning, 266K parameters), while surpassing the best PEFT method (LoRA, 1.7M parameters) with an order of magnitude fewer parameters.
However, \method{} shows suboptimal performance on MATH for \texttt{Mistral-7B-Instruct-v0.3}, which is consistent with other PEFT methods with few trainable parameters (e.g., (IA)$^3$, LoRA, LN-Tuning). This is likely due to \texttt{Mistral-7B-Instruct-v0.3}'s limited mathematical reasoning capabilities, requiring significant parameter updates to improve performance in this domain.

\method{} incurs negligible inference overhead. For example, on GSM8K with \texttt{Llama-3.1-8B-Instruct} and a budget of 32, the best test-time scaling baseline (FTTT) requires 4 GPU hours, whereas \method{} uses only 1.5 GPU hours, benefiting from shorter yet accurate predictions.

% fastest adapter 6.5 GPU hours

\section{Example}
\label{app:example}

Table~\ref{tab:example} shows a counseling session between a client agent and CAMI, with the client's profile shown in Table~\ref{tab:example profile}. During the session, CAMI explores Health and Relationships topics in the first 6 turns.  It subsequently performs Step Into operation from Relationships to Parenting topics in Turn 7, and from Parenting to Paternal Bond topics in Turn 9, before a Switch operation to Habituation topic in Turn 11.  The full exploration path is visualized in Figure~\ref{fig:exploration path}.

\begin{table*}[ht]
\centering
\begin{tabularx}{\textwidth}{c X}
\toprule
        {\bf Turn ID} & {\bf Utterances}                     \\ \midrule
1       & Counselor: Hello. How are you?                                                                                                                                                                                                                                                                                                                     \\ \hline
2       & Client: I am good. What about you?                                                                                                                                                                                                                                                                                                                 \\ \hline
3       & Counselor: I'm doing well, thanks for asking! Can you share how you've been feeling about your health lately, especially in relation to smoking?                                                                                                                                                                                                   \\ \hline
4       & Client: I've been pretty busy with work and juggling family stuff. Sometimes I just think about how nice it is to take a break with my coworkers outside, chatting and during those moments, life feels a bit more relaxed. You know how it is—it’s nice to unwind.                                                                                \\ \hline
5       & Counselor: It sounds like smoking is a comforting routine for you amidst your busy life with work and your daughter. I wonder how it might impact the time you spend with your family.                                                                                                                                                             \\ \hline
6       & Client: I get what you mean, but I really don't see my smoking as an issue. It helps me manage stress, and I don’t think it affects my time with my daughter. We still have good moments together, and I wouldn't want to change that.                                                                                                             \\ \hline
7       & Counselor: It sounds like you find comfort in smoking to manage stress, and that feels crucial to your parenting now. I wonder, though, how you might feel if your daughter started to see smoking as a way to unwind, too. What impact would that have on your relationship with her?                                                             \\ \hline
8       & Client: That’s a tough thought. I remember starting young and how hard it was to quit later. I wouldn’t want that for her. It makes me reflect on what I’m modeling.                                                                                                   \\ \hline
9       & Counselor: It's great that you're reflecting on your modeling for your daughter. Perhaps seeing smoking as something that may weaken your bond could help reshape how you approach stress—considering healthier ways to connect can actually strengthen your relationship and set a positive example for her.                                      \\ \hline
10      & Client: I appreciate that perspective. I do want to set a strong example for her. My experience with my own family and their struggles makes me cautious. I see the patterns and don’t want her to feel that smoking is a go-to solution for stress, like I did.                                                                                   \\ \hline
11      & Counselor: It sounds like your family’s struggles have really shaped your perspective on smoking. You want to protect your daughter from those same patterns, knowing that smoking can not only impact her stress management but also her overall learning and adaptation to challenges. That’s a powerful motivation.                             \\ \hline
12      & Client: Exactly. I want her to learn healthier coping strategies instead of following in my footsteps. I realize that I need to make a change for both our sakes. It’s not just about me; it’s about creating an environment where she can thrive without those habits. It's time to find better ways to manage stress that we can share together.   
\\ \bottomrule
\end{tabularx}
\caption{Counseling Session between CAMI and a Simulated Client.}
\label{tab:example}
\end{table*}



\begin{table*}[ht]
\centering
\begin{tabularx}{\textwidth}{l X}
\toprule
{\bf Profile Item}        & {\bf Description}                     \\ \midrule
Behavioral Problem  & Smoking \\ \hline
State of Mind       & Precontemplation \\ \hline
Personas            & You smoked when you were young.", "You have a 12-year-old daughter. \newline The friends of the your daughter have tried smoking. \newline When you are at work, you all go out and smoke and chit chat and talk about things. \newline You like to have a cigarette after eating. \\ \hline
Motivation          & You are motivated because of the risk of smoking in habituation for your daughter, as you don't want your daughter to think smoking is good thing.    \\ \hline
Beliefs             & You feel smoking enjoyable and relaxing. \newline You would be left out if you would hate to not smoke. \newline You use smoking as a way to manage weight and control eating habits. \newline You have been smoking for a long time and feels it's almost impossible to quit. \\ \hline
Motivation Topics          & Interpersonal Relationships, Parenting, Habituation
\\ \bottomrule
\end{tabularx}
\caption{Client's Profile in the Example.}
\label{tab:example profile}
\end{table*}


\begin{figure*}
    \centering
    \includegraphics[width=\linewidth]{figs/example_explore_path.pdf}
    \caption{Topic Exploration Path by the Counselor in the Example.}
    \label{fig:exploration path}
\end{figure*}



Finally, Figure~\ref{fig:tune-scale} examines the scaling behavior of PEFT baselines and \method{}. Initially, \method{} underperforms compared to other PEFT methods and FTTT, but it mostly achieves superior results when the budget exceeds 2. \method{}'s weaker performance with smaller budgets arises from its reliance on an initial attempt sampled from the raw LLM to initiate the process. This initial attempt often fails but is still counted as a valid attempt, making \method{} less competitive in low-budget settings.

\subsection{Analysis}

\paragraph{Ablation Study.}

Table~\ref{tab:ablation} presents an ablation study on the architecture design of \method{}. The results demonstrate that all components are essential, as removing any of them significantly degrades performance. Notably, normalization is the most critical component, as it addresses the varying gradient scales of different weights.

% how to combine predicted updates when scaling

\paragraph{Case Study.}

Table~\ref{tab:case} provides two examples on GSM8K where the leading PEFT method, LoRA, fails, but \method{} succeeds using \texttt{Llama-3.1-8B-Instruct}. These examples highlight \method{}'s superior ability to correctly interpret and reason through questions, unlike LoRA.

\begin{table*}
  [t]
  \centering
  \resizebox{\textwidth}{!}{%
  \begin{tabular}{cccccccccccc}
    \toprule \multicolumn{2}{c}{Components}                                                             & \multicolumn{5}{c}{Re-executability Rate (\%)} & \multicolumn{5}{c}{Readability (\#)} \\
    \cmidrule(lr){1-2} \cmidrule(lr){3-7} \cmidrule(lr){8-12}        \hspace{8pt}\labelemoji\hspace{8pt}                                                                & \hspace{8pt}\toolemoji\hspace{8pt}                                      & O0                                 & O1             & O2             & O3             & AVG            & O0             & O1             & O2             & O3             & AVG            \\
    \hline
    \rowcolor[rgb]{0.93,0.93,0.93}\multicolumn{12}{c}{\textbf{Initialize with LLM4Decompile-End-6.7B~\citep{llm4decompile}}}   \\
    \xmark                                                                                              & \xmark                                    & 69.51                              & 46.95          & 50.61          & 46.34          & 53.35          & 3.98 & 3.41 & 3.44 & 3.38 & 3.55 \\
    \cmark                                                                                              & \xmark                                    & 75.61                              & 50.61          & 50.00          & 50.00          & 56.55          & 4.01 & 3.44 & 3.39 & \textbf{3.49} & 3.58 \\
    \xmark                                                                                              & \cmark                                    & 83.54                     & \textbf{56.10}          & 51.22          & 50.61 & 60.37 & 4.05 & 3.51 & 3.51 & 3.42 & 3.62 \\
    \cmark                                                                                              & \cmark                                    & \textbf{85.37}                            & \textbf{56.10}                     & \textbf{51.83} & \textbf{52.43}          & \textbf{61.43} & \textbf{4.13} & \textbf{3.60} & \textbf{3.54} & \textbf{3.49} & \textbf{3.69} \\

    \rowcolor[rgb]{0.93,0.93,0.93}\multicolumn{12}{c}{\textbf{Initialize with Deepseek-Coder-6.7B-base~\citep{deepseekcoder}}} \\
    \xmark                                                                                              & \xmark                                    & 59.15                              & 35.98          & 39.02          & 37.80          & 42.99          & 3.71 & 3.05 & 3.16 & 3.05 & 3.24 \\
    \cmark                                                                                              & \xmark                                    & 66.46                              & 41.46          & 38.41          & 36.59          & 45.73          & 3.76 & 3.17 & \textbf{3.21} & 3.08 & 3.31 \\
    \xmark                                                                                              & \cmark                                    & 70.73                              & 39.63          & 39.02          & 40.24          & 47.41          & 3.90 & 3.17 & 3.08 & 3.11 & 3.31 \\
    \cmark                                                                                              & \cmark                                    & \textbf{79.88}                     & \textbf{45.73} & \textbf{43.90} & \textbf{42.68} & \textbf{53.05} & \textbf{3.96} & \textbf{3.21} & 3.18 & \textbf{3.19} & \textbf{3.38} \\
    \bottomrule
  \end{tabular}%
  }
  \caption{The ablation study of different methods across four optimization levels
  (O0, O1, O2, O3), as well as their average scores (AVG). The results in bold represent the optimal performance. The ~\labelemoji~ and ~\toolemoji~ means Relabedling and Function Call. \textbf{Bold} denotes the best performance.}
  \label{tab:ablation}
\end{table*}

% generalize to multiple-choice QA like GPQA

\section{Related Work}

\paragraph{Learning from Feedback.}

Other than the heuristic binary feedback studied in this work, prior research has explored feedback from various sources, such as humans~\cite{DBLP:conf/nips/Ouyang0JAWMZASR22}, other models~\cite{yang-etal-2022-re3}, tools~\cite{DBLP:conf/nips/SchickDDRLHZCS23}, and knowledge bases~\cite{gao-etal-2023-rarr}. This paper focuses on demonstrating the effectiveness of the proposed method and other feedback types are beyond the scope of this paper.

\paragraph{Test-Time Training.}

Test-Time Training (TTT) has shown success in the image modality by addressing distribution shifts and enhancing model capacity through self-supervised fine-tuning on each test case~\cite{DBLP:conf/icml/SunWLMEH20,DBLP:conf/nips/LiuKDBMA21,DBLP:journals/corr/abs-2310-13807}. Recent studies have extended TTT to the text modality~\cite{DBLP:conf/iclr/Hardt024,DBLP:journals/corr/abs-2401-11504}. The most relevant work, by~\citet{akyürek2024surprisingeffectivenesstesttimetraining}, uses TTT to enhance the reasoning ability of LLMs. However, their method relies heavily on human scaffolding for self-supervision and does not generalize beyond ARC-AGI~\cite{chollet2019measureintelligence}. In contrast, FTTT is generally applicable.

\paragraph{Learning to Optimize.}

Learning to Optimize (L2O) trains a network to act as an optimizer for another network~\cite{DBLP:journals/jmlr/ChenCCH0WY22}. Early approaches used reinforcement learning to train such optimizers~\cite{DBLP:conf/iclr/LiM17,DBLP:conf/icml/ChenHCDLBF17}, while recent work focuses on discovering analytical white-box optimizers~\cite{DBLP:conf/icml/BelloZVL17,DBLP:conf/nips/ChenLHRW0DLHLL23}. The most relevant work, MEND~\cite{DBLP:conf/iclr/MitchellLBFM22}, trains a network to predict weight updates from training gradients. \method{} builds on this idea, extending it to learn from test-time feedback with a distinct architecture.

\section{Conclusion}

In this paper, we propose a novel paradigm that leverages optimization to address the challenge of exploiting test-time feedback, resulting in improved scaling performance. We further present a learnable test-time optimizer, \method{}, which surpasses various PEFT baselines. Both FTTT and \method{} are efficient in terms of speed and trainable parameter count.

\section*{Limitations}

The current evaluation setting limits FTTT's potential by providing only binary feedback (i.e., correct or incorrect) for each attempt. However, developing complex reasoning environments with rich feedback is beyond the scope of this work. Additionally, while continuous feedback, such as that from reward models~\cite{DBLP:journals/corr/abs-2409-12122}, has been extensively studied, it is not examined here. Our method can be adapted to continuous feedback with minimal modifications, such as using REINFORCE~\cite{DBLP:journals/ml/Williams92}. For coherence, we leave this exploration to future work.

% \section*{Acknowledgments}


% Bibliography entries for the entire Anthology, followed by custom entries
%\bibliography{anthology,custom}
% Custom bibliography entries only
\bibliography{custom}

\appendix

\section{Hyperparameter Settings}
\label{app:setting}

Below is the detailed configurations of different fine-tuning methods:
\begin{itemize}[noitemsep, nolistsep]
    \item \textbf{Adapter} uses a learning rate of 1e-4 and the reduction factor of the bottleneck is 16.
    % \item \textbf{Prefix-Tuning} uses a learning rate of 2e-4 for \texttt{Llama-3.1-8B-Instruct} and 5e-5 \texttt{Mistral-7B-Instruct-v0.3} and the prefix length is 10.
    \item \textbf{(IA)$^3$} uses a learning rate of 5e-5.
    \item \textbf{LoRA} uses a learning rate of 2e-5. We only apply LoRA to the query and value projections in the last 8 layers, with a rank of 16 and a dropout ratio of 0.05.
    \item \textbf{LN-Tuning} uses a learning rate of 4e-4.
    \item \textbf{Full Fine-Tuning} uses a learning rate of 1e-5.
\end{itemize}
The number of training epochs is 100, 10, and 3 for MBPP, GSM8K, and MATH respectively. We use the Adam optimizer with a batch size of 20 for all methods in all datasets, including \method{}.

For \method{}, we sample 10 attempts for each training example together with the raw question to construct the model input. We employ nucleus sampling~\cite{DBLP:conf/iclr/HoltzmanBDFC20} with a temperature of 0.6 and $p=0.95$ to generate attempts. The number of training epochs for MBPP, GSM8K, and MATH is set to 10, 3, and 3 respectively. The learning rate is 1e-5. In inference, we sample an attempt using the same hyperparameters as in data generation before applying \method{} to mitigate the train-test discrepancy. We alternate between sampling attempts from the raw LLM and predicting refined attempts from sampled attempts when scaling \method{} with more budgets.

%\vspace{-15pt}

\section{Separation Logic Predicate Synthesis via \tool}
\label{sec:SLsynthesis}

Having described the enhanced \emph{general-purpose} predicate
synthesis algorithm from positive-only examples,
%
we now show how to instantiate it for synthesis of inductive SL
predicates and improve the efficiency of the search algorithm by
exploiting domain-specific SL insights. We further discuss the
SL-validity of the synthesised predicates and the completeness of the
search algorithm.

\subsection{SL Predicates: Basics and Intricacies}
\label{sec:default}
 
\begin{figure}[!t]
  \centering
  \[
\begin{aligned}
  \sym{predicate} & ::= \sym{main\_pred} \;|\; \sym{main\_pred}  \sym{invented\_pred}\ast \\
  \sym{main\_pred} & ::= \sym{base\_clause}(\pre{main\_head}) \;|\; \sym{rec\_clause}(\pre{main\_head})\ast \\
  \sym{invented\_pred} & ::= \sym{base\_clause}(\pre{inv\_head}) \;|\; \sym{rec\_clause}(\pre{inv\_head})\ast \\
  \sym{base\_clause}(H) & ::= H(\codeinmath{This}, \sym{args}) \leftarrow \sym{base\_lit}\ast, \sym{pure\_lit}\ast \\
  \sym{rec\_clause}(H) & ::= H(\codeinmath{This}, \sym{args}) \leftarrow \sym{pointer\_lit}\ast, \sym{rec\_lit}\ast, \sym{pure\_lit}\ast \\
  \sym{literal}(R) & ::= R(\sym{args}) \\
  % Define the specific types of literals
  \sym{base\_lit} & ::= \sym{literal}(\pre{base\_pred}) \qquad\qquad \texttt{\% Pre-defined  for spatial relations} \\
  \sym{pure\_lit} & ::= \sym{literal}(\pre{pure\_pred}) \qquad \qquad\enspace \quad \texttt{\% Pre-defined  for pure relations} \\
  \sym{pointer\_lit} & ::= \pre{domain}(\codeinmath{This}, \sym{var}) \qquad\qquad\quad\enspace \texttt{\% Extract from the memory graphs} \\
  \sym{rec\_lit} & ::= \sym{literal}(\sym{head}) \\
  % General concepts
  \sym{args} & ::= \sym{var} \;|\; \sym{var}, \sym{args} \\
  \sym{var} & ::= \codeinmath{X1} \;|\; \codeinmath{X2} \;|\; \dots \;|\; \codeinmath{This} \\
  \sym{head} & ::= \pre{main\_head} \;|\; \pre{inv\_head} \quad \texttt{\% From the task or randomly generated}
\end{aligned}
\]
\caption{The grammar of the SL predicates, in basic Backus–Naur form
  (BNF), extended with (1) meta-variables $(\cdot)$ for specialising
  the symbols, and (2) pre-defined atoms denoted by $\pre{X}$ (with
  comments of their origins).}
  \label{fig:grammar}
\end{figure}

We define the space of SL predicates in a way standard for
Syntax-Guided Synthesis (SyGuS)~\cite{Alur-al:FMCAD13}.
%
The grammar of the SL predicates is shown in \autoref{fig:grammar}. An
SL predicate is either having a shape with a single main predicate, or
shaped by a main predicate together with a set of invented
\emph{auxiliary} predicates, which are required in the case of nested
linked structures.
%

Specific to the predicates,
both main predicate and invented predicates consist of the base and recursive clauses, where the base clause is the one that does not have any recursive calls, and the recursive clause is the one that has recursive calls. The head literal (\ie, before $\leftarrow$) in each clause has a fixed argument \pcode{This} that denotes the base address of the data structure (similar to the \textit{this} reference in object-oriented programming).
% 
The body literals (\ie, after $\leftarrow$) in the clauses are defined in terms of different predicates: the base (and pure) predicates are pre-defined, but extensible, to capture the spatial relation among the pointer for the base clause (the pure constraints among variables in clauses, respectively); the domain predicates describe the points-to relations can be obtained from the memory graphs; the recursive predicates are the recursive calls to the main or invented predicates.

% To define a tractable search procedure


Three aspects in the grammar in \autoref{fig:grammar} contribute to the
infinite synthesis search space: (1) the length of clauses, (2) the
number of sub-clauses for each predicate, (3) the arity of the
invented predicates. 
%
% As customary in SyGuS, we bound them with constants.
%
For our task, we noticed that predicates for real-world structures
rarely require more than 10 literals in their bodies; two sub-clauses
for each predicate are sufficient to capture the common structures;
and the arity of the invented predicates is set to be not more than
the arity of the main predicates. Such bounds for hypothesis space are
common in almost all synthesis-by-example tools~(\eg,
\cite{cropper2021learning,lee2021combining,Si-al:FSE18}), not only to
make the synthesis tractable, but also to avoid
overfitting~\cite{PadhiMN019} (\eg, a predicate disjointing facts of
all examples).
%

Below, we discuss two challenges in make SL predicate synthesis
effective and efficient, together with how we address them in \tool.

% %
% The restriction of the search space is also a
% common solution to \emph{overfitting}, which is common in
% synthesis-by-example methods: there is always a
% predicate disjointing facts of all examples, but it is
% likely to be overfitted for specific examples. By providing a finite
% search space, such problem is eliminated.

% The outline  approaches here are
% presented in the context of our SL-specific setting, but are also
% applicable to other ILP tasks.


\subsubsection{Semantic-Based Pruning.}
\label{sec:semantics}

In most existing syntax-guided synthesisers \cite{cropper2021learning,Alur-al:FMCAD13,Si-al:FSE18}, the search is accelerated by pruning of the hypothesis search
space by employing the general \emph{syntax}
restrictions.
%
Other than limiting the syntax, we apply the following \emph{semantic}
properties' restriction of Separation Logic predicates to the search.
%
% Specifically, we encode the properties of SL predicates (\eg, \emph{minimum
%   reachability, pointer functionality}) with ASP so that many invalid
% outputs from \popper are eliminated. 
%
\begin{enumerate}
  \item \emph{Basic reachability}: no points-to relation appears in the
    body other than the ones from the \pcode{this} pointer. Thus, the clause \pcode{p(X, Y) :- next(X, Y), next(Y, Z), ...} is not 
    allowed as a candidate, because we expect all locations in the body to be
    accessible from \pcode{this} via fields.
  %
  \item \emph{Basic assumptions}: the base (non-recursive) clause
    restricts \pcode{this} pointer to either be \code{null} or to equal to
    another pointer parameter variable. \Eg, \pcode{p(X, Y) :-
      nullptr(X), ...} is allowed, but \pcode{p(X, Y) :-
      next(X, Y), ...} cannot be a base clause.
  %
  \item \emph{Restricted use of} \code{null}: if a variable \pcode{X} is
    a null-pointer (denoted by \pcode{nullptr(X)}), no
    more \pcode{X} occurs in the clause. \Eg, the clause \pcode{p(X, Y) :- nullptr(X), next(X, Y)}
    is not allowed.
  %
  \item \emph{Quasi-well-founded recursion of payload}: the pure argument passed to a
    recursive call should (non-strictly) decrease. \Eg, for a clause
    \pcode{p(X, S) :- next(X, Z), p(Z, S1), ...}, the set \pcode{S} should contains \pcode{S1}. This
    is a common assumption in recursive program synthesis \cite{albarghouthi2013recursive,lee2021combining}.
  %
  \item \emph{Heap functionality}: points-to relations of the same field
    should not target different locations. \Eg, a candidate clause cannot be \pcode{p(X, Y) :- next(X, Z), next(X, Z1), ...}.
  %
  \end{enumerate}

\noindent
%
This list of search constraints represents a combination of the
properties implied by SL semantics (in a Java-style field-based memory
model) as well as by common properties of data structures, which are
essential for the efficient search of SL predicates.
%
The exact encodings of these constraints in ASP are provided and explained in \autoref{app:slsemantics}.

\subsubsection{Free Variables and Auxiliary Placeholders.}
\label{sec:auxiliary}

Free variables are common in SL predicates, \eg, the (implicitly
existentially-quantified) location \pcode{Y} in the base clause of the
 doubly linked list below:
%
\begin{minted}[fontsize=\small]{prolog}
  dll(X, Y) :- nullptr(X).
  dll(X, Y) :- next(X, Z), prev(X, Y), dll(Z, X).
\end{minted}
%
Unfortunately, completeness guarantees of pruning discussed in \autoref{sec:popper2}  do not hold for
predicates with free variables in the sense that
 a complete (\ie, valid) hypothesis with free
variables might  be wrongfully pruned during the search~\cite[\S{4.5}]{cropper2021learning}.
%
To address this problem, we introduce \emph{auxiliary placeholders}
into the search as a way to express predicate clauses with free
variables.
%
For example, the following doubly linked list predicate can be
regarded the same as the one above with \pcode{anypointer()}
placeholder, and \emph{can} be synthesised.
%
\begin{minted}[fontsize=\small]{prolog}
  dll(X, Y) :- nullptr(X), anypointer(Y).
  dll(X, Y) :- next(X, Z), prev(X, Y), dll(Z, X).
\end{minted}
%
On a technical level, this requires adding an ASP constraint (shown in \autoref{app:auxiliary})
that forces the parameter of the placeholder predicate (\pcode{Y}
here) to appear \emph{twice} in the whole clause, so it could be later
translated into a single occurrence of a free variable.

% \subsubsection{Hypothesis Specificity.}
% \label{sec:specificity}

% Having applied the clause minimisation for redundancy elimination, the
% synthesiser is often left with the problem to choose the best
% hypothesis from the set of ``canonical'' candidates, none of which
% entail each other.
% %
% Our novel notion of specificity is aimed to provide an ordering that
% helps to make such a choice.
% %
% As an example, consider
% %
% the predicates \pcode{p()} and \pcode{q()} of the same size, defined
% as \pcode{p(A, B) :- succ(A, B)} and \pcode{q(A, B) :- less_than(A,
%   B)}.
% %
% But based on the meaning of the predicates, we should know
% that \pcode{succ(A, B)} is a stronger statement than
% \pcode{less_than(A, B)}, so \pcode{p(A, B)} is more specific than
% \pcode{q(A, B)}. With this to be considered, the specificity of a
% hypothesis is defined by the following (strict) partial order.


% \begin{definition}[Hypothesis Specificity]
% \label{def:spec}
% Given two hypotheses $A, B$ with the same arity and the same number of
% clauses, $A$ is \emph{more specific than} $B$ (denoted by $A \prec B$)
% \Iff either ($i$) $\mathit{size}(B) < \mathit{size}(A)$ (\ie, $A$ has
% strictly more literals), or ($ii$)
% $\mathit{size}(A)=\mathit{size}(B)$, and $\exists\mathit{l}_1$,
% $\mathit{l}_2$, s.t. $B(\mathit{l}_1/\mathit{l}_2) = A$, and
% $\mathit{l}_1 \models \mathit{l}_2$, and
% $\mathit{l}_2 \not\models \mathit{l}_1$, where
% $\mathit{l}_1/\mathit{l}_2$ denotes the replacement of the literal
% $\mathit{l}_2$ in a sub-clause of $B$ by $\mathit{l}_1$.
% \end{definition}

% \todo{}
% We conclude this section with the following formal proposition stating
% that our synthesis algorithm returns a \emph{locally-optimal} hypothesis 
%  in the search space with specificity as the
% metric.

% \begin{theorem}
%   \label{thm:specific}
%   The hypothesis returned by the positive-only learning in
%   \emph{\autoref{alg:popper}} is the most specific (i.e., the local
%   minimum of the specificity) predicate that is complete in the search
%   space defined by the algorithm's initial constraints
%   (\pcode{in_cons}) and the size limit (\pcode{max_size}) parameters.
% \end{theorem}
% \begin{proof}[Proof]
%   By induction on the size limit \pcode{max_size} of the predicate: when \pcode{max_size} is 0, there is no predicate hypotheses, so \pcode{None} is the most specific one. Then assume that the theorem holds for \pcode{max_size} $n$, \ie, \pcode{sol_i} is the most specific; we prove it for \pcode{max_size} $n+1$.

%   When \pcode{max_size} is $n+1$, based on the while loop in
%   \autoref{alg:popper}, the search space for $n+1$ is the search space
%   for $n$ plus when \pcode{size} is $n+1$. By the induction
%   hypothesis, \pcode{sol_i} is the most specific in the search space
%   for $n$, and the output \pcode{sol} is either \pcode{sol_i} or the
%   more specific one in $n+1$. Therefore, \pcode{sol} is the most
%   specific in the search space with $n+1$ as \pcode{max_size}.

% \end{proof}

\subsection{Ensuring SL Validity in \prolog}
\label{sec:sldomain}

An astute reader can notice that the validity of the synthesised
predicates is not immediate due to our treatment of \prolog clauses as
SL assertions: the conjunction in \prolog does not guarantee the
\emph{separating conjunction} (\pcode{*}) in the SL sense. As an
example, consider the following simplified \prolog predicate for
binary trees:
%
\begin{minted}[fontsize=\small]{prolog}
  bi_tree(X) :- nullptr(X). 
  bi_tree(X) :- t1(X, L), t2(X, R), bi_tree(L), bi_tree(R).
\end{minted}
%
In this case, an instance of \pcode{bi_tree(X)} being evaluated to be
\emph{true} in \prolog can imply \emph{false} under SL semantics that
enforces heap disjointness: considering a memory graph with two nodes
%
\begin{minted}[fontsize=\small]{prolog}
  t1(n1, n2). t2(n1, n2). t1(n2, null). t2(n2, null).
\end{minted}
%
so that the graph fact \pcode{bi_tree(n1)} is provable in \prolog, but
the clauses \pcode{bi_tree(L)} and \pcode{bi_tree(R)} are
\emph{non-disjoint}.
%
Notice that, in our inductive synthesis setting, this situation would
correspond to having \emph{multiple} occurrences of the same points-to
fact in a memory graph representing a positive example for the
predicate, but should not be allowed by the definition of separating
conjunction.

To avoid this source of unsoundness, we use a straightforward solution
that enforces such separating conjunction semantics in \prolog: a
valid SL predicate is a complete \prolog predicate where the positive
examples succeed using each points-to fact \emph{exactly} one time (a
semantic property of SL assertions known as \emph{linearity}).
%
For the complete \prolog but invalid SL predicates, we also use the
\textit{specialisation} rule in \autoref{sec:popper2} to prune them:
if a predicate violates the linearity, then a more constrained one
will also violate it; this contributes to the new pruning in
line~15 of \autoref{alg:popper}.

We establish the following property of our SL-specific predicate
synthesis stating that, for the predicates in \tool's search space in
\autoref{sec:default}, if a memory graph is provable in \prolog with
linearity, then the corresponding heap is valid under SL semantics.

\begin{theorem}[SL Validity]
\label{thm:validity}
Let \pcode{F(h)} denote the memory graph of a heap \pcode{h}. For any
output predicate \pcode{p(X)} of \tool and any heap \pcode{h}, the
following fact holds: 
%
  \pcode{F(h)} $\models_{\prolog+\text{Lin}}$
\pcode{p(X)} $\Rightarrow$ \pcode{h} $\models_{\text{SL}}$ \pcode{p(X)}.
% \begin{center}
% \end{center}
\end{theorem}




\subsection{The \tool Algorithm}
\label{sec:tool}

The only remaining step before putting all the pieces together is to
select the desired predicate from the set of non-comparable solutions
of positive-only learning. 
%
Even though predicates from POL can be conjuncted in general, the
semantics of SL predicates following the definition in
\autoref{sec:default} is more restrictive and the conjunction of valid
SL predicates may result in an ill-formed or a constantly false one. 
%
We found in practice that after the semantics-based normalisation from
\autoref{sec:normalise}, the number of literals can serve as a
\emph{good enough} specificity metric among incomparable predicates,
since containing more literals is likely to contain more information
or constraints about the heap structure. 
%
Following this intuition, we define the synthesis algorithm with
MAX\_POL function, which obtains the largest predicate from POL as per
\autoref{alg:popper}.

\begin{algorithm}[!t]
  \caption{The \tool loop for inductive predicate synthesis}
  \label{alg:sippy}
  \begin{algorithmic}[1]
  \small
  \Require memory graphs consist of \pcode{graph_bk, exs}
  \Procedure{Sippy}{\pcode{graph_bk, exs}}
      \State \pcode{graph_cons, shapes} = \pcode{graph_info(graph_bk, exs)}
      \State \pcode{max_var} = \pcode{max_body} = 1
      \State \pcode{sol} = \pcode{True} \Comment{The most general solution as initialisation}
      \For{\pcode{shape} in \pcode{shapes}}
        \State \pcode{max_size} = \pcode{maxsize(max_body, shape)}
        \State \pcode{h} = \Call{MAX\_POL}{\pcode{graph_bk, exs, graph_cons, max_size}}
        \If{\pcode{h} $\prec$ \pcode{sol}} \Comment{A more specific predicate is obtained}
            \State \pcode{max_var, max_body} = \pcode{(var_of(h), size_of(h))} + $\delta$
            \State \pcode{sol} = \pcode{h}
        \ElsIf{\pcode{sol} == True} \Comment{No predicate is yet learned}
            \If{\pcode{max_var} == \pcode{upper_bound}}
                \State \textbf{continue} \Comment{Try the next shape}
            \EndIf
            \State \pcode{max_var, max_body} += (1, 1)
        \Else
            \State \textbf{break} \Comment{No more specific predicate is found}
        \EndIf
      \EndFor
      \State \Return \pcode{sol}
  \EndProcedure
  \end{algorithmic}
\end{algorithm}



\autoref{alg:sippy} summarises the internal workings of \tool.
%
Our synthesiser takes as inputs memory graphs encoded as sets of logic
facts (\eg, \pcode{graph_bk}, such as \pcode{next(..)} and
\pcode{value(..)} from \autoref{fig:sorted-list}), positive examples of
heaps on which a predicate holds (\eg, \pcode{exs} as \pcode{srtl(..)}
from \autoref{fig:sorted-list}), so that the shape (matched with
pre-defined shapes in \autoref{sec:default}) a set of ASP constraints
(\pcode{graph_cons}) describing the information in the graphs (such as
the arity and types of the predicates to be learned) are obtained
(line~2).
%
Two parameters (line~3) for positive-only learning (MAX\_POL), (1)~the maximum number of
variables and (2)~the maximum size of the body of a predicate clause
for restricting the search space, are gradually increased during the
search using the following empirical strategy:
%
if no solution is valid (line~11), we either increase both parameters
by one to enlarge the space until finding one (line~14), or the
attempt on the current predicate shape fails (\ie, the upper bound of
the search space is reached), then
\tool will try synthesising using the next shape (line~13, \ie, more auxiliary predicates);
%
when obtaining one new better predicate than the existing, the search
parameters are both increased by a parameter~$\delta$ to find a
possibly more specific predicate (line~9), and the solution is
updated (line~10); if the learned predicate in the larger search space
is not better than the previous, we stop the search and output
(line~15-16).
%
The role of the parameter $\delta$ is, thus, to provide a ``margin''
for the completeness of the search: it is not guaranteed that \tool
will find the most specific solution \emph{across all possible search
  spaces}, but only in the search-space that is bound by the returned
output's parameters \emph{plus}~$\delta$.\footnote{We choose it to be (1,2) in our experiment from the natural observation: for our domain, we expect to have one body literal where the predicate is generating a new variable, and one more body literal that uses the new variable.}
%
Note that line~6 of \autoref{alg:sippy} features a function
\pcode{maxsize()} that calculates the maximum size of the search space based on the maximum number of variables and the predicate shape setting.

Finally, we provide a correctness argument for \tool. The soundness of
synthesising \emph{consistent} (\ie, inhabited) and \emph{well-formed}
(\ie, finitely provable) SL predicates is guaranteed by the soundness
of classic ILP and \autoref{thm:validity}. The following ``local''
completeness states that, given the output of \tool, no more specific
output can be discovered, \emph{even in} the larger search space
obtained by increasing the search parameters \emph{once} by $\delta$
at the line~9 of \autoref{alg:sippy}.

\begin{theorem}[Local Completeness of \tool]
\label{thm:completeness}
If the output of \tool is a predicate with the maximum number of
variables $m$ and the maximum length of the body $n$, then there is no
predicate with the maximum length of the body $m'$ and the maximum
number of variables $n'$, where $(m',~n')-(m,~n) = \delta$, that is
more specific than the output predicate.
\end{theorem}
\begin{proof}[Proof]
  Directly by contradiction and Theorem 3.1. Assume that the output solution \pcode{sol} is with size $(m,~n)$, and it is not the most specific one in size $(m',~n') = (m,~n) + \delta$.

 Because \pcode{sol} is the output, the search space is set to be $(m',~n')$ after the loop it is obtained. With Theorem 3.1 and the assumption, there is a solution \pcode{sol}$'$ in $(m',~n')$ that is more specific than \pcode{sol}, which is a contradiction with the output \pcode{sol}. Thus, \pcode{sol} is the most specific one in $(m',~n')$.
\end{proof}







% \subsection{Domain-Specific Pruning}
% \label{sec:dsc}

% So far we have shown how to encode the syntax of SL predicates
% (\autoref{sec:default}) as well as their basic semantic properties
% that guarantee validity of the solutions (\autoref{sec:sldomain}).
% %
% To enable even more aggressive yet sound search space pruning, we have
% encoded more SL properties as ASP search constraints:
% %
% \begin{enumerate}
% \item \emph{Basic Reachability}: no points-to relation appears in the
%   body other than the ones from the \pcode{this} pointer. For instance, the
%   predicate like \pcode{p(X, Y) :- next(X, Y), next(Y, Z), ...}, is not
%   allowed, because we expect all locations in the body to be
%   accessible from \pcode{this} via fields.
% %
% \item \emph{Basic Assumptions}: the base (non-recursive) clause
%   restricts \pcode{this} pointer to either be \code{null} or to equal to
%   another pointer parameter variable. For instance, \pcode{p(X, Y) :-
%     nullptr(X), ...} can be the base clause, but \pcode{p(X, Y) :-
%     next(X, Y), ...} cannot.
% %
% \item \emph{Restricted use of} \code{null}: if a variable \pcode{X} is
%   a null-pointer (denoted by \pcode{nullptr(X)}), no
%   more \pcode{X} should occur in the clause body. For example, \pcode{p(X, Y) :- nullptr(X), next(X, Y)}
%   is not allowed.
% %
% \item \emph{Quasi-well-founded recursion}: the pure argument passed to a
%   recursive call should (non-strictly) decrease. For instance,
%   \pcode{p(X, Y) :- next(X, Z), Y1 == Y+1, p(Z, Y1)} is not valid. This
%   is a common assumption in recursive program synthesis, which is also
%   suitable for our task.
% %
% \item \emph{Heap functionality}: points-to relations of the same field
%   should not target different locations. For instance, \pcode{p(X, Y)
%     :- next(X, Z), next(X, Z1), ...} is not allowed.
% %
% \end{enumerate}
% %
% This list of search constraints represents a combination of the
% properties implied by SL semantics (in a Java-style field-based memory
% model) as well as by common properties of shapes of data structures
% considered.
% %
% The rules above are merely optimisations: they are not necessary to
% ensure correctness of \tool and serve to restrict the search space to
% be a refined (but expressive) domain of SL predicate. 




\section{Prompts}
\label{sec:prompt}

Below is the reflection generation prompts $P$ for \texttt{Llama-3.1-8B-Instruct} and \texttt{Mistral-7B-Instruct-v0.3}:

\patchcmd{\quote}{\rightmargin}{\leftmargin 15pt \rightmargin}{}{}
\begin{quote}
\small %\it
\begin{tcolorbox}[breakable, colback=white, colbacktitle=blue!5!white, colframe=black, boxrule=1pt, title={\textcolor{black}{\textbf{Llama-3.1-8B-Instruct}}}]
\textbf{User:} Solve the following math problem $\ldots$\\
\textbf{Assistant:} $\ldots$ the final answer is: $\ldots$\\
\textbf{User:} Your answer is incorrect. \textcolor{gray}{Please carefully check the solution and summarize all mistakes in short. Do NOT provide the corrected solution. Do NOT say ``my solution''.}\\
\textbf{Assistant:} \textcolor{royalblue}{Here is the summary of the mistakes in the previous solution $\ldots$}
\end{tcolorbox}
\end{quote}
\patchcmd{\quote}{\rightmargin}{\leftmargin 26pt \rightmargin}{}{}

\patchcmd{\quote}{\rightmargin}{\leftmargin 15pt \rightmargin}{}{}
\begin{quote}
\small %\it
\begin{tcolorbox}[breakable, colback=white, colbacktitle=blue!5!white, colframe=black, boxrule=1pt, title={\textcolor{black}{\textbf{Mistral-7B-Instruct-v0.3}}}]
\textbf{User:} Solve the following math problem $\ldots$\\
\textbf{Assistant:} $\ldots$ the final answer is: $\ldots$\\
\textbf{User:} Your answer is incorrect. \textcolor{gray}{Carefully check the solution step-by-step and list all mistakes in short. MUST NOT provide the correct answer. Your response MUST be in the third person tone..}\\
\textbf{Assistant:} \textcolor{royalblue}{Here is the summary of the mistakes in the previous solution $\ldots$}
\end{tcolorbox}
\end{quote}
\patchcmd{\quote}{\rightmargin}{\leftmargin 26pt \rightmargin}{}{}
Sentences in \textcolor{gray}{gray} are the prompt and the one in \textcolor{royalblue}{blue} is the generated reflection.

\section{Additional Results}

Figure~\ref{fig:learning} is the training curves of various PEFT methods, including \method{}. We observe that Adapter shows a clear signal of overfitting, where it has a training loss close to 0, while its performance on the test set is the worst. All PEFT methods seem to converge smoothly. For \method{}, its training is not as stable as baselines, suggesting the difficulty of learning to optimize problem.

\end{document}
