\begin{table*}[t!]
    \centering
    % \setlength{\tabcolsep}{3pt}
    \small
    % \setlength{\tabcolsep}{1.5mm}{\scalebox{0.97}{
    \begin{tabular}{p{0.52\textwidth}|p{0.42\textwidth}}
    \toprule
    % id 243 + 330
    % \textbf{Question}: Who is the mascot of the university related to Randy Conrads? & 
    % \textbf{Question}: What is the record label of the co-writer and recording artist of Permission to Fly?\\
    % \textbf{Document} [4](Title: Benny Beaver): \blue{Benny Beaver is the official mascot of Oregon State University} and winner of the 2011 Capital One Mascot of the Year write - in campaign. \ldots & 
    % \textbf{Document} [8](Title: No Ordinary Girl): \blue{No Ordinary Girl is the debut album by Jordan Pruitt, released in the United States on February 6, 2007 by Hollywood Records}. The album debuted and peaked at number sixty-four \ldots\\
    % \textbf{Document} [7](Title: Randy Conrads): \blue{Randy Conrads attended Oregon State University}, graduating in 1972 with a bachelor's degree in industrial engineering. \ldots & 
    % \textbf{Document} [19](Title: Permission to Fly): \blue{Permission to Fly is the second and final studio album by American singer-songwriter Jordan Pruitt}. On July 22, 2008, the album was released \ldots\\
    % \gray{(Other irrelevant documents are omitted.)} & \gray{(Other irrelevant documents are omitted.)} \\
    % \textbf{Vicuna}: \red{the university. \ \xmark} & 
    % \textbf{Vicuna}: The answer is: \red{The answer is a question. \ \xmark}\\
    % \textbf{\method{}}: Randy Conrads attended Oregon State University \org{[7]}. The mascot of Oregon State University is Benny Beaver \org{[4]}. The answer is: \gr{Benny Beaver \ \cmark} & 
    % \textbf{\method{}}: The co-writer and recording artist of ``Permission to Fly'' is Jordan Pruitt \org{[19]}. The record label of Jordan Pruitt is Hollywood Records \org{[8]}. The answer is: \gr{Hollywood Records \ \cmark}\\

    % GSM8K 147 125 *157 214 *340
    \makecell*[{{p{0.52\textwidth}}}]{
    \textbf{Question:} Blake and Kelly are having a contest to see who can run the most in 15 minutes. They decide to do it on a football field that is 100 yards long. Blake runs back and forth 15 times. Kelly runs back and forth once, and then decides that she doesn't want to run next to Blake, so she starts to run to the 40-yard line and back. She does this 34 times. How much farther does the winner run than the loser?
    \\
    \rule[0.4em]{0.52\textwidth}{0.4pt}
    \\
    \textbf{LoRA:} Blake runs 15 * 100 * 2 = 3000 yards.\\Kelly runs 100 * 2 = 200 yards for each run to the 40-yard line and back.\\\org{Kelly runs 200 * 34 = 6800 yards.}\\Kelly runs 6800 - 3000 = 3800 yards more than Blake.\\The final answer is \red{3800 \xmark}.
    \\
    \rule[0.4em]{0.52\textwidth}{0.4pt}
    \\
    \textbf{\method{}:} Blake runs back and forth 15 times, and each time, he runs 100 yards twice because he runs to one end of the field and back. So, Blake's total running distance is 15 x 200 = 3000 yards.\\Kelly runs back and forth once, which is 200 yards, and then runs to the 40-yard line and back, which is 40 x 2 = 80 yards, and does this 34 times. \blue{Kelly's total running distance is 200 + (34 x 80) = 200 + 2720 = 2920 yards.}\\Therefore, Blake runs 3000 - 2920 = 80 yards farther than Kelly.\\The final answer is \gr{80 \cmark}.
    }
    &
    \makecell*[{{p{0.42\textwidth}}}]{
    \textbf{Question:} Josh decides to take up juggling to perform at the school talent show a month in the future. He starts off practicing juggling 3 balls, and slowly gets better adding 1 ball to his juggling act each week. After the end of the fourth week the talent show begins, but when Josh walks on stage he slips and drops three of his balls. 2 of them are caught by people in the crowd as they roll off the stage, but one gets lost completely since the auditorium is dark. With a sigh, Josh starts to juggle on stage with how many balls?
    \\
    \rule[0.4em]{0.42\textwidth}{0.4pt}
    \\
    \textbf{LoRA:} Josh started with 3 balls and added 1 ball each week for 4 weeks, so he had 3 + 4 = 7 balls.\\\org{He dropped 3 balls, but 2 were caught, so he lost 3 - 2 = 1 ball.}\\He was left with 7 - 1 = 6 balls.\\The final answer is \red{6 \xmark}.
    \\
    \rule[0.4em]{0.42\textwidth}{0.4pt}
    \\
    \textbf{\method{}:} Josh started with juggling 3 balls. After the end of the fourth week, Josh would have added 4 balls, making it a total of 3+4=7 balls.\\\blue{Josh had 3 balls left after 2 of the balls were caught by people in the crowd, so he was left with 7-3=4 balls.}\\The final answer is \gr{4 \cmark}.
    }
    \\
    \bottomrule
    \end{tabular}
    % }}
    \caption{Two GSM8K examples with model predictions. \gr{\cmark} means correct and \red{\xmark} means incorrect. Sentences marked in \org{orange} are incorrect reasoning steps and those marked in \blue{blue} are the correct counterparts.}
    \label{tab:case}
    % \vspace{-0.4cm}
\end{table*}