\begin{figure*}[t!]
    \newdimen\base
    \base=0.7cm

    \tikzstyle{textnode} = [rectangle,font=\scriptsize,draw=black,inner sep=0pt,outer sep=0pt,minimum width=2\base,minimum height=\base,rounded corners=2pt]

    \hspace*{\fill}
    \subfigure[Sequential Revision]
    {
        \centering
        \begin{tikzpicture}
        
        \node[textnode, fill=lyyblue!40!white] (question) at (0,0) {Question};
        \node[textnode, anchor=north, fill=red!30!white] (answer1) at ([yshift=-0.5\base]question.south) {Attempt \#1};
        \node[textnode, anchor=north, fill=red!30!white] (answer2) at ([yshift=-0.5\base]answer1.south) {Attempt \#2};
        \node[textnode, anchor=north, fill=red!30!white] (answer3) at ([yshift=-0.5\base]answer2.south) {Attempt \#3};
        \node[textnode, anchor=north, fill=ugreen!30!white] (answer4) at ([yshift=-0.5\base]answer3.south) {Attempt \#4};
        
        \draw[-latex] (question.east) to[out=0, in=0, looseness=3] ([xshift=\base, yshift=-0.25\base]question.south) to ([xshift=-\base, yshift=0.25\base]answer1.north) to[out=180, in=180, looseness=3] (answer1.west);
        \draw[-latex] (answer1.east) to[out=0, in=0, looseness=3] ([xshift=\base, yshift=-0.25\base]answer1.south) to ([xshift=-\base, yshift=0.25\base]answer2.north) to[out=180, in=180, looseness=3] (answer2.west);
        \draw[-latex] (answer2.east) to[out=0, in=0, looseness=3] ([xshift=\base, yshift=-0.25\base]answer2.south) to ([xshift=-\base, yshift=0.25\base]answer3.north) to[out=180, in=180, looseness=3] (answer3.west);
        \draw[-latex] (answer3.east) to[out=0, in=0, looseness=3] ([xshift=\base, yshift=-0.25\base]answer3.south) to ([xshift=-\base, yshift=0.25\base]answer4.north) to[out=180, in=180, looseness=3] (answer4.west);
        
        \end{tikzpicture}
        \label{fig:revision}
    }
    \hfill
    \subfigure[Parallel Sampling]
    {
        \centering
        \begin{tikzpicture}
        
        \node[textnode, fill=lyyblue!40!white] (question) at (0,0) {Question};
        \node[textnode, anchor=west, fill=red!30!white] (answer1) at ([xshift=2\base, yshift=3\base]question.east) {Attempt \#1};
        \node[textnode, anchor=west, fill=ugreen!30!white] (answer2) at ([xshift=2\base, yshift=\base]question.east) {Attempt \#2};
        \node[textnode, anchor=west, fill=red!30!white] (answer3) at ([xshift=2\base, yshift=-\base]question.east) {Attempt \#3};
        \node[textnode, anchor=west, fill=red!30!white] (answer4) at ([xshift=2\base, yshift=-3\base]question.east) {Attempt \#4};

        \draw[-latex] (question.east) to[out=0, in=180] (answer1.west);
        \draw[-latex] (question.east) to[out=0, in=180] (answer2.west);
        \draw[-latex] (question.east) to[out=0, in=180] (answer3.west);
        \draw[-latex] (question.east) to[out=0, in=180] (answer4.west);
        
        \end{tikzpicture}
      \label{fig:search}
    }
    \hfill
    \subfigure[Feedback-based Test-Time Training]
    {
        \centering
        \begin{tikzpicture}
        
        \node[textnode, fill=lyyblue!40!white] (question) at (0,0) {Question};
        \node[textnode, anchor=west, fill=red!30!white] (answer1) at ([xshift=2\base, yshift=3\base]question.east) {Attempt \#1};
        \node[textnode, anchor=west, fill=red!30!white] (answer2) at ([xshift=2\base, yshift=\base]question.east) {Attempt \#2};
        \node[textnode, anchor=west, fill=red!30!white] (answer3) at ([xshift=2\base, yshift=-\base]question.east) {Attempt \#3};
        \node[textnode, anchor=west, fill=ugreen!30!white] (answer4) at ([xshift=2\base, yshift=-3\base]question.east) {Attempt \#4};

        \draw[-latex] (question.east) to[out=0, in=180] (answer1.west);
        \draw[-latex] (question.east) to[out=0, in=180] (answer2.west);
        \draw[-latex] (question.east) to[out=0, in=180] (answer3.west);
        \draw[-latex] (question.east) to[out=0, in=180] (answer4.west);
        
        \draw[-latex, densely dashed] (answer1.east) to[out=0, in=0, looseness=2] ([xshift=\base, yshift=-0.5\base]answer1.south) to ([xshift=-\base, yshift=0.5\base]answer2.north) to[out=180, in=180, looseness=2] (answer2.west);
        \draw[-latex, densely dashed] (answer2.east) to[out=0, in=0, looseness=2] ([xshift=\base, yshift=-0.5\base]answer2.south) to ([xshift=-\base, yshift=0.5\base]answer3.north) to[out=180, in=180, looseness=2] (answer3.west);
        \draw[-latex, densely dashed] (answer3.east) to[out=0, in=0, looseness=2] ([xshift=\base, yshift=-0.5\base]answer3.south) to ([xshift=-\base, yshift=0.5\base]answer4.north) to[out=180, in=180, looseness=2] (answer4.west);
        
        \end{tikzpicture}
      \label{fig:ttt}
    }
    \hspace*{\fill}
    \caption{Comparison between sequential revision, parallel sampling, and feedback-based test-time training. {\protect\tikz \protect\draw[color=black, fill=red!30!white] plot[mark=square*, mark options={scale=1.4}] (0,0);} is the failed attempt and {\protect\tikz \protect\draw[color=black, fill=ugreen!30!white] plot[mark=square*, mark options={scale=1.4}] (0,0);} is the successful attempt. {\protect\tikz {\protect\draw[-latex, thick, color=black] (0,0.5) -- (0.5,0.5);\protect\draw[opacity=0] (0,0.4) -- (0.5,0.4);}} indicates the LLM generation with the input on the left of the arrow and the output on the right. {\protect\tikz {\protect\draw[-latex, thick, densely dashed, color=black] (0,0.5) -- (0.5,0.5);\protect\draw[opacity=0] (0,0.4) -- (0.5,0.4);}} denotes the LLM training, where the left of the arrow is the training data.}
    \label{fig:compare}
\end{figure*}