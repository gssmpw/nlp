

\documentclass{article}

% Recommended, but optional, packages for figures and better typesetting:
\usepackage{microtype}
\usepackage{graphicx}
\usepackage{subfigure}
\usepackage{booktabs} % for professional tables

% hyperref makes hyperlinks in the resulting PDF.
% If your build breaks (sometimes temporarily if a hyperlink spans a page)
% please comment out the following usepackage line and replace
% \usepackage{icml2025} with \usepackage[nohyperref]{icml2025} above.
\usepackage{hyperref}
\usepackage{authblk}
\usepackage{amsmath}
\usepackage{amssymb}
\usepackage{amsfonts}
\usepackage{amstext}
\usepackage{algorithmic}
\usepackage{graphicx}
\usepackage{textcomp}
\usepackage{xcolor}
\usepackage{xspace}
\usepackage{subfigure}
\usepackage{subfloat}
\usepackage{multirow}
\usepackage{wrapfig}
\usepackage{graphicx}
\usepackage{tikz}
\usepackage{flushend}

\usepackage{enumitem}

% Attempt to make hyperref and algorithmic work together better:
\newcommand{\theHalgorithm}{\arabic{algorithm}}
\newcommand{\name}{\texttt{ADMN}\xspace}

\newcommand{\says}[3]{\todo[size=\small,color=#2,inline]{#1 says: #3}}
\newcommand\ziqi[1]{\says{Ziqi}{yellow}{#1}}
\newcommand\jason[1]{\says{Jason}{pink}{#1}}
\newcommand\mani[1]{\says{Mani}{cyan}{#1}}
\newcommand\kang[1]{\says{Kang}{lightgray}{#1}}
\newcommand{\highlight}[1]{{\color{blue}{#1}}}

% Use the following line for the initial blind version submitted for review:
%\usepackage{icml2025}

% If accepted, instead use the following line for the camera-ready submission:
\usepackage[accepted]{icml2025}

% For theorems and such
\usepackage{amsmath}
\usepackage{amssymb}
\usepackage{mathtools}
\usepackage{amsthm}

% if you use cleveref..
\usepackage[capitalize,noabbrev]{cleveref}

%%%%%%%%%%%%%%%%%%%%%%%%%%%%%%%%
% THEOREMS
%%%%%%%%%%%%%%%%%%%%%%%%%%%%%%%%
\theoremstyle{plain}
\newtheorem{theorem}{Theorem}[section]
\newtheorem{proposition}[theorem]{Proposition}
\newtheorem{lemma}[theorem]{Lemma}
\newtheorem{corollary}[theorem]{Corollary}
\theoremstyle{definition}
\newtheorem{definition}[theorem]{Definition}
\newtheorem{assumption}[theorem]{Assumption}
\theoremstyle{remark}
\newtheorem{remark}[theorem]{Remark}

% Todonotes is useful during development; simply uncomment the next line
%    and comment out the line below the next line to turn off comments
%\usepackage[disable,textsize=tiny]{todonotes}
\usepackage[textsize=tiny]{todonotes}


% The \icmltitle you define below is probably too long as a header.
% Therefore, a short form for the running title is supplied here:
\icmltitlerunning{\name: A Layer-Wise Adaptive Multimodal Network for
Dynamic Input Noise and Compute Resources}

\makeatletter
\hypersetup{
    pdfauthor={},         % Remove author metadata
    pdftitle={},          % Remove title metadata
    pdfsubject={},        % Remove subject metadata
    pdfkeywords={},       % Remove keywords
    pdfproducer={},       % No producer info
    pdfcreator={}         % No creator info
}
\pdfinfoomitdate=1        % Removes creation date metadata
\pdftrailerid{}           % Removes document ID
\pdfsuppressptexinfo=1    % Suppresses LaTeX-generated metadata
\makeatother


\begin{document}

\twocolumn[
\icmltitle{\name: A Layer-Wise Adaptive Multimodal Network for \\
Dynamic Input Noise and Compute Resources}

% It is OKAY to include author information, even for blind
% submissions: the style file will automatically remove it for you
% unless you've provided the [accepted] option to the icml2025
% package.

% List of affiliations: The first argument should be a (short)
% identifier you will use later to specify author affiliations
% Academic affiliations should list Department, University, City, Region, Country
% Industry affiliations should list Company, City, Region, Country

% You can specify symbols, otherwise they are numbered in order.
% Ideally, you should not use this facility. Affiliations will be numbered
% in order of appearance and this is the preferred way.
\icmlsetsymbol{equal}{*}
\begin{icmlauthorlist}
% \icmlauthor{Anonymous authors}{yyy}
%\icmlauthor{Anonymous authors}{}


\icmlauthor{Jason Wu}{ucla}
\icmlauthor{Kang Yang}{ucla}
\icmlauthor{Lance Kaplan}{arl}
\icmlauthor{Mani Srivastava}{ucla,amazon,note}
% %\icmlauthor{}{sch}
% \icmlauthor{Firstname8 Lastname8}{sch}
% \icmlauthor{Firstname8 Lastname8}{yyy,comp}
%\icmlauthor{}{sch}
%\icmlauthor{}{sch}
\end{icmlauthorlist}


\icmlaffiliation{ucla}{University of California, Los Angeles}
\icmlaffiliation{amazon}{Amazon}
\icmlaffiliation{arl}{US Army DEVCOM Army Research Laboratory}
\icmlaffiliation{note}{Mani Srivastava holds concurrent appointments as an Amazon Scholar and a Professor at UCLA, but the work in this paper is unrelated to Amazon}

\icmlcorrespondingauthor{Jason Wu}{jaysunwu@g.ucla.edu}
% \icmlcorrespondingauthor{Firstname2 Lastname2}{first2.last2@www.uk}

% You may provide any keywords that you
% find helpful for describing your paper; these are used to populate
% the "keywords" metadata in the PDF but will not be shown in the document
% \icmlkeywords{Machine Learning, ICML}

\vskip 0.3in

]

% this must go after the closing bracket ] following \twocolumn[ ...

% This command actually creates the footnote in the first column
% listing the affiliations and the copyright notice.
% The command takes one argument, which is text to display at the start of the footnote.
% The \icmlEqualContribution command is standard text for equal contribution.
% Remove it (just {}) if you do not need this facility.

\printAffiliations{}  % leave blank if no need to mention equal contribution
% \printAffiliationsAndNotice{\icmlEqualContribution} % otherwise use the standard text.


%\mani{Be more explicit about what factors can cause the changes in compute (hardware platforms w thermal throttling, power availability, latency requirements, faults, and noise with environmental + sensor failure. Focus on root cause rather than downstream application - autonomous driving has redundancy and may not be a good example}

%\mani{Be careful about fidelity vs noise vs quality}
\begin{abstract}

Multimodal deep learning systems are deployed in dynamic scenarios due to the robustness afforded by multiple sensing modalities. Nevertheless, they struggle with varying compute resource availability~(due to multi-tenancy, device heterogeneity, etc.) and fluctuating quality of inputs~(from sensor feed corruption, environmental noise, etc.). Current multimodal systems employ static resource provisioning and cannot easily adapt when compute resources change over time. Additionally, their reliance on processing sensor data with fixed feature extractors is ill-equipped to handle variations in modality quality. Consequently, uninformative modalities, such as those with high noise, needlessly consume resources better allocated towards other modalities. We propose \name, a layer-wise \textbf{A}daptive \textbf{D}epth \textbf{M}ultimodal \textbf{N}etwork capable of tackling both challenges - it adjusts the total number of active layers across all modalities to meet compute resource constraints, and continually reallocates layers across input modalities according to their modality quality. 
Our evaluations showcase \name can match the accuracy of state-of-the-art networks while reducing up to 75\% of their floating-point operations.









\end{abstract}


In recent years, there has been a notable increase in the development and research of tethered UAVs, reflecting a growing interest in their diverse applications. One of the main motivations is to carry out long-term missions with aerial vehicles, as these present significant challenges due to the limitations of current battery solutions \cite{robotics12040117}. A UAV tethered to a UGV is an interesting configuration, as the UGV can power the UAV through the tether for longer times given the higher payload of the former.  %According to this, an interesting configuration to allow long-duration flights of a UAV is a tethered robot configuration in which a UGV is tied to the UAV and powering it. 
This introduces a paradigm in robotic collaboration, offering distinct advantages over traditional standalone systems by combining the strengths of each of the robotic agents \cite{MooreIROS2018}. %As we venture UGV tied to UAV into scenarios requiring heightened enhanced situational awareness involving an extended operational endurance, the tethered approach proves invaluable, due to the capability to provide energy to the UAV, thus increasing fly time \cite{6961531}. 
When deploying a UGV tethered to a UAV in scenarios requiring increased situational awareness and extended operational endurance, the tethered configuration can become even more invaluable, not only providing the UAV with power to significantly extend its flight time \cite{6961531},  %In this way, the cable plays an important role in providing 
but also with safe high-bandwidth communications \cite{850822,9202196}. 

However, the tethering mechanism introduces several challenges, particularly in modeling the hanging tether state \cite{XiaoSSRR2018}. Unlike standalone systems, where each vehicle operates independently, the tether requires intricate and permanent coordination between the UGV and the UAV. Understanding and managing the state of the tether becomes a critical aspect, which requires sophisticated algorithms and real-time processing capabilities \cite{9561062}. 

\begin{figure}
  \includegraphics[width=0.2\textwidth]{Figures/setup1.png}
  \hfill
  \includegraphics[width=0.2\textwidth]{Figures/setup2.png}
  \caption{Simplified 2D sketch showing an example for motion planning of a tethered UAV-UGV with a hanging tether. (Left) Initial robots and tether configuration, and UAV goal (red circle). (Right) Sequence of robots positions and tether length to reach the given goal. Notice how the goal cannot be reached by means of a taut tether, a hanging tether must be considered in this case.}
  \label{fig:planning-setup}
\end{figure}

The state of the tether has traditionally been analyzed through parameterization, an approach that employs equations to represent its physical behavior, especially the catenary curve \cite{BOOKOFCURVES}. Numerous methodologies, with the aim of simplifying this process, approximate the tether as a straight line \cite{autonomousvisual}\cite{framworktether}\cite{uavfire}. This straight-line approximation is only suitable in scenarios where there is a direct line of sight between the tether endpoints, and thus it inherently restricts the exploratory range of the UAV.

In general, hanging-tether approaches allow UAVs to access a broader range of areas compared to straight tether setups; see Fig. \ref{fig:planning-setup} for an example. This concept has been explored by incorporating tether parameterization into localization or planning processes. For instance, Lima and Pereira \cite{9476778} use the catenary equation to determine the UAV's position.  % This concept has been explored by incorporating tether parameterization into the localization or planning processes, such as in the work conducted by Lima and Pereira \cite{9476778}, where using the catenary equation is feasible to find the UAV position. 
Similarly, in \cite{9364354}, the focus is on computing the state of a catenary tether to localize two UAVs attached at each end. This setup is specifically designed to suspend an object, providing a novel approach to object manipulation using UAVs while maintaining a constant tether length. Another interesting application of the catenary model is presented in \cite{LARANJEIRA2020107018} for underwater operations, where the catenary is used to monitor the status of a cable connected to an \emph{N}-number of ROVs (Remotely Operated Vehicles) performing exploration tasks, also with a constant tether length.

In \cite{8848946}, the parameterization of the tether is used in the localization and control stages to perform two autonomous motion primitives, reactive feedback-based position control and model-predictive feedforward velocity control, but is not used in the planning stage. An interesting approach is presented in \cite{drones7020073}, where a tied unmanned aerial vehicle (TUAV), named ``Oxpecke'', was designed for the inspection of stone-mine pillars. This system uses a sweeping (lawnmower) pattern path planning method intended to map and inspect an entire rectangular area, such as the surface of a pillar. However, the surface to inspect is simple (a rectangle), and the tether length is not directly included in the path planning.

%A general approach about the consideration of the tether in the planning stage is introduced in \cite{battocletti2024entanglementdefinitionstetheredrobots}, where the authors present the definition of tether entanglement problems. Specifically, it addresses the challenges posed by the presence of a tether, including the geometric constraints on the robot's motion due to the finite tether length. For that, different constraints are considered in the planning stage. However, the method is too general and mainly tested in ground points, so UAV implementation are not considering, and algo this method allow tether contact with the floor while entanglement desnt exit.

A comprehensive approach to incorporate a tether in the planning stage is presented in \cite{battocletti2024entanglementdefinitionstetheredrobots}, where the authors define the challenges associated with tether entanglement. Specifically, this work addresses the constraints imposed by the tether on the motion of the robot, particularly the limitations arising from the finite length of the tether. Various constraints are integrated into the planning stage to account for these challenges. However, the proposed method is limited and mainly focused on ground applications, 
thus limiting its applicability to UAVs. Additionally, the approach allows for tether contact with the ground, as long as it does not result in entanglement.

On the other hand, \cite{capitán2024efficientstrategypathplanning} focuses on the development of a path planning strategy for marsupial robotic systems composed of a UGV tethered to a UAV. The article introduces a sequential planning strategy called MASPA (Marsupial Sequential Path-Planning Approach), which allows calculating collision-free 3D trajectories for the tethered UAV-UGV system in complex scenarios, for which the UGV advances to a point where the UAV executes the take-off and then advances to a desired point. This method considers both the geometric limitations imposed by obstacles and the cable and the properties of the joint motion of both robots. A novel algorithm, the PVA (Polygonal Visibility Algorithm), is also presented to identify feasible take-off points and solve visibility problems for the UAV in a three-dimensional space. Despite the novelty of the approach, it is not able to consider coordinated planning of the UGV and the UAV at the same time.

In \cite{smartinezr2023}, the catenary approximation is used to parameterize the state of the tether and plan a collision-free trajectory, in which the UAV must achieve objectives using a hanging tether. However, using the catenary equation, the planning process becomes a time-consuming task, allowing only offline computations. %which makes the planning process to be carried out offline.

%This paper will focus on reducing the complexity associated with the calculation of the variable length hanging tether. We will propose an approach that efficiently calculates the tether state with a minimum representation error concerning the real state, and integrating it into a trajectory planning algorithm for a UGV-UAV tethered team. To this end, we test our approach in the motion planning method for a mobile UGV-UAV tethered system presented in \cite{smartinezr2023}, which is based on two stages. The first stage computes a free-collision path planning for UAV, UGV, and tether, using the RRT* algorithm. The second stage corresponds to a trajectory planning method based on nonlinear optimization that considers smoothness, speed, acceleration limitations of the UGV and UAV, and optimizes the tether configuration to maximize the distance from obstacles. %Unfortunately, considering the real catenary curve in the planner could make it computationally demanding, as shown in our previous work \cite{martinez2021optimization}. In it, we manage to design, implement and test in experiments a two-step optimized planner which considers the catenary shape. For this reason, we propose to approximate the shape of the tether as a parabola without affecting the safety of the planning system and making use of its simpler description to speed-up the computation of optimal paths.

%Our approach is based on the motion planning mentioned above due to the robustness of computed trajectories. Thus, we include in the first stage, a decision problem to set the initial tether length, to quickly obtain a collision-free state for the whole system. Furthermore, we propose a new planner-state parameterization and replace the use of the catenary equation with a parabola equation for estimating the shape of the tether. Thus, the main contributions of the article are:
%\begin{itemize}
%
%\item In Planner Stage: Solving the decision problem to find a collision-free parabola curve instead of the traditional catenary curve. This change allows the RRT* (Rapidly-exploring Random Tree) planner to calculate trajectories faster and more efficiently, since it avoids the computational complexity associated with the calculation of the catenary. The parabolic curve simplifies the collision decision process and increases the success rate in three-dimensional environments with obstacles.
%
%\item  In Optimizer stage: This stage introduces a direct parameterization of the tether in the trajectory state function, which includes the parameters of the curve (parabola or catenary) in the system state vector. This allows a more accurate evaluation of geometric constraints (such as distance to obstacles) and reduces the optimization time by up to an order of magnitude compared to previous methods, achieving safer and smoother trajectories for the UAV-UGV system.
%\end{itemize}

This paper focuses on reducing the complexity associated with the calculation of the variable length hanging tether. %The paper builds on the previous work of the authors \cite{smartinezr2023}, extending it with a new approach that efficiently calculates the tether state with a minimum representation error related to the actual state and a new parameterization of the tether curve in the trajectory optimizer for faster computation. Thus, 
The main contributions are listed below.

\begin{itemize}
    \item A new method for efficient computation of a collision-free catenary curve based on the parabola approximation. This paper proposes using the parabola curve to model the hanging tether curve, detailing the full pipeline, including the computation of the final catenary model. This method reduces the execution time of the path planner to great extent, since it avoids the computational complexity associated with the calculation of the catenary model for tether collision detection. This model also increases the feasibility of the trajectory planner approach, reaching an averaged 98\% of feasibility in the validation scenarios. 

    \item A direct parameterization of the tether in the trajectory state definition, which includes the parameters of the curve (parabola or catenary) in the system state vector. This allows a more accurate evaluation of geometric constraints (such as distance to obstacles) and reduces the optimization time \rev{by more than an order of magnitude} compared to previous methods, achieving safer and smoother trajectories for the UAV-UGV system. \rev{Such improvement opens the door to apply the proposed method to real-time local re-planning.}
\end{itemize}

%The experimental results will show how this new parameterization boosts the computation, while the parabola model will clearly improve the feasibility of the method over the catenary. 

%Thus, we include in the first stage, a decision problem to set the initial tether length, to quickly obtain a collision-free state for the whole system. Additionally, we replace the traditional catenary equation with a parabolic approximation to estimate the tether shape more efficiently. In the second stage of nonlinear optimization stage, we further simplify the process by parameterizing the tether instead of relying on the catenary model. This approach not only streamlines the representation of the curve but also facilitates more straightforward and efficient gradient calculations during optimization.

The paper is structured as follows. In Section \ref{sec:overview}, we show the general problem to be solved, whereas Section \ref{sec:approach} formalizes the solutions proposed. Section \ref{sec:path_planning} details the implementation of the solution within the planning stage. In Section \ref{sec:optimization_process}, we describe how curve parameterization is utilized to enhance the optimization process for trajectory computation. The experimental results are discussed in Section \ref{sec:experiments}. Finally, the paper is concluded in Section \ref{sec:conclusions}.

\section{Related Work}

% \iffalse
% \subsection{Dynamic Neural Networks} 
% Talk about neural networks that modify their characteristics based on an external input or characteristics of the input. Have a paragraph devoted to early-exit neural networks

% Early exit drawbacks: Early exit looks at if we can exit at one layer, we generate the entire set of pass or not at the beginning, so we know which layers are active. This can result in more efficient data batching, batch data that has same layer allocation and can be executed similarly. Can't do this with early exit.

% I can also propose dynamic batching, where we can batch data that has same layer allocation and can be executed similarly. Could also be a contribution

% \subsection{Reinforcement Learning Techniques}
% How has RL been used to solve this problem in the unimodal setting?


% \fi

\textbf{Early Exiting in Unimodal Networks.}
Early Exiting has been explored extensively in unimodal networks to improve inference efficiency~\cite{xin2020deebert, neurips2020_d4dd111a, meng2022adavit}. Methods like DeeBERT~\cite{xin2020deebert} and PABEE~\cite{neurips2020_d4dd111a} use confidence thresholds to halt computation for simpler inputs. However, these unimodal approaches fail to consider multimodal challenges such as QoI-aware resource allocation. Moreover, Early Exiting is incompatible with redistributing resources away from low-QoI modalities, as these low-confidence samples propagate through the entire network.



\begin{figure*}
    \centering
    \includegraphics[width=1\linewidth]{Figures/ADMN_Architecture.png}
    \vspace{-0.2in}
    % \caption{\name architecture. Gray boxes indicate a dropped layer, blue indicates a frozen layer, and red indicates a tunable layer. TE: transformer encoder.}
\caption{\name architecture. \textcolor{gray}{[Gray box]}: dropped layer, \textcolor{blue}{[Blue box]}: frozen layer, \textcolor{red}{[Red box]}: tunable layer. TE: Transformer Encoder.}
    \label{fig:admn_architecture}
    \vspace{-0.1in}
\end{figure*}

\textbf{Dynamic Inference for Multimodal Systems.}
Multimodal networks traditionally rely on input-agnostic static provisioning, resulting in inefficiencies when modality QoI varies, and also incompatibility with dynamic computational resource availability. 
To enhance computational efficiency, dynamic networks have been proposed~\cite{xue2023dynamic, panda2021adamml, gao2020listen,cai2024ACF, mullapudi2018hydranets}. 
DynMM~\cite{xue2023dynamic} trains a set of expert networks representing different modality combinations, processing simple inputs with a subset of available modalities. 
AdaMML~\cite{panda2021adamml} and Listen to Look~\cite{gao2020listen} improve inference efficiency by leveraging multimodal information to eliminate temporal redundancy in videos. ACF~\cite{cai2024ACF} dynamically replaces certain modules with lightweight networks according to the input for greater efficiency in edge devices. 
Despite these advances, these existing methods (1) overlook significant QoI variations arising from input noise, (2) utilize or discard entire modalities without fine-grained control, and (3) fail to consider \emph{fixed} resource budgets. In contrast, \name allocates a fixed number of layers among modalities in a fine-grained manner according to input QoI, which accounts for both relative modality importance and noise.  



% Channel exchanging~\cite{wang2020deep} focuses on inter-modality interactions to enhance feature integration, emphasizing representation quality over computational savings. 
% E2E-VLP~\cite{xu2021e2e} improves robustness in noisy conditions via vision-language pretraining, indirectly boosting efficiency. 
% HydraNets~\cite{mullapudi2018hydranets} generalize dynamic architectures by adapting computation paths based on input complexity, offering foundational insights for multimodal systems. 

% Despite these advances, existing methods underutilize complementary and redundant information across modalities and fail to optimize backbone depth and layer allocation, limiting their applicability under stringent efficiency and robustness requirements. 
% In contrast, our work introduces a novel fidelity-aware mechanism that dynamically allocates computational resources to modality backbones, explicitly optimizing layer depth and resource distribution at the backbone level. Moreover, Early-Exit techniques reduce computation in the \emph{average case} and are not capable of meeting fixed compute budgets. 




% \textbf{Adaptive Computation in Vision Transformers.}
% Adaptive computation has been explored in Transformer architectures~\cite{wu2018blockdrop, meng2022adavit}. 
% BlockDrop utilizes reinforcement learning to dynamically select residual blocks, while AdaViT~\cite{meng2022adavit} employs lightweight decision networks to adaptively choose patches, attention heads, and layers within Vision Transformers. 
% Although these methods achieve high efficiency, they are designed for unimodal tasks and fail to address the unique challenges of multimodal networks, such as cross-modal interactions and dynamic resource allocation. 
% Our work builds upon these ideas, introducing adaptive mechanisms tailored for multimodal systems that optimize computation across both modalities and backbone layers, enabling efficient and robust cross-modal inference.




\section{MMTEB Construction}

\subsection{Open science effort}
\label{sec:open-source-effort}
To ensure the broad applicability of MMTEB across various domains, we recruited a diverse group of contributors. We actively encouraged participation from industry professionals, low-resource language communities, and academic researchers. To clarify authorship assignment and recognize desired contributions, we implemented a point-based system, similar to \citet{lovenia2024seacrowd}.
To facilitate transparency, coordination was managed through GitHub. 
A detailed breakdown of contributors and the point system can be found in Appendix~\ref{sec:contributions}.

\subsection{Ensuring task quality}

To guarantee the quality of the added tasks,\footnote{A task includes a dataset and an implementation for model evaluation.} each task was reviewed by at least one of the main contributors. In addition, we required task submissions to include metadata fields. These fields included details such as annotation source, dataset source, license, dialects, and citation information. Appendix~\ref{appendix:task_metadata} provides a comprehensive description of each field. 

Furthermore, we ensured that the performance on submitted tasks fell within a reasonable range to avoid trivially low or unrealistically high performance. Therefore, we required two multilingual models to be run on the task; multilingual-e5-small
% \footnote{\url{https://huggingface.co/intfloat/multilingual-e5-small}}
~\citep{wang2022text} and MiniLM-L12
% \footnote{\url{https://huggingface.co/sentence-transformers/paraphrase-multilingual-MiniLM-L12-v2}}
~\citep{reimers2019sentencebert}.
A task was examined further if the models obtained scores close to a random baseline (within a 2\% margin), a near-perfect score, or if both models obtained roughly similar scores. 
% Similarly, if the two models obtained roughly similar scores. 
These tasks were examined for flawed implementation or poor data quality. Afterwards, a decision was made to either exclude or include the task. We consulted with contributors who are familiar with the target language whenever possible before the final decision. A task could be included despite failing these checks. For example, scores close to the random baseline might be due to the task's inherent difficulty rather than poor data quality.

\subsection{Accessibility and benchmark optimization}
\label{sec:benchmark-optimization}

As detailed in \autoref{sec:intro}, extensive benchmark evaluations often require significant computational resources. This trend is also observed in \texttt{MTEB(eng, v1)} \citep{muennighoff2023mteb}, where running moderately sized LLMs can take up to two days on a single A100 GPU. Accessibility for low-resource communities is particularly important for MMTEB, considering the common co-occurrence of computational constraints \citep{ahia-etal-2021-low-resource}. 

Below, we discuss three main strategies implemented to make our benchmark more efficient.  We additionally elaborate further code optimization in Appendix~\ref{sec:appendix-code-optimizations}.

\subsubsection{Downsampling and caching embeddings} 
The first strategy involves optimizing the evaluation process by downsampling datasets and caching embeddings. Encoding a large volume of documents for tasks such as retrieval and clustering can be a significant bottleneck in evaluation. Downsampling involves selecting a representative subset of the dataset and reducing the number of documents that require processing. Caching embeddings prevents redundant encoding by using already processed documents.

\paragraph{Clustering.} In MTEB, clustering is evaluated by computing the v-measure score \citep{rosenberg-hirschberg-2007-v} on text embeddings clustered using k-means. This process is repeated over multiple distinct sets, inevitably resulting in a large number of documents being encoded. To reduce this encoding burden, we propose a bootstrapping approach that reuses encoded documents across sets. We first encode a 4\% subsample of the corpus and sample 10 sets without replacement. Each set undergoes k-means clustering, and we record performance estimates. For certain tasks, this approach reduces the number of documents encoded by 100$\times$. In Appendix \ref{sec:task-construction}, we compare both approaches and find an average speedup of 16.11x across tasks, while preserving the relative ranking of models (Average Spearman correlation: 0.96).

\paragraph{Retrieval.} A key challenge in retrieval tasks is encoding large document collections, which can contain millions of entries \cite{nguyenhendriksen2024multimodal}. To maintain performance comparable to the original datasets while reducing the collection size, we adopted the TREC pooling strategy \citep{buckley2007bias,soboroff2003building}, which aggregates scores from multiple models to select representative documents.\footnote{We utilized a range of models: BM25 for lexical hard negatives, e5-multilingual-large as a top-performing BERT-large multilingual model, and e5-Mistral-Instruct 7B, the largest model leveraging instruction-based data.}  For each dataset, we retained the top 250 ranked documents per query, a threshold determined through initial tests that showed negligible differences in absolute scores and no changes in relative rankings across representative models (see Appendix~\ref{app:retrieval_downsample} for details on downsampling effects). These documents are merged to form a smaller representative collection. For datasets exceeding 1,000 queries, we randomly sampled 1,000 queries, reducing the largest datasets from over 5 million documents to a maximum of 250,000. This approach accelerated evaluation while preserving ranking performance.

\paragraph{Bitext Mining.} We apply similar optimization to bitext mining tasks. Some datasets, such as Flores \citep{nllb2022flores} share the same sentences across several language pairs (e.g., English sentences are the same in the English-Hindi pair and the English-Bosnian pair). By caching the embeddings, we reduce the number of embedding computations, making it linear in the number of languages instead of quadratic. For the English documents within Flores this results in a reduction of documents needed to be embedded from ~410,000 in \texttt{MTEB(eng, v1)} to just 1,012 in our benchmark.

\subsubsection{Encouraging smaller dataset submissions} 
\label{sec:smaller-dataset-submissions}
The second strategy focused on encouraging contributors to downsample datasets before submission. To achieve this, we used a stratified split based on target categories. This helped us to ensure that the downsampled datasets could effectively differentiate between candidate models. To validate the process, we compared scores before and after downsampling. For details, we refer to Appendix~\ref{sec:speedup}.

\subsubsection{Task Selection}
\label{sec:taskselection}

To further reduce the computation overhead we seek to construct a task subset that can reliably predict task scores outside the subset.

For task selection, we followed an approach inspired by \citet{Xia2020PredictingPerformance}. We seek to estimate the model $m_i \in M$ scores $s_{t, m_i}$ on an unobserved task $t$ based on scores on observed tasks $s_{j, m_k} \in S, j \neq t$. This allows us to consider the performance of tasks as features within a prediction problem. Thus we can treat task selection as feature reduction, a well-formulated task within machine learning. Note that this formulation allows us to keep the unobserved task arbitrary, representing generalization to unseen tasks \citep{cholletMeasureIntelligence2019}. We used a backward selection method, where one task is left out to be predicted, an estimator\footnote{We use the term ``estimator" to differentiate between the evaluated embedding model. For our estimator, we use linear regression.}
is fitted on the performance of all models except one, and the score of the held-out model is predicted. This process is repeated until predicted scores are generated for all models on all tasks.
% We used a backward selection method, where one model-task pair is left out to be predicted. An estimator\footnote{We use the term "estimator" to differentiate between the evaluated embedding model. For our estimator, we use linear regression.} is fitted on the performance scores of all other model-task pairs, and the score for the held-out pair is predicted. This process is repeated until predicted scores are generated for all models across all tasks.
The most predictable task is then removed, leaving the estimators in the task subset group. Optionally, we can add additional criteria to ensure task diversity and language representation. Spearman's rank correlation was chosen as the similarity score, as it best preserved the relative ranking when applied to the \texttt{MTEB(eng, v1)}.


\subsection{Benchmark construction}
\label{sec:benchmarkconstruction}
From the extensive collection of tasks in MMTEB, we developed several representative benchmarks, including a highly multilingual benchmark, \texttt{MTEB(Multilingual)}, as well as regional geopolitical benchmarks, \texttt{MTEB(Europe)} and \texttt{MTEB(Indic)}. Additionally, we introduce a faster version of \texttt{MTEB(eng, v1)} \citep{muennighoff2023mteb}, which we refer to as \texttt{MTEB(eng, v2)}. MMTEB also integrates domain-specific benchmarks like CoIR for code retrieval \citep{li2024coircomprehensivebenchmarkcode} and LongEmbed for long document retrieval  \citep{zhu2024longembed}. MMTEB also introduces language-specific benchmarks, extending the existing suite that includes Scandinavian \citep{enevoldsen2024scandinavian}, Chinese \citep{xiao2024cpack}, Polish \citep{poswiata2024plmteb}, and French \citep{ciancone2024extending}. For an overview of the benchmarks, we refer to Appendix~\ref{sec:benchmark-creation}.

In the following section, we detail a methodology that we designed to create more targeted and concise benchmarks. This methodology includes: 1) clearly defining the initial scope of the benchmark \textbf{(Initial Scope)}, 2) reducing the number of tasks by iterative task selection tasks based on intertask correlation \textbf{(Refined Scope)}, and 3) performing a thorough manual review \textbf{(Task Selection and Review)}. We provide an overview in \autoref{tab:numberoftasks}.

In addition to these benchmarks, we provide accompanying code to facilitate the creation of new benchmarks, to allow communities and companies to create tailored benchmarks. In the following, we present \texttt{MTEB(Multilingual)} and \texttt{MTEB(eng, v2)} as two example cases. For a comprehensive overview of benchmark construction and the tasks included in each benchmark, we refer to Appendix~\ref{sec:appendix-benchmark-overview}.
\newline




\begin{table}
\centering
{\footnotesize
    \begin{tabular}{lcccc}
\toprule
\textbf{Benchmark} & \textbf{Initial Scope}  & \textbf{Refined Scope} & \textbf{Task Selection and Review} \\
\midrule
\texttt{MTEB(Multilingual)} & >500 & 343 & 132 \\
\texttt{MTEB(Europe)} & 420 & 228 & 74 \\
\texttt{MTEB(Indic)} & 55 & 44 & 23 \\
\texttt{MTEB(eng, v2)} & 56 & 54 & 41 \\
\bottomrule
    \end{tabular}
}
    \caption{Number of tasks in each benchmark after each filtering step. The initial scope includes tasks relevant to the benchmark goal, notably language of interest. The refined scope further reduced the scope, e.g. removing datasets with underspecified licenses.}
    \label{tab:numberoftasks}
    \vspace{-3mm}
\end{table}

\noindent
\header{MTEB(Multilingual)}:
We select all available languages within MMTEB as the initial scope of the benchmark. This results in 550 tasks. We reduce this selection by removing machine-translated datasets, datasets with under-specified licenses, and highly domain-specific datasets such as code-retrieval datasets. This results in 343 tasks covering $>$250 languages. Following this selection, we evaluate this subset using a representative selection of models (See Section~\ref{sec:models}) and apply task selection to remove the most predictable tasks. To ensure language diversity and representation across task categories, we avoid removing a task that would eliminate a language from the respective task category. Additionally, we did not remove a task if the mean squared error between predicted and observed scores exceeded 0.5 standard deviations. This is to avoid inadvertantly overindexing to easier tasks. The process of iterative task removal (Section~\ref{sec:taskselection}) is repeated until the most predictable held-out task obtained a Spearman correlation of less than 0.8 between predicted and observed scores, or if no tasks were available for filtering. This results in a final selection of 131 diverse tasks. Finally, the selected tasks were reviewed, if possible, by contributors who spoke the target language. If needed, the selection criteria were updated, and some tasks were manually replaced with higher-quality alternatives. 
\newline

\noindent
\header{MTEB(eng, v2)}:
Unlike the multilingual benchmarks which target a language group, this benchmark is designed to match \texttt{MTEB(eng, v1)}, incorporating computational efficiencies (see Section~\ref{sec:benchmark-optimization}) and reducing the intertask correlation using task selection. To prevent overfitting, we intend it as a zero-shot benchmark, excluding tasks like MS MARCO \citep{NguyenRSGTMD16} and Natural Questions \citep{kwiatkowski2019natural}, which are frequently used in fine-tuning.

We start the construction by replacing each task with its optimized variant. This updated set obtains a Spearman correlation of $0.97$, $p<.0001$ (Pearson $0.99$, $p<.0001$) with \texttt{MTEB(eng, v1)} using mean aggregation for the selected models  (see \autoref{sec:models}).
The task selection process then proceeds similarly to \texttt{MTEB(Multilingual)}, ensuring task diversity by retaining a task if its removal would eliminate a task category. Tasks, where the mean squared error between predicted and observed performance exceeds 0.2 standard deviations, are also retained. This process continues until the most predictable held-out task yields a Spearman correlation below 0.9 between predicted and observed scores. The final selection consists of 41 tasks. We compare this with \texttt{MTEB(eng, v1)} \citep{muennighoff2023mteb} in Section~\ref{sec:mteb_english_vs_lite}.

\section{Experimental Settings}

\subsection{Models} 
\label{sec:models}

We select a representative set of models, focusing on multilingual models across various size categories. We benchmark the multilingual LaBSE \citep{feng-etal-2022-language}, trained on paraphrase corpora, English and multilingual versions of MPNet \citep{song2020mpnet}, and MiniLM \citep{wang-etal-2021-minilmv2} model, trained on diverse datasets. We also evaluate the multilingual e5 series models \citep{wang2024multilingual, wang2022text} trained using a two-step approach utilizing weak supervision. Additionally, to understand the role of scale as well as instruction finetuning, we benchmark GritLM-7B \citep{muennighoff2024generative} and e5-multilingual-7b-instruct \citep{wang2023improving}, which are both based on the Mistral 7B model \citep{jiang2023mistral}.

Revision IDs, model implementation, and prompts used are available in \autoref{sec:appendix-models}. We ran the models on all the implemented tasks to encourage further analysis of the model results.
Results, including multiple performance metrics, runtime, CO2 emissions, model metadata, etc., are publicly available in the versioned results repository.\footnote{\url{https://github.com/embeddings-benchmark/results}.}

\subsection{Evaluation Scores}
For our performance metrics, we report average scores across all tasks, scores per task category, and weighted by task category. We compute model ranks using the Borda count method \citep{NEURIPS2022_ac4920f4}, derived from social choice theory. This method, which is also employed in election systems based on preference ranking, has been shown to be more robust for comparing NLP systems. To compute this score, we consider each task as a preference voter voting for each model, and scores are aggregated according to the Borda Count method. In the case of ties, we use the tournament Borda count method.

\subsection{Multilingual performance} 

While MMTEB includes multiple benchmarks (see Appendix~\ref{sec:benchmark-creation}), we select three multilingual benchmarks to showcase. These constitute a fully multilingual benchmark \texttt{MTEB(Multilingual)} and two targeting languages with varying levels of resources: \texttt{MTEB(Europe)} and \texttt{MTEB(Indic)}. The performance of our selected models on these tasks can be seen in \autoref{tab:overall-performance}.
For performance metrics per task, across domains, etc., we refer to \autoref{sec:fullresults}. 
\begin{figure}
    \centering
    \includegraphics[width=0.95\linewidth]{figures/performance-x-parameters.pdf}
\caption{Mean performance across tasks on MTEB(Multilingual) according to the number of parameters. The circle size denotes the embedding size, while the color denotes the maximum sequence length of the model. To improve readability, only certain labels are shown. We refer to the public leaderboard
%\footnote{https://huggingface.co/spaces/mteb/leaderboard} 
for interactive visualization. We see that the notably smaller model obtains comparable performance to Mistral 7B and GritLM-7B, note that these overlap in the figure due to the similarity of the two models.}
    \label{fig:performance-x-speed}
    \vspace{-3mm}
\end{figure}


\begin{table*}[!th]
\centering
\resizebox{\textwidth}{!}{  
\setlength{\tabcolsep}{1pt}
{\footnotesize
\begin{tabular}{llcc|cccccccc}
\toprule
& \multicolumn{1}{c}{\textbf{Rank}  ($\downarrow$)} &  \multicolumn{2}{c}{\textbf{Average Across}} & \multicolumn{7}{c}{\textbf{Average per Category}} \\
\cmidrule(r){2-2} \cmidrule{3-4} \cmidrule(l){5-12}
\textbf{Model} ($\downarrow$) & Borda Count & All & \multicolumn{1}{r}{Category}  & \multicolumn{1}{c}{Btxt} & Pr Clf  & Clf & STS & Rtrvl & M. Clf  & Clust & Rrnk \\
\midrule
\multicolumn{12}{c}{\vspace{2mm} \normalsize \texttt{MTEB(Multilingual)}} \\
\textcolor{gray}{Number of datasets ($\rightarrow$) } & \textcolor{gray}{(132)} & \textcolor{gray}{(132)} & \multicolumn{1}{c}{\textcolor{gray}{(132)}} &   \multicolumn{1}{c}{\textcolor{gray}{(13)}} &   \textcolor{gray}{(11)}  &   \textcolor{gray}{(43)}  &   \textcolor{gray}{(16)}  &   \textcolor{gray}{(18)}  &   \textcolor{gray}{(5)}  &   \textcolor{gray}{(17)}  &   \textcolor{gray}{(6)} \\
\midrule
multilingual-e5-large-instruct & 1 (1375) & \textbf{63.2} & \textbf{62.1} & \textbf{80.1} & 80.9 & \textbf{64.9} & \textbf{76.8} & 57.1 & \textbf{22.9} & \textbf{51.5} & 62.6 \\
GritLM-7B & 2 (1258) & 60.9 & 60.1 & 70.5 & 79.9 & 61.8 & 73.3 & \textbf{58.3} & 22.8 & 50.5 & \textbf{63.8} \\
e5-mistral-7b-instruct & 3 (1233) & 60.3 & 59.9 & 70.6 & 81.1 & 60.3 & 74.0 & 55.8 & 22.2 & 51.4 & \textbf{63.8} \\
multilingual-e5-large & 4 (1109) & 58.6 & 58.2 & 71.7 & 79.0 & 59.9 & 73.5 & 54.1 & 21.3 & 42.9 & \textbf{62.8} \\
multilingual-e5-base & 5 (944) & 57.0 & 56.5 & 69.4 & 77.2 & 58.2 & 71.4 & 52.7 & 20.2 & 42.7 & 60.2 \\
multilingual-mpnet-base & 6 (830) & 52.0 & 51.1 & 52.1 & \textbf{81.2} & 55.1 & 69.7 & 39.8 & 16.4 & 41.1 & 53.4 \\
multilingual-e5-small & 7 (784) & 55.5 & 55.2 & 67.5 & 76.3 & 56.5 & 70.4 & 49.3 & 19.1 & 41.7 & 60.4 \\
LaBSE & 8 (719) & 52.1 & 51.9 & 76.4 & 76.0 & 54.6 & 65.3 & 33.2 & 20.1 & 39.2 & 50.2 \\
multilingual-MiniLM-L12 & 9 (603) & 48.8 & 48.0 & 44.6 & 79.0 & 51.7 & 66.6 & 36.6 & 14.9 & 39.3 & 51.0 \\
all-mpnet-base & 10 (526) & 42.5 & 41.1 & 21.2 & 70.9 & 47.0 & 57.6 & 32.8 & 16.3 & 40.8 & 42.2 \\
all-MiniLM-L12 & 11 (490) & 42.2 & 40.9 & 22.9 & 71.7 & 46.8 & 57.2 & 32.5 & 14.6 & 36.8 & 44.3 \\
all-MiniLM-L6 & 12 (418) & 41.4 & 39.9 & 20.1 & 71.2 & 46.2 & 56.1 & 32.5 & 15.1 & 38.0 & 40.3 \\
\midrule
\multicolumn{12}{c}{\vspace{2mm} \normalsize \texttt{MTEB(Europe)}} \\
\textcolor{gray}{Number of datasets ($\rightarrow$) } & \textcolor{gray}{(74)} & \textcolor{gray}{(74)} & \multicolumn{1}{c}{\textcolor{gray}{(74)}} &   \multicolumn{1}{c}{\textcolor{gray}{(7)}} &   \textcolor{gray}{(6)}  &   \textcolor{gray}{(21)}   &   \textcolor{gray}{(9)}  &   \textcolor{gray}{(15)} &   \textcolor{gray}{(2)}  &   \textcolor{gray}{(6)} &   \textcolor{gray}{(3)}  \\
\midrule
GritLM-7B & 1 (757) & \textbf{63.0} & \textbf{62.7} & \textbf{90.4} & 89.9 & \textbf{64.7} & 76.1 & \textbf{57.1} & \textbf{17.6} & 45.3 & \textbf{60.3} \\
multilingual-e5-large-instruct & 2 (732) & 62.2 & 62.3 & 90.4 & 90.0 & 63.2 & \textbf{77.4} & 54.8 & 17.3 & \textbf{46.9} & 58.4 \\
e5-mistral-7b-instruct & 3 (725) & 61.7 & 61.9 & 89.6 & \textbf{91.2} & 62.9 &  76.5 & 53.6 & 15.5 & 46.5 & 59.8 \\
multilingual-e5-large & 4 (586) & 58.5 & 58.7 & 84.5 & 88.8 & 60.4 & 75.8 & 50.8 & 15.0 & 38.2 & 55.9 \\
multilingual-e5-base & 5 (499) & 57.2 & 57.5 & 84.1 & 87.4 & 57.9 & 73.7 & 50.2 & 14.9 & 38.2 & 53.9 \\
multilingual-mpnet-base & 6 (463) & 54.4 & 54.7 & 79.5 & 90.7 & 56.6 & 74.3 & 41.2 & 6.9 & 35.8 & 52.3 \\
multilingual-e5-small & 7 (399) & 55.0 & 55.7 & 80.9 & 86.4 & 56.1 & 71.6 & 46.1 & 14.0 & 36.5 & 54.1 \\
LaBSE & 8 (358) & 51.8 & 53.5 & 88.8 & 85.2 & 55.1 & 65.7 & 34.4 & 16.3 & 34.3 & 48.7 \\
multilingual-MiniLM-L12 & 9 (328) & 51.7 & 52.4 & 77.0 & 88.9 & 52.7 & 72.5 & 37.6 & 5.7 & 34.4 & 50.2 \\
all-mpnet-base & 10 (310) & 44.7 & 44.7 & 29.8 & 80.5 & 49.2 & 63.9 & 37.3 & 10.9 & 36.2 & 49.6 \\
all-MiniLM-L12 & 11 (292) & 44.4 & 44.1 & 32.1 & 81.5 & 49.2 & 64.2 & 36.2 & 7.6 & 32.5 & 49.2 \\
all-MiniLM-L6 & 12 (237) & 43.4 & 43.2 & 27.2 & 80.2 & 47.8 & 62.7 & 37.3 & 8.8 & 33.6 & 47.7 \\
\midrule
\multicolumn{12}{c}{\vspace{2mm} \normalsize \texttt{MTEB(Indic)}} \\
\textcolor{gray}{Number of datasets ($\rightarrow$) } & \textcolor{gray}{(23)} & \textcolor{gray}{(23)} & \multicolumn{1}{c}{\textcolor{gray}{(23)}} &   \multicolumn{1}{c}{\textcolor{gray}{(4)}} &   \textcolor{gray}{(1)}  &   \textcolor{gray}{(13)}   &   \textcolor{gray}{(1)}  &   \textcolor{gray}{(2)} &   \textcolor{gray}{(0)}  &   \textcolor{gray}{(1)} &   \textcolor{gray}{(1)}  \\
\midrule
multilingual-e5-large-instruct & 1 (209) & \textbf{70.2} & \textbf{71.6} & \textbf{80.4} & 76.3 & \textbf{67.0} & \textbf{53.7} & \textbf{84.9} & & \textbf{51.7} & \textbf{87.5} \\
multilingual-e5-large & 2 (188) & 66.4 & 65.1 & 77.7 & 75.1 & 64.7 & 43.9 & 82.6 & & 25.6 & 86.0 \\
multilingual-e5-base & 3 (173) & 64.6 & 62.6 & 74.2 & 72.8 & 63.8 & 41.1 & 77.8 & & 24.6 & 83.8 \\
multilingual-e5-small & 4 (164) & 64.7 & 63.2 & 73.7 & 73.8 & 63.8 & 40.8 & 76.8 & & 29.1 & 84.4 \\
GritLM-7B & 5 (151) & 60.2 & 58.0 & 58.4 & 67.8 & 60.0 & 27.2 & 79.5 & & 28.0 & 84.7 \\
e5-mistral-7b-instruct & 6 (144) & 60.0 & 58.4 & 59.1 & 73.0 & 59.6 & 23.0 & 77.3 & & 32.7 & 84.4 \\
LaBSE & 7 (139) & 61.9 & 59.7 & 74.1 & 64.6 & 61.9 & 52.8 & 64.3 & & 21.1 & 79.0 \\
multilingual-mpnet-base & 8 (137) & 58.5 & 55.2 & 44.2 & \textbf{82.0} & 61.9 & 34.1 & 57.9 & & 32.1 & 74.3 \\
multilingual-MiniLM-L12 & 9 (98) & 49.7 & 42.2 & 15.3 & 77.8 & 57.6 & 19.8 & 48.8 & & 16.7 & 59.3 \\
all-mpnet-base & 10 (68) & 33.6 & 22.6 & 3.7 & 52.6 & 45.2 & -2.5 & 12.9 & & 4.0 & 42.6 \\
all-MiniLM-L12 & 11 (49) & 33.1 & 23.2 & 3.5 & 55.0 & 43.9 & -5.3 & 13.9 & & 3.7 & 47.6 \\
all-MiniLM-L6 & 12 (40) & 31.8 & 20.4 & 2.5 & 53.7 & 44.1 & -6.3 & 6.2 & & 3.1 & 39.2 \\
\bottomrule

\end{tabular}
}
}  % edn resizebox
\caption{
% The results on three multilingual benchmarks. For each benchmark, we sort the score by rank (based on Borda count). We additionally supply an average across all tasks, an average per task category and an average weighted by task category.
The results for three multilingual benchmarks are ranked using Borda count. We provide averages across all tasks, per task category, and weighted by task category. The task categories are shortened as follows: Bitext Mining (Btxt), Pair Classification (Pr Clf), Classification (Clf), Semantic text similarity (STS), Retrieval (Rtrvl), Multilabel Classification (M. Clf), Clustering and Hierarchical Clustering (Clust) and Reranking (Rrnk). We highlight the best score in \textbf{bold}. Note that while Instruction retrieval \citep{weller2024followir} is included in \texttt{MTEB(Europe)} and \texttt{MTEB(Multilingual)}, but is excluded from the average by task category due to limited model support. For a broader model evaluation, refer to the public leaderboard.
}
\label{tab:overall-performance}
\end{table*}

\subsection{Theoretical Implications}

Our study explored the progress of online argument-making with the assistance of GenAI. To answer the first research question, our findings suggest that participants prompted GenAI by providing the specific context of the debate, trying to provoke aggressive responses. In this process, they also tried to balance their original stances and opinions with the content provided by GenAI. Various patterns emerged from the online forum posts, and participants combined different patterns for argumentation. They also committed logical fallacies in collaboration with GenAI. After a new person joined the debate, participants tended to maintain the original workflow of interacting with GenAI, while some reduced the usage of GenAI. In the free debate, two participants in the one-on-one debate formed teams with either a new member or GenAI, depending on their stances.

\subsubsection{Balancing the role of GenAI in the debate (RQ1)}

Previous research has primarily focused on the outcomes of co-writing with GenAI, evaluating the benefits and challenges. However, few studies have delved into the detailed process of argument-making. In our study, we observed that participants adapted their strategies to tailor GenAI to fit the debate scenario in online forums better. Some strategies are adjusting prompts from general to specific, providing detailed context, or assigning a particular role for GenAI, such as "a football fan" or "a professional debater". These findings extend the understanding of previous work as prompting can be challenging for participants in teamwork~\cite{han_when_2024}, and can also be challenging for non-experts to prompt GenAI~\cite{zamfirescu-pereira_why_2023}. In our context, where GenAI was used simultaneously by online forum members, the prompting process was straightforward for participants.

With support from GenAI, participants gained the confidence to express their opinions. Previous research has also shown that GenAI-powered assistance is beneficial for lifting people's confidence in writing~\cite{li_value_2024}. GenAI tools such as ChatGPT efficiently extract information from the Internet, allowing participants to create more straightforward outlines in academic writing and draw direct inspiration from it~\cite{tu_augmenting_2024}. However, in our study, participants noted that the content provided by ChatGPT was too formal and unnatural for forum posts. As a result, they adjusted the posting style to better fit the online forum's tone. The adoption of ChatGPT produced posts with similar content. Although GenAI has been utilized as a tool for enhancing critical thinking skills~\cite{tanprasert_debate_2024}, our findings revealed its potential harmfulness in inhibiting participants from developing a dialectical perspective and depth of thought.

In our research, participants did not tend to embrace opinions from ChatGPT or build up reciprocal relationships with it. In other words, they did not tend to adapt their opinions or stances to fulfill ChatGPT's expectations. Instead, they tended to maintain control over the entire debate. This aligns with previous literature implying that GenAI has limited normative influence on the co-writing process~\cite{jakesch_co-writing_2023}. Our research also suggests the situational use of GenAI, as participants chose to ignore ChatGPT's responses when there were disagreements of opinions among them. Participants wanted to integrate GenAI's content with their thoughts or utilize it to support their ideas. This notion corroborates with previous research on GenAI's roles when doing creative design tasks, showing that there was a latent hierarchy placing human thoughts above GenAI's content. Specifically, participants viewed GenAI as a validator when disagreements arose, whereas they treated GenAI as a supporter when agreements were reached~\cite{han_when_2024}. This observation also aligns with previous findings about GenAI's limitations in changing people's stances~\cite{tanprasert_debate_2024}, as participants reported that when disagreements arose, they chose to insist on their own opinions rather than follow the guidance of GenAI. In conclusion, participants strategically prompted ChatGPT to acquire information and support their opinions, and they even gave up using GenAI when facing disagreements, resulting in the situational use of ChatGPT. These findings, to some extent, challenged previous studies which suggest that  ChatGPT could decrease users' sense of ownership for argumentative writing~\cite{lee_design_2024, li_value_2024}. We infer that when polarized fans have a clear stance in online forums, they have a sense of accountability to take control of the debate.

\subsubsection{Creating similar posts and logical fallacies (RQ2)}

Our research indicates that participants brainstormed debate strategies with GenAI, acquired vital information, such as statistics and examples from GenAI, and incorporated them into their arguments. This finding echoes prior research which indicates that  GenAI could shift participants' opinions by exerting informational influence, emphasizing its capability of providing new information and persuasive arguments~\cite{jakesch_co-writing_2023}, which may escalate into ethical concerns on the manipulation of people's opinions~\cite{hancock_ai-mediated_2020}.

Participants in an online debate produced posts with similar content when collaborating with ChatGPT. For example, P4 made  arguments based on the same angle of "vision and creativity" three times. Within the context of argumentative essay writing, previous studies have also reported that utilizing GenAI could largely reduce the diversity of people's writing~\cite{li_value_2024}. In addition, homogenization of content may further undermine people's critical thinking skills~\cite{razi_not_2024}.


In addition to similar content, participants also committed logical fallacies in their posts. Previous research has found that deficiencies of GenAI caused by the internally synthesized algorithm of language models~\cite{fischer_generative_2023, razi_not_2024}, which include biased information~\cite{razi_not_2024} and misinformation~\cite{fischer_generative_2023, zhou_understanding_2024}. In contrast, we focused on the behaviors being manifested in collaboration with GenAI. We explored logical fallacies users commit, such as hasty generalizations, ad hominem attacks, and straw man arguments.

Although it is widely recognized that the sports community was overwhelmed with inter-group conflicts and hostile comments~\cite{wang_making_2023, zhang_intergroup_2019}, in our study, ad hominem attacks in the posts were relatively low compared to other kinds of fallacies (\autoref{fig7}). In light of this, future research may explore GenAI's latent persuasive abilities and its potential for alleviating hostile online debates~\cite{jakesch_co-writing_2023}.

\subsubsection{Maintaining the original workflow while reducing the usage of GenAI after a new member joined (RQ3)}

Our research also revealed the impact of GenAI on human behaviors. Previous work found that GenAI may disrupt the argument-making process and force participants to evaluate GenAI's suggestions~\cite{jakesch_co-writing_2023}.  However, prior research did not explore the detailed workflow of this process. In contrast, our research revealed that participants derived a behavioral route of prompting, obtaining information, and organizing thoughts in their interactions with GenAI and tended to maintain this behavior throughout the process. 

After the third participant came into the forum, participants' perceptions toward GenAI changed. We observed that participants teamed up with GenAI during the debate, especially those without a human teammate in Part 2 whose feelings of isolation urged them to do so. This finding extends prior literature on the relationship between humans and GenAI~\cite{han_when_2024}. However, after teaming up with ChatGPT and spending more time interacting with it, the participants without a teammate may give up using GenAI for a more timely response. This finding contradicts previous quantitative measurements showing that GenAI-powered assistance benefits people's productivity~\cite{li_value_2024}. Even though the time for writing may decrease for argumentative essay writing~\cite{li_value_2024}, participants can spend more time interacting with GenAI. This disparity might be caused by the differences between the formal setting of essay writing and the informal setting in online forums. Furthermore, there might also be discrepancies between participants' thoughts and actions, and thus, even though they may improve their productivity with the assistance of GenAI, they could still perceive this process as time-consuming.

While previous research has suggested that GenAI can help students become more engaged with asynchronous online discussions~\cite{lin_case_2024}, our study within a debate setting contradicts this to some extent. Participants found communication with GenAI to be distracting, which hindered their engagement in the debate. This may be explained by GenAI's strengths in providing information, coupled with its limitations in reasoning.

\subsection{Practical Implications}

\subsubsection{Visualizing logical constructs by GenAI}
Participants committed logical fallacies in their posts, highlighting issues in logical construction during the GenAI-mediated online argument-making process. With the continuous evolution of GenAI, it is becoming increasingly flexible in supporting various multimodal input/output (I/O) combinations. Practitioners may consider leveraging various techniques to visualize content structure and logical flow when writing opinion-based pieces. For example, the system could explicitly highlight the logical relationships among different pieces of content. This practice could help enhance users' awareness of the structural and logical aspects of their arguments, promoting iterative rethinking and critical evaluation of logic during argument formation. By doing so, users might create more logically coherent content, thereby enhancing efficient and constructive argument-making on online platforms.

\subsubsection{Developing intent-based argument-writing AI assistants}
We observed that participants adopted diverse methods to interact with ChatGPT, negotiating and balancing their own thoughts with the content provided by ChatGPT when drafting posts. This practice is often time-consuming and sometimes fails to meet participants' personalized needs when arguing with others online. In light of this, practitioners may consider adapting the characteristics of AI agents to better fit users' argument-writing needs based on their previous argument-making styles and human-AI interaction records. This may involve analyzing the patterns they commonly use when arguing with others and the types of information they retrieve from AI agents. This approach could create a more personalized argument-writing companion, reducing the direct prompt engineering effort required and promoting intent-AI interaction~\cite{ding_towards_2024}. Consequently, this may be helpful in improving users' experience, attitudes, and continued intention to use GenAI.


\subsection{Limitations and Future Work}

\subsubsection{Generalizability of participant characteristics}
Although we selected a topic that is relatively well-known globally and tried to include participants with diverse demographic characteristics, the majority of our recruited participants were non-native English speakers from Asia. As a result, the debate in the study may reflect culture-specific perspectives and vary across different ethnic backgrounds. In addition, all participants had an educational background as undergraduate students or even received postgraduate education. Thus, we probably ignored some marginalized groups on online forums. Therefore, other research may consider further diversifying the pool of participants to improve the generalizability of the study and pay much more attention to the marginalized groups, who might be vulnerable to hostile opinions and have less training in critical thinking skills.

\subsubsection{Modalities of content in online forum posts}
One limitation is that participants were required to post text-based content and emojis to the online forum. This meant that content with other modalities (e.g., images, audio, video, etc.) was excluded from this study. However, online forums in the real world usually support posting content in various formats, each of which can help forum members express their opinions and feelings. In light of this, future research may consider including richer modalities in online posts such as images co-created with GenAI in diverse contexts~\cite{fu_being_2024, lc_speculative_2024, lc_together_2023, lc_time_2024}, and exploring the patterns that emerge from these posts.

\subsubsection{Number of online forum members}
Real online discussion often involves multiple members, some joining early and others joining later. In our study, the first two participants were introduced in Part 1, and the third participant was introduced in Part 2, representing those who joined subsequently. The number of participants was limited to three to prevent potential chaos during data collection and presentation. However, the limited number of forum members may not fully capture the dynamics of real online forum discussions. A larger scale of the forum discussion might lead to more intricate discussions and interactions between participants and ChatGPT, potentially influencing the depth and complexity of the discourse. Therefore, further investigation on this topic may consider involving more forum members to understand people in real-world scenarios better.

\subsubsection{User interface and interaction design}
Participants were required to share their screens throughout the entire study process, during which we observed a degree of incoherence when they accessed ChatGPT to construct arguments on the forum. Participants needed to interact with ChatGPT while communicating with other forum members in separate panels. Frequently switching between ChatGPT and the forum may have reduced participants' willingness to use ChatGPT and distracted them from the online discussion. Future work may consider seamlessly integrating GenAI into the online forum interface to promote both human-AI interaction and human-human communication.

\subsubsection{Evaluation methods of online arguments' persuasiveness}
We primarily employed qualitative methods to interpret data from forum posts, ChatGPT records, and interviews. While qualitative methods are effective for probing participants' perceptions, behaviors, and experiences, we did not measure the persuasiveness of their arguments. Therefore, future research may consider adopting quantitative methods to assess the persuasiveness of writing outcomes in collaboration with GenAI. This approach may provide direct evidence to evaluate the effectiveness of GenAI in co-creating arguments with humans.


\subsubsection{Lack of representation of actual online posting environments}
To better observe the argument-making process, we designed both a turn-based debate and a free debate, aiming to gain a nuanced understanding of argument-making behavior in online forums and participants' usage of ChatGPT. However, this artificial setup cannot perfectly replicate natural online debate in a forum where members might hold a variety of stances rather than being extremely polarized as we assumed, either supporting Messi or Ronaldo. If the research setting were based on real online forums instead of the one we designed, it might better represent actual online communication environments and reduce the Hawthorne effect caused by the research.

\subsubsection{Constrained use of ChatGPT and other tools}
To better understand how people use ChatGPT, participants were not allowed to use third-party search engines such as Google during the study. However, in reality, forum members are not forced to use ChatGPT or other specific tools in a constrained way. Additionally, as we used only one GenAI tool, ChatGPT (GPT-4o), as our study apparatus, it also constrained how people obtained the data. Consequently, it may be worth exploring the interplay between GenAI and other types of tools complementing each other to see how GenAI can integrate with participants' information acquisition more naturally.


\section*{Impact Statement}

% This work dynamically adjusts resource allocation based on input quality and compute availability, optimizing efficiency without compromising performance. 
% By reducing unnecessary computations, \name significantly lowers energy consumption and carbon emissions, contributing to sustainable AI practices. 
% Its ability to adapt to varying computational constraints makes it valuable for resource-limited environments, such as edge devices and real-time systems.


This paper presents work whose goal is to advance the field of 
Machine Learning. There are many potential societal consequences 
of our work, none which we feel must be specifically highlighted here.


\section*{Acknowledgments}
The research reported in this paper was sponsored in part by the DEVCOM Army Research Laboratory (award \# W911NF1720196 ), the Air Force Office of Scientific Research (awards \#  FA95502210193 and FA95502310559), and the National Institutes of Health (award \# 1P41EB028242). The views and conclusions contained in this document are those of the authors and should not be interpreted as representing the official policies, either expressed or implied, of the funding agencies. Jason Wu was supported by the Department of Defense (DoD) through the National Defense Science \& Engineering Graduate (NDSEG) Fellowship Program.



\bibliography{main}
\bibliographystyle{icml2025}


%%%%%%%%%%%%%%%%%%%%%%%%%%%%%%%%%%%%%%%%%%%%%%%%%%%%%%%%%%%%%%%%%%%%%%%%%%%%%%%
%%%%%%%%%%%%%%%%%%%%%%%%%%%%%%%%%%%%%%%%%%%%%%%%%%%%%%%%%%%%%%%%%%%%%%%%%%%%%%%
% APPENDIX
%%%%%%%%%%%%%%%%%%%%%%%%%%%%%%%%%%%%%%%%%%%%%%%%%%%%%%%%%%%%%%%%%%%%%%%%%%%%%%%
%%%%%%%%%%%%%%%%%%%%%%%%%%%%%%%%%%%%%%%%%%%%%%%%%%%%%%%%%%%%%%%%%%%%%%%%%%%%%%%
\newpage
\appendix
\onecolumn
\section{Appendix}

\subsection{Controller Training Overhead}
\label{subsec:controller_overhead}
We found it sufficient to train the controller for 10 and 15 epochs on the localization and classification tasks, respectively. We attribute this to the simple end-to-end training recipe in which we avoid complex reinforcement learning. On a Nvidia RTX 4090, training the localization controller took only 27 min. 


% Add the following packages to your preamble:
% \usepackage{booktabs}

\begin{table*}[h]
\centering
\begin{tabular}{lcccccccc}
\toprule
\textbf{Noise Type}       & \textbf{Seed} & \textbf{6 Layers} & \textbf{8 Layers} & \textbf{12 Layers} & \textbf{16 Layers} \\ 
\midrule
\multirow{3}{*}{\textbf{Binary}} 
                    & 100           & 25.62             & 22.40             & 21.37              & 20.87              \\
                    & 200           & 24.49             & 22.56             & 21.88              & 20.66              \\
                    & 300           & 24.60             & 22.56             & 22.10              & 20.75              \\ 
\midrule
\multirow{3}{*}{\textbf{Discrete}} 
                    & 100           & 51.96             & 36.35             & 32.24              & 29.38              \\
                    & 200           & 47.29             & 39.12             & 33.01              & 30.95              \\
                    & 300           & 56.09             & 37.02             & 32.11              & 31.27              \\ 
\midrule
\multirow{3}{*}{\textbf{Continuous}} 
                    & 100           & 54.47             & 42.92             & 36.18              & 32.55              \\
                    & 200           & 53.25             & 42.20             & 36.96              & 33.66              \\
                    & 300           & 53.59             & 41.56             & 36.73              & 35.25              \\ 
\bottomrule
\end{tabular}
\caption{\name Localization Error (cm) $\downarrow$. Shown previously in Table \ref{tab:results} with averaging}
\label{tab:admn_full_loc}
\end{table*}



% Add the following packages to your preamble:
% \usepackage{booktabs}
% Add the following packages to your preamble:
% \usepackage{booktabs}
% Add the following packages to your preamble:
% \usepackage{booktabs}

\begin{table}[h]
\centering
\begin{tabular}{lcccccc}
\toprule
\textbf{Task}       & \textbf{Seed} & \textbf{6 Layers} & \textbf{8 Layers} & \textbf{12 Layers} & \textbf{16 Layers} \\ 
\midrule
\multirow{3}{*}{\textbf{Finite}}     
                    & 100           & 73.60             & 64.32             & 56.15              & 21.54              \\
                    & 200           & 98.77             & 49.90             & 24.30              & 22.17              \\
                    & 300           & 62.36             & 63.68             & 23.23              & 21.90              \\ 
\midrule
\multirow{3}{*}{\textbf{Discrete}}   
                    & 100           & 118.14            & 90.36             & 73.87              & 32.15              \\
                    & 200           & 94.43             & 58.66             & 50.76              & 51.32              \\
                    & 300           & 103.21            & 82.20             & 51.84              & 29.81              \\ 
\midrule
\multirow{3}{*}{\textbf{Continuous}} 
                    & 100           & 87.90             & 55.42             & 75.52              & 36.51              \\
                    & 200           & 96.32             & 118.54            & 42.46              & 31.16              \\
                    & 300           & 78.65             & 86.94             & 78.57              & 34.23              \\ 
\bottomrule
\end{tabular}
\caption{Straight-Through Estimator Localization Error (cm) $\downarrow$}
\label{tab:st_full_loc}
\end{table}


\subsection{Justification of Gradient Propagation Technique in the Controller}
\label{subsec:grad_justify}
\textbf{Directly Employing the Straight-Through Estimator:}
\name utilizes the combination of standard temperature Gumbel-Softmax sampling and the straight-through estimator to propagate gradients over the discretization to the continuous logits. One natural question is whether Gumbel-Softmax Sampling is necessary, as one could theoretically apply discretization on the raw logits and propagate gradients with the straight-through estimator. In Table \ref{tab:admn_full_loc} and Table \ref{tab:st_full_loc}, we present the localization results on the GTDM dataset across different layer configurations and noise categories, with three seeds for each experiment. The results highlight that Gumbel-Softmax sampling plays an important role in model training.

This behavior can be attributed to several reasons. First, by applying the softmax function to the logits, we convert them into probability values where one logit's high probabilities come at the expense of the others. As a result, the softmax function encourages the controller to select only the $L$ best performing layers for some value of noise and minimize the probability of the remaining layers. Additionally, utilizing the Gumbel distribution also introduces \emph{stochasticity} into the sampling process. Instead of always selecting the top-L logits as the active layers, the stochasticity intuitively serves to encourage \emph{exploration} of different layer configurations. 


\textbf{Progressive Top-L Gumbel Softmax Sampling:} 
Instead of employing the straight-through estimator, one can also utilize repeated Gumbel-Softmax Sampling to emulate discrete top-L sampling. \cite{xie2019reparameterizable} proposed a method to emulate discrete top-L sampling by repeatedly applying the softmax function $L$ times while adjusting the logits each iteration. However, these methods may cause issues when applied to \name. First, methods utilizing Gumbel-Softmax Sampling to emulate discrete distributions typically have to undergo \emph{temperature annealing} \cite{maddison2016gumbel}, where the temperature is slowly decreased until the distribution is approximately categorical. Utilizing annealing can lead to a longer and more complicated training process for the controller. Additionally, the lack of explicit discretization during training may also result in a distribution shift at inference time, where the controller may learn to overrely upon partially activated layers during training. 


\subsection{Training Details: }\label{sec_appendix_training}

% Add the following packages to your preamble:
% \usepackage{booktabs}
% \usepackage{multicol}
% \usepackage{caption}

\begin{table*}[h]
\centering
\label{tab:training_configs}
\begin{minipage}[t]{0.33\linewidth}
\centering
\begin{tabular}{lc}
\toprule
\textbf{Parameter}      & \textbf{Value} \\ 
\midrule
Epochs                  & 400            \\
Learning Rate           & 1E-4       \\
Scheduler               & LinearLR  \\
Optimizer               & Adam           \\
LayerDrop               & 0.2            \\
Fusion Layers           & 6              \\
Fusion Dimension        & 64            \\
Fusion Heads            & 4              \\
Modality Dropout        & 0.1            \\
Depth Noise             & [0, 3]      \\
Image Noise             & [0, 2]         \\
\bottomrule
\end{tabular}
\caption{\textbf{MM-Fi Finetuning}}
\label{tab:mmfi_details}
\end{minipage}%
\hfill
\begin{minipage}[t]{0.33\linewidth}
\centering
\begin{tabular}{lc}
\toprule
\textbf{Parameter}      & \textbf{Value} \\ 
\midrule
Epochs                  & 400            \\
Learning Rate           & 5.00E-04       \\
Optimizer               & Adam           \\
LayerDrop               & 0.2            \\
Fusion Layers           & 6              \\
Fusion Dimension        & 256            \\
Fusion Heads            & 4              \\
Modality Dropout        & 0.1            \\
Depth Noise             & [0, 0.75]      \\
Image Noise             & [0, 3]         \\
\bottomrule
\end{tabular}
\caption{\textbf{GDTM Finetuning}}
\label{tab:gdtm_details}
\end{minipage}%
\hfill
\begin{minipage}[t]{0.33\linewidth}
\centering
\begin{tabular}{lc}
\toprule
\textbf{Parameter}      & \textbf{Value} \\ 
\midrule
Epochs                  & 10/15             \\
Learning Rate           & 1.00E-03       \\
Scheduler               & LinearLR       \\
Gumbel Temperature      & 1              \\
\bottomrule
\end{tabular}
\caption{\textbf{Controller Training}}
\label{tab:controller_details}
\end{minipage}
\end{table*}



\textbf{MM-Fi Classification: } We depict the Stage 1 details for the MMFI Classification main network in Table \ref{tab:mmfi_details}. We employ the same 12 Layer ViT-Base model from pretraining. After we obtain the embeddings from the backbones, they are fused through a stack of Transformer Encoders with the details shown above. We employ a 0.2 LayerDrop rate and add Gaussian Noise with a randomly drawn standard deviation to the input for every batch. The image Gaussian Noise standard deviation is uniformly drawn from the range 0 to 2 and the depth noise is drawn from the range 0 to 3.  

\textbf{GDTM Localization: } The training details are shown in Table \ref{tab:gdtm_details}. Similarly to the MM-Fi Classification Model, we use a Transformer Encoder to perform multimodal fusion. The range of standard deviations for depth is smaller due to increased sensitivity of the depth modality in this dataset. 



\subsection{Additional LayerDrop Results}
\label{appendix:layerdrop}

% Add the following packages to your preamble:
% \usepackage{booktabs}
% Add the following packages to your preamble:
% \usepackage{booktabs}
% Add the following packages to your preamble:
% \usepackage{booktabs}

\begin{table}[h]
\centering
\begin{tabular}{lcccc}
\toprule
\textbf{Removed Layer Indices} & \textbf{Normal Pre} & \textbf{LayerDrop Pre} & \textbf{Normal Pre} & \textbf{LayerDrop Pre} \\
                        & \textbf{+ Normal FT} & \textbf{+ Normal FT}   & \textbf{+ LayerDrop FT} & \textbf{+ LayerDrop FT} \\
\midrule
None           & 81.16\%                & 80.36\%                   & 79.91\%                   & 78.92\%                      \\
6              & 79.99\%                & 79.35\%                   & 79.61\%                   & 78.59\%                      \\
6, 8           & 77.20\%                & 76.77\%                   & 78.68\%                   & 77.72\%                      \\
4, 6, 8        & 72.97\%                & 73.94\%                   & 77.64\%                   & 76.84\%                      \\
2, 4, 6, 8     & 70.01\%                & 71.48\%                   & 76.90\%                   & 76.37\%                      \\
2, 4, 6, 8, 10 & 64.27\%                & 65.07\%                   & 75.44\%                   & 74.70\%                      \\
2, 4, 6, 7, 8, 10   & 33.91\%            & 38.76\%                   & 69.83\%                   & 69.28\%                      \\
1, 2, 4, 6, 7, 8, 10 & 13.74\%            & 23.60\%                   & 66.67\%                   & 66.94\%                      \\
\bottomrule
\end{tabular}
\caption{ImageNet-1K performance with different layer indices removed. Normal refers to a LayerDrop rate of 0, while LayerDrop refers to utilizing a LayerDrop rate of 0.2. Pre indicates the MAE pretraining stage, while FT refers to supervised finetuning on ImageNet-1K. }
\label{tab:layer_removal}
\end{table}

The ViT models are first pretrained on the ImageNet dataset with Masked Autoencoder pretraining, in which we add LayerDrop. To understand its performance on the ImageNet-1K dataset, we perform a subsequent stage of supervised learning on the ImageNet dataset to obtain the validation accuracy. Table \ref{tab:layer_removal} reveals the validation accuracy on the ImageNet-1K dataset with various dropped layers, and with LayerDrop integrated into different stages of training. When comparing the model trained without any usage of LayerDrop to the one in which LayerDrop was employed in both stages, we can observe an accuracy improvement of over 50\% when 7 layers are dropped during inference time. Curiously, given that LayerDrop is added during supervised finetuning, applying MAE pretraining with LayerDrop does not appear to be necessary in ImageNet-1K. However, the results in Figure \ref{fig:layerdrop_plot} showcase that it has an impact on downstream tasks. 

\subsection{Qualitative Results}

\begin{figure}[h]
    \centering
    \includegraphics[width=0.8\linewidth]{Figures/Qualitative_Results_First_Part.png}
\end{figure}
\begin{figure}[h]
    \centering
    \includegraphics[width=0.8\linewidth]{Figures/Qualitative_Results_Second_Part.png}
    \caption{Visual Results on the GDTM Dataset highlighting the impact of noise and featuring the controller layer allocation}
    \label{fig:qualitative}
\end{figure}

In Figure \ref{fig:qualitative}, we visually showcase the noise corrupted multimodal inputs, and the corresponding layer allocation output by the \name controller with a budget of 6 layers. Given that the depth modality is naturally lower QoI, we can see that the controller gives preference towards the image modality. When Image Noise Standard Deviation is 3 and Depth Noise Standard Deviation is 0.25, the controller equally allocates layers among the two modalities. However, when the depth is entirely clean, the controller recognizes this and allocates all the layers towards the clean depth modality. These results reveal the intelligent allocation of the \name controller. For stability reasons, the first layer of the backbone is always activated. 


%%%%%%%%%%%%%%%%%%%%%%%%%%%%%%%%%%%%%%%%%%%%%%%%%%%%%%%%%%%%%%%%%%%%%%%%%%%%%%%
%%%%%%%%%%%%%%%%%%%%%%%%%%%%%%%%%%%%%%%%%%%%%%%%%%%%%%%%%%%%%%%%%%%%%%%%%%%%%%%


\end{document}


% This document was modified from the file originally made available by
% Pat Langley and Andrea Danyluk for ICML-2K. This version was created
% by Iain Murray in 2018, and modified by Alexandre Bouchard in
% 2019 and 2021 and by Csaba Szepesvari, Gang Niu and Sivan Sabato in 2022.
% Modified again in 2023 and 2024 by Sivan Sabato and Jonathan Scarlett.
% Previous contributors include Dan Roy, Lise Getoor and Tobias
% Scheffer, which was slightly modified from the 2010 version by
% Thorsten Joachims & Johannes Fuernkranz, slightly modified from the
% 2009 version by Kiri Wagstaff and Sam Roweis's 2008 version, which is
% slightly modified from Prasad Tadepalli's 2007 version which is a
% lightly changed version of the previous year's version by Andrew
% Moore, which was in turn edited from those of Kristian Kersting and
% Codrina Lauth. Alex Smola contributed to the algorithmic style files.
